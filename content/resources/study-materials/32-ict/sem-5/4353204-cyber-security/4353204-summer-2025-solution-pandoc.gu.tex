\documentclass[10pt,a4paper]{article}

% content/resources/templates/preamble.tex
\usepackage[margin=0.6in]{geometry}
\author{Milav Dabgar}
\usepackage{amsmath,amssymb,amsthm}
\usepackage{booktabs}
\usepackage{multirow}
\usepackage{xcolor}
\usepackage{tcolorbox}
\tcbuselibrary{breakable,skins}
\usepackage[colorlinks=true,linkcolor=blue]{hyperref}
\usepackage{titlesec}
\usepackage{enumitem}
\usepackage{tikz}
\usepackage{pgfplots}
\usepackage{circuitikz}
\usepackage[version=4]{mhchem}
\usepackage{longtable}
\usepackage{array}
\usepackage{float}
\usepackage{caption}
\usepackage{listings}

\lstset{
  basicstyle=\small\ttfamily,
  breaklines=true,
  breakatwhitespace=false,
  postbreak=\mbox{\textcolor{red}{$\hookrightarrow$}\space},
  float=false,
  numbers=left,
  numberstyle=\tiny\color{gray},
  numbersep=10pt,
  xleftmargin=2em,
  keywordstyle=\color{blue},
  commentstyle=\color{green!60!black},
  stringstyle=\color{purple},
  backgroundcolor=\color{gray!5},
  showstringspaces=false,
  tabsize=2,
  captionpos=b,
  keepspaces=true,
  columns=flexible
}

\pgfplotsset{compat=1.18}
\usetikzlibrary{shapes,arrows,positioning,calc,patterns,decorations.pathmorphing,decorations.markings,arrows.meta}

% Color scheme
\definecolor{headcolor}{RGB}{0,102,204}
\definecolor{keycolor}{RGB}{220,20,60}
\definecolor{solutioncolor}{RGB}{34,139,34}
\definecolor{mnemoniccolor}{RGB}{148,0,211}
\definecolor{codecolor}{RGB}{0,0,100}

% Spacing
\setlength{\parskip}{3pt}
\setlist[itemize]{nosep}
\setlist[enumerate]{nosep}

% Title formatting
\titleformat{\section}{\Large\bfseries\color{headcolor}}{\thesection}{1em}{}
\titleformat{\subsection}{\large\bfseries\color{headcolor}}{\thesubsection}{1em}{}

% Pandoc tightlist compatibility
\providecommand{\tightlist}{%
  \setlength{\itemsep}{0pt}\setlength{\parskip}{0pt}}

% Pandoc longtable compatibility
\newcounter{none}
\def\thenone{}


% content/resources/templates/gujarati-boxes.tex
\usepackage{fontspec}
\usepackage{polyglossia}

% Set Gujarati as main language (document is primarily in Gujarati)
% Note: gloss-gujarati.ldf doesn't exist in polyglossia, but it will use hyphenation patterns
\setdefaultlanguage{gujarati}
\setotherlanguage{english}

% Configure Gujarati font properly
% Use Language=Default to prevent polyglossia from trying to add language-specific features
% that don't exist for Gujarati, which causes "empty feature" warnings
\newfontfamily\gujaratifont[Script=Gujarati,AutoFakeBold=2.5,AutoFakeSlant=0.3]{Noto Sans Gujarati}
\setmainfont[Script=Gujarati,AutoFakeBold=2.5,AutoFakeSlant=0.3]{Noto Sans Gujarati}
% Use Noto Sans Gujarati for monospace to support Gujarati in text
\setmonofont[Scale=0.9]{Noto Sans Gujarati}

% Configure English to use the same font
\newfontfamily\englishfont[Script=Gujarati,AutoFakeBold=2.5,AutoFakeSlant=0.3]{Noto Sans Gujarati}

% Translations for polyglossia
\gappto\captionsgujarati{
  \renewcommand{\tablename}{કોષ્ટક}
  \renewcommand{\figurename}{આકૃતિ}
}

% Helper for TikZ nodes to ensure Gujarati font
\newcommand{\gu}[1]{{\gujaratifont #1}}

% Custom environments
\newtcolorbox{solutionbox}{
    breakable,
    enhanced,
    colback=solutioncolor!5!white,
    colframe=solutioncolor!75!black,
    fonttitle=\bfseries,
    title=જવાબ
}

\newtcolorbox{solutionboxnobreak}{
 colback=solutioncolor!5!white,
 colframe=solutioncolor!75!black,
 fonttitle=\bfseries,
 title=જવાબ
}

\newtcolorbox{keyformula}{
 breakable,
 enhanced,
 colback=keycolor!5!white,
 colframe=keycolor!75!black,
 fonttitle=\bfseries,
 title=રાસાયણિક સમીકરણ/સૂત્ર
}

\newtcolorbox{mnemonicbox}{
 breakable,
 enhanced,
 colback=mnemoniccolor!5!white,
 colframe=mnemoniccolor!75!black,
 fonttitle=\bfseries,
 title=મેમરી ટ્રીક
}


\begin{document}

\begin{center}
{\Huge\bfseries\color{headcolor} Subject Name (Gujarati)}\\[5pt]
{\LARGE 4353204 -- Summer 2025}\\[3pt]
{\large Semester 1 Study Material}\\[3pt]
{\normalsize\textit{Detailed Solutions and Explanations}}
\end{center}

\vspace{10pt}

\subsection*{પ્રશ્ન 1(અ) [3
ગુણ]}\label{uxaaauxab0uxab6uxaa8-1uxa85-3-uxa97uxaa3}

\textbf{ઉદાહરણ સાથે CIA ત્રિપુટીનું વર્ણન કરો.}

\begin{solutionbox}

\textbf{CIA ત્રિપુટીના ઘટકો:}

\begin{figure}
\centering
\pandocbounded{\includesvg[keepaspectratio]{diagrams/cia-triad.svg}}
\caption{CIA Triad}
\end{figure}

{\def\LTcaptype{none} % do not increment counter
\begin{longtable}[]{@{}
  >{\raggedright\arraybackslash}p{(\linewidth - 4\tabcolsep) * \real{0.2273}}
  >{\raggedright\arraybackslash}p{(\linewidth - 4\tabcolsep) * \real{0.3636}}
  >{\raggedright\arraybackslash}p{(\linewidth - 4\tabcolsep) * \real{0.4091}}@{}}
\toprule\noalign{}
\begin{minipage}[b]{\linewidth}\raggedright
ઘટક
\end{minipage} & \begin{minipage}[b]{\linewidth}\raggedright
વ્યાખ્યા
\end{minipage} & \begin{minipage}[b]{\linewidth}\raggedright
ઉદાહરણ
\end{minipage} \\
\midrule\noalign{}
\endhead
\bottomrule\noalign{}
\endlastfoot
\textbf{કન્ફિડેન્શિયાલિટી} & અનધિકૃત એક્સેસથી ડેટાનું રક્ષણ & બેંક એકાઉન્ટ પર પાસવર્ડ
પ્રોટેક્શન \\
\textbf{ઇન્ટેગ્રિટી} & ડેટાની ચોકસાઈ અને સંપૂર્ણતા & ડોક્યુમેન્ટ પર ડિજિટલ સહી \\
\textbf{એવેઇલેબિલિટી} & જરૂરિયાત મુજબ સિસ્ટમની ઉપલબ્ધતા & 24/7 ઓનલાઇન બેંકિંગ
સેવાઓ \\
\end{longtable}
}

\begin{itemize}
\tightlist
\item
  \textbf{કન્ફિડેન્શિયાલિટી}: માત્ર અધિકૃત વપરાશકર્તાઓ જ સંવેદનશીલ માહિતી એક્સેસ
  કરી શકે
\item
  \textbf{ઇન્ટેગ્રિટી}: ટ્રાન્સમિશન દરમિયાન ડેટા ચોક્કસ અને અપરિવર્તિત રહે
\item
  \textbf{એવેઇલેબિલિટી}: સિસ્ટમો કાયદેસર વપરાશકર્તાઓ માટે કાર્યરત અને સુલભ રહે
\end{itemize}

\end{solutionbox}
\begin{mnemonicbox}
``CIA માહિતી ને સુરક્ષિત રાખે''

\end{mnemonicbox}
\subsection*{પ્રશ્ન 1(બ) [4
ગુણ]}\label{uxaaauxab0uxab6uxaa8-1uxaac-4-uxa97uxaa3}

\textbf{પબ્લિક કી અને પ્રાઇવેટ કી ક્રિપ્ટોગ્રાફી સમજાવો.}

\begin{solutionbox}

\textbf{પબ્લિક કી ક્રિપ્ટોગ્રાફી (એસિમેટ્રિક):}

\begin{figure}
\centering
\pandocbounded{\includesvg[keepaspectratio]{diagrams/public-key-cryptography.svg}}
\caption{Public Key Cryptography}
\end{figure}

\begin{center}
\textbf{Mermaid Diagram (Code)}
\begin{verbatim}
{Shaded}
{Highlighting}[]
graph LR
    A[મોકલનાર] {-{-}{}|પબ્લિક કી સાથે એન્ક્રિપ્ટ| B[એન્ક્રિપ્ટેડ મેસેજ]}
    B {-{-}{}|પ્રાઇવેટ કી સાથે ડિક્રિપ્ટ| C[પ્રાપ્તકર્તા]}
{Highlighting}
{Shaded}
\end{verbatim}
\end{center}

\textbf{મુખ્ય લક્ષણો:}

{\def\LTcaptype{none} % do not increment counter
\begin{longtable}[]{@{}lll@{}}
\toprule\noalign{}
વિશેષતા & પબ્લિક કી & પ્રાઇવેટ કી \\
\midrule\noalign{}
\endhead
\bottomrule\noalign{}
\endlastfoot
\textbf{વિતરણ} & મુક્તપણે શેર કરાય & ગુપ્ત રાખાય \\
\textbf{ઉપયોગ} & એન્ક્રિપ્શન/વેરિફિકેશન & ડિક્રિપ્શન/સાઇનિંગ \\
\textbf{સુરક્ષા} & જાહેર હોઈ શકે & સુરક્ષિત રાખવી જરૂરી \\
\end{longtable}
}

\begin{itemize}
\tightlist
\item
  \textbf{પબ્લિક કી}: એન્ક્રિપ્શન અને સિગ્નેચર વેરિફિકેશન માટે
\item
  \textbf{પ્રાઇવેટ કી}: ડિક્રિપ્શન અને ડિજિટલ સાઇનિંગ માટે
\item
  \textbf{સુરક્ષા}: ગાણિતિક જટિલતા પર આધારિત (RSA, ECC અલ્ગોરિધમ)
\end{itemize}

\end{solutionbox}
\begin{mnemonicbox}
``પબ્લિક એન્ક્રિપ્ટ કરે, પ્રાઇવેટ ડિક્રિપ્ટ કરે''

\end{mnemonicbox}
\subsection*{પ્રશ્ન 1(ક) [7
ગુણ]}\label{uxaaauxab0uxab6uxaa8-1uxa95-7-uxa97uxaa3}

\textbf{OSI મોડેલના દરેક સ્તર સાથે સંકળાયેલ વિવિધ સુરક્ષા હુમલાઓ, પદ્ધતિઓ અને સેવાઓ
સમજાવો.}

\begin{solutionbox}

\textbf{OSI સુરક્ષા ફ્રેમવર્ક:}

\begin{figure}
\centering
\pandocbounded{\includesvg[keepaspectratio]{diagrams/osi-security-framework.svg}}
\caption{OSI Security Framework}
\end{figure}

{\def\LTcaptype{none} % do not increment counter
\begin{longtable}[]{@{}
  >{\raggedright\arraybackslash}p{(\linewidth - 6\tabcolsep) * \real{0.1562}}
  >{\raggedright\arraybackslash}p{(\linewidth - 6\tabcolsep) * \real{0.2812}}
  >{\raggedright\arraybackslash}p{(\linewidth - 6\tabcolsep) * \real{0.3125}}
  >{\raggedright\arraybackslash}p{(\linewidth - 6\tabcolsep) * \real{0.2500}}@{}}
\toprule\noalign{}
\begin{minipage}[b]{\linewidth}\raggedright
સ્તર
\end{minipage} & \begin{minipage}[b]{\linewidth}\raggedright
હુમલાઓ
\end{minipage} & \begin{minipage}[b]{\linewidth}\raggedright
પદ્ધતિઓ
\end{minipage} & \begin{minipage}[b]{\linewidth}\raggedright
સેવાઓ
\end{minipage} \\
\midrule\noalign{}
\endhead
\bottomrule\noalign{}
\endlastfoot
\textbf{ફિઝિકલ} & વાયરટેપિંગ, જેમિંગ & ફિઝિકલ સિક્યોરિટી, શિલ્ડિંગ & એક્સેસ
કંટ્રોલ \\
\textbf{ડેટા લિંક} & MAC ફ્લડિંગ, ARP પોઇઝનિંગ & એન્ક્રિપ્શન, ઓથેન્ટિકેશન & ફ્રેમ
ઇન્ટેગ્રિટી \\
\textbf{નેટવર્ક} & IP સ્પૂફિંગ, રાઉટિંગ એટેક & IPSec, ફાયરવોલ & પેકેટ ફિલ્ટરિંગ \\
\textbf{ટ્રાન્સપોર્ટ} & સેશન હાઇજેકિંગ, SYN ફ્લડિંગ & SSL/TLS, પોર્ટ સિક્યોરિટી &
એન્ડ-ટુ-એન્ડ સિક્યોરિટી \\
\textbf{સેશન} & સેશન રિપ્લે, હાઇજેકિંગ & સેશન ટોકન, ટાઇમઆઉટ & સેશન મેનેજમેન્ટ \\
\textbf{પ્રેઝન્ટેશન} & ડેટા કરપ્શન, ફોર્મેટ એટેક & એન્ક્રિપ્શન, કમ્પ્રેશન & ડેટા
ટ્રાન્સફોર્મેશન \\
\textbf{એપ્લિકેશન} & મેલવેર, સોશિયલ એન્જિનિયરિંગ & એન્ટિવાયરસ, યુઝર ટ્રેનિંગ &
એપ્લિકેશન સિક્યોરિટી \\
\end{longtable}
}

\textbf{મુખ્ય સુરક્ષા સેવાઓ:}

\begin{itemize}
\tightlist
\item
  \textbf{ઓથેન્ટિકેશન}: યુઝર આઇડેન્ટિટી વેરિફિકેશન
\item
  \textbf{ઓથોરાઇઝેશન}: એક્સેસ પરમિશન કંટ્રોલ
\item
  \textbf{નોન-રિપ્યુડિએશન}: ક્રિયાઓનો ઇનકાર અટકાવવો
\item
  \textbf{ડેટા ઇન્ટેગ્રિટી}: ડેટાની ચોકસાઈ સુનિશ્ચિત કરવી
\end{itemize}

\end{solutionbox}
\begin{mnemonicbox}
``બધા લોકોને ડેટા પ્રોટેક્શનની જરૂર છે''

\end{mnemonicbox}
\subsection*{પ્રશ્ન 1(ક અથવા) [7
ગુણ]}\label{uxaaauxab0uxab6uxaa8-1uxa95-uxa85uxaa5uxab5-7-uxa97uxaa3}

\textbf{MD5 હેશિંગ અને સિક્યોર હેશ ફંક્શન (SHA) અલ્ગોરિધમ સમજાવો.}

\begin{solutionbox}

\textbf{હેશ ફંક્શન સરખામણી:}

\begin{figure}
\centering
\pandocbounded{\includesvg[keepaspectratio]{diagrams/hash-function-process.svg}}
\caption{Hash Function Process}
\end{figure}

{\def\LTcaptype{none} % do not increment counter
\begin{longtable}[]{@{}llll@{}}
\toprule\noalign{}
વિશેષતા & MD5 & SHA-1 & SHA-256 \\
\midrule\noalign{}
\endhead
\bottomrule\noalign{}
\endlastfoot
\textbf{આઉટપુટ સાઇઝ} & 128 બિટ્સ & 160 બિટ્સ & 256 બિટ્સ \\
\textbf{સુરક્ષા સ્તર} & નબળું & નબળું & મજબૂત \\
\textbf{ઝડપ} & ઝડપી & મધ્યમ & ધીમું \\
\textbf{વર્તમાન સ્થિતિ} & અપ્રચલિત & અપ્રચલિત & ભલામણ કરેલ \\
\end{longtable}
}

\begin{center}
\textbf{Mermaid Diagram (Code)}
\begin{verbatim}
{Shaded}
{Highlighting}[]
graph LR
    A[ઇનપુટ મેસેજ] {-{-}{} B[હેશ ફંક્શન]}
    B {-{-}{} C[ફિક્સ્ડ{-}સાઇઝ હેશ]}
    C {-{-}{} D[ડિજિટલ ફિંગરપ્રિન્ટ]}
{Highlighting}
{Shaded}
\end{verbatim}
\end{center}

\textbf{હેશ ગુણધર્મો:}

\begin{itemize}
\tightlist
\item
  \textbf{ડિટર્મિનિસ્ટિક}: સમાન ઇનપુટ સમાન હેશ આપે
\item
  \textbf{એવેલાન્ચ ઇફેક્ટ}: નાનો ઇનપુટ ફેરફાર મોટો હેશ ફેરફાર લાવે
\item
  \textbf{વન-વે ફંક્શન}: હેશથી મૂળ ડેટા મેળવી શકાતો નથી
\item
  \textbf{કોલિઝન રેઝિસ્ટન્ટ}: બે અલગ ઇનપુટ માટે સમાન હેશ મળવો મુશ્કેલ
\end{itemize}

\textbf{એપ્લિકેશન:}

\begin{itemize}
\tightlist
\item
  પાસવર્ડ સ્ટોરેજ અને વેરિફિકેશન
\item
  ડિજિટલ સિગ્નેચર અને સર્ટિફિકેટ
\item
  ડેટા ઇન્ટેગ્રિટી ચેકિંગ
\end{itemize}

\end{solutionbox}
\begin{mnemonicbox}
``હેશ હંમેશા સમાન આઉટપુટ આપે''

\end{mnemonicbox}
\subsection*{પ્રશ્ન 2(અ) [3
ગુણ]}\label{uxaaauxab0uxab6uxaa8-2uxa85-3-uxa97uxaa3}

\textbf{ફાયરવોલ શું છે? તેના પ્રકારોની યાદી આપો.}

\begin{solutionbox}

\textbf{ફાયરવોલ વ્યાખ્યા:} નેટવર્ક સિક્યોરિટી ડિવાઇસ જે સુરક્ષા નિયમોના આધારે
આવતા-જતા ટ્રાફિકને મોનિટર અને કંટ્રોલ કરે છે.

\begin{figure}
\centering
\pandocbounded{\includesvg[keepaspectratio]{diagrams/firewall-architecture.svg}}
\caption{Firewall Architecture}
\end{figure}

\textbf{ફાયરવોલના પ્રકારો:}

{\def\LTcaptype{none} % do not increment counter
\begin{longtable}[]{@{}lll@{}}
\toprule\noalign{}
પ્રકાર & ફંક્શન & સ્તર \\
\midrule\noalign{}
\endhead
\bottomrule\noalign{}
\endlastfoot
\textbf{પેકેટ ફિલ્ટર} & પેકેટ હેડર તપાસે & નેટવર્ક લેયર \\
\textbf{સ્ટેટફુલ} & કનેક્શન સ્ટેટ ટ્રેક કરે & ટ્રાન્સપોર્ટ લેયર \\
\textbf{એપ્લિકેશન પ્રોક્સી} & એપ્લિકેશન ડેટા તપાસે & એપ્લિકેશન લેયર \\
\textbf{પર્સનલ ફાયરવોલ} & વ્યક્તિગત ડિવાઇસ સુરક્ષા & હોસ્ટ-બેસ્ડ \\
\end{longtable}
}

\begin{itemize}
\tightlist
\item
  \textbf{હાર્ડવેર ફાયરવોલ}: સમર્પિત નેટવર્ક ઉપકરણ
\item
  \textbf{સોફ્ટવેર ફાયરવોલ}: વ્યક્તિગત કમ્પ્યુટર પર ઇન્સ્ટોલ
\item
  \textbf{ક્લાઉડ ફાયરવોલ}: સેવા તરીકે પૂરો પાડવામાં આવે (FWaaS)
\end{itemize}

\end{solutionbox}
\begin{mnemonicbox}
``ફાયરવોલ હંમેશા નેટવર્કનું રક્ષણ કરે''

\end{mnemonicbox}
\subsection*{પ્રશ્ન 2(બ) [4
ગુણ]}\label{uxaaauxab0uxab6uxaa8-2uxaac-4-uxa97uxaa3}

\textbf{વ્યાખ્યાયિત કરો: HTTPS અને HTTPS ના કાર્યનું વર્ણન કરો.}

\begin{solutionbox}

\textbf{HTTPS વ્યાખ્યા:} Hypertext Transfer Protocol Secure - SSL/TLS
એન્ક્રિપ્શન પર HTTP.

\begin{figure}
\centering
\pandocbounded{\includesvg[keepaspectratio]{diagrams/https-process.svg}}
\caption{HTTPS Process}
\end{figure}

\textbf{HTTPS કાર્ય પ્રક્રિયા:}

\begin{verbatim}
sequenceDiagram
    participant C as ક્લાઇન્ટ
    participant S as સર્વર
    C{-S: 1. HTTPS વિનંતી}
    S{-C: 2. SSL સર્ટિફિકેટ}
    C{-S: 3. એન્ક્રિપ્ટેડ સેશન કી}
    S{-C: 4. એન્ક્રિપ્ટેડ રિસ્પોન્સ}
    Note over C,S: સુરક્ષિત કમ્યુનિકેશન સ્થાપિત
\end{verbatim}

\textbf{HTTPS ઘટકો:}

\begin{itemize}
\tightlist
\item
  \textbf{પોર્ટ 443}: સ્ટાન્ડર્ડ HTTPS પોર્ટ
\item
  \textbf{SSL/TLS}: એન્ક્રિપ્શન પ્રોટોકોલ
\item
  \textbf{ડિજિટલ સર્ટિફિકેટ}: સર્વર ઓથેન્ટિકેશન
\item
  \textbf{સિમેટ્રિક એન્ક્રિપ્શન}: ડેટા ટ્રાન્સમિશન
\end{itemize}

\textbf{ફાયદાઓ:}

\begin{itemize}
\tightlist
\item
  ટ્રાન્સમિશન દરમિયાન ડેટા એન્ક્રિપ્શન
\item
  સર્વર ઓથેન્ટિકેશન વેરિફિકેશન
\item
  ડેટા ઇન્ટેગ્રિટી પ્રોટેક્શન
\item
  SEO રેંકિંગ સુધારો
\end{itemize}

\end{solutionbox}
\begin{mnemonicbox}
``HTTPS વેબ ટ્રાફિકને સુરક્ષિત કરે''

\end{mnemonicbox}
\subsection*{પ્રશ્ન 2(ક) [7
ગુણ]}\label{uxaaauxab0uxab6uxaa8-2uxa95-7-uxa97uxaa3}

\textbf{વિવિધ પ્રકારના દુર્ભાવનાપૂર્ણ સોફ્ટવેર અને તેમની અસર સમજાવો.}

\begin{solutionbox}

\textbf{મેલવેર વર્કિંગ્રેડ:}

\begin{figure}
\centering
\pandocbounded{\includesvg[keepaspectratio]{diagrams/malware-classification.svg}}
\caption{Malware Classification}
\end{figure}

{\def\LTcaptype{none} % do not increment counter
\begin{longtable}[]{@{}llll@{}}
\toprule\noalign{}
પ્રકાર & વર્તન & અસર & ઉદાહરણ \\
\midrule\noalign{}
\endhead
\bottomrule\noalign{}
\endlastfoot
\textbf{વાયરસ} & ફાઇલો સાથે જોડાય & ફાઇલ કરપ્શન & બૂટ સેક્ટર વાયરસ \\
\textbf{વોર્મ} & સ્વ-પ્રતિકૃતિ & નેટવર્ક ભીડ & કન્ફિકર વોર્મ \\
\textbf{ટ્રોજન} & છદ્મવેશી મેલવેર & ડેટા ચોરી & બેંકિંગ ટ્રોજન \\
\textbf{રેન્સમવેર} & ફાઇલો એન્ક્રિપ્ટ કરે & ડેટા બંધક & WannaCry \\
\textbf{સ્પાયવેર} & પ્રવૃત્તિ મોનિટર કરે & ગોપનીયતા ભંગ & કીલોગર \\
\textbf{એડવેર} & અનચાહેલી જાહેરાતો & પ્રદર્શન ઘટાડો & પોપ-અપ જાહેરાતો \\
\textbf{રૂટકિટ} & હાજરી છુપાવે & સિસ્ટમ સમાધાન & કર્નલ રૂટકિટ \\
\end{longtable}
}

\textbf{સિસ્ટમ પર અસરો:}

\begin{itemize}
\tightlist
\item
  \textbf{પ્રદર્શન}: ધીમી સિસ્ટમ પ્રતિક્રિયા
\item
  \textbf{ડેટા}: નુકસાન, કરપ્શન અથવા ચોરી
\item
  \textbf{ગોપનીયતા}: અનધિકૃત મોનિટરિંગ
\item
  \textbf{નાણાકીય}: પ્રત્યક્ષ નાણાકીય નુકસાન
\end{itemize}

\textbf{રોકથામના પદ્ધતિઓ:}

\begin{itemize}
\tightlist
\item
  નિયમિત એન્ટિવાયરસ અપડેટ
\item
  સુરક્ષિત બ્રાઉઝિંગ પ્રેક્ટિસ
\item
  ઇમેઇલ એટેચમેન્ટમાં સાવધાની
\item
  સિસ્ટમ સિક્યોરિટી પેચ
\end{itemize}

\end{solutionbox}
\begin{mnemonicbox}
``વાયરસ વોર્મ ટ્રોજન ખરેખર બધા સંસાધનો ચોરે''

\end{mnemonicbox}
\subsection*{પ્રશ્ન 2(અ અથવા) [3
ગુણ]}\label{uxaaauxab0uxab6uxaa8-2uxa85-uxa85uxaa5uxab5-3-uxa97uxaa3}

\textbf{પ્રમાણીકરણ(ઓથેન્ટિકેશન) શું છે? પ્રમાણીકરણ(ઓથેન્ટિકેશન) ની વિવિધ પદ્ધતિઓ
સમજાવો.}

\begin{solutionbox}

\textbf{ઓથેન્ટિકેશન વ્યાખ્યા:} સિસ્ટમ એક્સેસ આપતા પહેલા યુઝર આઇડેન્ટિટી વેરિફાઇ
કરવાની પ્રક્રિયા.

\textbf{ઓથેન્ટિકેશન પદ્ધતિઓ:}

\begin{figure}
\centering
\pandocbounded{\includesvg[keepaspectratio]{diagrams/authentication-methods.svg}}
\caption{Authentication Methods}
\end{figure}

{\def\LTcaptype{none} % do not increment counter
\begin{longtable}[]{@{}lll@{}}
\toprule\noalign{}
પદ્ધતિ & વર્ણન & ઉદાહરણ \\
\midrule\noalign{}
\endhead
\bottomrule\noalign{}
\endlastfoot
\textbf{પાસવર્ડ} & તમે જે જાણો છો & PIN, પાસફ્રેઝ \\
\textbf{બાયોમેટ્રિક} & તમે જે છો & ફિંગરપ્રિન્ટ, આઇરિસ \\
\textbf{ટોકન} & તમારી પાસે જે છે & સ્માર્ટ કાર્ડ, USB કી \\
\end{longtable}
}

\begin{itemize}
\tightlist
\item
  \textbf{સિંગલ-ફેક્ટર}: એક ઓથેન્ટિકેશન પદ્ધતિ વાપરે
\item
  \textbf{મલ્ટિ-ફેક્ટર}: અનેક પદ્ધતિઓ જોડે
\item
  \textbf{ટુ-ફેક્ટર (2FA)}: બરાબર બે ફેક્ટર વાપરે
\end{itemize}

\end{solutionbox}
\begin{mnemonicbox}
``પાસવર્ડ બાયોમેટ્રિક ટોકન ઓથેન્ટિકેશન''

\end{mnemonicbox}
\subsection*{પ્રશ્ન 2(બ અથવા) [4
ગુણ]}\label{uxaaauxab0uxab6uxaa8-2uxaac-uxa85uxaa5uxab5-4-uxa97uxaa3}

\textbf{વ્યાખ્યાયિત કરો: ટ્રોજન્સ, રૂટકિટ, બેકડોર્સ, કીલોગર}

\begin{solutionbox}

\textbf{મેલવેર વ્યાખ્યાઓ:}

{\def\LTcaptype{none} % do not increment counter
\begin{longtable}[]{@{}
  >{\raggedright\arraybackslash}p{(\linewidth - 4\tabcolsep) * \real{0.2273}}
  >{\raggedright\arraybackslash}p{(\linewidth - 4\tabcolsep) * \real{0.3636}}
  >{\raggedright\arraybackslash}p{(\linewidth - 4\tabcolsep) * \real{0.4091}}@{}}
\toprule\noalign{}
\begin{minipage}[b]{\linewidth}\raggedright
શબ્દ
\end{minipage} & \begin{minipage}[b]{\linewidth}\raggedright
વ્યાખ્યા
\end{minipage} & \begin{minipage}[b]{\linewidth}\raggedright
લક્ષણો
\end{minipage} \\
\midrule\noalign{}
\endhead
\bottomrule\noalign{}
\endlastfoot
\textbf{ટ્રોજન્સ} & કાયદેસર સોફ્ટવેરના છદ્મવેશમાં મેલવેર & હાનિકારક દેખાય, છુપાયેલ
પેલોડ \\
\textbf{રૂટકિટ} & મેલવેરની હાજરી છુપાવતો સોફ્ટવેર & ઊંડી સિસ્ટમ એક્સેસ, સ્ટેલ્થ
ઓપરેશન \\
\textbf{બેકડોર્સ} & અનધિકૃત એક્સેસ પદ્ધતિ & સામાન્ય ઓથેન્ટિકેશન બાયપાસ કરે \\
\textbf{કીલોગર} & કીબોર્ડ ઇનપુટ રેકોર્ડ કરે & પાસવર્ડ, સંવેદનશીલ ડેટા કેપ્ચર કરે \\
\end{longtable}
}

\begin{itemize}
\tightlist
\item
  \textbf{ટ્રોજન્સ}: ગ્રીક ટ્રોજન હોર્સ પરથી નામ
\item
  \textbf{રૂટકિટ}: કર્નલ લેવલ પર કામ કરે
\item
  \textbf{બેકડોર્સ}: હાર્ડવેર અથવા સોફ્ટવેર આધારિત હોઈ શકે
\item
  \textbf{કીલોગર}: સોફ્ટવેર અથવા હાર્ડવેર ડિવાઇસ હોઈ શકે
\end{itemize}

\end{solutionbox}
\begin{mnemonicbox}
``ટ્રોજન રૂટ બેકડોર કીલોગ''

\end{mnemonicbox}
\subsection*{પ્રશ્ન 2(ક અથવા) [7
ગુણ]}\label{uxaaauxab0uxab6uxaa8-2uxa95-uxa85uxaa5uxab5-7-uxa97uxaa3}

\textbf{સિક્યોર સોકેટ લેયર (SSL) અને ટ્રાન્સપોર્ટ લેયર સિક્યોરિટી (TLS) પ્રોટોકોલ
સમજાવો.}

\begin{solutionbox}

\textbf{SSL/TLS પ્રોટોકોલ ઉત્ક્રાંતિ:}

\begin{figure}
\centering
\pandocbounded{\includesvg[keepaspectratio]{diagrams/ssl-tls-handshake.svg}}
\caption{SSL/TLS Handshake}
\end{figure}

{\def\LTcaptype{none} % do not increment counter
\begin{longtable}[]{@{}llll@{}}
\toprule\noalign{}
વર્ઝન & વર્ષ & સ્થિતિ & સુરક્ષા સ્તર \\
\midrule\noalign{}
\endhead
\bottomrule\noalign{}
\endlastfoot
\textbf{SSL 2.0} & 1995 & અપ્રચલિત & નબળું \\
\textbf{SSL 3.0} & 1996 & અપ્રચલિત & સંવેદનશીલ \\
\textbf{TLS 1.0} & 1999 & લેગસી & મર્યાદિત \\
\textbf{TLS 1.2} & 2008 & વ્યાપક ઉપયોગ & સારું \\
\textbf{TLS 1.3} & 2018 & વર્તમાન & મજબૂત \\
\end{longtable}
}

\textbf{TLS હેન્ડશેક પ્રક્રિયા:}

\begin{verbatim}
sequenceDiagram
    participant C as ક્લાઇન્ટ
    participant S as સર્વર
    C{-S: ClientHello}
    S{-C: ServerHello + સર્ટિફિકેટ}
    C{-S: કી એક્સચેન્જ}
    S{-C: પૂર્ણ}
    Note over C,S: સુરક્ષિત ચેનલ સ્થાપિત
\end{verbatim}

\textbf{મુખ્ય વિશેષતાઓ:}

\begin{itemize}
\tightlist
\item
  \textbf{એન્ક્રિપ્શન}: સિમેટ્રિક અને એસિમેટ્રિક અલ્ગોરિધમ
\item
  \textbf{ઓથેન્ટિકેશન}: સર્વર અને ક્લાયન્ટ વેરિફિકેશન
\item
  \textbf{ઇન્ટેગ્રિટી}: મેસેજ ઓથેન્ટિકેશન કોડ
\item
  \textbf{ફોરવર્ડ સિક્રેસી}: સેશન કી પ્રોટેક્શન
\end{itemize}

\textbf{એપ્લિકેશન:}

\begin{itemize}
\tightlist
\item
  HTTPS વેબ બ્રાઉઝિંગ
\item
  ઇમેઇલ સિક્યોરિટી (SMTPS)
\item
  VPN કનેક્શન
\item
  સુરક્ષિત ફાઇલ ટ્રાન્સફર
\end{itemize}

\end{solutionbox}
\begin{mnemonicbox}
``TLS બધા નેટવર્ક ટ્રાફિકને એન્ક્રિપ્ટ કરે''

\end{mnemonicbox}
\subsection*{પ્રશ્ન 3(અ) [3
ગુણ]}\label{uxaaauxab0uxab6uxaa8-3uxa85-3-uxa97uxaa3}

\textbf{સાયબર ક્રાઇમ અને સાયબરક્રિમિનલ ને વિગતવાર સમજાવો.}

\begin{solutionbox}

\textbf{સાયબર ક્રાઇમ વ્યાખ્યા:} કમ્પ્યુટર અથવા ઇન્ટરનેટ નેટવર્ક દ્વારા કરવામાં આવતી
ગુનાહિત પ્રવૃત્તિઓ.

\textbf{ડાયાગ્રામ:}

\begin{figure}
\centering
\pandocbounded{\includesvg[keepaspectratio]{diagrams/cybercrime-overview.svg}}
\caption{Cybercrime Overview}
\end{figure}

\textbf{સાયબરક્રિમિનલ પ્રકારો:}

{\def\LTcaptype{none} % do not increment counter
\begin{longtable}[]{@{}llll@{}}
\toprule\noalign{}
પ્રકાર & પ્રેરણા & કુશળતા & લક્ષ્ય \\
\midrule\noalign{}
\endhead
\bottomrule\noalign{}
\endlastfoot
\textbf{સ્ક્રિપ્ટ કિડીઝ} & મજા/પ્રસિદ્ધિ & ઓછી & અવ્યવસ્થિત \\
\textbf{હેક્ટિવિસ્ટ} & રાજકીય/સામાજિક & મધ્યમ & સંસ્થાઓ \\
\textbf{સાયબરક્રિમિનલ} & નાણાકીય લાભ & ઉચ્ચ & વ્યક્તિઓ/બેંકો \\
\end{longtable}
}

\begin{itemize}
\tightlist
\item
  \textbf{સાયબર ક્રાઇમ}: ડિજિટલ ટેકનોલોજીનો ઉપયોગ કરીને ગેરકાયદેસર પ્રવૃત્તિઓ
\item
  \textbf{સાયબરક્રિમિનલ}: સાયબર ક્રાઇમ કરનાર વ્યક્તિ
\item
  \textbf{અસર}: નાણાકીય નુકસાન, ગોપનીયતા ભંગ, સિસ્ટમ નુકસાન
\end{itemize}

\end{solutionbox}
\begin{mnemonicbox}
``સાયબર ક્રિમિનલો અરાજકતા સર્જે છે''

\end{mnemonicbox}
\subsection*{પ્રશ્ન 3(બ) [4
ગુણ]}\label{uxaaauxab0uxab6uxaa8-3uxaac-4-uxa97uxaa3}

\textbf{સાયબર સ્ટોકિંગ અને સાયબર બુલીંગ નું વર્ણન કરો.}

\begin{solutionbox}

\textbf{ડિજિટલ પજવણી સરખામણી:}

\begin{figure}
\centering
\pandocbounded{\includesvg[keepaspectratio]{diagrams/cyber-stalking-vs-bullying.svg}}
\caption{Cyber Stalking vs Cyber Bullying}
\end{figure}

{\def\LTcaptype{none} % do not increment counter
\begin{longtable}[]{@{}lll@{}}
\toprule\noalign{}
પાસું & સાયબર સ્ટોકિંગ & સાયબર બુલીંગ \\
\midrule\noalign{}
\endhead
\bottomrule\noalign{}
\endlastfoot
\textbf{લક્ષ્ય} & વિશિષ્ટ વ્યક્તિ & મોટેભાગે નાબાલિગો \\
\textbf{અવધિ} & સતત, લાંબા ગાળાની & એપિસોડિક હોઈ શકે \\
\textbf{હેતુ} & ભીતિ, નિયંત્રણ & પજવણી, અપમાન \\
\textbf{પ્લેટફોર્મ} & સોશિયલ મીડિયા, ઇમેઇલ & શાળાઓ, ગેમિંગ પ્લેટફોર્મ \\
\end{longtable}
}

\textbf{સાયબર સ્ટોકિંગ લક્ષણો:}

\begin{itemize}
\tightlist
\item
  સતત અનચાહેલ સંપર્ક
\item
  પીડિતની ઓનલાઇન પ્રવૃત્તિનું મોનિટરિંગ
\item
  ધમકીભર્યા સંદેશાઓ અથવા વર્તન
\item
  ઓળખની ચોરી અથવા ઢોંગ
\end{itemize}

\textbf{સાયબર બુલીંગ સ્વરૂપો:}

\begin{itemize}
\tightlist
\item
  ઓનલાઇન જાહેર અપમાન
\item
  ડિજિટલ જૂથોમાંથી બાકાત
\item
  ખોટી માહિતી ફેલાવવી
\item
  સંમતિ વિના ખાનગી સામગ્રી શેર કરવી
\end{itemize}

\textbf{રોકથામના પગલાં:}

\begin{itemize}
\tightlist
\item
  સોશિયલ મીડિયા પર ગોપનીયતા સેટિંગ્સ
\item
  પ્લેટફોર્મને પજવણીની જાણ કરવી
\item
  જરૂર પડે ત્યારે કાયદેસીની કાર્યવાહી
\item
  ડિજિટલ સાક્ષરતા શિક્ષણ
\end{itemize}

\end{solutionbox}
\begin{mnemonicbox}
``બુલીંગ બંધ કરો, સ્ટોકિંગની જાણ કરો''

\end{mnemonicbox}
\subsection*{પ્રશ્ન 3(ક) [7
ગુણ]}\label{uxaaauxab0uxab6uxaa8-3uxa95-7-uxa97uxaa3}

\textbf{સાયબર ક્રાઇમમાં પ્રોપર્ટી બેઇઝ્ડ ક્લાસિફિકેશન સમજાવો.}

\begin{solutionbox}

\textbf{પ્રોપર્ટી-આધારિત સાયબર ક્રાઇમ શ્રેણીઓ:}

\begin{figure}
\centering
\pandocbounded{\includesvg[keepaspectratio]{diagrams/cybercrime-classification.svg}}
\caption{Cybercrime Classification}
\end{figure}

{\def\LTcaptype{none} % do not increment counter
\begin{longtable}[]{@{}
  >{\raggedright\arraybackslash}p{(\linewidth - 6\tabcolsep) * \real{0.1944}}
  >{\raggedright\arraybackslash}p{(\linewidth - 6\tabcolsep) * \real{0.3611}}
  >{\raggedright\arraybackslash}p{(\linewidth - 6\tabcolsep) * \real{0.1944}}
  >{\raggedright\arraybackslash}p{(\linewidth - 6\tabcolsep) * \real{0.2500}}@{}}
\toprule\noalign{}
\begin{minipage}[b]{\linewidth}\raggedright
શ્રેણી
\end{minipage} & \begin{minipage}[b]{\linewidth}\raggedright
ક્રાઇમ પ્રકાર
\end{minipage} & \begin{minipage}[b]{\linewidth}\raggedright
વર્ણન
\end{minipage} & \begin{minipage}[b]{\linewidth}\raggedright
ઉદાહરણ
\end{minipage} \\
\midrule\noalign{}
\endhead
\bottomrule\noalign{}
\endlastfoot
\textbf{બૌદ્ધિક સંપત્તિ} & કોપીરાઇટ ઉલ્લંઘન & કોપીરાઇટ સામગ્રીનો અનધિકૃત ઉપયોગ
& સોફ્ટવેર પાયરેસી \\
\textbf{નાણાકીય સંપત્તિ} & ક્રેડિટ કાર્ડ ફ્રોડ & નાણાકીય માહિતીનો અનધિકૃત ઉપયોગ
& ઓનલાઇન શોપિંગ ફ્રોડ \\
\textbf{ડિજિટલ સંપત્તિ} & ડેટા ચોરી & ડિજિટલ માહિતીની ચોરી & ડેટાબેસ બ્રીચ \\
\textbf{વર્ચ્યુઅલ સંપત્તિ} & ગેમિંગ એસેટ ચોરી & વર્ચ્યુઅલ વસ્તુઓની ચોરી & ઓનલાઇન ગેમ
કરન્સી ચોરી \\
\end{longtable}
}

\textbf{ડાયાગ્રામ:}

\begin{figure}
\centering
\pandocbounded{\includesvg[keepaspectratio]{diagrams/property-based-cybercrime.svg}}
\caption{Property-Based Cybercrime Classification}
\end{figure}

\textbf{કાયદેસરના પાસાઓ:}

\begin{itemize}
\tightlist
\item
  \textbf{કોપીરાઇટ કાયદાઓ}: સર્જનાત્મક કાર્યોનું રક્ષણ
\item
  \textbf{ટ્રેડમાર્ક કાયદાઓ}: બ્રાન્ડ ઓળખનું રક્ષણ
\item
  \textbf{પેટન્ટ કાયદાઓ}: આવિષ્કારોનું રક્ષણ
\item
  \textbf{ટ્રેડ સિક્રેટ કાયદાઓ}: ગોપનીય માહિતીનું રક્ષણ
\end{itemize}

\textbf{અર્થતંત્ર પર અસર:}

\begin{itemize}
\tightlist
\item
  કાયદેસર વ્યવસાયો માટે આવકનું નુકસાન
\item
  નવીનતાની પ્રેરણામાં ઘટાડો
\item
  ગ્રાહક વિશ્વાસનું ધોવાણ
\item
  કાયદેસર અમલીકરણના ખર્ચ
\end{itemize}

\textbf{રોકથામ વ્યૂહરચનાઓ:}

\begin{itemize}
\tightlist
\item
  ડિજિટલ રાઇટ્સ મેનેજમેન્ટ (DRM)
\item
  વોટરમાર્કિંગ અને ટ્રેકિંગ
\item
  કાયદેસર અમલીકરણ મિકેનિઝમ
\item
  જાહેર જાગૃતિ ઝુંબેશ
\end{itemize}

\end{solutionbox}
\begin{mnemonicbox}
``પ્રોપર્ટી પ્રોટેક્શન પાયરેસી અટકાવે''

\end{mnemonicbox}
\subsection*{પ્રશ્ન 3(અ અથવા) [3
ગુણ]}\label{uxaaauxab0uxab6uxaa8-3uxa85-uxa85uxaa5uxab5-3-uxa97uxaa3}

\textbf{ડેટા ડિડલિંગ સમજાવો.}

\begin{solutionbox}

\textbf{ડેટા ડિડલિંગ વ્યાખ્યા:} કમ્પ્યુટર સિસ્ટમમાં ડેટા દાખલ કરતા પહેલા અથવા
દરમિયાન અનધિકૃત ફેરફાર.

\begin{figure}
\centering
\pandocbounded{\includesvg[keepaspectratio]{diagrams/data-diddling-process.svg}}
\caption{Data Diddling Process}
\end{figure}

\textbf{લક્ષણો:}

{\def\LTcaptype{none} % do not increment counter
\begin{longtable}[]{@{}ll@{}}
\toprule\noalign{}
પાસું & વર્ણન \\
\midrule\noalign{}
\endhead
\bottomrule\noalign{}
\endlastfoot
\textbf{પદ્ધતિ} & ડેટા વેલ્યુમાં ફેરફાર \\
\textbf{સમય} & સિસ્ટમ પ્રોસેસિંગ પહેલા \\
\textbf{શોધ} & ઘણીવાર ઓળખવું મુશ્કેલ \\
\end{longtable}
}

\begin{itemize}
\tightlist
\item
  \textbf{ઉદાહરણો}: સેલેરી આંકડાઓમાં ફેરફાર, પરીક્ષાના સ્કોરમાં ફેરફાર
\item
  \textbf{લક્ષ્ય}: એન્ટ્રી પ્રક્રિયા દરમિયાન ઇનપુટ ડેટા
\item
  \textbf{અસર}: નાણાકીય નુકસાન, ખોટા રેકોર્ડ
\end{itemize}

\end{solutionbox}
\begin{mnemonicbox}
``ડેટા ડિડલિંગ ડેટાબેસને નુકસાન પહોંચાડે''

\end{mnemonicbox}
\subsection*{પ્રશ્ન 3(બ અથવા) [4
ગુણ]}\label{uxaaauxab0uxab6uxaa8-3uxaac-uxa85uxaa5uxab5-4-uxa97uxaa3}

\textbf{સાયબર સ્પાઇંગ અને સાયબર ટેરરીઝમ સમજાવો.}

\begin{solutionbox}

\textbf{સાયબર ધમકીઓની સરખામણી:}

\begin{figure}
\centering
\pandocbounded{\includesvg[keepaspectratio]{diagrams/cyber-spying-vs-terrorism.svg}}
\caption{Cyber Spying vs Terrorism}
\end{figure}

{\def\LTcaptype{none} % do not increment counter
\begin{longtable}[]{@{}lll@{}}
\toprule\noalign{}
પાસું & સાયબર સ્પાઇંગ & સાયબર ટેરરીઝમ \\
\midrule\noalign{}
\endhead
\bottomrule\noalign{}
\endlastfoot
\textbf{હેતુ} & માહિતી એકત્રીકરણ & ભય/વિક્ષેપ સર્જવો \\
\textbf{લક્ષ્ય} & સરકાર, કોર્પોરેશન & નિર્ધારક ઇન્ફ્રાસ્ટ્રક્ચર \\
\textbf{પદ્ધતિઓ} & છુપી ઘૂસણખોરી & વિનાશક હુમલાઓ \\
\textbf{અસર} & ગુપ્ત માહિતીનું નુકસાન & જાહેર સુરક્ષા જોખમ \\
\end{longtable}
}

\textbf{સાયબર સ્પાઇંગ પ્રવૃત્તિઓ:}

\begin{itemize}
\tightlist
\item
  કોર્પોરેટ જાસૂસી
\item
  સરકારી દેખરેખ
\item
  ટ્રેડ સિક્રેટ ચોરી
\item
  વ્યક્તિગત માહિતી એકત્રીકરણ
\end{itemize}

\textbf{સાયબર ટેરરીઝમ પદ્ધતિઓ:}

\begin{itemize}
\tightlist
\item
  ઇન્ફ્રાસ્ટ્રક્ચર હુમલાઓ
\item
  મોટા પાયે વિક્ષેપ ઝુંબેશ
\item
  મનોવૈજ્ઞાનિક યુદ્ધ
\item
  આર્થિક નુકસાન
\end{itemize}

\textbf{રોકથામના પગલાં:}

\begin{itemize}
\tightlist
\item
  નેટવર્ક સિક્યોરિટી મોનિટરિંગ
\item
  ઘટના પ્રતિક્રિયા આયોજન
\item
  આંતરરાષ્ટ્રીય સહયોગ
\item
  જાહેર-ખાનગી ભાગીદારી
\end{itemize}

\end{solutionbox}
\begin{mnemonicbox}
``જાસૂસો ચોરે, આતંકવાદીઓ આતંક''

\end{mnemonicbox}
\subsection*{પ્રશ્ન 3(ક અથવા) [7
ગુણ]}\label{uxaaauxab0uxab6uxaa8-3uxa95-uxa85uxaa5uxab5-7-uxa97uxaa3}

\textbf{સાયબર સુરક્ષામાં ડિજિટલ સહીઓ અને ડિજિટલ પ્રમાણપત્રોની ભૂમિકા સમજાવો.}

\begin{solutionbox}

\textbf{ડિજિટલ સુરક્ષા ઘટકો:}

\begin{figure}
\centering
\pandocbounded{\includesvg[keepaspectratio]{diagrams/digital-signatures-certificates.svg}}
\caption{Digital Signatures and Certificates}
\end{figure}

{\def\LTcaptype{none} % do not increment counter
\begin{longtable}[]{@{}
  >{\raggedright\arraybackslash}p{(\linewidth - 6\tabcolsep) * \real{0.2000}}
  >{\raggedright\arraybackslash}p{(\linewidth - 6\tabcolsep) * \real{0.2000}}
  >{\raggedright\arraybackslash}p{(\linewidth - 6\tabcolsep) * \real{0.3200}}
  >{\raggedright\arraybackslash}p{(\linewidth - 6\tabcolsep) * \real{0.2800}}@{}}
\toprule\noalign{}
\begin{minipage}[b]{\linewidth}\raggedright
ઘટક
\end{minipage} & \begin{minipage}[b]{\linewidth}\raggedright
હેતુ
\end{minipage} & \begin{minipage}[b]{\linewidth}\raggedright
ફંક્શન
\end{minipage} & \begin{minipage}[b]{\linewidth}\raggedright
ફાયદો
\end{minipage} \\
\midrule\noalign{}
\endhead
\bottomrule\noalign{}
\endlastfoot
\textbf{ડિજિટલ સિગ્નેચર} & ઓથેન્ટિકેશન & મોકલનારની ઓળખ સાબિત કરે &
નોન-રિપ્યુડિએશન \\
\textbf{ડિજિટલ સર્ટિફિકેટ} & વેરિફિકેશન & પબ્લિક કીની માન્યતા & વિશ્વાસ
સ્થાપના \\
\end{longtable}
}

\textbf{ડિજિટલ સિગ્નેચર પ્રક્રિયા:}

\begin{center}
\textbf{Mermaid Diagram (Code)}
\begin{verbatim}
{Shaded}
{Highlighting}[]
graph LR
    A[ડોક્યુમેન્ટ] {-{-}{} B[હેશ ફંક્શન]}
    B {-{-}{} C[મેસેજ ડાઇજેસ્ટ]}
    C {-{-}{} D[પ્રાઇવેટ કી એન્ક્રિપ્શન]}
    D {-{-}{} E[ડિજિટલ સિગ્નેચર]}
    E {-{-}{} F[પબ્લિક કી સાથે વેરિફિકેશન]}
{Highlighting}
{Shaded}
\end{verbatim}
\end{center}

\textbf{ડિજિટલ સર્ટિફિકેટ ઘટકો:}

\begin{itemize}
\tightlist
\item
  \textbf{વિષય માહિતી}: સર્ટિફિકેટ માલિકની વિગતો
\item
  \textbf{પબ્લિક કી}: એન્ક્રિપ્શન/વેરિફિકેશન માટે
\item
  \textbf{ડિજિટલ સિગ્નેચર}: CA ની સહી
\item
  \textbf{માન્યતા અવધિ}: સર્ટિફિકેટની સમાપ્તિ તારીખ
\end{itemize}

\textbf{સર્ટિફિકેટ ઓથોરિટી (CA) ભૂમિકા:}

\begin{itemize}
\tightlist
\item
  ડિજિટલ સર્ટિફિકેટ જારી કરે
\item
  જારી કરતા પહેલા ઓળખ ચકાસે
\item
  સર્ટિફિકેટ રદ કરવાની યાદીઓ જાળવે
\item
  વિશ્વાસ ઇન્ફ્રાસ્ટ્રક્ચર પૂરું પાડે
\end{itemize}

\textbf{સાયબર સિક્યોરિટીમાં એપ્લિકેશન:}

\begin{itemize}
\tightlist
\item
  ઇમેઇલ સિક્યોરિટી (S/MIME)
\item
  સોફ્ટવેર માટે કોડ સાઇનિંગ
\item
  વેબસાઇટો માટે SSL/TLS સર્ટિફિકેટ
\item
  ડોક્યુમેન્ટ ઓથેન્ટિકેશન
\end{itemize}

\textbf{સુરક્ષા ફાયદાઓ:}

\begin{itemize}
\tightlist
\item
  \textbf{ઓથેન્ટિકેશન}: મોકલનારની ઓળખ ચકાસે
\item
  \textbf{ઇન્ટેગ્રિટી}: ડેટામાં ફેરફાર થયો નથી તેની ખાતરી
\item
  \textbf{નોન-રિપ્યુડિએશન}: ક્રિયાઓનો ઇનકાર અટકાવે
\item
  \textbf{ગોપનીયતા}: સુરક્ષિત કમ્યુનિકેશન સક્ષમ કરે
\end{itemize}

\end{solutionbox}
\begin{mnemonicbox}
``ડિજિટલ સિગ્નેચર ડોક્યુમેન્ટને સુરક્ષિત રીતે પ્રમાણિત કરે''

\end{mnemonicbox}
\subsection*{પ્રશ્ન 4(અ) [3
ગુણ]}\label{uxaaauxab0uxab6uxaa8-4uxa85-3-uxa97uxaa3}

\textbf{હેકિંગ શું છે? હેકર્સના પ્રકારોની યાદી બનાવો.}

\begin{solutionbox}

\textbf{હેકિંગ વ્યાખ્યા:} નબળાઈઓનો ફાયદો ઉઠાવવા માટે કમ્પ્યુટર સિસ્ટમ અથવા
નેટવર્કમાં અનધિકૃત એક્સેસ.

\textbf{હેકર વર્ગીકરણ:}

\begin{figure}
\centering
\pandocbounded{\includesvg[keepaspectratio]{diagrams/hacker-types.svg}}
\caption{Hacker Types}
\end{figure}

{\def\LTcaptype{none} % do not increment counter
\begin{longtable}[]{@{}lll@{}}
\toprule\noalign{}
પ્રકાર & હેતુ & કાયદેસર સ્થિતિ \\
\midrule\noalign{}
\endhead
\bottomrule\noalign{}
\endlastfoot
\textbf{વ્હાઇટ હેટ} & સુરક્ષા સુધારણા & કાયદેસર \\
\textbf{બ્લેક હેટ} & દુર્ભાવનાપૂર્ણ પ્રવૃત્તિઓ & ગેરકાયદેસર \\
\textbf{ગ્રે હેટ} & મિશ્ર પ્રેરણા & શંકાસ્પદ \\
\end{longtable}
}

\begin{itemize}
\tightlist
\item
  \textbf{વ્હાઇટ હેટ}: નૈતિક હેકર, સુરક્ષા સંશોધકો
\item
  \textbf{બ્લેક હેટ}: સાયબરક્રિમિનલ, દુર્ભાવનાપૂર્ણ હેતુ
\item
  \textbf{ગ્રે હેટ}: કેટલીકવાર કાયદેસર, કેટલીકવાર નહીં
\end{itemize}

\end{solutionbox}
\begin{mnemonicbox}
``સફેદ સારું, કાળું ખરાબ, ગ્રે શંકાસ્પદ''

\end{mnemonicbox}
\subsection*{પ્રશ્ન 4(બ) [4
ગુણ]}\label{uxaaauxab0uxab6uxaa8-4uxaac-4-uxa97uxaa3}

\textbf{હેકિંગની વલ્નરેબિલિટી અને 0-દિવસની પરિભાષા સમજાવો.}

\begin{solutionbox}

\textbf{સુરક્ષા પરિભાષા:}

\begin{figure}
\centering
\pandocbounded{\includesvg[keepaspectratio]{diagrams/vulnerability-vs-0day.svg}}
\caption{Vulnerability vs 0-Day}
\end{figure}

{\def\LTcaptype{none} % do not increment counter
\begin{longtable}[]{@{}llll@{}}
\toprule\noalign{}
શબ્દ & વ્યાખ્યા & જોખમ સ્તર & ઉદાહરણ \\
\midrule\noalign{}
\endhead
\bottomrule\noalign{}
\endlastfoot
\textbf{વલ્નરેબિલિટી} & સિસ્ટમની નબળાઈ & વિવિધ & અનપેચ્ડ સોફ્ટવેર \\
\textbf{0-દિવસ} & અજાણી નબળાઈ & ગંભીર & અશોધાયેલી ખામી \\
\end{longtable}
}

\textbf{વલ્નરેબિલિટી લક્ષણો:}

\begin{itemize}
\tightlist
\item
  \textbf{શોધ}: સુરક્ષા પરીક્ષણ દ્વારા મળે
\item
  \textbf{જાહેરાત}: વેન્ડરને જવાબદાર રિપોર્ટિંગ
\item
  \textbf{પેચિંગ}: વેન્ડર સુરક્ષા અપડેટ પૂરું પાડે
\item
  \textbf{વિંડો}: શોધ અને પેચ વચ્ચેનો સમય
\end{itemize}

\textbf{0-દિવસ હુમલો પ્રક્રિયા:}

\begin{itemize}
\tightlist
\item
  હેકર અજાણી નબળાઈ શોધે
\item
  વેન્ડરની જાણકારી પહેલા ખામીનો ફાયદો ઉઠાવે
\item
  કોઈ ઉપલબ્ધ પેચ અથવા સંરક્ષણ નથી
\item
  આશ્ચર્યના કારણે ઉચ્ચ સફળતા દર
\end{itemize}

\textbf{સંરક્ષણ વ્યૂહરચના:}

\begin{itemize}
\tightlist
\item
  નિયમિત સુરક્ષા અપડેટ
\item
  ઇન્ટ્રુઝન ડિટેક્શન સિસ્ટમ
\item
  વર્તણૂકીય વિશ્લેષણ સાધનો
\item
  ઝીરો-ટ્રસ્ટ સુરક્ષા મોડેલ
\end{itemize}

\end{solutionbox}
\begin{mnemonicbox}
``નબળાઈઓને પેચની જરૂર, ઝીરો-ડેને સાવચેતીની જરૂર''

\end{mnemonicbox}
\subsection*{પ્રશ્ન 4(ક) [7
ગુણ]}\label{uxaaauxab0uxab6uxaa8-4uxa95-7-uxa97uxaa3}

\textbf{હેકિંગના પાંચ સ્ટેપ્સ સમજાવો.}

\begin{solutionbox}

\textbf{હેકિંગ પદ્ધતિ:}

\begin{figure}
\centering
\pandocbounded{\includesvg[keepaspectratio]{diagrams/hacking-steps.svg}}
\caption{Hacking Steps}
\end{figure}

\begin{center}
\textbf{Mermaid Diagram (Code)}
\begin{verbatim}
{Shaded}
{Highlighting}[]
graph LR
    A[રિકોનેસન્સ] {-{-}{} B[સ્કેનિંગ]}
    B {-{-}{} C[એક્સેસ મેળવવી]}
    C {-{-}{} D[એક્સેસ જાળવી રાખવી]}
    D {-{-}{} E[ટ્રેક્સ ઢાંકવા]}
{Highlighting}
{Shaded}
\end{verbatim}
\end{center}

\textbf{વિગતવાર પગલાંઓ:}

{\def\LTcaptype{none} % do not increment counter
\begin{longtable}[]{@{}
  >{\raggedright\arraybackslash}p{(\linewidth - 6\tabcolsep) * \real{0.1842}}
  >{\raggedright\arraybackslash}p{(\linewidth - 6\tabcolsep) * \real{0.1842}}
  >{\raggedright\arraybackslash}p{(\linewidth - 6\tabcolsep) * \real{0.4211}}
  >{\raggedright\arraybackslash}p{(\linewidth - 6\tabcolsep) * \real{0.2105}}@{}}
\toprule\noalign{}
\begin{minipage}[b]{\linewidth}\raggedright
પગલું
\end{minipage} & \begin{minipage}[b]{\linewidth}\raggedright
વર્ણન
\end{minipage} & \begin{minipage}[b]{\linewidth}\raggedright
સાધનો/પદ્ધતિઓ
\end{minipage} & \begin{minipage}[b]{\linewidth}\raggedright
ઉદ્દેશ્ય
\end{minipage} \\
\midrule\noalign{}
\endhead
\bottomrule\noalign{}
\endlastfoot
\textbf{રિકોનેસન્સ} & માહિતી એકત્રીકરણ & Google dorking, સોશિયલ મીડિયા &
લક્ષ્ય પ્રોફાઇલિંગ \\
\textbf{સ્કેનિંગ} & સિસ્ટમ ગણતરી & Nmap, Nessus & નબળાઈ ઓળખ \\
\textbf{એક્સેસ મેળવવી} & નબળાઈઓનો ફાયદો & Metasploit, કસ્ટમ એક્સપ્લોઇટ & સિસ્ટમ
સમાધાન \\
\textbf{એક્સેસ જાળવી રાખવી} & સતત હાજરી & બેકડોર, રૂટકિટ & લાંબા ગાળાનું
નિયંત્રણ \\
\textbf{ટ્રેક્સ ઢાંકવા} & પુરાવા દૂર કરવા & લોગ સફાઇ, ફાઇલ કાઢવી & શોધ
ટાળવી \\
\end{longtable}
}

\textbf{માહિતી એકત્રીકરણ પ્રકારો:}

\begin{itemize}
\tightlist
\item
  \textbf{પેસિવ}: લક્ષ્ય સાથે સીધો સંપર્ક નહીં
\item
  \textbf{એક્ટિવ}: લક્ષ્ય સિસ્ટમ સાથે સીધી ક્રિયાપ્રતિક્રિયા
\end{itemize}

\textbf{સ્કેનિંગ તકનીકો:}

\begin{itemize}
\tightlist
\item
  ખુલ્લી સેવાઓ માટે પોર્ટ સ્કેનિંગ
\item
  નબળાઈઓ માટે વલ્નરેબિલિટી સ્કેનિંગ
\item
  ટોપોલોજી માટે નેટવર્ક મેપિંગ
\end{itemize}

\textbf{એક્સેસ પદ્ધતિઓ:}

\begin{itemize}
\tightlist
\item
  પાસવર્ડ હુમલાઓ (બ્રુટ ફોર્સ, ડિક્શનેરી)
\item
  નબળાઈઓનો ફાયદો ઉઠાવવો
\item
  સોશિયલ એન્જિનિયરિંગ
\item
  ભૌતિક એક્સેસ
\end{itemize}

\textbf{સ્થાયિત્વ મિકેનિઝમ:}

\begin{itemize}
\tightlist
\item
  બેકડોર ઇન્સ્ટોલ કરવા
\item
  યુઝર એકાઉન્ટ બનાવવા
\item
  ટાસ્ક શેડ્યુલ કરવા
\item
  રજિસ્ટ્રી ફેરફારો
\end{itemize}

\textbf{ટ્રેક કવરિંગ પદ્ધતિઓ:}

\begin{itemize}
\tightlist
\item
  સિસ્ટમ લોગ સાફ કરવા
\item
  કામચલાઉ ફાઇલો કાઢવી
\item
  ટાઇમસ્ટેમ્પ ફેરવવા
\item
  એન્ક્રિપ્શનનો ઉપયોગ
\end{itemize}

\end{solutionbox}
\begin{mnemonicbox}
``રિકોનેસન્સ સ્કેન્સ એક્સેસ જનરેટ કરે, કવરેજ જાળવે''

\end{mnemonicbox}
\subsection*{પ્રશ્ન 4(અ અથવા) [3
ગુણ]}\label{uxaaauxab0uxab6uxaa8-4uxa85-uxa85uxaa5uxab5-3-uxa97uxaa3}

\textbf{કાલી લિનક્સના કોઈપણ ત્રણ બેઝિક કમાન્ડ યોગ્ય ઉદાહરણ સાથે સમજાવો.}

\begin{solutionbox}

\textbf{અત્યાવશ્યક કાલી લિનક્સ કમાન્ડ્સ:}

\begin{figure}
\centering
\pandocbounded{\includesvg[keepaspectratio]{diagrams/kali-linux-commands-detailed.svg}}
\caption{Kali Linux Commands}
\end{figure}

{\def\LTcaptype{none} % do not increment counter
\begin{longtable}[]{@{}
  >{\raggedright\arraybackslash}p{(\linewidth - 4\tabcolsep) * \real{0.3462}}
  >{\raggedright\arraybackslash}p{(\linewidth - 4\tabcolsep) * \real{0.3077}}
  >{\raggedright\arraybackslash}p{(\linewidth - 4\tabcolsep) * \real{0.3462}}@{}}
\toprule\noalign{}
\begin{minipage}[b]{\linewidth}\raggedright
કમાન્ડ
\end{minipage} & \begin{minipage}[b]{\linewidth}\raggedright
ફંક્શન
\end{minipage} & \begin{minipage}[b]{\linewidth}\raggedright
ઉદાહરણ
\end{minipage} \\
\midrule\noalign{}
\endhead
\bottomrule\noalign{}
\endlastfoot
\textbf{nmap} & નેટવર્ક સ્કેનિંગ & \texttt{nmap\ -sS\ 192.168.1.1} \\
\textbf{netcat} & નેટવર્ક કમ્યુનિકેશન & \texttt{nc\ -l\ -p\ 1234} \\
\textbf{hydra} & પાસવર્ડ ક્રેકિંગ &
\texttt{hydra\ -l\ admin\ -P\ passwords.txt\ ssh://target} \\
\end{longtable}
}

\begin{itemize}
\tightlist
\item
  \textbf{Nmap}: નેટવર્ક પર હોસ્ટ અને સેવાઓ શોધે છે
\item
  \textbf{Netcat}: ડેટા ટ્રાન્સફર માટે નેટવર્ક કનેક્શન બનાવે છે
\item
  \textbf{Hydra}: બ્રુટ-ફોર્સ પાસવર્ડ હુમલાઓ કરે છે
\end{itemize}

\end{solutionbox}
\begin{mnemonicbox}
``નેટવર્ક મેપ, કનેક્ટ, ક્રેક''

\end{mnemonicbox}
\subsection*{પ્રશ્ન 4(બ) [4
ગુણ]}\label{uxaaauxab0uxab6uxaa8-4uxaac-4-uxa97uxaa3-1}

\textbf{સેશન હાઇજેકિંગનું વિગતવાર વર્ણન કરો.}

\begin{solutionbox}

\textbf{સેશન હાઇજેકિંગ ઓવરવ્યુ:} હુમલાખોર કાયદેસર યુઝરના સેશનને કબજે કરે છે તે હુમલો.

\begin{figure}
\centering
\pandocbounded{\includesvg[keepaspectratio]{diagrams/session-hijacking-process.svg}}
\caption{Session Hijacking Process}
\end{figure}

\textbf{સેશન હાઇજેકિંગના પ્રકારો:}

{\def\LTcaptype{none} % do not increment counter
\begin{longtable}[]{@{}lll@{}}
\toprule\noalign{}
પ્રકાર & પદ્ધતિ & રોકથામ \\
\midrule\noalign{}
\endhead
\bottomrule\noalign{}
\endlastfoot
\textbf{એક્ટિવ} & સેશન કબજે કરે & મજબૂત સેશન મેનેજમેન્ટ \\
\textbf{પેસિવ} & સેશન મોનિટર કરે & એન્ક્રિપ્શન (HTTPS) \\
\textbf{નેટવર્ક-લેવલ} & TCP હાઇજેકિંગ & સુરક્ષિત પ્રોટોકોલ \\
\textbf{એપ્લિકેશન-લેવલ} & કુકી ચોરી & સુરક્ષિત કુકી એટ્રિબ્યુટ \\
\end{longtable}
}

\textbf{હુમલાની પ્રક્રિયા:}

\begin{enumerate}
\tightlist
\item
  નેટવર્ક ટ્રાફિક મોનિટર કરવું
\item
  સેશન ઓળખકર્તાઓ કેપ્ચર કરવા
\item
  સેશન ટોકન્સ રિપ્લે કરવા
\item
  યુઝર એકાઉન્ટ એક્સેસ કરવું
\end{enumerate}

\textbf{રોકથામના પગલાં:}

\begin{itemize}
\tightlist
\item
  બધા કમ્યુનિકેશન માટે HTTPS નો ઉપયોગ
\item
  સુરક્ષિત સેશન મેનેજમેન્ટ અમલીકરણ
\item
  સુરક્ષિત કુકી એટ્રિબ્યુટ સેટ કરવા
\item
  શંકાસ્પદ પ્રવૃત્તિ માટે મોનિટરિંગ
\end{itemize}

\end{solutionbox}
\begin{mnemonicbox}
``સેશન હાઇજેકને સુરક્ષિત હેન્ડલિંગની જરૂર''

\end{mnemonicbox}
\subsection*{પ્રશ્ન 4(ક અથવા) [7
ગુણ]}\label{uxaaauxab0uxab6uxaa8-4uxa95-uxa85uxaa5uxab5-7-uxa97uxaa3}

\textbf{વર્ચ્યુઅલ પ્રાઇવેટ નેટવર્ક્સ (VPNs) જાહેર નેટવર્ક્સ પર કેવી રીતે સુરક્ષિત,
એન્ક્રિપ્ટેડ કનેક્શન બનાવે છે તે સમજાવો.}

\begin{solutionbox}

\textbf{VPN આર્કિટેક્ચર:}

\begin{figure}
\centering
\pandocbounded{\includesvg[keepaspectratio]{diagrams/vpn-architecture.svg}}
\caption{VPN Architecture}
\end{figure}

\begin{center}
\textbf{Mermaid Diagram (Code)}
\begin{verbatim}
{Shaded}
{Highlighting}[]
graph LR
    A[યુઝર ડિવાઇસ] {-{-}{}|એન્ક્રિપ્ટેડ ટનલ| B[VPN સર્વર]}
    B {-{-}{} C[ઇન્ટરનેટ]}
    C {-{-}{} D[ડેસ્ટિનેશન સર્વર]}
    E[ISP] {-.{-}{}|ટ્રાફિક જોઈ શકતું નથી| A}
{Highlighting}
{Shaded}
\end{verbatim}
\end{center}

\textbf{VPN ઘટકો:}

{\def\LTcaptype{none} % do not increment counter
\begin{longtable}[]{@{}lll@{}}
\toprule\noalign{}
ઘટક & ફંક્શન & ફાયદો \\
\midrule\noalign{}
\endhead
\bottomrule\noalign{}
\endlastfoot
\textbf{ટનલિંગ} & સુરક્ષિત પાથવે બનાવે & ડેટા પ્રોટેક્શન \\
\textbf{એન્ક્રિપ્શન} & ડેટાને ઝીણવટથી બદલે & ગોપનીયતા \\
\textbf{ઓથેન્ટિકેશન} & ઓળખ ચકાસે & એક્સેસ કંટ્રોલ \\
\textbf{IP માસ્કિંગ} & વાસ્તવિક IP છુપાવે & અનામત્વ \\
\end{longtable}
}

\textbf{VPN પ્રોટોકોલ:}

{\def\LTcaptype{none} % do not increment counter
\begin{longtable}[]{@{}llll@{}}
\toprule\noalign{}
પ્રોટોકોલ & સુરક્ષા સ્તર & ઝડપ & ઉપયોગ કેસ \\
\midrule\noalign{}
\endhead
\bottomrule\noalign{}
\endlastfoot
\textbf{OpenVPN} & ઉચ્ચ & સારી & સામાન્ય હેતુ \\
\textbf{IPSec} & અત્યંત ઉચ્ચ & મધ્યમ & એન્ટરપ્રાઇઝ \\
\textbf{WireGuard} & ઉચ્ચ & ઉત્કૃષ્ટ & આધુનિક સોલ્યુશન \\
\textbf{PPTP} & ઓછું & ઝડપી & લેગસી (અપ્રચલિત) \\
\end{longtable}
}

\textbf{VPN કાર્ય પ્રક્રિયા:}

\begin{enumerate}
\tightlist
\item
  \textbf{કનેક્શન}: ક્લાઇન્ટ VPN સર્વર સાથે જોડાય
\item
  \textbf{ઓથેન્ટિકેશન}: યુઝર ક્રેડેન્શિયલ ચકાસાય
\item
  \textbf{ટનલ ક્રિએશન}: એન્ક્રિપ્ટેડ પાથવે સ્થાપિત થાય
\item
  \textbf{ડેટા એન્ક્રિપ્શન}: બધો ટ્રાફિક એન્ક્રિપ્ટ થાય
\item
  \textbf{રાઉટિંગ}: ટ્રાફિક VPN સર્વર દ્વારા રાઉટ થાય
\item
  \textbf{ડિક્રિપ્શન}: ગંતવ્ય પર ડેટા ડિક્રિપ્ટ થાય
\end{enumerate}

\textbf{સુરક્ષા ફાયદાઓ:}

\begin{itemize}
\tightlist
\item
  \textbf{ડેટા પ્રોટેક્શન}: એન્ક્રિપ્શન ઇવ્સડ્રોપિંગ અટકાવે
\item
  \textbf{ગોપનીયતા}: IP એડ્રેસ માસ્કિંગ
\item
  \textbf{એક્સેસ કંટ્રોલ}: કનેક્શન પહેલા ઓથેન્ટિકેટ કરવું
\item
  \textbf{પ્રતિબંધો બાયપાસ}: જીઓ-બ્લોક્ડ કન્ટેન્ટ એક્સેસ કરવું
\end{itemize}

\textbf{વ્યાવસાયિક એપ્લિકેશન:}

\begin{itemize}
\tightlist
\item
  રિમોટ વર્કર એક્સેસ
\item
  સાઇટ-ટુ-સાઇટ કનેક્ટિવિટી
\item
  સુરક્ષિત ક્લાઉડ એક્સેસ
\item
  અનુપાલન આવશ્યકતાઓ
\end{itemize}

\textbf{વ્યક્તિગત ઉપયોગ કેસ:}

\begin{itemize}
\tightlist
\item
  પબ્લિક વાઇ-ફાઇ પ્રોટેક્શન
\item
  ગોપનીયતા વૃદ્ધિ
\item
  કન્ટેન્ટ એક્સેસ
\item
  લોકેશન ગોપનીયતા
\end{itemize}

\end{solutionbox}
\begin{mnemonicbox}
``VPN નેટવર્ક પ્રાઇવસી પ્રદાન કરે''

\end{mnemonicbox}
\subsection*{પ્રશ્ન 5(અ) [3
ગુણ]}\label{uxaaauxab0uxab6uxaa8-5uxa85-3-uxa97uxaa3}

\textbf{નેટવર્ક ફોરેન્સિક્સ સમજાવો.}

\begin{solutionbox}

\textbf{નેટવર્ક ફોરેન્સિક્સ વ્યાખ્યા:} સુરક્ષા ઘટનાઓ શોધવા અને વિશ્લેષણ કરવા માટે
નેટવર્ક ટ્રાફિકની તપાસ.

\begin{figure}
\centering
\pandocbounded{\includesvg[keepaspectratio]{diagrams/network-forensics-process.svg}}
\caption{Network Forensics Process}
\end{figure}

\textbf{મુખ્ય ઘટકો:}

{\def\LTcaptype{none} % do not increment counter
\begin{longtable}[]{@{}lll@{}}
\toprule\noalign{}
ઘટક & હેતુ & સાધનો \\
\midrule\noalign{}
\endhead
\bottomrule\noalign{}
\endlastfoot
\textbf{ટ્રાફિક કેપ્ચર} & નેટવર્ક ડેટા રેકોર્ડ કરવો & Wireshark, tcpdump \\
\textbf{વિશ્લેષણ} & પેટર્ન તપાસવા & NetworkMiner, Snort \\
\textbf{પુરાવા} & શોધોનો ઇનકાર અટકાવવો & ફોરેન્સિક રિપોર્ટ \\
\end{longtable}
}

\begin{itemize}
\tightlist
\item
  \textbf{અવકાશ}: પેકેટ્સ, ફ્લોઝ અને નેટવર્ક વર્તણૂકનું વિશ્લેષણ
\item
  \textbf{ઉદ્દેશ્ય}: સુરક્ષા ભંગ અને હુમલાના પેટર્ન ઓળખવા
\item
  \textbf{પડકાર}: મોટા ડેટા વોલ્યુમ અને રીઅલ-ટાઇમ પ્રોસેસિંગ
\end{itemize}

\end{solutionbox}
\begin{mnemonicbox}
``નેટવર્ક ફોરેન્સિક્સ તથ્યો શોધે''

\end{mnemonicbox}
\subsection*{પ્રશ્ન 5(બ) [4
ગુણ]}\label{uxaaauxab0uxab6uxaa8-5uxaac-4-uxa97uxaa3}

\textbf{ડિજિટલ ફોરેન્સિક તપાસમાં પુરાવા તરીકે CCTV શા માટે મહત્વની ભૂમિકા ભજવે છે
તે સમજાવો.}

\begin{solutionbox}

\textbf{ડિજિટલ ફોરેન્સિક્સમાં CCTV:}

\begin{figure}
\centering
\pandocbounded{\includesvg[keepaspectratio]{diagrams/cctv-digital-forensics.svg}}
\caption{CCTV Digital Forensics}
\end{figure}

{\def\LTcaptype{none} % do not increment counter
\begin{longtable}[]{@{}lll@{}}
\toprule\noalign{}
પાસું & મહત્વ & મૂલ્ય \\
\midrule\noalign{}
\endhead
\bottomrule\noalign{}
\endlastfoot
\textbf{વિઝ્યુઅલ પુરાવા} & સીધું અવલોકન & ઉચ્ચ વિશ્વસનીયતા \\
\textbf{ટાઇમલાઇન} & સમય-સ્ટેમ્પ રેકોર્ડ & ઘટના સહસંબંધ \\
\textbf{ડિજિટલ ફોર્મેટ} & વિશ્લેષણ કરવામાં સરળ & મેટાડેટા એક્સટ્રેક્શન \\
\textbf{બેકઅપ} & બહુવિધ કોપીઓ & પુરાવા સંરક્ષણ \\
\end{longtable}
}

\textbf{પુરાવાનું મૂલ્ય:}

\begin{itemize}
\tightlist
\item
  \textbf{સમર્થન}: અન્ય ડિજિટલ પુરાવાઓને સમર્થન આપે
\item
  \textbf{ટાઇમલાઇન}: ઘટનાઓનો ક્રમ સ્થાપિત કરે
\item
  \textbf{ઓળખ}: ગુનેગારની ઓળખ પ્રગટ કરી શકે
\item
  \textbf{સંદર્ભ}: ઘટના દરમિયાન ભૌતિક વાતાવરણ દર્શાવે
\end{itemize}

\textbf{ફોરેન્સિક વિચારણાઓ:}

\begin{itemize}
\tightlist
\item
  \textbf{ચેઇન ઓફ કસ્ટોડી}: યોગ્ય પુરાવા હેન્ડલિંગ
\item
  \textbf{ઓથેન્ટિકેશન}: વિડિયો અખંડિતતા ચકાસવી
\item
  \textbf{વિશ્લેષણ}: વૃદ્ધિ અને અર્થઘટન
\item
  \textbf{કાયદેસરી સ્વીકાર્યતા}: કોર્ટ-સ્વીકાર્ય ફોર્મેટ
\end{itemize}

\end{solutionbox}
\begin{mnemonicbox}
``CCTV ગુનાહિત વર્તણૂકને સ્પષ્ટ રીતે કેપ્ચર કરે''

\end{mnemonicbox}
\subsection*{પ્રશ્ન 5(ક) [7
ગુણ]}\label{uxaaauxab0uxab6uxaa8-5uxa95-7-uxa97uxaa3}

\textbf{ડિજિટલ ફોરેન્સિક તપાસના તબક્કાઓ સમજાવો.}

\begin{solutionbox}

\textbf{ડિજિટલ ફોરેન્સિક્સ તપાસના તબક્કાઓ:}

\begin{figure}
\centering
\pandocbounded{\includesvg[keepaspectratio]{diagrams/digital-forensics-phases.svg}}
\caption{Digital Forensics Phases}
\end{figure}

\begin{center}
\textbf{Mermaid Diagram (Code)}
\begin{verbatim}
{Shaded}
{Highlighting}[]
graph LR
    A[ઓળખ] {-{-}{} B[સંરક્ષણ]}
    B {-{-}{} C[સંગ્રહ]}
    C {-{-}{} D[પરીક્ષા]}
    D {-{-}{} E[વિશ્લેષણ]}
    E {-{-}{} F[પ્રસ્તુતિ]}
{Highlighting}
{Shaded}
\end{verbatim}
\end{center}

\textbf{વિગતવાર તબક્કાનું વિભાજન:}

{\def\LTcaptype{none} % do not increment counter
\begin{longtable}[]{@{}llll@{}}
\toprule\noalign{}
તબક્કો & પ્રવૃત્તિઓ & સાધનો & ઉદ્દેશ્ય \\
\midrule\noalign{}
\endhead
\bottomrule\noalign{}
\endlastfoot
\textbf{ઓળખ} & સંભવિત પુરાવાઓ ઓળખવા & વિઝ્યુઅલ નિરીક્ષણ & અવકાશ વ્યાખ્યા \\
\textbf{સંરક્ષણ} & પુરાવા દૂષણ અટકાવવું & રાઇટ બ્લોકર & પુરાવા અખંડતા \\
\textbf{સંગ્રહ} & ડિજિટલ પુરાવા મેળવવા & ફોરેન્સિક ઇમેજિંગ & સંપૂર્ણ ડેટા કેપ્ચર \\
\textbf{પરીક્ષા} & સંબંધિત ડેટા એક્સટ્રેક્ટ કરવો & Autopsy, FTK & ડેટા રિકવરી \\
\textbf{વિશ્લેષણ} & શોધોનું અર્થઘટન & ટાઇમલાઇન સાધનો & પેટર્ન ઓળખ \\
\textbf{પ્રસ્તુતિ} & પરિણામોનો દસ્તાવેજ & રિપોર્ટ જનરેટર & કાયદેસર પ્રસ્તુતિ \\
\end{longtable}
}

\textbf{તબક્કો 1 - ઓળખ:}

\begin{itemize}
\tightlist
\item
  દૃશ્યનું સર્વેક્ષણ કરવું
\item
  સંભવિત પુરાવા સ્ત્રોતોની ઓળખ
\item
  પ્રારંભિક અવલોકનોનો દસ્તાવેજ
\item
  તપાસનો અવકાશ સ્થાપિત કરવો
\end{itemize}

\textbf{તબક્કો 2 - સંરક્ષણ:}

\begin{itemize}
\tightlist
\item
  અપરાધ સ્થળ સુરક્ષિત કરવું
\item
  પુરાવા દૂષણ અટકાવવું
\item
  રાઇટ-પ્રોટેક્શન મિકેનિઝમનો ઉપયોગ
\item
  પુરાવાની સ્થિતિનો દસ્તાવેજ
\end{itemize}

\textbf{તબક્કો 3 - સંગ્રહ:}

\begin{itemize}
\tightlist
\item
  ફોરેન્સિક ઇમેજ બનાવવી
\item
  ચેઇન ઓફ કસ્ટડી જાળવે
\item
  યોગ્ય સંગ્રહ તકનીકોનો ઉપયોગ
\item
  વેરિફિકેશન માટે હેશ વેલ્યુ જનરેટ કરવી
\end{itemize}

\textbf{તબક્કો 4 - પરીક્ષા:}

\begin{itemize}
\tightlist
\item
  ફાઇલ સિસ્ટમ એક્સટ્રેક્ટ કરવી
\item
  કાઢી નાખેલ ડેટા રિકવર કરવો
\item
  સંબંધિત ફાઇલો ઓળખવી
\item
  શોધોનો દસ્તાવેજ
\end{itemize}

\textbf{તબક્કો 5 - વિશ્લેષણ:}

\begin{itemize}
\tightlist
\item
  પુરાવાઓને સહસંબંધિત કરવા
\item
  ઘટનાઓનું પુનઃનિર્માણ
\item
  પેટર્ન ઓળખવા
\item
  નિષ્કર્ષ ખખડાવવા
\end{itemize}

\textbf{તબક્કો 6 - પ્રસ્તુતિ:}

\begin{itemize}
\tightlist
\item
  વિગતવાર રિપોર્ટ તૈયાર કરવો
\item
  વિઝ્યુઅલ પ્રસ્તુતિઓ બનાવવી
\item
  તકનીકી શોધો સમજાવવા
\item
  કાયદેસરની કાર્યવાહીનું સમર્થન
\end{itemize}

\textbf{ગુણવત્તા ખાતરી:}

\begin{itemize}
\tightlist
\item
  \textbf{દસ્તાવેજીકરણ}: દરેક તબક્કે વિગતવાર રેકોર્ડ
\item
  \textbf{માન્યતા}: પ્રક્રિયાઓ અને પરિણામો ચકાસવા
\item
  \textbf{પુનઃઉત્પાદનક્ષમતા}: પરિણામો ડુપ્લિકેટ કરી શકાય તેની ખાતરી
\item
  \textbf{કાયદેસર અનુપાલન}: ન્યાયક્ષેત્રીય આવશ્યકતાઓનું પાલન
\end{itemize}

\end{solutionbox}
\begin{mnemonicbox}
``તપાસકર્તાઓ સંરક્ષિત કરે, એકત્ર કરે, તપાસે, વિશ્લેષણ કરે,
પ્રસ્તુત કરે''

\end{mnemonicbox}
\subsection*{પ્રશ્ન 5(અ અથવા) [3
ગુણ]}\label{uxaaauxab0uxab6uxaa8-5uxa85-uxa85uxaa5uxab5-3-uxa97uxaa3}

\textbf{સાયબર સુરક્ષા સંબંધિત વિવિધ ક્ષેત્રોમાં માઇક્રોકન્ટ્રોલરની એપ્લિકેશનોની યાદી
બનાવો.}

\begin{solutionbox}

\textbf{માઇક્રોકન્ટ્રોલર સુરક્ષા એપ્લિકેશન:}

\begin{figure}
\centering
\pandocbounded{\includesvg[keepaspectratio]{diagrams/microcontroller-security-applications.svg}}
\caption{Microcontroller Security Applications}
\end{figure}

{\def\LTcaptype{none} % do not increment counter
\begin{longtable}[]{@{}lll@{}}
\toprule\noalign{}
ક્ષેત્ર & એપ્લિકેશન & સુરક્ષા ફંક્શન \\
\midrule\noalign{}
\endhead
\bottomrule\noalign{}
\endlastfoot
\textbf{IoT સુરક્ષા} & સ્માર્ટ હોમ ડિવાઇસ & ઓથેન્ટિકેશન, એન્ક્રિપ્શન \\
\textbf{એક્સેસ કંટ્રોલ} & કી કાર્ડ, બાયોમેટ્રિક & ઓળખ ચકાસણી \\
\textbf{નેટવર્ક સુરક્ષા} & હાર્ડવેર ફાયરવોલ & પેકેટ ફિલ્ટરિંગ \\
\end{longtable}
}

\begin{itemize}
\tightlist
\item
  \textbf{સ્માર્ટ કાર્ડ}: સુરક્ષિત ઓથેન્ટિકેશન ટોકન
\item
  \textbf{HSM (હાર્ડવેર સિક્યોરિટી મોડ્યુલ)}: ક્રિપ્ટોગ્રાફિક પ્રોસેસિંગ
\item
  \textbf{એમ્બેડેડ સિસ્ટમ}: સિક્યોર બૂટ, ટેમ્પર ડિટેક્શન
\end{itemize}

\end{solutionbox}
\begin{mnemonicbox}
``માઇક્રોકન્ટ્રોલર બહુવિધ સુરક્ષા ફંક્શન મેનેજ કરે''

\end{mnemonicbox}
\subsection*{પ્રશ્ન 5(બ અથવા) [4
ગુણ]}\label{uxaaauxab0uxab6uxaa8-5uxaac-uxa85uxaa5uxab5-4-uxa97uxaa3}

\textbf{નૈતિક (એથિકલ) હેકિંગમાં પોર્ટ સ્કેનિંગનું મહત્વ સમજાવો.}

\begin{solutionbox}

\textbf{એથિકલ હેકિંગમાં પોર્ટ સ્કેનિંગ:}

\begin{figure}
\centering
\pandocbounded{\includesvg[keepaspectratio]{diagrams/port-scanning-ethical-hacking.svg}}
\caption{Port Scanning in Ethical Hacking}
\end{figure}

{\def\LTcaptype{none} % do not increment counter
\begin{longtable}[]{@{}lll@{}}
\toprule\noalign{}
પાસું & મહત્વ & ફાયદો \\
\midrule\noalign{}
\endhead
\bottomrule\noalign{}
\endlastfoot
\textbf{સેવા શોધ} & ચાલતી સેવાઓ ઓળખવી & હુમલા સપાટીનું મેપિંગ \\
\textbf{વલ્નરેબિલિટી એસેસમેન્ટ} & ખુલ્લા પોર્ટ શોધવા & સુરક્ષા ગેપ ઓળખ \\
\textbf{નેટવર્ક મેપિંગ} & ટોપોલોજી સમજવી & ઇન્ફ્રાસ્ટ્રક્ચર વિશ્લેષણ \\
\textbf{સુરક્ષા પરીક્ષણ} & કોન્ફિગરેશન માન્ય કરવી & અનુપાલન ચકાસણી \\
\end{longtable}
}

\textbf{પોર્ટ સ્કેનિંગ તકનીકો:}

\begin{itemize}
\tightlist
\item
  \textbf{TCP કનેક્ટ}: સંપૂર્ણ કનેક્શન સ્થાપના
\item
  \textbf{SYN સ્કેન}: સ્ટેલ્થ સ્કેનિંગ પદ્ધતિ
\item
  \textbf{UDP સ્કેન}: યુઝર ડેટાગ્રામ પ્રોટોકોલ સ્કેનિંગ
\item
  \textbf{સેવા ડિટેક્શન}: સેવા વર્ઝન ઓળખવી
\end{itemize}

\textbf{નૈતિક વિચારણાઓ:}

\begin{itemize}
\tightlist
\item
  \textbf{અધિકૃતતા}: યોગ્ય પરવાનગી મેળવવી
\item
  \textbf{અવકાશ}: નિર્ધારિત સીમાઓમાં રહેવું
\item
  \textbf{દસ્તાવેજીકરણ}: બધી પ્રવૃત્તિઓ રેકોર્ડ કરવી
\item
  \textbf{રિપોર્ટિંગ}: વિગતવાર શોધો પ્રદાન કરવા
\end{itemize}

\end{solutionbox}
\begin{mnemonicbox}
``પોર્ટ સ્કેનિંગ સુરક્ષા આંતરદૃષ્ટિ પ્રદાન કરે''

\end{mnemonicbox}
\subsection*{પ્રશ્ન 5(ક અથવા) [7
ગુણ]}\label{uxaaauxab0uxab6uxaa8-5uxa95-uxa85uxaa5uxab5-7-uxa97uxaa3}

\textbf{કાલી લિનક્સ ટૂલ્સનો ઉપયોગ કરીને વલ્નરેબિલિટી એસેસમેન્ટ કરવાની પ્રક્રિયાનું
વર્ણન કરો.}

\begin{solutionbox}

\textbf{વલ્નરેબિલિટી એસેસમેન્ટ પ્રક્રિયા:}

\begin{figure}
\centering
\pandocbounded{\includesvg[keepaspectratio]{diagrams/vulnerability-assessment-process.svg}}
\caption{Vulnerability Assessment Process}
\end{figure}

\begin{center}
\textbf{Mermaid Diagram (Code)}
\begin{verbatim}
{Shaded}
{Highlighting}[]
graph LR
    A[રિકોનેસન્સ] {-{-}{} B[પોર્ટ સ્કેનિંગ]}
    B {-{-}{} C[સેવા ગણતરી]}
    C {-{-}{} D[વલ્નરેબિલિટી સ્કેનિંગ]}
    D {-{-}{} E[વિશ્લેષણ અને રિપોર્ટિંગ]}
{Highlighting}
{Shaded}
\end{verbatim}
\end{center}

\textbf{પગલું-દર-પગલું પ્રક્રિયા:}

{\def\LTcaptype{none} % do not increment counter
\begin{longtable}[]{@{}
  >{\raggedright\arraybackslash}p{(\linewidth - 6\tabcolsep) * \real{0.1795}}
  >{\raggedright\arraybackslash}p{(\linewidth - 6\tabcolsep) * \real{0.2564}}
  >{\raggedright\arraybackslash}p{(\linewidth - 6\tabcolsep) * \real{0.4103}}
  >{\raggedright\arraybackslash}p{(\linewidth - 6\tabcolsep) * \real{0.1538}}@{}}
\toprule\noalign{}
\begin{minipage}[b]{\linewidth}\raggedright
પગલું
\end{minipage} & \begin{minipage}[b]{\linewidth}\raggedright
કાલી ટૂલ
\end{minipage} & \begin{minipage}[b]{\linewidth}\raggedright
કમાન્ડ ઉદાહરણ
\end{minipage} & \begin{minipage}[b]{\linewidth}\raggedright
હેતુ
\end{minipage} \\
\midrule\noalign{}
\endhead
\bottomrule\noalign{}
\endlastfoot
\textbf{રિકોનેસન્સ} & Nmap & \texttt{nmap\ -sn\ 192.168.1.0/24} & હોસ્ટ
શોધ \\
\textbf{પોર્ટ સ્કેનિંગ} & Nmap & \texttt{nmap\ -sS\ -O\ target} & ખુલ્લા
પોર્ટની ઓળખ \\
\textbf{સેવા ગણતરી} & Nmap, બેનર ગ્રેબિંગ & \texttt{nmap\ -sV\ target} & સેવા
વર્ઝન ડિટેક્શન \\
\textbf{વલ્નરેબિલિટી સ્કેનિંગ} & OpenVAS, Nessus & \texttt{openvas-start} &
ઓટોમેટેડ વલ્નરેબિલિટી ડિટેક્શન \\
\textbf{વેબ એપ્લિકેશન પરીક્ષણ} & Nikto, Dirb & \texttt{nikto\ -h\ target} &
વેબ વલ્નરેબિલિટી સ્કેનિંગ \\
\end{longtable}
}

\textbf{વિગતવાર પ્રક્રિયા:}

\textbf{તબક્કો 1 - લક્ષ્ય ઓળખ:}

\begin{itemize}
\tightlist
\item
  નેટવર્ક ડિસ્કવરી માટે Nmap નો ઉપયોગ
\item
  લાઇવ હોસ્ટ અને તેમના IP એડ્રેસની ઓળખ
\item
  નેટવર્ક ટોપોલોજીનો દસ્તાવેજ
\item
  લક્ષ્ય અવકાશ નિર્ધારણ
\end{itemize}

\textbf{તબક્કો 2 - પોર્ટ અને સેવા વિશ્લેષણ:}

\begin{itemize}
\tightlist
\item
  વ્યાપક પોર્ટ સ્કેન કરવા
\item
  ચાલતી સેવાઓ અને વર્ઝન ઓળખવા
\item
  ડિફોલ્ટ ક્રેડેન્શિયલ ચકાસવા
\item
  સેવા કોન્ફિગરેશન વિશ્લેષણ
\end{itemize}

\textbf{તબક્કો 3 - ઓટોમેટેડ વલ્નરેબિલિટી સ્કેનિંગ:}

\begin{itemize}
\tightlist
\item
  વલ્નરેબિલિટી સ્કેનર (OpenVAS) કોન્ફિગર કરવા
\item
  વ્યાપક સ્કેન ચલાવવા
\item
  સ્કેન પરિણામોનું વિશ્લેષણ
\item
  ગંભીરતા અનુસાર શોધોને પ્રાથમિકતા આપવી
\end{itemize}

\textbf{તબક્કો 4 - મેન્યુઅલ પરીક્ષણ:}

\begin{itemize}
\tightlist
\item
  ઓટોમેટેડ શોધોની ચકાસણી
\item
  લક્ષિત પરીક્ષણ કરવું
\item
  વિશિષ્ટ વલ્નરેબિલિટી માટે પરીક્ષણ
\item
  ફોલ્સ પોઝિટિવ માન્ય કરવા
\end{itemize}

\textbf{તબક્કો 5 - વેબ એપ્લિકેશન એસેસમેન્ટ:}

\begin{itemize}
\tightlist
\item
  વેબ વલ્નરેબિલિટી સ્કેનરનો ઉપયોગ
\item
  OWASP ટોપ 10 વલ્નરેબિલિટી માટે પરીક્ષણ
\item
  એપ્લિકેશન લોજિકનું વિશ્લેષણ
\item
  મિસકોન્ફિગરેશન ચકાસવા
\end{itemize}

\textbf{સામાન્ય કાલી ટૂલ્સ:}

{\def\LTcaptype{none} % do not increment counter
\begin{longtable}[]{@{}lll@{}}
\toprule\noalign{}
ટૂલ & ફંક્શન & ઉપયોગ કેસ \\
\midrule\noalign{}
\endhead
\bottomrule\noalign{}
\endlastfoot
\textbf{Nmap} & નેટવર્ક સ્કેનિંગ & પોર્ટ અને સેવા શોધ \\
\textbf{OpenVAS} & વલ્નરેબિલિટી સ્કેનિંગ & ઓટોમેટેડ એસેસમેન્ટ \\
\textbf{Nikto} & વેબ સ્કેનિંગ & વેબ સર્વર વલ્નરેબિલિટી \\
\textbf{Dirb} & ડિરેક્ટરી બ્રુટ ફોર્સિંગ & છુપાયેલ ફાઇલ શોધ \\
\textbf{SQLmap} & SQL ઇન્જેક્શન પરીક્ષણ & ડેટાબેસ વલ્નરેબિલિટી \\
\textbf{Burp Suite} & વેબ પ્રોક્સી & મેન્યુઅલ વેબ પરીક્ષણ \\
\textbf{Metasploit} & એક્સપ્લોઇટેશન ફ્રેમવર્ક & વલ્નરેબિલિટી માન્યતા \\
\end{longtable}
}

\textbf{એસેસમેન્ટ પદ્ધતિ:}

\begin{itemize}
\tightlist
\item
  \textbf{અવકાશ વ્યાખ્યા}: એસેસમેન્ટ સીમાઓ સ્પષ્ટ રીતે વ્યાખ્યાયિત કરવી
\item
  \textbf{માહિતી એકત્રીકરણ}: લક્ષ્ય ઇન્ટેલિજન્સ એકત્ર કરવી
\item
  \textbf{વલ્નરેબિલિટી ડિટેક્શન}: બહુવિધ સ્કેનિંગ પદ્ધતિઓનો ઉપયોગ
\item
  \textbf{જોખમ એસેસમેન્ટ}: અસર અને સંભાવનાનું મૂલ્યાંકન
\item
  \textbf{રેમેડિએશન પ્લાનિંગ}: કાર્યક્ષમ ભલામણો પ્રદાન કરવી
\end{itemize}

\textbf{રિપોર્ટિંગ ઘટકો:}

\begin{itemize}
\tightlist
\item
  \textbf{એક્ઝિક્યુટિવ સમરી}: મેનેજમેન્ટ માટે ઉચ્ચ-સ્તરીય શોધો
\item
  \textbf{તકનીકી વિગતો}: વલ્નરેબિલિટીના વિગતવાર વર્ણનો
\item
  \textbf{જોખમ રેટિંગ}: CVSS સ્કોર અને બિઝનેસ અસર
\item
  \textbf{રેમેડિએશન સ્ટેપ્સ}: વિશિષ્ટ મિટિગેશન ભલામણો
\item
  \textbf{સપોર્ટિંગ એવિડન્સ}: સ્ક્રીનશોટ અને પ્રૂફ-ઓફ-કોન્સેપ્ટ
\end{itemize}

\textbf{બેસ્ટ પ્રેક્ટિસિસ:}

\begin{itemize}
\tightlist
\item
  \textbf{અધિકૃતતા}: હંમેશા લેખિત પરવાનગી મેળવવી
\item
  \textbf{દસ્તાવેજીકરણ}: બધી પ્રવૃત્તિઓના વિગતવાર લોગ જાળવવા
\item
  \textbf{ન્યૂનતમ અસર}: પ્રોડક્શન સિસ્ટમને ખલેલ પહોંચાડવાનું ટાળવું
\item
  \textbf{ગોપનીયતા}: શોધાયેલ સંવેદનશીલ માહિતીનું રક્ષણ કરવું
\end{itemize}

\end{solutionbox}
\begin{mnemonicbox}
``વલ્નરેબિલિટી એસેસમેન્ટ એપ્લિકેશન સિક્યોરિટીને માન્ય કરે''

\end{mnemonicbox}

\end{document}
