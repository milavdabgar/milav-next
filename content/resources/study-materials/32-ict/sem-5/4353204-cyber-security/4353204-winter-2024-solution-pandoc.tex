\documentclass[10pt,a4paper]{article}

% content/resources/templates/preamble.tex
\usepackage[margin=0.6in]{geometry}
\author{Milav Dabgar}
\usepackage{amsmath,amssymb,amsthm}
\usepackage{booktabs}
\usepackage{multirow}
\usepackage{xcolor}
\usepackage{tcolorbox}
\tcbuselibrary{breakable,skins}
\usepackage[colorlinks=true,linkcolor=blue]{hyperref}
\usepackage{titlesec}
\usepackage{enumitem}
\usepackage{tikz}
\usepackage{pgfplots}
\usepackage{circuitikz}
\usepackage[version=4]{mhchem}
\usepackage{longtable}
\usepackage{array}
\usepackage{float}
\usepackage{caption}
\usepackage{listings}

\lstset{
  basicstyle=\small\ttfamily,
  breaklines=true,
  breakatwhitespace=false,
  postbreak=\mbox{\textcolor{red}{$\hookrightarrow$}\space},
  float=false,
  numbers=left,
  numberstyle=\tiny\color{gray},
  numbersep=10pt,
  xleftmargin=2em,
  keywordstyle=\color{blue},
  commentstyle=\color{green!60!black},
  stringstyle=\color{purple},
  backgroundcolor=\color{gray!5},
  showstringspaces=false,
  tabsize=2,
  captionpos=b,
  keepspaces=true,
  columns=flexible
}

\pgfplotsset{compat=1.18}
\usetikzlibrary{shapes,arrows,positioning,calc,patterns,decorations.pathmorphing,decorations.markings,arrows.meta}

% Color scheme
\definecolor{headcolor}{RGB}{0,102,204}
\definecolor{keycolor}{RGB}{220,20,60}
\definecolor{solutioncolor}{RGB}{34,139,34}
\definecolor{mnemoniccolor}{RGB}{148,0,211}
\definecolor{codecolor}{RGB}{0,0,100}

% Spacing
\setlength{\parskip}{3pt}
\setlist[itemize]{nosep}
\setlist[enumerate]{nosep}

% Title formatting
\titleformat{\section}{\Large\bfseries\color{headcolor}}{\thesection}{1em}{}
\titleformat{\subsection}{\large\bfseries\color{headcolor}}{\thesubsection}{1em}{}

% Pandoc tightlist compatibility
\providecommand{\tightlist}{%
  \setlength{\itemsep}{0pt}\setlength{\parskip}{0pt}}

% Pandoc longtable compatibility
\newcounter{none}
\def\thenone{}


% content/resources/templates/english-boxes.tex
% This file is currently empty - it exists to maintain consistency with the import structure.
% Add custom environments here if needed in the future.


\begin{document}

\begin{center}
{\Huge\bfseries\color{headcolor} Subject Name Solutions}\\[5pt]
{\LARGE 4353204 -- Winter 2024}\\[3pt]
{\large Semester 1 Study Material}\\[3pt]
{\normalsize\textit{Detailed Solutions and Explanations}}
\end{center}

\vspace{10pt}

\subsection*{Question 1(a) [3 marks]}\label{q1a}

\textbf{Define cyber security \& computer security.}

\begin{solutionbox}

\begin{itemize}
\tightlist
\item
  \textbf{Cyber Security}: Protection of internet-connected systems
  including hardware, software, and data from cyber threats. It focuses
  on defending networks, devices, and programs from unauthorized digital
  attacks.
\item
  \textbf{Computer Security}: Protection of individual computer systems
  and data from theft, damage, or unauthorized access. It focuses on
  safeguarding the physical computer hardware and the software installed
  on it.
\end{itemize}

\textbf{Diagram:}

\begin{figure}
\centering
\pandocbounded{\includesvg[keepaspectratio]{diagrams/cyber-vs-computer-security-comparison.svg}}
\caption{Cyber Security vs Computer Security}
\end{figure}

\end{solutionbox}
\begin{mnemonicbox}
``Cyber Circles Networks, Computer Covers Machines''

\end{mnemonicbox}
\subsection*{Question 1(b) [4 marks]}\label{q1b}

\textbf{Explain CIA triad.}

\begin{solutionbox}
The CIA triad represents the three fundamental
principles of information security:

{\def\LTcaptype{none} % do not increment counter
\begin{longtable}[]{@{}
  >{\raggedright\arraybackslash}p{(\linewidth - 2\tabcolsep) * \real{0.4583}}
  >{\raggedright\arraybackslash}p{(\linewidth - 2\tabcolsep) * \real{0.5417}}@{}}
\toprule\noalign{}
\begin{minipage}[b]{\linewidth}\raggedright
Principle
\end{minipage} & \begin{minipage}[b]{\linewidth}\raggedright
Description
\end{minipage} \\
\midrule\noalign{}
\endhead
\bottomrule\noalign{}
\endlastfoot
\textbf{Confidentiality} & Ensures that sensitive information is
accessible only to authorized parties \\
\textbf{Integrity} & Guarantees that data remains accurate and unaltered
during storage and transmission \\
\textbf{Availability} & Ensures systems and data are accessible when
needed by authorized users \\
\end{longtable}
}

\textbf{Diagram:}

\begin{figure}
\centering
\pandocbounded{\includesvg[keepaspectratio]{diagrams/cia-triad.svg}}
\caption{CIA Triad}
\end{figure}

\end{solutionbox}
\begin{mnemonicbox}
``CIA Keeps Information Properly Accessible''

\end{mnemonicbox}
\subsection*{Question 1(c) [7 marks]}\label{q1c}

\textbf{Define adversary, attack, countermeasure, risk, security policy,
system resource, and threat in the context of computer security.}

\begin{solutionbox}

{\def\LTcaptype{none} % do not increment counter
\begin{longtable}[]{@{}
  >{\raggedright\arraybackslash}p{(\linewidth - 2\tabcolsep) * \real{0.3333}}
  >{\raggedright\arraybackslash}p{(\linewidth - 2\tabcolsep) * \real{0.6667}}@{}}
\toprule\noalign{}
\begin{minipage}[b]{\linewidth}\raggedright
Term
\end{minipage} & \begin{minipage}[b]{\linewidth}\raggedright
Definition
\end{minipage} \\
\midrule\noalign{}
\endhead
\bottomrule\noalign{}
\endlastfoot
\textbf{Adversary} & Individual or group that attempts to exploit
vulnerabilities for malicious purposes \\
\textbf{Attack} & Deliberate action to compromise security by exploiting
vulnerabilities in a system \\
\textbf{Countermeasure} & Controls implemented to mitigate or eliminate
security vulnerabilities \\
\textbf{Risk} & Potential for loss or damage when a threat exploits a
vulnerability \\
\textbf{Security Policy} & Documented rules that define acceptable use
and protection requirements \\
\textbf{System Resource} & Hardware, software, data, or network
components that require protection \\
\textbf{Threat} & Potential danger that might exploit a vulnerability to
breach security \\
\end{longtable}
}

\textbf{Diagram:}

\begin{figure}
\centering
\pandocbounded{\includesvg[keepaspectratio]{diagrams/security-threat-model.svg}}
\caption{Security Threat Model}
\end{figure}

\end{solutionbox}
\begin{mnemonicbox}
``ARTSVSC: All Resources Typically Secure Various
System Components''

\end{mnemonicbox}
\subsection*{Question 1(c OR) [7
marks]}\label{question-1c-or-7-marks}

\textbf{Explain MD5 hashing algorithm.}

\begin{solutionbox}
MD5 (Message Digest 5) is a widely used cryptographic
hash function that produces a 128-bit (16-byte) hash value:

\begin{enumerate}
\tightlist
\item
  \textbf{Input Processing}: Message is padded and divided into 512-bit
  blocks
\item
  \textbf{Initialization}: Sets up four 32-bit registers with fixed
  values
\item
  \textbf{Compression}: Processes message in 16-word blocks through four
  rounds of operations
\item
  \textbf{Output}: Produces 128-bit digest as final hash value
\end{enumerate}

\textbf{Diagram:}

\begin{figure}
\centering
\pandocbounded{\includesvg[keepaspectratio]{diagrams/md5-algorithm.svg}}
\caption{MD5 Algorithm}
\end{figure}

\begin{itemize}
\tightlist
\item
  \textbf{Weakness}: Not collision-resistant; shouldn't be used for
  security-critical applications
\item
  \textbf{Usage}: File integrity verification and non-security critical
  applications
\end{itemize}

\end{solutionbox}
\begin{mnemonicbox}
``Pad, Divide, Process, Output - Don't Use For
Security!''

\end{mnemonicbox}
\subsection*{Question 2(a) [3 marks]}\label{q2a}

\textbf{Define authentication in context of cyber security.}

\begin{solutionbox}
Authentication is the process of verifying the identity
of a user, system, or entity before granting access to resources:

\begin{itemize}
\tightlist
\item
  \textbf{Confirms}: ``You are who you claim to be''
\item
  \textbf{Verifies}: Identity using credentials (passwords, biometrics,
  tokens)
\item
  \textbf{Precedes}: Authorization (what you can access after
  authentication)
\end{itemize}

\textbf{Diagram:}

\begin{figure}
\centering
\pandocbounded{\includesvg[keepaspectratio]{diagrams/authentication-process.svg}}
\caption{Authentication Process}
\end{figure}

\end{solutionbox}
\begin{mnemonicbox}
``Verify Before Entry''

\end{mnemonicbox}
\subsection*{Question 2(b) [4 marks]}\label{q2b}

\textbf{Explain public key cryptography with example.}

\begin{solutionbox}
Public key cryptography uses two mathematically related
keys for secure communication:

{\def\LTcaptype{none} % do not increment counter
\begin{longtable}[]{@{}ll@{}}
\toprule\noalign{}
Component & Function \\
\midrule\noalign{}
\endhead
\bottomrule\noalign{}
\endlastfoot
\textbf{Public Key} & Shared openly and used to encrypt messages \\
\textbf{Private Key} & Kept secret and used to decrypt messages \\
\end{longtable}
}

\textbf{Example}: In RSA encryption, if Alice wants to send Bob a
message:

\begin{enumerate}
\tightlist
\item
  Alice encrypts with Bob's public key
\item
  Only Bob can decrypt using his private key
\end{enumerate}

\textbf{Diagram:}

\begin{figure}
\centering
\pandocbounded{\includesvg[keepaspectratio]{diagrams/public-key-cryptography-example.svg}}
\caption{Public Key Cryptography Example}
\end{figure}

\end{solutionbox}
\begin{mnemonicbox}
``Public to Lock, Private to Unlock''

\end{mnemonicbox}
\subsection*{Question 2(c) [7 marks]}\label{q2c}

\textbf{Explain working of packet filter and application proxy.}

\begin{solutionbox}

{\def\LTcaptype{none} % do not increment counter
\begin{longtable}[]{@{}
  >{\raggedright\arraybackslash}p{(\linewidth - 2\tabcolsep) * \real{0.6250}}
  >{\raggedright\arraybackslash}p{(\linewidth - 2\tabcolsep) * \real{0.3750}}@{}}
\toprule\noalign{}
\begin{minipage}[b]{\linewidth}\raggedright
Firewall Type
\end{minipage} & \begin{minipage}[b]{\linewidth}\raggedright
Working
\end{minipage} \\
\midrule\noalign{}
\endhead
\bottomrule\noalign{}
\endlastfoot
\textbf{Packet Filter} & Examines packet headers based on predefined
rules. Makes decisions based on source/destination IP addresses, ports,
and protocols. Works at OSI network and transport layers. Offers
high-speed filtering with low resource usage. \\
\textbf{Application Proxy} & Acts as intermediary between client and
server applications. Processes all traffic at application layer. Creates
two connections (client-to-proxy and proxy-to-server). Provides content
inspection and user authentication capabilities. \\
\end{longtable}
}

\textbf{Diagram:}

\begin{figure}
\centering
\pandocbounded{\includesvg[keepaspectratio]{diagrams/packet-filter-vs-proxy.svg}}
\caption{Packet Filter vs Application Proxy}
\end{figure}

\end{solutionbox}
\begin{mnemonicbox}
``Packets Check Headers, Proxies Check Content''

\end{mnemonicbox}
\subsection*{Question 2(a OR) [3
marks]}\label{question-2a-or-3-marks}

\textbf{Explain multi-factor authentication.}

\begin{solutionbox}
Multi-factor authentication (MFA) requires users to
provide two or more verification factors to gain access to a resource:

\begin{itemize}
\tightlist
\item
  \textbf{Something you know}: Password, PIN, security question
\item
  \textbf{Something you have}: Mobile phone, smart card, security token
\item
  \textbf{Something you are}: Fingerprint, facial recognition, voice
  pattern
\end{itemize}

\textbf{Diagram:}

\begin{figure}
\centering
\pandocbounded{\includesvg[keepaspectratio]{diagrams/multi-factor-authentication.svg}}
\caption{Multi-Factor Authentication}
\end{figure}

\end{solutionbox}
\begin{mnemonicbox}
``Know, Have, Are - Triple Security''

\end{mnemonicbox}
\subsection*{Question 2(b OR) [4
marks]}\label{question-2b-or-4-marks}

\textbf{Explain the process of password verification.}

\begin{solutionbox}
Password verification is the process of authenticating
user credentials against stored values:

\begin{enumerate}
\tightlist
\item
  \textbf{User Input}: User enters username and password
\item
  \textbf{Hash Generation}: System hashes the entered password
\item
  \textbf{Comparison}: Hash is compared with stored hash in database
\item
  \textbf{Access Decision}: Access granted if hashes match, denied if
  not
\end{enumerate}

\textbf{Diagram:}

\begin{figure}
\centering
\pandocbounded{\includesvg[keepaspectratio]{diagrams/password-verification-process.svg}}
\caption{Password Verification Process}
\end{figure}

\end{solutionbox}
\begin{mnemonicbox}
``Enter, Hash, Compare, Decide''

\end{mnemonicbox}
\subsection*{Question 2(c OR) [7
marks]}\label{question-2c-or-7-marks}

\textbf{List out malicious software and explain any three malicious
software attacks.}

\begin{solutionbox}

\textbf{Malicious Software Types}:

\begin{itemize}
\tightlist
\item
  Viruses, Worms, Trojans, Ransomware, Spyware, Adware, Rootkits,
  Keyloggers, Bots
\end{itemize}

\textbf{Three Common Attacks}:

{\def\LTcaptype{none} % do not increment counter
\begin{longtable}[]{@{}
  >{\raggedright\arraybackslash}p{(\linewidth - 2\tabcolsep) * \real{0.5000}}
  >{\raggedright\arraybackslash}p{(\linewidth - 2\tabcolsep) * \real{0.5000}}@{}}
\toprule\noalign{}
\begin{minipage}[b]{\linewidth}\raggedright
Attack Type
\end{minipage} & \begin{minipage}[b]{\linewidth}\raggedright
Explanation
\end{minipage} \\
\midrule\noalign{}
\endhead
\bottomrule\noalign{}
\endlastfoot
\textbf{Ransomware} & Encrypts victim's files and demands payment for
decryption key. Spreads through phishing emails, malicious downloads, or
exploiting vulnerabilities. Example: WannaCry. \\
\textbf{Trojans} & Disguised as legitimate software but performs
malicious actions. Creates backdoors for attackers to access systems.
Example: Remote Access Trojans (RATs). \\
\textbf{Spyware} & Collects user information without consent. Monitors
activities, keystrokes, and browsing habits. Can steal passwords and
financial information. \\
\end{longtable}
}

\textbf{Diagram:}

\begin{figure}
\centering
\pandocbounded{\includesvg[keepaspectratio]{diagrams/malware-attacks-detailed.svg}}
\caption{Malicious Software Attacks}
\end{figure}

\end{solutionbox}
\begin{mnemonicbox}
``RTS: Ransom Takes Systems, Trojans Sneak In,
Spyware Steals Info''

\end{mnemonicbox}
\subsection*{Question 3(a) [3 marks]}\label{q3a}

\textbf{Explain the importance of ports in cyber security.}

\begin{solutionbox}
Ports are virtual endpoints for network communications
that:

\begin{itemize}
\tightlist
\item
  \textbf{Identify Services}: Each service uses specific port numbers
  (HTTP:80, HTTPS:443)
\item
  \textbf{Enable Filtering}: Firewalls control traffic by
  allowing/blocking specific ports
\item
  \textbf{Reduce Attack Surface}: Closing unnecessary ports enhances
  security
\end{itemize}

\textbf{Diagram:}

\begin{figure}
\centering
\pandocbounded{\includesvg[keepaspectratio]{diagrams/port-security-importance.svg}}
\caption{Port Security Importance}
\end{figure}

\end{solutionbox}
\begin{mnemonicbox}
``Every Port Is An Entry Point''

\end{mnemonicbox}
\subsection*{Question 3(b) [4 marks]}\label{q3b}

\textbf{Explain Virtual private network.}

\begin{solutionbox}
A Virtual Private Network (VPN) is a technology that:

{\def\LTcaptype{none} % do not increment counter
\begin{longtable}[]{@{}
  >{\raggedright\arraybackslash}p{(\linewidth - 2\tabcolsep) * \real{0.4091}}
  >{\raggedright\arraybackslash}p{(\linewidth - 2\tabcolsep) * \real{0.5909}}@{}}
\toprule\noalign{}
\begin{minipage}[b]{\linewidth}\raggedright
Feature
\end{minipage} & \begin{minipage}[b]{\linewidth}\raggedright
Description
\end{minipage} \\
\midrule\noalign{}
\endhead
\bottomrule\noalign{}
\endlastfoot
\textbf{Encrypted Tunnel} & Creates secure connection over public
networks \\
\textbf{IP Masking} & Hides user's IP address and location \\
\textbf{Data Protection} & Encrypts data during transmission \\
\textbf{Remote Access} & Enables secure connection to private
networks \\
\end{longtable}
}

\textbf{Diagram:}

\begin{figure}
\centering
\pandocbounded{\includesvg[keepaspectratio]{diagrams/vpn-architecture.svg}}
\caption{VPN Architecture}
\end{figure}

\end{solutionbox}
\begin{mnemonicbox}
``Tunnel, Encrypt, Protect, Connect''

\end{mnemonicbox}
\subsection*{Question 3(c) [7 marks]}\label{q3c}

\textbf{Explain the impact of web security threats.}

\begin{solutionbox}
Web security threats have significant impacts on
organizations and individuals:

{\def\LTcaptype{none} % do not increment counter
\begin{longtable}[]{@{}
  >{\raggedright\arraybackslash}p{(\linewidth - 2\tabcolsep) * \real{0.3810}}
  >{\raggedright\arraybackslash}p{(\linewidth - 2\tabcolsep) * \real{0.6190}}@{}}
\toprule\noalign{}
\begin{minipage}[b]{\linewidth}\raggedright
Impact
\end{minipage} & \begin{minipage}[b]{\linewidth}\raggedright
Description
\end{minipage} \\
\midrule\noalign{}
\endhead
\bottomrule\noalign{}
\endlastfoot
\textbf{Data Breaches} & Exposure of sensitive information leading to
financial losses and reputation damage \\
\textbf{Financial Loss} & Direct monetary theft, fraud, recovery costs,
and regulatory fines \\
\textbf{Operational Disruption} & System downtime affecting business
continuity and customer service \\
\textbf{Reputation Damage} & Loss of customer trust and brand value
after security incidents \\
\textbf{Legal Consequences} & Litigation, regulatory penalties, and
compliance violations \\
\end{longtable}
}

\textbf{Diagram:}

\begin{figure}
\centering
\pandocbounded{\includesvg[keepaspectratio]{diagrams/web-security-threats-impact.svg}}
\caption{Web Security Threats Impact}
\end{figure}

\end{solutionbox}
\begin{mnemonicbox}
``DFROL: Data, Finances, Resources, Opinion, Legal''

\end{mnemonicbox}
\subsection*{Question 3(a OR) [3
marks]}\label{question-3a-or-3-marks}

\textbf{Explain working of digital signature.}

\begin{solutionbox}
Digital signatures authenticate electronic documents
and verify their integrity:

\begin{enumerate}
\tightlist
\item
  \textbf{Hash Creation}: Document is hashed to create a unique digest
\item
  \textbf{Encryption}: Sender encrypts the hash using their private key
\item
  \textbf{Verification}: Recipient decrypts using sender's public key
\item
  \textbf{Validation}: Comparing decrypted hash with newly generated
  hash
\end{enumerate}

\textbf{Diagram:}

\begin{figure}
\centering
\pandocbounded{\includesvg[keepaspectratio]{diagrams/digital-signature-working.svg}}
\caption{Digital Signature Working}
\end{figure}

\end{solutionbox}
\begin{mnemonicbox}
``Hash, Sign, Send, Verify''

\end{mnemonicbox}
\subsection*{Question 3(b OR) [4
marks]}\label{question-3b-or-4-marks}

\textbf{Describe HTTPS.}

\begin{solutionbox}
HTTPS (Hypertext Transfer Protocol Secure) is a secure
version of HTTP:

{\def\LTcaptype{none} % do not increment counter
\begin{longtable}[]{@{}ll@{}}
\toprule\noalign{}
Feature & Description \\
\midrule\noalign{}
\endhead
\bottomrule\noalign{}
\endlastfoot
\textbf{TLS/SSL} & Uses Transport Layer Security to encrypt data \\
\textbf{Authentication} & Verifies website identity through
certificates \\
\textbf{Data Integrity} & Prevents tampering of transmitted data \\
\textbf{Port 443} & Uses default port 443 instead of HTTP's port 80 \\
\end{longtable}
}

\textbf{Diagram:}

\begin{figure}
\centering
\pandocbounded{\includesvg[keepaspectratio]{diagrams/https-process.svg}}
\caption{HTTPS Process}
\end{figure}

\end{solutionbox}
\begin{mnemonicbox}
``Secured Pages Show Padlock''

\end{mnemonicbox}
\subsection*{Question 3(c OR) [7
marks]}\label{question-3c-or-7-marks}

\textbf{Explain social engineering, vishing and machine in the middle
attack.}

\begin{solutionbox}

{\def\LTcaptype{none} % do not increment counter
\begin{longtable}[]{@{}
  >{\raggedright\arraybackslash}p{(\linewidth - 2\tabcolsep) * \real{0.5000}}
  >{\raggedright\arraybackslash}p{(\linewidth - 2\tabcolsep) * \real{0.5000}}@{}}
\toprule\noalign{}
\begin{minipage}[b]{\linewidth}\raggedright
Attack Type
\end{minipage} & \begin{minipage}[b]{\linewidth}\raggedright
Explanation
\end{minipage} \\
\midrule\noalign{}
\endhead
\bottomrule\noalign{}
\endlastfoot
\textbf{Social Engineering} & Psychological manipulation to trick users
into revealing sensitive information. Exploits human trust rather than
technical vulnerabilities. Common techniques include pretexting,
baiting, and phishing. \\
\textbf{Vishing} & Voice phishing using phone calls to steal
information. Attackers impersonate legitimate organizations. Often uses
urgency or fear to manipulate victims. \\
\textbf{Machine in the Middle} & Attacker secretly intercepts and relays
communication between two parties. Victims believe they're communicating
directly with each other. Allows attackers to steal/modify sensitive
information during transmission. \\
\end{longtable}
}

\textbf{Diagram:}

\begin{figure}
\centering
\pandocbounded{\includesvg[keepaspectratio]{diagrams/mitm-attack.svg}}
\caption{Man-in-the-Middle Attack}
\end{figure}

\end{solutionbox}
\begin{mnemonicbox}
``SEVeM: Social Engineers Voice Messages and Mediate
connections''

\end{mnemonicbox}
\subsection*{Question 4(a) [3 marks]}\label{q4a}

\textbf{Match the following.}

\begin{solutionbox}

{\def\LTcaptype{none} % do not increment counter
\begin{longtable}[]{@{}
  >{\raggedright\arraybackslash}p{(\linewidth - 2\tabcolsep) * \real{0.5000}}
  >{\raggedright\arraybackslash}p{(\linewidth - 2\tabcolsep) * \real{0.5000}}@{}}
\toprule\noalign{}
\begin{minipage}[b]{\linewidth}\raggedright
Column A
\end{minipage} & \begin{minipage}[b]{\linewidth}\raggedright
Column B
\end{minipage} \\
\midrule\noalign{}
\endhead
\bottomrule\noalign{}
\endlastfoot
1. Denial of Service (DoS) & f.~Attack that disrupts network services \\
2. Port 443 & c.~Default port for HTTPS \\
3. Secure Socket Layer (SSL) & e. Predecessor of TLS for secure
communication \\
4. Port 80 & b. Default port for HTTP \\
5. Integrity & a. Ensures data is not altered during transmission \\
6. VPN (Virtual Private Network) & d.~Creates a secure connection over
the internet \\
\end{longtable}
}

\textbf{Diagram:}

\begin{figure}
\centering
\pandocbounded{\includesvg[keepaspectratio]{diagrams/cybersecurity-matching-diagram.svg}}
\caption{Cybersecurity Terms Matching}
\end{figure}

\end{solutionbox}
\begin{mnemonicbox}
``Disrupt HTTPS, Secure HTTP, Intact VPN''

\end{mnemonicbox}
\subsection*{Question 4(b) [4 marks]}\label{q4b}

\textbf{List out types of hackers and explain role of each.}

\begin{solutionbox}

{\def\LTcaptype{none} % do not increment counter
\begin{longtable}[]{@{}
  >{\raggedright\arraybackslash}p{(\linewidth - 2\tabcolsep) * \real{0.6842}}
  >{\raggedright\arraybackslash}p{(\linewidth - 2\tabcolsep) * \real{0.3158}}@{}}
\toprule\noalign{}
\begin{minipage}[b]{\linewidth}\raggedright
Hacker Type
\end{minipage} & \begin{minipage}[b]{\linewidth}\raggedright
Role
\end{minipage} \\
\midrule\noalign{}
\endhead
\bottomrule\noalign{}
\endlastfoot
\textbf{White Hat} & Ethical hackers who test systems with permission to
improve security \\
\textbf{Black Hat} & Malicious hackers who exploit vulnerabilities for
personal gain or damage \\
\textbf{Gray Hat} & Operate between ethical and malicious; may hack
without permission but disclose findings \\
\textbf{Script Kiddies} & Inexperienced hackers using pre-written
scripts without understanding the technology \\
\end{longtable}
}

\textbf{Diagram:}

\begin{figure}
\centering
\pandocbounded{\includesvg[keepaspectratio]{diagrams/hacker-types.svg}}
\caption{Hacker Types}
\end{figure}

\end{solutionbox}
\begin{mnemonicbox}
``White Protects, Black Attacks, Gray Mixes, Kids
Script''

\end{mnemonicbox}
\subsection*{Question 4(c) [7 marks]}\label{q4c}

\textbf{Explain SSH (Secure shell) protocol stack.}

\begin{solutionbox}
SSH (Secure Shell) protocol stack provides secure
remote access and file transfers:

{\def\LTcaptype{none} % do not increment counter
\begin{longtable}[]{@{}
  >{\raggedright\arraybackslash}p{(\linewidth - 2\tabcolsep) * \real{0.4118}}
  >{\raggedright\arraybackslash}p{(\linewidth - 2\tabcolsep) * \real{0.5882}}@{}}
\toprule\noalign{}
\begin{minipage}[b]{\linewidth}\raggedright
Layer
\end{minipage} & \begin{minipage}[b]{\linewidth}\raggedright
Function
\end{minipage} \\
\midrule\noalign{}
\endhead
\bottomrule\noalign{}
\endlastfoot
\textbf{Transport Layer} & Handles encryption, server authentication,
and data integrity \\
\textbf{User Authentication Layer} & Verifies client identity using
passwords, keys, or certificates \\
\textbf{Connection Layer} & Manages multiple channels within a single
SSH connection \\
\end{longtable}
}

\textbf{Key Features}:

\begin{itemize}
\tightlist
\item
  Strong encryption (AES, 3DES)
\item
  Public key authentication
\item
  Data integrity checking
\item
  Port forwarding and tunneling
\end{itemize}

\textbf{Diagram:}

\begin{figure}
\centering
\pandocbounded{\includesvg[keepaspectratio]{diagrams/ssh-protocol-stack.svg}}
\caption{SSH Protocol Stack}
\end{figure}

\end{solutionbox}
\begin{mnemonicbox}
``Transport Secures, Users Authenticate, Connections
Multiplex''

\end{mnemonicbox}
\subsection*{Question 4(a OR) [3
marks]}\label{question-4a-or-3-marks}

\textbf{Explain foot printing in ethical hacking.}

\begin{solutionbox}
Footprinting is the first phase of ethical hacking
where information is gathered about the target:

\begin{itemize}
\tightlist
\item
  \textbf{Purpose}: Collecting data about network, systems, and
  organization
\item
  \textbf{Methods}: WHOIS lookup, DNS analysis, social media research
\item
  \textbf{Outcomes}: Identifying potential entry points and
  vulnerabilities
\end{itemize}

\textbf{Diagram:}

\begin{figure}
\centering
\pandocbounded{\includesvg[keepaspectratio]{diagrams/footprinting-ethical-hacking.svg}}
\caption{Footprinting in Ethical Hacking}
\end{figure}

\end{solutionbox}
\begin{mnemonicbox}
``Gather Before Attack''

\end{mnemonicbox}
\subsection*{Question 4(b OR) [4
marks]}\label{question-4b-or-4-marks}

\textbf{Explain scanning in ethical hacking.}

\begin{solutionbox}
Scanning is the process of actively probing a target
system to identify live hosts, open ports, and services:

{\def\LTcaptype{none} % do not increment counter
\begin{longtable}[]{@{}ll@{}}
\toprule\noalign{}
Technique & Purpose \\
\midrule\noalign{}
\endhead
\bottomrule\noalign{}
\endlastfoot
\textbf{Port Scanning} & Identifies open ports and running services \\
\textbf{Vulnerability Scanning} & Detects known security weaknesses \\
\textbf{Network Mapping} & Discovers network topology and devices \\
\textbf{OS Fingerprinting} & Determines operating system versions \\
\end{longtable}
}

\textbf{Diagram:}

\begin{figure}
\centering
\pandocbounded{\includesvg[keepaspectratio]{diagrams/port-scanning-ethical-hacking.svg}}
\caption{Port Scanning Process}
\end{figure}

\end{solutionbox}
\begin{mnemonicbox}
``PONS: Ports Open, Network Services''

\end{mnemonicbox}
\subsection*{Question 4(c OR) [7
marks]}\label{question-4c-or-7-marks}

\textbf{Describe injection attack and phishing attack.}

\begin{solutionbox}

{\def\LTcaptype{none} % do not increment counter
\begin{longtable}[]{@{}
  >{\raggedright\arraybackslash}p{(\linewidth - 2\tabcolsep) * \real{0.5000}}
  >{\raggedright\arraybackslash}p{(\linewidth - 2\tabcolsep) * \real{0.5000}}@{}}
\toprule\noalign{}
\begin{minipage}[b]{\linewidth}\raggedright
Attack Type
\end{minipage} & \begin{minipage}[b]{\linewidth}\raggedright
Description
\end{minipage} \\
\midrule\noalign{}
\endhead
\bottomrule\noalign{}
\endlastfoot
\textbf{Injection Attack} & Inserts malicious code into vulnerable
applications. Common types include SQL injection, command injection, and
XSS. Exploits poor input validation. Can lead to data theft,
modification, or destruction. Prevented through input sanitization and
parameterized queries. \\
\textbf{Phishing Attack} & Social engineering attack using fake
websites/emails. Attempts to steal credentials, financial information,
or install malware. Often mimics trusted organizations. Contains urgent
call-to-action to create panic. Prevented through education, email
filtering, and multi-factor authentication. \\
\end{longtable}
}

\textbf{Diagram:}

\begin{figure}
\centering
\pandocbounded{\includesvg[keepaspectratio]{diagrams/injection-vs-phishing-attacks.svg}}
\caption{Injection vs Phishing Attacks}
\end{figure}

\end{solutionbox}
\begin{mnemonicbox}
``Inject Code, Phish People''

\end{mnemonicbox}
\subsection*{Question 5(a) [3 marks]}\label{q5a}

\textbf{Explain disk forensics.}

\begin{solutionbox}
Disk forensics is the examination of storage media to
recover, analyze, and preserve digital evidence:

\begin{itemize}
\tightlist
\item
  \textbf{Purpose}: Recover deleted files, analyze file systems, and
  establish timelines
\item
  \textbf{Methods}: Bit-by-bit imaging, hash verification, and
  specialized tools
\item
  \textbf{Applications}: Criminal investigations, corporate security
  incidents, data recovery
\end{itemize}

\textbf{Diagram:}

\begin{figure}
\centering
\pandocbounded{\includesvg[keepaspectratio]{diagrams/disk-forensics-process.svg}}
\caption{Disk Forensics Process}
\end{figure}

\end{solutionbox}
\begin{mnemonicbox}
``Recover, Analyze, Present''

\end{mnemonicbox}
\subsection*{Question 5(b) [4 marks]}\label{q5b}

\textbf{Explain password cracking methods.}

\begin{solutionbox}

{\def\LTcaptype{none} % do not increment counter
\begin{longtable}[]{@{}
  >{\raggedright\arraybackslash}p{(\linewidth - 2\tabcolsep) * \real{0.3810}}
  >{\raggedright\arraybackslash}p{(\linewidth - 2\tabcolsep) * \real{0.6190}}@{}}
\toprule\noalign{}
\begin{minipage}[b]{\linewidth}\raggedright
Method
\end{minipage} & \begin{minipage}[b]{\linewidth}\raggedright
Description
\end{minipage} \\
\midrule\noalign{}
\endhead
\bottomrule\noalign{}
\endlastfoot
\textbf{Brute Force} & Tries all possible character combinations
systematically \\
\textbf{Dictionary Attack} & Uses list of common words and variations \\
\textbf{Rainbow Table} & Pre-computed tables of password hashes for
quick lookup \\
\textbf{Social Engineering} & Manipulates users to reveal passwords \\
\end{longtable}
}

\textbf{Diagram:}

\begin{figure}
\centering
\pandocbounded{\includesvg[keepaspectratio]{diagrams/password-cracking-methods.svg}}
\caption{Password Cracking Methods}
\end{figure}

\end{solutionbox}
\begin{mnemonicbox}
``BDRS: Brute Dictionary Rainbow Social''

\end{mnemonicbox}
\subsection*{Question 5(c) [7 marks]}\label{q5c}

\textbf{Describe Remote Administration Tool (RAT).}

\begin{solutionbox}
A Remote Administration Tool (RAT) is software that
enables remote control of a computer system:

{\def\LTcaptype{none} % do not increment counter
\begin{longtable}[]{@{}
  >{\raggedright\arraybackslash}p{(\linewidth - 2\tabcolsep) * \real{0.3810}}
  >{\raggedright\arraybackslash}p{(\linewidth - 2\tabcolsep) * \real{0.6190}}@{}}
\toprule\noalign{}
\begin{minipage}[b]{\linewidth}\raggedright
Aspect
\end{minipage} & \begin{minipage}[b]{\linewidth}\raggedright
Description
\end{minipage} \\
\midrule\noalign{}
\endhead
\bottomrule\noalign{}
\endlastfoot
\textbf{Functionality} & Provides complete control over target system
including file access, screen viewing, and keylogging \\
\textbf{Deployment} & Often installed through phishing, bundled with
legitimate software, or via exploited vulnerabilities \\
\textbf{Architecture} & Client-server model where server runs on
victim's machine and client is controlled by attacker \\
\textbf{Legitimate Uses} & IT support, remote work, and system
administration \\
\textbf{Malicious Uses} & Unauthorized surveillance, data theft, and
sabotage \\
\end{longtable}
}

\textbf{Diagram:}

\begin{figure}
\centering
\pandocbounded{\includesvg[keepaspectratio]{diagrams/rat-attack-architecture.svg}}
\caption{RAT Attack Architecture}
\end{figure}

\end{solutionbox}
\begin{mnemonicbox}
``RCASD: Remote Control Access Steals Data''

\end{mnemonicbox}
\subsection*{Question 5(a OR) [3
marks]}\label{question-5a-or-3-marks}

\textbf{List out challenges of cybercrime.}

\begin{solutionbox}
Major challenges in combating cybercrime include:

\begin{itemize}
\tightlist
\item
  \textbf{Jurisdiction Issues}: Crimes crossing international boundaries
\item
  \textbf{Technical Complexity}: Constantly evolving attack methods
\item
  \textbf{Attribution Problems}: Difficulty identifying perpetrators
\item
  \textbf{Evidence Collection}: Volatile and easily altered digital
  evidence
\end{itemize}

\textbf{Diagram:}

\begin{figure}
\centering
\pandocbounded{\includesvg[keepaspectratio]{diagrams/cybercrime-challenges.svg}}
\caption{Cybercrime Challenges}
\end{figure}

\end{solutionbox}
\begin{mnemonicbox}
``JTAE: Jurisdictions, Technology, Attribution,
Evidence''

\end{mnemonicbox}
\subsection*{Question 5(b OR) [4
marks]}\label{question-5b-or-4-marks}

\textbf{Explain mobile forensics.}

\begin{solutionbox}
Mobile forensics is the science of recovering digital
evidence from mobile devices:

{\def\LTcaptype{none} % do not increment counter
\begin{longtable}[]{@{}
  >{\raggedright\arraybackslash}p{(\linewidth - 2\tabcolsep) * \real{0.3810}}
  >{\raggedright\arraybackslash}p{(\linewidth - 2\tabcolsep) * \real{0.6190}}@{}}
\toprule\noalign{}
\begin{minipage}[b]{\linewidth}\raggedright
Aspect
\end{minipage} & \begin{minipage}[b]{\linewidth}\raggedright
Description
\end{minipage} \\
\midrule\noalign{}
\endhead
\bottomrule\noalign{}
\endlastfoot
\textbf{Data Types} & Call logs, messages, location data, photos, app
data \\
\textbf{Challenges} & Encryption, diverse operating systems,
anti-forensic techniques \\
\textbf{Methods} & Physical extraction, logical acquisition, file system
analysis \\
\textbf{Tools} & Cellebrite UFED, Oxygen Forensic, Magnet AXIOM \\
\end{longtable}
}

\textbf{Diagram:}

\begin{figure}
\centering
\pandocbounded{\includesvg[keepaspectratio]{diagrams/mobile-forensics-process.svg}}
\caption{Mobile Forensics Process}
\end{figure}

\end{solutionbox}
\begin{mnemonicbox}
``GEAR: Get Evidence, Analyze, Report''

\end{mnemonicbox}
\subsection*{Question 5(c OR) [7
marks]}\label{question-5c-or-7-marks}

\textbf{Explain Salami Attack, Web Jacking, Data diddling and Ransomware
attack.}

\begin{solutionbox}

{\def\LTcaptype{none} % do not increment counter
\begin{longtable}[]{@{}
  >{\raggedright\arraybackslash}p{(\linewidth - 2\tabcolsep) * \real{0.5000}}
  >{\raggedright\arraybackslash}p{(\linewidth - 2\tabcolsep) * \real{0.5000}}@{}}
\toprule\noalign{}
\begin{minipage}[b]{\linewidth}\raggedright
Attack Type
\end{minipage} & \begin{minipage}[b]{\linewidth}\raggedright
Description
\end{minipage} \\
\midrule\noalign{}
\endhead
\bottomrule\noalign{}
\endlastfoot
\textbf{Salami Attack} & Series of minor theft actions that go unnoticed
individually. Often involves modifying financial transactions by taking
small amounts. Cumulative effect can be significant over time. Example:
Rounding bank transactions and collecting fractions. \\
\textbf{Web Jacking} & Hijacking a website by changing its content or
redirecting to fake site. Involves domain theft or DNS manipulation.
Used for distributing malware or collecting sensitive information. \\
\textbf{Data Diddling} & Unauthorized modification of data before/during
input to system. Changes are typically small and hard to detect. Affects
data integrity and can lead to wrong business decisions. \\
\textbf{Ransomware} & Malware that encrypts victim's files and demands
payment for decryption. Typically spreads through phishing or exploiting
vulnerabilities. Notable examples include WannaCry and Ryuk. \\
\end{longtable}
}

\textbf{Diagram:}

\begin{figure}
\centering
\pandocbounded{\includesvg[keepaspectratio]{diagrams/cybercrime-attack-types.svg}}
\caption{Cybercrime Attack Types}
\end{figure}

\end{solutionbox}
\begin{mnemonicbox}
``SWDR: Small slices, Websites hijacked, Data
altered, Ransom demanded''

\end{mnemonicbox}

\end{document}
