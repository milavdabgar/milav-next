\documentclass[10pt,a4paper]{article}

% content/resources/templates/preamble.tex
\usepackage[margin=0.6in]{geometry}
\author{Milav Dabgar}
\usepackage{amsmath,amssymb,amsthm}
\usepackage{booktabs}
\usepackage{multirow}
\usepackage{xcolor}
\usepackage{tcolorbox}
\tcbuselibrary{breakable,skins}
\usepackage[colorlinks=true,linkcolor=blue]{hyperref}
\usepackage{titlesec}
\usepackage{enumitem}
\usepackage{tikz}
\usepackage{pgfplots}
\usepackage{circuitikz}
\usepackage[version=4]{mhchem}
\usepackage{longtable}
\usepackage{array}
\usepackage{float}
\usepackage{caption}
\usepackage{listings}

\lstset{
  basicstyle=\small\ttfamily,
  breaklines=true,
  breakatwhitespace=false,
  postbreak=\mbox{\textcolor{red}{$\hookrightarrow$}\space},
  float=false,
  numbers=left,
  numberstyle=\tiny\color{gray},
  numbersep=10pt,
  xleftmargin=2em,
  keywordstyle=\color{blue},
  commentstyle=\color{green!60!black},
  stringstyle=\color{purple},
  backgroundcolor=\color{gray!5},
  showstringspaces=false,
  tabsize=2,
  captionpos=b,
  keepspaces=true,
  columns=flexible
}

\pgfplotsset{compat=1.18}
\usetikzlibrary{shapes,arrows,positioning,calc,patterns,decorations.pathmorphing,decorations.markings,arrows.meta}

% Color scheme
\definecolor{headcolor}{RGB}{0,102,204}
\definecolor{keycolor}{RGB}{220,20,60}
\definecolor{solutioncolor}{RGB}{34,139,34}
\definecolor{mnemoniccolor}{RGB}{148,0,211}
\definecolor{codecolor}{RGB}{0,0,100}

% Spacing
\setlength{\parskip}{3pt}
\setlist[itemize]{nosep}
\setlist[enumerate]{nosep}

% Title formatting
\titleformat{\section}{\Large\bfseries\color{headcolor}}{\thesection}{1em}{}
\titleformat{\subsection}{\large\bfseries\color{headcolor}}{\thesubsection}{1em}{}

% Pandoc tightlist compatibility
\providecommand{\tightlist}{%
  \setlength{\itemsep}{0pt}\setlength{\parskip}{0pt}}

% Pandoc longtable compatibility
\newcounter{none}
\def\thenone{}


% content/resources/templates/gujarati-boxes.tex
\usepackage{fontspec}
\usepackage{polyglossia}

% Set Gujarati as main language (document is primarily in Gujarati)
% Note: gloss-gujarati.ldf doesn't exist in polyglossia, but it will use hyphenation patterns
\setdefaultlanguage{gujarati}
\setotherlanguage{english}

% Configure Gujarati font properly
% Use Language=Default to prevent polyglossia from trying to add language-specific features
% that don't exist for Gujarati, which causes "empty feature" warnings
\newfontfamily\gujaratifont[Script=Gujarati,AutoFakeBold=2.5,AutoFakeSlant=0.3]{Noto Sans Gujarati}
\setmainfont[Script=Gujarati,AutoFakeBold=2.5,AutoFakeSlant=0.3]{Noto Sans Gujarati}
% Use Noto Sans Gujarati for monospace to support Gujarati in text
\setmonofont[Scale=0.9]{Noto Sans Gujarati}

% Configure English to use the same font
\newfontfamily\englishfont[Script=Gujarati,AutoFakeBold=2.5,AutoFakeSlant=0.3]{Noto Sans Gujarati}

% Translations for polyglossia
\gappto\captionsgujarati{
  \renewcommand{\tablename}{કોષ્ટક}
  \renewcommand{\figurename}{આકૃતિ}
}

% Helper for TikZ nodes to ensure Gujarati font
\newcommand{\gu}[1]{{\gujaratifont #1}}

% Custom environments
\newtcolorbox{solutionbox}{
    breakable,
    enhanced,
    colback=solutioncolor!5!white,
    colframe=solutioncolor!75!black,
    fonttitle=\bfseries,
    title=જવાબ
}

\newtcolorbox{solutionboxnobreak}{
 colback=solutioncolor!5!white,
 colframe=solutioncolor!75!black,
 fonttitle=\bfseries,
 title=જવાબ
}

\newtcolorbox{keyformula}{
 breakable,
 enhanced,
 colback=keycolor!5!white,
 colframe=keycolor!75!black,
 fonttitle=\bfseries,
 title=રાસાયણિક સમીકરણ/સૂત્ર
}

\newtcolorbox{mnemonicbox}{
 breakable,
 enhanced,
 colback=mnemoniccolor!5!white,
 colframe=mnemoniccolor!75!black,
 fonttitle=\bfseries,
 title=મેમરી ટ્રીક
}


\begin{document}

\begin{center}
{\Huge\bfseries\color{headcolor} Subject Name (Gujarati)}\\[5pt]
{\LARGE 4353206 -- Summer 2025}\\[3pt]
{\large Semester 1 Study Material}\\[3pt]
{\normalsize\textit{Detailed Solutions and Explanations}}
\end{center}

\vspace{10pt}

\subsection*{પ્રશ્ન 1(અ) [3
ગુણ]}\label{uxaaauxab0uxab6uxaa8-1uxa85-3-uxa97uxaa3}

\textbf{n-ચેનલ MOSFET ના ભૌતિક બંધારણનું સાફ લેબલવાળું આકૃતિ દોરો.}

\begin{solutionbox}

\textbf{આકૃતિ:}

\begin{verbatim}
           Gate (G)
              |
    +{-{-}{-}{-}{-}{-}{-}{-}{-}+{-}{-}{-}{-}{-}{-}{-}{-}{-}{-}+}
    |    SiO2 (oxide)    |
+{-{-}{-}+{-}{-}{-}{-}{-}{-}{-}{-}{-}{-}{-}{-}{-}{-}{-}{-}{-}{-}{-}{-}+{-}{-}{-}+}
|S  |                    | D |
|o  |  p{-type substrate  | r |}
|u  |                    | a |
|r  |     n+      n+     | i |
|c  |    {-{-}{-}{-}    {-}{-}{-}{-}    | n |  }
|e  |   Source   Drain   |   |
+{-{-}{-}+{-}{-}{-}{-}{-}{-}{-}{-}{-}{-}{-}{-}{-}{-}{-}{-}{-}{-}{-}{-}+{-}{-}{-}+}
              |
           Substrate/Body
\end{verbatim}

\textbf{મુખ્ય ઘટકો:}

\begin{itemize}
\tightlist
\item
  \textbf{સોર્સ}: n+ ડોપ્ડ વિસ્તાર જે ઈલેક્ટ્રોન પૂરો પાડે છે
\item
  \textbf{ડ્રેઇન}: n+ ડોપ્ડ વિસ્તાર જે ઈલેક્ટ્રોન એકત્ર કરે છે
\item
  \textbf{ગેટ}: ચેનલને નિયંત્રિત કરતું મેટલ ઈલેક્ટ્રોડ
\item
  \textbf{ઓક્સાઇડ}: SiO2 ઇન્સ્યુલેટિંગ સ્તર
\item
  \textbf{સબ્સ્ટ્રેટ}: p-ટાઇપ સિલિકોન બોડી
\end{itemize}

\end{solutionbox}
\begin{mnemonicbox}
``SOGD - સોર્સ, ઓક્સાઇડ, ગેટ, ડ્રેઇન''

\end{mnemonicbox}
\subsection*{પ્રશ્ન 1(બ) [4
ગુણ]}\label{uxaaauxab0uxab6uxaa8-1uxaac-4-uxa97uxaa3}

\textbf{એક્સટર્નલ બાયાસ હેઠળ MOS ના ડિપ્લીશન અને ઇન્વર્શનનું એનર્જી બેન્ડ ડાયાગ્રામ
MOS બાયાસિંગ ડાયાગ્રામ સાથે દોરો. ઇન્વર્શન રીજનને વિગતવાર સમજાવો.}

\begin{solutionbox}

\textbf{MOS બાયાસિંગ સર્કિટ:}

\begin{verbatim}
    VG
     |
     |    Gate
    +++++++++++
    |  SiO2   |
    +{-{-}{-}{-}{-}{-}{-}{-}{-}+}
    | p{-type  |}
    +{-{-}{-}{-}{-}{-}{-}{-}{-}+}
         |
        VB
\end{verbatim}

\textbf{એનર્જી બેન્ડ ડાયાગ્રામ:}

{\def\LTcaptype{none} % do not increment counter
\begin{longtable}[]{@{}ll@{}}
\toprule\noalign{}
બાયાસ સ્થિતિ & એનર્જી બેન્ડ વર્તન \\
\midrule\noalign{}
\endhead
\bottomrule\noalign{}
\endlastfoot
\textbf{ડિપ્લીશન} & બેન્ડ ઉપરની તરફ વળે છે, હોલ્સ ખતમ થાય છે \\
\textbf{ઇન્વર્શન} & મજબૂત બેન્ડ બેન્ડિંગ, ઈલેક્ટ્રોન ચેનલ બને છે \\
\end{longtable}
}

\textbf{ઇન્વર્શન રીજન વિગતો:}

\begin{itemize}
\tightlist
\item
  \textbf{મજબૂત ઇન્વર્શન}: VG \textgreater{} VT (થ્રેશોલ્ડ વોલ્ટેજ)
\item
  \textbf{ઈલેક્ટ્રોન ચેનલ}: Si-SiO2 ઇન્ટરફેસ પર બને છે
\item
  \textbf{ચેનલ કન્ડક્ટિવિટી}: ગેટ વોલ્ટેજ સાથે વધે છે
\item
  \textbf{થ્રેશોલ્ડ શરત}: સરફેસ પોટેન્શિયલ = 2φF
\end{itemize}

\end{solutionbox}
\begin{mnemonicbox}
``DIVE - ડિપ્લીશન, ઇન્વર્શન, વોલ્ટેજ, ઈલેક્ટ્રોન્સ''

\end{mnemonicbox}
\subsection*{પ્રશ્ન 1(ક) [7
ગુણ]}\label{uxaaauxab0uxab6uxaa8-1uxa95-7-uxa97uxaa3}

\textbf{MOSFET ની I-V લાક્ષણિકતા સમજાવો.}

\begin{solutionbox}

\textbf{I-V લાક્ષણિકતા વિસ્તારો:}

{\def\LTcaptype{none} % do not increment counter
\begin{longtable}[]{@{}
  >{\raggedright\arraybackslash}p{(\linewidth - 4\tabcolsep) * \real{0.2353}}
  >{\raggedright\arraybackslash}p{(\linewidth - 4\tabcolsep) * \real{0.3235}}
  >{\raggedright\arraybackslash}p{(\linewidth - 4\tabcolsep) * \real{0.4412}}@{}}
\toprule\noalign{}
\begin{minipage}[b]{\linewidth}\raggedright
વિસ્તાર
\end{minipage} & \begin{minipage}[b]{\linewidth}\raggedright
શરત
\end{minipage} & \begin{minipage}[b]{\linewidth}\raggedright
ડ્રેઇન કરંટ
\end{minipage} \\
\midrule\noalign{}
\endhead
\bottomrule\noalign{}
\endlastfoot
\textbf{કટઓફ} & VGS \textless{} VT & ID \approx 0 \\
\textbf{લિનિયર} & VGS \textgreater{} VT, VDS \textless{} VGS-VT & ID =
μnCox(W/L)[(VGS-VT)VDS - VDS^{2}/2] \\
\textbf{સેચ્યુરેશન} & VGS \textgreater{} VT, VDS \geq VGS-VT & ID =
(μnCox/2)(W/L)(VGS-VT)^{2} \\
\end{longtable}
}

\textbf{લાક્ષણિકતા વક્ર:}

\begin{verbatim}
    ID
     |
     |     Saturation
     |    +{-{-}{-}{-}{-}{-}{-}{-}{-}}
     |   /
     |  / Linear
     | /
     |/
  {-{-}{-}+{-}{-}{-}{-}{-}{-}{-}{-}{-}{-}{-} VDS}
     0    VGS{-VT}
\end{verbatim}

\textbf{મુખ્ય પેરામીટર્સ:}

\begin{itemize}
\tightlist
\item
  \textbf{μn}: ઈલેક્ટ્રોન મોબિલિટી
\item
  \textbf{Cox}: ગેટ ઓક્સાઇડ કેપેસિટન્સ
\item
  \textbf{W/L}: પહોળાઈ અને લંબાઈનો રેશિયો
\item
  \textbf{VT}: થ્રેશોલ્ડ વોલ્ટેજ
\end{itemize}

\textbf{ઓપરેટિંગ મોડ્સ:}

\begin{itemize}
\tightlist
\item
  \textbf{એન્હાન્સમેન્ટ}: પોઝિટિવ VGS સાથે ચેનલ બને છે
\item
  \textbf{સ્ક્વેર લો}: સેચ્યુરેશન વિસ્તાર ચતુર્ભુજ સંબંધ અનુસરે છે
\end{itemize}

\end{solutionbox}
\begin{mnemonicbox}
``CLS - કટઓફ, લિનિયર, સેચ્યુરેશન''

\end{mnemonicbox}
\subsection*{પ્રશ્ન 1(ક) OR [7
ગુણ]}\label{uxaaauxab0uxab6uxaa8-1uxa95-or-7-uxa97uxaa3}

\textbf{સ્કેલિંગ વ્યાખ્યાયિત કરો. સ્કેલિંગની જરૂરિયાત સમજાવો. સ્કેલિંગની નકારાત્મક
અસરોની સૂચિ બનાવો અને સમજાવો.}

\begin{solutionbox}

\textbf{વ્યાખ્યા:} \textbf{સ્કેલિંગ} એટલે પર્ફોર્મન્સ અને ડેન્સિટી સુધારવા માટે MOSFET
ના પરિમાણોમાં વ્યવસ્થિત ઘટાડો.

\textbf{સ્કેલિંગની જરૂરિયાત:}

{\def\LTcaptype{none} % do not increment counter
\begin{longtable}[]{@{}ll@{}}
\toprule\noalign{}
ફાયદો & વર્ણન \\
\midrule\noalign{}
\endhead
\bottomrule\noalign{}
\endlastfoot
\textbf{વધુ ડેન્સિટી} & ચિપ વિસ્તાર દીઠ વધુ ટ્રાન્ઝિસ્ટર \\
\textbf{ઝડપી સ્પીડ} & ઘટેલી ગેટ ડીલે \\
\textbf{ઓછી પાવર} & ઘટેલી સ્વિચિંગ એનર્જી \\
\textbf{કોસ્ટ રિડક્શન} & વેફર દીઠ વધુ ચિપ્સ \\
\end{longtable}
}

\textbf{સ્કેલિંગ પ્રકારો:}

{\def\LTcaptype{none} % do not increment counter
\begin{longtable}[]{@{}llll@{}}
\toprule\noalign{}
પ્રકાર & ગેટ લંબાઈ & સપ્લાય વોલ્ટેજ & ઓક્સાઇડ જાડાઈ \\
\midrule\noalign{}
\endhead
\bottomrule\noalign{}
\endlastfoot
\textbf{કોન્સ્ટન્ટ વોલ્ટેજ} & ↓α & સ્થિર & ↓α \\
\textbf{કોન્સ્ટન્ટ ફીલ્ડ} & ↓α & ↓α & ↓α \\
\end{longtable}
}

\textbf{નકારાત્મક અસરો:}

\begin{itemize}
\tightlist
\item
  \textbf{શોર્ટ ચેનલ અસરો}: થ્રેશોલ્ડ વોલ્ટેજ રોલ-ઓફ
\item
  \textbf{હોટ કેરિયર અસરો}: ડિવાઇસ ડીગ્રેડેશન
\item
  \textbf{ગેટ લીકેજ}: વધેલો ટનલિંગ કરંટ
\item
  \textbf{પ્રોસેસ વેરિએશન્સ}: મેન્યુફેક્ચરિંગ પડકારો
\item
  \textbf{પાવર ડેન્સિટી}: હીટ ડિસિપેશન સમસ્યાઓ
\end{itemize}

\end{solutionbox}
\begin{mnemonicbox}
``SHGPP - શોર્ટ ચેનલ, હોટ કેરિયર, ગેટ લીકેજ, પ્રોસેસ,
પાવર''

\end{mnemonicbox}
\subsection*{પ્રશ્ન 2(અ) [3
ગુણ]}\label{uxaaauxab0uxab6uxaa8-2uxa85-3-uxa97uxaa3}

\textbf{CMOS નો ઉપયોગ કરીને Y' = (AB' + A'B) અમલમાં મૂકો.}

\begin{solutionbox}

\textbf{લોજિક વિશ્લેષણ:} Y' = (AB' + A'B) = A \oplus B (XOR ફંક્શન)

\textbf{CMOS અમલીકરણ:}

\begin{verbatim}
    VDD
     |
   +{-+{-}+   +{-}+{-}+}
   |pA |   |pB |
   +{-{-}{-}+   +{-}{-}{-}+}
     |       |
     +{-{-}{-}Y{-}{-}{-}+}
     |       |
   +{-{-}{-}+   +{-}{-}{-}+}
   |nA |   |nB{|}
   +{-+{-}+   +{-}+{-}+}
     |       |
    GND     GND
\end{verbatim}

\textbf{ટ્રુથ ટેબલ:}

{\def\LTcaptype{none} % do not increment counter
\begin{longtable}[]{@{}lllll@{}}
\toprule\noalign{}
A & B & AB' & A'B & Y' \\
\midrule\noalign{}
\endhead
\bottomrule\noalign{}
\endlastfoot
0 & 0 & 0 & 0 & 1 \\
0 & 1 & 0 & 1 & 0 \\
1 & 0 & 1 & 0 & 0 \\
1 & 1 & 0 & 0 & 1 \\
\end{longtable}
}

\end{solutionbox}
\begin{mnemonicbox}
``XOR ને કોમ્પ્લિમેન્ટરી સ્વિચિંગ જોઈએ''

\end{mnemonicbox}
\subsection*{પ્રશ્ન 2(બ) [4
ગુણ]}\label{uxaaauxab0uxab6uxaa8-2uxaac-4-uxa97uxaa3}

\textbf{એન્હાન્સમેન્ટ લોડ ઇન્વર્ટરને તેના સર્કિટ ડાયાગ્રામ સાથે સમજાવો.}

\begin{solutionbox}

\textbf{સર્કિટ ડાયાગ્રામ:}

\begin{verbatim}
    VDD
     |
     +{-{-}{-}o VG2}
     |
   +{-+{-}+ Enhancement}
   |ME |  Load
   +{-{-}{-}+}
     |
     +{-{-}{-}o Vout}
     |
   +{-+{-}+}
   |MD | Driver
   +{-{-}{-}+}
     |
    GND
     |
     +{-{-}{-}o Vin}
\end{verbatim}

\textbf{કન્ફિગરેશન:}

{\def\LTcaptype{none} % do not increment counter
\begin{longtable}[]{@{}lll@{}}
\toprule\noalign{}
ઘટક & પ્રકાર & કનેક્શન \\
\midrule\noalign{}
\endhead
\bottomrule\noalign{}
\endlastfoot
\textbf{લોડ (ME)} & એન્હાન્સમેન્ટ NMOS & ગેટ VDD સાથે જોડાયેલું \\
\textbf{ડ્રાઇવર (MD)} & એન્હાન્સમેન્ટ NMOS & ગેટ ઇનપુટ છે \\
\end{longtable}
}

\textbf{ઓપરેશન:}

\begin{itemize}
\tightlist
\item
  \textbf{લોડ ટ્રાન્ઝિસ્ટર}: એક્ટિવ લોડ રેઝિસ્ટર તરીકે કામ કરે છે
\item
  \textbf{હાઇ આઉટપુટ}: લોડ ટ્રાન્ઝિસ્ટરના VT દ્વારા મર્યાદિત
\item
  \textbf{લો આઉટપુટ}: ડ્રાઇવરની તાકાત પર આધાર રાખે છે
\item
  \textbf{ગેરફાયદો}: થ્રેશોલ્ડ ડ્રોપ કારણે ખરાબ VOH
\end{itemize}

\textbf{ટ્રાન્સફર લાક્ષણિકતાઓ:}

\begin{itemize}
\tightlist
\item
  \textbf{VOH}: VDD - VT (બગડેલો હાઇ લેવલ)
\item
  \textbf{VOL}: ગ્રાઉન્ડ પોટેન્શિયલની નજીક
\item
  \textbf{નોઇઝ માર્જિન}: થ્રેશોલ્ડ લોસ કારણે ઘટેલો
\end{itemize}

\end{solutionbox}
\begin{mnemonicbox}
``ELI - એન્હાન્સમેન્ટ લોડ ઇન્વર્ટરમાં થ્રેશોલ્ડ સમસ્યા''

\end{mnemonicbox}
\subsection*{પ્રશ્ન 2(ક) [7
ગુણ]}\label{uxaaauxab0uxab6uxaa8-2uxa95-7-uxa97uxaa3}

\textbf{ઇન્વર્ટરની વોલ્ટેજ ટ્રાન્સફર કેરેક્ટરિસ્ટિક સમજાવો.}

\begin{solutionbox}

\textbf{VTC પેરામીટર્સ:}

{\def\LTcaptype{none} % do not increment counter
\begin{longtable}[]{@{}lll@{}}
\toprule\noalign{}
પેરામીટર & વર્ણન & આદર્શ કિંમત \\
\midrule\noalign{}
\endhead
\bottomrule\noalign{}
\endlastfoot
\textbf{VOH} & આઉટપુટ હાઇ વોલ્ટેજ & VDD \\
\textbf{VOL} & આઉટપુટ લો વોલ્ટેજ & 0V \\
\textbf{VIH} & ઇનપુટ હાઇ વોલ્ટેજ & VDD/2 \\
\textbf{VIL} & ઇનપુટ લો વોલ્ટેજ & VDD/2 \\
\textbf{VM} & સ્વિચિંગ થ્રેશોલ્ડ & VDD/2 \\
\end{longtable}
}

\textbf{VTC વક્ર:}

\begin{verbatim}
   Vout
     |
   VDD+         +{-{-}{-}{-}{-}{-}{-}}
     |         /
     |        /
   VM+{-{-}{-}{-}{-}{-}{-}+  VM}
     |        {}
     |         {}
     0          +{-{-}{-}{-}{-}{-}{-}}
     +{-{-}{-}{-}+{-}{-}{-}{-}+{-}{-}{-}{-}{-} Vin}
      VIL VM  VIH   VDD
\end{verbatim}

\textbf{નોઇઝ માર્જિન્સ:}

\begin{itemize}
\tightlist
\item
  \textbf{NMH} = VOH - VIH (હાઇ નોઇઝ માર્જિન)
\item
  \textbf{NML} = VIL - VOL (લો નોઇઝ માર્જિન)
\end{itemize}

\textbf{વિસ્તારો:}

\begin{itemize}
\tightlist
\item
  \textbf{વિસ્તાર 1}: ઇનપુટ લો, આઉટપુટ હાઇ
\item
  \textbf{વિસ્તાર 2}: ટ્રાન્ઝિશન વિસ્તાર
\item
  \textbf{વિસ્તાર 3}: ઇનપુટ હાઇ, આઉટપુટ લો
\end{itemize}

\textbf{ગુણવત્તા મેટ્રિક્સ:}

\begin{itemize}
\tightlist
\item
  \textbf{તીક્ષ્ણ ટ્રાન્ઝિશન}: બહેતર નોઇઝ ઇમ્યુનિટી
\item
  \textbf{સિમેટ્રિક સ્વિચિંગ}: VM = VDD/2
\item
  \textbf{ફુલ સ્વિંગ}: VOH = VDD, VOL = 0
\end{itemize}

\end{solutionbox}
\begin{mnemonicbox}
``VTC દર્શાવે છે VOICE - VOH, VOL, ઇનપુટ થ્રેશોલ્ડ,
લાક્ષણિકતા, બધું''

\end{mnemonicbox}
\subsection*{પ્રશ્ન 2(અ) OR [3
ગુણ]}\label{uxaaauxab0uxab6uxaa8-2uxa85-or-3-uxa97uxaa3}

\textbf{CMOS નો ઉપયોગ કરીને NAND2 ગેટ સમજાવો.}

\begin{solutionbox}

\textbf{CMOS NAND2 સર્કિટ:}

\begin{verbatim}
       VDD
        |
    +{-{-}{-}+{-}{-}{-}+}
    |       |
  +{-+{-}+   +{-}+{-}+}
  |pA |   |pB | PMOS
  +{-{-}{-}+   +{-}{-}{-}+ (Parallel)}
    |       |
    +{-{-}{-}Y{-}{-}{-}+}
        |
      +{-+{-}+}
      |nA | NMOS
      +{-{-}{-}+ (Series)}
        |
      +{-+{-}+}
      |nB |
      +{-{-}{-}+}
        |
       GND
\end{verbatim}

\textbf{ટ્રુથ ટેબલ:}

{\def\LTcaptype{none} % do not increment counter
\begin{longtable}[]{@{}lll@{}}
\toprule\noalign{}
A & B & Y \\
\midrule\noalign{}
\endhead
\bottomrule\noalign{}
\endlastfoot
0 & 0 & 1 \\
0 & 1 & 1 \\
1 & 0 & 1 \\
1 & 1 & 0 \\
\end{longtable}
}

\textbf{ઓપરેશન:}

\begin{itemize}
\tightlist
\item
  \textbf{PMOS નેટવર્ક}: પેરેલલ કનેક્શન (પુલ-અપ)
\item
  \textbf{NMOS નેટવર્ક}: સીરીઝ કનેક્શન (પુલ-ડાઉન)
\item
  \textbf{આઉટપુટ લો}: ફક્ત જ્યારે બંને ઇનપુટ હાઇ
\end{itemize}

\end{solutionbox}
\begin{mnemonicbox}
``NAND - નોટ AND, પેરેલલ PMOS, સીરીઝ NMOS''

\end{mnemonicbox}
\subsection*{પ્રશ્ન 2(બ) OR [4
ગુણ]}\label{uxaaauxab0uxab6uxaa8-2uxaac-or-4-uxa97uxaa3}

\textbf{રેઝિસ્ટિવ લોડ ઇન્વર્ટર સર્કિટના ઓપરેટિંગ મોડ અને VTC સમજાવો.}

\begin{solutionbox}

\textbf{સર્કિટ કન્ફિગરેશન:}

\begin{verbatim}
    VDD
     |
     R (Load Resistor)
     |
     +{-{-}{-}o Vout}
     |
   +{-+{-}+}
   |MN | NMOS Driver
   +{-{-}{-}+}
     |
    GND
     |
     +{-{-}{-}o Vin}
\end{verbatim}

\textbf{ઓપરેટિંગ મોડ્સ:}

{\def\LTcaptype{none} % do not increment counter
\begin{longtable}[]{@{}lll@{}}
\toprule\noalign{}
ઇનપુટ સ્થિતિ & NMOS સ્થિતિ & આઉટપુટ \\
\midrule\noalign{}
\endhead
\bottomrule\noalign{}
\endlastfoot
\textbf{Vin = 0} & OFF & VOH = VDD \\
\textbf{Vin = VDD} & ON & VOL = R·ID/(R+RDS) \\
\end{longtable}
}

\textbf{VTC લાક્ષણિકતાઓ:}

\begin{itemize}
\tightlist
\item
  \textbf{VOH}: ઉત્તમ (VDD)
\item
  \textbf{VOL}: R અને RDS રેશિયો પર આધાર રાખે છે
\item
  \textbf{પાવર વપરાશ}: ઇનપુટ હાઇ હોય ત્યારે સ્ટેટિક કરંટ
\item
  \textbf{ટ્રાન્ઝિશન}: રેઝિસ્ટિવ લોડ કારણે ધીમું
\end{itemize}

\textbf{ડિઝાઇન ટ્રેડ-ઓફ્સ:}

\begin{itemize}
\tightlist
\item
  \textbf{મોટો R}: બહેતર VOL, ધીમું સ્વિચિંગ
\item
  \textbf{નાનો R}: ઝડપી સ્વિચિંગ, વધુ પાવર
\item
  \textbf{એરિયા}: રેઝિસ્ટર નોંધપાત્ર જગ્યા લે છે
\end{itemize}

\end{solutionbox}
\begin{mnemonicbox}
``RLI - રેઝિસ્ટિવ લોડમાં અનિવાર્ય પાવર વપરાશ''

\end{mnemonicbox}
\subsection*{પ્રશ્ન 2(ક) OR [7
ગુણ]}\label{uxaaauxab0uxab6uxaa8-2uxa95-or-7-uxa97uxaa3}

\textbf{CMOS ઇન્વર્ટર દોરો અને VTC સાથે તેની કામગીરી સમજાવો.}

\begin{solutionbox}

\textbf{CMOS ઇન્વર્ટર સર્કિટ:}

\begin{verbatim}
       VDD
        |
      +{-+{-}+}
      |MP | PMOS
      +{-{-}{-}+}
        |
        +{-{-}{-}o Vout}
        |
      +{-+{-}+}
      |MN | NMOS  
      +{-{-}{-}+}
        |
       GND
        |
        +{-{-}{-}o Vin}
\end{verbatim}

\textbf{ઓપરેશન વિસ્તારો:}

{\def\LTcaptype{none} % do not increment counter
\begin{longtable}[]{@{}lllll@{}}
\toprule\noalign{}
Vin રેન્જ & PMOS & NMOS & Vout & વિસ્તાર \\
\midrule\noalign{}
\endhead
\bottomrule\noalign{}
\endlastfoot
\textbf{0 થી VTN} & ON & OFF & VDD & 1 \\
**VTN થી VDD- & VTP & ** & ON & ON \\
**VDD- & VTP & થી VDD** & OFF & ON \\
\end{longtable}
}

\textbf{VTC વિશ્લેષણ:}

\begin{verbatim}
   Vout
     |
   VDD+
     |{}
     | {}
     |  {\_\_\_}
   VM+    {\_\_\_}
     |        {\_\_\_}
     |            {}
     0             +{-{-}{-}}
     +{-{-}{-}{-}+{-}{-}{-}{-}+{-}{-}{-}{-}+{-}{-}{-} Vin}
        VTN  VM  VTP  VDD
\end{verbatim}

\textbf{મુખ્ય લક્ષણો:}

\begin{itemize}
\tightlist
\item
  \textbf{ઝીરો સ્ટેટિક પાવર}: કોઈ DC કરંટ પાથ નથી
\item
  \textbf{ફુલ સ્વિંગ}: VOH = VDD, VOL = 0V
\item
  \textbf{હાઇ નોઇઝ માર્જિન્સ}: NMH = NML \approx 0.4VDD
\item
  \textbf{તીક્ષ્ણ ટ્રાન્ઝિશન}: ટ્રાન્ઝિશન વિસ્તારમાં હાઇ ગેઇન
\end{itemize}

\textbf{ડિઝાઇન વિચારણાઓ:}

\begin{itemize}
\tightlist
\item
  \textbf{β રેશિયો}: સિમેટ્રિક સ્વિચિંગ માટે βN/βP
\item
  \textbf{થ્રેશોલ્ડ મેચિંગ}: VTN \approx \textbar VTP\textbar{} પસંદીદા
\end{itemize}

\end{solutionbox}
\begin{mnemonicbox}
``CMOS માં ઝીરો સ્ટેટિક પાવર અને ફુલ સ્વિંગ''

\end{mnemonicbox}
\subsection*{પ્રશ્ન 3(અ) [3
ગુણ]}\label{uxaaauxab0uxab6uxaa8-3uxa85-3-uxa97uxaa3}

\textbf{ડિપ્લીશન લોડનો ઉપયોગ કરીને Y= (A̅+B̅)C̅+D̅+E̅ અમલમાં મૂકો.}

\begin{solutionbox}

\textbf{લોજિક સરળીકરણ:} Y = (A̅+B̅)C̅+D̅+E̅ = A̅C̅+B̅C̅+D̅+E̅

\textbf{ડિપ્લીશન લોડ અમલીકરણ:}

\begin{verbatim}
       VDD
        |
      +{-+{-}+ VGS=0}
      |MD | Depletion
      +{-{-}{-}+ Load}
        |
        +{-{-}{-}o Y}
        |
    +{-{-}{-}+{-}{-}{-}+{-}{-}{-}+{-}{-}{-}+}
    |   |   |   |   |
  +{-+{-}{-}+{-}+{-}+{-}+{-}+{-}+{-}+{-}+{-}+}
  |A{| |B| |C| |D| |E| Pull{-}down}
  +{-{-}+ +{-}{-}+ +{-}{-}+ +{-}{-}+ +{-}{-}+ Network}
    |   |   |   |   |
   GND GND GND GND GND
\end{verbatim}

\textbf{પુલ-ડાઉન નેટવર્ક:}

\begin{itemize}
\tightlist
\item
  \textbf{સીરીઝ}: A̅C̅ પાથ અને B̅C̅ પાથ
\item
  \textbf{પેરેલલ}: બધા પાથ પેરેલલમાં જોડાયેલા
\item
  \textbf{અમલીકરણ}: યોગ્ય ટ્રાન્ઝિસ્ટર સાઇઝિંગ જરૂરી
\end{itemize}

\end{solutionbox}
\begin{mnemonicbox}
``ડિપ્લીશન લોડ પેરેલલ પુલ-ડાઉન પાથ સાથે''

\end{mnemonicbox}
\subsection*{પ્રશ્ન 3(બ) [4
ગુણ]}\label{uxaaauxab0uxab6uxaa8-3uxaac-4-uxa97uxaa3}

\textbf{FPGA પર ટૂંકી નોંધ લખો.}

\begin{solutionbox}

\textbf{FPGA વ્યાખ્યા:} \textbf{ફીલ્ડ પ્રોગ્રામેબલ ગેટ એરે} - રીકન્ફિગરેબલ
ઇન્ટિગ્રેટેડ સર્કિટ.

\textbf{આર્કિટેક્ચર ઘટકો:}

{\def\LTcaptype{none} % do not increment counter
\begin{longtable}[]{@{}ll@{}}
\toprule\noalign{}
ઘટક & કાર્ય \\
\midrule\noalign{}
\endhead
\bottomrule\noalign{}
\endlastfoot
\textbf{CLB} & કન્ફિગરેબલ લોજિક બ્લોક \\
\textbf{IOB} & ઇનપુટ/આઉટપુટ બ્લોક \\
\textbf{ઇન્ટરકનેક્ટ} & રાઉટિંગ રિસોર્સ \\
\textbf{સ્વિચ મેટ્રિક્સ} & કનેક્શન પોઇન્ટ્સ \\
\end{longtable}
}

\textbf{પ્રોગ્રામિંગ ટેક્નોલોજીઝ:}

\begin{itemize}
\tightlist
\item
  \textbf{SRAM-આધારિત}: વોલેટાઇલ, ઝડપી રીકન્ફિગરેશન
\item
  \textbf{એન્ટીફ્યુઝ}: નોન-વોલેટાઇલ, એક વખતનું પ્રોગ્રામેબલ
\item
  \textbf{ફ્લેશ-આધારિત}: નોન-વોલેટાઇલ, રીપ્રોગ્રામેબલ
\end{itemize}

\textbf{એપ્લિકેશન્સ:}

\begin{itemize}
\tightlist
\item
  \textbf{પ્રોટોટાઇપિંગ}: ડિજિટલ સિસ્ટમ ડેવલપમેન્ટ
\item
  \textbf{DSP}: સિગ્નલ પ્રોસેસિંગ એપ્લિકેશન્સ
\item
  \textbf{કંટ્રોલ સિસ્ટમ્સ}: ઇન્ડસ્ટ્રિયલ ઓટોમેશન
\item
  \textbf{કોમ્યુનિકેશન્સ}: પ્રોટોકોલ અમલીકરણ
\end{itemize}

\textbf{ASIC સામે ફાયદા:}

\begin{itemize}
\tightlist
\item
  \textbf{લવચીકતા}: રીકન્ફિગરેબલ ડિઝાઇન
\item
  \textbf{ટાઇમ-ટુ-માર્કેટ}: ઝડપી વિકાસ
\item
  \textbf{કોસ્ટ}: નાના વોલ્યુમ માટે ઓછો
\item
  \textbf{જોખમ}: ઘટેલો ડિઝાઇન જોખમ
\end{itemize}

\end{solutionbox}
\begin{mnemonicbox}
``FPGA - લવચીક પ્રોગ્રામિંગ ફાયદા આપે છે''

\end{mnemonicbox}
\subsection*{પ્રશ્ન 3(ક) [7
ગુણ]}\label{uxaaauxab0uxab6uxaa8-3uxa95-7-uxa97uxaa3}

\textbf{Y ચાર્ટ ડિઝાઇન ફ્લો દોરો અને સમજાવો.}

\begin{solutionbox}

\textbf{Y-ચાર્ટ ડાયાગ્રામ:}

\begin{verbatim}
graph TB
    subgraph "Behavioral Domain"
        B1[Algorithm]
        B2[Register Transfer]
        B3[Boolean Equations]
    end

    subgraph "Structural Domain"
        S1[Processor]
        S2[ALU, Register]
        S3[Gates]
    end
    
    subgraph "Physical Domain"
        P1[Floor Plan]
        P2[Module Layout]
        P3[Cell Layout]
    end
    
    B1 {-{-} S1}
    B2 {-{-} S2}
    B3 {-{-} S3}
    S1 {-{-} P1}
    S2 {-{-} P2}
    S3 {-{-} P3}
\end{verbatim}

\textbf{ડિઝાઇન ડોમેઇન્સ:}

{\def\LTcaptype{none} % do not increment counter
\begin{longtable}[]{@{}lll@{}}
\toprule\noalign{}
ડોમેઇન & લેવલ & વર્ણન \\
\midrule\noalign{}
\endhead
\bottomrule\noalign{}
\endlastfoot
\textbf{બિહેવિયરલ} & એલ્ગોરિધમ \rightarrow RT \rightarrow બુલિયન & સિસ્ટમ શું કરે છે \\
\textbf{સ્ટ્રક્ચરલ} & પ્રોસેસર \rightarrow ALU \rightarrow ગેટ્સ & સિસ્ટમ કેવી રીતે બનાવેલ છે \\
\textbf{ફિઝિકલ} & ફ્લોર પ્લાન \rightarrow લેઆઉટ \rightarrow સેલ્સ & ભૌતિક અમલીકરણ \\
\end{longtable}
}

\textbf{ડિઝાઇન ફ્લો પ્રક્રિયા:}

\begin{itemize}
\tightlist
\item
  \textbf{ટોપ-ડાઉન}: બિહેવિયરલથી શરૂ કરી ફિઝિકલ તરફ જાઓ
\item
  \textbf{બોટમ-અપ}: ઘટકોથી ઉપરની તરફ બનાવો
\item
  \textbf{મિક્સ્ડ એપ્રોચ}: બંનેની સંયુક્ત પદ્ધતિ
\end{itemize}

\textbf{એબ્સ્ટ્રેક્શન લેવલ્સ:}

\begin{itemize}
\tightlist
\item
  \textbf{સિસ્ટમ લેવલ}: સૌથી વધુ એબ્સ્ટ્રેક્શન
\item
  \textbf{RT લેવલ}: રજિસ્ટર ટ્રાન્સફર ઓપરેશન્સ
\item
  \textbf{ગેટ લેવલ}: બુલિયન લોજિક અમલીકરણ
\item
  \textbf{લેઆઉટ લેવલ}: ભૌતિક જ્યોમેટ્રી
\end{itemize}

\textbf{ડિઝાઇન વેરિફિકેશન:}

\begin{itemize}
\tightlist
\item
  \textbf{હોરિઝોન્ટલ}: સમાન લેવલે ડોમેઇન્સ વચ્ચે
\item
  \textbf{વર્ટિકલ}: સમાન ડોમેઇનમાં લેવલ્સ વચ્ચે
\end{itemize}

\end{solutionbox}
\begin{mnemonicbox}
``Y-ચાર્ટ: બિહેવિયરલ, સ્ટ્રક્ચરલ, ફિઝિકલ - BSP ડોમેઇન્સ''

\end{mnemonicbox}
\subsection*{પ્રશ્ન 3(અ) OR [3
ગુણ]}\label{uxaaauxab0uxab6uxaa8-3uxa85-or-3-uxa97uxaa3}

\textbf{ડિપ્લીશન લોડનો ઉપયોગ કરીને NOR2 ગેટ સમજાવો.}

\begin{solutionbox}

\textbf{ડિપ્લીશન લોડ NOR2 સર્કિટ:}

\begin{verbatim}
       VDD
        |
      +{-+{-}+ VGS=0}
      |MD | Depletion  
      +{-{-}{-}+ Load}
        |
        +{-{-}{-}o Y}
        |
    +{-{-}{-}+{-}{-}{-}+}
    |       |
  +{-+{-}+   +{-}+{-}+}
  |nA |   |nB | NMOS
  +{-{-}{-}+   +{-}{-}{-}+ (Parallel)}
    |       |
   GND     GND
\end{verbatim}

\textbf{ટ્રુથ ટેબલ:}

{\def\LTcaptype{none} % do not increment counter
\begin{longtable}[]{@{}lll@{}}
\toprule\noalign{}
A & B & Y \\
\midrule\noalign{}
\endhead
\bottomrule\noalign{}
\endlastfoot
0 & 0 & 1 \\
0 & 1 & 0 \\
1 & 0 & 0 \\
1 & 1 & 0 \\
\end{longtable}
}

\textbf{ઓપરેશન:}

\begin{itemize}
\tightlist
\item
  \textbf{બંને ઇનપુટ લો}: બંને NMOS OFF, Y = VDD
\item
  \textbf{કોઈપણ ઇનપુટ હાઇ}: સંબંધિત NMOS ON, Y = VOL
\item
  \textbf{લોડ ટ્રાન્ઝિસ્ટર}: પુલ-અપ કરંટ પૂરો પાડે છે
\end{itemize}

\end{solutionbox}
\begin{mnemonicbox}
``ડિપ્લીશન સાથે NOR - પેરેલલ NMOS પુલ-ડાઉન''

\end{mnemonicbox}
\subsection*{પ્રશ્ન 3(બ) OR [4
ગુણ]}\label{uxaaauxab0uxab6uxaa8-3uxaac-or-4-uxa97uxaa3}

\textbf{ફુલ કસ્ટમ અને સેમી કસ્ટમ ડિઝાઇન શૈલીઓની તુલના કરો.}

\begin{solutionbox}

\textbf{તુલના ટેબલ:}

{\def\LTcaptype{none} % do not increment counter
\begin{longtable}[]{@{}lll@{}}
\toprule\noalign{}
પેરામીટર & ફુલ કસ્ટમ & સેમી કસ્ટમ \\
\midrule\noalign{}
\endhead
\bottomrule\noalign{}
\endlastfoot
\textbf{ડિઝાઇન ટાઇમ} & લાંબો (6-18 મહિના) & ટૂંકો (2-6 મહિના) \\
\textbf{પર્ફોર્મન્સ} & શ્રેષ્ઠ & સારું \\
\textbf{એરિયા} & લઘુત્તમ & મધ્યમ \\
\textbf{પાવર} & ઓપ્ટિમાઇઝ્ડ & સ્વીકાર્ય \\
\textbf{કોસ્ટ} & હાઇ NRE & લોઅર NRE \\
\textbf{લવચીકતા} & મહત્તમ & મર્યાદિત \\
\textbf{જોખમ} & વધુ & ઓછું \\
\end{longtable}
}

\textbf{ફુલ કસ્ટમ લાક્ષણિકતાઓ:}

\begin{itemize}
\tightlist
\item
  \textbf{દરેક ટ્રાન્ઝિસ્ટર}: મેન્યુઅલી ડિઝાઇન અને પ્લેસ કરેલું
\item
  \textbf{લેઆઉટ ઓપ્ટિમાઇઝેશન}: મહત્તમ ડેન્સિટી હાંસલ
\item
  \textbf{એપ્લિકેશન્સ}: હાઇ-વોલ્યુમ, પર્ફોર્મન્સ-ક્રિટિકલ
\end{itemize}

\textbf{સેમી કસ્ટમ પ્રકારો:}

\begin{itemize}
\tightlist
\item
  \textbf{ગેટ એરે}: પૂર્વ-વ્યાખ્યાયિત ટ્રાન્ઝિસ્ટર એરે
\item
  \textbf{સ્ટેન્ડર્ડ સેલ}: પૂર્વ-ડિઝાઇન કરેલા સેલ્સની લાઇબ્રેરી
\item
  \textbf{FPGA}: ફીલ્ડ પ્રોગ્રામેબલ લોજિક
\end{itemize}

\textbf{ડિઝાઇન ફ્લો તુલના:}

\begin{itemize}
\tightlist
\item
  \textbf{ફુલ કસ્ટમ}: સ્પેસિફિકેશન \rightarrow સ્કીમેટિક \rightarrow લેઆઉટ \rightarrow વેરિફિકેશન
\item
  \textbf{સેમી કસ્ટમ}: સ્પેસિફિકેશન \rightarrow HDL \rightarrow સિન્થેસિસ \rightarrow પ્લેસ \& રાઉટ
\end{itemize}

\end{solutionbox}
\begin{mnemonicbox}
``ફુલ કસ્ટમ - મહત્તમ નિયંત્રણ, સેમી કસ્ટમ - સ્પીડ સમજૂતી''

\end{mnemonicbox}
\subsection*{પ્રશ્ન 3(ક) OR [7
ગુણ]}\label{uxaaauxab0uxab6uxaa8-3uxa95-or-7-uxa97uxaa3}

\textbf{ASIC ડિઝાઇન ફ્લો વિગતવાર દોરો અને સમજાવો.}

\begin{solutionbox}

\textbf{ASIC ડિઝાઇન ફ્લો:}

\begin{verbatim}
flowchart TD
    A[સિસ્ટમ સ્પેક્સિફિકેશન] {-{-} B[આર્કિટેક્ચર ડિઝાઇન]}
    B {-{-} C[RTL ડિઝાઇન]}
    C {-{-} D[ફંક્શનલ વેરિફિકેશન]}
    D {-{-} E[લોજિક સિન્થેસિસ]}
    E {-{-} F[ગેટ{-}લેવલ સિમ્યુલેશન]}
    F {-{-} G[ફ્લોર પ્લાનિંગ]}
    G {-{-} H[પ્લેસમેન્ટ]}
    H {-{-} I[ક્લોક ટ્રી સિન્થેસિસ]}
    I {-{-} J[રાઉટિંગ]}
    J {-{-} K[ફિઝિકલ વેરિફિકેશન]}
    K {-{-} L[સ્ટેટિક ટાઇમિંગ એનાલિસિસ]}
    L {-{-} M[ટેપ{-}આઉટ]}
\end{verbatim}

\textbf{ડિઝાઇન સ્ટેજો:}

{\def\LTcaptype{none} % do not increment counter
\begin{longtable}[]{@{}lll@{}}
\toprule\noalign{}
સ્ટેજ & વર્ણન & ટૂલ્સ/પદ્ધતિઓ \\
\midrule\noalign{}
\endhead
\bottomrule\noalign{}
\endlastfoot
\textbf{RTL ડિઝાઇન} & હાર્ડવેર વર્ણન & વેરિલોગ/VHDL \\
\textbf{સિન્થેસિસ} & RTL ને ગેટ્સમાં કન્વર્ટ & લોજિક સિન્થેસિસ ટૂલ્સ \\
\textbf{ફ્લોર પ્લાનિંગ} & ચિપ એરિયા વિતરણ & ફ્લોર પ્લાનિંગ ટૂલ્સ \\
\textbf{પ્લેસમેન્ટ} & ગેટ્સ/બ્લોક્સ સ્થાન & પ્લેસમેન્ટ એલ્ગોરિધમ \\
\textbf{રાઉટિંગ} & પ્લેસ કરેલા એલિમેન્ટ્સ જોડો & રાઉટિંગ એલ્ગોરિધમ \\
\end{longtable}
}

\textbf{વેરિફિકેશન સ્ટેપ્સ:}

\begin{itemize}
\tightlist
\item
  \textbf{ફંક્શનલ}: RTL સિમ્યુલેશન અને વેરિફિકેશન
\item
  \textbf{ગેટ-લેવલ}: પોસ્ટ-સિન્થેસિસ સિમ્યુલેશન
\item
  \textbf{ફિઝિકલ}: DRC, LVS, એન્ટેના ચેક્સ
\item
  \textbf{ટાઇમિંગ}: સેટઅપ/હોલ્ડ વાયોલેશન માટે STA
\end{itemize}

\textbf{ડિઝાઇન કન્સ્ટ્રેઇન્ટ્સ:}

\begin{itemize}
\tightlist
\item
  \textbf{ટાઇમિંગ}: ક્લોક ફ્રીક્વન્સી જરૂરિયાતો
\item
  \textbf{એરિયા}: સિલિકોન એરિયા મર્યાદાઓ
\item
  \textbf{પાવર}: પાવર વપરાશ લક્ષ્યો
\item
  \textbf{ટેસ્ટ}: ટેસ્ટેબિલિટી માટે ડિઝાઇન
\end{itemize}

\textbf{સાઇન-ઓફ ચેક્સ:}

\begin{itemize}
\tightlist
\item
  \textbf{DRC}: ડિઝાઇન રૂલ ચેક
\item
  \textbf{LVS}: લેઆઉટ વર્સીસ સ્કીમેટિક
\item
  \textbf{STA}: સ્ટેટિક ટાઇમિંગ એનાલિસિસ
\item
  \textbf{પાવર}: પાવર ઇન્ટેગ્રિટી એનાલિસિસ
\end{itemize}

\end{solutionbox}
\begin{mnemonicbox}
``ASIC ફ્લો: RTL \rightarrow સિન્થેસિસ \rightarrow ફિઝિકલ \rightarrow વેરિફિકેશન''

\end{mnemonicbox}
\subsection*{પ્રશ્ન 4(અ) [3
ગુણ]}\label{uxaaauxab0uxab6uxaa8-4uxa85-3-uxa97uxaa3}

\textbf{CMOS સાથે લોજિક ફંક્શન G = (A(D+E)+BC)̅ અમલમાં મૂકો}

\begin{solutionbox}

\textbf{લોજિક વિશ્લેષણ:} G = (A(D+E)+BC)̅ = (AD+AE+BC)̅

\textbf{CMOS અમલીકરણ:}

\begin{verbatim}
         VDD
          |
    +{-{-}{-}{-}{-}+{-}{-}{-}{-}{-}+{-}{-}{-}{-}{-}+}
    |     |     |     |
  +{-+{-}+ +{-}+{-}+ +{-}+{-}+ +{-}+{-}+}
  |pA | |pD | |pA | |pB | PMOS
  +{-{-}{-}+ +{-}{-}{-}+ +{-}{-}{-}+ +{-}{-}{-}+ (સીરીઝ બ્રાન્ચ)}
    |     |     |     |
    +{-{-}{-}{-}{-}+     +{-}{-}{-}{-}{-}+}
          |           |
          +{-{-}{-}{-}{-}G{-}{-}{-}{-}{-}+}
                |
        +{-{-}{-}{-}{-}{-}{-}+{-}{-}{-}{-}{-}{-}{-}+}
        |       |       |
      +{-+{-}+   +{-}+{-}+   +{-}+{-}+}
      |nA |   |nA |   |nB | NMOS  
      +{-{-}{-}+   +{-}{-}{-}+   +{-}{-}{-}+ (પેરેલલ)}
        |       |       |
      +{-+{-}+   +{-}+{-}+     |}
      |nD |   |nE |     |
      +{-{-}{-}+   +{-}{-}{-}+     |}
        |       |       |
       GND     GND    +{-+{-}+}
                      |nC |
                      +{-{-}{-}+}
                        |
                       GND
\end{verbatim}

\textbf{નેટવર્ક કન્ફિગરેશન:}

\begin{itemize}
\tightlist
\item
  \textbf{PMOS}: કોમ્પ્લિમેન્ટનું સીરીઝ અમલીકરણ
\item
  \textbf{NMOS}: મૂળ ફંક્શનનું પેરેલલ અમલીકરણ
\end{itemize}

\end{solutionbox}
\begin{mnemonicbox}
``કોમ્પ્લેક્સ CMOS - PMOS સીરીઝ, NMOS પેરેલલ''

\end{mnemonicbox}
\subsection*{પ્રશ્ન 4(બ) [4
ગુણ]}\label{uxaaauxab0uxab6uxaa8-4uxaac-4-uxa97uxaa3}

\textbf{3 બિટ પેરિટી ચેકર માટે વેરિલોગ કોડ લખો.}

\begin{solutionbox}

\textbf{વેરિલોગ કોડ:}

\begin{verbatim}
module parity\_checker\_3bit(
    input [2:0] data\_in,
    output parity\_even,
    output parity\_odd
);

// ઇવન પેરિટી ચેકર
assign parity\_even = \^{}data\_in;

// ઓડ પેરિટી ચેકર
assign parity\_odd = {(\^{}}data\_in);

// વૈકલ્પિક અમલીકરણ
/*
assign parity\_even = data\_in[0] \^{ data\_in[1] \^{} data\_in[2];}
assign parity\_odd = {(data\_in[0] \^{} data\_in[1] \^{} data\_in[2]);}
*/

endmodule
\end{verbatim}

\textbf{ટ્રુથ ટેબલ:}

{\def\LTcaptype{none} % do not increment counter
\begin{longtable}[]{@{}llll@{}}
\toprule\noalign{}
ઇનપુટ [2:0] & 1 નું સંખ્યા & ઇવન પેરિટી & ઓડ પેરિટી \\
\midrule\noalign{}
\endhead
\bottomrule\noalign{}
\endlastfoot
000 & 0 & 0 & 1 \\
001 & 1 & 1 & 0 \\
010 & 1 & 1 & 0 \\
011 & 2 & 0 & 1 \\
100 & 1 & 1 & 0 \\
101 & 2 & 0 & 1 \\
110 & 2 & 0 & 1 \\
111 & 3 & 1 & 0 \\
\end{longtable}
}

\textbf{મુખ્ય લક્ષણો:}

\begin{itemize}
\tightlist
\item
  \textbf{XOR રિડક્શન}: \texttt{\^{}data\_in} ઇવન પેરિટી આપે છે
\item
  \textbf{કોમ્પ્લિમેન્ટ}: \texttt{\textasciitilde{}(\^{}data\_in)} ઓડ પેરિટી
  આપે છે
\end{itemize}

\end{solutionbox}
\begin{mnemonicbox}
``પેરિટી ચેક: બધા બિટ્સનું XOR''

\end{mnemonicbox}
\subsection*{પ્રશ્ન 4(ક) [7
ગુણ]}\label{uxaaauxab0uxab6uxaa8-4uxa95-7-uxa97uxaa3}

\textbf{અમલીકરણ:} \textbf{1) G = (AD +BC+EF) CMOS નો ઉપયોગ કરીને [3
ગુણ]} \textbf{2) Y' = (ABCD + EF(G+H)+ J) CMOS નો ઉપયોગ કરીને [4
ગુણ]}

\begin{solutionbox}

\textbf{ભાગ 1: G = (AD +BC+EF) [3 ગુણ]}

\textbf{CMOS સર્કિટ:}

\begin{verbatim}
         VDD
          |
    +{-{-}{-}{-}{-}+{-}{-}{-}{-}{-}+{-}{-}{-}{-}{-}+}
    |     |     |     |
  +{-+{-}+ +{-}+{-}+ +{-}+{-}+ +{-}+{-}+}
  |pA | |pB | |pE | |pA | PMOS
  +{-{-}{-}+ +{-}{-}{-}+ +{-}{-}{-}+ +{-}{-}{-}+ (સીરીઝ બ્રાન્ચ)}
    |     |     |     |
  +{-+{-}+ +{-}+{-}+ +{-}+{-}+   |}
  |pD | |pC | |pF |   |
  +{-{-}{-}+ +{-}{-}{-}+ +{-}{-}{-}+   |}
    |     |     |     |
    +{-{-}{-}{-}{-}+{-}{-}{-}{-}{-}+{-}{-}{-}{-}{-}+}
              |
              G
              |
        +{-{-}{-}{-}{-}+{-}{-}{-}{-}{-}+{-}{-}{-}{-}{-}+}
        |     |     |     |
      +{-+{-}+ +{-}+{-}+ +{-}+{-}+ NMOS}
      |nA | |nB | |nE | (પેરેલલ)
      +{-{-}{-}+ +{-}{-}{-}+ +{-}{-}{-}+}
        |     |     |
      +{-+{-}+ +{-}+{-}+ +{-}+{-}+}
      |nD | |nC | |nF |
      +{-{-}{-}+ +{-}{-}{-}+ +{-}{-}{-}+}
        |     |     |
       GND   GND   GND
\end{verbatim}

\textbf{ભાગ 2: Y' = (ABCD + EF(G+H)+ J) [4 ગુણ]}

આ એક કોમ્પ્લેક્સ ફંક્શન છે જેને બહુવિધ સ્ટેજની જરૂર છે:

\textbf{સ્ટેજ 1}: (G+H) અમલીકરણ \textbf{સ્ટેજ 2}: EF(G+H) અમલીકરણ
\textbf{સ્ટેજ 3}: બધા ટર્મ્સ કોમ્બાઇન

\textbf{આવા કોમ્પ્લેક્સ ફંક્શન માટે ટ્રાન્સમિશન ગેટ્સ અને બહુવિધ સ્ટેજનો ઉપયોગ વધુ
પ્રેક્ટિકલ છે.}

\end{solutionbox}
\begin{mnemonicbox}
``કોમ્પ્લેક્સ ફંક્શન્સને સ્ટેજ્ડ અમલીકરણ જોઈએ''

\end{mnemonicbox}
\subsection*{પ્રશ્ન 4(અ) OR [3
ગુણ]}\label{uxaaauxab0uxab6uxaa8-4uxa85-or-3-uxa97uxaa3}

\textbf{ઉદાહરણ સાથે AOI લોજિક સમજાવો.}

\begin{solutionbox}

\textbf{AOI વ્યાખ્યા:} \textbf{AND-OR-Invert} લોજિક આ પ્રકારના ફંક્શન્સ
અમલીકરણ કરે છે: Y = (AB + CD + \ldots)̅

\textbf{ઉદાહરણ: Y = (AB + CD)̅}

\textbf{AOI અમલીકરણ:}

\begin{verbatim}
         VDD
          |
    +{-{-}{-}{-}{-}+{-}{-}{-}{-}{-}+}
    |           |
  +{-+{-}+       +{-}+{-}+}
  |pA |       |pC | PMOS
  +{-{-}{-}+       +{-}{-}{-}+ (સીરીઝ બ્રાન્ચ)}
    |           |
  +{-+{-}+       +{-}+{-}+}
  |pB |       |pD |
  +{-{-}{-}+       +{-}{-}{-}+}
    |           |
    +{-{-}{-}{-}{-}{-}{-}{-}{-}{-}{-}+}
            |
            Y
            |
      +{-{-}{-}{-}{-}+{-}{-}{-}{-}{-}+}
      |           |
    +{-+{-}+       +{-}+{-}+}
    |nA |       |nC | NMOS
    +{-{-}{-}+       +{-}{-}{-}+ (પેરેલલ બ્રાન્ચ)}
      |           |
    +{-+{-}+       +{-}+{-}+}
    |nB |       |nD |
    +{-{-}{-}+       +{-}{-}{-}+}
      |           |
     GND         GND
\end{verbatim}

\textbf{ફાયદા:}

\begin{itemize}
\tightlist
\item
  \textbf{સિંગલ સ્ટેજ}: ડાયરેક્ટ અમલીકરણ
\item
  \textbf{ઝડપી}: બહુવિધ લેવલ્સ દ્વારા પ્રોપેગેશન નહીં
\item
  \textbf{એરિયા એફિશિઅન્ટ}: અલગ ગેટ્સ કરતાં ઓછા ટ્રાન્ઝિસ્ટર
\end{itemize}

\textbf{એપ્લિકેશન્સ:}

\begin{itemize}
\tightlist
\item
  \textbf{કોમ્પ્લેક્સ ગેટ્સ}: મલ્ટિ-ઇનપુટ ફંક્શન્સ
\item
  \textbf{સ્પીડ-ક્રિટિકલ પાથ}: ઘટેલી ડીલે
\end{itemize}

\end{solutionbox}
\begin{mnemonicbox}
``AOI - AND-OR-Invert એક સ્ટેજમાં''

\end{mnemonicbox}
\subsection*{પ્રશ્ન 4(બ) OR [4
ગુણ]}\label{uxaaauxab0uxab6uxaa8-4uxaac-or-4-uxa97uxaa3}

\textbf{4-બિટ સીરિયલ IN પેરેલલ આઉટ શિફ્ટ રજિસ્ટર માટે વેરિલોગ કોડ લખો.}

\begin{solutionbox}

\textbf{વેરિલોગ કોડ:}

\begin{verbatim}
module sipo\_4bit(
    input clk,
    input reset,
    input serial\_in,
    output reg [3:0] parallel\_out
);

always @(posedge clk or posedge reset) begin
    if (reset) begin
        parallel\_out {=} 4{b0000};
    end else begin
        // બાકીને શિફ્ટ કરો અને LSB પર નવો બિટ મૂકો
        parallel\_out {=} \{parallel\_out[2:0], serial\_in\;}
    end
end

endmodule
\end{verbatim}

\textbf{ટેસ્ટબેન્ચ ઉદાહરણ:}

\begin{verbatim}
module tb\_sipo\_4bit;
    reg clk, reset, serial\_in;
    wire [3:0] parallel\_out;
    
    sipo\_4bit dut(.clk(clk), .reset(reset), 
                  .serial\_in(serial\_in), 
                  .parallel\_out(parallel\_out));
                  
    initial begin
        clk = 0;
        forever \#5 clk = {}clk;
    end
    
    initial begin
        reset = 1; serial\_in = 0;
        \#10 reset = 0;
        \#10 serial\_in = 1; // LSB પહેલાં
        \#10 serial\_in = 0;
        \#10 serial\_in = 1; 
        \#10 serial\_in = 1; // MSB
        \#20 $finish;
    end
endmodule
\end{verbatim}

\textbf{ઓપરેશન ટાઇમલાઇન:}

{\def\LTcaptype{none} % do not increment counter
\begin{longtable}[]{@{}lll@{}}
\toprule\noalign{}
ક્લોક & Serial\_in & Parallel\_out \\
\midrule\noalign{}
\endhead
\bottomrule\noalign{}
\endlastfoot
1 & 1 & 0001 \\
2 & 0 & 0010 \\
3 & 1 & 0101 \\
4 & 1 & 1011 \\
\end{longtable}
}

\end{solutionbox}
\begin{mnemonicbox}
``SIPO - સીરિયલ ઇન, પેરેલલ આઉટ શિફ્ટ લેફ્ટ સાથે''

\end{mnemonicbox}
\subsection*{પ્રશ્ન 4(ક) OR [7
ગુણ]}\label{uxaaauxab0uxab6uxaa8-4uxa95-or-7-uxa97uxaa3}

\textbf{CMOS નો ઉપયોગ કરીને ક્લોક્ડ NOR2 SR લેચ અને D-લેચ અમલીકરણ કરો.}

\begin{solutionbox}

\textbf{ક્લોક્ડ NOR2 SR લેચ:}

\begin{verbatim}
    S {-{-}{-}+    CLK}
         |     |
       +{-+{-}+ +{-}+{-}+}
       |TG1| |TG2| ટ્રાન્સમિશન ગેટ્સ
       +{-{-}{-}+ +{-}{-}{-}+}
         |     |
    +{-{-}{-}{-}+     +{-}{-}{-}{-}+}
    |               |
  +{-+{-}+           +{-}+{-}+}
  |NOR|{-{-}{-}{-}Q{-}{-}{-}{-}{-}{-}|NOR| ક્રોસ{-}કપ્લ્ડ}
  +{-{-}{-}+           +{-}{-}{-}+ NOR ગેટ્સ}
    |               |
    +{-{-}{-}{-}{-}{-}{-}R{-}{-}{-}{-}{-}{-}{-}+}
\end{verbatim}

\textbf{D-લેચ અમલીકરણ:}

\begin{verbatim}
    D {-{-}{-}{-}+}
          |
        +{-+{-}+  CLK}
        |TG1|{-{-}{-}+}
        +{-{-}{-}+   |}
          |     |
          +{-{-}+  |}
             |  |
           +{-+{-}+{-}+{-}+}
           | માસ્ટર |
           |  લેચ |
           +{-{-}{-}+{-}{-}{-}+}
               |
               Q
\end{verbatim}

\textbf{CMOS D-લેચ સર્કિટ:}

\begin{verbatim}
       VDD                    VDD
        |                      |
      +{-+{-}+  CLK            +{-}+{-}+  CLK}
      |pTG|{-{-}{-}+             |pTG|}
      +{-{-}{-}+   |             +{-}{-}{-}+}
        |     |               |
    D{-{-}{-}+     |               +{-}{-}{-}Q}
        |     |               |
      +{-+{-}+   |             +{-}+{-}+}
      |nTG|{-{-}{-}+             |nTG|}
      +{-{-}{-}+                 +{-}{-}{-}+}
        |                     |
       GND                   GND
    
    માસ્ટર સેક્શન        સ્લેવ સેક્શન
\end{verbatim}

\textbf{ઓપરેશન:}

\begin{itemize}
\tightlist
\item
  \textbf{CLK = 1}: માસ્ટર ટ્રાન્સપેરન્ટ, સ્લેવ હોલ્ડ
\item
  \textbf{CLK = 0}: માસ્ટર હોલ્ડ, સ્લેવ ટ્રાન્સપેરન્ટ
\item
  \textbf{ડેટા ટ્રાન્સફર}: ક્લોક એજ પર
\end{itemize}

\textbf{SR લેચ માટે ટ્રુથ ટેબલ:}

{\def\LTcaptype{none} % do not increment counter
\begin{longtable}[]{@{}lllll@{}}
\toprule\noalign{}
S & R & CLK & Q & Q' \\
\midrule\noalign{}
\endhead
\bottomrule\noalign{}
\endlastfoot
0 & 0 & 1 & હોલ્ડ & હોલ્ડ \\
0 & 1 & 1 & 0 & 1 \\
1 & 0 & 1 & 1 & 0 \\
1 & 1 & 1 & અવૈધ & અવૈધ \\
\end{longtable}
}

\end{solutionbox}
\begin{mnemonicbox}
``ક્લોક્ડ લેચ ટાઇમિંગ કંટ્રોલ માટે ટ્રાન્સમિશન ગેટ ઉપયોગ કરે
છે''

\end{mnemonicbox}
\subsection*{પ્રશ્ન 5(અ) [3
ગુણ]}\label{uxaaauxab0uxab6uxaa8-5uxa85-3-uxa97uxaa3}

\textbf{યુલર પાથ એપ્રોચને ધ્યાનમાં લેતા CMOS નો ઉપયોગ કરીને Y = (PQ +U)' માટે
સ્ટિક ડાયાગ્રામ દોરો.}

\begin{solutionbox}

\textbf{લોજિક વિશ્લેષણ:} Y = (PQ + U)' માટે જરૂરી છે PMOS: (PQ)' · U' = (P' +
Q') · U' NMOS: PQ + U

\textbf{સ્ટિક ડાયાગ્રામ:}

\begin{verbatim}
    VDD {-{-}{-}{-}{-}{-}{-}{-}{-}{-}{-}{-}{-}{-}{-}{-}{-}{-}{-}{-}{-}{-}{-} VDD Rail}
     |                         |
   +{-+{-}+     +{-}+{-}+           +{-}+{-}+}
   |P{|લીલો |Q|લીલો        |U|લીલો  PMOS}
   +{-+{-}+     +{-}+{-}+           +{-}+{-}+}
     |         |               |
     +{-{-}{-}{-}{-}{-}{-}{-}{-}+{-}{-}{-}{-}{-}{-}{-}{-}{-}{-}{-}{-}{-}{-}{-}+}
                    |
                    Y {-{-}{-}{-}{-}{-}{-}{-}{-}{-}{-} આઉટપુટ}
                    |
     +{-{-}{-}{-}{-}{-}{-}{-}{-}{-}{-}{-}{-}+}
     |                         |
   +{-+{-}+     +{-}+{-}+           +{-}+{-}+}
   |P |લાલ   |Q |લાલ          |U |લાલ   NMOS  
   +{-+{-}+     +{-}+{-}+           +{-}+{-}+}
     |         |               |
     +{-{-}{-}{-}{-}{-}{-}{-}{-}+               |}
               |               |
    GND {-{-}{-}{-}{-}{-}{-}+{-}{-}{-}{-}{-}{-}{-}{-}{-}{-}{-}{-}{-}{-}{-}+{-}{-} GND Rail}

કિવદંતી:
{- લીલો: P{-}diffusion (PMOS)}
{- લાલ: N{-}diffusion (NMOS)  }
{- વાદળી: Polysilicon (ગેટ્સ)}
{- મેટલ: ઇન્ટરકનેક્શન્સ}
\end{verbatim}

\textbf{યુલર પાથ:}

\begin{enumerate}
\tightlist
\item
  \textbf{PMOS}: P' \rightarrow Q' (સીરીઝ), પછી U' સાથે પેરેલલ
\item
  \textbf{NMOS}: P \rightarrow Q (સીરીઝ), પછી U સાથે પેરેલલ
\item
  \textbf{ઓપ્ટિમલ રાઉટિંગ}: ક્રોસઓવર મિનિમાઇઝ કરે છે
\end{enumerate}

\textbf{લેઆઉટ વિચારણાઓ:}

\begin{itemize}
\tightlist
\item
  \textbf{ડિફ્યુઝન બ્રેક્સ}: બહેતર પર્ફોર્મન્સ માટે મિનિમાઇઝ
\item
  \textbf{કોન્ટેક્ટ પ્લેસમેન્ટ}: યોગ્ય VDD/GND કનેક્શન્સ
\item
  \textbf{મેટલ રાઉટિંગ}: DRC વાયોલેશન ટાળો
\end{itemize}

\end{solutionbox}
\begin{mnemonicbox}
``સ્ટિક ડાયાગ્રામ યુલર પાથ ઓપ્ટિમાઇઝેશન સાથે ફિઝિકલ લેઆઉટ
દર્શાવે છે''

\end{mnemonicbox}
\subsection*{પ્રશ્ન 5(બ) [4
ગુણ]}\label{uxaaauxab0uxab6uxaa8-5uxaac-4-uxa97uxaa3}

\textbf{વેરિલોગનો ઉપયોગ કરીને 8\times1 મલ્ટિપ્લેક્સર અમલમાં મૂકો}

\begin{solutionbox}

\textbf{વેરિલોગ કોડ:}

\begin{verbatim}
module mux\_8x1(
    input [7:0] data\_in,    // 8 ડેટા ઇનપુટ્સ
    input [2:0] select,     // 3{-બિટ સિલેક્ટ સિગ્નલ}
    output reg data\_out     // આઉટપુટ
);

always @(*) begin
    case (select)
        3{b000}: data\_out = data\_in[0];
        3{b001}: data\_out = data\_in[1];
        3{b010}: data\_out = data\_in[2];
        3{b011}: data\_out = data\_in[3];
        3{b100}: data\_out = data\_in[4];
        3{b101}: data\_out = data\_in[5];
        3{b110}: data\_out = data\_in[6];
        3{b111}: data\_out = data\_in[7];
        default: data\_out = 1{b0};
    endcase
end

endmodule
\end{verbatim}

\textbf{વૈકલ્પિક અમલીકરણ:}

\begin{verbatim}
module mux\_8x1\_dataflow(
    input [7:0] data\_in,
    input [2:0] select,
    output data\_out
);

assign data\_out = data\_in[select];

endmodule
\end{verbatim}

\textbf{ટ્રુથ ટેબલ:}

{\def\LTcaptype{none} % do not increment counter
\begin{longtable}[]{@{}ll@{}}
\toprule\noalign{}
Select[2:0] & આઉટપુટ \\
\midrule\noalign{}
\endhead
\bottomrule\noalign{}
\endlastfoot
000 & data\_in[0] \\
001 & data\_in[1] \\
010 & data\_in[2] \\
011 & data\_in[3] \\
100 & data\_in[4] \\
101 & data\_in[5] \\
110 & data\_in[6] \\
111 & data\_in[7] \\
\end{longtable}
}

\textbf{ટેસ્ટબેન્ચ:}

\begin{verbatim}
module tb\_mux\_8x1;
    reg [7:0] data\_in;
    reg [2:0] select;
    wire data\_out;
    
    mux\_8x1 dut(.data\_in(data\_in), .select(select), .data\_out(data\_out));
    
    initial begin
        data\_in = 8{b10110100};
        for (int i = 0; i {} 8; i++) begin
            select = i;
            \#10;
            $display("Select=\%d, Output=\%b", select, data\_out);
        end
    end
endmodule
\end{verbatim}

\end{solutionbox}
\begin{mnemonicbox}
``MUX સિલેક્ટ લાઇન્સના આધારે ઘણા ઇનપુટ્સમાંથી એક પસંદ કરે છે''

\end{mnemonicbox}
\subsection*{પ્રશ્ન 5(ક) [7
ગુણ]}\label{uxaaauxab0uxab6uxaa8-5uxa95-7-uxa97uxaa3}

\textbf{વેરિલોગમાં બિહેવિયરલ મોડેલિંગ સ્ટાઇલનો ઉપયોગ કરીને ફુલ એડર અમલમાં મૂકો.}

\begin{solutionbox}

\textbf{વેરિલોગ કોડ:}

\begin{verbatim}
module full\_adder\_behavioral(
    input A,
    input B, 
    input Cin,
    output reg Sum,
    output reg Cout
);

// બિહેવિયરલ મોડેલિંગ always બ્લોક સાથે
always @(*) begin
    case (\{A, B, Cin\)}
        3{b000}: begin Sum = 1{b0}; Cout = 1{b0}; end
        3{b001}: begin Sum = 1{b1}; Cout = 1{b0}; end
        3{b010}: begin Sum = 1{b1}; Cout = 1{b0}; end
        3{b011}: begin Sum = 1{b0}; Cout = 1{b1}; end
        3{b100}: begin Sum = 1{b1}; Cout = 1{b0}; end
        3{b101}: begin Sum = 1{b0}; Cout = 1{b1}; end
        3{b110}: begin Sum = 1{b0}; Cout = 1{b1}; end
        3{b111}: begin Sum = 1{b1}; Cout = 1{b1}; end
        default: begin Sum = 1{b0}; Cout = 1{b0}; end
    endcase
end

endmodule
\end{verbatim}

\textbf{વૈકલ્પિક બિહેવિયરલ સ્ટાઇલ:}

\begin{verbatim}
module full\_adder\_behavioral\_alt(
    input A, B, Cin,
    output reg Sum, Cout
);

always @(*) begin
    \{Cout, Sum\} = A + B + Cin;
end

endmodule
\end{verbatim}

\textbf{ટ્રુથ ટેબલ:}

{\def\LTcaptype{none} % do not increment counter
\begin{longtable}[]{@{}lllll@{}}
\toprule\noalign{}
A & B & Cin & Sum & Cout \\
\midrule\noalign{}
\endhead
\bottomrule\noalign{}
\endlastfoot
0 & 0 & 0 & 0 & 0 \\
0 & 0 & 1 & 1 & 0 \\
0 & 1 & 0 & 1 & 0 \\
0 & 1 & 1 & 0 & 1 \\
1 & 0 & 0 & 1 & 0 \\
1 & 0 & 1 & 0 & 1 \\
1 & 1 & 0 & 0 & 1 \\
1 & 1 & 1 & 1 & 1 \\
\end{longtable}
}

\textbf{ટેસ્ટબેન્ચ:}

\begin{verbatim}
module tb\_full\_adder;
    reg A, B, Cin;
    wire Sum, Cout;
    
    full\_adder\_behavioral dut(.A(A), .B(B), .Cin(Cin), 
                             .Sum(Sum), .Cout(Cout));
    
    initial begin
$monitor("A=\%b

B=\%b Cin=\%b | Sum=\%b Cout=\%b",

                 A, B, Cin, Sum, Cout);
        
        \{A, B, Cin\} = 3{b000}; \#10;
        \{A, B, Cin\} = 3{b001}; \#10;
        \{A, B, Cin\} = 3{b010}; \#10;
        \{A, B, Cin\} = 3{b011}; \#10;
        \{A, B, Cin\} = 3{b100}; \#10;
        \{A, B, Cin\} = 3{b101}; \#10;
        \{A, B, Cin\} = 3{b110}; \#10;
        \{A, B, Cin\} = 3{b111}; \#10;
        
        $finish;
    end
endmodule
\end{verbatim}

\textbf{બિહેવિયરલ લક્ષણો:}

\begin{itemize}
\tightlist
\item
  \textbf{Always બ્લોક}: સ્ટ્રક્ચર નહીં, બિહેવિયર વર્ણવે છે
\item
  \textbf{Case સ્ટેટમેન્ટ}: ટ્રુથ ટેબલ અમલીકરણ
\item
  \textbf{ઓટોમેટિક સિન્થેસિસ}: ટૂલ્સ ઓપ્ટિમાઇઝ્ડ સર્કિટ જનરેટ કરે છે
\end{itemize}

\end{solutionbox}
\begin{mnemonicbox}
``બિહેવિયરલ મોડેલિંગ સર્કિટ કેવી રીતે નહીં, શું કરે છે તે વર્ણવે
છે''

\end{mnemonicbox}
\subsection*{પ્રશ્ન 5(અ) OR [3
ગુણ]}\label{uxaaauxab0uxab6uxaa8-5uxa85-or-3-uxa97uxaa3}

\textbf{NOR2 ગેટ CMOS સર્કિટને તેના સ્ટિક ડાયાગ્રામ સાથે અમલમાં મૂકો.}

\begin{solutionbox}

\textbf{CMOS NOR2 સર્કિટ:}

\begin{verbatim}
       VDD
        |
    +{-{-}{-}+{-}{-}{-}+}
    |       |
  +{-+{-}+   +{-}+{-}+}
  |pA |   |pB | PMOS (પેરેલલ)
  +{-{-}{-}+   +{-}{-}{-}+}
    |       |
    +{-{-}{-}Y{-}{-}{-}+}
        |
      +{-+{-}+}
      |nA | NMOS (સીરીઝ)
      +{-{-}{-}+}
        |
      +{-+{-}+  }
      |nB |
      +{-{-}{-}+}
        |
       GND
\end{verbatim}

\textbf{સ્ટિક ડાયાગ્રામ:}

\begin{verbatim}
    VDD {-{-}{-}{-}{-}{-}{-}{-}{-}{-}{-}{-}{-}{-}{-}{-}{-}{-}{-}{-}{-}{-}{-} VDD Rail}
     |           |
   +{-+{-}+       +{-}+{-}+}
   |pA|લીલો    |pB|લીલો        PMOS (પેરેલલ)
   +{-+{-}+       +{-}+{-}+}
     |           |
     +{-{-}{-}{-}{-}{-}{-}{-}{-}{-}{-}+}
           |
           Y {-{-}{-}{-}{-}{-}{-}{-}{-}{-}{-}{-}{-}{-}{-}{-}{-}{-}{-} આઉટપુટ}
           |
         +{-+{-}+}
         |nA|લાલ                NMOS (સીરીઝ)
         +{-+{-}+}
           |
         +{-+{-}+}
         |nB|લાલ
         +{-+{-}+}
           |
    GND {-{-}{-}+{-}{-}{-}{-}{-}{-}{-}{-}{-}{-}{-}{-}{-}{-}{-}{-}{-}{-}{-} GND Rail}

કિવદંતી:
{- લીલો: P{-}diffusion}
{- લાલ: N{-}diffusion  }
{- વાદળી: Polysilicon ગેટ્સ}
{- મેટલ: કનેક્શન્સ}
\end{verbatim}

\textbf{લેઆઉટ નિયમો:}

\begin{itemize}
\tightlist
\item
  \textbf{PMOS}: પુલ-અપ માટે પેરેલલ કનેક્શન
\item
  \textbf{NMOS}: પુલ-ડાઉન માટે સીરીઝ કનેક્શન
\item
  \textbf{કોન્ટેક્ટ્સ}: યોગ્ય VDD/GND કનેક્શન્સ
\item
  \textbf{સ્પેસિંગ}: લઘુત્તમ ડિઝાઇન નિયમો પૂરા કરવા
\end{itemize}

\end{solutionbox}
\begin{mnemonicbox}
``NOR ગેટ: પેરેલલ PMOS, સીરીઝ NMOS''

\end{mnemonicbox}
\subsection*{પ્રશ્ન 5(બ) OR [4
ગુણ]}\label{uxaaauxab0uxab6uxaa8-5uxaac-or-4-uxa97uxaa3}

\textbf{વેરિલોગનો ઉપયોગ કરીને 4 બિટ અપ કાઉન્ટર અમલમાં મૂકો}

\begin{solutionbox}

\textbf{વેરિલોગ કોડ:}

\begin{verbatim}
module counter\_4bit\_up(
    input clk,
    input reset,
    input enable,
    output reg [3:0] count
);

always @(posedge clk or posedge reset) begin
    if (reset) begin
        count {=} 4{b0000};
    end else if (enable) begin
        if (count == 4{b1111}) begin
            count {=} 4{b0000};  // રોલઓવર
        end else begin
            count {=} count + 1;
        end
    end
    // જો enable લો છે તો વર્તમાન કિંમત રાખો
end

endmodule
\end{verbatim}

\textbf{ઓવરફ્લો સાથે વિસ્તૃત વર્ઝન:}

\begin{verbatim}
module counter\_4bit\_enhanced(
    input clk,
    input reset, 
    input enable,
    output reg [3:0] count,
    output overflow
);

always @(posedge clk or posedge reset) begin
    if (reset) begin
        count {=} 4{b0000};
    end else if (enable) begin
        count {=} count + 1;  // કુદરતી રોલઓવર
    end
end

assign overflow = (count == 4{b1111}) \& enable;

endmodule
\end{verbatim}

\textbf{કાઉન્ટ સિક્વન્સ:}

{\def\LTcaptype{none} % do not increment counter
\begin{longtable}[]{@{}lll@{}}
\toprule\noalign{}
ક્લોક & Count[3:0] & દશાંશ \\
\midrule\noalign{}
\endhead
\bottomrule\noalign{}
\endlastfoot
1 & 0000 & 0 \\
2 & 0001 & 1 \\
3 & 0010 & 2 \\
\ldots{} & \ldots{} & \ldots{} \\
15 & 1110 & 14 \\
16 & 1111 & 15 \\
17 & 0000 & 0 (રોલઓવર) \\
\end{longtable}
}

\textbf{ટેસ્ટબેન્ચ:}

\begin{verbatim}
module tb\_counter\_4bit;
    reg clk, reset, enable;
    wire [3:0] count;
    
    counter\_4bit\_up dut(.clk(clk), .reset(reset), 
                       .enable(enable), .count(count));
    
    // ક્લોક જનરેશન
    initial begin
        clk = 0;
        forever \#5 clk = {}clk;
    end
    
    // ટેસ્ટ સિક્વન્સ
    initial begin
        reset = 1; enable = 0;
        \#10 reset = 0; enable = 1;
        \#200 enable = 0;  // કાઉન્ટિંગ બંધ
        \#20 enable = 1;   // ફરી શરૂ
        \#100 $finish;
    end
    
    // મોનિટર
    always @(posedge clk) begin
        $display("Time=\%t Count=\%d", $time, count);
    end
endmodule
\end{verbatim}

\end{solutionbox}
\begin{mnemonicbox}
``અપ કાઉન્ટર: ઇનેબલ હોય ત્યારે દરેક ક્લોક પર વધારો''

\end{mnemonicbox}
\subsection*{પ્રશ્ન 5(ક) OR [7
ગુણ]}\label{uxaaauxab0uxab6uxaa8-5uxa95-or-7-uxa97uxaa3}

\textbf{વેરિલોગમાં બિહેવિયરલ મોડેલિંગ સ્ટાઇલનો ઉપયોગ કરીને 3:8 ડીકોડર અમલમાં
મૂકો.}

\begin{solutionbox}

\textbf{વેરિલોગ કોડ:}

\begin{verbatim}
module decoder\_3x8\_behavioral(
    input [2:0] address,    // 3{-બિટ એડ્રેસ ઇનપુટ}
    input enable,           // ઇનેબલ સિગ્નલ
    output reg [7:0] decode\_out  // 8{-બિટ ડીકોડેડ આઉટપુટ}
);

always @(*) begin
    if (enable) begin
        case (address)
            3{b000}: decode\_out = 8{b00000001};  // Y0
            3{b001}: decode\_out = 8{b00000010};  // Y1  
            3{b010}: decode\_out = 8{b00000100};  // Y2
            3{b011}: decode\_out = 8{b00001000};  // Y3
            3{b100}: decode\_out = 8{b00010000};  // Y4
            3{b101}: decode\_out = 8{b00100000};  // Y5
            3{b110}: decode\_out = 8{b01000000};  // Y6
            3{b111}: decode\_out = 8{b10000000};  // Y7
            default: decode\_out = 8{b00000000};
        endcase
    end else begin
        decode\_out = 8{b00000000};  // ડિસેબલ હોય ત્યારે બધા આઉટપુટ લો
    end
end

endmodule
\end{verbatim}

\textbf{વૈકલ્પિક અમલીકરણ:}

\begin{verbatim}
module decoder\_3x8\_shift(
    input [2:0] address,
    input enable,
    output [7:0] decode\_out
);

assign decode\_out = enable ? (8{b00000001} {} address) : 8{b00000000};

endmodule
\end{verbatim}

\textbf{ટ્રુથ ટેબલ:}

{\def\LTcaptype{none} % do not increment counter
\begin{longtable}[]{@{}lll@{}}
\toprule\noalign{}
Enable & Address[2:0] & decode\_out[7:0] \\
\midrule\noalign{}
\endhead
\bottomrule\noalign{}
\endlastfoot
0 & XXX & 00000000 \\
1 & 000 & 00000001 \\
1 & 001 & 00000010 \\
1 & 010 & 00000100 \\
1 & 011 & 00001000 \\
1 & 100 & 00010000 \\
1 & 101 & 00100000 \\
1 & 110 & 01000000 \\
1 & 111 & 10000000 \\
\end{longtable}
}

\textbf{ટેસ્ટબેન્ચ:}

\begin{verbatim}
module tb\_decoder\_3x8;
    reg [2:0] address;
    reg enable;
    wire [7:0] decode\_out;
    
    decoder\_3x8\_behavioral dut(.address(address), .enable(enable), 
                              .decode\_out(decode\_out));
    
    initial begin
        $monitor("Enable=\%b Address=\%b | Output=\%b", 
                 enable, address, decode\_out);
        
        // Enable = 0 સાથે ટેસ્ટ
        enable = 0;
        for (int i = 0; i {} 8; i++) begin
            address = i;
            \#10;
        end
        
        // Enable = 1 સાથે ટેસ્ટ
        enable = 1;
        for (int i = 0; i {} 8; i++) begin
            address = i;
            \#10;
        end
        
        $finish;
    end
endmodule
\end{verbatim}

\textbf{એપ્લિકેશન્સ:}

\begin{itemize}
\tightlist
\item
  \textbf{મેમોરી એડ્રેસિંગ}: 8 મેમોરી લોકેશનમાંથી એક પસંદ કરો
\item
  \textbf{ડિવાઇસ સિલેક્શન}: 8 પેરિફેરલ ડિવાઇસમાંથી એક ઇનેબલ કરો
\item
  \textbf{ડીમલ્ટિપ્લેક્સિંગ}: સિંગલ ઇનપુટને પસંદ કરેલા આઉટપુટ પર રાઉટ કરો
\end{itemize}

\textbf{ડિઝાઇન લક્ષણો:}

\begin{itemize}
\tightlist
\item
  \textbf{વન-હોટ એન્કોડિંગ}: એક સમયે ફક્ત એક આઉટપુટ હાઇ
\item
  \textbf{ઇનેબલ કંટ્રોલ}: ગ્લોબલ ઇનેબલ/ડિસેબલ ફંક્શનાલિટી
\item
  \textbf{ફુલ ડીકોડિંગ}: બધા શક્ય ઇનપુટ કોમ્બિનેશન હેન્ડલ
\end{itemize}

\end{solutionbox}
\begin{mnemonicbox}
``3:8 ડીકોડર - 3 ઇનપુટ્સ 8 આઉટપુટમાંથી 1 પસંદ કરે છે''

\end{mnemonicbox}

\end{document}
