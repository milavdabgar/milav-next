\documentclass[10pt,a4paper]{article}

% content/resources/templates/preamble.tex
\usepackage[margin=0.6in]{geometry}
\author{Milav Dabgar}
\usepackage{amsmath,amssymb,amsthm}
\usepackage{booktabs}
\usepackage{multirow}
\usepackage{xcolor}
\usepackage{tcolorbox}
\tcbuselibrary{breakable,skins}
\usepackage[colorlinks=true,linkcolor=blue]{hyperref}
\usepackage{titlesec}
\usepackage{enumitem}
\usepackage{tikz}
\usepackage{pgfplots}
\usepackage{circuitikz}
\usepackage[version=4]{mhchem}
\usepackage{longtable}
\usepackage{array}
\usepackage{float}
\usepackage{caption}
\usepackage{listings}

\lstset{
  basicstyle=\small\ttfamily,
  breaklines=true,
  breakatwhitespace=false,
  postbreak=\mbox{\textcolor{red}{$\hookrightarrow$}\space},
  float=false,
  numbers=left,
  numberstyle=\tiny\color{gray},
  numbersep=10pt,
  xleftmargin=2em,
  keywordstyle=\color{blue},
  commentstyle=\color{green!60!black},
  stringstyle=\color{purple},
  backgroundcolor=\color{gray!5},
  showstringspaces=false,
  tabsize=2,
  captionpos=b,
  keepspaces=true,
  columns=flexible
}

\pgfplotsset{compat=1.18}
\usetikzlibrary{shapes,arrows,positioning,calc,patterns,decorations.pathmorphing,decorations.markings,arrows.meta}

% Color scheme
\definecolor{headcolor}{RGB}{0,102,204}
\definecolor{keycolor}{RGB}{220,20,60}
\definecolor{solutioncolor}{RGB}{34,139,34}
\definecolor{mnemoniccolor}{RGB}{148,0,211}
\definecolor{codecolor}{RGB}{0,0,100}

% Spacing
\setlength{\parskip}{3pt}
\setlist[itemize]{nosep}
\setlist[enumerate]{nosep}

% Title formatting
\titleformat{\section}{\Large\bfseries\color{headcolor}}{\thesection}{1em}{}
\titleformat{\subsection}{\large\bfseries\color{headcolor}}{\thesubsection}{1em}{}

% Pandoc tightlist compatibility
\providecommand{\tightlist}{%
  \setlength{\itemsep}{0pt}\setlength{\parskip}{0pt}}

% Pandoc longtable compatibility
\newcounter{none}
\def\thenone{}


% content/resources/templates/english-boxes.tex
% This file is currently empty - it exists to maintain consistency with the import structure.
% Add custom environments here if needed in the future.


\begin{document}

\begin{center}
{\Huge\bfseries\color{headcolor} Subject Name Solutions}\\[5pt]
{\LARGE 4353206 -- Summer 2025}\\[3pt]
{\large Semester 1 Study Material}\\[3pt]
{\normalsize\textit{Detailed Solutions and Explanations}}
\end{center}

\vspace{10pt}

\subsection*{Question 1(a) [3 marks]}\label{q1a}

\textbf{Draw neat labeled diagram of physical structure of n-channel
MOSFET.}

\begin{solutionbox}

\textbf{Diagram:}

\begin{lstlisting}
           Gate (G)
              |
    +---------+----------+
    |    SiO2 (oxide)    |
+---+--------------------+---+
|S  |                    | D |
|o  |  p-type substrate  | r |
|u  |                    | a |
|r  |     n+      n+     | i |
|c  |    ----    ----    | n |  
|e  |   Source   Drain   |   |
+---+--------------------+---+
              |
           Substrate/Body
\end{lstlisting}

\textbf{Key Components:}

\begin{itemize}
\tightlist
\item
  \textbf{Source}: n+ doped region providing electrons
\item
  \textbf{Drain}: n+ doped region collecting electrons\\
\item
  \textbf{Gate}: Metal electrode controlling channel
\item
  \textbf{Oxide}: SiO2 insulating layer
\item
  \textbf{Substrate}: p-type silicon body
\end{itemize}

\end{solutionbox}
\begin{mnemonicbox}
``SOGD - Source, Oxide, Gate, Drain''

\end{mnemonicbox}
\subsection*{Question 1(b) [4 marks]}\label{q1b}

\textbf{Draw energy band diagram of depletion and inversion of MOS under
external bias with MOS biasing diagram. Explain inversion region in
detail.}

\begin{solutionbox}

\textbf{MOS Biasing Circuit:}

\begin{lstlisting}
    VG
     |
     |    Gate
    ++++++++++++
    |  SiO2   |
    +---------+
    | p-type  |
    +---------+
         |
        VB
\end{lstlisting}

\textbf{Energy Band Diagrams:}

{\def\LTcaptype{none} % do not increment counter
\begin{longtable}[]{@{}ll@{}}
\toprule\noalign{}
Bias Condition & Energy Band Behavior \\
\midrule\noalign{}
\endhead
\bottomrule\noalign{}
\endlastfoot
\textbf{Depletion} & Bands bend upward, holes depleted \\
\textbf{Inversion} & Strong band bending, electron channel forms \\
\end{longtable}
}

\textbf{Inversion Region Details:}

\begin{itemize}
\tightlist
\item
  \textbf{Strong inversion}: VG \textgreater{} VT (threshold voltage)
\item
  \textbf{Electron channel}: Forms at Si-SiO2 interface
\item
  \textbf{Channel conductivity}: Increases with gate voltage
\item
  \textbf{Threshold condition}: Surface potential = 2φF
\end{itemize}

\end{solutionbox}
\begin{mnemonicbox}
``DIVE - Depletion, Inversion, Voltage, Electrons''

\end{mnemonicbox}
\subsection*{Question 1(c) [7 marks]}\label{q1c}

\textbf{Explain I-V characteristics of MOSFET.}

\begin{solutionbox}

\textbf{I-V Characteristic Regions:}

{\def\LTcaptype{none} % do not increment counter
\begin{longtable}[]{@{}
  >{\raggedright\arraybackslash}p{(\linewidth - 4\tabcolsep) * \real{0.2353}}
  >{\raggedright\arraybackslash}p{(\linewidth - 4\tabcolsep) * \real{0.3235}}
  >{\raggedright\arraybackslash}p{(\linewidth - 4\tabcolsep) * \real{0.4412}}@{}}
\toprule\noalign{}
\begin{minipage}[b]{\linewidth}\raggedright
Region
\end{minipage} & \begin{minipage}[b]{\linewidth}\raggedright
Condition
\end{minipage} & \begin{minipage}[b]{\linewidth}\raggedright
Drain Current
\end{minipage} \\
\midrule\noalign{}
\endhead
\bottomrule\noalign{}
\endlastfoot
\textbf{Cutoff} & VGS \textless{} VT & ID \approx 0 \\
\textbf{Linear} & VGS \textgreater{} VT, VDS \textless{} VGS-VT & ID =
μnCox(W/L)[(VGS-VT)VDS - VDS^{2}/2] \\
\textbf{Saturation} & VGS \textgreater{} VT, VDS \geq VGS-VT & ID =
(μnCox/2)(W/L)(VGS-VT)^{2} \\
\end{longtable}
}

\textbf{Characteristic Curve:}

\begin{lstlisting}
    ID
     |
     |     Saturation
     |    +---------
     |   /
     |  / Linear
     | /
     |/
  ---+-----------> VDS
     0    VGS-VT
\end{lstlisting}

\textbf{Key Parameters:}

\begin{itemize}
\tightlist
\item
  \textbf{μn}: Electron mobility
\item
  \textbf{Cox}: Gate oxide capacitance
\item
  \textbf{W/L}: Width to length ratio
\item
  \textbf{VT}: Threshold voltage
\end{itemize}

\textbf{Operating Modes:}

\begin{itemize}
\tightlist
\item
  \textbf{Enhancement}: Channel forms with positive VGS
\item
  \textbf{Square law}: Saturation region follows quadratic relationship
\end{itemize}

\end{solutionbox}
\begin{mnemonicbox}
``CLS - Cutoff, Linear, Saturation''

\end{mnemonicbox}
\subsection*{Question 1(c) OR [7
marks]}\label{q1c}

\textbf{Define scaling. Explain the need of scaling. List and explain
the negative effects of scaling.}

\begin{solutionbox}

\textbf{Definition:} \textbf{Scaling} is the systematic reduction of
MOSFET dimensions to improve performance and density.

\textbf{Need for Scaling:}

{\def\LTcaptype{none} % do not increment counter
\begin{longtable}[]{@{}ll@{}}
\toprule\noalign{}
Benefit & Description \\
\midrule\noalign{}
\endhead
\bottomrule\noalign{}
\endlastfoot
\textbf{Higher Density} & More transistors per chip area \\
\textbf{Faster Speed} & Reduced gate delays \\
\textbf{Lower Power} & Decreased switching energy \\
\textbf{Cost Reduction} & More chips per wafer \\
\end{longtable}
}

\textbf{Scaling Types:}

{\def\LTcaptype{none} % do not increment counter
\begin{longtable}[]{@{}llll@{}}
\toprule\noalign{}
Type & Gate Length & Supply Voltage & Oxide Thickness \\
\midrule\noalign{}
\endhead
\bottomrule\noalign{}
\endlastfoot
\textbf{Constant Voltage} & ↓α & Constant & ↓α \\
\textbf{Constant Field} & ↓α & ↓α & ↓α \\
\end{longtable}
}

\textbf{Negative Effects:}

\begin{itemize}
\tightlist
\item
  \textbf{Short channel effects}: Threshold voltage roll-off
\item
  \textbf{Hot carrier effects}: Device degradation
\item
  \textbf{Gate leakage}: Increased tunneling current
\item
  \textbf{Process variations}: Manufacturing challenges
\item
  \textbf{Power density}: Heat dissipation issues
\end{itemize}

\end{solutionbox}
\begin{mnemonicbox}
``SHGPP - Short channel, Hot carrier, Gate leakage,
Process, Power''

\end{mnemonicbox}
\subsection*{Question 2(a) [3 marks]}\label{q2a}

\textbf{Implement Y' = (AB' + A'B) using CMOS.}

\begin{solutionbox}

\textbf{Logic Analysis:} Y' = (AB' + A'B) = A \oplus B (XOR function)

\textbf{CMOS Implementation:}

\begin{lstlisting}
    VDD
     |
   +-+-+   +-+-+
   |pA |   |pB |
   +---+   +---+
     |       |
     +---Y---+
     |       |
   +---+   +---+
   |nA |   |nB'|
   +-+-+   +-+-+
     |       |
    GND     GND
\end{lstlisting}

\textbf{Truth Table:}

{\def\LTcaptype{none} % do not increment counter
\begin{longtable}[]{@{}lllll@{}}
\toprule\noalign{}
A & B & AB' & A'B & Y' \\
\midrule\noalign{}
\endhead
\bottomrule\noalign{}
\endlastfoot
0 & 0 & 0 & 0 & 1 \\
0 & 1 & 0 & 1 & 0 \\
1 & 0 & 1 & 0 & 0 \\
1 & 1 & 0 & 0 & 1 \\
\end{longtable}
}

\end{solutionbox}
\begin{mnemonicbox}
``XOR needs complementary switching''

\end{mnemonicbox}
\subsection*{Question 2(b) [4 marks]}\label{q2b}

\textbf{Explain enhancement load inverter with its circuit diagrams.}

\begin{solutionbox}

\textbf{Circuit Diagram:}

\begin{lstlisting}
    VDD
     |
     +---o VG2
     |
   +-+-+ Enhancement
   |ME |  Load
   +---+
     |
     +---o Vout
     |
   +-+-+
   |MD | Driver
   +---+
     |
    GND
     |
     +---o Vin
\end{lstlisting}

\textbf{Configuration:}

{\def\LTcaptype{none} % do not increment counter
\begin{longtable}[]{@{}lll@{}}
\toprule\noalign{}
Component & Type & Connection \\
\midrule\noalign{}
\endhead
\bottomrule\noalign{}
\endlastfoot
\textbf{Load (ME)} & Enhancement NMOS & Gate connected to VDD \\
\textbf{Driver (MD)} & Enhancement NMOS & Gate is input \\
\end{longtable}
}

\textbf{Operation:}

\begin{itemize}
\tightlist
\item
  \textbf{Load transistor}: Acts as active load resistor
\item
  \textbf{High output}: Limited by VT of load transistor
\item
  \textbf{Low output}: Depends on driver strength
\item
  \textbf{Disadvantage}: Poor VOH due to threshold drop
\end{itemize}

\textbf{Transfer Characteristics:}

\begin{itemize}
\tightlist
\item
  \textbf{VOH}: VDD - VT (degraded high level)
\item
  \textbf{VOL}: Close to ground potential
\item
  \textbf{Noise margin}: Reduced due to threshold loss
\end{itemize}

\end{solutionbox}
\begin{mnemonicbox}
``ELI - Enhancement Load Inverter has threshold
Issues''

\end{mnemonicbox}
\subsection*{Question 2(c) [7 marks]}\label{q2c}

\textbf{Explain Voltage Transfer Characteristic of inverter.}

\begin{solutionbox}

\textbf{VTC Parameters:}

{\def\LTcaptype{none} % do not increment counter
\begin{longtable}[]{@{}lll@{}}
\toprule\noalign{}
Parameter & Description & Ideal Value \\
\midrule\noalign{}
\endhead
\bottomrule\noalign{}
\endlastfoot
\textbf{VOH} & Output High Voltage & VDD \\
\textbf{VOL} & Output Low Voltage & 0V \\
\textbf{VIH} & Input High Voltage & VDD/2 \\
\textbf{VIL} & Input Low Voltage & VDD/2 \\
\textbf{VM} & Switching Threshold & VDD/2 \\
\end{longtable}
}

\textbf{VTC Curve:}

\begin{lstlisting}
   Vout
     |
   VDD+         +-------
     |         /
     |        /
   VM+-------+  VM
     |        \
     |         \
     0         +-------
     +----+----+-----> Vin
      VIL VM  VIH   VDD
\end{lstlisting}

\textbf{Noise Margins:}

\begin{itemize}
\tightlist
\item
  \textbf{NMH} = VOH - VIH (High noise margin)
\item
  \textbf{NML} = VIL - VOL (Low noise margin)
\end{itemize}

\textbf{Regions:}

\begin{itemize}
\tightlist
\item
  \textbf{Region 1}: Input low, output high
\item
  \textbf{Region 2}: Transition region
\item
  \textbf{Region 3}: Input high, output low
\end{itemize}

\textbf{Quality Metrics:}

\begin{itemize}
\tightlist
\item
  \textbf{Sharp transition}: Better noise immunity
\item
  \textbf{Symmetric switching}: VM = VDD/2
\item
  \textbf{Full swing}: VOH = VDD, VOL = 0
\end{itemize}

\end{solutionbox}
\begin{mnemonicbox}
``VTC shows VOICE - VOH, VOL, Input thresholds,
Characteristics, Everything''

\end{mnemonicbox}
\subsection*{Question 2(a) OR [3
marks]}\label{q2a}

\textbf{Explain NAND2 gate using CMOS.}

\begin{solutionbox}

\textbf{CMOS NAND2 Circuit:}

\begin{lstlisting}
       VDD
        |
    +---+---+
    |       |
  +-+-+   +-+-+
  |pA |   |pB | PMOS
  +---+   +---+ (Parallel)
    |       |
    +---Y---+
        |
      +-+-+
      |nA | NMOS
      +---+ (Series)
        |
      +-+-+
      |nB |
      +---+
        |
       GND
\end{lstlisting}

\textbf{Truth Table:}

{\def\LTcaptype{none} % do not increment counter
\begin{longtable}[]{@{}lll@{}}
\toprule\noalign{}
A & B & Y \\
\midrule\noalign{}
\endhead
\bottomrule\noalign{}
\endlastfoot
0 & 0 & 1 \\
0 & 1 & 1 \\
1 & 0 & 1 \\
1 & 1 & 0 \\
\end{longtable}
}

\textbf{Operation:}

\begin{itemize}
\tightlist
\item
  \textbf{PMOS network}: Parallel connection (pull-up)
\item
  \textbf{NMOS network}: Series connection (pull-down)
\item
  \textbf{Output low}: Only when both inputs high
\end{itemize}

\end{solutionbox}
\begin{mnemonicbox}
``NAND - Not AND, Parallel PMOS, Series NMOS''

\end{mnemonicbox}
\subsection*{Question 2(b) OR [4
marks]}\label{q2b}

\textbf{Explain operating mode and VTC of Resistive load inverter
circuit.}

\begin{solutionbox}

\textbf{Circuit Configuration:}

\begin{lstlisting}
    VDD
     |
     R (Load Resistor)
     |
     +---o Vout
     |
   +-+-+
   |MN | NMOS Driver
   +---+
     |
    GND
     |
     +---o Vin
\end{lstlisting}

\textbf{Operating Modes:}

{\def\LTcaptype{none} % do not increment counter
\begin{longtable}[]{@{}lll@{}}
\toprule\noalign{}
Input State & NMOS State & Output \\
\midrule\noalign{}
\endhead
\bottomrule\noalign{}
\endlastfoot
\textbf{Vin = 0} & OFF & VOH = VDD \\
\textbf{Vin = VDD} & ON & VOL = R·ID/(R+RDS) \\
\end{longtable}
}

\textbf{VTC Characteristics:}

\begin{itemize}
\tightlist
\item
  \textbf{VOH}: Excellent (VDD)\\
\item
  \textbf{VOL}: Depends on R and RDS ratio
\item
  \textbf{Power consumption}: Static current when input high
\item
  \textbf{Transition}: Gradual due to resistive load
\end{itemize}

\textbf{Design Trade-offs:}

\begin{itemize}
\tightlist
\item
  \textbf{Large R}: Better VOL, slower switching
\item
  \textbf{Small R}: Faster switching, higher power
\item
  \textbf{Area}: Resistor occupies significant space
\end{itemize}

\end{solutionbox}
\begin{mnemonicbox}
``RLI - Resistive Load has Inevitable power
consumption''

\end{mnemonicbox}
\subsection*{Question 2(c) OR [7
marks]}\label{q2c}

\textbf{Draw CMOS inverter and explain its operation with VTC.}

\begin{solutionbox}

\textbf{CMOS Inverter Circuit:}

\begin{lstlisting}
       VDD
        |
      +-+-+
      |MP | PMOS
      +---+
        |
        +---o Vout
        |
      +-+-+
      |MN | NMOS  
      +---+
        |
       GND
        |
        +---o Vin
\end{lstlisting}

\textbf{Operation Regions:}

{\def\LTcaptype{none} % do not increment counter
\begin{longtable}[]{@{}lllll@{}}
\toprule\noalign{}
Vin Range & PMOS & NMOS & Vout & Region \\
\midrule\noalign{}
\endhead
\bottomrule\noalign{}
\endlastfoot
\textbf{0 to VTN} & ON & OFF & VDD & 1 \\
**VTN to VDD- & VTP & ** & ON & ON \\
**VDD- & VTP & to VDD** & OFF & ON \\
\end{longtable}
}

\textbf{VTC Analysis:}

\begin{lstlisting}
   Vout
     |
   VDD+
     |\
     | \
     |  \___
   VM+    \___
     |        \___
     |            \
     0             +---
     +----+----+----+---> Vin
        VTN  VM  VTP  VDD
\end{lstlisting}

\textbf{Key Features:}

\begin{itemize}
\tightlist
\item
  \textbf{Zero static power}: No DC current path
\item
  \textbf{Full swing}: VOH = VDD, VOL = 0V
\item
  \textbf{High noise margins}: NMH = NML \approx 0.4VDD
\item
  \textbf{Sharp transition}: High gain in transition region
\end{itemize}

\textbf{Design Considerations:}

\begin{itemize}
\tightlist
\item
  \textbf{β ratio}: βN/βP for symmetric switching
\item
  \textbf{Threshold matching}: VTN \approx \textbar VTP\textbar{} preferred
\end{itemize}

\end{solutionbox}
\begin{mnemonicbox}
``CMOS has Zero Static Power with Full Swing''

\end{mnemonicbox}
\subsection*{Question 3(a) [3 marks]}\label{q3a}

\textbf{Realize Y= (A̅+B̅)C̅+D̅+E̅ using depletion load.}

\begin{solutionbox}

\textbf{Logic Simplification:} Y = (A̅+B̅)C̅+D̅+E̅ = A̅C̅+B̅C̅+D̅+E̅

\textbf{Depletion Load Implementation:}

\begin{lstlisting}
       VDD
        |
      +-+-+ VGS=0
      |MD | Depletion
      +---+ Load
        |
        +---o Y
        |
    +---+---+---+---+
    |   |   |   |   |
  +-+-+-+-+-+-+-+-+-+-+
  |A'| |B'| |C'| |D'| |E'| Pull-down
  +--+ +--+ +--+ +--+ +--+ Network
    |   |   |   |   |
   GND GND GND GND GND
\end{lstlisting}

\textbf{Pull-down Network:}

\begin{itemize}
\tightlist
\item
  \textbf{Series}: A̅C̅ path and B̅C̅ path\\
\item
  \textbf{Parallel}: All paths connected in parallel
\item
  \textbf{Implementation}: Requires proper transistor sizing
\end{itemize}

\end{solutionbox}
\begin{mnemonicbox}
``Depletion Load with Parallel pull-down Paths''

\end{mnemonicbox}
\subsection*{Question 3(b) [4 marks]}\label{q3b}

\textbf{Write a short note on FPGA.}

\begin{solutionbox}

\textbf{FPGA Definition:} \textbf{Field Programmable Gate Array} -
Reconfigurable integrated circuit.

\textbf{Architecture Components:}

{\def\LTcaptype{none} % do not increment counter
\begin{longtable}[]{@{}ll@{}}
\toprule\noalign{}
Component & Function \\
\midrule\noalign{}
\endhead
\bottomrule\noalign{}
\endlastfoot
\textbf{CLB} & Configurable Logic Block \\
\textbf{IOB} & Input/Output Block \\
\textbf{Interconnect} & Routing resources \\
\textbf{Switch Matrix} & Connection points \\
\end{longtable}
}

\textbf{Programming Technologies:}

\begin{itemize}
\tightlist
\item
  \textbf{SRAM-based}: Volatile, fast reconfiguration
\item
  \textbf{Antifuse}: Non-volatile, one-time programmable\\
\item
  \textbf{Flash-based}: Non-volatile, reprogrammable
\end{itemize}

\textbf{Applications:}

\begin{itemize}
\tightlist
\item
  \textbf{Prototyping}: Digital system development
\item
  \textbf{DSP}: Signal processing applications
\item
  \textbf{Control systems}: Industrial automation
\item
  \textbf{Communications}: Protocol implementation
\end{itemize}

\textbf{Advantages vs ASIC:}

\begin{itemize}
\tightlist
\item
  \textbf{Flexibility}: Reconfigurable design
\item
  \textbf{Time-to-market}: Faster development
\item
  \textbf{Cost}: Lower for small volumes
\item
  \textbf{Risk}: Reduced design risk
\end{itemize}

\end{solutionbox}
\begin{mnemonicbox}
``FPGA - Flexible Programming Gives Advantages''

\end{mnemonicbox}
\subsection*{Question 3(c) [7 marks]}\label{q3c}

\textbf{Draw and explain Y chart design flow.}

\begin{solutionbox}

\textbf{Y-Chart Diagram:}

\includegraphics[width=1\linewidth,height=\textheight,keepaspectratio]{mermaid-2db7fb0e.pdf}

\textbf{Design Domains:}

{\def\LTcaptype{none} % do not increment counter
\begin{longtable}[]{@{}lll@{}}
\toprule\noalign{}
Domain & Levels & Description \\
\midrule\noalign{}
\endhead
\bottomrule\noalign{}
\endlastfoot
\textbf{Behavioral} & Algorithm \rightarrow RT \rightarrow Boolean & What the system does \\
\textbf{Structural} & Processor \rightarrow ALU \rightarrow Gates & How system is
constructed \\
\textbf{Physical} & Floor plan \rightarrow Layout \rightarrow Cells & Physical
implementation \\
\end{longtable}
}

\textbf{Design Flow Process:}

\begin{itemize}
\tightlist
\item
  \textbf{Top-down}: Start from behavioral, move to physical
\item
  \textbf{Bottom-up}: Build from components upward\\
\item
  \textbf{Mixed approach}: Combination of both methods
\end{itemize}

\textbf{Abstraction Levels:}

\begin{itemize}
\tightlist
\item
  \textbf{System level}: Highest abstraction
\item
  \textbf{RT level}: Register transfer operations
\item
  \textbf{Gate level}: Boolean logic implementation
\item
  \textbf{Layout level}: Physical geometry
\end{itemize}

\textbf{Design Verification:}

\begin{itemize}
\tightlist
\item
  \textbf{Horizontal}: Between domains at same level
\item
  \textbf{Vertical}: Between levels in same domain
\end{itemize}

\end{solutionbox}
\begin{mnemonicbox}
``Y-Chart: Behavioral, Structural, Physical - BSP
domains''

\end{mnemonicbox}
\subsection*{Question 3(a) OR [3
marks]}\label{q3a}

\textbf{Explain NOR2 gate using depletion load.}

\begin{solutionbox}

\textbf{Depletion Load NOR2 Circuit:}

\begin{lstlisting}
       VDD
        |
      +-+-+ VGS=0
      |MD | Depletion  
      +---+ Load
        |
        +---o Y
        |
    +---+---+
    |       |
  +-+-+   +-+-+
  |nA |   |nB | NMOS
  +---+   +---+ (Parallel)
    |       |
   GND     GND
\end{lstlisting}

\textbf{Truth Table:}

{\def\LTcaptype{none} % do not increment counter
\begin{longtable}[]{@{}lll@{}}
\toprule\noalign{}
A & B & Y \\
\midrule\noalign{}
\endhead
\bottomrule\noalign{}
\endlastfoot
0 & 0 & 1 \\
0 & 1 & 0 \\
1 & 0 & 0 \\
1 & 1 & 0 \\
\end{longtable}
}

\textbf{Operation:}

\begin{itemize}
\tightlist
\item
  \textbf{Both inputs low}: Both NMOS OFF, Y = VDD
\item
  \textbf{Any input high}: Corresponding NMOS ON, Y = VOL
\item
  \textbf{Load transistor}: Provides pull-up current
\end{itemize}

\end{solutionbox}
\begin{mnemonicbox}
``NOR with Depletion - Parallel NMOS pull-down''

\end{mnemonicbox}
\subsection*{Question 3(b) OR [4
marks]}\label{q3b}

\textbf{Compare full custom and semi-custom design styles.}

\begin{solutionbox}

\textbf{Comparison Table:}

{\def\LTcaptype{none} % do not increment counter
\begin{longtable}[]{@{}lll@{}}
\toprule\noalign{}
Parameter & Full Custom & Semi-Custom \\
\midrule\noalign{}
\endhead
\bottomrule\noalign{}
\endlastfoot
\textbf{Design Time} & Long (6-18 months) & Short (2-6 months) \\
\textbf{Performance} & Optimal & Good \\
\textbf{Area} & Minimum & Moderate \\
\textbf{Power} & Optimized & Acceptable \\
\textbf{Cost} & High NRE & Lower NRE \\
\textbf{Flexibility} & Maximum & Limited \\
\textbf{Risk} & High & Lower \\
\end{longtable}
}

\textbf{Full Custom Characteristics:}

\begin{itemize}
\tightlist
\item
  \textbf{Every transistor}: Manually designed and placed
\item
  \textbf{Layout optimization}: Maximum density achieved
\item
  \textbf{Applications}: High-volume, performance-critical
\end{itemize}

\textbf{Semi-Custom Types:}

\begin{itemize}
\tightlist
\item
  \textbf{Gate Array}: Pre-defined transistor array
\item
  \textbf{Standard Cell}: Library of pre-designed cells
\item
  \textbf{FPGA}: Field programmable logic
\end{itemize}

\textbf{Design Flow Comparison:}

\begin{itemize}
\tightlist
\item
  \textbf{Full Custom}: Specification \rightarrow Schematic \rightarrow Layout \rightarrow
  Verification
\item
  \textbf{Semi-Custom}: Specification \rightarrow HDL \rightarrow Synthesis \rightarrow Place \& Route
\end{itemize}

\end{solutionbox}
\begin{mnemonicbox}
``Full Custom - Maximum control, Semi-Custom - Speed
compromise''

\end{mnemonicbox}
\subsection*{Question 3(c) OR [7
marks]}\label{q3c}

\textbf{Draw and explain ASIC design flow in detail.}

\begin{solutionbox}

\textbf{ASIC Design Flow:}

\includegraphics[width=1\linewidth,height=\textheight,keepaspectratio]{mermaid-bbd930a6.pdf}

\textbf{Design Stages:}

{\def\LTcaptype{none} % do not increment counter
\begin{longtable}[]{@{}lll@{}}
\toprule\noalign{}
Stage & Description & Tools/Methods \\
\midrule\noalign{}
\endhead
\bottomrule\noalign{}
\endlastfoot
\textbf{RTL Design} & Hardware description & Verilog/VHDL \\
\textbf{Synthesis} & Convert RTL to gates & Logic synthesis tools \\
\textbf{Floor Planning} & Chip area allocation & Floor planning tools \\
\textbf{Placement} & Position gates/blocks & Placement algorithms \\
\textbf{Routing} & Connect placed elements & Routing algorithms \\
\end{longtable}
}

\textbf{Verification Steps:}

\begin{itemize}
\tightlist
\item
  \textbf{Functional}: RTL simulation and verification
\item
  \textbf{Gate-level}: Post-synthesis simulation\\
\item
  \textbf{Physical}: DRC, LVS, antenna checks
\item
  \textbf{Timing}: STA for setup/hold violations
\end{itemize}

\textbf{Design Constraints:}

\begin{itemize}
\tightlist
\item
  \textbf{Timing}: Clock frequency requirements
\item
  \textbf{Area}: Silicon area limitations
\item
  \textbf{Power}: Power consumption targets
\item
  \textbf{Test}: Design for testability
\end{itemize}

\textbf{Sign-off Checks:}

\begin{itemize}
\tightlist
\item
  \textbf{DRC}: Design Rule Check
\item
  \textbf{LVS}: Layout Versus Schematic\\
\item
  \textbf{STA}: Static Timing Analysis
\item
  \textbf{Power}: Power integrity analysis
\end{itemize}

\end{solutionbox}
\begin{mnemonicbox}
``ASIC flow: RTL \rightarrow Synthesis \rightarrow Physical \rightarrow
Verification''

\end{mnemonicbox}
\subsection*{Question 4(a) [3 marks]}\label{q4a}

\textbf{Implement the logic function G = (A(D+E)+BC)̅ using CMOS}

\begin{solutionbox}

\textbf{Logic Analysis:} G = (A(D+E)+BC)̅ = (AD+AE+BC)̅

\textbf{CMOS Implementation:}

\begin{lstlisting}
         VDD
          |
    +-----+-----+-----+
    |     |     |     |
  +-+-+ +-+-+ +-+-+ +-+-+
  |pA | |pD | |pA | |pB | PMOS
  +---+ +---+ +---+ +---+ (Series branches)
    |     |     |     |
    +-----+     +-----+
          |           |
          +-----G-----+
                |
        +-------+-------+
        |       |       |
      +-+-+   +-+-+   +-+-+
      |nA |   |nA |   |nB | NMOS  
      +---+   +---+   +---+ (Parallel)
        |       |       |
      +-+-+   +-+-+     |
      |nD |   |nE |     |
      +---+   +---+     |
        |       |       |
       GND     GND    +-+-+
                      |nC |
                      +---+
                        |
                       GND
\end{lstlisting}

\textbf{Network Configuration:}

\begin{itemize}
\tightlist
\item
  \textbf{PMOS}: Series implementation of complement
\item
  \textbf{NMOS}: Parallel implementation of original function
\end{itemize}

\end{solutionbox}
\begin{mnemonicbox}
``Complex CMOS - PMOS series, NMOS parallel''

\end{mnemonicbox}
\subsection*{Question 4(b) [4 marks]}\label{q4b}

\textbf{Write a Verilog code for 3 bit parity checker.}

\begin{solutionbox}

\textbf{Verilog Code:}

\begin{lstlisting}[language=Verilog]
module parity_checker_3bit(
    input [2:0] data_in,
    output parity_even,
    output parity_odd
);

// Even parity checker
assign parity_even = ^data_in;

// Odd parity checker  
assign parity_odd = ~(^data_in);

// Alternative implementation
/*
assign parity_even = data_in[0] ^ data_in[1] ^ data_in[2];
assign parity_odd = ~(data_in[0] ^ data_in[1] ^ data_in[2]);
*/

endmodule
\end{lstlisting}

\textbf{Truth Table:}

{\def\LTcaptype{none} % do not increment counter
\begin{longtable}[]{@{}llll@{}}
\toprule\noalign{}
Input [2:0] & Number of 1s & Even Parity & Odd Parity \\
\midrule\noalign{}
\endhead
\bottomrule\noalign{}
\endlastfoot
000 & 0 & 0 & 1 \\
001 & 1 & 1 & 0 \\
010 & 1 & 1 & 0 \\
011 & 2 & 0 & 1 \\
100 & 1 & 1 & 0 \\
101 & 2 & 0 & 1 \\
110 & 2 & 0 & 1 \\
111 & 3 & 1 & 0 \\
\end{longtable}
}

\textbf{Key Features:}

\begin{itemize}
\tightlist
\item
  \textbf{XOR reduction}: \passthrough{\lstinline!\^data\_in!} gives
  even parity
\item
  \textbf{Complement}: \passthrough{\lstinline!\~(\^data\_in)!} gives
  odd parity
\end{itemize}

\end{solutionbox}
\begin{mnemonicbox}
``Parity Check: XOR all bits''

\end{mnemonicbox}
\subsection*{Question 4(c) [7 marks]}\label{q4c}

\textbf{Implement:} \textbf{1) G = (AD +BC+EF) using CMOS [3 marks]}
\textbf{2) Y' = (ABCD + EF(G+H)+ J) using CMOS [4 marks]}

\begin{solutionbox}

\textbf{Part 1: G = (AD +BC+EF) [3 marks]}

\textbf{CMOS Circuit:}

\begin{lstlisting}
         VDD
          |
    +-----+-----+-----+
    |     |     |     |
  +-+-+ +-+-+ +-+-+ +-+-+
  |pA | |pB | |pE | |pA | PMOS
  +---+ +---+ +---+ +---+ (Series branches)
    |     |     |     |
  +-+-+ +-+-+ +-+-+   |
  |pD | |pC | |pF |   |
  +---+ +---+ +---+   |
    |     |     |     |
    +-----+-----+-----+
              |
              G
              |
        +-----+-----+-----+
        |     |     |     |
      +-+-+ +-+-+ +-+-+ NMOS
      |nA | |nB | |nE | (Parallel)
      +---+ +---+ +---+
        |     |     |
      +-+-+ +-+-+ +-+-+
      |nD | |nC | |nF |
      +---+ +---+ +---+
        |     |     |
       GND   GND   GND
\end{lstlisting}

\textbf{Part 2: Y' = (ABCD + EF(G+H)+ J) [4 marks]}

This requires a complex implementation with multiple stages:

\textbf{Stage 1}: Implement (G+H) \textbf{Stage 2}: Implement EF(G+H)\\
\textbf{Stage 3}: Combine all terms

\textbf{Simplified approach using transmission gates and multiple stages
would be more practical for this complex function.}

\end{solutionbox}
\begin{mnemonicbox}
``Complex functions need staged implementation''

\end{mnemonicbox}
\subsection*{Question 4(a) OR [3
marks]}\label{q4a}

\textbf{Explain AOI logic with example.}

\begin{solutionbox}

\textbf{AOI Definition:} \textbf{AND-OR-Invert} logic implements
functions of form: Y = (AB + CD + \ldots)̅

\textbf{Example: Y = (AB + CD)̅}

\textbf{AOI Implementation:}

\begin{lstlisting}
         VDD
          |
    +-----+-----+
    |           |
  +-+-+       +-+-+
  |pA |       |pC | PMOS
  +---+       +---+ (Series branches)
    |           |
  +-+-+       +-+-+
  |pB |       |pD |
  +---+       +---+
    |           |
    +-----------+
            |
            Y
            |
      +-----+-----+
      |           |
    +-+-+       +-+-+
    |nA |       |nC | NMOS
    +---+       +---+ (Parallel branches)
      |           |
    +-+-+       +-+-+
    |nB |       |nD |
    +---+       +---+
      |           |
     GND         GND
\end{lstlisting}

\textbf{Advantages:}

\begin{itemize}
\tightlist
\item
  \textbf{Single stage}: Direct implementation
\item
  \textbf{Fast}: No propagation through multiple levels
\item
  \textbf{Area efficient}: Fewer transistors than separate gates
\end{itemize}

\textbf{Applications:}

\begin{itemize}
\tightlist
\item
  \textbf{Complex gates}: Multi-input functions
\item
  \textbf{Speed-critical paths}: Reduced delay
\end{itemize}

\end{solutionbox}
\begin{mnemonicbox}
``AOI - AND-OR-Invert in one stage''

\end{mnemonicbox}
\subsection*{Question 4(b) OR [4
marks]}\label{q4b}

\textbf{Write Verilog Code for 4-bit Serial IN Parallel out shift
register.}

\begin{solutionbox}

\textbf{Verilog Code:}

\begin{lstlisting}[language=Verilog]
module sipo_4bit(
    input clk,
    input reset,
    input serial_in,
    output reg [3:0] parallel_out
);

always @(posedge clk or posedge reset) begin
    if (reset) begin
        parallel_out <= 4'b0000;
    end else begin
        // Shift left and insert new bit at LSB
        parallel_out <= {parallel_out[2:0], serial_in};
    end
end

endmodule
\end{lstlisting}

\textbf{Testbench Example:}

\begin{lstlisting}[language=Verilog]
module tb_sipo_4bit;
    reg clk, reset, serial_in;
    wire [3:0] parallel_out;
    
    sipo_4bit dut(.clk(clk), .reset(reset), 
                  .serial_in(serial_in), 
                  .parallel_out(parallel_out));
                  
    initial begin
        clk = 0;
        forever #5 clk = ~clk;
    end
    
    initial begin
        reset = 1; serial_in = 0;
        #10 reset = 0;
        #10 serial_in = 1; // LSB first
        #10 serial_in = 0;
        #10 serial_in = 1; 
        #10 serial_in = 1; // MSB
        #20 $finish;
    end
endmodule
\end{lstlisting}

\textbf{Operation Timeline:}

{\def\LTcaptype{none} % do not increment counter
\begin{longtable}[]{@{}lll@{}}
\toprule\noalign{}
Clock & Serial\_in & Parallel\_out \\
\midrule\noalign{}
\endhead
\bottomrule\noalign{}
\endlastfoot
1 & 1 & 0001 \\
2 & 0 & 0010 \\
3 & 1 & 0101 \\
4 & 1 & 1011 \\
\end{longtable}
}

\end{solutionbox}
\begin{mnemonicbox}
``SIPO - Serial In, Parallel Out with shift left''

\end{mnemonicbox}
\subsection*{Question 4(c) OR [7
marks]}\label{q4c}

\textbf{Implement clocked NOR2 SR latch and D-latch using CMOS.}

\begin{solutionbox}

\textbf{Clocked NOR2 SR Latch:}

\begin{lstlisting}
    S ---+    CLK
         |     |
       +-+-+ +-+-+
       |TG1| |TG2| Transmission Gates
       +---+ +---+
         |     |
    +----+     +----+
    |               |
  +-+-+           +-+-+
  |NOR|----Q------|NOR| Cross-coupled
  +---+           +---+ NOR gates
    |               |
    +-------R-------+
\end{lstlisting}

\textbf{D-Latch Implementation:}

\begin{lstlisting}
    D ----+
          |
        +-+-+  CLK
        |TG1|---+
        +---+   |
          |     |
          +--+  |
             |  |
           +-+-+-+-+
           | Master |
           |  Latch |
           +---+---+
               |
               Q
\end{lstlisting}

\textbf{CMOS D-Latch Circuit:}

\begin{lstlisting}
       VDD                    VDD
        |                      |
      +-+-+  CLK            +-+-+  CLK'
      |pTG|---+             |pTG|
      +---+   |             +---+
        |     |               |
    D---+     |               +---Q
        |     |               |
      +-+-+   |             +-+-+
      |nTG|---+             |nTG|
      +---+                 +---+
        |                     |
       GND                   GND
    
    Master Section        Slave Section
\end{lstlisting}

\textbf{Operation:}

\begin{itemize}
\tightlist
\item
  \textbf{CLK = 1}: Master transparent, slave holds
\item
  \textbf{CLK = 0}: Master holds, slave transparent
\item
  \textbf{Data transfer}: On clock edge
\end{itemize}

\textbf{Truth Table for SR Latch:}

{\def\LTcaptype{none} % do not increment counter
\begin{longtable}[]{@{}lllll@{}}
\toprule\noalign{}
S & R & CLK & Q & Q' \\
\midrule\noalign{}
\endhead
\bottomrule\noalign{}
\endlastfoot
0 & 0 & 1 & Hold & Hold \\
0 & 1 & 1 & 0 & 1 \\
1 & 0 & 1 & 1 & 0 \\
1 & 1 & 1 & Invalid & Invalid \\
\end{longtable}
}

\end{solutionbox}
\begin{mnemonicbox}
``Clocked latches use transmission gates for timing
control''

\end{mnemonicbox}
\subsection*{Question 5(a) [3 marks]}\label{q5a}

\textbf{Draw the stick diagram for Y = (PQ +U)' using CMOS considering
Euler path approach.}

\begin{solutionbox}

\textbf{Logic Analysis:} Y = (PQ + U)' requires PMOS: (PQ)' · U' = (P' +
Q') · U' NMOS: PQ + U

\textbf{Stick Diagram:}

\begin{lstlisting}
    VDD ----------------------- VDD Rail
     |                         |
   +-+-+     +-+-+           +-+-+
   |P'|green |Q'|green       |U'|green  PMOS
   +-+-+     +-+-+           +-+-+
     |         |               |
     +---------+---------------+
                    |
                    Y ----------- Output
                    |
     +-------------+
     |                         |
   +-+-+     +-+-+           +-+-+
   |P |red   |Q |red         |U |red   NMOS  
   +-+-+     +-+-+           +-+-+
     |         |               |
     +---------+               |
               |               |
    GND -------+---------------+-- GND Rail

Legend:
- Green: P-diffusion (PMOS)
- Red: N-diffusion (NMOS)  
- Blue: Polysilicon (Gates)
- Metal: Interconnections
\end{lstlisting}

\textbf{Euler Path:}

\begin{enumerate}
\tightlist
\item
  \textbf{PMOS}: P' \rightarrow Q' (series), then parallel to U'
\item
  \textbf{NMOS}: P \rightarrow Q (series), then parallel to U
\item
  \textbf{Optimal routing}: Minimizes crossovers
\end{enumerate}

\textbf{Layout Considerations:}

\begin{itemize}
\tightlist
\item
  \textbf{Diffusion breaks}: Minimize for better performance
\item
  \textbf{Contact placement}: Proper VDD/GND connections
\item
  \textbf{Metal routing}: Avoid DRC violations
\end{itemize}

\end{solutionbox}
\begin{mnemonicbox}
``Stick diagram shows physical layout with Euler path
optimization''

\end{mnemonicbox}
\subsection*{Question 5(b) [4 marks]}\label{q5b}

\textbf{Implement 8\times1 multiplexer using Verilog}

\begin{solutionbox}

\textbf{Verilog Code:}

\begin{lstlisting}[language=Verilog]
module mux_8x1(
    input [7:0] data_in,    // 8 data inputs
    input [2:0] select,     // 3-bit select signal
    output reg data_out     // Output
);

always @(*) begin
    case (select)
        3'b000: data_out = data_in[0];
        3'b001: data_out = data_in[1];
        3'b010: data_out = data_in[2];
        3'b011: data_out = data_in[3];
        3'b100: data_out = data_in[4];
        3'b101: data_out = data_in[5];
        3'b110: data_out = data_in[6];
        3'b111: data_out = data_in[7];
        default: data_out = 1'b0;
    endcase
end

endmodule
\end{lstlisting}

\textbf{Alternative Implementation:}

\begin{lstlisting}[language=Verilog]
module mux_8x1_dataflow(
    input [7:0] data_in,
    input [2:0] select,
    output data_out
);

assign data_out = data_in[select];

endmodule
\end{lstlisting}

\textbf{Truth Table:}

{\def\LTcaptype{none} % do not increment counter
\begin{longtable}[]{@{}ll@{}}
\toprule\noalign{}
Select[2:0] & Output \\
\midrule\noalign{}
\endhead
\bottomrule\noalign{}
\endlastfoot
000 & data\_in[0] \\
001 & data\_in[1] \\
010 & data\_in[2] \\
011 & data\_in[3] \\
100 & data\_in[4] \\
101 & data\_in[5] \\
110 & data\_in[6] \\
111 & data\_in[7] \\
\end{longtable}
}

\textbf{Testbench:}

\begin{lstlisting}[language=Verilog]
module tb_mux_8x1;
    reg [7:0] data_in;
    reg [2:0] select;
    wire data_out;
    
    mux_8x1 dut(.data_in(data_in), .select(select), .data_out(data_out));
    
    initial begin
        data_in = 8'b10110100;
        for (int i = 0; i < 8; i++) begin
            select = i;
            #10;
            $display("Select=%d, Output=%b", select, data_out);
        end
    end
endmodule
\end{lstlisting}

\end{solutionbox}
\begin{mnemonicbox}
``MUX selects one of many inputs based on select
lines''

\end{mnemonicbox}
\subsection*{Question 5(c) [7 marks]}\label{q5c}

\textbf{Implement full adder using behavioral modeling style in
Verilog.}

\begin{solutionbox}

\textbf{Verilog Code:}

\begin{lstlisting}[language=Verilog]
module full_adder_behavioral(
    input A,
    input B, 
    input Cin,
    output reg Sum,
    output reg Cout
);

// Behavioral modeling using always block
always @(*) begin
    case ({A, B, Cin})
        3'b000: begin Sum = 1'b0; Cout = 1'b0; end
        3'b001: begin Sum = 1'b1; Cout = 1'b0; end
        3'b010: begin Sum = 1'b1; Cout = 1'b0; end
        3'b011: begin Sum = 1'b0; Cout = 1'b1; end
        3'b100: begin Sum = 1'b1; Cout = 1'b0; end
        3'b101: begin Sum = 1'b0; Cout = 1'b1; end
        3'b110: begin Sum = 1'b0; Cout = 1'b1; end
        3'b111: begin Sum = 1'b1; Cout = 1'b1; end
        default: begin Sum = 1'b0; Cout = 1'b0; end
    endcase
end

endmodule
\end{lstlisting}

\textbf{Alternative Behavioral Style:}

\begin{lstlisting}[language=Verilog]
module full_adder_behavioral_alt(
    input A, B, Cin,
    output reg Sum, Cout
);

always @(*) begin
    {Cout, Sum} = A + B + Cin;
end

endmodule
\end{lstlisting}

\textbf{Truth Table:}

{\def\LTcaptype{none} % do not increment counter
\begin{longtable}[]{@{}lllll@{}}
\toprule\noalign{}
A & B & Cin & Sum & Cout \\
\midrule\noalign{}
\endhead
\bottomrule\noalign{}
\endlastfoot
0 & 0 & 0 & 0 & 0 \\
0 & 0 & 1 & 1 & 0 \\
0 & 1 & 0 & 1 & 0 \\
0 & 1 & 1 & 0 & 1 \\
1 & 0 & 0 & 1 & 0 \\
1 & 0 & 1 & 0 & 1 \\
1 & 1 & 0 & 0 & 1 \\
1 & 1 & 1 & 1 & 1 \\
\end{longtable}
}

\textbf{Testbench:}

\begin{lstlisting}[language=Verilog]
module tb_full_adder;
    reg A, B, Cin;
    wire Sum, Cout;
    
    full_adder_behavioral dut(.A(A), .B(B), .Cin(Cin), 
                             .Sum(Sum), .Cout(Cout));
    
    initial begin
$monitor("A=%b

B=%b Cin=%b | Sum=%b Cout=%b",

                 A, B, Cin, Sum, Cout);
        
        {A, B, Cin} = 3'b000; #10;
        {A, B, Cin} = 3'b001; #10;
        {A, B, Cin} = 3'b010; #10;
        {A, B, Cin} = 3'b011; #10;
        {A, B, Cin} = 3'b100; #10;
        {A, B, Cin} = 3'b101; #10;
        {A, B, Cin} = 3'b110; #10;
        {A, B, Cin} = 3'b111; #10;
        
        $finish;
    end
endmodule
\end{lstlisting}

\textbf{Behavioral Features:}

\begin{itemize}
\tightlist
\item
  \textbf{Always block}: Describes behavior, not structure
\item
  \textbf{Case statement}: Truth table implementation
\item
  \textbf{Automatic synthesis}: Tools generate optimized circuit
\end{itemize}

\end{solutionbox}
\begin{mnemonicbox}
``Behavioral modeling describes what circuit does,
not how''

\end{mnemonicbox}
\subsection*{Question 5(a) OR [3
marks]}\label{q5a}

\textbf{Implement NOR2 gate CMOS circuit with its stick diagram.}

\begin{solutionbox}

\textbf{CMOS NOR2 Circuit:}

\begin{lstlisting}
       VDD
        |
    +---+---+
    |       |
  +-+-+   +-+-+
  |pA |   |pB | PMOS (Parallel)
  +---+   +---+
    |       |
    +---Y---+
        |
      +-+-+
      |nA | NMOS (Series)
      +---+
        |
      +-+-+  
      |nB |
      +---+
        |
       GND
\end{lstlisting}

\textbf{Stick Diagram:}

\begin{lstlisting}
    VDD ----------------------- VDD Rail
     |           |
   +-+-+       +-+-+
   |pA|green  |pB|green        PMOS (Parallel)
   +-+-+       +-+-+
     |           |
     +-----------+
           |
           Y ------------------- Output
           |
         +-+-+
         |nA|red                NMOS (Series)
         +-+-+
           |
         +-+-+
         |nB|red
         +-+-+
           |
    GND ---+------------------- GND Rail

Legend:
- Green: P-diffusion
- Red: N-diffusion  
- Blue: Polysilicon gates
- Metal: Connections
\end{lstlisting}

\textbf{Layout Rules:}

\begin{itemize}
\tightlist
\item
  \textbf{PMOS}: Parallel connection for pull-up
\item
  \textbf{NMOS}: Series connection for pull-down\\
\item
  \textbf{Contacts}: Proper VDD/GND connections
\item
  \textbf{Spacing}: Meet minimum design rules
\end{itemize}

\end{solutionbox}
\begin{mnemonicbox}
``NOR gate: Parallel PMOS, Series NMOS''

\end{mnemonicbox}
\subsection*{Question 5(b) OR [4
marks]}\label{q5b}

\textbf{Implement 4 bit up counter using Verilog}

\begin{solutionbox}

\textbf{Verilog Code:}

\begin{lstlisting}[language=Verilog]
module counter_4bit_up(
    input clk,
    input reset,
    input enable,
    output reg [3:0] count
);

always @(posedge clk or posedge reset) begin
    if (reset) begin
        count <= 4'b0000;
    end else if (enable) begin
        if (count == 4'b1111) begin
            count <= 4'b0000;  // Rollover
        end else begin
            count <= count + 1;
        end
    end
    // If enable is low, hold current value
end

endmodule
\end{lstlisting}

\textbf{Enhanced Version with Overflow:}

\begin{lstlisting}[language=Verilog]
module counter_4bit_enhanced(
    input clk,
    input reset, 
    input enable,
    output reg [3:0] count,
    output overflow
);

always @(posedge clk or posedge reset) begin
    if (reset) begin
        count <= 4'b0000;
    end else if (enable) begin
        count <= count + 1;  // Natural rollover
    end
end

assign overflow = (count == 4'b1111) & enable;

endmodule
\end{lstlisting}

\textbf{Count Sequence:}

{\def\LTcaptype{none} % do not increment counter
\begin{longtable}[]{@{}lll@{}}
\toprule\noalign{}
Clock & Count[3:0] & Decimal \\
\midrule\noalign{}
\endhead
\bottomrule\noalign{}
\endlastfoot
1 & 0000 & 0 \\
2 & 0001 & 1 \\
3 & 0010 & 2 \\
\ldots{} & \ldots{} & \ldots{} \\
15 & 1110 & 14 \\
16 & 1111 & 15 \\
17 & 0000 & 0 (rollover) \\
\end{longtable}
}

\textbf{Testbench:}

\begin{lstlisting}[language=Verilog]
module tb_counter_4bit;
    reg clk, reset, enable;
    wire [3:0] count;
    
    counter_4bit_up dut(.clk(clk), .reset(reset), 
                       .enable(enable), .count(count));
    
    // Clock generation
    initial begin
        clk = 0;
        forever #5 clk = ~clk;
    end
    
    // Test sequence
    initial begin
        reset = 1; enable = 0;
        #10 reset = 0; enable = 1;
        #200 enable = 0;  // Stop counting
        #20 enable = 1;   // Resume
        #100 $finish;
    end
    
    // Monitor
    always @(posedge clk) begin
        $display("Time=%t Count=%d", $time, count);
    end
endmodule
\end{lstlisting}

\end{solutionbox}
\begin{mnemonicbox}
``Up counter: increment on each clock when enabled''

\end{mnemonicbox}
\subsection*{Question 5(c) OR [7
marks]}\label{q5c}

\textbf{Implement 3:8 decoder using behavioral modeling style in
Verilog.}

\begin{solutionbox}

\textbf{Verilog Code:}

\begin{lstlisting}[language=Verilog]
module decoder_3x8_behavioral(
    input [2:0] address,    // 3-bit address input
    input enable,           // Enable signal
    output reg [7:0] decode_out  // 8-bit decoded output
);

always @(*) begin
    if (enable) begin
        case (address)
            3'b000: decode_out = 8'b00000001;  // Y0
            3'b001: decode_out = 8'b00000010;  // Y1  
            3'b010: decode_out = 8'b00000100;  // Y2
            3'b011: decode_out = 8'b00001000;  // Y3
            3'b100: decode_out = 8'b00010000;  // Y4
            3'b101: decode_out = 8'b00100000;  // Y5
            3'b110: decode_out = 8'b01000000;  // Y6
            3'b111: decode_out = 8'b10000000;  // Y7
            default: decode_out = 8'b00000000;
        endcase
    end else begin
        decode_out = 8'b00000000;  // All outputs low when disabled
    end
end

endmodule
\end{lstlisting}

\textbf{Alternative Implementation:}

\begin{lstlisting}[language=Verilog]
module decoder_3x8_shift(
    input [2:0] address,
    input enable,
    output [7:0] decode_out
);

assign decode_out = enable ? (8'b00000001 << address) : 8'b00000000;

endmodule
\end{lstlisting}

\textbf{Truth Table:}

{\def\LTcaptype{none} % do not increment counter
\begin{longtable}[]{@{}lll@{}}
\toprule\noalign{}
Enable & Address[2:0] & decode\_out[7:0] \\
\midrule\noalign{}
\endhead
\bottomrule\noalign{}
\endlastfoot
0 & XXX & 00000000 \\
1 & 000 & 00000001 \\
1 & 001 & 00000010 \\
1 & 010 & 00000100 \\
1 & 011 & 00001000 \\
1 & 100 & 00010000 \\
1 & 101 & 00100000 \\
1 & 110 & 01000000 \\
1 & 111 & 10000000 \\
\end{longtable}
}

\textbf{Testbench:}

\begin{lstlisting}[language=Verilog]
module tb_decoder_3x8;
    reg [2:0] address;
    reg enable;
    wire [7:0] decode_out;
    
    decoder_3x8_behavioral dut(.address(address), .enable(enable), 
                              .decode_out(decode_out));
    
    initial begin
        $monitor("Enable=%b Address=%b | Output=%b", 
                 enable, address, decode_out);
        
        // Test with enable = 0
        enable = 0;
        for (int i = 0; i < 8; i++) begin
            address = i;
            #10;
        end
        
        // Test with enable = 1
        enable = 1;
        for (int i = 0; i < 8; i++) begin
            address = i;
            #10;
        end
        
        $finish;
    end
endmodule
\end{lstlisting}

\textbf{Applications:}

\begin{itemize}
\tightlist
\item
  \textbf{Memory addressing}: Select one of 8 memory locations
\item
  \textbf{Device selection}: Enable one of 8 peripheral devices
\item
  \textbf{Demultiplexing}: Route single input to selected output
\end{itemize}

\textbf{Design Features:}

\begin{itemize}
\tightlist
\item
  \textbf{One-hot encoding}: Only one output high at a time
\item
  \textbf{Enable control}: Global enable/disable functionality
\item
  \textbf{Full decoding}: All possible input combinations handled
\end{itemize}

\end{solutionbox}
\begin{mnemonicbox}
``3:8 Decoder - 3 inputs select 1 of 8 outputs''

\end{mnemonicbox}

\end{document}
