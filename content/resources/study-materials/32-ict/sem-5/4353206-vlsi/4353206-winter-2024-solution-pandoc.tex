\documentclass[10pt,a4paper]{article}

% content/resources/templates/preamble.tex
\usepackage[margin=0.6in]{geometry}
\author{Milav Dabgar}
\usepackage{amsmath,amssymb,amsthm}
\usepackage{booktabs}
\usepackage{multirow}
\usepackage{xcolor}
\usepackage{tcolorbox}
\tcbuselibrary{breakable,skins}
\usepackage[colorlinks=true,linkcolor=blue]{hyperref}
\usepackage{titlesec}
\usepackage{enumitem}
\usepackage{tikz}
\usepackage{pgfplots}
\usepackage{circuitikz}
\usepackage[version=4]{mhchem}
\usepackage{longtable}
\usepackage{array}
\usepackage{float}
\usepackage{caption}
\usepackage{listings}

\lstset{
  basicstyle=\small\ttfamily,
  breaklines=true,
  breakatwhitespace=false,
  postbreak=\mbox{\textcolor{red}{$\hookrightarrow$}\space},
  float=false,
  numbers=left,
  numberstyle=\tiny\color{gray},
  numbersep=10pt,
  xleftmargin=2em,
  keywordstyle=\color{blue},
  commentstyle=\color{green!60!black},
  stringstyle=\color{purple},
  backgroundcolor=\color{gray!5},
  showstringspaces=false,
  tabsize=2,
  captionpos=b,
  keepspaces=true,
  columns=flexible
}

\pgfplotsset{compat=1.18}
\usetikzlibrary{shapes,arrows,positioning,calc,patterns,decorations.pathmorphing,decorations.markings,arrows.meta}

% Color scheme
\definecolor{headcolor}{RGB}{0,102,204}
\definecolor{keycolor}{RGB}{220,20,60}
\definecolor{solutioncolor}{RGB}{34,139,34}
\definecolor{mnemoniccolor}{RGB}{148,0,211}
\definecolor{codecolor}{RGB}{0,0,100}

% Spacing
\setlength{\parskip}{3pt}
\setlist[itemize]{nosep}
\setlist[enumerate]{nosep}

% Title formatting
\titleformat{\section}{\Large\bfseries\color{headcolor}}{\thesection}{1em}{}
\titleformat{\subsection}{\large\bfseries\color{headcolor}}{\thesubsection}{1em}{}

% Pandoc tightlist compatibility
\providecommand{\tightlist}{%
  \setlength{\itemsep}{0pt}\setlength{\parskip}{0pt}}

% Pandoc longtable compatibility
\newcounter{none}
\def\thenone{}


% content/resources/templates/english-boxes.tex
% This file is currently empty - it exists to maintain consistency with the import structure.
% Add custom environments here if needed in the future.


\begin{document}

\begin{center}
{\Huge\bfseries\color{headcolor} Subject Name Solutions}\\[5pt]
{\LARGE 4353206 -- Winter 2024}\\[3pt]
{\large Semester 1 Study Material}\\[3pt]
{\normalsize\textit{Detailed Solutions and Explanations}}
\end{center}

\vspace{10pt}

\subsection*{Question 1(a) [3 marks]}\label{q1a}

\textbf{Draw all symbols for enhancement and depletion type MOSFET.}

\begin{solutionbox}

\textbf{Diagram:}

\begin{lstlisting}
Enhancement Type NMOS:           Enhancement Type PMOS:
    
    D                               D
    |                               |
G --+-- S                       G --+-- S
    |                               |
    B                               B
(No channel exists               (No channel exists
 without gate voltage)            without gate voltage)

Depletion Type NMOS:            Depletion Type PMOS:
    
    D                               D
    |                               |
G --+==-- S                     G --+==-- S
    |                               |
    B                               B
(Channel exists even            (Channel exists even
 without gate voltage)            without gate voltage)
\end{lstlisting}

\begin{itemize}
\tightlist
\item
  \textbf{Enhancement MOSFET}: Normal connection line between source and
  drain
\item
  \textbf{Depletion MOSFET}: Thick solid line indicating existing
  channel
\item
  \textbf{Arrow direction}: Points inward for NMOS, outward for PMOS
\end{itemize}

\end{solutionbox}
\begin{mnemonicbox}
``Enhancement Needs voltage, Depletion has Default
channel''

\end{mnemonicbox}
\subsection*{Question 1(b) [4 marks]}\label{q1b}

\textbf{Define: 1) Hierarchy 2) Regularity}

\begin{solutionbox}

{\def\LTcaptype{none} % do not increment counter
\begin{longtable}[]{@{}
  >{\raggedright\arraybackslash}p{(\linewidth - 4\tabcolsep) * \real{0.1935}}
  >{\raggedright\arraybackslash}p{(\linewidth - 4\tabcolsep) * \real{0.3871}}
  >{\raggedright\arraybackslash}p{(\linewidth - 4\tabcolsep) * \real{0.4194}}@{}}
\toprule\noalign{}
\begin{minipage}[b]{\linewidth}\raggedright
Term
\end{minipage} & \begin{minipage}[b]{\linewidth}\raggedright
Definition
\end{minipage} & \begin{minipage}[b]{\linewidth}\raggedright
Application
\end{minipage} \\
\midrule\noalign{}
\endhead
\bottomrule\noalign{}
\endlastfoot
\textbf{Hierarchy} & Top-down design approach where complex systems are
broken into smaller, manageable modules & Used in VLSI design flow from
system level to transistor level \\
\textbf{Regularity} & Design technique using repeated identical
structures to reduce complexity & Memory arrays, processor datapaths use
regular structures \\
\end{longtable}
}

\begin{itemize}
\tightlist
\item
  \textbf{Hierarchy benefits}: Easier design verification, modular
  testing, team collaboration
\item
  \textbf{Regularity advantages}: Reduced design time, better yield,
  simplified layout
\item
  \textbf{Design flow}: System \rightarrow Behavioral \rightarrow RTL \rightarrow Gate \rightarrow Layout
\item
  \textbf{Regular structures}: ROM arrays, cache memories, ALU blocks
\end{itemize}

\end{solutionbox}
\begin{mnemonicbox}
``Hierarchy Helps organize, Regularity Reduces
complexity''

\end{mnemonicbox}
\subsection*{Question 1(c) [7 marks]}\label{q1c}

\textbf{Explain MOS under external bias.}

\begin{solutionbox}


{\def\LTcaptype{none} % do not increment counter
\vspace{-5pt}
\captionof{table}{MOS Bias Conditions}
\vspace{-10pt}
\begin{longtable}[]{@{}
  >{\raggedright\arraybackslash}p{(\linewidth - 6\tabcolsep) * \real{0.2540}}
  >{\raggedright\arraybackslash}p{(\linewidth - 6\tabcolsep) * \real{0.2222}}
  >{\raggedright\arraybackslash}p{(\linewidth - 6\tabcolsep) * \real{0.3016}}
  >{\raggedright\arraybackslash}p{(\linewidth - 6\tabcolsep) * \real{0.2222}}@{}}
\toprule\noalign{}
\begin{minipage}[b]{\linewidth}\raggedright
Bias Condition
\end{minipage} & \begin{minipage}[b]{\linewidth}\raggedright
Gate Voltage
\end{minipage} & \begin{minipage}[b]{\linewidth}\raggedright
Channel Formation
\end{minipage} & \begin{minipage}[b]{\linewidth}\raggedright
Current Flow
\end{minipage} \\
\midrule\noalign{}
\endhead
\bottomrule\noalign{}
\endlastfoot
\textbf{Accumulation} & VG \textless{} 0 (NMOS) & Majority carriers
accumulate & No channel \\
\textbf{Depletion} & 0 \textless{} VG \textless{} VT & Depletion region
forms & Minimal current \\
\textbf{Inversion} & VG \textgreater{} VT & Minority carriers form
channel & Channel conducts \\
\end{longtable}
}

\textbf{Diagram:}

\includegraphics[width=1\linewidth,height=\textheight,keepaspectratio]{mermaid-7ca2b1e9.pdf}

\begin{itemize}
\tightlist
\item
  \textbf{Band bending}: External voltage bends energy bands at
  oxide-silicon interface
\item
  \textbf{Threshold voltage}: Minimum gate voltage needed for channel
  formation
\item
  \textbf{Surface potential}: Controls carrier concentration at silicon
  surface
\item
  \textbf{Capacitance variation}: Changes with bias conditions
\end{itemize}

\end{solutionbox}
\begin{mnemonicbox}
``Accumulation Attracts, Depletion Depletes,
Inversion Inverts carriers''

\end{mnemonicbox}
\subsection*{Question 1(c) OR [7
marks]}\label{q1c}

\textbf{What is the need for scaling? Explain types of scaling with its
effect.}

\begin{solutionbox}

\textbf{Need for Scaling:}

{\def\LTcaptype{none} % do not increment counter
\begin{longtable}[]{@{}
  >{\raggedright\arraybackslash}p{(\linewidth - 4\tabcolsep) * \real{0.3793}}
  >{\raggedright\arraybackslash}p{(\linewidth - 4\tabcolsep) * \real{0.3103}}
  >{\raggedright\arraybackslash}p{(\linewidth - 4\tabcolsep) * \real{0.3103}}@{}}
\toprule\noalign{}
\begin{minipage}[b]{\linewidth}\raggedright
Parameter
\end{minipage} & \begin{minipage}[b]{\linewidth}\raggedright
Benefit
\end{minipage} & \begin{minipage}[b]{\linewidth}\raggedright
Impact
\end{minipage} \\
\midrule\noalign{}
\endhead
\bottomrule\noalign{}
\endlastfoot
\textbf{Area reduction} & More transistors per chip & Higher integration
density \\
\textbf{Speed increase} & Reduced delays & Better performance \\
\textbf{Power reduction} & Lower power consumption & Portable devices \\
\textbf{Cost reduction} & Cheaper per function & Market
competitiveness \\
\end{longtable}
}

\textbf{Types of Scaling:}

\includegraphics[width=1\linewidth,height=\textheight,keepaspectratio]{mermaid-c0c0783b.pdf}

\begin{itemize}
\tightlist
\item
  \textbf{Full voltage scaling}: Length, width, voltage all scaled by
  factor α
\item
  \textbf{Constant voltage scaling}: Dimensions scaled, voltage
  unchanged
\item
  \textbf{Power density}: Remains constant in full scaling, increases in
  constant voltage
\item
  \textbf{Electric field}: Maintained in full scaling
\end{itemize}

\end{solutionbox}
\begin{mnemonicbox}
``Scaling Saves Space, Speed, and Spending''

\end{mnemonicbox}
\subsection*{Question 2(a) [3 marks]}\label{q2a}

\textbf{Write short note on FPGA.}

\begin{solutionbox}


{\def\LTcaptype{none} % do not increment counter
\vspace{-5pt}
\captionof{table}{FPGA Characteristics}
\vspace{-10pt}
\begin{longtable}[]{@{}
  >{\raggedright\arraybackslash}p{(\linewidth - 4\tabcolsep) * \real{0.2727}}
  >{\raggedright\arraybackslash}p{(\linewidth - 4\tabcolsep) * \real{0.3939}}
  >{\raggedright\arraybackslash}p{(\linewidth - 4\tabcolsep) * \real{0.3333}}@{}}
\toprule\noalign{}
\begin{minipage}[b]{\linewidth}\raggedright
Feature
\end{minipage} & \begin{minipage}[b]{\linewidth}\raggedright
Description
\end{minipage} & \begin{minipage}[b]{\linewidth}\raggedright
Advantage
\end{minipage} \\
\midrule\noalign{}
\endhead
\bottomrule\noalign{}
\endlastfoot
\textbf{Field Programmable} & Configurable after manufacturing &
Flexibility in design \\
\textbf{Gate Array} & Array of logic blocks & Parallel processing \\
\textbf{Reconfigurable} & Can be reprogrammed & Prototype development \\
\end{longtable}
}

\begin{itemize}
\tightlist
\item
  \textbf{Applications}: Digital signal processing, embedded systems,
  prototyping
\item
  \textbf{Architecture}: CLBs (Configurable Logic Blocks) connected by
  routing matrix
\item
  \textbf{Programming}: SRAM-based configuration memory
\item
  \textbf{Vendors}: Xilinx, Altera (Intel), Microsemi
\end{itemize}

\end{solutionbox}
\begin{mnemonicbox}
``FPGA: Flexible Programming for Gate Arrays''

\end{mnemonicbox}
\subsection*{Question 2(b) [4 marks]}\label{q2b}

\textbf{Compare semi-custom and full custom design methodologies.}

\begin{solutionbox}

{\def\LTcaptype{none} % do not increment counter
\begin{longtable}[]{@{}
  >{\raggedright\arraybackslash}p{(\linewidth - 4\tabcolsep) * \real{0.2973}}
  >{\raggedright\arraybackslash}p{(\linewidth - 4\tabcolsep) * \real{0.3514}}
  >{\raggedright\arraybackslash}p{(\linewidth - 4\tabcolsep) * \real{0.3514}}@{}}
\toprule\noalign{}
\begin{minipage}[b]{\linewidth}\raggedright
Parameter
\end{minipage} & \begin{minipage}[b]{\linewidth}\raggedright
Semi-Custom
\end{minipage} & \begin{minipage}[b]{\linewidth}\raggedright
Full Custom
\end{minipage} \\
\midrule\noalign{}
\endhead
\bottomrule\noalign{}
\endlastfoot
\textbf{Design Time} & Shorter (weeks) & Longer (months) \\
\textbf{Cost} & Lower development cost & Higher development cost \\
\textbf{Performance} & Moderate performance & Highest performance \\
\textbf{Area Efficiency} & Less efficient & Most efficient \\
\textbf{Applications} & ASICs, moderate volume & Microprocessors, high
volume \\
\textbf{Design Effort} & Standard cells used & Every transistor
designed \\
\end{longtable}
}

\begin{itemize}
\tightlist
\item
  \textbf{Semi-custom}: Uses pre-designed standard cells and gate arrays
\item
  \textbf{Full custom}: Complete transistor-level design optimization
\item
  \textbf{Trade-offs}: Time vs performance, cost vs efficiency
\item
  \textbf{Market fit}: Semi-custom for most applications, full custom
  for specialized needs
\end{itemize}

\end{solutionbox}
\begin{mnemonicbox}
``Semi-custom is Standard, Full custom is Finest''

\end{mnemonicbox}
\subsection*{Question 2(c) [7 marks]}\label{q2c}

\textbf{Explain MOSFET operation for 1) 0\textless VDS\textless VDSAT 2)
VDS = VDSAT 3) VDS \textgreater{} VDSAT}

\begin{solutionbox}

\textbf{Operating Regions:}

{\def\LTcaptype{none} % do not increment counter
\begin{longtable}[]{@{}
  >{\raggedright\arraybackslash}p{(\linewidth - 6\tabcolsep) * \real{0.1739}}
  >{\raggedright\arraybackslash}p{(\linewidth - 6\tabcolsep) * \real{0.2391}}
  >{\raggedright\arraybackslash}p{(\linewidth - 6\tabcolsep) * \real{0.1957}}
  >{\raggedright\arraybackslash}p{(\linewidth - 6\tabcolsep) * \real{0.3913}}@{}}
\toprule\noalign{}
\begin{minipage}[b]{\linewidth}\raggedright
Region
\end{minipage} & \begin{minipage}[b]{\linewidth}\raggedright
Condition
\end{minipage} & \begin{minipage}[b]{\linewidth}\raggedright
Channel
\end{minipage} & \begin{minipage}[b]{\linewidth}\raggedright
Current Behavior
\end{minipage} \\
\midrule\noalign{}
\endhead
\bottomrule\noalign{}
\endlastfoot
\textbf{Linear} & 0 \textless{} VDS \textless{} VDSAT & Uniform channel
& ID ∝ VDS \\
\textbf{Saturation onset} & VDS = VDSAT & Pinch-off begins & Maximum
linear current \\
\textbf{Saturation} & VDS \textgreater{} VDSAT & Pinched channel & ID
constant \\
\end{longtable}
}

\textbf{Diagram:}

\includegraphics[width=1\linewidth,height=\textheight,keepaspectratio]{mermaid-a6b0f986.pdf}

\begin{itemize}
\tightlist
\item
  \textbf{Linear region}: Channel acts as voltage-controlled resistor
\item
  \textbf{Saturation region}: Current controlled by gate voltage only
\item
  \textbf{VDSAT calculation}: VDSAT = VGS - VT
\item
  \textbf{Current equations}: Different mathematical models for each
  region
\end{itemize}

\end{solutionbox}
\begin{mnemonicbox}
``Linear Likes VDS, Saturation Says no more''

\end{mnemonicbox}
\subsection*{Question 2(a) OR [3
marks]}\label{q2a}

\textbf{Explain standard cell-based design.}

\begin{solutionbox}


{\def\LTcaptype{none} % do not increment counter
\vspace{-5pt}
\captionof{table}{Standard Cell Design}
\vspace{-10pt}
\begin{longtable}[]{@{}
  >{\raggedright\arraybackslash}p{(\linewidth - 4\tabcolsep) * \real{0.3333}}
  >{\raggedright\arraybackslash}p{(\linewidth - 4\tabcolsep) * \real{0.3939}}
  >{\raggedright\arraybackslash}p{(\linewidth - 4\tabcolsep) * \real{0.2727}}@{}}
\toprule\noalign{}
\begin{minipage}[b]{\linewidth}\raggedright
Component
\end{minipage} & \begin{minipage}[b]{\linewidth}\raggedright
Description
\end{minipage} & \begin{minipage}[b]{\linewidth}\raggedright
Benefit
\end{minipage} \\
\midrule\noalign{}
\endhead
\bottomrule\noalign{}
\endlastfoot
\textbf{Standard Cells} & Pre-designed logic gates & Faster design \\
\textbf{Cell Library} & Collection of characterized cells & Predictable
performance \\
\textbf{Place \& Route} & Automated layout generation & Reduced design
time \\
\end{longtable}
}

\begin{itemize}
\tightlist
\item
  \textbf{Process}: Logic synthesis \rightarrow Placement \rightarrow Routing \rightarrow Verification
\item
  \textbf{Cell types}: Basic gates, flip-flops, latches, complex
  functions
\item
  \textbf{Automation}: EDA tools handle physical implementation
\item
  \textbf{Quality}: Balanced performance, area, and power
\end{itemize}

\end{solutionbox}
\begin{mnemonicbox}
``Standard Cells Speed up Synthesis''

\end{mnemonicbox}
\subsection*{Question 2(b) OR [4
marks]}\label{q2b}

\textbf{Draw and explain Y-chart.}

\begin{solutionbox}

\textbf{Diagram:}

\includegraphics[width=1\linewidth,height=\textheight,keepaspectratio]{mermaid-b5e14ed9.pdf}

{\def\LTcaptype{none} % do not increment counter
\begin{longtable}[]{@{}lll@{}}
\toprule\noalign{}
Domain & Description & Examples \\
\midrule\noalign{}
\endhead
\bottomrule\noalign{}
\endlastfoot
\textbf{Behavioral} & What system does & Algorithms, RTL code \\
\textbf{Structural} & How system is built & Gates, modules,
processors \\
\textbf{Physical} & Physical implementation & Layout, floorplan,
masks \\
\end{longtable}
}

\begin{itemize}
\tightlist
\item
  \textbf{Design flow}: Move from outer ring (system) to inner ring
  (device)
\item
  \textbf{Abstraction levels}: Each ring represents different detail
  level
\item
  \textbf{Domain interaction}: Can move between domains at same
  abstraction
\item
  \textbf{VLSI design}: Covers all three domains and abstraction levels
\end{itemize}

\end{solutionbox}
\begin{mnemonicbox}
``Y-chart: behaVior, Structure, PhYsical''

\end{mnemonicbox}
\subsection*{Question 2(c) OR [7
marks]}\label{q2c}

\textbf{Explain gradual channel approximation for MOSFET current-voltage
characteristics.}

\begin{solutionbox}

\textbf{Assumptions:}

{\def\LTcaptype{none} % do not increment counter
\begin{longtable}[]{@{}
  >{\raggedright\arraybackslash}p{(\linewidth - 4\tabcolsep) * \real{0.3000}}
  >{\raggedright\arraybackslash}p{(\linewidth - 4\tabcolsep) * \real{0.3250}}
  >{\raggedright\arraybackslash}p{(\linewidth - 4\tabcolsep) * \real{0.3750}}@{}}
\toprule\noalign{}
\begin{minipage}[b]{\linewidth}\raggedright
Assumption
\end{minipage} & \begin{minipage}[b]{\linewidth}\raggedright
Description
\end{minipage} & \begin{minipage}[b]{\linewidth}\raggedright
Justification
\end{minipage} \\
\midrule\noalign{}
\endhead
\bottomrule\noalign{}
\endlastfoot
\textbf{Gradual channel} & Channel length \textgreater\textgreater{}
channel depth & Long channel devices \\
\textbf{1D analysis} & Current flows only in x-direction & Simplifies
mathematics \\
\textbf{Drift current} & Neglect diffusion current & High field
conditions \\
\textbf{Charge sheet} & Mobile charge in thin sheet & Small inversion
layer \\
\end{longtable}
}

\textbf{Current Derivation:}

\begin{itemize}
\tightlist
\item
  \textbf{Drain current}: ID = μn Cox (W/L) [(VGS-VT)VDS - VDS^{2}/2]
\item
  \textbf{Linear region}: When VDS \textless{} VGS-VT
\item
  \textbf{Saturation}: When VDS \geq VGS-VT, ID = μn Cox (W/2L)(VGS-VT)^{2}
\item
  \textbf{Channel charge}: Varies linearly from source to drain
\end{itemize}

\textbf{Limitations:}

\begin{itemize}
\tightlist
\item
  \textbf{Short channel effects}: Gradual approximation breaks down
\item
  \textbf{Velocity saturation}: High field effects not included
\item
  \textbf{2D effects}: Ignored in simple model
\end{itemize}

\end{solutionbox}
\begin{mnemonicbox}
``Gradual change Gives simple Gain equations''

\end{mnemonicbox}
\subsection*{Question 3(a) [3 marks]}\label{q3a}

\textbf{Draw symbol and write truth table of ideal inverter. Draw and
explain VTC of ideal inverter.}

\begin{solutionbox}

\textbf{Symbol and Truth Table:}

\begin{lstlisting}
    VIN ------>|>o----- VOUT
              NOT
\end{lstlisting}

{\def\LTcaptype{none} % do not increment counter
\begin{longtable}[]{@{}ll@{}}
\toprule\noalign{}
VIN & VOUT \\
\midrule\noalign{}
\endhead
\bottomrule\noalign{}
\endlastfoot
0 & 1 \\
1 & 0 \\
\end{longtable}
}

\textbf{VTC (Voltage Transfer Characteristic):}

\begin{lstlisting}
VOUT ^
     |
 VDD +-----+
     |     |
     |     |
     |     +------
     |           
     +--------------> VIN
     0   VDD/2   VDD
\end{lstlisting}

\begin{itemize}
\tightlist
\item
  \textbf{Ideal characteristics}: Sharp transition at VDD/2
\item
  \textbf{Noise margins}: NMH = NML = VDD/2
\item
  \textbf{Gain}: Infinite at switching point
\item
  \textbf{Power consumption}: Zero static power
\end{itemize}

\end{solutionbox}
\begin{mnemonicbox}
``Ideal Inverter: Infinite gain, Instant switching''

\end{mnemonicbox}
\subsection*{Question 3(b) [4 marks]}\label{q3b}

\textbf{Explain generalized inverter circuit with its VTC.}

\begin{solutionbox}

\textbf{Circuit Configuration:}

{\def\LTcaptype{none} % do not increment counter
\begin{longtable}[]{@{}lll@{}}
\toprule\noalign{}
Component & Function & Characteristics \\
\midrule\noalign{}
\endhead
\bottomrule\noalign{}
\endlastfoot
\textbf{Driver transistor} & Pull-down device & Controls switching \\
\textbf{Load device} & Pull-up element & Provides high output \\
\textbf{Supply voltage} & Power source & Determines logic levels \\
\end{longtable}
}

\textbf{VTC Regions:}

\includegraphics[width=1\linewidth,height=\textheight,keepaspectratio]{mermaid-79851e26.pdf}

\begin{itemize}
\tightlist
\item
  \textbf{Load line analysis}: Intersection of driver and load
  characteristics
\item
  \textbf{Switching threshold}: Determined by device sizing ratio
\item
  \textbf{Noise margins}: Depend on transition sharpness
\item
  \textbf{Power dissipation}: Static current during transition
\end{itemize}

\end{solutionbox}
\begin{mnemonicbox}
``Generalized design: Driver pulls Down, Load lifts
Up''

\end{mnemonicbox}
\subsection*{Question 3(c) [7 marks]}\label{q3c}

\textbf{Describe depletion load nMOS inverter with its circuit,
operating region and VTC.}

\begin{solutionbox}

\textbf{Circuit Diagram:}

\begin{lstlisting}
           VDD
            |
         +--+--+ VGS = 0
    VG --|     |
         |  T2 | (Depletion load)
         +-----+
            |
         +--+--+
    VIN -+     +- VOUT
         |  T1 |
         +-----+
            |
           GND
\end{lstlisting}

\textbf{Operating Regions:}

{\def\LTcaptype{none} % do not increment counter
\begin{longtable}[]{@{}llll@{}}
\toprule\noalign{}
Input State & T1 State & T2 State & Output \\
\midrule\noalign{}
\endhead
\bottomrule\noalign{}
\endlastfoot
\textbf{VIN = 0} & OFF & ON (depletion) & VOUT = VDD-VT \\
\textbf{VIN = VDD} & ON & ON (resistive) & VOUT = VOL \\
\end{longtable}
}

\textbf{VTC Analysis:}

\includegraphics[width=1\linewidth,height=\textheight,keepaspectratio]{mermaid-771bbfba.pdf}

\begin{itemize}
\tightlist
\item
  \textbf{Advantages}: Simple fabrication, good drive capability
\item
  \textbf{Disadvantages}: Degraded high output, static power consumption
\item
  \textbf{Applications}: Early NMOS logic families
\item
  \textbf{Design considerations}: Width ratio affects switching point
\end{itemize}

\end{solutionbox}
\begin{mnemonicbox}
``Depletion Device Delivers Decent drive''

\end{mnemonicbox}
\subsection*{Question 3(a) OR [3
marks]}\label{q3a}

\textbf{Explain noise margin.}

\begin{solutionbox}

\textbf{Definition and Parameters:}

{\def\LTcaptype{none} % do not increment counter
\begin{longtable}[]{@{}lll@{}}
\toprule\noalign{}
Parameter & Description & Formula \\
\midrule\noalign{}
\endhead
\bottomrule\noalign{}
\endlastfoot
\textbf{NMH} & High noise margin & NMH = VOH - VIH \\
\textbf{NML} & Low noise margin & NML = VIL - VOL \\
\textbf{VOH} & Output high voltage & Minimum high output \\
\textbf{VOL} & Output low voltage & Maximum low output \\
\textbf{VIH} & Input high threshold & Minimum input high \\
\textbf{VIL} & Input low threshold & Maximum input low \\
\end{longtable}
}

\begin{itemize}
\tightlist
\item
  \textbf{Significance}: Measure of circuit's immunity to noise
\item
  \textbf{Design goal}: Maximize both NMH and NML
\item
  \textbf{Trade-offs}: Noise margin vs speed vs power
\item
  \textbf{Applications}: Critical in digital system design
\end{itemize}

\end{solutionbox}
\begin{mnemonicbox}
``Noise Margins Maintain signal integrity''

\end{mnemonicbox}
\subsection*{Question 3(b) OR [4
marks]}\label{q3b}

\textbf{Explain resistive load inverter.}

\begin{solutionbox}

\textbf{Circuit and Analysis:}

{\def\LTcaptype{none} % do not increment counter
\begin{longtable}[]{@{}lll@{}}
\toprule\noalign{}
Component & Function & Characteristics \\
\midrule\noalign{}
\endhead
\bottomrule\noalign{}
\endlastfoot
\textbf{NMOS transistor} & Switching device & Variable resistance \\
\textbf{Load resistor} & Pull-up element & Fixed resistance RL \\
\textbf{Power supply} & Voltage source & Provides VDD \\
\end{longtable}
}

\textbf{Operating Principle:}

\begin{itemize}
\tightlist
\item
  \textbf{High input}: Transistor ON, VOUT = ID \times RL (low)
\item
  \textbf{Low input}: Transistor OFF, VOUT = VDD (high)
\item
  \textbf{Current path}: Always through resistor when output low
\item
  \textbf{Power consumption}: Static power = VDD^{2}/RL
\end{itemize}

\textbf{Advantages and Disadvantages:}

\begin{itemize}
\tightlist
\item
  \textbf{Simple design}: Easy to understand and implement
\item
  \textbf{Poor performance}: High static power, slow switching
\item
  \textbf{Limited use}: Mainly for understanding concepts
\end{itemize}

\end{solutionbox}
\begin{mnemonicbox}
``Resistor Restricts current, Reduces performance''

\end{mnemonicbox}
\subsection*{Question 3(c) OR [7
marks]}\label{q3c}

\textbf{Explain CMOS inverter with its VTC.}

\begin{solutionbox}

\textbf{Circuit Configuration:}

\begin{lstlisting}
           VDD
            |
         +--+--+
    VIN -+     +- VOUT
         | PMOS|
         +-----+
            |
         +--+--+
    VIN -+     |
         | NMOS+- VOUT
         +-----+
            |
           GND
\end{lstlisting}

\textbf{VTC Regions:}

{\def\LTcaptype{none} % do not increment counter
\begin{longtable}[]{@{}lllll@{}}
\toprule\noalign{}
Region & Input Range & PMOS State & NMOS State & Output \\
\midrule\noalign{}
\endhead
\bottomrule\noalign{}
\endlastfoot
\textbf{1} & VIN \textless{} VTN & ON & OFF & VDD \\
\textbf{2} & VTN \textless{} VIN \textless{} VDD/2 & ON & ON &
Transition \\
\textbf{3} & VDD/2 \textless{} VIN \textless{} VDD+VTP & ON & ON &
Transition \\
\textbf{4} & VIN \textgreater{} VDD+VTP & OFF & ON & 0 \\
\end{longtable}
}

\textbf{Key Characteristics:}

\includegraphics[width=1\linewidth,height=\textheight,keepaspectratio]{mermaid-2b62c17e.pdf}

\begin{itemize}
\tightlist
\item
  \textbf{Complementary operation}: Only one transistor conducts in
  steady state
\item
  \textbf{Switching point}: Determined by PMOS/NMOS ratio
\item
  \textbf{Power efficiency}: Minimal static power consumption
\item
  \textbf{Noise immunity}: Excellent noise margins
\end{itemize}

\end{solutionbox}
\begin{mnemonicbox}
``CMOS: Complementary for Complete performance''

\end{mnemonicbox}
\subsection*{Question 4(a) [3 marks]}\label{q4a}

\textbf{Draw AOI with CMOS implementation.}

\begin{solutionbox}

\textbf{AOI (AND-OR-INVERT) Logic:} Y = (AB + CD)'

\textbf{CMOS Implementation:}

\begin{lstlisting}
        VDD
         |
    +----+----+
    |         |
   PMOS     PMOS  (A')
    A'       B'
    |         |
    +----+----+
         |
    +----+----+
    |         |
   PMOS     PMOS  (C')
    C'       D'
    |         |
    +----+----+-- VOUT
         |
    +----+----+
    |         |
   NMOS     NMOS  (Series: AB)
    A        B
    |         |
    +----+----+
         |
    +----+----+
    |         |
   NMOS     NMOS  (Parallel: CD)
    C        D
    |         |
    +----+----+
         |
        GND
\end{lstlisting}

\begin{itemize}
\tightlist
\item
  \textbf{Pull-up network}: PMOS transistors in series-parallel
\item
  \textbf{Pull-down network}: NMOS transistors in parallel-series
\item
  \textbf{Duality}: Pull-up and pull-down are complements
\end{itemize}

\end{solutionbox}
\begin{mnemonicbox}
``AOI: AND-OR then Invert''

\end{mnemonicbox}
\subsection*{Question 4(b) [4 marks]}\label{q4b}

\textbf{Implement two input NOR and NAND gate using depletion load
nMOS.}

\begin{solutionbox}

\textbf{NOR Gate:}

\begin{lstlisting}
        VDD
         |
      +--+--+ (Depletion load)
 VG --|     |
      |     |
      +-----+-- VOUT
         |
    +----+----+
    |         |
   NMOS     NMOS  (Parallel)
    A        B
    |         |
    +----+----+
         |
        GND
\end{lstlisting}

\textbf{NAND Gate:}

\begin{lstlisting}
        VDD
         |
      +--+--+ (Depletion load)
 VG --|     |
      |     |
      +-----+-- VOUT
         |
      +--+--+
  A --|     |
      | NMOS|  (Series)
      +-----+
         |
      +--+--+
  B --|     |
      | NMOS|
      +-----+
         |
        GND
\end{lstlisting}

\textbf{Truth Tables:}

{\def\LTcaptype{none} % do not increment counter
\begin{longtable}[]{@{}llll@{}}
\toprule\noalign{}
A & B & NOR & NAND \\
\midrule\noalign{}
\endhead
\bottomrule\noalign{}
\endlastfoot
0 & 0 & 1 & 1 \\
0 & 1 & 0 & 1 \\
1 & 0 & 0 & 1 \\
1 & 1 & 0 & 0 \\
\end{longtable}
}

\end{solutionbox}
\begin{mnemonicbox}
``NOR needs None high, NAND Needs All high to be
low''

\end{mnemonicbox}
\subsection*{Question 4(c) [7 marks]}\label{q4c}

\textbf{Implement CMOS SR latch using NOR2 and NAND2 gates.}

\begin{solutionbox}

\textbf{SR Latch using NOR Gates:}

\begin{lstlisting}
    S ----+---[NOR]---+---- Q
          |           |
          +-----+     |
                |     |
          +-----+     |
          |           |
    R ----+---[NOR]---+---- Q'
                      |
                      +-----
\end{lstlisting}

\textbf{CMOS NOR Gate Implementation:}

\includegraphics[width=1\linewidth,height=\textheight,keepaspectratio]{mermaid-f44db2d3.pdf}

\textbf{State Table:}

{\def\LTcaptype{none} % do not increment counter
\begin{longtable}[]{@{}lllll@{}}
\toprule\noalign{}
S & R & Q(n+1) & Q'(n+1) & Action \\
\midrule\noalign{}
\endhead
\bottomrule\noalign{}
\endlastfoot
0 & 0 & Q(n) & Q'(n) & Hold \\
0 & 1 & 0 & 1 & Reset \\
1 & 0 & 1 & 0 & Set \\
1 & 1 & 0 & 0 & Invalid \\
\end{longtable}
}

\begin{itemize}
\tightlist
\item
  \textbf{Cross-coupled structure}: Output of each gate feeds other's
  input
\item
  \textbf{Bistable operation}: Two stable states (Set and Reset)
\item
  \textbf{Memory element}: Stores one bit of information
\item
  \textbf{Clock independence}: Asynchronous operation
\end{itemize}

\end{solutionbox}
\begin{mnemonicbox}
``SR latch: Set-Reset with cross-coupled gates''

\end{mnemonicbox}
\subsection*{Question 4(a) OR [3
marks]}\label{q4a}

\textbf{Implement XOR function using CMOS.}

\begin{solutionbox}

\textbf{XOR Truth Table:}

{\def\LTcaptype{none} % do not increment counter
\begin{longtable}[]{@{}lll@{}}
\toprule\noalign{}
A & B & Y = A\oplusB \\
\midrule\noalign{}
\endhead
\bottomrule\noalign{}
\endlastfoot
0 & 0 & 0 \\
0 & 1 & 1 \\
1 & 0 & 1 \\
1 & 1 & 0 \\
\end{longtable}
}

\textbf{CMOS XOR Implementation:}

\begin{lstlisting}
        VDD
         |
    +----+----+
    |         |
 A'-+PMOS  PMOS+-B'
    |         |
    +----+----+
         |
    +----+----+
    |         |
 B'-+PMOS  PMOS+-A'
    |         |
    +----+----+-- VOUT
         |
    +----+----+
    |         |
 A--+NMOS  NMOS+-B
    |         |
    +----+----+
         |
    +----+----+
    |         |
 B--+NMOS  NMOS+-A
    |         |
    +----+----+
         |
        GND
\end{lstlisting}

\begin{itemize}
\tightlist
\item
  \textbf{Function}: Y = AB' + A'B
\item
  \textbf{Transistor count}: 8 transistors (4 PMOS + 4 NMOS)
\item
  \textbf{Alternative}: Transmission gate implementation
\end{itemize}

\end{solutionbox}
\begin{mnemonicbox}
``XOR: eXclusive OR, different inputs give 1''

\end{mnemonicbox}
\subsection*{Question 4(b) OR [4
marks]}\label{q4b}

\textbf{Implement two input NOR and NAND gate using CMOS.}

\begin{solutionbox}

\textbf{CMOS NOR Gate:}

\begin{lstlisting}
        VDD
         |
    +----+----+
    |         |
 A'-+PMOS  PMOS+-B'  (Series)
    |         |
    +----+----+-- VOUT
         |
    +----+----+
    |         |
 A--+NMOS     +-B
    |    NMOS |      (Parallel)
    +----+----+
         |
        GND
\end{lstlisting}

\textbf{CMOS NAND Gate:}

\begin{lstlisting}
        VDD
         |
    +----+----+
    |         |
 A'-+PMOS     +-B'  (Parallel)
    |    PMOS |
    +----+----+-- VOUT
         |
    +----+----+
    |         |
 A--+NMOS  NMOS+-B  (Series)
    |         |
    +----+----+
         |
        GND
\end{lstlisting}

\textbf{Design Rules:}

{\def\LTcaptype{none} % do not increment counter
\begin{longtable}[]{@{}lll@{}}
\toprule\noalign{}
Gate & Pull-up Network & Pull-down Network \\
\midrule\noalign{}
\endhead
\bottomrule\noalign{}
\endlastfoot
\textbf{NAND} & PMOS in parallel & NMOS in series \\
\textbf{NOR} & PMOS in series & NMOS in parallel \\
\end{longtable}
}

\end{solutionbox}
\begin{mnemonicbox}
``NAND: Not AND, NOR: Not OR - complement the
networks''

\end{mnemonicbox}
\subsection*{Question 4(c) OR [7
marks]}\label{q4c}

\textbf{Implement Y=[PQ+R(S+T)]' Boolean equation using depletion
load nMOS and CMOS.}

\begin{solutionbox}

\textbf{Boolean Analysis:}

\begin{itemize}
\tightlist
\item
  Function: Y = [PQ + R(S+T)]'
\item
  Expanded: Y = [PQ + RS + RT]'
\item
  De Morgan: Y = (PQ)' · (RS)' · (RT)'
\item
  Final: Y = (P'+Q') · (R'+S') · (R'+T')
\end{itemize}

\textbf{nMOS Implementation:}

\begin{lstlisting}
        VDD
         |
      +--+--+ (Depletion load)
      |     |
      +-----+-- VOUT
         |
    P--+NMOS+--+
              |
    Q--+NMOS+--+
              |
              +-- (PQ branch)
              |
    R--+NMOS+--+
              |
         +----+----+
         |         |
    S--+NMOS   NMOS+--T
         |         |
         +---------+
              |
             GND
\end{lstlisting}

\textbf{CMOS Implementation:}

\includegraphics[width=1\linewidth,height=\textheight,keepaspectratio]{mermaid-dcf0c763.pdf}

\begin{itemize}
\tightlist
\item
  \textbf{nMOS characteristics}: Simple but with static power
\item
  \textbf{CMOS advantages}: No static power, full swing
\item
  \textbf{Complexity}: 7 transistors for nMOS, 14 for CMOS
\item
  \textbf{Performance}: CMOS faster and more efficient
\end{itemize}

\end{solutionbox}
\begin{mnemonicbox}
``Boolean to Circuit: nMOS simple, CMOS Complete''

\end{mnemonicbox}
\subsection*{Question 5(a) [3 marks]}\label{q5a}

\textbf{Explain design styles used in Verilog.}

\begin{solutionbox}

\textbf{Verilog Design Styles:}

{\def\LTcaptype{none} % do not increment counter
\begin{longtable}[]{@{}lll@{}}
\toprule\noalign{}
Style & Description & Application \\
\midrule\noalign{}
\endhead
\bottomrule\noalign{}
\endlastfoot
\textbf{Gate Level} & Using primitive gates & Low-level hardware
modeling \\
\textbf{Data Flow} & Using assign statements & Combinational logic \\
\textbf{Behavioral} & Using always blocks & Sequential and complex
logic \\
\textbf{Mixed} & Combination of styles & Complete system design \\
\end{longtable}
}

\begin{itemize}
\tightlist
\item
  \textbf{Gate level}: and, or, not, nand, nor primitives
\item
  \textbf{Data flow}: Continuous assignments with operators
\item
  \textbf{Behavioral}: Procedural assignments in always blocks
\item
  \textbf{Hierarchy}: Modules can use different styles
\end{itemize}

\end{solutionbox}
\begin{mnemonicbox}
``Gate-Data-Behavior: Three ways to Model''

\end{mnemonicbox}
\subsection*{Question 5(b) [4 marks]}\label{q5b}

\textbf{Write Verilog program for full adder using behavioral modeling.}

\begin{solutionbox}

\begin{lstlisting}[language=Verilog]
module full_adder_behavioral (
    input wire a, b, cin,
    output reg sum, cout
);

always @(*) begin
    case ({a, b, cin})
        3'b000: {cout, sum} = 2'b00;
        3'b001: {cout, sum} = 2'b01;
        3'b010: {cout, sum} = 2'b01;
        3'b011: {cout, sum} = 2'b10;
        3'b100: {cout, sum} = 2'b01;
        3'b101: {cout, sum} = 2'b10;
        3'b110: {cout, sum} = 2'b10;
        3'b111: {cout, sum} = 2'b11;
        default: {cout, sum} = 2'b00;
    endcase
end

endmodule
\end{lstlisting}

\textbf{Key Features:}

\begin{itemize}
\tightlist
\item
  \textbf{Always block}: Behavioral modeling construct
\item
  \textbf{Case statement}: Truth table implementation
\item
  \textbf{Concatenation}: \{cout, sum\} for combined output
\item
  \textbf{Sensitivity list}: @(*) for combinational logic
\end{itemize}

\end{solutionbox}
\begin{mnemonicbox}
``Behavioral uses Always with Case statements''

\end{mnemonicbox}
\subsection*{Question 5(c) [7 marks]}\label{q5c}

\textbf{Describe the function of CASE statement. Write Verilog code of
3x8 decoder using CASE statement.}

\begin{solutionbox}

\textbf{CASE Statement Function:}

{\def\LTcaptype{none} % do not increment counter
\begin{longtable}[]{@{}
  >{\raggedright\arraybackslash}p{(\linewidth - 4\tabcolsep) * \real{0.3103}}
  >{\raggedright\arraybackslash}p{(\linewidth - 4\tabcolsep) * \real{0.4483}}
  >{\raggedright\arraybackslash}p{(\linewidth - 4\tabcolsep) * \real{0.2414}}@{}}
\toprule\noalign{}
\begin{minipage}[b]{\linewidth}\raggedright
Feature
\end{minipage} & \begin{minipage}[b]{\linewidth}\raggedright
Description
\end{minipage} & \begin{minipage}[b]{\linewidth}\raggedright
Usage
\end{minipage} \\
\midrule\noalign{}
\endhead
\bottomrule\noalign{}
\endlastfoot
\textbf{Multi-way branch} & Selects one of many alternatives & Like
switch in C \\
\textbf{Pattern matching} & Compares expression with constants & Exact
bit matching \\
\textbf{Priority encoding} & First match wins & Top-down evaluation \\
\textbf{Default clause} & Handles unspecified cases & Prevents
latches \\
\end{longtable}
}

\textbf{3x8 Decoder Verilog Code:}

\begin{lstlisting}[language=Verilog]
module decoder_3x8 (
    input wire [2:0] select,
    input wire enable,
    output reg [7:0] out
);

always @(*) begin
    if (enable) begin
        case (select)
            3'b000: out = 8'b00000001;
            3'b001: out = 8'b00000010;
            3'b010: out = 8'b00000100;
            3'b011: out = 8'b00001000;
            3'b100: out = 8'b00010000;
            3'b101: out = 8'b00100000;
            3'b110: out = 8'b01000000;
            3'b111: out = 8'b10000000;
            default: out = 8'b00000000;
        endcase
    end else begin
        out = 8'b00000000;
    end
end

endmodule
\end{lstlisting}

\textbf{CASE Statement Features:}

\begin{itemize}
\tightlist
\item
  \textbf{Exact matching}: All bits must match exactly
\item
  \textbf{Parallel evaluation}: Hardware implementation is parallel
\item
  \textbf{Complete specification}: All possible input combinations
  covered
\item
  \textbf{Default clause}: Prevents unintended latches in synthesis
\end{itemize}

\end{solutionbox}
\begin{mnemonicbox}
``CASE Compares All Specified Exactly''

\end{mnemonicbox}
\subsection*{Question 5(a) OR [3
marks]}\label{q5a}

\textbf{Write Verilog code to implement 2:1 multiplexer.}

\begin{solutionbox}

\begin{lstlisting}[language=Verilog]
module mux_2to1 (
    input wire a, b, sel,
    output wire y
);

assign y = sel ? b : a;

endmodule
\end{lstlisting}

\textbf{Alternative Implementations:}

{\def\LTcaptype{none} % do not increment counter
\begin{longtable}[]{@{}lll@{}}
\toprule\noalign{}
Style & Code & Use Case \\
\midrule\noalign{}
\endhead
\bottomrule\noalign{}
\endlastfoot
\textbf{Data Flow} & assign y = sel ? b : a; & Simple logic \\
\textbf{Gate Level} & Uses and, or, not gates & Teaching purposes \\
\textbf{Behavioral} & always block with if-else & Complex conditions \\
\end{longtable}
}

\begin{itemize}
\tightlist
\item
  \textbf{Conditional operator}: ? : provides multiplexer function
\item
  \textbf{Continuous assignment}: assign for combinational logic
\item
  \textbf{Synthesis}: Tools convert to gate-level implementation
\end{itemize}

\end{solutionbox}
\begin{mnemonicbox}
``MUX: sel ? b : a - select between inputs''

\end{mnemonicbox}
\subsection*{Question 5(b) OR [4
marks]}\label{q5b}

\textbf{Write Verilog program for D flip-flop using behavioral
modeling.}

\begin{solutionbox}

\begin{lstlisting}[language=Verilog]
module d_flipflop (
    input wire clk, reset, d,
    output reg q, qbar
);

always @(posedge clk or posedge reset) begin
    if (reset) begin
        q <= 1'b0;
        qbar <= 1'b1;
    end else begin
        q <= d;
        qbar <= ~d;
    end
end

endmodule
\end{lstlisting}

\textbf{Key Features:}

{\def\LTcaptype{none} % do not increment counter
\begin{longtable}[]{@{}lll@{}}
\toprule\noalign{}
Element & Function & Syntax \\
\midrule\noalign{}
\endhead
\bottomrule\noalign{}
\endlastfoot
\textbf{posedge clk} & Rising edge trigger & Clock synchronization \\
\textbf{posedge reset} & Asynchronous reset & Immediate reset action \\
\textbf{Non-blocking} & \textless= operator & Sequential logic \\
\textbf{Complementary} & qbar = \textasciitilde q & True flip-flop
behavior \\
\end{longtable}
}

\begin{itemize}
\tightlist
\item
  \textbf{Edge sensitivity}: Responds only to clock edges
\item
  \textbf{Asynchronous reset}: Reset takes precedence over clock
\item
  \textbf{Sequential logic}: Uses non-blocking assignments
\item
  \textbf{State storage}: Maintains data between clock cycles
\end{itemize}

\end{solutionbox}
\begin{mnemonicbox}
``D Flip-flop: Data follows Clock with Reset''

\end{mnemonicbox}
\subsection*{Question 5(c) OR [7
marks]}\label{q5c}

\textbf{Explain testbench in brief. Write Verilog code to implement
4-bit down counter.}

\begin{solutionbox}

\textbf{Testbench Overview:}

{\def\LTcaptype{none} % do not increment counter
\begin{longtable}[]{@{}
  >{\raggedright\arraybackslash}p{(\linewidth - 4\tabcolsep) * \real{0.3056}}
  >{\raggedright\arraybackslash}p{(\linewidth - 4\tabcolsep) * \real{0.2500}}
  >{\raggedright\arraybackslash}p{(\linewidth - 4\tabcolsep) * \real{0.4444}}@{}}
\toprule\noalign{}
\begin{minipage}[b]{\linewidth}\raggedright
Component
\end{minipage} & \begin{minipage}[b]{\linewidth}\raggedright
Purpose
\end{minipage} & \begin{minipage}[b]{\linewidth}\raggedright
Implementation
\end{minipage} \\
\midrule\noalign{}
\endhead
\bottomrule\noalign{}
\endlastfoot
\textbf{Stimulus generation} & Provide test inputs & Clock, reset,
control signals \\
\textbf{Response monitoring} & Check outputs & Compare with expected
values \\
\textbf{Coverage analysis} & Verify completeness & All states and
transitions \\
\textbf{Debugging support} & Identify issues & Waveform analysis \\
\end{longtable}
}

\textbf{4-bit Down Counter:}

\begin{lstlisting}[language=Verilog]
module down_counter_4bit (
    input wire clk, reset, enable,
    output reg [3:0] count
);

always @(posedge clk or posedge reset) begin
    if (reset) begin
        count <= 4'b1111;  // Start from maximum value
    end else if (enable) begin
        if (count == 4'b0000)
            count <= 4'b1111;  // Wrap around
        else
            count <= count - 1;  // Decrement
    end
end

endmodule

// Testbench for down counter
module tb_down_counter;
    reg clk, reset, enable;
    wire [3:0] count;
    
    down_counter_4bit dut (
        .clk(clk), 
        .reset(reset), 
        .enable(enable), 
        .count(count)
    );
    
    // Clock generation
    always #5 clk = ~clk;
    
    initial begin
        clk = 0;
        reset = 1;
        enable = 0;
        
        #10 reset = 0;
        #10 enable = 1;
        
        #200 $finish;
    end
    
    // Monitor outputs
    initial begin
        $monitor("Time=%0t, Reset=%b, Enable=%b, Count=%b", 
                 $time, reset, enable, count);
    end
    
endmodule
\end{lstlisting}

\textbf{Testbench Components:}

\begin{itemize}
\tightlist
\item
  \textbf{Clock generation}: Continuous clock using always block
\item
  \textbf{Stimulus}: Reset and enable signal control
\item
  \textbf{Monitoring}: \$monitor for continuous output display
\item
  \textbf{Simulation control}: \$finish to end simulation
\end{itemize}

\textbf{Counter Features:}

\begin{itemize}
\tightlist
\item
  \textbf{Down counting}: Decrements from 15 to 0
\item
  \textbf{Wrap around}: Returns to 15 after reaching 0
\item
  \textbf{Enable control}: Counting only when enabled
\item
  \textbf{Synchronous operation}: All changes on clock edge
\end{itemize}

\end{solutionbox}
\begin{mnemonicbox}
``Testbench Tests with Clock, Stimulus, and Monitor''

\end{mnemonicbox}

\end{document}
