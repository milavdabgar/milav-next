\documentclass[10pt,a4paper]{article}

% content/resources/templates/preamble.tex
\usepackage[margin=0.6in]{geometry}
\author{Milav Dabgar}
\usepackage{amsmath,amssymb,amsthm}
\usepackage{booktabs}
\usepackage{multirow}
\usepackage{xcolor}
\usepackage{tcolorbox}
\tcbuselibrary{breakable,skins}
\usepackage[colorlinks=true,linkcolor=blue]{hyperref}
\usepackage{titlesec}
\usepackage{enumitem}
\usepackage{tikz}
\usepackage{pgfplots}
\usepackage{circuitikz}
\usepackage[version=4]{mhchem}
\usepackage{longtable}
\usepackage{array}
\usepackage{float}
\usepackage{caption}
\usepackage{listings}

\lstset{
  basicstyle=\small\ttfamily,
  breaklines=true,
  breakatwhitespace=false,
  postbreak=\mbox{\textcolor{red}{$\hookrightarrow$}\space},
  float=false,
  numbers=left,
  numberstyle=\tiny\color{gray},
  numbersep=10pt,
  xleftmargin=2em,
  keywordstyle=\color{blue},
  commentstyle=\color{green!60!black},
  stringstyle=\color{purple},
  backgroundcolor=\color{gray!5},
  showstringspaces=false,
  tabsize=2,
  captionpos=b,
  keepspaces=true,
  columns=flexible
}

\pgfplotsset{compat=1.18}
\usetikzlibrary{shapes,arrows,positioning,calc,patterns,decorations.pathmorphing,decorations.markings,arrows.meta}

% Color scheme
\definecolor{headcolor}{RGB}{0,102,204}
\definecolor{keycolor}{RGB}{220,20,60}
\definecolor{solutioncolor}{RGB}{34,139,34}
\definecolor{mnemoniccolor}{RGB}{148,0,211}
\definecolor{codecolor}{RGB}{0,0,100}

% Spacing
\setlength{\parskip}{3pt}
\setlist[itemize]{nosep}
\setlist[enumerate]{nosep}

% Title formatting
\titleformat{\section}{\Large\bfseries\color{headcolor}}{\thesection}{1em}{}
\titleformat{\subsection}{\large\bfseries\color{headcolor}}{\thesubsection}{1em}{}

% Pandoc tightlist compatibility
\providecommand{\tightlist}{%
  \setlength{\itemsep}{0pt}\setlength{\parskip}{0pt}}

% Pandoc longtable compatibility
\newcounter{none}
\def\thenone{}


% content/resources/templates/gujarati-boxes.tex
\usepackage{fontspec}
\usepackage{polyglossia}

% Set Gujarati as main language (document is primarily in Gujarati)
% Note: gloss-gujarati.ldf doesn't exist in polyglossia, but it will use hyphenation patterns
\setdefaultlanguage{gujarati}
\setotherlanguage{english}

% Configure Gujarati font properly
% Use Language=Default to prevent polyglossia from trying to add language-specific features
% that don't exist for Gujarati, which causes "empty feature" warnings
\newfontfamily\gujaratifont[Script=Gujarati,AutoFakeBold=2.5,AutoFakeSlant=0.3]{Noto Sans Gujarati}
\setmainfont[Script=Gujarati,AutoFakeBold=2.5,AutoFakeSlant=0.3]{Noto Sans Gujarati}
% Use Noto Sans Gujarati for monospace to support Gujarati in text
\setmonofont[Scale=0.9]{Noto Sans Gujarati}

% Configure English to use the same font
\newfontfamily\englishfont[Script=Gujarati,AutoFakeBold=2.5,AutoFakeSlant=0.3]{Noto Sans Gujarati}

% Translations for polyglossia
\gappto\captionsgujarati{
  \renewcommand{\tablename}{કોષ્ટક}
  \renewcommand{\figurename}{આકૃતિ}
}

% Helper for TikZ nodes to ensure Gujarati font
\newcommand{\gu}[1]{{\gujaratifont #1}}

% Custom environments
\newtcolorbox{solutionbox}{
    breakable,
    enhanced,
    colback=solutioncolor!5!white,
    colframe=solutioncolor!75!black,
    fonttitle=\bfseries,
    title=જવાબ
}

\newtcolorbox{solutionboxnobreak}{
 colback=solutioncolor!5!white,
 colframe=solutioncolor!75!black,
 fonttitle=\bfseries,
 title=જવાબ
}

\newtcolorbox{keyformula}{
 breakable,
 enhanced,
 colback=keycolor!5!white,
 colframe=keycolor!75!black,
 fonttitle=\bfseries,
 title=રાસાયણિક સમીકરણ/સૂત્ર
}

\newtcolorbox{mnemonicbox}{
 breakable,
 enhanced,
 colback=mnemoniccolor!5!white,
 colframe=mnemoniccolor!75!black,
 fonttitle=\bfseries,
 title=મેમરી ટ્રીક
}


\begin{document}

\begin{center}
{\Huge\bfseries\color{headcolor} Subject Name (Gujarati)}\\[5pt]
{\LARGE 4353206 -- Winter 2024}\\[3pt]
{\large Semester 1 Study Material}\\[3pt]
{\normalsize\textit{Detailed Solutions and Explanations}}
\end{center}

\vspace{10pt}

\subsection*{પ્રશ્ન 1(અ) [3
ગુણ]}\label{uxaaauxab0uxab6uxaa8-1uxa85-3-uxa97uxaa3}

\textbf{એનહેન્સમેન્ટ અને ડીપ્લેશન ટાઈપ MOSFET માટે બધા સિમ્બોલ દોરો.}

\begin{solutionbox}

\textbf{આકૃતિ:}

\begin{lstlisting}
Enhancement Type NMOS:           Enhancement Type PMOS:
    
    D                               D
    |                               |
G --+-- S                       G --+-- S
    |                               |
    B                               B
(ગેટ વોલ્ટેજ વિના ચેનલ નથી)        (ગેટ વોલ્ટેજ વિના ચેનલ નથી)

Depletion Type NMOS:            Depletion Type PMOS:
    
    D                               D
    |                               |
G --+==-- S                     G --+==-- S
    |                               |
    B                               B
(ગેટ વોલ્ટેજ વિના પણ ચેનલ છે)       (ગેટ વોલ્ટેજ વિના પણ ચેનલ છે)
\end{lstlisting}

\begin{itemize}
\tightlist
\item
  \textbf{એનહેન્સમેન્ટ MOSFET}: સોર્સ અને ડ્રેઇન વચ્ચે સામાન્ય કનેક્શન લાઇન
\item
  \textbf{ડીપ્લેશન MOSFET}: અસ્તિત્વમાં રહેલ ચેનલ દર્શાવતી જાડી લાઇન
\item
  \textbf{એરો દિશા}: NMOS માટે અંદરની તરફ, PMOS માટે બહારની તરફ
\end{itemize}

\end{solutionbox}
\begin{mnemonicbox}
``Enhancement ને વોલ્ટેજ જોઈએ, Depletion માં ડિફોલ્ટ
ચેનલ''

\end{mnemonicbox}
\subsection*{પ્રશ્ન 1(બ) [4
ગુણ]}\label{uxaaauxab0uxab6uxaa8-1uxaac-4-uxa97uxaa3}

\textbf{વ્યાખ્યા આપો: 1) હાઈરારકી 2) રેગ્યુલારીટી}

\begin{solutionbox}

{\def\LTcaptype{none} % do not increment counter
\begin{longtable}[]{@{}
  >{\raggedright\arraybackslash}p{(\linewidth - 4\tabcolsep) * \real{0.3333}}
  >{\raggedright\arraybackslash}p{(\linewidth - 4\tabcolsep) * \real{0.3704}}
  >{\raggedright\arraybackslash}p{(\linewidth - 4\tabcolsep) * \real{0.2963}}@{}}
\toprule\noalign{}
\begin{minipage}[b]{\linewidth}\raggedright
પરિભાષા
\end{minipage} & \begin{minipage}[b]{\linewidth}\raggedright
વ્યાખ્યા
\end{minipage} & \begin{minipage}[b]{\linewidth}\raggedright
ઉપયોગ
\end{minipage} \\
\midrule\noalign{}
\endhead
\bottomrule\noalign{}
\endlastfoot
\textbf{હાઈરારકી} & ટોપ-ડાઉન ડિઝાઇન અભિગમ જેમાં જટિલ સિસ્ટમને નાના, વ્યવસ્થિત
મોડ્યુલમાં વિભાજિત કરાય છે & VLSI ડિઝાઇન ફ્લોમાં સિસ્ટમ લેવલથી ટ્રાન્ઝિસ્ટર લેવલ સુધી
વપરાય છે \\
\textbf{રેગ્યુલારીટી} & જટિલતા ઘટાડવા માટે પુનરાવર્તિત સમાન સ્ટ્રક્ચરનો ઉપયોગ
કરતી ડિઝાઇન તકનીક & મેમરી એરે, પ્રોસેસર ડેટાપાથમાં નિયમિત સ્ટ્રક્ચર વપરાય છે \\
\end{longtable}
}

\begin{itemize}
\tightlist
\item
  \textbf{હાઈરારકીના ફાયદા}: સરળ ડિઝાઇન વેરિફિકેશન, મોડ્યુલર ટેસ્ટિંગ, ટીમ
  કોલેબોરેશન
\item
  \textbf{રેગ્યુલારીટીના ફાયદા}: ઓછો ડિઝાઇન સમય, બહેતર યીલ્ડ, સરળ લેઆઉટ
\item
  \textbf{ડિઝાઇન ફ્લો}: સિસ્ટમ \rightarrow બિહેવિયરલ \rightarrow RTL \rightarrow ગેટ \rightarrow લેઆઉટ
\item
  \textbf{નિયમિત સ્ટ્રક્ચર}: ROM એરે, કેશ મેમરી, ALU બ્લોક
\end{itemize}

\end{solutionbox}
\begin{mnemonicbox}
``હાઈરારકી હેલ્પ કરે ઓર્ગેનાઇઝ કરવામાં, રેગ્યુલારીટી રિડ્યુસ કરે
કોમ્પ્લેક્સિટી''

\end{mnemonicbox}
\subsection*{પ્રશ્ન 1(ક) [7
ગુણ]}\label{uxaaauxab0uxab6uxaa8-1uxa95-7-uxa97uxaa3}

\textbf{MOS અન્ડર એક્સટર્નલ બાયસ સમજાવો.}

\begin{solutionbox}

\textbf{MOS બાયસ કન્ડિશન કોષ્ટક:}

{\def\LTcaptype{none} % do not increment counter
\begin{longtable}[]{@{}llll@{}}
\toprule\noalign{}
બાયસ કન્ડિશન & ગેટ વોલ્ટેજ & ચેનલ નિર્માણ & કરંટ ફ્લો \\
\midrule\noalign{}
\endhead
\bottomrule\noalign{}
\endlastfoot
\textbf{એક્યુમ્યુલેશન} & VG \textless{} 0 (NMOS) & મેજોરિટી કેરિયર એકત્ર થાય છે &
ચેનલ નથી \\
\textbf{ડીપ્લેશન} & 0 \textless{} VG \textless{} VT & ડીપ્લેશન રીજન બને છે &
ન્યૂનતમ કરંટ \\
\textbf{ઇન્વર્શન} & VG \textgreater{} VT & માઇનોરિટી કેરિયર ચેનલ બનાવે છે &
ચેનલ વહન કરે છે \\
\end{longtable}
}

\textbf{આકૃતિ:}

\includegraphics[width=1\linewidth,height=\textheight,keepaspectratio]{mermaid-f03d606e.pdf}

\begin{itemize}
\tightlist
\item
  \textbf{બેન્ડ બેન્ડિંગ}: એક્સટર્નલ વોલ્ટેજ ઓક્સાઇડ-સિલિકોન ઇન્ટરફેસ પર એનર્જી બેન્ડ
  વાળે છે
\item
  \textbf{થ્રેશોલ્ડ વોલ્ટેજ}: ચેનલ નિર્માણ માટે જરૂરી ન્યૂનતમ ગેટ વોલ્ટેજ
\item
  \textbf{સરફેસ પોટેન્શિયલ}: સિલિકોન સરફેસ પર કેરિયર કોન્સંટ્રેશન કંટ્રોલ કરે છે
\item
  \textbf{કેપેસિટન્સ વેરિએશન}: બાયસ કન્ડિશન સાથે બદલાય છે
\end{itemize}

\end{solutionbox}
\begin{mnemonicbox}
``એક્યુમ્યુલેશન આકર્ષે, ડીપ્લેશન ડિપ્લીટ કરે, ઇન્વર્શન ઇન્વર્ટ કરે
કેરિયર''

\end{mnemonicbox}
\subsection*{પ્રશ્ન 1(ક) OR [7
ગુણ]}\label{uxaaauxab0uxab6uxaa8-1uxa95-or-7-uxa97uxaa3}

\textbf{સ્કેલિંગની શું જરૂરિયાત છે? સ્કેલિંગના ટાઈપ તેની ઈફેક્ટ સાથે સમજાવો.}

\begin{solutionbox}

\textbf{સ્કેલિંગની જરૂરિયાત:}

{\def\LTcaptype{none} % do not increment counter
\begin{longtable}[]{@{}lll@{}}
\toprule\noalign{}
પેરામીટર & ફાયદો & પ્રભાવ \\
\midrule\noalign{}
\endhead
\bottomrule\noalign{}
\endlastfoot
\textbf{એરિયા રિડક્શન} & ચિપ દીઠ વધુ ટ્રાન્ઝિસ્ટર & ઊંચી ઇન્ટિગ્રેશન ડેન્સિટી \\
\textbf{સ્પીડ ઇન્ક્રીઝ} & ઓછી ડીલે & બહેતર પરફોર્મન્સ \\
\textbf{પાવર રિડક્શન} & ઓછો પાવર વપરાશ & પોર્ટેબલ ડિવાઇસ \\
\textbf{કોસ્ટ રિડક્શન} & ફંક્શન દીઠ સસ્તું & માર્કેટ કોમ્પિટિટિવનેસ \\
\end{longtable}
}

\textbf{સ્કેલિંગના પ્રકાર:}

\includegraphics[width=1\linewidth,height=\textheight,keepaspectratio]{mermaid-dd5bfb30.pdf}

\begin{itemize}
\tightlist
\item
  \textbf{ફુલ વોલ્ટેજ સ્કેલિંગ}: લેન્થ, વિડથ, વોલ્ટેજ બધું α ફેક્ટર દ્વારા સ્કેલ
\item
  \textbf{કોન્સ્ટન્ટ વોલ્ટેજ સ્કેલિંગ}: ડાઇમેન્શન સ્કેલ, વોલ્ટેજ અપરિવર્તિત
\item
  \textbf{પાવર ડેન્સિટી}: ફુલ સ્કેલિંગમાં કોન્સ્ટન્ટ રહે, કોન્સ્ટન્ટ વોલ્ટેજમાં વધે
\item
  \textbf{ઇલેક્ટ્રિક ફીલ્ડ}: ફુલ સ્કેલિંગમાં મેન્ટેઇન થાય છે
\end{itemize}

\end{solutionbox}
\begin{mnemonicbox}
``સ્કેલિંગ સેવ કરે સ્પેસ, સ્પીડ અને સ્પેન્ડિંગ''

\end{mnemonicbox}
\subsection*{પ્રશ્ન 2(અ) [3
ગુણ]}\label{uxaaauxab0uxab6uxaa8-2uxa85-3-uxa97uxaa3}

\textbf{FPGA પર ટૂંકનોંધ લખો.}

\begin{solutionbox}

\textbf{FPGA લાક્ષણિકતાઓ કોષ્ટક:}

{\def\LTcaptype{none} % do not increment counter
\begin{longtable}[]{@{}lll@{}}
\toprule\noalign{}
લક્ષણ & વર્ણન & ફાયદો \\
\midrule\noalign{}
\endhead
\bottomrule\noalign{}
\endlastfoot
\textbf{ફીલ્ડ પ્રોગ્રામેબલ} & મેન્યુફેક્ચરિંગ પછી કોન્ફિગરેબલ & ડિઝાઇનમાં લવચીકતા \\
\textbf{ગેટ એરે} & લોજિક બ્લોકનું એરે & પેરેલલ પ્રોસેસિંગ \\
\textbf{રિકોન્ફિગરેબલ} & ફરીથી પ્રોગ્રામ કરી શકાય & પ્રોટોટાઇપ ડેવલપમેન્ટ \\
\end{longtable}
}

\begin{itemize}
\tightlist
\item
  \textbf{એપ્લિકેશન}: ડિજિટલ સિગ્નલ પ્રોસેસિંગ, એમ્બેડેડ સિસ્ટમ, પ્રોટોટાઇપિંગ
\item
  \textbf{આર્કિટેક્ચર}: CLBs (કોન્ફિગરેબલ લોજિક બ્લોક) રાઉટિંગ મેટ્રિક્સ દ્વારા
  કનેક્ટેડ
\item
  \textbf{પ્રોગ્રામિંગ}: SRAM-આધારિત કોન્ફિગરેશન મેમરી
\item
  \textbf{વેન્ડર}: Xilinx, Altera (Intel), Microsemi
\end{itemize}

\end{solutionbox}
\begin{mnemonicbox}
``FPGA: ફ્લેક્સિબલ પ્રોગ્રામિંગ ફોર ગેટ એરે''

\end{mnemonicbox}
\subsection*{પ્રશ્ન 2(બ) [4
ગુણ]}\label{uxaaauxab0uxab6uxaa8-2uxaac-4-uxa97uxaa3}

\textbf{સેમી કસ્ટમ અને ફુલ કસ્ટમ ડિઝાઇન મેથોડોલોજી સરખાવો.}

\begin{solutionbox}

{\def\LTcaptype{none} % do not increment counter
\begin{longtable}[]{@{}lll@{}}
\toprule\noalign{}
પેરામીટર & સેમી-કસ્ટમ & ફુલ કસ્ટમ \\
\midrule\noalign{}
\endhead
\bottomrule\noalign{}
\endlastfoot
\textbf{ડિઝાઇન ટાઇમ} & ઓછો (અઠવાડિયા) & વધુ (મહિના) \\
\textbf{કોસ્ટ} & ઓછો ડેવલપમેન્ટ કોસ્ટ & વધુ ડેવલપમેન્ટ કોસ્ટ \\
\textbf{પરફોર્મન્સ} & મધ્યમ પરફોર્મન્સ & સર્વોચ્ચ પરફોર્મન્સ \\
\textbf{એરિયા એફિશિયન્સી} & ઓછી કાર્યક્ષમ & સૌથી કાર્યક્ષમ \\
\textbf{એપ્લિકેશન} & ASICs, મધ્યમ વોલ્યુમ & માઇક્રોપ્રોસેસર, ઊંચો વોલ્યુમ \\
\textbf{ડિઝાઇન એફર્ટ} & સ્ટાન્ડર્ડ સેલ વપરાય છે & દરેક ટ્રાન્ઝિસ્ટર ડિઝાઇન કરાય
છે \\
\end{longtable}
}

\begin{itemize}
\tightlist
\item
  \textbf{સેમી-કસ્ટમ}: પ્રી-ડિઝાઇન્ડ સ્ટાન્ડર્ડ સેલ અને ગેટ એરેનો ઉપયોગ કરે છે
\item
  \textbf{ફુલ કસ્ટમ}: સંપૂર્ણ ટ્રાન્ઝિસ્ટર-લેવલ ડિઝાઇન ઓપ્ટિમાઇઝેશન
\item
  \textbf{ટ્રેડ-ઓફ}: સમય વર્સિસ પરફોર્મન્સ, કોસ્ટ વર્સિસ એફિશિયન્સી
\item
  \textbf{માર્કેટ ફિટ}: મોટાભાગના એપ્લિકેશન માટે સેમી-કસ્ટમ, સ્પેશિયલાઇઝડ જરૂરિયાત
  માટે ફુલ કસ્ટમ
\end{itemize}

\end{solutionbox}
\begin{mnemonicbox}
``સેમી-કસ્ટમ છે સ્ટાન્ડર્ડ, ફુલ કસ્ટમ છે ફાઇનેસ્ટ''

\end{mnemonicbox}
\subsection*{પ્રશ્ન 2(ક) [7
ગુણ]}\label{uxaaauxab0uxab6uxaa8-2uxa95-7-uxa97uxaa3}

\textbf{1) 0\textless VDS\textless VDSAT 2) VDS = VDSAT 3) VDS
\textgreater{} VDSAT માટે MOSFET ઓપરેશન સમજાવો.}

\begin{solutionbox}

\textbf{ઓપરેટિંગ રીજન:}

{\def\LTcaptype{none} % do not increment counter
\begin{longtable}[]{@{}llll@{}}
\toprule\noalign{}
રીજન & કન્ડિશન & ચેનલ & કરંટ બિહેવિયર \\
\midrule\noalign{}
\endhead
\bottomrule\noalign{}
\endlastfoot
\textbf{લિનિયર} & 0 \textless{} VDS \textless{} VDSAT & યુનિફોર્મ ચેનલ & ID
∝ VDS \\
\textbf{સેચ્યુરેશન ઓન્સેટ} & VDS = VDSAT & પિંચ-ઓફ શરૂ થાય છે & મેક્સિમમ લિનિયર
કરંટ \\
\textbf{સેચ્યુરેશન} & VDS \textgreater{} VDSAT & પિંચ્ડ ચેનલ & ID કોન્સ્ટન્ટ \\
\end{longtable}
}

\textbf{આકૃતિ:}

\includegraphics[width=1\linewidth,height=\textheight,keepaspectratio]{mermaid-f4ffe185.pdf}

\begin{itemize}
\tightlist
\item
  \textbf{લિનિયર રીજન}: ચેનલ વોલ્ટેજ-કંટ્રોલ્ડ રેઝિસ્ટર તરીકે કામ કરે છે
\item
  \textbf{સેચ્યુરેશન રીજન}: કરંટ માત્ર ગેટ વોલ્ટેજ દ્વારા કંટ્રોલ થાય છે
\item
  \textbf{VDSAT કેલ્ક્યુલેશન}: VDSAT = VGS - VT
\item
  \textbf{કરંટ સમીકરણો}: દરેક રીજન માટે અલગ મેથેમેટિકલ મોડેલ
\end{itemize}

\end{solutionbox}
\begin{mnemonicbox}
``લિનિયર લાઇક્સ VDS, સેચ્યુરેશન સેઝ નો મોર''

\end{mnemonicbox}
\subsection*{પ્રશ્ન 2(અ) OR [3
ગુણ]}\label{uxaaauxab0uxab6uxaa8-2uxa85-or-3-uxa97uxaa3}

\textbf{સ્ટાન્ડર્ડ સેલ બેઝ્ડ ડિઝાઇન સમજાવો.}

\begin{solutionbox}

\textbf{સ્ટાન્ડર્ડ સેલ ડિઝાઇન કોષ્ટક:}

{\def\LTcaptype{none} % do not increment counter
\begin{longtable}[]{@{}lll@{}}
\toprule\noalign{}
કમ્પોનન્ટ & વર્ણન & ફાયદો \\
\midrule\noalign{}
\endhead
\bottomrule\noalign{}
\endlastfoot
\textbf{સ્ટાન્ડર્ડ સેલ} & પ્રી-ડિઝાઇન્ડ લોજિક ગેટ & ઝડપી ડિઝાઇન \\
\textbf{સેલ લાઇબ્રેરી} & કેરેક્ટરાઇઝ્ડ સેલનો સંગ્રહ & અનુમાનિત પરફોર્મન્સ \\
\textbf{પ્લેસ એન્ડ રાઉટ} & ઓટોમેટેડ લેઆઉટ જનરેશન & ઓછો ડિઝાઇન સમય \\
\end{longtable}
}

\begin{itemize}
\tightlist
\item
  \textbf{પ્રોસેસ}: લોજિક સિન્થેસિસ \rightarrow પ્લેસમેન્ટ \rightarrow રાઉટિંગ \rightarrow વેરિફિકેશન
\item
  \textbf{સેલ પ્રકાર}: બેસિક ગેટ, ફ્લિપ-ફ્લોપ, લેચ, કોમ્પ્લેક્સ ફંક્શન
\item
  \textbf{ઓટોમેશન}: EDA ટૂલ ફિઝિકલ ઇમ્પ્લિમેન્ટેશન હેન્ડલ કરે છે
\item
  \textbf{ક્વોલિટી}: બેલેન્સ્ડ પરફોર્મન્સ, એરિયા અને પાવર
\end{itemize}

\end{solutionbox}
\begin{mnemonicbox}
``સ્ટાન્ડર્ડ સેલ સ્પીડ અપ કરે સિન્થેસિસ''

\end{mnemonicbox}
\subsection*{પ્રશ્ન 2(બ) OR [4
ગુણ]}\label{uxaaauxab0uxab6uxaa8-2uxaac-or-4-uxa97uxaa3}

\textbf{Y ચાર્ટ દોરો અને સમજાવો.}

\begin{solutionbox}

\textbf{આકૃતિ:}

\includegraphics[width=1\linewidth,height=\textheight,keepaspectratio]{mermaid-69afcb8a.pdf}

{\def\LTcaptype{none} % do not increment counter
\begin{longtable}[]{@{}lll@{}}
\toprule\noalign{}
ડોમેઇન & વર્ણન & ઉદાહરણ \\
\midrule\noalign{}
\endhead
\bottomrule\noalign{}
\endlastfoot
\textbf{બિહેવિયરલ} & સિસ્ટમ શું કરે છે & એલ્ગોરિધમ, RTL કોડ \\
\textbf{સ્ટ્રક્ચરલ} & સિસ્ટમ કેવી રીતે બને છે & ગેટ, મોડ્યુલ, પ્રોસેસર \\
\textbf{ફિઝિકલ} & ફિઝિકલ ઇમ્પ્લિમેન્ટેશન & લેઆઉટ, ફ્લોરપ્લાન, માસ્ક \\
\end{longtable}
}

\begin{itemize}
\tightlist
\item
  \textbf{ડિઝાઇન ફ્લો}: બાહ્ય રિંગ (સિસ્ટમ) થી અંદરની રિંગ (ડિવાઇસ) તરફ જવું
\item
  \textbf{એબ્સ્ટ્રેક્શન લેવલ}: દરેક રિંગ અલગ વિગતનું સ્તર દર્શાવે છે
\item
  \textbf{ડોમેઇન ઇન્ટરેક્શન}: સમાન એબ્સ્ટ્રેક્શન પર ડોમેઇન વચ્ચે મૂવ થઈ શકાય
\item
  \textbf{VLSI ડિઝાઇન}: ત્રણેય ડોમેઇન અને એબ્સ્ટ્રેક્શન લેવલ કવર કરે છે
\end{itemize}

\end{solutionbox}
\begin{mnemonicbox}
``Y-ચાર્ટ: બિહેવિયર, સ્ટ્રક્ચર, ફિઝિકલ''

\end{mnemonicbox}
\subsection*{પ્રશ્ન 2(ક) OR [7
ગુણ]}\label{uxaaauxab0uxab6uxaa8-2uxa95-or-7-uxa97uxaa3}

\textbf{MOSFET કરંટ-વોલ્ટેજ કેરેક્ટરિસ્ટિક માટે ગ્રેજુઅલ ચેનલ એપ્રોક્સિમેશન સમજાવો.}

\begin{solutionbox}

\textbf{ધારણાઓ:}

{\def\LTcaptype{none} % do not increment counter
\begin{longtable}[]{@{}lll@{}}
\toprule\noalign{}
ધારણા & વર્ણન & જસ્ટિફિકેશન \\
\midrule\noalign{}
\endhead
\bottomrule\noalign{}
\endlastfoot
\textbf{ગ્રેજુઅલ ચેનલ} & ચેનલ લેન્થ \textgreater\textgreater{} ચેનલ ડેપ્થ & લોંગ
ચેનલ ડિવાઇસ \\
\textbf{1D એનાલિસિસ} & કરંટ માત્ર x-દિશામાં ફ્લો થાય છે & મેથેમેટિક્સ સરળ બનાવે
છે \\
\textbf{ડ્રિફ્ટ કરંટ} & ડિફ્યુઝન કરંટ નેગ્લેક્ટ કરવો & હાઇ ફીલ્ડ કન્ડિશન \\
\textbf{ચાર્જ શીટ} & મોબાઇલ ચાર્જ પાતળી શીટમાં & નાની ઇન્વર્શન લેયર \\
\end{longtable}
}

\textbf{કરંટ ડેરિવેશન:}

\begin{itemize}
\tightlist
\item
  \textbf{ડ્રેઇન કરંટ}: ID = μn Cox (W/L) [(VGS-VT)VDS - VDS^{2}/2]
\item
  \textbf{લિનિયર રીજન}: જ્યારે VDS \textless{} VGS-VT
\item
  \textbf{સેચ્યુરેશન}: જ્યારે VDS \geq VGS-VT, ID = μn Cox (W/2L)(VGS-VT)^{2}
\item
  \textbf{ચેનલ ચાર્જ}: સોર્સથી ડ્રેઇન સુધી લિનિયર રીતે વેરી થાય છે
\end{itemize}

\textbf{મર્યાદાઓ:}

\begin{itemize}
\tightlist
\item
  \textbf{શોર્ટ ચેનલ ઇફેક્ટ}: ગ્રેજુઅલ એપ્રોક્સિમેશન બ્રેક ડાઉન થાય છે
\item
  \textbf{વેલોસિટી સેચ્યુરેશન}: હાઇ ફીલ્ડ ઇફેક્ટ ઇન્ક્લુડ નથી
\item
  \textbf{2D ઇફેક્ટ}: સિમ્પલ મોડેલમાં અવગણાય છે
\end{itemize}

\end{solutionbox}
\begin{mnemonicbox}
``ગ્રેજુઅલ ચેન્જ ગિવ સિમ્પલ ગેઇન એક્વેશન''

\end{mnemonicbox}
\subsection*{પ્રશ્ન 3(અ) [3
ગુણ]}\label{uxaaauxab0uxab6uxaa8-3uxa85-3-uxa97uxaa3}

\textbf{આઈડલ ઇન્વર્ટરનો સિમ્બોલ દોરો અને ટ્રુથ ટેબલ લખો. આઈડલ ઇન્વર્ટર માટે VTC
દોરો અને સમજાવો.}

\begin{solutionbox}

\textbf{સિમ્બોલ અને ટ્રુથ ટેબલ:}

\begin{lstlisting}
    VIN ------>|>o----- VOUT
              NOT
\end{lstlisting}

{\def\LTcaptype{none} % do not increment counter
\begin{longtable}[]{@{}ll@{}}
\toprule\noalign{}
VIN & VOUT \\
\midrule\noalign{}
\endhead
\bottomrule\noalign{}
\endlastfoot
0 & 1 \\
1 & 0 \\
\end{longtable}
}

\textbf{VTC (વોલ્ટેજ ટ્રાન્સફર કેરેક્ટરિસ્ટિક):}

\begin{lstlisting}
VOUT ^
     |
 VDD +-----+
     |     |
     |     |
     |     +------
     |           
     +--------------> VIN
     0   VDD/2   VDD
\end{lstlisting}

\begin{itemize}
\tightlist
\item
  \textbf{આઈડલ લાક્ષણિકતા}: VDD/2 પર તીવ્ર સંક્રમણ
\item
  \textbf{નોઇઝ માર્જિન}: NMH = NML = VDD/2
\item
  \textbf{ગેઇન}: સ્વિચિંગ પોઇન્ટ પર અનંત
\item
  \textbf{પાવર કન્ઝમ્પશન}: શૂન્ય સ્ટેટિક પાવર
\end{itemize}

\end{solutionbox}
\begin{mnemonicbox}
``આઈડલ ઇન્વર્ટર: અનંત ગેઇન, ઇન્સ્ટન્ટ સ્વિચિંગ''

\end{mnemonicbox}
\subsection*{પ્રશ્ન 3(બ) [4
ગુણ]}\label{uxaaauxab0uxab6uxaa8-3uxaac-4-uxa97uxaa3}

\textbf{જનરાલાઇઝ્ડ ઇન્વર્ટર સર્કિટ VTC સાથે સમજાવો.}

\begin{solutionbox}

\textbf{સર્કિટ કોન્ફિગરેશન:}

{\def\LTcaptype{none} % do not increment counter
\begin{longtable}[]{@{}lll@{}}
\toprule\noalign{}
કમ્પોનન્ટ & ફંક્શન & લાક્ષણિકતા \\
\midrule\noalign{}
\endhead
\bottomrule\noalign{}
\endlastfoot
\textbf{ડ્રાઇવર ટ્રાન્ઝિસ્ટર} & પુલ-ડાઉન ડિવાઇસ & સ્વિચિંગ કંટ્રોલ કરે છે \\
\textbf{લોડ ડિવાઇસ} & પુલ-અપ એલિમેન્ટ & હાઇ આઉટપુટ પ્રદાન કરે છે \\
\textbf{સપ્લાય વોલ્ટેજ} & પાવર સોર્સ & લોજિક લેવલ નક્કી કરે છે \\
\end{longtable}
}

\textbf{VTC રીજન:}

\includegraphics[width=1\linewidth,height=\textheight,keepaspectratio]{mermaid-fd0543e3.pdf}

\begin{itemize}
\tightlist
\item
  \textbf{લોડ લાઇન એનાલિસિસ}: ડ્રાઇવર અને લોડની લાક્ષણિકતાઓનું આંતરછેદ
\item
  \textbf{સ્વિચિંગ થ્રેશોલ્ડ}: ડિવાઇસ સાઇઝિંગ રેશિયો દ્વારા નક્કી થાય છે
\item
  \textbf{નોઇઝ માર્જિન}: ટ્રાન્ઝિશન શાર્પનેસ પર આધાર રાખે છે
\item
  \textbf{પાવર ડિસિપેશન}: ટ્રાન્ઝિશન દરમિયાન સ્ટેટિક કરંટ
\end{itemize}

\end{solutionbox}
\begin{mnemonicbox}
``જનરાલાઇઝ્ડ ડિઝાઇન: ડ્રાઇવર પુલ ડાઉન, લોડ લિફ્ટ અપ''

\end{mnemonicbox}
\subsection*{પ્રશ્ન 3(ક) [7
ગુણ]}\label{uxaaauxab0uxab6uxaa8-3uxa95-7-uxa97uxaa3}

\textbf{ડીપ્લેશન લોડ nMOS ઇન્વર્ટર તેની સર્કિટ, ઓપરેટિંગ રીજન અને VTC સાથે
સમજાવો.}

\begin{solutionbox}

\textbf{સર્કિટ આકૃતિ:}

\begin{lstlisting}
           VDD
            |
         +--+--+ VGS = 0
    VG --|     |
         |  T2 | (ડીપ્લેશન લોડ)
         +-----+
            |
         +--+--+
    VIN -+     +- VOUT
         |  T1 |
         +-----+
            |
           GND
\end{lstlisting}

\textbf{ઓપરેટિંગ રીજન:}

{\def\LTcaptype{none} % do not increment counter
\begin{longtable}[]{@{}llll@{}}
\toprule\noalign{}
ઇનપુટ સ્ટેટ & T1 સ્ટેટ & T2 સ્ટેટ & આઉટપુટ \\
\midrule\noalign{}
\endhead
\bottomrule\noalign{}
\endlastfoot
\textbf{VIN = 0} & બંધ & ચાલુ (ડીપ્લેશન) & VOUT = VDD-VT \\
\textbf{VIN = VDD} & ચાલુ & ચાલુ (રેઝિસ્ટિવ) & VOUT = VOL \\
\end{longtable}
}

\textbf{VTC એનાલિસિસ:}

\includegraphics[width=1\linewidth,height=\textheight,keepaspectratio]{mermaid-feb73f72.pdf}

\begin{itemize}
\tightlist
\item
  \textbf{ફાયદા}: સિમ્પલ ફેબ્રિકેશન, સારી ડ્રાઇવ કેપેબિલિટી
\item
  \textbf{નુકસાન}: ડિગ્રેડેડ હાઇ આઉટપુટ, સ્ટેટિક પાવર કન્ઝમ્પશન
\item
  \textbf{એપ્લિકેશન}: પ્રારંભિક NMOS લોજિક ફેમિલી
\item
  \textbf{ડિઝાઇન વિચારણા}: વિડથ રેશિયો સ્વિચિંગ પોઇન્ટને અસર કરે છે
\end{itemize}

\end{solutionbox}
\begin{mnemonicbox}
``ડીપ્લેશન ડિવાઇસ ડિલિવર કરે ડીસેન્ટ ડ્રાઇવ''

\end{mnemonicbox}
\subsection*{પ્રશ્ન 3(અ) OR [3
ગુણ]}\label{uxaaauxab0uxab6uxaa8-3uxa85-or-3-uxa97uxaa3}

\textbf{નોઇઝ માર્જિન સમજાવો.}

\begin{solutionbox}

\textbf{વ્યાખ્યા અને પેરામીટર:}

{\def\LTcaptype{none} % do not increment counter
\begin{longtable}[]{@{}lll@{}}
\toprule\noalign{}
પેરામીટર & વર્ણન & ફોર્મ્યુલા \\
\midrule\noalign{}
\endhead
\bottomrule\noalign{}
\endlastfoot
\textbf{NMH} & હાઇ નોઇઝ માર્જિન & NMH = VOH - VIH \\
\textbf{NML} & લો નોઇઝ માર્જિન & NML = VIL - VOL \\
\textbf{VOH} & આઉટપુટ હાઇ વોલ્ટેજ & મિનિમમ હાઇ આઉટપુટ \\
\textbf{VOL} & આઉટપુટ લો વોલ્ટેજ & મેક્સિમમ લો આઉટપુટ \\
\textbf{VIH} & ઇનપુટ હાઇ થ્રેશોલ્ડ & મિનિમમ ઇનપુટ હાઇ \\
\textbf{VIL} & ઇનપુટ લો થ્રેશોલ્ડ & મેક્સિમમ ઇનપુટ લો \\
\end{longtable}
}

\begin{itemize}
\tightlist
\item
  \textbf{મહત્વ}: સર્કિટની નોઇઝ સામે પ્રતિરોધકતાનું માપ
\item
  \textbf{ડિઝાઇન લક્ષ્ય}: NMH અને NML બન્નેને મેક્સિમાઇઝ કરો
\item
  \textbf{ટ્રેડ-ઓફ}: નોઇઝ માર્જિન વર્સિસ સ્પીડ વર્સિસ પાવર
\item
  \textbf{એપ્લિકેશન}: ડિજિટલ સિસ્ટમ ડિઝાઇનમાં મહત્વપૂર્ણ
\end{itemize}

\end{solutionbox}
\begin{mnemonicbox}
``નોઇઝ માર્જિન મેઇન્ટેઇન કરે સિગ્નલ ઇન્ટેગ્રિટી''

\end{mnemonicbox}
\subsection*{પ્રશ્ન 3(બ) OR [4
ગુણ]}\label{uxaaauxab0uxab6uxaa8-3uxaac-or-4-uxa97uxaa3}

\textbf{રેઝિસ્ટિવ લોડ ઇન્વર્ટર સમજાવો.}

\begin{solutionbox}

\textbf{સર્કિટ અને એનાલિસિસ:}

{\def\LTcaptype{none} % do not increment counter
\begin{longtable}[]{@{}lll@{}}
\toprule\noalign{}
કમ્પોનન્ટ & ફંક્શન & લાક્ષણિકતા \\
\midrule\noalign{}
\endhead
\bottomrule\noalign{}
\endlastfoot
\textbf{NMOS ટ્રાન્ઝિસ્ટર} & સ્વિચિંગ ડિવાઇસ & વેરિએબલ રેઝિસ્ટન્સ \\
\textbf{લોડ રેઝિસ્ટર} & પુલ-અપ એલિમેન્ટ & ફિક્સ્ડ રેઝિસ્ટન્સ RL \\
\textbf{પાવર સપ્લાય} & વોલ્ટેજ સોર્સ & VDD પ્રદાન કરે છે \\
\end{longtable}
}

\textbf{ઓપરેટિંગ પ્રિન્સિપલ:}

\begin{itemize}
\tightlist
\item
  \textbf{હાઇ ઇનપુટ}: ટ્રાન્ઝિસ્ટર ચાલુ, VOUT = ID \times RL (લો)
\item
  \textbf{લો ઇનપુટ}: ટ્રાન્ઝિસ્ટર બંધ, VOUT = VDD (હાઇ)
\item
  \textbf{કરંટ પાથ}: આઉટપુટ લો હોય ત્યારે હંમેશા રેઝિસ્ટર દ્વારા
\item
  \textbf{પાવર કન્ઝમ્પશન}: સ્ટેટિક પાવર = VDD^{2}/RL
\end{itemize}

\textbf{ફાયદા અને નુકસાન:}

\begin{itemize}
\tightlist
\item
  \textbf{સિમ્પલ ડિઝાઇન}: સમજવામાં અને ઇમ્પ્લિમેન્ટ કરવામાં સરળ
\item
  \textbf{ખરાબ પરફોર્મન્સ}: હાઇ સ્ટેટિક પાવર, સ્લો સ્વિચિંગ
\item
  \textbf{મર્યાદિત ઉપયોગ}: મુખ્યત્વે કોન્સેપ્ટ સમજવા માટે
\end{itemize}

\end{solutionbox}
\begin{mnemonicbox}
``રેઝિસ્ટર રિસ્ટ્રિક્ટ કરે કરંટ, રિડ્યુસ કરે પરફોર્મન્સ''

\end{mnemonicbox}
\subsection*{પ્રશ્ન 3(ક) OR [7
ગુણ]}\label{uxaaauxab0uxab6uxaa8-3uxa95-or-7-uxa97uxaa3}

\textbf{CMOS ઇન્વર્ટર તેની VTC સાથે સમજાવો.}

\begin{solutionbox}

\textbf{સર્કિટ કોન્ફિગરેશન:}

\begin{lstlisting}
           VDD
            |
         +--+--+
    VIN -+     +- VOUT
         | PMOS|
         +-----+
            |
         +--+--+
    VIN -+     |
         | NMOS+- VOUT
         +-----+
            |
           GND
\end{lstlisting}

\textbf{VTC રીજન:}

{\def\LTcaptype{none} % do not increment counter
\begin{longtable}[]{@{}lllll@{}}
\toprule\noalign{}
રીજન & ઇનપુટ રેન્જ & PMOS સ્ટેટ & NMOS સ્ટેટ & આઉટપુટ \\
\midrule\noalign{}
\endhead
\bottomrule\noalign{}
\endlastfoot
\textbf{1} & VIN \textless{} VTN & ચાલુ & બંધ & VDD \\
\textbf{2} & VTN \textless{} VIN \textless{} VDD/2 & ચાલુ & ચાલુ &
ટ્રાન્ઝિશન \\
\textbf{3} & VDD/2 \textless{} VIN \textless{} VDD+VTP & ચાલુ & ચાલુ &
ટ્રાન્ઝિશન \\
\textbf{4} & VIN \textgreater{} VDD+VTP & બંધ & ચાલુ & 0 \\
\end{longtable}
}

\textbf{મુખ્ય લાક્ષણિકતાઓ:}

\includegraphics[width=1\linewidth,height=\textheight,keepaspectratio]{mermaid-0e5866af.pdf}

\begin{itemize}
\tightlist
\item
  \textbf{કોમ્પ્લિમેન્ટરી ઓપરેશન}: સ્ટેડી સ્ટેટમાં માત્ર એક ટ્રાન્ઝિસ્ટર વહન કરે છે
\item
  \textbf{સ્વિચિંગ પોઇન્ટ}: PMOS/NMOS રેશિયો દ્વારા નક્કી થાય છે
\item
  \textbf{પાવર એફિશિયન્સી}: ન્યૂનતમ સ્ટેટિક પાવર કન્ઝમ્પશન
\item
  \textbf{નોઇઝ ઇમ્યુનિટી}: ઉત્તમ નોઇઝ માર્જિન
\end{itemize}

\end{solutionbox}
\begin{mnemonicbox}
``CMOS: કોમ્પ્લિમેન્ટરી ફોર કોમ્પ્લીટ પરફોર્મન્સ''

\end{mnemonicbox}
\subsection*{પ્રશ્ન 4(અ) [3
ગુણ]}\label{uxaaauxab0uxab6uxaa8-4uxa85-3-uxa97uxaa3}

\textbf{AOI CMOS ઇમ્પ્લિમેન્ટેશન સાથે દોરો.}

\begin{solutionbox}

\textbf{AOI (AND-OR-INVERT) લોજિક:} Y = (AB + CD)'

\textbf{CMOS ઇમ્પ્લિમેન્ટેશન:}

\begin{lstlisting}
        VDD
         |
    +----+----+
    |         |
   PMOS     PMOS  (A')
    A'       B'
    |         |
    +----+----+
         |
    +----+----+
    |         |
   PMOS     PMOS  (C')
    C'       D'
    |         |
    +----+----+-- VOUT
         |
    +----+----+
    |         |
   NMOS     NMOS  (સીરીઝ: AB)
    A        B
    |         |
    +----+----+
         |
    +----+----+
    |         |
   NMOS     NMOS  (પેરેલલ: CD)
    C        D
    |         |
    +----+----+
         |
        GND
\end{lstlisting}

\begin{itemize}
\tightlist
\item
  \textbf{પુલ-અપ નેટવર્ક}: PMOS ટ્રાન્ઝિસ્ટર સીરીઝ-પેરેલલમાં
\item
  \textbf{પુલ-ડાઉન નેટવર્ક}: NMOS ટ્રાન્ઝિસ્ટર પેરેલલ-સીરીઝમાં
\item
  \textbf{ડ્યુઆલિટી}: પુલ-અપ અને પુલ-ડાઉન કોમ્પ્લિમેન્ટ છે
\end{itemize}

\end{solutionbox}
\begin{mnemonicbox}
``AOI: AND-OR પછી ઇન્વર્ટ''

\end{mnemonicbox}
\subsection*{પ્રશ્ન 4(બ) [4
ગુણ]}\label{uxaaauxab0uxab6uxaa8-4uxaac-4-uxa97uxaa3}

\textbf{બે ઇનપુટ NOR અને NAND ગેટ ડીપ્લેશન લોડ nMOS થી બનાવો.}

\begin{solutionbox}

\textbf{NOR ગેટ:}

\begin{lstlisting}
        VDD
         |
      +--+--+ (ડીપ્લેશન લોડ)
 VG --|     |
      |     |
      +-----+-- VOUT
         |
    +----+----+
    |         |
   NMOS     NMOS  (પેરેલલ)
    A        B
    |         |
    +----+----+
         |
        GND
\end{lstlisting}

\textbf{NAND ગેટ:}

\begin{lstlisting}
        VDD
         |
      +--+--+ (ડીપ્લેશન લોડ)
 VG --|     |
      |     |
      +-----+-- VOUT
         |
      +--+--+
  A --|     |
      | NMOS|  (સીરીઝ)
      +-----+
         |
      +--+--+
  B --|     |
      | NMOS|
      +-----+
         |
        GND
\end{lstlisting}

\textbf{ટ્રુથ ટેબલ:}

{\def\LTcaptype{none} % do not increment counter
\begin{longtable}[]{@{}llll@{}}
\toprule\noalign{}
A & B & NOR & NAND \\
\midrule\noalign{}
\endhead
\bottomrule\noalign{}
\endlastfoot
0 & 0 & 1 & 1 \\
0 & 1 & 0 & 1 \\
1 & 0 & 0 & 1 \\
1 & 1 & 0 & 0 \\
\end{longtable}
}

\end{solutionbox}
\begin{mnemonicbox}
``NOR ને કંઈ હાઇ નહીં જોઈએ, NAND ને બધું હાઇ જોઈએ લો થવા
માટે''

\end{mnemonicbox}
\subsection*{પ્રશ્ન 4(ક) [7
ગુણ]}\label{uxaaauxab0uxab6uxaa8-4uxa95-7-uxa97uxaa3}

\textbf{NOR2 અને NAND2 ગેટનો ઉપયોગ કરીને CMOS SR લેચ ઇમ્પ્લિમેન્ટ કરો.}

\begin{solutionbox}

\textbf{NOR ગેટ વડે SR લેચ:}

\begin{lstlisting}
    S ----+---[NOR]---+---- Q
          |           |
          +-----+     |
                |     |
          +-----+     |
          |           |
    R ----+---[NOR]---+---- Q'
                      |
                      +-----
\end{lstlisting}

\textbf{CMOS NOR ગેટ ઇમ્પ્લિમેન્ટેશન:}

\includegraphics[width=1\linewidth,height=\textheight,keepaspectratio]{mermaid-24780da1.pdf}

\textbf{સ્ટેટ ટેબલ:}

{\def\LTcaptype{none} % do not increment counter
\begin{longtable}[]{@{}lllll@{}}
\toprule\noalign{}
S & R & Q(n+1) & Q'(n+1) & એક્શન \\
\midrule\noalign{}
\endhead
\bottomrule\noalign{}
\endlastfoot
0 & 0 & Q(n) & Q'(n) & હોલ્ડ \\
0 & 1 & 0 & 1 & રીસેટ \\
1 & 0 & 1 & 0 & સેટ \\
1 & 1 & 0 & 0 & અમાન્ય \\
\end{longtable}
}

\begin{itemize}
\tightlist
\item
  \textbf{ક્રોસ-કપ્લ્ડ સ્ટ્રક્ચર}: દરેક ગેટનું આઉટપુટ બીજાના ઇનપુટને ફીડ કરે છે
\item
  \textbf{બાઇસ્ટેબલ ઓપરેશન}: બે સ્થિર અવસ્થા (સેટ અને રીસેટ)
\item
  \textbf{મેમરી એલિમેન્ટ}: એક બિટ માહિતી સ્ટોર કરે છે
\item
  \textbf{ક્લોક ઇન્ડિપેન્ડન્સ}: એસિંક્રોનસ ઓપરેશન
\end{itemize}

\end{solutionbox}
\begin{mnemonicbox}
``SR લેચ: સેટ-રીસેટ વિથ ક્રોસ-કપ્લ્ડ ગેટ''

\end{mnemonicbox}
\subsection*{પ્રશ્ન 4(અ) OR [3
ગુણ]}\label{uxaaauxab0uxab6uxaa8-4uxa85-or-3-uxa97uxaa3}

\textbf{CMOS નો ઉપયોગ કરીને XOR ફંક્શન ઇમ્પ્લિમેન્ટ કરો.}

\begin{solutionbox}

\textbf{XOR ટ્રુથ ટેબલ:}

{\def\LTcaptype{none} % do not increment counter
\begin{longtable}[]{@{}lll@{}}
\toprule\noalign{}
A & B & Y = A\oplusB \\
\midrule\noalign{}
\endhead
\bottomrule\noalign{}
\endlastfoot
0 & 0 & 0 \\
0 & 1 & 1 \\
1 & 0 & 1 \\
1 & 1 & 0 \\
\end{longtable}
}

\textbf{CMOS XOR ઇમ્પ્લિમેન્ટેશન:}

\begin{lstlisting}
        VDD
         |
    +----+----+
    |         |
 A'-+PMOS  PMOS+-B'
    |         |
    +----+----+
         |
    +----+----+
    |         |
 B'-+PMOS  PMOS+-A'
    |         |
    +----+----+-- VOUT
         |
    +----+----+
    |         |
 A--+NMOS  NMOS+-B
    |         |
    +----+----+
         |
    +----+----+
    |         |
 B--+NMOS  NMOS+-A
    |         |
    +----+----+
         |
        GND
\end{lstlisting}

\begin{itemize}
\tightlist
\item
  \textbf{ફંક્શન}: Y = AB' + A'B
\item
  \textbf{ટ્રાન્ઝિસ્ટર કાઉન્ટ}: 8 ટ્રાન્ઝિસ્ટર (4 PMOS + 4 NMOS)
\item
  \textbf{વિકલ્પ}: ટ્રાન્સમિશન ગેટ ઇમ્પ્લિમેન્ટેશન
\end{itemize}

\end{solutionbox}
\begin{mnemonicbox}
``XOR: એક્સક્લુસિવ OR, અલગ ઇનપુટ આપે 1''

\end{mnemonicbox}
\subsection*{પ્રશ્ન 4(બ) OR [4
ગુણ]}\label{uxaaauxab0uxab6uxaa8-4uxaac-or-4-uxa97uxaa3}

\textbf{બે ઇનપુટ NOR અને NAND ગેટ CMOS થી બનાવો.}

\begin{solutionbox}

\textbf{CMOS NOR ગેટ:}

\begin{lstlisting}
        VDD
         |
    +----+----+
    |         |
 A'-+PMOS  PMOS+-B'  (સીરીઝ)
    |         |
    +----+----+-- VOUT
         |
    +----+----+
    |         |
 A--+NMOS     +-B
    |    NMOS |      (પેરેલલ)
    +----+----+
         |
        GND
\end{lstlisting}

\textbf{CMOS NAND ગેટ:}

\begin{lstlisting}
        VDD
         |
    +----+----+
    |         |
 A'-+PMOS     +-B'  (પેરેલલ)
    |    PMOS |
    +----+----+-- VOUT
         |
    +----+----+
    |         |
 A--+NMOS  NMOS+-B  (સીરીઝ)
    |         |
    +----+----+
         |
        GND
\end{lstlisting}

\textbf{ડિઝાઇન નિયમો:}

{\def\LTcaptype{none} % do not increment counter
\begin{longtable}[]{@{}lll@{}}
\toprule\noalign{}
ગેટ & પુલ-અપ નેટવર્ક & પુલ-ડાઉન નેટવર્ક \\
\midrule\noalign{}
\endhead
\bottomrule\noalign{}
\endlastfoot
\textbf{NAND} & PMOS પેરેલલમાં & NMOS સીરીઝમાં \\
\textbf{NOR} & PMOS સીરીઝમાં & NMOS પેરેલલમાં \\
\end{longtable}
}

\end{solutionbox}
\begin{mnemonicbox}
``NAND: નોટ AND, NOR: નોટ OR - નેટવર્ક કોમ્પ્લિમેન્ટ કરો''

\end{mnemonicbox}
\subsection*{પ્રશ્ન 4(ક) OR [7
ગુણ]}\label{uxaaauxab0uxab6uxaa8-4uxa95-or-7-uxa97uxaa3}

\textbf{Y=[PQ+R(S+T)]' બુલિયન સમીકરણ ડીપ્લેશન લોડ nMOS અને CMOS થી
ઇમ્પ્લિમેન્ટ કરો.}

\begin{solutionbox}

\textbf{બુલિયન એનાલિસિસ:}

\begin{itemize}
\tightlist
\item
  ફંક્શન: Y = [PQ + R(S+T)]'
\item
  વિસ્તૃત: Y = [PQ + RS + RT]'
\item
  ડે મોર્ગન: Y = (PQ)' · (RS)' · (RT)'
\item
  અંતિમ: Y = (P'+Q') · (R'+S') · (R'+T')
\end{itemize}

\textbf{nMOS ઇમ્પ્લિમેન્ટેશન:}

\begin{lstlisting}
        VDD
         |
      +--+--+ (ડીપ્લેશન લોડ)
      |     |
      +-----+-- VOUT
         |
    P--+NMOS+--+
              |
    Q--+NMOS+--+
              |
              +-- (PQ બ્રાન્ચ)
              |
    R--+NMOS+--+
              |
         +----+----+
         |         |
    S--+NMOS   NMOS+--T
         |         |
         +---------+
              |
             GND
\end{lstlisting}

\textbf{CMOS ઇમ્પ્લિમેન્ટેશન:}

\includegraphics[width=1\linewidth,height=\textheight,keepaspectratio]{mermaid-bed3ca59.pdf}

\begin{itemize}
\tightlist
\item
  \textbf{nMOS લાક્ષણિકતા}: સિમ્પલ પણ સ્ટેટિક પાવર સાથે
\item
  \textbf{CMOS ફાયદા}: સ્ટેટિક પાવર નથી, ફુલ સ્વિંગ
\item
  \textbf{જટિલતા}: nMOS માટે 7 ટ્રાન્ઝિસ્ટર, CMOS માટે 14
\item
  \textbf{પરફોર્મન્સ}: CMOS ઝડપી અને વધુ કાર્યક્ષમ
\end{itemize}

\end{solutionbox}
\begin{mnemonicbox}
``બુલિયન ટુ સર્કિટ: nMOS સિમ્પલ, CMOS કોમ્પ્લીટ''

\end{mnemonicbox}
\subsection*{પ્રશ્ન 5(અ) [3
ગુણ]}\label{uxaaauxab0uxab6uxaa8-5uxa85-3-uxa97uxaa3}

\textbf{વેરિલોગમાં ઉપયોગ થતી ડિઝાઇન સ્ટાઇલ સમજાવો.}

\begin{solutionbox}

\textbf{વેરિલોગ ડિઝાઇન સ્ટાઇલ:}

{\def\LTcaptype{none} % do not increment counter
\begin{longtable}[]{@{}lll@{}}
\toprule\noalign{}
સ્ટાઇલ & વર્ણન & એપ્લિકેશન \\
\midrule\noalign{}
\endhead
\bottomrule\noalign{}
\endlastfoot
\textbf{ગેટ લેવલ} & પ્રિમિટિવ ગેટનો ઉપયોગ & લો-લેવલ હાર્ડવેર મોડેલિંગ \\
\textbf{ડેટા ફ્લો} & assign સ્ટેટમેન્ટનો ઉપયોગ & કોમ્બિનેશનલ લોજિક \\
\textbf{બિહેવિયરલ} & always બ્લોકનો ઉપયોગ & સિક્વેન્શિયલ અને કોમ્પ્લેક્સ લોજિક \\
\textbf{મિક્સ્ડ} & સ્ટાઇલનું કોમ્બિનેશન & સંપૂર્ણ સિસ્ટમ ડિઝાઇન \\
\end{longtable}
}

\begin{itemize}
\tightlist
\item
  \textbf{ગેટ લેવલ}: and, or, not, nand, nor પ્રિમિટિવ
\item
  \textbf{ડેટા ફ્લો}: ઓપરેટર સાથે કંટિન્યુઅસ એસાઇનમેન્ટ
\item
  \textbf{બિહેવિયરલ}: always બ્લોકમાં પ્રોસિજરલ એસાઇનમેન્ટ
\item
  \textbf{હાઇરારકી}: મોડ્યુલ અલગ સ્ટાઇલ વાપરી શકે છે
\end{itemize}

\end{solutionbox}
\begin{mnemonicbox}
``ગેટ-ડેટા-બિહેવિયર: મોડેલ કરવાની ત્રણ રીત''

\end{mnemonicbox}
\subsection*{પ્રશ્ન 5(બ) [4
ગુણ]}\label{uxaaauxab0uxab6uxaa8-5uxaac-4-uxa97uxaa3}

\textbf{બિહેવિયરલ મોડેલિંગ થી ફુલ એડર માટે વેરિલોગ પ્રોગ્રામ લખો.}

\begin{solutionbox}

\begin{lstlisting}[language=Verilog]
module full_adder_behavioral (
    input wire a, b, cin,
    output reg sum, cout
);

always @(*) begin
    case ({a, b, cin})
        3'b000: {cout, sum} = 2'b00;
        3'b001: {cout, sum} = 2'b01;
        3'b010: {cout, sum} = 2'b01;
        3'b011: {cout, sum} = 2'b10;
        3'b100: {cout, sum} = 2'b01;
        3'b101: {cout, sum} = 2'b10;
        3'b110: {cout, sum} = 2'b10;
        3'b111: {cout, sum} = 2'b11;
        default: {cout, sum} = 2'b00;
    endcase
end

endmodule
\end{lstlisting}

\textbf{મુખ્ય લક્ષણો:}

\begin{itemize}
\tightlist
\item
  \textbf{Always બ્લોક}: બિહેવિયરલ મોડેલિંગ કન્સ્ટ્રક્ટ
\item
  \textbf{Case સ્ટેટમેન્ટ}: ટ્રુથ ટેબલ ઇમ્પ્લિમેન્ટેશન
\item
  \textbf{કોનકેટેનેશન}: કોમ્બાઇન्ड આઉટપુટ માટે \{cout, sum\}
\item
  \textbf{સેન્સિટિવિટી લિસ્ટ}: કોમ્બિનેશનલ લોજિક માટે @(*)
\end{itemize}

\end{solutionbox}
\begin{mnemonicbox}
``બિહેવિયરલ યુઝ કરે Always વિથ Case સ્ટેટમેન્ટ''

\end{mnemonicbox}
\subsection*{પ્રશ્ન 5(ક) [7
ગુણ]}\label{uxaaauxab0uxab6uxaa8-5uxa95-7-uxa97uxaa3}

\textbf{CASE સ્ટેટમેન્ટનું ફંક્શન વર્ણવો. CASE સ્ટેટમેન્ટનો ઉપયોગ કરીને 3x8 ડિકોડરનો
વેરિલોગ કોડ લખો.}

\begin{solutionbox}

\textbf{CASE સ્ટેટમેન્ટ ફંક્શન:}

{\def\LTcaptype{none} % do not increment counter
\begin{longtable}[]{@{}lll@{}}
\toprule\noalign{}
લક્ષણ & વર્ણન & ઉપયોગ \\
\midrule\noalign{}
\endhead
\bottomrule\noalign{}
\endlastfoot
\textbf{મલ્ટિ-વે બ્રાન્ચ} & અનેક વિકલ્પોમાંથી એક પસંદ કરે & C માં switch જેવું \\
\textbf{પેટર્ન મેચિંગ} & એક્સપ્રેશનને કોન્સ્ટન્ટ સાથે કોમ્પેર કરે & બરાબર બિટ મેચિંગ \\
\textbf{પ્રાયોરિટી એન્કોડિંગ} & પહેલું મેચ જીતે છે & ટોપ-ડાઉન ઇવેલ્યુએશન \\
\textbf{ડિફોલ્ટ ક્લોઝ} & અનસ્પેસિફાઇડ કેસ હેન્ડલ કરે & લેચ અટકાવે છે \\
\end{longtable}
}

\textbf{3x8 ડિકોડર વેરિલોગ કોડ:}

\begin{lstlisting}[language=Verilog]
module decoder_3x8 (
    input wire [2:0] select,
    input wire enable,
    output reg [7:0] out
);

always @(*) begin
    if (enable) begin
        case (select)
            3'b000: out = 8'b00000001;
            3'b001: out = 8'b00000010;
            3'b010: out = 8'b00000100;
            3'b011: out = 8'b00001000;
            3'b100: out = 8'b00010000;
            3'b101: out = 8'b00100000;
            3'b110: out = 8'b01000000;
            3'b111: out = 8'b10000000;
            default: out = 8'b00000000;
        endcase
    end else begin
        out = 8'b00000000;
    end
end

endmodule
\end{lstlisting}

\textbf{CASE સ્ટેટમેન્ટ લક્ષણો:}

\begin{itemize}
\tightlist
\item
  \textbf{બરાબર મેચિંગ}: બધા બિટ બરાબર મેચ થવા જોઈએ
\item
  \textbf{પેરેલલ ઇવેલ્યુએશન}: હાર્ડવેર ઇમ્પ્લિમેન્ટેશન પેરેલલ છે
\item
  \textbf{સંપૂર્ણ સ્પેસિફિકેશન}: બધા શક્ય ઇનપુટ કોમ્બિનેશન કવર કર્યા
\item
  \textbf{ડિફોલ્ટ ક્લોઝ}: સિન્થેસિસમાં અનઇન્ટેન્ડેડ લેચ અટકાવે છે
\end{itemize}

\end{solutionbox}
\begin{mnemonicbox}
``CASE કોમ્પેર કરે બધું સ્પેસિફાઇડ એક્ઝેક્ટલી''

\end{mnemonicbox}
\subsection*{પ્રશ્ન 5(અ) OR [3
ગુણ]}\label{uxaaauxab0uxab6uxaa8-5uxa85-or-3-uxa97uxaa3}

\textbf{2:1 મલ્ટિપ્લેક્સર ઇમ્પ્લિમેન્ટ કરતો વેરિલોગ કોડ લખો.}

\begin{solutionbox}

\begin{lstlisting}[language=Verilog]
module mux_2to1 (
    input wire a, b, sel,
    output wire y
);

assign y = sel ? b : a;

endmodule
\end{lstlisting}

\textbf{વિકલ્પિત ઇમ્પ્લિમેન્ટેશન:}

{\def\LTcaptype{none} % do not increment counter
\begin{longtable}[]{@{}lll@{}}
\toprule\noalign{}
સ્ટાઇલ & કોડ & ઉપયોગ કેસ \\
\midrule\noalign{}
\endhead
\bottomrule\noalign{}
\endlastfoot
\textbf{ડેટા ફ્લો} & assign y = sel ? b : a; & સિમ્પલ લોજિક \\
\textbf{ગેટ લેવલ} & and, or, not ગેટ વાપરે & શીખવાનો હેતુ \\
\textbf{બિહેવિયરલ} & if-else સાથે always બ્લોક & કોમ્પ્લેક્સ કન્ડિશન \\
\end{longtable}
}

\begin{itemize}
\tightlist
\item
  \textbf{કન્ડિશનલ ઓપરેટર}: ? : મલ્ટિપ્લેક્સર ફંક્શન પ્રદાન કરે
\item
  \textbf{કંટિન્યુઅસ એસાઇનમેન્ટ}: કોમ્બિનેશનલ લોજિક માટે assign
\item
  \textbf{સિન્થેસિસ}: ટૂલ ગેટ-લે lველ ઇમ્પ્લિમેન્ટેશનમાં કન્વર્ટ કરે
\end{itemize}

\end{solutionbox}
\begin{mnemonicbox}
``MUX: sel ? b : a - ઇનપુટ વચ્ચે પસંદગી''

\end{mnemonicbox}
\subsection*{પ્રશ્ન 5(બ) OR [4
ગુણ]}\label{uxaaauxab0uxab6uxaa8-5uxaac-or-4-uxa97uxaa3}

\textbf{બિહેવિયરલ મોડેલિંગ થી D ફ્લિપ-ફ્લોપ માટે વેરિલોગ પ્રોગ્રામ લખો.}

\begin{solutionbox}

\begin{lstlisting}[language=Verilog]
module d_flipflop (
    input wire clk, reset, d,
    output reg q, qbar
);

always @(posedge clk or posedge reset) begin
    if (reset) begin
        q <= 1'b0;
        qbar <= 1'b1;
    end else begin
        q <= d;
        qbar <= ~d;
    end
end

endmodule
\end{lstlisting}

\textbf{મુખ્ય લક્ષણો:}

{\def\LTcaptype{none} % do not increment counter
\begin{longtable}[]{@{}lll@{}}
\toprule\noalign{}
એલિમેન્ટ & ફંક્શન & સિન્ટેક્સ \\
\midrule\noalign{}
\endhead
\bottomrule\noalign{}
\endlastfoot
\textbf{posedge clk} & રાઇઝિંગ એજ ટ્રિગર & ક્લોક સિંક્રોનાઇઝેશન \\
\textbf{posedge reset} & એસિંક્રોનસ રીસેટ & તાત્કાલિક રીસેટ એક્શન \\
\textbf{નોન-બ્લોકિંગ} & \textless= ઓપરેટર & સિક્વેન્શિયલ લોજિક \\
\textbf{કોમ્પ્લિમેન્ટરી} & qbar = \textasciitilde q & સાચી ફ્લિપ-ફ્લોપ
બિહેવિયર \\
\end{longtable}
}

\begin{itemize}
\tightlist
\item
  \textbf{એજ સેન્સિટિવિટી}: માત્ર ક્લોક એજ પર પ્રતિભાવ આપે
\item
  \textbf{એસિંક્રોનસ રીસેટ}: રીસેટ ક્લોક કરતાં પ્રાથમિકતા લે
\item
  \textbf{સિક્વેન્શિયલ લોజિક}: નોન-બ્લોકિંગ એસાઇનમેન્ટ વાપરે
\item
  \textbf{સ્ટેટ સ્ટોરેજ}: ક્લોક સાઇકલ વચ્ચે ડેટા જાળવે
\end{itemize}

\end{solutionbox}
\begin{mnemonicbox}
``D ફ્લિપ-ફ્લોપ: ડેટા ફોલો કરે ક્લોક વિથ રીસેટ''

\end{mnemonicbox}
\subsection*{પ્રશ્ન 5(ક) OR [7
ગુણ]}\label{uxaaauxab0uxab6uxaa8-5uxa95-or-7-uxa97uxaa3}

\textbf{ટેસ્ટબેંચ ટૂંકમાં વર્ણવો. 4-બિટ ડાઉન કાઉન્ટર ઇમ્પ્લિમેન્ટ કરવાનો વેરિલોગ કોડ
લખો.}

\begin{solutionbox}

\textbf{ટેસ્ટબેંચ ઓવરવ્યુ:}

{\def\LTcaptype{none} % do not increment counter
\begin{longtable}[]{@{}lll@{}}
\toprule\noalign{}
કમ્પોનન્ટ & ઉદ્દેશ્ય & ઇમ્પ્લિમેન્ટેશન \\
\midrule\noalign{}
\endhead
\bottomrule\noalign{}
\endlastfoot
\textbf{સ્ટિમ્યુલસ જનરેશન} & ટેસ્ટ ઇનપુટ પ્રદાન કરવું & ક્લોક, રીસેટ, કંટ્રોલ સિગ્નલ \\
\textbf{રિસ્પોન્સ મોનિટરિંગ} & આઉટપુટ ચેક કરવું & અપેક્ષિત મૂલ્ય સાથે સરખાવો \\
\textbf{કવરેજ એનાલિસિસ} & સંપૂર્ણતા ચકાસવી & બધી સ્ટેટ અને ટ્રાન્ઝિશન \\
\textbf{ડિબગિંગ સપોર્ટ} & સમસ્યા ઓળખવી & વેવફોર્મ એનાલિસિસ \\
\end{longtable}
}

\textbf{4-બિટ ડાઉન કાઉન્ટર:}

\begin{lstlisting}[language=Verilog]
module down_counter_4bit (
    input wire clk, reset, enable,
    output reg [3:0] count
);

always @(posedge clk or posedge reset) begin
    if (reset) begin
        count <= 4'b1111;  // મેક્સિમમ વેલ્યુથી શરૂ કરો
    end else if (enable) begin
        if (count == 4'b0000)
            count <= 4'b1111;  // રેપ અરાઉન્ડ
        else
            count <= count - 1;  // ડેક્રિમેન્ટ
    end
end

endmodule

// ડાઉન કાઉન્ટર માટે ટેસ્ટબેંચ
module tb_down_counter;
    reg clk, reset, enable;
    wire [3:0] count;
    
    down_counter_4bit dut (
        .clk(clk), 
        .reset(reset), 
        .enable(enable), 
        .count(count)
    );
    
    // ક્લોક જનરેશન
    always #5 clk = ~clk;
    
    initial begin
        clk = 0;
        reset = 1;
        enable = 0;
        
        #10 reset = 0;
        #10 enable = 1;
        
        #200 $finish;
    end
    
    // આઉટપુટ મોનિટર
    initial begin
        $monitor("Time=%0t, Reset=%b, Enable=%b, Count=%b", 
                 $time, reset, enable, count);
    end
    
endmodule
\end{lstlisting}

\textbf{ટેસ્ટબેંચ કમ્પોનન્ટ:}

\begin{itemize}
\tightlist
\item
  \textbf{ક્લોક જનરેશન}: always બ્લોક વડે કંટિન્યુઅસ ક્લોક
\item
  \textbf{સ્ટિમ્યુલસ}: રીસેટ અને enable સિગ્નલ કંટ્રોલ
\item
  \textbf{મોનિટરિંગ}: કંટિન્યુઅસ આઉટપુટ ડિસ્પ્લે
\item
  \textbf{સિમ્યુલેશન કંટ્રોલ}: સિમ્યુલેશન અંત કરવા માટે \$finish
\end{itemize}

\textbf{કાઉન્ટર લક્ષણો:}

\begin{itemize}
\tightlist
\item
  \textbf{ડાઉન કાઉન્ટિંગ}: 15 થી 0 સુધી ડેક્રિમેન્ટ
\item
  \textbf{રેપ અરાઉન્ડ}: 0 પહોંચ્યા પછી 15 પર પાછું ફરે
\item
  \textbf{enable કંટ્રોલ}: માત્ર enable હોય ત્યારે જ કાઉન્ટિંગ
\item
  \textbf{સિંક્રોનસ ઓપરેશન}: ક્લોક એજ પર બધા ફેરફાર
\end{itemize}

\end{solutionbox}
\begin{mnemonicbox}
``ટેસ્ટબેંચ ટેસ્ટ કરે ક્લોક, સ્ટિમ્યુલસ અને મોનિટર સાથે''

\end{mnemonicbox}

\end{document}
