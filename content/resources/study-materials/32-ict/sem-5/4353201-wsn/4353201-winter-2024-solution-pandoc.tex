\documentclass[10pt,a4paper]{article}

% content/resources/templates/preamble.tex
\usepackage[margin=0.6in]{geometry}
\author{Milav Dabgar}
\usepackage{amsmath,amssymb,amsthm}
\usepackage{booktabs}
\usepackage{multirow}
\usepackage{xcolor}
\usepackage{tcolorbox}
\tcbuselibrary{breakable,skins}
\usepackage[colorlinks=true,linkcolor=blue]{hyperref}
\usepackage{titlesec}
\usepackage{enumitem}
\usepackage{tikz}
\usepackage{pgfplots}
\usepackage{circuitikz}
\usepackage[version=4]{mhchem}
\usepackage{longtable}
\usepackage{array}
\usepackage{float}
\usepackage{caption}
\usepackage{listings}

\lstset{
  basicstyle=\small\ttfamily,
  breaklines=true,
  breakatwhitespace=false,
  postbreak=\mbox{\textcolor{red}{$\hookrightarrow$}\space},
  float=false,
  numbers=left,
  numberstyle=\tiny\color{gray},
  numbersep=10pt,
  xleftmargin=2em,
  keywordstyle=\color{blue},
  commentstyle=\color{green!60!black},
  stringstyle=\color{purple},
  backgroundcolor=\color{gray!5},
  showstringspaces=false,
  tabsize=2,
  captionpos=b,
  keepspaces=true,
  columns=flexible
}

\pgfplotsset{compat=1.18}
\usetikzlibrary{shapes,arrows,positioning,calc,patterns,decorations.pathmorphing,decorations.markings,arrows.meta}

% Color scheme
\definecolor{headcolor}{RGB}{0,102,204}
\definecolor{keycolor}{RGB}{220,20,60}
\definecolor{solutioncolor}{RGB}{34,139,34}
\definecolor{mnemoniccolor}{RGB}{148,0,211}
\definecolor{codecolor}{RGB}{0,0,100}

% Spacing
\setlength{\parskip}{3pt}
\setlist[itemize]{nosep}
\setlist[enumerate]{nosep}

% Title formatting
\titleformat{\section}{\Large\bfseries\color{headcolor}}{\thesection}{1em}{}
\titleformat{\subsection}{\large\bfseries\color{headcolor}}{\thesubsection}{1em}{}

% Pandoc tightlist compatibility
\providecommand{\tightlist}{%
  \setlength{\itemsep}{0pt}\setlength{\parskip}{0pt}}

% Pandoc longtable compatibility
\newcounter{none}
\def\thenone{}


% content/resources/templates/english-boxes.tex
% This file is currently empty - it exists to maintain consistency with the import structure.
% Add custom environments here if needed in the future.


\begin{document}

\begin{center}
{\Huge\bfseries\color{headcolor} Subject Name Solutions}\\[5pt]
{\LARGE 4353201 -- Winter 2024}\\[3pt]
{\large Semester 1 Study Material}\\[3pt]
{\normalsize\textit{Detailed Solutions and Explanations}}
\end{center}

\vspace{10pt}

\subsection*{Question 1(a) [3 marks]}\label{q1a}

\textbf{Compare Single hop and Multihop Network.}

\begin{solutionbox}

{\def\LTcaptype{none} % do not increment counter
\begin{longtable}[]{@{}
  >{\raggedright\arraybackslash}p{(\linewidth - 4\tabcolsep) * \real{0.2292}}
  >{\raggedright\arraybackslash}p{(\linewidth - 4\tabcolsep) * \real{0.3958}}
  >{\raggedright\arraybackslash}p{(\linewidth - 4\tabcolsep) * \real{0.3750}}@{}}
\toprule\noalign{}
\begin{minipage}[b]{\linewidth}\raggedright
Parameter
\end{minipage} & \begin{minipage}[b]{\linewidth}\raggedright
Single Hop Network
\end{minipage} & \begin{minipage}[b]{\linewidth}\raggedright
Multihop Network
\end{minipage} \\
\midrule\noalign{}
\endhead
\bottomrule\noalign{}
\endlastfoot
\textbf{Communication} & Direct to base station & Via intermediate
nodes \\
\textbf{Energy consumption} & High for distant nodes & Distributed among
nodes \\
\textbf{Network coverage} & Limited by transmission range & Extended
coverage area \\
\textbf{Complexity} & Simple routing & Complex routing protocols \\
\end{longtable}
}

\begin{itemize}
\tightlist
\item
  \textbf{Single hop}: All nodes communicate directly with base station
\item
  \textbf{Multihop}: Data passes through multiple intermediate nodes to
  reach destination
\end{itemize}

\end{solutionbox}
\begin{mnemonicbox}
``Single Direct, Multi Relay''

\end{mnemonicbox}
\subsection*{Question 1(b) [4 marks]}\label{q1b}

\textbf{Explain the Basic Components of Sensor Node.}

\begin{solutionbox}

\includegraphics[width=1\linewidth,height=\textheight,keepaspectratio]{mermaid-a6a09802.pdf}

\textbf{Basic Components:}

\begin{itemize}
\tightlist
\item
  \textbf{Sensing subsystem}: Collects data from environment using
  sensors and ADC
\item
  \textbf{Processing subsystem}: Microcontroller/processor with memory
  for data processing
\item
  \textbf{Communication subsystem}: Radio transceiver for wireless data
  transmission
\item
  \textbf{Power subsystem}: Battery or energy harvesting unit for power
  supply
\end{itemize}

\end{solutionbox}
\begin{mnemonicbox}
``Sense Process Communicate Power''

\end{mnemonicbox}
\subsection*{Question 1(c) [7 marks]}\label{q1c}

\textbf{List out any four technologies to reduce power consumption in
WSN and explain any two technologies in detail.}

\begin{solutionbox}

\textbf{Four Power Reduction Technologies:}

{\def\LTcaptype{none} % do not increment counter
\begin{longtable}[]{@{}
  >{\raggedright\arraybackslash}p{(\linewidth - 2\tabcolsep) * \real{0.4800}}
  >{\raggedright\arraybackslash}p{(\linewidth - 2\tabcolsep) * \real{0.5200}}@{}}
\toprule\noalign{}
\begin{minipage}[b]{\linewidth}\raggedright
Technology
\end{minipage} & \begin{minipage}[b]{\linewidth}\raggedright
Description
\end{minipage} \\
\midrule\noalign{}
\endhead
\bottomrule\noalign{}
\endlastfoot
\textbf{Sleep scheduling} & Nodes alternate between active and sleep
modes \\
\textbf{Data aggregation} & Combines multiple data packets into single
transmission \\
\textbf{Topology control} & Optimizes network structure to reduce
energy \\
\textbf{Energy harvesting} & Uses renewable sources like solar,
vibration \\
\end{longtable}
}

\textbf{Detailed Explanation:}

\textbf{1. Sleep Scheduling:}

\begin{itemize}
\tightlist
\item
  \textbf{Active mode}: Node performs sensing, processing, communication
\item
  \textbf{Sleep mode}: Node powers down non-essential components
\item
  \textbf{Benefits}: Reduces idle listening energy consumption by 90\%
\end{itemize}

\textbf{2. Data Aggregation:}

\begin{itemize}
\tightlist
\item
  \textbf{Process}: Multiple sensor readings combined at intermediate
  nodes
\item
  \textbf{Techniques}: Average, maximum, minimum functions applied
\item
  \textbf{Advantage}: Reduces total number of transmissions
  significantly
\end{itemize}

\end{solutionbox}
\begin{mnemonicbox}
``Sleep Aggregate Topology Harvest''

\end{mnemonicbox}
\subsection*{Question 1(c) OR [7
marks]}\label{q1c}

\textbf{List out any four challenges of wireless sensor network and
explain any two in detail.}

\begin{solutionbox}

\textbf{Four WSN Challenges:}

{\def\LTcaptype{none} % do not increment counter
\begin{longtable}[]{@{}ll@{}}
\toprule\noalign{}
Challenge & Impact \\
\midrule\noalign{}
\endhead
\bottomrule\noalign{}
\endlastfoot
\textbf{Limited energy} & Affects network lifetime \\
\textbf{Limited bandwidth} & Constrains data transmission \\
\textbf{Security vulnerabilities} & Threatens data integrity \\
\textbf{Scalability issues} & Affects large network performance \\
\end{longtable}
}

\textbf{Detailed Explanation:}

\textbf{1. Limited Energy:}

\begin{itemize}
\tightlist
\item
  \textbf{Battery constraint}: Nodes operate on small batteries with
  limited capacity
\item
  \textbf{Energy depletion}: High energy consumption during transmission
  and reception
\item
  \textbf{Solution approaches}: Power management protocols,
  energy-efficient routing
\end{itemize}

\textbf{2. Security Vulnerabilities:}

\begin{itemize}
\tightlist
\item
  \textbf{Physical attacks}: Nodes can be physically captured or damaged
\item
  \textbf{Network attacks}: Eavesdropping, jamming, denial of service
  attacks
\item
  \textbf{Countermeasures}: Encryption, authentication, secure routing
  protocols
\end{itemize}

\end{solutionbox}
\begin{mnemonicbox}
``Energy Bandwidth Security Scale''

\end{mnemonicbox}
\subsection*{Question 2(a) [3 marks]}\label{q2a}

\textbf{``IEEE 802.15.4 standard and the Zigbee specifications are
popular protocol choices for Wireless Sensor Network'' - Justify}

\begin{solutionbox}

\textbf{Justification Table:}

{\def\LTcaptype{none} % do not increment counter
\begin{longtable}[]{@{}ll@{}}
\toprule\noalign{}
Feature & Benefit for WSN \\
\midrule\noalign{}
\endhead
\bottomrule\noalign{}
\endlastfoot
\textbf{Low power consumption} & Extends battery life \\
\textbf{Low data rate} & Suitable for sensor data \\
\textbf{Short range} & Perfect for clustered sensors \\
\textbf{Low cost} & Economical for large deployments \\
\end{longtable}
}

\begin{itemize}
\tightlist
\item
  \textbf{IEEE 802.15.4}: Provides PHY and MAC layer specifications
\item
  \textbf{ZigBee}: Adds network and application layers on top
\item
  \textbf{Perfect match}: WSN requirements align with protocol
  capabilities
\end{itemize}

\end{solutionbox}
\begin{mnemonicbox}
``Low Power, Low Data, Low Cost, Low Range''

\end{mnemonicbox}
\subsection*{Question 2(b) [4 marks]}\label{q2b}

\textbf{Explain Energy Efficient routing with the help of suitable
example}

\begin{solutionbox}

\includegraphics[width=1\linewidth,height=\textheight,keepaspectratio]{mermaid-4e440745.pdf}

\textbf{Energy Efficient Routing:}

\begin{itemize}
\tightlist
\item
  \textbf{Objective}: Select paths that maximize network lifetime
\item
  \textbf{Approach}: Consider remaining battery levels of nodes
\item
  \textbf{Example}: Route through Node 1 (80\% battery) instead of Node
  2 (30\% battery)
\end{itemize}

\textbf{Key Techniques:}

\begin{itemize}
\tightlist
\item
  \textbf{Battery awareness}: Monitor remaining energy levels
\item
  \textbf{Load balancing}: Distribute traffic among multiple paths
\item
  \textbf{Clustering}: Group nearby nodes to reduce long-distance
  transmissions
\end{itemize}

\end{solutionbox}
\begin{mnemonicbox}
``Battery Balance Cluster''

\end{mnemonicbox}
\subsection*{Question 2(c) [7 marks]}\label{q2c}

\textbf{Explain setup and steady state phase of LEACH protocol with the
help of suitable sketch.}

\begin{solutionbox}

\includegraphics[width=1\linewidth,height=\textheight,keepaspectratio]{mermaid-01111285.pdf}

\textbf{LEACH Protocol Phases:}

\textbf{Setup Phase:}

\begin{itemize}
\tightlist
\item
  \textbf{Cluster head selection}: Random selection based on probability
  threshold
\item
  \textbf{Advertisement}: Selected CHs broadcast announcement messages
\item
  \textbf{Cluster formation}: Non-CH nodes join nearest cluster head
\item
  \textbf{Schedule creation}: CH creates TDMA schedule for cluster
  members
\end{itemize}

\textbf{Steady State Phase:}

\begin{itemize}
\tightlist
\item
  \textbf{Data transmission}: Nodes send data to CH according to TDMA
  schedule
\item
  \textbf{Data aggregation}: CH combines received data from cluster
  members
\item
  \textbf{Data forwarding}: CH transmits aggregated data to base station
\end{itemize}

\textbf{Advantages:}

\begin{itemize}
\tightlist
\item
  \textbf{Energy distribution}: Rotates CH role among nodes
\item
  \textbf{Collision avoidance}: TDMA scheduling prevents interference
\end{itemize}

\end{solutionbox}
\begin{mnemonicbox}
``Select Advertise Join Schedule, Send Aggregate
Forward''

\end{mnemonicbox}
\subsection*{Question 2(a) OR [3
marks]}\label{q2a}

\textbf{Give Classification of routing protocols in Wireless Sensor
Network.}

\begin{solutionbox}

\textbf{WSN Routing Protocol Classification:}

{\def\LTcaptype{none} % do not increment counter
\begin{longtable}[]{@{}ll@{}}
\toprule\noalign{}
Classification Basis & Types \\
\midrule\noalign{}
\endhead
\bottomrule\noalign{}
\endlastfoot
\textbf{Network Structure} & Flat, Hierarchical, Location-based \\
\textbf{Protocol Operation} & Multipath, Query-based,
Negotiation-based \\
\textbf{Path Establishment} & Proactive, Reactive, Hybrid \\
\end{longtable}
}

\textbf{Main Categories:}

\begin{itemize}
\tightlist
\item
  \textbf{Flat routing}: All nodes have equal roles (e.g., Flooding,
  SPIN)
\item
  \textbf{Hierarchical routing}: Cluster-based approach (e.g., LEACH,
  TEEN)
\item
  \textbf{Location-based routing}: Uses geographic information (e.g.,
  GEAR)
\end{itemize}

\end{solutionbox}
\begin{mnemonicbox}
``Flat Hierarchical Location''

\end{mnemonicbox}
\subsection*{Question 2(b) OR [4
marks]}\label{q2b}

\textbf{Explain the wakeup concept of low duty cycle protocol with the
help of sketch.}

\begin{solutionbox}

\begin{lstlisting}
Time -->
Node A: [Sleep]---[Wake]--[Listen]--[Sleep]---[Wake]--[Listen]--[Sleep]
Node B: [Sleep]-----[Wake]--[Tx]--[Sleep]-----[Wake]--[Listen]--[Sleep]
         |     |     |     |   |     |       |     |     |       |
         0    T1    T2    T3  T4    T5      T6    T7    T8      T9
\end{lstlisting}

\textbf{Low Duty Cycle Wakeup Concept:}

\begin{itemize}
\tightlist
\item
  \textbf{Sleep period}: Nodes turn off radio to save energy
\item
  \textbf{Wake period}: Nodes periodically wake up to check for
  communication
\item
  \textbf{Synchronization}: Sender must know receiver's wakeup schedule
\end{itemize}

\textbf{Key Benefits:}

\begin{itemize}
\tightlist
\item
  \textbf{Energy savings}: Reduces idle listening by up to 99\%
\item
  \textbf{Coordinated access}: Prevents collisions during wakeup periods
\end{itemize}

\end{solutionbox}
\begin{mnemonicbox}
``Sleep Wake Listen Repeat''

\end{mnemonicbox}
\subsection*{Question 2(c) OR [7
marks]}\label{q2c}

\textbf{Explain Synch, RTS \& CTS Phases of S-MAC Protocol and message
passing approach of it.}

\begin{solutionbox}

\includegraphics[width=1\linewidth,height=\textheight,keepaspectratio]{mermaid-e884686b.pdf}

\textbf{S-MAC Protocol Phases:}

\textbf{1. Synchronization Phase:}

\begin{itemize}
\tightlist
\item
  \textbf{Purpose}: Establish common sleep/wake schedule
\item
  \textbf{Process}: Nodes exchange SYNC packets containing schedule
  information
\item
  \textbf{Benefit}: Ensures coordinated sleep patterns across network
\end{itemize}

\textbf{2. RTS Phase (Request to Send):}

\begin{itemize}
\tightlist
\item
  \textbf{Initiation}: Sender transmits RTS packet to intended receiver
\item
  \textbf{Content}: Source address, destination address, transmission
  duration
\end{itemize}

\textbf{3. CTS Phase (Clear to Send):}

\begin{itemize}
\tightlist
\item
  \textbf{Response}: Receiver sends CTS packet confirming availability
\item
  \textbf{Virtual sensing}: Neighboring nodes overhear CTS and defer
  transmission
\end{itemize}

\textbf{Message Passing Approach:}

\begin{itemize}
\tightlist
\item
  \textbf{Collision avoidance}: RTS/CTS handshake prevents hidden
  terminal problem
\item
  \textbf{Energy conservation}: Overhearing nodes enter sleep mode
  during data exchange
\item
  \textbf{Periodic synchronization}: Maintains network-wide schedule
  coordination
\end{itemize}

\end{solutionbox}
\begin{mnemonicbox}
``Sync Request Clear Transmit''

\end{mnemonicbox}
\subsection*{Question 3(a) [3 marks]}\label{q3a}

\textbf{Explain Super Frame structure of IEEE 802.15.4 standard.}

\begin{solutionbox}

\begin{lstlisting}
|<-------------- Super Frame (15.36 ms) -------------->|
|<---CAP--->|<----------CFP---------->|<--Inactive-->|
| Beacon |Slot|Slot|Slot|GTS|GTS|GTS|    Period    |
|   8    | 0 | 1 | 2 | 1 | 2 | 3 |              |
\end{lstlisting}

\textbf{Super Frame Components:}

{\def\LTcaptype{none} % do not increment counter
\begin{longtable}[]{@{}lll@{}}
\toprule\noalign{}
Component & Description & Duration \\
\midrule\noalign{}
\endhead
\bottomrule\noalign{}
\endlastfoot
\textbf{Beacon} & Network synchronization & Fixed \\
\textbf{CAP} & Contention Access Period & Variable \\
\textbf{CFP} & Contention Free Period & Variable \\
\textbf{Inactive} & Sleep period & Variable \\
\end{longtable}
}

\begin{itemize}
\tightlist
\item
  \textbf{CAP}: Uses CSMA/CA for channel access
\item
  \textbf{CFP}: Uses GTS (Guaranteed Time Slots) for real-time data
\item
  \textbf{Inactive period}: Devices can enter low-power mode
\end{itemize}

\end{solutionbox}
\begin{mnemonicbox}
``Beacon Contend Guarantee Sleep''

\end{mnemonicbox}
\subsection*{Question 3(b) [4 marks]}\label{q3b}

\textbf{Compare M2M and IoT Technology.}

\begin{solutionbox}

{\def\LTcaptype{none} % do not increment counter
\begin{longtable}[]{@{}lll@{}}
\toprule\noalign{}
Parameter & M2M & IoT \\
\midrule\noalign{}
\endhead
\bottomrule\noalign{}
\endlastfoot
\textbf{Communication} & Point-to-point & Internet-based \\
\textbf{Data processing} & Local & Cloud-based \\
\textbf{Connectivity} & Cellular/Wired & Multiple protocols \\
\textbf{Applications} & Specific industries & Consumer \& industrial \\
\end{longtable}
}

\textbf{Key Differences:}

\begin{itemize}
\tightlist
\item
  \textbf{M2M}: Machine-to-Machine direct communication
\item
  \textbf{IoT}: Internet of Things with cloud integration
\item
  \textbf{Scope}: M2M is subset of broader IoT ecosystem
\item
  \textbf{Intelligence}: IoT provides more advanced analytics and AI
\end{itemize}

\end{solutionbox}
\begin{mnemonicbox}
``M2M Direct, IoT Internet''

\end{mnemonicbox}
\subsection*{Question 3(c) [7 marks]}\label{q3c}

\textbf{Draw Block Diagram of IoT Architecture and explain it}

\begin{solutionbox}

\includegraphics[width=1\linewidth,height=\textheight,keepaspectratio]{mermaid-9651aecd.pdf}

\textbf{IoT Architecture Layers:}

\textbf{1. Physical Layer:}

\begin{itemize}
\tightlist
\item
  \textbf{Components}: Sensors (temperature, humidity), actuators
  (motors, valves)
\item
  \textbf{Function}: Data collection from physical environment
\end{itemize}

\textbf{2. Connectivity Layer:}

\begin{itemize}
\tightlist
\item
  \textbf{Protocols}: WiFi, Bluetooth, Zigbee, LoRaWAN, cellular
\item
  \textbf{Function}: Transmit data from devices to processing centers
\end{itemize}

\textbf{3. Data Processing Layer:}

\begin{itemize}
\tightlist
\item
  \textbf{Technologies}: Edge computing, fog computing
\item
  \textbf{Function}: Real-time processing and filtering of sensor data
\end{itemize}

\textbf{4. Data Accumulation Layer:}

\begin{itemize}
\tightlist
\item
  \textbf{Infrastructure}: Cloud storage, data warehouses
\item
  \textbf{Function}: Store massive amounts of IoT data
\end{itemize}

\textbf{5. Data Abstraction Layer:}

\begin{itemize}
\tightlist
\item
  \textbf{Components}: Databases, data analytics engines
\item
  \textbf{Function}: Organize and prepare data for applications
\end{itemize}

\textbf{6. Application Layer:}

\begin{itemize}
\tightlist
\item
  \textbf{Services}: Web applications, mobile apps, dashboards
\item
  \textbf{Function}: Provide user interfaces and business logic
\end{itemize}

\textbf{7. Collaboration Layer:}

\begin{itemize}
\tightlist
\item
  \textbf{Integration}: ERP systems, business processes
\item
  \textbf{Function}: Enable collaboration between different stakeholders
\end{itemize}

\end{solutionbox}
\begin{mnemonicbox}
``Physical Connect Process Accumulate Abstract Apply
Collaborate''

\end{mnemonicbox}
\subsection*{Question 3(a) OR [3
marks]}\label{q3a}

\textbf{Explain Energy problems of MAC Protocol}

\begin{solutionbox}

\textbf{Energy Problems in MAC Protocols:}

{\def\LTcaptype{none} % do not increment counter
\begin{longtable}[]{@{}
  >{\raggedright\arraybackslash}p{(\linewidth - 4\tabcolsep) * \real{0.2903}}
  >{\raggedright\arraybackslash}p{(\linewidth - 4\tabcolsep) * \real{0.4194}}
  >{\raggedright\arraybackslash}p{(\linewidth - 4\tabcolsep) * \real{0.2903}}@{}}
\toprule\noalign{}
\begin{minipage}[b]{\linewidth}\raggedright
Problem
\end{minipage} & \begin{minipage}[b]{\linewidth}\raggedright
Description
\end{minipage} & \begin{minipage}[b]{\linewidth}\raggedright
Impact
\end{minipage} \\
\midrule\noalign{}
\endhead
\bottomrule\noalign{}
\endlastfoot
\textbf{Idle listening} & Radio stays on without communication & 50-60\%
energy waste \\
\textbf{Collision} & Multiple transmissions interfere & Retransmission
overhead \\
\textbf{Overhearing} & Receiving irrelevant packets & Unnecessary energy
consumption \\
\end{longtable}
}

\textbf{Main Issues:}

\begin{itemize}
\tightlist
\item
  \textbf{Idle listening}: Most energy-consuming activity in WSN
\item
  \textbf{Protocol overhead}: Control packets consume additional energy
\item
  \textbf{Poor scheduling}: Inefficient channel access increases energy
  usage
\end{itemize}

\end{solutionbox}
\begin{mnemonicbox}
``Idle Collide Overhear''

\end{mnemonicbox}
\subsection*{Question 3(b) OR [4
marks]}\label{q3b}

\textbf{Explain modified OSI model for IoT system}

\begin{solutionbox}

\textbf{Modified OSI Model for IoT:}

{\def\LTcaptype{none} % do not increment counter
\begin{longtable}[]{@{}
  >{\raggedright\arraybackslash}p{(\linewidth - 4\tabcolsep) * \real{0.1707}}
  >{\raggedright\arraybackslash}p{(\linewidth - 4\tabcolsep) * \real{0.3902}}
  >{\raggedright\arraybackslash}p{(\linewidth - 4\tabcolsep) * \real{0.4390}}@{}}
\toprule\noalign{}
\begin{minipage}[b]{\linewidth}\raggedright
Layer
\end{minipage} & \begin{minipage}[b]{\linewidth}\raggedright
Traditional OSI
\end{minipage} & \begin{minipage}[b]{\linewidth}\raggedright
IoT Modification
\end{minipage} \\
\midrule\noalign{}
\endhead
\bottomrule\noalign{}
\endlastfoot
\textbf{Application} & User applications & IoT applications, cloud
services \\
\textbf{Presentation} & Data formatting & JSON, XML, CoAP \\
\textbf{Session} & Session management & MQTT, HTTP sessions \\
\textbf{Transport} & TCP, UDP & UDP, CoAP, MQTT \\
\textbf{Network} & IP routing & 6LoWPAN, IPv6 \\
\textbf{Data Link} & Ethernet, WiFi & IEEE 802.15.4, LoRa \\
\textbf{Physical} & Physical medium & Sensors, actuators, radio \\
\end{longtable}
}

\textbf{Key Modifications:}

\begin{itemize}
\tightlist
\item
  \textbf{Lightweight protocols}: Optimized for resource-constrained
  devices
\item
  \textbf{Energy efficiency}: Protocols designed for low power
  consumption
\item
  \textbf{Interoperability}: Support for diverse IoT devices and
  platforms
\end{itemize}

\end{solutionbox}
\begin{mnemonicbox}
``Apps Present Session Transport Network Link
Physical''

\end{mnemonicbox}
\subsection*{Question 3(c) OR [7
marks]}\label{q3c}

\textbf{Explain Sources of IoT in detail}

\begin{solutionbox}

\textbf{IoT Sources Classification:}

\includegraphics[width=1\linewidth,height=\textheight,keepaspectratio]{mermaid-cbd7a85c.pdf}

\textbf{1. Technology Evolution Sources:}

\begin{itemize}
\tightlist
\item
  \textbf{Internet expansion}: Global connectivity infrastructure
  development
\item
  \textbf{Mobile revolution}: Smartphones and tablets creating connected
  ecosystem
\item
  \textbf{Cloud computing}: Scalable computing and storage resources
\item
  \textbf{Big data analytics}: Ability to process massive data volumes
\end{itemize}

\textbf{2. Business Drivers:}

\begin{itemize}
\tightlist
\item
  \textbf{Operational efficiency}: Automation and optimization of
  business processes
\item
  \textbf{Cost reduction}: Lower operational and maintenance costs
\item
  \textbf{New business models}: Data-driven services and products
\item
  \textbf{Customer satisfaction}: Enhanced user experience through smart
  services
\end{itemize}

\textbf{3. Technological Enablers:}

\begin{itemize}
\tightlist
\item
  \textbf{Sensor advancement}: Smaller, cheaper, more accurate sensors
\item
  \textbf{Communication progress}: Improved wireless protocols and
  standards
\item
  \textbf{Processing evolution}: More powerful yet energy-efficient
  processors
\item
  \textbf{Storage revolution}: Cheaper and more reliable data storage
  solutions
\end{itemize}

\textbf{4. Market Demands:}

\begin{itemize}
\tightlist
\item
  \textbf{Smart cities}: Urban planning and infrastructure management
\item
  \textbf{Healthcare}: Remote monitoring and telemedicine
\item
  \textbf{Industrial automation}: Industry 4.0 and smart manufacturing
\item
  \textbf{Environmental monitoring}: Climate change and sustainability
  concerns
\end{itemize}

\textbf{Key Convergence Factors:}

\begin{itemize}
\tightlist
\item
  \textbf{IPv6 adoption}: Unlimited addressing for billions of devices
\item
  \textbf{5G networks}: High-speed, low-latency communication
\item
  \textbf{AI integration}: Machine learning for intelligent decision
  making
\end{itemize}

\end{solutionbox}
\begin{mnemonicbox}
``Technology Business Enable Market''

\end{mnemonicbox}
\subsection*{Question 4(a) [3 marks]}\label{q4a}

\textbf{Explain basic Components of IoT in brief.}

\begin{solutionbox}

\textbf{Basic IoT Components:}

{\def\LTcaptype{none} % do not increment counter
\begin{longtable}[]{@{}lll@{}}
\toprule\noalign{}
Component & Function & Examples \\
\midrule\noalign{}
\endhead
\bottomrule\noalign{}
\endlastfoot
\textbf{Sensors} & Data collection & Temperature, pressure, motion \\
\textbf{Connectivity} & Data transmission & WiFi, Bluetooth, cellular \\
\textbf{Data processing} & Information analysis & Edge/cloud
computing \\
\textbf{User interface} & Human interaction & Mobile apps, dashboards \\
\end{longtable}
}

\textbf{Core Functions:}

\begin{itemize}
\tightlist
\item
  \textbf{Sensing}: Collect environmental data
\item
  \textbf{Connecting}: Transmit data to processing centers
\item
  \textbf{Processing}: Analyze and extract insights
\item
  \textbf{Acting}: Control actuators based on analysis
\end{itemize}

\end{solutionbox}
\begin{mnemonicbox}
``Sense Connect Process Interface''

\end{mnemonicbox}
\subsection*{Question 4(b) [4 marks]}\label{q4b}

\textbf{Discuss Constrained Application Protocol (CoAP) in brief.}

\begin{solutionbox}

\textbf{CoAP Protocol Overview:}

\begin{lstlisting}
Client                    Server
  |                         |
  |------- GET /temp ------>|
  |                         |
  |<----- 2.05 Content -----|
  |    Payload: 25^\circC        |
  |                         |
\end{lstlisting}

\textbf{CoAP Features:}

{\def\LTcaptype{none} % do not increment counter
\begin{longtable}[]{@{}lll@{}}
\toprule\noalign{}
Feature & Description & Benefit \\
\midrule\noalign{}
\endhead
\bottomrule\noalign{}
\endlastfoot
\textbf{Lightweight} & Simple protocol design & Low resource usage \\
\textbf{UDP-based} & Uses UDP transport & Reduced overhead \\
\textbf{RESTful} & REST architecture & Easy integration \\
\textbf{Reliable} & Built-in retransmission & Ensures delivery \\
\end{longtable}
}

\textbf{Key Characteristics:}

\begin{itemize}
\tightlist
\item
  \textbf{Request/Response}: Similar to HTTP but optimized for IoT
\item
  \textbf{Confirmable messages}: Reliability through acknowledgments
\item
  \textbf{Resource discovery}: Built-in service discovery mechanism
\item
  \textbf{Block transfer}: Support for large data transfers
\end{itemize}

\end{solutionbox}
\begin{mnemonicbox}
``Light UDP REST Reliable''

\end{mnemonicbox}
\subsection*{Question 4(c) [7 marks]}\label{q4c}

\textbf{Explain Process of Sensor and controlling device (actuator)
management through cloud.}

\begin{solutionbox}

\includegraphics[width=1\linewidth,height=\textheight,keepaspectratio]{mermaid-25573933.pdf}

\textbf{Cloud-based IoT Management Process:}

\textbf{1. Data Collection Phase:}

\begin{itemize}
\tightlist
\item
  \textbf{Sensors}: Collect environmental data (temperature, humidity,
  motion)
\item
  \textbf{Local processing}: Basic filtering and formatting at edge
  devices
\item
  \textbf{Data transmission}: Send data to cloud via WiFi/cellular
  connection
\end{itemize}

\textbf{2. Cloud Processing Phase:}

\begin{itemize}
\tightlist
\item
  \textbf{Data ingestion}: Receive and store sensor data in cloud
  databases
\item
  \textbf{Real-time analytics}: Process data streams for immediate
  insights
\item
  \textbf{Machine learning}: Apply AI algorithms for pattern recognition
  and prediction
\end{itemize}

\textbf{3. Decision Making Phase:}

\begin{itemize}
\tightlist
\item
  \textbf{Rule engine}: Apply business rules to determine required
  actions
\item
  \textbf{Threshold monitoring}: Trigger alerts when values exceed
  limits
\item
  \textbf{Automated responses}: Generate control commands for actuators
\end{itemize}

\textbf{4. Control Execution Phase:}

\begin{itemize}
\tightlist
\item
  \textbf{Command dispatch}: Send control signals to appropriate
  actuators
\item
  \textbf{Device management}: Monitor actuator status and performance
\item
  \textbf{Feedback loop}: Collect confirmation of successful command
  execution
\end{itemize}

\textbf{5. User Interaction:}

\begin{itemize}
\tightlist
\item
  \textbf{Dashboard}: Real-time visualization of sensor data and system
  status
\item
  \textbf{Mobile apps}: Remote monitoring and manual control
  capabilities
\item
  \textbf{Notifications}: Alerts and warnings sent to users
\end{itemize}

\textbf{Benefits:}

\begin{itemize}
\tightlist
\item
  \textbf{Scalability}: Handle thousands of devices simultaneously
\item
  \textbf{Remote access}: Control devices from anywhere with internet
\item
  \textbf{Data analytics}: Historical analysis and predictive
  maintenance
\item
  \textbf{Integration}: Connect with other business systems and services
\end{itemize}

\end{solutionbox}
\begin{mnemonicbox}
``Collect Process Decide Control Interact''

\end{mnemonicbox}
\subsection*{Question 4(a) OR [3
marks]}\label{q4a}

\textbf{Define Internet of Things and state its Vision.}

\begin{solutionbox}

\textbf{Definition:} Internet of Things (IoT) is a network of
interconnected physical devices embedded with sensors, software, and
connectivity to collect and exchange data over the internet.

\textbf{IoT Vision:}

{\def\LTcaptype{none} % do not increment counter
\begin{longtable}[]{@{}ll@{}}
\toprule\noalign{}
Aspect & Vision \\
\midrule\noalign{}
\endhead
\bottomrule\noalign{}
\endlastfoot
\textbf{Connectivity} & Everything connected everywhere \\
\textbf{Intelligence} & Smart decision making \\
\textbf{Automation} & Minimal human intervention \\
\textbf{Integration} & Seamless system interaction \\
\end{longtable}
}

\textbf{Core Vision Elements:}

\begin{itemize}
\tightlist
\item
  \textbf{Ubiquitous computing}: Technology embedded in everyday objects
\item
  \textbf{Seamless interaction}: Natural human-device communication
\item
  \textbf{Intelligent environment}: Context-aware responsive systems
\end{itemize}

\end{solutionbox}
\begin{mnemonicbox}
``Connect Intelligence Automate Integrate''

\end{mnemonicbox}
\subsection*{Question 4(b) OR [4
marks]}\label{q4b}

\textbf{Discuss (Message Queue Telemetry Transport) MQTT protocol in
brief.}

\begin{solutionbox}

\textbf{MQTT Protocol Architecture:}

\begin{lstlisting}
Publisher                Broker               Subscriber
    |                       |                       |
    |-- Publish(topic) ---->|                       |
    |                       |<-- Subscribe(topic) --|
    |                       |                       |
    |                       |-- Forward Message  -->|
\end{lstlisting}

\textbf{MQTT Characteristics:}

{\def\LTcaptype{none} % do not increment counter
\begin{longtable}[]{@{}
  >{\raggedright\arraybackslash}p{(\linewidth - 4\tabcolsep) * \real{0.2727}}
  >{\raggedright\arraybackslash}p{(\linewidth - 4\tabcolsep) * \real{0.3939}}
  >{\raggedright\arraybackslash}p{(\linewidth - 4\tabcolsep) * \real{0.3333}}@{}}
\toprule\noalign{}
\begin{minipage}[b]{\linewidth}\raggedright
Feature
\end{minipage} & \begin{minipage}[b]{\linewidth}\raggedright
Description
\end{minipage} & \begin{minipage}[b]{\linewidth}\raggedright
Advantage
\end{minipage} \\
\midrule\noalign{}
\endhead
\bottomrule\noalign{}
\endlastfoot
\textbf{Lightweight} & Minimal protocol overhead & Suitable for IoT
devices \\
\textbf{Publish/Subscribe} & Decoupled communication & Scalable
architecture \\
\textbf{QoS levels} & Quality of service options & Reliable delivery \\
\textbf{Persistent sessions} & Session state maintained & Connection
resilience \\
\end{longtable}
}

\textbf{MQTT Components:}

\begin{itemize}
\tightlist
\item
  \textbf{Publisher}: Sends messages to broker
\item
  \textbf{Subscriber}: Receives messages from broker
\item
  \textbf{Broker}: Central message router
\item
  \textbf{Topics}: Message categorization system
\end{itemize}

\textbf{Quality of Service Levels:}

\begin{itemize}
\tightlist
\item
  \textbf{QoS 0}: At most once delivery
\item
  \textbf{QoS 1}: At least once delivery\\
\item
  \textbf{QoS 2}: Exactly once delivery
\end{itemize}

\end{solutionbox}
\begin{mnemonicbox}
``Publish Subscribe Broker Topic''

\end{mnemonicbox}
\subsection*{Question 4(c) OR [7
marks]}\label{q4c}

\textbf{Draw Architecture block diagram of Raspberry Pi and explain it.}

\begin{solutionbox}

\begin{lstlisting}
+----------------------------------------------------------+
|                    Raspberry Pi 4                       |
|  +----------+  +----------+  +----------+  +----------+ |
|  |   CPU    |  |   GPU    |  |  Memory  |  | Storage  | |
|  |Quad-core |  |VideoCore |  | 4GB RAM  |  |MicroSD   | |
|  |ARM A72   |  |    VI    |  |  LPDDR4  |  |   Card   | |
|  +----------+  +----------+  +----------+  +----------+ |
|                                                         |
|  +----------+  +----------+  +----------+  +----------+ |
|  |   GPIO   |  |   USB    |  | Network  |  |  Audio   | |
|  | 40 pins  |  | 4 ports  |  |Ethernet  |  |3.5mm jack| |
|  |          |  |  USB 3.0 |  |WiFi/BT   |  |   HDMI   | |
|  +----------+  +----------+  +----------+  +----------+ |
+----------------------------------------------------------+
\end{lstlisting}

\textbf{Raspberry Pi Architecture Components:}

\textbf{1. Processing Unit:}

\begin{itemize}
\tightlist
\item
  \textbf{CPU}: Quad-core ARM Cortex-A72 processor running at 1.5GHz
\item
  \textbf{GPU}: VideoCore VI for graphics processing and video
  acceleration
\item
  \textbf{Performance}: Capable of running full operating systems like
  Linux
\end{itemize}

\textbf{2. Memory System:}

\begin{itemize}
\tightlist
\item
  \textbf{RAM}: 4GB LPDDR4 system memory for program execution
\item
  \textbf{Storage}: MicroSD card slot for operating system and data
  storage
\item
  \textbf{Cache}: On-chip cache memory for improved performance
\end{itemize}

\textbf{3. Input/Output Interfaces:}

\begin{itemize}
\tightlist
\item
  \textbf{GPIO}: 40-pin general purpose input/output for sensor
  connectivity
\item
  \textbf{USB ports}: 4x USB 3.0 ports for peripherals and storage
  devices
\item
  \textbf{Display}: 2x micro-HDMI ports supporting 4K video output
\end{itemize}

\textbf{4. Connectivity Options:}

\begin{itemize}
\tightlist
\item
  \textbf{Ethernet}: Gigabit Ethernet port for wired network connection
\item
  \textbf{Wireless}: Dual-band WiFi 802.11ac and Bluetooth 5.0
\item
  \textbf{Camera}: Dedicated camera serial interface (CSI) port
\end{itemize}

\textbf{5. Power and Audio:}

\begin{itemize}
\tightlist
\item
  \textbf{Power}: USB-C power input with efficient power management
\item
  \textbf{Audio}: 3.5mm audio jack and HDMI audio output
\item
  \textbf{Power consumption}: Optimized for continuous operation
\end{itemize}

\textbf{IoT Applications:}

\begin{itemize}
\tightlist
\item
  \textbf{Home automation}: Control lights, fans, security systems
\item
  \textbf{Industrial monitoring}: Temperature, pressure, vibration
  sensing
\item
  \textbf{Robotics}: Motor control, sensor integration, computer vision
\item
  \textbf{Data logging}: Environmental monitoring and data collection
\end{itemize}

\textbf{Advantages for IoT:}

\begin{itemize}
\tightlist
\item
  \textbf{Cost-effective}: Low-cost computing platform
\item
  \textbf{Versatile}: Supports multiple programming languages
\item
  \textbf{Community support}: Large ecosystem of tutorials and projects
\item
  \textbf{Expandability}: Compatible with numerous sensors and modules
\end{itemize}

\end{solutionbox}
\begin{mnemonicbox}
``Process Memory Interface Connect Power''

\end{mnemonicbox}
\subsection*{Question 5(a) [3 marks]}\label{q5a}

\textbf{Draw Block Diagram of Smart Health Monitoring System with IoT.}

\begin{solutionbox}

\includegraphics[width=1\linewidth,height=\textheight,keepaspectratio]{mermaid-8642594f.pdf}

\textbf{System Components:}

\begin{itemize}
\tightlist
\item
  \textbf{Sensors}: Collect vital signs (heart rate, blood pressure,
  temperature)
\item
  \textbf{Microcontroller}: Process sensor data and manage communication
\item
  \textbf{Connectivity}: Transmit data to cloud via WiFi/cellular
  networks
\item
  \textbf{Cloud platform}: Store data and provide analytics services
\item
  \textbf{User interfaces}: Mobile apps and web dashboards for
  monitoring
\end{itemize}

\end{solutionbox}
\begin{mnemonicbox}
``Sense Process Connect Store Monitor''

\end{mnemonicbox}
\subsection*{Question 5(b) [4 marks]}\label{q5b}

\textbf{List out different types of sensors in IoT and briefly explain
working of any two.}

\begin{solutionbox}

\textbf{IoT Sensor Types:}

{\def\LTcaptype{none} % do not increment counter
\begin{longtable}[]{@{}lll@{}}
\toprule\noalign{}
Sensor Type & Measurement & Applications \\
\midrule\noalign{}
\endhead
\bottomrule\noalign{}
\endlastfoot
\textbf{Temperature} & Heat/cold levels & HVAC, weather monitoring \\
\textbf{Humidity} & Moisture content & Agriculture, storage \\
\textbf{Pressure} & Force per unit area & Weather, industrial \\
\textbf{Motion/PIR} & Movement detection & Security, automation \\
\textbf{Gas} & Chemical composition & Air quality, safety \\
\textbf{Light} & Illumination levels & Smart lighting \\
\end{longtable}
}

\textbf{Detailed Working:}

\textbf{1. Temperature Sensor (DHT22):}

\begin{itemize}
\tightlist
\item
  \textbf{Principle}: Thermistor resistance changes with temperature
\item
  \textbf{Process}: Microcontroller reads resistance value and converts
  to temperature
\item
  \textbf{Output}: Digital signal with temperature and humidity data
\item
  \textbf{Applications}: Smart thermostat, environmental monitoring
\end{itemize}

\textbf{2. PIR Motion Sensor:}

\begin{itemize}
\tightlist
\item
  \textbf{Principle}: Detects infrared radiation emitted by moving
  objects
\item
  \textbf{Components}: Pyroelectric sensor with fresnel lens
\item
  \textbf{Working}: Changes in infrared levels trigger digital output
  signal
\item
  \textbf{Applications}: Security systems, automatic lighting, occupancy
  detection
\end{itemize}

\end{solutionbox}
\begin{mnemonicbox}
``Temperature Humidity Pressure Motion Gas Light''

\end{mnemonicbox}
\subsection*{Question 5(c) [7 marks]}\label{q5c}

\textbf{Draw Block diagram of smart home automation with IoT and Explain
its working.}

\begin{solutionbox}

\includegraphics[width=1\linewidth,height=\textheight,keepaspectratio]{mermaid-831c74eb.pdf}

\textbf{Smart Home Automation Working:}

\textbf{1. Data Collection:}

\begin{itemize}
\tightlist
\item
  \textbf{Environmental sensors}: Monitor temperature, humidity, light
  levels
\item
  \textbf{Security sensors}: Detect motion, door/window status,
  smoke/gas
\item
  \textbf{User presence}: PIR sensors determine occupancy in different
  rooms
\end{itemize}

\textbf{2. Data Processing:}

\begin{itemize}
\tightlist
\item
  \textbf{Local processing}: Immediate responses for critical situations
  (fire alarm)
\item
  \textbf{Cloud processing}: Complex analytics and pattern recognition
\item
  \textbf{Machine learning}: Learn user preferences and habits over time
\end{itemize}

\textbf{3. Decision Making:}

\begin{itemize}
\tightlist
\item
  \textbf{Rule-based control}: If temperature \textgreater{} 25^\circC, turn
  on AC
\item
  \textbf{Scheduled operations}: Turn on lights at sunset, water plants
  at 6 AM
\item
  \textbf{User preferences}: Adjust lighting and temperature based on
  learned patterns
\end{itemize}

\textbf{4. Control Execution:}

\begin{itemize}
\tightlist
\item
  \textbf{Lighting control}: Automatic dimming based on ambient light
  and time
\item
  \textbf{Climate control}: Optimize heating/cooling based on occupancy
  and weather
\item
  \textbf{Security management}: Arm/disarm security system, lock/unlock
  doors
\end{itemize}

\textbf{5. User Interaction:}

\begin{itemize}
\tightlist
\item
  \textbf{Mobile app}: Remote monitoring and control from anywhere
\item
  \textbf{Voice commands}: Integration with Alexa, Google Assistant
\item
  \textbf{Manual override}: Physical switches and controls remain
  functional
\end{itemize}

\textbf{6. Communication Flow:}

\begin{itemize}
\tightlist
\item
  \textbf{Sensor data}: Collected every few seconds and transmitted to
  controller
\item
  \textbf{Cloud synchronization}: Data backup and remote access
  capabilities
\item
  \textbf{Status updates}: Real-time notifications to mobile devices
\end{itemize}

\textbf{Key Features:}

\begin{itemize}
\tightlist
\item
  \textbf{Energy efficiency}: Automatic control reduces power
  consumption by 30-40\%
\item
  \textbf{Security enhancement}: Real-time monitoring and alert systems
\item
  \textbf{Convenience}: Voice control and smartphone integration
\item
  \textbf{Cost savings}: Optimized usage of electricity and water
  resources
\end{itemize}

\textbf{System Benefits:}

\begin{itemize}
\tightlist
\item
  \textbf{Remote monitoring}: Check home status from office or vacation
\item
  \textbf{Automated responses}: Immediate action during emergencies
\item
  \textbf{Personalization}: Customized environment based on individual
  preferences
\item
  \textbf{Integration}: Works with existing home appliances and systems
\end{itemize}

\textbf{Technical Specifications:}

\begin{itemize}
\tightlist
\item
  \textbf{Protocols}: WiFi, Zigbee, Z-Wave for device communication
\item
  \textbf{Power backup}: Battery backup for critical sensors during
  power outage
\item
  \textbf{Data encryption}: Secure communication between devices and
  cloud
\item
  \textbf{Scalability}: Easy addition of new devices and sensors
\end{itemize}

\end{solutionbox}
\begin{mnemonicbox}
``Collect Process Decide Control Interact Secure''

\end{mnemonicbox}
\subsection*{Question 5(a) OR [3
marks]}\label{q5a}

\textbf{List out any three Industrial and Military IoT applications.}

\begin{solutionbox}

\textbf{Industrial IoT Applications:}

{\def\LTcaptype{none} % do not increment counter
\begin{longtable}[]{@{}
  >{\raggedright\arraybackslash}p{(\linewidth - 4\tabcolsep) * \real{0.3611}}
  >{\raggedright\arraybackslash}p{(\linewidth - 4\tabcolsep) * \real{0.3611}}
  >{\raggedright\arraybackslash}p{(\linewidth - 4\tabcolsep) * \real{0.2778}}@{}}
\toprule\noalign{}
\begin{minipage}[b]{\linewidth}\raggedright
Application
\end{minipage} & \begin{minipage}[b]{\linewidth}\raggedright
Description
\end{minipage} & \begin{minipage}[b]{\linewidth}\raggedright
Benefits
\end{minipage} \\
\midrule\noalign{}
\endhead
\bottomrule\noalign{}
\endlastfoot
\textbf{Predictive maintenance} & Monitor equipment health in real-time
& Reduce downtime, lower costs \\
\textbf{Supply chain tracking} & Track goods from factory to customer &
Improve efficiency, reduce losses \\
\textbf{Energy management} & Monitor and optimize power consumption &
Reduce energy costs by 20-30\% \\
\end{longtable}
}

\textbf{Military IoT Applications:}

{\def\LTcaptype{none} % do not increment counter
\begin{longtable}[]{@{}
  >{\raggedright\arraybackslash}p{(\linewidth - 4\tabcolsep) * \real{0.3611}}
  >{\raggedright\arraybackslash}p{(\linewidth - 4\tabcolsep) * \real{0.3611}}
  >{\raggedright\arraybackslash}p{(\linewidth - 4\tabcolsep) * \real{0.2778}}@{}}
\toprule\noalign{}
\begin{minipage}[b]{\linewidth}\raggedright
Application
\end{minipage} & \begin{minipage}[b]{\linewidth}\raggedright
Description
\end{minipage} & \begin{minipage}[b]{\linewidth}\raggedright
Benefits
\end{minipage} \\
\midrule\noalign{}
\endhead
\bottomrule\noalign{}
\endlastfoot
\textbf{Battlefield surveillance} & Real-time monitoring of combat zones
& Enhanced situational awareness \\
\textbf{Asset tracking} & Monitor military equipment and vehicles &
Prevent theft, optimize logistics \\
\textbf{Soldier health monitoring} & Track vital signs of personnel &
Improve safety, medical response \\
\end{longtable}
}

\end{solutionbox}
\begin{mnemonicbox}
``Predict Track Energy, Survey Track Monitor''

\end{mnemonicbox}
\subsection*{Question 5(b) OR [4
marks]}\label{q5b}

\textbf{List out different types of actuators in IoT and briefly explain
working of any two.}

\begin{solutionbox}

\textbf{IoT Actuator Types:}

{\def\LTcaptype{none} % do not increment counter
\begin{longtable}[]{@{}lll@{}}
\toprule\noalign{}
Actuator Type & Function & Applications \\
\midrule\noalign{}
\endhead
\bottomrule\noalign{}
\endlastfoot
\textbf{Servo motor} & Precise angular positioning & Robotics,
automation \\
\textbf{Relay} & Electrical switching & Lights, fans, appliances \\
\textbf{Solenoid valve} & Fluid flow control & Irrigation, HVAC \\
\textbf{LED} & Light emission & Indicators, displays \\
\textbf{Buzzer} & Sound generation & Alarms, notifications \\
\textbf{Stepper motor} & Precise rotational control & 3D printers,
CNC \\
\end{longtable}
}

\textbf{Detailed Working:}

\textbf{1. Servo Motor:}

\begin{itemize}
\tightlist
\item
  \textbf{Control signal}: PWM (Pulse Width Modulation) signal
  determines position
\item
  \textbf{Feedback system}: Internal potentiometer provides position
  feedback
\item
  \textbf{Working}: Control circuit compares desired vs actual position
\item
  \textbf{Applications}: Robotic arms, camera pan/tilt, automatic doors
\end{itemize}

\textbf{2. Relay Module:}

\begin{itemize}
\tightlist
\item
  \textbf{Electromagnetic principle}: Coil creates magnetic field when
  energized
\item
  \textbf{Switching action}: Magnetic field moves mechanical contacts
\item
  \textbf{Isolation}: Electrical isolation between control and load
  circuits
\item
  \textbf{Applications}: Home automation, industrial control, safety
  systems
\end{itemize}

\end{solutionbox}
\begin{mnemonicbox}
``Servo Relay Solenoid LED Buzzer Stepper''

\end{mnemonicbox}
\subsection*{Question 5(c) OR [7
marks]}\label{q5c}

\textbf{Draw Block diagram of smart parking system with IoT and Explain
its working.}

\begin{solutionbox}

\includegraphics[width=1\linewidth,height=\textheight,keepaspectratio]{mermaid-73586cd7.pdf}

\textbf{Smart Parking System Working:}

\textbf{1. Vehicle Detection:}

\begin{itemize}
\tightlist
\item
  \textbf{Sensor placement}: IR or ultrasonic sensors installed at each
  parking space
\item
  \textbf{Detection mechanism}: Sensors detect presence/absence of
  vehicles
\item
  \textbf{Status monitoring}: Continuous monitoring of space occupancy
\item
  \textbf{Data accuracy}: Multiple sensors reduce false positive
  readings
\end{itemize}

\textbf{2. Data Collection and Processing:}

\begin{itemize}
\tightlist
\item
  \textbf{Microcontroller}: NodeMCU/Arduino processes sensor data
  locally
\item
  \textbf{Status determination}: Occupied (sensor blocked) or Free
  (sensor clear)
\item
  \textbf{Time stamping}: Record entry and exit times for billing
\item
  \textbf{Data validation}: Filter out temporary obstructions (leaves,
  debris)
\end{itemize}

\textbf{3. Communication and Cloud Integration:}

\begin{itemize}
\tightlist
\item
  \textbf{WiFi transmission}: Real-time data sent to cloud server
\item
  \textbf{Database storage}: Maintain records of parking space status
\item
  \textbf{Analytics processing}: Generate usage patterns and statistics
\item
  \textbf{API integration}: Connect with mobile apps and display systems
\end{itemize}

\textbf{4. User Interface and Services:}

\begin{itemize}
\tightlist
\item
  \textbf{Mobile application}: Users can find and reserve parking spaces
\item
  \textbf{Real-time updates}: Live status of available parking spaces
\item
  \textbf{Navigation assistance}: GPS guidance to selected parking space
\item
  \textbf{Payment integration}: Online payment for parking fees
\end{itemize}

\textbf{5. Visual Indicators:}

\begin{itemize}
\tightlist
\item
  \textbf{LED indicators}: Green (free), Red (occupied) for each space
\item
  \textbf{Display boards}: Electronic signs showing total available
  spaces
\item
  \textbf{Mobile notifications}: Alerts when reserved time is expiring
\item
  \textbf{Admin dashboard}: Management interface for monitoring and
  control
\end{itemize}

\textbf{6. Advanced Features:}

\begin{itemize}
\tightlist
\item
  \textbf{Space reservation}: Book parking space in advance
\item
  \textbf{Automatic billing}: Calculate charges based on parking
  duration
\item
  \textbf{Violation detection}: Alert for unauthorized parking
\item
  \textbf{Data analytics}: Peak usage hours, revenue analysis
\end{itemize}

\textbf{System Benefits:}

\begin{itemize}
\tightlist
\item
  \textbf{Time saving}: Reduces time spent searching for parking
\item
  \textbf{Traffic reduction}: Less circling around looking for spaces
\item
  \textbf{Revenue optimization}: Dynamic pricing based on demand
\item
  \textbf{Environmental impact}: Reduced fuel consumption and emissions
\end{itemize}

\textbf{Technical Components:}

\begin{itemize}
\tightlist
\item
  \textbf{Sensors}: IR proximity sensors or ultrasonic distance sensors
\item
  \textbf{Microcontrollers}: ESP8266/ESP32 based development boards
\item
  \textbf{Communication}: WiFi, LoRaWAN, or cellular connectivity
\item
  \textbf{Power supply}: Solar panels with battery backup for remote
  locations
\end{itemize}

\textbf{Implementation Challenges:}

\begin{itemize}
\tightlist
\item
  \textbf{Weather resistance}: Sensors must work in rain, snow, extreme
  temperatures
\item
  \textbf{Power management}: Battery-powered sensors need efficient
  power usage
\item
  \textbf{Network reliability}: Backup communication methods for
  connectivity issues
\item
  \textbf{Maintenance}: Regular cleaning and calibration of sensors
\end{itemize}

\textbf{Cost-Benefit Analysis:}

\begin{itemize}
\tightlist
\item
  \textbf{Initial investment}: Sensor installation and system setup
  costs
\item
  \textbf{Operational savings}: Reduced management overhead
\item
  \textbf{Revenue increase}: Improved space utilization and dynamic
  pricing
\item
  \textbf{Payback period}: Typically 12-18 months for commercial
  installations
\end{itemize}

\textbf{Integration Possibilities:}

\begin{itemize}
\tightlist
\item
  \textbf{Smart city systems}: Connect with traffic management systems
\item
  \textbf{Building automation}: Integration with shopping mall or office
  building systems
\item
  \textbf{Public transportation}: Coordinate with bus/metro schedules
\item
  \textbf{Emergency services}: Priority access for emergency vehicles
\end{itemize}

\textbf{Future Enhancements:}

\begin{itemize}
\tightlist
\item
  \textbf{AI integration}: Predict parking demand using machine learning
\item
  \textbf{Electric vehicle charging}: Integration with EV charging
  stations
\item
  \textbf{Autonomous vehicles}: Support for self-parking cars
\item
  \textbf{Mobile payment expansion}: Integration with digital wallets
\end{itemize}

\end{solutionbox}
\begin{mnemonicbox}
``Detect Process Communicate Interface Indicate
Serve''

\end{mnemonicbox}

\end{document}
