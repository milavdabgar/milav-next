\documentclass{article}

% content/resources/templates/preamble.tex
\usepackage[margin=0.6in]{geometry}
\author{Milav Dabgar}
\usepackage{amsmath,amssymb,amsthm}
\usepackage{booktabs}
\usepackage{multirow}
\usepackage{xcolor}
\usepackage{tcolorbox}
\tcbuselibrary{breakable,skins}
\usepackage[colorlinks=true,linkcolor=blue]{hyperref}
\usepackage{titlesec}
\usepackage{enumitem}
\usepackage{tikz}
\usepackage{pgfplots}
\usepackage{circuitikz}
\usepackage[version=4]{mhchem}
\usepackage{longtable}
\usepackage{array}
\usepackage{float}
\usepackage{caption}
\usepackage{listings}

\lstset{
  basicstyle=\small\ttfamily,
  breaklines=true,
  breakatwhitespace=false,
  postbreak=\mbox{\textcolor{red}{$\hookrightarrow$}\space},
  float=false,
  numbers=left,
  numberstyle=\tiny\color{gray},
  numbersep=10pt,
  xleftmargin=2em,
  keywordstyle=\color{blue},
  commentstyle=\color{green!60!black},
  stringstyle=\color{purple},
  backgroundcolor=\color{gray!5},
  showstringspaces=false,
  tabsize=2,
  captionpos=b,
  keepspaces=true,
  columns=flexible
}

\pgfplotsset{compat=1.18}
\usetikzlibrary{shapes,arrows,positioning,calc,patterns,decorations.pathmorphing,decorations.markings,arrows.meta}

% Color scheme
\definecolor{headcolor}{RGB}{0,102,204}
\definecolor{keycolor}{RGB}{220,20,60}
\definecolor{solutioncolor}{RGB}{34,139,34}
\definecolor{mnemoniccolor}{RGB}{148,0,211}
\definecolor{codecolor}{RGB}{0,0,100}

% Spacing
\setlength{\parskip}{3pt}
\setlist[itemize]{nosep}
\setlist[enumerate]{nosep}

% Title formatting
\titleformat{\section}{\Large\bfseries\color{headcolor}}{\thesection}{1em}{}
\titleformat{\subsection}{\large\bfseries\color{headcolor}}{\thesubsection}{1em}{}

% Pandoc tightlist compatibility
\providecommand{\tightlist}{%
  \setlength{\itemsep}{0pt}\setlength{\parskip}{0pt}}

% Pandoc longtable compatibility
\newcounter{none}
\def\thenone{}


% content/resources/templates/gujarati-boxes.tex
\usepackage{fontspec}
\usepackage{polyglossia}

% Set Gujarati as main language (document is primarily in Gujarati)
% Note: gloss-gujarati.ldf doesn't exist in polyglossia, but it will use hyphenation patterns
\setdefaultlanguage{gujarati}
\setotherlanguage{english}

% Configure Gujarati font properly
% Use Language=Default to prevent polyglossia from trying to add language-specific features
% that don't exist for Gujarati, which causes "empty feature" warnings
\newfontfamily\gujaratifont[Script=Gujarati,AutoFakeBold=2.5,AutoFakeSlant=0.3]{Noto Sans Gujarati}
\setmainfont[Script=Gujarati,AutoFakeBold=2.5,AutoFakeSlant=0.3]{Noto Sans Gujarati}
% Use Noto Sans Gujarati for monospace to support Gujarati in text
\setmonofont[Scale=0.9]{Noto Sans Gujarati}

% Configure English to use the same font
\newfontfamily\englishfont[Script=Gujarati,AutoFakeBold=2.5,AutoFakeSlant=0.3]{Noto Sans Gujarati}

% Translations for polyglossia
\gappto\captionsgujarati{
  \renewcommand{\tablename}{કોષ્ટક}
  \renewcommand{\figurename}{આકૃતિ}
}

% Helper for TikZ nodes to ensure Gujarati font
\newcommand{\gu}[1]{{\gujaratifont #1}}

% Custom environments
\newtcolorbox{solutionbox}{
    breakable,
    enhanced,
    colback=solutioncolor!5!white,
    colframe=solutioncolor!75!black,
    fonttitle=\bfseries,
    title=જવાબ
}

\newtcolorbox{solutionboxnobreak}{
 colback=solutioncolor!5!white,
 colframe=solutioncolor!75!black,
 fonttitle=\bfseries,
 title=જવાબ
}

\newtcolorbox{keyformula}{
 breakable,
 enhanced,
 colback=keycolor!5!white,
 colframe=keycolor!75!black,
 fonttitle=\bfseries,
 title=રાસાયણિક સમીકરણ/સૂત્ર
}

\newtcolorbox{mnemonicbox}{
 breakable,
 enhanced,
 colback=mnemoniccolor!5!white,
 colframe=mnemoniccolor!75!black,
 fonttitle=\bfseries,
 title=મેમરી ટ્રીક
}


% Custom commands for GTU solutions
% This file defines semantic commands for consistent formatting

% Question command with automatic formatting
\newcommand{\question}[2]{%
  \section*{Question #1}%
  \textbf{#2}%
}

% OR question variant
\newcommand{\questionor}[2]{%
  \section*{Question #1 OR}%
  \textbf{#2}%
}

% Proper table environment with caption
\newenvironment{answertable}[1]{%
  \begin{table}[htbp]
  \centering
  \caption{#1}
}{%
  \end{table}
}

% Proper figure environment for diagrams
\newenvironment{answerdiagram}[1]{%
  \begin{figure}[htbp]
  \centering
  \caption{#1}
}{%
  \end{figure}
}

% Semantic markup for key terms
\newcommand{\keyword}[1]{\textbf{#1}}
\newcommand{\code}[1]{\texttt{#1}}
\newcommand{\classname}[1]{\texttt{#1}}
\newcommand{\methodname}[1]{\texttt{#1}}

% Proper quotation marks
\newcommand{\mnemonic}[1]{``#1''}


\title{વાયરલેસ સેન્સર નેટવર્ક અને IoT (4353201) - શિયાળો 2024 સોલ્યુશન}
\date{નવેમ્બર 21, 2024}

\begin{document}
\maketitle

\questionmarks{1(a)}{3}{સિંગલ હોપ અને મલ્ટિહોપ નેટવર્કની સરખામણી કરો.}

\begin{solutionbox}
\begin{tabulary}{\linewidth}{|L|L|L|}
\hline
\textbf{પેરામીટર} & \textbf{સિંગલ હોપ નેટવર્ક} & \textbf{મલ્ટિહોપ નેટવર્ક} \\
\hline
\textbf{કમ્યુનિકેશન} & સીધું બેઝ સ્ટેશન સાથે & મધ્યવર્તી નોડ્સ દ્વારા \\
\textbf{એનર્જી વપરાશ} & દૂરના નોડ્સ માટે વધુ & નોડ્સ વચ્ચે વિતરિત \\
\textbf{નેટવર્ક કવરેજ} & ટ્રાન્સમિશન રેન્જ દ્વારા મર્યાદિત & વિસ્તૃત કવરેજ વિસ્તાર \\
\textbf{જટિલતા} & સરળ રાઉટિંગ & જટિલ રાઉટિંગ પ્રોટોકોલ \\
\hline
\end{tabulary}

\begin{itemize}
    \item \textbf{સિંગલ હોપ}: બધા નોડ્સ બેઝ સ્ટેશન સાથે સીધો સંપર્ક કરે છે
    \item \textbf{મલ્ટિહોપ}: ડેટા ગંતવ્ય સુધી પહોંચવા માટે અનેક મધ્યવર્તી નોડ્સમાંથી પસાર થાય છે
\end{itemize}

\begin{mnemonicbox}Single Direct, Multi Relay\end{mnemonicbox}
\end{solutionbox}

\questionmarks{1(b)}{4}{સેન્સર નોડના મૂળભૂત ઘટકો સમજાવો.}

\begin{solutionbox}
\begin{center}
\begin{tikzpicture}[gtu block]
    \node (sn) {Sensor Node};
    
    \node (sensing) [below left=1.5cm and 0.5cm of sn] {Sensing Unit};
    \node (proc) [below right=1.5cm and 0.5cm of sn] {Processing Unit};
    \node (comm) [below left=3.5cm and 0.5cm of sn] {Communication Unit};
    \node (power) [below right=3.5cm and 0.5cm of sn] {Power Unit};
    
    \node (sensors) [below=0.5cm of sensing, font=\small] {Sensors \& ADC};
    \node (mem) [below=0.5cm of proc, font=\small] {Processor \& Memory};
    \node (trans) [below=0.5cm of comm, font=\small] {Transceiver};
    \node (batt) [below=0.5cm of power, font=\small] {Battery};

    \draw [gtu arrow] (sn) -- (sensing);
    \draw [gtu arrow] (sn) -- (proc);
    \draw [gtu arrow] (sn) -- (comm);
    \draw [gtu arrow] (sn) -- (power);
    
    \draw [gtu arrow] (sensing) -- (sensors);
    \draw [gtu arrow] (proc) -- (mem);
    \draw [gtu arrow] (comm) -- (trans);
    \draw [gtu arrow] (power) -- (batt);
\end{tikzpicture}
\end{center}

\textbf{મૂળભૂત ઘટકો:}
\begin{itemize}
    \item \textbf{સેન્સિંગ સબસિસ્ટમ}: સેન્સર્સ અને ADC નો ઉપયોગ કરીને પર્યાવરણમાંથી ડેટા એકત્રિત કરે છે
    \item \textbf{પ્રોસેસિંગ સબસિસ્ટમ}: ડેટા પ્રોસેસિંગ માટે મેમોરી સાથે માઇક્રોકંટ્રોલર/પ્રોસેસર
    \item \textbf{કમ્યુનિકેશન સબસિસ્ટમ}: વાયરલેસ ડેટા ટ્રાન્સમિશન માટે રેડિયો ટ્રાન્સીવર
    \item \textbf{પાવર સબસિસ્ટમ}: પાવર સપ્લાય માટે બેટરી અથવા એનર્જી હાર્વેસ્ટિંગ યુનિટ
\end{itemize}

\begin{mnemonicbox}Sense Process Communicate Power\end{mnemonicbox}
\end{solutionbox}

\questionmarks{1(c)}{7}{WSN માં પાવર કન્ઝમ્પશન ઘટાડવા માટે કોઈપણ ચાર ટેકનોલોજીની યાદી બનાવો અને કોઈપણ બે ટેકનોલોજીને વિગતવાર સમજાવો.}

\begin{solutionbox}
\begin{center}
\captionof{table}{ચાર પાવર રિડક્શન ટેકનોલોજીઓ:}
\begin{tabulary}{\linewidth}{|L|L|}
\hline
\textbf{ટેકનોલોજી} & \textbf{વર્ણન} \\
\hline
\textbf{સ્લીપ શેડ્યુલિંગ} & નોડ્સ સક્રિય અને સ્લીપ મોડ વચ્ચે ફેરફાર કરે છે \\
\textbf{ડેટા એગ્રિગેશન} & અનેક ડેટા પેકેટ્સને એક જ ટ્રાન્સમિશનમાં જોડે છે \\
\textbf{ટોપોલોજી કંટ્રોલ} & એનર્જી ઘટાડવા માટે નેટવર્ક સ્ટ્રક્ચર ઓપ્ટિમાઇઝ કરે છે \\
\textbf{એનર્જી હાર્વેસ્ટિંગ} & સોલાર, વાઇબ્રેશન જેવા રિન્યુએબલ સોર્સનો ઉપયોગ કરે છે \\
\hline
\end{tabulary}
\end{center}

\textbf{વિગતવાર સમજૂતી:}

\textbf{1. સ્લીપ શેડ્યુલિંગ:}
\begin{itemize}
    \item \textbf{એક્ટિવ મોડ}: નોડ સેન્સિંગ, પ્રોસેસિંગ, કમ્યુનિકેશન કરે છે
    \item \textbf{સ્લીપ મોડ}: નોડ બિનજરૂરી ઘટકોને પાવર ડાઉન કરે છે
    \item \textbf{ફાયદા}: આઇડલ લિસનિંગ એનર્જી કન્ઝમ્પશન 90\% સુધી ઘટાડે છે
\end{itemize}

\textbf{2. ડેટા એગ્રિગેશન:}
\begin{itemize}
    \item \textbf{પ્રક્રિયા}: મધ્યવર્તી નોડ્સ પર અનેક સેન્સર રીડિંગ્સ જોડવામાં આવે છે
    \item \textbf{ટેકનિક્સ}: એવરેજ, મેક્સિમમ, મિનિમમ ફંક્શન્સ લાગુ કરવામાં આવે છે
    \item \textbf{ફાયદો}: કુલ ટ્રાન્સમિશનની સંખ્યા નોંધપાત્ર રીતે ઘટાડે છે
\end{itemize}

\begin{mnemonicbox}Sleep Aggregate Topology Harvest\end{mnemonicbox}
\end{solutionbox}

\vspace{0.5em}\centerline{\textbf{OR}}\questionmarks{1(c)}{ 7 }{વાયરલેસ સેન્સર નેટવર્કના કોઈપણ ચાર પડકારોની યાદી બનાવો અને કોઈપણ બેને વિગતવાર સમજાવો.}

\begin{solutionbox}
\begin{center}
\captionof{table}{ચાર WSN પડકારો:}
\begin{tabulary}{\linewidth}{|L|L|}
\hline
\textbf{પડકાર} & \textbf{અસર} \\
\hline
\textbf{મર્યાદિત એનર્જી} & નેટવર્ક લાઇફટાઇમને અસર કરે છે \\
\textbf{મર્યાદિત બેન્ડવિડ્થ} & ડેટા ટ્રાન્સમિશનને મર્યાદિત કરે છે \\
\textbf{સિક્યુરિટી વલ્નરેબિલિટીઝ} & ડેટા ઇન્ટેગ્રિટીને જોખમમાં મૂકે છે \\
\textbf{સ્કેલેબિલિટી ઇશ્યુઝ} & મોટા નેટવર્ક પરફોર્મન્સને અસર કરે છે \\
\hline
\end{tabulary}
\end{center}

\textbf{વિગતવાર સમજૂતી:}

\textbf{1. મર્યાદિત એનર્જી:}
\begin{itemize}
    \item \textbf{બેટરી કન્સ્ટ્રેઈન્ટ}: નોડ્સ મર્યાદિત કેપેસિટી સાથે નાની બેટરીઓ પર કામ કરે છે
    \item \textbf{એનર્જી ડિપ્લીશન}: ટ્રાન્સમિશન અને રિસેપ્શન દરમિયાન ઉચ્ચ એનર્જી વપરાશ
    \item \textbf{સોલ્યુશન એપ્રોચ}: પાવર મેનેજમેન્ટ પ્રોટોકોલ્સ, એનર્જી-એફિશિયન્ટ રાઉટિંગ
\end{itemize}

\textbf{2. સિક્યુરિટી વલ્નરેબિલિટીઝ:}
\begin{itemize}
    \item \textbf{ફિઝિકલ એટેક્સ}: નોડ્સને ભૌતિક રીતે કેપ્ચર અથવા નુકસાન થઈ શકે છે
    \item \textbf{નેટવર્ક એટેક્સ}: ઇવ્સડ્રોપિંગ, જેમિંગ, ડિનાયલ ઓફ સર્વિસ એટેક્સ
    \item \textbf{કાઉન્ટરમેઝર્સ}: એન્ક્રિપ્શન, ઓથેન્ટિકેશન, સિક્યોર રાઉટિંગ પ્રોટોકોલ્સ
\end{itemize}

\begin{mnemonicbox}Energy Bandwidth Security Scale\end{mnemonicbox}
\end{solutionbox}

\questionmarks{2(a)}{3}{``IEEE 802.15.4 સ્ટાન્ડર્ડ અને ZigBee સ્પેસિફિકેશન્સ વાયરલેસ સેન્સર નેટવર્ક માટે લોકપ્રિય પ્રોટોકોલ પસંદગીઓ છે'' - જસ્ટિફાઈ}

\begin{solutionbox}
\begin{center}
\captionof{table}{જસ્ટિફિકેશન ટેબલ:}
\begin{tabulary}{\linewidth}{|L|L|}
\hline
\textbf{ફીચર} & \textbf{WSN માટે ફાયદો} \\
\hline
\textbf{લો પાવર કન્ઝમ્પશન} & બેટરી લાઇફ વધારે છે \\
\textbf{લો ડેટા રેટ} & સેન્સર ડેટા માટે યોગ્ય \\
\textbf{શોર્ટ રેન્જ} & ક્લસ્ટર્ડ સેન્સર્સ માટે પરફેક્ટ \\
\textbf{લો કોસ્ટ} & મોટા ડિપ્લોયમેન્ટ માટે આર્થિક \\
\hline
\end{tabulary}
\end{center}

\begin{itemize}
    \item \textbf{IEEE 802.15.4}: PHY અને MAC લેયર સ્પેસિફિકેશન્સ પ્રદાન કરે છે
    \item \textbf{ZigBee}: ટોચ પર નેટવર્ક અને એપ્લિકેશન લેયર્સ ઉમેરે છે
    \item \textbf{પરફેક્ટ મેચ}: WSN આવશ્યકતાઓ પ્રોટોકોલ ક્ષમતાઓ સાથે સંરેખિત થાય છે
\end{itemize}

\begin{mnemonicbox}Low Power, Low Data, Low Cost, Low Range\end{mnemonicbox}
\end{solutionbox}

\questionmarks{2(b)}{4}{યોગ્ય ઉદાહરણની મદદથી એનર્જી એફિશિયન્ટ રાઉટિંગ સમજાવો}

\begin{solutionbox}
\begin{center}
\begin{tikzpicture}[gtu state]
    \node (A) {Source Node};
    \node (B) [above right=1cm and 2cm of A, fill=green!30] {Node 1\\(Battery: 80\%)};
    \node (C) [below right=1cm and 2cm of A, fill=red!30] {Node 2\\(Battery: 30\%)};
    \node (D) [right=2cm of B] at ($(B)!0.5!(C) + (3,0)$) {Destination};

    \draw [gtu arrow] (A) -- (B);
    \draw [gtu arrow, dashed] (A) -- (C);
    \draw [gtu arrow] (B) -- (D);
    \draw [gtu arrow, dashed] (C) -- (D);
\end{tikzpicture}
\end{center}

\textbf{એનર્જી એફિશિયન્ટ રાઉટિંગ:}
\begin{itemize}
    \item \textbf{ઉદ્દેશ્ય}: નેટવર્ક લાઇફટાઇમ મહત્તમ કરતા પાથ્સ પસંદ કરો
    \item \textbf{એપ્રોચ}: નોડ્સના બાકી બેટરી લેવલ્સ ધ્યાનમાં લો
    \item \textbf{ઉદાહરણ}: નોડ 2 (30\% બેટરી) ને બદલે નોડ 1 (80\% બેટરી) દ્વારા રૂટ કરો
\end{itemize}

\textbf{મુખ્ય ટેકનિક્સ:}
\begin{itemize}
    \item \textbf{બેટરી અવેરનેસ}: બાકી એનર્જી લેવલ્સનું નિરીક્ષણ કરો
    \item \textbf{લોડ બેલેન્સિંગ}: અનેક પાથ્સ વચ્ચે ટ્રાફિક વિતરણ કરો
    \item \textbf{ક્લસ્ટરિંગ}: લાંબા-અંતરના ટ્રાન્સમિશન ઘટાડવા માટે નજીકના નોડ્સને ગ્રુપ કરો
\end{itemize}

\begin{mnemonicbox}Battery Balance Cluster\end{mnemonicbox}
\end{solutionbox}

\questionmarks{2(c)}{7}{યોગ્ય સ્કેચની મદદથી LEACH પ્રોટોકોલના સેટઅપ અને સ્ટેડી સ્ટેટ ફેઝ સમજાવો.}

\begin{solutionbox}
\begin{center}
\begin{tikzpicture}[gtu state, node distance=2cm]
    \node (n1) {Node 1};
    \node (n2) [right=of n1] {Node 2 (CH)};
    \node (n3) [right=of n2] {Node 3};
    \node (bs) [below=3cm of n2] {Base Station};

    % Setup Phase
    \node [above=0.5cm of n2, font=\bfseries] {Setup Phase};
    \draw [gtu arrow, bend left] (n2) to node[above, font=\small] {Adv} (n1);
    \draw [gtu arrow, bend right] (n2) to node[above, font=\small] {Adv} (n3);
    \draw [gtu arrow, bend left] (n1) to node[below, font=\small] {Join} (n2);
    \draw [gtu arrow, bend right] (n3) to node[below, font=\small] {Join} (n2);

    % Steady State Phase
    \node [right=4cm of n2, font=\bfseries] (steady) {Steady State Phase};
    \draw [gtu arrow] (n1) -- node[above, font=\small, sloped] {Data} (n2);
    \draw [gtu arrow] (n3) -- node[above, font=\small, sloped] {Data} (n2);
    \draw [gtu arrow, line width=1.5pt] (n2) -- node[right, font=\small] {Aggregated Data} (bs);
\end{tikzpicture}
\end{center}

\textbf{LEACH પ્રોટોકોલ ફેઝિસ:}

\textbf{સેટઅપ ફેઝ:}
\begin{itemize}
    \item \textbf{ક્લસ્ટર હેડ સિલેક્શન}: પ્રોબેબિલિટી થ્રેશોલ્ડ આધારિત રેન્ડમ સિલેક્શન
    \item \textbf{એડવર્ટાઇઝમેન્ટ}: પસંદ કરેલા CHs એનાઉન્સમેન્ટ મેસેજિસ બ્રોડકાસ્ટ કરે છે
    \item \textbf{ક્લસ્ટર ફોર્મેશન}: નોન-CH નોડ્સ નજીકના ક્લસ્ટર હેડમાં જોડાય છે
    \item \textbf{શેડ્યુલ ક્રિએશન}: CH ક્લસ્ટર મેમ્બર્સ માટે TDMA શેડ્યુલ બનાવે છે
\end{itemize}

\textbf{સ્ટેડી સ્ટેટ ફેઝ:}
\begin{itemize}
    \item \textbf{ડેટા ટ્રાન્સમિશન}: નોડ્સ TDMA શેડ્યુલ અનુસાર CH ને ડેટા મોકલે છે
    \item \textbf{ડેટા એગ્રિગેશન}: CH ક્લસ્ટર મેમ્બર્સ પાસેથી પ્રાપ્ત ડેટાને જોડે છે
    \item \textbf{ડેટા ફોરવર્ડિંગ}: CH એગ્રિગેટેડ ડેટાને બેઝ સ્ટેશન પર ટ્રાન્સમિટ કરે છે
\end{itemize}

\textbf{ફાયદા:}
\begin{itemize}
    \item \textbf{એનર્જી ડિસ્ટ્રિબ્યુશન}: નોડ્સ વચ્ચે CH રોલ રોટેટ કરે છે
    \item \textbf{કોલિઝન એવોઇડન્સ}: TDMA શેડ્યુલિંગ ઇન્ટરફેરન્સ અટકાવે છે
\end{itemize}

\begin{mnemonicbox}Select Advertise Join Schedule, Send Aggregate Forward\end{mnemonicbox}
\end{solutionbox}

\vspace{0.5em}\centerline{\textbf{OR}}\questionmarks{2(a)}{ 3 }{વાયરલેસ સેન્સર નેટવર્કમાં રાઉટિંગ પ્રોટોકોલ્સનું વર્ગીકરણ આપો.}

\begin{solutionbox}
\begin{center}
\captionof{table}{WSN રાઉટિંગ પ્રોટોકોલ વર્ગીકરણ:}
\begin{tabulary}{\linewidth}{|L|L|}
\hline
\textbf{વર્ગીકરણ આધાર} & \textbf{પ્રકારો} \\
\hline
\textbf{નેટવર્ક સ્ટ્રક્ચર} & ફ્લેટ, હાઇરાર્કિકલ, લોકેશન-બેઝ્ડ \\
\textbf{પ્રોટોકોલ ઓપરેશન} & મલ્ટિપાથ, ક્વેરી-બેઝ્ડ, નેગોસિએશન-બેઝ્ડ \\
\textbf{પાથ એસ્ટેબ્લિશમેન્ટ} & પ્રોએક્ટિવ, રિએક્ટિવ, હાઇબ્રિડ \\
\hline
\end{tabulary}
\end{center}

\textbf{મુખ્ય કેટેગરીઝ:}
\begin{itemize}
    \item \textbf{ફ્લેટ રાઉટિંગ}: બધા નોડ્સની સમાન ભૂમિકા (જેમ કે, ફ્લડિંગ, SPIN)
    \item \textbf{હાઇરાર્કિકલ રાઉટિંગ}: ક્લસ્ટર-બેઝ્ડ એપ્રોચ (જેમ કે, LEACH, TEEN)
    \item \textbf{લોકેશન-બેઝ્ડ રાઉટિંગ}: જિયોગ્રાફિક ઇન્ફોર્મેશનનો ઉપયોગ (જેમ કે, GEAR)
\end{itemize}

\begin{mnemonicbox}Flat Hierarchical Location\end{mnemonicbox}
\end{solutionbox}

\vspace{0.5em}\centerline{\textbf{OR}}\questionmarks{2(b)}{ 4 }{સ્કેચની મદદથી લો ડ્યુટી સાઇકલ પ્રોટોકોલના વેકઅપ કોન્સેપ્ટને સમજાવો.}

\begin{solutionbox}
\begin{center}
\begin{tikzpicture}[gtu block]
    \node (time) {Time $\rightarrow$};
    
    % Node A Timeline
    \node (a_lbl) [below=0.5cm of time] {Node A:};
    \foreach \x/\t in {0/Sleep, 2/Wake, 3/Listen, 4/Sleep, 6/Wake, 7/Listen, 8/Sleep} {
        \node [draw, minimum width=1cm, minimum height=0.6cm, right=\x cm of a_lbl] {\t};
    }

    % Node B Timeline  
    \node (b_lbl) [below=1.5cm of a_lbl] {Node B:};
    \foreach \x/\t in {0/Sleep, 3/Wake, 4/Tx, 5/Sleep, 7/Wake, 8/Listen, 9/Sleep} {
        \node [draw, minimum width=1cm, minimum height=0.6cm, right=\x cm of b_lbl] {\t};
    }
\end{tikzpicture}
\end{center}

\textbf{લો ડ્યુટી સાઇકલ વેકઅપ કોન્સેપ્ટ:}
\begin{itemize}
    \item \textbf{સ્લીપ પીરિયડ}: એનર્જી બચાવવા માટે નોડ્સ રેડિયો બંધ કરે છે
    \item \textbf{વેક પીરિયડ}: નોડ્સ સમયાંતરે કમ્યુનિકેશન ચેક કરવા માટે જાગે છે
    \item \textbf{સિંક્રોનાઇઝેશન}: સેન્ડરને રિસીવરના વેકઅપ શેડ્યુલની જાણ હોવી જરૂરી
\end{itemize}

\textbf{મુખ્ય ફાયદા:}
\begin{itemize}
    \item \textbf{એનર્જી સેવિંગ્સ}: આઇડલ લિસનિંગ 99\% સુધી ઘટાડે છે
    \item \textbf{કોઓર્ડિનેટેડ એક્સેસ}: વેકઅપ પીરિયડ દરમિયાન કોલિઝન અટકાવે છે
\end{itemize}

\begin{mnemonicbox}Sleep Wake Listen Repeat\end{mnemonicbox}
\end{solutionbox}

\vspace{0.5em}\centerline{\textbf{OR}}\questionmarks{2(c)}{ 7 }{S-MAC પ્રોટોકોલના Synch, RTS અને CTS તબક્કાઓ અને તેના મેસેજ પાસિંગ એપ્રોચ સમજાવો.}

\begin{solutionbox}
\begin{center}
\begin{tikzpicture}[gtu state, node distance=2cm]
    \node (A) {Node A};
    \node (B) [right=of A] {Node B};
    \node (C) [right=of B] {Node C};

    % Phases
    \node (sync) [below=1cm of A, font=\bfseries] {1. SYNC Phase};
    \draw [gtu arrow] (A) -- node[above] {SYNC} (B);
    \draw [gtu arrow] (A) to[bend left] node[above] {SYNC} (C);
    
    \node (rts) [below=2cm of sync, font=\bfseries] {2. RTS/CTS Phase};
    \draw [gtu arrow] (A) -- node[above] {RTS} (B);
    \draw [gtu arrow] (B) -- node[below] {CTS} (A);
    \node [right=0.5cm of rts, font=\small, text width=3cm] {Node C overhears CTS and sleeps};

    \node (data) [below=2cm of rts, font=\bfseries] {3. Data Phase};
    \draw [gtu arrow] (A) -- node[above] {DATA} (B);
    \draw [gtu arrow] (B) -- node[below] {ACK} (A);
\end{tikzpicture}
\end{center}

\textbf{S-MAC પ્રોટોકોલ ફેઝિસ:}

\textbf{1. સિંક્રોનાઇઝેશન ફેઝ:}
\begin{itemize}
    \item \textbf{હેતુ}: સામાન્ય સ્લીપ/વેક શેડ્યુલ સ્થાપિત કરવું
    \item \textbf{પ્રક્રિયા}: નોડ્સ શેડ્યુલ ઇન્ફોર્મેશન સાથે SYNC પેકેટ્સનું વિનિમય કરે છે
    \item \textbf{ફાયદો}: નેટવર્ક વ્યાપી કોઓર્ડિનેટેડ સ્લીપ પેટર્ન સુનિશ્ચિત કરે છે
\end{itemize}

\textbf{2. RTS ફેઝ (રિક્વેસ્ટ ટુ સેન્ડ):}
\begin{itemize}
    \item \textbf{શરૂઆત}: સેન્ડર ઇન્ટેન્ડેડ રિસીવર ને RTS પેકેટ ટ્રાન્સમિટ કરે છે
    \item \textbf{કન્ટેન્ટ}: સોર્સ એડ્રેસ, ડેસ્ટિનેશન એડ્રેસ, ટ્રાન્સમિશન ડ્યુરેશન
\end{itemize}

\textbf{3. CTS ફેઝ (ક્લિયર ટુ સેન્ડ):}
\begin{itemize}
    \item \textbf{રિસ્પોન્સ}: રિસીવર ઉપલબ્ધતાની પુષ્ટિ કરતું CTS પેકેટ મોકલે છે
    \item \textbf{વર્ચ્યુઅલ સેન્સિંગ}: પડોશી નોડ્સ CTS સાંભળે છે અને ટ્રાન્સમિશન મુલતવી રાખે છે
\end{itemize}

\textbf{મેસેજ પાસિંગ એપ્રોચ:}
\begin{itemize}
    \item \textbf{કોલિઝન એવોઇડન્સ}: RTS/CTS હેન્ડશેક હિડન ટર્મિનલ પ્રોબ્લેમ અટકાવે છે
    \item \textbf{એનર્જી કન્ઝર્વેશન}: ઓવરહિયરિંગ નોડ્સ ડેટા એક્સચેન્જ દરમિયાન સ્લીપ મોડમાં જાય છે
    \item \textbf{પીરિયોડિક સિંક્રોનાઇઝેશન}: નેટવર્ક-વાઇડ શેડ્યુલ કોઓર્ડિનેશન જાળવે છે
\end{itemize}

\begin{mnemonicbox}Sync Request Clear Transmit\end{mnemonicbox}
\end{solutionbox}

\questionmarks{3(a)}{3}{IEEE 802.15.4 સ્ટાન્ડર્ડનું સુપર ફ્રેમ સ્ટ્રક્ચર સમજાવો.}

\begin{solutionbox}
\begin{center}
\begin{tikzpicture}[gtu block]
    \draw (0,0) rectangle (12,1);
    \node at (6, 1.3) {Super Frame (15.36 ms)};
    
    \draw (0,0) rectangle (1,1) node[pos=0.5] {Beacon};
    \draw (1,0) rectangle (4,1) node[pos=0.5] {CAP};
    \draw (4,0) rectangle (8,1) node[pos=0.5] {CFP (GTS)};
    \draw (8,0) rectangle (12,1) node[pos=0.5] {Inactive Period};
    
    \node at (2.5, -0.5) {Contention Access};
    \node at (6, -0.5) {Contention Free};
    \node at (10, -0.5) {Sleep Mode};
\end{tikzpicture}
\end{center}

\begin{center}
\captionof{table}{સુપર ફ્રેમ ઘટકો:}
\begin{tabulary}{\linewidth}{|L|L|L|}
\hline
\textbf{ઘટક} & \textbf{વર્ણન} & \textbf{અવધિ} \\
\hline
\textbf{બીકન} & નેટવર્ક સિંક્રોનાઇઝેશન & નિશ્ચિત \\
\textbf{CAP} & કન્ટેન્શન એક્સેસ પીરિયડ & ચલ \\
\textbf{CFP} & કન્ટેન્શન ફ્રી પીરિયડ & ચલ \\
\textbf{ઇનએક્ટિવ} & સ્લીપ પીરિયડ & ચલ \\
\hline
\end{tabulary}
\end{center}

\begin{itemize}
    \item \textbf{CAP}: ચેનલ એક્સેસ માટે CSMA/CA નો ઉપયોગ કરે છે
    \item \textbf{CFP}: રિયલ-ટાઇમ ડેટા માટે GTS (ગેરેન્ટીડ ટાઇમ સ્લોટ્સ) નો ઉપયોગ કરે છે
    \item \textbf{ઇનએક્ટિવ પીરિયડ}: ડિવાઇસિસ લો-પાવર મોડમાં જઈ શકે છે
\end{itemize}

\begin{mnemonicbox}Beacon Contend Guarantee Sleep\end{mnemonicbox}
\end{solutionbox}

\questionmarks{3(b)}{4}{M2M અને IoT ટેકનોલોજીની સરખામણી કરો.}

\begin{solutionbox}
\begin{tabulary}{\linewidth}{|L|L|L|}
\hline
\textbf{પેરામીટર} & \textbf{M2M} & \textbf{IoT} \\
\hline
\textbf{કમ્યુનિકેશન} & પોઇન્ટ-ટુ-પોઇન્ટ & ઇન્ટરનેટ-બેઝ્ડ \\
\textbf{ડેટા પ્રોસેસિંગ} & લોકલ & ક્લાઉડ-બેઝ્ડ \\
\textbf{કનેક્ટિવિટી} & સેલ્યુલર/વાયર્ડ & અનેક પ્રોટોકોલ્સ \\
\textbf{એપ્લિકેશન્સ} & વિશિષ્ટ ઇન્ડસ્ટ્રીઝ & કન્ઝ્યુમર અને ઇન્ડસ્ટ્રિયલ \\
\hline
\end{tabulary}

\textbf{મુખ્ય તફાવતો:}
\begin{itemize}
    \item \textbf{M2M}: મશીન-ટુ-મશીન ડાયરેક્ટ કમ્યુનિકેશન
    \item \textbf{IoT}: ક્લાઉડ ઇન્ટિગ્રેશન સાથે ઇન્ટરનેટ ઓફ થિંગ્સ
    \item \textbf{સ્કોપ}: M2M એ વ્યાપક IoT ઇકોસિસ્ટમનો ઉપસમૂહ છે
    \item \textbf{ઇન્ટેલિજન્સ}: IoT વધુ એડવાન્સ્ડ એનાલિટિક્સ અને AI પ્રદાન કરે છે
\end{itemize}

\begin{mnemonicbox}M2M Direct, IoT Internet\end{mnemonicbox}
\end{solutionbox}

\questionmarks{3(c)}{7}{IoT આર્કિટેક્ચરનો બ્લોક ડાયાગ્રામ દોરો અને તેને સમજાવો}

\begin{solutionbox}
\begin{center}
\begin{tikzpicture}[gtu block]
    \node (phys) [draw, rectangle, minimum width=4cm, minimum height=1cm] {1. Physical Layer\\(Sensors, Actuators)};
    \node (conn) [draw, rectangle, minimum width=4cm, minimum height=1cm, above=0.5cm of phys] {2. Connectivity Layer\\(WiFi, BT, Cellular)};
    \node (proc) [draw, rectangle, minimum width=4cm, minimum height=1cm, above=0.5cm of conn] {3. Data Processing Layer\\(Edge/Fog)};
    \node (accum) [draw, rectangle, minimum width=4cm, minimum height=1cm, above=0.5cm of proc] {4. Data Accumulation Layer\\(Cloud Storage)};
    \node (abstr) [draw, rectangle, minimum width=4cm, minimum height=1cm, above=0.5cm of accum] {5. Data Abstraction Layer\\(Databases)};
    \node (app) [draw, rectangle, minimum width=4cm, minimum height=1cm, above=0.5cm of abstr] {6. Application Layer\\(Analytics, Apps)};
    \node (collab) [draw, rectangle, minimum width=4cm, minimum height=1cm, above=0.5cm of app] {7. Collaboration Layer\\(Business Processes)};

    \draw [gtu arrow] (phys) -- (conn);
    \draw [gtu arrow] (conn) -- (proc);
    \draw [gtu arrow] (proc) -- (accum);
    \draw [gtu arrow] (accum) -- (abstr);
    \draw [gtu arrow] (abstr) -- (app);
    \draw [gtu arrow] (app) -- (collab);
\end{tikzpicture}
\end{center}

\textbf{IoT આર્કિટેક્ચર લેયર્સ:}

\textbf{1. ફિઝિકલ લેયર:}
\begin{itemize}
    \item \textbf{ઘટકો}: સેન્સર્સ (તાપમાન, ભેજ), એક્ચ્યુએટર્સ (મોટર્સ, વાલ્વ્સ)
    \item \textbf{કાર્ય}: ભૌતિક પર્યાવરણમાંથી ડેટા કલેક્શન
\end{itemize}

\textbf{2. કનેક્ટિવિટી લેયર:}
\begin{itemize}
    \item \textbf{પ્રોટોકોલ્સ}: WiFi, Bluetooth, Zigbee, LoRaWAN, સેલ્યુલર
    \item \textbf{કાર્ય}: ડિવાઇસિસમાંથી પ્રોસેસિંગ સેન્ટર્સ સુધી ડેટા ટ્રાન્સમિટ કરવું
\end{itemize}

\textbf{3. ડેટા પ્રોસેસિંગ લેયર:}
\begin{itemize}
    \item \textbf{ટેકનોલોજીઝ}: એજ કમ્પ્યુટિંગ, ફોગ કમ્પ્યુટિંગ
    \item \textbf{કાર્ય}: સેન્સર ડેટાની રિયલ-ટાઇમ પ્રોસેસિંગ અને ફિલ્ટરિંગ
\end{itemize}

\textbf{4. ડેટા એક્યુમ્યુલેશન લેયર:}
\begin{itemize}
    \item \textbf{ઇન્ફ્રાસ્ટ્રક્ચર}: ક્લાઉડ સ્ટોરેજ, ડેટા વેરહાઉસિસ
    \item \textbf{કાર્ય}: IoT ડેટાના વિશાળ પ્રમાણને સ્ટોર કરવું
\end{itemize}

\textbf{5. ડેટા એબ્સ્ટ્રેક્શન લેયર:}
\begin{itemize}
    \item \textbf{ઘટકો}: ડેટાબેસિસ, ડેટા એનાલિટિક્સ એન્જિન્સ
    \item \textbf{કાર્ય}: એપ્લિકેશન્સ માટે ડેટાને ઓર્ગેનાઇઝ અને તૈયાર કરવું
\end{itemize}

\textbf{6. એપ્લિકેશન લેયર:}
\begin{itemize}
    \item \textbf{સર્વિસિસ}: વેબ એપ્લિકેશન્સ, મોબાઇલ એપ્સ, ડેશબોર્ડ્સ
    \item \textbf{કાર્ય}: યુઝર ઇન્ટરફેસિસ અને બિઝનેસ લોજિક પ્રદાન કરવું
\end{itemize}

\textbf{7. કોલાબોરેશન લેયર:}
\begin{itemize}
    \item \textbf{ઇન્ટિગ્રેશન}: ERP સિસ્ટમ્સ, બિઝનેસ પ્રોસેસિસ
    \item \textbf{કાર્ય}: વિવિધ સ્ટેકહોલ્ડર્સ વચ્ચે કોલાબોરેશન સક્ષમ કરવું
\end{itemize}

\begin{mnemonicbox}Physical Connect Process Accumulate Abstract Apply Collaborate\end{mnemonicbox}
\end{solutionbox}

\vspace{0.5em}\centerline{\textbf{OR}}\questionmarks{3(a)}{ 3 }{MAC પ્રોટોકોલની એનર્જી સમસ્યાઓ સમજાવો}

\begin{solutionbox}
\begin{center}
\captionof{table}{MAC પ્રોટોકોલ્સમાં એનર્જી સમસ્યાઓ:}
\begin{tabulary}{\linewidth}{|L|L|L|}
\hline
\textbf{સમસ્યા} & \textbf{વર્ણન} & \textbf{અસર} \\
\hline
\textbf{આઇડલ લિસનિંગ} & કમ્યુનિકેશન વિના રેડિયો ચાલુ રહે છે & 50-60\% એનર્જી વેસ્ટ \\
\textbf{કોલિઝન} & અનેક ટ્રાન્સમિશન્સ ઇન્ટરફેર કરે છે & રિટ્રાન્સમિશન ઓવરહેડ \\
\textbf{ઓવરહિયરિંગ} & અપ્રસ્તુત પેકેટ્સ પ્રાપ્ત કરવું & બિનજરૂરી એનર્જી વપરાશ \\
\hline
\end{tabulary}
\end{center}

\textbf{મુખ્ય મુદ્દાઓ:}
\begin{itemize}
    \item \textbf{આઇડલ લિસનિંગ}: WSN માં સૌથી વધુ એનર્જી-વપરાતી પ્રવૃત્તિ
    \item \textbf{પ્રોટોકોલ ઓવરહેડ}: કંટ્રોલ પેકેટ્સ વધારાની એનર્જી વાપરે છે
    \item \textbf{પૂર ગરીબ શેડ્યુલિંગ}: બિનકાર્યક્ષમ ચેનલ એક્સેસ એનર્જી વધારે છે
\end{itemize}

\begin{mnemonicbox}Idle Collide Overhear\end{mnemonicbox}
\end{solutionbox}

\vspace{0.5em}\centerline{\textbf{OR}}\questionmarks{3(b)}{ 4 }{IoT સિસ્ટમ માટે મોડિફાઇડ OSI મોડેલ સમજાવો}

\begin{solutionbox}
\begin{center}
\captionof{table}{IoT માટે મોડિફાઇડ OSI મોડેલ:}
\begin{tabulary}{\linewidth}{|L|L|L|}
\hline
\textbf{લેયર} & \textbf{પરંપરાગત OSI} & \textbf{IoT મોડિફિકેશન} \\
\hline
\textbf{એપ્લિકેશન} & યુઝર એપ્લિકેશન્સ & IoT એપ્લિકેશન્સ, ક્લાઉડ સર્વિસિસ \\
\textbf{પ્રેઝન્ટેશન} & ડેટા ફોર્મેટિંગ & JSON, XML, CoAP \\
\textbf{સેશન} & સેશન મેનેજમેન્ટ & MQTT, HTTP સેશન્સ \\
\textbf{ટ્રાન્સપોર્ટ} & TCP, UDP & UDP, CoAP, MQTT \\
\textbf{નેટવર્ક} & IP રાઉટિંગ & 6LoWPAN, IPv6 \\
\textbf{ડેટા લિંક} & Ethernet, WiFi & IEEE 802.15.4, LoRa \\
\textbf{ફિઝિકલ} & ફિઝિકલ મીડિયમ & સેન્સર્સ, એક્ચ્યુએટર્સ, રેડિયો \\
\hline
\end{tabulary}
\end{center}

\textbf{મુખ્ય મોડિફિકેશન્સ:}
\begin{itemize}
    \item \textbf{લાઇટવેઇટ પ્રોટોકોલ્સ}: રિસોર્સ-કન્સ્ટ્રેઇન્ડ ડિવાઇસિસ માટે ઓપ્ટિમાઇઝ્ડ
    \item \textbf{એનર્જી એફિશિયન્સી}: લો પાવર કન્ઝમ્પશન માટે ડિઝાઇન કરેલા પ્રોટોકોલ્સ
    \item \textbf{ઇન્ટરઓપરેબિલિટી}: વિવિધ IoT ડિવાઇસિસ અને પ્લેટફોર્મ્સ માટે સપોર્ટ
\end{itemize}

\begin{mnemonicbox}Apps Present Session Transport Network Link Physical\end{mnemonicbox}
\end{solutionbox}

\vspace{0.5em}\centerline{\textbf{OR}}\questionmarks{3(c)}{ 7 }{IoT ના સ્રોતો વિગતવાર સમજાવો}

\begin{solutionbox}
\begin{center}
\begin{tikzpicture}[gtu block]
    \node (root) [draw, circle, minimum size=2cm, font=\bfseries] {IoT Sources};
    
    \node (tech) [above left=2cm of root, align=center] {Technology Evolution\\(Internet, Mobile,\\Cloud, Big Data)};
    \node (bus) [above right=2cm of root, align=center] {Business Drivers\\(Cost Reduction,\\Efficiency, CX)};
    \node (enable) [below left=2cm of root, align=center] {Tech Enablers\\(Sensors, Wireless,\\Processing, Storage)};
    \node (market) [below right=2cm of root, align=center] {Market Demands\\(Smart Cities,\\Healthcare, Industry)};
    
    \draw [gtu arrow] (root) -- (tech);
    \draw [gtu arrow] (root) -- (bus);
    \draw [gtu arrow] (root) -- (enable);
    \draw [gtu arrow] (root) -- (market);
\end{tikzpicture}
\end{center}

\textbf{1. ટેકનોલોજી ઇવોલ્યુશન સ્રોતો:}
\begin{itemize}
    \item \textbf{ઇન્ટરનેટ વિસ્તરણ}: ગ્લોબલ કનેક્ટિવિટી ઇન્ફ્રાસ્ટ્રક્ચર ડેવલપમેન્ટ
    \item \textbf{મોબાઇલ રિવોલ્યુશન}: સ્માર્ટફોન અને ટેબ્લેટ્સ કનેક્ટેડ ઇકોસિસ્ટમ બનાવે છે
    \item \textbf{ક્લાઉડ કમ્પ્યુટિંગ}: સ્કેલેબલ કમ્પ્યુટિંગ અને સ્ટોરેજ રિસોર્સિસ
    \item \textbf{બિગ ડેટા એનાલિટિક્સ}: વિશાળ ડેટા વોલ્યુમ્સ પ્રોસેસ કરવાની ક્ષમતા
\end{itemize}

\textbf{2. બિઝનેસ ડ્રાઇવર્સ:}
\begin{itemize}
    \item \textbf{ઓપરેશનલ એફિશિયન્સી}: બિઝનેસ પ્રોસેસિસનું ઓટોમેશન અને ઓપ્ટિમાઇઝેશન
    \item \textbf{કોસ્ટ રિડક્શન}: ઓપરેશનલ અને મેઇન્ટેનન્સ કોસ્ટ ઓછી
    \item \textbf{નવા બિઝનેસ મોડલ્સ}: ડેટા-ડ્રિવન સર્વિસિસ અને પ્રોડક્ટ્સ
    \item \textbf{કસ્ટમર સેટિસફેક્શન}: સ્માર્ટ સર્વિસિસ દ્વારા યુઝર એક્સપિરિયન્સ વધારવું
\end{itemize}

\textbf{3. ટેકનોલોજિકલ એનેબલર્સ:}
\begin{itemize}
    \item \textbf{સેન્સર એડવાન્સમેન્ટ}: નાના, સસ્તા, વધુ સચોટ સેન્સર્સ
    \item \textbf{કમ્યુનિકેશન પ્રોગ્રેસ}: બહેતર વાયરલેસ પ્રોટોકોલ્સ અને સ્ટાન્ડર્ડ્સ
    \item \textbf{પ્રોસેસિંગ ઇવોલ્યુશન}: વધુ શક્તિશાળી છતાં એનર્જી-એફિશિયન્ટ પ્રોસેસર્સ
    \item \textbf{સ્ટોરેજ રિવોલ્યુશન}: સસ્તું અને વધુ વિશ્વસનીય ડેટા સ્ટોરેજ સોલ્યુશન્સ
\end{itemize}

\textbf{4. માર્કેટ ડિમાન્ડ્સ:}
\begin{itemize}
    \item \textbf{સ્માર્ટ સિટીઝ}: શહેરી આયોજન અને ઇન્ફ્રાસ્ટ્રક્ચર મેનેજમેન્ટ
    \item \textbf{હેલ્થકેર}: રિમોટ મોનિટરિંગ અને ટેલિમેડિસિન
    \item \textbf{ઇન્ડસ્ટ્રિયલ ઓટોમેશન}: ઇન્ડસ્ટ્રી 4.0 અને સ્માર્ટ મેન્યુફેક્ચરિંગ
    \item \textbf{એન્વાયરન્મેન્ટલ મોનિટરિંગ}: ક્લાઇમેટ ચેન્જ અને સસ્ટેનેબિલિટી ચિંતાઓ
\end{itemize}

\textbf{મુખ્ય કન્વર્જન્સ ફેક્ટર્સ:}
\begin{itemize}
    \item \textbf{IPv6 એડોપ્શન}: અબજો ડિવાઇસિસ માટે અનલિમિટેડ એડ્રેસિંગ
    \item \textbf{5G નેટવર્ક્સ}: હાઇ-સ્પીડ, લો-લેટન્સી કમ્યુનિકેશન
    \item \textbf{AI ઇન્ટિગ્રેશન}: ઇન્ટેલિજન્ટ ડિસિઝન મેકિંગ માટે મશીન લર્નિંગ
\end{itemize}

\begin{mnemonicbox}Technology Business Enable Market\end{mnemonicbox}
\end{solutionbox}

\questionmarks{4(a)}{3}{મૂળભૂત IoT ઘટકો સમજાવો.}

\begin{solutionbox}
\begin{center}
\captionof{table}{મૂળભૂત IoT ઘટકો:}
\begin{tabulary}{\linewidth}{|L|L|L|}
\hline
\textbf{ઘટક} & \textbf{કાર્ય} & \textbf{ઉદાહરણો} \\
\hline
\textbf{સેન્સર્સ} & ડેટા કલેક્શન & તાપમાન, દબાણ, ગતિ \\
\textbf{કનેક્ટિવિટી} & ડેટા ટ્રાન્સમિશન & WiFi, Bluetooth, સેલ્યુલર \\
\textbf{ડેટા પ્રોસેસિંગ} & ઇન્ફોર્મેશન એનાલિસિસ & એજ/ક્લાઉડ કમ્પ્યુટિંગ \\
\textbf{યુઝર ઇન્ટરફેસ} & હ્યુમન ઇન્ટરેક્શન & મોબાઇલ એપ્સ, ડેશબોર્ડ્સ \\
\hline
\end{tabulary}
\end{center}

\textbf{મુખ્ય કાર્યો:}
\begin{itemize}
    \item \textbf{સેન્સિંગ}: પર્યાવરણ ડેટા એકત્રિત કરો
    \item \textbf{કનેક્ટિંગ}: પ્રોસેસિંગ સેન્ટર્સ પર ડેટા મોકલો
    \item \textbf{પ્રોસેસિંગ}: વિશ્લેષણ કરો અને આંતરદૃષ્ટિ કાઢો
    \item \textbf{એક્ટિંગ}: વિશ્લેષણના આધારે એક્ચ્યુએટર્સને નિયંત્રિત કરો
\end{itemize}

\begin{mnemonicbox}Sense Connect Process Interface\end{mnemonicbox}
\end{solutionbox}

\questionmarks{4(b)}{4}{કન્સ્ટ્રેઇન્ડ એપ્લિકેશન પ્રોટોકોલ (CoAP) વિશે સંક્ષિપ્તમાં ચર્ચા કરો.}

\begin{solutionbox}
\begin{center}
\begin{tikzpicture}[gtu state, node distance=4cm]
    \node (client) {Client};
    \node (server) [right=of client] {Server};
    
    \draw [gtu arrow] (client) -- node[above] {GET /temp} (server);
    \draw [gtu arrow] (server) -- node[below] {2.05 Content (25$^\circ$C)} (client);
\end{tikzpicture}
\end{center}

\begin{center}
\captionof{table}{CoAP ફીચર્સ:}
\begin{tabulary}{\linewidth}{|L|L|L|}
\hline
\textbf{ફીચર} & \textbf{વર્ણન} & \textbf{ફાયદો} \\
\hline
\textbf{લાઇટવેઇટ} & સરળ પ્રોટોકોલ ડિઝાઇન & લો રિસોર્સ વપરાશ \\
\textbf{UDP-બેઝ્ડ} & UDP ટ્રાન્સપોર્ટનો ઉપયોગ કરે છે & ઘટાડેલો ઓવરહેડ \\
\textbf{RESTful} & REST આર્કિટેક્ચર & સરળ ઇન્ટિગ્રેશન \\
\textbf{રિલાયબલ} & બિલ્ટ-ઇન રિટ્રાન્સમિશન & ડિલિવરી સુનિશ્ચિત કરે છે \\
\hline
\end{tabulary}
\end{center}

\textbf{મુખ્ય લાક્ષણિકતાઓ:}
\begin{itemize}
    \item \textbf{રિક્વેસ્ટ/રિસ્પોન્સ}: HTTP જેવું જ પરંતુ IoT માટે ઓપ્ટિમાઇઝ્ડ
    \item \textbf{કન્ફર્મેબલ મેસેજિસ}: એક્નોલેજમેન્ટ્સ દ્વારા વિશ્વસનીયતા
    \item \textbf{રિસોર્સ ડિસ્કવરી}: બિલ્ટ-ઇન સર્વિસ ડિસ્કવરી મિકેનિઝમ
    \item \textbf{બ્લોક ટ્રાન્સફર}: મોટા ડેટા ટ્રાન્સફર માટે સપોર્ટ
\end{itemize}

\begin{mnemonicbox}Light UDP REST Reliable\end{mnemonicbox}
\end{solutionbox}

\questionmarks{4(c)}{7}{ક્લાઉડ દ્વારા સેન્સર અને કંટ્રોલિંગ ડિવાઇસ (એક્ચ્યુએટર) મેનેજમેન્ટની પ્રક્રિયા સમજાવો.}

\begin{solutionbox}
\begin{center}
\begin{tikzpicture}[gtu state, node distance=2.5cm]
    \node (sens) {Sensor};
    \node (gw) [right=of sens] {Gateway};
    \node (cloud) [right=of gw] {Cloud};
    \node (act) [below=of gw] {Actuator};
    \node (user) [below=of cloud] {User App};

    \draw [gtu arrow] (sens) -- node[above, font=\small] {Data} (gw);
    \draw [gtu arrow] (gw) -- node[above, font=\small] {MQTT} (cloud);
    \draw [gtu arrow] (cloud) -- node[right, font=\small] {Dash} (user);
    \draw [gtu arrow] (user) -- node[left, font=\small] {Cmd} (cloud);
    \draw [gtu arrow] (cloud) to[bend right] node[above, font=\small] {Action} (gw);
    \draw [gtu arrow] (gw) -- node[left, font=\small] {Sig} (act);
\end{tikzpicture}
\end{center}

\textbf{ક્લાઉડ-બેઝ્ડ IoT મેનેજમેન્ટ પ્રક્રિયા:}

\textbf{1. ડેટા કલેક્શન ફેઝ:}
\begin{itemize}
    \item \textbf{સેન્સર્સ}: પર્યાવરણ ડેટા એકત્રિત કરો (તાપમાન, ભેજ, ગતિ)
    \item \textbf{લોકલ પ્રોસેસિંગ}: એજ ડિવાઇસિસ પર મૂળભૂત ફિલ્ટરિંગ અને ફોર્મેટિંગ
    \item \textbf{ડેટા ટ્રાન્સમિશન}: WiFi/સેલ્યુલર કનેક્શન દ્વારા ક્લાઉડ પર ડેટા મોકલો
\end{itemize}

\textbf{2. ક્લાઉડ પ્રોસેસિંગ ફેઝ:}
\begin{itemize}
    \item \textbf{ડેટા ઇન્જેશન}: ક્લાઉડ ડેટાબેસેસમાં સેન્સર ડેટા પ્રાપ્ત કરો અને સ્ટોર કરો
    \item \textbf{રિયલ-ટાઇમ એનાલિટિક્સ}: તાત્કાલિક આંતરદૃષ્ટિ માટે ડેટા સ્ટ્રીમ્સ પ્રોસેસ કરો
    \item \textbf{મશીન લર્નિંગ}: પેટર્ન રેકગ્નિશન અને પ્રેડિક્શન માટે AI અલ્ગોરિધમ્સ લાગુ કરો
\end{itemize}

\textbf{3. ડિસિઝન મેકિંગ ફેઝ:}
\begin{itemize}
    \item \textbf{રૂલ એન્જિન}: જરૂરી ક્રિયાઓ નક્કી કરવા માટે બિઝનેસ રૂલ્સ લાગુ કરો
    \item \textbf{થ્રેશોલ્ડ મોનિટરિંગ}: જ્યારે મૂલ્યો મર્યાદા કરતા વધારે હોય ત્યારે એલર્ટ્સ ટ્રિગર કરો
    \item \textbf{ઓટોમેટેડ રિસ્પોન્સિસ}: એક્ચ્યુએટર્સ માટે કંટ્રોલ કમાન્ડ્સ જનરેટ કરો
\end{itemize}

\textbf{4. કંટ્રોલ એક્ઝીક્યુશન ફેઝ:}
\begin{itemize}
    \item \textbf{કમાન્ડ ડિસ્પેચ}: યોગ્ય એક્ચ્યુએટર્સને કંટ્રોલ સિગ્નલ્સ મોકલો
    \item \textbf{ડિવાઇસ મેનેજમેન્ટ}: એક્ચ્યુએટર સ્ટેટસ અને પરફોર્મન્સ મોનિટર કરો
    \item \textbf{ફીડબેક લૂપ}: સફળ કમાન્ડ એક્ઝીક્યુશનની પુષ્ટિ એકત્રિત કરો
\end{itemize}

\textbf{5. યુઝર ઇન્ટરેક્શન:}
\begin{itemize}
    \item \textbf{ડેશબોર્ડ}: સેન્સર ડેટા અને સિસ્ટમ સ્ટેટસનું રિયલ-ટાઇમ વિઝ્યુલાઇઝેશન
    \item \textbf{મોબાઇલ એપ્સ}: રિમોટ મોનિટરિંગ અને મેન્યુઅલ કંટ્રોલ ક્ષમતાઓ
    \item \textbf{નોટિફિકેશન્સ}: યુઝર્સને મોકલેલા એલર્ટ્સ અને ચેતવણીઓ
\end{itemize}

\textbf{ફાયદા:}
\begin{itemize}
    \item \textbf{સ્કેલેબિલિટી}: હજારો ડિવાઇસિસ એક સાથે હેન્ડલ કરો
    \item \textbf{રિમોટ એક્સેસ}: ઇન્ટરનેટ સાથે ગમે ત્યાંથી ડિવાઇસિસ કંટ્રોલ કરો
    \item \textbf{ડેટા એનાલિટિક્સ}: ઐતિહાસિક વિશ્લેષણ અને પ્રિડિક્ટિવ મેઇન્ટેનન્સ
    \item \textbf{ઇન્ટિગ્રેશન}: અન્ય બિઝનેસ સિસ્ટમ્સ અને સર્વિસિસ સાથે કનેક્ટ કરો
\end{itemize}

\begin{mnemonicbox}Collect Process Decide Control Interact\end{mnemonicbox}
\end{solutionbox}

\vspace{0.5em}\centerline{\textbf{OR}}\questionmarks{4(a)}{ 3 }{ઇન્ટરનેટ ઓફ થિંગ્સની વ્યાખ્યા આપો અને તેનો વિઝન જણાવો.}

\begin{solutionbox}
\textbf{વ્યાખ્યા:}
ઇન્ટરનેટ ઓફ થિંગ્સ (IoT) એ ઇન્ટરકનેક્ટેડ ફિઝિકલ ડિવાઇસિસનું નેટવર્ક છે જે સેન્સર્સ, સોફ્ટવેર અને કનેક્ટિવિટી સાથે એમ્બેડેડ છે જે ઇન્ટરનેટ પર ડેટા એકત્રિત અને વિનિમય કરે છે.

\begin{center}
\captionof{table}{IoT વિઝન:}
\begin{tabulary}{\linewidth}{|L|L|}
\hline
\textbf{પાસું} & \textbf{વિઝન} \\
\hline
\textbf{કનેક્ટિવિટી} & બધું દરેક જગ્યાએ જોડાયેલું \\
\textbf{ઇન્ટેલિજન્સ} & સ્માર્ટ ડિસિઝન મેકિંગ \\
\textbf{ઓટોમેશન} & ન્યૂનતમ માનવ હસ્તક્ષેપ \\
\textbf{ઇન્ટિગ્રેશન} & સીમલેસ સિસ્ટમ ઇન્ટરેક્શન \\
\hline
\end{tabulary}
\end{center}

\textbf{કોર વિઝન એલિમેન્ટ્સ:}
\begin{itemize}
    \item \textbf{સર્વવ્યાપી કમ્પ્યુટિંગ}: રોજિંદા પદાર્થોમાં એમ્બેડેડ ટેકનોલોજી
    \item \textbf{સીમલેસ ઇન્ટરેક્શન}: કુદરતી માનવ-ડિવાઇસ કમ્યુનિકેશન
    \item \textbf{ઇન્ટેલિજન્ટ એન્વાયરન્મેન્ટ}: સંદર્ભ-જાગૃત રિસ્પોન્સિવ સિસ્ટમ્સ
\end{itemize}

\begin{mnemonicbox}Connect Intelligence Automate Integrate\end{mnemonicbox}
\end{solutionbox}

\vspace{0.5em}\centerline{\textbf{OR}}\questionmarks{4(b)}{ 4 }{(Message Queue Telemetry Transport) MQTT પ્રોટોકોલ વિશે સંક્ષિપ્તમાં ચર્ચા કરો.}

\begin{solutionbox}
\begin{center}
\begin{tikzpicture}[gtu state, node distance=3cm]
    \node (pub) {Publisher};
    \node (broker) [right=of pub] {Broker};
    \node (sub) [right=of broker] {Subscriber};

    \draw [gtu arrow] (pub) -- node[above, font=\small] {Publish(TopicA)} (broker);
    \draw [gtu arrow] (sub) -- node[below, font=\small] {Subscribe(TopicA)} (broker);
    \draw [gtu arrow, dashed] (broker) -- node[above, font=\small] {Forward Msg} (sub);
\end{tikzpicture}
\end{center}

\begin{center}
\captionof{table}{MQTT લાક્ષણિકતાઓ:}
\begin{tabulary}{\linewidth}{|L|L|L|}
\hline
\textbf{ફીચર} & \textbf{વર્ણન} & \textbf{ફાયદો} \\
\hline
\textbf{લાઇટવેઇટ} & ન્યૂનતમ પ્રોટોકોલ ઓવરહેડ & IoT ડિવાઇસિસ માટે યોગ્ય \\
\textbf{પબ્લિશ/સબ્સ્ક્રાઇબ} & ડિકપલ્ડ કમ્યુનિકેશન & સ્કેલેબલ આર્કિટેક્ચર \\
\textbf{QoS લેવલ્સ} & ક્વોલિટી ઓફ સર્વિસ વિકલ્પો & વિશ્વસનીય ડિલિવરી \\
\textbf{પર્સિસ્ટન્ટ સેશન્સ} & સેશન સ્ટેટ જળવાયેલ & કનેક્શન રિઝિલિયન્સ \\
\hline
\end{tabulary}
\end{center}

\textbf{MQTT ઘટકો:}
\begin{itemize}
    \item \textbf{પબ્લિશર}: બ્રોકરને મેસેજિસ મોકલે છે
    \item \textbf{સબ્સ્ક્રાઇબર}: બ્રોકર પાસેથી મેસેજિસ મેળવે છે
    \item \textbf{બ્રોકર}: સેન્ટ્રલ મેસેજ રાઉટર
    \item \textbf{ટોપિક્સ}: મેસેજ કેટેગરાઇઝેશન સિસ્ટમ
\end{itemize}

\textbf{ક્વોલિટી ઓફ સર્વિસ (QoS) લેવલ્સ:}
\begin{itemize}
    \item \textbf{QoS 0}: વધુમાં વધુ એક વાર ડિલિવરી (At most once)
    \item \textbf{QoS 1}: ઓછામાં ઓછું એક વાર ડિલિવરી (At least once)
    \item \textbf{QoS 2}: બરાબર એક વાર ડિલિવરી (Exactly once)
\end{itemize}

\begin{mnemonicbox}Publish Subscribe Broker Topic\end{mnemonicbox}
\end{solutionbox}

\vspace{0.5em}\centerline{\textbf{OR}}\questionmarks{4(c)}{ 7 }{Raspberry Pi નો આર્કિટેક્ચર બ્લોક ડાયાગ્રામ દોરો અને સમજાવો.}

\begin{solutionbox}
\begin{center}
\begin{tikzpicture}[gtu block]
    \node (cpu) [draw, rectangle, minimum width=2cm] {CPU (ARM A72)};
    \node (gpu) [draw, rectangle, minimum width=2cm, right=0.5cm of cpu] {GPU (VideoCore)};
    \node (ram) [draw, rectangle, minimum width=2cm, right=0.5cm of gpu] {RAM (4GB)};
    \node (store) [draw, rectangle, minimum width=2cm, right=0.5cm of ram] {MicroSD};
    
    \node (gpio) [draw, rectangle, minimum width=2cm, below=1cm of cpu] {GPIO (40 pins)};
    \node (usb) [draw, rectangle, minimum width=2cm, below=1cm of gpu] {USB (3.0)};
    \node (net) [draw, rectangle, minimum width=2cm, below=1cm of ram] {Network (WiFi/Eth)};
    \node (av) [draw, rectangle, minimum width=2cm, below=1cm of store] {A/V (HDMI/Audio)};
    
    \node [draw, rectangle, inner sep=0.5cm, fit=(cpu)(av)] (board) {};
    \node [above=0.1cm of board] {\textbf{Raspberry Pi 4}};
\end{tikzpicture}
\end{center}

\textbf{Raspberry Pi આર્કિટેક્ચર ઘટકો:}

\textbf{1. પ્રોસેસિંગ યુનિટ:}
\begin{itemize}
    \item \textbf{CPU}: ક્વાડ-કોર ARM Cortex-A72 પ્રોસેસર 1.5GHz પર ચાલે છે
    \item \textbf{GPU}: ગ્રાફિક્સ પ્રોસેસિંગ અને વિડિયો પ્રવેગક માટે VideoCore VI
    \item \textbf{પરફોર્મન્સ}: Linux જેવી સંપૂર્ણ ઓપરેટિંગ સિસ્ટમ્સ ચલાવવા માટે સક્ષમ
\end{itemize}

\textbf{2. મેમોરી સિસ્ટમ:}
\begin{itemize}
    \item \textbf{RAM}: પ્રોગ્રામ એક્ઝીક્યુશન માટે 4GB LPDDR4 સિસ્ટમ મેમોરી
    \item \textbf{સ્ટોરેજ}: ઓપરેટિંગ સિસ્ટમ અને ડેટા સ્ટોરેજ માટે માઇક્રોએસડી કાર્ડ સ્લોટ
\end{itemize}

\textbf{3. ઇનપુટ/આઉટપુટ ઇન્ટરફેસિસ:}
\begin{itemize}
    \item \textbf{GPIO}: સેન્સર કનેક્ટિવિટી માટે 40-પિન જનરલ પર્પઝ ઇનપુટ/આઉટપુટ
    \item \textbf{USB પોર્ટ્સ}: પેરિફેરલ્સ અને સ્ટોરેજ ઉપકરણો માટે 4x USB 3.0 પોર્ટ્સ
    \item \textbf{ડિસ્પ્લે}: 4K વિડિયો આઉટપુટને સપોર્ટ કરતા 2x માઇક્રો-HDMI પોર્ટ્સ
\end{itemize}

\textbf{4. કનેક્ટિવિટી વિકલ્પો:}
\begin{itemize}
    \item \textbf{ઇથરનેટ}: વાયર્ડ નેટવર્ક કનેક્શન માટે ગીગાબીટ ઇથરનેટ પોર્ટ
    \item \textbf{વાયરલેસ}: ડ્યુઅલ-બેન્ડ WiFi 802.11ac અને Bluetooth 5.0
\end{itemize}

\textbf{IoT એપ્લિકેશન્સ:}
\begin{itemize}
    \item \textbf{હોમ ઓટોમેશન}, \textbf{ઇન્ડસ્ટ્રિયલ મોનિટરિંગ}, \textbf{રોબોટિક્સ}
\end{itemize}

\begin{mnemonicbox}Process Memory Interface Connect Power\end{mnemonicbox}
\end{solutionbox}

\questionmarks{5(a)}{3}{IoT સાથે સ્માર્ટ હેલ્થ મોનિટરિંગ સિસ્ટમનો બ્લોક ડાયાગ્રામ દોરો.}

\begin{solutionbox}
\begin{center}
\begin{tikzpicture}[gtu state, node distance=2cm]
    \node (patient) {Patient};
    \node (sensor) [right=of patient] {Sensors (HR, Temp)};
    \node (mcu) [right=of sensor] {MCU (ESP32)};
    \node (cloud) [below=of mcu] {Cloud Server};
    \node (app) [left=of cloud] {Mobile/Doctor App};
    \node (alert) [right=of cloud] {Alerts (SMS)};

    \draw [gtu arrow] (patient) -- (sensor);
    \draw [gtu arrow] (sensor) -- (mcu);
    \draw [gtu arrow] (mcu) -- node[right] {WiFi} (cloud);
    \draw [gtu arrow] (cloud) -- (app);
    \draw [gtu arrow] (cloud) -- (alert);
\end{tikzpicture}
\end{center}

\textbf{સિસ્ટમ ઘટકો:}
\begin{itemize}
    \item \textbf{સેન્સર્સ}: વાઇટલ સાઇન્સ એકત્રિત કરો (હાર્ટ રેટ, બ્લડ પ્રેશર)
    \item \textbf{માઇક્રોકંટ્રોલર}: સેન્સર ડેટા પ્રોસેસ અને કમ્યુનિકેશન મેનેજ કરો
    \item \textbf{ક્લાઉડ પ્લેટફોર્મ}: ડેટા સ્ટોર કરો અને એનાલિટિક્સ સેવાઓ પ્રદાન કરો
    \item \textbf{યુઝર ઇન્ટરફેસિસ}: મોનિટરિંગ માટે મોબાઇલ એપ્સ અને વેબ ડેશબોર્ડ્સ
\end{itemize}

\begin{mnemonicbox}Sense Process Connect Store Monitor\end{mnemonicbox}
\end{solutionbox}

\questionmarks{5(b)}{4}{IoT માં વિવિધ પ્રકારના સેન્સરની યાદી આપો અને કોઈપણ બેની કાર્યપદ્ધતિ સંક્ષિપ્તમાં સમજાવો.}

\begin{solutionbox}
\begin{center}
\captionof{table}{IoT સેન્સર પ્રકારો:}
\begin{tabulary}{\linewidth}{|L|L|L|}
\hline
\textbf{સેન્સર પ્રકાર} & \textbf{માપન} & \textbf{એપ્લિકેશન્સ} \\
\hline
\textbf{તાપમાન} & ગરમી/શરદી સ્તર & થર્મોસ્ટેટ, હવામાન \\
\textbf{ભેજ} & ભેજ સામગ્રી & કૃષિ, સ્ટોરેજ \\
\textbf{દબાણ} & એકમ વિસ્તાર દીઠ બળ & હવામાન, ઔદ્યોગિક \\
\textbf{મોશન/PIR} & હલનચલન શોધ & સુરક્ષા, ઓટોમેશન \\
\textbf{ગેસ} & રાસાયણિક રચના & હવાની ગુણવત્તા \\
\textbf{પ્રકાશ} & રોશની સ્તર & સ્માર્ટ લાઇટિંગ \\
\hline
\end{tabulary}
\end{center}

\textbf{વિગતવાર કાર્યપદ્ધતિ:}

\textbf{1. તાપમાન સેન્સર (DHT22):}
\begin{itemize}
    \item \textbf{સિદ્ધાંત}: તાપમાન સાથે થર્મિસ્ટર રેઝિસ્ટન્સ બદલાય છે
    \item \textbf{પ્રક્રિયા}: માઇક્રોકંટ્રોલર રેઝિસ્ટન્સ મૂલ્ય વાંચે છે અને તાપમાનમાં રૂપાંતરિત કરે છે
    \item \textbf{આઉટપુટ}: તાપમાન અને ભેજ ડેટા સાથે ડિજિટલ સિગ્નલ
\end{itemize}

\textbf{2. PIR મોશન સેન્સર:}
\begin{itemize}
    \item \textbf{સિદ્ધાંત}: ગતિશીલ પદાર્થો દ્વારા ઉત્સર્જિત ઇન્ફ્રારેડ રેડિયેશન શોધે છે
    \item \textbf{કાર્ય}: ઇન્ફ્રારેડ લેવલ્સમાં ફેરફાર ડિજિટલ આઉટપુટ સિગ્નલ ટ્રિગર કરે છે
    \item \textbf{એપ્લિકેશન્સ}: સુરક્ષા સિસ્ટમ્સ, ઓટોમેટિક લાઇટિંગ
\end{itemize}

\begin{mnemonicbox}Temperature Humidity Pressure Motion Gas Light\end{mnemonicbox}
\end{solutionbox}

\questionmarks{5(c)}{7}{IoT સાથે સ્માર્ટ હોમ ઓટોમેશનનો બ્લોક ડાયાગ્રામ દોરો અને તેની કાર્યપદ્ધતિ સમજાવો.}

\begin{solutionbox}
\begin{center}
\begin{tikzpicture}[gtu block]
    \node (ctrl) [draw, rectangle, minimum width=3cm] {Controller (RPi)};
    \node (sens) [left=of ctrl, align=center] {Sensors\\(Temp, Motion)};
    \node (act) [right=of ctrl, align=center] {Actuators\\(Lights, AC)};
    \node (comm) [below=of ctrl] {Comm (WiFi)};
    \node (cloud) [below=of comm] {Cloud Server};
    \node (user) [left=of cloud] {User App};
    \node (voice) [right=of cloud] {Voice Asst};

    \draw [gtu arrow] (sens) -- (ctrl);
    \draw [gtu arrow] (ctrl) -- (act);
    \draw [gtu arrow] (ctrl) -- (comm);
    \draw [gtu arrow] (comm) -- (cloud);
    \draw [gtu arrow] (cloud) -- (user);
    \draw [gtu arrow] (cloud) -- (voice);
\end{tikzpicture}
\end{center}

\textbf{સ્માર્ટ હોમ ઓટોમેશન કાર્યપદ્ધતિ:}
\begin{itemize}
    \item \textbf{ડેટા કલેક્શન}: સેન્સર્સ (પર્યાવરણ, સુરક્ષા) ઘરની સ્થિતિનું નિરીક્ષણ કરે છે.
    \item \textbf{ડેટા પ્રોસેસિંગ}: લોકલ (ક્રિટિકલ) અને ક્લાઉડ (એનાલિટિક્સ) પ્રોસેસિંગ.
    \item \textbf{ડિસિઝન મેકિંગ}: રૂલ્સ અને AI ક્રિયાઓ નિયંત્રિત કરે છે.
    \item \textbf{કંટ્રોલ એક્ઝીક્યુશન}: કંટ્રોલર એક્ચ્યુએટર્સને સિગ્નલ મોકલે છે (લાઇટ ડીમ, દરવાજા લોક).
    \item \textbf{યુઝર ઇન્ટરેક્શન}: એપ્સ અને વોઇસ આસિસ્ટન્ટ્સ રિમોટ કંટ્રોલ આપે છે.
\end{itemize}

\textbf{મુખ્ય વિશેષતાઓ:}
\begin{itemize}
    \item \textbf{એનર્જી એફિશિયન્સી}: ઓપ્ટિમાઇઝ્ડ વપરાશ 30-40\% પાવર બચાવે છે.
    \item \textbf{સિક્યુરિટી}: રિયલ-ટાઇમ એલર્ટ્સ અને મોનિટરિંગ.
    \item \textbf{સગવડ}: ઓટોમેટેડ રૂટિન અને વોઇસ કંટ્રોલ.
\end{itemize}

\begin{mnemonicbox}Collect Process Decide Control Interact Secure\end{mnemonicbox}
\end{solutionbox}

\vspace{0.5em}\centerline{\textbf{OR}}\questionmarks{5(a)}{ 3 }{કોઈપણ ત્રણ ઔદ્યોગિક અને લશ્કરી IoT એપ્લિકેશન્સની યાદી આપો.}

\begin{solutionbox}
\begin{center}
\captionof{table}{ઔદ્યોગિક IoT એપ્લિકેશન્સ:}
\begin{tabulary}{\linewidth}{|L|L|L|}
\hline
\textbf{એપ્લિકેશન} & \textbf{વર્ણન} & \textbf{ફાયદા} \\
\hline
\textbf{પ્રિડિક્ટિવ મેઇન્ટેનન્સ} & સાધનોની તંદુરસ્તી મોનિટર & ડાઉનટાઇમ ઘટાડો \\
\textbf{સપ્લાય ચેઇન ટ્રેકિંગ} & માલની હિલચાલ ટ્રેક & કાર્યક્ષમતા વધારો \\
\textbf{એનર્જી મેનેજમેન્ટ} & પાવર વપરાશ ઓપ્ટિમાઇઝ & ખર્ચ ઘટાડો \\
\hline
\end{tabulary}
\end{center}

\begin{center}
\captionof{table}{લશ્કરી IoT એપ્લિકેશન્સ (IoMT):}
\begin{tabulary}{\linewidth}{|L|L|L|}
\hline
\textbf{એપ્લિકેશન} & \textbf{વર્ણન} & \textbf{ફાયદા} \\
\hline
\textbf{બેટલફિલ્ડ સર્વેલન્સ} & રિયલ-ટાઇમ કોમ્બેટ ઝોન & પરિસ્થિતિ જાગૃતિ \\
\textbf{એસેટ ટ્રેકિંગ} & સાધનો/વાહનો મોનિટર & લોજિસ્ટિક્સ ઓપ્ટિમાઇઝેશન \\
\textbf{સોલ્જર હેલ્થ} & સૈનિક વાઇટલ સાઇન્સ & સલામતી અને પ્રતિભાવ \\
\hline
\end{tabulary}
\end{center}

\begin{mnemonicbox}Predict Track Energy, Survey Track Monitor\end{mnemonicbox}
\end{solutionbox}

\vspace{0.5em}\centerline{\textbf{OR}}\questionmarks{5(b)}{ 4 }{IoT માં વિવિધ પ્રકારના એક્ચ્યુએટર્સની યાદી આપો અને કોઈપણ બેની કાર્યપદ્ધતિ સંક્ષિપ્તમાં સમજાવો.}

\begin{solutionbox}
\begin{center}
\captionof{table}{IoT એક્ચ્યુએટર પ્રકારો:}
\begin{tabulary}{\linewidth}{|L|L|L|}
\hline
\textbf{એક્ચ્યુએટર પ્રકાર} & \textbf{કાર્ય} & \textbf{એપ્લિકેશન્સ} \\
\hline
\textbf{સર્વો મોટર} & કોણીય પોઝિશનિંગ & રોબોટિક્સ \\
\textbf{રિલે} & ઇલેક્ટ્રિકલ સ્વિચિંગ & લાઇટ્સ, ઉપકરણો \\
\textbf{સોલેનોઇડ વાલ્વ} & પ્રવાહી નિયંત્રણ & સિંચાઈ \\
\textbf{LED} & પ્રકાશ ઉત્સર્જન & ઇન્ડિકેટર્સ \\
\textbf{બઝર} & અવાજ જનરેશન & એલાર્મ્સ \\
\textbf{સ્ટેપર મોટર} & રોટેશનલ કંટ્રોલ & 3D પ્રિન્ટર્સ \\
\hline
\end{tabulary}
\end{center}

\textbf{વિગતવાર કાર્યપદ્ધતિ:}

\textbf{1. સર્વો મોટર:}
\begin{itemize}
    \item \textbf{કંટ્રોલ સિગ્નલ}: PWM સિગ્નલ પોઝિશન નક્કી કરે છે
    \item \textbf{ફીડબેક}: આંતરિક પોટેન્શિયોમીટર ચોકસાઈ સુનિશ્ચિત કરે છે
    \item \textbf{કાર્ય}: સર્કિટ ઇચ્છિત વિ વાસ્તવિક પોઝિશનની તુલના કરે છે
\end{itemize}

\textbf{2. રિલે મોડ્યુલ:}
\begin{itemize}
    \item \textbf{સિદ્ધાંત}: ઇલેક્ટ્રોમેગ્નેટ મિકેનિકલ સ્વિચ ખસેડે છે
    \item \textbf{સ્વિચિંગ}: હાઇ વોલ્ટેજ સર્કિટને કનેક્ટ/ડિસ્કનેક્ટ કરે છે
    \item \textbf{આઇસોલેશન}: લો વોલ્ટેજ MCU થી હાઇ લોડ્સ સુરક્ષિત રીતે નિયંત્રિત કરે છે
\end{itemize}

\begin{mnemonicbox}Servo Relay Solenoid LED Buzzer Stepper\end{mnemonicbox}
\end{solutionbox}

\vspace{0.5em}\centerline{\textbf{OR}}\questionmarks{5(c)}{ 7 }{IoT સાથે સ્માર્ટ પાર્કિંગ સિસ્ટમનો બ્લોક ડાયાગ્રામ દોરો અને તેની કાર્યપદ્ધતિ સમજાવો.}

\begin{solutionbox}
\begin{center}
\begin{tikzpicture}[gtu block]
    \node (space) [draw, rectangle, minimum width=2.5cm] {Parking Space};
    \node (sensor) [draw, rectangle, minimum width=2.5cm, right=0.5cm of space] {Sensor (IR/US)};
    \node (mcu) [draw, rectangle, minimum width=2.5cm, right=0.5cm of sensor] {MCU (NodeMCU)};
    \node (cloud) [draw, rectangle, minimum width=2.5cm, below=1cm of mcu] {Cloud Server};
    \node (app) [draw, rectangle, minimum width=2.5cm, left=0.5cm of cloud] {Mobile App};
    \node (disp) [draw, rectangle, minimum width=2.5cm, right=0.5cm of cloud] {Display Board};
    \node (led) [draw, rectangle, minimum width=2.5cm, above=0.5cm of mcu] {LED Indicator};

    \draw [gtu arrow] (space) -- (sensor);
    \draw [gtu arrow] (sensor) -- (mcu);
    \draw [gtu arrow] (mcu) -- (led);
    \draw [gtu arrow] (mcu) -- node[right] {WiFi} (cloud);
    \draw [gtu arrow] (cloud) -- (app);
    \draw [gtu arrow] (cloud) -- (disp);
\end{tikzpicture}
\end{center}

\textbf{સ્માર્ટ પાર્કિંગ સિસ્ટમ કાર્યપદ્ધતિ:}

\textbf{1. વાહન શોધ (Vehicle Detection):}
\begin{itemize}
    \item દરેક જગ્યાએ IR/Ultrasonic સેન્સર્સ વાહનની હાજરી શોધે છે.
    \item સતત દેખરેખ ચોક્કસ ઓક્યુપન્સી સ્ટેટસ સુનિશ્ચિત કરે છે.
\end{itemize}

\textbf{2. ડેટા કલેક્શન અને પ્રોસેસિંગ:}
\begin{itemize}
    \item માઇક્રોકંટ્રોલર સેન્સર ડેટા (Occupied/Free) પ્રોસેસ કરે છે.
    \item કચરામાંથી ખોટા પોઝિટિવ્સ ટાળવા માટે ડેટા માન્ય કરે છે.
\end{itemize}

\textbf{3. કમ્યુનિકેશન:}
\begin{itemize}
    \item WiFi ક્લાઉડ સર્વર પર રિયલ-ટાઇમ સ્ટેટસ ટ્રાન્સમિટ કરે છે.
    \item ક્લાઉડ ડેટાબેઝ રેકોર્ડ્સ સ્ટોર કરે છે અને એનાલિટિક્સ કરે છે.
\end{itemize}

\textbf{4. યુઝર સર્વિસિસ:}
\begin{itemize}
    \item મોબાઇલ એપ્લિકેશન જગ્યાઓ શોધવા અને આરક્ષિત કરવાની મંજૂરી આપે છે.
    \item ઉપલબ્ધ સ્થળો પર રિયલ-ટાઇમ નેવિગેશન.
\end{itemize}

\textbf{5. ઇન્ડિકેટર્સ:}
\begin{itemize}
    \item ઓન-સાઇટ LED ઇન્ડિકેટર્સ (લાલ/લીલા) અને ડિસ્પ્લે બોર્ડ્સ.
\end{itemize}

\textbf{ફાયદા:}
\begin{itemize}
    \item \textbf{સમય બચત}: ઝડપી પાર્કિંગ સ્થળ સ્થાન.
    \item \textbf{ટ્રાફિક ઘટાડો}: ઓછું ચક્કર લગાવવું.
    \item \textbf{આવક}: ઓપ્ટિમાઇઝ્ડ જગ્યા ઉપયોગ.
\end{itemize}

\begin{mnemonicbox}Detect Process Communicate Interface Indicate Serve\end{mnemonicbox}
\end{solutionbox}

\end{document}
