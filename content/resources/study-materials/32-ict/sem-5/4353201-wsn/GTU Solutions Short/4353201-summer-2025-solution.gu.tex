\documentclass{article}

% content/resources/templates/preamble.tex
\usepackage[margin=0.6in]{geometry}
\author{Milav Dabgar}
\usepackage{amsmath,amssymb,amsthm}
\usepackage{booktabs}
\usepackage{multirow}
\usepackage{xcolor}
\usepackage{tcolorbox}
\tcbuselibrary{breakable,skins}
\usepackage[colorlinks=true,linkcolor=blue]{hyperref}
\usepackage{titlesec}
\usepackage{enumitem}
\usepackage{tikz}
\usepackage{pgfplots}
\usepackage{circuitikz}
\usepackage[version=4]{mhchem}
\usepackage{longtable}
\usepackage{array}
\usepackage{float}
\usepackage{caption}
\usepackage{listings}

\lstset{
  basicstyle=\small\ttfamily,
  breaklines=true,
  breakatwhitespace=false,
  postbreak=\mbox{\textcolor{red}{$\hookrightarrow$}\space},
  float=false,
  numbers=left,
  numberstyle=\tiny\color{gray},
  numbersep=10pt,
  xleftmargin=2em,
  keywordstyle=\color{blue},
  commentstyle=\color{green!60!black},
  stringstyle=\color{purple},
  backgroundcolor=\color{gray!5},
  showstringspaces=false,
  tabsize=2,
  captionpos=b,
  keepspaces=true,
  columns=flexible
}

\pgfplotsset{compat=1.18}
\usetikzlibrary{shapes,arrows,positioning,calc,patterns,decorations.pathmorphing,decorations.markings,arrows.meta}

% Color scheme
\definecolor{headcolor}{RGB}{0,102,204}
\definecolor{keycolor}{RGB}{220,20,60}
\definecolor{solutioncolor}{RGB}{34,139,34}
\definecolor{mnemoniccolor}{RGB}{148,0,211}
\definecolor{codecolor}{RGB}{0,0,100}

% Spacing
\setlength{\parskip}{3pt}
\setlist[itemize]{nosep}
\setlist[enumerate]{nosep}

% Title formatting
\titleformat{\section}{\Large\bfseries\color{headcolor}}{\thesection}{1em}{}
\titleformat{\subsection}{\large\bfseries\color{headcolor}}{\thesubsection}{1em}{}

% Pandoc tightlist compatibility
\providecommand{\tightlist}{%
  \setlength{\itemsep}{0pt}\setlength{\parskip}{0pt}}

% Pandoc longtable compatibility
\newcounter{none}
\def\thenone{}


% content/resources/templates/gujarati-boxes.tex
\usepackage{fontspec}
\usepackage{polyglossia}

% Set Gujarati as main language (document is primarily in Gujarati)
% Note: gloss-gujarati.ldf doesn't exist in polyglossia, but it will use hyphenation patterns
\setdefaultlanguage{gujarati}
\setotherlanguage{english}

% Configure Gujarati font properly
% Use Language=Default to prevent polyglossia from trying to add language-specific features
% that don't exist for Gujarati, which causes "empty feature" warnings
\newfontfamily\gujaratifont[Script=Gujarati,AutoFakeBold=2.5,AutoFakeSlant=0.3]{Noto Sans Gujarati}
\setmainfont[Script=Gujarati,AutoFakeBold=2.5,AutoFakeSlant=0.3]{Noto Sans Gujarati}
% Use Noto Sans Gujarati for monospace to support Gujarati in text
\setmonofont[Scale=0.9]{Noto Sans Gujarati}

% Configure English to use the same font
\newfontfamily\englishfont[Script=Gujarati,AutoFakeBold=2.5,AutoFakeSlant=0.3]{Noto Sans Gujarati}

% Translations for polyglossia
\gappto\captionsgujarati{
  \renewcommand{\tablename}{કોષ્ટક}
  \renewcommand{\figurename}{આકૃતિ}
}

% Helper for TikZ nodes to ensure Gujarati font
\newcommand{\gu}[1]{{\gujaratifont #1}}

% Custom environments
\newtcolorbox{solutionbox}{
    breakable,
    enhanced,
    colback=solutioncolor!5!white,
    colframe=solutioncolor!75!black,
    fonttitle=\bfseries,
    title=જવાબ
}

\newtcolorbox{solutionboxnobreak}{
 colback=solutioncolor!5!white,
 colframe=solutioncolor!75!black,
 fonttitle=\bfseries,
 title=જવાબ
}

\newtcolorbox{keyformula}{
 breakable,
 enhanced,
 colback=keycolor!5!white,
 colframe=keycolor!75!black,
 fonttitle=\bfseries,
 title=રાસાયણિક સમીકરણ/સૂત્ર
}

\newtcolorbox{mnemonicbox}{
 breakable,
 enhanced,
 colback=mnemoniccolor!5!white,
 colframe=mnemoniccolor!75!black,
 fonttitle=\bfseries,
 title=મેમરી ટ્રીક
}


% Custom commands for GTU solutions
% This file defines semantic commands for consistent formatting

% Question command with automatic formatting
\newcommand{\question}[2]{%
  \section*{Question #1}%
  \textbf{#2}%
}

% OR question variant
\newcommand{\questionor}[2]{%
  \section*{Question #1 OR}%
  \textbf{#2}%
}

% Proper table environment with caption
\newenvironment{answertable}[1]{%
  \begin{table}[htbp]
  \centering
  \caption{#1}
}{%
  \end{table}
}

% Proper figure environment for diagrams
\newenvironment{answerdiagram}[1]{%
  \begin{figure}[htbp]
  \centering
  \caption{#1}
}{%
  \end{figure}
}

% Semantic markup for key terms
\newcommand{\keyword}[1]{\textbf{#1}}
\newcommand{\code}[1]{\texttt{#1}}
\newcommand{\classname}[1]{\texttt{#1}}
\newcommand{\methodname}[1]{\texttt{#1}}

% Proper quotation marks
\newcommand{\mnemonic}[1]{``#1''}


\title{વાયરલેસ સેન્સર નેટવર્ક્સ અને IoT (4353201) - સમર 2025 સોલ્યુશન}
\date{મે 12, 2025}

\begin{document}
\maketitle

\questionmarks{1(અ)}{3}{વાયરલેસ સેન્સર નેટવર્ક (WSN) ની વ્યાખ્યા આપો અને તેના મુખ્ય ઘટકોની યાદી આપો.}

\begin{solutionbox}
\textbf{WSN વ્યાખ્યા}: વાયરલેસ સેન્સર નેટવર્ક એ અવકાશીય રીતે વિતરિત સ્વાયત્ત સેન્સર્સનો સંગ્રહ છે જે ભૌતિક અથવા પર્યાવરણીય સ્થિતિઓનું નિરીક્ષણ કરે છે અને નેટવર્ક દ્વારા સહકારી રીતે મુખ્ય સ્થાને ડેટા પસાર કરે છે.

\begin{center}
\captionof{table}{મુખ્ય ઘટકો}
\begin{tabulary}{\linewidth}{|L|L|}
\hline
\textbf{ઘટક} & \textbf{કાર્ય} \\ \hline
\textbf{સેન્સર નોડ્સ} & પર્યાવરણીય ડેટા સંગ્રહ કરે છે \\ \hline
\textbf{બેઝ સ્ટેશન} & ડેટા સંગ્રહ અને પ્રક્રિયા કેન્દ્ર \\ \hline
\textbf{કમ્યુનિકેશન લિંક્સ} & વાયરલેસ ડેટા ટ્રાન્સમિશન \\ \hline
\textbf{ગેટવે} & WSN અને બાહ્ય નેટવર્ક વચ્ચે ઇન્ટરફેસ \\ \hline
\end{tabulary}
\end{center}
\end{solutionbox}

\begin{mnemonicbox}
\mnemonic{SBCG - Sensors Base Communication Gateway}
\end{mnemonicbox}

\questionmarks{1(બ)}{4}{WSNs માં ફિઝિકલ લેયરની ભૂમિકા સમજાવો.}

\begin{solutionbox}
\textbf{ફિઝિકલ લેયર કાર્યો:}

\begin{itemize}
    \item \textbf{સિગ્નલ ટ્રાન્સમિશન}: વાયરલેસ કમ્યુનિકેશન માટે ડિજિટલ ડેટાને રેડિયો તરંગોમાં કન્વર્ટ કરે છે
    \item \textbf{ફ્રીક્વન્સી મૅનેજમેન્ટ}: ISM બેન્ડ્સમાં કાર્ય કરે છે (2.4 GHz, 915 MHz, 433 MHz)
    \item \textbf{પાવર કંટ્રોલ}: બેટરી લાઇફ ઑપ્ટિમાઇઝ કરવા માટે ટ્રાન્સમિશન પાવર મૅનેજ કરે છે
    \item \textbf{મોડ્યુલેશન}: ડેટા એન્કોડિંગ માટે BPSK, QPSK જેવી તકનીકોનો ઉપયોગ કરે છે
\end{itemize}

\begin{center}
\begin{tikzpicture}[node distance=2.2cm, auto]
    \node [gtu block] (data) {ડિજિટલ\\ડેટા};
    \node [gtu block, right of=data, node distance=3.5cm] (phy) {ફિઝિકલ\\લેયર};
    \node [gtu block, right of=phy, node distance=3.5cm] (ant) {એન્ટેના\\ટ્રાન્સમિશન};

    \path [gtu arrow] (data) -- (phy);
    \path [gtu arrow] (phy) -- (ant);
\end{tikzpicture}
\captionof{figure}{સરળ બ્લોક ડાયાગ્રામ}
\end{center}
\end{solutionbox}

\begin{mnemonicbox}
\mnemonic{SFPM - Signal Frequency Power Modulation}
\end{mnemonicbox}

\questionmarks{1(ક)}{7}{WSNs માં ટ્રાન્સીવર્સ માટેની ડિઝાઇન વિચારણાઓની ચર્ચા કરો.}

\begin{solutionbox}
\textbf{મુખ્ય ડિઝાઇન વિચારણાઓ:}

\begin{itemize}
    \item \textbf{પાવર એફિશિયન્સી}: વિસ્તૃત બેટરી લાઇફ માટે અતિ-નીચો પાવર વપરાશ
    \item \textbf{કમ્યુનિકેશન રેન્જ}: રેન્જ (10m-1km) અને પાવર વપરાશ વચ્ચે સંતુલન
    \item \textbf{ડેટા રેટ}: સેન્સર એપ્લિકેશન્સ માટે સામાન્ય રીતે 20-250 kbps
    \item \textbf{ફ્રીક્વન્સી બેન્ડ}: લાઇસન્સિંગ આવશ્યકતાઓ ટાળવા માટે ISM બેન્ડ્સ
    \item \textbf{મોડ્યુલેશન સ્કીમ}: ઓછા પાવર માટે OOK, FSK જેવી સરળ સ્કીમ્સ
    \item \textbf{એન્ટેના ડિઝાઇન}: કોમ્પેક્ટ, ઓમ્નિડાયરેક્શનલ એન્ટેના
    \item \textbf{કોસ્ટ ફેક્ટર}: લાર્જ-સ્કેલ ડિપ્લોયમેન્ટ માટે ઓછી કિંમતના ઘટકો
\end{itemize}

\textbf{ટ્રાન્સીવર આર્કિટેક્ચર:}

\begin{center}
\begin{tikzpicture}[node distance=2.5cm, auto]
    \node [gtu block] (mcu) {MCU};
    \node [gtu block, right of=mcu] (rf) {RF\\Frontend};
    \node [gtu block, right of=rf] (palna) {PA/LNA};
    \node [gtu block, right of=palna] (ant) {Antenna};

    \path [gtu arrow] (mcu) edge [latex-latex] (rf);
    \path [gtu arrow] (rf) edge [latex-latex] (palna);
    \path [gtu arrow] (palna) edge [latex-latex] (ant);
\end{tikzpicture}
\captionof{figure}{ટ્રાન્સીવર આર્કિટેક્ચર}
\end{center}

\begin{center}
\captionof{table}{ટ્રેડ-ઑફ્સ}
\begin{tabulary}{\linewidth}{|L|L|L|}
\hline
\textbf{પેરામીટર} & \textbf{હાઇ પર્ફોર્મન્સ} & \textbf{લો પાવર} \\ \hline
રેન્જ & લાંબી (1km) & ટૂંકી (100m) \\ \hline
પાવર & વધારે (100mW) & ઓછી (1mW) \\ \hline
કિંમત & મંહગું & સસ્તું \\ \hline
\end{tabulary}
\end{center}
\end{solutionbox}

\begin{mnemonicbox}
\mnemonic{PCRFMAC - Power Communication Range Frequency Modulation Antenna Cost}
\end{mnemonicbox}

\questionmarks{1(ક OR)}{7}{WSN માં ઑપ્ટિમાઇઝેશન ગોલ્સ અને ફિગર્સ ઑફ મેરિટને સમજાવો.}

\begin{solutionbox}
\textbf{ઑપ્ટિમાઇઝેશન ગોલ્સ:}

\begin{itemize}
    \item \textbf{એનર્જી એફિશિયન્સી}: પાવર વપરાશ ઘટાડીને નેટવર્ક લાઇફટાઇમ વધારવી
    \item \textbf{કવરેજ}: ન્યૂનતમ સેન્સર નોડ્સ સાથે સંપૂર્ણ વિસ્તાર મૉનિટરિંગ સુનિશ્ચિત કરવું
    \item \textbf{કનેક્ટિવિટી}: નોડ ફેઇલ્યુર સાથે પણ નેટવર્ક કનેક્ટિવિટી જાળવવી
    \item \textbf{ડેટા ક્વોલિટી}: એકત્રિત ડેટાની ઉચ્ચ ચોકસાઇ અને વિશ્વસનીયતા
    \item \textbf{સ્કેલેબિલિટી}: મોટી સંખ્યામાં નોડ્સને સપોર્ટ કરવું (100-10000)
    \item \textbf{કોસ્ટ ઇફેક્ટિવનેસ}: ડિપ્લોયમેન્ટ અને મેઇન્ટેનન્સ કોસ્ટ ઘટાડવી
\end{itemize}

\begin{center}
\captionof{table}{ફિગર્સ ઑફ મેરિટ}
\begin{tabulary}{\linewidth}{|L|L|L|}
\hline
\textbf{મેટ્રિક} & \textbf{વર્ણન} & \textbf{સામાન્ય મૂલ્ય} \\ \hline
\textbf{નેટવર્ક લાઇફટાઇમ} & પ્રથમ નોડ મૃત્યુ સુધીનો સમય & 1-5 વર્ષ \\ \hline
\textbf{કવરેજ રેશિયો} & કવર કરેલું વિસ્તાર/કુલ વિસ્તાર & >95\% \\ \hline
\textbf{કનેક્ટિવિટી} & કનેક્ટેડ નોડ્સ/કુલ નોડ્સ & >90\% \\ \hline
\textbf{લેટન્સી} & એન્ડ-ટુ-એન્ડ વિલંબ & <1 સેકન્ડ \\ \hline
\textbf{થ્રુપુટ} & નોડ દીઠ ડેટા રેટ & 1-100 kbps \\ \hline
\end{tabulary}
\end{center}

\textbf{ઑપ્ટિમાઇઝેશન ટેકનિક્સ:}

\begin{itemize}
    \item \textbf{ક્લસ્ટરિંગ}: કમ્યુનિકેશન ઓવરહેડ ઘટાડવું
    \item \textbf{ડેટા એગ્રિગેશન}: રિડન્ડન્ટ ટ્રાન્સમિશન્સ ઘટાડવા
    \item \textbf{સ્લીપ શેડ્યુલિંગ}: જરૂર ન હોય ત્યારે નોડ્સ બંધ કરવા
\end{itemize}
\end{solutionbox}

\begin{mnemonicbox}
\mnemonic{ECCDC - Energy Coverage Connectivity Data Cost}
\end{mnemonicbox}

\questionmarks{2(અ)}{3}{WSNs માં સેન્સર MAC પ્રોટોકોલની લાક્ષણિકતાઓની યાદી આપો.}

\begin{solutionbox}
\textbf{S-MAC પ્રોટોકોલ લાક્ષણિકતાઓ:}

\begin{center}
\captionof{table}{લાક્ષણિકતાઓ}
\begin{tabulary}{\linewidth}{|L|L|}
\hline
\textbf{લાક્ષણિકતા} & \textbf{વર્ણન} \\ \hline
\textbf{ડ્યુટી સાયક્લિંગ} & સમયાંતરે સ્લીપ અને વેક-અપ સાયકલ \\ \hline
\textbf{કોલિઝન એવોઇડન્સ} & RTS/CTS મેકેનિઝમ \\ \hline
\textbf{ઓવરહિયરિંગ એવોઇડન્સ} & અપ્રાસંગિક ટ્રાન્સમિશન દરમિયાન નોડ્સ સૂઈ જાય છે \\ \hline
\textbf{મેસેજ પાસિંગ} & લાંબા મેસેજીસ ફ્રેગમેન્ટ્સમાં વિભાજિત \\ \hline
\end{tabulary}
\end{center}
\end{solutionbox}

\begin{mnemonicbox}
\mnemonic{DCOM - Duty Collision Overhearing Message}
\end{mnemonicbox}

\questionmarks{2(બ)}{4}{WSNs માં એનર્જી-એફિશિયન્ટ રૂટિંગની વિભાવના વર્ણન કરો.}

\begin{solutionbox}
\textbf{એનર્જી-એફિશિયન્ટ રૂટિંગ કોન્સેપ્ટ:}

એનર્જી-એફિશિયન્ટ રૂટિંગ નેટવર્ક કનેક્ટિવિટી અને ડેટા ડિલિવરી જાળવીને પાવર વપરાશ ઘટાડે છે.

\textbf{મુખ્ય ટેકનિક્સ:}

\begin{itemize}
    \item \textbf{મલ્ટિ-હોપ કમ્યુનિકેશન}: ટૂંકા હોપ્સ લાંબા હોપ્સ કરતાં ઓછા પાવરનો વપરાશ કરે છે
    \item \textbf{લોડ બેલેન્સિંગ}: નોડ ડિપ્લીશન ટાળવા માટે ટ્રાફિક વિતરિત કરવું
    \item \textbf{ડેટા એગ્રિગેશન}: અનેક સ્ત્રોતોમાંથી ડેટા સંયોજિત કરવું
    \item \textbf{જિયોગ્રાફિક રૂટિંગ}: કાર્યક્ષમ પાથ માટે સ્થાન માહિતીનો ઉપયોગ
\end{itemize}

\textbf{એનર્જી મોડલ:}

\begin{lstlisting}[language={},caption={એનર્જી મોડલ}]
E_tx = E_elec * k + e_amp * k * d^2
E_rx = E_elec * k
\end{lstlisting}

\begin{center}
\captionof{table}{રૂટિંગ સ્ટ્રેટેજીસ}
\begin{tabulary}{\linewidth}{|L|L|L|}
\hline
\textbf{સ્ટ્રેટેજી} & \textbf{પાવર સેવિંગ} & \textbf{ઇમ્પ્લિમેન્ટેશન} \\ \hline
\textbf{શોર્ટેસ્ટ પાથ} & મધ્યમ & સરળ \\ \hline
\textbf{મિન-એનર્જી} & ઊંચું & જટિલ \\ \hline
\textbf{મેક્સ-લાઇફટાઇમ} & ખૂબ ઊંચું & ખૂબ જટિલ \\ \hline
\end{tabulary}
\end{center}
\end{solutionbox}

\begin{mnemonicbox}
\mnemonic{MLDG - Multi-hop Load Data Geographic}
\end{mnemonicbox}

\questionmarks{2(ક)}{7}{WSNs માટે MAC પ્રોટોકોલ્સનું વર્ગીકરણ ઉદાહરણો સાથે સમજાવો.}

\begin{solutionbox}
\textbf{MAC પ્રોટોકોલ વર્ગીકરણ:}

\begin{center}
\begin{tikzpicture}[
    level 1/.style={sibling distance=4cm},
    level 2/.style={sibling distance=1.5cm},
    edge from parent/.style={draw, -latex},
    every node/.style={gtu block, font=\small, align=center}
]
    \node {MAC પ્રોટોકોલ્સ}
        child {node {કન્ટેન્શન\\બેઝ્ડ}
            child {node {CSMA/CA}}
            child {node {S-MAC}}
            child {node {T-MAC}}
        }
        child {node {શેડ્યુલ\\બેઝ્ડ}
            child {node {TDMA}}
            child {node {LEACH}}
            child {node {TRAMA}}
        }
        child {node {હાઇબ્રિડ}
            child {node {Z-MAC}}
            child {node {Funneling\\MAC}}
        };
\end{tikzpicture}
\captionof{figure}{MAC પ્રોટોકોલ વર્ગીકરણ}
\end{center}

\textbf{વિગતવાર વર્ગીકરણ:}

\textbf{1. કન્ટેન્શન-બેઝ્ડ પ્રોટોકોલ્સ:}
\begin{itemize}
    \item \textbf{CSMA/CA}: ટ્રાન્સમિશન પહેલાં કેરિયર સેન્સિંગ
    \item \textbf{S-MAC}: સ્લીપ શેડ્યુલ્સ સાથે સિંક્રોનાઇઝ્ડ ડ્યુટી સાયકલ્સ
    \item \textbf{T-MAC}: ટ્રાફિક આધારિત એડાપ્ટિવ ડ્યુટી સાયકલ
\end{itemize}

\textbf{2. શેડ્યુલ-બેઝ્ડ પ્રોટોકોલ્સ:}
\begin{itemize}
    \item \textbf{TDMA}: નોડ્સને ટાઇમ સ્લોટ્સ ફાળવવામાં આવે છે
    \item \textbf{LEACH}: રોટેટિંગ ક્લસ્ટર હેડ્સ સાથે ક્લસ્ટર-બેઝ્ડ
    \item \textbf{TRAMA}: ટ્રાફિક-એડાપ્ટિવ મીડિયમ એક્સેસ
\end{itemize}

\textbf{3. હાઇબ્રિડ પ્રોટોકોલ્સ:}
\begin{itemize}
    \item \textbf{Z-MAC}: CSMA અને TDMA ફાયદાઓને સંયોજિત કરે છે
    \item \textbf{Funneling-MAC}: વિવિધ નેટવર્ક રીજન્સ માટે વિવિધ પ્રોટોકોલ્સ
\end{itemize}

\begin{center}
\captionof{table}{તુલના}
\begin{tabulary}{\linewidth}{|L|L|L|L|}
\hline
\textbf{પ્રોટોકોલ પ્રકાર} & \textbf{એનર્જી એફિશિયન્સી} & \textbf{લેટન્સી} & \textbf{સ્કેલેબિલિટી} \\ \hline
\textbf{કન્ટેન્શન} & મધ્યમ & ઓછું & ઊંચું \\ \hline
\textbf{શેડ્યુલ} & ઊંચું & મધ્યમ & મધ્યમ \\ \hline
\textbf{હાઇબ્રિડ} & ઊંચું & ઓછું & ઊંચું \\ \hline
\end{tabulary}
\end{center}
\end{solutionbox}

\begin{mnemonicbox}
\mnemonic{CSH - Contention Schedule Hybrid}
\end{mnemonicbox}

\questionmarks{2(અ OR)}{3}{WSNs માં એડ્રેસ મેનેજમેન્ટનો હેતુ જણાવો.}

\begin{solutionbox}
\textbf{એડ્રેસ મેનેજમેન્ટ હેતુ:}

\begin{center}
\captionof{table}{હેતુઓ}
\begin{tabulary}{\linewidth}{|L|L|}
\hline
\textbf{હેતુ} & \textbf{વર્ણન} \\ \hline
\textbf{નોડ આઇડેન્ટિફિકેશન} & દરેક સેન્સર નોડની અનન્ય ઓળખ \\ \hline
\textbf{રૂટિંગ સપોર્ટ} & કાર્યક્ષમ ડેટા ફોરવર્ડિંગ સક્ષમ કરવું \\ \hline
\textbf{નેટવર્ક ઓર્ગેનાઇઝેશન} & સ્કેલેબિલિટી માટે હાયરાર્કિકલ એડ્રેસિંગ \\ \hline
\end{tabulary}
\end{center}
\end{solutionbox}

\begin{mnemonicbox}
\mnemonic{NIR - Node Identification Routing}
\end{mnemonicbox}

\questionmarks{2(બ OR)}{4}{જિયોગ્રાફિક રૂટિંગને વિસ્તારથી સમજાવો.}

\begin{solutionbox}
\textbf{જિયોગ્રાફિક રૂટિંગ:}

જિયોગ્રાફિક રૂટિંગ રૂટિંગ ટેબલ્સ જાળવ્યા વિના ફોરવર્ડિંગ નિર્ણયો લેવા માટે ભૌતિક સ્થાન માહિતીનો ઉપયોગ કરે છે.

\textbf{મુખ્ય ઘટકો:}

\begin{itemize}
    \item \textbf{લોકેશન સર્વિસ}: GPS અથવા લોકેલાઇઝેશન એલ્ગોરિધમ્સ
    \item \textbf{ગ્રીડી ફોરવર્ડિંગ}: ડેસ્ટિનેશનની સૌથી નજીકના નેઇબર પાસે ફોરવર્ડ કરવું
    \item \textbf{ફેસ રૂટિંગ}: લોકલ મિનિમા પરિસ્થિતિઓ હેન્ડલ કરવી
    \item \textbf{કોઓર્ડિનેટ સિસ્ટમ}: 2D/3D પોઝિશનિંગ
\end{itemize}

\textbf{ફોરવર્ડિંગ એલ્ગોરિધમ:}

\begin{lstlisting}[language={},caption={ફોરવર્ડિંગ એલ્ગોરિધમ}]
1. ડેસ્ટિનેશન કોઓર્ડિનેટ્સ સાથે પેકેટ મેળવો
2. ડેસ્ટિનેશનની સૌથી નજીકનો નેઇબર શોધો
3. જો વર્તમાન નોડ કરતાં નજીક છે, તો ફોરવર્ડ કરો
4. નહીં તો ફેસ રૂટિંગનો ઉપયોગ કરો અથવા ડ્રોપ કરો
\end{lstlisting}

\begin{center}
\captionof{table}{ફાયદાઓ/નુકસાનો}
\begin{tabulary}{\linewidth}{|L|L|L|}
\hline
\textbf{પાસું} & \textbf{ફાયદો} & \textbf{નુકસાન} \\ \hline
\textbf{સ્કેલેબિલિટી} & કોઈ રૂટિંગ ટેબલ્સ નહીં & લોકેશન ઓવરહેડ \\ \hline
\textbf{એડાપ્ટેબિલિટી} & મોબિલિટી હેન્ડલ કરે છે & લોકલ મિનિમા સમસ્યા \\ \hline
\end{tabulary}
\end{center}
\end{solutionbox}

\begin{mnemonicbox}
\mnemonic{LGFC - Location Greedy Face Coordinate}
\end{mnemonicbox}

\questionmarks{2(ક OR)}{7}{WSN માં LEACH પ્રોટોકોલની કાર્યપ્રણાલી સમજાવો.}

\begin{solutionbox}
\textbf{LEACH પ્રોટોકોલ (લો-એનર્જી એડાપ્ટિવ ક્લસ્ટરિંગ હાયરાર્કી):}

\textbf{પ્રોટોકોલ તબક્કાઓ:}

\begin{center}
\begin{tikzpicture}[node distance=2.5cm, auto]
    \node [gtu state] (setup) {સેટઅપ\\ફેઝ};
    \node [gtu state, right of=setup, node distance=5cm] (steady) {સ્ટેડી સ્ટેટ\\ફેઝ};

    \path [gtu arrow] (setup) edge [bend left] (steady);
    \path [gtu arrow] (steady) edge [bend left] (setup);
    
    % Details for Setup
    \node [below of=setup, node distance=2cm, align=center, font=\footnotesize] {ક્લસ્ટર હેડ સિલેક્શન\\ક્લસ્ટર ફોર્મેશન\\શેડ્યુલ ક્રિએશન};

    % Details for Steady
    \node [below of=steady, node distance=2cm, align=center, font=\footnotesize] {ડેટા કલેક્શન\\ડેટા એગ્રિગેશન\\ડેટા ટ્રાન્સમિશન};
\end{tikzpicture}
\captionof{figure}{LEACH પ્રોટોકોલ તબક્કાઓ}
\end{center}

\textbf{વિગતવાર કાર્યપ્રણાલી:}

\textbf{1. સેટઅપ ફેઝ:}
\begin{itemize}
    \item \textbf{ક્લસ્ટર હેડ સિલેક્શન}: નોડ્સ સંભાવના આધારે ક્લસ્ટર હેડ બનવાનું નક્કી કરે છે
    \item \textbf{એડવર્ટાઇઝમેન્ટ}: ક્લસ્ટર હેડ્સ એડવર્ટાઇઝમેન્ટ મેસેજીસ બ્રોડકાસ્ટ કરે છે
    \item \textbf{ક્લસ્ટર ફોર્મેશન}: નોન-ક્લસ્ટર હેડ નોડ્સ નજીકના ક્લસ્ટર હેડ સાથે જોડાય છે
    \item \textbf{શેડ્યુલ ક્રિએશન}: ક્લસ્ટર સભ્યો માટે TDMA શેડ્યુલ બનાવવામાં આવે છે
\end{itemize}

\textbf{2. સ્ટેડી સ્ટેટ ફેઝ:}
\begin{itemize}
    \item \textbf{ડેટા કલેક્શન}: ક્લસ્ટર સભ્યો ડેટા એકત્રિત કરીને ક્લસ્ટર હેડને મોકલે છે
    \item \textbf{ડેટા એગ્રિગેશન}: ક્લસ્ટર હેડ પ્રાપ્ત ડેટાને એકીકૃત કરે છે
    \item \textbf{ડેટા ટ્રાન્સમિશન}: એકીકૃત ડેટા બેઝ સ્ટેશનને મોકલવામાં આવે છે
\end{itemize}

\textbf{ક્લસ્ટર હેડ સિલેક્શન ફોર્મ્યુલા:}

\begin{lstlisting}[language={},caption={ક્લસ્ટર હેડ સંભાવના}]
P(n) = k / (N - k * (r mod N/k))
Where: 
k = desired cluster heads
N = total nodes
r = current round
\end{lstlisting}

\textbf{એનર્જી ફાયદાઓ:}
\begin{itemize}
    \item \textbf{લોડ ડિસ્ટ્રિબ્યુશન}: ક્લસ્ટર હેડ ભૂમિકા નોડ્સ વચ્ચે ફરે છે
    \item \textbf{ડેટા એગ્રિગેશન}: બેઝ સ્ટેશનને ટ્રાન્સમિશન્સ ઘટાડે છે
    \item \textbf{શોર્ટ રેન્જ કમ્યુનિકેશન}: મોટાભાગના ટ્રાન્સમિશન્સ ક્લસ્ટરની અંદર હોય છે
\end{itemize}

\begin{center}
\captionof{table}{પર્ફોર્મન્સ મેટ્રિક્સ}
\begin{tabulary}{\linewidth}{|L|L|L|}
\hline
\textbf{મેટ્રિક} & \textbf{LEACH} & \textbf{ડાયરેક્ટ ટ્રાન્સમિશન} \\ \hline
\textbf{નેટવર્ક લાઇફટાઇમ} & 8x લાંબી & બેઝલાઇન \\ \hline
\textbf{એનર્જી ડિસ્ટ્રિબ્યુશન} & યુનિફોર્મ & અસમાન \\ \hline
\textbf{સ્કેલેબિલિટી} & ઊંચી & ઓછી \\ \hline
\end{tabulary}
\end{center}
\end{solutionbox}

\begin{mnemonicbox}
\mnemonic{SSCADT - Setup Steady Cluster Aggregation Data Transmission}
\end{mnemonicbox}

\questionmarks{3(અ)}{3}{IoT ની વ્યાખ્યા આપો અને તેના મુખ્ય સ્ત્રોતો જણાવો.}

\begin{solutionbox}
\textbf{IoT વ્યાખ્યા}: ઇન્ટરનેટ ઑફ થિંગ્સ એ સેન્સર્સ, સોફ્ટવેર અને કનેક્ટિવિટી સાથે એમ્બેડેડ ભૌતિક ઉપકરણોનું નેટવર્ક છે જે ડેટા એકત્રિત કરવા અને તેની આપ-લે કરવા માટે છે.

\begin{center}
\captionof{table}{મુખ્ય સ્ત્રોતો}
\begin{tabulary}{\linewidth}{|L|L|}
\hline
\textbf{સ્ત્રોત} & \textbf{વર્ણન} \\ \hline
\textbf{RFID ટેક્નોલોજી} & પદાર્થ ટ્રેકિંગ માટે રેડિયો ફ્રીક્વન્સી આઇડેન્ટિફિકેશન \\ \hline
\textbf{સેન્સર નેટવર્ક્સ} & WSNs અને પર્યાવરણીય મોનિટરિંગ સિસ્ટમ્સ \\ \hline
\textbf{મોબાઇલ કમ્પ્યુટિંગ} & સ્માર્ટફોન્સ અને પોર્ટેબલ ઉપકરણો \\ \hline
\textbf{ક્લાઉડ કમ્પ્યુટિંગ} & સ્કેલેબલ ડેટા સ્ટોરેજ અને પ્રોસેસિંગ \\ \hline
\end{tabulary}
\end{center}
\end{solutionbox}

\begin{mnemonicbox}
\mnemonic{RSMC - RFID Sensor Mobile Cloud}
\end{mnemonicbox}

\questionmarks{3(બ)}{4}{IoT/M2M સિસ્ટમ્સ માટે મોડિફાઇડ OSI મોડલ સમજાવો.}

\begin{solutionbox}
\textbf{IoT માટે મોડિફાઇડ OSI મોડલ:}

\begin{center}
\captionof{table}{લેયર તુલના}
\begin{tabulary}{\linewidth}{|L|L|L|}
\hline
\textbf{લેયર} & \textbf{પરંપરાગત OSI} & \textbf{IoT/M2M મોડિફિકેશન} \\ \hline
\textbf{એપ્લિકેશન} & એન્ડ-યુઝર એપ્લિકેશન્સ & IoT એપ્લિકેશન્સ, ડેટા એનાલિટિક્સ \\ \hline
\textbf{પ્રેઝન્ટેશન} & ડેટા ફોર્મેટિંગ & ડેટા એગ્રિગેશન, સિમેન્ટિક પ્રોસેસિંગ \\ \hline
\textbf{સેશન} & સેશન મેનેજમેન્ટ & ડિવાઇસ મેનેજમેન્ટ, સિક્યુરિટી \\ \hline
\textbf{ટ્રાન્સપોર્ટ} & એન્ડ-ટુ-એન્ડ ડિલિવરી & વિશ્વસનીય/અવિશ્વસનીય ડિલિવરી (UDP/TCP) \\ \hline
\textbf{નેટવર્ક} & રૂટિંગ & IPv6, 6LoWPAN, RPL રૂટિંગ \\ \hline
\textbf{ડેટા લિંક} & ફ્રેમ ડિલિવરી & IEEE 802.15.4, વાઇફાઇ, બ્લૂટૂથ \\ \hline
\textbf{ફિઝિકલ} & બિટ ટ્રાન્સમિશન & રેડિયો, ઓપ્ટિકલ, વાયર્ડ ટ્રાન્સમિશન \\ \hline
\end{tabulary}
\end{center}

\textbf{પ્રોટોકોલ સ્ટેક ઉદાહરણ:}

\begin{center}
\begin{tikzpicture}[node distance=1cm, auto]
    \node [gtu block, minimum width=4cm] (app) {IoT એપ્લિકેશન};
    \node [gtu block, minimum width=4cm, below of=app] (coap) {CoAP/MQTT};
    \node [gtu block, minimum width=4cm, below of=coap] (udp) {UDP};
    \node [gtu block, minimum width=4cm, below of=udp] (lowpan) {6LoWPAN};
    \node [gtu block, minimum width=4cm, below of=lowpan] (phy) {IEEE 802.15.4};
\end{tikzpicture}
\captionof{figure}{IoT પ્રોટોકોલ સ્ટેક}
\end{center}
\end{solutionbox}

\begin{mnemonicbox}
\mnemonic{Six-Layer Low-Power WAN - 6LoWPAN}
\end{mnemonicbox}

\questionmarks{3(ક)}{7}{IoT સિસ્ટમના મુખ્ય ઘટકોની આકૃત સાથે ચર્ચા કરો.}

\begin{solutionbox}
\textbf{IoT સિસ્ટમ આર્કિટેક્ચર:}

\begin{center}
\begin{tikzpicture}[node distance=2.2cm, auto]
    \node [gtu block] (sensor) {સેન્સર્સ};
    \node [gtu block, right of=sensor] (gw) {ગેટવે};
    \node [gtu block, right of=gw] (net) {નેટવર્ક};
    \node [gtu block, right of=net] (cloud) {ક્લાઉડ};
    \node [gtu block, right of=cloud] (app) {એનાલિટિક્સ};
    
    \path [gtu arrow] (sensor) -- (gw);
    \path [gtu arrow] (gw) -- (net);
    \path [gtu arrow] (net) -- (cloud);
    \path [gtu arrow] (cloud) -- (app);

    % Additional blocks
    \node [gtu block, below of=gw, node distance=1.5cm] (sec) {સિક્યુરિટી};
    \node [gtu block, below of=sensor, node distance=1.5cm] (dev) {ડિવાઇસ મેનેજમેન્ટ};
    
    \path [gtu arrow, dashed] (sec) -- (gw);
    \path [gtu arrow, dashed] (sec) -- (net);
    \path [gtu arrow, dashed] (sec) -- (cloud);
    \path [gtu arrow, dashed] (dev) -- (sensor);
\end{tikzpicture}
\captionof{figure}{IoT સિસ્ટમ ઘટકો}
\end{center}

\textbf{મુખ્ય ઘટકો:}
\begin{enumerate}
    \item \textbf{ડિવાઇસ લેયર}: સેન્સર્સ, એક્ચ્યુએટર્સ, MCUs (ESP32), કમ્યુનિકેશન મોડ્યુલ્સ.
    \item \textbf{કનેક્ટિવિટી લેયર}: ગેટવેઝ, નેટવર્ક ઇન્ફ્રાસ્ટ્રક્ચર, પ્રોટોકોલ્સ (MQTT, HTTP).
    \item \textbf{ડેટા પ્રોસેસિંગ લેયર}: ક્લાઉડ પ્લેટફોર્મ્સ, એજ કમ્પ્યુટિંગ, સ્ટોરેજ.
    \item \textbf{એપ્લિકેશન લેયર}: એનાલિટિક્સ એન્જિન, ML, APIs.
    \item \textbf{બિઝનેસ લેયર}: UI, બિઝનેસ લોજિક, ઇન્ટિગ્રેશન.
\end{enumerate}

\begin{center}
\captionof{table}{ઘટક કાર્યો}
\begin{tabulary}{\linewidth}{|L|L|L|L|}
\hline
\textbf{ઘટક} & \textbf{ઇનપુટ} & \textbf{પ્રોસેસિંગ} & \textbf{આઉટપુટ} \\ \hline
\textbf{સેન્સર્સ} & ભૌતિક પેરામીટર્સ & એનાલોગ ટુ ડિજિટલ & ડિજિટલ ડેટા \\ \hline
\textbf{ગેટવે} & સેન્સર ડેટા & પ્રોટોકોલ કન્વર્ઝન & નેટવર્ક પેકેટ્સ \\ \hline
\textbf{ક્લાઉડ} & કાચો ડેટા & સ્ટોરેજ અને એનાલિટિક્સ & પ્રોસેસ્ડ માહિતી \\ \hline
\textbf{એપ્લિકેશન્સ} & પ્રોસેસ્ડ ડેટા & બિઝનેસ લોજિક & યુઝર એક્શન્સ \\ \hline
\end{tabulary}
\end{center}
\end{solutionbox}

\begin{mnemonicbox}
\mnemonic{DCDA-B - Device Connectivity Data Application Business}
\end{mnemonicbox}

\questionmarks{3(અ OR)}{3}{IoT અમલીકરણની ત્રણ પડકારોની યાદી આપો.}

\begin{solutionbox}
\textbf{IoT અમલીકરણ પડકારો:}

\begin{center}
\captionof{table}{પડકારો}
\begin{tabulary}{\linewidth}{|L|L|}
\hline
\textbf{પડકાર} & \textbf{વર્ણન} \\ \hline
\textbf{સિક્યુરિટી અને પ્રાઇવસી} & ડેટા અને ડિવાઇસ એક્સેસનું સુરક્ષણ \\ \hline
\textbf{ઇન્ટરઓપરેબિલિટી} & વિવિધ પ્રોટોકોલ્સ અને સ્ટાન્ડર્ડ્સ \\ \hline
\textbf{સ્કેલેબિલિટી} & લાખો કનેક્ટેડ ડિવાઇસીસનું મેનેજમેન્ટ \\ \hline
\end{tabulary}
\end{center}
\end{solutionbox}

\begin{mnemonicbox}
\mnemonic{SIS - Security Interoperability Scalability}
\end{mnemonicbox}

\questionmarks{3(બ OR)}{4}{IoT પાછળની ટેક્નોલોજીને ઉદાહરણો સાથે વર્ણન કરો.}

\begin{solutionbox}
\textbf{મુખ્ય ટેક્નોલોજીઓ:}

\begin{itemize}
    \item \textbf{સેન્સિંગ ટેક્નોલોજી}: MEMS, એન્વાયરનમેન્ટલ (DHT22), બાયોમેટ્રિક સેન્સર્સ.
    \item \textbf{કમ્યુનિકેશન ટેક્નોલોજી}: શોર્ટ રેન્જ (વાઇફાઇ, Zigbee), લોંગ રેન્જ (LoRaWAN, 5G).
    \item \textbf{કમ્પ્યુટિંગ ટેક્નોલોજી}: MCUs (ESP32), SBCs (Raspberry Pi).
    \item \textbf{ક્લાઉડ ટેક્નોલોજી}: AWS IoT, Azure IoT, ડેટા એનાલિટિક્સ.
\end{itemize}

\textbf{ટેક્નોલોજી સ્ટેક ઉદાહરણ:}

\begin{center}
\begin{tikzpicture}[node distance=1cm, auto]
    \node [gtu block, minimum width=4cm] (cloud) {ક્લાઉડ (AWS)};
    \node [gtu block, minimum width=4cm, below of=cloud] (wifi) {વાઇફાઇ નેટવર્ક};
    \node [gtu block, minimum width=4cm, below of=wifi] (mcu) {ESP32 MCU};
    \node [gtu block, minimum width=4cm, below of=mcu] (sensor) {DHT22 સેન્સર};
\end{tikzpicture}
\captionof{figure}{ટેક્નોલોજી સ્ટેક ઉદાહરણ}
\end{center}
\end{solutionbox}

\begin{mnemonicbox}
\mnemonic{SCCC - Sensing Communication Computing Cloud}
\end{mnemonicbox}

\questionmarks{3(ક OR)}{7}{IoT માં M2M કમ્યુનિકેશનની ભૂમિકા ઉદાહરણ એપ્લિકેશન સાથે સમજાવો.}

\begin{solutionbox}
\textbf{IoT માં M2M કમ્યુનિકેશન:}

મશીન-ટુ-મશીન (M2M) કમ્યુનિકેશન માનવી હસ્તક્ષેપ વિના ઉપકરણો વચ્ચે સ્વયંચાલિત ડેટા આપ-લે સક્ષમ કરે છે.

\textbf{M2M આર્કિટેક્ચર:}

\begin{center}
\begin{tikzpicture}[node distance=2.5cm, auto]
    \node [gtu block] (gw) {M2M ગેટવે};
    \node [gtu block, left of=gw] (d1) {ડિવાઇસ 1};
    \node [gtu block, above of=d1, node distance=1.5cm] (d2) {ડિવાઇસ 2};
    \node [gtu block, below of=d1, node distance=1.5cm] (d3) {ડિવાઇસ 3};
    
    \node [gtu block, right of=gw, node distance=3cm] (server) {M2M સર્વર};
    \node [gtu block, right of=server] (app) {એપ સર્વર};
    \node [gtu block, right of=app] (user) {એન્ડ યુઝર};

    \path [gtu arrow] (d1) -- (gw);
    \path [gtu arrow] (d2) -- (gw);
    \path [gtu arrow] (d3) -- (gw);
    \path [gtu arrow] (gw) -- (server);
    \path [gtu arrow] (server) -- (app);
    \path [gtu arrow] (app) -- (user);
\end{tikzpicture}
\captionof{figure}{M2M આર્કિટેક્ચર}
\end{center}

\textbf{ઉદાહરણ એપ્લિકેશન: સ્માર્ટ સ્ટ્રીટ લાઇટિંગ સિસ્ટમ}
\begin{itemize}
    \item \textbf{મોશન સેન્સર્સ}: ચળવળ શોધે છે અને Zigbee દ્વારા ડેટા મોકલે છે.
    \item \textbf{લાઇટ્સ}: "લાઇટિંગ પાથ" બનાવવા માટે વાતચીત કરે છે, બ્રાઇટનેસ એડજસ્ટ કરે છે.
    \item \textbf{કંટ્રોલર}: સેલ્યુલર કનેક્શન દ્વારા શેડ્યુલ્સ ઑપ્ટિમાઇઝ કરે છે.
    \item \textbf{ફાયદાઓ}: એનર્જી એફિશિયન્સી (60\%), પ્રેડિક્ટિવ મેઇન્ટેનન્સ.
\end{itemize}

\begin{center}
\captionof{table}{પર્ફોર્મન્સ મેટ્રિક્સ}
\begin{tabulary}{\linewidth}{|L|L|L|}
\hline
\textbf{મેટ્રિક} & \textbf{પરંપરાગત} & \textbf{M2M સ્માર્ટ સિસ્ટમ} \\ \hline
\textbf{એનર્જી વપરાશ} & 100\% & 40\% \\ \hline
\textbf{રિસ્પોન્સ ટાઇમ} & મેન્યુઅલ (કલાકો) & સ્વયંચાલિત (સેકન્ડો) \\ \hline
\end{tabulary}
\end{center}
\end{solutionbox}

\begin{mnemonicbox}
\mnemonic{ARSR - Autonomous Real-time Scalable Reliable}
\end{mnemonicbox}

\questionmarks{4(અ)}{3}{IoT માં વપરાતા ત્રણ એપ્લિકેશન લેયર પ્રોટોકોલ્સના નામ આપો.}

\begin{solutionbox}
\textbf{IoT એપ્લિકેશન લેયર પ્રોટોકોલ્સ:}

\begin{center}
\captionof{table}{પ્રોટોકોલ્સ}
\begin{tabulary}{\linewidth}{|L|L|}
\hline
\textbf{પ્રોટોકોલ} & \textbf{હેતુ} \\ \hline
\textbf{MQTT} & લાઇટવેઇટ પબ્લિશ-સબ્સ્ક્રાઇબ મેસેજિંગ \\ \hline
\textbf{CoAP} & રિસોર્સ-લિમિટેડ ડિવાઇસીસ માટે કન્સ્ટ્રેઇન્ડ એપ્લિકેશન પ્રોટોકોલ \\ \hline
\textbf{HTTP/HTTPS} & વેબ-આધારિત RESTful કમ્યુનિકેશન \\ \hline
\end{tabulary}
\end{center}
\end{solutionbox}

\begin{mnemonicbox}
\mnemonic{MCH - MQTT CoAP HTTP}
\end{mnemonicbox}

\questionmarks{4(બ)}{4}{IoT સિસ્ટમ્સમાં MQTT ની ભૂમિકા સમજાવો.}

\begin{solutionbox}
\textbf{MQTT ભૂમિકા:}
MQTT એ મર્યાદિત સંસાધનો સાથેના IoT ઉપકરણો માટે ડિઝાઇન કરેલ લાઇટવેઇટ પબ્લિશ-સબ્સ્ક્રાઇબ મેસેજિંગ પ્રોટોકોલ છે.

\textbf{આર્કિટેક્ચર:}

\begin{center}
\begin{tikzpicture}[node distance=3cm, auto]
    \node [gtu block] (broker) {MQTT\\બ્રોકર};
    \node [gtu block, left of=broker] (pub) {પબ્લિશર\\(સેન્સર)};
    \node [gtu block, right of=broker] (sub) {સબ્સ્ક્રાઇબર\\(ડિસ્પ્લે)};

    \path [gtu arrow] (pub) -- node [above] {Publish} (broker);
    \path [gtu arrow] (broker) -- node [above] {Subscribe} (sub);
\end{tikzpicture}
\captionof{figure}{MQTT આર્કિટેક્ચર}
\end{center}

\textbf{QoS સ્તરો:}
\begin{itemize}
    \item \textbf{QoS 0}: વધુમાં વધુ એક વખત (ફાયર એન્ડ ફરગેટ).
    \item \textbf{QoS 1}: ઓછામાં ઓછું એક વખત (ગેરેન્ટીડ ડિલિવરી).
    \item \textbf{QoS 2}: બરાબર એક વખત (જટિલ ડેટા).
\end{itemize}
\end{solutionbox}

\begin{mnemonicbox}
\mnemonic{PQPL - Publish QoS Persistent Last-will}
\end{mnemonicbox}

\questionmarks{4(ક)}{7}{NodeMCU નો ઉપયોગ કરીને તાપમાન સેન્સર ડેટા વાંચીને ક્લાઉડ પ્લેટફોર્મ પર ટ્રાન્સમિટ કરવા માટે સિસ્ટમ ડિઝાઇન કરો.}

\begin{solutionbox}
\textbf{સિસ્ટમ ડિઝાઇન: તાપમાન મોનિટરિંગ સિસ્ટમ}

\begin{center}
\begin{tikzpicture}[node distance=2.2cm, auto]
    \node [gtu block] (sensor) {DHT22};
    \node [gtu block, right of=sensor] (mcu) {NodeMCU};
    \node [gtu block, right of=mcu] (wifi) {રાઉટર};
    \node [gtu block, right of=wifi] (cloud) {ક્લાઉડ};
    \node [gtu block, right of=cloud] (dash) {ડેશબોર્ડ};

    \path [gtu arrow] (sensor) -- (mcu);
    \path [gtu arrow] (mcu) -- (wifi);
    \path [gtu arrow] (wifi) -- (cloud);
    \path [gtu arrow] (cloud) -- (dash);
\end{tikzpicture}
\captionof{figure}{સિસ્ટમ આર્કિટેક્ચર}
\end{center}

\textbf{સર્કિટ ડાયાગ્રામ:}

\begin{center}
\begin{tikzpicture}[node distance=3cm, auto]
    \node [gtu block, minimum height=2cm] (mcu) {NodeMCU\\ESP8266};
    \node [gtu block, right of=mcu, minimum height=2cm] (sensor) {DHT22\\સેન્સર};

    \draw [thick] (mcu.east) -- (sensor.west) node [midway, above] {D4 $\leftrightarrow$ DATA};
    \draw [thick] ([yshift=0.5cm]mcu.east) -- ([yshift=0.5cm]sensor.west) node [midway, above] {3.3V $\leftrightarrow$ VCC};
    \draw [thick] ([yshift=-0.5cm]mcu.east) -- ([yshift=-0.5cm]sensor.west) node [midway, above] {GND $\leftrightarrow$ GND};
\end{tikzpicture}
\captionof{figure}{હાર્ડવેર કનેક્શન્સ}
\end{center}

\textbf{કોડ સ્નિપેટ:}
\begin{lstlisting}[language=C++,caption={Arduino કોડ}]
#include <ESP8266WiFi.h>
#include <DHT.h>
#define DHT_PIN D4
// Setup and Loop to read temp and publish via MQTT
float temp = dht.readTemperature();
client.publish("sensor/data", String(temp).c_str());
\end{lstlisting}
\end{solutionbox}

\begin{mnemonicbox}
\mnemonic{HSCDP - Hardware Software Cloud Data Platform}
\end{mnemonicbox}

\questionmarks{4(અ OR)}{3}{IoT એપ્લિકેશન્સમાં વપરાતા સેન્સર્સના પ્રકારોની યાદી આપો.}

\begin{solutionbox}
\textbf{IoT સેન્સર પ્રકારો:}

\begin{center}
\captionof{table}{સેન્સર પ્રકારો}
\begin{tabulary}{\linewidth}{|L|L|}
\hline
\textbf{સેન્સર પ્રકાર} & \textbf{માપણ} \\ \hline
\textbf{તાપમાન} & આસપાસ અને સપાટીનું તાપમાન \\ \hline
\textbf{મોશન/PIR} & હિલચાલ અને હાજરી શોધવી \\ \hline
\textbf{લાઇટ/LDR} & આસપાસના પ્રકાશની તીવ્રતા \\ \hline
\end{tabulary}
\end{center}
\end{solutionbox}

\begin{mnemonicbox}
\mnemonic{TML - Temperature Motion Light}
\end{mnemonicbox}

\questionmarks{4(બ OR)}{4}{IoT સિસ્ટમ્સમાં સિક્યુરિટી પડકારોની ચર્ચા કરો.}

\begin{solutionbox}
\textbf{IoT સિક્યુરિટી પડકારો:}

\begin{itemize}
    \item \textbf{ડિવાઇસ-લેવલ}: નબળી ઓથેન્ટિકેશન, ફર્મવેર વલ્નરેબિલિટીઝ.
    \item \textbf{નેટવર્ક-લેવલ}: અનએન્ક્રિપ્ટેડ કમ્યુનિકેશન, મેન-ઇન-ધ-મિડલ.
    \item \textbf{ક્લાઉડ-લેવલ}: ડેટા પ્રાઇવસી, API સિક્યુરિટી, ડેટા બ્રીચીસ.
\end{itemize}

\begin{center}
\captionof{table}{સોલ્યુશન્સ}
\begin{tabulary}{\linewidth}{|L|L|}
\hline
\textbf{પડકાર} & \textbf{સોલ્યુશન} \\ \hline
\textbf{નબળી ઓથેન્ટિકેશન} & મજબૂત પાસવર્ડ્સ, મલ્ટિ-ફેક્ટર ઓથેન્ટિકેશન \\ \hline
\textbf{ડેટા ટ્રાન્સમિશન} & એન્ડ-ટુ-એન્ડ એન્ક્રિપ્શન (TLS/SSL) \\ \hline
\textbf{ફર્મવેર} & સિક્યોર OTA અપડેટ્સ \\ \hline
\end{tabulary}
\end{center}
\end{solutionbox}

\begin{mnemonicbox}
\mnemonic{DNCI - Device Network Cloud Identity}
\end{mnemonicbox}

\questionmarks{4(ક OR)}{7}{મોબાઇલ એપ દ્વારા Raspberry Pi નો ઉપયોગ કરીને બલ્બને કંટ્રોલ કરવા માટે બ્લોક ડાયાગ્રામ દોરો અને બ્લોક્સને વિસ્તારથી સમજાવો.}

\begin{solutionbox}
\textbf{સ્માર્ટ બલ્બ કંટ્રોલ સિસ્ટમ:}

\begin{center}
\begin{tikzpicture}[node distance=2cm, auto]
    \node [gtu block] (app) {મોબાઇલ એપ};
    \node [gtu block, right of=app] (router) {રાઉટર};
    \node [gtu block, right of=router] (pi) {Raspberry Pi};
    \node [gtu block, right of=pi] (relay) {રિલે};
    \node [gtu block, right of=relay] (bulb) {AC બલ્બ};

    \path [gtu arrow] (app) -- (router);
    \path [gtu arrow] (router) -- (pi);
    \path [gtu arrow] (pi) -- (relay);
    \path [gtu arrow] (relay) -- (bulb);
\end{tikzpicture}
\captionof{figure}{કંટ્રોલ ફ્લો}
\end{center}

\textbf{સિસ્ટમ ઓપરેશન ફ્લો:}
\begin{itemize}
    \item \textbf{Tap ON}: એપ HTTP રિક્વેસ્ટ મોકલે છે -> વેબ સર્વર (Pi) -> GPIO HIGH -> રિલે ON -> બલ્બ ON.
    \item \textbf{Tap OFF}: એપ HTTP રિક્વેસ્ટ મોકલે છે -> વેબ સર્વર (Pi) -> GPIO LOW -> રિલે OFF -> બલ્બ OFF.
\end{itemize}

\textbf{હાર્ડવેર કનેક્શન્સ:}
\begin{itemize}
    \item \textbf{Raspberry Pi}: GPIO 18, 5V, GND રિલે સાથે જોડાયેલ.
    \item \textbf{રિલે}: AC સર્કિટના લાઇવ વાયરને કંટ્રોલ કરે છે.
    \item \textbf{સેફ્ટી}: રિલે મોડ્યુલમાં ઓપ્ટોકપલર આઇસોલેશન.
\end{itemize}
\end{solutionbox}

\begin{mnemonicbox}
\mnemonic{MIHRBA - Mobile Internet Home-router Raspberry-pi Relay Bulb}
\end{mnemonicbox}

\questionmarks{5(અ)}{3}{IoT એપ્લિકેશન્સને વ્યાપક શ્રેણીઓમાં વર્ગીકૃત કરો.}

\begin{solutionbox}
\textbf{IoT એપ્લિકેશન શ્રેણીઓ:}

\begin{center}
\captionof{table}{શ્રેણીઓ}
\begin{tabulary}{\linewidth}{|L|L|}
\hline
\textbf{શ્રેણી} & \textbf{વર્ણન} \\ \hline
\textbf{કન્ઝ્યુમર IoT} & સ્માર્ટ હોમ્સ, વિયરેબલ્સ, મનોરંજન \\ \hline
\textbf{ઇન્ડસ્ટ્રિયલ IoT} & મેન્યુફેક્ચરિંગ, સપ્લાય ચેઇન, પ્રેડિક્ટિવ મેઇન્ટેનન્સ \\ \hline
\textbf{ઇન્ફ્રાસ્ટ્રક્ચર IoT} & સ્માર્ટ સિટીઝ, ટ્રાન્સપોર્ટેશન, યુટિલિટીઝ \\ \hline
\end{tabulary}
\end{center}
\end{solutionbox}

\begin{mnemonicbox}
\mnemonic{CII - Consumer Industrial Infrastructure}
\end{mnemonicbox}

\questionmarks{5(બ)}{4}{IoT નો ઉપયોગ કરીને સ્માર્ટ હોમ ઓટોમેશન સિસ્ટમની કાર્યપ્રણાલી સમજાવો.}

\begin{solutionbox}
\textbf{સ્માર્ટ હોમ ઓટોમેશન સિસ્ટમ:}

\begin{center}
\begin{tikzpicture}[node distance=2.5cm, auto]
    \node [gtu block] (hub) {સેન્ટ્રલ હબ};
    \node [gtu block, left of=hub] (sensor) {સેન્સર્સ\\(ઇનપુટ)};
    \node [gtu block, right of=hub] (actuator) {એક્ચ્યુએટર્સ\\(આઉટપુટ)};
    \node [gtu block, below of=hub] (cloud) {ક્લાઉડ\\સર્વિસીસ};
    \node [gtu block, left of=cloud] (app) {મોબાઇલ એપ};

    \path [gtu arrow] (sensor) -- (hub);
    \path [gtu arrow] (hub) -- (actuator);
    \path [gtu arrow] (hub) edge [latex-latex] (cloud);
    \path [gtu arrow] (cloud) edge [latex-latex] (app);
\end{tikzpicture}
\captionof{figure}{ઓટોમેશન સિસ્ટમ}
\end{center}

\textbf{ફાયદાઓ:}
\begin{itemize}
    \item \textbf{એનર્જી એફિશિયન્સી}: 20-30\% ઘટાડો.
    \item \textbf{સિક્યુરિટી}: રીઅલ-ટાઇમ અલર્ટ્સ.
    \item \textbf{સુવિધા}: વોઇસ કમાન્ડ્સ.
\end{itemize}
\end{solutionbox}

\begin{mnemonicbox}
\mnemonic{HCSA - Hub Communication Sensors Actuators}
\end{mnemonicbox}

\questionmarks{5(ક)}{7}{IoT આધારિત હેલ્થકેર મોનિટરિંગ સિસ્ટમ માટે બ્લોક ડાયાગ્રામ અને કાર્યસિદ્ધાંત સૂચવો.}

\begin{solutionbox}
\textbf{IoT હેલ્થકેર મોનિટરિંગ સિસ્ટમ:}

\begin{center}
\begin{tikzpicture}[node distance=2.5cm, auto]
    \node [gtu block] (wear) {વિયરેબલ્સ};
    \node [gtu block, right of=wear] (phone) {સ્માર્ટફોન};
    \node [gtu block, right of=phone] (cloud) {ક્લાઉડ\\પ્લેટફોર્મ};
    \node [gtu block, below of=cloud] (analytics) {એનાલિટિક્સ\\AI};
    \node [gtu block, left of=analytics] (ui) {ડોક્ટર/પેશન્ટ\\ડેશબોર્ડ};

    \path [gtu arrow] (wear) -- (phone);
    \path [gtu arrow] (phone) -- (cloud);
    \path [gtu arrow] (cloud) -- (analytics);
    \path [gtu arrow] (analytics) -- (ui);
    
    % Feedback loop
    \path [gtu arrow, dashed] (ui) -| (phone);
\end{tikzpicture}
\captionof{figure}{હેલ્થકેર આર્કિટેક્ચર}
\end{center}

\textbf{વિગતવાર ઘટકો:}
\begin{enumerate}
    \item \textbf{પેશન્ટ ડિવાઇસીસ}: સ્માર્ટવોચ, BP મોનિટર, સ્માર્ટ પેચીસ.
    \item \textbf{કમ્યુનિકેશન}: BLE થી ફોન, વાઇફાઇ/સેલ્યુલર થી ક્લાઉડ.
    \item \textbf{ક્લાઉડ ઇન્ફ્રાસ્ટ્રક્ચર}: HIPAA કમ્પ્લાયન્ટ સ્ટોરેજ, રીઅલ-ટાઇમ પ્રોસેસિંગ.
    \item \textbf{એનાલિટિક્સ}: વાઇટલ સાઇન્સ એનાલિસિસ, પ્રેડિક્ટિવ અલર્ટ્સ.
    \item \textbf{ઇન્ટરફેસીસ}: પેશન્ટ એપ, ડોક્ટર પોર્ટલ, ઇમર્જન્સી ડેશબોર્ડ.
\end{enumerate}

\textbf{કાર્યસિદ્ધાંત:}
\begin{itemize}
    \item \textbf{કલેક્શન}: દર 15-30 સેકન્ડે વાઇટલ સાઇન્સ.
    \item \textbf{એનાલિસિસ}: ML એલ્ગોરિધમ્સ વિસંગતતાઓ તપાસે છે.
    \item \textbf{રિસ્પોન્સ}: જો ક્રિટિકલ હોય તો ડોક્ટર્સ/કુટુંબને અલર્ટ્સ.
\end{itemize}
\end{solutionbox}

\begin{mnemonicbox}
\mnemonic{WHDCA-UI - Wearables Home-devices Data Communication Analytics User-interface}
\end{mnemonicbox}

\questionmarks{5(અ OR)}{3}{ત્રણ વાસ્તવિક IoT એપ્લિકેશન્સની યાદી આપો.}

\begin{solutionbox}
\textbf{વાસ્તવિક IoT એપ્લિકેશન્સ:}

\begin{center}
\captionof{table}{એપ્લિકેશન્સ}
\begin{tabulary}{\linewidth}{|L|L|}
\hline
\textbf{એપ્લિકેશન} & \textbf{વર્ણન} \\ \hline
\textbf{સ્માર્ટ એગ્રિકલ્ચર} & સ્વયંચાલિત સિંચાઈ \\ \hline
\textbf{ઇન્ડસ્ટ્રિયલ મોનિટરિંગ} & પ્રેડિક્ટિવ મેઇન્ટેનન્સ \\ \hline
\textbf{સ્માર્ટ ટ્રાન્સપોર્ટેશન} & ટ્રાફિક મેનેજમેન્ટ \\ \hline
\end{tabulary}
\end{center}
\end{solutionbox}

\begin{mnemonicbox}
\mnemonic{AIT - Agriculture Industrial Transportation}
\end{mnemonicbox}

\questionmarks{5(બ OR)}{4}{સ્માર્ટ પાર્કિંગ સિસ્ટમમાં IoT ની ભૂમિકા વર્ણન કરો.}

\begin{solutionbox}
\textbf{સ્માર્ટ પાર્કિંગ સિસ્ટમ:}
\begin{itemize}
    \item \textbf{સેન્સર્સ}: વાહનની હાજરી શોધે છે.
    \item \textbf{ક્લાઉડ}: ઉપલબ્ધતા ગણતરી કરે છે.
    \item \textbf{એપ}: ડ્રાઇવરોને ખાલી જગ્યાઓ બતાવે છે.
\end{itemize}

\textbf{ફાયદાઓ:}
\begin{itemize}
    \item રીઅલ-ટાઇમ ઉપલબ્ધતા અપડેટ્સ.
    \item સ્વયંચાલિત પેમેન્ટ્સ.
    \item ઇંધણ વપરાશમાં ઘટાડો.
\end{itemize}
\end{solutionbox}

\begin{mnemonicbox}
\mnemonic{DCPN - Detection Collection Processing Notification}
\end{mnemonicbox}

\questionmarks{5(ક OR)}{7}{Raspberry Pi ના આર્કિટેક્ચર બ્લોક ડાયાગ્રામ દોરો અને તેને સમજાવો.}

\begin{solutionbox}
\textbf{Raspberry Pi 4B આર્કિટેક્ચર:}

\begin{center}
\begin{tikzpicture}[node distance=2.5cm, auto]
    \node [gtu block, minimum width=3cm] (soc) {SoC BCM2711\\(CPU+GPU)};
    \node [gtu block, above of=soc] (ram) {LPDDR4 RAM};
    \node [gtu block, below of=soc] (power) {પાવર મેનેજમેન્ટ};
    \node [gtu block, left of=soc] (io) {USB/ઇથરનેટ};
    \node [gtu block, right of=soc] (gpio) {GPIO/કેમેરા};

    \path [gtu arrow] (soc) -- (ram);
    \path [gtu arrow] (power) -- (soc);
    \path [gtu arrow] (soc) -- (io);
    \path [gtu arrow] (soc) -- (gpio);
\end{tikzpicture}
\captionof{figure}{સરળ આર્કિટેક્ચર}
\end{center}

\textbf{વિગતવાર ઘટકો:}
\begin{enumerate}
    \item \textbf{CPU}: ક્વાડ-કોર ARM Cortex-A72 (1.5 GHz).
    \item \textbf{GPU}: VideoCore VI (4K વિડિયો સપોર્ટ).
    \item \textbf{RAM}: 1GB - 8GB LPDDR4 વિકલ્પો.
    \item \textbf{કનેક્ટિવિટી}: ગિગાબિટ ઇથરનેટ, ડ્યુઅલ-બેન્ડ વાઇફાઇ, BT 5.0.
    \item \textbf{I/O}: 40-pin GPIO, 2x માઇક્રો-HDMI, CSI/DSI પોર્ટ્સ.
\end{enumerate}

\textbf{ફાયદાઓ:}
\begin{itemize}
    \item સંપૂર્ણ Linux OS સપોર્ટ.
    \item સમૃદ્ધ સમુદાય ઇકોસિસ્ટમ.
    \item કોસ્ટ-ઇફેક્ટિવ (\$35+).
\end{itemize}
\end{solutionbox}

\begin{mnemonicbox}
\mnemonic{CPU-GPU-SoC-MEM-CONN-IO-PWR-BOOT - Complete Pi Architecture}
\end{mnemonicbox}

\end{document}
