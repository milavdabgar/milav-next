\documentclass[10pt,a4paper]{article}

% content/resources/templates/preamble.tex
\usepackage[margin=0.6in]{geometry}
\author{Milav Dabgar}
\usepackage{amsmath,amssymb,amsthm}
\usepackage{booktabs}
\usepackage{multirow}
\usepackage{xcolor}
\usepackage{tcolorbox}
\tcbuselibrary{breakable,skins}
\usepackage[colorlinks=true,linkcolor=blue]{hyperref}
\usepackage{titlesec}
\usepackage{enumitem}
\usepackage{tikz}
\usepackage{pgfplots}
\usepackage{circuitikz}
\usepackage[version=4]{mhchem}
\usepackage{longtable}
\usepackage{array}
\usepackage{float}
\usepackage{caption}
\usepackage{listings}

\lstset{
  basicstyle=\small\ttfamily,
  breaklines=true,
  breakatwhitespace=false,
  postbreak=\mbox{\textcolor{red}{$\hookrightarrow$}\space},
  float=false,
  numbers=left,
  numberstyle=\tiny\color{gray},
  numbersep=10pt,
  xleftmargin=2em,
  keywordstyle=\color{blue},
  commentstyle=\color{green!60!black},
  stringstyle=\color{purple},
  backgroundcolor=\color{gray!5},
  showstringspaces=false,
  tabsize=2,
  captionpos=b,
  keepspaces=true,
  columns=flexible
}

\pgfplotsset{compat=1.18}
\usetikzlibrary{shapes,arrows,positioning,calc,patterns,decorations.pathmorphing,decorations.markings,arrows.meta}

% Color scheme
\definecolor{headcolor}{RGB}{0,102,204}
\definecolor{keycolor}{RGB}{220,20,60}
\definecolor{solutioncolor}{RGB}{34,139,34}
\definecolor{mnemoniccolor}{RGB}{148,0,211}
\definecolor{codecolor}{RGB}{0,0,100}

% Spacing
\setlength{\parskip}{3pt}
\setlist[itemize]{nosep}
\setlist[enumerate]{nosep}

% Title formatting
\titleformat{\section}{\Large\bfseries\color{headcolor}}{\thesection}{1em}{}
\titleformat{\subsection}{\large\bfseries\color{headcolor}}{\thesubsection}{1em}{}

% Pandoc tightlist compatibility
\providecommand{\tightlist}{%
  \setlength{\itemsep}{0pt}\setlength{\parskip}{0pt}}

% Pandoc longtable compatibility
\newcounter{none}
\def\thenone{}


% content/resources/templates/english-boxes.tex
% This file is currently empty - it exists to maintain consistency with the import structure.
% Add custom environments here if needed in the future.


\begin{document}

\begin{center}
{\Huge\bfseries\color{headcolor} Subject Name Solutions}\\[5pt]
{\LARGE 4353201 -- Summer 2025}\\[3pt]
{\large Semester 1 Study Material}\\[3pt]
{\normalsize\textit{Detailed Solutions and Explanations}}
\end{center}

\vspace{10pt}

\subsection*{Question 1(a) [3 marks]}\label{q1a}

\textbf{Define Wireless Sensor Networks (WSN) and list its key
components.}

\begin{solutionbox}

\textbf{WSN Definition}: A Wireless Sensor Network is a collection of
spatially distributed autonomous sensors that monitor physical or
environmental conditions and cooperatively pass data through the network
to a main location.

\textbf{Key Components Table:}

{\def\LTcaptype{none} % do not increment counter
\begin{longtable}[]{@{}ll@{}}
\toprule\noalign{}
Component & Function \\
\midrule\noalign{}
\endhead
\bottomrule\noalign{}
\endlastfoot
\textbf{Sensor Nodes} & Collect environmental data \\
\textbf{Base Station} & Data collection and processing center \\
\textbf{Communication Links} & Wireless data transmission \\
\textbf{Gateway} & Interface between WSN and external networks \\
\end{longtable}
}

\end{solutionbox}
\begin{mnemonicbox}
``SBCG - Sensors Base Communication Gateway''

\end{mnemonicbox}
\subsection*{Question 1(b) [4 marks]}\label{q1b}

\textbf{Explain the role of the physical layer in WSNs.}

\begin{solutionbox}

\textbf{Physical Layer Functions:}

\begin{itemize}
\tightlist
\item
  \textbf{Signal Transmission}: Converts digital data into radio waves
  for wireless communication
\item
  \textbf{Frequency Management}: Operates in ISM bands (2.4 GHz, 915
  MHz, 433 MHz)
\item
  \textbf{Power Control}: Manages transmission power to optimize battery
  life
\item
  \textbf{Modulation}: Uses techniques like BPSK, QPSK for data encoding
\end{itemize}

\textbf{Simple Block Diagram:}

\begin{lstlisting}
┌─────────────┐    ┌─────────────┐    ┌─────────────┐
│   Digital   │───▶│  Physical   │───▶│  Antenna    │
│    Data     │    │   Layer     │    │ Transmission│
└─────────────┘    └─────────────┘    └─────────────┘
\end{lstlisting}

\end{solutionbox}
\begin{mnemonicbox}
``SFPM - Signal Frequency Power Modulation''

\end{mnemonicbox}
\subsection*{Question 1(c) [7 marks]}\label{q1c}

\textbf{Discuss the design considerations for transceivers in WSNs.}

\begin{solutionbox}

\textbf{Key Design Considerations:}

\begin{itemize}
\tightlist
\item
  \textbf{Power Efficiency}: Ultra-low power consumption for extended
  battery life
\item
  \textbf{Communication Range}: Balance between range (10m-1km) and
  power consumption\\
\item
  \textbf{Data Rate}: Typically 20-250 kbps for sensor applications
\item
  \textbf{Frequency Band}: ISM bands to avoid licensing requirements
\item
  \textbf{Modulation Scheme}: Simple schemes like OOK, FSK for low power
\item
  \textbf{Antenna Design}: Compact, omnidirectional antennas
\item
  \textbf{Cost Factor}: Low-cost components for large-scale deployment
\end{itemize}

\textbf{Transceiver Architecture:}

\begin{lstlisting}
┌───────────┐    ┌──────────┐    ┌─────────┐    ┌─────────┐
│   MCU     │◄──►│   RF     │◄──►│  PA/LNA │◄──►│ Antenna │
│           │    │ Frontend │    │         │    │         │
└───────────┘    └──────────┘    └─────────┘    └─────────┘
\end{lstlisting}

\textbf{Trade-offs Table:}

{\def\LTcaptype{none} % do not increment counter
\begin{longtable}[]{@{}lll@{}}
\toprule\noalign{}
Parameter & High Performance & Low Power \\
\midrule\noalign{}
\endhead
\bottomrule\noalign{}
\endlastfoot
Range & Long (1km) & Short (100m) \\
Power & High (100mW) & Low (1mW) \\
Cost & Expensive & Cheap \\
\end{longtable}
}

\end{solutionbox}
\begin{mnemonicbox}
``PCRFMAC - Power Communication Range Frequency
Modulation Antenna Cost''

\end{mnemonicbox}
\subsection*{Question 1(c) OR [7
marks]}\label{q1c}

\textbf{Explain optimization goals and figures of merit in WSN.}

\begin{solutionbox}

\textbf{Optimization Goals:}

\begin{itemize}
\tightlist
\item
  \textbf{Energy Efficiency}: Maximize network lifetime by minimizing
  power consumption
\item
  \textbf{Coverage}: Ensure complete area monitoring with minimum sensor
  nodes
\item
  \textbf{Connectivity}: Maintain network connectivity even with node
  failures
\item
  \textbf{Data Quality}: High accuracy and reliability of collected data
\item
  \textbf{Scalability}: Support large number of nodes (100-10000)
\item
  \textbf{Cost Effectiveness}: Minimize deployment and maintenance costs
\end{itemize}

\textbf{Figures of Merit Table:}

{\def\LTcaptype{none} % do not increment counter
\begin{longtable}[]{@{}lll@{}}
\toprule\noalign{}
Metric & Description & Typical Value \\
\midrule\noalign{}
\endhead
\bottomrule\noalign{}
\endlastfoot
\textbf{Network Lifetime} & Time until first node dies & 1-5 years \\
\textbf{Coverage Ratio} & Area covered/Total area & \textgreater95\% \\
\textbf{Connectivity} & Connected nodes/Total nodes &
\textgreater90\% \\
\textbf{Latency} & End-to-end delay & \textless1 second \\
\textbf{Throughput} & Data rate per node & 1-100 kbps \\
\end{longtable}
}

\textbf{Optimization Techniques:}

\begin{itemize}
\tightlist
\item
  \textbf{Clustering}: Reduce communication overhead
\item
  \textbf{Data Aggregation}: Minimize redundant transmissions\\
\item
  \textbf{Sleep Scheduling}: Turn off nodes when not needed
\end{itemize}

\end{solutionbox}
\begin{mnemonicbox}
``ECCDC - Energy Coverage Connectivity Data Cost''

\end{mnemonicbox}
\subsection*{Question 2(a) [3 marks]}\label{q2a}

\textbf{List the characteristics of Sensor MAC protocol in WSNs.}

\begin{solutionbox}

\textbf{S-MAC Protocol Characteristics:}

{\def\LTcaptype{none} % do not increment counter
\begin{longtable}[]{@{}
  >{\raggedright\arraybackslash}p{(\linewidth - 2\tabcolsep) * \real{0.5517}}
  >{\raggedright\arraybackslash}p{(\linewidth - 2\tabcolsep) * \real{0.4483}}@{}}
\toprule\noalign{}
\begin{minipage}[b]{\linewidth}\raggedright
Characteristic
\end{minipage} & \begin{minipage}[b]{\linewidth}\raggedright
Description
\end{minipage} \\
\midrule\noalign{}
\endhead
\bottomrule\noalign{}
\endlastfoot
\textbf{Duty Cycling} & Periodic sleep and wake-up cycles \\
\textbf{Collision Avoidance} & RTS/CTS mechanism \\
\textbf{Overhearing Avoidance} & Nodes sleep during irrelevant
transmissions \\
\textbf{Message Passing} & Long messages broken into fragments \\
\end{longtable}
}

\end{solutionbox}
\begin{mnemonicbox}
``DCOM - Duty Collision Overhearing Message''

\end{mnemonicbox}
\subsection*{Question 2(b) [4 marks]}\label{q2b}

\textbf{Describe the concept of energy-efficient routing in WSNs.}

\begin{solutionbox}

\textbf{Energy-Efficient Routing Concept:}

Energy-efficient routing minimizes power consumption while maintaining
network connectivity and data delivery.

\textbf{Key Techniques:}

\begin{itemize}
\tightlist
\item
  \textbf{Multi-hop Communication}: Short hops consume less power than
  long hops
\item
  \textbf{Load Balancing}: Distribute traffic to avoid node depletion
\item
  \textbf{Data Aggregation}: Combine data from multiple sources
\item
  \textbf{Geographic Routing}: Use location information for efficient
  paths
\end{itemize}

\textbf{Energy Model:}

\begin{lstlisting}
E_tx = E_elec \times k + ε_amp \times k \times d^{2}
E_rx = E_elec \times k
\end{lstlisting}

\textbf{Routing Strategies Table:}

{\def\LTcaptype{none} % do not increment counter
\begin{longtable}[]{@{}lll@{}}
\toprule\noalign{}
Strategy & Power Saving & Implementation \\
\midrule\noalign{}
\endhead
\bottomrule\noalign{}
\endlastfoot
\textbf{Shortest Path} & Medium & Simple \\
\textbf{Min-Energy} & High & Complex \\
\textbf{Max-Lifetime} & Very High & Very Complex \\
\end{longtable}
}

\end{solutionbox}
\begin{mnemonicbox}
``MLDG - Multi-hop Load Data Geographic''

\end{mnemonicbox}
\subsection*{Question 2(c) [7 marks]}\label{q2c}

\textbf{Explain the classification of MAC protocols for WSNs with
examples.}

\begin{solutionbox}

\textbf{MAC Protocol Classification:}

\includegraphics[width=1\linewidth,height=\textheight,keepaspectratio]{mermaid-1829af38.pdf}

\textbf{Detailed Classification:}

\textbf{1. Contention-Based Protocols:}

\begin{itemize}
\tightlist
\item
  \textbf{CSMA/CA}: Carrier sensing before transmission
\item
  \textbf{S-MAC}: Synchronized duty cycles with sleep schedules
\item
  \textbf{T-MAC}: Adaptive duty cycle based on traffic
\end{itemize}

\textbf{2. Schedule-Based Protocols:}

\begin{itemize}
\tightlist
\item
  \textbf{TDMA}: Time slots allocated to nodes
\item
  \textbf{LEACH}: Cluster-based with rotating cluster heads
\item
  \textbf{TRAMA}: Traffic-adaptive medium access
\end{itemize}

\textbf{3. Hybrid Protocols:}

\begin{itemize}
\tightlist
\item
  \textbf{Z-MAC}: Combines CSMA and TDMA benefits
\item
  \textbf{Funneling-MAC}: Different protocols for different network
  regions
\end{itemize}

\textbf{Comparison Table:}

{\def\LTcaptype{none} % do not increment counter
\begin{longtable}[]{@{}llll@{}}
\toprule\noalign{}
Protocol Type & Energy Efficiency & Latency & Scalability \\
\midrule\noalign{}
\endhead
\bottomrule\noalign{}
\endlastfoot
\textbf{Contention} & Medium & Low & High \\
\textbf{Schedule} & High & Medium & Medium \\
\textbf{Hybrid} & High & Low & High \\
\end{longtable}
}

\end{solutionbox}
\begin{mnemonicbox}
``CSH - Contention Schedule Hybrid''

\end{mnemonicbox}
\subsection*{Question 2(a) OR [3
marks]}\label{q2a}

\textbf{State the purpose of address management in WSNs.}

\begin{solutionbox}

\textbf{Address Management Purpose:}

{\def\LTcaptype{none} % do not increment counter
\begin{longtable}[]{@{}ll@{}}
\toprule\noalign{}
Purpose & Description \\
\midrule\noalign{}
\endhead
\bottomrule\noalign{}
\endlastfoot
\textbf{Node Identification} & Unique identification of each sensor
node \\
\textbf{Routing Support} & Enable efficient data forwarding \\
\textbf{Network Organization} & Hierarchical addressing for
scalability \\
\end{longtable}
}

\end{solutionbox}
\begin{mnemonicbox}
``NIR - Node Identification Routing''

\end{mnemonicbox}
\subsection*{Question 2(b) OR [4
marks]}\label{q2b}

\textbf{Explain geographic routing in Detail.}

\begin{solutionbox}

\textbf{Geographic Routing:}

Geographic routing uses physical location information to make forwarding
decisions without maintaining routing tables.

\textbf{Key Components:}

\begin{itemize}
\tightlist
\item
  \textbf{Location Service}: GPS or localization algorithms
\item
  \textbf{Greedy Forwarding}: Forward to neighbor closest to destination
\item
  \textbf{Face Routing}: Handle local minima situations
\item
  \textbf{Coordinate System}: 2D/3D positioning
\end{itemize}

\textbf{Forwarding Algorithm:}

\begin{lstlisting}
1. Receive packet with destination coordinates
2. Find neighbor closest to destination
3. If closer than current node, forward
4. Else use face routing or drop
\end{lstlisting}

\textbf{Advantages/Disadvantages:}

{\def\LTcaptype{none} % do not increment counter
\begin{longtable}[]{@{}lll@{}}
\toprule\noalign{}
Aspect & Advantage & Disadvantage \\
\midrule\noalign{}
\endhead
\bottomrule\noalign{}
\endlastfoot
\textbf{Scalability} & No routing tables & Location overhead \\
\textbf{Adaptability} & Handles mobility & Local minima problem \\
\end{longtable}
}

\end{solutionbox}
\begin{mnemonicbox}
``LGFC - Location Greedy Face Coordinate''

\end{mnemonicbox}
\subsection*{Question 2(c) OR [7
marks]}\label{q2c}

\textbf{Explain the working of the LEACH protocol in WSN.}

\begin{solutionbox}

\textbf{LEACH Protocol (Low-Energy Adaptive Clustering Hierarchy):}

\textbf{Protocol Phases:}

\includegraphics[width=1\linewidth,height=\textheight,keepaspectratio]{mermaid-8a28d90a.pdf}

\textbf{Detailed Working:}

\textbf{1. Setup Phase:}

\begin{itemize}
\tightlist
\item
  \textbf{Cluster Head Selection}: Nodes decide to become cluster heads
  based on probability
\item
  \textbf{Advertisement}: Cluster heads broadcast advertisement messages
\item
  \textbf{Cluster Formation}: Non-cluster head nodes join nearest
  cluster head
\item
  \textbf{Schedule Creation}: TDMA schedule created for cluster members
\end{itemize}

\textbf{2. Steady State Phase:}

\begin{itemize}
\tightlist
\item
  \textbf{Data Collection}: Cluster members collect and send data to
  cluster head
\item
  \textbf{Data Aggregation}: Cluster head aggregates received data
\item
  \textbf{Data Transmission}: Aggregated data sent to base station
\end{itemize}

\textbf{Cluster Head Selection Formula:}

\begin{lstlisting}
P(n) = k / (N - k \times (r mod N/k))
\end{lstlisting}

Where:

k = desired cluster heads,

N = total nodes,

r = current round


\textbf{Energy Benefits:}

\begin{itemize}
\tightlist
\item
  \textbf{Load Distribution}: Cluster head role rotates among nodes
\item
  \textbf{Data Aggregation}: Reduces transmissions to base station
\item
  \textbf{Short Range Communication}: Most transmissions are within
  cluster
\end{itemize}

\textbf{Performance Metrics:}

{\def\LTcaptype{none} % do not increment counter
\begin{longtable}[]{@{}lll@{}}
\toprule\noalign{}
Metric & LEACH & Direct Transmission \\
\midrule\noalign{}
\endhead
\bottomrule\noalign{}
\endlastfoot
\textbf{Network Lifetime} & 8x longer & Baseline \\
\textbf{Energy Distribution} & Uniform & Uneven \\
\textbf{Scalability} & High & Low \\
\end{longtable}
}

\end{solutionbox}
\begin{mnemonicbox}
``SSCADT - Setup Steady Cluster Aggregation Data
Transmission''

\end{mnemonicbox}
\subsection*{Question 3(a) [3 marks]}\label{q3a}

\textbf{Define IoT and mention its key sources.}

\begin{solutionbox}

\textbf{IoT Definition}: Internet of Things is a network of
interconnected physical devices embedded with sensors, software, and
connectivity to collect and exchange data.

\textbf{Key Sources Table:}

{\def\LTcaptype{none} % do not increment counter
\begin{longtable}[]{@{}
  >{\raggedright\arraybackslash}p{(\linewidth - 2\tabcolsep) * \real{0.3810}}
  >{\raggedright\arraybackslash}p{(\linewidth - 2\tabcolsep) * \real{0.6190}}@{}}
\toprule\noalign{}
\begin{minipage}[b]{\linewidth}\raggedright
Source
\end{minipage} & \begin{minipage}[b]{\linewidth}\raggedright
Description
\end{minipage} \\
\midrule\noalign{}
\endhead
\bottomrule\noalign{}
\endlastfoot
\textbf{RFID Technology} & Radio frequency identification for object
tracking \\
\textbf{Sensor Networks} & WSNs and environmental monitoring systems \\
\textbf{Mobile Computing} & Smartphones and portable devices \\
\textbf{Cloud Computing} & Scalable data storage and processing \\
\end{longtable}
}

\end{solutionbox}
\begin{mnemonicbox}
``RSMC - RFID Sensor Mobile Cloud''

\end{mnemonicbox}
\subsection*{Question 3(b) [4 marks]}\label{q3b}

\textbf{Explain the modified OSI model for IoT/M2M systems.}

\begin{solutionbox}

\textbf{Modified OSI Model for IoT:}

{\def\LTcaptype{none} % do not increment counter
\begin{longtable}[]{@{}
  >{\raggedright\arraybackslash}p{(\linewidth - 4\tabcolsep) * \real{0.1591}}
  >{\raggedright\arraybackslash}p{(\linewidth - 4\tabcolsep) * \real{0.3636}}
  >{\raggedright\arraybackslash}p{(\linewidth - 4\tabcolsep) * \real{0.4773}}@{}}
\toprule\noalign{}
\begin{minipage}[b]{\linewidth}\raggedright
Layer
\end{minipage} & \begin{minipage}[b]{\linewidth}\raggedright
Traditional OSI
\end{minipage} & \begin{minipage}[b]{\linewidth}\raggedright
IoT/M2M Modification
\end{minipage} \\
\midrule\noalign{}
\endhead
\bottomrule\noalign{}
\endlastfoot
\textbf{Application} & End-user applications & IoT applications, data
analytics \\
\textbf{Presentation} & Data formatting & Data aggregation, semantic
processing \\
\textbf{Session} & Session management & Device management, security \\
\textbf{Transport} & End-to-end delivery & Reliable/unreliable delivery
(UDP/TCP) \\
\textbf{Network} & Routing & IPv6, 6LoWPAN, RPL routing \\
\textbf{Data Link} & Frame delivery & IEEE 802.15.4, WiFi, Bluetooth \\
\textbf{Physical} & Bit transmission & Radio, optical, wired
transmission \\
\end{longtable}
}

\textbf{IoT-Specific Modifications:}

\begin{itemize}
\tightlist
\item
  \textbf{6LoWPAN}: IPv6 over Low-Power Wireless Personal Area Networks
\item
  \textbf{CoAP}: Constrained Application Protocol for resource-limited
  devices
\item
  \textbf{MQTT}: Message Queuing Telemetry Transport for lightweight
  communication
\end{itemize}

\textbf{Protocol Stack Example:}

\begin{lstlisting}
┌─────────────────┐
│  IoT Application│
├─────────────────┤
│   CoAP/MQTT     │
├─────────────────┤
│      UDP        │
├─────────────────┤
│    6LoWPAN      │
├─────────────────┤
│  IEEE 802.15.4  │
└─────────────────┘
\end{lstlisting}

\end{solutionbox}
\begin{mnemonicbox}
``Six-Layer Low-Power WAN - 6LoWPAN''

\end{mnemonicbox}
\subsection*{Question 3(c) [7 marks]}\label{q3c}

\textbf{Discuss the major components of an IoT system with a diagram.}

\begin{solutionbox}

\textbf{IoT System Architecture:}

\includegraphics[width=1\linewidth,height=\textheight,keepaspectratio]{mermaid-da34d992.pdf}

\textbf{Major Components:}

\textbf{1. Device Layer:}

\begin{itemize}
\tightlist
\item
  \textbf{Sensors}: Temperature, humidity, motion, light sensors
\item
  \textbf{Actuators}: Motors, relays, valves for control
\item
  \textbf{Microcontrollers}: ESP32, Arduino, Raspberry Pi
\item
  \textbf{Communication Modules}: WiFi, Bluetooth, LoRa, Cellular
\end{itemize}

\textbf{2. Connectivity Layer:}

\begin{itemize}
\tightlist
\item
  \textbf{Gateways}: Protocol translation and data aggregation
\item
  \textbf{Network Infrastructure}: Internet, cellular, satellite
\item
  \textbf{Communication Protocols}: HTTP, MQTT, CoAP, WebSocket
\end{itemize}

\textbf{3. Data Processing Layer:}

\begin{itemize}
\tightlist
\item
  \textbf{Cloud Platforms}: AWS IoT, Azure IoT, Google Cloud IoT
\item
  \textbf{Edge Computing}: Local data processing and filtering
\item
  \textbf{Data Storage}: Time-series databases, NoSQL databases
\end{itemize}

\textbf{4. Application Layer:}

\begin{itemize}
\tightlist
\item
  \textbf{Analytics Engine}: Real-time and batch processing
\item
  \textbf{Machine Learning}: Predictive analytics and pattern
  recognition
\item
  \textbf{APIs}: RESTful services for data access
\end{itemize}

\textbf{5. Business Layer:}

\begin{itemize}
\tightlist
\item
  \textbf{User Interfaces}: Web dashboards, mobile apps
\item
  \textbf{Business Logic}: Rules engines and workflow management
\item
  \textbf{Integration}: ERP, CRM system integration
\end{itemize}

\textbf{Component Functions Table:}

{\def\LTcaptype{none} % do not increment counter
\begin{longtable}[]{@{}llll@{}}
\toprule\noalign{}
Component & Input & Processing & Output \\
\midrule\noalign{}
\endhead
\bottomrule\noalign{}
\endlastfoot
\textbf{Sensors} & Physical parameters & Analog to digital & Digital
data \\
\textbf{Gateway} & Sensor data & Protocol conversion & Network
packets \\
\textbf{Cloud} & Raw data & Storage and analytics & Processed
information \\
\textbf{Applications} & Processed data & Business logic & User
actions \\
\end{longtable}
}

\textbf{Data Flow:}

\begin{lstlisting}
Sensors \rightarrow Gateway \rightarrow Internet \rightarrow Cloud \rightarrow Analytics \rightarrow Applications \rightarrow Users
\end{lstlisting}

\end{solutionbox}
\begin{mnemonicbox}
``DCDA-B - Device Connectivity Data Application
Business''

\end{mnemonicbox}
\subsection*{Question 3(a) OR [3
marks]}\label{q3a}

\textbf{List three challenges of IoT implementation.}

\begin{solutionbox}

\textbf{IoT Implementation Challenges:}

{\def\LTcaptype{none} % do not increment counter
\begin{longtable}[]{@{}ll@{}}
\toprule\noalign{}
Challenge & Description \\
\midrule\noalign{}
\endhead
\bottomrule\noalign{}
\endlastfoot
\textbf{Security and Privacy} & Protecting data and device access \\
\textbf{Interoperability} & Different protocols and standards \\
\textbf{Scalability} & Managing millions of connected devices \\
\end{longtable}
}

\end{solutionbox}
\begin{mnemonicbox}
``SIS - Security Interoperability Scalability''

\end{mnemonicbox}
\subsection*{Question 3(b) OR [4
marks]}\label{q3b}

\textbf{Describe the technology behind IoT with examples.}

\begin{solutionbox}

\textbf{Core Technologies:}

\textbf{1. Sensing Technology:}

\begin{itemize}
\tightlist
\item
  \textbf{MEMS Sensors}: Accelerometers, gyroscopes
\item
  \textbf{Environmental Sensors}: Temperature, humidity (DHT22)
\item
  \textbf{Biometric Sensors}: Heart rate, fingerprint
\item
  \textbf{Example}: Smart thermostat using temperature sensors
\end{itemize}

\textbf{2. Communication Technology:}

\begin{itemize}
\tightlist
\item
  \textbf{Short Range}: Bluetooth, WiFi, Zigbee
\item
  \textbf{Long Range}: LoRaWAN, Cellular (4G/5G), Satellite
\item
  \textbf{Example}: Smart home using WiFi for local control
\end{itemize}

\textbf{3. Computing Technology:}

\begin{itemize}
\tightlist
\item
  \textbf{Microcontrollers}: ESP32, Arduino Uno
\item
  \textbf{Single Board Computers}: Raspberry Pi
\item
  \textbf{Example}: Smart irrigation using NodeMCU
\end{itemize}

\textbf{4. Cloud Technology:}

\begin{itemize}
\tightlist
\item
  \textbf{Platforms}: AWS IoT Core, Microsoft Azure IoT
\item
  \textbf{Services}: Data analytics, machine learning
\item
  \textbf{Example}: Industrial monitoring using AWS IoT
\end{itemize}

\textbf{Technology Stack Example:}

\begin{lstlisting}
┌─────────────────┐
│   Cloud (AWS)   │
├─────────────────┤
│  WiFi Network   │
├─────────────────┤
│    ESP32 MCU    │
├─────────────────┤
│  DHT22 Sensor   │
└─────────────────┘
\end{lstlisting}

\end{solutionbox}
\begin{mnemonicbox}
``SCCC - Sensing Communication Computing Cloud''

\end{mnemonicbox}
\subsection*{Question 3(c) OR [7
marks]}\label{q3c}

\textbf{Explain the role of M2M communication in IoT with an example
application.}

\begin{solutionbox}

\textbf{M2M Communication in IoT:}

Machine-to-Machine (M2M) communication enables automated data exchange
between devices without human intervention.

\textbf{Key Characteristics:}

\begin{itemize}
\tightlist
\item
  \textbf{Autonomous Operation}: Devices communicate without human input
\item
  \textbf{Real-time Response}: Immediate action based on data exchange
\item
  \textbf{Scalable Architecture}: Support for thousands of connected
  devices
\item
  \textbf{Reliable Communication}: Guaranteed message delivery
\end{itemize}

\textbf{M2M Architecture:}

\includegraphics[width=1\linewidth,height=\textheight,keepaspectratio]{mermaid-996dde7c.pdf}

\textbf{Communication Protocols:}

\begin{itemize}
\tightlist
\item
  \textbf{MQTT}: Lightweight publish-subscribe messaging
\item
  \textbf{CoAP}: Constrained Application Protocol for limited devices
\item
  \textbf{HTTP/REST}: Web-based communication
\item
  \textbf{WebSocket}: Real-time bidirectional communication
\end{itemize}

\textbf{Example Application: Smart Street Lighting System}

\textbf{System Components:}

\begin{itemize}
\tightlist
\item
  \textbf{Smart LED Lights}: Individual controllable street lights
\item
  \textbf{Motion Sensors}: Detect pedestrian and vehicle movement
\item
  \textbf{Light Sensors}: Measure ambient light levels
\item
  \textbf{Central Controller}: Manages entire lighting network
\end{itemize}

\textbf{M2M Communication Flow:}

\begin{lstlisting}
1. Motion sensor detects movement
2. Sensor sends data to nearby lights via Zigbee
3. Lights communicate with each other to create "lighting path"
4. Lights automatically adjust brightness based on traffic
5. Usage data sent to central controller via cellular
6. Controller optimizes lighting schedules
\end{lstlisting}

\textbf{M2M Benefits in this Application:}

\begin{itemize}
\tightlist
\item
  \textbf{Energy Efficiency}: Lights dim when no activity detected
\item
  \textbf{Predictive Maintenance}: Lights report their health status
\item
  \textbf{Adaptive Control}: System learns traffic patterns
\item
  \textbf{Cost Reduction}: 60\% energy savings compared to traditional
  lighting
\end{itemize}

\textbf{Communication Protocol Stack:}

\begin{lstlisting}
┌─────────────────┐
│   Application   │ \leftarrow Smart Lighting Control
├─────────────────┤
│      MQTT       │ \leftarrow Message Protocol
├─────────────────┤
│      TCP        │ \leftarrow Transport Layer
├─────────────────┤
│   Cellular/WiFi │ \leftarrow Network Layer
└─────────────────┘
\end{lstlisting}

\textbf{Performance Metrics:}

{\def\LTcaptype{none} % do not increment counter
\begin{longtable}[]{@{}lll@{}}
\toprule\noalign{}
Metric & Traditional & M2M Smart System \\
\midrule\noalign{}
\endhead
\bottomrule\noalign{}
\endlastfoot
\textbf{Energy Consumption} & 100\% & 40\% \\
\textbf{Maintenance Cost} & High & Low (predictive) \\
\textbf{Response Time} & Manual (hours) & Automatic (seconds) \\
\textbf{Flexibility} & Fixed schedule & Adaptive \\
\end{longtable}
}

\end{solutionbox}
\begin{mnemonicbox}
``ARSR - Autonomous Real-time Scalable Reliable''

\end{mnemonicbox}
\subsection*{Question 4(a) [3 marks]}\label{q4a}

\textbf{Name three application layer protocols used in IoT.}

\begin{solutionbox}

\textbf{IoT Application Layer Protocols:}

{\def\LTcaptype{none} % do not increment counter
\begin{longtable}[]{@{}
  >{\raggedright\arraybackslash}p{(\linewidth - 2\tabcolsep) * \real{0.5263}}
  >{\raggedright\arraybackslash}p{(\linewidth - 2\tabcolsep) * \real{0.4737}}@{}}
\toprule\noalign{}
\begin{minipage}[b]{\linewidth}\raggedright
Protocol
\end{minipage} & \begin{minipage}[b]{\linewidth}\raggedright
Purpose
\end{minipage} \\
\midrule\noalign{}
\endhead
\bottomrule\noalign{}
\endlastfoot
\textbf{MQTT} & Lightweight publish-subscribe messaging \\
\textbf{CoAP} & Constrained Application Protocol for resource-limited
devices \\
\textbf{HTTP/HTTPS} & Web-based RESTful communication \\
\end{longtable}
}

\end{solutionbox}
\begin{mnemonicbox}
``MCH - MQTT CoAP HTTP''

\end{mnemonicbox}
\subsection*{Question 4(b) [4 marks]}\label{q4b}

\textbf{Explain the role of MQTT in IoT systems.}

\begin{solutionbox}

\textbf{MQTT (Message Queuing Telemetry Transport) Role:}

MQTT is a lightweight publish-subscribe messaging protocol designed for
IoT devices with limited resources.

\textbf{Key Features:}

\begin{itemize}
\tightlist
\item
  \textbf{Publish-Subscribe Model}: Decoupled communication between
  devices
\item
  \textbf{Quality of Service}: Three levels (0, 1, 2) for message
  delivery
\item
  \textbf{Persistent Sessions}: Maintains connection state
\item
  \textbf{Last Will Testament}: Automatic notification when device
  disconnects
\end{itemize}

\textbf{MQTT Architecture:}

\begin{lstlisting}
┌─────────────┐    ┌─────────────┐    ┌─────────────┐
│  Publisher  │───▶│   Broker    │◄───│ Subscriber  │
│  (Sensor)   │    │  (Server)   │    │ (Display)   │
└─────────────┘    └─────────────┘    └─────────────┘
\end{lstlisting}

\textbf{QoS Levels:}

{\def\LTcaptype{none} % do not increment counter
\begin{longtable}[]{@{}lll@{}}
\toprule\noalign{}
Level & Description & Use Case \\
\midrule\noalign{}
\endhead
\bottomrule\noalign{}
\endlastfoot
\textbf{QoS 0} & At most once delivery & Non-critical data \\
\textbf{QoS 1} & At least once delivery & Important data \\
\textbf{QoS 2} & Exactly once delivery & Critical commands \\
\end{longtable}
}

\textbf{Benefits in IoT:}

\begin{itemize}
\tightlist
\item
  \textbf{Low Bandwidth}: Minimal protocol overhead
\item
  \textbf{Battery Efficient}: Optimized for low-power devices
\item
  \textbf{Scalable}: Supports thousands of concurrent connections
\end{itemize}

\end{solutionbox}
\begin{mnemonicbox}
``PQPL - Publish QoS Persistent Last-will''

\end{mnemonicbox}
\subsection*{Question 4(c) [7 marks]}\label{q4c}

\textbf{Design a system to read temperature sensor data using NodeMCU
and transmit it to a cloud platform.}

\begin{solutionbox}

\textbf{System Design: Temperature Monitoring System}

\textbf{System Architecture:}

\includegraphics[width=1\linewidth,height=\textheight,keepaspectratio]{mermaid-a3938936.pdf}

\textbf{Hardware Components:}

\begin{itemize}
\tightlist
\item
  \textbf{NodeMCU ESP8266}: Microcontroller with WiFi capability
\item
  \textbf{DHT22 Sensor}: Digital temperature and humidity sensor
\item
  \textbf{Breadboard and Jumper Wires}: For connections
\item
  \textbf{Power Supply}: USB or external 5V supply
\end{itemize}

\textbf{Circuit Diagram:}

\begin{lstlisting}
NodeMCU ESP8266        DHT22 Sensor
┌─────────────┐       ┌─────────────┐
│    3.3V     │──────▶│     VCC     │
│     GND     │──────▶│     GND     │
│     D4      │──────▶│    DATA     │
└─────────────┘       └─────────────┘
\end{lstlisting}

\textbf{Software Implementation:}

\textbf{Arduino Code (Simplified):}

\begin{lstlisting}[language={C++}]
#include <ESP8266WiFi.h>
#include <DHT.h>
#include <PubSubClient.h>

#define DHT_PIN D4
#define DHT_TYPE DHT22

DHT dht(DHT_PIN, DHT_TYPE);
WiFiClient espClient;
PubSubClient client(espClient);

void setup() {
  Serial.begin(115200);
  dht.begin();
  WiFi.begin("SSID", "PASSWORD");
  client.setServer("mqtt.broker.com", 1883);
}

void loop() {
  float temp = dht.readTemperature();
  float hum = dht.readHumidity();
  
  String payload = "{\"temperature\":" + String(temp) + 
                   ",\"humidity\":" + String(hum) + "}";
  
  client.publish("sensor/data", payload.c_str());
  delay(30000); // Send every 30 seconds
}
\end{lstlisting}

\textbf{Cloud Platform Setup (AWS IoT):}

\begin{enumerate}
\tightlist
\item
  \textbf{Device Registration}: Create IoT device certificate
\item
  \textbf{Topic Configuration}: Set up MQTT topics for data
\item
  \textbf{Rules Engine}: Process and route incoming data\\
\item
  \textbf{Database Storage}: Store data in DynamoDB/TimeStream
\item
  \textbf{API Gateway}: Create REST APIs for data access
\end{enumerate}

\textbf{Data Flow:}

\begin{lstlisting}
DHT22 \rightarrow NodeMCU \rightarrow WiFi \rightarrow Internet \rightarrow AWS IoT \rightarrow Database \rightarrow Dashboard
\end{lstlisting}

\textbf{System Features:}

\begin{itemize}
\tightlist
\item
  \textbf{Real-time Monitoring}: Temperature data every 30 seconds
\item
  \textbf{Historical Data}: Store data for trend analysis
\item
  \textbf{Alerts}: Email/SMS when temperature exceeds thresholds
\item
  \textbf{Remote Access}: View data from anywhere via web/mobile
\end{itemize}

\textbf{Performance Specifications:}

{\def\LTcaptype{none} % do not increment counter
\begin{longtable}[]{@{}ll@{}}
\toprule\noalign{}
Parameter & Specification \\
\midrule\noalign{}
\endhead
\bottomrule\noalign{}
\endlastfoot
\textbf{Accuracy} & \pm0.5^\circC temperature, \pm2\% humidity \\
\textbf{Range} & -40^\circC to 80^\circC \\
\textbf{Update Rate} & 30 seconds \\
\textbf{Power Consumption} & 70mA active, 20µA deep sleep \\
\textbf{WiFi Range} & 50-100 meters \\
\end{longtable}
}

\textbf{Cost Analysis:}

{\def\LTcaptype{none} % do not increment counter
\begin{longtable}[]{@{}ll@{}}
\toprule\noalign{}
Component & Cost (USD) \\
\midrule\noalign{}
\endhead
\bottomrule\noalign{}
\endlastfoot
\textbf{NodeMCU ESP8266} & \$3 \\
\textbf{DHT22 Sensor} & \$5 \\
\textbf{Miscellaneous} & \$2 \\
\textbf{Total Hardware} & \$10 \\
\textbf{Cloud Service} & \$5/month \\
\end{longtable}
}

\end{solutionbox}
\begin{mnemonicbox}
``HSCDP - Hardware Software Cloud Data Platform''

\end{mnemonicbox}
\subsection*{Question 4(a) OR [3
marks]}\label{q4a}

\textbf{List the types of sensors used in IoT applications.}

\begin{solutionbox}

\textbf{IoT Sensor Types:}

{\def\LTcaptype{none} % do not increment counter
\begin{longtable}[]{@{}ll@{}}
\toprule\noalign{}
Sensor Type & Measurement \\
\midrule\noalign{}
\endhead
\bottomrule\noalign{}
\endlastfoot
\textbf{Temperature} & Ambient and surface temperature \\
\textbf{Motion/PIR} & Movement and presence detection \\
\textbf{Light/LDR} & Ambient light intensity \\
\end{longtable}
}

\end{solutionbox}
\begin{mnemonicbox}
``TML - Temperature Motion Light''

\end{mnemonicbox}
\subsection*{Question 4(b) OR [4
marks]}\label{q4b}

\textbf{Discuss the security challenges in IoT systems.}

\begin{solutionbox}

\textbf{IoT Security Challenges:}

\textbf{1. Device-Level Security:}

\begin{itemize}
\tightlist
\item
  \textbf{Weak Authentication}: Default passwords and poor access
  control
\item
  \textbf{Firmware Vulnerabilities}: Unpatched security flaws
\item
  \textbf{Physical Security}: Device tampering and theft
\item
  \textbf{Resource Constraints}: Limited processing power for encryption
\end{itemize}

\textbf{2. Network-Level Security:}

\begin{itemize}
\tightlist
\item
  \textbf{Data Transmission}: Unencrypted communication channels
\item
  \textbf{Network Protocols}: Vulnerabilities in wireless protocols
\item
  \textbf{Man-in-the-Middle}: Interception of communication
\item
  \textbf{DDoS Attacks}: Overwhelming network infrastructure
\end{itemize}

\textbf{3. Cloud-Level Security:}

\begin{itemize}
\tightlist
\item
  \textbf{Data Privacy}: Unauthorized access to stored data
\item
  \textbf{API Security}: Vulnerabilities in application interfaces
\item
  \textbf{Identity Management}: Poor user authentication and
  authorization
\item
  \textbf{Data Breaches}: Large-scale data theft
\end{itemize}

\textbf{Security Solutions Table:}

{\def\LTcaptype{none} % do not increment counter
\begin{longtable}[]{@{}
  >{\raggedright\arraybackslash}p{(\linewidth - 2\tabcolsep) * \real{0.5238}}
  >{\raggedright\arraybackslash}p{(\linewidth - 2\tabcolsep) * \real{0.4762}}@{}}
\toprule\noalign{}
\begin{minipage}[b]{\linewidth}\raggedright
Challenge
\end{minipage} & \begin{minipage}[b]{\linewidth}\raggedright
Solution
\end{minipage} \\
\midrule\noalign{}
\endhead
\bottomrule\noalign{}
\endlastfoot
\textbf{Weak Authentication} & Strong passwords, multi-factor
authentication \\
\textbf{Data Transmission} & End-to-end encryption (TLS/SSL) \\
\textbf{Firmware Updates} & Secure OTA update mechanisms \\
\textbf{Access Control} & Role-based permissions \\
\end{longtable}
}

\end{solutionbox}
\begin{mnemonicbox}
``DNCI - Device Network Cloud Identity''

\end{mnemonicbox}
\subsection*{Question 4(c) OR [7
marks]}\label{q4c}

\textbf{Draw a block diagram for controlling a bulb using Raspberry Pi
via a mobile app. Explain the blocks in detail.}

\begin{solutionbox}

\textbf{Smart Bulb Control System:}

\includegraphics[width=1\linewidth,height=\textheight,keepaspectratio]{mermaid-bc3ebf03.pdf}

\textbf{Detailed Block Explanation:}

\textbf{1. Mobile Application:}

\begin{itemize}
\tightlist
\item
  \textbf{Platform}: Android/iOS native app or web app
\item
  \textbf{Interface}: ON/OFF buttons, dimming slider, scheduling
\item
  \textbf{Communication}: HTTP requests to Raspberry Pi web server
\item
  \textbf{Features}: Real-time status, timer controls, voice commands
\end{itemize}

\textbf{2. Internet/WiFi Network:}

\begin{itemize}
\tightlist
\item
  \textbf{Local Network}: Home WiFi router for local control
\item
  \textbf{Internet}: Remote access via port forwarding or VPN
\item
  \textbf{Protocols}: HTTP/HTTPS for web communication
\item
  \textbf{Security}: WPA2/WPA3 encryption
\end{itemize}

\textbf{3. Home Router:}

\begin{itemize}
\tightlist
\item
  \textbf{Function}: Network gateway and DHCP server
\item
  \textbf{Port Forwarding}: External access to Raspberry Pi
\item
  \textbf{Firewall}: Security for home network
\item
  \textbf{QoS}: Traffic prioritization
\end{itemize}

\textbf{4. Raspberry Pi Controller:}

\begin{itemize}
\tightlist
\item
  \textbf{Model}: Raspberry Pi 4B with WiFi capability
\item
  \textbf{OS}: Raspberry Pi OS (Linux-based)
\item
  \textbf{Web Server}: Flask/Apache serving control interface
\item
  \textbf{GPIO Control}: Python libraries for hardware control
\end{itemize}

\textbf{5. Relay Module:}

\begin{itemize}
\tightlist
\item
  \textbf{Type}: 5V single-channel relay module
\item
  \textbf{Function}: Electrical isolation and AC switching
\item
  \textbf{Control Signal}: 3.3V GPIO from Raspberry Pi
\item
  \textbf{Safety}: Optocoupler isolation
\end{itemize}

\textbf{6. AC Bulb:}

\begin{itemize}
\tightlist
\item
  \textbf{Type}: Standard 230V AC incandescent/LED bulb
\item
  \textbf{Power}: Up to 100W capacity
\item
  \textbf{Control}: ON/OFF switching via relay
\item
  \textbf{Connection}: Series connection through relay contacts
\end{itemize}

\textbf{System Operation Flow:}

\begin{lstlisting}
Mobile App          Raspberry Pi         Relay Module        AC Bulb
┌─────────┐        ┌─────────────┐       ┌─────────────┐     ┌─────────┐
│ Tap ON  │───────▶│ Web Server  │──────▶│   GPIO=HIGH │────▶│ Bulb ON │
│         │        │   Process   │       │             │     │         │
│ Tap OFF │───────▶│   Request   │──────▶│   GPIO=LOW  │────▶│ Bulb OFF│
└─────────┘        └─────────────┘       └─────────────┘     └─────────┘
\end{lstlisting}

\textbf{Software Components:}

\textbf{Python Code (Simplified):}

\begin{lstlisting}[language=Python]
import RPi.GPIO as GPIO
from flask import Flask, request, jsonify

app = Flask(__name__)
RELAY_PIN = 18
GPIO.setmode(GPIO.BCM)
GPIO.setup(RELAY_PIN, GPIO.OUT)

@app.route('/bulb/<state>')
def control_bulb(state):
    if state == 'on':
        GPIO.output(RELAY_PIN, GPIO.HIGH)
        return jsonify({'status': 'Bulb ON'})
    elif state == 'off':
        GPIO.output(RELAY_PIN, GPIO.LOW)
        return jsonify({'status': 'Bulb OFF'})

if __name__ == '__main__':
    app.run(host='0.0.0.0', port=5000)
\end{lstlisting}

\textbf{Mobile App Interface:}

\begin{itemize}
\tightlist
\item
  \textbf{Connection}: HTTP requests to Pi's IP address
\item
  \textbf{URL Format}:
  \passthrough{\lstinline!http://192.168.1.100:5000/bulb/on!}
\item
  \textbf{Response}: JSON status confirmation
\item
  \textbf{UI Elements}: Toggle switch, status indicator
\end{itemize}

\textbf{Hardware Connections:}

{\def\LTcaptype{none} % do not increment counter
\begin{longtable}[]{@{}lll@{}}
\toprule\noalign{}
Raspberry Pi & Relay Module & AC Circuit \\
\midrule\noalign{}
\endhead
\bottomrule\noalign{}
\endlastfoot
GPIO 18 & IN & - \\
5V & VCC & - \\
GND & GND & - \\
- & COM & Live Wire \\
- & NO & Bulb Live \\
\end{longtable}
}

\textbf{Safety Considerations:}

\begin{itemize}
\tightlist
\item
  \textbf{Electrical Isolation}: Relay provides galvanic isolation
\item
  \textbf{Proper Wiring}: Follow electrical safety codes
\item
  \textbf{Enclosure}: Protect connections from moisture
\item
  \textbf{Circuit Breaker}: Include in AC circuit for safety
\end{itemize}

\textbf{System Advantages:}

\begin{itemize}
\tightlist
\item
  \textbf{Remote Control}: Access from anywhere with internet
\item
  \textbf{Scheduling}: Automated ON/OFF timers
\item
  \textbf{Energy Monitoring}: Track power consumption
\item
  \textbf{Voice Control}: Integration with Alexa/Google Assistant
\item
  \textbf{Multiple Bulbs}: Expandable to control multiple devices
\end{itemize}

\textbf{Cost Breakdown:}

{\def\LTcaptype{none} % do not increment counter
\begin{longtable}[]{@{}ll@{}}
\toprule\noalign{}
Component & Cost (USD) \\
\midrule\noalign{}
\endhead
\bottomrule\noalign{}
\endlastfoot
\textbf{Raspberry Pi 4B} & \$35 \\
\textbf{Relay Module} & \$3 \\
\textbf{Jumper Wires} & \$2 \\
\textbf{Enclosure} & \$5 \\
\textbf{Total} & \$45 \\
\end{longtable}
}

\end{solutionbox}
\begin{mnemonicbox}
``MIHRBA - Mobile Internet Home-router Raspberry-pi
Relay Bulb''

\end{mnemonicbox}
\subsection*{Question 5(a) [3 marks]}\label{q5a}

\textbf{Classify IoT applications into broad categories.}

\begin{solutionbox}

\textbf{IoT Application Categories:}

{\def\LTcaptype{none} % do not increment counter
\begin{longtable}[]{@{}
  >{\raggedright\arraybackslash}p{(\linewidth - 2\tabcolsep) * \real{0.4348}}
  >{\raggedright\arraybackslash}p{(\linewidth - 2\tabcolsep) * \real{0.5652}}@{}}
\toprule\noalign{}
\begin{minipage}[b]{\linewidth}\raggedright
Category
\end{minipage} & \begin{minipage}[b]{\linewidth}\raggedright
Description
\end{minipage} \\
\midrule\noalign{}
\endhead
\bottomrule\noalign{}
\endlastfoot
\textbf{Consumer IoT} & Smart homes, wearables, entertainment \\
\textbf{Industrial IoT} & Manufacturing, supply chain, predictive
maintenance \\
\textbf{Infrastructure IoT} & Smart cities, transportation, utilities \\
\end{longtable}
}

\end{solutionbox}
\begin{mnemonicbox}
``CII - Consumer Industrial Infrastructure''

\end{mnemonicbox}
\subsection*{Question 5(b) [4 marks]}\label{q5b}

\textbf{Explain the working of a smart home automation system using
IoT.}

\begin{solutionbox}

\textbf{Smart Home Automation System:}

Smart home automation integrates various IoT devices to provide
centralized control and intelligent automation of home functions.

\textbf{System Components:}

\begin{itemize}
\tightlist
\item
  \textbf{Central Hub}: Smart home controller (like Amazon Echo, Google
  Home)
\item
  \textbf{Sensors}: Motion, temperature, light, door/window sensors
\item
  \textbf{Actuators}: Smart switches, thermostats, door locks, cameras
\item
  \textbf{Communication}: WiFi, Zigbee, Z-Wave protocols
\end{itemize}

\textbf{Working Principle:}

\begin{lstlisting}
┌─────────────┐    ┌─────────────┐    ┌─────────────┐
│   Sensors   │───▶│ Central Hub │───▶│  Actuators  │
│  (Input)    │    │ (Process)   │    │  (Output)   │
└─────────────┘    └─────────────┘    └─────────────┘
        ▲                 │                 │
        │                 ▼                 │
┌─────────────┐    ┌─────────────┐          │
│ Mobile App  │◄───│    Cloud    │          │
│   Control   │    │  Services   │          │
└─────────────┘    └─────────────┘          │
        ▲                                   │
        └───────────────────────────────────┘
\end{lstlisting}

\textbf{Automation Examples:}

\begin{itemize}
\tightlist
\item
  \textbf{Security}: Motion sensors trigger lights and cameras
\item
  \textbf{Energy Management}: Temperature sensors control HVAC systems
\item
  \textbf{Convenience}: Voice commands control multiple devices
\item
  \textbf{Safety}: Smoke detectors trigger alarms and notifications
\end{itemize}

\textbf{Benefits:}

\begin{itemize}
\tightlist
\item
  \textbf{Energy Efficiency}: 20-30\% reduction in power consumption
\item
  \textbf{Security}: Real-time monitoring and alerts
\item
  \textbf{Convenience}: Remote control and automation
\item
  \textbf{Cost Savings}: Reduced utility bills and insurance premiums
\end{itemize}

\end{solutionbox}
\begin{mnemonicbox}
``HCSA - Hub Communication Sensors Actuators''

\end{mnemonicbox}
\subsection*{Question 5(c) [7 marks]}\label{q5c}

\textbf{Propose a block diagram and working principle for an IoT-based
healthcare monitoring system.}

\begin{solutionbox}

\textbf{IoT Healthcare Monitoring System:}

\textbf{System Architecture:}

\includegraphics[width=1\linewidth,height=\textheight,keepaspectratio]{mermaid-fa5c5cb7.pdf}

\textbf{Detailed Components:}

\textbf{1. Patient-Side Devices:}

\textbf{Wearable Sensors:}

\begin{itemize}
\tightlist
\item
  \textbf{Smartwatch}: Heart rate, activity tracking, ECG
\item
  \textbf{Fitness Bands}: Steps, sleep patterns, calories
\item
  \textbf{Smart Patches}: Continuous glucose monitoring, temperature
\item
  \textbf{Smart Clothing}: Respiratory rate, posture monitoring
\end{itemize}

\textbf{Home Monitoring Devices:}

\begin{itemize}
\tightlist
\item
  \textbf{Smart Blood Pressure Monitor}: Automatic readings with
  timestamps
\item
  \textbf{Smart Weighing Scale}: Body composition analysis
\item
  \textbf{Smart Thermometer}: Non-contact temperature measurement
\item
  \textbf{Smart Pill Dispenser}: Medication adherence tracking
\end{itemize}

\textbf{Environmental Sensors:}

\begin{itemize}
\tightlist
\item
  \textbf{Air Quality Monitor}: PM2.5, CO2, humidity levels
\item
  \textbf{Smart Bedroom}: Sleep quality analysis
\item
  \textbf{Fall Detection}: Accelerometer-based emergency detection
\end{itemize}

\textbf{2. Communication Layer:}

\begin{itemize}
\tightlist
\item
  \textbf{Smartphone Gateway}: Data aggregation and transmission
\item
  \textbf{Bluetooth LE}: Low-power device connectivity
\item
  \textbf{WiFi/4G/5G}: Internet connectivity for data upload
\item
  \textbf{Edge Processing}: Local data filtering and analysis
\end{itemize}

\textbf{3. Cloud Infrastructure:}

\begin{itemize}
\tightlist
\item
  \textbf{Healthcare Cloud Platform}: HIPAA-compliant data storage
\item
  \textbf{Real-time Data Processing}: Stream processing for vital signs
\item
  \textbf{Machine Learning Models}: Anomaly detection and prediction
\item
  \textbf{API Gateway}: Secure data access for applications
\end{itemize}

\textbf{4. Analytics and Intelligence:}

\begin{itemize}
\tightlist
\item
  \textbf{Vital Signs Analysis}: Trend detection and threshold
  monitoring
\item
  \textbf{Predictive Analytics}: Early warning system for health issues
\item
  \textbf{Personalized Insights}: Individual health recommendations
\item
  \textbf{Population Health}: Aggregate health statistics
\end{itemize}

\textbf{5. User Interfaces:}

\begin{itemize}
\tightlist
\item
  \textbf{Patient Mobile App}: Personal health dashboard
\item
  \textbf{Doctor Web Portal}: Patient monitoring and management
\item
  \textbf{Emergency Dashboard}: Critical alerts and response
  coordination
\item
  \textbf{Family App}: Caregiver notifications and updates
\end{itemize}

\textbf{Working Principle:}

\textbf{Data Collection Phase:}

\begin{lstlisting}
Sensors \rightarrow Smartphone \rightarrow Data Validation \rightarrow Cloud Upload
\end{lstlisting}

\textbf{Processing Phase:}

\begin{lstlisting}
Raw Data \rightarrow Preprocessing \rightarrow ML Analysis \rightarrow Alert Generation
\end{lstlisting}

\textbf{Response Phase:}

\begin{lstlisting}
Alerts \rightarrow Classification \rightarrow Notification \rightarrow Action Taken
\end{lstlisting}

\textbf{Detailed Workflow:}

\begin{enumerate}
\tightlist
\item
  \textbf{Continuous Monitoring}: Wearable devices collect vital signs
  every 15-30 seconds
\item
  \textbf{Data Aggregation}: Smartphone app aggregates data from
  multiple sensors
\item
  \textbf{Quality Check}: Data validation and error correction
  algorithms
\item
  \textbf{Secure Transmission}: Encrypted data sent to cloud via
  cellular/WiFi
\item
  \textbf{Real-time Analysis}: ML algorithms analyze incoming data
  streams
\item
  \textbf{Pattern Recognition}: Identify normal vs abnormal health
  patterns
\item
  \textbf{Alert Generation}: Automated alerts for threshold violations
\item
  \textbf{Notification Dispatch}: Alerts sent to patients, doctors, and
  family
\item
  \textbf{Emergency Response}: Critical alerts trigger emergency
  services
\item
  \textbf{Data Storage}: Historical data stored for long-term analysis
\end{enumerate}

\textbf{Clinical Use Cases:}

\textbf{Chronic Disease Management:}

\begin{itemize}
\tightlist
\item
  \textbf{Diabetes}: Continuous glucose monitoring with insulin
  recommendations
\item
  \textbf{Hypertension}: Blood pressure tracking with medication
  reminders
\item
  \textbf{Heart Disease}: ECG monitoring with arrhythmia detection
\item
  \textbf{COPD}: Respiratory rate and oxygen saturation monitoring
\end{itemize}

\textbf{Emergency Detection:}

\begin{itemize}
\tightlist
\item
  \textbf{Cardiac Events}: Heart rate anomalies trigger immediate alerts
\item
  \textbf{Falls}: Accelerometer data detects falls in elderly patients
\item
  \textbf{Medication Non-compliance}: Smart pill dispensers track
  adherence
\item
  \textbf{Sleep Apnea}: Respiratory monitoring during sleep
\end{itemize}

\textbf{Performance Metrics:}

{\def\LTcaptype{none} % do not increment counter
\begin{longtable}[]{@{}lll@{}}
\toprule\noalign{}
Metric & Target Value & Current Achievement \\
\midrule\noalign{}
\endhead
\bottomrule\noalign{}
\endlastfoot
\textbf{Data Accuracy} & \textgreater95\% & 97\% \\
\textbf{False Alarm Rate} & \textless5\% & 3\% \\
\textbf{Response Time} & \textless30 seconds & 15 seconds \\
\textbf{Battery Life} & 7 days & 5 days \\
\textbf{User Adoption} & \textgreater80\% & 75\% \\
\end{longtable}
}

\textbf{Technical Specifications:}

\textbf{Sensor Specifications:}

\begin{itemize}
\tightlist
\item
  \textbf{Heart Rate}: \pm2 BPM accuracy
\item
  \textbf{Blood Pressure}: \pm3 mmHg accuracy\\
\item
  \textbf{Temperature}: \pm0.1^\circC accuracy
\item
  \textbf{Activity}: \textgreater95\% step counting accuracy
\end{itemize}

\textbf{Communication Specifications:}

\begin{itemize}
\tightlist
\item
  \textbf{Data Rate}: 1-10 Kbps per device
\item
  \textbf{Latency}: \textless100ms for critical alerts
\item
  \textbf{Range}: 10m Bluetooth, unlimited cellular
\item
  \textbf{Security}: AES-256 encryption
\end{itemize}

\textbf{Privacy and Security:}

\begin{itemize}
\tightlist
\item
  \textbf{Data Encryption}: End-to-end encryption for all communications
\item
  \textbf{Access Control}: Role-based permissions for healthcare
  providers
\item
  \textbf{Compliance}: HIPAA, GDPR compliant data handling
\item
  \textbf{Audit Trails}: Complete logging of data access and
  modifications
\end{itemize}

\textbf{Cost-Benefit Analysis:}

\textbf{Implementation Costs:}

\begin{itemize}
\tightlist
\item
  \textbf{Hardware per Patient}: \$200-500
\item
  \textbf{Cloud Infrastructure}: \$10-20 per patient per month
\item
  \textbf{Development}: \$500K-1M initial investment
\item
  \textbf{Maintenance}: 15-20\% of development cost annually
\end{itemize}

\textbf{Benefits:}

\begin{itemize}
\tightlist
\item
  \textbf{Hospital Readmission Reduction}: 25-30\%
\item
  \textbf{Emergency Response Time}: 50\% improvement
\item
  \textbf{Healthcare Cost Savings}: \$1000-2000 per patient annually
\item
  \textbf{Patient Satisfaction}: 85\% improvement in care quality
\end{itemize}

\textbf{Challenges and Solutions:}

{\def\LTcaptype{none} % do not increment counter
\begin{longtable}[]{@{}ll@{}}
\toprule\noalign{}
Challenge & Solution \\
\midrule\noalign{}
\endhead
\bottomrule\noalign{}
\endlastfoot
\textbf{Data Privacy} & End-to-end encryption, data anonymization \\
\textbf{Device Battery Life} & Low-power protocols, energy harvesting \\
\textbf{False Alarms} & AI-based pattern recognition, adaptive
thresholds \\
\textbf{User Compliance} & Gamification, family involvement \\
\textbf{Interoperability} & Standard protocols (HL7 FHIR, MQTT) \\
\end{longtable}
}

\textbf{Future Enhancements:}

\begin{itemize}
\tightlist
\item
  \textbf{AI-Powered Diagnosis}: Advanced machine learning for disease
  prediction
\item
  \textbf{Telemedicine Integration}: Video consultations based on sensor
  data
\item
  \textbf{Blockchain}: Secure, distributed health record management
\item
  \textbf{5G Connectivity}: Ultra-low latency for real-time monitoring
\end{itemize}

\end{solutionbox}
\begin{mnemonicbox}
``WHDCA-UI - Wearables Home-devices Data
Communication Analytics User-interface''

\end{mnemonicbox}
\subsection*{Question 5(a) OR [3
marks]}\label{q5a}

\textbf{List three real-world IoT applications.}

\begin{solutionbox}

\textbf{Real-World IoT Applications:}

{\def\LTcaptype{none} % do not increment counter
\begin{longtable}[]{@{}
  >{\raggedright\arraybackslash}p{(\linewidth - 2\tabcolsep) * \real{0.5000}}
  >{\raggedright\arraybackslash}p{(\linewidth - 2\tabcolsep) * \real{0.5000}}@{}}
\toprule\noalign{}
\begin{minipage}[b]{\linewidth}\raggedright
Application
\end{minipage} & \begin{minipage}[b]{\linewidth}\raggedright
Description
\end{minipage} \\
\midrule\noalign{}
\endhead
\bottomrule\noalign{}
\endlastfoot
\textbf{Smart Agriculture} & Soil moisture monitoring and automated
irrigation \\
\textbf{Industrial Monitoring} & Predictive maintenance of manufacturing
equipment \\
\textbf{Smart Transportation} & Traffic management and vehicle tracking
systems \\
\end{longtable}
}

\end{solutionbox}
\begin{mnemonicbox}
``AIT - Agriculture Industrial Transportation''

\end{mnemonicbox}
\subsection*{Question 5(b) OR [4
marks]}\label{q5b}

\textbf{Describe the role of IoT in a smart parking system.}

\begin{solutionbox}

\textbf{IoT in Smart Parking System:}

IoT enables intelligent parking management by providing real-time
information about parking space availability and automating payment
processes.

\textbf{System Components:}

\begin{itemize}
\tightlist
\item
  \textbf{Parking Sensors}: Ultrasonic/magnetic sensors detect vehicle
  presence
\item
  \textbf{Gateway Devices}: Collect data from multiple sensors
\item
  \textbf{Cloud Platform}: Process and store parking data
\item
  \textbf{Mobile Application}: User interface for parking information
\end{itemize}

\textbf{IoT Benefits:}

{\def\LTcaptype{none} % do not increment counter
\begin{longtable}[]{@{}ll@{}}
\toprule\noalign{}
Traditional Parking & IoT Smart Parking \\
\midrule\noalign{}
\endhead
\bottomrule\noalign{}
\endlastfoot
Manual space searching & Real-time availability \\
Cash/card payments & Mobile payments \\
No data analytics & Usage analytics \\
High fuel wastage & 30\% fuel savings \\
\end{longtable}
}

\textbf{Working Process:}

\begin{enumerate}
\tightlist
\item
  \textbf{Detection}: Sensors detect empty/occupied spaces
\item
  \textbf{Data Collection}: Gateway aggregates sensor data
\item
  \textbf{Cloud Processing}: Real-time space availability calculation
\item
  \textbf{User Notification}: Mobile app shows available spaces
\item
  \textbf{Navigation}: GPS-guided parking assistance
\item
  \textbf{Payment}: Automated mobile payment processing
\end{enumerate}

\textbf{Key Features:}

\begin{itemize}
\tightlist
\item
  \textbf{Real-time Updates}: Space availability updated every 30
  seconds
\item
  \textbf{Predictive Analytics}: Parking demand forecasting
\item
  \textbf{Dynamic Pricing}: Rates adjusted based on demand
\item
  \textbf{Violation Detection}: Overstay and illegal parking alerts
\end{itemize}

\end{solutionbox}
\begin{mnemonicbox}
``DCPN - Detection Collection Processing
Notification''

\end{mnemonicbox}
\subsection*{Question 5(c) OR [7
marks]}\label{q5c}

\textbf{Draw Architecture block diagram of Raspberry Pi and explain it.}

\begin{solutionbox}

\textbf{Raspberry Pi 4B Architecture:}

\includegraphics[width=1\linewidth,height=\textheight,keepaspectratio]{mermaid-9278b194.pdf}

\textbf{Detailed Architecture Explanation:}

\textbf{1. Central Processing Unit (CPU):}

\begin{itemize}
\tightlist
\item
  \textbf{Processor}: Quad-core ARM Cortex-A72 64-bit
\item
  \textbf{Clock Speed}: 1.5 GHz (can be overclocked to 2.0 GHz)
\item
  \textbf{Architecture}: ARMv8-A with NEON SIMD support
\item
  \textbf{Cache}: L1: 32KB instruction + 32KB data per core, L2: 1MB
  shared
\item
  \textbf{Performance}: \textasciitilde4x faster than Raspberry Pi 3B+
\end{itemize}

\textbf{2. Graphics Processing Unit (GPU):}

\begin{itemize}
\tightlist
\item
  \textbf{Model}: Broadcom VideoCore VI
\item
  \textbf{Features}: OpenGL ES 3.0, Hardware video decode
\item
  \textbf{Video}: 4K60 HEVC decode, 1080p60 H.264 encode
\item
  \textbf{Display}: Dual 4K display support via micro-HDMI
\end{itemize}

\textbf{3. System on Chip (SoC):}

\begin{itemize}
\tightlist
\item
  \textbf{Chip}: Broadcom BCM2711
\item
  \textbf{Process}: 28nm technology
\item
  \textbf{Integration}: CPU, GPU, memory controller, I/O controllers
\item
  \textbf{Thermal Management}: Heat spreader and thermal throttling
\end{itemize}

\textbf{4. Memory Subsystem:}

\begin{itemize}
\tightlist
\item
  \textbf{RAM}: LPDDR4-3200 (1GB, 2GB, 4GB, or 8GB variants)
\item
  \textbf{Memory Controller}: 64-bit wide bus
\item
  \textbf{Bandwidth}: Up to 25.6 GB/s theoretical
\item
  \textbf{Storage}: MicroSD card slot (UHS-I support)
\end{itemize}

\textbf{5. Connectivity Options:}

\textbf{USB Connectivity:}

\begin{itemize}
\tightlist
\item
  \textbf{USB 3.0}: 2 ports with 5 Gbps speed
\item
  \textbf{USB 2.0}: 2 ports with 480 Mbps speed
\item
  \textbf{Power}: Bus-powered devices supported up to 1.2A total
\end{itemize}

\textbf{Network Connectivity:}

\begin{itemize}
\tightlist
\item
  \textbf{Ethernet}: Gigabit Ethernet (1000 Mbps) via USB 3.0
\item
  \textbf{WiFi}: 802.11ac dual-band (2.4GHz + 5GHz)
\item
  \textbf{Bluetooth}: Bluetooth 5.0 with Low Energy support
\end{itemize}

\textbf{6. Input/Output Interfaces:}

\textbf{GPIO (General Purpose Input/Output):}

\begin{itemize}
\tightlist
\item
  \textbf{Pins}: 40-pin header (26 GPIO + power + ground)
\item
  \textbf{Protocols}: SPI, I2C, UART, PWM support
\item
  \textbf{Voltage}: 3.3V logic levels
\item
  \textbf{Current}: 16mA per pin, 50mA total
\end{itemize}

\textbf{Specialized Interfaces:}

\begin{itemize}
\tightlist
\item
  \textbf{Camera Serial Interface (CSI)}: 15-pin connector for camera
  modules
\item
  \textbf{Display Serial Interface (DSI)}: 15-pin connector for touch
  displays
\item
  \textbf{Audio}: 3.5mm TRRS jack (audio + composite video)
\item
  \textbf{HDMI}: 2x micro-HDMI ports supporting 4K60
\end{itemize}

\textbf{7. Power Management:}

\begin{itemize}
\tightlist
\item
  \textbf{Input}: USB-C connector, 5V 3A minimum
\item
  \textbf{Power Consumption}: 2.7W idle, 6.4W under stress
\item
  \textbf{Power Management IC}: Efficient voltage regulation
\item
  \textbf{GPIO Power}: 3.3V and 5V rails available
\end{itemize}

\textbf{8. Boot and Storage:}

\begin{itemize}
\tightlist
\item
  \textbf{Boot Options}: MicroSD card, USB storage, network boot
\item
  \textbf{File Systems}: Supports ext4, FAT32, NTFS
\item
  \textbf{OS Support}: Raspberry Pi OS, Ubuntu, Windows 10 IoT
\end{itemize}

\textbf{Performance Comparison:}

{\def\LTcaptype{none} % do not increment counter
\begin{longtable}[]{@{}lll@{}}
\toprule\noalign{}
Specification & RPi 3B+ & RPi 4B \\
\midrule\noalign{}
\endhead
\bottomrule\noalign{}
\endlastfoot
\textbf{CPU Cores} & 4 & 4 \\
\textbf{CPU Speed} & 1.4 GHz & 1.5 GHz \\
\textbf{RAM Options} & 1GB & 1/2/4/8GB \\
\textbf{Ethernet} & 300 Mbps & 1 Gbps \\
\textbf{USB} & 2.0 only & 3.0 + 2.0 \\
\textbf{WiFi} & 802.11n & 802.11ac \\
\end{longtable}
}

\textbf{GPIO Pinout (Key Pins):}

{\def\LTcaptype{none} % do not increment counter
\begin{longtable}[]{@{}llll@{}}
\toprule\noalign{}
Pin & Function & Pin & Function \\
\midrule\noalign{}
\endhead
\bottomrule\noalign{}
\endlastfoot
1 & 3.3V Power & 2 & 5V Power \\
3 & GPIO 2 (SDA) & 4 & 5V Power \\
5 & GPIO 3 (SCL) & 6 & Ground \\
7 & GPIO 4 & 8 & GPIO 14 (TXD) \\
9 & Ground & 10 & GPIO 15 (RXD) \\
\end{longtable}
}

\textbf{Software Architecture:}

\begin{lstlisting}
┌─────────────────────────────────────┐
│          Applications               │
├─────────────────────────────────────┤
│      Python/C++/Java Libraries      │
├─────────────────────────────────────┤
│         Raspberry Pi OS             │
├─────────────────────────────────────┤
│          Linux Kernel               │
├─────────────────────────────────────┤
│         Hardware (BCM2711)          │
└─────────────────────────────────────┘
\end{lstlisting}

\textbf{Typical IoT Use Cases:}

\begin{itemize}
\tightlist
\item
  \textbf{IoT Gateway}: Collect data from sensors via GPIO/USB
\item
  \textbf{Edge Computing}: Local data processing and ML inference
\item
  \textbf{Home Automation}: Control devices via GPIO and network
\item
  \textbf{Industrial Monitoring}: Interface with industrial sensors
\item
  \textbf{Robotics}: Motor control and sensor integration
\end{itemize}

\textbf{Advantages in IoT:}

\begin{itemize}
\tightlist
\item
  \textbf{Full Linux OS}: Complete development environment
\item
  \textbf{Rich I/O}: Multiple communication protocols supported
\item
  \textbf{Community Support}: Extensive documentation and libraries
\item
  \textbf{Cost-Effective}: \$35-75 depending on RAM configuration
\item
  \textbf{Power Efficient}: Can run on battery with proper power
  management
\end{itemize}

\textbf{Limitations:}

\begin{itemize}
\tightlist
\item
  \textbf{Real-time Performance}: Not suitable for hard real-time
  applications
\item
  \textbf{Industrial Temperature}: Consumer-grade temperature range
\item
  \textbf{GPIO Drive}: Limited current output per pin
\item
  \textbf{Analog Input}: No built-in ADC (requires external ADC)
\end{itemize}

\textbf{Development Tools:}

\begin{itemize}
\tightlist
\item
  \textbf{Programming Languages}: Python, C/C++, Java, Node.js
\item
  \textbf{IDEs}: Thonny, Visual Studio Code, Eclipse
\item
  \textbf{Libraries}: RPi.GPIO, gpiozero, OpenCV, TensorFlow Lite
\item
  \textbf{Remote Development}: SSH, VNC, VS Code Remote
\end{itemize}

\end{solutionbox}
\begin{mnemonicbox}
``CPU-GPU-SoC-MEM-CONN-IO-PWR-BOOT - Complete Pi
Architecture''

\end{mnemonicbox}

\end{document}
