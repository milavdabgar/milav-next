\documentclass{article}

% content/resources/templates/preamble.tex
\usepackage[margin=0.6in]{geometry}
\author{Milav Dabgar}
\usepackage{amsmath,amssymb,amsthm}
\usepackage{booktabs}
\usepackage{multirow}
\usepackage{xcolor}
\usepackage{tcolorbox}
\tcbuselibrary{breakable,skins}
\usepackage[colorlinks=true,linkcolor=blue]{hyperref}
\usepackage{titlesec}
\usepackage{enumitem}
\usepackage{tikz}
\usepackage{pgfplots}
\usepackage{circuitikz}
\usepackage[version=4]{mhchem}
\usepackage{longtable}
\usepackage{array}
\usepackage{float}
\usepackage{caption}
\usepackage{listings}

\lstset{
  basicstyle=\small\ttfamily,
  breaklines=true,
  breakatwhitespace=false,
  postbreak=\mbox{\textcolor{red}{$\hookrightarrow$}\space},
  float=false,
  numbers=left,
  numberstyle=\tiny\color{gray},
  numbersep=10pt,
  xleftmargin=2em,
  keywordstyle=\color{blue},
  commentstyle=\color{green!60!black},
  stringstyle=\color{purple},
  backgroundcolor=\color{gray!5},
  showstringspaces=false,
  tabsize=2,
  captionpos=b,
  keepspaces=true,
  columns=flexible
}

\pgfplotsset{compat=1.18}
\usetikzlibrary{shapes,arrows,positioning,calc,patterns,decorations.pathmorphing,decorations.markings,arrows.meta}

% Color scheme
\definecolor{headcolor}{RGB}{0,102,204}
\definecolor{keycolor}{RGB}{220,20,60}
\definecolor{solutioncolor}{RGB}{34,139,34}
\definecolor{mnemoniccolor}{RGB}{148,0,211}
\definecolor{codecolor}{RGB}{0,0,100}

% Spacing
\setlength{\parskip}{3pt}
\setlist[itemize]{nosep}
\setlist[enumerate]{nosep}

% Title formatting
\titleformat{\section}{\Large\bfseries\color{headcolor}}{\thesection}{1em}{}
\titleformat{\subsection}{\large\bfseries\color{headcolor}}{\thesubsection}{1em}{}

% Pandoc tightlist compatibility
\providecommand{\tightlist}{%
  \setlength{\itemsep}{0pt}\setlength{\parskip}{0pt}}

% Pandoc longtable compatibility
\newcounter{none}
\def\thenone{}


% content/resources/templates/english-boxes.tex

% Custom environments
\newtcolorbox{solutionbox}{
 breakable,
 enhanced,
 colback=solutioncolor!5!white,
 colframe=solutioncolor!75!black,
 fonttitle=\bfseries,
 title=Solution
}

\newtcolorbox{solutionboxnobreak}{
 colback=solutioncolor!5!white,
 colframe=solutioncolor!75!black,
 fonttitle=\bfseries,
 title=Solution
}

\newtcolorbox{keyformula}{
 breakable,
 enhanced,
 colback=keycolor!5!white,
 colframe=keycolor!75!black,
 fonttitle=\bfseries,
 title=Key Formula
}

\newtcolorbox{mnemonicboxenv}{
 breakable,
 enhanced,
 colback=mnemoniccolor!5!white,
 colframe=mnemoniccolor!75!black,
 fonttitle=\bfseries,
 title=Mnemonic
}

\newcommand{\mnemonicbox}[1]{%
  \begin{mnemonicboxenv}
    #1
  \end{mnemonicboxenv}
}


% Custom commands for GTU solutions
% This file defines semantic commands for consistent formatting

% Question command with automatic formatting
\newcommand{\question}[2]{%
  \section*{Question #1}%
  \textbf{#2}%
}

% OR question variant
\newcommand{\questionor}[2]{%
  \section*{Question #1 OR}%
  \textbf{#2}%
}

% Proper table environment with caption
\newenvironment{answertable}[1]{%
  \begin{table}[htbp]
  \centering
  \caption{#1}
}{%
  \end{table}
}

% Proper figure environment for diagrams
\newenvironment{answerdiagram}[1]{%
  \begin{figure}[htbp]
  \centering
  \caption{#1}
}{%
  \end{figure}
}

% Semantic markup for key terms
\newcommand{\keyword}[1]{\textbf{#1}}
\newcommand{\code}[1]{\texttt{#1}}
\newcommand{\classname}[1]{\texttt{#1}}
\newcommand{\methodname}[1]{\texttt{#1}}

% Proper quotation marks
\newcommand{\mnemonic}[1]{``#1''}


\title{Wireless Sensor Networks and IoT (4353201) - Winter 2024 Solution}
\date{November 21, 2024}

\begin{document}
\maketitle

\questionmarks{1(a)}{3}{Compare Single hop and Multihop Network.}

\begin{solutionbox}
\begin{tabulary}{\linewidth}{|L|L|L|}
\hline
\textbf{Parameter} & \textbf{Single Hop Network} & \textbf{Multihop Network} \\
\hline
\textbf{Communication} & Direct to base station & Via intermediate nodes \\
\textbf{Energy consumption} & High for distant nodes & Distributed among nodes \\
\textbf{Network coverage} & Limited by transmission range & Extended coverage area \\
\textbf{Complexity} & Simple routing & Complex routing protocols \\
\hline
\end{tabulary}

\begin{itemize}
    \item \textbf{Single hop}: All nodes communicate directly with base station
    \item \textbf{Multihop}: Data passes through multiple intermediate nodes to reach destination
\end{itemize}

\begin{mnemonicbox}Single Direct, Multi Relay\end{mnemonicbox}
\end{solutionbox}

\questionmarks{1(b)}{4}{Explain the Basic Components of Sensor Node.}

\begin{solutionbox}
\begin{center}
\begin{tikzpicture}[gtu block]
    \node (sn) {Sensor Node};
    
    \node (sensing) [below left=1.5cm and 0.5cm of sn] {Sensing Unit};
    \node (proc) [below right=1.5cm and 0.5cm of sn] {Processing Unit};
    \node (comm) [below left=3.5cm and 0.5cm of sn] {Communication Unit};
    \node (power) [below right=3.5cm and 0.5cm of sn] {Power Unit};
    
    \node (sensors) [below=0.5cm of sensing, font=\small] {Sensors \& ADC};
    \node (mem) [below=0.5cm of proc, font=\small] {Processor \& Memory};
    \node (trans) [below=0.5cm of comm, font=\small] {Transceiver};
    \node (batt) [below=0.5cm of power, font=\small] {Battery};

    \draw [gtu arrow] (sn) -- (sensing);
    \draw [gtu arrow] (sn) -- (proc);
    \draw [gtu arrow] (sn) -- (comm);
    \draw [gtu arrow] (sn) -- (power);
    
    \draw [gtu arrow] (sensing) -- (sensors);
    \draw [gtu arrow] (proc) -- (mem);
    \draw [gtu arrow] (comm) -- (trans);
    \draw [gtu arrow] (power) -- (batt);
\end{tikzpicture}
\end{center}

\textbf{Basic Components:}
\begin{itemize}
    \item \textbf{Sensing subsystem}: Collects data from environment using sensors and ADC
    \item \textbf{Processing subsystem}: Microcontroller/processor with memory for data processing
    \item \textbf{Communication subsystem}: Radio transceiver for wireless data transmission
    \item \textbf{Power subsystem}: Battery or energy harvesting unit for power supply
\end{itemize}

\begin{mnemonicbox}Sense Process Communicate Power\end{mnemonicbox}
\end{solutionbox}

\questionmarks{1(c)}{7}{List out any four technologies to reduce power consumption in WSN and explain any two technologies in detail.}

\begin{solutionbox}
\begin{center}
\captionof{table}{Four Power Reduction Technologies:}
\begin{tabulary}{\linewidth}{|L|L|}
\hline
\textbf{Technology} & \textbf{Description} \\
\hline
\textbf{Sleep scheduling} & Nodes alternate between active and sleep modes \\
\textbf{Data aggregation} & Combines multiple data packets into single transmission \\
\textbf{Topology control} & Optimizes network structure to reduce energy \\
\textbf{Energy harvesting} & Uses renewable sources like solar, vibration \\
\hline
\end{tabulary}
\end{center}

\textbf{Detailed Explanation:}

\textbf{1. Sleep Scheduling:}
\begin{itemize}
    \item \textbf{Active mode}: Node performs sensing, processing, communication
    \item \textbf{Sleep mode}: Node powers down non-essential components
    \item \textbf{Benefits}: Reduces idle listening energy consumption by 90\%
\end{itemize}

\textbf{2. Data Aggregation:}
\begin{itemize}
    \item \textbf{Process}: Multiple sensor readings combined at intermediate nodes
    \item \textbf{Techniques}: Average, maximum, minimum functions applied
    \item \textbf{Advantage}: Reduces total number of transmissions significantly
\end{itemize}

\begin{mnemonicbox}Sleep Aggregate Topology Harvest\end{mnemonicbox}
\end{solutionbox}

\vspace{0.5em}\centerline{\textbf{OR}}\questionmarks{1(c)}{ 7 }{List out any four challenges of wireless sensor network and explain any two in detail.}

\begin{solutionbox}
\begin{center}
\captionof{table}{Four WSN Challenges:}
\begin{tabulary}{\linewidth}{|L|L|}
\hline
\textbf{Challenge} & \textbf{Impact} \\
\hline
\textbf{Limited energy} & Affects network lifetime \\
\textbf{Limited bandwidth} & Constrains data transmission \\
\textbf{Security vulnerabilities} & Threatens data integrity \\
\textbf{Scalability issues} & Affects large network performance \\
\hline
\end{tabulary}
\end{center}

\textbf{Detailed Explanation:}

\textbf{1. Limited Energy:}
\begin{itemize}
    \item \textbf{Battery constraint}: Nodes operate on small batteries with limited capacity
    \item \textbf{Energy depletion}: High energy consumption during transmission and reception
    \item \textbf{Solution approaches}: Power management protocols, energy-efficient routing
\end{itemize}

\textbf{2. Security Vulnerabilities:}
\begin{itemize}
    \item \textbf{Physical attacks}: Nodes can be physically captured or damaged
    \item \textbf{Network attacks}: Eavesdropping, jamming, denial of service attacks
    \item \textbf{Countermeasures}: Encryption, authentication, secure routing protocols
\end{itemize}

\begin{mnemonicbox}Energy Bandwidth Security Scale\end{mnemonicbox}
\end{solutionbox}

\questionmarks{2(a)}{3}{``IEEE 802.15.4 standard and the Zigbee specifications are popular protocol choices for Wireless Sensor Network'' - Justify}

\begin{solutionbox}
\begin{center}
\captionof{table}{Justification Table:}
\begin{tabulary}{\linewidth}{|L|L|}
\hline
\textbf{Feature} & \textbf{Benefit for WSN} \\
\hline
\textbf{Low power consumption} & Extends battery life \\
\textbf{Low data rate} & Suitable for sensor data \\
\textbf{Short range} & Perfect for clustered sensors \\
\textbf{Low cost} & Economical for large deployments \\
\hline
\end{tabulary}
\end{center}

\begin{itemize}
    \item \textbf{IEEE 802.15.4}: Provides PHY and MAC layer specifications
    \item \textbf{ZigBee}: Adds network and application layers on top
    \item \textbf{Perfect match}: WSN requirements align with protocol capabilities
\end{itemize}

\begin{mnemonicbox}Low Power, Low Data, Low Cost, Low Range\end{mnemonicbox}
\end{solutionbox}

\questionmarks{2(b)}{4}{Explain Energy Efficient routing with the help of suitable example}

\begin{solutionbox}
\begin{center}
\begin{tikzpicture}[gtu state]
    \node (A) {Source Node};
    \node (B) [above right=1cm and 2cm of A, fill=green!30] {Node 1\\(Battery: 80\%)};
    \node (C) [below right=1cm and 2cm of A, fill=red!30] {Node 2\\(Battery: 30\%)};
    \node (D) [right=2cm of B] at ($(B)!0.5!(C) + (3,0)$) {Destination};

    \draw [gtu arrow] (A) -- (B);
    \draw [gtu arrow, dashed] (A) -- (C);
    \draw [gtu arrow] (B) -- (D);
    \draw [gtu arrow, dashed] (C) -- (D);
\end{tikzpicture}
\end{center}

\textbf{Energy Efficient Routing:}
\begin{itemize}
    \item \textbf{Objective}: Select paths that maximize network lifetime
    \item \textbf{Approach}: Consider remaining battery levels of nodes
    \item \textbf{Example}: Route through Node 1 (80\% battery) instead of Node 2 (30\% battery)
\end{itemize}

\textbf{Key Techniques:}
\begin{itemize}
    \item \textbf{Battery awareness}: Monitor remaining energy levels
    \item \textbf{Load balancing}: Distribute traffic among multiple paths
    \item \textbf{Clustering}: Group nearby nodes to reduce long-distance transmissions
\end{itemize}

\begin{mnemonicbox}Battery Balance Cluster\end{mnemonicbox}
\end{solutionbox}

\questionmarks{2(c)}{7}{Explain setup and steady state phase of LEACH protocol with the help of suitable sketch.}

\begin{solutionbox}
\begin{center}
\begin{tikzpicture}[gtu state, node distance=2cm]
    \node (n1) {Node 1};
    \node (n2) [right=of n1] {Node 2 (CH)};
    \node (n3) [right=of n2] {Node 3};
    \node (bs) [below=3cm of n2] {Base Station};

    % Setup Phase
    \node [above=0.5cm of n2, font=\bfseries] {Setup Phase};
    \draw [gtu arrow, bend left] (n2) to node[above, font=\small] {Adv} (n1);
    \draw [gtu arrow, bend right] (n2) to node[above, font=\small] {Adv} (n3);
    \draw [gtu arrow, bend left] (n1) to node[below, font=\small] {Join} (n2);
    \draw [gtu arrow, bend right] (n3) to node[below, font=\small] {Join} (n2);

    % Steady State Phase
    \node [right=4cm of n2, font=\bfseries] (steady) {Steady State Phase};
    \draw [gtu arrow] (n1) -- node[above, font=\small, sloped] {Data} (n2);
    \draw [gtu arrow] (n3) -- node[above, font=\small, sloped] {Data} (n2);
    \draw [gtu arrow, line width=1.5pt] (n2) -- node[right, font=\small] {Aggregated Data} (bs);
\end{tikzpicture}
\end{center}

\textbf{LEACH Protocol Phases:}

\textbf{Setup Phase:}
\begin{itemize}
    \item \textbf{Cluster head selection}: Random selection based on probability threshold
    \item \textbf{Advertisement}: Selected CHs broadcast announcement messages
    \item \textbf{Cluster formation}: Non-CH nodes join nearest cluster head
    \item \textbf{Schedule creation}: CH creates TDMA schedule for cluster members
\end{itemize}

\textbf{Steady State Phase:}
\begin{itemize}
    \item \textbf{Data transmission}: Nodes send data to CH according to TDMA schedule
    \item \textbf{Data aggregation}: CH combines received data from cluster members
    \item \textbf{Data forwarding}: CH transmits aggregated data to base station
\end{itemize}

\textbf{Advantages:}
\begin{itemize}
    \item \textbf{Energy distribution}: Rotates CH role among nodes
    \item \textbf{Collision avoidance}: TDMA scheduling prevents interference
\end{itemize}

\begin{mnemonicbox}Select Advertise Join Schedule, Send Aggregate Forward\end{mnemonicbox}
\end{solutionbox}

\vspace{0.5em}\centerline{\textbf{OR}}\questionmarks{2(a)}{ 3 }{Give Classification of routing protocols in Wireless Sensor Network.}

\begin{solutionbox}
\begin{center}
\captionof{table}{WSN Routing Protocol Classification:}
\begin{tabulary}{\linewidth}{|L|L|}
\hline
\textbf{Classification Basis} & \textbf{Types} \\
\hline
\textbf{Network Structure} & Flat, Hierarchical, Location-based \\
\textbf{Protocol Operation} & Multipath, Query-based, Negotiation-based \\
\textbf{Path Establishment} & Proactive, Reactive, Hybrid \\
\hline
\end{tabulary}
\end{center}

\textbf{Main Categories:}
\begin{itemize}
    \item \textbf{Flat routing}: All nodes have equal roles (e.g., Flooding, SPIN)
    \item \textbf{Hierarchical routing}: Cluster-based approach (e.g., LEACH, TEEN)
    \item \textbf{Location-based routing}: Uses geographic information (e.g., GEAR)
\end{itemize}

\begin{mnemonicbox}Flat Hierarchical Location\end{mnemonicbox}
\end{solutionbox}

\vspace{0.5em}\centerline{\textbf{OR}}\questionmarks{2(b)}{ 4 }{Explain the wakeup concept of low duty cycle protocol with the help of sketch.}

\begin{solutionbox}
\begin{center}
\begin{tikzpicture}[gtu block]
    \node (time) {Time $\rightarrow$};
    
    % Node A Timeline
    \node (a_lbl) [below=0.5cm of time] {Node A:};
    \foreach \x/\t in {0/Sleep, 2/Wake, 3/Listen, 4/Sleep, 6/Wake, 7/Listen, 8/Sleep} {
        \node [draw, minimum width=1cm, minimum height=0.6cm, right=\x cm of a_lbl] {\t};
    }

    % Node B Timeline  
    \node (b_lbl) [below=1.5cm of a_lbl] {Node B:};
    \foreach \x/\t in {0/Sleep, 3/Wake, 4/Tx, 5/Sleep, 7/Wake, 8/Listen, 9/Sleep} {
        \node [draw, minimum width=1cm, minimum height=0.6cm, right=\x cm of b_lbl] {\t};
    }
\end{tikzpicture}
\end{center}

\textbf{Low Duty Cycle Wakeup Concept:}
\begin{itemize}
    \item \textbf{Sleep period}: Nodes turn off radio to save energy
    \item \textbf{Wake period}: Nodes periodically wake up to check for communication
    \item \textbf{Synchronization}: Sender must know receiver's wakeup schedule
\end{itemize}

\textbf{Key Benefits:}
\begin{itemize}
    \item \textbf{Energy savings}: Reduces idle listening by up to 99\%
    \item \textbf{Coordinated access}: Prevents collisions during wakeup periods
\end{itemize}

\begin{mnemonicbox}Sleep Wake Listen Repeat\end{mnemonicbox}
\end{solutionbox}

\vspace{0.5em}\centerline{\textbf{OR}}\questionmarks{2(c)}{ 7 }{Explain Synch, RTS \& CTS Phases of S-MAC Protocol and message passing approach of it.}

\begin{solutionbox}
\begin{center}
\begin{tikzpicture}[gtu state, node distance=2cm]
    \node (A) {Node A};
    \node (B) [right=of A] {Node B};
    \node (C) [right=of B] {Node C};

    % Phases
    \node (sync) [below=1cm of A, font=\bfseries] {1. SYNC Phase};
    \draw [gtu arrow] (A) -- node[above] {SYNC} (B);
    \draw [gtu arrow] (A) to[bend left] node[above] {SYNC} (C);
    
    \node (rts) [below=2cm of sync, font=\bfseries] {2. RTS/CTS Phase};
    \draw [gtu arrow] (A) -- node[above] {RTS} (B);
    \draw [gtu arrow] (B) -- node[below] {CTS} (A);
    \node [right=0.5cm of rts, font=\small, text width=3cm] {Node C overhears CTS and sleeps};

    \node (data) [below=2cm of rts, font=\bfseries] {3. Data Phase};
    \draw [gtu arrow] (A) -- node[above] {DATA} (B);
    \draw [gtu arrow] (B) -- node[below] {ACK} (A);
\end{tikzpicture}
\end{center}

\textbf{S-MAC Protocol Phases:}

\textbf{1. Synchronization Phase:}
\begin{itemize}
    \item \textbf{Purpose}: Establish common sleep/wake schedule
    \item \textbf{Process}: Nodes exchange SYNC packets containing schedule information
    \item \textbf{Benefit}: Ensures coordinated sleep patterns across network
\end{itemize}

\textbf{2. RTS Phase (Request to Send):}
\begin{itemize}
    \item \textbf{Initiation}: Sender transmits RTS packet to intended receiver
    \item \textbf{Content}: Source address, destination address, transmission duration
\end{itemize}

\textbf{3. CTS Phase (Clear to Send):}
\begin{itemize}
    \item \textbf{Response}: Receiver sends CTS packet confirming availability
    \item \textbf{Virtual sensing}: Neighboring nodes overhear CTS and defer transmission
\end{itemize}

\textbf{Message Passing Approach:}
\begin{itemize}
    \item \textbf{Collision avoidance}: RTS/CTS handshake prevents hidden terminal problem
    \item \textbf{Energy conservation}: Overhearing nodes enter sleep mode during data exchange
    \item \textbf{Periodic synchronization}: Maintains network-wide schedule coordination
\end{itemize}

\begin{mnemonicbox}Sync Request Clear Transmit\end{mnemonicbox}
\end{solutionbox}

\questionmarks{3(a)}{3}{Explain Super Frame structure of IEEE 802.15.4 standard.}

\begin{solutionbox}
\begin{center}
\begin{tikzpicture}[gtu block]
    \draw (0,0) rectangle (12,1);
    \node at (6, 1.3) {Super Frame (15.36 ms)};
    
    \draw (0,0) rectangle (1,1) node[pos=0.5] {Beacon};
    \draw (1,0) rectangle (4,1) node[pos=0.5] {CAP};
    \draw (4,0) rectangle (8,1) node[pos=0.5] {CFP (GTS)};
    \draw (8,0) rectangle (12,1) node[pos=0.5] {Inactive Period};
    
    \node at (2.5, -0.5) {Contention Access};
    \node at (6, -0.5) {Contention Free};
    \node at (10, -0.5) {Sleep Mode};
\end{tikzpicture}
\end{center}

\begin{center}
\captionof{table}{Super Frame Components:}
\begin{tabulary}{\linewidth}{|L|L|L|}
\hline
\textbf{Component} & \textbf{Description} & \textbf{Duration} \\
\hline
\textbf{Beacon} & Network synchronization & Fixed \\
\textbf{CAP} & Contention Access Period & Variable \\
\textbf{CFP} & Contention Free Period & Variable \\
\textbf{Inactive} & Sleep period & Variable \\
\hline
\end{tabulary}
\end{center}

\begin{itemize}
    \item \textbf{CAP}: Uses CSMA/CA for channel access
    \item \textbf{CFP}: Uses GTS (Guaranteed Time Slots) for real-time data
    \item \textbf{Inactive period}: Devices can enter low-power mode
\end{itemize}

\begin{mnemonicbox}Beacon Contend Guarantee Sleep\end{mnemonicbox}
\end{solutionbox}

\questionmarks{3(b)}{4}{Compare M2M and IoT Technology.}

\begin{solutionbox}
\begin{tabulary}{\linewidth}{|L|L|L|}
\hline
\textbf{Parameter} & \textbf{M2M} & \textbf{IoT} \\
\hline
\textbf{Communication} & Point-to-point & Internet-based \\
\textbf{Data processing} & Local & Cloud-based \\
\textbf{Connectivity} & Cellular/Wired & Multiple protocols \\
\textbf{Applications} & Specific industries & Consumer \& industrial \\
\hline
\end{tabulary}

\textbf{Key Differences:}
\begin{itemize}
    \item \textbf{M2M}: Machine-to-Machine direct communication
    \item \textbf{IoT}: Internet of Things with cloud integration
    \item \textbf{Scope}: M2M is subset of broader IoT ecosystem
    \item \textbf{Intelligence}: IoT provides more advanced analytics and AI
\end{itemize}

\begin{mnemonicbox}M2M Direct, IoT Internet\end{mnemonicbox}
\end{solutionbox}

\questionmarks{3(c)}{7}{Draw Block Diagram of IoT Architecture and explain it}

\begin{solutionbox}
\begin{center}
\begin{tikzpicture}[gtu block]
    \node (phys) [draw, rectangle, minimum width=4cm, minimum height=1cm] {1. Physical Layer\\(Sensors, Actuators)};
    \node (conn) [draw, rectangle, minimum width=4cm, minimum height=1cm, above=0.5cm of phys] {2. Connectivity Layer\\(WiFi, BT, Cellular)};
    \node (proc) [draw, rectangle, minimum width=4cm, minimum height=1cm, above=0.5cm of conn] {3. Data Processing Layer\\(Edge/Fog)};
    \node (accum) [draw, rectangle, minimum width=4cm, minimum height=1cm, above=0.5cm of proc] {4. Data Accumulation Layer\\(Cloud Storage)};
    \node (abstr) [draw, rectangle, minimum width=4cm, minimum height=1cm, above=0.5cm of accum] {5. Data Abstraction Layer\\(Databases)};
    \node (app) [draw, rectangle, minimum width=4cm, minimum height=1cm, above=0.5cm of abstr] {6. Application Layer\\(Analytics, Apps)};
    \node (collab) [draw, rectangle, minimum width=4cm, minimum height=1cm, above=0.5cm of app] {7. Collaboration Layer\\(Business Processes)};

    \draw [gtu arrow] (phys) -- (conn);
    \draw [gtu arrow] (conn) -- (proc);
    \draw [gtu arrow] (proc) -- (accum);
    \draw [gtu arrow] (accum) -- (abstr);
    \draw [gtu arrow] (abstr) -- (app);
    \draw [gtu arrow] (app) -- (collab);
\end{tikzpicture}
\end{center}

\textbf{IoT Architecture Layers:}

\textbf{1. Physical Layer:}
\begin{itemize}
    \item \textbf{Components}: Sensors (temperature, humidity), actuators (motors, valves)
    \item \textbf{Function}: Data collection from physical environment
\end{itemize}

\textbf{2. Connectivity Layer:}
\begin{itemize}
    \item \textbf{Protocols}: WiFi, Bluetooth, Zigbee, LoRaWAN, cellular
    \item \textbf{Function}: Transmit data from devices to processing centers
\end{itemize}

\textbf{3. Data Processing Layer:}
\begin{itemize}
    \item \textbf{Technologies}: Edge computing, fog computing
    \item \textbf{Function}: Real-time processing and filtering of sensor data
\end{itemize}

\textbf{4. Data Accumulation Layer:}
\begin{itemize}
    \item \textbf{Infrastructure}: Cloud storage, data warehouses
    \item \textbf{Function}: Store massive amounts of IoT data
\end{itemize}

\textbf{5. Data Abstraction Layer:}
\begin{itemize}
    \item \textbf{Components}: Databases, data analytics engines
    \item \textbf{Function}: Organize and prepare data for applications
\end{itemize}

\textbf{6. Application Layer:}
\begin{itemize}
    \item \textbf{Services}: Web applications, mobile apps, dashboards
    \item \textbf{Function}: Provide user interfaces and business logic
\end{itemize}

\textbf{7. Collaboration Layer:}
\begin{itemize}
    \item \textbf{Integration}: ERP systems, business processes
    \item \textbf{Function}: Enable collaboration between different stakeholders
\end{itemize}

\begin{mnemonicbox}Physical Connect Process Accumulate Abstract Apply Collaborate\end{mnemonicbox}
\end{solutionbox}

\vspace{0.5em}\centerline{\textbf{OR}}\questionmarks{3(a)}{ 3 }{Explain Energy problems of MAC Protocol}

\begin{solutionbox}
\begin{center}
\captionof{table}{Energy Problems in MAC Protocols:}
\begin{tabulary}{\linewidth}{|L|L|L|}
\hline
\textbf{Problem} & \textbf{Description} & \textbf{Impact} \\
\hline
\textbf{Idle listening} & Radio stays on without communication & 50-60\% energy waste \\
\textbf{Collision} & Multiple transmissions interfere & Retransmission overhead \\
\textbf{Overhearing} & Receiving irrelevant packets & Unnecessary energy consumption \\
\hline
\end{tabulary}
\end{center}

\textbf{Main Issues:}
\begin{itemize}
    \item \textbf{Idle listening}: Most energy-consuming activity in WSN
    \item \textbf{Protocol overhead}: Control packets consume additional energy
    \item \textbf{Poor scheduling}: Inefficient channel access increases energy usage
\end{itemize}

\begin{mnemonicbox}Idle Collide Overhear\end{mnemonicbox}
\end{solutionbox}

\vspace{0.5em}\centerline{\textbf{OR}}\questionmarks{3(b)}{ 4 }{Explain modified OSI model for IoT system}

\begin{solutionbox}
\begin{center}
\captionof{table}{Modified OSI Model for IoT:}
\begin{tabulary}{\linewidth}{|L|L|L|}
\hline
\textbf{Layer} & \textbf{Traditional OSI} & \textbf{IoT Modification} \\
\hline
\textbf{Application} & User applications & IoT applications, cloud services \\
\textbf{Presentation} & Data formatting & JSON, XML, CoAP \\
\textbf{Session} & Session management & MQTT, HTTP sessions \\
\textbf{Transport} & TCP, UDP & UDP, CoAP, MQTT \\
\textbf{Network} & IP routing & 6LoWPAN, IPv6 \\
\textbf{Data Link} & Ethernet, WiFi & IEEE 802.15.4, LoRa \\
\textbf{Physical} & Physical medium & Sensors, actuators, radio \\
\hline
\end{tabulary}
\end{center}

\textbf{Key Modifications:}
\begin{itemize}
    \item \textbf{Lightweight protocols}: Optimized for resource-constrained devices
    \item \textbf{Energy efficiency}: Protocols designed for low power consumption
    \item \textbf{Interoperability}: Support for diverse IoT devices and platforms
\end{itemize}

\begin{mnemonicbox}Apps Present Session Transport Network Link Physical\end{mnemonicbox}
\end{solutionbox}

\vspace{0.5em}\centerline{\textbf{OR}}\questionmarks{3(c)}{ 7 }{Explain Sources of IoT in detail}

\begin{solutionbox}
\begin{center}
\begin{tikzpicture}[gtu block]
    \node (root) [draw, circle, minimum size=2cm, font=\bfseries] {IoT Sources};
    
    \node (tech) [above left=2cm of root, align=center] {Technology Evolution\\(Internet, Mobile,\\Cloud, Big Data)};
    \node (bus) [above right=2cm of root, align=center] {Business Drivers\\(Cost Reduction,\\Efficiency, CX)};
    \node (enable) [below left=2cm of root, align=center] {Tech Enablers\\(Sensors, Wireless,\\Processing, Storage)};
    \node (market) [below right=2cm of root, align=center] {Market Demands\\(Smart Cities,\\Healthcare, Industry)};
    
    \draw [gtu arrow] (root) -- (tech);
    \draw [gtu arrow] (root) -- (bus);
    \draw [gtu arrow] (root) -- (enable);
    \draw [gtu arrow] (root) -- (market);
\end{tikzpicture}
\end{center}

\textbf{1. Technology Evolution Sources:}
\begin{itemize}
    \item \textbf{Internet expansion}: Global connectivity infrastructure development
    \item \textbf{Mobile revolution}: Smartphones and tablets creating connected ecosystem
    \item \textbf{Cloud computing}: Scalable computing and storage resources
    \item \textbf{Big data analytics}: Ability to process massive data volumes
\end{itemize}

\textbf{2. Business Drivers:}
\begin{itemize}
    \item \textbf{Operational efficiency}: Automation and optimization of business processes
    \item \textbf{Cost reduction}: Lower operational and maintenance costs
    \item \textbf{New business models}: Data-driven services and products
    \item \textbf{Customer satisfaction}: Enhanced user experience through smart services
\end{itemize}

\textbf{3. Technological Enablers:}
\begin{itemize}
    \item \textbf{Sensor advancement}: Smaller, cheaper, more accurate sensors
    \item \textbf{Communication progress}: Improved wireless protocols and standards
    \item \textbf{Processing evolution}: More powerful yet energy-efficient processors
    \item \textbf{Storage revolution}: Cheaper and more reliable data storage solutions
\end{itemize}

\textbf{4. Market Demands:}
\begin{itemize}
    \item \textbf{Smart cities}: Urban planning and infrastructure management
    \item \textbf{Healthcare}: Remote monitoring and telemedicine
    \item \textbf{Industrial automation}: Industry 4.0 and smart manufacturing
    \item \textbf{Environmental monitoring}: Climate change and sustainability concerns
\end{itemize}

\textbf{Key Convergence Factors:}
\begin{itemize}
    \item \textbf{IPv6 adoption}: Unlimited addressing for billions of devices
    \item \textbf{5G networks}: High-speed, low-latency communication
    \item \textbf{AI integration}: Machine learning for intelligent decision making
\end{itemize}

\begin{mnemonicbox}Technology Business Enable Market\end{mnemonicbox}
\end{solutionbox}

\questionmarks{4(a)}{3}{Explain basic Components of IoT in brief.}

\begin{solutionbox}
\begin{center}
\captionof{table}{Basic IoT Components:}
\begin{tabulary}{\linewidth}{|L|L|L|}
\hline
\textbf{Component} & \textbf{Function} & \textbf{Examples} \\
\hline
\textbf{Sensors} & Data collection & Temperature, pressure, motion \\
\textbf{Connectivity} & Data transmission & WiFi, Bluetooth, cellular \\
\textbf{Data processing} & Information analysis & Edge/cloud computing \\
\textbf{User interface} & Human interaction & Mobile apps, dashboards \\
\hline
\end{tabulary}
\end{center}

\textbf{Core Functions:}
\begin{itemize}
    \item \textbf{Sensing}: Collect environmental data
    \item \textbf{Connecting}: Transmit data to processing centers
    \item \textbf{Processing}: Analyze and extract insights
    \item \textbf{Acting}: Control actuators based on analysis
\end{itemize}

\begin{mnemonicbox}Sense Connect Process Interface\end{mnemonicbox}
\end{solutionbox}

\questionmarks{4(b)}{4}{Discuss Constrained Application Protocol (CoAP) in brief.}

\begin{solutionbox}
\begin{center}
\begin{tikzpicture}[gtu state, node distance=4cm]
    \node (client) {Client};
    \node (server) [right=of client] {Server};
    
    \draw [gtu arrow] (client) -- node[above] {GET /temp} (server);
    \draw [gtu arrow] (server) -- node[below] {2.05 Content (25$^\circ$C)} (client);
\end{tikzpicture}
\end{center}

\begin{center}
\captionof{table}{CoAP Features:}
\begin{tabulary}{\linewidth}{|L|L|L|}
\hline
\textbf{Feature} & \textbf{Description} & \textbf{Benefit} \\
\hline
\textbf{Lightweight} & Simple protocol design & Low resource usage \\
\textbf{UDP-based} & Uses UDP transport & Reduced overhead \\
\textbf{RESTful} & REST architecture & Easy integration \\
\textbf{Reliable} & Built-in retransmission & Ensures delivery \\
\hline
\end{tabulary}
\end{center}

\textbf{Key Characteristics:}
\begin{itemize}
    \item \textbf{Request/Response}: Similar to HTTP but optimized for IoT
    \item \textbf{Confirmable messages}: Reliability through acknowledgments
    \item \textbf{Resource discovery}: Built-in service discovery mechanism
    \item \textbf{Block transfer}: Support for large data transfers
\end{itemize}

\begin{mnemonicbox}Light UDP REST Reliable\end{mnemonicbox}
\end{solutionbox}

\questionmarks{4(c)}{7}{Explain Process of Sensor and controlling device (actuator) management through cloud.}

\begin{solutionbox}
\begin{center}
\begin{tikzpicture}[gtu state, node distance=2.5cm]
    \node (sens) {Sensor};
    \node (gw) [right=of sens] {Gateway};
    \node (cloud) [right=of gw] {Cloud};
    \node (act) [below=of gw] {Actuator};
    \node (user) [below=of cloud] {User App};

    \draw [gtu arrow] (sens) -- node[above, font=\small] {Data} (gw);
    \draw [gtu arrow] (gw) -- node[above, font=\small] {MQTT} (cloud);
    \draw [gtu arrow] (cloud) -- node[right, font=\small] {Dash} (user);
    \draw [gtu arrow] (user) -- node[left, font=\small] {Cmd} (cloud);
    \draw [gtu arrow] (cloud) to[bend right] node[above, font=\small] {Action} (gw);
    \draw [gtu arrow] (gw) -- node[left, font=\small] {Sig} (act);
\end{tikzpicture}
\end{center}

\textbf{Cloud-based IoT Management Process:}

\textbf{1. Data Collection Phase:}
\begin{itemize}
    \item \textbf{Sensors}: Collect environmental data (temperature, humidity, motion)
    \item \textbf{Local processing}: Basic filtering and formatting at edge devices
    \item \textbf{Data transmission}: Send data to cloud via WiFi/cellular connection
\end{itemize}

\textbf{2. Cloud Processing Phase:}
\begin{itemize}
    \item \textbf{Data ingestion}: Receive and store sensor data in cloud databases
    \item \textbf{Real-time analytics}: Process data streams for immediate insights
    \item \textbf{Machine learning}: Apply AI algorithms for pattern recognition and prediction
\end{itemize}

\textbf{3. Decision Making Phase:}
\begin{itemize}
    \item \textbf{Rule engine}: Apply business rules to determine required actions
    \item \textbf{Threshold monitoring}: Trigger alerts when values exceed limits
    \item \textbf{Automated responses}: Generate control commands for actuators
\end{itemize}

\textbf{4. Control Execution Phase:}
\begin{itemize}
    \item \textbf{Command dispatch}: Send control signals to appropriate actuators
    \item \textbf{Device management}: Monitor actuator status and performance
    \item \textbf{Feedback loop}: Collect confirmation of successful command execution
\end{itemize}

\textbf{5. User Interaction:}
\begin{itemize}
    \item \textbf{Dashboard}: Real-time visualization of sensor data and system status
    \item \textbf{Mobile apps}: Remote monitoring and manual control capabilities
    \item \textbf{Notifications}: Alerts and warnings sent to users
\end{itemize}

\textbf{Benefits:}
\begin{itemize}
    \item \textbf{Scalability}: Handle thousands of devices simultaneously
    \item \textbf{Remote access}: Control devices from anywhere with internet
    \item \textbf{Data analytics}: Historical analysis and predictive maintenance
    \item \textbf{Integration}: Connect with other business systems and services
\end{itemize}

\begin{mnemonicbox}Collect Process Decide Control Interact\end{mnemonicbox}
\end{solutionbox}

\vspace{0.5em}\centerline{\textbf{OR}}\questionmarks{4(a)}{ 3 }{Define Internet of Things and state its Vision.}

\begin{solutionbox}
\textbf{Definition:}
Internet of Things (IoT) is a network of interconnected physical devices embedded with sensors, software, and connectivity to collect and exchange data over the internet.

\begin{center}
\captionof{table}{IoT Vision:}
\begin{tabulary}{\linewidth}{|L|L|}
\hline
\textbf{Aspect} & \textbf{Vision} \\
\hline
\textbf{Connectivity} & Everything connected everywhere \\
\textbf{Intelligence} & Smart decision making \\
\textbf{Automation} & Minimal human intervention \\
\textbf{Integration} & Seamless system interaction \\
\hline
\end{tabulary}
\end{center}

\textbf{Core Vision Elements:}
\begin{itemize}
    \item \textbf{Ubiquitous computing}: Technology embedded in everyday objects
    \item \textbf{Seamless interaction}: Natural human-device communication
    \item \textbf{Intelligent environment}: Context-aware responsive systems
\end{itemize}

\begin{mnemonicbox}Connect Intelligence Automate Integrate\end{mnemonicbox}
\end{solutionbox}

\vspace{0.5em}\centerline{\textbf{OR}}\questionmarks{4(b)}{ 4 }{Discuss (Message Queue Telemetry Transport) MQTT protocol in brief.}

\begin{solutionbox}
\begin{center}
\begin{tikzpicture}[gtu state, node distance=3cm]
    \node (pub) {Publisher};
    \node (broker) [right=of pub] {Broker};
    \node (sub) [right=of broker] {Subscriber};

    \draw [gtu arrow] (pub) -- node[above, font=\small] {Publish(TopicA)} (broker);
    \draw [gtu arrow] (sub) -- node[below, font=\small] {Subscribe(TopicA)} (broker);
    \draw [gtu arrow, dashed] (broker) -- node[above, font=\small] {Forward Msg} (sub);
\end{tikzpicture}
\end{center}

\begin{center}
\captionof{table}{MQTT Characteristics:}
\begin{tabulary}{\linewidth}{|L|L|L|}
\hline
\textbf{Feature} & \textbf{Description} & \textbf{Advantage} \\
\hline
\textbf{Lightweight} & Minimal protocol overhead & Suitable for IoT devices \\
\textbf{Publish/Subscribe} & Decoupled communication & Scalable architecture \\
\textbf{QoS levels} & Quality of service options & Reliable delivery \\
\textbf{Persistent sessions} & Session state maintained & Connection resilience \\
\hline
\end{tabulary}
\end{center}

\textbf{MQTT Components:}
\begin{itemize}
    \item \textbf{Publisher}: Sends messages to broker
    \item \textbf{Subscriber}: Receives messages from broker
    \item \textbf{Broker}: Central message router
    \item \textbf{Topics}: Message categorization system
\end{itemize}

\textbf{Quality of Service Levels:}
\begin{itemize}
    \item \textbf{QoS 0}: At most once delivery
    \item \textbf{QoS 1}: At least once delivery
    \item \textbf{QoS 2}: Exactly once delivery
\end{itemize}

\begin{mnemonicbox}Publish Subscribe Broker Topic\end{mnemonicbox}
\end{solutionbox}

\vspace{0.5em}\centerline{\textbf{OR}}\questionmarks{4(c)}{ 7 }{Draw Architecture block diagram of Raspberry Pi and explain it.}

\begin{solutionbox}
\begin{center}
\begin{tikzpicture}[gtu block]
    \node (cpu) [draw, rectangle, minimum width=2cm] {CPU (ARM A72)};
    \node (gpu) [draw, rectangle, minimum width=2cm, right=0.5cm of cpu] {GPU (VideoCore)};
    \node (ram) [draw, rectangle, minimum width=2cm, right=0.5cm of gpu] {RAM (4GB)};
    \node (store) [draw, rectangle, minimum width=2cm, right=0.5cm of ram] {MicroSD};
    
    \node (gpio) [draw, rectangle, minimum width=2cm, below=1cm of cpu] {GPIO (40 pins)};
    \node (usb) [draw, rectangle, minimum width=2cm, below=1cm of gpu] {USB (3.0)};
    \node (net) [draw, rectangle, minimum width=2cm, below=1cm of ram] {Network (WiFi/Eth)};
    \node (av) [draw, rectangle, minimum width=2cm, below=1cm of store] {A/V (HDMI/Audio)};
    
    \node [draw, rectangle, inner sep=0.5cm, fit=(cpu)(av)] (board) {};
    \node [above=0.1cm of board] {\textbf{Raspberry Pi 4}};
\end{tikzpicture}
\end{center}

\textbf{Raspberry Pi Architecture Components:}

\textbf{1. Processing Unit:}
\begin{itemize}
    \item \textbf{CPU}: Quad-core ARM Cortex-A72 processor running at 1.5GHz
    \item \textbf{GPU}: VideoCore VI for graphics processing and video acceleration
    \item \textbf{Performance}: Capable of running full operating systems like Linux
\end{itemize}

\textbf{2. Memory System:}
\begin{itemize}
    \item \textbf{RAM}: 4GB LPDDR4 system memory for program execution
    \item \textbf{Storage}: MicroSD card slot for operating system and data storage
    \item \textbf{Cache}: On-chip cache memory for improved performance
\end{itemize}

\textbf{3. Input/Output Interfaces:}
\begin{itemize}
    \item \textbf{GPIO}: 40-pin general purpose input/output for sensor connectivity
    \item \textbf{USB ports}: 4x USB 3.0 ports for peripherals and storage devices
    \item \textbf{Display}: 2x micro-HDMI ports supporting 4K video output
\end{itemize}

\textbf{4. Connectivity Options:}
\begin{itemize}
    \item \textbf{Ethernet}: Gigabit Ethernet port for wired network connection
    \item \textbf{Wireless}: Dual-band WiFi 802.11ac and Bluetooth 5.0
    \item \textbf{Camera}: Dedicated camera serial interface (CSI) port
\end{itemize}

\textbf{5. Power and Audio:}
\begin{itemize}
    \item \textbf{Power}: USB-C power input with efficient power management
    \item \textbf{Audio}: 3.5mm audio jack and HDMI audio output
    \item \textbf{Power consumption}: Optimized for continuous operation
\end{itemize}

\textbf{IoT Applications:}
\begin{itemize}
    \item \textbf{Home automation}: Control lights, fans, security systems
    \item \textbf{Industrial monitoring}: Temperature, pressure, vibration sensing
    \item \textbf{Robotics}: Motor control, sensor integration, computer vision
    \item \textbf{Data logging}: Environmental monitoring and data collection
\end{itemize}

\textbf{Advantages for IoT:}
\begin{itemize}
    \item \textbf{Cost-effective}: Low-cost computing platform
    \item \textbf{Versatile}: Supports multiple programming languages
    \item \textbf{Community support}: Large ecosystem of tutorials and projects
    \item \textbf{Expandability}: Compatible with numerous sensors and modules
\end{itemize}

\begin{mnemonicbox}Process Memory Interface Connect Power\end{mnemonicbox}
\end{solutionbox}

\questionmarks{5(a)}{3}{Draw Block Diagram of Smart Health Monitoring System with IoT.}

\begin{solutionbox}
\begin{center}
\begin{tikzpicture}[gtu state, node distance=2cm]
    \node (patient) {Patient};
    \node (sensor) [right=of patient] {Sensors (HR, Temp)};
    \node (mcu) [right=of sensor] {MCU (ESP32)};
    \node (cloud) [below=of mcu] {Cloud Server};
    \node (app) [left=of cloud] {Mobile/Doctor App};
    \node (alert) [right=of cloud] {Alerts (SMS)};

    \draw [gtu arrow] (patient) -- (sensor);
    \draw [gtu arrow] (sensor) -- (mcu);
    \draw [gtu arrow] (mcu) -- node[right] {WiFi} (cloud);
    \draw [gtu arrow] (cloud) -- (app);
    \draw [gtu arrow] (cloud) -- (alert);
\end{tikzpicture}
\end{center}

\textbf{System Components:}
\begin{itemize}
    \item \textbf{Sensors}: Collect vital signs (heart rate, blood pressure, temperature)
    \item \textbf{Microcontroller}: Process sensor data and manage communication
    \item \textbf{Connectivity}: Transmit data to cloud via WiFi/cellular networks
    \item \textbf{Cloud platform}: Store data and provide analytics services
    \item \textbf{User interfaces}: Mobile apps and web dashboards for monitoring
\end{itemize}

\begin{mnemonicbox}Sense Process Connect Store Monitor\end{mnemonicbox}
\end{solutionbox}

\questionmarks{5(b)}{4}{List out different types of sensors in IoT and briefly explain working of any two.}

\begin{solutionbox}
\begin{center}
\captionof{table}{IoT Sensor Types:}
\begin{tabulary}{\linewidth}{|L|L|L|}
\hline
\textbf{Sensor Type} & \textbf{Measurement} & \textbf{Applications} \\
\hline
\textbf{Temperature} & Heat/cold levels & HVAC, weather monitoring \\
\textbf{Humidity} & Moisture content & Agriculture, storage \\
\textbf{Pressure} & Force per unit area & Weather, industrial \\
\textbf{Motion/PIR} & Movement detection & Security, automation \\
\textbf{Gas} & Chemical composition & Air quality, safety \\
\textbf{Light} & Illumination levels & Smart lighting \\
\hline
\end{tabulary}
\end{center}

\textbf{Detailed Working:}

\textbf{1. Temperature Sensor (DHT22):}
\begin{itemize}
    \item \textbf{Principle}: Thermistor resistance changes with temperature
    \item \textbf{Process}: Microcontroller reads resistance value and converts to temperature
    \item \textbf{Output}: Digital signal with temperature and humidity data
    \item \textbf{Applications}: Smart thermostat, environmental monitoring
\end{itemize}

\textbf{2. PIR Motion Sensor:}
\begin{itemize}
    \item \textbf{Principle}: Detects infrared radiation emitted by moving objects
    \item \textbf{Components}: Pyroelectric sensor with fresnel lens
    \item \textbf{Working}: Changes in infrared levels trigger digital output signal
    \item \textbf{Applications}: Security systems, automatic lighting, occupancy detection
\end{itemize}

\begin{mnemonicbox}Temperature Humidity Pressure Motion Gas Light\end{mnemonicbox}
\end{solutionbox}

\questionmarks{5(c)}{7}{Draw Block diagram of smart home automation with IoT and Explain its working.}

\begin{solutionbox}
\begin{center}
\begin{tikzpicture}[gtu block]
    \node (ctrl) [draw, rectangle, minimum width=3cm] {Controller (RPi)};
    \node (sens) [left=of ctrl, align=center] {Sensors\\(Temp, Motion)};
    \node (act) [right=of ctrl, align=center] {Actuators\\(Lights, AC)};
    \node (comm) [below=of ctrl] {Comm (WiFi)};
    \node (cloud) [below=of comm] {Cloud Server};
    \node (user) [left=of cloud] {User App};
    \node (voice) [right=of cloud] {Voice Asst};

    \draw [gtu arrow] (sens) -- (ctrl);
    \draw [gtu arrow] (ctrl) -- (act);
    \draw [gtu arrow] (ctrl) -- (comm);
    \draw [gtu arrow] (comm) -- (cloud);
    \draw [gtu arrow] (cloud) -- (user);
    \draw [gtu arrow] (cloud) -- (voice);
\end{tikzpicture}
\end{center}

\textbf{Smart Home Automation Working:}
\begin{itemize}
    \item \textbf{Data Collection}: Sensors (environment, security, presence) monitor home status.
    \item \textbf{Data Processing}: Local (critical) and cloud (analytics) processing of sensor data.
    \item \textbf{Decision Making}: Rules (if temp > 25 then AC on) and AI (learning habits) control actions.
    \item \textbf{Control Execution}: Controller sends signals to actuators (lights dimmed, doors locked).
    \item \textbf{User Interaction}: Apps and voice assistants allow remote monitoring and control.
\end{itemize}

\textbf{Key Features:}
\begin{itemize}
    \item \textbf{Energy efficiency}: Optimized usage saves 30-40\% power.
    \item \textbf{Security}: Real-time alerts and monitoring.
    \item \textbf{Convenience}: Automated routines and voice control.
\end{itemize}

\begin{mnemonicbox}Collect Process Decide Control Interact Secure\end{mnemonicbox}
\end{solutionbox}

\vspace{0.5em}\centerline{\textbf{OR}}\questionmarks{5(a)}{ 3 }{List out any three Industrial and Military IoT applications.}

\begin{solutionbox}
\begin{center}
\captionof{table}{Industrial IoT Applications:}
\begin{tabulary}{\linewidth}{|L|L|L|}
\hline
\textbf{Application} & \textbf{Description} & \textbf{Benefits} \\
\hline
\textbf{Predictive maintenance} & Monitor equipment health & Reduce downtime \\
\textbf{Supply chain tracking} & Track goods movement & Improve efficiency \\
\textbf{Energy management} & Optimize power consumption & Reduce costs \\
\hline
\end{tabulary}
\end{center}

\begin{center}
\captionof{table}{Military IoT Applications:}
\begin{tabulary}{\linewidth}{|L|L|L|}
\hline
\textbf{Application} & \textbf{Description} & \textbf{Benefits} \\
\hline
\textbf{Battlefield surveillance} & Real-time combat zone monitoring & Situational awareness \\
\textbf{Asset tracking} & Monitor equipment/vehicles & Logistics optimization \\
\textbf{Soldier health} & Track personnel vital signs & Safety and response \\
\hline
\end{tabulary}
\end{center}

\begin{mnemonicbox}Predict Track Energy, Survey Track Monitor\end{mnemonicbox}
\end{solutionbox}

\vspace{0.5em}\centerline{\textbf{OR}}\questionmarks{5(b)}{ 4 }{List out different types of actuators in IoT and briefly explain working of any two.}

\begin{solutionbox}
\begin{center}
\captionof{table}{IoT Actuator Types:}
\begin{tabulary}{\linewidth}{|L|L|L|}
\hline
\textbf{Actuator Type} & \textbf{Function} & \textbf{Applications} \\
\hline
\textbf{Servo motor} & Angular positioning & Robotics \\
\textbf{Relay} & Electrical switching & Lights, appliances \\
\textbf{Solenoid valve} & Fluid control & Irrigation \\
\textbf{LED} & Light emission & Indicators \\
\textbf{Buzzer} & Sound generation & Alarms \\
\textbf{Stepper motor} & Rotational control & 3D printers \\
\hline
\end{tabulary}
\end{center}

\textbf{Detailed Working:}

\textbf{1. Servo Motor:}
\begin{itemize}
    \item \textbf{Control signal}: PWM signal determines position
    \item \textbf{Feedback}: Internal potentiometer ensures accuracy
    \item \textbf{Working}: Circuit compares desired vs actual position
    \item \textbf{Applications}: Robotic arms, automatic doors
\end{itemize}

\textbf{2. Relay Module:}
\begin{itemize}
    \item \textbf{Principle}: Electromagnet moves mechanical switch
    \item \textbf{Switching}: Connects/disconnects high voltage circuit
    \item \textbf{Isolation}: Safely controls high loads from low voltage MCU
    \item \textbf{Applications}: Home automation switching
\end{itemize}

\begin{mnemonicbox}Servo Relay Solenoid LED Buzzer Stepper\end{mnemonicbox}
\end{solutionbox}

\vspace{0.5em}\centerline{\textbf{OR}}\questionmarks{5(c)}{ 7 }{Draw Block diagram of smart parking system with IoT and Explain its working.}

\begin{solutionbox}
\begin{center}
\begin{tikzpicture}[gtu block]
    \node (space) [draw, rectangle, minimum width=2.5cm] {Parking Space};
    \node (sensor) [draw, rectangle, minimum width=2.5cm, right=0.5cm of space] {Sensor (IR/US)};
    \node (mcu) [draw, rectangle, minimum width=2.5cm, right=0.5cm of sensor] {MCU (NodeMCU)};
    \node (cloud) [draw, rectangle, minimum width=2.5cm, below=1cm of mcu] {Cloud Server};
    \node (app) [draw, rectangle, minimum width=2.5cm, left=0.5cm of cloud] {Mobile App};
    \node (disp) [draw, rectangle, minimum width=2.5cm, right=0.5cm of cloud] {Display Board};
    \node (led) [draw, rectangle, minimum width=2.5cm, above=0.5cm of mcu] {LED Indicator};

    \draw [gtu arrow] (space) -- (sensor);
    \draw [gtu arrow] (sensor) -- (mcu);
    \draw [gtu arrow] (mcu) -- (led);
    \draw [gtu arrow] (mcu) -- node[right] {WiFi} (cloud);
    \draw [gtu arrow] (cloud) -- (app);
    \draw [gtu arrow] (cloud) -- (disp);
\end{tikzpicture}
\end{center}

\textbf{Smart Parking System Working:}

\textbf{1. Vehicle Detection:}
\begin{itemize}
    \item IR/Ultrasonic sensors at each space detect vehicle presence.
    \item Continuous monitoring ensures accurate occupancy status.
\end{itemize}

\textbf{2. Data Collection \& Processing:}
\begin{itemize}
    \item Microcontroller processes sensor data (Occupied/Free).
    \item Validates data to avoid false positives from debris.
\end{itemize}

\textbf{3. Communication:}
\begin{itemize}
    \item WiFi transmits real-time status to cloud server.
    \item Cloud database stores records and performs analytics.
\end{itemize}

\textbf{4. User Services:}
\begin{itemize}
    \item Mobile app allows finding and reserving spaces.
    \item Real-time navigation to available spots.
    \item Online payment integration.
\end{itemize}

\textbf{5. Indicators:}
\begin{itemize}
    \item On-site LED indicators (Red/Green) and display boards.
    \item Admin dashboard for management.
\end{itemize}

\textbf{Benefits:}
\begin{itemize}
    \item \textbf{Time saving}: Quick parking spot location.
    \item \textbf{Traffic reduction}: Less circling.
    \item \textbf{Revenue}: Optimized space utilization.
\end{itemize}

\begin{mnemonicbox}
Detect Process Communicate Interface Indicate Serve
\end{mnemonicbox}
\end{solutionbox}

\end{document}
