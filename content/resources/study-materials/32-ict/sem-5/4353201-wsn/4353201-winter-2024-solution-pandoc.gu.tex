\documentclass[10pt,a4paper]{article}

% content/resources/templates/preamble.tex
\usepackage[margin=0.6in]{geometry}
\author{Milav Dabgar}
\usepackage{amsmath,amssymb,amsthm}
\usepackage{booktabs}
\usepackage{multirow}
\usepackage{xcolor}
\usepackage{tcolorbox}
\tcbuselibrary{breakable,skins}
\usepackage[colorlinks=true,linkcolor=blue]{hyperref}
\usepackage{titlesec}
\usepackage{enumitem}
\usepackage{tikz}
\usepackage{pgfplots}
\usepackage{circuitikz}
\usepackage[version=4]{mhchem}
\usepackage{longtable}
\usepackage{array}
\usepackage{float}
\usepackage{caption}
\usepackage{listings}

\lstset{
  basicstyle=\small\ttfamily,
  breaklines=true,
  breakatwhitespace=false,
  postbreak=\mbox{\textcolor{red}{$\hookrightarrow$}\space},
  float=false,
  numbers=left,
  numberstyle=\tiny\color{gray},
  numbersep=10pt,
  xleftmargin=2em,
  keywordstyle=\color{blue},
  commentstyle=\color{green!60!black},
  stringstyle=\color{purple},
  backgroundcolor=\color{gray!5},
  showstringspaces=false,
  tabsize=2,
  captionpos=b,
  keepspaces=true,
  columns=flexible
}

\pgfplotsset{compat=1.18}
\usetikzlibrary{shapes,arrows,positioning,calc,patterns,decorations.pathmorphing,decorations.markings,arrows.meta}

% Color scheme
\definecolor{headcolor}{RGB}{0,102,204}
\definecolor{keycolor}{RGB}{220,20,60}
\definecolor{solutioncolor}{RGB}{34,139,34}
\definecolor{mnemoniccolor}{RGB}{148,0,211}
\definecolor{codecolor}{RGB}{0,0,100}

% Spacing
\setlength{\parskip}{3pt}
\setlist[itemize]{nosep}
\setlist[enumerate]{nosep}

% Title formatting
\titleformat{\section}{\Large\bfseries\color{headcolor}}{\thesection}{1em}{}
\titleformat{\subsection}{\large\bfseries\color{headcolor}}{\thesubsection}{1em}{}

% Pandoc tightlist compatibility
\providecommand{\tightlist}{%
  \setlength{\itemsep}{0pt}\setlength{\parskip}{0pt}}

% Pandoc longtable compatibility
\newcounter{none}
\def\thenone{}


% content/resources/templates/gujarati-boxes.tex
\usepackage{fontspec}
\usepackage{polyglossia}

% Set Gujarati as main language (document is primarily in Gujarati)
% Note: gloss-gujarati.ldf doesn't exist in polyglossia, but it will use hyphenation patterns
\setdefaultlanguage{gujarati}
\setotherlanguage{english}

% Configure Gujarati font properly
% Use Language=Default to prevent polyglossia from trying to add language-specific features
% that don't exist for Gujarati, which causes "empty feature" warnings
\newfontfamily\gujaratifont[Script=Gujarati,AutoFakeBold=2.5,AutoFakeSlant=0.3]{Noto Sans Gujarati}
\setmainfont[Script=Gujarati,AutoFakeBold=2.5,AutoFakeSlant=0.3]{Noto Sans Gujarati}
% Use Noto Sans Gujarati for monospace to support Gujarati in text
\setmonofont[Scale=0.9]{Noto Sans Gujarati}

% Configure English to use the same font
\newfontfamily\englishfont[Script=Gujarati,AutoFakeBold=2.5,AutoFakeSlant=0.3]{Noto Sans Gujarati}

% Translations for polyglossia
\gappto\captionsgujarati{
  \renewcommand{\tablename}{કોષ્ટક}
  \renewcommand{\figurename}{આકૃતિ}
}

% Helper for TikZ nodes to ensure Gujarati font
\newcommand{\gu}[1]{{\gujaratifont #1}}

% Custom environments
\newtcolorbox{solutionbox}{
    breakable,
    enhanced,
    colback=solutioncolor!5!white,
    colframe=solutioncolor!75!black,
    fonttitle=\bfseries,
    title=જવાબ
}

\newtcolorbox{solutionboxnobreak}{
 colback=solutioncolor!5!white,
 colframe=solutioncolor!75!black,
 fonttitle=\bfseries,
 title=જવાબ
}

\newtcolorbox{keyformula}{
 breakable,
 enhanced,
 colback=keycolor!5!white,
 colframe=keycolor!75!black,
 fonttitle=\bfseries,
 title=રાસાયણિક સમીકરણ/સૂત્ર
}

\newtcolorbox{mnemonicbox}{
 breakable,
 enhanced,
 colback=mnemoniccolor!5!white,
 colframe=mnemoniccolor!75!black,
 fonttitle=\bfseries,
 title=મેમરી ટ્રીક
}


\begin{document}

\begin{center}
{\Huge\bfseries\color{headcolor} Subject Name (Gujarati)}\\[5pt]
{\LARGE 4353201 -- Winter 2024}\\[3pt]
{\large Semester 1 Study Material}\\[3pt]
{\normalsize\textit{Detailed Solutions and Explanations}}
\end{center}

\vspace{10pt}

\subsection*{પ્રશ્ન 1(અ) [3
ગુણ]}\label{uxaaauxab0uxab6uxaa8-1uxa85-3-uxa97uxaa3}

\textbf{સિંગલ હોપ અને મલ્ટિહોપ નેટવર્કની સરખામણી કરો.}

\begin{solutionbox}

{\def\LTcaptype{none} % do not increment counter
\begin{longtable}[]{@{}lll@{}}
\toprule\noalign{}
પેરામીટર & સિંગલ હોપ નેટવર્ક & મલ્ટિહોપ નેટવર્ક \\
\midrule\noalign{}
\endhead
\bottomrule\noalign{}
\endlastfoot
\textbf{કમ્યુનિકેશન} & સીધું બેઝ સ્ટેશન સાથે & મધ્યવર્તી નોડ્સ દ્વારા \\
\textbf{એનર્જી વપરાશ} & દૂરના નોડ્સ માટે વધુ & નોડ્સ વચ્ચે વિતરિત \\
\textbf{નેટવર્ક કવરેજ} & ટ્રાન્સમિશન રેન્જ દ્વારા મર્યાદિત & વિસ્તૃત કવરેજ વિસ્તાર \\
\textbf{જટિલતા} & સરળ રાઉટિંગ & જટિલ રાઉટિંગ પ્રોટોકોલ \\
\end{longtable}
}

\begin{itemize}
\tightlist
\item
  \textbf{સિંગલ હોપ}: બધા નોડ્સ બેઝ સ્ટેશન સાથે સીધો સંપર્ક કરે છે
\item
  \textbf{મલ્ટિહોપ}: ડેટા ગંતવ્ય સુધી પહોંચવા માટે અનેક મધ્યવર્તી નોડ્સમાંથી પસાર
  થાય છે
\end{itemize}

\end{solutionbox}
\begin{mnemonicbox}
``સિંગલ ડાયરેક્ટ, મલ્ટિ રિલે''

\end{mnemonicbox}
\subsection*{પ્રશ્ન 1(બ) [4
ગુણ]}\label{uxaaauxab0uxab6uxaa8-1uxaac-4-uxa97uxaa3}

\textbf{સેન્સર નોડના મૂળભૂત ઘટકો સમજાવો.}

\begin{solutionbox}

\includegraphics[width=1\linewidth,height=\textheight,keepaspectratio]{mermaid-57c7b82d.pdf}

\textbf{મૂળભૂત ઘટકો:}

\begin{itemize}
\tightlist
\item
  \textbf{સેન્સિંગ સબસિસ્ટમ}: સેન્સર્સ અને ADC નો ઉપયોગ કરીને પર્યાવરણમાંથી ડેટા
  એકત્રિત કરે છે
\item
  \textbf{પ્રોસેસિંગ સબસિસ્ટમ}: ડેટા પ્રોસેસિંગ માટે મેમોરી સાથે
  માઇક્રોકંટ્રોલર/પ્રોસેસર
\item
  \textbf{કમ્યુનિકેશન સબસિસ્ટમ}: વાયરલેસ ડેટા ટ્રાન્સમિશન માટે રેડિયો ટ્રાન્સીવર
\item
  \textbf{પાવર સબસિસ્ટમ}: પાવર સપ્લાય માટે બેટરી અથવા એનર્જી હાર્વેસ્ટિંગ યુનિટ
\end{itemize}

\end{solutionbox}
\begin{mnemonicbox}
``સેન્સ પ્રોસેસ કમ્યુનિકેટ પાવર''

\end{mnemonicbox}
\subsection*{પ્રશ્ન 1(ક) [7
ગુણ]}\label{uxaaauxab0uxab6uxaa8-1uxa95-7-uxa97uxaa3}

\textbf{WSN માં પાવર કન્ઝમ્પશન ઘટાડવા માટે કોઈપણ ચાર ટેકનોલોજીની યાદી બનાવો
અને કોઈપણ બે ટેકનોલોજીને વિગતવાર સમજાવો.}

\begin{solutionbox}

\textbf{ચાર પાવર રિડક્શન ટેકનોલોજીઓ:}

{\def\LTcaptype{none} % do not increment counter
\begin{longtable}[]{@{}ll@{}}
\toprule\noalign{}
ટેકનોલોજી & વર્ણન \\
\midrule\noalign{}
\endhead
\bottomrule\noalign{}
\endlastfoot
\textbf{સ્લીપ શેડ્યુલિંગ} & નોડ્સ સક્રિય અને સ્લીપ મોડ વચ્ચે ફેરફાર કરે છે \\
\textbf{ડેટા એગ્રિગેશન} & અનેક ડેટા પેકેટ્સને એક જ ટ્રાન્સમિશનમાં જોડે છે \\
\textbf{ટોપોલોજી કંટ્રોલ} & એનર્જી ઘટાડવા માટે નેટવર્ક સ્ટ્રક્ચર ઓપ્ટિમાઇઝ કરે છે \\
\textbf{એનર્જી હાર્વેસ્ટિંગ} & સોલાર, વાઇબ્રેશન જેવા રિન્યુએબલ સોર્સનો ઉપયોગ કરે
છે \\
\end{longtable}
}

\textbf{વિગતવાર સમજૂતી:}

\textbf{1. સ્લીપ શેડ્યુલિંગ:}

\begin{itemize}
\tightlist
\item
  \textbf{એક્ટિવ મોડ}: નોડ સેન્સિંગ, પ્રોસેસિંગ, કમ્યુનિકેશન કરે છે
\item
  \textbf{સ્લીપ મોડ}: નોડ બિનજરૂરી ઘટકોને પાવર ડાઉન કરે છે
\item
  \textbf{ફાયદા}: આઇડલ લિસનિંગ એનર્જી કન્ઝમ્પશન 90\% સુધી ઘટાડે છે
\end{itemize}

\textbf{2. ડેટા એગ્રિગેશન:}

\begin{itemize}
\tightlist
\item
  \textbf{પ્રક્રિયા}: મધ્યવર્તી નોડ્સ પર અનેક સેન્સર રીડિંગ્સ જોડવામાં આવે છે
\item
  \textbf{ટેકનિક્સ}: એવરેજ, મેક્સિમમ, મિનિમમ ફંક્શન્સ લાગુ કરવામાં આવે છે
\item
  \textbf{ફાયદો}: કુલ ટ્રાન્સમિશનની સંખ્યા નોંધપાત્ર રીતે ઘટાડે છે
\end{itemize}

\end{solutionbox}
\begin{mnemonicbox}
``સ્લીપ એગ્રિગેટ ટોપોલોજી હાર્વેસ્ટ''

\end{mnemonicbox}
\subsection*{પ્રશ્ન 1(ક) OR [7
ગુણ]}\label{uxaaauxab0uxab6uxaa8-1uxa95-or-7-uxa97uxaa3}

\textbf{વાયરલેસ સેન્સર નેટવર્કના કોઈપણ ચાર પડકારોની યાદી બનાવો અને કોઈપણ બેને
વિગતવાર સમજાવો.}

\begin{solutionbox}

\textbf{ચાર WSN પડકારો:}

{\def\LTcaptype{none} % do not increment counter
\begin{longtable}[]{@{}ll@{}}
\toprule\noalign{}
પડકાર & અસર \\
\midrule\noalign{}
\endhead
\bottomrule\noalign{}
\endlastfoot
\textbf{મર્યાદિત એનર્જી} & નેટવર્ક લાઇફટાઇમને અસર કરે છે \\
\textbf{મર્યાદિત બેન્ડવિડ્થ} & ડેટા ટ્રાન્સમિશનને મર્યાદિત કરે છે \\
\textbf{સિક્યુરિટી વલ્નરેબિલિટીઝ} & ડેટા ઇન્ટેગ્રિટીને જોખમમાં મૂકે છે \\
\textbf{સ્કેલેબિલિટી ઇશ્યુઝ} & મોટા નેટવર્ક પરફોર્મન્સને અસર કરે છે \\
\end{longtable}
}

\textbf{વિગતવાર સમજૂતી:}

\textbf{1. મર્યાદિત એનર્જી:}

\begin{itemize}
\tightlist
\item
  \textbf{બેટરી કન્સ્ટ્રેઈન્ટ}: નોડ્સ મર્યાદિત કેપેસિટી સાથે નાની બેટરીઓ પર કામ કરે છે
\item
  \textbf{એનર્જી ડિપ્લીશન}: ટ્રાન્સમિશન અને રિસેપ્શન દરમિયાન ઉચ્ચ એનર્જી વપરાશ
\item
  \textbf{સોલ્યુશન એપ્રોચ}: પાવર મેનેજમેન્ટ પ્રોટોકોલ્સ, એનર્જી-એફિશિયન્ટ રાઉટિંગ
\end{itemize}

\textbf{2. સિક્યુરિટી વલ્નરેબિલિટીઝ:}

\begin{itemize}
\tightlist
\item
  \textbf{ફિઝિકલ એટેક્સ}: નોડ્સને ભૌતિક રીતે કેપ્ચર અથવા નુકસાન થઈ શકે છે
\item
  \textbf{નેટવર્ક એટેક્સ}: ઇવ્સડ્રોપિંગ, જેમિંગ, ડિનાયલ ઓફ સર્વિસ એટેક્સ
\item
  \textbf{કાઉન્ટરમેઝર્સ}: એન્ક્રિપ્શન, ઓથેન્ટિકેશન, સિક્યોર રાઉટિંગ પ્રોટોકોલ્સ
\end{itemize}

\end{solutionbox}
\begin{mnemonicbox}
``એનર્જી બેન્ડવિડ્થ સિક્યુરિટી સ્કેલ''

\end{mnemonicbox}
\subsection*{પ્રશ્ન 2(અ) [3
ગુણ]}\label{uxaaauxab0uxab6uxaa8-2uxa85-3-uxa97uxaa3}

\textbf{``IEEE 802.15.4 સ્ટાન્ડર્ડ અને ZigBee સ્પેસિફિકેશન્સ વાયરલેસ સેન્સર નેટવર્ક
માટે લોકપ્રિય પ્રોટોકોલ પસંદગીઓ છે'' - જસ્ટિફાઈ}

\begin{solutionbox}

\textbf{જસ્ટિફિકેશન ટેબલ:}

{\def\LTcaptype{none} % do not increment counter
\begin{longtable}[]{@{}ll@{}}
\toprule\noalign{}
ફીચર & WSN માટે ફાયદો \\
\midrule\noalign{}
\endhead
\bottomrule\noalign{}
\endlastfoot
\textbf{લો પાવર કન્ઝમ્પશન} & બેટરી લાઇફ વધારે છે \\
\textbf{લો ડેટા રેટ} & સેન્સર ડેટા માટે યોગ્ય \\
\textbf{શોર્ટ રેન્જ} & ક્લસ્ટર્ડ સેન્સર્સ માટે પરફેક્ટ \\
\textbf{લો કોસ્ટ} & મોટા ડિપ્લોયમેન્ટ માટે આર્થિક \\
\end{longtable}
}

\begin{itemize}
\tightlist
\item
  \textbf{IEEE 802.15.4}: PHY અને MAC લેયર સ્પેસિફિકેશન્સ પ્રદાન કરે છે
\item
  \textbf{ZigBee}: ટોચ પર નેટવર્ક અને એપ્લિકેશન લેયર્સ ઉમેરે છે
\item
  \textbf{પરફેક્ટ મેચ}: WSN આવશ્યકતાઓ પ્રોટોકોલ ક્ષમતાઓ સાથે સંરેખિત થાય છે
\end{itemize}

\end{solutionbox}
\begin{mnemonicbox}
``લો પાવર, લો ડેટા, લો કોસ્ટ, લો રેન્જ''

\end{mnemonicbox}
\subsection*{પ્રશ્ન 2(બ) [4
ગુણ]}\label{uxaaauxab0uxab6uxaa8-2uxaac-4-uxa97uxaa3}

\textbf{યોગ્ય ઉદાહરણની મદદથી એનર્જી એફિશિયન્ટ રાઉટિંગ સમજાવો}

\begin{solutionbox}

\includegraphics[width=1\linewidth,height=\textheight,keepaspectratio]{mermaid-765054f9.pdf}

\textbf{એનર્જી એફિશિયન્ટ રાઉટિંગ:}

\begin{itemize}
\tightlist
\item
  \textbf{ઉદ્દેશ્ય}: નેટવર્ક લાઇફટાઇમ મહત્તમ કરતા પાથ્સ પસંદ કરો
\item
  \textbf{એપ્રોચ}: નોડ્સના બાકી બેટરી લેવલ્સ ધ્યાનમાં લો
\item
  \textbf{ઉદાહરણ}: નોડ 2 (30\% બેટરી) ને બદલે નોડ 1 (80\% બેટરી) દ્વારા રૂટ
  કરો
\end{itemize}

\textbf{મુખ્ય ટેકનિક્સ:}

\begin{itemize}
\tightlist
\item
  \textbf{બેટરી અવેરનેસ}: બાકી એનર્જી લેવલ્સનું નિરીક્ષણ કરો
\item
  \textbf{લોડ બેલેન્સિંગ}: અનેક પાથ્સ વચ્ચે ટ્રાફિક વિતરણ કરો
\item
  \textbf{ક્લસ્ટરિંગ}: લાંબા-અંતરના ટ્રાન્સમિશન ઘટાડવા માટે નજીકના નોડ્સને ગ્રુપ કરો
\end{itemize}

\end{solutionbox}
\begin{mnemonicbox}
``બેટરી બેલેન્સ ક્લસ્ટર''

\end{mnemonicbox}
\subsection*{પ્રશ્ન 2(ક) [7
ગુણ]}\label{uxaaauxab0uxab6uxaa8-2uxa95-7-uxa97uxaa3}

\textbf{યોગ્ય સ્કેચની મદદથી LEACH પ્રોટોકોલના સેટઅપ અને સ્ટેડી સ્ટેટ ફેઝ સમજાવો.}

\begin{solutionbox}

\includegraphics[width=1\linewidth,height=\textheight,keepaspectratio]{mermaid-bff68e26.pdf}

\textbf{LEACH પ્રોટોકોલ ફેઝિસ:}

\textbf{સેટઅપ ફેઝ:}

\begin{itemize}
\tightlist
\item
  \textbf{ક્લસ્ટર હેડ સિલેક્શન}: પ્રોબેબિલિટી થ્રેશોલ્ડ આધારિત રેન્ડમ સિલેક્શન
\item
  \textbf{એડવર્ટાઇઝમેન્ટ}: પસંદ કરેલા CHs એનાઉન્સમેન્ટ મેસેજિસ બ્રોડકાસ્ટ કરે છે
\item
  \textbf{ક્લસ્ટર ફોર્મેશન}: નોન-CH નોડ્સ નજીકના ક્લસ્ટર હેડમાં જોડાય છે
\item
  \textbf{શેડ્યુલ ક્રિએશન}: CH ક્લસ્ટર મેમ્બર્સ માટે TDMA શેડ્યુલ બનાવે છે
\end{itemize}

\textbf{સ્ટેડી સ્ટેટ ફેઝ:}

\begin{itemize}
\tightlist
\item
  \textbf{ડેટા ટ્રાન્સમિશન}: નોડ્સ TDMA શેડ્યુલ અનુસાર CH ને ડેટા મોકલે છે
\item
  \textbf{ડેટા એગ્રિગેશન}: CH ક્લસ્ટર મેમ્બર્સ પાસેથી પ્રાપ્ત ડેટાને જોડે છે
\item
  \textbf{ડેટા ફોરવર્ડિંગ}: CH એગ્રિગેટેડ ડેટાને બેઝ સ્ટેશન પર ટ્રાન્સમિટ કરે છે
\end{itemize}

\textbf{ફાયદા:}

\begin{itemize}
\tightlist
\item
  \textbf{એનર્જી ડિસ્ટ્રિબ્યુશન}: નોડ્સ વચ્ચે CH રોલ રોટેટ કરે છે
\item
  \textbf{કોલિઝન એવોઇડન્સ}: TDMA શેડ્યુલિંગ ઇન્ટરફેરન્સ અટકાવે છે
\end{itemize}

\end{solutionbox}
\begin{mnemonicbox}
``સિલેક્ટ એડવર્ટાઇઝ જોઇન શેડ્યુલ, સેન્ડ એગ્રિગેટ ફોરવર્ડ''

\end{mnemonicbox}
\subsection*{પ્રશ્ન 2(અ) OR [3
ગુણ]}\label{uxaaauxab0uxab6uxaa8-2uxa85-or-3-uxa97uxaa3}

\textbf{વાયરલેસ સેન્સર નેટવર્કમાં રાઉટિંગ પ્રોટોકોલ્સનું વર્ગીકરણ આપો.}

\begin{solutionbox}

\textbf{WSN રાઉટિંગ પ્રોટોકોલ વર્ગીકરણ:}

{\def\LTcaptype{none} % do not increment counter
\begin{longtable}[]{@{}ll@{}}
\toprule\noalign{}
વર્ગીકરણ આધાર & પ્રકારો \\
\midrule\noalign{}
\endhead
\bottomrule\noalign{}
\endlastfoot
\textbf{નેટવર્ક સ્ટ્રક્ચર} & ફ્લેટ, હાઇરાર્કિકલ, લોકેશન-બેઝ્ડ \\
\textbf{પ્રોટોકોલ ઓપરેશન} & મલ્ટિપાથ, ક્વેરી-બેઝ્ડ, નેગોસિએશન-બેઝ્ડ \\
\textbf{પાથ એસ્ટેબ્લિશમેન્ટ} & પ્રોએક્ટિવ, રિએક્ટિવ, હાઇબ્રિડ \\
\end{longtable}
}

\textbf{મુખ્ય કેટેગરીઝ:}

\begin{itemize}
\tightlist
\item
  \textbf{ફ્લેટ રાઉટિંગ}: બધા નોડ્સની સમાન ભૂમિકા (જેમ કે, ફ્લડિંગ, SPIN)
\item
  \textbf{હાઇરાર્કિકલ રાઉટિંગ}: ક્લસ્ટર-બેઝ્ડ એપ્રોચ (જેમ કે, LEACH, TEEN)
\item
  \textbf{લોકેશન-બેઝ્ડ રાઉટિંગ}: જિયોગ્રાફિક ઇન્ફોર્મેશનનો ઉપયોગ (જેમ કે, GEAR)
\end{itemize}

\end{solutionbox}
\begin{mnemonicbox}
``ફ્લેટ હાઇરાર્કિકલ લોકેશન''

\end{mnemonicbox}
\subsection*{પ્રશ્ન 2(બ) OR [4
ગુણ]}\label{uxaaauxab0uxab6uxaa8-2uxaac-or-4-uxa97uxaa3}

\textbf{સ્કેચની મદદથી લો ડ્યુટી સાઇકલ પ્રોટોકોલના વેકઅપ કોન્સેપ્ટને સમજાવો.}

\begin{solutionbox}

\begin{lstlisting}
Time -->
Node A: [Sleep]---[Wake]--[Listen]--[Sleep]---[Wake]--[Listen]--[Sleep]
Node B: [Sleep]-----[Wake]--[Tx]--[Sleep]-----[Wake]--[Listen]--[Sleep]
         |     |     |     |   |     |       |     |     |       |
         0    T1    T2    T3  T4    T5      T6    T7    T8      T9
\end{lstlisting}

\textbf{લો ડ્યુટી સાઇકલ વેકઅપ કોન્સેપ્ટ:}

\begin{itemize}
\tightlist
\item
  \textbf{સ્લીપ પીરિયડ}: એનર્જી બચાવવા માટે નોડ્સ રેડિયો બંધ કરે છે
\item
  \textbf{વેક પીરિયડ}: નોડ્સ સમયાંતરે કમ્યુનિકેશન ચેક કરવા માટે જાગે છે
\item
  \textbf{સિંક્રોનાઇઝેશન}: સેન્ડરને રિસીવરના વેકઅપ શેડ્યુલની જાણ હોવી જરૂરી
\end{itemize}

\textbf{મુખ્ય ફાયદા:}

\begin{itemize}
\tightlist
\item
  \textbf{એનર્જી સેવિંગ્સ}: આઇડલ લિસનિંગ 99\% સુધી ઘટાડે છે
\item
  \textbf{કોઓર્ડિનેટેડ એક્સેસ}: વેકઅપ પીરિયડ દરમિયાન કોલિઝન અટકાવે છે
\end{itemize}

\end{solutionbox}
\begin{mnemonicbox}
``સ્લીપ વેક લિસન રિપીટ''

\end{mnemonicbox}
\subsection*{પ્રશ્ન 2(ક) OR [7
ગુણ]}\label{uxaaauxab0uxab6uxaa8-2uxa95-or-7-uxa97uxaa3}

\textbf{S-MAC પ્રોટોકોલના Synch, RTS અને CTS તબક્કાઓ અને તેના મેસેજ પાસિંગ એપ્રોચ
સમજાવો.}

\begin{solutionbox}

\includegraphics[width=1\linewidth,height=\textheight,keepaspectratio]{mermaid-a5968456.pdf}

\textbf{S-MAC પ્રોટોકોલ ફેઝિસ:}

\textbf{1. સિંક્રોનાઇઝેશન ફેઝ:}

\begin{itemize}
\tightlist
\item
  \textbf{હેતુ}: સામાન્ય સ્લીપ/વેક શેડ્યુલ સ્થાપિત કરવું
\item
  \textbf{પ્રક્રિયા}: નોડ્સ શેડ્યુલ ઇન્ફોર્મેશન સાથે SYNC પેકેટ્સનું વિનિમય કરે છે
\item
  \textbf{ફાયદો}: નેટવર્ક વ્યાપી કોઓર્ડિનેટેડ સ્લીપ પેટર્ન સુનિશ્ચિત કરે છે
\end{itemize}

\textbf{2. RTS ફેઝ (રિક્વેસ્ટ ટુ સેન્ડ):}

\begin{itemize}
\tightlist
\item
  \textbf{શરૂઆત}: સેન્ડર ઇન્ટેન્ડેડ રિસીવર ને RTS પેકેટ ટ્રાન્સમિટ કરે છે
\item
  \textbf{કન્ટેન્ટ}: સોર્સ એડ્રેસ, ડેસ્ટિનેશન એડ્રેસ, ટ્રાન્સમિશન ડ્યુરેશન
\end{itemize}

\textbf{3. CTS ફેઝ (ક્લિયર ટુ સેન્ડ):}

\begin{itemize}
\tightlist
\item
  \textbf{રિસ્પોન્સ}: રિસીવર ઉપલબ્ધતાની પુષ્ટિ કરતું CTS પેકેટ મોકલે છે
\item
  \textbf{વર્ચ્યુઅલ સેન્સિંગ}: પડોશી નોડ્સ CTS સાંભળે છે અને ટ્રાન્સમિશન મુલતવી રાખે છે
\end{itemize}

\textbf{મેસેજ પાસિંગ એપ્રોચ:}

\begin{itemize}
\tightlist
\item
  \textbf{કોલિઝન એવોઇડન્સ}: RTS/CTS હેન્ડશેક હિડન ટર્મિનલ પ્રોબ્લેમ અટકાવે છે
\item
  \textbf{એનર્જી કન્ઝર્વેશન}: ઓવરહિયરિંગ નોડ્સ ડેટા એક્સચેન્જ દરમિયાન સ્લીપ મોડમાં
  જાય છે
\item
  \textbf{પીરિયોડિક સિંક્રોનાઇઝેશન}: નેટવર્ક-વાઇડ શેડ્યુલ કોઓર્ડિનેશન જાળવે છે
\end{itemize}

\end{solutionbox}
\begin{mnemonicbox}
``સિંક રિક્વેસ્ટ ક્લિયર ટ્રાન્સમિટ''

\end{mnemonicbox}
\subsection*{પ્રશ્ન 3(અ) [3
ગુણ]}\label{uxaaauxab0uxab6uxaa8-3uxa85-3-uxa97uxaa3}

\textbf{IEEE 802.15.4 સ્ટાન્ડર્ડનું સુપર ફ્રેમ સ્ટ્રક્ચર સમજાવો.}

\begin{solutionbox}

\begin{lstlisting}
|<-------------- Super Frame (15.36 ms) -------------->|
|<---CAP--->|<----------CFP---------->|<--Inactive-->|
| Beacon |Slot|Slot|Slot|GTS|GTS|GTS|    Period    |
|   8    | 0 | 1 | 2 | 1 | 2 | 3 |              |
\end{lstlisting}

\textbf{સુપર ફ્રેમ ઘટકો:}

{\def\LTcaptype{none} % do not increment counter
\begin{longtable}[]{@{}lll@{}}
\toprule\noalign{}
ઘટક & વર્ણન & અવધિ \\
\midrule\noalign{}
\endhead
\bottomrule\noalign{}
\endlastfoot
\textbf{બીકન} & નેટવર્ક સિંક્રોનાઇઝેશન & નિશ્ચિત \\
\textbf{CAP} & કન્ટેન્શન એક્સેસ પીરિયડ & ચલ \\
\textbf{CFP} & કન્ટેન્શન ફ્રી પીરિયડ & ચલ \\
\textbf{ઇનએક્ટિવ} & સ્લીપ પીરિયડ & ચલ \\
\end{longtable}
}

\begin{itemize}
\tightlist
\item
  \textbf{CAP}: ચેનલ એક્સેસ માટે CSMA/CA નો ઉપયોગ કરે છે
\item
  \textbf{CFP}: રિયલ-ટાઇમ ડેટા માટે GTS (ગેરેન્ટીડ ટાઇમ સ્લોટ્સ) નો ઉપયોગ કરે છે
\item
  \textbf{ઇનએક્ટિવ પીરિયડ}: ડિવાઇસિસ લો-પાવર મોડમાં જઈ શકે છે
\end{itemize}

\end{solutionbox}
\begin{mnemonicbox}
``બીકન કન્ટેન્ડ ગેરેન્ટી સ્લીપ''

\end{mnemonicbox}
\subsection*{પ્રશ્ન 3(બ) [4
ગુણ]}\label{uxaaauxab0uxab6uxaa8-3uxaac-4-uxa97uxaa3}

\textbf{M2M અને IoT ટેકનોલોજીની સરખામણી કરો.}

\begin{solutionbox}

{\def\LTcaptype{none} % do not increment counter
\begin{longtable}[]{@{}lll@{}}
\toprule\noalign{}
પેરામીટર & M2M & IoT \\
\midrule\noalign{}
\endhead
\bottomrule\noalign{}
\endlastfoot
\textbf{કમ્યુનિકેશન} & પોઇન્ટ-ટુ-પોઇન્ટ & ઇન્ટરનેટ-બેઝ્ડ \\
\textbf{ડેટા પ્રોસેસિંગ} & લોકલ & ક્લાઉડ-બેઝ્ડ \\
\textbf{કનેક્ટિવિટી} & સેલ્યુલર/વાયર્ડ & અનેક પ્રોટોકોલ્સ \\
\textbf{એપ્લિકેશન્સ} & વિશિષ્ટ ઇન્ડસ્ટ્રીઝ & કન્ઝ્યુમર અને ઇન્ડસ્ટ્રિયલ \\
\end{longtable}
}

\textbf{મુખ્ય તફાવતો:}

\begin{itemize}
\tightlist
\item
  \textbf{M2M}: મશીન-ટુ-મશીન ડાયરેક્ટ કમ્યુનિકેશન
\item
  \textbf{IoT}: ક્લાઉડ ઇન્ટિગ્રેશન સાથે ઇન્ટરનેટ ઓફ થિંગ્સ
\item
  \textbf{સ્કોપ}: M2M એ વ્યાપક IoT ઇકોસિસ્ટમનો ઉપસમૂહ છે
\item
  \textbf{ઇન્ટેલિજન્સ}: IoT વધુ એડવાન્સ્ડ એનાલિટિક્સ અને AI પ્રદાન કરે છે
\end{itemize}

\end{solutionbox}
\begin{mnemonicbox}
``M2M ડાયરેક્ટ, IoT ઇન્ટરનેટ''

\end{mnemonicbox}
\subsection*{પ્રશ્ન 3(ક) [7
ગુણ]}\label{uxaaauxab0uxab6uxaa8-3uxa95-7-uxa97uxaa3}

\textbf{IoT આર્કિટેક્ચરનો બ્લોક ડાયાગ્રામ દોરો અને તેને સમજાવો}

\begin{solutionbox}

\includegraphics[width=1\linewidth,height=\textheight,keepaspectratio]{mermaid-dd578ded.pdf}

\textbf{IoT આર્કિટેક્ચર લેયર્સ:}

\textbf{1. ફિઝિકલ લેયર:}

\begin{itemize}
\tightlist
\item
  \textbf{ઘટકો}: સેન્સર્સ (તાપમાન, ભેજ), એક્ચ્યુએટર્સ (મોટર્સ, વાલ્વ્સ)
\item
  \textbf{કાર્ય}: ભૌતિક પર્યાવરણમાંથી ડેટા કલેક્શન
\end{itemize}

\textbf{2. કનેક્ટિવિટી લેયર:}

\begin{itemize}
\tightlist
\item
  \textbf{પ્રોટોકોલ્સ}: WiFi, Bluetooth, Zigbee, LoRaWAN, સેલ્યુલર
\item
  \textbf{કાર્ય}: ડિવાઇસિસમાંથી પ્રોસેસિંગ સેન્ટર્સ સુધી ડેટા ટ્રાન્સમિટ કરવું
\end{itemize}

\textbf{3. ડેટા પ્રોસેસિંગ લેયર:}

\begin{itemize}
\tightlist
\item
  \textbf{ટેકનોલોજીઝ}: એજ કમ્પ્યુટિંગ, ફોગ કમ્પ્યુટિંગ
\item
  \textbf{કાર્ય}: સેન્સર ડેટાની રિયલ-ટાઇમ પ્રોસેસિંગ અને ફિલ્ટરિંગ
\end{itemize}

\textbf{4. ડેટા એક્યુમ્યુલેશન લેયર:}

\begin{itemize}
\tightlist
\item
  \textbf{ઇન્ફ્રાસ્ટ્રક્ચર}: ક્લાઉડ સ્ટોરેજ, ડેટા વેરહાઉસિસ
\item
  \textbf{કાર્ય}: IoT ડેટાના વિશાળ પ્રમાણને સ્ટોર કરવું
\end{itemize}

\textbf{5. ડેટા એબ્સ્ટ્રેક્શન લેયર:}

\begin{itemize}
\tightlist
\item
  \textbf{ઘટકો}: ડેટાબેસિસ, ડેટા એનાલિટિક્સ એન્જિન્સ
\item
  \textbf{કાર્ય}: એપ્લિકેશન્સ માટે ડેટાને ઓર્ગેનાઇઝ અને તૈયાર કરવું
\end{itemize}

\textbf{6. એપ્લિકેશન લેયર:}

\begin{itemize}
\tightlist
\item
  \textbf{સર્વિસિસ}: વેબ એપ્લિકેશન્સ, મોબાઇલ એપ્સ, ડેશબોર્ડ્સ
\item
  \textbf{કાર્ય}: યુઝર ઇન્ટરફેસિસ અને બિઝનેસ લોજિક પ્રદાન કરવું
\end{itemize}

\textbf{7. કોલાબોરેશન લેયર:}

\begin{itemize}
\tightlist
\item
  \textbf{ઇન્ટિગ્રેશન}: ERP સિસ્ટમ્સ, બિઝનેસ પ્રોસેસિસ
\item
  \textbf{કાર્ય}: વિવિધ સ્ટેકહોલ્ડર્સ વચ્ચે કોલાબોરેશન સક્ષમ કરવું
\end{itemize}

\end{solutionbox}
\begin{mnemonicbox}
``ફિઝિકલ કનેક્ટ પ્રોસેસ એક્યુમ્યુલેટ એબ્સ્ટ્રેક્ટ એપ્લાઈ કોલાબોરેટ''

\end{mnemonicbox}
\subsection*{પ્રશ્ન 3(અ) OR [3
ગુણ]}\label{uxaaauxab0uxab6uxaa8-3uxa85-or-3-uxa97uxaa3}

\textbf{MAC પ્રોટોકોલની એનર્જી સમસ્યાઓ સમજાવો}

\begin{solutionbox}

\textbf{MAC પ્રોટોકોલ્સમાં એનર્જી સમસ્યાઓ:}

{\def\LTcaptype{none} % do not increment counter
\begin{longtable}[]{@{}lll@{}}
\toprule\noalign{}
સમસ્યા & વર્ણન & અસર \\
\midrule\noalign{}
\endhead
\bottomrule\noalign{}
\endlastfoot
\textbf{આઇડલ લિસનિંગ} & કમ્યુનિકેશન વિના રેડિયો ચાલુ રહે છે & 50-60\% એનર્જી
વેસ્ટ \\
\textbf{કોલિઝન} & અનેક ટ્રાન્સમિશન્સ ઇન્ટરફેર કરે છે & રિટ્રાન્સમિશન ઓવરહેડ \\
\textbf{ઓવરહિયરિંગ} & અપ્રસ્તુત પેકેટ્સ પ્રાપ્ત કરવું & બિનજરૂરી એનર્જી વપરાશ \\
\end{longtable}
}

\textbf{મુખ્ય મુદ્દાઓ:}

\begin{itemize}
\tightlist
\item
  \textbf{આઇડલ લિસનિંગ}: WSN માં સૌથી વધુ એનર્જી-વપરાતી પ્રવૃત્તિ
\item
  \textbf{પ્રોટોકોલ ઓવરહેડ}: કંટ્રોલ પેકેટ્સ વધારાની એનર્જી વાપરે છે
\item
  \textbf{પૂર ગરીબ શેડ્યુલિંગ}: બિનકાર્યક્ષમ ચેનલ એક્સેસ એનર્જી વધારે છે
\end{itemize}

\end{solutionbox}
\begin{mnemonicbox}
``આઇડલ કોલાઇડ ઓવરહિયર''

\end{mnemonicbox}
\subsection*{પ્રશ્ન 3(બ) OR [4
ગુણ]}\label{uxaaauxab0uxab6uxaa8-3uxaac-or-4-uxa97uxaa3}

\textbf{IoT સિસ્ટમ માટે મોડિફાઇડ OSI મોડેલ સમજાવો}

\begin{solutionbox}

\textbf{IoT માટે મોડિફાઇડ OSI મોડેલ:}

{\def\LTcaptype{none} % do not increment counter
\begin{longtable}[]{@{}lll@{}}
\toprule\noalign{}
લેયર & પરંપરાગત OSI & IoT મોડિફિકેશન \\
\midrule\noalign{}
\endhead
\bottomrule\noalign{}
\endlastfoot
\textbf{એપ્લિકેશન} & યુઝર એપ્લિકેશન્સ & IoT એપ્લિકેશન્સ, ક્લાઉડ સર્વિસિસ \\
\textbf{પ્રેઝન્ટેશન} & ડેટા ફોર્મેટિંગ & JSON, XML, CoAP \\
\textbf{સેશન} & સેશન મેનેજમેન્ટ & MQTT, HTTP સેશન્સ \\
\textbf{ટ્રાન્સપોર્ટ} & TCP, UDP & UDP, CoAP, MQTT \\
\textbf{નેટવર્ક} & IP રાઉટિંગ & 6LoWPAN, IPv6 \\
\textbf{ડેટા લિંક} & Ethernet, WiFi & IEEE 802.15.4, LoRa \\
\textbf{ફિઝિકલ} & ફિઝિકલ મીડિયમ & સેન્સર્સ, એક્ચ્યુએટર્સ, રેડિયો \\
\end{longtable}
}

\textbf{મુખ્ય મોડિફિકેશન્સ:}

\begin{itemize}
\tightlist
\item
  \textbf{લાઇટવેઇટ પ્રોટોકોલ્સ}: રિસોર્સ-કન્સ્ટ્રેઇન્ડ ડિવાઇસિસ માટે ઓપ્ટિમાઇઝ્ડ
\item
  \textbf{એનર્જી એફિશિયન્સી}: લો પાવર કન્ઝમ્પશન માટે ડિઝાઇન કરેલા પ્રોટોકોલ્સ
\item
  \textbf{ઇન્ટરઓપરેબિલિટી}: વિવિધ IoT ડિવાઇસિસ અને પ્લેટફોર્મ્સ માટે સપોર્ટ
\end{itemize}

\end{solutionbox}
\begin{mnemonicbox}
``એપ્સ પ્રેઝન્ટ સેશન ટ્રાન્સપોર્ટ નેટવર્ક લિંક ફિઝિકલ''

\end{mnemonicbox}
\subsection*{પ્રશ્ન 3(ક) OR [7
ગુણ]}\label{uxaaauxab0uxab6uxaa8-3uxa95-or-7-uxa97uxaa3}

\textbf{IoT ના સ્રોતો વિગતવાર સમજાવો}

\begin{solutionbox}

\textbf{IoT સ્રોતો વર્ગીકરણ:}

\includegraphics[width=1\linewidth,height=\textheight,keepaspectratio]{mermaid-4c6326bd.pdf}

\textbf{1. ટેકનોલોજી ઇવોલ્યુશન સ્રોતો:}

\begin{itemize}
\tightlist
\item
  \textbf{ઇન્ટરનેટ વિસ્તરણ}: ગ્લોબલ કનેક્ટિવિટી ઇન્ફ્રાસ્ટ્રક્ચર ડેવલપમેન્ટ
\item
  \textbf{મોબાઇલ રિવોલ્યુશન}: સ્માર્ટફોન અને ટેબ્લેટ્સ કનેક્ટેડ ઇકોસિસ્ટમ બનાવે છે
\item
  \textbf{ક્લાઉડ કમ્પ્યુટિંગ}: સ્કેલેબલ કમ્પ્યુટિંગ અને સ્ટોરેજ રિસોર્સિસ
\item
  \textbf{બિગ ડેટા એનાલિટિક્સ}: વિશાળ ડેટા વોલ્યુમ્સ પ્રોસેસ કરવાની ક્ષમતા
\end{itemize}

\textbf{2. બિઝનેસ ડ્રાઇવર્સ:}

\begin{itemize}
\tightlist
\item
  \textbf{ઓપરેશનલ એફિશિયન્સી}: બિઝનેસ પ્રોસેસિસનું ઓટોમેશન અને ઓપ્ટિમાઇઝેશન
\item
  \textbf{કોસ્ટ રિડક્શન}: ઓપરેશનલ અને મેઇન્ટેનન્સ કોસ્ટ ઓછી
\item
  \textbf{નવા બિઝનેસ મોડલ્સ}: ડેટા-ડ્રિવન સર્વિસિસ અને પ્રોડક્ટ્સ
\item
  \textbf{કસ્ટમર સેટિસફેક્શન}: સ્માર્ટ સર્વિસિસ દ્વારા યુઝર એક્સપિરિયન્સ વધારવું
\end{itemize}

\textbf{3. ટેકનોલોજિકલ એનેબલર્સ:}

\begin{itemize}
\tightlist
\item
  \textbf{સેન્સર એડવાન્સમેન્ટ}: નાના, સસ્તા, વધુ સચોટ સેન્સર્સ
\item
  \textbf{કમ્યુનિકેશન પ્રોગ્રેસ}: બહેતર વાયરલેસ પ્રોટોકોલ્સ અને સ્ટાન્ડર્ડ્સ
\item
  \textbf{પ્રોસેસિંગ ઇવોલ્યુશન}: વધુ શક્તિશાળી છતાં એનર્જી-એફિશિયન્ટ પ્રોસેસર્સ
\item
  \textbf{સ્ટોરેજ રિવોલ્યુશન}: સસ્તું અને વધુ વિશ્વસનીય ડેટા સ્ટોરેજ સોલ્યુશન્સ
\end{itemize}

\textbf{4. માર્કેટ ડિમાન્ડ્સ:}

\begin{itemize}
\tightlist
\item
  \textbf{સ્માર્ટ સિટીઝ}: શહેરી આયોજન અને ઇન્ફ્રાસ્ટ્રક્ચર મેનેજમેન્ટ
\item
  \textbf{હેલ્થકેર}: રિમોટ મોનિટરિંગ અને ટેલિમેડિસિન
\item
  \textbf{ઇન્ડસ્ટ્રિયલ ઓટોમેશન}: ઇન્ડસ્ટ્રી 4.0 અને સ્માર્ટ મેન્યુફેક્ચરિંગ
\item
  \textbf{એન્વાયરન્મેન્ટલ મોનિટરિંગ}: ક્લાઇમેટ ચેન્જ અને સસ્ટેનેબિલિટી ચિંતાઓ
\end{itemize}

\textbf{મુખ્ય કન્વર્જન્સ ફેક્ટર્સ:}

\begin{itemize}
\tightlist
\item
  \textbf{IPv6 એડોપ્શન}: અબજો ડિવાઇસિસ માટે અનલિમિટેડ એડ્રેસિંગ
\item
  \textbf{5G નેટવર્ક્સ}: હાઇ-સ્પીડ, લો-લેટન્સી કમ્યુનિકેશન
\item
  \textbf{AI ઇન્ટિગ્રેશન}: ઇન્ટેલિજન્ટ ડિસિઝન મેકિંગ માટે મશીન લર્નિંગ
\end{itemize}

\end{solutionbox}
\begin{mnemonicbox}
``ટેકનોલોજી બિઝનેસ એનેબલ માર્કેટ''

\end{mnemonicbox}
\subsection*{પ્રશ્ન 4(અ) [3
ગુણ]}\label{uxaaauxab0uxab6uxaa8-4uxa85-3-uxa97uxaa3}

\textbf{IoT ના મૂળભૂત ઘટકોને સંક્ષિપ્તમાં સમજાવો.}

\begin{solutionbox}

\textbf{મૂળભૂત IoT ઘટકો:}

{\def\LTcaptype{none} % do not increment counter
\begin{longtable}[]{@{}lll@{}}
\toprule\noalign{}
ઘટક & કાર્ય & ઉદાહરણ \\
\midrule\noalign{}
\endhead
\bottomrule\noalign{}
\endlastfoot
\textbf{સેન્સર્સ} & ડેટા કલેક્શન & તાપમાન, દબાણ, ગતિ \\
\textbf{કનેક્ટિવિટી} & ડેટા ટ્રાન્સમિશન & WiFi, Bluetooth, સેલ્યુલર \\
\textbf{ડેટા પ્રોસેસિંગ} & ઇન્ફોર્મેશન એનાલિસિસ & એજ/ક્લાઉડ કમ્પ્યુટિંગ \\
\textbf{યુઝર ઇન્ટરફેસ} & હ્યુમન ઇન્ટરેક્શન & મોબાઇલ એપ્સ, ડેશબોર્ડ્સ \\
\end{longtable}
}

\textbf{કોર ફંક્શન્સ:}

\begin{itemize}
\tightlist
\item
  \textbf{સેન્સિંગ}: પર્યાવરણીય ડેટા એકત્રિત કરવું
\item
  \textbf{કનેક્ટિંગ}: પ્રોસેસિંગ સેન્ટર્સ સુધી ડેટા ટ્રાન્સમિટ કરવું
\item
  \textbf{પ્રોસેસિંગ}: એનાલિસિસ અને ઇનસાઇટ્સ કાઢવા
\item
  \textbf{એક્ટિંગ}: એનાલિસિસ આધારે એક્ચ્યુએટર્સને કંટ્રોલ કરવું
\end{itemize}

\end{solutionbox}
\begin{mnemonicbox}
``સેન્સ કનેક્ટ પ્રોસેસ ઇન્ટરફેસ''

\end{mnemonicbox}
\subsection*{પ્રશ્ન 4(બ) [4
ગુણ]}\label{uxaaauxab0uxab6uxaa8-4uxaac-4-uxa97uxaa3}

\textbf{કન્સ્ટ્રેઇન્ડ એપ્લિકેશન પ્રોટોકોલ (CoAP) ની સંક્ષિપ્તમાં ચર્ચા કરો.}

\begin{solutionbox}

\textbf{CoAP પ્રોટોકોલ ઓવરવ્યુ:}

\begin{lstlisting}
Client                    Server
  |                         |
  |------- GET /temp ------>|
  |                         |
  |<----- 2.05 Content -----|
  |    Payload: 25^\circC        |
  |                         |
\end{lstlisting}

\textbf{CoAP ફીચર્સ:}

{\def\LTcaptype{none} % do not increment counter
\begin{longtable}[]{@{}lll@{}}
\toprule\noalign{}
ફીચર & વર્ણન & ફાયદો \\
\midrule\noalign{}
\endhead
\bottomrule\noalign{}
\endlastfoot
\textbf{લાઇટવેઇટ} & સિમ્પલ પ્રોટોકોલ ડિઝાઇન & લો રિસોર્સ વેજ \\
\textbf{UDP-બેઝ્ડ} & UDP ટ્રાન્સપોર્ટ વાપરે છે & રિડ્યુસ્ડ ઓવરહેડ \\
\textbf{RESTful} & REST આર્કિટેક્ચર & ઇઝી ઇન્ટિગ્રેશન \\
\textbf{રિલાયેબલ} & બિલ્ટ-ઇન રિટ્રાન્સમિશન & એન્શ્યોર્સ ડિલિવરી \\
\end{longtable}
}

\textbf{મુખ્ય લક્ષણો:}

\begin{itemize}
\tightlist
\item
  \textbf{રિક્વેસ્ટ/રિસ્પોન્સ}: HTTP સમાન પરંતુ IoT માટે ઓપ્ટિમાઇઝ્ડ
\item
  \textbf{કન્ફર્મેબલ મેસેજિસ}: એકનોલેજમેન્ટ્સ દ્વારા રિલાયબિલિટી
\item
  \textbf{રિસોર્સ ડિસ્કવરી}: બિલ્ટ-ઇન સર્વિસ ડિસ્કવરી મેકેનિઝ્મ
\item
  \textbf{બ્લોક ટ્રાન્સફર}: મોટા ડેટા ટ્રાન્સફર્સ માટે સપોર્ટ
\end{itemize}

\end{solutionbox}
\begin{mnemonicbox}
``લાઇટ UDP REST રિલાયેબલ''

\end{mnemonicbox}
\subsection*{પ્રશ્ન 4(ક) [7
ગુણ]}\label{uxaaauxab0uxab6uxaa8-4uxa95-7-uxa97uxaa3}

\textbf{ક્લાઉડ દ્વારા સેન્સર અને કંટ્રોલિંગ ડિવાઇસ (એક્ચ્યુએટર) મેનેજમેન્ટની પ્રક્રિયા
સમજાવો.}

\begin{solutionbox}

\includegraphics[width=1\linewidth,height=\textheight,keepaspectratio]{mermaid-4154c96b.pdf}

\textbf{ક્લાઉડ-બેઝ્ડ IoT મેનેજમેન્ટ પ્રોસેસ:}

\textbf{1. ડેટા કલેક્શન ફેઝ:}

\begin{itemize}
\tightlist
\item
  \textbf{સેન્સર્સ}: પર્યાવરણીય ડેટા એકત્રિત કરે છે (તાપમાન, ભેજ, ગતિ)
\item
  \textbf{લોકલ પ્રોસેસિંગ}: એજ ડિવાઇસિસ પર બેઝિક ફિલ્ટરિંગ અને ફોર્મેટિંગ
\item
  \textbf{ડેટા ટ્રાન્સમિશન}: WiFi/સેલ્યુલર કનેક્શન દ્વારા ક્લાઉડ પર ડેટા મોકલવું
\end{itemize}

\textbf{2. ક્લાઉડ પ્રોસેસિંગ ફેઝ:}

\begin{itemize}
\tightlist
\item
  \textbf{ડેટા ઈન્જેસ્શન}: ક્લાઉડ ડેટાબેસિસમાં સેન્સર ડેટા પ્રાપ્ત અને સ્ટોર કરવું
\item
  \textbf{રિયલ-ટાઇમ એનાલિટિક્સ}: તાત્કાલિક ઇનસાઇટ્સ માટે ડેટા સ્ટ્રીમ્સ પ્રોસેસ
  કરવા
\item
  \textbf{મશીન લર્નિંગ}: પેટર્ન રેકગ્નિશન અને પ્રિડિક્શન માટે AI એલ્ગોરિધમ્સ લાગુ
  કરવા
\end{itemize}

\textbf{3. ડિસિઝન મેકિંગ ફેઝ:}

\begin{itemize}
\tightlist
\item
  \textbf{રૂલ એન્જિન}: જરૂરી એક્શન્સ નક્કી કરવા માટે બિઝનેસ રૂલ્સ લાગુ કરવા
\item
  \textbf{થ્રેશોલ્ડ મોનિટરિંગ}: વેલ્યુઝ લિમિટ્સ ઓતરી જાય ત્યારે એલર્ટ ટ્રિગર કરવા
\item
  \textbf{ઓટોમેટેડ રિસ્પોન્સિસ}: એક્ચ્યુએટર્સ માટે કંટ્રોલ કમાન્ડ્સ જનરેટ કરવા
\end{itemize}

\textbf{4. કંટ્રોલ એક્ઝીક્યુશન ફેઝ:}

\begin{itemize}
\tightlist
\item
  \textbf{કમાન્ડ ડિસ્પેચ}: યોગ્ય એક્ચ્યુએટર્સ પર કંટ્રોલ સિગ્નલ્સ મોકલવા
\item
  \textbf{ડિવાઇસ મેનેજમેન્ટ}: એક્ચ્યુએટર સ્ટેટસ અને પરફોર્મન્સ મોનિટર કરવું
\item
  \textbf{ફીડબેક લૂપ}: સફળ કમાન્ડ એક્ઝીક્યુશનની કન્ફર્મેશન એકત્રિત કરવી
\end{itemize}

\textbf{5. યુઝર ઇન્ટરેક્શન:}

\begin{itemize}
\tightlist
\item
  \textbf{ડેશબોર્ડ}: સેન્સર ડેટા અને સિસ્ટમ સ્ટેટસનું રિયલ-ટાઇમ વિઝ્યુઅલાઇઝેશન
\item
  \textbf{મોબાઇલ એપ્સ}: રિમોટ મોનિટરિંગ અને મેન્યુઅલ કંટ્રોલ ક્ષમતાઓ
\item
  \textbf{નોટિફિકેશન્સ}: યુઝર્સને એલર્ટ્સ અને વોર્નિંગ્સ મોકલવા
\end{itemize}

\textbf{ફાયદા:}

\begin{itemize}
\tightlist
\item
  \textbf{સ્કેલેબિલિટી}: હજારો ડિવાઇસિસને એકસાથે હેન્ડલ કરી શકે છે
\item
  \textbf{રિમોટ એક્સેસ}: ઇન્ટરનેટ સાથે ગમે ત્યાંથી ડિવાઇસિસ કંટ્રોલ કરી શકાય છે
\item
  \textbf{ડેટા એનાલિટિક્સ}: હિસ્ટોરિકલ એનાલિસિસ અને પ્રિડિક્ટિવ મેઇન્ટેનન્સ
\item
  \textbf{ઇન્ટિગ્રેશન}: અન્ય બિઝનેસ સિસ્ટમ્સ અને સર્વિસિસ સાથે કનેક્ટ કરી શકાય છે
\end{itemize}

\end{solutionbox}
\begin{mnemonicbox}
``કલેક્ટ પ્રોસેસ ડિસાઇડ કંટ્રોલ ઇન્ટરેક્ટ''

\end{mnemonicbox}
\subsection*{પ્રશ્ન 4(અ) OR [3
ગુણ]}\label{uxaaauxab0uxab6uxaa8-4uxa85-or-3-uxa97uxaa3}

\textbf{ઇન્ટરનેટ ઓફ થિંગ્સને વ્યાખ્યાયિત કરો અને તેનું વિઝન જણાવો.}

\begin{solutionbox}

\textbf{વ્યાખ્યા:} ઇન્ટરનેટ ઓફ થિંગ્સ (IoT) એ સેન્સર્સ, સોફ્ટવેર, અને કનેક્ટિવિટી સાથે
એમ્બેડેડ ભૌતિક ડિવાઇસિસનું નેટવર્ક છે જે ઇન્ટરનેટ પર ડેટા એકત્રિત અને વિનિમય કરવા માટે
છે.

\textbf{IoT વિઝન:}

{\def\LTcaptype{none} % do not increment counter
\begin{longtable}[]{@{}ll@{}}
\toprule\noalign{}
પાસું & વિઝન \\
\midrule\noalign{}
\endhead
\bottomrule\noalign{}
\endlastfoot
\textbf{કનેક્ટિવિટી} & બધું બધે કનેક્ટેડ \\
\textbf{ઇન્ટેલિજન્સ} & સ્માર્ટ ડિસિઝન મેકિંગ \\
\textbf{ઓટોમેશન} & મિનિમલ હ્યુમન ઇન્ટરવેન્શન \\
\textbf{ઇન્ટિગ્રેશન} & સીમલેસ સિસ્ટમ ઇન્ટરેક્શન \\
\end{longtable}
}

\textbf{કોર વિઝન એલિમેન્ટ્સ:}

\begin{itemize}
\tightlist
\item
  \textbf{યુબિક્વિટસ કમ્પ્યુટિંગ}: રોજિંદા વસ્તુઓમાં એમ્બેડેડ ટેકનોલોજી
\item
  \textbf{સીમલેસ ઇન્ટરેક્શન}: કુદરતી હ્યુમન-ડિવાઇસ કમ્યુનિકેશન
\item
  \textbf{ઇન્ટેલિજન્ટ એન્વાયરન્મેન્ટ}: કન્ટેક્સ્ટ-અવેર રિસ્પોન્સિવ સિસ્ટમ્સ
\end{itemize}

\end{solutionbox}
\begin{mnemonicbox}
``કનેક્ટ ઇન્ટેલિજન્સ ઓટોમેટ ઇન્ટિગ્રેટ''

\end{mnemonicbox}
\subsection*{પ્રશ્ન 4(બ) OR [4
ગુણ]}\label{uxaaauxab0uxab6uxaa8-4uxaac-or-4-uxa97uxaa3}

\textbf{મેસેજ ક્યુ ટેલિમેટ્રી ટ્રાન્સપોર્ટ (MQTT) પ્રોટોકોલની સંક્ષિપ્તમાં ચર્ચા કરો.}

\begin{solutionbox}

\textbf{MQTT પ્રોટોકોલ આર્કિટેક્ચર:}

\begin{lstlisting}
Publisher                Broker               Subscriber
    |                       |                      |
    |-- Publish(topic) ---->|                      |
    |                       |<-- Subscribe(topic)--|
    |                       |                      |
    |                       |-- Forward Message -->|
\end{lstlisting}

\textbf{MQTT લક્ષણો:}

{\def\LTcaptype{none} % do not increment counter
\begin{longtable}[]{@{}lll@{}}
\toprule\noalign{}
ફીચર & વર્ણન & ફાયદો \\
\midrule\noalign{}
\endhead
\bottomrule\noalign{}
\endlastfoot
\textbf{લાઇટવેઇટ} & મિનિમલ પ્રોટોકોલ ઓવરહેડ & IoT ડિવાઇસિસ માટે યોગ્ય \\
\textbf{પબ્લિશ/સબ્સ્ક્રાઇબ} & ડિકપલ્ડ કમ્યુનિકેશન & સ્કેલેબલ આર્કિટેક્ચર \\
\textbf{QoS લેવલ્સ} & ક્વોલિટી ઓફ સર્વિસ ઓપ્શન્સ & રિલાયેબલ ડિલિવરી \\
\textbf{પર્સિસ્ટન્ટ સેશન્સ} & સેશન સ્ટેટ જાળવવામાં આવે છે & કનેક્શન રેઝિલિયન્સ \\
\end{longtable}
}

\textbf{MQTT ઘટકો:}

\begin{itemize}
\tightlist
\item
  \textbf{પબ્લિશર}: બ્રોકર પર મેસેજિસ મોકલે છે
\item
  \textbf{સબ્સ્ક્રાઇબર}: બ્રોકર પાસેથી મેસેજિસ પ્રાપ્ત કરે છે
\item
  \textbf{બ્રોકર}: સેન્ટ્રલ મેસેજ રાઉટર
\item
  \textbf{ટોપિક્સ}: મેસેજ કેટેગોરાઇઝેશન સિસ્ટમ
\end{itemize}

\textbf{ક્વોલિટી ઓફ સર્વિસ લેવલ્સ:}

\begin{itemize}
\tightlist
\item
  \textbf{QoS 0}: સૌથી વધુ એક વાર ડિલિવરી
\item
  \textbf{QoS 1}: ઓછામાં ઓછું એક વાર ડિલિવરી
\item
  \textbf{QoS 2}: બરાબર એક વાર ડિલિવરી
\end{itemize}

\end{solutionbox}
\begin{mnemonicbox}
``પબ્લિશ સબ્સ્ક્રાઇબ બ્રોકર ટોપિક''

\end{mnemonicbox}
\subsection*{પ્રશ્ન 4(ક) OR [7
ગુણ]}\label{uxaaauxab0uxab6uxaa8-4uxa95-or-7-uxa97uxaa3}

\textbf{રાસ્પબેરી પાઇનો આર્કિટેક્ચર બ્લોક ડાયાગ્રામ દોરો અને તેને સમજાવો.}

\begin{solutionbox}

\begin{lstlisting}
+----------------------------------------------------------+
|                    Raspberry Pi 4                       |
|  +----------+  +----------+  +----------+  +----------+ |
|  |   CPU    |  |   GPU    |  |  Memory  |  | Storage  | |
|  |Quad-core |  |VideoCore |  | 4GB RAM  |  |MicroSD   | |
|  |ARM A72   |  |    VI    |  |  LPDDR4  |  |   Card   | |
|  +----------+  +----------+  +----------+  +----------+ |
|                                                         |
|  +----------+  +----------+  +----------+  +----------+ |
|  |   GPIO   |  |   USB    |  | Network  |  |  Audio   | |
|  | 40 pins  |  | 4 ports  |  |Ethernet  |  |3.5mm jack| |
|  |          |  |  USB 3.0 |  |WiFi/BT   |  |   HDMI   | |
|  +----------+  +----------+  +----------+  +----------+ |
+----------------------------------------------------------+
\end{lstlisting}

\textbf{રાસ્પબેરી પાઇ આર્કિટેક્ચર ઘટકો:}

\textbf{1. પ્રોસેસિંગ યુનિટ:}

\begin{itemize}
\tightlist
\item
  \textbf{CPU}: 1.5GHz પર ચાલતું ક્વાડ-કોર ARM Cortex-A72 પ્રોસેસર
\item
  \textbf{GPU}: ગ્રાફિક્સ પ્રોસેસિંગ અને વિડિયો એક્સિલરેશન માટે VideoCore VI
\item
  \textbf{પરફોર્મન્સ}: Linux જેવા સંપૂર્ણ ઓપરેટિંગ સિસ્ટમ્સ ચલાવવા સક્ષમ
\end{itemize}

\textbf{2. મેમોરી સિસ્ટમ:}

\begin{itemize}
\tightlist
\item
  \textbf{RAM}: પ્રોગ્રામ એક્ઝીક્યુશન માટે 4GB LPDDR4 સિસ્ટમ મેમોરી
\item
  \textbf{સ્ટોરેજ}: ઓપરેટિંગ સિસ્ટમ અને ડેટા સ્ટોરેજ માટે MicroSD કાર્ડ સ્લોટ
\item
  \textbf{કેશ}: બહેતર પરફોર્મન્સ માટે ઓન-ચિપ કેશ મેમોરી
\end{itemize}

\textbf{3. ઇનપુટ/આઉટપુટ ઇન્ટરફેસિસ:}

\begin{itemize}
\tightlist
\item
  \textbf{GPIO}: સેન્સર કનેક્ટિવિટી માટે 40-પિન જનરલ પર્પઝ ઇનપુટ/આઉટપુટ
\item
  \textbf{USB પોર્ટ્સ}: પેરિફેરલ્સ અને સ્ટોરેજ ડિવાઇસિસ માટે 4x USB 3.0 પોર્ટ્સ
\item
  \textbf{ડિસ્પ્લે}: 4K વિડિયો આઉટપુટ સપોર્ટિંગ 2x માઇક્રો-HDMI પોર્ટ્સ
\end{itemize}

\textbf{4. કનેક્ટિવિટી ઓપ્શન્સ:}

\begin{itemize}
\tightlist
\item
  \textbf{ઇથરનેટ}: વાયર્ડ નેટવર્ક કનેક્શન માટે ગિગાબિટ ઇથરનેટ પોર્ટ
\item
  \textbf{વાયરલેસ}: ડ્યુઅલ-બેન્ડ WiFi 802.11ac અને Bluetooth 5.0
\item
  \textbf{કેમેરા}: ડેડિકેટેડ કેમેરા સીરિયલ ઇન્ટરફેસ (CSI) પોર્ટ
\end{itemize}

\textbf{5. પાવર અને ઓડિયો:}

\begin{itemize}
\tightlist
\item
  \textbf{પાવર}: એફિશિયન્ટ પાવર મેનેજમેન્ટ સાથે USB-C પાવર ઇનપુટ
\item
  \textbf{ઓડિયો}: 3.5mm ઓડિયો જેક અને HDMI ઓડિયો આઉટપુટ
\item
  \textbf{પાવર કન્ઝમ્પશન}: સતત ઓપરેશન માટે ઓપ્ટિમાઇઝ્ડ
\end{itemize}

\textbf{IoT એપ્લિકેશન્સ:}

\begin{itemize}
\tightlist
\item
  \textbf{હોમ ઓટોમેશન}: લાઇટ્સ, ફેન્સ, સિક્યુરિટી સિસ્ટમ્સ કંટ્રોલ
\item
  \textbf{ઇન્ડસ્ટ્રિયલ મોનિટરિંગ}: તાપમાન, દબાણ, વાઇબ્રેશન સેન્સિંગ
\item
  \textbf{રોબોટિક્સ}: મોટર કંટ્રોલ, સેન્સર ઇન્ટિગ્રેશન, કમ્પ્યુટર વિઝન
\item
  \textbf{ડેટા લોગિંગ}: પર્યાવરણીય મોનિટરિંગ અને ડેટા કલેક્શન
\end{itemize}

\textbf{IoT માટે ફાયદા:}

\begin{itemize}
\tightlist
\item
  \textbf{કોસ્ટ-ઇફેક્ટિવ}: લો-કોસ્ટ કમ્પ્યુટિંગ પ્લેટફોર્મ
\item
  \textbf{વર્સેટાઇલ}: અનેક પ્રોગ્રામિંગ લેંગ્વેજિસ સપોર્ટ કરે છે
\item
  \textbf{કમ્યુનિટી સપોર્ટ}: ટ્યુટોરિયલ્સ અને પ્રોજેક્ટ્સનું વિશાળ ઇકોસિસ્ટમ
\item
  \textbf{એક્સપેન્ડેબિલિટી}: અનેક સેન્સર્સ અને મોડ્યુલ્સ સાથે કમ્પેટિબલ
\end{itemize}

\end{solutionbox}
\begin{mnemonicbox}
``પ્રોસેસ મેમોરી ઇન્ટરફેસ કનેક્ટ પાવર''

\end{mnemonicbox}
\subsection*{પ્રશ્ન 5(અ) [3
ગુણ]}\label{uxaaauxab0uxab6uxaa8-5uxa85-3-uxa97uxaa3}

\textbf{IoT નો ઉપયોગ કરીને સ્માર્ટ હેલ્થ મોનિટરિંગ સિસ્ટમનો બ્લોક ડાયાગ્રામ
દોરો.}

\begin{solutionbox}

\includegraphics[width=1\linewidth,height=\textheight,keepaspectratio]{mermaid-c906660a.pdf}

\textbf{સિસ્ટમ ઘટકો:}

\begin{itemize}
\tightlist
\item
  \textbf{સેન્સર્સ}: વાઇટલ સાઇન્સ એકત્રિત કરે છે (હાર્ટ રેટ, બ્લડ પ્રેશર, તાપમાન)
\item
  \textbf{માઇક્રોકંટ્રોલર}: સેન્સર ડેટા પ્રોસેસ કરે છે અને કમ્યુનિકેશન મેનેજ કરે છે
\item
  \textbf{કનેક્ટિવિટી}: WiFi/સેલ્યુલર નેટવર્ક્સ દ્વારા ક્લાઉડ પર ડેટા ટ્રાન્સમિટ કરે છે
\item
  \textbf{ક્લાઉડ પ્લેટફોર્મ}: ડેટા સ્ટોર કરે છે અને એનાલિટિક્સ સર્વિસિસ પ્રદાન કરે છે
\item
  \textbf{યુઝર ઇન્ટરફેસિસ}: મોનિટરિંગ માટે મોબાઇલ એપ્સ અને વેબ ડેશબોર્ડ્સ
\end{itemize}

\end{solutionbox}
\begin{mnemonicbox}
``સેન્સ પ્રોસેસ કનેક્ટ સ્ટોર મોનિટર''

\end{mnemonicbox}
\subsection*{પ્રશ્ન 5(બ) [4
ગુણ]}\label{uxaaauxab0uxab6uxaa8-5uxaac-4-uxa97uxaa3}

\textbf{IoT માં વિવિધ પ્રકારના સેન્સર્સની યાદી બનાવો અને કોઈપણ બેના કાર્યને
સંક્ષિપ્તમાં સમજાવો.}

\begin{solutionbox}

\textbf{IoT સેન્સર પ્રકારો:}

{\def\LTcaptype{none} % do not increment counter
\begin{longtable}[]{@{}lll@{}}
\toprule\noalign{}
સેન્સર પ્રકાર & માપન & એપ્લિકેશન્સ \\
\midrule\noalign{}
\endhead
\bottomrule\noalign{}
\endlastfoot
\textbf{તાપમાન} & ગરમી/ઠંડક લેવલ્સ & HVAC, હવામાન મોનિટરિંગ \\
\textbf{ભેજ} & ભેજનું પ્રમાણ & કૃષિ, સ્ટોરેજ \\
\textbf{દબાણ} & એકમ વિસ્તાર દીઠ બળ & હવામાન, ઇન્ડસ્ટ્રિયલ \\
\textbf{ગતિ/PIR} & હલચલ શોધ & સિક્યુરિટી, ઓટોમેશન \\
\textbf{ગેસ} & રસાયણિક રચના & હવાની ગુણવત્તા, સલામતી \\
\textbf{પ્રકાશ} & પ્રકાશ સ્તર & સ્માર્ટ લાઇટિંગ \\
\end{longtable}
}

\textbf{વિગતવાર કાર્ય:}

\textbf{1. તાપમાન સેન્સર (DHT22):}

\begin{itemize}
\tightlist
\item
  \textbf{સિદ્ધાંત}: થર્મિસ્ટર રેઝિસ્ટન્સ તાપમાન સાથે બદલાય છે
\item
  \textbf{પ્રક્રિયા}: માઇક્રોકંટ્રોલર રેઝિસ્ટન્સ વેલ્યુ વાંચે છે અને તાપમાનમાં કન્વર્ટ કરે
  છે
\item
  \textbf{આઉટપુટ}: તાપમાન અને ભેજ ડેટા સાથે ડિજિટલ સિગ્નલ
\item
  \textbf{એપ્લિકેશન્સ}: સ્માર્ટ થર્મોસ્ટેટ, પર્યાવરણીય મોનિટરિંગ
\end{itemize}

\textbf{2. PIR મોશન સેન્સર:}

\begin{itemize}
\tightlist
\item
  \textbf{સિદ્ધાંત}: હલતા પદાર્થો દ્વારા ઉત્સર્જિત ઇન્ફ્રારેડ રેડિયેશન શોધે છે
\item
  \textbf{ઘટકો}: ફ્રેસ્નેલ લેન્સ સાથે પાયરોઇલેક્ટ્રિક સેન્સર
\item
  \textbf{કાર્ય}: ઇન્ફ્રારેડ લેવલ્સમાં ફેરફાર ડિજિટલ આઉટપુટ સિગ્નલ ટ્રિગર કરે છે
\item
  \textbf{એપ્લિકેશન્સ}: સિક્યુરિટી સિસ્ટમ્સ, ઓટોમેટિક લાઇટિંગ, ઓક્યુપેન્સી ડિટેક્શન
\end{itemize}

\end{solutionbox}
\begin{mnemonicbox}
``તાપમાન ભેજ દબાણ ગતિ ગેસ પ્રકાશ''

\end{mnemonicbox}
\subsection*{પ્રશ્ન 5(ક) [7
ગુણ]}\label{uxaaauxab0uxab6uxaa8-5uxa95-7-uxa97uxaa3}

\textbf{IoT નો ઉપયોગ કરીને સ્માર્ટ હોમ ઓટોમેશનનો બ્લોક ડાયાગ્રામ દોરો અને તેનું
કાર્ય સમજાવો.}

\begin{solutionbox}

\includegraphics[width=1\linewidth,height=\textheight,keepaspectratio]{mermaid-eb569bde.pdf}

\textbf{સ્માર્ટ હોમ ઓટોમેશન કાર્ય:}

\textbf{1. ડેટા કલેક્શન:}

\begin{itemize}
\tightlist
\item
  \textbf{પર્યાવરણીય સેન્સર્સ}: તાપમાન, ભેજ, પ્રકાશ સ્તરનું નિરીક્ષણ કરે છે
\item
  \textbf{સિક્યુરિટી સેન્સર્સ}: ગતિ, દરવાજા/બારીની સ્થિતિ, સ્મોક/ગેસ શોધે છે
\item
  \textbf{યુઝર પ્રેઝન્સ}: વિવિધ રૂમ્સમાં ઓક્યુપેન્સી નિર્ધારિત કરવા માટે PIR સેન્સર્સ
\end{itemize}

\textbf{2. ડેટા પ્રોસેસિંગ:}

\begin{itemize}
\tightlist
\item
  \textbf{લોકલ પ્રોસેસિંગ}: ક્રિટિકલ પરિસ્થિતિઓ (ફાયર એલાર્મ) માટે તાત્કાલિક
  પ્રતિક્રિયા
\item
  \textbf{ક્લાઉડ પ્રોસેસિંગ}: જટિલ એનાલિટિક્સ અને પેટર્ન રેકગ્નિશન
\item
  \textbf{મશીન લર્નિંગ}: સમય સાથે યુઝર પ્રાથમિકતાઓ અને આદતો શીખવી
\end{itemize}

\textbf{3. ડિસિઝન મેકિંગ:}

\begin{itemize}
\tightlist
\item
  \textbf{રૂલ-બેઝ્ડ કંટ્રોલ}: જો તાપમાન \textgreater{} 25^\circC, તો AC ચાલુ કરો
\item
  \textbf{શેડ્યુલ્ડ ઓપરેશન્સ}: સૂર્યાસ્ત સમયે લાઇટ્સ ચાલુ કરો, સવારે 6 વાગ્યે છોડવાઓને
  પાણી આપો
\item
  \textbf{યુઝર પ્રાથમિકતાઓ}: શીખેલા પેટર્ન આધારે લાઇટિંગ અને તાપમાન એડજસ્ટ કરો
\end{itemize}

\textbf{4. કંટ્રોલ એક્ઝીક્યુશન:}

\begin{itemize}
\tightlist
\item
  \textbf{લાઇટિંગ કંટ્રોલ}: એમ્બિઅન્ટ લાઇટ અને સમય આધારે ઓટોમેટિક ડિમિંગ
\item
  \textbf{ક્લાઇમેટ કંટ્રોલ}: ઓક્યુપેન્સી અને હવામાન આધારે હીટિંગ/કૂલિંગ ઓપ્ટિમાઇઝ કરો
\item
  \textbf{સિક્યુરિટી મેનેજમેન્ટ}: સિક્યુરિટી સિસ્ટમ આર્મ/ડિસઆર્મ, દરવાજા લોક/અનલોક
\end{itemize}

\textbf{5. યુઝર ઇન્ટરેક્શન:}

\begin{itemize}
\tightlist
\item
  \textbf{મોબાઇલ એપ}: ગમે ત્યાંથી રિમોટ મોનિટરિંગ અને કંટ્રોલ
\item
  \textbf{વૉઇસ કમાન્ડ્સ}: Alexa, Google Assistant સાથે ઇન્ટિગ્રેશન
\item
  \textbf{મેન્યુઅલ ઓવરરાઇડ}: ફિઝિકલ સ્વિચિસ અને કંટ્રોલ્સ કાર્યક્ષમ રહે છે
\end{itemize}

\textbf{6. કમ્યુનિકેશન ફ્લો:}

\begin{itemize}
\tightlist
\item
  \textbf{સેન્સર ડેટા}: દર થોડી સેકન્ડે એકત્રિત કરવામાં આવે છે અને કંટ્રોલર પર
  ટ્રાન્સમિટ કરવામાં આવે છે
\item
  \textbf{ક્લાઉડ સિંક્રોનાઇઝેશન}: ડેટા બેકઅપ અને રિમોટ એક્સેસ ક્ષમતાઓ
\item
  \textbf{સ્ટેટસ અપડેટ્સ}: મોબાઇલ ડિવાઇસિસ પર રિયલ-ટાઇમ નોટિફિકેશન્સ
\end{itemize}

\textbf{મુખ્ય ફીચર્સ:}

\begin{itemize}
\tightlist
\item
  \textbf{એનર્જી એફિશિયન્સી}: ઓટોમેટિક કંટ્રોલ વીજ વપરાશ 30-40\% ઘટાડે છે
\item
  \textbf{સિક્યુરિટી એન્હાન્સમેન્ટ}: રિયલ-ટાઇમ મોનિટરિંગ અને એલર્ટ સિસ્ટમ્સ
\item
  \textbf{કન્વીનિયન્સ}: વૉઇસ કંટ્રોલ અને સ્માર્ટફોન ઇન્ટિગ્રેશન
\item
  \textbf{કોસ્ટ સેવિંગ્સ}: વીજ અને પાણીના સંસાધનોનો ઓપ્ટિમાઇઝ્ડ ઉપયોગ
\end{itemize}

\textbf{સિસ્ટમ ફાયદા:}

\begin{itemize}
\tightlist
\item
  \textbf{રિમોટ મોનિટરિંગ}: ઓફિસ અથવા વેકેશનથી ઘરની સ્થિતિ ચેક કરો
\item
  \textbf{ઓટોમેટેડ રિસ્પોન્સિસ}: ઇમર્જન્સી દરમિયાન તાત્કાલિક પગલાં
\item
  \textbf{પર્સનલાઇઝેશન}: વ્યક્તિગત પ્રાથમિકતાઓ આધારે કસ્ટમાઇઝ્ડ વાતાવરણ
\item
  \textbf{ઇન્ટિગ્રેશન}: હાલના ઘરેલું ઉપકરણો અને સિસ્ટમ્સ સાથે કામ કરે છે
\end{itemize}

\textbf{ટેકનિકલ સ્પેસિફિકેશન્સ:}

\begin{itemize}
\tightlist
\item
  \textbf{પ્રોટોકોલ્સ}: ડિવાઇસ કમ્યુનિકેશન માટે WiFi, Zigbee, Z-Wave
\item
  \textbf{પાવર બેકઅપ}: પાવર કટ દરમિયાન ક્રિટિકલ સેન્સર્સ માટે બેટરી બેકઅપ
\item
  \textbf{ડેટા એન્ક્રિપ્શન}: ડિવાઇસિસ અને ક્લાઉડ વચ્ચે સિક્યોર કમ્યુનિકેશન
\item
  \textbf{સ્કેલેબિલિટી}: નવા ડિવાઇસિસ અને સેન્સર્સનો સરળ ઉમેરો
\end{itemize}

\end{solutionbox}
\begin{mnemonicbox}
``કલેક્ટ પ્રોસેસ ડિસાઇડ કંટ્રોલ ઇન્ટરેક્ટ સિક્યોર''

\end{mnemonicbox}
\subsection*{પ્રશ્ન 5(અ) OR [3
ગુણ]}\label{uxaaauxab0uxab6uxaa8-5uxa85-or-3-uxa97uxaa3}

\textbf{કોઈપણ ત્રણ ઇન્ડસ્ટ્રિયલ અને મિલિટરી IoT એપ્લિકેશન્સની યાદી બનાવો.}

\begin{solutionbox}

\textbf{ઇન્ડસ્ટ્રિયલ IoT એપ્લિકેશન્સ:}

{\def\LTcaptype{none} % do not increment counter
\begin{longtable}[]{@{}
  >{\raggedright\arraybackslash}p{(\linewidth - 4\tabcolsep) * \real{0.3611}}
  >{\raggedright\arraybackslash}p{(\linewidth - 4\tabcolsep) * \real{0.3611}}
  >{\raggedright\arraybackslash}p{(\linewidth - 4\tabcolsep) * \real{0.2778}}@{}}
\toprule\noalign{}
\begin{minipage}[b]{\linewidth}\raggedright
એપ્લિકેશન
\end{minipage} & \begin{minipage}[b]{\linewidth}\raggedright
વર્ણન
\end{minipage} & \begin{minipage}[b]{\linewidth}\raggedright
ફાયદા
\end{minipage} \\
\midrule\noalign{}
\endhead
\bottomrule\noalign{}
\endlastfoot
\textbf{પ્રિડિક્ટિવ મેઇન્ટેનન્સ} & રિયલ-ટાઇમમાં સાધનોના સ્વાસ્થ્યનું નિરીક્ષણ &
ડાઉનટાઇમ ઘટાડો, ખર્ચ ઓછો \\
\textbf{સપ્લાય ચેઇન ટ્રેકિંગ} & ફેક્ટરીથી ગ્રાહક સુધી માલનો ટ્રેક & કાર્યક્ષમતા
સુધારો, નુકસાન ઘટાડો \\
\textbf{એનર્જી મેનેજમેન્ટ} & વીજ વપરાશનું નિરીક્ષણ અને ઓપ્ટિમાઇઝેશન & એનર્જી કોસ્ટ
20-30\% ઘટાડો \\
\end{longtable}
}

\textbf{મિલિટરી IoT એપ્લિકેશન્સ:}

{\def\LTcaptype{none} % do not increment counter
\begin{longtable}[]{@{}
  >{\raggedright\arraybackslash}p{(\linewidth - 4\tabcolsep) * \real{0.3611}}
  >{\raggedright\arraybackslash}p{(\linewidth - 4\tabcolsep) * \real{0.3611}}
  >{\raggedright\arraybackslash}p{(\linewidth - 4\tabcolsep) * \real{0.2778}}@{}}
\toprule\noalign{}
\begin{minipage}[b]{\linewidth}\raggedright
એપ્લિકેશન
\end{minipage} & \begin{minipage}[b]{\linewidth}\raggedright
વર્ણન
\end{minipage} & \begin{minipage}[b]{\linewidth}\raggedright
ફાયદા
\end{minipage} \\
\midrule\noalign{}
\endhead
\bottomrule\noalign{}
\endlastfoot
\textbf{બેટલફીલ્ડ સર્વેલન્સ} & લડાઇ ઝોનનું રિયલ-ટાઇમ મોનિટરિંગ & વધારેલ સિચ્યુએશનલ
અવેરનેસ \\
\textbf{એસેટ ટ્રેકિંગ} & મિલિટરી સાધનો અને વાહનોનું નિરીક્ષણ & ચોરી અટકાવો,
લોજિસ્ટિક્સ ઓપ્ટિમાઇઝ કરો \\
\textbf{સોલ્જર હેલ્થ મોનિટરિંગ} & કર્મચારીઓના વાઇટલ સાઇન્સનો ટ્રેક & સલામતી
સુધારો, મેડિકલ રિસ્પોન્સ \\
\end{longtable}
}

\end{solutionbox}
\begin{mnemonicbox}
``પ્રિડિક્ટ ટ્રેક એનર્જી, સર્વે ટ્રેક મોનિટર''

\end{mnemonicbox}
\subsection*{પ્રશ્ન 5(બ) OR [4
ગુણ]}\label{uxaaauxab0uxab6uxaa8-5uxaac-or-4-uxa97uxaa3}

\textbf{IoT માં વિવિધ પ્રકારના એક્ચ્યુએટર્સની યાદી બનાવો અને કોઈપણ બેના કાર્યને
સંક્ષિપ્તમાં સમજાવો.}

\begin{solutionbox}

\textbf{IoT એક્ચ્યુએટર પ્રકારો:}

{\def\LTcaptype{none} % do not increment counter
\begin{longtable}[]{@{}lll@{}}
\toprule\noalign{}
એક્ચ્યુએટર પ્રકાર & કાર્ય & એપ્લિકેશન્સ \\
\midrule\noalign{}
\endhead
\bottomrule\noalign{}
\endlastfoot
\textbf{સર્વો મોટર} & ચોક્કસ કોણીય સ્થિતિ & રોબોટિક્સ, ઓટોમેશન \\
\textbf{રિલે} & ઇલેક્ટ્રિકલ સ્વિચિંગ & લાઇટ્સ, ફેન્સ, ઉપકરણો \\
\textbf{સોલેનોઇડ વાલ્વ} & પ્રવાહી પ્રવાહ નિયંત્રણ & સિંચાઈ, HVAC \\
\textbf{LED} & પ્રકાશ ઉત્સર્જન & સૂચકાંકો, ડિસ્પ્લે \\
\textbf{બઝર} & અવાજ ઉત્પાદન & એલાર્મ્સ, નોટિફિકેશન્સ \\
\textbf{સ્ટેપર મોટર} & ચોક્કસ રોટેશનલ કંટ્રોલ & 3D પ્રિન્ટર્સ, CNC \\
\end{longtable}
}

\textbf{વિગતવાર કાર્ય:}

\textbf{1. સર્વો મોટર:}

\begin{itemize}
\tightlist
\item
  \textbf{કંટ્રોલ સિગ્નલ}: PWM (પલ્સ વિડ્થ મોડ્યુલેશન) સિગ્નલ સ્થિતિ નિર્ધારિત કરે છે
\item
  \textbf{ફીડબેક સિસ્ટમ}: આંતરિક પોટેન્શિયોમીટર પોઝિશન ફીડબેક પ્રદાન કરે છે
\item
  \textbf{કાર્ય}: કંટ્રોલ સર્કિટ ઇચ્છિત વિ એક્ચ્યુઅલ પોઝિશનની સરખામણી કરે છે
\item
  \textbf{એપ્લિકેશન્સ}: રોબોટિક આર્મ્સ, કેમેરા પેન/ટિલ્ટ, ઓટોમેટિક દરવાજા
\end{itemize}

\textbf{2. રિલે મોડ્યુલ:}

\begin{itemize}
\tightlist
\item
  \textbf{ઇલેક્ટ્રોમેગ્નેટિક સિદ્ધાંત}: એનર્જાઇઝ થાય ત્યારે કોઇલ મેગ્નેટિક ફીલ્ડ બનાવે છે
\item
  \textbf{સ્વિચિંગ એક્શન}: મેગ્નેટિક ફીલ્ડ મેકેનિકલ કોન્ટેક્ટ્સને ખસેડે છે
\item
  \textbf{આઇસોલેશન}: કંટ્રોલ અને લોડ સર્કિટ્સ વચ્ચે ઇલેક્ટ્રિકલ આઇસોલેશન
\item
  \textbf{એપ્લિકેશન્સ}: હોમ ઓટોમેશન, ઇન્ડસ્ટ્રિયલ કંટ્રોલ, સેફ્ટી સિસ્ટમ્સ
\end{itemize}

\end{solutionbox}
\begin{mnemonicbox}
``સર્વો રિલે સોલેનોઇડ LED બઝર સ્ટેપર''

\end{mnemonicbox}
\subsection*{પ્રશ્ન 5(ક) OR [7
ગુણ]}\label{uxaaauxab0uxab6uxaa8-5uxa95-or-7-uxa97uxaa3}

\textbf{IoT નો ઉપયોગ કરીને સ્માર્ટ પાર્કિંગ સિસ્ટમનો બ્લોક ડાયાગ્રામ દોરો અને તેનું
કાર્ય સમજાવો.}

\begin{solutionbox}

\includegraphics[width=1\linewidth,height=\textheight,keepaspectratio]{mermaid-32916c23.pdf}

\textbf{સ્માર્ટ પાર્કિંગ સિસ્ટમ કાર્ય:}

\textbf{1. વાહન શોધ:}

\begin{itemize}
\tightlist
\item
  \textbf{સેન્સર પ્લેસમેન્ટ}: દરેક પાર્કિંગ સ્પેસ પર IR અથવા અલ્ટ્રાસોનિક સેન્સર્સ
  ઇન્સ્ટોલ કરવામાં આવે છે
\item
  \textbf{ડિટેક્શન મેકેનિઝ્મ}: સેન્સર્સ વાહનોની હાજરી/ગેરહાજરી શોધે છે
\item
  \textbf{સ્ટેટસ મોનિટરિંગ}: સ્પેસ ઓક્યુપેન્સીનું સતત નિરીક્ષણ
\item
  \textbf{ડેટા એક્યુરસી}: અનેક સેન્સર્સ ખોટા પોઝિટિવ રીડિંગ્સ ઘટાડે છે
\end{itemize}

\textbf{2. ડેટા કલેક્શન અને પ્રોસેસિંગ:}

\begin{itemize}
\tightlist
\item
  \textbf{માઇક્રોકંટ્રોલર}: NodeMCU/Arduino સેન્સર ડેટાને સ્થાનિક રીતે પ્રોસેસ કરે છે
\item
  \textbf{સ્ટેટસ ડિટર્મિનેશન}: ઓક્યુપાઇડ (સેન્સર બ્લોક્ડ) અથવા ફ્રી (સેન્સર ક્લિયર)
\item
  \textbf{ટાઇમ સ્ટેમ્પિંગ}: બિલિંગ માટે એન્ટ્રી અને એક્ઝિટ ટાઇમ રેકોર્ડ કરવા
\item
  \textbf{ડેટા વેલિડેશન}: અસ્થાયી અવરોધો (પાંદડા, કચરો) ફિલ્ટર કરવા
\end{itemize}

\textbf{3. કમ્યુનિકેશન અને ક્લાઉડ ઇન્ટિગ્રેશન:} (ચાલુ)

\begin{itemize}
\tightlist
\item
  \textbf{WiFi ટ્રાન્સમિશન}: ક્લાઉડ સર્વર પર રિયલ-ટાઇમ ડેટા મોકલવામાં આવે છે
\item
  \textbf{ડેટાબેસ સ્ટોરેજ}: પાર્કિંગ સ્પેસ સ્ટેટસના રેકોર્ડ્સ જાળવવા
\item
  \textbf{એનાલિટિક્સ પ્રોસેસિંગ}: ઉપયોગના પેટર્ન અને આંકડા જનરેટ કરવા
\item
  \textbf{API ઇન્ટિગ્રેશન}: મોબાઇલ એપ્સ અને ડિસ્પ્લે સિસ્ટમ્સ સાથે કનેક્ટ કરવું
\end{itemize}

\textbf{4. યુઝર ઇન્ટરફેસ અને સર્વિસિસ:}

\begin{itemize}
\tightlist
\item
  \textbf{મોબાઇલ એપ્લિકેશન}: યુઝર્સ પાર્કિંગ સ્પેસિસ શોધી અને રિઝર્વ કરી શકે છે
\item
  \textbf{રિયલ-ટાઇમ અપડેટ્સ}: ઉપલબ્ધ પાર્કિંગ સ્પેસિસનું લાઇવ સ્ટેટસ
\item
  \textbf{નેવિગેશન આસિસ્ટન્સ}: પસંદ કરેલી પાર્કિંગ સ્પેસ સુધી GPS માર્ગદર્શન
\item
  \textbf{પેમેન્ટ ઇન્ટિગ્રેશન}: પાર્કિંગ ફી માટે ઓનલાઇન પેમેન્ટ
\end{itemize}

\textbf{5. વિઝ્યુઅલ ઇન્ડિકેટર્સ:}

\begin{itemize}
\tightlist
\item
  \textbf{LED ઇન્ડિકેટર્સ}: દરેક સ્પેસ માટે લીલો (ફ્રી), લાલ (ઓક્યુપાઇડ)
\item
  \textbf{ડિસ્પ્લે બોર્ડ્સ}: કુલ ઉપલબ્ધ સ્પેસિસ દર્શાવતા ઇલેક્ટ્રોનિક સાઇન્સ
\item
  \textbf{મોબાઇલ નોટિફિકેશન્સ}: રિઝર્વ્ડ ટાઇમ એક્સપાયર થતો હોય ત્યારે એલર્ટ્સ
\item
  \textbf{એડમિન ડેશબોર્ડ}: મોનિટરિંગ અને કંટ્રોલ માટે મેનેજમેન્ટ ઇન્ટરફેસ
\end{itemize}

\textbf{6. એડવાન્સ્ડ ફીચર્સ:}

\begin{itemize}
\tightlist
\item
  \textbf{સ્પેસ રિઝર્વેશન}: અગાઉથી પાર્કિંગ સ્પેસ બુક કરવી
\item
  \textbf{ઓટોમેટિક બિલિંગ}: પાર્કિંગ અવધિ આધારે ચાર્જિસ કેલ્ક્યુલેટ કરવા
\item
  \textbf{વાયોલેશન ડિટેક્શન}: અનધિકૃત પાર્કિંગ માટે એલર્ટ
\item
  \textbf{ડેટા એનાલિટિક્સ}: પીક ઉપયોગ કલાકો, રેવન્યુ એનાલિસિસ
\end{itemize}

\textbf{સિસ્ટમ ફાયદા:}

\begin{itemize}
\tightlist
\item
  \textbf{ટાઇમ સેવિંગ}: પાર્કિંગ શોધવામાં લાગતો સમય ઘટાડે છે
\item
  \textbf{ટ્રાફિક રિડક્શન}: સ્પેસિસ શોધતાં ફરવાનું ઓછું
\item
  \textbf{રેવન્યુ ઓપ્ટિમાઇઝેશન}: માંગ આધારિત ડાયનેમિક પ્રાઇસિંગ
\item
  \textbf{એન્વાયરન્મેન્ટલ ઇમ્પેક્ટ}: ઇંધણ વપરાશ અને ઉત્સર્જન ઘટાડે છે
\end{itemize}

\textbf{ટેકનિકલ ઘટકો:}

\begin{itemize}
\tightlist
\item
  \textbf{સેન્સર્સ}: IR પ્રોક્સિમિટી સેન્સર્સ અથવા અલ્ટ્રાસોનિક ડિસ્ટન્સ સેન્સર્સ
\item
  \textbf{માઇક્રોકંટ્રોલર્સ}: ESP8266/ESP32 બેઝ્ડ ડેવલપમેન્ટ બોર્ડ્સ
\item
  \textbf{કમ્યુનિકેશન}: WiFi, LoRaWAN, અથવા સેલ્યુલર કનેક્ટિવિટી
\item
  \textbf{પાવર સપ્લાય}: રિમોટ લોકેશન્સ માટે બેટરી બેકઅપ સાથે સોલાર પેનલ્સ
\end{itemize}

\textbf{ઇમ્પ્લિમેન્ટેશન પડકારો:}

\begin{itemize}
\tightlist
\item
  \textbf{વેધર રેઝિસ્ટન્સ}: સેન્સર્સએ વરસાદ, બરફ, આત્યંતિક તાપમાનમાં કામ કરવું જોઈએ
\item
  \textbf{પાવર મેનેજમેન્ટ}: બેટરી-પાવર્ડ સેન્સર્સને કાર્યક્ષમ પાવર વપરાશની જરૂર છે
\item
  \textbf{નેટવર્ક રિલાયબિલિટી}: કનેક્ટિવિટી ઇશ્યુઝ માટે બેકઅપ કમ્યુનિકેશન મેથડ્સ
\item
  \textbf{મેઇન્ટેનન્સ}: સેન્સર્સની નિયમિત સફાઈ અને કેલિબ્રેશન
\end{itemize}

\textbf{કોસ્ટ-બેનિફિટ એનાલિસિસ:}

\begin{itemize}
\tightlist
\item
  \textbf{પ્રારંભિક રોકાણ}: સેન્સર ઇન્સ્ટોલેશન અને સિસ્ટમ સેટઅપ કોસ્ટ
\item
  \textbf{ઓપરેશનલ સેવિંગ્સ}: મેનેજમેન્ટ ઓવરહેડ ઘટાડવું
\item
  \textbf{રેવન્યુ ઇન્ક્રીઝ}: સુધારેલ સ્પેસ યુટિલાઇઝેશન અને ડાયનેમિક પ્રાઇસિંગ
\item
  \textbf{પેબેક પીરિયડ}: કમર્શિયલ ઇન્સ્ટોલેશન્સ માટે સામાન્ય રીતે 12-18 મહિના
\end{itemize}

\textbf{ઇન્ટિગ્રેશન પોસિબિલિટીઝ:}

\begin{itemize}
\tightlist
\item
  \textbf{સ્માર્ટ સિટી સિસ્ટમ્સ}: ટ્રાફિક મેનેજમેન્ટ સિસ્ટમ્સ સાથે કનેક્ટ કરવું
\item
  \textbf{બિલ્ડિંગ ઓટોમેશન}: શોપિંગ મોલ અથવા ઓફિસ બિલ્ડિંગ સિસ્ટમ્સ સાથે ઇન્ટિગ્રેશન
\item
  \textbf{પબ્લિક ટ્રાન્સપોર્ટેશન}: બસ/મેટ્રો શેડ્યુલ્સ સાથે કોઓર્ડિનેટ કરવું
\item
  \textbf{ઇમર્જન્સી સર્વિસિસ}: ઇમર્જન્સી વાહનો માટે પ્રાયોરિટી એક્સેસ
\end{itemize}

\textbf{ભવિષ્યની એન્હાન્સમેન્ટ્સ:}

\begin{itemize}
\tightlist
\item
  \textbf{AI ઇન્ટિગ્રેશન}: મશીન લર્નિંગ વાપરીને પાર્કિંગ ડિમાન્ડ પ્રિડિક્ટ કરવી
\item
  \textbf{ઇલેક્ટ્રિક વાહન ચાર્જિંગ}: EV ચાર્જિંગ સ્ટેશન્સ સાથે ઇન્ટિગ્રેશન
\item
  \textbf{ઓટોનોમસ વાહન્સ}: સેલ્ફ-પાર્કિંગ કાર્સ માટે સપોર્ટ
\item
  \textbf{મોબાઇલ પેમેન્ટ એક્સપેન્શન}: ડિજિટલ વોલેટ્સ સાથે ઇન્ટિગ્રેશન
\end{itemize}

\end{solutionbox}
\begin{mnemonicbox}
``ડિટેક્ટ પ્રોસેસ કમ્યુનિકેટ ઇન્ટરફેસ ઇન્ડિકેટ સર્વ''

\end{mnemonicbox}

\end{document}
