\documentclass[10pt,a4paper]{article}

% content/resources/templates/preamble.tex
\usepackage[margin=0.6in]{geometry}
\author{Milav Dabgar}
\usepackage{amsmath,amssymb,amsthm}
\usepackage{booktabs}
\usepackage{multirow}
\usepackage{xcolor}
\usepackage{tcolorbox}
\tcbuselibrary{breakable,skins}
\usepackage[colorlinks=true,linkcolor=blue]{hyperref}
\usepackage{titlesec}
\usepackage{enumitem}
\usepackage{tikz}
\usepackage{pgfplots}
\usepackage{circuitikz}
\usepackage[version=4]{mhchem}
\usepackage{longtable}
\usepackage{array}
\usepackage{float}
\usepackage{caption}
\usepackage{listings}

\lstset{
  basicstyle=\small\ttfamily,
  breaklines=true,
  breakatwhitespace=false,
  postbreak=\mbox{\textcolor{red}{$\hookrightarrow$}\space},
  float=false,
  numbers=left,
  numberstyle=\tiny\color{gray},
  numbersep=10pt,
  xleftmargin=2em,
  keywordstyle=\color{blue},
  commentstyle=\color{green!60!black},
  stringstyle=\color{purple},
  backgroundcolor=\color{gray!5},
  showstringspaces=false,
  tabsize=2,
  captionpos=b,
  keepspaces=true,
  columns=flexible
}

\pgfplotsset{compat=1.18}
\usetikzlibrary{shapes,arrows,positioning,calc,patterns,decorations.pathmorphing,decorations.markings,arrows.meta}

% Color scheme
\definecolor{headcolor}{RGB}{0,102,204}
\definecolor{keycolor}{RGB}{220,20,60}
\definecolor{solutioncolor}{RGB}{34,139,34}
\definecolor{mnemoniccolor}{RGB}{148,0,211}
\definecolor{codecolor}{RGB}{0,0,100}

% Spacing
\setlength{\parskip}{3pt}
\setlist[itemize]{nosep}
\setlist[enumerate]{nosep}

% Title formatting
\titleformat{\section}{\Large\bfseries\color{headcolor}}{\thesection}{1em}{}
\titleformat{\subsection}{\large\bfseries\color{headcolor}}{\thesubsection}{1em}{}

% Pandoc tightlist compatibility
\providecommand{\tightlist}{%
  \setlength{\itemsep}{0pt}\setlength{\parskip}{0pt}}

% Pandoc longtable compatibility
\newcounter{none}
\def\thenone{}


% content/resources/templates/gujarati-boxes.tex
\usepackage{fontspec}
\usepackage{polyglossia}

% Set Gujarati as main language (document is primarily in Gujarati)
% Note: gloss-gujarati.ldf doesn't exist in polyglossia, but it will use hyphenation patterns
\setdefaultlanguage{gujarati}
\setotherlanguage{english}

% Configure Gujarati font properly
% Use Language=Default to prevent polyglossia from trying to add language-specific features
% that don't exist for Gujarati, which causes "empty feature" warnings
\newfontfamily\gujaratifont[Script=Gujarati,AutoFakeBold=2.5,AutoFakeSlant=0.3]{Noto Sans Gujarati}
\setmainfont[Script=Gujarati,AutoFakeBold=2.5,AutoFakeSlant=0.3]{Noto Sans Gujarati}
% Use Noto Sans Gujarati for monospace to support Gujarati in text
\setmonofont[Scale=0.9]{Noto Sans Gujarati}

% Configure English to use the same font
\newfontfamily\englishfont[Script=Gujarati,AutoFakeBold=2.5,AutoFakeSlant=0.3]{Noto Sans Gujarati}

% Translations for polyglossia
\gappto\captionsgujarati{
  \renewcommand{\tablename}{કોષ્ટક}
  \renewcommand{\figurename}{આકૃતિ}
}

% Helper for TikZ nodes to ensure Gujarati font
\newcommand{\gu}[1]{{\gujaratifont #1}}

% Custom environments
\newtcolorbox{solutionbox}{
    breakable,
    enhanced,
    colback=solutioncolor!5!white,
    colframe=solutioncolor!75!black,
    fonttitle=\bfseries,
    title=જવાબ
}

\newtcolorbox{solutionboxnobreak}{
 colback=solutioncolor!5!white,
 colframe=solutioncolor!75!black,
 fonttitle=\bfseries,
 title=જવાબ
}

\newtcolorbox{keyformula}{
 breakable,
 enhanced,
 colback=keycolor!5!white,
 colframe=keycolor!75!black,
 fonttitle=\bfseries,
 title=રાસાયણિક સમીકરણ/સૂત્ર
}

\newtcolorbox{mnemonicbox}{
 breakable,
 enhanced,
 colback=mnemoniccolor!5!white,
 colframe=mnemoniccolor!75!black,
 fonttitle=\bfseries,
 title=મેમરી ટ્રીક
}


\begin{document}

\begin{center}
{\Huge\bfseries\color{headcolor} Subject Name (Gujarati)}\\[5pt]
{\LARGE 4353202 -- Summer 2025}\\[3pt]
{\large Semester 1 Study Material}\\[3pt]
{\normalsize\textit{Detailed Solutions and Explanations}}
\end{center}

\vspace{10pt}

\subsection*{પ્રશ્ન 1(અ) [3
ગુણ]}\label{uxaaauxab0uxab6uxaa8-1uxa85-3-uxa97uxaa3}

\textbf{બધા જ પ્રકારના સૉફ્ટવેર એપ્લિકેશન ડોમેઇન ની યાદી બનાવો અને Embedded
Software સમજાવો}

\begin{solutionbox}

\textbf{સૉફ્ટવેર એપ્લિકેશન ડોમેઇન:}

{\def\LTcaptype{none} % do not increment counter
\begin{longtable}[]{@{}ll@{}}
\toprule\noalign{}
ડોમેઇન & વર્ણન \\
\midrule\noalign{}
\endhead
\bottomrule\noalign{}
\endlastfoot
\textbf{સિસ્ટમ સૉફ્ટવેર} & ઓપરેટિંગ સિસ્ટમ, ડિવાઇસ ડ્રાઇવર \\
\textbf{એપ્લિકેશન સૉફ્ટવેર} & વર્ડ પ્રોસેસર, ગેમ્સ, બિઝનેસ એપ્સ \\
\textbf{એન્જિનિયરિંગ/સાયન્ટિફિક સૉફ્ટવેર} & CAD, સિમ્યુલેશન ટૂલ \\
\textbf{એમ્બેડેડ સૉફ્ટવેર} & રિયલ-ટાઇમ કંટ્રોલ સિસ્ટમ \\
\textbf{વેબ એપ્લિકેશન} & બ્રાઉઝર-આધારિત એપ્લિકેશન \\
\textbf{AI સૉફ્ટવેર} & મશીન લર્નિંગ, એક્સપર્ટ સિસ્ટમ \\
\end{longtable}
}

\textbf{એમ્બેડેડ સૉફ્ટવેર} એ વિશેષ સૉફ્ટવેર છે જે ચોક્કસ હાર્ડવેર સાથે એમ્બેડેડ સિસ્ટમ પર
ચાલે છે. આ વોશિંગ મશીન, કાર અને મેડિકલ ઉપકરણોમાં વપરાય છે.

\begin{itemize}
\tightlist
\item
  \textbf{રિયલ-ટાઇમ ઓપરેશન}: નિર્ધારિત સમયમર્યાદામાં જવાબ આપવો જોઈએ
\item
  \textbf{રિસોર્સ મર્યાદાઓ}: મર્યાદિત મેમરી અને પ્રોસેસિંગ પાવર
\item
  \textbf{હાર્ડવેર પર નિર્ભરતા}: ચોક્કસ હાર્ડવેર સાથે ગાઢ એકીકરણ
\end{itemize}

\end{solutionbox}
\begin{mnemonicbox}
``SAEEWA'' - System, Application, Engineering,
Embedded, Web, AI

\end{mnemonicbox}
\subsection*{પ્રશ્ન 1(બ) [4
ગુણ]}\label{uxaaauxab0uxab6uxaa8-1uxaac-4-uxa97uxaa3}

\textbf{જેનેરિક ફ્રેમવર્ક એક્ટિવિટીસ અને અમ્બ્રેલા એક્ટિવિટીસ સમજાવો}

\begin{solutionbox}

\textbf{જેનેરિક ફ્રેમવર્ક એક્ટિવિટીસ:}

{\def\LTcaptype{none} % do not increment counter
\begin{longtable}[]{@{}ll@{}}
\toprule\noalign{}
એક્ટિવિટી & હેતુ \\
\midrule\noalign{}
\endhead
\bottomrule\noalign{}
\endlastfoot
\textbf{કોમ્યુનિકેશન} & હિતધારકોથી જરૂરિયાતો એકત્રિત કરવી \\
\textbf{પ્લાનિંગ} & કાર્ય યોજના અને શેડ્યૂલ બનાવવું \\
\textbf{મોડેલિંગ} & વિશ્લેષણ અને ડિઝાઇન મોડેલ બનાવવા \\
\textbf{કન્સ્ટ્રક્શન} & કોડ જનરેશન અને ટેસ્ટિંગ \\
\textbf{ડિપ્લોયમેન્ટ} & સૉફ્ટવેર ડિલિવરી અને સપોર્ટ \\
\end{longtable}
}

\textbf{અમ્બ્રેલા એક્ટિવિટીસ:}

{\def\LTcaptype{none} % do not increment counter
\begin{longtable}[]{@{}ll@{}}
\toprule\noalign{}
એક્ટિવિટી & હેતુ \\
\midrule\noalign{}
\endhead
\bottomrule\noalign{}
\endlastfoot
\textbf{પ્રોજેક્ટ મેનેજમેન્ટ} & પ્રગતિ ટ્રેક કરવી અને નિયંત્રણ \\
\textbf{રિસ્ક મેનેજમેન્ટ} & જોખમો ઓળખવા અને ઘટાડવા \\
\textbf{ક્વોલિટી એશ્યોરન્સ} & સૉફ્ટવેર ગુણવત્તા સુનિશ્ચિત કરવી \\
\textbf{કન્ફિગરેશન મેનેજમેન્ટ} & ફેરફારોને નિયંત્રિત કરવા \\
\textbf{વર્ક પ્રોડક્ટ પ્રિપરેશન} & દસ્તાવેજીકરણ બનાવવું \\
\end{longtable}
}

\begin{itemize}
\tightlist
\item
  \textbf{ફ્રેમવર્ક એક્ટિવિટીસ}: દરેક પ્રોજેક્ટમાં મુખ્ય ક્રમિક પ્રવૃત્તિઓ
\item
  \textbf{અમ્બ્રેલા એક્ટિવિટીસ}: પ્રોજેક્ટ જીવનકાળ દરમિયાન સતત પ્રવૃત્તિઓ
\end{itemize}

\end{solutionbox}
\begin{mnemonicbox}
``CPMCD'' ફ્રેમવર્ક માટે, ``PRQCW'' અમ્બ્રેલા માટે

\end{mnemonicbox}
\subsection*{પ્રશ્ન 1(ક) [7
ગુણ]}\label{uxaaauxab0uxab6uxaa8-1uxa95-7-uxa97uxaa3}

\textbf{સૉફ્ટવેર ડેવલપમેંટ લાઇફ સાઇકલની આકૃતિ દોરી તેના તબક્કાઓ સમજાવો}

\begin{solutionbox}

\textbf{SDLC આકૃતિ:}

\begin{center}
\textbf{Mermaid Diagram (Code)}
\begin{verbatim}
{Shaded}
{Highlighting}[]
graph LR
    A[Requirements Analysis] {-{-}{} B[System Design]}
    B {-{-}{} C[Implementation]}
    C {-{-}{} D[Testing]}
    D {-{-}{} E[Deployment]}
    E {-{-}{} F[Maintenance]}
    F {-{-}{} A}
{Highlighting}
{Shaded}
\end{verbatim}
\end{center}

\textbf{SDLC તબક્કાઓ:}

{\def\LTcaptype{none} % do not increment counter
\begin{longtable}[]{@{}
  >{\raggedright\arraybackslash}p{(\linewidth - 4\tabcolsep) * \real{0.2800}}
  >{\raggedright\arraybackslash}p{(\linewidth - 4\tabcolsep) * \real{0.4000}}
  >{\raggedright\arraybackslash}p{(\linewidth - 4\tabcolsep) * \real{0.3200}}@{}}
\toprule\noalign{}
\begin{minipage}[b]{\linewidth}\raggedright
તબક્કો
\end{minipage} & \begin{minipage}[b]{\linewidth}\raggedright
પ્રવૃત્તિઓ
\end{minipage} & \begin{minipage}[b]{\linewidth}\raggedright
પરિણામો
\end{minipage} \\
\midrule\noalign{}
\endhead
\bottomrule\noalign{}
\endlastfoot
\textbf{જરૂરિયાત વિશ્લેષણ} & વપરાશકર્તા જરૂરિયાતો એકત્રિત કરવી, SRS બનાવવું &
SRS દસ્તાવેજ \\
\textbf{સિસ્ટમ ડિઝાઇન} & આર્કિટેક્ચર ડિઝાઇન, UI ડિઝાઇન & ડિઝાઇન દસ્તાવેજ \\
\textbf{અમલીકરણ} & કોડ ડેવલપમેન્ટ, યુનિટ ટેસ્ટિંગ & સોર્સ કોડ \\
\textbf{ટેસ્ટિંગ} & એકીકરણ, સિસ્ટમ ટેસ્ટિંગ & ટેસ્ટ રિપોર્ટ \\
\textbf{ડિપ્લોયમેન્ટ} & ઇન્સ્ટોલેશન, વપરાશકર્તા તાલીમ & ડિપ્લોય થયેલ સિસ્ટમ \\
\textbf{જાળવણી} & બગ ફિક્સ, સુધારાઓ & અપડેટ થયેલ સિસ્ટમ \\
\end{longtable}
}

\begin{itemize}
\tightlist
\item
  \textbf{વ્યવસ્થિત અભિગમ}: દરેક તબક્કાના ચોક્કસ ઇનપુટ અને આઉટપુટ
\item
  \textbf{ગુણવત્તા ગેટ}: તબક્કાઓ વચ્ચે સમીક્ષા ગુણવત્તા સુનિશ્ચિત કરે છે
\item
  \textbf{પુનરાવર્તિત પ્રકૃતિ}: પ્રતિપુષ્ટિ આગામી ચક્રો સુધારે છે
\end{itemize}

\end{solutionbox}
\begin{mnemonicbox}
``વાસ્તવિક સિસ્ટમ અમલીકરણ ટેસ્ટ દરમિયાન જાળવણી''

\end{mnemonicbox}
\subsection*{પ્રશ્ન 1(ક) OR [7
ગુણ]}\label{uxaaauxab0uxab6uxaa8-1uxa95-or-7-uxa97uxaa3}

\textbf{સોફ્ટવેર ડેવલપમેંટ મોડેલ્સની યાદી બનાવી કોઈ પણ બે મોડલ જરૂરી આકૃતિ સાથે
સમજાવો}

\begin{solutionbox}

\textbf{સૉફ્ટવેર ડેવલપમેન્ટ મોડેલ્સ:}

{\def\LTcaptype{none} % do not increment counter
\begin{longtable}[]{@{}ll@{}}
\toprule\noalign{}
મોડેલ & લાક્ષણિકતાઓ \\
\midrule\noalign{}
\endhead
\bottomrule\noalign{}
\endlastfoot
\textbf{વોટરફોલ મોડેલ} & ક્રમિક, રેખીય અભિગમ \\
\textbf{પુનરાવર્તિત મોડેલ} & ડેવલપમેન્ટના પુનરાવર્તિત ચક્રો \\
\textbf{સ્પાઇરલ મોડેલ} & જોખમ-સંચાલિત, પુનરાવર્તિત \\
\textbf{એજાઇલ મોડેલ} & લવચીક, ગ્રાહક સહયોગ \\
\textbf{RAD મોડેલ} & ઝડપી પ્રોટોટાઇપિંગ \\
\textbf{V-મોડેલ} & વેરિફિકેશન અને વેલિડેશન પર ધ્યાન \\
\end{longtable}
}

\textbf{1. વોટરફોલ મોડેલ:}

\begin{center}
\textbf{Mermaid Diagram (Code)}
\begin{verbatim}
{Shaded}
{Highlighting}[]
graph LR
    A[Requirements] {-{-}{} B[Design]}
    B {-{-}{} C[Implementation]}
    C {-{-}{} D[Testing]}
    D {-{-}{} E[Deployment]}
    E {-{-}{} F[Maintenance]}
{Highlighting}
{Shaded}
\end{verbatim}
\end{center}

\textbf{2. સ્પાઇરલ મોડેલ:}

\begin{center}
\textbf{Mermaid Diagram (Code)}
\begin{verbatim}
{Shaded}
{Highlighting}[]
graph LR
    A[Planning] {-{-}{} B[Risk Analysis]}
    B {-{-}{} C[Engineering]}
    C {-{-}{} D[Evaluation]}
    D {-{-}{} A}
{Highlighting}
{Shaded}
\end{verbatim}
\end{center}

\begin{itemize}
\tightlist
\item
  \textbf{વોટરફોલ}: સરળ, સારી રીતે સમજાયેલ જરૂરિયાતો માટે યોગ્ય
\item
  \textbf{સ્પાઇરલ}: ઉચ્ચ જોખમવાળા પ્રોજેક્ટને પુનરાવર્તિત જોખમ મૂલ્યાંકન સાથે હેન્ડલ
  કરે છે
\end{itemize}

\end{solutionbox}
\begin{mnemonicbox}
``WIRRAV'' - Waterfall, Iterative, RAD, Risk-driven,
Agile, V-model

\end{mnemonicbox}
\subsection*{પ્રશ્ન 2(અ) [3
ગુણ]}\label{uxaaauxab0uxab6uxaa8-2uxa85-3-uxa97uxaa3}

\textbf{SCRUM એજાઇલ પ્રોસેસ મોડલ અને SPIRAL પ્રોસેસ મોડલ વચ્ચેના તફાવત લખો}

\begin{solutionbox}

{\def\LTcaptype{none} % do not increment counter
\begin{longtable}[]{@{}lll@{}}
\toprule\noalign{}
પાસું & SCRUM & SPIRAL \\
\midrule\noalign{}
\endhead
\bottomrule\noalign{}
\endlastfoot
\textbf{અભિગમ} & એજાઇલ, પુનરાવર્તિત & જોખમ-સંચાલિત, પુનરાવર્તિત \\
\textbf{અવધિ} & નિશ્ચિત સ્પ્રિન્ટ (2-4 અઠવાડિયા) & ચલ સ્પાઇરલ ચક્રો \\
\textbf{ધ્યાન} & ગ્રાહક સહયોગ & જોખમ વ્યવસ્થાપન \\
\textbf{આયોજન} & સ્પ્રિન્ટ પ્લાનિંગ & વ્યાપક આયોજન \\
\textbf{દસ્તાવેજીકરણ} & ન્યૂનતમ દસ્તાવેજીકરણ & વિગતવાર દસ્તાવેજીકરણ \\
\textbf{ટીમ સાઇઝ} & નાની ટીમ (5-9 સભ્યો) & કોઈપણ ટીમ સાઇઝ \\
\end{longtable}
}

\begin{itemize}
\tightlist
\item
  \textbf{SCRUM}: ઝડપી ડિલિવરી અને ગ્રાહક પ્રતિપુષ્ટિ પર ભાર
\item
  \textbf{SPIRAL}: જોખમ ઓળખ અને શમન પર ધ્યાન
\end{itemize}

\end{solutionbox}
\begin{mnemonicbox}
``SCRUM=સ્પીડ, SPIRAL=સેફ્ટી''

\end{mnemonicbox}
\subsection*{પ્રશ્ન 2(બ) [4
ગુણ]}\label{uxaaauxab0uxab6uxaa8-2uxaac-4-uxa97uxaa3}

\textbf{જરૂરિયાત એકત્રીકરણ તકનીકોની યાદી આપો અને કોઇ પણ એક સમજાવો}

\begin{solutionbox}

\textbf{જરૂરિયાત એકત્રીકરણ તકનીકો:}

{\def\LTcaptype{none} % do not increment counter
\begin{longtable}[]{@{}ll@{}}
\toprule\noalign{}
તકનીક & વર્ણન \\
\midrule\noalign{}
\endhead
\bottomrule\noalign{}
\endlastfoot
\textbf{ઇન્ટરવ્યુ} & હિતધારકો સાથે સીધી વાતચીત \\
\textbf{પ્રશ્નાવલી} & માળખાગત લેખિત પ્રશ્નો \\
\textbf{અવલોકન} & વપરાશકર્તાઓને કાર્ય કરતા જોવા \\
\textbf{દસ્તાવેજ વિશ્લેષણ} & હાલના દસ્તાવેજોની સમીક્ષા \\
\textbf{પ્રોટોટાઇપિંગ} & કાર્યશીલ મોડેલ બનાવવા \\
\textbf{બ્રેઇનસ્ટોર્મિંગ} & ગ્રૂપ આઇડિયા જનરેશન \\
\end{longtable}
}

\textbf{ઇન્ટરવ્યુ તકનીક સમજાવેલ:}

\begin{itemize}
\tightlist
\item
  \textbf{માળખાગત ઇન્ટરવ્યુ}: પૂર્વનિર્ધારિત પ્રશ્નો, ઔપચારિક અભિગમ
\item
  \textbf{અમાળખાગત ઇન્ટરવ્યુ}: ખુલ્લી ચર્ચા, લવચીક
\item
  \textbf{અર્ધ-માળખાગત}: બંનેનું મિશ્રણ
\end{itemize}

\textbf{ફાયદાઓ}: સીધી હિતધારક ઇનપુટ, સ્પષ્ટીકરણ શક્ય, વિગતવાર માહિતી
\textbf{પડકારો}: સમય વપરાશ, ઇન્ટરવ્યુઅર પૂર્વગ્રહ, અધૂરી માહિતી

\end{solutionbox}
\begin{mnemonicbox}
``IQDPBB'' - Interview, Questionnaire, Document,
Prototype, Brainstorm, Observe

\end{mnemonicbox}
\subsection*{પ્રશ્ન 2(ક) [7
ગુણ]}\label{uxaaauxab0uxab6uxaa8-2uxa95-7-uxa97uxaa3}

\textbf{યુઝ કેસ ડાયગ્રામ વ્યાખ્યાપિત કરો. તેને ઉદાહરણ સાથે સમજાવો}

\begin{solutionbox}

\textbf{યુઝ કેસ ડાયગ્રામ વ્યાખ્યા:} યુઝ કેસ ડાયગ્રામ એક્ટર્સ અને તેમની યુઝ કેસ સાથેની
ક્રિયાપ્રતિક્રિયા દર્શાવીને સિસ્ટમની કાર્યાત્મક જરૂરિયાતો બતાવે છે.

\textbf{ઘટકો:}

{\def\LTcaptype{none} % do not increment counter
\begin{longtable}[]{@{}lll@{}}
\toprule\noalign{}
ઘટક & પ્રતીક & હેતુ \\
\midrule\noalign{}
\endhead
\bottomrule\noalign{}
\endlastfoot
\textbf{એક્ટર} & લાકડી આકૃતિ & બાહ્ય એન્ટિટી \\
\textbf{યુઝ કેસ} & અંડાકાર & સિસ્ટમ ફંક્શન \\
\textbf{એસોસિએશન} & લાઇન & એક્ટર-યુઝ કેસ સંબંધ \\
\textbf{સિસ્ટમ બાઉન્ડરી} & લંબચોરસ & સિસ્ટમ સ્કોપ \\
\end{longtable}
}

\textbf{ઉદાહરણ: લાઇબ્રેરી મેનેજમેન્ટ સિસ્ટમ}

\begin{center}
\textbf{Mermaid Diagram (Code)}
\begin{verbatim}
{Shaded}
{Highlighting}[]
graph TD
    A[લાઇબ્રેરિયન] {-{-}{} B(પુસ્તક ઇશ્યૂ કરવું)}
    A {-{-}{} C(પુસ્તક પરત કરવું)}
    A {-{-}{} D(પુસ્તક ઉમેરવું)}
    E[વિદ્યાર્થી] {-{-}{} B}
    E {-{-}{} C}
    E {-{-}{} F(પુસ્તક શોધવું)}
{Highlighting}
{Shaded}
\end{verbatim}
\end{center}

\textbf{સંબંધો:}

\begin{itemize}
\tightlist
\item
  \textbf{Include}: યુઝ કેસ દ્વારા શેર કરાયેલ સામાન્ય કાર્યક્ષમતા
\item
  \textbf{Extend}: બેઝ યુઝ કેસમાં વૈકલ્પિક કાર્યક્ષમતા ઉમેરવી
\item
  \textbf{સામાન્યીકરણ}: એક્ટર્સ અથવા યુઝ કેસ વચ્ચે વારસો
\end{itemize}

\textbf{ફાયદાઓ}: સ્પષ્ટ કાર્યાત્મક ઝાંખી, કોમ્યુનિકેશન ટૂલ, ટેસ્ટિંગ માટે આધાર

\end{solutionbox}
\begin{mnemonicbox}
``એક્ટર્સ યુઝ કેસ સિસ્ટમની અંદર''

\end{mnemonicbox}
\subsection*{પ્રશ્ન 2(અ) OR [3
ગુણ]}\label{uxaaauxab0uxab6uxaa8-2uxa85-or-3-uxa97uxaa3}

\textbf{વોટર ફોલ મોડલ અને ઈટરેટિવ વોટર ફોલ મોડલ ની સરખામણી કરો}

\begin{solutionbox}

{\def\LTcaptype{none} % do not increment counter
\begin{longtable}[]{@{}lll@{}}
\toprule\noalign{}
પાસું & વોટરફોલ મોડેલ & ઇટરેટિવ વોટરફોલ \\
\midrule\noalign{}
\endhead
\bottomrule\noalign{}
\endlastfoot
\textbf{તબક્કાઓ} & ક્રમિક, એક વખત & પુનરાવર્તનમાં પુનરાવૃત્તિ \\
\textbf{પ્રતિપુષ્ટિ} & પ્રોજેક્ટના અંતે & દરેક પુનરાવર્તન પછી \\
\textbf{જોખમ} & મોડેથી જોખમ ઓળખ & વહેલી જોખમ ઓળખ \\
\textbf{લવચીકતા} & કઠોર, કોઈ ફેરફાર નહીં & ફેરફારોને સમાવે છે \\
\textbf{ટેસ્ટિંગ} & ડેવલપમેન્ટ પછી & સતત ટેસ્ટિંગ \\
\textbf{ડિલિવરી} & એક અંતિમ ડિલિવરી & બહુવિધ વૃદ્ધિશીલ ડિલિવરી \\
\end{longtable}
}

\begin{itemize}
\tightlist
\item
  \textbf{વોટરફોલ}: સ્થિર, સારી રીતે વ્યાખ્યાયિત જરૂરિયાતો માટે યોગ્ય
\item
  \textbf{ઇટરેટિવ વોટરફોલ}: પ્રતિપુષ્ટિ સાથે વિકસિત જરૂરિયાતો માટે બહેતર
\end{itemize}

\end{solutionbox}
\begin{mnemonicbox}
``PFRTFD'' - Phases, Feedback, Risk, Testing,
Flexibility, Delivery

\end{mnemonicbox}
\subsection*{પ્રશ્ન 2(બ) OR [4
ગુણ]}\label{uxaaauxab0uxab6uxaa8-2uxaac-or-4-uxa97uxaa3}

\textbf{ફંકશનલ અને નોન-ફંકશનલ જરૂરિયાતની વ્યાખ્યા લખી બંનેના ઉદાહરણ આપો}

\begin{solutionbox}

\textbf{ફંકશનલ જરૂરિયાતો:} સિસ્ટમે શું કરવું જોઈએ - ચોક્કસ વર્તણૂકો અને કાર્યોને
વ્યાખ્યાયિત કરતી જરૂરિયાતો.

\textbf{નોન-ફંકશનલ જરૂરિયાતો:} સિસ્ટમ કેવી રીતે કાર્ય કરે છે - ગુણવત્તા લક્ષણો અને
મર્યાદાઓને વ્યાખ્યાયિત કરતી જરૂરિયાતો.

{\def\LTcaptype{none} % do not increment counter
\begin{longtable}[]{@{}lll@{}}
\toprule\noalign{}
પ્રકાર & ફંકશનલ & નોન-ફંકશનલ \\
\midrule\noalign{}
\endhead
\bottomrule\noalign{}
\endlastfoot
\textbf{વ્યાખ્યા} & સિસ્ટમ વર્તણૂક & સિસ્ટમ ગુણવત્તા \\
\textbf{ઉદાહરણો} & લોગિન, ગણતરી, સંગ્રહ & પ્રદર્શન, સુરક્ષા \\
\textbf{ટેસ્ટિંગ} & બ્લેક-બોક્સ ટેસ્ટિંગ & લોડ, સ્ટ્રેસ ટેસ્ટિંગ \\
\textbf{દસ્તાવેજીકરણ} & યુઝ કેસ, દૃશ્યો & ગુણવત્તા મેટ્રિક્સ \\
\end{longtable}
}

\textbf{ફંકશનલ ઉદાહરણો:}

\begin{itemize}
\tightlist
\item
  વપરાશકર્તા પ્રમાણીકરણ અને લોગિન
\item
  કુલ બિલ રકમની ગણતરી કરવી
\item
  માસિક રિપોર્ટ જનરેટ કરવી
\end{itemize}

\textbf{નોન-ફંકશનલ ઉદાહરણો:}

\begin{itemize}
\tightlist
\item
  સિસ્ટમ રિસ્પોન્સ ટાઇમ \textless{} 2 સેકન્ડ (પ્રદર્શન)
\item
  99.9\% સિસ્ટમ ઉપલબ્ધતા (વિશ્વસનીયતા)
\item
  1000 સમવર્તી વપરાશકર્તાઓને સપોર્ટ (સ્કેલેબિલિટી)
\end{itemize}

\end{solutionbox}
\begin{mnemonicbox}
``ફંકશનલ=શું, નોન-ફંકશનલ=કેવી રીતે''

\end{mnemonicbox}
\subsection*{પ્રશ્ન 2(ક) OR [7
ગુણ]}\label{uxaaauxab0uxab6uxaa8-2uxa95-or-7-uxa97uxaa3}

\textbf{કોહેશનની વ્યાખ્યા આપો. કોહેશનનું વર્ગીકરણ સમજાવો}

\begin{solutionbox}

\textbf{કોહેશન વ્યાખ્યા:} કોહેશન માપે છે કે મોડ્યુલની અંદરના ત્તત્વો કેટલા નજીકથી
સંબંધિત છે. ઉચ્ચ કોહેશન સારી રીતે ડિઝાઇન કરાયેલ મોડ્યુલ દર્શાવે છે.

\textbf{કોહેશનનું વર્ગીકરણ (સૌથી મજબૂતથી સૌથી નબળું):}

{\def\LTcaptype{none} % do not increment counter
\begin{longtable}[]{@{}lll@{}}
\toprule\noalign{}
પ્રકાર & વર્ણન & ઉદાહરણ \\
\midrule\noalign{}
\endhead
\bottomrule\noalign{}
\endlastfoot
\textbf{ફંકશનલ} & એક, સારી રીતે વ્યાખ્યાયિત કાર્ય & વર્ગમૂળ ગણતરી \\
\textbf{સિક્વન્શિયલ} & એકનું આઉટપુટ = બીજાનું ઇનપુટ & વાંચવું\rightarrowપ્રોસેસ કરવું\rightarrowલખવું \\
\textbf{કોમ્યુનિકેશનલ} & સમાન ડેટા પર કામ કરવું & ગ્રાહક રેકોર્ડ અપડેટ \\
\textbf{પ્રોસિજરલ} & અમલીકરણનો ક્રમ અનુસરવો & પેરોલ પ્રોસેસિંગ સ્ટેપ્સ \\
\textbf{ટેમ્પોરલ} & સમાન સમયે અમલ & સિસ્ટમ પ્રારંભીકરણ \\
\textbf{લોજિકલ} & સમાન લોજિકલ ફંક્શન & બધા ઇનપુટ/આઉટપુટ ઓપરેશન \\
\textbf{કોઇન્સિડેન્ટલ} & કોઈ અર્થપૂર્ણ સંબંધ નહીં & રેન્ડમ યુટિલિટીઝ \\
\end{longtable}
}

\begin{center}
\textbf{Mermaid Diagram (Code)}
\begin{verbatim}
{Shaded}
{Highlighting}[]
graph LR
    A[ફંકશનલ {- સૌથી મજબૂત] {-}{-}{} B[સિક્વન્શિયલ]}
    B {-{-}{} C[કોમ્યુનિકેશનલ]}
    C {-{-}{} D[પ્રોસિજરલ]}
    D {-{-}{} E[ટેમ્પોરલ]}
    E {-{-}{} F[લોજિકલ]}
    F {-{-}{} G[કોઇન્સિડેન્ટલ {-} સૌથી નબળું]}
{Highlighting}
{Shaded}
\end{verbatim}
\end{center}

\textbf{લક્ષ્ય}: જાળવણીયોગ્ય, વિશ્વસનીય મોડ્યુલ માટે ફંકશનલ કોહેશન હાંસલ કરવું

\end{solutionbox}
\begin{mnemonicbox}
``ફ્રેન્કની સ્માર્ટ બિલાડી ટેનિસ લોજિકલી રમે છે''

\end{mnemonicbox}
\subsection*{પ્રશ્ન 3(અ) [3
ગુણ]}\label{uxaaauxab0uxab6uxaa8-3uxa85-3-uxa97uxaa3}

\textbf{સારા સોફ્ટવેર ડિઝાઇનની લાક્ષણિકતાઓની યાદી બનાવો}

\begin{solutionbox}

\textbf{સારા સૉફ્ટવેર ડિઝાઇનની લાક્ષણિકતાઓ:}

{\def\LTcaptype{none} % do not increment counter
\begin{longtable}[]{@{}ll@{}}
\toprule\noalign{}
લાક્ષણિકતા & વર્ણન \\
\midrule\noalign{}
\endhead
\bottomrule\noalign{}
\endlastfoot
\textbf{મોડ્યુલારિટી} & સ્વતંત્ર મોડ્યુલમાં વિભાજિત \\
\textbf{એબ્સ્ટ્રેક્શન} & અમલીકરણ વિગતો છુપાવવી \\
\textbf{એન્કેપ્સ્યુલેશન} & ડેટા અને મેથડ્સ એકસાથે બંડલ કરવા \\
\textbf{હાયરાર્કી} & સ્તરો/લેવલમાં સંગઠિત \\
\textbf{સરળતા} & સમજવામાં અને જાળવવામાં સરળ \\
\textbf{લવચીકતા} & ભવિષ્યના ફેરફારોને સમાવવા \\
\end{longtable}
}

\begin{itemize}
\tightlist
\item
  \textbf{ઉચ્ચ કોહેશન}: સંબંધિત ત્તત્વો એકસાથે જૂથબદ્ધ
\item
  \textbf{નીચું કપલિંગ}: મોડ્યુલ વચ્ચે ન્યૂનતમ નિર્ભરતાઓ
\item
  \textbf{પુનઃઉપયોગિતા}: ઘટકોને અન્ય સિસ્ટમમાં ફરીથી વાપરી શકાય
\end{itemize}

\end{solutionbox}
\begin{mnemonicbox}
``MAEHSF'' - Modularity, Abstraction, Encapsulation,
Hierarchy, Simplicity, Flexibility

\end{mnemonicbox}
\subsection*{પ્રશ્ન 3(બ) [4
ગુણ]}\label{uxaaauxab0uxab6uxaa8-3uxaac-4-uxa97uxaa3}

\textbf{ઈંટરમીડીયેટ COCOMO મોડલ દ્વારા પ્રોજેક્ટ એસ્ટીમેશન પધ્ધતિ સમજાવો}

\begin{solutionbox}

\textbf{ઇન્ટરમીડિયેટ COCOMO મોડેલ:} ઉત્પાદકતાને અસર કરતા કોસ્ટ ડ્રાઇવરોને ધ્યાનમાં
લઈને બેઝિક COCOMO ને વિસ્તૃત કરે છે.

\textbf{સૂત્ર:} Effort = a \times (KLOC)\^{}b \times EAF

\textbf{કોસ્ટ ડ્રાઇવર્સ:}

{\def\LTcaptype{none} % do not increment counter
\begin{longtable}[]{@{}lll@{}}
\toprule\noalign{}
કેટેગરી & ડ્રાઇવર્સ & પ્રભાવ \\
\midrule\noalign{}
\endhead
\bottomrule\noalign{}
\endlastfoot
\textbf{પ્રોડક્ટ} & વિશ્વસનીયતા, જટિલતા & પ્રયત્ન ગુણક \\
\textbf{હાર્ડવેર} & એક્ઝિક્યુશન ટાઇમ, સ્ટોરેજ & પ્રદર્શન મર્યાદાઓ \\
\textbf{કર્મચારીવર્ગ} & વિશ્લેષક ક્ષમતા, અનુભવ & ટીમ કુશળતા \\
\textbf{પ્રોજેક્ટ} & આધુનિક પ્રથાઓ, શેડ્યૂલ & ડેવલપમેન્ટ વાતાવરણ \\
\end{longtable}
}

\textbf{પ્રયત્ન સમાયોજન ફેક્ટર (EAF):} EAF = બધા કોસ્ટ ડ્રાઇવર ગુણકોનું ગુણાકાર

\textbf{પગલાં:}

\begin{enumerate}
\tightlist
\item
  KLOC (કોડની હજારો લાઇન) નો અંદાજ કાઢવો
\item
  પ્રોજેક્ટ પ્રકાર આધારે યોગ્ય a, b મૂલ્યો પસંદ કરવા
\item
  કોસ્ટ ડ્રાઇવર્સનું મૂલ્યાંકન (સ્કેલ 0.70 થી 1.65)
\item
  EAF ની ગણતરી કરવી
\item
  પર્સન-મંથમાં પ્રયત્ન મેળવવા માટે સૂત્ર લાગુ કરવું
\end{enumerate}

\end{solutionbox}
\begin{mnemonicbox}
``PHPP'' - Product, Hardware, Personnel, Project
drivers

\end{mnemonicbox}
\subsection*{પ્રશ્ન 3(ક) [7
ગુણ]}\label{uxaaauxab0uxab6uxaa8-3uxa95-7-uxa97uxaa3}

\textbf{ઓનલાઇન શોપિંગ સિસ્ટમ માટે લેવલ-1 નો ડેટા ફ્લો ડાયગ્રામ દોરો અને સમજાવો}

\begin{solutionbox}

\textbf{ઓનલાઇન શોપિંગ સિસ્ટમ માટે લેવલ-1 DFD:}

\begin{verbatim}
    +{-{-}{-}{-}{-}{-}{-}{-}{-}{-}+                    +{-}{-}{-}{-}{-}{-}{-}{-}{-}{-}+}
    |          |     Order Info     |          |
    |Customer  |{{-}{-}{-}{-}{-}{-}{-}{-}{-}{-}{-}{-}{-}{-}{-}{-}{-}{-}| Process  |}
    |          |     Product Info   | Order    |
    +{-{-}{-}{-}{-}{-}{-}{-}{-}{-}+                    +{-}{-}{-}{-}{-}{-}{-}{-}{-}{-}+}
                                           |
                                           | Order Details
                                           v
    +{-{-}{-}{-}{-}{-}{-}{-}{-}{-}+     Payment Info   +{-}{-}{-}{-}{-}{-}{-}{-}{-}{-}+}
    |Payment   |{{-}{-}{-}{-}{-}{-}{-}{-}{-}{-}{-}{-}{-}{-}{-}{-}{-}{-}| Process  |}
    |Gateway   |                    | Payment  |
    +{-{-}{-}{-}{-}{-}{-}{-}{-}{-}+                    +{-}{-}{-}{-}{-}{-}{-}{-}{-}{-}+}
                                           |
                                           | Inventory Update
                                           v
    +{-{-}{-}{-}{-}{-}{-}{-}{-}{-}+     Stock Info     +{-}{-}{-}{-}{-}{-}{-}{-}{-}{-}+}
    |Inventory |{{-}{-}{-}{-}{-}{-}{-}{-}{-}{-}{-}{-}{-}{-}{-}{-}{-}{-}| Manage   |}
    |Manager   |                    |Inventory |
    +{-{-}{-}{-}{-}{-}{-}{-}{-}{-}+                    +{-}{-}{-}{-}{-}{-}{-}{-}{-}{-}+}
\end{verbatim}

\textbf{પ્રોસેસ:}

{\def\LTcaptype{none} % do not increment counter
\begin{longtable}[]{@{}llll@{}}
\toprule\noalign{}
પ્રોસેસ & ઇનપુટ & આઉટપુટ & વર્ણન \\
\midrule\noalign{}
\endhead
\bottomrule\noalign{}
\endlastfoot
\textbf{ઓર્ડર પ્રોસેસ} & ગ્રાહક ઓર્ડર & ઓર્ડર પુષ્ટિકરણ & ઓર્ડર પ્લેસમેન્ટ હેન્ડલ
કરવું \\
\textbf{પેમેન્ટ પ્રોસેસ} & પેમેન્ટ વિગતો & પેમેન્ટ સ્ટેટસ & ટ્રાન્ઝેક્શન પ્રોસેસ કરવા \\
\textbf{ઇન્વેન્ટરી મેનેજ} & સ્ટોક ક્વેરી & સ્ટોક સ્ટેટસ & પ્રોડક્ટ ઉપલબ્ધતા ટ્રેક
કરવી \\
\end{longtable}
}

\textbf{ડેટા સ્ટોર:}

\begin{itemize}
\tightlist
\item
  \textbf{પ્રોડક્ટ ડેટાબેઝ}: પ્રોડક્ટ માહિતી સંગ્રહિત કરવી
\item
  \textbf{ઓર્ડર ડેટાબેઝ}: ઓર્ડર વિગતો સંગ્રહિત કરવી
\item
  \textbf{ગ્રાહક ડેટાબેઝ}: ગ્રાહક પ્રોફાઇલ સંગ્રહિત કરવી
\end{itemize}

\textbf{બાહ્ય એન્ટિટીઝ:}

\begin{itemize}
\tightlist
\item
  \textbf{ગ્રાહક}: ઓર્ડર મૂકે છે, પેમેન્ટ કરે છે
\item
  \textbf{પેમેન્ટ ગેટવે}: પેમેન્ટ પ્રોસેસ કરે છે
\item
  \textbf{ઇન્વેન્ટરી મેનેજર}: સ્ટોક લેવલ અપડેટ કરે છે
\end{itemize}

\end{solutionbox}
\begin{mnemonicbox}
``PPMI'' - Process order, Process payment, Manage
inventory

\end{mnemonicbox}
\subsection*{પ્રશ્ન 3(અ) OR [3
ગુણ]}\label{uxaaauxab0uxab6uxaa8-3uxa85-or-3-uxa97uxaa3}

\textbf{એનાલિસિસ અને ડિઝાઇન વચ્ચેનો તફાવત લખો}

\begin{solutionbox}

{\def\LTcaptype{none} % do not increment counter
\begin{longtable}[]{@{}lll@{}}
\toprule\noalign{}
પાસું & એનાલિસિસ & ડિઝાઇન \\
\midrule\noalign{}
\endhead
\bottomrule\noalign{}
\endlastfoot
\textbf{ધ્યાન} & સિસ્ટમે શું કરવું જોઈએ & સિસ્ટમ કેવી રીતે કામ કરશે \\
\textbf{તબક્કો} & જરૂરિયાત તબક્કો & ડિઝાઇન તબક્કો \\
\textbf{આઉટપુટ} & સમસ્યાની સમજ & સોલ્યુશન સ્ટ્રક્ચર \\
\textbf{મોડેલ} & યુઝ કેસ, જરૂરિયાતો & આર્કિટેક્ચર, ક્લાસ \\
\textbf{દૃષ્ટિકોણ} & વપરાશકર્તાનો દૃષ્ટિકોણ & ડેવલપરનો દૃષ્ટિકોણ \\
\textbf{સ્તર} & અમૂર્ત, સંકલ્પનાત્મક & નક્કર, વિગતવાર \\
\end{longtable}
}

\begin{itemize}
\tightlist
\item
  \textbf{એનાલિસિસ}: સમસ્યા-કેન્દ્રિત, જરૂરિયાતોની સમજ
\item
  \textbf{ડિઝાઇન}: સોલ્યુશન-કેન્દ્રિત, સિસ્ટમ આર્કિટેક્ચર બનાવવું
\end{itemize}

\end{solutionbox}
\begin{mnemonicbox}
``એનાલિસિસ=શું, ડિઝાઇન=કેવી રીતે''

\end{mnemonicbox}
\subsection*{પ્રશ્ન 3(બ) OR [4
ગુણ]}\label{uxaaauxab0uxab6uxaa8-3uxaac-or-4-uxa97uxaa3}

\textbf{બેઝિક COCOMO મોડલ દ્વારા પ્રોજેક્ટ એસ્ટીમેશન પધ્ધતિ સમજાવો}

\begin{solutionbox}

\textbf{બેઝિક COCOMO મોડેલ:} કોડની લાઇન આધારે સૉફ્ટવેર ડેવલપમેન્ટ પ્રયત્નનો અંદાજ
કાઢે છે.

\textbf{સૂત્ર:}

\begin{itemize}
\tightlist
\item
  Effort = a \times (KLOC)\^{}b person-months
\item
  Time = c \times (Effort)\^{}d months
\end{itemize}

\textbf{પ્રોજેક્ટ પ્રકારો:}

{\def\LTcaptype{none} % do not increment counter
\begin{longtable}[]{@{}llllll@{}}
\toprule\noalign{}
પ્રકાર & a & b & c & d & વર્ણન \\
\midrule\noalign{}
\endhead
\bottomrule\noalign{}
\endlastfoot
\textbf{ઓર્ગેનિક} & 2.4 & 1.05 & 2.5 & 0.38 & નાની, અનુભવી ટીમ \\
\textbf{સેમી-ડિટેચ્ડ} & 3.0 & 1.12 & 2.5 & 0.35 & મધ્યમ કદ, મિશ્ર ટીમ \\
\textbf{એમ્બેડેડ} & 3.6 & 1.20 & 2.5 & 0.32 & જટિલ, કડક મર્યાદાઓ \\
\end{longtable}
}

\textbf{પગલાં:}

\begin{enumerate}
\tightlist
\item
  KLOC (કોડની હજારો લાઇન) નો અંદાજ કાઢવો
\item
  પ્રોજેક્ટ પ્રકાર ઓળખવો (organic/semi-detached/embedded)
\item
  યોગ્ય ગુણાંકો લાગુ કરવા
\item
  પ્રયત્ન અને ડેવલપમેન્ટ સમયની ગણતરી કરવી
\end{enumerate}

\textbf{ઉદાહરણ}: 10 KLOC ઓર્ગેનિક પ્રોજેક્ટ

\begin{itemize}
\tightlist
\item
  Effort = 2.4 \times (10)\^{}1.05 = 25.2 person-months
\item
  Time = 2.5 \times (25.2)\^{}0.38 = 8.4 months
\end{itemize}

\end{solutionbox}
\begin{mnemonicbox}
``OSE'' - Organic, Semi-detached, Embedded

\end{mnemonicbox}
\subsection*{પ્રશ્ન 3(ક) OR [7
ગુણ]}\label{uxaaauxab0uxab6uxaa8-3uxa95-or-7-uxa97uxaa3}

\textbf{લાઇબ્રેરી મેનેજમેન્ટ સિસ્ટમ માટે ક્લાસ ડાયગ્રામ દોરો અને સમજાવો}

\begin{solutionbox}

\textbf{લાઇબ્રેરી મેનેજમેન્ટ સિસ્ટમ માટે ક્લાસ ડાયગ્રામ:}

\begin{verbatim}
classDiagram
class Library \{
    +name: String
    +address: String
    +addBook()
    +removeBook()
    +searchBook()
\}

class Book \{
    +bookId: String
    +title: String
    +author: String
    +ISBN: String
    +isAvailable: Boolean
    +getDetails()
\}

class Member \{
    +memberId: String
    +name: String
    +email: String
    +phone: String
    +issueBook()
    +returnBook()
\}

class Transaction \{
    +transactionId: String
    +issueDate: Date
    +returnDate: Date
    +fine: Double
    +calculateFine()
\}

Library "1" o{-{-} "many" Book}
Member "1" o{-{-} "many" Transaction}
Book "1" o{-{-} "many" Transaction}
\end{verbatim}

\textbf{સંબંધો:}

{\def\LTcaptype{none} % do not increment counter
\begin{longtable}[]{@{}lll@{}}
\toprule\noalign{}
સંબંધ & વર્ણન & મલ્ટિપ્લિસિટી \\
\midrule\noalign{}
\endhead
\bottomrule\noalign{}
\endlastfoot
\textbf{લાઇબ્રેરી-બુક} & લાઇબ્રેરીમાં પુસ્તકો છે & 1 થી ઘણા \\
\textbf{મેમ્બર-ટ્રાન્ઝેક્શન} & મેમ્બરના ટ્રાન્ઝેક્શન છે & 1 થી ઘણા \\
\textbf{બુક-ટ્રાન્ઝેક્શન} & પુસ્તક ટ્રાન્ઝેક્શનમાં સામેલ & 1 થી ઘણા \\
\end{longtable}
}

\textbf{મુખ્ય લક્ષણો:}

\begin{itemize}
\tightlist
\item
  \textbf{એટ્રિબ્યુટ્સ}: દરેક ક્લાસના ડેટા સભ્યો
\item
  \textbf{મેથડ્સ}: ક્લાસ ડેટા પર કામ કરતા ફંક્શન
\item
  \textbf{એસોસિએશન}: ક્લાસો વચ્ચેના સંબંધો બતાવે છે કે તેઓ કેવી રીતે ક્રિયાપ્રતિક્રિયા
  કરે છે
\end{itemize}

\end{solutionbox}
\begin{mnemonicbox}
``LBMT'' - Library, Book, Member, Transaction

\end{mnemonicbox}
\subsection*{પ્રશ્ન 4(અ) [3
ગુણ]}\label{uxaaauxab0uxab6uxaa8-4uxa85-3-uxa97uxaa3}

\textbf{પ્રોજેક્ટ સાઇઝ નક્કી કરવાના મેટ્રિક્સની યાદી બનાવી તેની વ્યાખ્યા લખો}

\begin{solutionbox}

\textbf{પ્રોજેક્ટ સાઇઝ એસ્ટીમેશન મેટ્રિક્સ:}

{\def\LTcaptype{none} % do not increment counter
\begin{longtable}[]{@{}lll@{}}
\toprule\noalign{}
મેટ્રિક & વ્યાખ્યા & ઉપયોગ \\
\midrule\noalign{}
\endhead
\bottomrule\noalign{}
\endlastfoot
\textbf{લાઇન્સ ઓફ કોડ (LOC)} & એક્ઝિક્યુટેબલ કોડ લાઇનની ગણતરી & પરંપરાગત
સાઇઝિંગ \\
\textbf{ફંક્શન પોઇન્ટ્સ (FP)} & કાર્યક્ષમતા આધારિત માપ & ભાષા-સ્વતંત્ર \\
\textbf{ફીચર પોઇન્ટ્સ} & વિસ્તૃત ફંક્શન પોઇન્ટ્સ & રિયલ-ટાઇમ સિસ્ટમ \\
\textbf{ઓબ્જેક્ટ પોઇન્ટ્સ} & ઓબ્જેક્ટ અને મેથડ્સની ગણતરી & ઓબ્જેક્ટ-ઓરિએન્ટેડ સિસ્ટમ \\
\textbf{યુઝ કેસ પોઇન્ટ્સ} & યુઝ કેસ જટિલતા આધારિત & જરૂરિયાત-આધારિત \\
\end{longtable}
}

\textbf{ફંક્શન પોઇન્ટ્સ ઘટકો:}

\begin{itemize}
\tightlist
\item
  \textbf{એક્સટર્નલ ઇનપુટ્સ}: ડેટા એન્ટ્રી સ્ક્રીન
\item
  \textbf{એક્સટર્નલ આઉટપુટ્સ}: રિપોર્ટ્સ, મેસેજ
\item
  \textbf{એક્સટર્નલ ઇન્ક્વાયરીઝ}: ઇન્ટરેક્ટિવ ક્વેરીઝ
\item
  \textbf{ઇન્ટર્નલ ફાઇલ્સ}: માસ્ટર ફાઇલ્સ
\item
  \textbf{એક્સટર્નલ ઇન્ટરફેસ}: શેર કરેલ ડેટા
\end{itemize}

\textbf{ફાયદાઓ}: વહેલું અનુમાન, ટેકનોલોજી-સ્વતંત્ર, માનકીકૃત અભિગમ

\end{solutionbox}
\begin{mnemonicbox}
``LFFOU'' - LOC, Function Points, Feature Points,
Object Points, Use Case Points

\end{mnemonicbox}
\subsection*{પ્રશ્ન 4(બ) [4
ગુણ]}\label{uxaaauxab0uxab6uxaa8-4uxaac-4-uxa97uxaa3}

\textbf{જોખમની ઓળખને વિસ્તારથી સમજાવો}

\begin{solutionbox}

\textbf{જોખમ ઓળખ:} પ્રોજેક્ટની સફળતાને અસર કરી શકે તેવા સંભવિત જોખમોને શોધવા,
ઓળખવા અને વર્ણવવાની પ્રક્રિયા.

\textbf{જોખમ કેટેગરીઝ:}

{\def\LTcaptype{none} % do not increment counter
\begin{longtable}[]{@{}lll@{}}
\toprule\noalign{}
કેટેગરી & ઉદાહરણો & પ્રભાવ \\
\midrule\noalign{}
\endhead
\bottomrule\noalign{}
\endlastfoot
\textbf{ટેકનિકલ} & નવી ટેકનોલોજી, જટિલતા & ડેવલપમેન્ટ વિલંબ \\
\textbf{પ્રોજેક્ટ} & શેડ્યૂલ, બજેટ મર્યાદાઓ & કોસ્ટ ઓવરરન \\
\textbf{બિઝનેસ} & માર્કેટ ફેરફારો, સ્પર્ધા & પ્રોજેક્ટ રદ્દીકરણ \\
\textbf{બાહ્ય} & વેન્ડર મુદ્દાઓ, નિયમો & નિર્ભરતાઓ \\
\end{longtable}
}

\textbf{ઓળખ તકનીકો:}

\begin{itemize}
\tightlist
\item
  \textbf{બ્રેઇનસ્ટોર્મિંગ}: જોખમો ઓળખવા માટે ટીમ ચર્ચા
\item
  \textbf{ચેકલિસ્ટ}: માનક જોખમ કેટેગરીઝની સમીક્ષા
\item
  \textbf{એક્સપર્ટ જજમેન્ટ}: અનુભવ આધારિત ઓળખ
\item
  \textbf{SWOT એનાલિસિસ}: શક્તિઓ, નબળાઈઓ, તકો, ધમકીઓ
\end{itemize}

\textbf{રિસ્ક રજિસ્ટર:} ઓળખાયેલ જોખમો સાથેનો દસ્તાવેજ જેમાં છે:

\begin{itemize}
\tightlist
\item
  જોખમ વર્ણન
\item
  ઘટનાની સંભાવના
\item
  પ્રભાવની ગંભીરતા
\item
  જોખમ કેટેગરી
\item
  જવાબદાર વ્યક્તિ
\end{itemize}

\end{solutionbox}
\begin{mnemonicbox}
``TPBE'' - Technical, Project, Business, External
risks

\end{mnemonicbox}
\subsection*{પ્રશ્ન 4(ક) [7
ગુણ]}\label{uxaaauxab0uxab6uxaa8-4uxa95-7-uxa97uxaa3}

\textbf{તમારી પસંદની કોઇ સિસ્ટમ માટે Gantt Chart દોરો}

\begin{solutionbox}

\textbf{ઓનલાઇન બેંકિંગ સિસ્ટમ માટે ગેન્ટ ચાર્ટ:}

{\def\LTcaptype{none} % do not increment counter
\begin{longtable}[]{@{}
  >{\raggedright\arraybackslash}p{(\linewidth - 16\tabcolsep) * \real{0.0588}}
  >{\raggedright\arraybackslash}p{(\linewidth - 16\tabcolsep) * \real{0.1176}}
  >{\raggedright\arraybackslash}p{(\linewidth - 16\tabcolsep) * \real{0.1176}}
  >{\raggedright\arraybackslash}p{(\linewidth - 16\tabcolsep) * \real{0.1176}}
  >{\raggedright\arraybackslash}p{(\linewidth - 16\tabcolsep) * \real{0.1176}}
  >{\raggedright\arraybackslash}p{(\linewidth - 16\tabcolsep) * \real{0.1176}}
  >{\raggedright\arraybackslash}p{(\linewidth - 16\tabcolsep) * \real{0.1176}}
  >{\raggedright\arraybackslash}p{(\linewidth - 16\tabcolsep) * \real{0.1176}}
  >{\raggedright\arraybackslash}p{(\linewidth - 16\tabcolsep) * \real{0.1176}}@{}}
\toprule\noalign{}
\begin{minipage}[b]{\linewidth}\raggedright
કાર્ય
\end{minipage} & \begin{minipage}[b]{\linewidth}\raggedright
અઠવાડિયું 1
\end{minipage} & \begin{minipage}[b]{\linewidth}\raggedright
અઠવાડિયું 2
\end{minipage} & \begin{minipage}[b]{\linewidth}\raggedright
અઠવાડિયું 3
\end{minipage} & \begin{minipage}[b]{\linewidth}\raggedright
અઠવાડિયું 4
\end{minipage} & \begin{minipage}[b]{\linewidth}\raggedright
અઠવાડિયું 5
\end{minipage} & \begin{minipage}[b]{\linewidth}\raggedright
અઠવાડિયું 6
\end{minipage} & \begin{minipage}[b]{\linewidth}\raggedright
અઠવાડિયું 7
\end{minipage} & \begin{minipage}[b]{\linewidth}\raggedright
અઠવાડિયું 8
\end{minipage} \\
\midrule\noalign{}
\endhead
\bottomrule\noalign{}
\endlastfoot
\textbf{જરૂરિયાત વિશ્લેષણ} & ████████ & ████████ & & & & & & \\
\textbf{સિસ્ટમ ડિઝાઇન} & & ████████ & ████████ & & & & & \\
\textbf{ડેટાબેઝ ડિઝાઇન} & & & ████████ & ████████ & & & & \\
\textbf{UI ડેવલપમેન્ટ} & & & & ████████ & ████████ & & & \\
\textbf{બેકએન્ડ ડેવલપમેન્ટ} & & & & & ████████ & ████████ & & \\
\textbf{ટેસ્ટિંગ} & & & & & & ████████ & ████████ & \\
\textbf{ડિપ્લોયમેન્ટ} & & & & & & & ████████ & ████████ \\
\end{longtable}
}

\textbf{પ્રોજેક્ટ કાર્યો:}

{\def\LTcaptype{none} % do not increment counter
\begin{longtable}[]{@{}llll@{}}
\toprule\noalign{}
કાર્ય & અવધિ & નિર્ભરતાઓ & સંસાધનો \\
\midrule\noalign{}
\endhead
\bottomrule\noalign{}
\endlastfoot
\textbf{જરૂરિયાત વિશ્લેષણ} & 2 અઠવાડિયા & કોઈ નહીં & બિઝનેસ એનાલિસ્ટ \\
\textbf{સિસ્ટમ ડિઝાઇન} & 2 અઠવાડિયા & જરૂરિયાતો & સિસ્ટમ ડિઝાઇનર \\
\textbf{ડેટાબેઝ ડિઝાઇન} & 2 અઠવાડિયા & સિસ્ટમ ડિઝાઇન & ડેટાબેઝ ડિઝાઇનર \\
\textbf{UI ડેવલપમેન્ટ} & 2 અઠવાડિયા & સિસ્ટમ ડિઝાઇન & UI ડેવલપર \\
\textbf{બેકએન્ડ ડેવલપમેન્ટ} & 2 અઠવાડિયા & ડેટાબેઝ ડિઝાઇન & બેકએન્ડ ડેવલપર \\
\textbf{ટેસ્ટિંગ} & 2 અઠવાડિયા & UI + બેકએન્ડ & QA ટેસ્ટર \\
\textbf{ડિપ્લોયમેન્ટ} & 2 અઠવાડિયા & ટેસ્ટિંગ & DevOps એન્જિનિયર \\
\end{longtable}
}

\textbf{ફાયદાઓ}: દ્રશ્ય પ્રગતિ ટ્રેકિંગ, સંસાધન ફાળવણી, નિર્ભરતા વ્યવસ્થાપન

\end{solutionbox}
\begin{mnemonicbox}
``RSDUBtd'' - Requirements, System design, Database,
UI, Backend, Testing, Deployment

\end{mnemonicbox}
\subsection*{પ્રશ્ન 4(અ) OR [3
ગુણ]}\label{uxaaauxab0uxab6uxaa8-4uxa85-or-3-uxa97uxaa3}

\textbf{પ્રોજેક્ટ મેનેજરની જવાબદારીઓની યાદી બનાવો}

\begin{solutionbox}

\textbf{પ્રોજેક્ટ મેનેજરની જવાબદારીઓ:}

{\def\LTcaptype{none} % do not increment counter
\begin{longtable}[]{@{}ll@{}}
\toprule\noalign{}
ક્ષેત્ર & જવાબદારીઓ \\
\midrule\noalign{}
\endhead
\bottomrule\noalign{}
\endlastfoot
\textbf{આયોજન} & પ્રોજેક્ટ પ્લાન બનાવવા, સ્કોપ વ્યાખ્યાયિત કરવો \\
\textbf{સંગઠન} & સંસાધનો ફાળવવા, ટીમ બનાવવી \\
\textbf{નેતૃત્વ} & ટીમને પ્રેરણા આપવી, સંઘર્ષ ઉકેલવો \\
\textbf{નિયંત્રણ} & પ્રગતિ મોનિટર કરવી, ફેરફારો વ્યવસ્થિત કરવા \\
\textbf{કોમ્યુનિકેશન} & હિતધારક અપડેટ્સ, ટીમ કોર્ડિનેશન \\
\textbf{રિસ્ક મેનેજમેન્ટ} & જોખમો ઓળખવા અને શમન કરવા \\
\end{longtable}
}

\textbf{મુખ્ય પ્રવૃત્તિઓ:}

\begin{itemize}
\tightlist
\item
  \textbf{પ્રોજેક્ટ શરૂઆત}: ઉદ્દેશ્યો અને મર્યાદાઓ વ્યાખ્યાયિત કરવા
\item
  \textbf{શેડ્યૂલ મેનેજમેન્ટ}: ટાઇમલાઇન બનાવવી અને જાળવવી
\item
  \textbf{બજેટ નિયંત્રણ}: ખર્ચ અને વ્યય મોનિટર કરવા
\item
  \textbf{ગુણવત્તા આશ્વાસન}: ડિલિવરેબલ સ્ટાન્ડર્ડ સુનિશ્ચિત કરવા
\item
  \textbf{ટીમ મેનેજમેન્ટ}: ટીમ સભ્યોનું નેતૃત્વ અને વિકાસ
\end{itemize}

\end{solutionbox}
\begin{mnemonicbox}
``POLCR'' - Planning, Organizing, Leading,
Controlling, Risk management

\end{mnemonicbox}
\subsection*{પ્રશ્ન 4(બ) OR [4
ગુણ]}\label{uxaaauxab0uxab6uxaa8-4uxaac-or-4-uxa97uxaa3}

\textbf{જોખમ આકારણીને વિસ્તારથી સમજાવો}

\begin{solutionbox}

\textbf{જોખમ આકારણી:} પ્રોજેક્ટની સફળતા પર તેમની સંભાવના અને પ્રભાવ નક્કી કરવા
માટે ઓળખાયેલ જોખમોનું મૂલ્યાંકન કરવાની પ્રક્રિયા.

\textbf{આકારણી ઘટકો:}

{\def\LTcaptype{none} % do not increment counter
\begin{longtable}[]{@{}lll@{}}
\toprule\noalign{}
ઘટક & સ્કેલ & વર્ણન \\
\midrule\noalign{}
\endhead
\bottomrule\noalign{}
\endlastfoot
\textbf{સંભાવના} & 1-5 અથવા \% & જોખમ ઘટનાની સંભાવના \\
\textbf{પ્રભાવ} & 1-5 અથવા \$ & જો જોખમ થાય તો તીવ્રતા \\
\textbf{રિસ્ક સ્કોર} & P \times I & એકંદર જોખમ પ્રાથમિકતા \\
\end{longtable}
}

\textbf{જોખમ આકારણી મેટ્રિક્સ:}

{\def\LTcaptype{none} % do not increment counter
\begin{longtable}[]{@{}llll@{}}
\toprule\noalign{}
સંભાવના/પ્રભાવ & નીચું (1) & મધ્યમ (2) & ઉચ્ચ (3) \\
\midrule\noalign{}
\endhead
\bottomrule\noalign{}
\endlastfoot
\textbf{નીચું (1)} & 1 & 2 & 3 \\
\textbf{મધ્યમ (2)} & 2 & 4 & 6 \\
\textbf{ઉચ્ચ (3)} & 3 & 6 & 9 \\
\end{longtable}
}

\textbf{આકારણી તકનીકો:}

\begin{itemize}
\tightlist
\item
  \textbf{ગુણાત્મક આકારણી}: વર્ણનાત્મક સ્કેલ (ઉચ્ચ/મધ્યમ/નીચું)
\item
  \textbf{માત્રાત્મક આકારણી}: સંખ્યાત્મક મૂલ્યો અને ગણતરીઓ
\item
  \textbf{એક્સપર્ટ જજમેન્ટ}: અનુભવ આધારિત મૂલ્યાંકન
\item
  \textbf{ઐતિહાસિક ડેટા}: ભૂતકાળના પ્રોજેક્ટ વિશ્લેષણ
\end{itemize}

\textbf{જોખમ વર્ગીકરણ:}

\begin{itemize}
\tightlist
\item
  \textbf{ઉચ્ચ જોખમ} (7-9): તાત્કાલિક ધ્યાન જરૂરી
\item
  \textbf{મધ્યમ જોખમ} (4-6): મોનિટર કરવું અને શમન આયોજન કરવું
\item
  \textbf{નીચું જોખમ} (1-3): સ્વીકારવું અથવા ન્યૂનતમ શમન
\end{itemize}

\end{solutionbox}
\begin{mnemonicbox}
``PIS'' - Probability, Impact, Score

\end{mnemonicbox}
\subsection*{પ્રશ્ન 4(ક) OR [7
ગુણ]}\label{uxaaauxab0uxab6uxaa8-4uxa95-or-7-uxa97uxaa3}

\textbf{તમારી પસંદની કોઇ સિસ્ટમ માટે સ્પ્રિન્ટ બર્ન ડાઉન ચાર્ટ દોરો}

\begin{solutionbox}

\textbf{E-commerce મોબાઇલ એપ માટે સ્પ્રિન્ટ બર્ન ડાઉન ચાર્ટ (2-અઠવાડિયાનો
સ્પ્રિન્ટ):}

\begin{verbatim}
સ્ટોરી પોઇન્ટ્સ
    |
 40 +{-{-}{-}*}
    |    {}
 35 +     *
    |      {}
 30 +       *
    |        {}
 25 +         *{-{-}{-}*}
    |              {}
 20 +               *
    |                {}
 15 +                 *
    |                  {}
 10 +                   *
    |                    {}
  5 +                     *
    |                      {}
  0 +\_\_\_\_\_\_\_\_\_\_\_\_\_\_\_\_\_\_\_\_\_\_\_\_*
    1  2  3  4  5  6  7  8  9  10 દિવસ
    
    * = વાસ્તવિક પ્રગતિ
    {-{-}{-} = આદર્શ પ્રગતિ}
\end{verbatim}

\textbf{સ્પ્રિન્ટ વિગતો:}

{\def\LTcaptype{none} % do not increment counter
\begin{longtable}[]{@{}llll@{}}
\toprule\noalign{}
દિવસ & આદર્શ બાકી & વાસ્તવિક બાકી & પૂર્ણ થયેલ કાર્ય \\
\midrule\noalign{}
\endhead
\bottomrule\noalign{}
\endlastfoot
\textbf{દિવસ 1} & 36 & 40 & સ્પ્રિન્ટ પ્લાનિંગ \\
\textbf{દિવસ 2} & 32 & 35 & યુઝર લોગિન ફીચર \\
\textbf{દિવસ 3} & 28 & 30 & પ્રોડક્ટ કેટાલોગ \\
\textbf{દિવસ 4} & 24 & 25 & શોપિંગ કાર્ટ \\
\textbf{દિવસ 5} & 20 & 25 & API મુદ્દાથી અવરોધ \\
\textbf{દિવસ 6} & 16 & 20 & પેમેન્ટ એકીકરણ \\
\textbf{દિવસ 7} & 12 & 15 & ઓર્ડર મેનેજમેન્ટ \\
\textbf{દિવસ 8} & 8 & 10 & ટેસ્ટિંગ અને ફિક્સ \\
\textbf{દિવસ 9} & 4 & 5 & અંતિમ ટેસ્ટિંગ \\
\textbf{દિવસ 10} & 0 & 0 & સ્પ્રિન્ટ પૂર્ણ \\
\end{longtable}
}

\textbf{મુખ્ય અંતર્દૃષ્ટિ:}

\begin{itemize}
\tightlist
\item
  \textbf{ઢાળ}: આદર્શ સાથે સરખામણીએ પ્રગતિ દર
\item
  \textbf{સપાટ વિસ્તારો}: અવરોધિત કાર્ય અથવા સ્કોપ ફેરફારો
\item
  \textbf{આદર્શથી નીચે}: શેડ્યૂલ આગળ
\item
  \textbf{આદર્શથી ઉપર}: શેડ્યૂલ પાછળ
\end{itemize}

\end{solutionbox}
\begin{mnemonicbox}
``DABC'' - Days, Actual, Burn-down, Chart

\end{mnemonicbox}
\subsection*{પ્રશ્ન 5(અ) [3
ગુણ]}\label{uxaaauxab0uxab6uxaa8-5uxa85-3-uxa97uxaa3}

\textbf{કોડ રિવ્યુ તકનીકની યાદી બનાવી કોઈ એક સમજાવો}

\begin{solutionbox}

\textbf{કોડ રિવ્યુ તકનીકો:}

{\def\LTcaptype{none} % do not increment counter
\begin{longtable}[]{@{}lll@{}}
\toprule\noalign{}
તકનીક & વર્ણન & સહભાગીઓ \\
\midrule\noalign{}
\endhead
\bottomrule\noalign{}
\endlastfoot
\textbf{કોડ વોકથ્રુ} & લેખક દ્વારા અનૌપચારિક સમીક્ષા & લેખક + સમીક્ષકો \\
\textbf{કોડ ઇન્સ્પેક્શન} & ઔપચારિક, વ્યવસ્થિત સમીક્ષા & પ્રશિક્ષિત નિરીક્ષકો \\
\textbf{પીઅર રિવ્યુ} & સાથીદાર કોડ તપાસે છે & ડેવલપર સાથીદારો \\
\textbf{ટૂલ-આધારિત રિવ્યુ} & સ્વચાલિત વિશ્લેષણ & ટૂલ્સ + ડેવલપર્સ \\
\end{longtable}
}

\textbf{કોડ ઇન્સ્પેક્શન સમજાવેલ:}

\textbf{પ્રક્રિયા:}

\begin{enumerate}
\tightlist
\item
  \textbf{આયોજન}: કોડ પસંદ કરવો, ભૂમિકાઓ સોંપવી
\item
  \textbf{ઝાંખી}: લેખક કોડ સ્ટ્રક્ચર સમજાવે છે
\item
  \textbf{તૈયારી}: કોડની વ્યક્તિગત સમીક્ષા
\item
  \textbf{ઇન્સ્પેક્શન મીટિંગ}: ગ્રૂપ કોડ તપાસે છે
\item
  \textbf{રિવર્ક}: ઓળખાયેલ ખામીઓ ઠીક કરવી
\item
  \textbf{ફોલો-અપ}: સુધારાઓ ચકાસવા
\end{enumerate}

\textbf{ભૂમિકાઓ:}

\begin{itemize}
\tightlist
\item
  \textbf{મોડરેટર}: ઇન્સ્પેક્શન પ્રક્રિયાનું નેતૃત્વ
\item
  \textbf{લેખક}: કોડ ડેવલપર, લોજિક સમજાવે છે
\item
  \textbf{સમીક્ષકો}: ખામીઓ અને મુદ્દાઓ શોધે છે
\item
  \textbf{રેકોર્ડર}: તારણો દસ્તાવેજીકૃત કરે છે
\end{itemize}

\textbf{ફાયદાઓ}: ઉચ્ચ ખામી શોધ દર, જ્ઞાન શેરિંગ, સુધારેલ કોડ ગુણવત્તા

\end{solutionbox}
\begin{mnemonicbox}
``CWIP'' - Code Walkthrough, Inspection, Peer review

\end{mnemonicbox}
\subsection*{પ્રશ્ન 5(બ) [4
ગુણ]}\label{uxaaauxab0uxab6uxaa8-5uxaac-4-uxa97uxaa3}

\textbf{ઓનલાઇન શોપિંગ સિસ્ટમ માટે ટેસ્ટ કેસ તૈયાર કરો}

\begin{solutionbox}

\textbf{ઓનલાઇન શોપિંગ સિસ્ટમ માટે ટેસ્ટ કેસ:}

{\def\LTcaptype{none} % do not increment counter
\begin{longtable}[]{@{}
  >{\raggedright\arraybackslash}p{(\linewidth - 6\tabcolsep) * \real{0.2653}}
  >{\raggedright\arraybackslash}p{(\linewidth - 6\tabcolsep) * \real{0.2245}}
  >{\raggedright\arraybackslash}p{(\linewidth - 6\tabcolsep) * \real{0.2449}}
  >{\raggedright\arraybackslash}p{(\linewidth - 6\tabcolsep) * \real{0.2653}}@{}}
\toprule\noalign{}
\begin{minipage}[b]{\linewidth}\raggedright
ટેસ્ટ કેસ ID
\end{minipage} & \begin{minipage}[b]{\linewidth}\raggedright
ટેસ્ટ દૃશ્ય
\end{minipage} & \begin{minipage}[b]{\linewidth}\raggedright
ટેસ્ટ સ્ટેપ્સ
\end{minipage} & \begin{minipage}[b]{\linewidth}\raggedright
અપેક્ષિત પરિણામ
\end{minipage} \\
\midrule\noalign{}
\endhead
\bottomrule\noalign{}
\endlastfoot
\textbf{TC001} & વપરાશકર્તા નોંધણી & 1. માન્ય વિગતો દાખલ કરો2. રજિસ્ટર ક્લિક
કરો & એકાઉન્ટ સફળતાપૂર્વક બનાવ્યું \\
\textbf{TC002} & વપરાશકર્તા લોગિન & 1. વપરાશકર્તાનામ/પાસવર્ડ દાખલ કરો2.
લોગિન ક્લિક કરો & વપરાશકર્તા લોગ ઇન થયો \\
\textbf{TC003} & કાર્ટમાં ઉમેરો & 1. પ્રોડક્ટ પસંદ કરો2. કાર્ટમાં ઉમેરો ક્લિક કરો
& પ્રોડક્ટ કાર્ટમાં ઉમેર્યું \\
\textbf{TC004} & ચેકઆઉટ પ્રક્રિયા & 1. કાર્ટમાં જાઓ2. ચેકઆઉટ ક્લિક કરો3. પેમેન્ટ
વિગતો દાખલ કરો & ઓર્ડર સફળતાપૂર્વક મૂક્યો \\
\end{longtable}
}

\textbf{વિગતવાર ટેસ્ટ કેસ ઉદાહરણ:}

\textbf{ટેસ્ટ કેસ ID}: TC003 \textbf{ટેસ્ટ ટાઇટલ}: શોપિંગ કાર્ટમાં પ્રોડક્ટ ઉમેરવું
\textbf{પ્રી-કન્ડિશન}: વપરાશકર્તા લોગ ઇન છે, પ્રોડક્ટ ઉપલબ્ધ છે \textbf{ટેસ્ટ
સ્ટેપ્સ}:

\begin{enumerate}
\tightlist
\item
  પ્રોડક્ટ કેટાલોગ પર નેવિગેટ કરો
\item
  પ્રોડક્ટ પસંદ કરો
\item
  જથ્થો પસંદ કરો
\item
  ``કાર્ટમાં ઉમેરો'' બટન ક્લિક કરો
\end{enumerate}

\textbf{અપેક્ષિત પરિણામ}: સાચા જથ્થા અને કિંમત સાથે પ્રોડક્ટ કાર્ટમાં દેખાય છે
\textbf{પોસ્ટ-કન્ડિશન}: કાર્ટ કાઉન્ટ વધે છે, કુલ રકમ અપડેટ થાય છે

\end{solutionbox}
\begin{mnemonicbox}
``RAULC'' - Registration, Authentication, User cart,
Login, Checkout

\end{mnemonicbox}
\subsection*{પ્રશ્ન 5(ક) [7
ગુણ]}\label{uxaaauxab0uxab6uxaa8-5uxa95-7-uxa97uxaa3}

\textbf{વ્હાઇટ બોક્સ ટેકનિકની વ્યાખ્યા કરો. વિવિધ વ્હાઇટ બોક્સ તકનીકની સૂચિ
બનાવો. કોઈપણ બે સમજાવો}

\begin{solutionbox}

\textbf{વ્હાઇટ બોક્સ ટેસ્ટિંગ વ્યાખ્યા:} આંતરિક કોડ સ્ટ્રક્ચર, લોજિક પાથ અને અમલીકરણ
વિગતોની તપાસ કરતી ટેસ્ટિંગ તકનીક.

\textbf{વ્હાઇટ બોક્સ તકનીકો:}

{\def\LTcaptype{none} % do not increment counter
\begin{longtable}[]{@{}lll@{}}
\toprule\noalign{}
તકનીક & કવરેજ ક્રાઇટેરિયા & હેતુ \\
\midrule\noalign{}
\endhead
\bottomrule\noalign{}
\endlastfoot
\textbf{સ્ટેટમેન્ટ કવરેજ} & બધા સ્ટેટમેન્ટ એક્ઝિક્યુટ & બેસિક કોડ કવરેજ \\
\textbf{બ્રાન્ચ કવરેજ} & બધી બ્રાન્ચ લેવાય & નિર્ણય ટેસ્ટિંગ \\
\textbf{પાથ કવરેજ} & બધા પાથ એક્ઝિક્યુટ & સંપૂર્ણ ફ્લો ટેસ્ટિંગ \\
\textbf{કન્ડિશન કવરેજ} & બધી શરતો ટેસ્ટ & લોજિકલ એક્સપ્રેશન ટેસ્ટિંગ \\
\textbf{લૂપ ટેસ્ટિંગ} & બધા લૂપ વેરિએશન & પુનરાવર્તક સ્ટ્રક્ચર ટેસ્ટિંગ \\
\end{longtable}
}

\textbf{1. સ્ટેટમેન્ટ કવરેજ:} કોડમાં દરેક એક્ઝિક્યુટેબલ સ્ટેટમેન્ટ ઓછામાં ઓછું એક વાર
એક્ઝિક્યુટ થાય તેની ખાતરી કરે છે.

\textbf{સૂત્ર}: (એક્ઝિક્યુટ થયેલ સ્ટેટમેન્ટ / કુલ સ્ટેટમેન્ટ) \times 100\%

\textbf{ઉદાહરણ:}

\begin{verbatim}
if (x > 0)        // સ્ટેટમેન્ટ 1
    y = x + 1;    // સ્ટેટમેન્ટ 2
else
    y = x - 1;    // સ્ટેટમેન્ટ 3
z = y * 2;        // સ્ટેટમેન્ટ 4
\end{verbatim}

\textbf{ટેસ્ટ કેસ}: x = 5 (સ્ટેટમેન્ટ 1,2,4 કવર કરે), x = -1 (સ્ટેટમેન્ટ 1,3,4 કવર
કરે) \textbf{કવરેજ}: 100\% સ્ટેટમેન્ટ કવરેજ હાંસલ

\textbf{2. બ્રાન્ચ કવરેજ:} નિર્ણય બિંદુઓની દરેક બ્રાન્ચ (true/false) એક્ઝિક્યુટ થાય
તેની ખાતરી કરે છે.

\textbf{ઉદાહરણ:}

\begin{verbatim}
if (a > b && c > d)    // બે શરતો
    result = 1;        // True બ્રાન્ચ
else
    result = 0;        // False બ્રાન્ચ
\end{verbatim}

\textbf{ટેસ્ટ કેસ}:

\begin{itemize}
\tightlist
\item
a=5,

b=3,

c=7,

d=2 (true બ્રાન્ચ)

\item
a=1,

b=3,

c=7,

d=2 (false બ્રાન્ચ)

\end{itemize}

\textbf{ફાયદાઓ}: સ્ટેટમેન્ટ કવરેજ કરતાં ઉચ્ચ ખામી શોધ

\end{solutionbox}
\begin{mnemonicbox}
``SBPCL'' - Statement, Branch, Path, Condition, Loop

\end{mnemonicbox}
\subsection*{પ્રશ્ન 5(અ) OR [3
ગુણ]}\label{uxaaauxab0uxab6uxaa8-5uxa85-or-3-uxa97uxaa3}

\textbf{સૉફ્ટવેર ડોક્યુમેન્ટેશન સમજાવો}

\begin{solutionbox}

\textbf{સૉફ્ટવેર ડોક્યુમેન્ટેશન:} સૉફ્ટવેર સિસ્ટમ, તેની ડિઝાઇન, અમલીકરણ અને ઉપયોગનું
વર્ણન કરતી લેખિત સામગ્રી.

\textbf{ડોક્યુમેન્ટેશનના પ્રકારો:}

{\def\LTcaptype{none} % do not increment counter
\begin{longtable}[]{@{}lll@{}}
\toprule\noalign{}
પ્રકાર & હેતુ & પ્રેક્ષકો \\
\midrule\noalign{}
\endhead
\bottomrule\noalign{}
\endlastfoot
\textbf{આંતરિક ડોક્યુમેન્ટેશન} & કોડ સમજ & ડેવલપર્સ \\
\textbf{બાહ્ય ડોક્યુમેન્ટેશન} & સિસ્ટમ ઉપયોગ & વપરાશકર્તાઓ, ઓપરેટર્સ \\
\textbf{સિસ્ટમ ડોક્યુમેન્ટેશન} & ડિઝાઇન અને આર્કિટેક્ચર & જાળવણીકર્તાઓ \\
\textbf{વપરાશકર્તા ડોક્યુમેન્ટેશન} & ઓપરેશન સૂચનાઓ & અંતિમ વપરાશકર્તાઓ \\
\end{longtable}
}

\textbf{આંતરિક ડોક્યુમેન્ટેશન:}

\begin{itemize}
\tightlist
\item
  \textbf{ટિપ્પણીઓ}: કોડ લોજિક અને હેતુ સમજાવે છે
\item
  \textbf{કોડ સ્ટ્રક્ચર}: ક્લાસ અને મેથડ વર્ણનો
\item
  \textbf{ડિઝાઇન તર્ક}: શા માટે ચોક્કસ અભિગમ પસંદ કર્યો
\end{itemize}

\textbf{બાહ્ય ડોક્યુમેન્ટેશન:}

\begin{itemize}
\tightlist
\item
  \textbf{વપરાશકર્તા મેન્યુઅલ}: સ્ટેપ-બાય-સ્ટેપ ઉપયોગ સૂચનાઓ
\item
  \textbf{ઇન્સ્ટોલેશન ગાઇડ}: સેટઅપ પ્રક્રિયાઓ
\item
  \textbf{API ડોક્યુમેન્ટેશન}: ઇન્ટરફેસ સ્પેસિફિકેશન
\end{itemize}

\textbf{ફાયદાઓ}: સરળ જાળવણી, જ્ઞાન સ્થાનાંતરણ, ઘટાડેલ તાલીમ સમય

\end{solutionbox}
\begin{mnemonicbox}
``IESU'' - Internal, External, System, User
documentation

\end{mnemonicbox}
\subsection*{પ્રશ્ન 5(બ) OR [4
ગુણ]}\label{uxaaauxab0uxab6uxaa8-5uxaac-or-4-uxa97uxaa3}

\textbf{ATM સિસ્ટમ માટે 4 ટેસ્ટ કેસ બનાવો}

\begin{solutionbox}

\textbf{ATM સિસ્ટમ માટે ટેસ્ટ કેસ:}

{\def\LTcaptype{none} % do not increment counter
\begin{longtable}[]{@{}
  >{\raggedright\arraybackslash}p{(\linewidth - 6\tabcolsep) * \real{0.2653}}
  >{\raggedright\arraybackslash}p{(\linewidth - 6\tabcolsep) * \real{0.2245}}
  >{\raggedright\arraybackslash}p{(\linewidth - 6\tabcolsep) * \real{0.2449}}
  >{\raggedright\arraybackslash}p{(\linewidth - 6\tabcolsep) * \real{0.2653}}@{}}
\toprule\noalign{}
\begin{minipage}[b]{\linewidth}\raggedright
ટેસ્ટ કેસ ID
\end{minipage} & \begin{minipage}[b]{\linewidth}\raggedright
ટેસ્ટ દૃશ્ય
\end{minipage} & \begin{minipage}[b]{\linewidth}\raggedright
ટેસ્ટ સ્ટેપ્સ
\end{minipage} & \begin{minipage}[b]{\linewidth}\raggedright
અપેક્ષિત પરિણામ
\end{minipage} \\
\midrule\noalign{}
\endhead
\bottomrule\noalign{}
\endlastfoot
\textbf{TC001} & માન્ય PIN એન્ટ્રી & 1. કાર્ડ દાખલ કરો2. સાચો PIN દાખલ કરો3.
Enter દબાવો & મુખ્ય મેનુમાં પ્રવેશ મળ્યો \\
\textbf{TC002} & અમાન્ય PIN એન્ટ્રી & 1. કાર્ડ દાખલ કરો2. ખોટો PIN દાખલ
કરો3. Enter દબાવો & ``અમાન્ય PIN'' સંદેશ દેખાય છે \\
\textbf{TC003} & રોકડ ઉપાડ & 1. સફળતાપૂર્વક લોગિન કરો2. ``રોકડ ઉપાડ'' પસંદ
કરો3. રકમ દાખલ કરો4. પુષ્ટિ કરો & રોકડ આપવામાં આવી, બેલેન્સ અપડેટ થયું \\
\textbf{TC004} & અપૂરતું બેલેન્સ & 1. સફળતાપૂર્વક લોગિન કરો2. ``રોકડ ઉપાડ'' પસંદ
કરો3. બેલેન્સ કરતાં વધુ રકમ દાખલ કરો & ``અપૂરતું બેલેન્સ'' સંદેશ \\
\end{longtable}
}

\textbf{વિગતવાર ટેસ્ટ કેસ:}

\textbf{ટેસ્ટ કેસ ID}: TC003 \textbf{ટેસ્ટ વર્ણન}: પૂરતા બેલેન્સ સાથે રોકડ ઉપાડવી
\textbf{પ્રી-કન્ડિશન}: માન્ય ATM કાર્ડ, પૂરતું એકાઉન્ટ બેલેન્સ \textbf{ટેસ્ટ ડેટા}:
PIN=1234, ઉપાડની રકમ=₹1000, એકાઉન્ટ બેલેન્સ=₹5000

\textbf{પોસ્ટ-કન્ડિશન}: એકાઉન્ટ બેલેન્સ ₹1000 ઘટાડ્યું, ટ્રાન્ઝેક્શન રેકોર્ડ થયું

\end{solutionbox}
\begin{mnemonicbox}
``VPCI'' - Valid PIN, PIN error, Cash withdrawal,
Insufficient funds

\end{mnemonicbox}
\subsection*{પ્રશ્ન 5(ક) OR [7
ગુણ]}\label{uxaaauxab0uxab6uxaa8-5uxa95-or-7-uxa97uxaa3}

\textbf{બ્લેક બોક્સ ટેસ્ટિંગ પધ્ધતિની સૂચિ બનાવો. તેને ફંકશનલ ટેસ્ટિંગ કેમ કહેવાય છે તે
સમજાવો. થા કોઇ પણ બે પધ્ધતિ આકૃતિ સાથે વણવો}

\begin{solutionbox}

\textbf{બ્લેક બોક્સ ટેસ્ટિંગ પધ્ધતિઓ:}

{\def\LTcaptype{none} % do not increment counter
\begin{longtable}[]{@{}lll@{}}
\toprule\noalign{}
પધ્ધતિ & હેતુ & ઇનપુટ ફોકસ \\
\midrule\noalign{}
\endhead
\bottomrule\noalign{}
\endlastfoot
\textbf{સમકક્ષ વિભાજન} & ઇનપુટને વર્ગોમાં વહેંચવું & માન્ય/અમાન્ય વિભાજન \\
\textbf{બાઉન્ડરી વેલ્યુ એનાલિસિસ} & સીમા મૂલ્યોની ટેસ્ટ & સીમા શરતો \\
\textbf{ડિસિઝન ટેબલ ટેસ્ટિંગ} & જટિલ બિઝનેસ નિયમો & બહુવિધ ઇનપુટ સંયોજનો \\
\textbf{સ્ટેટ ટ્રાન્ઝિશન ટેસ્ટિંગ} & સ્ટેટ આધારિત સિસ્ટમ & સ્ટેટ ફેરફારો \\
\textbf{યુઝ કેસ ટેસ્ટિંગ} & કાર્યાત્મક દૃશ્યો & વપરાશકર્તા ક્રિયાપ્રતિક્રિયા \\
\textbf{એરર ગેસિંગ} & અનુભવ આધારિત ટેસ્ટિંગ & સંભવિત ભૂલ શરતો \\
\end{longtable}
}

\textbf{શા માટે ફંકશનલ ટેસ્ટિંગ કહેવાય છે?} બ્લેક બોક્સ ટેસ્ટિંગ \textbf{સિસ્ટમ શું કરે
છે} પર ધ્યાન આપે છે \textbf{તે કેવી રીતે કામ કરે છે} તેનાથી વિપરીત. તે આંતરિક કોડ
સ્ટ્રક્ચરનું જ્ઞાન વિના ઇનપુટ અને અપેક્ષિત આઉટપુટ ટેસ્ટ કરીને કાર્યાત્મક આવશ્યકતાઓને માન્ય
કરે છે.

\textbf{1. સમકક્ષ વિભાજન:}

\begin{verbatim}
ઇનપુટ રેન્જ: વય (0{-120)}

માન્ય વિભાજન:     અમાન્ય વિભાજન:
   18{-65 વર્ષ       0   0{-}17   66{-}120   120}
      |                |     |      |       |
      v                v     v      v       v
   [માન્ય]         [અમાન્ય ઇનપુટ્સ]
\end{verbatim}

\textbf{ઉદાહરણ}: જોબ એપ્લિકેશન માટે વય વેલિડેશન

\begin{itemize}
\tightlist
\item
  \textbf{માન્ય વિભાજન}: 18-65 વર્ષ
\item
  \textbf{અમાન્ય વિભાજન}: \textless0, 0-17, 66-120, \textgreater120
\item
  \textbf{ટેસ્ટ કેસ}: દરેક વિભાજનમાંથી એક (દા.ત., 25, -5, 10, 70, 130)
\end{itemize}

\textbf{2. બાઉન્ડરી વેલ્યુ એનાલિસિસ:}

\begin{verbatim}
    ઇનપુટ રેન્જ: સ્કોર (0{-100)}
    
    અમાન્ય  |  માન્ય રેન્જ  | અમાન્ય
      {-1  0  |  1    99  100 | 101}
       |  |  |   |    |   |  |  |
       v  v  v   v    v   v  v  v
    [ટેસ્ટ બાઉન્ડરી મૂલ્યો]
\end{verbatim}

\textbf{ઉદાહરણ}: વિદ્યાર્થી સ્કોર વેલિડેશન (0-100)

\begin{itemize}
\tightlist
\item
  \textbf{ટેસ્ટ મૂલ્યો}: -1, 0, 1, 50, 99, 100, 101
\item
  \textbf{ફોકસ}: સીમાની અંદર અને બહાર
\item
  \textbf{તર્ક}: મોટાભાગની ભૂલો સીમા પર થાય છે
\end{itemize}

\textbf{ફાયદાઓ:}

\begin{itemize}
\tightlist
\item
  \textbf{સ્વતંત્રતા}: પ્રોગ્રામિંગ જ્ઞાનની આવશ્યકતા નથી
\item
  \textbf{વપરાશકર્તા દૃષ્ટિકોણ}: વપરાશકર્તાના દૃષ્ટિકોણથી ટેસ્ટ
\item
  \textbf{જરૂરિયાત વેલિડેશન}: કાર્યાત્મક સ્પેસિફિકેશન ચકાસે છે
\end{itemize}

\end{solutionbox}
\begin{mnemonicbox}
``EBDSUE'' - Equivalence, Boundary, Decision, State,
Use case, Error guessing

\end{mnemonicbox}

\end{document}
