\documentclass[10pt,a4paper]{article}

% content/resources/templates/preamble.tex
\usepackage[margin=0.6in]{geometry}
\author{Milav Dabgar}
\usepackage{amsmath,amssymb,amsthm}
\usepackage{booktabs}
\usepackage{multirow}
\usepackage{xcolor}
\usepackage{tcolorbox}
\tcbuselibrary{breakable,skins}
\usepackage[colorlinks=true,linkcolor=blue]{hyperref}
\usepackage{titlesec}
\usepackage{enumitem}
\usepackage{tikz}
\usepackage{pgfplots}
\usepackage{circuitikz}
\usepackage[version=4]{mhchem}
\usepackage{longtable}
\usepackage{array}
\usepackage{float}
\usepackage{caption}
\usepackage{listings}

\lstset{
  basicstyle=\small\ttfamily,
  breaklines=true,
  breakatwhitespace=false,
  postbreak=\mbox{\textcolor{red}{$\hookrightarrow$}\space},
  float=false,
  numbers=left,
  numberstyle=\tiny\color{gray},
  numbersep=10pt,
  xleftmargin=2em,
  keywordstyle=\color{blue},
  commentstyle=\color{green!60!black},
  stringstyle=\color{purple},
  backgroundcolor=\color{gray!5},
  showstringspaces=false,
  tabsize=2,
  captionpos=b,
  keepspaces=true,
  columns=flexible
}

\pgfplotsset{compat=1.18}
\usetikzlibrary{shapes,arrows,positioning,calc,patterns,decorations.pathmorphing,decorations.markings,arrows.meta}

% Color scheme
\definecolor{headcolor}{RGB}{0,102,204}
\definecolor{keycolor}{RGB}{220,20,60}
\definecolor{solutioncolor}{RGB}{34,139,34}
\definecolor{mnemoniccolor}{RGB}{148,0,211}
\definecolor{codecolor}{RGB}{0,0,100}

% Spacing
\setlength{\parskip}{3pt}
\setlist[itemize]{nosep}
\setlist[enumerate]{nosep}

% Title formatting
\titleformat{\section}{\Large\bfseries\color{headcolor}}{\thesection}{1em}{}
\titleformat{\subsection}{\large\bfseries\color{headcolor}}{\thesubsection}{1em}{}

% Pandoc tightlist compatibility
\providecommand{\tightlist}{%
  \setlength{\itemsep}{0pt}\setlength{\parskip}{0pt}}

% Pandoc longtable compatibility
\newcounter{none}
\def\thenone{}


% content/resources/templates/english-boxes.tex
% This file is currently empty - it exists to maintain consistency with the import structure.
% Add custom environments here if needed in the future.


\begin{document}

\begin{center}
{\Huge\bfseries\color{headcolor} Subject Name Solutions}\\[5pt]
{\LARGE 4353202 -- Summer 2025}\\[3pt]
{\large Semester 1 Study Material}\\[3pt]
{\normalsize\textit{Detailed Solutions and Explanations}}
\end{center}

\vspace{10pt}

\subsection*{Question 1(a) [3 marks]}\label{q1a}

\textbf{Enlist Software Application Domain and explain Embedded
Software}

\begin{solutionbox}

\textbf{Software Application Domains:}

{\def\LTcaptype{none} % do not increment counter
\begin{longtable}[]{@{}ll@{}}
\toprule\noalign{}
Domain & Description \\
\midrule\noalign{}
\endhead
\bottomrule\noalign{}
\endlastfoot
\textbf{System Software} & Operating systems, device drivers \\
\textbf{Application Software} & Word processors, games, business apps \\
\textbf{Engineering/Scientific Software} & CAD, simulation tools \\
\textbf{Embedded Software} & Real-time control systems \\
\textbf{Web Applications} & Browser-based applications \\
\textbf{AI Software} & Machine learning, expert systems \\
\end{longtable}
}

\textbf{Embedded Software} is specialized software that runs on embedded
systems with dedicated hardware. It controls specific functions in
devices like washing machines, cars, and medical equipment.

\begin{itemize}
\tightlist
\item
  \textbf{Real-time operation}: Must respond within strict time limits
\item
  \textbf{Resource constraints}: Limited memory and processing power
\item
  \textbf{Hardware dependency}: Closely integrated with specific
  hardware
\end{itemize}

\end{solutionbox}
\begin{mnemonicbox}
``SAEEWA'' - System, Application, Engineering,
Embedded, Web, AI

\end{mnemonicbox}
\subsection*{Question 1(b) [4 marks]}\label{q1b}

\textbf{Explain Generic Framework activities and umbrella activities}

\begin{solutionbox}

\textbf{Generic Framework Activities:}

{\def\LTcaptype{none} % do not increment counter
\begin{longtable}[]{@{}ll@{}}
\toprule\noalign{}
Activity & Purpose \\
\midrule\noalign{}
\endhead
\bottomrule\noalign{}
\endlastfoot
\textbf{Communication} & Gather requirements from stakeholders \\
\textbf{Planning} & Define work plan and schedule \\
\textbf{Modeling} & Create analysis and design models \\
\textbf{Construction} & Code generation and testing \\
\textbf{Deployment} & Software delivery and support \\
\end{longtable}
}

\textbf{Umbrella Activities:}

{\def\LTcaptype{none} % do not increment counter
\begin{longtable}[]{@{}ll@{}}
\toprule\noalign{}
Activity & Purpose \\
\midrule\noalign{}
\endhead
\bottomrule\noalign{}
\endlastfoot
\textbf{Project Management} & Track progress and control \\
\textbf{Risk Management} & Identify and mitigate risks \\
\textbf{Quality Assurance} & Ensure software quality \\
\textbf{Configuration Management} & Control changes \\
\textbf{Work Product Preparation} & Document creation \\
\end{longtable}
}

\begin{itemize}
\tightlist
\item
  \textbf{Framework activities}: Core sequential activities in every
  project
\item
  \textbf{Umbrella activities}: Continuous activities throughout project
  lifecycle
\end{itemize}

\end{solutionbox}
\begin{mnemonicbox}
``CPMCD'' for Framework, ``PRQCW'' for Umbrella

\end{mnemonicbox}
\subsection*{Question 1(c) [7 marks]}\label{q1c}

\textbf{Recreate the software development life cycle diagram and explain
it's phases}

\begin{solutionbox}

\textbf{SDLC Diagram:}

\includegraphics[width=1\linewidth,height=\textheight,keepaspectratio]{mermaid-f9ffbda1.pdf}

\textbf{SDLC Phases:}

{\def\LTcaptype{none} % do not increment counter
\begin{longtable}[]{@{}
  >{\raggedright\arraybackslash}p{(\linewidth - 4\tabcolsep) * \real{0.2121}}
  >{\raggedright\arraybackslash}p{(\linewidth - 4\tabcolsep) * \real{0.3636}}
  >{\raggedright\arraybackslash}p{(\linewidth - 4\tabcolsep) * \real{0.4242}}@{}}
\toprule\noalign{}
\begin{minipage}[b]{\linewidth}\raggedright
Phase
\end{minipage} & \begin{minipage}[b]{\linewidth}\raggedright
Activities
\end{minipage} & \begin{minipage}[b]{\linewidth}\raggedright
Deliverables
\end{minipage} \\
\midrule\noalign{}
\endhead
\bottomrule\noalign{}
\endlastfoot
\textbf{Requirements Analysis} & Gather user needs, create SRS & SRS
Document \\
\textbf{System Design} & Architecture design, UI design & Design
Document \\
\textbf{Implementation} & Code development, unit testing & Source
Code \\
\textbf{Testing} & Integration, system testing & Test Reports \\
\textbf{Deployment} & Installation, user training & Deployed System \\
\textbf{Maintenance} & Bug fixes, enhancements & Updated System \\
\end{longtable}
}

\begin{itemize}
\tightlist
\item
  \textbf{Systematic approach}: Each phase has specific inputs and
  outputs
\item
  \textbf{Quality gates}: Reviews between phases ensure quality
\item
  \textbf{Iterative nature}: Feedback improves subsequent cycles
\end{itemize}

\end{solutionbox}
\begin{mnemonicbox}
``Real Systems Implement Tests During Maintenance''

\end{mnemonicbox}
\subsection*{Question 1(c) OR [7
marks]}\label{q1c}

\textbf{List software development models and explain any two models with
necessary diagrams}

\begin{solutionbox}

\textbf{Software Development Models:}

{\def\LTcaptype{none} % do not increment counter
\begin{longtable}[]{@{}ll@{}}
\toprule\noalign{}
Model & Characteristics \\
\midrule\noalign{}
\endhead
\bottomrule\noalign{}
\endlastfoot
\textbf{Waterfall Model} & Sequential, linear approach \\
\textbf{Iterative Model} & Repeated cycles of development \\
\textbf{Spiral Model} & Risk-driven, iterative \\
\textbf{Agile Model} & Flexible, customer collaboration \\
\textbf{RAD Model} & Rapid prototyping \\
\textbf{V-Model} & Verification and validation focus \\
\end{longtable}
}

\textbf{1. Waterfall Model:}

\includegraphics[width=1\linewidth,height=\textheight,keepaspectratio]{mermaid-dd03b038.pdf}

\textbf{2. Spiral Model:}

\includegraphics[width=1\linewidth,height=\textheight,keepaspectratio]{mermaid-176a4fcd.pdf}

\begin{itemize}
\tightlist
\item
  \textbf{Waterfall}: Simple, suitable for well-understood requirements
\item
  \textbf{Spiral}: Handles high-risk projects with iterative risk
  assessment
\end{itemize}

\end{solutionbox}
\begin{mnemonicbox}
``WIRRAV'' - Waterfall, Iterative, RAD, Risk-driven,
Agile, V-model

\end{mnemonicbox}
\subsection*{Question 2(a) [3 marks]}\label{q2a}

\textbf{Differentiate SCRUM Agile process model with SPIRAL process
model}

\begin{solutionbox}

{\def\LTcaptype{none} % do not increment counter
\begin{longtable}[]{@{}lll@{}}
\toprule\noalign{}
Aspect & SCRUM & SPIRAL \\
\midrule\noalign{}
\endhead
\bottomrule\noalign{}
\endlastfoot
\textbf{Approach} & Agile, iterative & Risk-driven, iterative \\
\textbf{Duration} & Fixed sprints (2-4 weeks) & Variable spiral
cycles \\
\textbf{Focus} & Customer collaboration & Risk management \\
\textbf{Planning} & Sprint planning & Comprehensive planning \\
\textbf{Documentation} & Minimal documentation & Detailed
documentation \\
\textbf{Team Size} & Small teams (5-9 members) & Any team size \\
\end{longtable}
}

\begin{itemize}
\tightlist
\item
  \textbf{SCRUM}: Emphasizes rapid delivery and customer feedback
\item
  \textbf{SPIRAL}: Focuses on risk identification and mitigation
\end{itemize}

\end{solutionbox}
\begin{mnemonicbox}
``SCRUM=Speed, SPIRAL=Safety''

\end{mnemonicbox}
\subsection*{Question 2(b) [4 marks]}\label{q2b}

\textbf{List requirement gathering techniques and explain anyone}

\begin{solutionbox}

\textbf{Requirement Gathering Techniques:}

{\def\LTcaptype{none} % do not increment counter
\begin{longtable}[]{@{}ll@{}}
\toprule\noalign{}
Technique & Description \\
\midrule\noalign{}
\endhead
\bottomrule\noalign{}
\endlastfoot
\textbf{Interviews} & Direct conversation with stakeholders \\
\textbf{Questionnaires} & Structured written questions \\
\textbf{Observation} & Watch users perform tasks \\
\textbf{Document Analysis} & Review existing documents \\
\textbf{Prototyping} & Build working models \\
\textbf{Brainstorming} & Group idea generation \\
\end{longtable}
}

\textbf{Interview Technique Explained:}

\begin{itemize}
\tightlist
\item
  \textbf{Structured interviews}: Predetermined questions, formal
  approach
\item
  \textbf{Unstructured interviews}: Open-ended discussion, flexible
\item
  \textbf{Semi-structured}: Combination of both approaches
\end{itemize}

\textbf{Benefits}: Direct stakeholder input, clarification possible,
detailed information \textbf{Challenges}: Time-consuming, interviewer
bias, incomplete information

\end{solutionbox}
\begin{mnemonicbox}
``IQDPBB'' - Interview, Questionnaire, Document,
Prototype, Brainstorm, Observe

\end{mnemonicbox}
\subsection*{Question 2(c) [7 marks]}\label{q2c}

\textbf{Define use case diagram. Explain it with example}

\begin{solutionbox}

\textbf{Use Case Diagram Definition:} A use case diagram shows the
functional requirements of a system by depicting actors and their
interactions with use cases.

\textbf{Components:}

{\def\LTcaptype{none} % do not increment counter
\begin{longtable}[]{@{}lll@{}}
\toprule\noalign{}
Component & Symbol & Purpose \\
\midrule\noalign{}
\endhead
\bottomrule\noalign{}
\endlastfoot
\textbf{Actor} & Stick figure & External entity \\
\textbf{Use Case} & Oval & System function \\
\textbf{Association} & Line & Actor-use case relationship \\
\textbf{System Boundary} & Rectangle & System scope \\
\end{longtable}
}

\textbf{Example: Library Management System}

\includegraphics[width=1\linewidth,height=\textheight,keepaspectratio]{mermaid-d1ad5455.pdf}

\textbf{Relationships:}

\begin{itemize}
\tightlist
\item
  \textbf{Include}: Common functionality shared by use cases
\item
  \textbf{Extend}: Optional functionality added to base use case
\item
  \textbf{Generalization}: Inheritance between actors or use cases
\end{itemize}

\textbf{Benefits}: Clear functional overview, communication tool, basis
for testing

\end{solutionbox}
\begin{mnemonicbox}
``Actors Use Cases Inside Systems''

\end{mnemonicbox}
\subsection*{Question 2(a) OR [3
marks]}\label{q2a}

\textbf{Compare Water fall model and Iterative waterfall model}

\begin{solutionbox}

{\def\LTcaptype{none} % do not increment counter
\begin{longtable}[]{@{}
  >{\raggedright\arraybackslash}p{(\linewidth - 4\tabcolsep) * \real{0.1860}}
  >{\raggedright\arraybackslash}p{(\linewidth - 4\tabcolsep) * \real{0.3721}}
  >{\raggedright\arraybackslash}p{(\linewidth - 4\tabcolsep) * \real{0.4419}}@{}}
\toprule\noalign{}
\begin{minipage}[b]{\linewidth}\raggedright
Aspect
\end{minipage} & \begin{minipage}[b]{\linewidth}\raggedright
Waterfall Model
\end{minipage} & \begin{minipage}[b]{\linewidth}\raggedright
Iterative Waterfall
\end{minipage} \\
\midrule\noalign{}
\endhead
\bottomrule\noalign{}
\endlastfoot
\textbf{Phases} & Sequential, one-time & Repeated in iterations \\
\textbf{Feedback} & At end of project & After each iteration \\
\textbf{Risk} & High risk detection late & Early risk identification \\
\textbf{Flexibility} & Rigid, no changes & Accommodates changes \\
\textbf{Testing} & After development & Continuous testing \\
\textbf{Delivery} & Single final delivery & Multiple incremental
deliveries \\
\end{longtable}
}

\begin{itemize}
\tightlist
\item
  \textbf{Waterfall}: Suitable for stable, well-defined requirements
\item
  \textbf{Iterative Waterfall}: Better for evolving requirements with
  feedback
\end{itemize}

\end{solutionbox}
\begin{mnemonicbox}
``PFRTFD'' - Phases, Feedback, Risk, Testing,
Flexibility, Delivery

\end{mnemonicbox}
\subsection*{Question 2(b) OR [4
marks]}\label{q2b}

\textbf{Define Functional and non-Functional Requirement and give
examples of both}

\begin{solutionbox}

\textbf{Functional Requirements:} Requirements that define what the
system should do - specific behaviors and functions.

\textbf{Non-Functional Requirements:} Requirements that define how the
system should perform - quality attributes and constraints.

{\def\LTcaptype{none} % do not increment counter
\begin{longtable}[]{@{}lll@{}}
\toprule\noalign{}
Type & Functional & Non-Functional \\
\midrule\noalign{}
\endhead
\bottomrule\noalign{}
\endlastfoot
\textbf{Definition} & System behavior & System quality \\
\textbf{Examples} & Login, Calculate, Store & Performance, Security \\
\textbf{Testing} & Black-box testing & Load, stress testing \\
\textbf{Documentation} & Use cases, scenarios & Quality metrics \\
\end{longtable}
}

\textbf{Functional Examples:}

\begin{itemize}
\tightlist
\item
  User authentication and login
\item
  Calculate total bill amount
\item
  Generate monthly reports
\end{itemize}

\textbf{Non-Functional Examples:}

\begin{itemize}
\tightlist
\item
  System response time \textless{} 2 seconds (Performance)
\item
  99.9\% system availability (Reliability)\\
\item
  Support 1000 concurrent users (Scalability)
\end{itemize}

\end{solutionbox}
\begin{mnemonicbox}
``Functional=What, Non-Functional=How''

\end{mnemonicbox}
\subsection*{Question 2(c) OR [7
marks]}\label{q2c}

\textbf{Define cohesion. Explain classification of cohesion}

\begin{solutionbox}

\textbf{Cohesion Definition:} Cohesion measures how closely related
elements within a module are. High cohesion indicates a well-designed
module.

\textbf{Classification of Cohesion (Strongest to Weakest):}

{\def\LTcaptype{none} % do not increment counter
\begin{longtable}[]{@{}
  >{\raggedright\arraybackslash}p{(\linewidth - 4\tabcolsep) * \real{0.2143}}
  >{\raggedright\arraybackslash}p{(\linewidth - 4\tabcolsep) * \real{0.4643}}
  >{\raggedright\arraybackslash}p{(\linewidth - 4\tabcolsep) * \real{0.3214}}@{}}
\toprule\noalign{}
\begin{minipage}[b]{\linewidth}\raggedright
Type
\end{minipage} & \begin{minipage}[b]{\linewidth}\raggedright
Description
\end{minipage} & \begin{minipage}[b]{\linewidth}\raggedright
Example
\end{minipage} \\
\midrule\noalign{}
\endhead
\bottomrule\noalign{}
\endlastfoot
\textbf{Functional} & Single, well-defined task & Calculate square
root \\
\textbf{Sequential} & Output of one = input of next &
Read\rightarrowProcess\rightarrowWrite \\
\textbf{Communicational} & Operate on same data & Update customer
record \\
\textbf{Procedural} & Follow sequence of execution & Process payroll
steps \\
\textbf{Temporal} & Execute at same time & System initialization \\
\textbf{Logical} & Similar logical functions & All input/output
operations \\
\textbf{Coincidental} & No meaningful relationship & Random utilities \\
\end{longtable}
}

\includegraphics[width=1\linewidth,height=\textheight,keepaspectratio]{mermaid-17214699.pdf}

\textbf{Goal}: Achieve functional cohesion for maintainable, reliable
modules

\end{solutionbox}
\begin{mnemonicbox}
``Frank's Smart Cat Plays Tennis Like Crazy''

\end{mnemonicbox}
\subsection*{Question 3(a) [3 marks]}\label{q3a}

\textbf{List characteristics of good software design}

\begin{solutionbox}

\textbf{Characteristics of Good Software Design:}

{\def\LTcaptype{none} % do not increment counter
\begin{longtable}[]{@{}ll@{}}
\toprule\noalign{}
Characteristic & Description \\
\midrule\noalign{}
\endhead
\bottomrule\noalign{}
\endlastfoot
\textbf{Modularity} & Divided into independent modules \\
\textbf{Abstraction} & Hide implementation details \\
\textbf{Encapsulation} & Bundle data and methods together \\
\textbf{Hierarchy} & Organized in layers/levels \\
\textbf{Simplicity} & Easy to understand and maintain \\
\textbf{Flexibility} & Accommodate future changes \\
\end{longtable}
}

\begin{itemize}
\tightlist
\item
  \textbf{High cohesion}: Related elements grouped together
\item
  \textbf{Low coupling}: Minimal dependencies between modules
\item
  \textbf{Reusability}: Components can be reused in other systems
\end{itemize}

\end{solutionbox}
\begin{mnemonicbox}
``MAEHSF'' - Modularity, Abstraction, Encapsulation,
Hierarchy, Simplicity, Flexibility

\end{mnemonicbox}
\subsection*{Question 3(b) [4 marks]}\label{q3b}

\textbf{Explain Project Estimation Techniques using intermediate COCOMO
model}

\begin{solutionbox}

\textbf{Intermediate COCOMO Model:} Extends basic COCOMO by considering
cost drivers that affect productivity.

\textbf{Formula:} Effort = a \times (KLOC)\^{}b \times EAF

\textbf{Cost Drivers:}

{\def\LTcaptype{none} % do not increment counter
\begin{longtable}[]{@{}lll@{}}
\toprule\noalign{}
Category & Drivers & Impact \\
\midrule\noalign{}
\endhead
\bottomrule\noalign{}
\endlastfoot
\textbf{Product} & Reliability, Complexity & Effort multiplier \\
\textbf{Hardware} & Execution time, Storage & Performance constraints \\
\textbf{Personnel} & Analyst capability, Experience & Team skills \\
\textbf{Project} & Modern practices, Schedule & Development
environment \\
\end{longtable}
}

\textbf{Effort Adjustment Factor (EAF):} EAF = Product of all cost
driver multipliers

\textbf{Steps:}

\begin{enumerate}
\tightlist
\item
  Estimate KLOC (thousands of lines of code)
\item
  Select appropriate a, b values based on project type
\item
  Evaluate cost drivers (scale 0.70 to 1.65)
\item
  Calculate EAF
\item
  Apply formula to get effort in person-months
\end{enumerate}

\end{solutionbox}
\begin{mnemonicbox}
``PHPP'' - Product, Hardware, Personnel, Project
drivers

\end{mnemonicbox}
\subsection*{Question 3(c) [7 marks]}\label{q3c}

\textbf{Draw and explain level-1 Data flow diagram for Online shopping
system}

\begin{solutionbox}

\textbf{Level-1 DFD for Online Shopping System:}

\begin{lstlisting}
    +----------+                    +----------+
    |          |     Order Info     |          |
    |Customer  |<------------------>| Process  |
    |          |     Product Info   | Order    |
    +----------+                    +----------+
                                           |
                                           | Order Details
                                           v
    +----------+     Payment Info   +----------+
    |Payment   |<------------------>| Process  |
    |Gateway   |                    | Payment  |
    +----------+                    +----------+
                                           |
                                           | Inventory Update
                                           v
    +----------+     Stock Info     +----------+
    |Inventory |<------------------>| Manage   |
    |Manager   |                    |Inventory |
    +----------+                    +----------+
\end{lstlisting}

\textbf{Processes:}

{\def\LTcaptype{none} % do not increment counter
\begin{longtable}[]{@{}
  >{\raggedright\arraybackslash}p{(\linewidth - 6\tabcolsep) * \real{0.2432}}
  >{\raggedright\arraybackslash}p{(\linewidth - 6\tabcolsep) * \real{0.1892}}
  >{\raggedright\arraybackslash}p{(\linewidth - 6\tabcolsep) * \real{0.2162}}
  >{\raggedright\arraybackslash}p{(\linewidth - 6\tabcolsep) * \real{0.3514}}@{}}
\toprule\noalign{}
\begin{minipage}[b]{\linewidth}\raggedright
Process
\end{minipage} & \begin{minipage}[b]{\linewidth}\raggedright
Input
\end{minipage} & \begin{minipage}[b]{\linewidth}\raggedright
Output
\end{minipage} & \begin{minipage}[b]{\linewidth}\raggedright
Description
\end{minipage} \\
\midrule\noalign{}
\endhead
\bottomrule\noalign{}
\endlastfoot
\textbf{Process Order} & Customer order & Order confirmation & Handle
order placement \\
\textbf{Process Payment} & Payment details & Payment status & Process
transactions \\
\textbf{Manage Inventory} & Stock queries & Stock status & Track product
availability \\
\end{longtable}
}

\textbf{Data Stores:}

\begin{itemize}
\tightlist
\item
  \textbf{Product Database}: Store product information
\item
  \textbf{Order Database}: Store order details
\item
  \textbf{Customer Database}: Store customer profiles
\end{itemize}

\textbf{External Entities:}

\begin{itemize}
\tightlist
\item
  \textbf{Customer}: Places orders, makes payments
\item
  \textbf{Payment Gateway}: Processes payments
\item
  \textbf{Inventory Manager}: Updates stock levels
\end{itemize}

\end{solutionbox}
\begin{mnemonicbox}
``PPMI'' - Process order, Process payment, Manage
inventory

\end{mnemonicbox}
\subsection*{Question 3(a) OR [3
marks]}\label{q3a}

\textbf{Differentiate analysis and design}

\begin{solutionbox}

{\def\LTcaptype{none} % do not increment counter
\begin{longtable}[]{@{}lll@{}}
\toprule\noalign{}
Aspect & Analysis & Design \\
\midrule\noalign{}
\endhead
\bottomrule\noalign{}
\endlastfoot
\textbf{Focus} & What system should do & How system will work \\
\textbf{Phase} & Requirements phase & Design phase \\
\textbf{Output} & Problem understanding & Solution structure \\
\textbf{Models} & Use cases, requirements & Architecture, classes \\
\textbf{Perspective} & User's viewpoint & Developer's viewpoint \\
\textbf{Level} & Abstract, conceptual & Concrete, detailed \\
\end{longtable}
}

\begin{itemize}
\tightlist
\item
  \textbf{Analysis}: Problem-focused, understanding requirements
\item
  \textbf{Design}: Solution-focused, creating system architecture
\end{itemize}

\end{solutionbox}
\begin{mnemonicbox}
``Analysis=WHAT, Design=HOW''

\end{mnemonicbox}
\subsection*{Question 3(b) OR [4
marks]}\label{q3b}

\textbf{Explain Project Estimation Techniques using basic COCOMO model}

\begin{solutionbox}

\textbf{Basic COCOMO Model:} Estimates software development effort based
on lines of code.

\textbf{Formula:}

\begin{itemize}
\tightlist
\item
  Effort = a \times (KLOC)\^{}b person-months
\item
  Time = c \times (Effort)\^{}d months
\end{itemize}

\textbf{Project Types:}

{\def\LTcaptype{none} % do not increment counter
\begin{longtable}[]{@{}
  >{\raggedright\arraybackslash}p{(\linewidth - 10\tabcolsep) * \real{0.1935}}
  >{\raggedright\arraybackslash}p{(\linewidth - 10\tabcolsep) * \real{0.0968}}
  >{\raggedright\arraybackslash}p{(\linewidth - 10\tabcolsep) * \real{0.0968}}
  >{\raggedright\arraybackslash}p{(\linewidth - 10\tabcolsep) * \real{0.0968}}
  >{\raggedright\arraybackslash}p{(\linewidth - 10\tabcolsep) * \real{0.0968}}
  >{\raggedright\arraybackslash}p{(\linewidth - 10\tabcolsep) * \real{0.4194}}@{}}
\toprule\noalign{}
\begin{minipage}[b]{\linewidth}\raggedright
Type
\end{minipage} & \begin{minipage}[b]{\linewidth}\raggedright
a
\end{minipage} & \begin{minipage}[b]{\linewidth}\raggedright
b
\end{minipage} & \begin{minipage}[b]{\linewidth}\raggedright
c
\end{minipage} & \begin{minipage}[b]{\linewidth}\raggedright
d
\end{minipage} & \begin{minipage}[b]{\linewidth}\raggedright
Description
\end{minipage} \\
\midrule\noalign{}
\endhead
\bottomrule\noalign{}
\endlastfoot
\textbf{Organic} & 2.4 & 1.05 & 2.5 & 0.38 & Small, experienced team \\
\textbf{Semi-detached} & 3.0 & 1.12 & 2.5 & 0.35 & Medium size, mixed
team \\
\textbf{Embedded} & 3.6 & 1.20 & 2.5 & 0.32 & Complex, tight
constraints \\
\end{longtable}
}

\textbf{Steps:}

\begin{enumerate}
\tightlist
\item
  Estimate KLOC (thousands of lines of code)
\item
  Identify project type (organic/semi-detached/embedded)
\item
  Apply appropriate coefficients
\item
  Calculate effort and development time
\end{enumerate}

\textbf{Example}: 10 KLOC organic project

\begin{itemize}
\tightlist
\item
  Effort = 2.4 \times (10)\^{}1.05 = 25.2 person-months
\item
  Time = 2.5 \times (25.2)\^{}0.38 = 8.4 months
\end{itemize}

\end{solutionbox}
\begin{mnemonicbox}
``OSE'' - Organic, Semi-detached, Embedded

\end{mnemonicbox}
\subsection*{Question 3(c) OR [7
marks]}\label{q3c}

\textbf{Draw and explain Class Diagram for Library Management system}

\begin{solutionbox}

\textbf{Class Diagram for Library Management System:}

\includegraphics[width=1\linewidth,height=\textheight,keepaspectratio]{mermaid-ff12c097.pdf}

\textbf{Relationships:}

{\def\LTcaptype{none} % do not increment counter
\begin{longtable}[]{@{}lll@{}}
\toprule\noalign{}
Relationship & Description & Multiplicity \\
\midrule\noalign{}
\endhead
\bottomrule\noalign{}
\endlastfoot
\textbf{Library-Book} & Library contains books & 1 to many \\
\textbf{Member-Transaction} & Member has transactions & 1 to many \\
\textbf{Book-Transaction} & Book involved in transactions & 1 to many \\
\end{longtable}
}

\textbf{Key Features:}

\begin{itemize}
\tightlist
\item
  \textbf{Attributes}: Data members of each class
\item
  \textbf{Methods}: Functions that operate on class data
\item
  \textbf{Associations}: Relationships between classes showing how they
  interact
\end{itemize}

\end{solutionbox}
\begin{mnemonicbox}
``LBMT'' - Library, Book, Member, Transaction

\end{mnemonicbox}
\subsection*{Question 4(a) [3 marks]}\label{q4a}

\textbf{List Project Size Estimation Metrics and define them}

\begin{solutionbox}

\textbf{Project Size Estimation Metrics:}

{\def\LTcaptype{none} % do not increment counter
\begin{longtable}[]{@{}
  >{\raggedright\arraybackslash}p{(\linewidth - 4\tabcolsep) * \real{0.2963}}
  >{\raggedright\arraybackslash}p{(\linewidth - 4\tabcolsep) * \real{0.4444}}
  >{\raggedright\arraybackslash}p{(\linewidth - 4\tabcolsep) * \real{0.2593}}@{}}
\toprule\noalign{}
\begin{minipage}[b]{\linewidth}\raggedright
Metric
\end{minipage} & \begin{minipage}[b]{\linewidth}\raggedright
Definition
\end{minipage} & \begin{minipage}[b]{\linewidth}\raggedright
Usage
\end{minipage} \\
\midrule\noalign{}
\endhead
\bottomrule\noalign{}
\endlastfoot
\textbf{Lines of Code (LOC)} & Count of executable code lines &
Traditional sizing \\
\textbf{Function Points (FP)} & Measure based on functionality &
Language-independent \\
\textbf{Feature Points} & Extended function points & Real-time
systems \\
\textbf{Object Points} & Count of objects and methods & Object-oriented
systems \\
\textbf{Use Case Points} & Based on use case complexity &
Requirements-based \\
\end{longtable}
}

\textbf{Function Points Components:}

\begin{itemize}
\tightlist
\item
  \textbf{External Inputs}: Data entry screens
\item
  \textbf{External Outputs}: Reports, messages
\item
  \textbf{External Inquiries}: Interactive queries
\item
  \textbf{Internal Files}: Master files
\item
  \textbf{External Interfaces}: Shared data
\end{itemize}

\textbf{Benefits}: Early estimation, technology-independent,
standardized approach

\end{solutionbox}
\begin{mnemonicbox}
``LFFOU'' - LOC, Function Points, Feature Points,
Object Points, Use Case Points

\end{mnemonicbox}
\subsection*{Question 4(b) [4 marks]}\label{q4b}

\textbf{Explain Risk identification in detail}

\begin{solutionbox}

\textbf{Risk Identification:} Process of finding, recognizing, and
describing potential risks that could affect project success.

\textbf{Risk Categories:}

{\def\LTcaptype{none} % do not increment counter
\begin{longtable}[]{@{}lll@{}}
\toprule\noalign{}
Category & Examples & Impact \\
\midrule\noalign{}
\endhead
\bottomrule\noalign{}
\endlastfoot
\textbf{Technical} & New technology, complexity & Development delays \\
\textbf{Project} & Schedule, budget constraints & Cost overruns \\
\textbf{Business} & Market changes, competition & Project
cancellation \\
\textbf{External} & Vendor issues, regulations & Dependencies \\
\end{longtable}
}

\textbf{Identification Techniques:}

\begin{itemize}
\tightlist
\item
  \textbf{Brainstorming}: Team discussions to identify risks
\item
  \textbf{Checklists}: Standard risk categories review
\item
  \textbf{Expert judgment}: Experience-based identification
\item
  \textbf{SWOT analysis}: Strengths, Weaknesses, Opportunities, Threats
\end{itemize}

\textbf{Risk Register:} Document containing identified risks with:

\begin{itemize}
\tightlist
\item
  Risk description
\item
  Probability of occurrence
\item
  Impact severity
\item
  Risk category
\item
  Responsible person
\end{itemize}

\end{solutionbox}
\begin{mnemonicbox}
``TPBE'' - Technical, Project, Business, External
risks

\end{mnemonicbox}
\subsection*{Question 4(c) [7 marks]}\label{q4c}

\textbf{Prepare Gantt Chart for any system of your choice}

\begin{solutionbox}

\textbf{Gantt Chart for Online Banking System:}

{\def\LTcaptype{none} % do not increment counter
\begin{longtable}[]{@{}
  >{\raggedright\arraybackslash}p{(\linewidth - 16\tabcolsep) * \real{0.0857}}
  >{\raggedright\arraybackslash}p{(\linewidth - 16\tabcolsep) * \real{0.1143}}
  >{\raggedright\arraybackslash}p{(\linewidth - 16\tabcolsep) * \real{0.1143}}
  >{\raggedright\arraybackslash}p{(\linewidth - 16\tabcolsep) * \real{0.1143}}
  >{\raggedright\arraybackslash}p{(\linewidth - 16\tabcolsep) * \real{0.1143}}
  >{\raggedright\arraybackslash}p{(\linewidth - 16\tabcolsep) * \real{0.1143}}
  >{\raggedright\arraybackslash}p{(\linewidth - 16\tabcolsep) * \real{0.1143}}
  >{\raggedright\arraybackslash}p{(\linewidth - 16\tabcolsep) * \real{0.1143}}
  >{\raggedright\arraybackslash}p{(\linewidth - 16\tabcolsep) * \real{0.1143}}@{}}
\toprule\noalign{}
\begin{minipage}[b]{\linewidth}\raggedright
Task
\end{minipage} & \begin{minipage}[b]{\linewidth}\raggedright
Week 1
\end{minipage} & \begin{minipage}[b]{\linewidth}\raggedright
Week 2
\end{minipage} & \begin{minipage}[b]{\linewidth}\raggedright
Week 3
\end{minipage} & \begin{minipage}[b]{\linewidth}\raggedright
Week 4
\end{minipage} & \begin{minipage}[b]{\linewidth}\raggedright
Week 5
\end{minipage} & \begin{minipage}[b]{\linewidth}\raggedright
Week 6
\end{minipage} & \begin{minipage}[b]{\linewidth}\raggedright
Week 7
\end{minipage} & \begin{minipage}[b]{\linewidth}\raggedright
Week 8
\end{minipage} \\
\midrule\noalign{}
\endhead
\bottomrule\noalign{}
\endlastfoot
\textbf{Requirements Analysis} & ████████ & ████████ & & & & & & \\
\textbf{System Design} & & ████████ & ████████ & & & & & \\
\textbf{Database Design} & & & ████████ & ████████ & & & & \\
\textbf{UI Development} & & & & ████████ & ████████ & & & \\
\textbf{Backend Development} & & & & & ████████ & ████████ & & \\
\textbf{Testing} & & & & & & ████████ & ████████ & \\
\textbf{Deployment} & & & & & & & ████████ & ████████ \\
\end{longtable}
}

\textbf{Project Tasks:}

{\def\LTcaptype{none} % do not increment counter
\begin{longtable}[]{@{}
  >{\raggedright\arraybackslash}p{(\linewidth - 6\tabcolsep) * \real{0.1463}}
  >{\raggedright\arraybackslash}p{(\linewidth - 6\tabcolsep) * \real{0.2439}}
  >{\raggedright\arraybackslash}p{(\linewidth - 6\tabcolsep) * \real{0.3415}}
  >{\raggedright\arraybackslash}p{(\linewidth - 6\tabcolsep) * \real{0.2683}}@{}}
\toprule\noalign{}
\begin{minipage}[b]{\linewidth}\raggedright
Task
\end{minipage} & \begin{minipage}[b]{\linewidth}\raggedright
Duration
\end{minipage} & \begin{minipage}[b]{\linewidth}\raggedright
Dependencies
\end{minipage} & \begin{minipage}[b]{\linewidth}\raggedright
Resources
\end{minipage} \\
\midrule\noalign{}
\endhead
\bottomrule\noalign{}
\endlastfoot
\textbf{Requirements Analysis} & 2 weeks & None & Business Analyst \\
\textbf{System Design} & 2 weeks & Requirements & System Designer \\
\textbf{Database Design} & 2 weeks & System Design & Database
Designer \\
\textbf{UI Development} & 2 weeks & System Design & UI Developer \\
\textbf{Backend Development} & 2 weeks & Database Design & Backend
Developer \\
\textbf{Testing} & 2 weeks & UI + Backend & QA Tester \\
\textbf{Deployment} & 2 weeks & Testing & DevOps Engineer \\
\end{longtable}
}

\textbf{Benefits}: Visual progress tracking, resource allocation,
dependency management

\end{solutionbox}
\begin{mnemonicbox}
``RSDUBtd'' - Requirements, System design, Database,
UI, Backend, Testing, Deployment

\end{mnemonicbox}
\subsection*{Question 4(a) OR [3
marks]}\label{q4a}

\textbf{List Responsibilities of Project manager}

\begin{solutionbox}

\textbf{Project Manager Responsibilities:}

{\def\LTcaptype{none} % do not increment counter
\begin{longtable}[]{@{}ll@{}}
\toprule\noalign{}
Area & Responsibilities \\
\midrule\noalign{}
\endhead
\bottomrule\noalign{}
\endlastfoot
\textbf{Planning} & Create project plans, define scope \\
\textbf{Organizing} & Allocate resources, form teams \\
\textbf{Leading} & Motivate team, resolve conflicts \\
\textbf{Controlling} & Monitor progress, manage changes \\
\textbf{Communication} & Stakeholder updates, team coordination \\
\textbf{Risk Management} & Identify and mitigate risks \\
\end{longtable}
}

\textbf{Key Activities:}

\begin{itemize}
\tightlist
\item
  \textbf{Project initiation}: Define objectives and constraints
\item
  \textbf{Schedule management}: Create and maintain timelines
\item
  \textbf{Budget control}: Monitor costs and expenses
\item
  \textbf{Quality assurance}: Ensure deliverable standards
\item
  \textbf{Team management}: Lead and develop team members
\end{itemize}

\end{solutionbox}
\begin{mnemonicbox}
``POLCR'' - Planning, Organizing, Leading,
Controlling, Risk management

\end{mnemonicbox}
\subsection*{Question 4(b) OR [4
marks]}\label{q4b}

\textbf{Explain Risk Assessment in detail}

\begin{solutionbox}

\textbf{Risk Assessment:} Process of evaluating identified risks to
determine their probability and impact on project success.

\textbf{Assessment Components:}

{\def\LTcaptype{none} % do not increment counter
\begin{longtable}[]{@{}lll@{}}
\toprule\noalign{}
Component & Scale & Description \\
\midrule\noalign{}
\endhead
\bottomrule\noalign{}
\endlastfoot
\textbf{Probability} & 1-5 or \% & Likelihood of risk occurrence \\
\textbf{Impact} & 1-5 or \$ & Severity if risk occurs \\
\textbf{Risk Score} & P \times I & Overall risk priority \\
\end{longtable}
}

\textbf{Risk Assessment Matrix:}

{\def\LTcaptype{none} % do not increment counter
\begin{longtable}[]{@{}llll@{}}
\toprule\noalign{}
Probability/Impact & Low (1) & Medium (2) & High (3) \\
\midrule\noalign{}
\endhead
\bottomrule\noalign{}
\endlastfoot
\textbf{Low (1)} & 1 & 2 & 3 \\
\textbf{Medium (2)} & 2 & 4 & 6 \\
\textbf{High (3)} & 3 & 6 & 9 \\
\end{longtable}
}

\textbf{Assessment Techniques:}

\begin{itemize}
\tightlist
\item
  \textbf{Qualitative assessment}: Descriptive scales (High/Medium/Low)
\item
  \textbf{Quantitative assessment}: Numerical values and calculations
\item
  \textbf{Expert judgment}: Experience-based evaluation
\item
  \textbf{Historical data}: Past project analysis
\end{itemize}

\textbf{Risk Categorization:}

\begin{itemize}
\tightlist
\item
  \textbf{High risk} (7-9): Immediate attention required
\item
  \textbf{Medium risk} (4-6): Monitor and plan mitigation
\item
  \textbf{Low risk} (1-3): Accept or minimal mitigation
\end{itemize}

\end{solutionbox}
\begin{mnemonicbox}
``PIS'' - Probability, Impact, Score

\end{mnemonicbox}
\subsection*{Question 4(c) OR [7
marks]}\label{q4c}

\textbf{Prepare Sprint burn down chart for any system of your choice}

\begin{solutionbox}

\textbf{Sprint Burn Down Chart for E-commerce Mobile App (2-week
Sprint):}

\begin{lstlisting}
Story Points
    |
 40 +---*
    |    \
 35 +     *
    |      \
 30 +       *
    |        \
 25 +         *---*
    |              \
 20 +               *
    |                \
 15 +                 *
    |                  \
 10 +                   *
    |                    \
  5 +                     *
    |                      \
  0 +________________________*
    1  2  3  4  5  6  7  8  9  10 Days
    
    * = Actual Progress
    --- = Ideal Progress
\end{lstlisting}

\textbf{Sprint Details:}

{\def\LTcaptype{none} % do not increment counter
\begin{longtable}[]{@{}llll@{}}
\toprule\noalign{}
Day & Ideal Remaining & Actual Remaining & Work Completed \\
\midrule\noalign{}
\endhead
\bottomrule\noalign{}
\endlastfoot
\textbf{Day 1} & 36 & 40 & Sprint planning \\
\textbf{Day 2} & 32 & 35 & User login feature \\
\textbf{Day 3} & 28 & 30 & Product catalog \\
\textbf{Day 4} & 24 & 25 & Shopping cart \\
\textbf{Day 5} & 20 & 25 & Blocked by API issue \\
\textbf{Day 6} & 16 & 20 & Payment integration \\
\textbf{Day 7} & 12 & 15 & Order management \\
\textbf{Day 8} & 8 & 10 & Testing and fixes \\
\textbf{Day 9} & 4 & 5 & Final testing \\
\textbf{Day 10} & 0 & 0 & Sprint completed \\
\end{longtable}
}

\textbf{Key Insights:}

\begin{itemize}
\tightlist
\item
  \textbf{Slope}: Progress rate compared to ideal
\item
  \textbf{Flat areas}: Blocked work or scope changes
\item
  \textbf{Below ideal}: Ahead of schedule
\item
  \textbf{Above ideal}: Behind schedule
\end{itemize}

\end{solutionbox}
\begin{mnemonicbox}
``DABC'' - Days, Actual, Burn-down, Chart

\end{mnemonicbox}
\subsection*{Question 5(a) [3 marks]}\label{q5a}

\textbf{List Code Review Techniques and explain anyone}

\begin{solutionbox}

\textbf{Code Review Techniques:}

{\def\LTcaptype{none} % do not increment counter
\begin{longtable}[]{@{}
  >{\raggedright\arraybackslash}p{(\linewidth - 4\tabcolsep) * \real{0.2895}}
  >{\raggedright\arraybackslash}p{(\linewidth - 4\tabcolsep) * \real{0.3421}}
  >{\raggedright\arraybackslash}p{(\linewidth - 4\tabcolsep) * \real{0.3684}}@{}}
\toprule\noalign{}
\begin{minipage}[b]{\linewidth}\raggedright
Technique
\end{minipage} & \begin{minipage}[b]{\linewidth}\raggedright
Description
\end{minipage} & \begin{minipage}[b]{\linewidth}\raggedright
Participants
\end{minipage} \\
\midrule\noalign{}
\endhead
\bottomrule\noalign{}
\endlastfoot
\textbf{Code Walkthrough} & Informal review by author & Author +
reviewers \\
\textbf{Code Inspection} & Formal, systematic review & Trained
inspectors \\
\textbf{Peer Review} & Colleague examines code & Developer peers \\
\textbf{Tool-based Review} & Automated analysis & Tools + developers \\
\end{longtable}
}

\textbf{Code Inspection Explained:}

\textbf{Process:}

\begin{enumerate}
\tightlist
\item
  \textbf{Planning}: Select code, assign roles
\item
  \textbf{Overview}: Author explains code structure
\item
  \textbf{Preparation}: Individual review of code
\item
  \textbf{Inspection meeting}: Group examines code
\item
  \textbf{Rework}: Fix identified defects
\item
  \textbf{Follow-up}: Verify corrections
\end{enumerate}

\textbf{Roles:}

\begin{itemize}
\tightlist
\item
  \textbf{Moderator}: Leads the inspection process
\item
  \textbf{Author}: Code developer, explains logic
\item
  \textbf{Reviewers}: Find defects and issues
\item
  \textbf{Recorder}: Documents findings
\end{itemize}

\textbf{Benefits}: High defect detection rate, knowledge sharing,
improved code quality

\end{solutionbox}
\begin{mnemonicbox}
``CWIP'' - Code Walkthrough, Inspection, Peer review

\end{mnemonicbox}
\subsection*{Question 5(b) [4 marks]}\label{q5b}

\textbf{Prepare test cases for online shopping system}

\begin{solutionbox}

\textbf{Test Cases for Online Shopping System:}

{\def\LTcaptype{none} % do not increment counter
\begin{longtable}[]{@{}
  >{\raggedright\arraybackslash}p{(\linewidth - 6\tabcolsep) * \real{0.2281}}
  >{\raggedright\arraybackslash}p{(\linewidth - 6\tabcolsep) * \real{0.2632}}
  >{\raggedright\arraybackslash}p{(\linewidth - 6\tabcolsep) * \real{0.2105}}
  >{\raggedright\arraybackslash}p{(\linewidth - 6\tabcolsep) * \real{0.2982}}@{}}
\toprule\noalign{}
\begin{minipage}[b]{\linewidth}\raggedright
Test Case ID
\end{minipage} & \begin{minipage}[b]{\linewidth}\raggedright
Test Scenario
\end{minipage} & \begin{minipage}[b]{\linewidth}\raggedright
Test Steps
\end{minipage} & \begin{minipage}[b]{\linewidth}\raggedright
Expected Result
\end{minipage} \\
\midrule\noalign{}
\endhead
\bottomrule\noalign{}
\endlastfoot
\textbf{TC001} & User Registration & 1. Enter valid details2. Click
Register & Account created successfully \\
\textbf{TC002} & User Login & 1. Enter username/password2. Click Login &
User logged in \\
\textbf{TC003} & Add to Cart & 1. Select product2. Click Add to Cart &
Product added to cart \\
\textbf{TC004} & Checkout Process & 1. Go to cart2. Click Checkout3.
Enter payment details & Order placed successfully \\
\end{longtable}
}

\textbf{Detailed Test Case Example:}

\textbf{Test Case ID}: TC003 \textbf{Test Title}: Add Product to
Shopping Cart \textbf{Pre-conditions}: User is logged in, product is
available \textbf{Test Steps}:

\begin{enumerate}
\tightlist
\item
  Navigate to product catalog
\item
  Select a product
\item
  Choose quantity
\item
  Click ``Add to Cart'' button
\end{enumerate}

\textbf{Expected Result}: Product appears in cart with correct quantity
and price \textbf{Post-conditions}: Cart count increases, total amount
updates

\end{solutionbox}
\begin{mnemonicbox}
``RAULC'' - Registration, Authentication, User cart,
Login, Checkout

\end{mnemonicbox}
\subsection*{Question 5(c) [7 marks]}\label{q5c}

\textbf{Define White box technique. List various white box technique.
Explain any two}

\begin{solutionbox}

\textbf{White Box Testing Definition:} Testing technique that examines
internal code structure, logic paths, and implementation details.

\textbf{White Box Techniques:}

{\def\LTcaptype{none} % do not increment counter
\begin{longtable}[]{@{}
  >{\raggedright\arraybackslash}p{(\linewidth - 4\tabcolsep) * \real{0.2895}}
  >{\raggedright\arraybackslash}p{(\linewidth - 4\tabcolsep) * \real{0.4737}}
  >{\raggedright\arraybackslash}p{(\linewidth - 4\tabcolsep) * \real{0.2368}}@{}}
\toprule\noalign{}
\begin{minipage}[b]{\linewidth}\raggedright
Technique
\end{minipage} & \begin{minipage}[b]{\linewidth}\raggedright
Coverage Criteria
\end{minipage} & \begin{minipage}[b]{\linewidth}\raggedright
Purpose
\end{minipage} \\
\midrule\noalign{}
\endhead
\bottomrule\noalign{}
\endlastfoot
\textbf{Statement Coverage} & All statements executed & Basic code
coverage \\
\textbf{Branch Coverage} & All branches taken & Decision testing \\
\textbf{Path Coverage} & All paths executed & Complete flow testing \\
\textbf{Condition Coverage} & All conditions tested & Logical expression
testing \\
\textbf{Loop Testing} & All loop variations & Iterative structure
testing \\
\end{longtable}
}

\textbf{1. Statement Coverage:} Ensures every executable statement in
code is executed at least once.

\textbf{Formula}: (Executed statements / Total statements) \times 100\%

\textbf{Example:}

\begin{lstlisting}
if (x > 0)        // Statement 1
    y = x + 1;    // Statement 2
else
    y = x - 1;    // Statement 3
z = y * 2;        // Statement 4
\end{lstlisting}

\textbf{Test Cases}: x = 5 (covers statements 1,2,4), x = -1 (covers
statements 1,3,4) \textbf{Coverage}: 100\% statement coverage achieved

\textbf{2. Branch Coverage:} Ensures every branch (true/false) of
decision points is executed.

\textbf{Example:}

\begin{lstlisting}
if (a > b && c > d)    // Two conditions
    result = 1;        // True branch
else
    result = 0;        // False branch
\end{lstlisting}

\textbf{Test Cases}:

\begin{itemize}
\tightlist
\item
a=5,

b=3,

c=7,

d=2 (true branch)

\item
a=1,

b=3,

c=7,

d=2 (false branch)

\end{itemize}

\textbf{Benefits}: Higher defect detection than statement coverage

\end{solutionbox}
\begin{mnemonicbox}
``SBPCL'' - Statement, Branch, Path, Condition, Loop

\end{mnemonicbox}
\subsection*{Question 5(a) OR [3
marks]}\label{q5a}

\textbf{Explain software documentation}

\begin{solutionbox}

\textbf{Software Documentation:} Written material that describes
software system, its design, implementation, and usage.

\textbf{Types of Documentation:}

{\def\LTcaptype{none} % do not increment counter
\begin{longtable}[]{@{}lll@{}}
\toprule\noalign{}
Type & Purpose & Audience \\
\midrule\noalign{}
\endhead
\bottomrule\noalign{}
\endlastfoot
\textbf{Internal Documentation} & Code understanding & Developers \\
\textbf{External Documentation} & System usage & Users, operators \\
\textbf{System Documentation} & Design and architecture & Maintainers \\
\textbf{User Documentation} & Operation instructions & End users \\
\end{longtable}
}

\textbf{Internal Documentation:}

\begin{itemize}
\tightlist
\item
  \textbf{Comments}: Explain code logic and purpose\\
\item
  \textbf{Code structure}: Class and method descriptions
\item
  \textbf{Design rationale}: Why specific approaches chosen
\end{itemize}

\textbf{External Documentation:}

\begin{itemize}
\tightlist
\item
  \textbf{User manuals}: Step-by-step usage instructions
\item
  \textbf{Installation guides}: Setup procedures
\item
  \textbf{API documentation}: Interface specifications
\end{itemize}

\textbf{Benefits}: Easier maintenance, knowledge transfer, reduced
training time

\end{solutionbox}
\begin{mnemonicbox}
``IESU'' - Internal, External, System, User
documentation

\end{mnemonicbox}
\subsection*{Question 5(b) OR [4
marks]}\label{q5b}

\textbf{Prepare at least 4 test cases for ATM System}

\begin{solutionbox}

\textbf{Test Cases for ATM System:}

{\def\LTcaptype{none} % do not increment counter
\begin{longtable}[]{@{}
  >{\raggedright\arraybackslash}p{(\linewidth - 6\tabcolsep) * \real{0.2281}}
  >{\raggedright\arraybackslash}p{(\linewidth - 6\tabcolsep) * \real{0.2632}}
  >{\raggedright\arraybackslash}p{(\linewidth - 6\tabcolsep) * \real{0.2105}}
  >{\raggedright\arraybackslash}p{(\linewidth - 6\tabcolsep) * \real{0.2982}}@{}}
\toprule\noalign{}
\begin{minipage}[b]{\linewidth}\raggedright
Test Case ID
\end{minipage} & \begin{minipage}[b]{\linewidth}\raggedright
Test Scenario
\end{minipage} & \begin{minipage}[b]{\linewidth}\raggedright
Test Steps
\end{minipage} & \begin{minipage}[b]{\linewidth}\raggedright
Expected Result
\end{minipage} \\
\midrule\noalign{}
\endhead
\bottomrule\noalign{}
\endlastfoot
\textbf{TC001} & Valid PIN Entry & 1. Insert card2. Enter correct PIN3.
Press Enter & Access granted to main menu \\
\textbf{TC002} & Invalid PIN Entry & 1. Insert card2. Enter wrong PIN3.
Press Enter & ``Invalid PIN'' message displayed \\
\textbf{TC003} & Cash Withdrawal & 1. Login successfully2. Select
``Withdraw Cash''3. Enter amount4. Confirm & Cash dispensed, balance
updated \\
\textbf{TC004} & Insufficient Balance & 1. Login successfully2. Select
``Withdraw Cash''3. Enter amount \textgreater{} balance & ``Insufficient
funds'' message \\
\end{longtable}
}

\textbf{Detailed Test Case:}

\textbf{Test Case ID}: TC003 \textbf{Test Description}: Withdraw cash
with sufficient balance \textbf{Pre-conditions}: Valid ATM card,
sufficient account balance \textbf{Test Data}: PIN=1234, Withdrawal
amount=₹1000, Account balance=₹5000

\textbf{Post-conditions}: Account balance reduced by ₹1000, transaction
recorded

\end{solutionbox}
\begin{mnemonicbox}
``VPCI'' - Valid PIN, PIN error, Cash withdrawal,
Insufficient funds

\end{mnemonicbox}
\subsection*{Question 5(c) OR [7
marks]}\label{q5c}

\textbf{Enlist all black box testing methodologies. Explain why it is
known as functional testing? Explain at least 2 methods with diagram}

\begin{solutionbox}

\textbf{Black Box Testing Methodologies:}

{\def\LTcaptype{none} % do not increment counter
\begin{longtable}[]{@{}
  >{\raggedright\arraybackslash}p{(\linewidth - 4\tabcolsep) * \real{0.2667}}
  >{\raggedright\arraybackslash}p{(\linewidth - 4\tabcolsep) * \real{0.3000}}
  >{\raggedright\arraybackslash}p{(\linewidth - 4\tabcolsep) * \real{0.4333}}@{}}
\toprule\noalign{}
\begin{minipage}[b]{\linewidth}\raggedright
Method
\end{minipage} & \begin{minipage}[b]{\linewidth}\raggedright
Purpose
\end{minipage} & \begin{minipage}[b]{\linewidth}\raggedright
Input Focus
\end{minipage} \\
\midrule\noalign{}
\endhead
\bottomrule\noalign{}
\endlastfoot
\textbf{Equivalence Partitioning} & Divide inputs into classes &
Valid/invalid partitions \\
\textbf{Boundary Value Analysis} & Test edge values & Boundary
conditions \\
\textbf{Decision Table Testing} & Complex business rules & Multiple
input combinations \\
\textbf{State Transition Testing} & State-based systems & State
changes \\
\textbf{Use Case Testing} & Functional scenarios & User interactions \\
\textbf{Error Guessing} & Experience-based testing & Likely error
conditions \\
\end{longtable}
}

\textbf{Why called Functional Testing?} Black box testing focuses on
\textbf{what the system does} rather than \textbf{how it works}. It
validates functional requirements by testing inputs and expected outputs
without knowledge of internal code structure.

\textbf{1. Equivalence Partitioning:}

\begin{lstlisting}
Input Range: Age (0-120)

Valid Partition:     Invalid Partitions:
   18-65 years       <0   0-17   66-120   >120
      |                |     |      |       |
      v                v     v      v       v
   [VALID]         [INVALID INPUTS]
\end{lstlisting}

\textbf{Example}: Age validation for job application

\begin{itemize}
\tightlist
\item
  \textbf{Valid partition}: 18-65 years
\item
  \textbf{Invalid partitions}: \textless0, 0-17, 66-120, \textgreater120
\item
  \textbf{Test cases}: One from each partition (e.g., 25, -5, 10, 70,
  130)
\end{itemize}

\textbf{2. Boundary Value Analysis:}

\begin{lstlisting}
    Input Range: Score (0-100)
    
    Invalid  |  Valid Range  | Invalid
      -1  0  |  1    99  100 | 101
       |  |  |   |    |   |  |  |
       v  v  v   v    v   v  v  v
    [Test boundary values]
\end{lstlisting}

\textbf{Example}: Student score validation (0-100)

\begin{itemize}
\tightlist
\item
  \textbf{Test values}: -1, 0, 1, 50, 99, 100, 101
\item
  \textbf{Focus}: Just inside and outside boundaries
\item
  \textbf{Rationale}: Most errors occur at boundaries
\end{itemize}

\textbf{Benefits:}

\begin{itemize}
\tightlist
\item
  \textbf{Independence}: No programming knowledge required
\item
  \textbf{User perspective}: Tests from user's viewpoint\\
\item
  \textbf{Requirement validation}: Verifies functional specifications
\end{itemize}

\end{solutionbox}
\begin{mnemonicbox}
``EBDSUE'' - Equivalence, Boundary, Decision, State,
Use case, Error guessing

\end{mnemonicbox}

\end{document}
