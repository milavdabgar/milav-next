\documentclass[10pt,a4paper]{article}

% content/resources/templates/preamble.tex
\usepackage[margin=0.6in]{geometry}
\author{Milav Dabgar}
\usepackage{amsmath,amssymb,amsthm}
\usepackage{booktabs}
\usepackage{multirow}
\usepackage{xcolor}
\usepackage{tcolorbox}
\tcbuselibrary{breakable,skins}
\usepackage[colorlinks=true,linkcolor=blue]{hyperref}
\usepackage{titlesec}
\usepackage{enumitem}
\usepackage{tikz}
\usepackage{pgfplots}
\usepackage{circuitikz}
\usepackage[version=4]{mhchem}
\usepackage{longtable}
\usepackage{array}
\usepackage{float}
\usepackage{caption}
\usepackage{listings}

\lstset{
  basicstyle=\small\ttfamily,
  breaklines=true,
  breakatwhitespace=false,
  postbreak=\mbox{\textcolor{red}{$\hookrightarrow$}\space},
  float=false,
  numbers=left,
  numberstyle=\tiny\color{gray},
  numbersep=10pt,
  xleftmargin=2em,
  keywordstyle=\color{blue},
  commentstyle=\color{green!60!black},
  stringstyle=\color{purple},
  backgroundcolor=\color{gray!5},
  showstringspaces=false,
  tabsize=2,
  captionpos=b,
  keepspaces=true,
  columns=flexible
}

\pgfplotsset{compat=1.18}
\usetikzlibrary{shapes,arrows,positioning,calc,patterns,decorations.pathmorphing,decorations.markings,arrows.meta}

% Color scheme
\definecolor{headcolor}{RGB}{0,102,204}
\definecolor{keycolor}{RGB}{220,20,60}
\definecolor{solutioncolor}{RGB}{34,139,34}
\definecolor{mnemoniccolor}{RGB}{148,0,211}
\definecolor{codecolor}{RGB}{0,0,100}

% Spacing
\setlength{\parskip}{3pt}
\setlist[itemize]{nosep}
\setlist[enumerate]{nosep}

% Title formatting
\titleformat{\section}{\Large\bfseries\color{headcolor}}{\thesection}{1em}{}
\titleformat{\subsection}{\large\bfseries\color{headcolor}}{\thesubsection}{1em}{}

% Pandoc tightlist compatibility
\providecommand{\tightlist}{%
  \setlength{\itemsep}{0pt}\setlength{\parskip}{0pt}}

% Pandoc longtable compatibility
\newcounter{none}
\def\thenone{}


% content/resources/templates/gujarati-boxes.tex
\usepackage{fontspec}
\usepackage{polyglossia}

% Set Gujarati as main language (document is primarily in Gujarati)
% Note: gloss-gujarati.ldf doesn't exist in polyglossia, but it will use hyphenation patterns
\setdefaultlanguage{gujarati}
\setotherlanguage{english}

% Configure Gujarati font properly
% Use Language=Default to prevent polyglossia from trying to add language-specific features
% that don't exist for Gujarati, which causes "empty feature" warnings
\newfontfamily\gujaratifont[Script=Gujarati,AutoFakeBold=2.5,AutoFakeSlant=0.3]{Noto Sans Gujarati}
\setmainfont[Script=Gujarati,AutoFakeBold=2.5,AutoFakeSlant=0.3]{Noto Sans Gujarati}
% Use Noto Sans Gujarati for monospace to support Gujarati in text
\setmonofont[Scale=0.9]{Noto Sans Gujarati}

% Configure English to use the same font
\newfontfamily\englishfont[Script=Gujarati,AutoFakeBold=2.5,AutoFakeSlant=0.3]{Noto Sans Gujarati}

% Translations for polyglossia
\gappto\captionsgujarati{
  \renewcommand{\tablename}{કોષ્ટક}
  \renewcommand{\figurename}{આકૃતિ}
}

% Helper for TikZ nodes to ensure Gujarati font
\newcommand{\gu}[1]{{\gujaratifont #1}}

% Custom environments
\newtcolorbox{solutionbox}{
    breakable,
    enhanced,
    colback=solutioncolor!5!white,
    colframe=solutioncolor!75!black,
    fonttitle=\bfseries,
    title=જવાબ
}

\newtcolorbox{solutionboxnobreak}{
 colback=solutioncolor!5!white,
 colframe=solutioncolor!75!black,
 fonttitle=\bfseries,
 title=જવાબ
}

\newtcolorbox{keyformula}{
 breakable,
 enhanced,
 colback=keycolor!5!white,
 colframe=keycolor!75!black,
 fonttitle=\bfseries,
 title=રાસાયણિક સમીકરણ/સૂત્ર
}

\newtcolorbox{mnemonicbox}{
 breakable,
 enhanced,
 colback=mnemoniccolor!5!white,
 colframe=mnemoniccolor!75!black,
 fonttitle=\bfseries,
 title=મેમરી ટ્રીક
}


\begin{document}

\begin{center}
{\Huge\bfseries\color{headcolor} Subject Name (Gujarati)}\\[5pt]
{\LARGE 4353202 -- Winter 2024}\\[3pt]
{\large Semester 1 Study Material}\\[3pt]
{\normalsize\textit{Detailed Solutions and Explanations}}
\end{center}

\vspace{10pt}

\subsection*{પ્રશ્ન 1(અ) [3
ગુણ]}\label{uxaaauxab0uxab6uxaa8-1uxa85-3-uxa97uxaa3}

\textbf{સોફ્ટવૅર ની વ્યાખ્યા આપો અને તેની લાક્ષણિકતા સમજાવો.}

\begin{solutionbox}

\textbf{સોફ્ટવૅર} એ કમ્પ્યુટર પ્રોગ્રામ્સ, પ્રક્રિયાઓ અને દસ્તાવેજીકરણનો સમૂહ છે જે
કમ્પ્યુટર સિસ્ટમ પર કાર્યો કરે છે.

\textbf{ટેબલ: સોફ્ટવૅર લાક્ષણિકતાઓ}

{\def\LTcaptype{none} % do not increment counter
\begin{longtable}[]{@{}ll@{}}
\toprule\noalign{}
લાક્ષણિકતા & વર્ણન \\
\midrule\noalign{}
\endhead
\bottomrule\noalign{}
\endlastfoot
\textbf{અસ્પર્શ્ય} & સ્પર્શ કરી શકાતું નથી, માત્ર અનુભવી શકાય છે \\
\textbf{વિકસિત} & એન્જિનિયર્ડ, મેન્યુફેક્ચર્ડ નહીં \\
\textbf{જાળવણીયોગ્ય} & સુધારણા અને અપડેટ કરી શકાય છે \\
\textbf{વિશ્વસનીય} & સતત કામ કરવું જોઈએ \\
\textbf{કાર્યક્ષમ} & સાધનોનો શ્રેષ્ઠ ઉપયોગ કરે છે \\
\end{longtable}
}

\begin{itemize}
\tightlist
\item
  \textbf{મુખ્ય મુદ્દો}: સોફ્ટવૅર = પ્રોગ્રામ્સ + દસ્તાવેજીકરણ + પ્રક્રિયાઓ
\item
\end{solutionbox}
\begin{mnemonicbox}
``I Don't Make Reliable Electronics''
\end{itemize}

\end{mnemonicbox}
\begin{center}\rule{0.5\linewidth}{0.5pt}\end{center}

\subsection*{પ્રશ્ન 1(બ) [4
ગુણ]}\label{uxaaauxab0uxab6uxaa8-1uxaac-4-uxa97uxaa3}

\textbf{ક્લાસિકલ વોટરફોલ મોડેલ સમજાવો.}

\begin{solutionbox}

\textbf{વોટરફોલ મોડેલ} એ રેખીય ક્રમિક સોફ્ટવૅર વિકાસ પદ્ધતિ છે જ્યાં દરેક તબક્કો
પૂર્ણ થયા પછી જ આગળનો તબક્કો શરૂ થાય છે.

\begin{center}
\textbf{Mermaid Diagram (Code)}
\begin{verbatim}
{Shaded}
{Highlighting}[]
graph LR
    A[આવશ્યકતા વિશ્લેષણ] {-{-}{} B[સિસ્ટમ ડિઝાઇન]}
    B {-{-}{} C[અમલીકરણ]}
    C {-{-}{} D[પરીક્ષણ]}
    D {-{-}{} E[જમાવટ]}
    E {-{-}{} F[જાળવણી]}
{Highlighting}
{Shaded}
\end{verbatim}
\end{center}

\textbf{મુખ્ય લક્ષણો}:

\begin{itemize}
\tightlist
\item
  \textbf{ક્રમિક તબક્કાઓ}: તબક્કાઓ વચ્ચે કોઈ ઓવરલેપ નથી
\item
  \textbf{દસ્તાવેજીકરણ આધારિત}: દરેક તબક્કે ભારે દસ્તાવેજીકરણ
\item
  \textbf{સરળ માળખું}: સમજવા અને મેનેજ કરવા સરળ
\item
  \textbf{નિશ્ચિત આવશ્યકતાઓ}: એકવાર શરૂ થયા પછી ફેરફાર મુશ્કેલ
\end{itemize}

\end{solutionbox}
\begin{mnemonicbox}
``Real Systems Include Testing, Deployment,
Maintenance''

\end{mnemonicbox}
\begin{center}\rule{0.5\linewidth}{0.5pt}\end{center}

\subsection*{પ્રશ્ન 1(ક) [7
ગુણ]}\label{uxaaauxab0uxab6uxaa8-1uxa95-7-uxa97uxaa3}

\textbf{સોફ્ટવૅર પ્રોસેસ ફ્રેમવર્ક અને અમ્બ્રેલા એક્ટિવિટી સમજાવો.}

\begin{solutionbox}

\textbf{સોફ્ટવૅર પ્રોસેસ ફ્રેમવર્ક} મુખ્ય પ્રોસેસ વિસ્તારો ઓળખીને સંપૂર્ણ સોફ્ટવૅર
એન્જિનિયરિંગ પ્રોસેસ માટે પાયો પ્રદાન કરે છે.

\begin{center}
\textbf{Mermaid Diagram (Code)}
\begin{verbatim}
{Shaded}
{Highlighting}[]
graph LR
    A[સંવાદ] {-{-}{} B[આયોજન]}
    B {-{-}{} C[મોડેલિંગ]}
    C {-{-}{} D[નિર્માણ]}
    D {-{-}{} E[જમાવટ]}
    E {-{-}{} A}
    
    F[અમ્બ્રેલા એક્ટિવિટીઝ] {-{-}{} A}
    F {-{-}{} B}
    F {-{-}{} C}
    F {-{-}{} D}
    F {-{-}{} E}
{Highlighting}
{Shaded}
\end{verbatim}
\end{center}

\textbf{ટેબલ: ફ્રેમવર્ક એક્ટિવિટીઝ વિ અમ્બ્રેલા એક્ટિવિટીઝ}

{\def\LTcaptype{none} % do not increment counter
\begin{longtable}[]{@{}ll@{}}
\toprule\noalign{}
ફ્રેમવર્ક એક્ટિવિટીઝ & અમ્બ્રેલા એક્ટિવિટીઝ \\
\midrule\noalign{}
\endhead
\bottomrule\noalign{}
\endlastfoot
સંવાદ & સોફ્ટવૅર પ્રોજેક્ટ ટ્રેકિંગ \\
આયોજન & જોખમ મેનેજમેન્ટ \\
મોડેલિંગ & ગુણવત્તા ખાતરી \\
નિર્માણ & તકનીકી સમીક્ષાઓ \\
જમાવટ & કન્ફિગરેશન મેનેજમેન્ટ \\
\end{longtable}
}

\textbf{ફ્રેમવર્ક એક્ટિવિટીઝ}:

\begin{itemize}
\tightlist
\item
  \textbf{સંવાદ}: સ્ટેકહોલ્ડર્સ પાસેથી આવશ્યકતાઓ એકત્રિત કરવી
\item
  \textbf{આયોજન}: પ્રોજેક્ટ પ્લાન અને શેડ્યુલ બનાવવું
\item
  \textbf{મોડેલિંગ}: ડિઝાઇન મોડેલ્સ બનાવવા
\item
  \textbf{નિર્માણ}: કોડ જનરેશન અને પરીક્ષણ
\item
  \textbf{જમાવટ}: સોફ્ટવૅર ડિલિવરી અને ફીડબેક
\end{itemize}

\textbf{અમ્બ્રેલા એક્ટિવિટીઝ} પ્રોજેક્ટ દરમિયાન ચાલે છે:

\begin{itemize}
\tightlist
\item
  \textbf{પ્રોજેક્ટ ટ્રેકિંગ}: પ્રગતિ નિરીક્ષણ
\item
  \textbf{જોખમ મેનેજમેન્ટ}: જોખમો ઓળખવા અને નિયંત્રિત કરવા
\item
  \textbf{ગુણવત્તા ખાતરી}: ગુણવત્તા ધોરણો સુનિશ્ચિત કરવા
\item
  \textbf{કન્ફિગરેશન મેનેજમેન્ટ}: ફેરફારો નિયંત્રિત કરવા
\end{itemize}

\end{solutionbox}
\begin{mnemonicbox}
``Can People Make Construction Deploy''

\end{mnemonicbox}
\begin{center}\rule{0.5\linewidth}{0.5pt}\end{center}

\subsection*{પ્રશ્ન 1(ક) OR [7
ગુણ]}\label{uxaaauxab0uxab6uxaa8-1uxa95-or-7-uxa97uxaa3}

\textbf{SCRUM મોડેલ પર ટૂંક નોંધ લખો.}

\begin{solutionbox}

\textbf{SCRUM} એ પુનરાવર્તક અને વૃદ્ધિશીલ પ્રથાઓનો ઉપયોગ કરીને સોફ્ટવૅર વિકાસ
પ્રોજેક્ટ્સનું મેનેજમેન્ટ કરવા માટેનું એક agile ફ્રેમવર્ક છે.

\begin{center}
\textbf{Mermaid Diagram (Code)}
\begin{verbatim}
{Shaded}
{Highlighting}[]
graph LR
    A[Product Backlog] {-{-}{} B[Sprint Planning]}
    B {-{-}{} C[Sprint Backlog]}
    C {-{-}{} D[Sprint 2{-}4 weeks]}
    D {-{-}{} E[Sprint Review]}
    E {-{-}{} F[Sprint Retrospective]}
    F {-{-}{} B}
    D {-{-}{} G[Daily Scrum]}
    G {-{-}{} D}
{Highlighting}
{Shaded}
\end{verbatim}
\end{center}

\textbf{ટેબલ: SCRUM ભૂમિકાઓ અને આર્ટિફેક્ટ્સ}

{\def\LTcaptype{none} % do not increment counter
\begin{longtable}[]{@{}ll@{}}
\toprule\noalign{}
ઘટક & વર્ણન \\
\midrule\noalign{}
\endhead
\bottomrule\noalign{}
\endlastfoot
\textbf{Product Owner} & આવશ્યકતાઓ અને પ્રાથમિકતાઓ વ્યાખ્યાયિત કરે છે \\
\textbf{Scrum Master} & પ્રક્રિયાને સુવિધા આપે છે અને અવરોધો દૂર કરે છે \\
\textbf{Development Team} & સ્વ-સંગઠિત ટીમ જે પ્રોડક્ટ બનાવે છે \\
\textbf{Product Backlog} & લક્ષણોની પ્રાથમિકતા આપેલી યાદી \\
\textbf{Sprint Backlog} & વર્તમાન sprint માટે પસંદ કરેલા કાર્યો \\
\end{longtable}
}

\textbf{મુખ્ય ઇવેન્ટ્સ}:

\begin{itemize}
\tightlist
\item
  \textbf{Sprint Planning}: આગામી sprint માટે કામ પસંદ કરવું
\item
  \textbf{Daily Scrum}: 15-મિનિટનું દૈનિક સિંક્રોનાઇઝેશન
\item
  \textbf{Sprint Review}: પૂર્ણ થયેલ કામ દર્શાવવું
\item
  \textbf{Sprint Retrospective}: પ્રક્રિયા પર વિચાર કરવો અને સુધારવું
\end{itemize}

\textbf{ફાયદાઓ}: ઝડપી ડિલિવરી, લવચીકતા, સતત સુધારણા, ગ્રાહક સહયોગ

\end{solutionbox}
\begin{mnemonicbox}
``People Sprint Daily Reviewing Retrospectively''

\end{mnemonicbox}
\begin{center}\rule{0.5\linewidth}{0.5pt}\end{center}

\subsection*{પ્રશ્ન 2(અ) [3
ગુણ]}\label{uxaaauxab0uxab6uxaa8-2uxa85-3-uxa97uxaa3}

\textbf{સારા SRS ની લાક્ષણિકતા સમજાવો.}

\begin{solutionbox}

\textbf{SRS (સોફ્ટવૅર આવશ્યકતા વિશિષ્ટતા)} દસ્તાવેજ અસરકારક બનવા માટે વિશિષ્ટ
ગુણો હોવા જોઈએ.

\textbf{ટેબલ: સારા SRS લાક્ષણિકતાઓ}

{\def\LTcaptype{none} % do not increment counter
\begin{longtable}[]{@{}ll@{}}
\toprule\noalign{}
લાક્ષણિકતા & અર્થ \\
\midrule\noalign{}
\endhead
\bottomrule\noalign{}
\endlastfoot
\textbf{સંપૂર્ણ} & બધી આવશ્યકતાઓ સમાવેશ \\
\textbf{સુસંગત} & કોઈ વિરોધાભાસી આવશ્યકતાઓ નથી \\
\textbf{અસ્પષ્ટ નથી} & સ્પષ્ટ અને એક અર્થઘટન \\
\textbf{ચકાસણીયોગ્ય} & પરીક્ષણ અને વેલિડેશન શક્ય \\
\textbf{સુધારણાયોગ્ય} & જરૂર પડે ત્યારે બદલવા સરળ \\
\end{longtable}
}

\begin{itemize}
\tightlist
\item
  \textbf{સંપૂર્ણ}: બધી functional અને non-functional આવશ્યકતાઓ સમાવે છે
\item
  \textbf{સુસંગત}: વિવિધ આવશ્યકતાઓ વચ્ચે કોઈ સંઘર્ષ નથી
\item
  \textbf{અસ્પષ્ટ નથી}: દરેક આવશ્યકતાનો માત્ર એક જ અર્થઘટન છે
\end{itemize}

\end{solutionbox}
\begin{mnemonicbox}
``Complete Computers Use Verified Modifications''

\end{mnemonicbox}
\begin{center}\rule{0.5\linewidth}{0.5pt}\end{center}

\subsection*{પ્રશ્ન 2(બ) [4
ગુણ]}\label{uxaaauxab0uxab6uxaa8-2uxaac-4-uxa97uxaa3}

\textbf{પ્રોટોટાઇપ મોડેલના લાભ અને ગેરલાભ વર્ણવો.}

\begin{solutionbox}

\textbf{પ્રોટોટાઇપ મોડેલ} આવશ્યકતાઓને વધુ સારી રીતે સમજવા માટે સોફ્ટવૅરનું કાર્યકારી
મોડેલ બનાવે છે.

\textbf{ટેબલ: પ્રોટોટાઇપ મોડેલ - ફાયદા અને ગેરફાયદા}

{\def\LTcaptype{none} % do not increment counter
\begin{longtable}[]{@{}ll@{}}
\toprule\noalign{}
ફાયદા & ગેરફાયદા \\
\midrule\noalign{}
\endhead
\bottomrule\noalign{}
\endlastfoot
\textbf{આવશ્યકતા સમજણ સુધારે છે} & \textbf{સમયનો વધારે ખર્ચ} \\
\textbf{વપરાશકર્તા સામેલગીરી} & \textbf{ખર્ચમાં વધારો} \\
\textbf{પ્રારંભિક ભૂલ શોધ} & \textbf{અપૂર્ણ વિશ્લેષણ} \\
\textbf{વપરાશકર્તા સંતુષ્ટિ} & \textbf{પ્રોટોટાઇપ મૂંઝવણ} \\
\end{longtable}
}

\textbf{ફાયદા}:

\begin{itemize}
\tightlist
\item
  \textbf{સ્પષ્ટ આવશ્યકતાઓ}: વપરાશકર્તા કાર્યકારી મોડેલ જુએ છે
\item
  \textbf{પ્રારંભિક ફીડબેક}: અંતિમ પ્રોડક્ટના જોખમો ઘટાડે છે
\item
  \textbf{વપરાશકર્તાનો સમાવેશ}: વધુ સારી વપરાશકર્તા સ્વીકૃતિ
\end{itemize}

\textbf{ગેરફાયદા}:

\begin{itemize}
\tightlist
\item
  \textbf{વધારાનો સમય}: પ્રોટોટાઇપ બનાવવામાં સમય લાગે છે
\item
  \textbf{વધારાનો ખર્ચ}: પ્રોટોટાઇપ માટે સાધનોની જરૂર
\item
  \textbf{અવકાશ વિસ્તરણ}: વપરાશકર્તા પ્રોટોટાઇપના ફીચર્સની અપેક્ષા રાખી શકે છે
\end{itemize}

\end{solutionbox}
\begin{mnemonicbox}
``Better Users Experience'' વિ ``Time Costs
Increase''

\end{mnemonicbox}
\begin{center}\rule{0.5\linewidth}{0.5pt}\end{center}

\subsection*{પ્રશ્ન 2(ક) [7
ગુણ]}\label{uxaaauxab0uxab6uxaa8-2uxa95-7-uxa97uxaa3}

\textbf{સ્પાઇરલ મોડેલ ડિઝાઇન વર્ણવો અને તેના લાભ અને ગેરલાભ વિશે લખો.}

\begin{solutionbox}

\textbf{સ્પાઇરલ મોડેલ} પુનરાવર્તક વિકાસને વ્યવસ્થિત જોખમ મેનેજમેન્ટ સાથે પુનરાવર્તિત
ચક્રો દ્વારા જોડે છે.

\begin{center}
\textbf{Mermaid Diagram (Code)}
\begin{verbatim}
{Shaded}
{Highlighting}[]
graph TD
    A[આયોજન] {-{-}{} B[જોખમ વિશ્લેષણ]}
    B {-{-}{} C[એન્જિનિયરિંગ]}
    C {-{-}{} D[ગ્રાહક મૂલ્યાંકન]}
    D {-{-}{} A}
    
    E[ચતુર્થાંશ 1: આયોજન] 
    F[ચતુર્થાંશ 2: જોખમ વિશ્લેષણ]
    G[ચતુર્થાંશ 3: એન્જિનિયરિંગ]
    H[ચતુર્થાંશ 4: ગ્રાહક મૂલ્યાંકન]
{Highlighting}
{Shaded}
\end{verbatim}
\end{center}

\textbf{ટેબલ: સ્પાઇરલ મોડેલ તબક્કાઓ}

{\def\LTcaptype{none} % do not increment counter
\begin{longtable}[]{@{}ll@{}}
\toprule\noalign{}
તબક્કો & પ્રવૃત્તિઓ \\
\midrule\noalign{}
\endhead
\bottomrule\noalign{}
\endlastfoot
\textbf{આયોજન} & આવશ્યકતા એકત્રીકરણ, સાધન આયોજન \\
\textbf{જોખમ વિશ્લેષણ} & જોખમો ઓળખવા અને ઉકેલવા \\
\textbf{એન્જિનિયરિંગ} & વિકાસ અને પરીક્ષણ \\
\textbf{ગ્રાહક મૂલ્યાંકન} & ગ્રાહક સમીક્ષા અને ફીડબેક \\
\end{longtable}
}

\textbf{ફાયદા}:

\begin{itemize}
\tightlist
\item
  \textbf{જોખમ મેનેજમેન્ટ}: પ્રારંભિક જોખમ ઓળખ
\item
  \textbf{લવચીકતા}: ફેરફારો સરળતાથી સમાવે છે
\item
  \textbf{ગ્રાહક સામેલગીરી}: નિયમિત ગ્રાહક ફીડબેક
\item
  \textbf{ગુણવત્તા ફોકસ}: સતત પરીક્ષણ અને વેલિડેશન
\end{itemize}

\textbf{ગેરફાયદા}:

\begin{itemize}
\tightlist
\item
  \textbf{જટિલ મેનેજમેન્ટ}: મેનેજ કરવું મુશ્કેલ
\item
  \textbf{ઊંચો ખર્ચ}: જોખમ વિશ્લેષણને કારણે મોંઘું
\item
  \textbf{સમય લેતું}: લાંબા વિકાસ ચક્રો
\item
  \textbf{જોખમ નિપુણતા જરૂરી}: જોખમ મૂલ્યાંકન કૌશલ્યની જરૂર
\end{itemize}

\textbf{શ્રેષ્ઠ માટે}: મોટા, જટિલ, ઉચ્ચ-જોખમ પ્રોજેક્ટ્સ

\end{solutionbox}
\begin{mnemonicbox}
``Plan Risks Engineering Customer'' તબક્કાઓ માટે

\end{mnemonicbox}
\begin{center}\rule{0.5\linewidth}{0.5pt}\end{center}

\subsection*{પ્રશ્ન 2(અ) OR [3
ગુણ]}\label{uxaaauxab0uxab6uxaa8-2uxa85-or-3-uxa97uxaa3}

\textbf{ઇન્ક્રિમેન્ટલ મોડેલ સમજાવો.}

\begin{solutionbox}

\textbf{ઇન્ક્રિમેન્ટલ મોડેલ} સોફ્ટવૅરને નાના, કાર્યાત્મક ટુકડાઓમાં જે ઇન્ક્રિમેન્ટ્સ કહેવાય
છે તેમાં ડિલિવર કરે છે.

\begin{center}
\textbf{Mermaid Diagram (Code)}
\begin{verbatim}
{Shaded}
{Highlighting}[]
graph LR
    A[મુખ્ય પ્રોડક્ટ] {-{-}{} B[ઇન્ક્રિમેન્ટ 1]}
    B {-{-}{} C[ઇન્ક્રિમેન્ટ 2]}
    C {-{-}{} D[ઇન્ક્રિમેન્ટ 3]}
    D {-{-}{} E[અંતિમ પ્રોડક્ટ]}
{Highlighting}
{Shaded}
\end{verbatim}
\end{center}

\textbf{મુખ્ય લક્ષણો}:

\begin{itemize}
\tightlist
\item
  \textbf{આંશિક અમલીકરણ}: દરેક ઇન્ક્રિમેન્ટ કાર્યક્ષમતા ઉમેરે છે
\item
  \textbf{પ્રારંભિક ડિલિવરી}: મુખ્ય ફીચર્સ પ્રથમ ડિલિવર થાય છે
\item
  \textbf{સમાંતર વિકાસ}: અનેક ઇન્ક્રિમેન્ટ્સ એકસાથે વિકસાવી શકાય છે
\end{itemize}

\textbf{ટેબલ: ઇન્ક્રિમેન્ટલ મોડેલ લાક્ષણિકતાઓ}

{\def\LTcaptype{none} % do not increment counter
\begin{longtable}[]{@{}ll@{}}
\toprule\noalign{}
પાસું & વર્ણન \\
\midrule\noalign{}
\endhead
\bottomrule\noalign{}
\endlastfoot
\textbf{ડિલિવરી} & અનેક રિલીઝ \\
\textbf{કાર્યક્ષમતા} & દરેક ઇન્ક્રિમેન્ટ સાથે વધે છે \\
\textbf{જોખમ} & પ્રારંભિક ડિલિવરી દ્વારા ઘટે છે \\
\textbf{ફીડબેક} & સતત વપરાશકર્તા ફીડબેક \\
\end{longtable}
}

\end{solutionbox}
\begin{mnemonicbox}
``Deliver Functionality Reducing Feedback''

\end{mnemonicbox}
\begin{center}\rule{0.5\linewidth}{0.5pt}\end{center}

\subsection*{પ્રશ્ન 2(બ) OR [4
ગુણ]}\label{uxaaauxab0uxab6uxaa8-2uxaac-or-4-uxa97uxaa3}

\textbf{રેપિડ એપ્લિકેશન ડેવલપમેન્ટ મોડેલનો ખ્યાલ આપી સમજાવો.}

\begin{solutionbox}

\textbf{RAD (રેપિડ એપ્લિકેશન ડેવલપમેન્ટ)} વ્યાપક આયોજનને બદલે ઝડપી પ્રોટોટાઇપિંગ અને
ઝડપી ફીડબેક પર ભાર મૂકે છે.

\textbf{ટેબલ: RAD મોડેલ તબક્કાઓ}

{\def\LTcaptype{none} % do not increment counter
\begin{longtable}[]{@{}lll@{}}
\toprule\noalign{}
તબક્કો & અવધિ & પ્રવૃત્તિઓ \\
\midrule\noalign{}
\endhead
\bottomrule\noalign{}
\endlastfoot
\textbf{બિઝનેસ મોડેલિંગ} & ટૂંકી & બિઝનેસ કાર્યો વ્યાખ્યાયિત કરવા \\
\textbf{ડેટા મોડેલિંગ} & ટૂંકી & ડેટા આવશ્યકતાઓ વ્યાખ્યાયિત કરવી \\
\textbf{પ્રોસેસ મોડેલિંગ} & ટૂંકી & ડેટાને બિઝનેસ માહિતીમાં રૂપાંતરિત કરવું \\
\textbf{એપ્લિકેશન જનરેશન} & ટૂંકી & સોફ્ટવૅર બનાવવા માટે ટૂલ્સનો ઉપયોગ \\
\textbf{ટેસ્ટિંગ અને ટર્નઓવર} & ટૂંકી & પરીક્ષણ અને જમાવટ \\
\end{longtable}
}

\textbf{મુખ્ય ખ્યાલો}:

\begin{itemize}
\tightlist
\item
  \textbf{પુનઃઉપયોગી ઘટકો}: પૂર્વ-નિર્મિત ઘટકો વિકાસ ગતિ વધારે છે
\item
  \textbf{શક્તિશાળી ટૂલ્સ}: CASE ટૂલ્સ અને કોડ જનરેટર્સ
\item
  \textbf{નાની ટીમો}: પ્રતિ ટીમ 2-6 લોકો
\item
  \textbf{સમય-બોક્સ્ડ}: કડક સમય મર્યાદા (60-90 દિવસ)
\end{itemize}

\textbf{RAD માટેની આવશ્યકતાઓ}:

\begin{itemize}
\tightlist
\item
  \textbf{સારી રીતે વ્યાખ્યાયિત બિઝનેસ આવશ્યકતાઓ}
\item
  \textbf{વપરાશકર્તાની સામેલગીરી} સમગ્ર પ્રક્રિયા દરમિયાન
\item
  \textbf{કુશળ ડેવલપર્સ} જે RAD ટૂલ્સથી પરિચિત છે
\end{itemize}

\end{solutionbox}
\begin{mnemonicbox}
``Business Data Process Application Testing''

\end{mnemonicbox}
\begin{center}\rule{0.5\linewidth}{0.5pt}\end{center}

\subsection*{પ્રશ્ન 2(ક) OR [7
ગુણ]}\label{uxaaauxab0uxab6uxaa8-2uxa95-or-7-uxa97uxaa3}

\textbf{SDLC ને વ્યાખ્યાયિત કરો અને દરેક તબક્કા સમજાવો.}

\begin{solutionbox}

\textbf{SDLC (સોફ્ટવૅર ડેવલપમેન્ટ લાઇફ સાઇકલ)} સારી રીતે વ્યાખ્યાયિત તબક્કાઓ
દ્વારા સોફ્ટવૅર બનાવવાની વ્યવસ્થિત પ્રક્રિયા છે.

\begin{center}
\textbf{Mermaid Diagram (Code)}
\begin{verbatim}
{Shaded}
{Highlighting}[]
graph LR
    A[આયોજન] {-{-}{} B[વિશ્લેષણ]}
    B {-{-}{} C[ડિઝાઇન]}
    C {-{-}{} D[અમલીકરણ]}
    D {-{-}{} E[પરીક્ષણ]}
    E {-{-}{} F[જમાવટ]}
    F {-{-}{} G[જાળવણી]}
    G {-{-}{} A}
{Highlighting}
{Shaded}
\end{verbatim}
\end{center}

\textbf{ટેબલ: SDLC તબક્કાઓ વિગતવાર}

{\def\LTcaptype{none} % do not increment counter
\begin{longtable}[]{@{}lll@{}}
\toprule\noalign{}
તબક્કો & પ્રવૃત્તિઓ & ડિલિવરેબલ્સ \\
\midrule\noalign{}
\endhead
\bottomrule\noalign{}
\endlastfoot
\textbf{આયોજન} & પ્રોજેક્ટ આયોજન, શક્યતા અભ્યાસ & પ્રોજેક્ટ પ્લાન \\
\textbf{વિશ્લેષણ} & આવશ્યકતા એકત્રીકરણ & SRS દસ્તાવેજ \\
\textbf{ડિઝાઇન} & સિસ્ટમ આર્કિટેક્ચર, UI ડિઝાઇન & ડિઝાઇન દસ્તાવેજ \\
\textbf{અમલીકરણ} & કોડિંગ, યુનિટ ટેસ્ટિંગ & સોર્સ કોડ \\
\textbf{પરીક્ષણ} & સિસ્ટમ ટેસ્ટિંગ, ઇન્ટિગ્રેશન & ટેસ્ટ રિપોર્ટ્સ \\
\textbf{જમાવટ} & ઇન્સ્ટોલેશન, વપરાશકર્તા તાલીમ & લાઇવ સિસ્ટમ \\
\textbf{જાળવણી} & બગ ફિક્સ, સુધારણાઓ & અપડેટેડ સિસ્ટમ \\
\end{longtable}
}

\textbf{તબક્કો વર્ણન}:

\begin{itemize}
\tightlist
\item
  \textbf{આયોજન}: પ્રોજેક્ટ અવકાશ અને સાધનો વ્યાખ્યાયિત કરવા
\item
  \textbf{વિશ્લેષણ}: સિસ્ટમ શું કરવું જોઈએ તે સમજવું
\item
  \textbf{ડિઝાઇન}: સિસ્ટમ કેવી રીતે કામ કરશે તેનું આયોજન
\item
  \textbf{અમલીકરણ}: વાસ્તવિક સિસ્ટમ બનાવવું
\item
  \textbf{પરીક્ષણ}: સિસ્ટમ યોગ્ય રીતે કામ કરે છે કે કેમ તે ચકાસવું
\item
  \textbf{જમાવટ}: વપરાશકર્તાઓ માટે સિસ્ટમ રિલીઝ કરવું
\item
  \textbf{જાળવણી}: ચાલુ સપોર્ટ અને અપડેટ્સ
\end{itemize}

\end{solutionbox}
\begin{mnemonicbox}
``People Always Design Implementation, Test
Deployment, Maintain''

\end{mnemonicbox}
\begin{center}\rule{0.5\linewidth}{0.5pt}\end{center}

\subsection*{પ્રશ્ન 3(અ) [3
ગુણ]}\label{uxaaauxab0uxab6uxaa8-3uxa85-3-uxa97uxaa3}

\textbf{સોફ્ટવૅર પ્રોજેક્ટને મેનેજ કરવાની સ્કિલ વર્ણવો.}

\begin{solutionbox}

\textbf{સોફ્ટવૅર પ્રોજેક્ટ મેનેજમેન્ટ} તકનીકી અને સોફ્ટ સ્કિલ્સના સંયોજનની જરૂર છે.

\textbf{ટેબલ: જરૂરી પ્રોજેક્ટ મેનેજમેન્ટ સ્કિલ્સ}

{\def\LTcaptype{none} % do not increment counter
\begin{longtable}[]{@{}ll@{}}
\toprule\noalign{}
સ્કિલ કેટેગરી & વિશિષ્ટ સ્કિલ્સ \\
\midrule\noalign{}
\endhead
\bottomrule\noalign{}
\endlastfoot
\textbf{તકનીકી} & SDLC, ટૂલ્સ, ટેક્નોલોજીઝની સમજ \\
\textbf{નેતૃત્વ} & ટીમ પ્રેરણા, નિર્ણય લેવો \\
\textbf{સંવાદ} & ટીમ અને ક્લાયન્ટ સાથે સ્પષ્ટ સંવાદ \\
\textbf{આયોજન} & સાધન ફાળવણી, શેડ્યુલિંગ \\
\textbf{સમસ્યા ઉકેલ} & જોખમ મેનેજમેન્ટ, સંઘર્ષ નિવારણ \\
\end{longtable}
}

\textbf{મુખ્ય સ્કિલ્સ}:

\begin{itemize}
\tightlist
\item
  \textbf{લોકો મેનેજમેન્ટ}: ટીમ સભ્યોનું નેતૃત્વ અને પ્રેરણા
\item
  \textbf{તકનીકી જ્ઞાન}: વિકાસ પ્રક્રિયા અને ટૂલ્સની સમજ
\item
  \textbf{સંવાદ}: તકનીકી ટીમ અને સ્ટેકહોલ્ડર્સ વચ્ચેનો સેતુ
\end{itemize}

\end{solutionbox}
\begin{mnemonicbox}
``Technical Leaders Communicate Planning Problems''

\end{mnemonicbox}
\begin{center}\rule{0.5\linewidth}{0.5pt}\end{center}

\subsection*{પ્રશ્ન 3(બ) [4
ગુણ]}\label{uxaaauxab0uxab6uxaa8-3uxaac-4-uxa97uxaa3}

\textbf{સોફ્ટવૅર પ્રોજેક્ટ મેનેજરની જવાબદારી ટૂંકમાં લખો.}

\begin{solutionbox}

\textbf{સોફ્ટવૅર પ્રોજેક્ટ મેનેજર} પ્રોજેક્ટની શરૂઆતથી પૂર્ણતા સુધી સમગ્ર પ્રોજેક્ટની દેખરેખ
રાખે છે.

\textbf{ટેબલ: પ્રોજેક્ટ મેનેજરની જવાબદારીઓ}

{\def\LTcaptype{none} % do not increment counter
\begin{longtable}[]{@{}ll@{}}
\toprule\noalign{}
વિસ્તાર & જવાબદારીઓ \\
\midrule\noalign{}
\endhead
\bottomrule\noalign{}
\endlastfoot
\textbf{આયોજન} & પ્રોજેક્ટ પ્લાન, શેડ્યુલ, બજેટ બનાવવા \\
\textbf{ટીમ મેનેજમેન્ટ} & ટીમ સભ્યોને હાયર, ટ્રેન અને મેનેજ કરવા \\
\textbf{સંવાદ} & સ્ટેકહોલ્ડર્સને નિયમિત અપડેટ્સ \\
\textbf{ગુણવત્તા નિયંત્રણ} & ડિલિવરેબલ્સ ગુણવત્તા ધોરણો પૂરા કરે તે સુનિશ્ચિત કરવું \\
\textbf{જોખમ મેનેજમેન્ટ} & પ્રોજેક્ટના જોખમો ઓળખવા અને ઘટાડવા \\
\end{longtable}
}

\textbf{પ્રાથમિક જવાબદારીઓ}:

\begin{itemize}
\tightlist
\item
  \textbf{પ્રોજેક્ટ આયોજન}: અવકાશ, સમયસીમા અને સાધનો વ્યાખ્યાયિત કરવા
\item
  \textbf{ટીમ નેતૃત્વ}: વિકાસ ટીમને માર્ગદર્શન અને સહાય આપવી
\item
  \textbf{સ્ટેકહોલ્ડર સંવાદ}: દરેકને પ્રગતિની માહિતી આપતા રહેવું
\item
  \textbf{ગુણવત્તા ખાતરી}: પ્રોજેક્ટ આવશ્યકતાઓ પૂરી કરે તે સુનિશ્ચિત કરવું
\item
  \textbf{જોખમ મેનેજમેન્ટ}: પ્રોજેક્ટના જોખમો અને મુદ્દાઓને હેન્ડલ કરવા
\end{itemize}

\textbf{સફળતાના પરિબળો}: સમયસર ડિલિવરી, બજેટની અંદર, આવશ્યકતાઓ પૂરી કરવી

\end{solutionbox}
\begin{mnemonicbox}
``Plan Team Communication Quality Risk''

\end{mnemonicbox}
\begin{center}\rule{0.5\linewidth}{0.5pt}\end{center}

\subsection*{પ્રશ્ન 3(ક) [7
ગુણ]}\label{uxaaauxab0uxab6uxaa8-3uxa95-7-uxa97uxaa3}

\textbf{SRS ની આવશ્યકતાનું વર્ગીકરણ કરો (1) ફંક્શનલ આવશ્યકતાઓ (2) નોન-ફંક્શનલ
આવશ્યકતાઓ.}

\begin{solutionbox}

\textbf{આવશ્યકતા વર્ગીકરણ} વિવિધ પ્રકારની સિસ્ટમ જરૂરિયાતોને વ્યવસ્થિત અને
સમજવામાં મદદ કરે છે.

\textbf{ટેબલ: ફંક્શનલ વિ નોન-ફંક્શનલ આવશ્યકતાઓ}

{\def\LTcaptype{none} % do not increment counter
\begin{longtable}[]{@{}lll@{}}
\toprule\noalign{}
પાસું & ફંક્શનલ આવશ્યકતાઓ & નોન-ફંક્શનલ આવશ્યકતાઓ \\
\midrule\noalign{}
\endhead
\bottomrule\noalign{}
\endlastfoot
\textbf{વ્યાખ્યા} & સિસ્ટમ શું કરવું જોઈએ & સિસ્ટમ કેવા પ્રદર્શન કરવું જોઈએ \\
\textbf{ફોકસ} & સિસ્ટમ કાર્યક્ષમતા & સિસ્ટમ ગુણવત્તા લક્ષણો \\
\textbf{ઉદાહરણો} & લોગિન, સર્ચ, કેલ્ક્યુલેટ & પ્રદર્શન, સુરક્ષા, ઉપયોગિતા \\
\textbf{પરીક્ષણ} & ફંક્શનલ ટેસ્ટિંગ & પ્રદર્શન ટેસ્ટિંગ \\
\end{longtable}
}

\textbf{ફંક્શનલ આવશ્યકતાઓ}:

\begin{itemize}
\tightlist
\item
  \textbf{વપરાશકર્તા ક્રિયાપ્રતિક્રિયાઓ}: લોગિન, રજિસ્ટ્રેશન, ડેટા એન્ટ્રી
\item
  \textbf{બિઝનેસ નિયમો}: વેલિડેશન નિયમો, ગણતરીઓ
\item
  \textbf{સિસ્ટમ ફીચર્સ}: રિપોર્ટ્સ, નોટિફિકેશન્સ, વર્કફ્લો
\item
  \textbf{ડેટા પ્રોસેસિંગ}: CRUD ઓપરેશન્સ
\end{itemize}

\textbf{ઉદાહરણો}:

\begin{itemize}
\tightlist
\item
  વપરાશકર્તા યુઝરનેમ/પાસવર્ડ સાથે લોગિન કરી શકે છે
\item
  સિસ્ટમ આપોઆપ ટેક્સની ગણતરી કરે છે
\item
  માસિક વેચાણ રિપોર્ટ જનરેટ કરવી
\end{itemize}

\textbf{નોન-ફંક્શનલ આવશ્યકતાઓ}:

\textbf{ટેબલ: નોન-ફંક્શનલ આવશ્યકતા પ્રકારો}

{\def\LTcaptype{none} % do not increment counter
\begin{longtable}[]{@{}lll@{}}
\toprule\noalign{}
પ્રકાર & વર્ણન & ઉદાહરણ \\
\midrule\noalign{}
\endhead
\bottomrule\noalign{}
\endlastfoot
\textbf{પ્રદર્શન} & ગતિ અને પ્રતિસાદ & પ્રતિસાદ સમય \textless{} 2 સેકન્ડ \\
\textbf{સુરક્ષા} & ડેટા સંરક્ષણ & એન્ક્રિપ્ટેડ ડેટા ટ્રાન્સમિશન \\
\textbf{ઉપયોગિતા} & વપરાશકર્તા અનુભવ & શીખવા માટે સરળ ઇન્ટરફેસ \\
\textbf{વિશ્વસનીયતા} & સિસ્ટમ વિશ્વસનીયતા & 99.9\% અપટાઇમ \\
\textbf{સ્કેલેબિલિટી} & વૃદ્ધિ હેન્ડલિંગ & 1000+ વપરાશકર્તાઓને સપોર્ટ \\
\end{longtable}
}

\textbf{ગુણવત્તા લક્ષણો}:

\begin{itemize}
\tightlist
\item
  \textbf{પ્રદર્શન}: પ્રતિસાદ સમય, થ્રુપુટ
\item
  \textbf{સુરક્ષા}: ઓથેન્ટિકેશન, ઓથોરાઇઝેશન, એન્ક્રિપ્શન
\item
  \textbf{ઉપયોગિતા}: વપરાશકર્તા-મૈત્રીપૂર્ણ ઇન્ટરફેસ, પહોંચતા
\item
  \textbf{વિશ્વસનીયતા}: અપટાઇમ, એરર હેન્ડલિંગ
\item
  \textbf{જાળવણીયોગ્યતા}: કોડ ગુણવત્તા, દસ્તાવેજીકરણ
\end{itemize}

\end{solutionbox}
\begin{mnemonicbox}
``Performance Security Usability Reliability
Maintainability''

\end{mnemonicbox}
\begin{center}\rule{0.5\linewidth}{0.5pt}\end{center}

\subsection*{પ્રશ્ન 3(અ) OR [3
ગુણ]}\label{uxaaauxab0uxab6uxaa8-3uxa85-or-3-uxa97uxaa3}

\textbf{SRS નું મહત્વ દર્શાવો.}

\begin{solutionbox}

\textbf{SRS (સોફ્ટવૅર આવશ્યકતા વિશિષ્ટતા)} એ મહત્વપૂર્ણ દસ્તાવેજ છે જે સોફ્ટવૅર શું કરવું
જોઈએ તે વ્યાખ્યાયિત કરે છે.

\textbf{ટેબલ: SRS મહત્વ}

{\def\LTcaptype{none} % do not increment counter
\begin{longtable}[]{@{}ll@{}}
\toprule\noalign{}
પાસું & ફાયદો \\
\midrule\noalign{}
\endhead
\bottomrule\noalign{}
\endlastfoot
\textbf{સ્પષ્ટ સંવાદ} & બધા સ્ટેકહોલ્ડર્સ આવશ્યકતાઓ સમજે છે \\
\textbf{પ્રોજેક્ટ આયોજન} & અંદાજ અને શેડ્યુલિંગ માટે આધાર \\
\textbf{ગુણવત્તા ખાતરી} & પરીક્ષણ માટે પાયો \\
\textbf{ફેરફાર મેનેજમેન્ટ} & નિયંત્રિત આવશ્યકતા ફેરફારો \\
\textbf{કાનૂની સંરક્ષણ} & કરાર સંદર્ભ દસ્તાવેજ \\
\end{longtable}
}

\textbf{મુખ્ય મહત્વ}:

\begin{itemize}
\tightlist
\item
  \textbf{સંવાદ સાધન}: ક્લાયન્ટ્સ અને ડેવલપર્સ વચ્ચેનો સેતુ
\item
  \textbf{આયોજન પાયો}: સમય, ખર્ચ અને સાધનોનો અંદાજ કાઢવામાં મદદ કરે છે
\item
  \textbf{પરીક્ષણ આધાર}: SRS આવશ્યકતાઓમાંથી ટેસ્ટ કેસ મેળવવા
\end{itemize}

\end{solutionbox}
\begin{mnemonicbox}
``Clear Planning Quality Change Legal''

\end{mnemonicbox}
\begin{center}\rule{0.5\linewidth}{0.5pt}\end{center}

\subsection*{પ્રશ્ન 3(બ) OR [4
ગુણ]}\label{uxaaauxab0uxab6uxaa8-3uxaac-or-4-uxa97uxaa3}

\textbf{Gantt ચાર્ટ વિશે સમજાવો.}

\begin{solutionbox}

\textbf{Gantt ચાર્ટ} એ દ્રશ્ય પ્રોજેક્ટ મેનેજમેન્ટ ટૂલ છે જે કાર્યો, સમયસીમા અને
નિર્ભરતા દર્શાવે છે.

\begin{verbatim}
gantt
    title પ્રોજેક્ટ શેડ્યુલ
    dateFormat  YYYY{-MM{-}DD}
    section તબક્કો 1
    આવશ્યકતાઓ    :a1, 2024{-01{-}01, 30d}
    ડિઝાઇન         :a2, after a1, 20d
    section તબક્કો 2
    કોડિંગ         :a3, after a2, 45d
    પરીક્ષણ        :a4, after a3, 15d
\end{verbatim}

\textbf{ટેબલ: Gantt ચાર્ટ ઘટકો}

{\def\LTcaptype{none} % do not increment counter
\begin{longtable}[]{@{}ll@{}}
\toprule\noalign{}
ઘટક & વર્ણન \\
\midrule\noalign{}
\endhead
\bottomrule\noalign{}
\endlastfoot
\textbf{કાર્યો} & પૂર્ણ કરવાના કાર્ય આઇટમ્સ \\
\textbf{ટાઇમલાઇન} & આડી સમય સ્કેલ \\
\textbf{બાર્સ} & કાર્યની અવધિ અને પ્રગતિ \\
\textbf{નિર્ભરતા} & કાર્યો વચ્ચેના સંબંધો \\
\textbf{માઇલસ્ટોન્સ} & મહત્વપૂર્ણ પ્રોજેક્ટ ઇવેન્ટ્સ \\
\end{longtable}
}

\textbf{ફાયદા}:

\begin{itemize}
\tightlist
\item
  \textbf{દ્રશ્ય ટાઇમલાઇન}: પ્રોજેક્ટ શેડ્યુલ જોવા સરળ
\item
  \textbf{પ્રગતિ ટ્રેકિંગ}: કાર્ય પૂર્ણતાનું નિરીક્ષણ
\item
  \textbf{સાધન આયોજન}: સાધનોને અસરકારક રીતે ફાળવવા
\item
  \textbf{નિર્ભરતા મેનેજમેન્ટ}: કાર્ય સંબંધો સમજવા
\end{itemize}

\end{solutionbox}
\begin{mnemonicbox}
``Tasks Timeline Bars Dependencies Milestones''

\end{mnemonicbox}
\begin{center}\rule{0.5\linewidth}{0.5pt}\end{center}

\subsection*{પ્રશ્ન 3(ક) OR [7
ગુણ]}\label{uxaaauxab0uxab6uxaa8-3uxa95-or-7-uxa97uxaa3}

\textbf{રિસ્ક મેનેજમેન્ટ પર ટૂંક નોંધ લખો.}

\begin{solutionbox}

\textbf{રિસ્ક મેનેજમેન્ટ} એ પ્રોજેક્ટના જોખમોને ઓળખવા, વિશ્લેષણ કરવા અને નિયંત્રિત
કરવાની વ્યવસ્થિત પ્રક્રિયા છે.

\begin{center}
\textbf{Mermaid Diagram (Code)}
\begin{verbatim}
{Shaded}
{Highlighting}[]
graph LR
    A[જોખમ ઓળખ] {-{-}{} B[જોખમ વિશ્લેષણ]}
    B {-{-}{} C[જોખમ આયોજન]}
    C {-{-}{} D[જોખમ નિરીક્ષણ]}
    D {-{-}{} A}
{Highlighting}
{Shaded}
\end{verbatim}
\end{center}

\textbf{ટેબલ: રિસ્ક મેનેજમેન્ટ પ્રક્રિયા}

{\def\LTcaptype{none} % do not increment counter
\begin{longtable}[]{@{}lll@{}}
\toprule\noalign{}
તબક્કો & પ્રવૃત્તિઓ & આઉટપુટ \\
\midrule\noalign{}
\endhead
\bottomrule\noalign{}
\endlastfoot
\textbf{ઓળખ} & સંભવિત જોખમો શોધવા & જોખમ યાદી \\
\textbf{વિશ્લેષણ} & સંભાવના અને અસરનું મૂલ્યાંકન & જોખમ પ્રાથમિકતા \\
\textbf{આયોજન} & પ્રતિસાદ વ્યૂહરચના વિકસાવવી & જોખમ પ્રતિસાદ પ્લાન \\
\textbf{નિરીક્ષણ} & જોખમોને ટ્રેક અને નિયંત્રિત કરવા & અપડેટેડ જોખમ સ્થિતિ \\
\end{longtable}
}

\textbf{જોખમ કેટેગરીઓ}:

\textbf{ટેબલ: સોફ્ટવૅર જોખમોના પ્રકારો}

{\def\LTcaptype{none} % do not increment counter
\begin{longtable}[]{@{}ll@{}}
\toprule\noalign{}
કેટેગરી & ઉદાહરણો \\
\midrule\noalign{}
\endhead
\bottomrule\noalign{}
\endlastfoot
\textbf{તકનીકી} & ટેક્નોલોજી ફેરફારો, જટિલતા \\
\textbf{પ્રોજેક્ટ} & શેડ્યુલ વિલંબ, સાધન અછત \\
\textbf{બિઝનેસ} & બજાર ફેરફારો, ફંડિંગ મુદ્દાઓ \\
\textbf{બાહ્ય} & વિક્રેતા સમસ્યાઓ, નિયમનકારી ફેરફારો \\
\end{longtable}
}

\textbf{જોખમ પ્રતિસાદ વ્યૂહરચના}:

\begin{itemize}
\tightlist
\item
  \textbf{ટાળવું}: જોખમ સ્ત્રોતને દૂર કરવું
\item
  \textbf{ઘટાડવું}: સંભાવના અથવા અસર ઘટાડવી\\
\item
  \textbf{સ્થાનાંતરિત કરવું}: અન્ય લોકો સાથે જોખમ વહેંચવું
\item
  \textbf{સ્વીકારવું}: જોખમ સાથે જીવવું
\end{itemize}

\textbf{જોખમ મૂલ્યાંકન}: સંભાવના \times અસર = જોખમ એક્સપોઝર

\textbf{ફાયદા}: પ્રો-એક્ટિવ સમસ્યા ઉકેલ, વધુ સારી પ્રોજેક્ટ સફળતા દર, સ્ટેકહોલ્ડર
વિશ્વાસ

\end{solutionbox}
\begin{mnemonicbox}
``Identify Analyze Plan Monitor'' પ્રક્રિયા માટે,
``Avoid Mitigate Transfer Accept'' વ્યૂહરચના માટે

\end{mnemonicbox}
\begin{center}\rule{0.5\linewidth}{0.5pt}\end{center}

\subsection*{પ્રશ્ન 4(અ) [3
ગુણ]}\label{uxaaauxab0uxab6uxaa8-4uxa85-3-uxa97uxaa3}

\textbf{પ્રોજેક્ટની સાઇઝના અંદાજ માટેના મેટ્રિક શું છે? FP ઉદાહરણ સાથે સમજાવો.}

\begin{solutionbox}

\textbf{સાઇઝ અંદાજ મેટ્રિક્સ} સોફ્ટવૅર પ્રોજેક્ટના સાઇઝ અને પ્રયત્નોની આગાહી કરવામાં
મદદ કરે છે.

\textbf{ટેબલ: સાઇઝ અંદાજ મેટ્રિક્સ}

{\def\LTcaptype{none} % do not increment counter
\begin{longtable}[]{@{}ll@{}}
\toprule\noalign{}
મેટ્રિક & વર્ણન \\
\midrule\noalign{}
\endhead
\bottomrule\noalign{}
\endlastfoot
\textbf{LOC} & કોડની લાઇન્સ \\
\textbf{Function Points} & કાર્યક્ષમતા-આધારિત માપ \\
\textbf{Object Points} & ઑબ્જેક્ટ-ઓરિએન્ટેડ સિસ્ટમ્સ માટે \\
\textbf{Feature Points} & વિસ્તૃત Function Points \\
\end{longtable}
}

\textbf{Function Points (FP)} વપરાશકર્તા કાર્યક્ષમતાના આધારે સોફ્ટવૅર સાઇઝ માપે
છે.

\textbf{FP ઘટકો}:

\begin{itemize}
\tightlist
\item
  \textbf{External Inputs}: ડેટા એન્ટ્રી સ્ક્રીન્સ
\item
  \textbf{External Outputs}: રિપોર્ટ્સ, સંદેશાઓ\\
\item
  \textbf{External Queries}: ડેટાબેસ ક્વેરીઝ
\item
  \textbf{Internal Files}: ડેટા સ્ટોર્સ
\item
  \textbf{External Interfaces}: સિસ્ટમ કનેક્શન્સ
\end{itemize}

\textbf{FP ગણતરી ઉદાહરણ}: લાઇબ્રેરી મેનેજમેન્ટ સિસ્ટમ માટે:

\begin{itemize}
\tightlist
\item
  External Inputs: 5 (પુસ્તક એન્ટ્રી, સભ્ય એન્ટ્રી, વગેરે)
\item
  External Outputs: 3 (રિપોર્ટ્સ)
\item
  External Queries: 4 (સર્ચ ફંક્શન્સ)
\item
  Internal Files: 2 (પુસ્તક DB, સભ્ય DB)
\item
  External Interfaces: 1 (ઓનલાઇન કેટલોગ)
\end{itemize}

\textbf{સિમ્પલ FP = 5 + 3 + 4 + 2 + 1 = 15 Function Points}

\end{solutionbox}
\begin{mnemonicbox}
``Inputs Outputs Queries Files Interfaces''

\end{mnemonicbox}
\begin{center}\rule{0.5\linewidth}{0.5pt}\end{center}

\subsection*{પ્રશ્ન 4(બ) [4
ગુણ]}\label{uxaaauxab0uxab6uxaa8-4uxaac-4-uxa97uxaa3}

\textbf{પ્રોજેક્ટ અંદાજની બેસિક ટેકનિક COCOMO મોડેલ સમજાવો.}

\begin{solutionbox}

\textbf{COCOMO (COnstructive COst MOdel)} સોફ્ટવૅર ડેવલપમેન્ટ પ્રયત્ન અને શેડ્યુલનો
અંદાજ લગાવે છે.

\textbf{ટેબલ: COCOMO મોડેલ પ્રકારો}

{\def\LTcaptype{none} % do not increment counter
\begin{longtable}[]{@{}lll@{}}
\toprule\noalign{}
પ્રકાર & વર્ણન & ચોકસાઈ \\
\midrule\noalign{}
\endhead
\bottomrule\noalign{}
\endlastfoot
\textbf{બેસિક} & સરળ સાઇઝ-આધારિત અંદાજ & \pm75\% \\
\textbf{મધ્યવર્તી} & કોસ્ટ ડ્રાઇવર્સ સમાવે છે & \pm25\% \\
\textbf{વિગતવાર} & તબક્કા-સ્તરીય અંદાજ & \pm10\% \\
\end{longtable}
}

\textbf{બેસિક COCOMO ફોર્મુલા}:

\begin{itemize}
\tightlist
\item
  \textbf{પ્રયત્ન} = a \times (KLOC)\^{}b person-months
\item
  \textbf{સમય} = c \times (પ્રયત્ન)\^{}d months
\item
  \textbf{લોકો} = પ્રયત્ન / સમય
\end{itemize}

\textbf{ટેબલ: COCOMO કોન્સ્ટન્ટ્સ}

{\def\LTcaptype{none} % do not increment counter
\begin{longtable}[]{@{}lllll@{}}
\toprule\noalign{}
પ્રોજેક્ટ પ્રકાર & a & b & c & d \\
\midrule\noalign{}
\endhead
\bottomrule\noalign{}
\endlastfoot
\textbf{Organic} & 2.4 & 1.05 & 2.5 & 0.38 \\
\textbf{Semi-detached} & 3.0 & 1.12 & 2.5 & 0.35 \\
\textbf{Embedded} & 3.6 & 1.20 & 2.5 & 0.32 \\
\end{longtable}
}

\textbf{ઉદાહરણ}: 10 KLOC organic પ્રોજેક્ટ માટે

\begin{itemize}
\tightlist
\item
  પ્રયત્ન = 2.4 \times (10)\^{}1.05 = 25.47 person-months
\item
  સમય = 2.5 \times (25.47)\^{}0.38 = 8.64 months
\item
  લોકો = 25.47 / 8.64 = 3 લોકો
\end{itemize}

\end{solutionbox}
\begin{mnemonicbox}
``Organic Semi Embedded'' પ્રોજેક્ટ પ્રકારો માટે

\end{mnemonicbox}
\begin{center}\rule{0.5\linewidth}{0.5pt}\end{center}

\subsection*{પ્રશ્ન 4(ક) [7
ગુણ]}\label{uxaaauxab0uxab6uxaa8-4uxa95-7-uxa97uxaa3}

\textbf{તમારી પસંદગીની સિસ્ટમ માટે સ્પ્રિન્ટ બર્ન ડાઉન ચાર્ટ તૈયાર કરો.}

\begin{solutionbox}

\textbf{સ્પ્રિન્ટ બર્ન ડાઉન ચાર્ટ} \textbf{ઓનલાઇન શોપિંગ સિસ્ટમ} માટે સ્પ્રિન્ટ
દરમિયાન બાકી કામને ટ્રેક કરે છે.

\begin{center}
\textbf{Mermaid Diagram (Code)}
\begin{verbatim}
{Shaded}
{Highlighting}[]
graph LR
    A[સ્પ્રિન્ટ ગોલ: વપરાશકર્તા ઓથેન્ટિકેશન મોડ્યુલ{br/{}સ્પ્રિન્ટ અવધિ: 2 અઠવાડિયા{}br/{}કુલ સ્ટોરી પોઇન્ટ્સ: 40]}
{Highlighting}
{Shaded}
\end{verbatim}
\end{center}

\textbf{સ્પ્રિન્ટ બેકલોગ}:

\textbf{ટેબલ: સ્પ્રિન્ટ કાર્યો}

{\def\LTcaptype{none} % do not increment counter
\begin{longtable}[]{@{}lll@{}}
\toprule\noalign{}
કાર્ય & સ્ટોરી પોઇન્ટ્સ & દિવસ સોંપાયેલ \\
\midrule\noalign{}
\endhead
\bottomrule\noalign{}
\endlastfoot
\textbf{વપરાશકર્તા રજિસ્ટ્રેશન} & 8 & દિવસ 1-2 \\
\textbf{વપરાશકર્તા લોગિન} & 6 & દિવસ 3-4 \\
\textbf{પાસવર્ડ રીસેટ} & 5 & દિવસ 5-6 \\
\textbf{પ્રોફાઇલ મેનેજમેન્ટ} & 8 & દિવસ 7-8 \\
\textbf{સેશન મેનેજમેન્ટ} & 6 & દિવસ 9-10 \\
\textbf{ટેસ્ટિંગ અને બગ ફિક્સ} & 7 & દિવસ 11-14 \\
\end{longtable}
}

\textbf{બર્ન ડાઉન ચાર્ટ ડેટા}:

\textbf{ટેબલ: દૈનિક પ્રગતિ}

{\def\LTcaptype{none} % do not increment counter
\begin{longtable}[]{@{}llll@{}}
\toprule\noalign{}
દિવસ & આદર્શ બાકી & વાસ્તવિક બાકી & પૂર્ણ થયેલ કામ \\
\midrule\noalign{}
\endhead
\bottomrule\noalign{}
\endlastfoot
\textbf{દિવસ 0} & 40 & 40 & સ્પ્રિન્ટ શરૂઆત \\
\textbf{દિવસ 2} & 36 & 38 & રજિસ્ટ્રેશન વિલંબ \\
\textbf{દિવસ 4} & 32 & 32 & લોગિન પૂર્ણ \\
\textbf{દિવસ 6} & 28 & 27 & પાસવર્ડ રીસેટ જલ્દી પૂર્ણ \\
\textbf{દિવસ 8} & 24 & 26 & પ્રોફાઇલ મેનેજમેન્ટ મુદ્દાઓ \\
\textbf{દિવસ 10} & 20 & 20 & પાછા ટ્રેક પર \\
\textbf{દિવસ 12} & 16 & 15 & ટેસ્ટિંગ સારી પ્રગતિ \\
\textbf{દિવસ 14} & 0 & 0 & સ્પ્રિન્ટ પૂર્ણ \\
\end{longtable}
}

\textbf{ચાર્ટ વિશ્લેષણ}:

\begin{itemize}
\tightlist
\item
  \textbf{લીલી લાઇન}: આદર્શ બર્ન ડાઉન
\item
  \textbf{લાલ લાઇન}: વાસ્તવિક પ્રગતિ\\
\item
  \textbf{વિવિધતાઓ}: પડકારો અને પુનઃપ્રાપ્તિ દર્શાવે છે
\item
  \textbf{પૂર્ણતા}: સ્પ્રિન્ટ સમયસર પૂર્ણ થયું
\end{itemize}

\textbf{ફાયદા}: દ્રશ્ય પ્રગતિ ટ્રેકિંગ, પ્રારંભિક સમસ્યા ઓળખ, ટીમ પ્રેરણા

\end{solutionbox}
\begin{mnemonicbox}
``Track Progress Daily, Identify Issues Early''

\end{mnemonicbox}
\begin{center}\rule{0.5\linewidth}{0.5pt}\end{center}

\subsection*{પ્રશ્ન 4(અ) OR [3
ગુણ]}\label{uxaaauxab0uxab6uxaa8-4uxa85-or-3-uxa97uxaa3}

\textbf{USE CASE ડાયાગ્રામના ઘટકો સમજાવો.}

\begin{solutionbox}

\textbf{યુઝ કેસ ડાયાગ્રામ} વપરાશકર્તાના દૃષ્ટિકોણથી સિસ્ટમ કાર્યક્ષમતા દર્શાવે છે.

\textbf{ટેબલ: યુઝ કેસ ડાયાગ્રામ ઘટકો}

{\def\LTcaptype{none} % do not increment counter
\begin{longtable}[]{@{}lll@{}}
\toprule\noalign{}
ઘટક & સિમ્બોલ & વર્ણન \\
\midrule\noalign{}
\endhead
\bottomrule\noalign{}
\endlastfoot
\textbf{એક્ટર} & Stick figure & સિસ્ટમ સાથે વાતચીત કરતી બાહ્ય એન્ટિટી \\
\textbf{યુઝ કેસ} & ઓવલ & સિસ્ટમ કાર્યક્ષમતા \\
\textbf{સિસ્ટમ બાઉન્ડરી} & રેક્ટેંગલ & સિસ્ટમ અવકાશ \\
\textbf{એસોસિએશન} & લાઇન & એક્ટર-યુઝ કેસ સંબંધ \\
\textbf{જનરલાઇઝેશન} & એરો & વારસા સંબંધ \\
\end{longtable}
}

\textbf{સંબંધો}:

\begin{itemize}
\tightlist
\item
  \textbf{ઇન્ક્લૂડ}: એક યુઝ કેસ બીજાને સમાવે છે (ફરજિયાત)
\item
  \textbf{એક્સટેન્ડ}: વૈકલ્પિક યુઝ કેસ વિસ્તરણ
\item
  \textbf{જનરલાઇઝેશન}: માતા-પિતા-બાળક સંબંધ
\end{itemize}

\textbf{ઉદાહરણ ઘટકો}:

\begin{itemize}
\tightlist
\item
  \textbf{પ્રાથમિક એક્ટર}: ગ્રાહક, એડમિન
\item
  \textbf{યુઝ કેસ}: લોગિન, પ્રોડક્ટ્સ સર્ચ કરો, ઓર્ડર આપો
\item
  \textbf{સિસ્ટમ}: ઓનલાઇન શોપિંગ સિસ્ટમ
\end{itemize}

\end{solutionbox}
\begin{mnemonicbox}
``Actors Use Systems, Associate Generally''

\end{mnemonicbox}
\begin{center}\rule{0.5\linewidth}{0.5pt}\end{center}

\subsection*{પ્રશ્ન 4(બ) OR [4
ગુણ]}\label{uxaaauxab0uxab6uxaa8-4uxaac-or-4-uxa97uxaa3}

\textbf{કોહેસન અને કપલિંગની સરખામણી કરો.}

\begin{solutionbox}

\textbf{કોહેસન અને કપલિંગ} જાળવણીયોગ્યતાને અસર કરતા મહત્વપૂર્ણ સોફ્ટવૅર ડિઝાઇન
સિદ્ધાંતો છે.

\textbf{ટેબલ: કોહેસન વિ કપલિંગ સરખામણી}

{\def\LTcaptype{none} % do not increment counter
\begin{longtable}[]{@{}lll@{}}
\toprule\noalign{}
પાસું & કોહેસન & કપલિંગ \\
\midrule\noalign{}
\endhead
\bottomrule\noalign{}
\endlastfoot
\textbf{વ્યાખ્યા} & મોડ્યુલની અંદર એકતા & મોડ્યુલો વચ્ચે નિર્ભરતા \\
\textbf{ઇચ્છનીય સ્તર} & ઉચ્ચ કોહેસન પસંદ & નીચું કપલિંગ પસંદ \\
\textbf{ફોકસ} & આંતરિક મોડ્યુલ એકતા & આંતર-મોડ્યુલ સંબંધો \\
\textbf{અસર} & મોડ્યુલ વિશ્વસનીયતા & સિસ્ટમ લવચીકતા \\
\textbf{માપ} & મોડ્યુલ તત્વો કેટલા સંબંધિત છે & મોડ્યુલો કેટલા નિર્ભર છે \\
\end{longtable}
}

\textbf{કોહેસન પ્રકારો} (નીચાથી ઉચ્ચા સુધી):

\begin{itemize}
\tightlist
\item
  \textbf{સંયોગજન્ય}: રેન્ડમ ગ્રુપિંગ
\item
  \textbf{તાર્કિક}: સમાન લોજિક
\item
  \textbf{ટેમ્પોરલ}: સમાન સમય અમલ
\item
  \textbf{પ્રોસેજ્યોરલ}: ક્રમિક પગલાં
\item
  \textbf{કમ્યુનિકેશનલ}: સમાન ડેટા
\item
  \textbf{સિક્વેન્શિયલ}: એકનું આઉટપુટ બીજાનું ઇનપુટ
\item
  \textbf{ફંક્શનલ}: એક જ હેતુ
\end{itemize}

\textbf{કપલિંગ પ્રકારો} (ઉચ્ચાથી નીચા સુધી):

\begin{itemize}
\tightlist
\item
  \textbf{કન્ટેન્ટ}: મોડ્યુલ આંતરિક બાબતોને સીધો એક્સેસ
\item
  \textbf{કોમન}: વહેંચાયેલ ગ્લોબલ ડેટા
\item
  \textbf{એક્સટર્નલ}: વહેંચાયેલ બાહ્ય ઇન્ટરફેસ
\item
  \textbf{કન્ટ્રોલ}: કન્ટ્રોલ માહિતી પાસ
\item
  \textbf{સ્ટેમ્પ}: ડેટા સ્ટ્રક્ચર પાસ
\item
  \textbf{ડેટા}: સરળ ડેટા પાસ
\end{itemize}

\textbf{ગોલ}: \textbf{ઉચ્ચ કોહેસન + નીચું કપલિંગ = સારી ડિઝાઇન}

\end{solutionbox}
\begin{mnemonicbox}
``High Cohesion, Low Coupling'' સારી ડિઝાઇન માટે

\end{mnemonicbox}
\begin{center}\rule{0.5\linewidth}{0.5pt}\end{center}

\subsection*{પ્રશ્ન 4(ક) OR [7
ગુણ]}\label{uxaaauxab0uxab6uxaa8-4uxa95-or-7-uxa97uxaa3}

\textbf{રિસ્ક એસેસમેન્ટને વિસ્તારથી સમજાવો.}

\begin{solutionbox}

\textbf{રિસ્ક એસેસમેન્ટ} મેનેજમેન્ટ પ્રયત્નોને પ્રાથમિકતા આપવા માટે ઓળખાયેલા જોખમોનું
મૂલ્યાંકન કરે છે.

\begin{center}
\textbf{Mermaid Diagram (Code)}
\begin{verbatim}
{Shaded}
{Highlighting}[]
graph LR
    A[જોખમ ઓળખ] {-{-}{} B[જોખમ એસેસમેન્ટ]}
    B {-{-}{} C[સંભાવના વિશ્લેષણ]}
    B {-{-}{} D[અસર વિશ્લેષણ]}
    C {-{-}{} E[જોખમ એક્સપોઝર ગણતરી]}
    D {-{-}{} E}
    E {-{-}{} F[જોખમ પ્રાથમિકતા]}
{Highlighting}
{Shaded}
\end{verbatim}
\end{center}

\textbf{રિસ્ક એસેસમેન્ટ ઘટકો}:

\textbf{ટેબલ: જોખમ એસેસમેન્ટ તત્વો}

{\def\LTcaptype{none} % do not increment counter
\begin{longtable}[]{@{}lll@{}}
\toprule\noalign{}
તત્વ & વર્ણન & સ્કેલ \\
\midrule\noalign{}
\endhead
\bottomrule\noalign{}
\endlastfoot
\textbf{સંભાવના} & જોખમ થવાની શક્યતા & 0.1 થી 1.0 \\
\textbf{અસર} & જોખમ થાય તો પરિણામો & 1 થી 10 \\
\textbf{જોખમ એક્સપોઝર} & સંભાવના \times અસર & ગણતરીયુક્ત મૂલ્ય \\
\textbf{જોખમ સ્તર} & પ્રાથમિકતા વર્ગીકરણ & ઉચ્ચ/મધ્યમ/નીચું \\
\end{longtable}
}

\textbf{એસેસમેન્ટ પ્રક્રિયા}:

\textbf{1. સંભાવના એસેસમેન્ટ}:

\begin{itemize}
\tightlist
\item
  \textbf{ખૂબ નીચી (0.1)}: થવાની શક્યતા નથી
\item
  \textbf{નીચી (0.3)}: શક્ય પણ સંભવિત નથી\\
\item
  \textbf{મધ્યમ (0.5)}: થઈ શકે કે ન પણ થાય
\item
  \textbf{ઉચ્ચ (0.7)}: થવાની શક્યતા છે
\item
  \textbf{ખૂબ ઉચ્ચ (0.9)}: લગભગ નિશ્ચિત
\end{itemize}

\textbf{2. અસર એસેસમેન્ટ}:

\begin{itemize}
\tightlist
\item
  \textbf{વિનાશકારી (9-10)}: પ્રોજેક્ટ નિષ્ફળતા
\item
  \textbf{ગંભીર (7-8)}: મોટા વિલંબ/કોસ્ટ ઓવરરન
\item
  \textbf{સીમાંત (4-6)}: શેડ્યુલ/બજેટ પર થોડી અસર
\item
  \textbf{નગણ્ય (1-3)}: ઓછી અસર
\end{itemize}

\textbf{3. જોખમ એક્સપોઝર ગણતરી}: \textbf{જોખમ એક્સપોઝર = સંભાવના \times અસર}

\textbf{ઉદાહરણ જોખમ એસેસમેન્ટ}:

\textbf{ટેબલ: નમૂના જોખમ વિશ્લેષણ}

{\def\LTcaptype{none} % do not increment counter
\begin{longtable}[]{@{}lllll@{}}
\toprule\noalign{}
જોખમ & સંભાવના & અસર & એક્સપોઝર & પ્રાથમિકતા \\
\midrule\noalign{}
\endhead
\bottomrule\noalign{}
\endlastfoot
\textbf{મુખ્ય ડેવલપર છોડી જાય} & 0.3 & 8 & 2.4 & મધ્યમ \\
\textbf{આવશ્યકતા ફેરફાર} & 0.7 & 6 & 4.2 & ઉચ્ચ \\
\textbf{ટેક્નોલોજી નિષ્ફળતા} & 0.2 & 9 & 1.8 & નીચું \\
\textbf{બજેટ કાપ} & 0.4 & 7 & 2.8 & મધ્યમ \\
\end{longtable}
}

\textbf{રિસ્ક મેટ્રિક્સ}:

\begin{itemize}
\tightlist
\item
  \textbf{ઉચ્ચ પ્રાથમિકતા}: એક્સપોઝર \textgreater{} 4.0
\item
  \textbf{મધ્યમ પ્રાથમિકતા}: એક્સપોઝર 2.0-4.0\\
\item
  \textbf{નીચી પ્રાથમિકતા}: એક્સપોઝર \textless{} 2.0
\end{itemize}

\textbf{એસેસમેન્ટ ફાયદા}:

\begin{itemize}
\tightlist
\item
  \textbf{ઉદ્દેશ્ય પ્રાથમિકતા}: ડેટા-આધારિત નિર્ણયો
\item
  \textbf{સાધન ફાળવણી}: ઉચ્ચ-જોખમ આઇટમ્સ પર ફોકસ
\item
  \textbf{સંવાદ સાધન}: સ્પષ્ટ જોખમ સંવાદ
\item
  \textbf{આયોજન ઇનપુટ}: પ્રોજેક્ટ આયોજનને પ્રભાવિત કરે છે
\end{itemize}

\end{solutionbox}
\begin{mnemonicbox}
``Probability Impact Exposure Priority''

\end{mnemonicbox}
\begin{center}\rule{0.5\linewidth}{0.5pt}\end{center}

\subsection*{પ્રશ્ન 5(અ) [3
ગુણ]}\label{uxaaauxab0uxab6uxaa8-5uxa85-3-uxa97uxaa3}

\textbf{કોડ રિવ્યુની કોડ ઇન્સ્પેક્શન ટેકનિક સમજાવો.}

\begin{solutionbox}

\textbf{કોડ ઇન્સ્પેક્શન} એ ખામીઓ શોધવા માટે કોડની ઔપચારિક, વ્યવસ્થિત તપાસ છે.

\textbf{ટેબલ: કોડ ઇન્સ્પેક્શન પ્રક્રિયા}

{\def\LTcaptype{none} % do not increment counter
\begin{longtable}[]{@{}lll@{}}
\toprule\noalign{}
તબક્કો & સહભાગીઓ & પ્રવૃત્તિઓ \\
\midrule\noalign{}
\endhead
\bottomrule\noalign{}
\endlastfoot
\textbf{આયોજન} & મોડરેટર & ઇન્સ્પેક્શન શેડ્યુલ કરવું, કોડ વિતરિત કરવો \\
\textbf{ઓવરવ્યૂ} & લેખક, ટીમ & લેખક કોડ સમજાવે છે \\
\textbf{તૈયારી} & વ્યક્તિગત & દરેક રિવ્યુઅર કોડનો અભ્યાસ કરે છે \\
\textbf{ઇન્સ્પેક્શન} & બધા રિવ્યુઅર્સ & વ્યવસ્થિત રીતે ખામીઓ શોધવી \\
\textbf{રિવર્ક} & લેખક & ઓળખાયેલી ખામીઓ સુધારવી \\
\textbf{ફોલો-અપ} & મોડરેટર & સુધારાઓ ચકાસવા \\
\end{longtable}
}

\textbf{મુખ્ય લક્ષણો}:

\begin{itemize}
\tightlist
\item
  \textbf{ઔપચારિક પ્રક્રિયા}: વ્યાખ્યાયિત ભૂમિકાઓ સાથે માળખાગત અભિગમ
\item
  \textbf{વ્યવસ્થિત સમીક્ષા}: લાઇન-બાય-લાઇન તપાસ
\item
  \textbf{ખામી કેન્દ્રિત}: ભૂલો શોધવી, ઉકેલો નહીં
\item
  \textbf{લેખકની ટીકા નહીં}: કોડ પર ફોકસ, કોડર પર નહીં
\end{itemize}

\textbf{ફાયદા}: પ્રારંભિક ખામી શોધ, જ્ઞાન વહેંચણી, કોડ ગુણવત્તા સુધારણા

\end{solutionbox}
\begin{mnemonicbox}
``Plan Overview Prepare Inspect Rework Follow-up''

\end{mnemonicbox}
\begin{center}\rule{0.5\linewidth}{0.5pt}\end{center}

\subsection*{પ્રશ્ન 5(બ) [4
ગુણ]}\label{uxaaauxab0uxab6uxaa8-5uxaac-4-uxa97uxaa3}

\textbf{ATM ના ઓછામાં ઓછા ચાર ટેસ્ટ કેસ તૈયાર કરો.}

\begin{solutionbox}

\textbf{ATM ટેસ્ટ કેસ} ઓટોમેટેડ ટેલર મશીનની કાર્યક્ષમતા ચકાસે છે.

\textbf{ટેબલ: ATM ટેસ્ટ કેસ}

{\def\LTcaptype{none} % do not increment counter
\begin{longtable}[]{@{}
  >{\raggedright\arraybackslash}p{(\linewidth - 8\tabcolsep) * \real{0.2000}}
  >{\raggedright\arraybackslash}p{(\linewidth - 8\tabcolsep) * \real{0.2000}}
  >{\raggedright\arraybackslash}p{(\linewidth - 8\tabcolsep) * \real{0.2000}}
  >{\raggedright\arraybackslash}p{(\linewidth - 8\tabcolsep) * \real{0.2000}}
  >{\raggedright\arraybackslash}p{(\linewidth - 8\tabcolsep) * \real{0.2000}}@{}}
\toprule\noalign{}
\begin{minipage}[b]{\linewidth}\raggedright
ટેસ્ટ કેસ ID
\end{minipage} & \begin{minipage}[b]{\linewidth}\raggedright
ટેસ્ટ સિનેરિયો
\end{minipage} & \begin{minipage}[b]{\linewidth}\raggedright
ઇનપુટ
\end{minipage} & \begin{minipage}[b]{\linewidth}\raggedright
અપેક્ષિત આઉટપુટ
\end{minipage} & \begin{minipage}[b]{\linewidth}\raggedright
પરિણામ
\end{minipage} \\
\midrule\noalign{}
\endhead
\bottomrule\noalign{}
\endlastfoot
\textbf{TC001} & માન્ય PIN એન્ટ્રી & સાચો 4-અંકનો PIN & પ્રવેશ મંજૂર, મુખ્ય મેનુ
દર્શાવવું & Pass/Fail \\
\textbf{TC002} & અમાન્ય PIN એન્ટ્રી & ખોટો PIN (3 પ્રયાસ) & કાર્ડ બ્લોક, એરર
સંદેશ & Pass/Fail \\
\textbf{TC003} & રોકડ ઉપાડ & રકમ \leq ખાતા બેલેન્સ & રોકડ આપવી, રસીદ પ્રિન્ટ
કરવી & Pass/Fail \\
\textbf{TC004} & અપૂરતો બેલેન્સ & રકમ \textgreater{} ખાતા બેલેન્સ & વ્યવહાર
નકારવો, બેલેન્સ બતાવવો & Pass/Fail \\
\end{longtable}
}

\textbf{વિગતવાર ટેસ્ટ કેસ}:

\textbf{ટેસ્ટ કેસ 1: માન્ય લોગિન}

\begin{itemize}
\tightlist
\item
  \textbf{પૂર્વશરત}: ATM કાર્યરત છે, કાર્ડ દાખલ કર્યું
\item
  \textbf{પગલાં}: સાચો PIN દાખલ કરો \rightarrow Enter દબાવો
\item
  \textbf{અપેક્ષિત}: વિકલ્પો સાથે મુખ્ય મેનુ દર્શાવવું
\end{itemize}

\textbf{ટેસ્ટ કેસ 2: રોકડ ઉપાડ}

\begin{itemize}
\tightlist
\item
  \textbf{પૂર્વશરત}: વપરાશકર્તા લોગ ઇન, પૂરતો બેલેન્સ
\item
  \textbf{પગલાં}: ઉપાડ પસંદ કરો \rightarrow રકમ દાખલ કરો \rightarrow કન્ફર્મ કરો
\item
  \textbf{અપેક્ષિત}: રોકડ આપવી, બેલેન્સ અપડેટ કરવો
\end{itemize}

\textbf{ટેસ્ટ કેસ 3: બેલેન્સ પૂછપરછ}

\begin{itemize}
\tightlist
\item
  \textbf{પૂર્વશરત}: વપરાશકર્તા લોગ ઇન
\item
  \textbf{પગલાં}: બેલેન્સ પૂછપરછ પસંદ કરો
\item
  \textbf{અપેક્ષિત}: વર્તમાન બેલેન્સ સ્ક્રીન પર દર્શાવવો
\end{itemize}

\textbf{ટેસ્ટ કેસ 4: PIN ફેરફાર}

\begin{itemize}
\tightlist
\item
  \textbf{પૂર્વશરત}: વપરાશકર્તા લોગ ઇન
\item
  \textbf{પગલાં}: PIN ફેરફાર પસંદ કરો \rightarrow જૂનો PIN દાખલ કરો \rightarrow નવો PIN દાખલ કરો
  \rightarrow કન્ફર્મ કરો
\item
  \textbf{અપેક્ષિત}: PIN સફળતાપૂર્વક બદલાયો, પુષ્ટિ સંદેશ
\end{itemize}

\end{solutionbox}
\begin{mnemonicbox}
``Login Withdraw Inquiry Change''

\end{mnemonicbox}
\begin{center}\rule{0.5\linewidth}{0.5pt}\end{center}

\subsection*{પ્રશ્ન 5(ક) [7
ગુણ]}\label{uxaaauxab0uxab6uxaa8-5uxa95-7-uxa97uxaa3}

\textbf{white box ટેસ્ટિંગ વર્ણવો.}

\begin{solutionbox}

\textbf{વ્હાઇટ બોક્સ ટેસ્ટિંગ} આંતરિક કોડ માળખું અને લોજિક પાથ્સની તપાસ કરે છે.

\begin{center}
\textbf{Mermaid Diagram (Code)}
\begin{verbatim}
{Shaded}
{Highlighting}[]
graph LR
    A[સોર્સ કોડ] {-{-}{} B[કન્ટ્રોલ ફ્લો વિશ્લેષણ]}
    B {-{-}{} C[પાથ કવરેજ]}
    C {-{-}{} D[ટેસ્ટ કેસ ડિઝાઇન]}
    D {-{-}{} E[ટેસ્ટ એક્ઝિક્યુશન]}
    E {-{-}{} F[કવરેજ વિશ્લેષણ]}
{Highlighting}
{Shaded}
\end{verbatim}
\end{center}

\textbf{ટેબલ: વ્હાઇટ બોક્સ ટેસ્ટિંગ લાક્ષણિકતાઓ}

{\def\LTcaptype{none} % do not increment counter
\begin{longtable}[]{@{}ll@{}}
\toprule\noalign{}
પાસું & વર્ણન \\
\midrule\noalign{}
\endhead
\bottomrule\noalign{}
\endlastfoot
\textbf{ફોકસ} & આંતરિક કોડ માળખું \\
\textbf{જ્ઞાન} & કોડ અમલીકરણ વિગતો \\
\textbf{કવરેજ} & સ્ટેટમેન્ટ્સ, બ્રાન્ચ, પાથ્સ \\
\textbf{ટેકનિક્સ} & બેસિસ પાથ, લુપ ટેસ્ટિંગ \\
\textbf{ટૂલ્સ} & કોડ કવરેજ એનાલાઇઝર્સ \\
\end{longtable}
}

\textbf{કવરેજ માપદંડો}:

\textbf{ટેબલ: કવરેજ પ્રકારો}

{\def\LTcaptype{none} % do not increment counter
\begin{longtable}[]{@{}lll@{}}
\toprule\noalign{}
કવરેજ પ્રકાર & વર્ણન & ગોલ \\
\midrule\noalign{}
\endhead
\bottomrule\noalign{}
\endlastfoot
\textbf{સ્ટેટમેન્ટ કવરેજ} & દરેક સ્ટેટમેન્ટ એક્ઝિક્યુટ કરવું & 100\% સ્ટેટમેન્ટ્સ \\
\textbf{બ્રાન્ચ કવરેજ} & દરેક બ્રાન્ચ એક્ઝિક્યુટ કરવું & બધા if-else પાથ્સ \\
\textbf{પાથ કવરેજ} & દરેક પાથ એક્ઝિક્યુટ કરવું & બધા શક્ય પાથ્સ \\
\textbf{કન્ડિશન કવરેજ} & બધી શરતો ટેસ્ટ કરવી & દરેક કન્ડિશન માટે true/false \\
\end{longtable}
}

\textbf{વ્હાઇટ બોક્સ ટેસ્ટિંગ ટેકનિક્સ}:

\textbf{1. બેસિસ પાથ ટેસ્ટિંગ}:

\begin{itemize}
\tightlist
\item
  \textbf{સાયક્લોમેટિક કોમ્પ્લેક્સિટી} ગણવી: V(G) = E - N + 2
\item
E = એજ્સ,

N = કન્ટ્રોલ ફ્લો ગ્રાફમાં નોડ્સ

\item
  V(G) બરાબર સ્વતંત્ર પાથ્સ જનરેટ કરવા
\end{itemize}

\textbf{2. લુપ ટેસ્ટિંગ}:

\begin{itemize}
\tightlist
\item
  \textbf{સિમ્પલ લુપ્સ}: 0, 1, 2, સામાન્ય, મહત્તમ પુનરાવર્તનો ટેસ્ટ કરવા
\item
  \textbf{નેસ્ટેડ લુપ્સ}: પહેલા આંતરિક લુપ, પછી બાહ્ય
\item
  \textbf{કોન્કેટેનેટેડ લુપ્સ}: અલગ લુપ્સ તરીકે ટેસ્ટ કરવા
\end{itemize}

\textbf{3. કન્ડિશન ટેસ્ટિંગ}:

\begin{itemize}
\tightlist
\item
  બધી લોજિકલ કન્ડિશન્સ ટેસ્ટ કરવી (AND, OR, NOT)
\item
  દરેક કન્ડિશન true અને false બંને માટે મૂલ્યાંકન સુનિશ્ચિત કરવું
\end{itemize}

\textbf{ઉદાહરણ: સિમ્પલ કોડ ટેસ્ટિંગ}

\begin{verbatim}
if (age >= 18 AND income > 25000)
    approve_loan();
else
    reject_loan();
\end{verbatim}

\textbf{ટેસ્ટ કેસ}:

\begin{itemize}
\tightlist
\item
  age=20, income=30000 (બંને true) \rightarrow approve
\item
  age=16, income=30000 (પહેલું false) \rightarrow reject\\
\item
  age=20, income=20000 (બીજું false) \rightarrow reject
\item
  age=16, income=20000 (બંને false) \rightarrow reject
\end{itemize}

\textbf{ફાયદા}:

\begin{itemize}
\tightlist
\item
  \textbf{સંપૂર્ણ ટેસ્ટિંગ}: આંતરિક લોજિક ટેસ્ટ કરે છે
\item
  \textbf{પ્રારંભિક ખામી શોધ}: લોજિક એરર્સ શોધે છે
\item
  \textbf{કવરેજ માપ}: મૂર્ત ટેસ્ટિંગ પ્રગતિ
\end{itemize}

\textbf{ગેરફાયદા}:

\begin{itemize}
\tightlist
\item
  \textbf{સમય લેતું}: કોડ જ્ઞાનની જરૂર
\item
  \textbf{મોંઘું}: કુશળ ટેસ્ટર્સની જરૂર
\item
  \textbf{જાળવણી}: કોડ સાથે ફેરફારો
\end{itemize}

\textbf{ટૂલ્સ}: JUnit (Java), NUnit (.NET), Coverage.py (Python)

\end{solutionbox}
\begin{mnemonicbox}
``Statement Branch Path Condition'' કવરેજ પ્રકારો માટે

\end{mnemonicbox}
\begin{center}\rule{0.5\linewidth}{0.5pt}\end{center}

\subsection*{પ્રશ્ન 5(અ) OR [3
ગુણ]}\label{uxaaauxab0uxab6uxaa8-5uxa85-or-3-uxa97uxaa3}

\textbf{કોડ રિવ્યુની કોડ વોક થ્રુ ટેકનિક સમજાવો.}

\begin{solutionbox}

\textbf{કોડ વોક થ્રુ} એ અનૌપચારિક કોડ રિવ્યુ ટેકનિક છે જ્યાં લેખક ટીમને કોડ રજૂ કરે
છે.

\textbf{ટેબલ: વોક થ્રુ પ્રક્રિયા}

{\def\LTcaptype{none} % do not increment counter
\begin{longtable}[]{@{}lll@{}}
\toprule\noalign{}
તબક્કો & વર્ણન & અવધિ \\
\midrule\noalign{}
\endhead
\bottomrule\noalign{}
\endlastfoot
\textbf{તૈયારી} & લેખક પ્રેઝન્ટેશન તૈયાર કરે છે & 30 મિનિટ \\
\textbf{પ્રેઝન્ટેશન} & લેખક કોડ લોજિક સમજાવે છે & 1-2 કલાક \\
\textbf{ચર્ચા} & ટીમ પ્રશ્નો પૂછે છે, સુધારાઓ સૂચવે છે & 30 મિનિટ \\
\textbf{દસ્તાવેજીકરણ} & મુદ્દાઓ અને એક્શન આઇટમ્સ રેકોર્ડ કરવા & 15 મિનિટ \\
\end{longtable}
}

\textbf{મુખ્ય લાક્ષણિકતાઓ}:

\begin{itemize}
\tightlist
\item
  \textbf{લેખક-આગેવાની}: કોડ લેખક સેશન ચલાવે છે
\item
  \textbf{અનૌપચારિક પ્રક્રિયા}: ઇન્સ્પેક્શન કરતાં ઓછું માળખાગત
\item
  \textbf{શિક્ષણાત્મક}: ટીમ કોડ કાર્યક્ષમતા વિશે શીખે છે
\item
  \textbf{સહયોગી}: ખુલ્લી ચર્ચાને પ્રોત્સાહન
\end{itemize}

\textbf{સહભાગીઓ}:

\begin{itemize}
\tightlist
\item
  \textbf{લેખક}: કોડ રજૂ કરે છે અને સમજાવે છે
\item
  \textbf{રિવ્યુઅર્સ}: પ્રશ્નો પૂછે છે અને ફીડબેક આપે છે
\item
  \textbf{મોડરેટર}: ચર્ચાને કેન્દ્રિત રાખે છે (વૈકલ્પિક)
\end{itemize}

\textbf{ફાયદા}: જ્ઞાન વહેંચણી, પ્રારંભિક સમસ્યા શોધ, ટીમ સહયોગ, શીખવાની તક

\end{solutionbox}
\begin{mnemonicbox}
``Prepare Present Discuss Document''

\end{mnemonicbox}
\begin{center}\rule{0.5\linewidth}{0.5pt}\end{center}

\subsection*{પ્રશ્ન 5(બ) OR [4
ગુણ]}\label{uxaaauxab0uxab6uxaa8-5uxaac-or-4-uxa97uxaa3}

\textbf{સોફ્ટવૅર ડોક્યુમેન્ટેશન વિશે સમજાવો.}

\begin{solutionbox}

\textbf{સોફ્ટવૅર ડોક્યુમેન્ટેશન} વિવિધ સ્ટેકહોલ્ડર્સ માટે સોફ્ટવૅર સિસ્ટમ વિશે માહિતી
પ્રદાન કરે છે.

\textbf{ટેબલ: ડોક્યુમેન્ટેશન પ્રકારો}

{\def\LTcaptype{none} % do not increment counter
\begin{longtable}[]{@{}
  >{\raggedright\arraybackslash}p{(\linewidth - 4\tabcolsep) * \real{0.3333}}
  >{\raggedright\arraybackslash}p{(\linewidth - 4\tabcolsep) * \real{0.3333}}
  >{\raggedright\arraybackslash}p{(\linewidth - 4\tabcolsep) * \real{0.3333}}@{}}
\toprule\noalign{}
\begin{minipage}[b]{\linewidth}\raggedright
પ્રકાર
\end{minipage} & \begin{minipage}[b]{\linewidth}\raggedright
હેતુ
\end{minipage} & \begin{minipage}[b]{\linewidth}\raggedright
પ્રેક્ષકો
\end{minipage} \\
\midrule\noalign{}
\endhead
\bottomrule\noalign{}
\endlastfoot
\textbf{વપરાશકર્તા ડોક્યુમેન્ટેશન} & સોફ્ટવૅરનો ઉપયોગ કેવી રીતે કરવો & અંતિમ
વપરાશકર્તાઓ \\
\textbf{સિસ્ટમ ડોક્યુમેન્ટેશન} & તકનીકી વિગતો & ડેવલપર્સ, જાળવણીકર્તાઓ \\
\textbf{પ્રોસેસ ડોક્યુમેન્ટેશન} & વિકાસ પ્રક્રિયા & પ્રોજેક્ટ ટીમ \\
\textbf{આવશ્યકતા ડોક્યુમેન્ટેશન} & સિસ્ટમ શું કરવું જોઈએ & બધા સ્ટેકહોલ્ડર્સ \\
\end{longtable}
}

\textbf{આંતરિક ડોક્યુમેન્ટેશન}:

\begin{itemize}
\tightlist
\item
  \textbf{કોડ કોમેન્ટ્સ}: જટિલ લોજિક સમજાવવી
\item
  \textbf{ફંક્શન હેડર્સ}: હેતુ અને પેરામીટર્સ વર્ણવવા\\
\item
  \textbf{વેરિએબલ નામો}: સ્વ-દસ્તાવેજીકરણ ઓળખકર્તાઓ
\item
  \textbf{README ફાઇલ્સ}: પ્રોજેક્ટ ઓવરવ્યુ અને સેટઅપ
\end{itemize}

\textbf{બાહ્ય ડોક્યુમેન્ટેશન}:

\begin{itemize}
\tightlist
\item
  \textbf{વપરાશકર્તા માન્યુઅલ્સ}: ચરણ-દર-ચરણ ઉપયોગ સૂચનાઓ
\item
  \textbf{ઇન્સ્ટોલેશન ગાઇડ્સ}: સેટઅપ પ્રક્રિયાઓ
\item
  \textbf{API ડોક્યુમેન્ટેશન} : ઇન્ટરફેસ વિશિષ્ટતાઓ
\item
  \textbf{તાલીમ સામગ્રી}: શિક્ષણાત્મક સામગ્રી
\end{itemize}

\textbf{ફાયદા}:

\begin{itemize}
\tightlist
\item
  \textbf{જાળવણીયોગ્યતા}: કોડ અપડેટ્સ સરળ
\item
  \textbf{જ્ઞાન સ્થાનાંતરણ}: નવા ટીમ સભ્યો ઝડપથી શીખે છે
\item
  \textbf{વપરાશકર્તા સપોર્ટ}: સપોર્ટ વિનંતીઓ ઘટાડે છે
\item
  \textbf{ગુણવત્તા ખાતરી}: આવશ્યકતાઓ અને ડિઝાઇન દસ્તાવેજીકરણ કરે છે
\end{itemize}

\textbf{ડોક્યુમેન્ટેશન ધોરણો}: સુસંગત ફોર્મેટ, નિયમિત અપડેટ્સ, વર્ઝન કન્ટ્રોલ, પહોંચતા

\end{solutionbox}
\begin{mnemonicbox}
``User System Process Requirements'' પ્રકારો માટે

\end{mnemonicbox}
\begin{center}\rule{0.5\linewidth}{0.5pt}\end{center}

\subsection*{પ્રશ્ન 5(ક) OR [7
ગુણ]}\label{uxaaauxab0uxab6uxaa8-5uxa95-or-7-uxa97uxaa3}

\textbf{black box ટેસ્ટિંગ પર ટૂંક નોંધ લખો.}

\begin{solutionbox}

\textbf{બ્લેક બોક્સ ટેસ્ટિંગ} આંતરિક કોડ માળખાના જ્ઞાન વિના સોફ્ટવૅર કાર્યક્ષમતાની
તપાસ કરે છે.

\begin{center}
\textbf{Mermaid Diagram (Code)}
\begin{verbatim}
{Shaded}
{Highlighting}[]
graph LR
    A[ઇનપુટ] {-{-}{} B[સોફ્ટવૅર સિસ્ટમ{}br/{}બ્લેક બોક્સ] {-}{-}{} C[આઉટપુટ]}
    D[ટેસ્ટ કેસ] {-{-}{} A}
    E[અપેક્ષિત પરિણામો] {-{-}{} F[સરખામણી]}
    C {-{-}{} F}
{Highlighting}
{Shaded}
\end{verbatim}
\end{center}

\textbf{ટેબલ: બ્લેક બોક્સ ટેસ્ટિંગ લાક્ષણિકતાઓ}

{\def\LTcaptype{none} % do not increment counter
\begin{longtable}[]{@{}ll@{}}
\toprule\noalign{}
પાસું & વર્ણન \\
\midrule\noalign{}
\endhead
\bottomrule\noalign{}
\endlastfoot
\textbf{ફોકસ} & બાહ્ય વર્તન \\
\textbf{જ્ઞાન} & આવશ્યકતાઓ અને વિશિષ્ટતાઓ \\
\textbf{અભિગમ} & ઇનપુટ-આઉટપુટ સંબંધ \\
\textbf{કવરેજ} & કાર્યાત્મક આવશ્યકતાઓ \\
\textbf{દૃષ્ટિકોણ} & વપરાશકર્તા દૃષ્ટિકોણ \\
\end{longtable}
}

\textbf{બ્લેક બોક્સ ટેસ્ટિંગ ટેકનિક્સ}:

\textbf{ટેબલ: ટેસ્ટિંગ ટેકનિક્સ}

{\def\LTcaptype{none} % do not increment counter
\begin{longtable}[]{@{}
  >{\raggedright\arraybackslash}p{(\linewidth - 4\tabcolsep) * \real{0.3333}}
  >{\raggedright\arraybackslash}p{(\linewidth - 4\tabcolsep) * \real{0.3333}}
  >{\raggedright\arraybackslash}p{(\linewidth - 4\tabcolsep) * \real{0.3333}}@{}}
\toprule\noalign{}
\begin{minipage}[b]{\linewidth}\raggedright
ટેકનિક
\end{minipage} & \begin{minipage}[b]{\linewidth}\raggedright
વર્ણન
\end{minipage} & \begin{minipage}[b]{\linewidth}\raggedright
ઉદાહરણ
\end{minipage} \\
\midrule\noalign{}
\endhead
\bottomrule\noalign{}
\endlastfoot
\textbf{ઇક્વિવેલન્સ પાર્ટિશનિંગ} & ઇનપુટ્સને માન્ય/અમાન્ય વર્ગોમાં વહેંચવા & વય: 0-17,
18-65, \textgreater65 \\
\textbf{બાઉન્ડરી વેલ્યુ એનાલિસિસ} & સીમાઓ પર ટેસ્ટ કરવું & વય ટેસ્ટ: 17, 18, 65,
66 \\
\textbf{ડિસિઝન ટેબલ} & જટિલ બિઝનેસ નિયમો & ઇન્શ્યોરન્સ પ્રીમિયમ ગણતરી \\
\textbf{સ્ટેટ ટ્રાન્ઝિશન} & સિસ્ટમ સ્ટેટ ફેરફારો & ATM સ્ટેટ્સ: idle, processing,
error \\
\end{longtable}
}

\textbf{1. ઇક્વિવેલન્સ પાર્ટિશનિંગ}:

\begin{itemize}
\tightlist
\item
  \textbf{માન્ય પાર્ટિશન્સ}: સ્વીકૃત ઇનપુટ્સ
\item
  \textbf{અમાન્ય પાર્ટિશન્સ}: નકારેલા ઇનપુટ્સ
\item
  દરેક પાર્ટિશનમાંથી \textbf{એક વેલ્યુ ટેસ્ટ} કરવી
\end{itemize}

\textbf{ઉદાહરણ}: પાસવર્ડ લંબાઈ (6-12 અક્ષરો)

\begin{itemize}
\tightlist
\item
  માન્ય: 6-12 અક્ષરો
\item
  અમાન્ય: \textless6 અક્ષરો, \textgreater12 અક્ષરો
\end{itemize}

\textbf{2. બાઉન્ડરી વેલ્યુ એનાલિસિસ}:

\begin{itemize}
\tightlist
\item
  \textbf{લઘુત્તમ, મહત્તમ, લઘુત્તમથી થોડું નીચે, મહત્તમથી થોડું ઉપર} ટેસ્ટ કરવું
\item
  મોટાભાગની ભૂલો સીમાઓ પર થાય છે
\end{itemize}

\textbf{ઉદાહરણ}: રેન્જ 1-100 માટે

\begin{itemize}
\tightlist
\item
  ટેસ્ટ: 0, 1, 2, 99, 100, 101
\end{itemize}

\textbf{3. ડિસિઝન ટેબલ ટેસ્ટિંગ}:

\begin{itemize}
\tightlist
\item
  \textbf{કન્ડિશન્સ}: ઇનપુટ કન્ડિશન્સ
\item
  \textbf{એક્શન્સ}: અપેક્ષિત આઉટપુટ્સ
\item
  \textbf{નિયમો}: કન્ડિશન-એક્શન સંયોજનો
\end{itemize}

\textbf{ફાયદા}:

\begin{itemize}
\tightlist
\item
  \textbf{વપરાશકર્તા દૃષ્ટિકોણ}: વપરાશકર્તાના દૃષ્ટિકોણથી ટેસ્ટ કરે છે
\item
  \textbf{કોડ જ્ઞાનની જરૂર નથી}: ટેસ્ટર્સને પ્રોગ્રામિંગ સ્કિલ્સની જરૂર નથી
\item
  \textbf{નિષ્પક્ષ}: કોડ અમલીકરણથી પ્રભાવિત નથી
\item
  \textbf{પ્રારંભિક ટેસ્ટિંગ}: આવશ્યકતાઓ સાથે શરૂ કરી શકાય છે
\end{itemize}

\textbf{ગેરફાયદા}:

\begin{itemize}
\tightlist
\item
  \textbf{મર્યાદિત કવરેજ}: કેટલાક કોડ પાથ્સ ચૂકાવી શકે છે
\item
  \textbf{રિડન્ડન્ટ ટેસ્ટિંગ}: સમાન લોજિકને વધુ વખત ટેસ્ટ કરી શકે છે
\item
  \textbf{મુશ્કેલ ટેસ્ટ કેસ ડિઝાઇન}: આંતરિક જ્ઞાન વિના મુશ્કેલ
\end{itemize}

\textbf{બ્લેક બોક્સ ટેસ્ટિંગના પ્રકારો}:

\begin{itemize}
\tightlist
\item
  \textbf{ફંક્શનલ ટેસ્ટિંગ}: મુખ્ય કાર્યક્ષમતા
\item
  \textbf{નોન-ફંક્શનલ ટેસ્ટિંગ}: પ્રદર્શન, ઉપયોગિતા
\item
  \textbf{રીગ્રેશન ટેસ્ટિંગ}: ફેરફારો પછી
\item
  \textbf{યુઝર એક્સેપ્ટન્સ ટેસ્ટિંગ}: અંતિમ વેલિડેશન
\end{itemize}

\textbf{ટૂલ્સ}: Selenium (વેબ), Appium (મોબાઇલ), TestComplete, QTP

\textbf{ક્યારે ઉપયોગ કરવો}:

\begin{itemize}
\tightlist
\item
  સિસ્ટમ ટેસ્ટિંગ તબક્કો
\item
  યુઝર એક્સેપ્ટન્સ ટેસ્ટિંગ
\item
  ઇન્ટિગ્રેશન ટેસ્ટિંગ
\item
  રીગ્રેશન ટેસ્ટિંગ
\end{itemize}

\end{solutionbox}
\begin{mnemonicbox}
``Equivalence Boundary Decision State'' ટેકનિક્સ માટે

\end{mnemonicbox}

\end{document}
