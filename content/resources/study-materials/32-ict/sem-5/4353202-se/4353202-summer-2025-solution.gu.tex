\documentclass{article}

% content/resources/templates/preamble.tex
\usepackage[margin=0.6in]{geometry}
\author{Milav Dabgar}
\usepackage{amsmath,amssymb,amsthm}
\usepackage{booktabs}
\usepackage{multirow}
\usepackage{xcolor}
\usepackage{tcolorbox}
\tcbuselibrary{breakable,skins}
\usepackage[colorlinks=true,linkcolor=blue]{hyperref}
\usepackage{titlesec}
\usepackage{enumitem}
\usepackage{tikz}
\usepackage{pgfplots}
\usepackage{circuitikz}
\usepackage[version=4]{mhchem}
\usepackage{longtable}
\usepackage{array}
\usepackage{float}
\usepackage{caption}
\usepackage{listings}

\lstset{
  basicstyle=\small\ttfamily,
  breaklines=true,
  breakatwhitespace=false,
  postbreak=\mbox{\textcolor{red}{$\hookrightarrow$}\space},
  float=false,
  numbers=left,
  numberstyle=\tiny\color{gray},
  numbersep=10pt,
  xleftmargin=2em,
  keywordstyle=\color{blue},
  commentstyle=\color{green!60!black},
  stringstyle=\color{purple},
  backgroundcolor=\color{gray!5},
  showstringspaces=false,
  tabsize=2,
  captionpos=b,
  keepspaces=true,
  columns=flexible
}

\pgfplotsset{compat=1.18}
\usetikzlibrary{shapes,arrows,positioning,calc,patterns,decorations.pathmorphing,decorations.markings,arrows.meta}

% Color scheme
\definecolor{headcolor}{RGB}{0,102,204}
\definecolor{keycolor}{RGB}{220,20,60}
\definecolor{solutioncolor}{RGB}{34,139,34}
\definecolor{mnemoniccolor}{RGB}{148,0,211}
\definecolor{codecolor}{RGB}{0,0,100}

% Spacing
\setlength{\parskip}{3pt}
\setlist[itemize]{nosep}
\setlist[enumerate]{nosep}

% Title formatting
\titleformat{\section}{\Large\bfseries\color{headcolor}}{\thesection}{1em}{}
\titleformat{\subsection}{\large\bfseries\color{headcolor}}{\thesubsection}{1em}{}

% Pandoc tightlist compatibility
\providecommand{\tightlist}{%
  \setlength{\itemsep}{0pt}\setlength{\parskip}{0pt}}

% Pandoc longtable compatibility
\newcounter{none}
\def\thenone{}


% content/resources/templates/gujarati-boxes.tex
\usepackage{fontspec}
\usepackage{polyglossia}

% Set Gujarati as main language (document is primarily in Gujarati)
% Note: gloss-gujarati.ldf doesn't exist in polyglossia, but it will use hyphenation patterns
\setdefaultlanguage{gujarati}
\setotherlanguage{english}

% Configure Gujarati font properly
% Use Language=Default to prevent polyglossia from trying to add language-specific features
% that don't exist for Gujarati, which causes "empty feature" warnings
\newfontfamily\gujaratifont[Script=Gujarati,AutoFakeBold=2.5,AutoFakeSlant=0.3]{Noto Sans Gujarati}
\setmainfont[Script=Gujarati,AutoFakeBold=2.5,AutoFakeSlant=0.3]{Noto Sans Gujarati}
% Use Noto Sans Gujarati for monospace to support Gujarati in text
\setmonofont[Scale=0.9]{Noto Sans Gujarati}

% Configure English to use the same font
\newfontfamily\englishfont[Script=Gujarati,AutoFakeBold=2.5,AutoFakeSlant=0.3]{Noto Sans Gujarati}

% Translations for polyglossia
\gappto\captionsgujarati{
  \renewcommand{\tablename}{કોષ્ટક}
  \renewcommand{\figurename}{આકૃતિ}
}

% Helper for TikZ nodes to ensure Gujarati font
\newcommand{\gu}[1]{{\gujaratifont #1}}

% Custom environments
\newtcolorbox{solutionbox}{
    breakable,
    enhanced,
    colback=solutioncolor!5!white,
    colframe=solutioncolor!75!black,
    fonttitle=\bfseries,
    title=જવાબ
}

\newtcolorbox{solutionboxnobreak}{
 colback=solutioncolor!5!white,
 colframe=solutioncolor!75!black,
 fonttitle=\bfseries,
 title=જવાબ
}

\newtcolorbox{keyformula}{
 breakable,
 enhanced,
 colback=keycolor!5!white,
 colframe=keycolor!75!black,
 fonttitle=\bfseries,
 title=રાસાયણિક સમીકરણ/સૂત્ર
}

\newtcolorbox{mnemonicbox}{
 breakable,
 enhanced,
 colback=mnemoniccolor!5!white,
 colframe=mnemoniccolor!75!black,
 fonttitle=\bfseries,
 title=મેમરી ટ્રીક
}


% Custom commands for GTU solutions
% This file defines semantic commands for consistent formatting

% Question command with automatic formatting
\newcommand{\question}[2]{%
  \section*{Question #1}%
  \textbf{#2}%
}

% OR question variant
\newcommand{\questionor}[2]{%
  \section*{Question #1 OR}%
  \textbf{#2}%
}

% Proper table environment with caption
\newenvironment{answertable}[1]{%
  \begin{table}[htbp]
  \centering
  \caption{#1}
}{%
  \end{table}
}

% Proper figure environment for diagrams
\newenvironment{answerdiagram}[1]{%
  \begin{figure}[htbp]
  \centering
  \caption{#1}
}{%
  \end{figure}
}

% Semantic markup for key terms
\newcommand{\keyword}[1]{\textbf{#1}}
\newcommand{\code}[1]{\texttt{#1}}
\newcommand{\classname}[1]{\texttt{#1}}
\newcommand{\methodname}[1]{\texttt{#1}}

% Proper quotation marks
\newcommand{\mnemonic}[1]{``#1''}


\title{Software Engineering (4353202) - Summer 2025 Solution}
\date{May 14, 2025}

\begin{document}
\maketitle

\questionmarks{1(અ)}{3}{બધા જ પ્રકારના સૉફ્ટવેર એપ્લિકેશન ડોમેઇન ની યાદી બનાવો અને Embedded Software સમજાવો}

\begin{solutionbox}
\textbf{સૉફ્ટવેર એપ્લિકેશન ડોમેઇન:}

\begin{center}
\captionof{table}{સૉફ્ટવેર એપ્લિકેશન ડોમેઇન}
\begin{tabulary}{\linewidth}{|L|L|}
\hline
\textbf{ડોમેઇન} & \textbf{વર્ણન} \\ \hline
સિસ્ટમ સૉફ્ટવેર & ઓપરેટિંગ સિસ્ટમ, ડિવાઇસ ડ્રાઇવર \\ \hline
એપ્લિકેશન સૉફ્ટવેર & વર્ડ પ્રોસેસર, ગેમ્સ, બિઝનેસ એપ્સ \\ \hline
એન્જિનિયરિંગ/સાયન્ટિફિક સૉફ્ટવેર & CAD, સિમ્યુલેશન ટૂલ \\ \hline
એમ્બેડેડ સૉફ્ટવેર & રિયલ-ટાઇમ કંટ્રોલ સિસ્ટમ \\ \hline
વેબ એપ્લિકેશન & બ્રાઉઝર-આધારિત એપ્લિકેશન \\ \hline
AI સૉફ્ટવેર & મશીન લર્નિંગ, એક્સપર્ટ સિસ્ટમ \\ \hline
\end{tabulary}
\end{center}

\textbf{એમ્બેડેડ સૉફ્ટવેર} એ વિશેષ સૉફ્ટવેર છે જે ચોક્કસ હાર્ડવેર સાથે એમ્બેડેડ સિસ્ટમ પર ચાલે છે. આ વોશિંગ મશીન, કાર અને મેડિકલ ઉપકરણોમાં વપરાય છે.

\begin{itemize}
    \item \keyword{રિયલ-ટાઇમ ઓપરેશન}: નિર્ધારિત સમયમર્યાદામાં જવાબ આપવો જોઈએ
    \item \keyword{રિસોર્સ મર્યાદાઓ}: મર્યાદિત મેમરી અને પ્રોસેસિંગ પાવર
    \item \keyword{હાર્ડવેર પર નિર્ભરતા}: ચોક્કસ હાર્ડવેર સાથે ગાઢ એકીકરણ
\end{itemize}
\end{solutionbox}

\begin{mnemonicbox}
\mnemonic{SAEEWA: System, Application, Engineering, Embedded, Web, AI}
\end{mnemonicbox}

\questionmarks{1(બ)}{4}{જેનેરિક ફ્રેમવર્ક એક્ટિવિટીસ અને અમ્બ્રેલા એક્ટિવિટીસ સમજાવો}

\begin{solutionbox}
\textbf{જેનેરિક ફ્રેમવર્ક એક્ટિવિટીસ:}

\begin{center}
\captionof{table}{જેનેરિક ફ્રેમવર્ક એક્ટિવિટીસ}
\begin{tabulary}{\linewidth}{|L|L|}
\hline
\textbf{એક્ટિવિટી} & \textbf{હેતુ} \\ \hline
કોમ્યુનિકેશન & હિતધારકોથી જરૂરિયાતો એકત્રિત કરવી \\ \hline
પ્લાનિંગ & કાર્ય યોજના અને શેડ્યૂલ બનાવવું \\ \hline
મોડેલિંગ & વિશ્લેષણ અને ડિઝાઇન મોડેલ બનાવવા \\ \hline
કન્સ્ટ્રક્શન & કોડ જનરેશન અને ટેસ્ટિંગ \\ \hline
ડિપ્લોયમેન્ટ & સૉફ્ટવેર ડિલિવરી અને સપોર્ટ \\ \hline
\end{tabulary}
\end{center}

\textbf{અમ્બ્રેલા એક્ટિવિટીસ:}

\begin{center}
\captionof{table}{અમ્બ્રેલા એક્ટિવિટીસ}
\begin{tabulary}{\linewidth}{|L|L|}
\hline
\textbf{એક્ટિવિટી} & \textbf{હેતુ} \\ \hline
પ્રોજેક્ટ મેનેજમેન્ટ & પ્રગતિ ટ્રેક કરવી અને નિયંત્રણ \\ \hline
રિસ્ક મેનેજમેન્ટ & જોખમો ઓળખવા અને ઘટાડવા \\ \hline
ક્વોલિટી એશ્યોરન્સ & સૉફ્ટવેર ગુણવત્તા સુનિશ્ચિત કરવી \\ \hline
કન્ફિગરેશન મેનેજમેન્ટ & ફેરફારોને નિયંત્રિત કરવા \\ \hline
વર્ક પ્રોડક્ટ પ્રિપરેશન & દસ્તાવેજીકરણ બનાવવું \\ \hline
\end{tabulary}
\end{center}

\begin{itemize}
    \item \keyword{ફ્રેમવર્ક એક્ટિવિટીસ}: દરેક પ્રોજેક્ટમાં મુખ્ય ક્રમિક પ્રવૃત્તિઓ
    \item \keyword{અમ્બ્રેલા એક્ટિવિટીસ}: પ્રોજેક્ટ જીવનકાળ દરમિયાન સતત પ્રવૃત્તિઓ
\end{itemize}
\end{solutionbox}

\begin{mnemonicbox}
\mnemonic{CPMCD ફ્રેમવર્ક માટે, PRQCW અમ્બ્રેલા માટે}
\end{mnemonicbox}

\questionmarks{1(ક)}{7}{સૉફ્ટવેર ડેવલપમેંટ લાઇફ સાઇકલની આકૃતિ દોરી તેના તબક્કાઓ સમજાવો}

\begin{solutionbox}
\textbf{SDLC આકૃતિ:}

\begin{center}
\begin{tikzpicture}[node distance=2.5cm, auto]
    \node [gtu state] (req) {જરૂરિયાત\\વિશ્લેષણ};
    \node [gtu state, right of=req] (design) {સિસ્ટમ\\ડિઝાઇન};
    \node [gtu state, right of=design] (impl) {અમલીકરણ};
    \node [gtu state, below of=impl] (test) {ટેસ્ટિંગ};
    \node [gtu state, left of=test] (deploy) {ડિપ્લોયમેન્ટ};
    \node [gtu state, left of=deploy] (maint) {જાળવણી};
    
    \path [gtu arrow] (req) -- (design);
    \path [gtu arrow] (design) -- (impl);
    \path [gtu arrow] (impl) -- (test);
    \path [gtu arrow] (test) -- (deploy);
    \path [gtu arrow] (deploy) -- (maint);
    \path [gtu arrow] (maint) to [bend right=45] (req);
\end{tikzpicture}
\captionof{figure}{સૉફ્ટવેર ડેવલપમેન્ટ લાઇફ સાઇકલ}
\end{center}

\textbf{SDLC તબક્કાઓ:}

\begin{center}
\captionof{table}{SDLC તબક્કાઓ વિગતો}
\begin{tabulary}{\linewidth}{|L|L|L|}
\hline
\textbf{તબક્કો} & \textbf{પ્રવૃત્તિઓ} & \textbf{પરિણામો} \\ \hline
જરૂરિયાત વિશ્લેષણ & વપરાશકર્તા જરૂરિયાતો એકત્રિત કરવી, SRS બનાવવું & SRS દસ્તાવેજ \\ \hline
સિસ્ટમ ડિઝાઇન & આર્કિટેક્ચર ડિઝાઇન, UI ડિઝાઇન & ડિઝાઇન દસ્તાવેજ \\ \hline
અમલીકરણ & કોડ ડેવલપમેન્ટ, યુનિટ ટેસ્ટિંગ & સોર્સ કોડ \\ \hline
ટેસ્ટિંગ & એકીકરણ, સિસ્ટમ ટેસ્ટિંગ & ટેસ્ટ રિપોર્ટ \\ \hline
ડિપ્લોયમેન્ટ & ઇન્સ્ટોલેશન, વપરાશકર્તા તાલીમ & ડિપ્લોય થયેલ સિસ્ટમ \\ \hline
જાળવણી & બગ ફિક્સ, સુધારાઓ & અપડેટ થયેલ સિસ્ટમ \\ \hline
\end{tabulary}
\end{center}

\begin{itemize}
    \item \keyword{વ્યવસ્થિત અભિગમ}: દરેક તબક્કાના ચોક્કસ ઇનપુટ અને આઉટપુટ
    \item \keyword{ગુણવત્તા ગેટ}: તબક્કાઓ વચ્ચે સમીક્ષા ગુણવત્તા સુનિશ્ચિત કરે છે
    \item \keyword{પુનરાવર્તિત પ્રકૃતિ}: પ્રતિપુષ્ટિ આગામી ચક્રો સુધારે છે
\end{itemize}
\end{solutionbox}

\begin{mnemonicbox}
\mnemonic{વાસ્તવિક સિસ્ટમ અમલીકરણ ટેસ્ટ દરમિયાન જાળવણી}
\end{mnemonicbox}

\questionmarks{1(ક) OR}{7}{સોફ્ટવેર ડેવલપમેંટ મોડેલ્સની યાદી બનાવી કોઈ પણ બે મોડલ જરૂરી આકૃતિ સાથે સમજાવો}

\begin{solutionbox}
\textbf{સૉફ્ટવેર ડેવલપમેન્ટ મોડેલ્સ:}

\begin{center}
\captionof{table}{સૉફ્ટવેર ડેવલપમેન્ટ મોડેલ્સ}
\begin{tabulary}{\linewidth}{|L|L|}
\hline
\textbf{મોડેલ} & \textbf{લાક્ષણિકતાઓ} \\ \hline
વોટરફોલ મોડેલ & ક્રમિક, રેખીય અભિગમ \\ \hline
પુનરાવર્તિત મોડેલ & ડેવલપમેન્ટના પુનરાવર્તિત ચક્રો \\ \hline
સ્પાઇરલ મોડેલ & જોખમ-સંચાલિત, પુનરાવર્તિત \\ \hline
એજાઇલ મોડેલ & લવચીક, ગ્રાહક સહયોગ \\ \hline
RAD મોડેલ & ઝડપી પ્રોટોટાઇપિંગ \\ \hline
V-મોડેલ & વેરિફિકેશન અને વેલિડેશન પર ધ્યાન \\ \hline
\end{tabulary}
\end{center}

\textbf{1. વોટરફોલ મોડેલ:}

\begin{center}
\begin{tikzpicture}[node distance=2cm, auto]
    \node [gtu block] (req) {Requirements};
    \node [gtu block, right of=req] (design) {Design};
    \node [gtu block, right of=design] (impl) {Implementation};
    \node [gtu block, below of=impl] (test) {Testing};
    \node [gtu block, left of=test] (deploy) {Deployment};
    \node [gtu block, left of=deploy] (maint) {Maintenance};
    
    \path [gtu arrow] (req) -- (design);
    \path [gtu arrow] (design) -- (impl);
    \path [gtu arrow] (impl) -- (test);
    \path [gtu arrow] (test) -- (deploy);
    \path [gtu arrow] (deploy) -- (maint);
\end{tikzpicture}
\captionof{figure}{વોટરફોલ મોડેલ}
\end{center}

\textbf{2. સ્પાઇરલ મોડેલ:}

\begin{center}
\begin{tikzpicture}[node distance=2.5cm, auto]
    \node [gtu state] (plan) {Planning};
    \node [gtu state, right of=plan] (risk) {Risk\\Analysis};
    \node [gtu state, below of=risk] (eng) {Engineering};
    \node [gtu state, left of=eng] (eval) {Evaluation};
    
    \path [gtu arrow] (plan) -- (risk);
    \path [gtu arrow] (risk) -- (eng);
    \path [gtu arrow] (eng) -- (eval);
    \path [gtu arrow] (eval) -- (plan);
\end{tikzpicture}
\captionof{figure}{સ્પાઇરલ મોડેલ}
\end{center}

\begin{itemize}
    \item \keyword{વોટરફોલ}: સરળ, સારી રીતે સમજાયેલ જરૂરિયાતો માટે યોગ્ય
    \item \keyword{સ્પાઇરલ}: ઉચ્ચ જોખમવાળા પ્રોજેક્ટને પુનરાવર્તિત જોખમ મૂલ્યાંકન સાથે હેન્ડલ કરે છે
\end{itemize}
\end{solutionbox}

\begin{mnemonicbox}
\mnemonic{WIRRAV: Waterfall, Iterative, RAD, Risk-driven, Agile, V-model}
\end{mnemonicbox}


\questionmarks{2(અ)}{3}{SCRUM એજાઇલ પ્રોસેસ મોડલ અને SPIRAL પ્રોસેસ મોડલ વચ્ચેના તફાવત લખો}

\begin{solutionbox}
\begin{center}
\captionof{table}{SCRUM vs SPIRAL}
\begin{tabulary}{\linewidth}{|L|L|L|}
\hline
\textbf{પાસું} & \textbf{SCRUM} & \textbf{SPIRAL} \\ \hline
અભિગમ & એજાઇલ, પુનરાવર્તિત & જોખમ-સંચાલિત, પુનરાવર્તિત \\ \hline
અવધિ & નિશ્ચિત સ્પ્રિન્ટ (2-4 અઠવાડિયા) & ચલ સ્પાઇરલ ચક્રો \\ \hline
ધ્યાન & ગ્રાહક સહયોગ & જોખમ વ્યવસ્થાપન \\ \hline
આયોજન & સ્પ્રિન્ટ પ્લાનિંગ & વ્યાપક આયોજન \\ \hline
દસ્તાવેજીકરણ & ન્યૂનતમ દસ્તાવેજીકરણ & વિગતવાર દસ્તાવેજીકરણ \\ \hline
ટીમ સાઇઝ & નાની ટીમ (5-9 સભ્યો) & કોઈપણ ટીમ સાઇઝ \\ \hline
\end{tabulary}
\end{center}

\begin{itemize}
    \item \keyword{SCRUM}: ઝડપી ડિલિવરી અને ગ્રાહક પ્રતિપુષ્ટિ પર ભાર
    \item \keyword{SPIRAL}: જોખમ ઓળખ અને શમન પર ધ્યાન
\end{itemize}
\end{solutionbox}

\begin{mnemonicbox}
\mnemonic{SCRUM=સ્પીડ, SPIRAL=સેફ્ટી}
\end{mnemonicbox}

\questionmarks{2(બ)}{4}{જરૂરિયાત એકત્રીકરણ તકનીકોની યાદી આપો અને કોઇ પણ એક સમજાવો}

\begin{solutionbox}
\textbf{જરૂરિયાત એકત્રીકરણ તકનીકો:}

\begin{center}
\captionof{table}{જરૂરિયાત એકત્રીકરણ તકનીકો}
\begin{tabulary}{\linewidth}{|L|L|}
\hline
\textbf{તકનીક} & \textbf{વર્ણન} \\ \hline
ઇન્ટરવ્યુ & હિતધારકો સાથે સીધી વાતચીત \\ \hline
પ્રશ્નાવલી & માળખાગત લેખિત પ્રશ્નો \\ \hline
અવલોકન & વપરાશકર્તાઓને કાર્ય કરતા જોવા \\ \hline
દસ્તાવેજ વિશ્લેષણ & હાલના દસ્તાવેજોની સમીક્ષા \\ \hline
પ્રોટોટાઇપિંગ & કાર્યશીલ મોડેલ બનાવવા \\ \hline
બ્રેઇનસ્ટોર્મિંગ & ગ્રૂપ આઇડિયા જનરેશન \\ \hline
\end{tabulary}
\end{center}

\textbf{ઇન્ટરવ્યુ તકનીક સમજાવેલ:}

\begin{itemize}
    \item \keyword{માળખાગત ઇન્ટરવ્યુ}: પૂર્વનિર્ધારિત પ્રશ્નો, ઔપચારિક અભિગમ
    \item \keyword{અમાળખાગત ઇન્ટરવ્યુ}: ખુલ્લી ચર્ચા, લવચીક
    \item \keyword{અર્ધ-માળખાગત}: બંનેનું મિશ્રણ
\end{itemize}

\textbf{ફાયદાઓ}: સીધી હિતધારક ઇનપુટ, સ્પષ્ટીકરણ શક્ય, વિગતવાર માહિતી

\textbf{પડકારો}: સમય વપરાશ, ઇન્ટરવ્યુઅર પૂર્વગ્રહ, અધૂરી માહિતી
\end{solutionbox}

\begin{mnemonicbox}
\mnemonic{IQDPBB: Interview, Questionnaire, Document, Prototype, Brainstorm, Observe}
\end{mnemonicbox}

\questionmarks{2(ક)}{7}{યુઝ કેસ ડાયગ્રામ વ્યાખ્યાપિત કરો. તેને ઉદાહરણ સાથે સમજાવો}

\begin{solutionbox}
\textbf{યુઝ કેસ ડાયગ્રામ વ્યાખ્યા:}

યુઝ કેસ ડાયગ્રામ એક્ટર્સ અને તેમની યુઝ કેસ સાથેની ક્રિયાપ્રતિક્રિયા દર્શાવીને સિસ્ટમની કાર્યાત્મક જરૂરિયાતો બતાવે છે.

\textbf{ઘટકો:}

\begin{center}
\captionof{table}{યુઝ કેસ ડાયગ્રામ ઘટકો}
\begin{tabulary}{\linewidth}{|L|L|L|}
\hline
\textbf{ઘટક} & \textbf{પ્રતીક} & \textbf{હેતુ} \\ \hline
એક્ટર & લાકડી આકૃતિ & બાહ્ય એન્ટિટી \\ \hline
યુઝ કેસ & અંડાકાર & સિસ્ટમ ફંક્શન \\ \hline
એસોસિએશન & લાઇન & એક્ટર-યુઝ કેસ સંબંધ \\ \hline
સિસ્ટમ બાઉન્ડરી & લંબચોરસ & સિસ્ટમ સ્કોપ \\ \hline
\end{tabulary}
\end{center}

\textbf{ઉદાહરણ: લાઇબ્રેરી મેનેજમેન્ટ સિસ્ટમ}

\begin{center}
\begin{tikzpicture}[node distance=3cm, auto]
    % Actors (using simple shapes)
    \node[circle, draw, minimum size=0.4cm] (lib-head) at (0,0) {};
    \draw (0,-0.2) -- (0,-0.8);
    \draw (-0.3,-0.5) -- (0.3,-0.5);
    \draw (0,-0.8) -- (-0.3,-1.3);
    \draw (0,-0.8) -- (0.3,-1.3);
    \node[below=1.5cm of lib-head] {લાઇબ્રેરિયન};
    
    \node[circle, draw, minimum size=0.4cm] (stu-head) at (0,-4) {};
    \draw (0,-4.2) -- (0,-4.8);
    \draw (-0.3,-4.5) -- (0.3,-4.5);
    \draw (0,-4.8) -- (-0.3,-5.3);
    \draw (0,-4.8) -- (0.3,-5.3);
    \node[below=1.5cm of stu-head] {વિદ્યાર્થી};
    
    % Use cases
    \node [draw, ellipse, minimum width=2.5cm, minimum height=1cm, right=3cm of lib-head, yshift=0.5cm] (issue) {પુસ્તક ઇશ્યૂ કરવું};
    \node [draw, ellipse, minimum width=2.5cm, minimum height=1cm, below=0.5cm of issue] (return) {પુસ્તક પરત કરવું};
    \node [draw, ellipse, minimum width=2.5cm, minimum height=1cm, below=0.5cm of return] (add) {પુસ્તક ઉમેરવું};
    \node [draw, ellipse, minimum width=2.5cm, minimum height=1cm, below=0.5cm of add] (search) {પુસ્તક શોધવું};
    
    % Associations
    \draw (lib-head) -- (issue);
    \draw (lib-head) -- (return);
    \draw (lib-head) -- (add);
    \draw (stu-head) -- (issue);
    \draw (stu-head) -- (return);
    \draw (stu-head) -- (search);
\end{tikzpicture}
\captionof{figure}{લાઇબ્રેરી મેનેજમેન્ટ સિસ્ટમ યુઝ કેસ ડાયગ્રામ}
\end{center}

\textbf{સંબંધો:}

\begin{itemize}
    \item \keyword{Include}: યુઝ કેસ દ્વારા શેર કરાયેલ સામાન્ય કાર્યક્ષમતા
    \item \keyword{Extend}: બેઝ યુઝ કેસમાં વૈકલ્પિક કાર્યક્ષમતા ઉમેરવી
    \item \keyword{સામાન્યીકરણ}: એક્ટર્સ અથવા યુઝ કેસ વચ્ચે વારસો
\end{itemize}

\textbf{ફાયદાઓ}: સ્પષ્ટ કાર્યાત્મક ઝાંખી, કોમ્યુનિકેશન ટૂલ, ટેસ્ટિંગ માટે આધાર
\end{solutionbox}

\begin{mnemonicbox}
\mnemonic{એક્ટર્સ યુઝ કેસ સિસ્ટમની અંદર}
\end{mnemonicbox}

\questionmarks{2(અ) OR}{3}{વોટર ફોલ મોડલ અને ઈટરેટિવ વોટર ફોલ મોડલ ની સરખામણી કરો}

\begin{solutionbox}
\begin{center}
\captionof{table}{વોટરફોલ vs ઇટરેટિવ વોટરફોલ}
\begin{tabulary}{\linewidth}{|L|L|L|}
\hline
\textbf{પાસું} & \textbf{વોટરફોલ મોડેલ} & \textbf{ઇટરેટિવ વોટરફોલ} \\ \hline
તબક્કાઓ & ક્રમિક, એક વખત & પુનરાવર્તનમાં પુનરાવૃત્તિ \\ \hline
પ્રતિપુષ્ટિ & પ્રોજેક્ટના અંતે & દરેક પુનરાવર્તન પછી \\ \hline
જોખમ & મોડેથી જોખમ ઓળખ & વહેલી જોખમ ઓળખ \\ \hline
લવચીકતા & કઠોર, કોઈ ફેરફાર નહીં & ફેરફારોને સમાવે છે \\ \hline
ટેસ્ટિંગ & ડેવલપમેન્ટ પછી & સતત ટેસ્ટિંગ \\ \hline
ડિલિવરી & એક અંતિમ ડિલિવરી & બહુવિધ વૃદ્ધિશીલ ડિલિવરી \\ \hline
\end{tabulary}
\end{center}

\begin{itemize}
    \item \keyword{વોટરફોલ}: સ્થિર, સારી રીતે વ્યાખ્યાયિત જરૂરિયાતો માટે યોગ્ય
    \item \keyword{ઇટરેટિવ વોટરફોલ}: પ્રતિપુષ્ટિ સાથે વિકસિત જરૂરિયાતો માટે બહેતર
\end{itemize}
\end{solutionbox}

\begin{mnemonicbox}
\mnemonic{PFRTFD: Phases, Feedback, Risk, Testing, Flexibility, Delivery}
\end{mnemonicbox}

\questionmarks{2(બ) OR}{4}{ફંકશનલ અને નોન-ફંકશનલ જરૂરિયાતની વ્યાખ્યા લખી બંનેના ઉદાહરણ આપો}

\begin{solutionbox}
\textbf{ફંકશનલ જરૂરિયાતો:}

સિસ્ટમે શું કરવું જોઈએ - ચોક્કસ વર્તણૂકો અને કાર્યોને વ્યાખ્યાયિત કરતી જરૂરિયાતો.

\textbf{નોન-ફંકશનલ જરૂરિયાતો:}

સિસ્ટમ કેવી રીતે કાર્ય કરે છે - ગુણવત્તા લક્ષણો અને મર્યાદાઓને વ્યાખ્યાયિત કરતી જરૂરિયાતો.

\begin{center}
\captionof{table}{ફંકશનલ vs નોન-ફંકશનલ જરૂરિયાતો}
\begin{tabulary}{\linewidth}{|L|L|L|}
\hline
\textbf{પ્રકાર} & \textbf{ફંકશનલ} & \textbf{નોન-ફંકશનલ} \\ \hline
વ્યાખ્યા & સિસ્ટમ વર્તણૂક & સિસ્ટમ ગુણવત્તા \\ \hline
ઉદાહરણો & લોગિન, ગણતરી, સંગ્રહ & પ્રદર્શન, સુરક્ષા \\ \hline
ટેસ્ટિંગ & બ્લેક-બોક્સ ટેસ્ટિંગ & લોડ, સ્ટ્રેસ ટેસ્ટિંગ \\ \hline
દસ્તાવેજીકરણ & યુઝ કેસ, દૃશ્યો & ગુણવત્તા મેટ્રિક્સ \\ \hline
\end{tabulary}
\end{center}

\textbf{ફંકશનલ ઉદાહરણો:}

\begin{itemize}
    \item વપરાશકર્તા પ્રમાણીકરણ અને લોગિન
    \item કુલ બિલ રકમની ગણતરી કરવી
    \item માસિક રિપોર્ટ જનરેટ કરવી
\end{itemize}

\textbf{નોન-ફંકશનલ ઉદાહરણો:}

\begin{itemize}
    \item સિસ્ટમ રિસ્પોન્સ ટાઇમ $<$ 2 સેકન્ડ (પ્રદર્શન)
    \item 99.9\% સિસ્ટમ ઉપલબ્ધતા (વિશ્વસનીયતા)
    \item 1000 સમવર્તી વપરાશકર્તાઓને સપોર્ટ (સ્કેલેબિલિટી)
\end{itemize}
\end{solutionbox}

\begin{mnemonicbox}
\mnemonic{ફંકશનલ=શું, નોન-ફંકશનલ=કેવી રીતે}
\end{mnemonicbox}

\questionmarks{2(ક) OR}{7}{કોહેશનની વ્યાખ્યા આપો. કોહેશનનું વર્ગીકરણ સમજાવો}

\begin{solutionbox}
\textbf{કોહેશન વ્યાખ્યા:}

કોહેશન માપે છે કે મોડ્યુલની અંદરના ત્તત્વો કેટલા નજીકથી સંબંધિત છે. ઉચ્ચ કોહેશન સારી રીતે ડિઝાઇન કરાયેલ મોડ્યુલ દર્શાવે છે.

\textbf{કોહેશનનું વર્ગીકરણ (સૌથી મજબૂતથી સૌથી નબળું):}

\begin{center}
\captionof{table}{કોહેશન પ્રકારો}
\begin{tabulary}{\linewidth}{|L|L|L|}
\hline
\textbf{પ્રકાર} & \textbf{વર્ણન} & \textbf{ઉદાહરણ} \\ \hline
ફંકશનલ & એક, સારી રીતે વ્યાખ્યાયિત કાર્ય & વર્ગમૂળ ગણતરી \\ \hline
સિક્વન્શિયલ & એકનું આઉટપુટ = બીજાનું ઇનપુટ & વાંચવું→પ્રોસેસ કરવું→લખવું \\ \hline
કોમ્યુનિકેશનલ & સમાન ડેટા પર કામ કરવું & ગ્રાહક રેકોર્ડ અપડેટ \\ \hline
પ્રોસિજરલ & અમલીકરણનો ક્રમ અનુસરવો & પેરોલ પ્રોસેસિંગ સ્ટેપ્સ \\ \hline
ટેમ્પોરલ & સમાન સમયે અમલ & સિસ્ટમ પ્રારંભીકરણ \\ \hline
લોજિકલ & સમાન લોજિકલ ફંક્શન & બધા ઇનપુટ/આઉટપુટ ઓપરેશન \\ \hline
કોઇન્સિડેન્ટલ & કોઈ અર્થપૂર્ણ સંબંધ નહીં & રેન્ડમ યુટિલિટીઝ \\ \hline
\end{tabulary}
\end{center}

\begin{center}
\begin{tikzpicture}[node distance=2cm, auto]
    \node [gtu block] (func) {ફંકશનલ\\સૌથી મજબૂત};
    \node [gtu block, right of=func] (seq) {સિક્વન્શિયલ};
    \node [gtu block, right of=seq] (comm) {કોમ્યુનિકેશનલ};
    \node [gtu block, below of=comm] (proc) {પ્રોસિજરલ};
    \node [gtu block, left of=proc] (temp) {ટેમ્પોરલ};
    \node [gtu block, left of=temp] (log) {લોજિકલ};
    \node [gtu block, below of=log] (coin) {કોઇન્સિડેન્ટલ\\સૌથી નબળું};
    
    \path [gtu arrow] (func) -- (seq);
    \path [gtu arrow] (seq) -- (comm);
    \path [gtu arrow] (comm) -- (proc);
    \path [gtu arrow] (proc) -- (temp);
    \path [gtu arrow] (temp) -- (log);
    \path [gtu arrow] (log) -- (coin);
\end{tikzpicture}
\captionof{figure}{કોહેશન વર્ગીકરણ હાયરાર્કી}
\end{center}

\textbf{લક્ષ્ય}: જાળવણીયોગ્ય, વિશ્વસનીય મોડ્યુલ માટે ફંકશનલ કોહેશન હાંસલ કરવું
\end{solutionbox}

\begin{mnemonicbox}
\mnemonic{ફ્રેન્કની સ્માર્ટ બિલાડી ટેનિસ લોજિકલી રમે છે}
\end{mnemonicbox}

\questionmarks{3(અ)}{3}{સારા સોફ્ટવેર ડિઝાઇનની લાક્ષણિકતાઓની યાદી બનાવો}

\begin{solutionbox}
\textbf{સારા સૉફ્ટવેર ડિઝાઇનની લાક્ષણિકતાઓ:}

\begin{center}
\captionof{table}{સારા ડિઝાઇનની લાક્ષણિકતાઓ}
\begin{tabulary}{\linewidth}{|L|L|}
\hline
\textbf{લાક્ષણિકતા} & \textbf{વર્ણન} \\ \hline
મોડ્યુલારિટી & સ્વતંત્ર મોડ્યુલમાં વિભાજિત \\ \hline
એબ્સ્ટ્રેક્શન & અમલીકરણ વિગતો છુપાવવી \\ \hline
એન્કેપ્સ્યુલેશન & ડેટા અને મેથડ્સ એકસાથે બંડલ કરવા \\ \hline
હાયરાર્કી & સ્તરો/લેવલમાં સંગઠિત \\ \hline
સરળતા & સમજવામાં અને જાળવવામાં સરળ \\ \hline
લવચીકતા & ભવિષ્યના ફેરફારોને સમાવવા \\ \hline
\end{tabulary}
\end{center}

\begin{itemize}
    \item \keyword{ઉચ્ચ કોહેશન}: સંબંધિત ત્તત્વો એકસાથે જૂથબદ્ધ
    \item \keyword{નીચું કપલિંગ}: મોડ્યુલ વચ્ચે ન્યૂનતમ નિર્ભરતાઓ
    \item \keyword{પુનઃઉપયોગિતા}: ઘટકોને અન્ય સિસ્ટમમાં ફરીથી વાપરી શકાય
\end{itemize}
\end{solutionbox}

\begin{mnemonicbox}
\mnemonic{MAEHSF: Modularity, Abstraction, Encapsulation, Hierarchy, Simplicity, Flexibility}
\end{mnemonicbox}

\questionmarks{3(બ)}{4}{ઈંટરમીડીયેટ COCOMO મોડલ દ્વારા પ્રોજેક્ટ એસ્ટીમેશન પધ્ધતિ સમજાવો}

\begin{solutionbox}
\textbf{ઇન્ટરમીડિયેટ COCOMO મોડેલ:}

ઉત્પાદકતાને અસર કરતા કોસ્ટ ડ્રાઇવરોને ધ્યાનમાં લઈને બેઝિક COCOMO ને વિસ્તૃત કરે છે.

\textbf{સૂત્ર:}

Effort = a $\times$ (KLOC)$^b$ $\times$ EAF

\textbf{કોસ્ટ ડ્રાઇવર્સ:}

\begin{center}
\captionof{table}{કોસ્ટ ડ્રાઇવર કેટેગરીઝ}
\begin{tabulary}{\linewidth}{|L|L|L|}
\hline
\textbf{કેટેગરી} & \textbf{ડ્રાઇવર્સ} & \textbf{પ્રભાવ} \\ \hline
પ્રોડક્ટ & વિશ્વસનીયતા, જટિલતા & પ્રયત્ન ગુણક \\ \hline
હાર્ડવેર & એક્ઝિક્યુશન ટાઇમ, સ્ટોરેજ & પ્રદર્શન મર્યાદાઓ \\ \hline
કર્મચારીવર્ગ & વિશ્લેષક ક્ષમતા, અનુભવ & ટીમ કુશળતા \\ \hline
પ્રોજેક્ટ & આધુનિક પ્રથાઓ, શેડ્યૂલ & ડેવલપમેન્ટ વાતાવરણ \\ \hline
\end{tabulary}
\end{center}

\textbf{પ્રયત્ન સમાયોજન ફેક્ટર (EAF):}

EAF = બધા કોસ્ટ ડ્રાઇવર ગુણકોનું ગુણાકાર

\textbf{પગલાં:}

\begin{enumerate}
    \item KLOC (કોડની હજારો લાઇન) નો અંદાજ કાઢવો
    \item પ્રોજેક્ટ પ્રકાર આધારે યોગ્ય a, b મૂલ્યો પસંદ કરવા
    \item કોસ્ટ ડ્રાઇવર્સનું મૂલ્યાંકન (સ્કેલ 0.70 થી 1.65)
    \item EAF ની ગણતરી કરવી
    \item પર્સન-મંથમાં પ્રયત્ન મેળવવા માટે સૂત્ર લાગુ કરવું
\end{enumerate}
\end{solutionbox}

\begin{mnemonicbox}
\mnemonic{PHPP: Product, Hardware, Personnel, Project drivers}
\end{mnemonicbox}

\questionmarks{3(ક)}{7}{ઓનલાઇન શોપિંગ સિસ્ટમ માટે લેવલ-1 નો ડેટા ફ્લો ડાયગ્રામ દોરો અને સમજાવો}

\begin{solutionbox}
\textbf{ઓનલાઇન શોપિંગ સિસ્ટમ માટે લેવલ-1 DFD:}

\begin{center}
\begin{tikzpicture}[node distance=3cm, auto]
    % External entities
    \node [draw, rectangle] (customer) {ગ્રાહક};
    \node [draw, rectangle, below=4cm of customer] (payment) {પેમેન્ટ\\ગેટવે};
    \node [draw, rectangle, below=8cm of customer] (inventory) {ઇન્વેન્ટરી\\મેનેજર};
    
    % Processes
    \node [draw, circle, minimum size=1.5cm, right=4cm of customer] (p1) {1\\ઓર્ડર\\પ્રોસેસ};
    \node [draw, circle, minimum size=1.5cm, below=2cm of p1] (p2) {2\\પેમેન્ટ\\પ્રોસેસ};
    \node [draw, circle, minimum size=1.5cm, below=2cm of p2] (p3) {3\\ઇન્વેન્ટરી\\મેનેજ};
    
    % Data flows
    \draw [->] (customer) -- node[above] {ઓર્ડર માહિતી} (p1);
    \draw [->] (p1) -- node[below] {પ્રોડક્ટ માહિતી} (customer);
    \draw [->] (p1) -- node[right] {ઓર્ડર વિગતો} (p2);
    \draw [->] (payment) -- node[above] {પેમેન્ટ માહિતી} (p2);
    \draw [->] (p2) -- node[right] {ઇન્વેન્ટરી અપડેટ} (p3);
    \draw [->] (inventory) -- node[above] {સ્ટોક માહિતી} (p3);
\end{tikzpicture}
\captionof{figure}{ઓનલાઇન શોપિંગ સિસ્ટમ DFD લેવલ-1}
\end{center}

\textbf{પ્રોસેસ:}

\begin{center}
\captionof{table}{DFD પ્રોસેસ વિગતો}
\begin{tabulary}{\linewidth}{|L|L|L|L|}
\hline
\textbf{પ્રોસેસ} & \textbf{ઇનપુટ} & \textbf{આઉટપુટ} & \textbf{વર્ણન} \\ \hline
ઓર્ડર પ્રોસેસ & ગ્રાહક ઓર્ડર & ઓર્ડર પુષ્ટિકરણ & ઓર્ડર પ્લેસમેન્ટ હેન્ડલ કરવું \\ \hline
પેમેન્ટ પ્રોસેસ & પેમેન્ટ વિગતો & પેમેન્ટ સ્ટેટસ & ટ્રાન્ઝેક્શન પ્રોસેસ કરવા \\ \hline
ઇન્વેન્ટરી મેનેજ & સ્ટોક ક્વેરી & સ્ટોક સ્ટેટસ & પ્રોડક્ટ ઉપલબ્ધતા ટ્રેક કરવી \\ \hline
\end{tabulary}
\end{center}

\textbf{ડેટા સ્ટોર:}

\begin{itemize}
    \item \keyword{પ્રોડક્ટ ડેટાબેઝ}: પ્રોડક્ટ માહિતી સંગ્રહિત કરવી
    \item \keyword{ઓર્ડર ડેટાબેઝ}: ઓર્ડર વિગતો સંગ્રહિત કરવી
    \item \keyword{ગ્રાહક ડેટાબેઝ}: ગ્રાહક પ્રોફાઇલ સંગ્રહિત કરવી
\end{itemize}

\textbf{બાહ્ય એન્ટિટીઝ:}

\begin{itemize}
    \item \keyword{ગ્રાહક}: ઓર્ડર મૂકે છે, પેમેન્ટ કરે છે
    \item \keyword{પેમેન્ટ ગેટવે}: પેમેન્ટ પ્રોસેસ કરે છે
    \item \keyword{ઇન્વેન્ટરી મેનેજર}: સ્ટોક લેવલ અપડેટ કરે છે
\end{itemize}
\end{solutionbox}

\begin{mnemonicbox}
\mnemonic{PPMI: Process order, Process payment, Manage inventory}
\end{mnemonicbox}

\questionmarks{3(અ) OR}{3}{એનાલિસિસ અને ડિઝાઇન વચ્ચેનો તફાવત લખો}

\begin{solutionbox}
\begin{center}
\captionof{table}{એનાલિસિસ vs ડિઝાઇન}
\begin{tabulary}{\linewidth}{|L|L|L|}
\hline
\textbf{પાસું} & \textbf{એનાલિસિસ} & \textbf{ડિઝાઇન} \\ \hline
ધ્યાન & સિસ્ટમે શું કરવું જોઈએ & સિસ્ટમ કેવી રીતે કામ કરશે \\ \hline
તબક્કો & જરૂરિયાત તબક્કો & ડિઝાઇન તબક્કો \\ \hline
આઉટપુટ & સમસ્યાની સમજ & સોલ્યુશન સ્ટ્રક્ચર \\ \hline
મોડેલ & યુઝ કેસ, જરૂરિયાતો & આર્કિટેક્ચર, ક્લાસ \\ \hline
દૃષ્ટિકોણ & વપરાશકર્તાનો દૃષ્ટિકોણ & ડેવલપરનો દૃષ્ટિકોણ \\ \hline
સ્તર & અમૂર્ત, સંકલ્પનાત્મક & નક્કર, વિગતવાર \\ \hline
\end{tabulary}
\end{center}

\begin{itemize}
    \item \keyword{એનાલિસિસ}: સમસ્યા-કેન્દ્રિત, જરૂરિયાતોની સમજ
    \item \keyword{ડિઝાઇન}: સોલ્યુશન-કેન્દ્રિત, સિસ્ટમ આર્કિટેક્ચર બનાવવું
\end{itemize}
\end{solutionbox}

\begin{mnemonicbox}
\mnemonic{એનાલિસિસ=શું, ડિઝાઇન=કેવી રીતે}
\end{mnemonicbox}

\questionmarks{3(બ) OR}{4}{બેઝિક COCOMO મોડલ દ્વારા પ્રોજેક્ટ એસ્ટીમેશન પધ્ધતિ સમજાવો}

\begin{solutionbox}
\textbf{બેઝિક COCOMO મોડેલ:}

કોડની લાઇન આધારે સૉફ્ટવેર ડેવલપમેન્ટ પ્રયત્નનો અંદાજ કાઢે છે.

\textbf{સૂત્ર:}

\begin{itemize}
    \item Effort = a $\times$ (KLOC)$^b$ person-months
    \item Time = c $\times$ (Effort)$^d$ months
\end{itemize}

\textbf{પ્રોજેક્ટ પ્રકારો:}

\begin{center}
\captionof{table}{COCOMO પ્રોજેક્ટ પ્રકારો}
\begin{tabulary}{\linewidth}{|L|C|C|C|C|L|}
\hline
\textbf{પ્રકાર} & \textbf{a} & \textbf{b} & \textbf{c} & \textbf{d} & \textbf{વર્ણન} \\ \hline
ઓર્ગેનિક & 2.4 & 1.05 & 2.5 & 0.38 & નાની, અનુભવી ટીમ \\ \hline
સેમી-ડિટેચ્ડ & 3.0 & 1.12 & 2.5 & 0.35 & મધ્યમ કદ, મિશ્ર ટીમ \\ \hline
એમ્બેડેડ & 3.6 & 1.20 & 2.5 & 0.32 & જટિલ, કડક મર્યાદાઓ \\ \hline
\end{tabulary}
\end{center}

\textbf{પગલાં:}

\begin{enumerate}
    \item KLOC (કોડની હજારો લાઇન) નો અંદાજ કાઢવો
    \item પ્રોજેક્ટ પ્રકાર ઓળખવો (organic/semi-detached/embedded)
    \item યોગ્ય ગુણાંકો લાગુ કરવા
    \item પ્રયત્ન અને ડેવલપમેન્ટ સમયની ગણતરી કરવી
\end{enumerate}

\textbf{ઉદાહરણ}: 10 KLOC ઓર્ગેનિક પ્રોજેક્ટ

\begin{itemize}
    \item Effort = 2.4 $\times$ (10)$^{1.05}$ = 25.2 person-months
    \item Time = 2.5 $\times$ (25.2)$^{0.38}$ = 8.4 months
\end{itemize}
\end{solutionbox}

\begin{mnemonicbox}
\mnemonic{OSE: Organic, Semi-detached, Embedded}
\end{mnemonicbox}

\questionmarks{3(ક) OR}{7}{લાઇબ્રેરી મેનેજમેન્ટ સિસ્ટમ માટે ક્લાસ ડાયગ્રામ દોરો અને સમજાવો}

\begin{solutionbox}
\textbf{લાઇબ્રેરી મેનેજમેન્ટ સિસ્ટમ માટે ક્લાસ ડાયગ્રામ:}

\begin{center}
\begin{tikzpicture}[node distance=3.5cm, auto]
    % Library class
    \node [draw, rectangle, minimum width=3cm, minimum height=2.5cm] (library) at (0,0) {
        \begin{tabular}{c}
        \textbf{Library} \\
        \hline
        +name: String \\
        +address: String \\
        \hline
        +addBook() \\
        +removeBook() \\
        +searchBook()
        \end{tabular}
    };
    
    % Book class
    \node [draw, rectangle, minimum width=3cm, minimum height=3cm, right=4cm of library] (book) {
        \begin{tabular}{c}
        \textbf{Book} \\
        \hline
        +bookId: String \\
        +title: String \\
        +author: String \\
        +ISBN: String \\
        +isAvailable: Boolean \\
        \hline
        +getDetails()
        \end{tabular}
    };
    
    % Member class
    \node [draw, rectangle, minimum width=3cm, minimum height=2.8cm, below=3cm of library] (member) {
        \begin{tabular}{c}
        \textbf{Member} \\
        \hline
        +memberId: String \\
        +name: String \\
        +email: String \\
        +phone: String \\
        \hline
        +issueBook() \\
        +returnBook()
        \end{tabular}
    };
    
    % Transaction class
    \node [draw, rectangle, minimum width=3cm, minimum height=2.5cm, below=3cm of book] (trans) {
        \begin{tabular}{c}
        \textbf{Transaction} \\
        \hline
        +transactionId: String \\
        +issueDate: Date \\
        +returnDate: Date \\
        +fine: Double \\
        \hline
        +calculateFine()
        \end{tabular}
    };
    
    % Relationships
    \draw [->] (library) -- node[above] {1..*} (book);
    \draw [->] (member) -- node[above] {1..*} (trans);
    \draw [->] (book) -- node[right] {1..*} (trans);
\end{tikzpicture}
\captionof{figure}{લાઇબ્રેરી મેનેજમેન્ટ સિસ્ટમ ક્લાસ ડાયગ્રામ}
\end{center}

\textbf{સંબંધો:}

\begin{center}
\captionof{table}{ક્લાસ સંબંધો}
\begin{tabulary}{\linewidth}{|L|L|L|}
\hline
\textbf{સંબંધ} & \textbf{વર્ણન} & \textbf{મલ્ટિપ્લિસિટી} \\ \hline
લાઇબ્રેરી-બુક & લાઇબ્રેરીમાં પુસ્તકો છે & 1 થી ઘણા \\ \hline
મેમ્બર-ટ્રાન્ઝેક્શન & મેમ્બરના ટ્રાન્ઝેક્શન છે & 1 થી ઘણા \\ \hline
બુક-ટ્રાન્ઝેક્શન & પુસ્તક ટ્રાન્ઝેક્શનમાં સામેલ & 1 થી ઘણા \\ \hline
\end{tabulary}
\end{center}

\textbf{મુખ્ય લક્ષણો:}

\begin{itemize}
    \item \keyword{એટ્રિબ્યુટ્સ}: દરેક ક્લાસના ડેટા સભ્યો
    \item \keyword{મેથડ્સ}: ક્લાસ ડેટા પર કામ કરતા ફંક્શન
    \item \keyword{એસોસિએશન}: ક્લાસો વચ્ચેના સંબંધો બતાવે છે કે તેઓ કેવી રીતે ક્રિયાપ્રતિક્રિયા કરે છે
\end{itemize}
\end{solutionbox}

\begin{mnemonicbox}
\mnemonic{LBMT: Library, Book, Member, Transaction}
\end{mnemonicbox}

\questionmarks{4(અ)}{3}{પ્રોજેક્ટ સાઇઝ નક્કી કરવાના મેટ્રિક્સની યાદી બનાવી તેની વ્યાખ્યા લખો}

\begin{solutionbox}
\textbf{પ્રોજેક્ટ સાઇઝ એસ્ટીમેશન મેટ્રિક્સ:}

\begin{center}
\captionof{table}{સાઇઝ એસ્ટીમેશન મેટ્રિક્સ}
\begin{tabulary}{\linewidth}{|L|L|L|}
\hline
\textbf{મેટ્રિક} & \textbf{વ્યાખ્યા} & \textbf{ઉપયોગ} \\ \hline
લાઇન્સ ઓફ કોડ (LOC) & એક્ઝિક્યુટેબલ કોડ લાઇનની ગણતરી & પરંપરાગત સાઇઝિંગ \\ \hline
ફંક્શન પોઇન્ટ્સ (FP) & કાર્યક્ષમતા આધારિત માપ & ભાષા-સ્વતંત્ર \\ \hline
ફીચર પોઇન્ટ્સ & વિસ્તૃત ફંક્શન પોઇન્ટ્સ & રિયલ-ટાઇમ સિસ્ટમ \\ \hline
ઓબ્જેક્ટ પોઇન્ટ્સ & ઓબ્જેક્ટ અને મેથડ્સની ગણતરી & ઓબ્જેક્ટ-ઓરિએન્ટેડ સિસ્ટમ \\ \hline
યુઝ કેસ પોઇન્ટ્સ & યુઝ કેસ જટિલતા આધારિત & જરૂરિયાત-આધારિત \\ \hline
\end{tabulary}
\end{center}

\textbf{ફંક્શન પોઇન્ટ્સ ઘટકો:}

\begin{itemize}
    \item \keyword{એક્સટર્નલ ઇનપુટ્સ}: ડેટા એન્ટ્રી સ્ક્રીન
    \item \keyword{એક્સટર્નલ આઉટપુટ્સ}: રિપોર્ટ્સ, મેસેજ
    \item \keyword{એક્સટર્નલ ઇન્ક્વાયરીઝ}: ઇન્ટરેક્ટિવ ક્વેરીઝ
    \item \keyword{ઇન્ટર્નલ ફાઇલ્સ}: માસ્ટર ફાઇલ્સ
    \item \keyword{એક્સટર્નલ ઇન્ટરફેસ}: શેર કરેલ ડેટા
\end{itemize}

\textbf{ફાયદાઓ}: વહેલું અનુમાન, ટેકનોલોજી-સ્વતંત્ર, માનકીકૃત અભિગમ
\end{solutionbox}

\begin{mnemonicbox}
\mnemonic{LFFOU: LOC, Function Points, Feature Points, Object Points, Use Case Points}
\end{mnemonicbox}

\questionmarks{4(બ)}{4}{જોખમની ઓળખને વિસ્તારથી સમજાવો}

\begin{solutionbox}
\textbf{જોખમ ઓળખ:}

પ્રોજેક્ટની સફળતાને અસર કરી શકે તેવા સંભવિત જોખમોને શોધવા, ઓળખવા અને વર્ણવવાની પ્રક્રિયા.

\textbf{જોખમ કેટેગરીઝ:}

\begin{center}
\captionof{table}{જોખમ કેટેગરીઝ}
\begin{tabulary}{\linewidth}{|L|L|L|}
\hline
\textbf{કેટેગરી} & \textbf{ઉદાહરણો} & \textbf{પ્રભાવ} \\ \hline
ટેકનિકલ & નવી ટેકનોલોજી, જટિલતા & ડેવલપમેન્ટ વિલંબ \\ \hline
પ્રોજેક્ટ & શેડ્યૂલ, બજેટ મર્યાદાઓ & કોસ્ટ ઓવરરન \\ \hline
બિઝનેસ & માર્કેટ ફેરફારો, સ્પર્ધા & પ્રોજેક્ટ રદ્દીકરણ \\ \hline
બાહ્ય & વેન્ડર મુદ્દાઓ, નિયમો & નિર્ભરતાઓ \\ \hline
\end{tabulary}
\end{center}

\textbf{ઓળખ તકનીકો:}

\begin{itemize}
    \item \keyword{બ્રેઇનસ્ટોર્મિંગ}: જોખમો ઓળખવા માટે ટીમ ચર્ચા
    \item \keyword{ચેકલિસ્ટ}: માનક જોખમ કેટેગરીઝની સમીક્ષા
    \item \keyword{એક્સપર્ટ જજમેન્ટ}: અનુભવ આધારિત ઓળખ
    \item \keyword{SWOT એનાલિસિસ}: શક્તિઓ, નબળાઈઓ, તકો, ધમકીઓ
\end{itemize}

\textbf{રિસ્ક રજિસ્ટર:}

ઓળખાયેલ જોખમો સાથેનો દસ્તાવેજ જેમાં છે:

\begin{itemize}
    \item જોખમ વર્ણન
    \item ઘટનાની સંભાવના
    \item પ્રભાવની ગંભીરતા
    \item જોખમ કેટેગરી
    \item જવાબદાર વ્યક્તિ
\end{itemize}
\end{solutionbox}

\begin{mnemonicbox}
\mnemonic{TPBE: Technical, Project, Business, External risks}
\end{mnemonicbox}

\questionmarks{4(ક)}{7}{તમારી પસંદની કોઇ સિસ્ટમ માટે Gantt Chart દોરો}

\begin{solutionbox}
\textbf{ઓનલાઇન બેંકિંગ સિસ્ટમ માટે ગેન્ટ ચાર્ટ:}

\begin{center}
\captionof{table}{ગેન્ટ ચાર્ટ - ઓનલાઇન બેંકિંગ સિસ્ટમ}
\begin{tabulary}{\linewidth}{|L|C|C|C|C|C|C|C|C|}
\hline
\textbf{કાર્ય} & \textbf{અઠ 1} & \textbf{અઠ 2} & \textbf{અઠ 3} & \textbf{અઠ 4} & \textbf{અઠ 5} & \textbf{અઠ 6} & \textbf{અઠ 7} & \textbf{અઠ 8} \\ \hline
જરૂરિયાત વિશ્લેષણ & ████████ & ████████ & & & & & & \\ \hline
સિસ્ટમ ડિઝાઇન & & ████████ & ████████ & & & & & \\ \hline
ડેટાબેઝ ડિઝાઇન & & & ████████ & ████████ & & & & \\ \hline
UI ડેવલપમેન્ટ & & & & ████████ & ████████ & & & \\ \hline
બેકએન્ડ ડેવલપમેન્ટ & & & & & ████████ & ████████ & & \\ \hline
ટેસ્ટિંગ & & & & & & ████████ & ████████ & \\ \hline
ડિપ્લોયમેન્ટ & & & & & & & ████████ & ████████ \\ \hline
\end{tabulary}
\end{center}

\textbf{પ્રોજેક્ટ કાર્યો:}

\begin{center}
\captionof{table}{કાર્ય વિગતો}
\begin{tabulary}{\linewidth}{|L|L|L|L|}
\hline
\textbf{કાર્ય} & \textbf{અવધિ} & \textbf{નિર્ભરતાઓ} & \textbf{સંસાધનો} \\ \hline
જરૂરિયાત વિશ્લેષણ & 2 અઠવાડિયા & કોઈ નહીં & બિઝનેસ એનાલિસ્ટ \\ \hline
સિસ્ટમ ડિઝાઇન & 2 અઠવાડિયા & જરૂરિયાતો & સિસ્ટમ ડિઝાઇનર \\ \hline
ડેટાબેઝ ડિઝાઇન & 2 અઠવાડિયા & સિસ્ટમ ડિઝાઇન & ડેટાબેઝ ડિઝાઇનર \\ \hline
UI ડેવલપમેન્ટ & 2 અઠવાડિયા & સિસ્ટમ ડિઝાઇન & UI ડેવલપર \\ \hline
બેકએન્ડ ડેવલપમેન્ટ & 2 અઠવાડિયા & ડેટાબેઝ ડિઝાઇન & બેકએન્ડ ડેવલપર \\ \hline
ટેસ્ટિંગ & 2 અઠવાડિયા & UI + બેકએન્ડ & QA ટેસ્ટર \\ \hline
ડિપ્લોયમેન્ટ & 2 અઠવાડિયા & ટેસ્ટિંગ & DevOps એન્જિનિયર \\ \hline
\end{tabulary}
\end{center}

\textbf{ફાયદાઓ}: દ્રશ્ય પ્રગતિ ટ્રેકિંગ, સંસાધન ફાળવણી, નિર્ભરતા વ્યવસ્થાપન
\end{solutionbox}

\begin{mnemonicbox}
\mnemonic{RSDUBtd: Requirements, System design, Database, UI, Backend, Testing, Deployment}
\end{mnemonicbox}

\questionmarks{4(અ) OR}{3}{પ્રોજેક્ટ મેનેજરની જવાબદારીઓની યાદી બનાવો}

\begin{solutionbox}
\textbf{પ્રોજેક્ટ મેનેજરની જવાબદારીઓ:}

\begin{center}
\captionof{table}{પ્રોજેક્ટ મેનેજર જવાબદારીઓ}
\begin{tabulary}{\linewidth}{|L|L|}
\hline
\textbf{ક્ષેત્ર} & \textbf{જવાબદારીઓ} \\ \hline
આયોજન & પ્રોજેક્ટ પ્લાન બનાવવા, સ્કોપ વ્યાખ્યાયિત કરવો \\ \hline
સંગઠન & સંસાધનો ફાળવવા, ટીમ બનાવવી \\ \hline
નેતૃત્વ & ટીમને પ્રેરણા આપવી, સંઘર્ષ ઉકેલવો \\ \hline
નિયંત્રણ & પ્રગતિ મોનિટર કરવી, ફેરફારો વ્યવસ્થિત કરવા \\ \hline
કોમ્યુનિકેશન & હિતધારક અપડેટ્સ, ટીમ કોર્ડિનેશન \\ \hline
રિસ્ક મેનેજમેન્ટ & જોખમો ઓળખવા અને શમન કરવા \\ \hline
\end{tabulary}
\end{center}

\textbf{મુખ્ય પ્રવૃત્તિઓ:}

\begin{itemize}
    \item \keyword{પ્રોજેક્ટ શરૂઆત}: ઉદ્દેશ્યો અને મર્યાદાઓ વ્યાખ્યાયિત કરવા
    \item \keyword{શેડ્યૂલ મેનેજમેન્ટ}: ટાઇમલાઇન બનાવવી અને જાળવવી
    \item \keyword{બજેટ નિયંત્રણ}: ખર્ચ અને વ્યય મોનિટર કરવા
    \item \keyword{ગુણવત્તા આશ્વાસન}: ડિલિવરેબલ સ્ટાન્ડર્ડ સુનિશ્ચિત કરવા
    \item \keyword{ટીમ મેનેજમેન્ટ}: ટીમ સભ્યોનું નેતૃત્વ અને વિકાસ
\end{itemize}
\end{solutionbox}

\begin{mnemonicbox}
\mnemonic{POLCR: Planning, Organizing, Leading, Controlling, Risk management}
\end{mnemonicbox}

\questionmarks{4(બ) OR}{4}{જોખમ આકારણીને વિસ્તારથી સમજાવો}

\begin{solutionbox}
\textbf{જોખમ આકારણી:}

પ્રોજેક્ટની સફળતા પર તેમની સંભાવના અને પ્રભાવ નક્કી કરવા માટે ઓળખાયેલ જોખમોનું મૂલ્યાંકન કરવાની પ્રક્રિયા.

\textbf{આકારણી ઘટકો:}

\begin{center}
\captionof{table}{જોખમ આકારણી ઘટકો}
\begin{tabulary}{\linewidth}{|L|L|L|}
\hline
\textbf{ઘટક} & \textbf{સ્કેલ} & \textbf{વર્ણન} \\ \hline
સંભાવના & 1-5 અથવા \% & જોખમ ઘટનાની સંભાવના \\ \hline
પ્રભાવ & 1-5 અથવા \$ & જો જોખમ થાય તો તીવ્રતા \\ \hline
રિસ્ક સ્કોર & P $\times$ I & એકંદર જોખમ પ્રાથમિકતા \\ \hline
\end{tabulary}
\end{center}

\textbf{જોખમ આકારણી મેટ્રિક્સ:}

\begin{center}
\captionof{table}{જોખમ મેટ્રિક્સ}
\begin{tabulary}{\linewidth}{|L|C|C|C|}
\hline
\textbf{સંભાવના/પ્રભાવ} & \textbf{નીચું (1)} & \textbf{મધ્યમ (2)} & \textbf{ઉચ્ચ (3)} \\ \hline
નીચું (1) & 1 & 2 & 3 \\ \hline
મધ્યમ (2) & 2 & 4 & 6 \\ \hline
ઉચ્ચ (3) & 3 & 6 & 9 \\ \hline
\end{tabulary}
\end{center}

\textbf{આકારણી તકનીકો:}

\begin{itemize}
    \item \keyword{ગુણાત્મક આકારણી}: વર્ણનાત્મક સ્કેલ (ઉચ્ચ/મધ્યમ/નીચું)
    \item \keyword{માત્રાત્મક આકારણી}: સંખ્યાત્મક મૂલ્યો અને ગણતરીઓ
    \item \keyword{એક્સપર્ટ જજમેન્ટ}: અનુભવ આધારિત મૂલ્યાંકન
    \item \keyword{ઐતિહાસિક ડેટા}: ભૂતકાળના પ્રોજેક્ટ વિશ્લેષણ
\end{itemize}

\textbf{જોખમ વર્ગીકરણ:}

\begin{itemize}
    \item \keyword{ઉચ્ચ જોખમ} (7-9): તાત્કાલિક ધ્યાન જરૂરી
    \item \keyword{મધ્યમ જોખમ} (4-6): મોનિટર કરવું અને શમન આયોજન કરવું
    \item \keyword{નીચું જોખમ} (1-3): સ્વીકારવું અથવા ન્યૂનતમ શમન
\end{itemize}
\end{solutionbox}

\begin{mnemonicbox}
\mnemonic{PIS: Probability, Impact, Score}
\end{mnemonicbox}

\questionmarks{4(ક) OR}{7}{તમારી પસંદની કોઇ સિસ્ટમ માટે સ્પ્રિન્ટ બર્ન ડાઉન ચાર્ટ દોરો}

\begin{solutionbox}
\textbf{E-commerce મોબાઇલ એપ માટે સ્પ્રિન્ટ બર્ન ડાઉન ચાર્ટ (2-અઠવાડિયાનો સ્પ્રિન્ટ):}

\begin{center}
\begin{tikzpicture}[scale=0.8]
    % Axes
    \draw[->] (0,0) -- (11,0) node[right] {દિવસ};
    \draw[->] (0,0) -- (0,9) node[above] {સ્ટોરી પોઇન્ટ્સ};
    
    % Y-axis labels
    \foreach \y/\label in {0/0, 2/10, 4/20, 6/30, 8/40}
        \node at (-0.5,\y) {\label};
    
    % X-axis labels
    \foreach \x in {1,...,10}
        \node at (\x,-0.5) {\x};
    
    % Ideal line
    \draw[dashed, thick] (1,8) -- (10,0);
    
    % Actual progress
    \draw[thick, blue] (1,8) -- (2,7) -- (3,6) -- (4,5.5) -- (5,5.5) -- (6,4) -- (7,3) -- (8,2) -- (9,1) -- (10,0);
    
    % Legend
    \node[right] at (11,7) {આદર્શ};
    \node[right, blue] at (11,6) {વાસ્તવિક};
\end{tikzpicture}
\captionof{figure}{સ્પ્રિન્ટ બર્ન ડાઉન ચાર્ટ}
\end{center}

\textbf{સ્પ્રિન્ટ વિગતો:}

\begin{center}
\captionof{table}{સ્પ્રિન્ટ પ્રગતિ}
\begin{tabulary}{\linewidth}{|C|C|C|L|}
\hline
\textbf{દિવસ} & \textbf{આદર્શ બાકી} & \textbf{વાસ્તવિક બાકી} & \textbf{પૂર્ણ થયેલ કાર્ય} \\ \hline
1 & 36 & 40 & સ્પ્રિન્ટ પ્લાનિંગ \\ \hline
2 & 32 & 35 & યુઝર લોગિન ફીચર \\ \hline
3 & 28 & 30 & પ્રોડક્ટ કેટાલોગ \\ \hline
4 & 24 & 25 & શોપિંગ કાર્ટ \\ \hline
5 & 20 & 25 & API મુદ્દાથી અવરોધ \\ \hline
6 & 16 & 20 & પેમેન્ટ એકીકરણ \\ \hline
7 & 12 & 15 & ઓર્ડર મેનેજમેન્ટ \\ \hline
8 & 8 & 10 & ટેસ્ટિંગ અને ફિક્સ \\ \hline
9 & 4 & 5 & અંતિમ ટેસ્ટિંગ \\ \hline
10 & 0 & 0 & સ્પ્રિન્ટ પૂર્ણ \\ \hline
\end{tabulary}
\end{center}

\textbf{મુખ્ય અંતર્દૃષ્ટિ:}

\begin{itemize}
    \item \keyword{ઢાળ}: આદર્શ સાથે સરખામણીએ પ્રગતિ દર
    \item \keyword{સપાટ વિસ્તારો}: અવરોધિત કાર્ય અથવા સ્કોપ ફેરફારો
    \item \keyword{આદર્શથી નીચે}: શેડ્યૂલ આગળ
    \item \keyword{આદર્શથી ઉપર}: શેડ્યૂલ પાછળ
\end{itemize}
\end{solutionbox}

\begin{mnemonicbox}
\mnemonic{DABC: Days, Actual, Burn-down, Chart}
\end{mnemonicbox}

\questionmarks{5(અ)}{3}{કોડ રિવ્યુ તકનીકની યાદી બનાવી કોઈ એક સમજાવો}

\begin{solutionbox}
\textbf{કોડ રિવ્યુ તકનીકો:}

\begin{center}
\captionof{table}{કોડ રિવ્યુ તકનીકો}
\begin{tabulary}{\linewidth}{|L|L|L|}
\hline
\textbf{તકનીક} & \textbf{વર્ણન} & \textbf{સહભાગીઓ} \\ \hline
કોડ વોકથ્રુ & લેખક દ્વારા અનૌપચારિક સમીક્ષા & લેખક + સમીક્ષકો \\ \hline
કોડ ઇન્સ્પેક્શન & ઔપચારિક, વ્યવસ્થિત સમીક્ષા & પ્રશિક્ષિત નિરીક્ષકો \\ \hline
પીઅર રિવ્યુ & સાથીદાર કોડ તપાસે છે & ડેવલપર સાથીદારો \\ \hline
ટૂલ-આધારિત રિવ્યુ & સ્વચાલિત વિશ્લેષણ & ટૂલ્સ + ડેવલપર્સ \\ \hline
\end{tabulary}
\end{center}

\textbf{કોડ ઇન્સ્પેક્શન સમજાવેલ:}

\textbf{પ્રક્રિયા:}

\begin{enumerate}
    \item \keyword{આયોજન}: કોડ પસંદ કરવો, ભૂમિકાઓ સોંપવી
    \item \keyword{ઝાંખી}: લેખક કોડ સ્ટ્રક્ચર સમજાવે છે
    \item \keyword{તૈયારી}: કોડની વ્યક્તિગત સમીક્ષા
    \item \keyword{ઇન્સ્પેક્શન મીટિંગ}: ગ્રૂપ કોડ તપાસે છે
    \item \keyword{રિવર્ક}: ઓળખાયેલ ખામીઓ ઠીક કરવી
    \item \keyword{ફોલો-અપ}: સુધારાઓ ચકાસવા
\end{enumerate}

\textbf{ભૂમિકાઓ:}

\begin{itemize}
    \item \keyword{મોડરેટર}: ઇન્સ્પેક્શન પ્રક્રિયાનું નેતૃત્વ
    \item \keyword{લેખક}: કોડ ડેવલપર, લોજિક સમજાવે છે
    \item \keyword{સમીક્ષકો}: ખામીઓ અને મુદ્દાઓ શોધે છે
    \item \keyword{રેકોર્ડર}: તારણો દસ્તાવેજીકૃત કરે છે
\end{itemize}

\textbf{ફાયદાઓ}: ઉચ્ચ ખામી શોધ દર, જ્ઞાન શેરિંગ, સુધારેલ કોડ ગુણવત્તા
\end{solutionbox}

\begin{mnemonicbox}
\mnemonic{CWIP: Code Walkthrough, Inspection, Peer review}
\end{mnemonicbox}

\questionmarks{5(બ)}{4}{ઓનલાઇન શોપિંગ સિસ્ટમ માટે ટેસ્ટ કેસ તૈયાર કરો}

\begin{solutionbox}
\textbf{ઓનલાઇન શોપિંગ સિસ્ટમ માટે ટેસ્ટ કેસ:}

\begin{center}
\captionof{table}{ટેસ્ટ કેસ}
\begin{tabulary}{\linewidth}{|L|L|L|L|}
\hline
\textbf{TC ID} & \textbf{ટેસ્ટ દૃશ્ય} & \textbf{ટેસ્ટ સ્ટેપ્સ} & \textbf{અપેક્ષિત પરિણામ} \\ \hline
TC001 & વપરાશકર્તા નોંધણી & 1. માન્ય વિગતો દાખલ કરો\newline 2. રજિસ્ટર ક્લિક કરો & એકાઉન્ટ સફળતાપૂર્વક બનાવ્યું \\ \hline
TC002 & વપરાશકર્તા લોગિન & 1. વપરાશકર્તાનામ/પાસવર્ડ દાખલ કરો\newline 2. લોગિન ક્લિક કરો & વપરાશકર્તા લોગ ઇન થયો \\ \hline
TC003 & કાર્ટમાં ઉમેરો & 1. પ્રોડક્ટ પસંદ કરો\newline 2. કાર્ટમાં ઉમેરો ક્લિક કરો & પ્રોડક્ટ કાર્ટમાં ઉમેર્યું \\ \hline
TC004 & ચેકઆઉટ પ્રક્રિયા & 1. કાર્ટમાં જાઓ\newline 2. ચેકઆઉટ ક્લિક કરો\newline 3. પેમેન્ટ વિગતો દાખલ કરો & ઓર્ડર સફળતાપૂર્વક મૂક્યો \\ \hline
\end{tabulary}
\end{center}

\textbf{વિગતવાર ટેસ્ટ કેસ ઉદાહરણ:}

\textbf{ટેસ્ટ કેસ ID}: TC003

\textbf{ટેસ્ટ ટાઇટલ}: શોપિંગ કાર્ટમાં પ્રોડક્ટ ઉમેરવું

\textbf{પ્રી-કન્ડિશન}: વપરાશકર્તા લોગ ઇન છે, પ્રોડક્ટ ઉપલબ્ધ છે

\textbf{ટેસ્ટ સ્ટેપ્સ}:

\begin{enumerate}
    \item પ્રોડક્ટ કેટાલોગ પર નેવિગેટ કરો
    \item પ્રોડક્ટ પસંદ કરો
    \item જથ્થો પસંદ કરો
    \item ``કાર્ટમાં ઉમેરો'' બટન ક્લિક કરો
\end{enumerate}

\textbf{અપેક્ષિત પરિણામ}: સાચા જથ્થા અને કિંમત સાથે પ્રોડક્ટ કાર્ટમાં દેખાય છે

\textbf{પોસ્ટ-કન્ડિશન}: કાર્ટ કાઉન્ટ વધે છે, કુલ રકમ અપડેટ થાય છે
\end{solutionbox}

\begin{mnemonicbox}
\mnemonic{RAULC: Registration, Authentication, User cart, Login, Checkout}
\end{mnemonicbox}

\questionmarks{5(ક)}{7}{વ્હાઇટ બોક્સ ટેકનિકની વ્યાખ્યા કરો. વિવિધ વ્હાઇટ બોક્સ તકનીકની સૂચિ બનાવો. કોઈપણ બે સમજાવો}

\begin{solutionbox}
\textbf{વ્હાઇટ બોક્સ ટેસ્ટિંગ વ્યાખ્યા:}

આંતરિક કોડ સ્ટ્રક્ચર, લોજિક પાથ અને અમલીકરણ વિગતોની તપાસ કરતી ટેસ્ટિંગ તકનીક.

\textbf{વ્હાઇટ બોક્સ તકનીકો:}

\begin{center}
\captionof{table}{વ્હાઇટ બોક્સ તકનીકો}
\begin{tabulary}{\linewidth}{|L|L|L|}
\hline
\textbf{તકનીક} & \textbf{કવરેજ ક્રાઇટેરિયા} & \textbf{હેતુ} \\ \hline
સ્ટેટમેન્ટ કવરેજ & બધા સ્ટેટમેન્ટ એક્ઝિક્યુટ & બેસિક કોડ કવરેજ \\ \hline
બ્રાન્ચ કવરેજ & બધી બ્રાન્ચ લેવાય & નિર્ણય ટેસ્ટિંગ \\ \hline
પાથ કવરેજ & બધા પાથ એક્ઝિક્યુટ & સંપૂર્ણ ફ્લો ટેસ્ટિંગ \\ \hline
કન્ડિશન કવરેજ & બધી શરતો ટેસ્ટ & લોજિકલ એક્સપ્રેશન ટેસ્ટિંગ \\ \hline
લૂપ ટેસ્ટિંગ & બધા લૂપ વેરિએશન & પુનરાવર્તક સ્ટ્રક્ચર ટેસ્ટિંગ \\ \hline
\end{tabulary}
\end{center}

\textbf{1. સ્ટેટમેન્ટ કવરેજ:}

કોડમાં દરેક એક્ઝિક્યુટેબલ સ્ટેટમેન્ટ ઓછામાં ઓછું એક વાર એક્ઝિક્યુટ થાય તેની ખાતરી કરે છે.

\textbf{સૂત્ર}: (એક્ઝિક્યુટ થયેલ સ્ટેટમેન્ટ / કુલ સ્ટેટમેન્ટ) $\times$ 100\%

\textbf{ઉદાહરણ:}

\begin{lstlisting}[language=C]
if (x > 0)        // Statement 1
    y = x + 1;    // Statement 2
else
    y = x - 1;    // Statement 3
z = y * 2;        // Statement 4
\end{lstlisting}

\textbf{ટેસ્ટ કેસ}: x = 5 (સ્ટેટમેન્ટ 1,2,4 કવર કરે), x = -1 (સ્ટેટમેન્ટ 1,3,4 કવર કરે)

\textbf{કવરેજ}: 100\% સ્ટેટમેન્ટ કવરેજ હાંસલ

\textbf{2. બ્રાન્ચ કવરેજ:}

નિર્ણય બિંદુઓની દરેક બ્રાન્ચ (true/false) એક્ઝિક્યુટ થાય તેની ખાતરી કરે છે.

\textbf{ઉદાહરણ:}

\begin{lstlisting}[language=C]
if (a > b && c > d)    // Two conditions
    result = 1;        // True branch
else
    result = 0;        // False branch
\end{lstlisting}

\textbf{ટેસ્ટ કેસ}:

\begin{itemize}
    \item a=5, b=3, c=7, d=2 (true બ્રાન્ચ)
    \item a=1, b=3, c=7, d=2 (false બ્રાન્ચ)
\end{itemize}

\textbf{ફાયદાઓ}: સ્ટેટમેન્ટ કવરેજ કરતાં ઉચ્ચ ખામી શોધ
\end{solutionbox}

\begin{mnemonicbox}
\mnemonic{SBPCL: Statement, Branch, Path, Condition, Loop}
\end{mnemonicbox}

\questionmarks{5(અ) OR}{3}{સૉફ્ટવેર ડોક્યુમેન્ટેશન સમજાવો}

\begin{solutionbox}
\textbf{સૉફ્ટવેર ડોક્યુમેન્ટેશન:}

સૉફ્ટવેર સિસ્ટમ, તેની ડિઝાઇન, અમલીકરણ અને ઉપયોગનું વર્ણન કરતી લેખિત સામગ્રી.

\textbf{ડોક્યુમેન્ટેશનના પ્રકારો:}

\begin{center}
\captionof{table}{ડોક્યુમેન્ટેશન પ્રકારો}
\begin{tabulary}{\linewidth}{|L|L|L|}
\hline
\textbf{પ્રકાર} & \textbf{હેતુ} & \textbf{પ્રેક્ષકો} \\ \hline
આંતરિક ડોક્યુમેન્ટેશન & કોડ સમજ & ડેવલપર્સ \\ \hline
બાહ્ય ડોક્યુમેન્ટેશન & સિસ્ટમ ઉપયોગ & વપરાશકર્તાઓ, ઓપરેટર્સ \\ \hline
સિસ્ટમ ડોક્યુમેન્ટેશન & ડિઝાઇન અને આર્કિટેક્ચર & જાળવણીકર્તાઓ \\ \hline
વપરાશકર્તા ડોક્યુમેન્ટેશન & ઓપરેશન સૂચનાઓ & અંતિમ વપરાશકર્તાઓ \\ \hline
\end{tabulary}
\end{center}

\textbf{આંતરિક ડોક્યુમેન્ટેશન:}

\begin{itemize}
    \item \keyword{ટિપ્પણીઓ}: કોડ લોજિક અને હેતુ સમજાવે છે
    \item \keyword{કોડ સ્ટ્રક્ચર}: ક્લાસ અને મેથડ વર્ણનો
    \item \keyword{ડિઝાઇન તર્ક}: શા માટે ચોક્કસ અભિગમ પસંદ કર્યો
\end{itemize}

\textbf{બાહ્ય ડોક્યુમેન્ટેશન:}

\begin{itemize}
    \item \keyword{વપરાશકર્તા મેન્યુઅલ}: સ્ટેપ-બાય-સ્ટેપ ઉપયોગ સૂચનાઓ
    \item \keyword{ઇન્સ્ટોલેશન ગાઇડ}: સેટઅપ પ્રક્રિયાઓ
    \item \keyword{API ડોક્યુમેન્ટેશન}: ઇન્ટરફેસ સ્પેસિફિકેશન
\end{itemize}

\textbf{ફાયદાઓ}: સરળ જાળવણી, જ્ઞાન સ્થાનાંતરણ, ઘટાડેલ તાલીમ સમય
\end{solutionbox}

\begin{mnemonicbox}
\mnemonic{IESU: Internal, External, System, User documentation}
\end{mnemonicbox}

\questionmarks{5(બ) OR}{4}{ATM સિસ્ટમ માટે 4 ટેસ્ટ કેસ બનાવો}

\begin{solutionbox}
\textbf{ATM સિસ્ટમ માટે ટેસ્ટ કેસ:}

\begin{center}
\captionof{table}{ATM ટેસ્ટ કેસ}
\begin{tabulary}{\linewidth}{|L|L|L|L|}
\hline
\textbf{TC ID} & \textbf{ટેસ્ટ દૃશ્ય} & \textbf{ટેસ્ટ સ્ટેપ્સ} & \textbf{અપેક્ષિત પરિણામ} \\ \hline
TC001 & માન્ય PIN એન્ટ્રી & 1. કાર્ડ દાખલ કરો\newline 2. સાચો PIN દાખલ કરો\newline 3. Enter દબાવો & મુખ્ય મેનુમાં પ્રવેશ મળ્યો \\ \hline
TC002 & અમાન્ય PIN એન્ટ્રી & 1. કાર્ડ દાખલ કરો\newline 2. ખોટો PIN દાખલ કરો\newline 3. Enter દબાવો & ``અમાન્ય PIN'' સંદેશ દેખાય છે \\ \hline
TC003 & રોકડ ઉપાડ & 1. સફળતાપૂર્વક લોગિન કરો\newline 2. ``રોકડ ઉપાડ'' પસંદ કરો\newline 3. રકમ દાખલ કરો\newline 4. પુષ્ટિ કરો & રોકડ આપવામાં આવી, બેલેન્સ અપડેટ થયું \\ \hline
TC004 & અપૂરતું બેલેન્સ & 1. સફળતાપૂર્વક લોગિન કરો\newline 2. ``રોકડ ઉપાડ'' પસંદ કરો\newline 3. બેલેન્સ કરતાં વધુ રકમ દાખલ કરો & ``અપૂરતું બેલેન્સ'' સંદેશ \\ \hline
\end{tabulary}
\end{center}

\textbf{વિગતવાર ટેસ્ટ કેસ:}

\textbf{ટેસ્ટ કેસ ID}: TC003

\textbf{ટેસ્ટ વર્ણન}: પૂરતા બેલેન્સ સાથે રોકડ ઉપાડવી

\textbf{પ્રી-કન્ડિશન}: માન્ય ATM કાર્ડ, પૂરતું એકાઉન્ટ બેલેન્સ

\textbf{ટેસ્ટ ડેટા}: PIN=1234, ઉપાડની રકમ=₹1000, એકાઉન્ટ બેલેન્સ=₹5000

\textbf{પોસ્ટ-કન્ડિશન}: એકાઉન્ટ બેલેન્સ ₹1000 ઘટાડ્યું, ટ્રાન્ઝેક્શન રેકોર્ડ થયું
\end{solutionbox}

\begin{mnemonicbox}
\mnemonic{VPCI: Valid PIN, PIN error, Cash withdrawal, Insufficient funds}
\end{mnemonicbox}

\questionmarks{5(ક) OR}{7}{બ્લેક બોક્સ ટેસ્ટિંગ પધ્ધતિની સૂચિ બનાવો. તેને ફંકશનલ ટેસ્ટિંગ કેમ કહેવાય છે તે સમજાવો. કોઇ પણ બે પધ્ધતિ આકૃતિ સાથે વણવો}

\begin{solutionbox}
\textbf{બ્લેક બોક્સ ટેસ્ટિંગ પધ્ધતિઓ:}

\begin{center}
\captionof{table}{બ્લેક બોક્સ ટેસ્ટિંગ પધ્ધતિઓ}
\begin{tabulary}{\linewidth}{|L|L|L|}
\hline
\textbf{પધ્ધતિ} & \textbf{હેતુ} & \textbf{ઇનપુટ ફોકસ} \\ \hline
સમકક્ષ વિભાજન & ઇનપુટને વર્ગોમાં વહેંચવું & માન્ય/અમાન્ય વિભાજન \\ \hline
બાઉન્ડરી વેલ્યુ એનાલિસિસ & સીમા મૂલ્યોની ટેસ્ટ & સીમા શરતો \\ \hline
ડિસિઝન ટેબલ ટેસ્ટિંગ & જટિલ બિઝનેસ નિયમો & બહુવિધ ઇનપુટ સંયોજનો \\ \hline
સ્ટેટ ટ્રાન્ઝિશન ટેસ્ટિંગ & સ્ટેટ આધારિત સિસ્ટમ & સ્ટેટ ફેરફારો \\ \hline
યુઝ કેસ ટેસ્ટિંગ & કાર્યાત્મક દૃશ્યો & વપરાશકર્તા ક્રિયાપ્રતિક્રિયા \\ \hline
એરર ગેસિંગ & અનુભવ આધારિત ટેસ્ટિંગ & સંભવિત ભૂલ શરતો \\ \hline
\end{tabulary}
\end{center}

\textbf{શા માટે ફંકશનલ ટેસ્ટિંગ કહેવાય છે?}

બ્લેક બોક્સ ટેસ્ટિંગ \textbf{સિસ્ટમ શું કરે છે} પર ધ્યાન આપે છે \textbf{તે કેવી રીતે કામ કરે છે} તેનાથી વિપરીત. તે આંતરિક કોડ સ્ટ્રક્ચરનું જ્ઞાન વિના ઇનપુટ અને અપેક્ષિત આઉટપુટ ટેસ્ટ કરીને કાર્યાત્મક આવશ્યકતાઓને માન્ય કરે છે.

\textbf{1. સમકક્ષ વિભાજન:}

\begin{center}
\begin{tikzpicture}[node distance=2cm, auto]
    \node [draw, rectangle, minimum width=3cm, align=center] (valid) {માન્ય વિભાજન\\ 18-65 વર્ષ};
    \node [draw, rectangle, above=1cm of valid, minimum width=2cm] (inv1) {$<$0};
    \node [draw, rectangle, right=0.5cm of inv1, minimum width=2cm] (inv2) {0-17};
    \node [draw, rectangle, below=1cm of valid, minimum width=2cm] (inv3) {66-120};
    \node [draw, rectangle, right=0.5cm of inv3, minimum width=2cm] (inv4) {$>$120};
    
    \node [above=2cm of valid] {અમાન્ય વિભાજન};
    \node [below=2cm of valid] {અમાન્ય વિભાજન};
\end{tikzpicture}
\captionof{figure}{સમકક્ષ વિભાજન ઉદાહરણ}
\end{center}

\textbf{ઉદાહરણ}: જોબ એપ્લિકેશન માટે વય વેલિડેશન

\begin{itemize}
    \item \keyword{માન્ય વિભાજન}: 18-65 વર્ષ
    \item \keyword{અમાન્ય વિભાજન}: $<$0, 0-17, 66-120, $>$120
    \item \keyword{ટેસ્ટ કેસ}: દરેક વિભાજનમાંથી એક (દા.ત., 25, -5, 10, 70, 130)
\end{itemize}

\textbf{2. બાઉન્ડરી વેલ્યુ એનાલિસિસ:}

\begin{center}
\begin{tikzpicture}[node distance=1.5cm, auto]
    \draw[<->] (0,0) -- (10,0);
    \foreach \x/\label in {1/-1, 2/0, 3/1, 5/50, 7/99, 8/100, 9/101}
        \node at (\x,-0.5) {\label};
    
    \node at (1,0.5) {$\times$};
    \node at (2,0.5) {$\bullet$};
    \node at (3,0.5) {$\bullet$};
    \node at (5,0.5) {$\bullet$};
    \node at (7,0.5) {$\bullet$};
    \node at (8,0.5) {$\bullet$};
    \node at (9,0.5) {$\times$};
    
    \node at (0.5,1) {અમાન્ય};
    \node at (5,1) {માન્ય રેન્જ};
    \node at (9.5,1) {અમાન્ય};
\end{tikzpicture}
\captionof{figure}{બાઉન્ડરી વેલ્યુ એનાલિસિસ ઉદાહરણ}
\end{center}

\textbf{ઉદાહરણ}: વિદ્યાર્થી સ્કોર વેલિડેશન (0-100)

\begin{itemize}
    \item \keyword{ટેસ્ટ મૂલ્યો}: -1, 0, 1, 50, 99, 100, 101
    \item \keyword{ફોકસ}: સીમાની અંદર અને બહાર
    \item \keyword{તર્ક}: મોટાભાગની ભૂલો સીમા પર થાય છે
\end{itemize}

\textbf{ફાયદાઓ:}

\begin{itemize}
    \item \keyword{સ્વતંત્રતા}: પ્રોગ્રામિંગ જ્ઞાનની આવશ્યકતા નથી
    \item \keyword{વપરાશકર્તા દૃષ્ટિકોણ}: વપરાશકર્તાના દૃષ્ટિકોણથી ટેસ્ટ
    \item \keyword{જરૂરિયાત વેલિડેશન}: કાર્યાત્મક સ્પેસિફિકેશન ચકાસે છે
\end{itemize}
\end{solutionbox}

\begin{mnemonicbox}
\mnemonic{EBDSUE: Equivalence, Boundary, Decision, State, Use case, Error guessing}
\end{mnemonicbox}

\end{document}
