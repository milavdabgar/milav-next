\documentclass{article}

% content/resources/templates/preamble.tex
\usepackage[margin=0.6in]{geometry}
\author{Milav Dabgar}
\usepackage{amsmath,amssymb,amsthm}
\usepackage{booktabs}
\usepackage{multirow}
\usepackage{xcolor}
\usepackage{tcolorbox}
\tcbuselibrary{breakable,skins}
\usepackage[colorlinks=true,linkcolor=blue]{hyperref}
\usepackage{titlesec}
\usepackage{enumitem}
\usepackage{tikz}
\usepackage{pgfplots}
\usepackage{circuitikz}
\usepackage[version=4]{mhchem}
\usepackage{longtable}
\usepackage{array}
\usepackage{float}
\usepackage{caption}
\usepackage{listings}

\lstset{
  basicstyle=\small\ttfamily,
  breaklines=true,
  breakatwhitespace=false,
  postbreak=\mbox{\textcolor{red}{$\hookrightarrow$}\space},
  float=false,
  numbers=left,
  numberstyle=\tiny\color{gray},
  numbersep=10pt,
  xleftmargin=2em,
  keywordstyle=\color{blue},
  commentstyle=\color{green!60!black},
  stringstyle=\color{purple},
  backgroundcolor=\color{gray!5},
  showstringspaces=false,
  tabsize=2,
  captionpos=b,
  keepspaces=true,
  columns=flexible
}

\pgfplotsset{compat=1.18}
\usetikzlibrary{shapes,arrows,positioning,calc,patterns,decorations.pathmorphing,decorations.markings,arrows.meta}

% Color scheme
\definecolor{headcolor}{RGB}{0,102,204}
\definecolor{keycolor}{RGB}{220,20,60}
\definecolor{solutioncolor}{RGB}{34,139,34}
\definecolor{mnemoniccolor}{RGB}{148,0,211}
\definecolor{codecolor}{RGB}{0,0,100}

% Spacing
\setlength{\parskip}{3pt}
\setlist[itemize]{nosep}
\setlist[enumerate]{nosep}

% Title formatting
\titleformat{\section}{\Large\bfseries\color{headcolor}}{\thesection}{1em}{}
\titleformat{\subsection}{\large\bfseries\color{headcolor}}{\thesubsection}{1em}{}

% Pandoc tightlist compatibility
\providecommand{\tightlist}{%
  \setlength{\itemsep}{0pt}\setlength{\parskip}{0pt}}

% Pandoc longtable compatibility
\newcounter{none}
\def\thenone{}


% content/resources/templates/gujarati-boxes.tex
\usepackage{fontspec}
\usepackage{polyglossia}

% Set Gujarati as main language (document is primarily in Gujarati)
% Note: gloss-gujarati.ldf doesn't exist in polyglossia, but it will use hyphenation patterns
\setdefaultlanguage{gujarati}
\setotherlanguage{english}

% Configure Gujarati font properly
% Use Language=Default to prevent polyglossia from trying to add language-specific features
% that don't exist for Gujarati, which causes "empty feature" warnings
\newfontfamily\gujaratifont[Script=Gujarati,AutoFakeBold=2.5,AutoFakeSlant=0.3]{Noto Sans Gujarati}
\setmainfont[Script=Gujarati,AutoFakeBold=2.5,AutoFakeSlant=0.3]{Noto Sans Gujarati}
% Use Noto Sans Gujarati for monospace to support Gujarati in text
\setmonofont[Scale=0.9]{Noto Sans Gujarati}

% Configure English to use the same font
\newfontfamily\englishfont[Script=Gujarati,AutoFakeBold=2.5,AutoFakeSlant=0.3]{Noto Sans Gujarati}

% Translations for polyglossia
\gappto\captionsgujarati{
  \renewcommand{\tablename}{કોષ્ટક}
  \renewcommand{\figurename}{આકૃતિ}
}

% Helper for TikZ nodes to ensure Gujarati font
\newcommand{\gu}[1]{{\gujaratifont #1}}

% Custom environments
\newtcolorbox{solutionbox}{
    breakable,
    enhanced,
    colback=solutioncolor!5!white,
    colframe=solutioncolor!75!black,
    fonttitle=\bfseries,
    title=જવાબ
}

\newtcolorbox{solutionboxnobreak}{
 colback=solutioncolor!5!white,
 colframe=solutioncolor!75!black,
 fonttitle=\bfseries,
 title=જવાબ
}

\newtcolorbox{keyformula}{
 breakable,
 enhanced,
 colback=keycolor!5!white,
 colframe=keycolor!75!black,
 fonttitle=\bfseries,
 title=રાસાયણિક સમીકરણ/સૂત્ર
}

\newtcolorbox{mnemonicbox}{
 breakable,
 enhanced,
 colback=mnemoniccolor!5!white,
 colframe=mnemoniccolor!75!black,
 fonttitle=\bfseries,
 title=મેમરી ટ્રીક
}


% Custom commands for GTU solutions
% This file defines semantic commands for consistent formatting

% Question command with automatic formatting
\newcommand{\question}[2]{%
  \section*{Question #1}%
  \textbf{#2}%
}

% OR question variant
\newcommand{\questionor}[2]{%
  \section*{Question #1 OR}%
  \textbf{#2}%
}

% Proper table environment with caption
\newenvironment{answertable}[1]{%
  \begin{table}[htbp]
  \centering
  \caption{#1}
}{%
  \end{table}
}

% Proper figure environment for diagrams
\newenvironment{answerdiagram}[1]{%
  \begin{figure}[htbp]
  \centering
  \caption{#1}
}{%
  \end{figure}
}

% Semantic markup for key terms
\newcommand{\keyword}[1]{\textbf{#1}}
\newcommand{\code}[1]{\texttt{#1}}
\newcommand{\classname}[1]{\texttt{#1}}
\newcommand{\methodname}[1]{\texttt{#1}}

% Proper quotation marks
\newcommand{\mnemonic}[1]{``#1''}


\usetikzlibrary{fit}

\title{સોફ્ટવૅર એન્જિનિયરિંગ (4353202) - શિયાળો 2024 ઉકેલ}
\date{November 25, 2024}

\begin{document}
\maketitle

\questionmarks{1(a)}{3}{સોફ્ટવૅર ની વ્યાખ્યા આપો અને તેની લાક્ષણિકતા સમજાવો.}

\begin{solutionbox}
\textbf{સોફ્ટવૅર} એ કમ્પ્યુટર પ્રોગ્રામ્સ, પ્રક્રિયાઓ અને દસ્તાવેજીકરણનો સમૂહ છે જે કમ્પ્યુટર સિસ્ટમ પર કાર્યો કરે છે.

\begin{center}
\captionof{table}{સોફ્ટવૅર લાક્ષણિકતાઓ}
\begin{tabulary}{\linewidth}{|L|L|}
\hline
\textbf{લાક્ષણિકતા} & \textbf{વર્ણન} \\ \hline
\textbf{અસ્પર્શ્ય} & સ્પર્શ કરી શકાતું નથી, માત્ર અનુભવી શકાય છે \\ \hline
\textbf{વિકસિત} & એન્જિનિયર્ડ, મેન્યુફેક્ચર્ડ નહીં \\ \hline
\textbf{જાળવણીયોગ્ય} & સુધારણા અને અપડેટ કરી શકાય છે \\ \hline
\textbf{વિશ્વસનીય} & સતત કામ કરવું જોઈએ \\ \hline
\textbf{કાર્યક્ષમ} & સાધનોનો શ્રેષ્ઠ ઉપયોગ કરે છે \\ \hline
\end{tabulary}
\end{center}

\begin{itemize}
    \item \textbf{મુખ્ય મુદ્દો}: સોફ્ટવૅર = પ્રોગ્રામ્સ + દસ્તાવેજીકરણ + પ્રક્રિયાઓ
\end{itemize}
\end{solutionbox}

\begin{mnemonicbox}
\mnemonic{I Don't Make Reliable Electronics (Intangible, Developed, Maintainable, Reliable, Efficient)}
\end{mnemonicbox}

\questionmarks{1(b)}{4}{ક્લાસિકલ વોટરફોલ મોડેલ સમજાવો.}

\begin{solutionbox}
\textbf{વોટરફોલ મોડેલ} એ રેખીય ક્રમિક સોફ્ટવૅર વિકાસ પદ્ધતિ છે જ્યાં દરેક તબક્કો પૂર્ણ થયા પછી જ આગળનો તબક્કો શરૂ થાય છે.

\begin{center}
\begin{tikzpicture}[node distance=1.5cm, auto]
    \node [gtu block] (req) {આવશ્યકતા વિશ્લેષણ};
    \node [gtu block, below of=req] (des) {સિસ્ટમ ડિઝાઇન};
    \node [gtu block, below of=des] (impl) {અમલીકરણ};
    \node [gtu block, below of=impl] (test) {પરીક્ષણ};
    \node [gtu block, below of=test] (dep) {જમાવટ};
    \node [gtu block, below of=dep] (maint) {જાળવણી};

    \draw [gtu arrow] (req) -- (des);
    \draw [gtu arrow] (des) -- (impl);
    \draw [gtu arrow] (impl) -- (test);
    \draw [gtu arrow] (test) -- (dep);
    \draw [gtu arrow] (dep) -- (maint);
\end{tikzpicture}
\captionof{figure}{ક્લાસિકલ વોટરફોલ મોડેલ}
\end{center}

\textbf{મુખ્ય લક્ષણો}:
\begin{itemize}
    \item \textbf{ક્રમિક તબક્કાઓ}: તબક્કાઓ વચ્ચે કોઈ ઓવરલેપ નથી
    \item \textbf{દસ્તાવેજીકરણ આધારિત}: દરેક તબક્કે ભારે દસ્તાવેજીકરણ
    \item \textbf{સરળ માળખું}: સમજવા અને મેનેજ કરવા સરળ
    \item \textbf{નિશ્ચિત આવશ્યકતાઓ}: એકવાર શરૂ થયા પછી ફેરફાર મુશ્કેલ
\end{itemize}
\end{solutionbox}

\begin{mnemonicbox}
\mnemonic{Real Systems Include Testing, Deployment, Maintenance}
\end{mnemonicbox}

\questionmarks{1(c)}{7}{સોફ્ટવૅર પ્રોસેસ ફ્રેમવર્ક અને અમ્બ્રેલા એક્ટિવિટી સમજાવો.}

\begin{solutionbox}
\textbf{સોફ્ટવૅર પ્રોસેસ ફ્રેમવર્ક} મુખ્ય પ્રોસેસ વિસ્તારો ઓળખીને સંપૂર્ણ સોફ્ટવૅર એન્જિનિયરિંગ પ્રોસેસ માટે પાયો પ્રદાન કરે છે.

\begin{center}
\begin{tikzpicture}[node distance=2cm, auto]
    % Framework Activities in a cycle
    \node [gtu state] (comm) {સંવાદ};
    \node [gtu state, right of=comm] (plan) {આયોજન};
    \node [gtu state, right of=plan] (model) {મોડેલિંગ};
    \node [gtu state, right of=model] (const) {નિર્માણ};
    \node [gtu state, right of=const] (dep) {જમાવટ};

    \path [gtu arrow] (comm) -- (plan);
    \path [gtu arrow] (plan) -- (model);
    \path [gtu arrow] (model) -- (const);
    \path [gtu arrow] (const) -- (dep);
    \path [gtu arrow] (dep) edge [bend left=30] (comm);
    
    % Umbrella Activities spanning across
    \node [draw, headcolor, dashed, fit=(comm) (dep), inner sep=1cm, label=above:\textbf{અમ્બ્રેલા એક્ટિવિટીઝ}] (umbrella) {};
    
\end{tikzpicture}
\captionof{figure}{પ્રોસેસ ફ્રેમવર્ક \& અમ્બ્રેલા એક્ટિવિટીઝ}
\end{center}

\begin{center}
\captionof{table}{ફ્રેમવર્ક એક્ટિવિટીઝ વિ અમ્બ્રેલા એક્ટિવિટીઝ}
\begin{tabulary}{\linewidth}{|L|L|}
\hline
\textbf{ફ્રેમવર્ક એક્ટિવિટીઝ} & \textbf{અમ્બ્રેલા એક્ટિવિટીઝ} \\ \hline
સંવાદ & સોફ્ટવૅર પ્રોજેક્ટ ટ્રેકિંગ \\ \hline
આયોજન & જોખમ મેનેજમેન્ટ \\ \hline
મોડેલિંગ & ગુણવત્તા ખાતરી \\ \hline
નિર્માણ & તકનીકી સમીક્ષાઓ \\ \hline
જમાવટ & કન્ફિગરેશન મેનેજમેન્ટ \\ \hline
\end{tabulary}
\end{center}

\textbf{ફ્રેમવર્ક એક્ટિવિટીઝ}:
\begin{itemize}
    \item \textbf{સંવાદ}: સ્ટેકહોલ્ડર્સ પાસેથી આવશ્યકતાઓ એકત્રિત કરવી
    \item \textbf{આયોજન}: પ્રોજેક્ટ પ્લાન અને શેડ્યુલ બનાવવું
    \item \textbf{મોડેલિંગ}: ડિઝાઇન મોડેલ્સ બનાવવા
    \item \textbf{નિર્માણ}: કોડ જનરેશન અને પરીક્ષણ
    \item \textbf{જમાવટ}: સોફ્ટવૅર ડિલિવરી અને ફીડબેક
\end{itemize}

\textbf{અમ્બ્રેલા એક્ટિવિટીઝ} પ્રોજેક્ટ દરમિયાન ચાલે છે:
\begin{itemize}
    \item \textbf{પ્રોજેક્ટ ટ્રેકિંગ}: પ્રગતિ નિરીક્ષણ
    \item \textbf{જોખમ મેનેજમેન્ટ}: જોખમો ઓળખવા અને નિયંત્રિત કરવા
    \item \textbf{ગુણવત્તા ખાતરી}: ગુણવત્તા ધોરણો સુનિશ્ચિત કરવા
    \item \textbf{કન્ફિગરેશન મેનેજમેન્ટ}: ફેરફારો નિયંત્રિત કરવા
\end{itemize}
\end{solutionbox}

\begin{mnemonicbox}
\mnemonic{Can People Make Construction Deploy (Communication, Planning, Modeling, Construction, Deployment)}
\end{mnemonicbox}

\questionmarks{1(c) OR}{7}{SCRUM મોડેલ પર ટૂંક નોંધ લખો.}

\begin{solutionbox}
\textbf{SCRUM} એ પુનરાવર્તક અને વૃદ્ધિશીલ પ્રથાઓનો ઉપયોગ કરીને સોફ્ટવૅર વિકાસ પ્રોજેક્ટ્સનું મેનેજમેન્ટ કરવા માટેનું એક agile ફ્રેમવર્ક છે.

\begin{center}
\begin{tikzpicture}[node distance=1.5cm, auto]
    \node [gtu block, align=center] (pb) {Product\\Backlog};
    \node [gtu block, right of=pb, xshift=1cm, align=center] (sp) {Sprint\\Planning};
    \node [gtu block, right of=sp, xshift=1cm, align=center] (sb) {Sprint\\Backlog};
    
    \node [draw, circle, minimum size=2.5cm, right of=sb, xshift=1.5cm, align=center] (sprint) {Sprint\\(2-4 weeks)};
    
    \node [gtu block, right of=sprint, xshift=1.5cm, align=center] (review) {Sprint\\Review};
    \node [gtu block, below of=review, align=center] (retro) {Sprint\\Retrospective};
    
    \draw [gtu arrow] (pb) -- (sp);
    \draw [gtu arrow] (sp) -- (sb);
    \draw [gtu arrow] (sb) -- (sprint);
    \draw [gtu arrow] (sprint) -- (review);
    \draw [gtu arrow] (review) -- (retro);
    \draw [gtu arrow] (retro) -| (sp);
    
    \node [gtu block, above of=sprint, align=center] (daily) {Daily\\Scrum};
    \draw [gtu arrow] (sprint) -- (daily);
    \draw [gtu arrow] (daily) -- (sprint);
\end{tikzpicture}
\captionof{figure}{SCRUM Process Flow}
\end{center}

\begin{center}
\captionof{table}{SCRUM ભૂમિકાઓ અને આર્ટિફેક્ટ્સ}
\begin{tabulary}{\linewidth}{|L|L|}
\hline
\textbf{ઘટક} & \textbf{વર્ણન} \\ \hline
\textbf{Product Owner} & આવશ્યકતાઓ અને પ્રાથમિકતાઓ વ્યાખ્યાયિત કરે છે \\ \hline
\textbf{Scrum Master} & પ્રક્રિયાને સુવિધા આપે છે અને અવરોધો દૂર કરે છે \\ \hline
\textbf{Development Team} & સ્વ-સંગઠિત ટીમ જે પ્રોડક્ટ બનાવે છે \\ \hline
\textbf{Product Backlog} & લક્ષણોની પ્રાથમિકતા આપેલી યાદી \\ \hline
\textbf{Sprint Backlog} & વર્તમાન sprint માટે પસંદ કરેલા કાર્યો \\ \hline
\end{tabulary}
\end{center}

\textbf{મુખ્ય ઇવેન્ટ્સ}:
\begin{itemize}
    \item \textbf{Sprint Planning}: આગામી sprint માટે કામ પસંદ કરવું
    \item \textbf{Daily Scrum}: 15-મિનિટનું દૈનિક સિંક્રોનાઇઝેશન
    \item \textbf{Sprint Review}: પૂર્ણ થયેલ કામ દર્શાવવું
    \item \textbf{Sprint Retrospective}: પ્રક્રિયા પર વિચાર કરવો અને સુધારવું
\end{itemize}

\textbf{ફાયદાઓ}: ઝડપી ડિલિવરી, લવચીકતા, સતત સુધારણા, ગ્રાહક સહયોગ
\end{solutionbox}

\begin{mnemonicbox}
\mnemonic{People Sprint Daily Reviewing Retrospectively}
\end{mnemonicbox}

\questionmarks{2(a)}{3}{સારા SRS ની લાક્ષણિકતા સમજાવો.}

\begin{solutionbox}
\textbf{SRS (સોફ્ટવૅર આવશ્યકતા વિશિષ્ટતા)} દસ્તાવેજ અસરકારક બનવા માટે વિશિષ્ટ ગુણો હોવા જોઈએ.

\begin{center}
\captionof{table}{સારા SRS લાક્ષણિકતાઓ}
\begin{tabulary}{\linewidth}{|L|L|}
\hline
\textbf{લાક્ષણિકતા} & \textbf{અર્થ} \\ \hline
\textbf{સંપૂર્ણ} & બધી આવશ્યકતાઓ સમાવેશ \\ \hline
\textbf{સુસંગત} & કોઈ વિરોધાભાસી આવશ્યકતાઓ નથી \\ \hline
\textbf{અસ્પષ્ટ નથી} & સ્પષ્ટ અને એક અર્થઘટન \\ \hline
\textbf{ચકાસણીયોગ્ય} & પરીક્ષણ અને વેલિડેશન શક્ય \\ \hline
\textbf{સુધારણાયોગ્ય} & જરૂર પડે ત્યારે બદલવા સરળ \\ \hline
\end{tabulary}
\end{center}

\begin{itemize}
    \item \textbf{સંપૂર્ણ}: બધી functional અને non-functional આવશ્યકતાઓ સમાવે છે
    \item \textbf{સુસંગત}: વિવિધ આવશ્યકતાઓ વચ્ચે કોઈ સંઘર્ષ નથી
    \item \textbf{અસ્પષ્ટ નથી}: દરેક આવશ્યકતાનો માત્ર એક જ અર્થઘટન છે
\end{itemize}
\end{solutionbox}

\begin{mnemonicbox}
\mnemonic{Complete Computers Use Verified Modifications}
\end{mnemonicbox}

\questionmarks{2(b)}{4}{પ્રોટોટાઇપ મોડેલના લાભ અને ગેરલાભ વર્ણવો.}

\begin{solutionbox}
\textbf{પ્રોટોટાઇપ મોડેલ} આવશ્યકતાઓને વધુ સારી રીતે સમજવા માટે સોફ્ટવૅરનું કાર્યકારી મોડેલ બનાવે છે.

\begin{center}
\captionof{table}{પ્રોટોટાઇપ મોડેલ - ફાયદા અને ગેરફાયદા}
\begin{tabulary}{\linewidth}{|L|L|}
\hline
\textbf{ફાયદા} & \textbf{ગેરફાયદા} \\ \hline
\textbf{આવશ્યકતા સમજણ સુધારે છે} & \textbf{સમયનો વધારે ખર્ચ} \\ \hline
\textbf{વપરાશકર્તા સામેલગીરી} & \textbf{ખર્ચમાં વધારો} \\ \hline
\textbf{પ્રારંભિક ભૂલ શોધ} & \textbf{અપૂર્ણ વિશ્લેષણ} \\ \hline
\textbf{વપરાશકર્તા સંતુષ્ટિ} & \textbf{પ્રોટોટાઇપ મૂંઝવણ} \\ \hline
\end{tabulary}
\end{center}

\textbf{ફાયદા}:
\begin{itemize}
    \item \textbf{સ્પષ્ટ આવશ્યકતાઓ}: વપરાશકર્તા કાર્યકારી મોડેલ જુએ છે
    \item \textbf{પ્રારંભિક ફીડબેક}: ખર્ચાળ ભૂલો ટાળે છે
    \item \textbf{વપરાશકર્તા સામેલગીરી}: શ્રેષ્ઠ વપરાશકર્તા સ્વીકૃતિ
\end{itemize}

\textbf{ગેરફાયદા}:
\begin{itemize}
    \item \textbf{વધારાનો સમય}: પ્રોટોટાઇપ બનાવવામાં સમય લાગે છે
    \item \textbf{વધારાનો ખર્ચ}: પ્રોટોટાઇપ માટે સંસાધનોની જરૂર
    \item \textbf{સ્કોપ ક્રીપ}: વપરાશકર્તાઓ વધુ સુવિધાઓ માંગી શકે છે
\end{itemize}
\end{solutionbox}

\begin{mnemonicbox}
\mnemonic{Better Users Experience vs Time Costs Increase}
\end{mnemonicbox}

\questionmarks{2(c)}{7}{Spiral મોડેલ ડિઝાઇન કરો, વર્ણવો અને તેના ફાયદા અને ગેરફાયદા આપો.}

\begin{solutionbox}
\textbf{Spiral મોડેલ} પુનરાવર્તક વિકાસને વ્યવસ્થિત જોખમ સંચાલન સાથે જોડે છે.

\begin{center}
\begin{tikzpicture}[scale=0.8, auto]
    % Spiral line
    \draw [headcolor, thick, domain=0:14.5, variable=\t, samples=200, smooth] plot ({\t r}: {0.3*\t});
    
    % Axes
    \draw [->] (-5,0) -- (5,0);
    \draw [->] (0,-5) -- (0,5);
    
    % Quadrant Labels
    \node [align=center] at (3.5, 3.5) {\textbf{આયોજન}};
    \node [align=center] at (-3.5, 3.5) {\textbf{જોખમ વિશ્લેષણ}};
    \node [align=center] at (-3.5, -3.5) {\textbf{એન્જિનિયરિંગ}};
    \node [align=center] at (3.5, -3.5) {\textbf{ગ્રાહક}\\\textbf{મૂલ્યાંકન}};
    
\end{tikzpicture}
\captionof{figure}{Spiral મોડેલ}
\end{center}

\begin{center}
\captionof{table}{Spiral મોડેલ તબક્કાઓ}
\begin{tabulary}{\linewidth}{|L|L|}
\hline
\textbf{તબક્કો} & \textbf{પ્રવૃત્તિઓ} \\ \hline
\textbf{આયોજન} & આવશ્યકતાઓ એકત્રિત કરવી, સંસાધન આયોજન \\ \hline
\textbf{જોખમ વિશ્લેષણ} & જોખમો ઓળખવા અને ઉકેલવા \\ \hline
\textbf{એન્જિનિયરિંગ} & વિકાસ અને પરીક્ષણ \\ \hline
\textbf{ગ્રાહક મૂલ્યાંકન} & ગ્રાહક સમીક્ષાઓ અને ફીડબેક \\ \hline
\end{tabulary}
\end{center}

\textbf{ફાયદા}:
\begin{itemize}
    \item \textbf{જોખમ સંચાલન}: પ્રારંભિક જોખમ ઓળખ
    \item \textbf{લવચીકતા}: ફેરફારો સરળતાથી સમાવી શકાય છે
    \item \textbf{ગ્રાહક સંડોવણી}: નિયમિત ગ્રાહક ફીડબેક
    \item \textbf{ગુણવત્તા ફોકસ}: સતત પરીક્ષણ અને માન્યતા
\end{itemize}

\textbf{ગેરફાયદા}:
\begin{itemize}
    \item \textbf{જટિલ મેનેજમેન્ટ}: સંચાલન કરવું મુશ્કેલ
    \item \textbf{વધુ ખર્ચ}: જોખમ વિશ્લેષણને કારણે ખર્ચાળ
    \item \textbf{સમય માંગી લે છે}: લાંબા વિકાસ ચક્ર
    \item \textbf{જોખમ નિષ્ણાતની જરૂર}: જોખમ આકારણી કુશળતા આવશ્યક છે
\end{itemize}

\textbf{શ્રેષ્ઠ આ માટે}: મોટા, જટિલ, ઉચ્ચ-જોખમવાળા પ્રોજેક્ટ્સ
\end{solutionbox}

\begin{mnemonicbox}
\mnemonic{Plan Risks Engineering Customer}
\end{mnemonicbox}

\questionmarks{2(a) OR}{3}{Incremental મોડેલ સમજાવો.}

\begin{solutionbox}
\textbf{Incremental મોડેલ} ઇન્ક્રીમેન્ટ્સ તરીકે ઓળખાતા નાના, વિધેયાત્મક ટુકડાઓમાં સોફ્ટવૅર વિતરિત કરે છે.

\begin{center}
\begin{tikzpicture}[node distance=1.5cm, auto]
    \node [gtu block] (core) {Core Product};
    \node [gtu block, right of=core, xshift=1.2cm] (inc1) {Increment 1};
    \node [gtu block, right of=inc1, xshift=1.2cm] (inc2) {Increment 2};
    \node [gtu block, right of=inc2, xshift=1.2cm] (inc3) {Increment 3};
    \node [gtu block, below of=inc1, xshift=1.35cm] (final) {Final Product};
    
    \draw [gtu arrow] (core) -- (inc1);
    \draw [gtu arrow] (inc1) -- (inc2);
    \draw [gtu arrow] (inc2) -- (inc3);
    \draw [gtu arrow] (inc3) -- (final);
\end{tikzpicture}
\captionof{figure}{Incremental Delivery}
\end{center}

\textbf{મુખ્ય લક્ષણો}:
\begin{itemize}
    \item \textbf{આંશિક અમલીકરણ}: દરેક ઇન્ક્રીમેન્ટ કાર્યક્ષમતા ઉમેરે છે
    \item \textbf{પ્રારંભિક ડિલિવરી}: મુખ્ય સુવિધાઓ પહેલા વિતરિત
    \item \textbf{સમાંતર વિકાસ}: બહુવિધ ઇન્ક્રીમેન્ટ્સ એકસાથે વિકસાવી શકાય છે
\end{itemize}

\begin{center}
\captionof{table}{Incremental મોડેલ લાક્ષણિકતાઓ}
\begin{tabulary}{\linewidth}{|L|L|}
\hline
\textbf{પાસા} & \textbf{વર્ણન} \\ \hline
\textbf{ડિલિવરી} & બહુવિધ રીલીઝ \\ \hline
\textbf{કાર્યક્ષમતા} & દરેક ઇન્ક્રીમેન્ટ સાથે વધે છે \\ \hline
\textbf{જોખમ} & પ્રારંભિક ડિલિવરી દ્વારા ઘટાડો \\ \hline
\textbf{ફીડબેક} & સતત વપરાશકર્તા ફીડબેક \\ \hline
\end{tabulary}
\end{center}
\end{solutionbox}

\begin{mnemonicbox}
\mnemonic{Deliver Functionality Reducing Feedback}
\end{mnemonicbox}

\questionmarks{2(b) OR}{4}{RAD મોડેલનો ખ્યાલ લખો અને સમજાવો.}

\begin{solutionbox}
\textbf{RAD (Rapid Application Development)} વિસ્તૃત આયોજન પર ઝડપી પ્રોટોટાઇપિંગ અને ઝડપી ફીડબેક પર ભાર મૂકે છે.

\begin{center}
\captionof{table}{RAD મોડેલ તબક્કાઓ}
\begin{tabulary}{\linewidth}{|L|C|L|}
\hline
\textbf{તબક્કો} & \textbf{સમય} & \textbf{પ્રવૃત્તિઓ} \\ \hline
\textbf{Business Modeling} & ટૂંકા & વ્યવસાય કાર્યો વ્યાખ્યાયિત કરો \\ \hline
\textbf{Data Modeling} & ટૂંકા & ડેટા આવશ્યકતાઓ વ્યાખ્યાયિત કરો \\ \hline
\textbf{Process Modeling} & ટૂંકા & ડેટાને માહિતીમાં રૂપાંતરિત કરો \\ \hline
\textbf{Application Generation} & ટૂંકા & સોફ્ટવૅર બનાવવા સાધનો ઉપયોગ કરો \\ \hline
\textbf{Testing \& Turnover} & ટૂંકા & પરીક્ષણ અને જમાવટ \\ \hline
\end{tabulary}
\end{center}

\textbf{મુખ્ય ખ્યાલો}:
\begin{itemize}
    \item \textbf{પુનઃઉપયોગી ઘટકો}: પૂર્વ-નિર્મિત ઘટકો વિકાસ ઝડપી બનાવે છે
    \item \textbf{શક્તિશાળી સાધનો}: CASE સાધનો અને કોડ જનરેટર્સ
    \item \textbf{નાની ટીમો}: ટીમ દીઠ 2-6 લોકો
    \item \textbf{સમય-બદ્ધ}: કડક સમય મર્યાદા (60-90 દિવસ)
\end{itemize}

\textbf{RAD માટે આવશ્યકતાઓ}:
\begin{itemize}
    \item \textbf{સારી રીતે વ્યાખ્યાયિત વ્યવસાય આવશ્યકતાઓ}
    \item \textbf{વપરાશકર્તા સામેલગીરી} પ્રક્રિયા દરમિયાન
    \item \textbf{કુશળ ડેવલપર્સ} RAD સાધનોથી પરિચિત
\end{itemize}
\end{solutionbox}

\begin{mnemonicbox}
\mnemonic{Business Data Process Application Testing}
\end{mnemonicbox}

\questionmarks{2(c) OR}{7}{SDLC વ્યાખ્યાયિત કરો અને દરેક તબક્કા સમજાવો.}

\begin{solutionbox}
\textbf{SDLC (સોફ્ટવૅર ડેવલપમેન્ટ લાઇફ સાયકલ)} એ સારી રીતે વ્યાખ્યાયિત તબક્કાઓ દ્વારા સોફ્ટવૅર બનાવવા માટેની વ્યવસ્થિત પ્રક્રિયા છે.

\begin{center}
\begin{tikzpicture}[node distance=1.6cm, auto]
    \node [gtu state] (plan) {Planning};
    \node [gtu state, right of=plan, xshift=0.5cm] (ana) {Analysis};
    \node [gtu state, right of=ana, xshift=0.5cm] (des) {Design};
    \node [gtu state, below of=des] (imp) {Implementation};
    \node [gtu state, left of=imp, xshift=-0.5cm] (test) {Testing};
    \node [gtu state, left of=test, xshift=-0.5cm] (dep) {Deployment};
    \node [gtu state, above of=dep] (maint) {Maintenance};
    
    \draw [gtu arrow] (plan) -- (ana);
    \draw [gtu arrow] (ana) -- (des);
    \draw [gtu arrow] (des) -- (imp);
    \draw [gtu arrow] (imp) -- (test);
    \draw [gtu arrow] (test) -- (dep);
    \draw [gtu arrow] (dep) -- (maint);
    \draw [gtu arrow] (maint) -- (plan);
\end{tikzpicture}
\captionof{figure}{SDLC ચક્ર}
\end{center}

\begin{center}
\captionof{table}{SDLC તબક્કાઓ વિગતવાર}
\begin{tabulary}{\linewidth}{|L|L|L|}
\hline
\textbf{તબક્કો} & \textbf{પ્રવૃત્તિઓ} & \textbf{ડિલિવરી} \\ \hline
\textbf{Planning} & પ્રોજેક્ટ આયોજન, શક્યતા અભ્યાસ & પ્રોજેક્ટ પ્લાન \\ \hline
\textbf{Analysis} & આવશ્યકતા એકત્રીકરણ & SRS દસ્તાવેજ \\ \hline
\textbf{Design} & સિસ્ટમ આર્કિટેક્ચર, UI ડિઝાઇન & ડિઝાઇન દસ્તાવેજ \\ \hline
\textbf{Implementation} & કોડિંગ, યુનિટ ટેસ્ટિંગ & સોર્સ કોડ \\ \hline
\textbf{Testing} & સિસ્ટમ ટેસ્ટિંગ, ઇન્ટીગ્રેશન & ટેસ્ટ રિપોર્ટ્સ \\ \hline
\textbf{Deployment} & ઇન્સ્ટોલેશન, તાલીમ & લાઈવ સિસ્ટમ \\ \hline
\textbf{Maintenance} & બગ ફિક્સ, સુધારાઓ & અપડેટ સિસ્ટમ \\ \hline
\end{tabulary}
\end{center}

\textbf{તબક્કા વર્ણન}:
\begin{itemize}
    \item \textbf{Planning}: પ્રોજેક્ટ સ્કોપ અને સંસાધનો નક્કી કરવા
    \item \textbf{Analysis}: સિસ્ટમે શું કરવું જોઈએ તે સમજવું
    \item \textbf{Design}: સિસ્ટમ કેવી રીતે કામ કરશે તે પ્લાન કરવું
    \item \textbf{Implementation}: વાસ્તવિક સિસ્ટમ બનાવવી
    \item \textbf{Testing}: સિસ્ટમ યોગ્ય રીતે કાર્ય કરે છે તે ચકાસવું
    \item \textbf{Deployment}: વપરાશકર્તાઓ માટે સિસ્ટમ રજૂ કરવી
    \item \textbf{Maintenance}: ચાલુ સપોર્ટ અને અપડેટ્સ
\end{itemize}
\end{solutionbox}

\begin{mnemonicbox}
\mnemonic{People Always Design Implementation, Test Deployment, Maintain}
\end{mnemonicbox}

\questionmarks{3(a)}{3}{સોફ્ટવૅર પ્રોજેક્ટ્સ મેનેજ કરવા માટેની કુશળતા વર્ણવો.}

\begin{solutionbox}
\textbf{સોફ્ટવૅર પ્રોજેક્ટ મેનેજમેન્ટ} માટે તકનીકી અને સોફ્ટ સ્કિલ્સના સંયોજનની જરૂર છે.

\begin{center}
\captionof{table}{આવશ્યક પ્રોજેક્ટ મેનેજમેન્ટ સ્કિલ્સ}
\begin{tabulary}{\linewidth}{|L|L|}
\hline
\textbf{કૌશલ્ય શ્રેણી} & \textbf{વિશિષ્ટ કૌશલ્યો} \\ \hline
\textbf{ટેકનિકલ} & SDLC, સાધનો, તકનીકોની સમજ \\ \hline
\textbf{નેતૃત્વ} & ટીમ પ્રેરણા, નિર્ણય લેવાની ક્ષમતા \\ \hline
\textbf{સંવાદ} & ટીમ અને ક્લાયન્ટ્સ સાથે સ્પષ્ટ વાતચીત \\ \hline
\textbf{આયોજન} & સંસાધન ફાળવણી, સમયપત્રક \\ \hline
\textbf{સમસ્યા નિવારણ} & જોખમ સંચાલન, સંઘર્ષ નિવારણ \\ \hline
\end{tabulary}
\end{center}

\textbf{મુખ્ય કૌશલ્યો}:
\begin{itemize}
    \item \textbf{પીપલ મેનેજમેન્ટ}: ટીમ સભ્યોનું નેતૃત્વ અને પ્રેરણા
    \item \textbf{ટેકનિકલ જ્ઞાન}: વિકાસ પ્રક્રિયા અને સાધનોની સમજ
    \item \textbf{સંવાદ}: ટેકનિકલ ટીમ અને સ્ટેકહોલ્ડર્સ વચ્ચે સેતુ
\end{itemize}
\end{solutionbox}

\begin{mnemonicbox}
\mnemonic{Technical Leaders Communicate Planning Problems}
\end{mnemonicbox}

\questionmarks{3(b)}{4}{સોફ્ટવૅર પ્રોજેક્ટ મેનેજરની જવાબદારી સંક્ષિપ્તમાં લખો.}

\begin{solutionbox}
\textbf{સોફ્ટવૅર પ્રોજેક્ટ મેનેજર} શરૂઆતથી અંત સુધી સમગ્ર પ્રોજેક્ટની દેખરેખ રાખે છે.

\begin{center}
\captionof{table}{પ્રોજેક્ટ મેનેજરની જવાબદારીઓ}
\begin{tabulary}{\linewidth}{|L|L|}
\hline
\textbf{ક્ષેત્ર} & \textbf{જવાબદારીઓ} \\ \hline
\textbf{આયોજન} & પ્રોજેક્ટ પ્લાન, સમયપત્રક, બજેટ બનાવવા \\ \hline
\textbf{ટીમ મેનેજમેન્ટ} & ભરતી, તાલીમ, અને ટીમ મેનેજમેન્ટ \\ \hline
\textbf{સંવાદ} & સ્ટેકહોલ્ડર્સને નિયમિત અપડેટ્સ \\ \hline
\textbf{ગુણવત્તા નિયંત્રણ} & ડિલિવરેબલ્સ ગુણવત્તા ધોરણો પૂર્ણ કરે છે \\ \hline
\textbf{જોખમ સંચાલન} & પ્રોજેક્ટ જોખમો ઓળખવા અને ઘટાડવા \\ \hline
\end{tabulary}
\end{center}

\textbf{પ્રાથમિક જવાબદારીઓ}:
\begin{itemize}
    \item \textbf{પ્રોજેક્ટ આયોજન}: સ્કોપ, સમયરેખા અને સંસાધનો વ્યાખ્યાયિત કરવા
    \item \textbf{ટીમ નેતૃત્વ}: વિકાસ ટીમને માર્ગદર્શન અને સપોર્ટ
    \item \textbf{સ્ટેકહોલ્ડર સંવાદ}: પ્રગતિ વિશે સૌને માહિતગાર રાખવા
    \item \textbf{ગુણવત્તા ખાતરી}: પ્રોજેક્ટ આવશ્યકતાઓ પૂર્ણ કરે તે સુનિશ્ચિત કરવું
    \item \textbf{જોખમ સંચાલન}: પ્રોજેક્ટ જોખમો અને સમસ્યાઓ સંભાળવી
\end{itemize}
\end{solutionbox}

\begin{mnemonicbox}
\mnemonic{Plan Team Communication Quality Risk}
\end{mnemonicbox}

\questionmarks{3(c)}{7}{SRS માં આવશ્યકતાઓના પ્રકારોનું વર્ગીકરણ કરો (1) Functional Requirements (2) Non-Functional Requirements.}

\begin{solutionbox}
\textbf{આવશ્યકતા વર્ગીકરણ} વિવિધ પ્રકારની સિસ્ટમ જરૂરિયાતોને વ્યવસ્થિત અને સમજવામાં મદદ કરે છે.

\begin{center}
\captionof{table}{Functional વિ Non-Functional Requirements}
\begin{tabulary}{\linewidth}{|L|L|L|}
\hline
\textbf{પાસા} & \textbf{Functional Requirements} & \textbf{Non-Functional Requirements} \\ \hline
\textbf{વ્યાખ્યા} & સિસ્ટમે શું કરવું જોઈએ & સિસ્ટમ કેવી રીતે કામ કરવી જોઈએ \\ \hline
\textbf{ફોકસ} & સિસ્ટમ કાર્યક્ષમતા & સિસ્ટમ ગુણવત્તા લક્ષણો \\ \hline
\textbf{ઉદાહરણો} & લોગિન, સર્ચ, ગણતરી & પ્રદર્શન, સુરક્ષા, ઉપયોગિતા \\ \hline
\textbf{પરીક્ષણ} & કાર્યકારી પરીક્ષણ & પ્રદર્શન પરીક્ષણ \\ \hline
\end{tabulary}
\end{center}

\textbf{Functional Requirements}:
\begin{itemize}
    \item \textbf{વપરાશકર્તા ક્રિયાપ્રતિક્રિયાઓ}: લોગિન, નોંધણી, ડેટા એન્ટ્રી
    \item \textbf{વ્યવસાય નિયમો}: વેલિડેશન નિયમો, ગણતરીઓ
    \item \textbf{સિસ્ટમ સુવિધાઓ}: રિપોર્ટ્સ, સૂચનાઓ, વર્કફ્લો
    \item \textbf{ડેટા પ્રોસેસિંગ}: CRUD operations
\end{itemize}

\textbf{Non-Functional Requirements}:

\begin{center}
\captionof{table}{Non-Functional Requirement પ્રકારો}
\begin{tabulary}{\linewidth}{|L|L|L|}
\hline
\textbf{પ્રકાર} & \textbf{વર્ણન} & \textbf{ઉદાહરણ} \\ \hline
\textbf{Performance} & ઝડપ અને પ્રતિભાવ & Response time < 2 સેકન્ડ \\ \hline
\textbf{Security} & ડેટા સુરક્ષા & Encrypted ડેટા ટ્રાન્સમિશન \\ \hline
\textbf{Usability} & વપરાશકર્તા અનુભવ & શીખવા માટે સરળ ઈન્ટરફેસ \\ \hline
\textbf{Reliability} & સિસ્ટમ વિશ્વસનીયતા & 99.9\% અપટાઇમ \\ \hline
\textbf{Scalability} & વૃદ્ધિ સંચાલન & 1000+ યુઝર્સ સપોર્ટ \\ \hline
\end{tabulary}
\end{center}
\end{solutionbox}

\begin{mnemonicbox}
\mnemonic{Performance Security Usability Reliability Maintainability}
\end{mnemonicbox}

\questionmarks{3(a) OR}{3}{SRS નું મહત્વ સમજાવો.}

\begin{solutionbox}
\textbf{SRS (સોફ્ટવૅર આવશ્યકતા વિશિષ્ટતા)} એક મહત્વપૂર્ણ દસ્તાવેજ છે જે વ્યાખ્યાયિત કરે છે કે સોફ્ટવૅરે શું કરવું જોઈએ.

\begin{center}
\captionof{table}{SRS નું મહત્વ}
\begin{tabulary}{\linewidth}{|L|L|}
\hline
\textbf{પાસા} & \textbf{લાભ} \\ \hline
\textbf{સ્પષ્ટ સંવાદ} & બધા સ્ટેકહોલ્ડર્સ આવશ્યકતાઓ સમજે છે \\ \hline
\textbf{પ્રોજેક્ટ આયોજન} & અંદાજ અને શેડ્યુલિંગ માટે આધાર \\ \hline
\textbf{ગુણવત્તા ખાતરી} & પરીક્ષણ માટે પાયો \\ \hline
\textbf{ફેરફાર વ્યવસ્થાપન} & નિયંત્રિત આવશ્યકતા ફેરફારો \\ \hline
\textbf{કાનૂની સુરક્ષા} & કરાર સંદર્ભ દસ્તાવેજ \\ \hline
\end{tabulary}
\end{center}
\end{solutionbox}

\begin{mnemonicbox}
\mnemonic{Clear Planning Quality Change Legal}
\end{mnemonicbox}

\questionmarks{3(b) OR}{4}{Gantt Chart સમજાવો.}

\begin{solutionbox}
\textbf{Gantt Chart} કાર્યો, સમયરેખા અને અવલંબન દર્શાવતું વિઝ્યુઅલ પ્રોજેક્ટ મેનેજમેન્ટ સાધન છે.

\begin{center}
\begin{tikzpicture}[scale=0.7]
    \draw[->] (0,0) -- (12,0) node[right] {Time};
    \draw[->] (0,0) -- (0,5) node[above] {Tasks};
    
    % Tasks
    \node[right] at (0, 4.5) {Requirements};
    \draw[fill=headcolor] (1,4.2) rectangle (4,4.8);
    
    \node[right] at (0, 3.5) {Design};
    \draw[fill=headcolor] (4,3.2) rectangle (6,3.8);
    
    \node[right] at (0, 2.5) {Coding};
    \draw[fill=headcolor] (6,2.2) rectangle (9,2.8);
    
    \node[right] at (0, 1.5) {Testing};
    \draw[fill=headcolor] (9,1.2) rectangle (11,1.8);
    
    % Dependencies
    \draw[dashed, ->] (4,4.5) -- (4,3.8);
    \draw[dashed, ->] (6,3.5) -- (6,2.8);
    \draw[dashed, ->] (9,2.5) -- (9,1.8);
    
    % Grid lines (optional for visual aid)
    \foreach \x in {1,...,11} \draw[gray!30] (\x,0) -- (\x,5);
    
\end{tikzpicture}
\captionof{figure}{સરળ Gantt Chart}
\end{center}

\begin{center}
\captionof{table}{Gantt Chart ઘટકો}
\begin{tabulary}{\linewidth}{|L|L|}
\hline
\textbf{ઘટક} & \textbf{વર્ણન} \\ \hline
\textbf{Tasks} & પૂર્ણ કરવાના કામો \\ \hline
\textbf{Timeline} & આડી સમયરેખા \\ \hline
\textbf{Bars} & કાર્ય અવધિ અને પ્રગતિ \\ \hline
\textbf{Dependencies} & કાર્ય સંબંધો \\ \hline
\textbf{Milestones} & મહત્વપૂર્ણ પ્રોજેક્ટ ઘટનાઓ \\ \hline
\end{tabulary}
\end{center}
\end{solutionbox}

\begin{mnemonicbox}
\mnemonic{Tasks Timeline Bars Dependencies Milestones}
\end{mnemonicbox}

\questionmarks{3(c) OR}{7}{Risk Management પર ટૂંક નોંધ લખો.}

\begin{solutionbox}
\textbf{Risk Management} એ પ્રોજેક્ટ જોખમોને ઓળખવા, વિશ્લેષણ કરવા અને નિયંત્રિત કરવાની વ્યવસ્થિત પ્રક્રિયા છે.

\begin{center}
\begin{tikzpicture}[node distance=2.5cm, auto]
    \node [gtu state] (id) {Risk\\Identification};
    \node [gtu state, right of=id] (ana) {Risk\\Analysis};
    \node [gtu state, right of=ana] (plan) {Risk\\Planning};
    \node [gtu state, right of=plan] (mon) {Risk\\Monitoring};

    \draw [gtu arrow] (id) -- (ana);
    \draw [gtu arrow] (ana) -- (plan);
    \draw [gtu arrow] (plan) -- (mon);
    \draw [gtu arrow] (mon)edge [bend left=30] (id);
\end{tikzpicture}
\captionof{figure}{Risk Management ચક્ર}
\end{center}

\begin{center}
\captionof{table}{Risk Management પ્રક્રિયા}
\begin{tabulary}{\linewidth}{|L|L|L|}
\hline
\textbf{તબક્કો} & \textbf{પ્રવૃત્તિઓ} & \textbf{આઉટપુટ} \\ \hline
\textbf{Identification} & સંભવિત જોખમો શોધો & Risk list \\ \hline
\textbf{Analysis} & સંભાવના અને અસરનું મૂલ્યાંકન & Risk priority \\ \hline
\textbf{Planning} & પ્રતિભાવ વ્યૂહરચના વિકસાવો & Risk response plan \\ \hline
\textbf{Monitoring} & જોખમો ટ્રેક અને કંટ્રોલ કરો & Updated risk status \\ \hline
\end{tabulary}
\end{center}

\textbf{Risk શ્રેણીઓ}:

\begin{center}
\captionof{table}{સોફ્ટવૅર જોખમોના પ્રકારો}
\begin{tabulary}{\linewidth}{|L|L|}
\hline
\textbf{શ્રેણી} & \textbf{ઉદાહરણો} \\ \hline
\textbf{Technical} & ટેકનોલોજી ફેરફારો, જટિલતા \\ \hline
\textbf{Project} & શિડ્યુલ વિલંબ, સંસાધન અછત \\ \hline
\textbf{Business} & બજાર ફેરફારો, ભંડોળ મુદ્દાઓ \\ \hline
\textbf{External} & વેન્ડર સમસ્યાઓ, નિયમનકારી ફેરફારો \\ \hline
\end{tabulary}
\end{center}

\textbf{Risk પ્રતિભાવ વ્યૂહરચનાઓ}:
\begin{itemize}
    \item \textbf{Avoid}: જોખમ સ્ત્રોત દૂર કરો
    \item \textbf{Mitigate}: સંભાવના અથવા અસર ઘટાડો
    \item \textbf{Transfer}: જોખમ અન્ય સાથે શેર કરો
    \item \textbf{Accept}: જોખમ સાથે જીવો
\end{itemize}
\end{solutionbox}

\begin{mnemonicbox}
\mnemonic{Identify Analyze Plan Monitor (Process), Avoid Mitigate Transfer Accept (Strategies)}
\end{mnemonicbox}

\questionmarks{4(a)}{3}{Size estimation માટે મેટ્રિક શું છે? ઉદાહરણ સાથે FP સમજાવો.}

\begin{solutionbox}
\textbf{Size Estimation Metrics} સોફ્ટવૅર પ્રોજેક્ટ કદ અને પ્રયત્નોની આગાહી કરવામાં મદદ કરે છે.

\begin{center}
\captionof{table}{Size Estimation Metrics}
\begin{tabulary}{\linewidth}{|L|L|}
\hline
\textbf{મેટ્રિક} & \textbf{વર્ણન} \\ \hline
\textbf{LOC} & Lines of Code \\ \hline
\textbf{Function Points} & Functionality-આધારિત માપદંડ \\ \hline
\textbf{Object Points} & Object-oriented સિસ્ટમ્સ માટે \\ \hline
\textbf{Feature Points} & ઉન્નત ફંક્શન પોઈન્ટ્સ \\ \hline
\end{tabulary}
\end{center}

\textbf{Function Points (FP)} વપરાશકર્તા કાર્યક્ષમતાના આધારે સોફ્ટવૅર કદને માપે છે.

\textbf{FP ઘટકો}:
\begin{itemize}
    \item \textbf{External Inputs}: ડેટા એન્ટ્રી સ્ક્રીન્સ
    \item \textbf{External Outputs}: રિપોર્ટ્સ, સંદેશાઓ
    \item \textbf{External Queries}: ડેટાબેઝ પ્રશ્નો
    \item \textbf{Internal Files}: ડેટા સ્ટોર્સ
    \item \textbf{External Interfaces}: સિસ્ટમ કનેક્શન્સ
\end{itemize}

\textbf{FP ગણતરી ઉદાહરણ}:
Library Management System માટે:
\begin{itemize}
    \item External Inputs: 5 (Book entry, Member entry, etc.)
    \item External Outputs: 3 (Reports)
    \item External Queries: 4 (Search functions)
    \item Internal Files: 2 (Book DB, Member DB)
    \item External Interfaces: 1 (Online catalog)
\end{itemize}

\textbf{Simple FP = 5 + 3 + 4 + 2 + 1 = 15 Function Points}
\end{solutionbox}

\begin{mnemonicbox}
\mnemonic{Inputs Outputs Queries Files Interfaces}
\end{mnemonicbox}

\questionmarks{4(b)}{4}{Basic COCOMO મોડેલનો ઉપયોગ કરીને પ્રોજેક્ટ અંદાજ તકનીકો સમજાવો.}

\begin{solutionbox}
\textbf{COCOMO (COnstructive COst MOdel)} સોફ્ટવૅર વિકાસ પ્રયત્ન અને શેડ્યુલનો અંદાજ કાઢે છે.

\begin{center}
\captionof{table}{COCOMO મોડેલ પ્રકારો}
\begin{tabulary}{\linewidth}{|L|L|L|}
\hline
\textbf{પ્રકાર} & \textbf{વર્ણન} & \textbf{ચોકસાઈ} \\ \hline
\textbf{Basic} & સરળ કદ-આધારિત અંદાજ & $\pm$75\% \\ \hline
\textbf{Intermediate} & કિંમત ડ્રાઇવરો સમાવે છે & $\pm$25\% \\ \hline
\textbf{Detailed} & તબક્કા-સ્તર અંદાજ & $\pm$10\% \\ \hline
\end{tabulary}
\end{center}

\textbf{Basic COCOMO ફોર્મ્યુલા}:
\begin{itemize}
    \item \textbf{Effort} = $a \times (KLOC)^b$ person-months
    \item \textbf{Time} = $c \times (Effort)^d$ months
    \item \textbf{People} = Effort / Time
\end{itemize}

\begin{center}
\captionof{table}{COCOMO Constants}
\begin{tabulary}{\linewidth}{|L|C|C|C|C|}
\hline
\textbf{Project Type} & \textbf{a} & \textbf{b} & \textbf{c} & \textbf{d} \\ \hline
\textbf{Organic} & 2.4 & 1.05 & 2.5 & 0.38 \\ \hline
\textbf{Semi-detached} & 3.0 & 1.12 & 2.5 & 0.35 \\ \hline
\textbf{Embedded} & 3.6 & 1.20 & 2.5 & 0.32 \\ \hline
\end{tabulary}
\end{center}
\end{solutionbox}

\begin{mnemonicbox}
\mnemonic{Organic Semi Embedded}
\end{mnemonicbox}

\questionmarks{4(c)}{7}{તમારી પસંદગીની સિસ્ટમ માટે Sprint burn down chart તૈયાર કરો.}

\begin{solutionbox}
\textbf{Sprint Burn Down Chart} sprint દરમિયાન બાકી રહેલા કામને ટ્રેક કરે છે. ઉદાહરણ: \textbf{Online Shopping System}.

\textbf{Sprint Goal}: User Authentication Module (40 Story Points, 2 Weeks Estimate)

\begin{center}
\begin{tikzpicture}[scale=0.8]
    % Axes
    \draw[->] (0,0) -- (11,0) node[right] {Day};
    \draw[->] (0,0) -- (0,9) node[above] {Story Points};
    
    % Y-axis labels
    \foreach \y/\label in {0/0, 2/10, 4/20, 6/30, 8/40}
        \node at (-0.5,\y) {\label};
    
    % X-axis labels
    \foreach \x/\label in {1/2, 2/4, 3/6, 4/8, 5/10, 6/12, 7/14}
        \node at (\x*1.4,-0.5) {\label};
    
    % Ideal line
    \draw[dashed, thick] (0,8) -- (10,0);
    
    % Actual progress
    \draw[thick, headcolor] (0,8) -- (1.4,7.6) -- (2.8,7) -- (4.2,6.4) -- (5.6,5.4) -- (7,5.0) -- (8.4,4) -- (10,0);
    
    % Legend
    \node[right] at (8,8) {Ideal};
    \node[right, headcolor] at (8,7.5) {Actual};
\end{tikzpicture}
\captionof{figure}{Sprint Burn Down Chart}
\end{center}

\textbf{Chart Analysis}:
\begin{itemize}
    \item \textbf{Green line}: Ideal burn down
    \item \textbf{Red line}: Actual progress  
    \item \textbf{પૂર્ણતા}: Sprint સમયસર પૂર્ણ થયું
\end{itemize}

\textbf{ફાયદા}: વિઝ્યુઅલ પ્રગતિ ટ્રેકિંગ, પ્રારંભિક સમસ્યા ઓળખ, ટીમ પ્રેરણા
\end{solutionbox}

\begin{mnemonicbox}
\mnemonic{Track Progress Daily, Identify Issues Early}
\end{mnemonicbox}

\questionmarks{4(a) OR}{3}{USE CASE ડાયાગ્રામના ઘટક સમજાવો.}

\begin{solutionbox}
\textbf{Use Case Diagram} વપરાશકર્તા પરિપ્રેક્ષ્યમાં સિસ્ટમ કાર્યક્ષમતા દર્શાવે છે.

\begin{center}
\captionof{table}{Use Case Diagram ઘટકો}
\begin{tabulary}{\linewidth}{|L|C|L|}
\hline
\textbf{ઘટક} & \textbf{પ્રતીક} & \textbf{વર્ણન} \\ \hline
\textbf{Actor} & Stick figure & બાહ્ય એન્ટિટી \\ \hline
\textbf{Use Case} & Oval & સિસ્ટમ કાર્યક્ષમતા \\ \hline
\textbf{System Boundary} & Rectangle & સિસ્ટમ સ્કોપ \\ \hline
\textbf{Association} & Line & Actor-Use Case સંબંધ \\ \hline
\textbf{Generalization} & Arrow & વારસાગત સંબંધ \\ \hline
\end{tabulary}
\end{center}

\textbf{સંબંધો}:
\begin{itemize}
    \item \textbf{Include}: એક use case બીજાને સમાવે છે (ફરજિયાત)
    \item \textbf{Extend}: વૈકલ્પિક use case વિસ્તરણ
\end{itemize}
\end{solutionbox}

\begin{mnemonicbox}
\mnemonic{Actors Use Systems, Associate Generally}
\end{mnemonicbox}

\questionmarks{4(b) OR}{4}{Cohesion અને Coupling સરખાવો.}

\begin{solutionbox}
\textbf{Cohesion અને Coupling} જાળવણીક્ષમતા પર અસર કરતા મહત્વપૂર્ણ સોફ્ટવૅર ડિઝાઇન સિદ્ધાંતો છે.

\begin{center}
\captionof{table}{Cohesion વિ Coupling}
\begin{tabulary}{\linewidth}{|L|L|L|}
\hline
\textbf{પાસા} & \textbf{Cohesion} & \textbf{Coupling} \\ \hline
\textbf{વ્યાખ્યા} & મોડ્યુલની અંદર એકતા & મોડ્યુલો વચ્ચે અવલંબન \\ \hline
\textbf{ઇચ્છનીય સ્તર} & ઉચ્ચ (High) & ઓછું (Low) \\ \hline
\textbf{ફોકસ} & આંતરિક મોડ્યુલ એકતા & આંતર-મોડ્યુલ સંબંધો \\ \hline
\textbf{અસર} & મોડ્યુલ વિશ્વસનીયતા & સિસ્ટમ લવચીકતા \\ \hline
\end{tabulary}
\end{center}

\textbf{લક્ષ્ય}: \textbf{High Cohesion + Low Coupling = Good Design}
\end{solutionbox}

\begin{mnemonicbox}
\mnemonic{High Cohesion, Low Coupling}
\end{mnemonicbox}

\questionmarks{4(c) OR}{7}{Risk Assessment વિગતવાર સમજાવો.}

\begin{solutionbox}
\textbf{Risk Assessment} મેનેજમેન્ટ પ્રયત્નોને પ્રાથમિકતા આપવા ઓળખાયેલા જોખમોનું મૂલ્યાંકન કરે છે.

\begin{center}
\begin{tikzpicture}[node distance=2cm, auto]
    \node [gtu block] (id) {Risk Identification};
    \node [gtu block, right of=id, xshift=1.5cm] (assess) {Risk Assessment};
    
    \node [gtu state, below of=assess, xshift=-1.5cm] (prob) {Probability Analysis};
    \node [gtu state, below of=assess, xshift=1.5cm] (imp) {Impact Analysis};
    
    \node [gtu block, below of=assess, yshift=-2cm] (exp) {Risk Exposure};
    \node [gtu block, below of=exp] (prio) {Risk Prioritization};
    
    \draw [gtu arrow] (id) -- (assess);
    \draw [gtu arrow] (assess) -- (prob);
    \draw [gtu arrow] (assess) -- (imp);
    \draw [gtu arrow] (prob) -- (exp);
    \draw [gtu arrow] (imp) -- (exp);
    \draw [gtu arrow] (exp) -- (prio);
\end{tikzpicture}
\captionof{figure}{Risk Assessment પ્રક્રિયા}
\end{center}

\begin{center}
\captionof{table}{Risk Assessment ઘટકો}
\begin{tabulary}{\linewidth}{|L|L|L|}
\hline
\textbf{ઘટક} & \textbf{વર્ણન} & \textbf{સ્કેલ} \\ \hline
\textbf{Probability} & જોખમની સંભાવના & 0.1 to 1.0 \\ \hline
\textbf{Impact} & જોખમ થાય તો પરિણામો & 1 to 10 \\ \hline
\textbf{Risk Exposure} & Probability $\times$ Impact & ગણતરી કરેલ કિંમત \\ \hline
\textbf{Risk Level} & પ્રાથમિકતા વર્ગીકરણ & High/Medium/Low \\ \hline
\end{tabulary}
\end{center}

\textbf{મૂલ્યાંકન પ્રક્રિયા}:
1. \textbf{Probability Assessment}: 0.1 (Very Low) - 0.9 (Very High)
2. \textbf{Impact Assessment}: 1-10 સ્કેલ
3. \textbf{Risk Exposure}: $P \times I$

\textbf{Risk Matrix}:
High (> 4.0), Medium (2.0-4.0), Low (< 2.0)
\end{solutionbox}

\begin{mnemonicbox}
\mnemonic{Probability Impact Exposure Priority}
\end{mnemonicbox}

\questionmarks{5(a)}{3}{કોડ રિવ્યૂ માં Code Inspection તકનીક સમજાવો.}

\begin{solutionbox}
\textbf{Code Inspection} એ ખામીઓ શોધવા માટે કોડની ઔપચારિક, વ્યવસ્થિત તપાસ છે.

\begin{center}
\captionof{table}{Code Inspection પ્રક્રિયા}
\begin{tabulary}{\linewidth}{|L|L|L|}
\hline
\textbf{તબક્કો} & \textbf{સહભાગીઓ} & \textbf{પ્રવૃત્તિઓ} \\ \hline
\textbf{Planning} & Moderator & શેડ્યુલ, કોડ વિતરણ \\ \hline
\textbf{Overview} & Author, Team & Author કોડ સમજાવે છે \\ \hline
\textbf{Preparation} & Individual & રિવ્યુઅર્સ કોડ અભ્યાસ કરે છે \\ \hline
\textbf{Inspection} & All reviewers & ખામીઓ શોધવી \\ \hline
\textbf{Rework} & Author & ખામીઓ સુધારવી \\ \hline
\textbf{Follow-up} & Moderator & સુધારાઓ ચકાસવા \\ \hline
\end{tabulary}
\end{center}
\end{solutionbox}

\begin{mnemonicbox}
\mnemonic{Plan Overview Prepare Inspect Rework Follow-up}
\end{mnemonicbox}

\questionmarks{5(b)}{4}{ATM ના ઓછામાં ઓછા ચાર Test Cases તૈયાર કરો.}

\begin{solutionbox}
\textbf{ATM Test Cases}:

\begin{center}
\captionof{table}{ATM Test Cases}
\begin{tabulary}{\linewidth}{|L|L|L|L|L|}
\hline
\textbf{ID} & \textbf{પરિસ્થિતિ} & \textbf{ઇનપુટ} & \textbf{આઉટપુટ} & \textbf{પરિણામ} \\ \hline
\textbf{TC1} & Valid PIN & સાચો PIN & એક્સેસ મંજૂર & Pass \\ \hline
\textbf{TC2} & Invalid PIN & ખોટો PIN & કાર્ડ બ્લોક & Pass \\ \hline
\textbf{TC3} & Withdrawal & રકમ $\le$ બેલેન્સ & રોકડ વિતરણ & Pass \\ \hline
\textbf{TC4} & Low Balance & રકમ $>$ બેલેન્સ & નકારવામાં આવ્યું & Pass \\ \hline
\end{tabulary}
\end{center}
\end{solutionbox}

\begin{mnemonicbox}
\mnemonic{Login Withdraw Inquiry Change}
\end{mnemonicbox}

\questionmarks{5(c)}{7}{White box testing વર્ણવો.}

\begin{solutionbox}
\textbf{White Box Testing} આંતરિક કોડ માળખું અને લોજિક પાથ તપાસે છે.

\begin{center}
\begin{tikzpicture}[node distance=1.5cm, auto]
    \node [gtu block] (code) {Source Code};
    \node [gtu block, right of=code, xshift=1cm] (cf) {Control Flow};
    \node [gtu block, right of=cf, xshift=1cm] (path) {Path Coverage};
    \node [gtu block, below of=code] (design) {Test Design};
    \node [gtu block, below of=cf] (exec) {Execution};
    \node [gtu block, below of=path] (anl) {Coverage Analysis};
    
    \draw [gtu arrow] (code) -- (cf);
    \draw [gtu arrow] (cf) -- (path);
    \draw [gtu arrow] (path) -- (design);
    \draw [gtu arrow] (design) -- (exec);
    \draw [gtu arrow] (exec) -- (anl);
\end{tikzpicture}
\captionof{figure}{White Box Testing પ્રક્રિયા}
\end{center}

\textbf{Coverage Criteria}:
\begin{itemize}
    \item \textbf{Statement Coverage}: દરેક સ્ટેટમેન્ટ એક્ઝિક્યુટ કરો
    \item \textbf{Branch Coverage}: બધા if-else પાથ એક્ઝિક્યુટ કરો
    \item \textbf{Path Coverage}: દરેક શક્ય પાથ એક્ઝિક્યુટ કરો
    \item \textbf{Condition Coverage}: બધી શરતો (true/false) ટેસ્ટ કરો
\end{itemize}

\textbf{ફાયદા}: સંપૂર્ણ પરીક્ષણ, લોજિક ભૂલો શોધે છે \\
\textbf{ગેરફાયદા}: સમય માંગી લે છે, મોંઘું, કોડ આધારિત
\end{solutionbox}

\begin{mnemonicbox}
\mnemonic{Statement Branch Path Condition}
\end{mnemonicbox}

\questionmarks{5(a) OR}{3}{કોડ રિવ્યૂ માં Code Walk Through તકનીક સમજાવો.}

\begin{solutionbox}
\textbf{Code Walk Through} એ અનૌપચારિક કોડ રિવ્યૂ તકનીક છે જ્યાં લેખક ટીમને કોડ રજૂ કરે છે.

\begin{center}
\captionof{table}{Walk Through પ્રક્રિયા}
\begin{tabulary}{\linewidth}{|L|L|L|}
\hline
\textbf{તબક્કો} & \textbf{વર્ણન} & \textbf{સમય} \\ \hline
\textbf{Preparation} & લેખક પ્રસ્તુતિ તૈયાર કરે છે & 30 min \\ \hline
\textbf{Presentation} & લેખક કોડ લોજિક સમજાવે છે & 1-2 hours \\ \hline
\textbf{Discussion} & ટીમ પ્રશ્નો પૂછે છે & 30 min \\ \hline
\textbf{Documentation} & મુદ્દાઓ રેકોર્ડ કરો & 15 min \\ \hline
\end{tabulary}
\end{center}
\end{solutionbox}

\begin{mnemonicbox}
\mnemonic{Prepare Present Discuss Document}
\end{mnemonicbox}

\questionmarks{5(b) OR}{4}{સોફ્ટવૅર દસ્તાવેજીકરણ સમજાવો.}

\begin{solutionbox}
\textbf{સોફ્ટવૅર દસ્તાવેજીકરણ} વિવિધ સ્ટેકહોલ્ડર્સ માટે સોફ્ટવૅર સિસ્ટમ વિશે માહિતી પ્રદાન કરે છે.

\begin{center}
\captionof{table}{દસ્તાવેજીકરણ પ્રકારો}
\begin{tabulary}{\linewidth}{|L|L|L|}
\hline
\textbf{પ્રકાર} & \textbf{હેતુ} & \textbf{પ્રેક્ષકો} \\ \hline
\textbf{User Doc} & સોફ્ટવૅર કેવી રીતે વાપરવું & End users \\ \hline
\textbf{System Doc} & ટેકનિકલ વિગતો & Developers \\ \hline
\textbf{Process Doc} & વિકાસ પ્રક્રિયા & Project team \\ \hline
\textbf{Requirements} & સિસ્ટમે શું કરવું જોઈએ & Stakeholders \\ \hline
\end{tabulary}
\end{center}

\textbf{Internal}: Code comments, README. \\
\textbf{External}: User manuals, API docs.
\end{solutionbox}

\begin{mnemonicbox}
\mnemonic{User System Process Requirements}
\end{mnemonicbox}

\questionmarks{5(c) OR}{7}{Black box testing પર ટૂંક નોંધ લખો.}

\begin{solutionbox}
\textbf{Black Box Testing} આંતરિક કોડ માળખાના જ્ઞાન વિના સોફ્ટવૅર કાર્યક્ષમતા તપાસે છે.

\begin{center}
\begin{tikzpicture}[node distance=1.5cm, auto]
    \node [gtu block] (input) {Input};
    \node [gtu block, right of=input, xshift=1.5cm, fill=black!80, text=white] (bb) {Black Box\\System};
    \node [gtu block, right of=bb, xshift=1.5cm] (output) {Output};
    
    \node [gtu state, below of=input] (tc) {Test Cases};
    \node [gtu state, below of=output] (exp) {Expected};
    \node [gtu block, below of=bb] (compare) {Compare};
    
    \draw [gtu arrow] (input) -- (bb);
    \draw [gtu arrow] (bb) -- (output);
    \draw [gtu arrow] (tc) -- (input);
    \draw [gtu arrow] (exp) -- (compare);
    \draw [gtu arrow] (output) -- (compare);
\end{tikzpicture}
\captionof{figure}{Black Box Testing ખ્યાલ}
\end{center}

\begin{center}
\captionof{table}{Testing તકનીકો}
\begin{tabulary}{\linewidth}{|L|L|L|}
\hline
\textbf{તકનીક} & \textbf{વર્ણન} & \textbf{ઉદાહરણ} \\ \hline
\textbf{Equivalence Partitioning} & Valid/Invalid વર્ગો & Age: 0-17, 18-65, >65 \\ \hline
\textbf{Boundary Value} & સીમાઓ પર પરીક્ષણ & Age: 17, 18, 65, 66 \\ \hline
\textbf{Decision Table} & જટિલ નિયમો & વીમા પ્રીમિયમ \\ \hline
\textbf{State Transition} & સ્થિતિ ફેરફારો & ATM states \\ \hline
\end{tabulary}
\end{center}

\textbf{સરખામણી}:
\begin{itemize}
    \item \textbf{User perspective} vs Code perspective
    \item \textbf{No code knowledge needed} vs Programming skills
    \item \textbf{Early testing} vs Testing after code
\end{itemize}
\end{solutionbox}

\begin{mnemonicbox}
\mnemonic{Equivalence Boundary Decision State}
\end{mnemonicbox}

\end{document}
