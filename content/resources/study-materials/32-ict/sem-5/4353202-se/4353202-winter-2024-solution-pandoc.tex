\documentclass[10pt,a4paper]{article}

% content/resources/templates/preamble.tex
\usepackage[margin=0.6in]{geometry}
\author{Milav Dabgar}
\usepackage{amsmath,amssymb,amsthm}
\usepackage{booktabs}
\usepackage{multirow}
\usepackage{xcolor}
\usepackage{tcolorbox}
\tcbuselibrary{breakable,skins}
\usepackage[colorlinks=true,linkcolor=blue]{hyperref}
\usepackage{titlesec}
\usepackage{enumitem}
\usepackage{tikz}
\usepackage{pgfplots}
\usepackage{circuitikz}
\usepackage[version=4]{mhchem}
\usepackage{longtable}
\usepackage{array}
\usepackage{float}
\usepackage{caption}
\usepackage{listings}

\lstset{
  basicstyle=\small\ttfamily,
  breaklines=true,
  breakatwhitespace=false,
  postbreak=\mbox{\textcolor{red}{$\hookrightarrow$}\space},
  float=false,
  numbers=left,
  numberstyle=\tiny\color{gray},
  numbersep=10pt,
  xleftmargin=2em,
  keywordstyle=\color{blue},
  commentstyle=\color{green!60!black},
  stringstyle=\color{purple},
  backgroundcolor=\color{gray!5},
  showstringspaces=false,
  tabsize=2,
  captionpos=b,
  keepspaces=true,
  columns=flexible
}

\pgfplotsset{compat=1.18}
\usetikzlibrary{shapes,arrows,positioning,calc,patterns,decorations.pathmorphing,decorations.markings,arrows.meta}

% Color scheme
\definecolor{headcolor}{RGB}{0,102,204}
\definecolor{keycolor}{RGB}{220,20,60}
\definecolor{solutioncolor}{RGB}{34,139,34}
\definecolor{mnemoniccolor}{RGB}{148,0,211}
\definecolor{codecolor}{RGB}{0,0,100}

% Spacing
\setlength{\parskip}{3pt}
\setlist[itemize]{nosep}
\setlist[enumerate]{nosep}

% Title formatting
\titleformat{\section}{\Large\bfseries\color{headcolor}}{\thesection}{1em}{}
\titleformat{\subsection}{\large\bfseries\color{headcolor}}{\thesubsection}{1em}{}

% Pandoc tightlist compatibility
\providecommand{\tightlist}{%
  \setlength{\itemsep}{0pt}\setlength{\parskip}{0pt}}

% Pandoc longtable compatibility
\newcounter{none}
\def\thenone{}


% content/resources/templates/english-boxes.tex
% This file is currently empty - it exists to maintain consistency with the import structure.
% Add custom environments here if needed in the future.


\begin{document}

\begin{center}
{\Huge\bfseries\color{headcolor} Subject Name Solutions}\\[5pt]
{\LARGE 4353202 -- Winter 2024}\\[3pt]
{\large Semester 1 Study Material}\\[3pt]
{\normalsize\textit{Detailed Solutions and Explanations}}
\end{center}

\vspace{10pt}

\subsection*{Question 1(a) [3 marks]}\label{q1a}

\textbf{Define software and explain its characteristics.}

\begin{solutionbox}

\textbf{Software} is a collection of computer programs, procedures, and
documentation that performs tasks on a computer system.


{\def\LTcaptype{none} % do not increment counter
\vspace{-5pt}
\captionof{table}{Software Characteristics}
\vspace{-10pt}
\begin{longtable}[]{@{}ll@{}}
\toprule\noalign{}
Characteristic & Description \\
\midrule\noalign{}
\endhead
\bottomrule\noalign{}
\endlastfoot
\textbf{Intangible} & Cannot be touched, only experienced \\
\textbf{Developed} & Engineered, not manufactured \\
\textbf{Maintainable} & Can be modified and updated \\
\textbf{Reliable} & Should work consistently \\
\textbf{Efficient} & Uses resources optimally \\
\end{longtable}
}

\begin{itemize}
\tightlist
\item
  \textbf{Key point}: Software = Programs + Documentation + Procedures
\item
\end{solutionbox}
\begin{mnemonicbox}
``I Don't Make Reliable Electronics'' (Intangible,
  Developed, Maintainable, Reliable, Efficient)
\end{itemize}

\end{mnemonicbox}
\begin{center}\rule{0.5\linewidth}{0.5pt}\end{center}

\subsection*{Question 1(b) [4 marks]}\label{q1b}

\textbf{Explain classical waterfall model.}

\begin{solutionbox}

\textbf{Waterfall Model} is a linear sequential software development
approach where each phase must be completed before the next begins.

\includegraphics[width=1\linewidth,height=\textheight,keepaspectratio]{mermaid-13a0ca14.pdf}

\textbf{Key Features}:

\begin{itemize}
\tightlist
\item
  \textbf{Sequential phases}: No overlap between phases
\item
  \textbf{Documentation-driven}: Heavy documentation at each phase
\item
  \textbf{Simple structure}: Easy to understand and manage
\item
  \textbf{Fixed requirements}: Changes are difficult once started
\end{itemize}

\end{solutionbox}
\begin{mnemonicbox}
``Real Systems Include Testing, Deployment,
Maintenance''

\end{mnemonicbox}
\begin{center}\rule{0.5\linewidth}{0.5pt}\end{center}

\subsection*{Question 1(c) [7 marks]}\label{q1c}

\textbf{Explain software process framework and umbrella activities.}

\begin{solutionbox}

\textbf{Software Process Framework} provides the foundation for complete
software engineering process by identifying key process areas.

\includegraphics[width=1\linewidth,height=\textheight,keepaspectratio]{mermaid-2eabb121.pdf}


{\def\LTcaptype{none} % do not increment counter
\vspace{-5pt}
\captionof{table}{Framework Activities vs Umbrella Activities}
\vspace{-10pt}
\begin{longtable}[]{@{}ll@{}}
\toprule\noalign{}
Framework Activities & Umbrella Activities \\
\midrule\noalign{}
\endhead
\bottomrule\noalign{}
\endlastfoot
Communication & Software project tracking \\
Planning & Risk management \\
Modeling & Quality assurance \\
Construction & Technical reviews \\
Deployment & Configuration management \\
\end{longtable}
}

\textbf{Framework Activities}:

\begin{itemize}
\tightlist
\item
  \textbf{Communication}: Gather requirements from stakeholders
\item
  \textbf{Planning}: Create project plan and schedule
\item
  \textbf{Modeling}: Create design models
\item
  \textbf{Construction}: Code generation and testing
\item
  \textbf{Deployment}: Software delivery and feedback
\end{itemize}

\textbf{Umbrella Activities} run throughout the project:

\begin{itemize}
\tightlist
\item
  \textbf{Project tracking}: Monitor progress
\item
  \textbf{Risk management}: Identify and control risks
\item
  \textbf{Quality assurance}: Ensure quality standards
\item
  \textbf{Configuration management}: Control changes
\end{itemize}

\end{solutionbox}
\begin{mnemonicbox}
``Can People Make Construction Deploy''
(Communication, Planning, Modeling, Construction, Deployment)

\end{mnemonicbox}
\begin{center}\rule{0.5\linewidth}{0.5pt}\end{center}

\subsection*{Question 1(c) OR [7
marks]}\label{q1c}

\textbf{Write a short note on SCRUM.}

\begin{solutionbox}

\textbf{SCRUM} is an agile framework for managing software development
projects using iterative and incremental practices.

\includegraphics[width=1\linewidth,height=\textheight,keepaspectratio]{mermaid-a935632e.pdf}


{\def\LTcaptype{none} % do not increment counter
\vspace{-5pt}
\captionof{table}{SCRUM Roles and Artifacts}
\vspace{-10pt}
\begin{longtable}[]{@{}ll@{}}
\toprule\noalign{}
Component & Description \\
\midrule\noalign{}
\endhead
\bottomrule\noalign{}
\endlastfoot
\textbf{Product Owner} & Defines requirements and priorities \\
\textbf{Scrum Master} & Facilitates process and removes obstacles \\
\textbf{Development Team} & Self-organizing team that builds product \\
\textbf{Product Backlog} & Prioritized list of features \\
\textbf{Sprint Backlog} & Tasks selected for current sprint \\
\end{longtable}
}

\textbf{Key Events}:

\begin{itemize}
\tightlist
\item
  \textbf{Sprint Planning}: Select work for upcoming sprint
\item
  \textbf{Daily Scrum}: 15-minute daily synchronization
\item
  \textbf{Sprint Review}: Demonstrate completed work
\item
  \textbf{Sprint Retrospective}: Reflect and improve process
\end{itemize}

\textbf{Benefits}: Fast delivery, flexibility, continuous improvement,
customer collaboration

\end{solutionbox}
\begin{mnemonicbox}
``People Sprint Daily Reviewing Retrospectively''

\end{mnemonicbox}
\begin{center}\rule{0.5\linewidth}{0.5pt}\end{center}

\subsection*{Question 2(a) [3 marks]}\label{q2a}

\textbf{Explain characteristic of good SRS.}

\begin{solutionbox}

\textbf{SRS (Software Requirements Specification)} document should have
specific qualities to be effective.


{\def\LTcaptype{none} % do not increment counter
\vspace{-5pt}
\captionof{table}{Good SRS Characteristics}
\vspace{-10pt}
\begin{longtable}[]{@{}ll@{}}
\toprule\noalign{}
Characteristic & Meaning \\
\midrule\noalign{}
\endhead
\bottomrule\noalign{}
\endlastfoot
\textbf{Complete} & All requirements included \\
\textbf{Consistent} & No contradictory requirements \\
\textbf{Unambiguous} & Clear and single interpretation \\
\textbf{Verifiable} & Can be tested and validated \\
\textbf{Modifiable} & Easy to change when needed \\
\end{longtable}
}

\begin{itemize}
\tightlist
\item
  \textbf{Complete}: Contains all functional and non-functional
  requirements
\item
  \textbf{Consistent}: No conflicts between different requirements
\item
  \textbf{Unambiguous}: Each requirement has only one interpretation
\end{itemize}

\end{solutionbox}
\begin{mnemonicbox}
``Complete Computers Use Verified Modifications''

\end{mnemonicbox}
\begin{center}\rule{0.5\linewidth}{0.5pt}\end{center}

\subsection*{Question 2(b) [4 marks]}\label{q2b}

\textbf{Describe advantage and disadvantages of prototype model.}

\begin{solutionbox}

\textbf{Prototype Model} creates a working model of software to
understand requirements better.


{\def\LTcaptype{none} % do not increment counter
\vspace{-5pt}
\captionof{table}{Prototype Model - Pros and Cons}
\vspace{-10pt}
\begin{longtable}[]{@{}ll@{}}
\toprule\noalign{}
Advantages & Disadvantages \\
\midrule\noalign{}
\endhead
\bottomrule\noalign{}
\endlastfoot
\textbf{Better requirement understanding} & \textbf{Time consuming} \\
\textbf{User involvement} & \textbf{Cost increase} \\
\textbf{Early error detection} & \textbf{Incomplete analysis} \\
\textbf{User satisfaction} & \textbf{Prototype confusion} \\
\end{longtable}
}

\textbf{Advantages}:

\begin{itemize}
\tightlist
\item
  \textbf{Clear requirements}: Users see working model
\item
  \textbf{Early feedback}: Reduces final product risks
\item
  \textbf{User involvement}: Better user acceptance
\end{itemize}

\textbf{Disadvantages}:

\begin{itemize}
\tightlist
\item
  \textbf{Extra time}: Building prototype takes time
\item
  \textbf{Additional cost}: Resources needed for prototype
\item
  \textbf{Scope creep}: Users may expect prototype features
\end{itemize}

\end{solutionbox}
\begin{mnemonicbox}
``Better Users Experience'' vs ``Time Costs
Increase''

\end{mnemonicbox}
\begin{center}\rule{0.5\linewidth}{0.5pt}\end{center}

\subsection*{Question 2(c) [7 marks]}\label{q2c}

\textbf{Design and describe Spiral model and give advantages and
disadvantages.}

\begin{solutionbox}

\textbf{Spiral Model} combines iterative development with systematic
risk management through repeated cycles.

\includegraphics[width=1\linewidth,height=\textheight,keepaspectratio]{mermaid-db8f8547.pdf}


{\def\LTcaptype{none} % do not increment counter
\vspace{-5pt}
\captionof{table}{Spiral Model Phases}
\vspace{-10pt}
\begin{longtable}[]{@{}ll@{}}
\toprule\noalign{}
Phase & Activities \\
\midrule\noalign{}
\endhead
\bottomrule\noalign{}
\endlastfoot
\textbf{Planning} & Requirements gathering, resource planning \\
\textbf{Risk Analysis} & Identify and resolve risks \\
\textbf{Engineering} & Development and testing \\
\textbf{Customer Evaluation} & Customer reviews and feedback \\
\end{longtable}
}

\textbf{Advantages}:

\begin{itemize}
\tightlist
\item
  \textbf{Risk management}: Early risk identification
\item
  \textbf{Flexibility}: Accommodates changes easily
\item
  \textbf{Customer involvement}: Regular customer feedback
\item
  \textbf{Quality focus}: Continuous testing and validation
\end{itemize}

\textbf{Disadvantages}:

\begin{itemize}
\tightlist
\item
  \textbf{Complex management}: Difficult to manage
\item
  \textbf{High cost}: Expensive due to risk analysis
\item
  \textbf{Time consuming}: Long development cycles
\item
  \textbf{Risk expertise needed}: Requires risk assessment skills
\end{itemize}

\textbf{Best for}: Large, complex, high-risk projects

\end{solutionbox}
\begin{mnemonicbox}
``Plan Risks Engineering Customer'' for phases

\end{mnemonicbox}
\begin{center}\rule{0.5\linewidth}{0.5pt}\end{center}

\subsection*{Question 2(a) OR [3
marks]}\label{q2a}

\textbf{Explain Incremental model.}

\begin{solutionbox}

\textbf{Incremental Model} delivers software in small, functional pieces
called increments.

\includegraphics[width=1\linewidth,height=\textheight,keepaspectratio]{mermaid-bedc97b7.pdf}

\textbf{Key Features}:

\begin{itemize}
\tightlist
\item
  \textbf{Partial implementation}: Each increment adds functionality
\item
  \textbf{Early delivery}: Core features delivered first
\item
  \textbf{Parallel development}: Multiple increments can be developed
  simultaneously
\end{itemize}


{\def\LTcaptype{none} % do not increment counter
\vspace{-5pt}
\captionof{table}{Incremental Model Characteristics}
\vspace{-10pt}
\begin{longtable}[]{@{}ll@{}}
\toprule\noalign{}
Aspect & Description \\
\midrule\noalign{}
\endhead
\bottomrule\noalign{}
\endlastfoot
\textbf{Delivery} & Multiple releases \\
\textbf{Functionality} & Grows with each increment \\
\textbf{Risk} & Reduced through early delivery \\
\textbf{Feedback} & Continuous user feedback \\
\end{longtable}
}

\end{solutionbox}
\begin{mnemonicbox}
``Deliver Functionality Reducing Feedback''

\end{mnemonicbox}
\begin{center}\rule{0.5\linewidth}{0.5pt}\end{center}

\subsection*{Question 2(b) OR [4
marks]}\label{q2b}

\textbf{Write concept of Rapid Application Development model and explain
it.}

\begin{solutionbox}

\textbf{RAD (Rapid Application Development)} emphasizes rapid
prototyping and quick feedback over extensive planning.


{\def\LTcaptype{none} % do not increment counter
\vspace{-5pt}
\captionof{table}{RAD Model Phases}
\vspace{-10pt}
\begin{longtable}[]{@{}lll@{}}
\toprule\noalign{}
Phase & Duration & Activities \\
\midrule\noalign{}
\endhead
\bottomrule\noalign{}
\endlastfoot
\textbf{Business Modeling} & Short & Define business functions \\
\textbf{Data Modeling} & Short & Define data requirements \\
\textbf{Process Modeling} & Short & Convert data to business info \\
\textbf{Application Generation} & Short & Use tools to create
software \\
\textbf{Testing \& Turnover} & Short & Test and deploy \\
\end{longtable}
}

\textbf{Key Concepts}:

\begin{itemize}
\tightlist
\item
  \textbf{Reusable components}: Pre-built components speed development
\item
  \textbf{Powerful tools}: CASE tools and code generators
\item
  \textbf{Small teams}: 2-6 people per team
\item
  \textbf{Time-boxed}: Strict time limits (60-90 days)
\end{itemize}

\textbf{Requirements for RAD}:

\begin{itemize}
\tightlist
\item
  \textbf{Well-defined business requirements}
\item
  \textbf{User involvement} throughout process
\item
  \textbf{Skilled developers} familiar with RAD tools
\end{itemize}

\end{solutionbox}
\begin{mnemonicbox}
``Business Data Process Application Testing''

\end{mnemonicbox}
\begin{center}\rule{0.5\linewidth}{0.5pt}\end{center}

\subsection*{Question 2(c) OR [7
marks]}\label{q2c}

\textbf{Define SDLC and explain each phase.}

\begin{solutionbox}

\textbf{SDLC (Software Development Life Cycle)} is a systematic process
for building software through well-defined phases.

\includegraphics[width=1\linewidth,height=\textheight,keepaspectratio]{mermaid-5d3c847e.pdf}


{\def\LTcaptype{none} % do not increment counter
\vspace{-5pt}
\captionof{table}{SDLC Phases Detailed}
\vspace{-10pt}
\begin{longtable}[]{@{}lll@{}}
\toprule\noalign{}
Phase & Activities & Deliverables \\
\midrule\noalign{}
\endhead
\bottomrule\noalign{}
\endlastfoot
\textbf{Planning} & Project planning, feasibility study & Project
plan \\
\textbf{Analysis} & Requirement gathering & SRS document \\
\textbf{Design} & System architecture, UI design & Design document \\
\textbf{Implementation} & Coding, unit testing & Source code \\
\textbf{Testing} & System testing, integration & Test reports \\
\textbf{Deployment} & Installation, user training & Live system \\
\textbf{Maintenance} & Bug fixes, enhancements & Updated system \\
\end{longtable}
}

\textbf{Phase Descriptions}:

\begin{itemize}
\tightlist
\item
  \textbf{Planning}: Define project scope and resources
\item
  \textbf{Analysis}: Understand what system should do
\item
  \textbf{Design}: Plan how system will work
\item
  \textbf{Implementation}: Build the actual system
\item
  \textbf{Testing}: Verify system works correctly
\item
  \textbf{Deployment}: Release system to users
\item
  \textbf{Maintenance}: Ongoing support and updates
\end{itemize}

\end{solutionbox}
\begin{mnemonicbox}
``People Always Design Implementation, Test
Deployment, Maintain''

\end{mnemonicbox}
\begin{center}\rule{0.5\linewidth}{0.5pt}\end{center}

\subsection*{Question 3(a) [3 marks]}\label{q3a}

\textbf{Describe skills to manage software projects.}

\begin{solutionbox}

\textbf{Software Project Management} requires combination of technical
and soft skills.


{\def\LTcaptype{none} % do not increment counter
\vspace{-5pt}
\captionof{table}{Essential Project Management Skills}
\vspace{-10pt}
\begin{longtable}[]{@{}ll@{}}
\toprule\noalign{}
Skill Category & Specific Skills \\
\midrule\noalign{}
\endhead
\bottomrule\noalign{}
\endlastfoot
\textbf{Technical} & Understanding SDLC, tools, technologies \\
\textbf{Leadership} & Team motivation, decision making \\
\textbf{Communication} & Clear communication with team and clients \\
\textbf{Planning} & Resource allocation, scheduling \\
\textbf{Problem-solving} & Risk management, conflict resolution \\
\end{longtable}
}

\textbf{Key Skills}:

\begin{itemize}
\tightlist
\item
  \textbf{People management}: Lead and motivate team members
\item
  \textbf{Technical knowledge}: Understand development process and tools
\item
  \textbf{Communication}: Bridge between technical team and stakeholders
\end{itemize}

\end{solutionbox}
\begin{mnemonicbox}
``Technical Leaders Communicate Planning Problems''

\end{mnemonicbox}
\begin{center}\rule{0.5\linewidth}{0.5pt}\end{center}

\subsection*{Question 3(b) [4 marks]}\label{q3b}

\textbf{Briefly write responsibility of Software Project manager.}

\begin{solutionbox}

\textbf{Software Project Manager} oversees entire project from
initiation to completion.


{\def\LTcaptype{none} % do not increment counter
\vspace{-5pt}
\captionof{table}{Project Manager Responsibilities}
\vspace{-10pt}
\begin{longtable}[]{@{}ll@{}}
\toprule\noalign{}
Area & Responsibilities \\
\midrule\noalign{}
\endhead
\bottomrule\noalign{}
\endlastfoot
\textbf{Planning} & Create project plans, schedules, budgets \\
\textbf{Team Management} & Hire, train, and manage team members \\
\textbf{Communication} & Regular updates to stakeholders \\
\textbf{Quality Control} & Ensure deliverables meet quality standards \\
\textbf{Risk Management} & Identify and mitigate project risks \\
\end{longtable}
}

\textbf{Primary Responsibilities}:

\begin{itemize}
\tightlist
\item
  \textbf{Project Planning}: Define scope, timeline, and resources
\item
  \textbf{Team Leadership}: Guide and support development team
\item
  \textbf{Stakeholder Communication}: Keep everyone informed of progress
\item
  \textbf{Quality Assurance}: Ensure project meets requirements
\item
  \textbf{Risk Management}: Handle project risks and issues
\end{itemize}

\textbf{Success Factors}: On-time delivery, within budget, meeting
requirements

\end{solutionbox}
\begin{mnemonicbox}
``Plan Team Communication Quality Risk''

\end{mnemonicbox}
\begin{center}\rule{0.5\linewidth}{0.5pt}\end{center}

\subsection*{Question 3(c) [7 marks]}\label{q3c}

\textbf{Classify types of Requirements in SRS (1) Functional
Requirements (2) Non-Functional Requirements.}

\begin{solutionbox}

\textbf{Requirements Classification} helps organize and understand
different types of system needs.


{\def\LTcaptype{none} % do not increment counter
\vspace{-5pt}
\captionof{table}{Functional vs Non-Functional Requirements}
\vspace{-10pt}
\begin{longtable}[]{@{}
  >{\raggedright\arraybackslash}p{(\linewidth - 4\tabcolsep) * \real{0.3333}}
  >{\raggedright\arraybackslash}p{(\linewidth - 4\tabcolsep) * \real{0.3333}}
  >{\raggedright\arraybackslash}p{(\linewidth - 4\tabcolsep) * \real{0.3333}}@{}}
\toprule\noalign{}
\begin{minipage}[b]{\linewidth}\raggedright
Aspect
\end{minipage} & \begin{minipage}[b]{\linewidth}\raggedright
Functional Requirements
\end{minipage} & \begin{minipage}[b]{\linewidth}\raggedright
Non-Functional Requirements
\end{minipage} \\
\midrule\noalign{}
\endhead
\bottomrule\noalign{}
\endlastfoot
\textbf{Definition} & What system should do & How system should
perform \\
\textbf{Focus} & System functionality & System quality attributes \\
\textbf{Examples} & Login, search, calculate & Performance, security,
usability \\
\textbf{Testing} & Functional testing & Performance testing \\
\end{longtable}
}

\textbf{Functional Requirements}:

\begin{itemize}
\tightlist
\item
  \textbf{User interactions}: Login, registration, data entry
\item
  \textbf{Business rules}: Validation rules, calculations
\item
  \textbf{System features}: Reports, notifications, workflows
\item
  \textbf{Data processing}: CRUD operations
\end{itemize}

\textbf{Examples}:

\begin{itemize}
\tightlist
\item
  User can login with username/password
\item
  System calculates tax automatically
\item
  Generate monthly sales report
\end{itemize}

\textbf{Non-Functional Requirements}:


{\def\LTcaptype{none} % do not increment counter
\vspace{-5pt}
\captionof{table}{Non-Functional Requirement Types}
\vspace{-10pt}
\begin{longtable}[]{@{}
  >{\raggedright\arraybackslash}p{(\linewidth - 4\tabcolsep) * \real{0.3333}}
  >{\raggedright\arraybackslash}p{(\linewidth - 4\tabcolsep) * \real{0.3333}}
  >{\raggedright\arraybackslash}p{(\linewidth - 4\tabcolsep) * \real{0.3333}}@{}}
\toprule\noalign{}
\begin{minipage}[b]{\linewidth}\raggedright
Type
\end{minipage} & \begin{minipage}[b]{\linewidth}\raggedright
Description
\end{minipage} & \begin{minipage}[b]{\linewidth}\raggedright
Example
\end{minipage} \\
\midrule\noalign{}
\endhead
\bottomrule\noalign{}
\endlastfoot
\textbf{Performance} & Speed and responsiveness & Response time
\textless{} 2 seconds \\
\textbf{Security} & Data protection & Encrypted data transmission \\
\textbf{Usability} & User experience & Easy to learn interface \\
\textbf{Reliability} & System dependability & 99.9\% uptime \\
\textbf{Scalability} & Growth handling & Support 1000+ users \\
\end{longtable}
}

\textbf{Quality Attributes}:

\begin{itemize}
\tightlist
\item
  \textbf{Performance}: Response time, throughput
\item
  \textbf{Security}: Authentication, authorization, encryption
\item
  \textbf{Usability}: User-friendly interface, accessibility
\item
  \textbf{Reliability}: Uptime, error handling
\item
  \textbf{Maintainability}: Code quality, documentation
\end{itemize}

\end{solutionbox}
\begin{mnemonicbox}
``Performance Security Usability Reliability
Maintainability''

\end{mnemonicbox}
\begin{center}\rule{0.5\linewidth}{0.5pt}\end{center}

\subsection*{Question 3(a) OR [3
marks]}\label{q3a}

\textbf{Illustrate importance of SRS.}

\begin{solutionbox}

\textbf{SRS (Software Requirements Specification)} is crucial document
that defines what software should do.


{\def\LTcaptype{none} % do not increment counter
\vspace{-5pt}
\captionof{table}{SRS Importance}
\vspace{-10pt}
\begin{longtable}[]{@{}ll@{}}
\toprule\noalign{}
Aspect & Benefit \\
\midrule\noalign{}
\endhead
\bottomrule\noalign{}
\endlastfoot
\textbf{Clear Communication} & All stakeholders understand
requirements \\
\textbf{Project Planning} & Basis for estimation and scheduling \\
\textbf{Quality Assurance} & Foundation for testing \\
\textbf{Change Management} & Controlled requirement changes \\
\textbf{Legal Protection} & Contract reference document \\
\end{longtable}
}

\textbf{Key Importance}:

\begin{itemize}
\tightlist
\item
  \textbf{Communication tool}: Bridge between clients and developers
\item
  \textbf{Planning foundation}: Helps estimate time, cost, and resources
\item
  \textbf{Testing basis}: Test cases derived from SRS requirements
\end{itemize}

\end{solutionbox}
\begin{mnemonicbox}
``Clear Planning Quality Change Legal''

\end{mnemonicbox}
\begin{center}\rule{0.5\linewidth}{0.5pt}\end{center}

\subsection*{Question 3(b) OR [4
marks]}\label{q3b}

\textbf{Explain Gantt Chart.}

\begin{solutionbox}

\textbf{Gantt Chart} is a visual project management tool showing tasks,
timelines, and dependencies.

\includegraphics[width=1\linewidth,height=\textheight,keepaspectratio]{mermaid-dcac1ca8.pdf}


{\def\LTcaptype{none} % do not increment counter
\vspace{-5pt}
\captionof{table}{Gantt Chart Components}
\vspace{-10pt}
\begin{longtable}[]{@{}ll@{}}
\toprule\noalign{}
Component & Description \\
\midrule\noalign{}
\endhead
\bottomrule\noalign{}
\endlastfoot
\textbf{Tasks} & Work items to be completed \\
\textbf{Timeline} & Horizontal time scale \\
\textbf{Bars} & Task duration and progress \\
\textbf{Dependencies} & Task relationships \\
\textbf{Milestones} & Important project events \\
\end{longtable}
}

\textbf{Benefits}:

\begin{itemize}
\tightlist
\item
  \textbf{Visual timeline}: Easy to see project schedule
\item
  \textbf{Progress tracking}: Monitor task completion
\item
  \textbf{Resource planning}: Allocate resources effectively
\item
  \textbf{Dependency management}: Understand task relationships
\end{itemize}

\end{solutionbox}
\begin{mnemonicbox}
``Tasks Timeline Bars Dependencies Milestones''

\end{mnemonicbox}
\begin{center}\rule{0.5\linewidth}{0.5pt}\end{center}

\subsection*{Question 3(c) OR [7
marks]}\label{q3c}

\textbf{Write a short note on Risk Management.}

\begin{solutionbox}

\textbf{Risk Management} is systematic process of identifying,
analyzing, and controlling project risks.

\includegraphics[width=1\linewidth,height=\textheight,keepaspectratio]{mermaid-73c5b521.pdf}


{\def\LTcaptype{none} % do not increment counter
\vspace{-5pt}
\captionof{table}{Risk Management Process}
\vspace{-10pt}
\begin{longtable}[]{@{}lll@{}}
\toprule\noalign{}
Phase & Activities & Output \\
\midrule\noalign{}
\endhead
\bottomrule\noalign{}
\endlastfoot
\textbf{Identification} & Find potential risks & Risk list \\
\textbf{Analysis} & Assess probability and impact & Risk priority \\
\textbf{Planning} & Develop response strategies & Risk response plan \\
\textbf{Monitoring} & Track and control risks & Updated risk status \\
\end{longtable}
}

\textbf{Risk Categories}:


{\def\LTcaptype{none} % do not increment counter
\vspace{-5pt}
\captionof{table}{Types of Software Risks}
\vspace{-10pt}
\begin{longtable}[]{@{}ll@{}}
\toprule\noalign{}
Category & Examples \\
\midrule\noalign{}
\endhead
\bottomrule\noalign{}
\endlastfoot
\textbf{Technical} & Technology changes, complexity \\
\textbf{Project} & Schedule delays, resource shortage \\
\textbf{Business} & Market changes, funding issues \\
\textbf{External} & Vendor problems, regulatory changes \\
\end{longtable}
}

\textbf{Risk Response Strategies}:

\begin{itemize}
\tightlist
\item
  \textbf{Avoid}: Eliminate risk source
\item
  \textbf{Mitigate}: Reduce probability or impact\\
\item
  \textbf{Transfer}: Share risk with others
\item
  \textbf{Accept}: Live with the risk
\end{itemize}

\textbf{Risk Assessment}: Probability \times Impact = Risk Exposure

\textbf{Benefits}: Proactive problem solving, better project success
rate, stakeholder confidence

\end{solutionbox}
\begin{mnemonicbox}
``Identify Analyze Plan Monitor'' for process,
``Avoid Mitigate Transfer Accept'' for strategies

\end{mnemonicbox}
\begin{center}\rule{0.5\linewidth}{0.5pt}\end{center}

\subsection*{Question 4(a) [3 marks]}\label{q4a}

\textbf{What is metric for size estimation? Explain FP with example.}

\begin{solutionbox}

\textbf{Size Estimation Metrics} help predict software project size and
effort.


{\def\LTcaptype{none} % do not increment counter
\vspace{-5pt}
\captionof{table}{Size Estimation Metrics}
\vspace{-10pt}
\begin{longtable}[]{@{}ll@{}}
\toprule\noalign{}
Metric & Description \\
\midrule\noalign{}
\endhead
\bottomrule\noalign{}
\endlastfoot
\textbf{LOC} & Lines of Code \\
\textbf{Function Points} & Functionality-based measurement \\
\textbf{Object Points} & For object-oriented systems \\
\textbf{Feature Points} & Enhanced function points \\
\end{longtable}
}

\textbf{Function Points (FP)} measure software size based on user
functionality.

\textbf{FP Components}:

\begin{itemize}
\tightlist
\item
  \textbf{External Inputs}: Data entry screens
\item
  \textbf{External Outputs}: Reports, messages\\
\item
  \textbf{External Queries}: Database queries
\item
  \textbf{Internal Files}: Data stores
\item
  \textbf{External Interfaces}: System connections
\end{itemize}

\textbf{FP Calculation Example}: For a Library Management System:

\begin{itemize}
\tightlist
\item
  External Inputs: 5 (Book entry, Member entry, etc.)
\item
  External Outputs: 3 (Reports)
\item
  External Queries: 4 (Search functions)
\item
  Internal Files: 2 (Book DB, Member DB)
\item
  External Interfaces: 1 (Online catalog)
\end{itemize}

\textbf{Simple FP = 5 + 3 + 4 + 2 + 1 = 15 Function Points}

\end{solutionbox}
\begin{mnemonicbox}
``Inputs Outputs Queries Files Interfaces''

\end{mnemonicbox}
\begin{center}\rule{0.5\linewidth}{0.5pt}\end{center}

\subsection*{Question 4(b) [4 marks]}\label{q4b}

\textbf{Explain project estimation techniques using basic COCOMO model.}

\begin{solutionbox}

\textbf{COCOMO (COnstructive COst MOdel)} estimates software development
effort and schedule.


{\def\LTcaptype{none} % do not increment counter
\vspace{-5pt}
\captionof{table}{COCOMO Model Types}
\vspace{-10pt}
\begin{longtable}[]{@{}lll@{}}
\toprule\noalign{}
Type & Description & Accuracy \\
\midrule\noalign{}
\endhead
\bottomrule\noalign{}
\endlastfoot
\textbf{Basic} & Simple size-based estimation & \pm75\% \\
\textbf{Intermediate} & Includes cost drivers & \pm25\% \\
\textbf{Detailed} & Phase-level estimation & \pm10\% \\
\end{longtable}
}

\textbf{Basic COCOMO Formula}:

\begin{itemize}
\tightlist
\item
  \textbf{Effort} = a \times (KLOC)\^{}b person-months
\item
  \textbf{Time} = c \times (Effort)\^{}d months
\item
  \textbf{People} = Effort / Time
\end{itemize}


{\def\LTcaptype{none} % do not increment counter
\vspace{-5pt}
\captionof{table}{COCOMO Constants}
\vspace{-10pt}
\begin{longtable}[]{@{}lllll@{}}
\toprule\noalign{}
Project Type & a & b & c & d \\
\midrule\noalign{}
\endhead
\bottomrule\noalign{}
\endlastfoot
\textbf{Organic} & 2.4 & 1.05 & 2.5 & 0.38 \\
\textbf{Semi-detached} & 3.0 & 1.12 & 2.5 & 0.35 \\
\textbf{Embedded} & 3.6 & 1.20 & 2.5 & 0.32 \\
\end{longtable}
}

\textbf{Example}: For 10 KLOC organic project

\begin{itemize}
\tightlist
\item
  Effort = 2.4 \times (10)\^{}1.05 = 25.47 person-months
\item
  Time = 2.5 \times (25.47)\^{}0.38 = 8.64 months
\item
  People = 25.47 / 8.64 = 3 people
\end{itemize}

\end{solutionbox}
\begin{mnemonicbox}
``Organic Semi Embedded'' for project types

\end{mnemonicbox}
\begin{center}\rule{0.5\linewidth}{0.5pt}\end{center}

\subsection*{Question 4(c) [7 marks]}\label{q4c}

\textbf{Prepare Sprint burn down chart for system of your choice.}

\begin{solutionbox}

\textbf{Sprint Burn Down Chart} tracks remaining work during a sprint
for \textbf{Online Shopping System}.

\includegraphics[width=1\linewidth,height=\textheight,keepaspectratio]{mermaid-ecd4426e.pdf}

\textbf{Sprint Backlog}:


{\def\LTcaptype{none} % do not increment counter
\vspace{-5pt}
\captionof{table}{Sprint Tasks}
\vspace{-10pt}
\begin{longtable}[]{@{}lll@{}}
\toprule\noalign{}
Task & Story Points & Day Assigned \\
\midrule\noalign{}
\endhead
\bottomrule\noalign{}
\endlastfoot
\textbf{User Registration} & 8 & Day 1-2 \\
\textbf{User Login} & 6 & Day 3-4 \\
\textbf{Password Reset} & 5 & Day 5-6 \\
\textbf{Profile Management} & 8 & Day 7-8 \\
\textbf{Session Management} & 6 & Day 9-10 \\
\textbf{Testing \& Bug Fixes} & 7 & Day 11-14 \\
\end{longtable}
}

\textbf{Burn Down Chart Data}:


{\def\LTcaptype{none} % do not increment counter
\vspace{-5pt}
\captionof{table}{Daily Progress}
\vspace{-10pt}
\begin{longtable}[]{@{}llll@{}}
\toprule\noalign{}
Day & Ideal Remaining & Actual Remaining & Work Completed \\
\midrule\noalign{}
\endhead
\bottomrule\noalign{}
\endlastfoot
\textbf{Day 0} & 40 & 40 & Sprint Start \\
\textbf{Day 2} & 36 & 38 & Registration delay \\
\textbf{Day 4} & 32 & 32 & Login completed \\
\textbf{Day 6} & 28 & 27 & Password reset done early \\
\textbf{Day 8} & 24 & 26 & Profile management issues \\
\textbf{Day 10} & 20 & 20 & Back on track \\
\textbf{Day 12} & 16 & 15 & Testing progressing well \\
\textbf{Day 14} & 0 & 0 & Sprint completed \\
\end{longtable}
}

\textbf{Chart Analysis}:

\begin{itemize}
\tightlist
\item
  \textbf{Green line}: Ideal burn down
\item
  \textbf{Red line}: Actual progress\\
\item
  \textbf{Variations}: Show challenges and recoveries
\item
  \textbf{Completion}: Sprint finished on time
\end{itemize}

\textbf{Benefits}: Visual progress tracking, early problem
identification, team motivation

\end{solutionbox}
\begin{mnemonicbox}
``Track Progress Daily, Identify Issues Early''

\end{mnemonicbox}
\begin{center}\rule{0.5\linewidth}{0.5pt}\end{center}

\subsection*{Question 4(a) OR [3
marks]}\label{q4a}

\textbf{Explain the component of USE CASE diagram.}

\begin{solutionbox}

\textbf{Use Case Diagram} shows system functionality from user
perspective.


{\def\LTcaptype{none} % do not increment counter
\vspace{-5pt}
\captionof{table}{Use Case Diagram Components}
\vspace{-10pt}
\begin{longtable}[]{@{}lll@{}}
\toprule\noalign{}
Component & Symbol & Description \\
\midrule\noalign{}
\endhead
\bottomrule\noalign{}
\endlastfoot
\textbf{Actor} & Stick figure & External entity interacting with
system \\
\textbf{Use Case} & Oval & System functionality \\
\textbf{System Boundary} & Rectangle & System scope \\
\textbf{Association} & Line & Actor-Use Case relationship \\
\textbf{Generalization} & Arrow & Inheritance relationship \\
\end{longtable}
}

\textbf{Relationships}:

\begin{itemize}
\tightlist
\item
  \textbf{Include}: One use case includes another (mandatory)
\item
  \textbf{Extend}: Optional use case extension
\item
  \textbf{Generalization}: Parent-child relationship
\end{itemize}

\textbf{Example Components}:

\begin{itemize}
\tightlist
\item
  \textbf{Primary Actor}: Customer, Admin
\item
  \textbf{Use Cases}: Login, Search Products, Place Order
\item
  \textbf{System}: Online Shopping System
\end{itemize}

\end{solutionbox}
\begin{mnemonicbox}
``Actors Use Systems, Associate Generally''

\end{mnemonicbox}
\begin{center}\rule{0.5\linewidth}{0.5pt}\end{center}

\subsection*{Question 4(b) OR [4
marks]}\label{q4b}

\textbf{Compare Cohesion and Coupling.}

\begin{solutionbox}

\textbf{Cohesion and Coupling} are important software design principles
affecting maintainability.


{\def\LTcaptype{none} % do not increment counter
\vspace{-5pt}
\captionof{table}{Cohesion vs Coupling Comparison}
\vspace{-10pt}
\begin{longtable}[]{@{}
  >{\raggedright\arraybackslash}p{(\linewidth - 4\tabcolsep) * \real{0.3333}}
  >{\raggedright\arraybackslash}p{(\linewidth - 4\tabcolsep) * \real{0.3333}}
  >{\raggedright\arraybackslash}p{(\linewidth - 4\tabcolsep) * \real{0.3333}}@{}}
\toprule\noalign{}
\begin{minipage}[b]{\linewidth}\raggedright
Aspect
\end{minipage} & \begin{minipage}[b]{\linewidth}\raggedright
Cohesion
\end{minipage} & \begin{minipage}[b]{\linewidth}\raggedright
Coupling
\end{minipage} \\
\midrule\noalign{}
\endhead
\bottomrule\noalign{}
\endlastfoot
\textbf{Definition} & Unity within module & Dependency between
modules \\
\textbf{Desirable Level} & High cohesion preferred & Low coupling
preferred \\
\textbf{Focus} & Internal module unity & Inter-module relationships \\
\textbf{Impact} & Module reliability & System flexibility \\
\textbf{Measurement} & How related are module elements & How dependent
modules are \\
\end{longtable}
}

\textbf{Cohesion Types} (Low to High):

\begin{itemize}
\tightlist
\item
  \textbf{Coincidental}: Random grouping
\item
  \textbf{Logical}: Similar logic
\item
  \textbf{Temporal}: Same time execution
\item
  \textbf{Procedural}: Sequential steps
\item
  \textbf{Communicational}: Same data
\item
  \textbf{Sequential}: Output of one is input of next
\item
  \textbf{Functional}: Single purpose
\end{itemize}

\textbf{Coupling Types} (High to Low):

\begin{itemize}
\tightlist
\item
  \textbf{Content}: Direct access to module internals
\item
  \textbf{Common}: Shared global data
\item
  \textbf{External}: Shared external interface
\item
  \textbf{Control}: Control information passed
\item
  \textbf{Stamp}: Data structure passed
\item
  \textbf{Data}: Simple data passed
\end{itemize}

\textbf{Goal}: \textbf{High Cohesion + Low Coupling = Good Design}

\end{solutionbox}
\begin{mnemonicbox}
``High Cohesion, Low Coupling'' for good design

\end{mnemonicbox}
\begin{center}\rule{0.5\linewidth}{0.5pt}\end{center}

\subsection*{Question 4(c) OR [7
marks]}\label{q4c}

\textbf{Explain Risk Assessment in detail.}

\begin{solutionbox}

\textbf{Risk Assessment} evaluates identified risks to prioritize
management efforts.

\includegraphics[width=1\linewidth,height=\textheight,keepaspectratio]{mermaid-7af19596.pdf}

\textbf{Risk Assessment Components}:


{\def\LTcaptype{none} % do not increment counter
\vspace{-5pt}
\captionof{table}{Risk Assessment Elements}
\vspace{-10pt}
\begin{longtable}[]{@{}lll@{}}
\toprule\noalign{}
Element & Description & Scale \\
\midrule\noalign{}
\endhead
\bottomrule\noalign{}
\endlastfoot
\textbf{Probability} & Likelihood of risk occurring & 0.1 to 1.0 \\
\textbf{Impact} & Consequences if risk occurs & 1 to 10 \\
\textbf{Risk Exposure} & Probability \times Impact & Calculated value \\
\textbf{Risk Level} & Priority classification & High/Medium/Low \\
\end{longtable}
}

\textbf{Assessment Process}:

\textbf{1. Probability Assessment}:

\begin{itemize}
\tightlist
\item
  \textbf{Very Low (0.1)}: Unlikely to happen
\item
  \textbf{Low (0.3)}: Possible but not probable\\
\item
  \textbf{Medium (0.5)}: May or may not happen
\item
  \textbf{High (0.7)}: Likely to happen
\item
  \textbf{Very High (0.9)}: Almost certain
\end{itemize}

\textbf{2. Impact Assessment}:

\begin{itemize}
\tightlist
\item
  \textbf{Catastrophic (9-10)}: Project failure
\item
  \textbf{Critical (7-8)}: Major delays/cost overrun
\item
  \textbf{Marginal (4-6)}: Some impact on schedule/budget
\item
  \textbf{Negligible (1-3)}: Little impact
\end{itemize}

\textbf{3. Risk Exposure Calculation}: \textbf{Risk Exposure =
Probability \times Impact}

\textbf{Example Risk Assessment}:


{\def\LTcaptype{none} % do not increment counter
\vspace{-5pt}
\captionof{table}{Sample Risk Analysis}
\vspace{-10pt}
\begin{longtable}[]{@{}lllll@{}}
\toprule\noalign{}
Risk & Probability & Impact & Exposure & Priority \\
\midrule\noalign{}
\endhead
\bottomrule\noalign{}
\endlastfoot
\textbf{Key developer leaves} & 0.3 & 8 & 2.4 & Medium \\
\textbf{Requirements change} & 0.7 & 6 & 4.2 & High \\
\textbf{Technology failure} & 0.2 & 9 & 1.8 & Low \\
\textbf{Budget cuts} & 0.4 & 7 & 2.8 & Medium \\
\end{longtable}
}

\textbf{Risk Matrix}:

\begin{itemize}
\tightlist
\item
  \textbf{High Priority}: Exposure \textgreater{} 4.0
\item
  \textbf{Medium Priority}: Exposure 2.0-4.0\\
\item
  \textbf{Low Priority}: Exposure \textless{} 2.0
\end{itemize}

\textbf{Assessment Benefits}:

\begin{itemize}
\tightlist
\item
  \textbf{Objective prioritization}: Data-driven decisions
\item
  \textbf{Resource allocation}: Focus on high-risk items
\item
  \textbf{Communication tool}: Clear risk communication
\item
  \textbf{Planning input}: Influences project planning
\end{itemize}

\end{solutionbox}
\begin{mnemonicbox}
``Probability Impact Exposure Priority''

\end{mnemonicbox}
\begin{center}\rule{0.5\linewidth}{0.5pt}\end{center}

\subsection*{Question 5(a) [3 marks]}\label{q5a}

\textbf{Explain code inspection technique in code review.}

\begin{solutionbox}

\textbf{Code Inspection} is formal, systematic examination of code to
find defects.


{\def\LTcaptype{none} % do not increment counter
\vspace{-5pt}
\captionof{table}{Code Inspection Process}
\vspace{-10pt}
\begin{longtable}[]{@{}lll@{}}
\toprule\noalign{}
Phase & Participants & Activities \\
\midrule\noalign{}
\endhead
\bottomrule\noalign{}
\endlastfoot
\textbf{Planning} & Moderator & Schedule inspection, distribute code \\
\textbf{Overview} & Author, Team & Author explains code \\
\textbf{Preparation} & Individual & Each reviewer studies code \\
\textbf{Inspection} & All reviewers & Find defects systematically \\
\textbf{Rework} & Author & Fix identified defects \\
\textbf{Follow-up} & Moderator & Verify fixes \\
\end{longtable}
}

\textbf{Key Features}:

\begin{itemize}
\tightlist
\item
  \textbf{Formal process}: Structured approach with defined roles
\item
  \textbf{Systematic review}: Line-by-line examination
\item
  \textbf{Defect focused}: Find errors, not solutions
\item
  \textbf{No author criticism}: Focus on code, not coder
\end{itemize}

\textbf{Benefits}: Early defect detection, knowledge sharing, improved
code quality

\end{solutionbox}
\begin{mnemonicbox}
``Plan Overview Prepare Inspect Rework Follow-up''

\end{mnemonicbox}
\begin{center}\rule{0.5\linewidth}{0.5pt}\end{center}

\subsection*{Question 5(b) [4 marks]}\label{q5b}

\textbf{Prepare at least four test cases of ATM.}

\begin{solutionbox}

\textbf{ATM Test Cases} verify automated teller machine functionality.


{\def\LTcaptype{none} % do not increment counter
\vspace{-5pt}
\captionof{table}{ATM Test Cases}
\vspace{-10pt}
\begin{longtable}[]{@{}
  >{\raggedright\arraybackslash}p{(\linewidth - 8\tabcolsep) * \real{0.2000}}
  >{\raggedright\arraybackslash}p{(\linewidth - 8\tabcolsep) * \real{0.2000}}
  >{\raggedright\arraybackslash}p{(\linewidth - 8\tabcolsep) * \real{0.2000}}
  >{\raggedright\arraybackslash}p{(\linewidth - 8\tabcolsep) * \real{0.2000}}
  >{\raggedright\arraybackslash}p{(\linewidth - 8\tabcolsep) * \real{0.2000}}@{}}
\toprule\noalign{}
\begin{minipage}[b]{\linewidth}\raggedright
Test Case ID
\end{minipage} & \begin{minipage}[b]{\linewidth}\raggedright
Test Scenario
\end{minipage} & \begin{minipage}[b]{\linewidth}\raggedright
Input
\end{minipage} & \begin{minipage}[b]{\linewidth}\raggedright
Expected Output
\end{minipage} & \begin{minipage}[b]{\linewidth}\raggedright
Result
\end{minipage} \\
\midrule\noalign{}
\endhead
\bottomrule\noalign{}
\endlastfoot
\textbf{TC001} & Valid PIN Entry & Correct 4-digit PIN & Access granted,
main menu displayed & Pass/Fail \\
\textbf{TC002} & Invalid PIN Entry & Wrong PIN (3 attempts) & Card
blocked, error message & Pass/Fail \\
\textbf{TC003} & Cash Withdrawal & Amount \leq Account balance & Cash
dispensed, receipt printed & Pass/Fail \\
\textbf{TC004} & Insufficient Balance & Amount \textgreater{} Account
balance & Transaction declined, balance shown & Pass/Fail \\
\end{longtable}
}

\textbf{Detailed Test Cases}:

\textbf{Test Case 1: Valid Login}

\begin{itemize}
\tightlist
\item
  \textbf{Precondition}: ATM is operational, card inserted
\item
  \textbf{Steps}: Enter correct PIN \rightarrow Press Enter
\item
  \textbf{Expected}: Main menu with options displayed
\end{itemize}

\textbf{Test Case 2: Cash Withdrawal}

\begin{itemize}
\tightlist
\item
  \textbf{Precondition}: User logged in, sufficient balance
\item
  \textbf{Steps}: Select Withdrawal \rightarrow Enter amount \rightarrow Confirm
\item
  \textbf{Expected}: Cash dispensed, balance updated
\end{itemize}

\textbf{Test Case 3: Balance Inquiry}

\begin{itemize}
\tightlist
\item
  \textbf{Precondition}: User logged in
\item
  \textbf{Steps}: Select Balance Inquiry
\item
  \textbf{Expected}: Current balance displayed on screen
\end{itemize}

\textbf{Test Case 4: PIN Change}

\begin{itemize}
\tightlist
\item
  \textbf{Precondition}: User logged in
\item
  \textbf{Steps}: Select Change PIN \rightarrow Enter old PIN \rightarrow Enter new PIN \rightarrow
  Confirm
\item
  \textbf{Expected}: PIN changed successfully, confirmation message
\end{itemize}

\end{solutionbox}
\begin{mnemonicbox}
``Login Withdraw Inquiry Change''

\end{mnemonicbox}
\begin{center}\rule{0.5\linewidth}{0.5pt}\end{center}

\subsection*{Question 5(c) [7 marks]}\label{q5c}

\textbf{Describe white box testing.}

\begin{solutionbox}

\textbf{White Box Testing} examines internal code structure and logic
paths.

\includegraphics[width=1\linewidth,height=\textheight,keepaspectratio]{mermaid-4b6a68dc.pdf}


{\def\LTcaptype{none} % do not increment counter
\vspace{-5pt}
\captionof{table}{White Box Testing Characteristics}
\vspace{-10pt}
\begin{longtable}[]{@{}ll@{}}
\toprule\noalign{}
Aspect & Description \\
\midrule\noalign{}
\endhead
\bottomrule\noalign{}
\endlastfoot
\textbf{Focus} & Internal code structure \\
\textbf{Knowledge} & Code implementation details \\
\textbf{Coverage} & Statements, branches, paths \\
\textbf{Techniques} & Basis path, loop testing \\
\textbf{Tools} & Code coverage analyzers \\
\end{longtable}
}

\textbf{Coverage Criteria}:


{\def\LTcaptype{none} % do not increment counter
\vspace{-5pt}
\captionof{table}{Coverage Types}
\vspace{-10pt}
\begin{longtable}[]{@{}
  >{\raggedright\arraybackslash}p{(\linewidth - 4\tabcolsep) * \real{0.3333}}
  >{\raggedright\arraybackslash}p{(\linewidth - 4\tabcolsep) * \real{0.3333}}
  >{\raggedright\arraybackslash}p{(\linewidth - 4\tabcolsep) * \real{0.3333}}@{}}
\toprule\noalign{}
\begin{minipage}[b]{\linewidth}\raggedright
Coverage Type
\end{minipage} & \begin{minipage}[b]{\linewidth}\raggedright
Description
\end{minipage} & \begin{minipage}[b]{\linewidth}\raggedright
Goal
\end{minipage} \\
\midrule\noalign{}
\endhead
\bottomrule\noalign{}
\endlastfoot
\textbf{Statement Coverage} & Execute every statement & 100\%
statements \\
\textbf{Branch Coverage} & Execute every branch & All if-else paths \\
\textbf{Path Coverage} & Execute every path & All possible paths \\
\textbf{Condition Coverage} & Test all conditions & True/false for each
condition \\
\end{longtable}
}

\textbf{White Box Testing Techniques}:

\textbf{1. Basis Path Testing}:

\begin{itemize}
\tightlist
\item
  Calculate \textbf{Cyclomatic Complexity}: V(G) = E - N + 2
\item
E = Edges,

N = Nodes in control flow graph

\item
  Generate independent paths equal to V(G)
\end{itemize}

\textbf{2. Loop Testing}:

\begin{itemize}
\tightlist
\item
  \textbf{Simple loops}: Test 0, 1, 2, typical, max iterations
\item
  \textbf{Nested loops}: Test inner loop first, then outer
\item
  \textbf{Concatenated loops}: Test as separate loops
\end{itemize}

\textbf{3. Condition Testing}:

\begin{itemize}
\tightlist
\item
  Test all logical conditions (AND, OR, NOT)
\item
  Ensure each condition evaluates to true and false
\end{itemize}

\textbf{Example: Simple Code Testing}

\begin{lstlisting}
if (age >= 18 AND income > 25000)
    approve_loan();
else
    reject_loan();
\end{lstlisting}

\textbf{Test Cases}:

\begin{itemize}
\tightlist
\item
  age=20, income=30000 (both true) \rightarrow approve
\item
  age=16, income=30000 (first false) \rightarrow reject\\
\item
  age=20, income=20000 (second false) \rightarrow reject
\item
  age=16, income=20000 (both false) \rightarrow reject
\end{itemize}

\textbf{Advantages}:

\begin{itemize}
\tightlist
\item
  \textbf{Thorough testing}: Tests internal logic
\item
  \textbf{Early defect detection}: Finds logic errors
\item
  \textbf{Coverage measurement}: Quantifiable testing progress
\end{itemize}

\textbf{Disadvantages}:

\begin{itemize}
\tightlist
\item
  \textbf{Time consuming}: Requires code knowledge
\item
  \textbf{Expensive}: Needs skilled testers
\item
  \textbf{Maintenance}: Changes with code updates
\end{itemize}

\textbf{Tools}: JUnit (Java), NUnit (.NET), Coverage.py (Python)

\end{solutionbox}
\begin{mnemonicbox}
``Statement Branch Path Condition'' for coverage
types

\end{mnemonicbox}
\begin{center}\rule{0.5\linewidth}{0.5pt}\end{center}

\subsection*{Question 5(a) OR [3
marks]}\label{q5a}

\textbf{Explain code walk through Technique in code review.}

\begin{solutionbox}

\textbf{Code Walk Through} is informal code review technique where
author presents code to team.


{\def\LTcaptype{none} % do not increment counter
\vspace{-5pt}
\captionof{table}{Walk Through Process}
\vspace{-10pt}
\begin{longtable}[]{@{}
  >{\raggedright\arraybackslash}p{(\linewidth - 4\tabcolsep) * \real{0.3333}}
  >{\raggedright\arraybackslash}p{(\linewidth - 4\tabcolsep) * \real{0.3333}}
  >{\raggedright\arraybackslash}p{(\linewidth - 4\tabcolsep) * \real{0.3333}}@{}}
\toprule\noalign{}
\begin{minipage}[b]{\linewidth}\raggedright
Phase
\end{minipage} & \begin{minipage}[b]{\linewidth}\raggedright
Description
\end{minipage} & \begin{minipage}[b]{\linewidth}\raggedright
Duration
\end{minipage} \\
\midrule\noalign{}
\endhead
\bottomrule\noalign{}
\endlastfoot
\textbf{Preparation} & Author prepares presentation & 30 minutes \\
\textbf{Presentation} & Author explains code logic & 1-2 hours \\
\textbf{Discussion} & Team asks questions, suggests improvements & 30
minutes \\
\textbf{Documentation} & Record issues and action items & 15 minutes \\
\end{longtable}
}

\textbf{Key Characteristics}:

\begin{itemize}
\tightlist
\item
  \textbf{Author-led}: Code author drives the session
\item
  \textbf{Informal process}: Less structured than inspection
\item
  \textbf{Educational}: Team learns about code functionality
\item
  \textbf{Collaborative}: Open discussion encouraged
\end{itemize}

\textbf{Participants}:

\begin{itemize}
\tightlist
\item
  \textbf{Author}: Presents and explains code
\item
  \textbf{Reviewers}: Ask questions and provide feedback
\item
  \textbf{Moderator}: Keeps discussion focused (optional)
\end{itemize}

\textbf{Benefits}: Knowledge sharing, early problem detection, team
collaboration, learning opportunity

\end{solutionbox}
\begin{mnemonicbox}
``Prepare Present Discuss Document''

\end{mnemonicbox}
\begin{center}\rule{0.5\linewidth}{0.5pt}\end{center}

\subsection*{Question 5(b) OR [4
marks]}\label{q5b}

\textbf{Explain software documentation.}

\begin{solutionbox}

\textbf{Software Documentation} provides information about software
system for various stakeholders.


{\def\LTcaptype{none} % do not increment counter
\vspace{-5pt}
\captionof{table}{Documentation Types}
\vspace{-10pt}
\begin{longtable}[]{@{}
  >{\raggedright\arraybackslash}p{(\linewidth - 4\tabcolsep) * \real{0.3333}}
  >{\raggedright\arraybackslash}p{(\linewidth - 4\tabcolsep) * \real{0.3333}}
  >{\raggedright\arraybackslash}p{(\linewidth - 4\tabcolsep) * \real{0.3333}}@{}}
\toprule\noalign{}
\begin{minipage}[b]{\linewidth}\raggedright
Type
\end{minipage} & \begin{minipage}[b]{\linewidth}\raggedright
Purpose
\end{minipage} & \begin{minipage}[b]{\linewidth}\raggedright
Audience
\end{minipage} \\
\midrule\noalign{}
\endhead
\bottomrule\noalign{}
\endlastfoot
\textbf{User Documentation} & How to use software & End users \\
\textbf{System Documentation} & Technical details & Developers,
maintainers \\
\textbf{Process Documentation} & Development process & Project team \\
\textbf{Requirements Documentation} & What system should do & All
stakeholders \\
\end{longtable}
}

\textbf{Internal Documentation}:

\begin{itemize}
\tightlist
\item
  \textbf{Code comments}: Explain complex logic
\item
  \textbf{Function headers}: Describe purpose and parameters\\
\item
  \textbf{Variable names}: Self-documenting identifiers
\item
  \textbf{README files}: Project overview and setup
\end{itemize}

\textbf{External Documentation}:

\begin{itemize}
\tightlist
\item
  \textbf{User manuals}: Step-by-step usage instructions
\item
  \textbf{Installation guides}: Setup procedures
\item
  \textbf{API documentation}: Interface specifications
\item
  \textbf{Training materials}: Educational content
\end{itemize}

\textbf{Benefits}:

\begin{itemize}
\tightlist
\item
  \textbf{Maintainability}: Easier code updates
\item
  \textbf{Knowledge transfer}: New team members learn faster
\item
  \textbf{User support}: Reduces support requests
\item
  \textbf{Quality assurance}: Documents requirements and design
\end{itemize}

\textbf{Documentation Standards}: Consistent format, regular updates,
version control, accessibility

\end{solutionbox}
\begin{mnemonicbox}
``User System Process Requirements'' for types

\end{mnemonicbox}
\begin{center}\rule{0.5\linewidth}{0.5pt}\end{center}

\subsection*{Question 5(c) OR [7
marks]}\label{q5c}

\textbf{Write a short note on black box testing.}

\begin{solutionbox}

\textbf{Black Box Testing} examines software functionality without
knowledge of internal code structure.

\includegraphics[width=1\linewidth,height=\textheight,keepaspectratio]{mermaid-75bf0c29.pdf}


{\def\LTcaptype{none} % do not increment counter
\vspace{-5pt}
\captionof{table}{Black Box Testing Characteristics}
\vspace{-10pt}
\begin{longtable}[]{@{}ll@{}}
\toprule\noalign{}
Aspect & Description \\
\midrule\noalign{}
\endhead
\bottomrule\noalign{}
\endlastfoot
\textbf{Focus} & External behavior \\
\textbf{Knowledge} & Requirements and specifications \\
\textbf{Approach} & Input-output relationship \\
\textbf{Coverage} & Functional requirements \\
\textbf{Perspective} & User viewpoint \\
\end{longtable}
}

\textbf{Black Box Testing Techniques}:


{\def\LTcaptype{none} % do not increment counter
\vspace{-5pt}
\captionof{table}{Testing Techniques}
\vspace{-10pt}
\begin{longtable}[]{@{}
  >{\raggedright\arraybackslash}p{(\linewidth - 4\tabcolsep) * \real{0.3333}}
  >{\raggedright\arraybackslash}p{(\linewidth - 4\tabcolsep) * \real{0.3333}}
  >{\raggedright\arraybackslash}p{(\linewidth - 4\tabcolsep) * \real{0.3333}}@{}}
\toprule\noalign{}
\begin{minipage}[b]{\linewidth}\raggedright
Technique
\end{minipage} & \begin{minipage}[b]{\linewidth}\raggedright
Description
\end{minipage} & \begin{minipage}[b]{\linewidth}\raggedright
Example
\end{minipage} \\
\midrule\noalign{}
\endhead
\bottomrule\noalign{}
\endlastfoot
\textbf{Equivalence Partitioning} & Divide inputs into valid/invalid
classes & Age: 0-17, 18-65, \textgreater65 \\
\textbf{Boundary Value Analysis} & Test at boundaries & Test age: 17,
18, 65, 66 \\
\textbf{Decision Table} & Complex business rules & Insurance premium
calculation \\
\textbf{State Transition} & System state changes & ATM states: idle,
processing, error \\
\end{longtable}
}

\textbf{1. Equivalence Partitioning}:

\begin{itemize}
\tightlist
\item
  \textbf{Valid partitions}: Accepted inputs
\item
  \textbf{Invalid partitions}: Rejected inputs
\item
  \textbf{Test one value} from each partition
\end{itemize}

\textbf{Example}: Password length (6-12 characters)

\begin{itemize}
\tightlist
\item
  Valid: 6-12 characters
\item
  Invalid: \textless6 characters, \textgreater12 characters
\end{itemize}

\textbf{2. Boundary Value Analysis}:

\begin{itemize}
\tightlist
\item
  Test \textbf{minimum, maximum, just below minimum, just above maximum}
\item
  Most errors occur at boundaries
\end{itemize}

\textbf{Example}: For range 1-100

\begin{itemize}
\tightlist
\item
  Test: 0, 1, 2, 99, 100, 101
\end{itemize}

\textbf{3. Decision Table Testing}:

\begin{itemize}
\tightlist
\item
  \textbf{Conditions}: Input conditions
\item
  \textbf{Actions}: Expected outputs
\item
  \textbf{Rules}: Condition-action combinations
\end{itemize}

\textbf{Advantages}:

\begin{itemize}
\tightlist
\item
  \textbf{User perspective}: Tests from user viewpoint
\item
  \textbf{No code knowledge needed}: Testers don't need programming
  skills
\item
  \textbf{Unbiased}: Not influenced by code implementation
\item
  \textbf{Early testing}: Can start with requirements
\end{itemize}

\textbf{Disadvantages}:

\begin{itemize}
\tightlist
\item
  \textbf{Limited coverage}: May miss some code paths
\item
  \textbf{Redundant testing}: Might test same logic multiple times
\item
  \textbf{Difficult test case design}: Hard without internal knowledge
\end{itemize}

\textbf{Types of Black Box Testing}:

\begin{itemize}
\tightlist
\item
  \textbf{Functional Testing}: Core functionality
\item
  \textbf{Non-functional Testing}: Performance, usability
\item
  \textbf{Regression Testing}: After changes
\item
  \textbf{User Acceptance Testing}: Final validation
\end{itemize}

\textbf{Tools}: Selenium (web), Appium (mobile), TestComplete, QTP

\textbf{When to Use}:

\begin{itemize}
\tightlist
\item
  System testing phase
\item
  User acceptance testing
\item
  Integration testing
\item
  Regression testing
\end{itemize}

\end{solutionbox}
\begin{mnemonicbox}
``Equivalence Boundary Decision State'' for
techniques

\end{mnemonicbox}

\end{document}
