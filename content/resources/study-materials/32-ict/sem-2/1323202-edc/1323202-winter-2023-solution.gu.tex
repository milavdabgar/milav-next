\documentclass{article}

% content/resources/templates/preamble.tex
\usepackage[margin=0.6in]{geometry}
\author{Milav Dabgar}
\usepackage{amsmath,amssymb,amsthm}
\usepackage{booktabs}
\usepackage{multirow}
\usepackage{xcolor}
\usepackage{tcolorbox}
\tcbuselibrary{breakable,skins}
\usepackage[colorlinks=true,linkcolor=blue]{hyperref}
\usepackage{titlesec}
\usepackage{enumitem}
\usepackage{tikz}
\usepackage{pgfplots}
\usepackage{circuitikz}
\usepackage[version=4]{mhchem}
\usepackage{longtable}
\usepackage{array}
\usepackage{float}
\usepackage{caption}
\usepackage{listings}

\lstset{
  basicstyle=\small\ttfamily,
  breaklines=true,
  breakatwhitespace=false,
  postbreak=\mbox{\textcolor{red}{$\hookrightarrow$}\space},
  float=false,
  numbers=left,
  numberstyle=\tiny\color{gray},
  numbersep=10pt,
  xleftmargin=2em,
  keywordstyle=\color{blue},
  commentstyle=\color{green!60!black},
  stringstyle=\color{purple},
  backgroundcolor=\color{gray!5},
  showstringspaces=false,
  tabsize=2,
  captionpos=b,
  keepspaces=true,
  columns=flexible
}

\pgfplotsset{compat=1.18}
\usetikzlibrary{shapes,arrows,positioning,calc,patterns,decorations.pathmorphing,decorations.markings,arrows.meta}

% Color scheme
\definecolor{headcolor}{RGB}{0,102,204}
\definecolor{keycolor}{RGB}{220,20,60}
\definecolor{solutioncolor}{RGB}{34,139,34}
\definecolor{mnemoniccolor}{RGB}{148,0,211}
\definecolor{codecolor}{RGB}{0,0,100}

% Spacing
\setlength{\parskip}{3pt}
\setlist[itemize]{nosep}
\setlist[enumerate]{nosep}

% Title formatting
\titleformat{\section}{\Large\bfseries\color{headcolor}}{\thesection}{1em}{}
\titleformat{\subsection}{\large\bfseries\color{headcolor}}{\thesubsection}{1em}{}

% Pandoc tightlist compatibility
\providecommand{\tightlist}{%
  \setlength{\itemsep}{0pt}\setlength{\parskip}{0pt}}

% Pandoc longtable compatibility
\newcounter{none}
\def\thenone{}


% content/resources/templates/gujarati-boxes.tex
\usepackage{fontspec}
\usepackage{polyglossia}

% Set Gujarati as main language (document is primarily in Gujarati)
% Note: gloss-gujarati.ldf doesn't exist in polyglossia, but it will use hyphenation patterns
\setdefaultlanguage{gujarati}
\setotherlanguage{english}

% Configure Gujarati font properly
% Use Language=Default to prevent polyglossia from trying to add language-specific features
% that don't exist for Gujarati, which causes "empty feature" warnings
\newfontfamily\gujaratifont[Script=Gujarati,AutoFakeBold=2.5,AutoFakeSlant=0.3]{Noto Sans Gujarati}
\setmainfont[Script=Gujarati,AutoFakeBold=2.5,AutoFakeSlant=0.3]{Noto Sans Gujarati}
% Use Noto Sans Gujarati for monospace to support Gujarati in text
\setmonofont[Scale=0.9]{Noto Sans Gujarati}

% Configure English to use the same font
\newfontfamily\englishfont[Script=Gujarati,AutoFakeBold=2.5,AutoFakeSlant=0.3]{Noto Sans Gujarati}

% Translations for polyglossia
\gappto\captionsgujarati{
  \renewcommand{\tablename}{કોષ્ટક}
  \renewcommand{\figurename}{આકૃતિ}
}

% Helper for TikZ nodes to ensure Gujarati font
\newcommand{\gu}[1]{{\gujaratifont #1}}

% Custom environments
\newtcolorbox{solutionbox}{
    breakable,
    enhanced,
    colback=solutioncolor!5!white,
    colframe=solutioncolor!75!black,
    fonttitle=\bfseries,
    title=જવાબ
}

\newtcolorbox{solutionboxnobreak}{
 colback=solutioncolor!5!white,
 colframe=solutioncolor!75!black,
 fonttitle=\bfseries,
 title=જવાબ
}

\newtcolorbox{keyformula}{
 breakable,
 enhanced,
 colback=keycolor!5!white,
 colframe=keycolor!75!black,
 fonttitle=\bfseries,
 title=રાસાયણિક સમીકરણ/સૂત્ર
}

\newtcolorbox{mnemonicbox}{
 breakable,
 enhanced,
 colback=mnemoniccolor!5!white,
 colframe=mnemoniccolor!75!black,
 fonttitle=\bfseries,
 title=મેમરી ટ્રીક
}


% Custom commands for GTU solutions
% This file defines semantic commands for consistent formatting

% Question command with automatic formatting
\newcommand{\question}[2]{%
  \section*{Question #1}%
  \textbf{#2}%
}

% OR question variant
\newcommand{\questionor}[2]{%
  \section*{Question #1 OR}%
  \textbf{#2}%
}

% Proper table environment with caption
\newenvironment{answertable}[1]{%
  \begin{table}[htbp]
  \centering
  \caption{#1}
}{%
  \end{table}
}

% Proper figure environment for diagrams
\newenvironment{answerdiagram}[1]{%
  \begin{figure}[htbp]
  \centering
  \caption{#1}
}{%
  \end{figure}
}

% Semantic markup for key terms
\newcommand{\keyword}[1]{\textbf{#1}}
\newcommand{\code}[1]{\texttt{#1}}
\newcommand{\classname}[1]{\texttt{#1}}
\newcommand{\methodname}[1]{\texttt{#1}}

% Proper quotation marks
\newcommand{\mnemonic}[1]{``#1''}


\title{Electronics Devices \& Circuits (1323202) - Winter 2023 Solution}
\date{January 24, 2024}

\begin{document}
\maketitle


\questionmarks{1(a)}{3}{સ્વચ્છ આકૃતિ સાથે ડીસી લોડ લાઈન વિષે સમજાવો.}

\begin{solutionbox}
DC લોડ લાઈન ટ્રાન્ઝિસ્ટરના આઉટપુટ ખાસિયતો પર એક સીધી રેખા છે જે બધા સંભવિત ઓપરેટિંગ પોઇન્ટ્સ બતાવે છે.

\textbf{આકૃતિ:}

\begin{center}
\begin{tikzpicture}[scale=1]
    % Axes
    \draw[->] (0,0) -- (5,0) node[right] {$V_{CE}$};
    \draw[->] (0,0) -- (0,4) node[above] {$I_C$};
    
    % Load Line
    \draw[thick] (0,3) node[left] {$I_{C(sat)} = \frac{V_{CC}}{R_C}$} -- (4,0) node[below] {$V_{CC} (Cutoff)$};
    
    % Curves
    \draw[gray, domain=0:4] plot (\x, {3*(1-exp(-\x))*0.8});
    \draw[gray, domain=0:4] plot (\x, {3*(1-exp(-\x))*0.6});
    \draw[gray, domain=0:4] plot (\x, {3*(1-exp(-\x))*0.4});
    
    % Q-point
    \filldraw (1.5, 1.875) circle (2pt) node[above right] {$Q (Operating Point)$};
    
    \node at (2.5, -1) {DC Load Line};
\end{tikzpicture}
\end{center}

\begin{itemize}
    \item \textbf{કલેક્ટર સેચુરેશન કરંટ}: જ્યારે $V_{CE} = 0$, ત્યારે $I_C = V_{CC}/R_C$
    \item \textbf{કટઓફ વોલ્ટેજ}: જ્યારે $I_C = 0$, ત્યારે $V_{CE} = V_{CC}$
    \item \textbf{Q-પોઇન્ટ}: લોડ લાઈન પર ઓપરેટિંગ પોઇન્ટ
\end{itemize}

\begin{mnemonicbox}
"LEVEL" - "Load line Establishes Voltage and current for Every Load condition"
\end{mnemonicbox}
\end{solutionbox}

\questionmarks{1(b)}{4}{થર્મલ રનઅવે વિગતવાર સમજાવો.}

\begin{solutionbox}
થર્મલ રનઅવે એક એવી સ્થિતિ છે જ્યાં ગરમી ટ્રાન્ઝિસ્ટરના કલેક્ટર કરંટમાં વધારો કરે છે, જે વધુ ગરમી ઉત્પન્ન કરે છે, જે ટ્રાન્ઝિસ્ટરને નુકસાન તરફ દોરી જાય છે.

\textbf{આકૃતિ:}

\begin{center}
\begin{tikzpicture}[gtu block]
    \node [draw, rectangle] (temp1) {તાપમાન વધે છે};
    \node [draw, rectangle, below=0.8cm of temp1] (leak) {લીકેજ કરંટ વધે છે};
    \node [draw, rectangle, below=0.8cm of leak] (ic) {કલેક્ટર કરંટ વધે છે};
    \node [draw, rectangle, below=0.8cm of ic] (power) {વધુ પાવર વપરાશ};
    \node [draw, rectangle, below=0.8cm of power] (temp2) {વધુ તાપમાન વધારો};
    
    \draw [gtu arrow] (temp1) -- (leak);
    \draw [gtu arrow] (leak) -- (ic);
    \draw [gtu arrow] (ic) -- (power);
    \draw [gtu arrow] (power) -- (temp2);
    \draw [gtu arrow] (temp2.east) to[out=0,in=0] (temp1.east);
\end{tikzpicture}
\end{center}

\begin{itemize}
    \item \textbf{ગરમી ઉત્પાદન}: પાવર વપરાશ = $V_{CE} \times I_C$
    \item \textbf{મહત્વપૂર્ણ અસર}: વધારેલ જંક્શન તાપમાન $V_{BE}$ ઘટાડે છે
    \item \textbf{નિવારણ}: હીટ સિંક, થર્મલ સ્ટેબલાઇઝેશન સર્કિટ્સ, યોગ્ય બાયસિંગ
    \item \textbf{ખતરો}: નિયંત્રિત ન કરવામાં આવે તો ટ્રાન્ઝિસ્ટરને નષ્ટ કરી શકે છે
\end{itemize}

\begin{mnemonicbox}
"HEAT" - "Higher Emission Amplifies Temperature"
\end{mnemonicbox}
\end{solutionbox}

\questionmarks{1(c)}{7}{ટુ સ્ટેજ R-C કપલ્ડ એમ્પ્લીફાયરનો સર્કિટ ડાયાગ્રામ અને ફ્રીક્વન્શી રિસ્પોન્સ દોરો. દરેક કમ્પોનન્ટનું મહત્વ સમજાવો.}

\begin{solutionbox}
R-C કપલ્ડ એમ્પ્લીફાયર મલ્ટીપલ ટ્રાન્ઝિસ્ટર સ્ટેજ્સને જોડવા માટે કેપેસિટર્સનો ઉપયોગ કરે છે.

\textbf{સર્કિટ ડાયાગ્રામ:}

\begin{center}
\begin{circuitikz}[american, scale=0.9, transform shape]
    % Stage 1
    \draw (0,0) node[ground] {} to[R, l=$R_{E1}$] (0,1) to[Tnpn, n=Q1] (0,3) to[R, l=$R_{C1}$] (0,5) -- (0,6) node[vcc] {$V_{CC}$};
    \draw (Q1.B) -- (-1.5, 1.85); 
    \draw (-1.5, 6) to[R, l=$R_1$] (-1.5, 1.85) to[R, l=$R_2$] (-1.5, 0) node[ground] {};
    \draw (-3.5, 1.85) to[C, l=$C_{in}$] (-1.5, 1.85);
    \draw (-3.5, 1.85) node[left] {$V_{in}$};
    \draw (0,1) -- (1,1) to[C, l=$C_{E1}$] (1,0) node[ground] {};
    
    % Coupling
    \draw (0,3) to[C, l=$C_C$] (3,3) -- (3, 1.85);
    
    % Stage 2
    \draw (4.5,0) node[ground] {} to[R, l=$R_{E2}$] (4.5,1) to[Tnpn, n=Q2] (4.5,3) to[R, l=$R_{C2}$] (4.5,5) -- (4.5,6) -- (0,6);
    \draw (Q2.B) -- (3, 1.85);
    \draw (3, 6) to[R, l=$R_3$] (3, 1.85) to[R, l=$R_4$] (3, 0) node[ground] {};
    \draw (4.5,1) -- (5.5,1) to[C, l=$C_{E2}$] (5.5,0) node[ground] {};
    
    % Output
    \draw (4.5,3) to[C, l=$C_{out}$] (6.5,3) node[right] {$V_{out}$};
    
    \draw (-1.5, 6) -- (0,6);
    \draw (3,6) -- (4.5,6);
\end{circuitikz}
\end{center}

\textbf{ફ્રીક્વન્સી રિસ્પોન્સ:}

\begin{center}
\begin{tikzpicture}[scale=0.8]
    \begin{axis}[
        xlabel={Frequency (log scale)},
        ylabel={Gain (dB)},
        xmode=log,
        ymin=0, ymax=50,
        grid=both,
        width=10cm, height=5cm
    ]
    \addplot[smooth, thick] coordinates {
        (10, 10) (100, 37) (1000, 40) (10000, 40) (100000, 37) (1000000, 10)
    };
    \draw[dashed] (100, 37) -- (100, 0) node[below] {$f_L$};
    \draw[dashed] (100000, 37) -- (100000, 0) node[below] {$f_H$};
    \draw[<->] (100, 20) -- (100000, 20) node[midway, fill=white] {Bandwidth};
    \end{axis}
\end{tikzpicture}
\end{center}

\begin{itemize}
    \item \textbf{કપલિંગ કેપેસિટર્સ ($C_C$)}: DC બ્લોક કરે છે, સ્ટેજ્સ વચ્ચે AC સિગ્નલ ટ્રાન્સફર કરે છે
    \item \textbf{બાયસિંગ રેસિસ્ટર્સ ($R_1, R_2$)}: ટ્રાન્ઝિસ્ટર ઓપરેશન માટે યોગ્ય Q-પોઇન્ટ સ્થાપિત કરે છે
    \item \textbf{બાયપાસ કેપેસિટર્સ ($C_E$)}: $R_E$ પર નેગેટિવ ફીડબેકથી ગેઇન ઘટાડો રોકે છે
    \item \textbf{બેન્ડવિડ્થ}: લો ($f_L$) અને હાઈ ($f_H$) કટઓફ ફ્રીક્વન્સી વચ્ચેનો રેન્જ
\end{itemize}

\begin{mnemonicbox}
"CARS" - "Coupling capacitors Allow Resistance Separation"
\end{mnemonicbox}
\end{solutionbox}

\vspace{0.5em}\centerline{\textbf{OR}}\questionmarks{1(c)}{7}{એમ્પ્લીફાયરમાં નેગેટીવ અને પોઝીટીવ ફીડબેક સરખાવો.}

\begin{solutionbox}
ફીડબેક સિસ્ટમ્સ આઉટપુટના એક ભાગને ઇનપુટ પર પાછો મોકલે છે.

\begin{center}
\captionof{table}{ફીડબેક પ્રકારોની સરખામણી}
\begin{tabulary}{\textwidth}{|L|L|L|}
\hline
\textbf{પેરામીટર} & \textbf{નેગેટિવ ફીડબેક} & \textbf{પોઝિટિવ ફીડબેક} \\
\hline
ગેઇન & ઘટાડે છે & વધારે છે \\
\hline
બેન્ડવિડ્થ & વધારે છે & ઘટાડે છે \\
\hline
સ્ટેબિલિટી & સુધારે છે & ઘટાડે છે \\
\hline
ડિસ્ટોર્શન & ઘટાડે છે & વધારે છે \\
\hline
નોઇઝ & ઘટાડે છે & વધારે છે \\
\hline
ઇનપુટ/આઉટપુટ ઇમ્પીડન્સ & નિયંત્રિત કરી શકાય છે & અનિશ્ચિત \\
\hline
એપ્લિકેશન્સ & એમ્પ્લિફાયર, રેગ્યુલેટર & ઓસિલેટર, શ્મિટ ટ્રિગર \\
\hline
\end{tabulary}
\end{center}

\begin{itemize}
    \item \textbf{નેગેટિવ ફીડબેક}: આઉટપુટ ઇનપુટથી 180° શિફ્ટ હોય છે
    \item \textbf{પોઝિટિવ ફીડબેક}: આઉટપુટ ઇનપુટથી 0° શિફ્ટ હોય છે
    \item \textbf{બાર્ખાઉસન ક્રાઇટેરિયા}: યુનિટી ગેઇન સાથે પોઝિટિવ ફીડબેક ઓસિલેશન ઉત્પન્ન કરે છે
\end{itemize}

\begin{mnemonicbox}
"SIGN" - "Stability Increases with Gain Negation"
\end{mnemonicbox}
\end{solutionbox}
\questionmarks{2(a)}{3}{ઓસિલેશન માટે બારખૌસન ક્રાઈટરીઆ (Barkhausen's criteria) જણાવો અને સમજાવો.}

\begin{solutionbox}
બાર્ખાઉસન ક્રાઇટેરિયા ફીડબેક સિસ્ટમમાં સતત ઓસિલેશન માટેની શરતો નિર્ધારિત કરે છે.

\textbf{આકૃતિ:}

\begin{center}
\begin{tikzpicture}[gtu block]
    \node [draw, rectangle] (amp) {એમ્પ્લિફાયર ($A$)};
    \node [draw, rectangle, below=1cm of amp] (beta) {ફીડબેક નેટવર્ક ($\beta$)};
    \node [left=1cm of amp] (in) {};
    \node [right=1cm of amp] (out) {};

    \draw [gtu arrow] (in) -- node[above] {$V_{in}=0$} (amp);
    \draw [gtu arrow] (amp) -- node[above] {$V_{out}$} (out);
    \draw [gtu arrow] (amp.east) -- ++(0.5,0) |- (beta.east);
    \draw [gtu arrow] (beta.west) -| (amp.west);
    
    \node [right, align=left] at (4,0) {શરતો:\\1. લૂપ ગેઇન $|A\beta| = 1$\\2. ફેઝ શિફ્ટ $\angle A\beta = 0^\circ$ કે $360^\circ$};
\end{tikzpicture}
\end{center}

\begin{itemize}
    \item \textbf{ગેઇન શરત}: લૂપ ગેઇન ($A\times\beta$) 1 (યુનિટી) હોવી જોઈએ
    \item \textbf{ફેઝ શરત}: લૂપની આસપાસ કુલ ફેઝ શિફ્ટ $0^\circ$ અથવા $360^\circ$ હોવી જોઈએ
    \item \textbf{વ્યવહારિક અમલીકરણ}: શરૂઆતમાં લૂપ ગેઇન $> 1$ હોય છે જેથી ઓસિલેશન શરૂ થાય, પછી નોન-લિનિયારિટીના કારણે 1 પર સ્થિર થાય છે
\end{itemize}

\begin{mnemonicbox}
"LOOP" - "Loop's Overall Output Phase"
\end{mnemonicbox}
\end{solutionbox}

\questionmarks{2(b)}{4}{ફિક્સ્ડ બાયસ, કલેક્ટર ટુ બેઝ બાયસ અને વોલ્ટેજ ડિવાઈડર બાયસ પદ્ધતિઓની સરખામણી કરો.}

\begin{solutionbox}
વિવિધ બાયસિંગ તકનીકો સ્થિરતા અને તાપમાન ક્ષતિપૂર્તિના વિવિધ સ્તરો પ્રદાન કરે છે.

\begin{center}
\captionof{table}{બાયસિંગ પદ્ધતિઓની સરખામણી}
\begin{tabulary}{\textwidth}{|L|L|L|L|}
\hline
\textbf{પેરામીટર} & \textbf{ફિક્સ્ડ બાયસ} & \textbf{કલેક્ટર-બેઝ બાયસ} & \textbf{વોલ્ટેજ ડિવાઇડર બાયસ} \\
\hline
સ્ટેબિલિટી & નબળી & વધુ સારી & ઉત્તમ \\
\hline
સર્કિટ જટિલતા & સરળ & મધ્યમ & જટિલ \\
\hline
તાપમાન સ્ટેબિલિટી & નબળી & મધ્યમ & સારી \\
\hline
કોમ્પોનેન્ટ્સ & 1 રેસિસ્ટર & 1 રેસિસ્ટર & 3-4 રેસિસ્ટર \\
\hline
સ્ટેબિલિટી ફેક્ટર (S) & ઉચ્ચ (અસ્થિર) & મધ્યમ & નીચો (સ્થિર) \\
\hline
\end{tabulary}
\end{center}

\begin{itemize}
    \item \textbf{ફિક્સ્ડ બાયસ}: બેઝથી $V_{CC}$ સુધી એક રેસિસ્ટર
    \item \textbf{કલેક્ટર-બેઝ બાયસ}: કલેક્ટરથી બેઝ સુધી ફીડબેક રેસિસ્ટર નેગેટિવ ફીડબેક આપે છે
    \item \textbf{વોલ્ટેજ ડિવાઇડર}: બે રેસિસ્ટર $\beta$ થી સ્વતંત્ર સ્થિર રેફરન્સ વોલ્ટેજ બનાવે છે
\end{itemize}

\begin{mnemonicbox}
"STORM" - "Stability Through Optimized Resistor Methods"
\end{mnemonicbox}
\end{solutionbox}

\questionmarks{2(c)}{7}{હાર્ટલી ઓસીલેટર પર ટૂંક નોંધ લખો.}

\begin{solutionbox}
હાર્ટલી ઓસિલેટર એક LC ઓસિલેટર છે જેમાં ફીડબેક માટે ટેપ્ડ ઇન્ડક્ટર હોય છે.

\textbf{સર્કિટ ડાયાગ્રામ:}

\begin{center}
\begin{circuitikz}[american, scale=0.9, transform shape]
    % Transistor
    \draw (0,0) node[ground] {} to[R, l=$R_E$] (0,1) to[Tnpn, n=Q1] (0,3) to[L, l=$RFC$] (0,5) -- (0,6) node[vcc] {$V_{CC}$};
    \draw (0,1) -- (1,1) to[C, l=$C_E$] (1,0) node[ground] {};
    
    % Biasing
    \draw (-1.5, 6) to[R, l=$R_1$] (-1.5, 1.85) -- (Q1.B);
    \draw (-1.5, 1.85) to[R, l=$R_2$] (-1.5, 0) node[ground] {};
    \draw (-1.5, 6) -- (0,6);
    
    % Tank Circuit
    \draw (3, 4) to[C, l=$C$] (3, 0);
    \draw (4.5, 4) to[L, l=$L_1$] (4.5, 2) node[ground] {} to[L, l=$L_2$] (4.5, 0);
    \draw (3,4) -- (4.5,4);
    \draw (3,0) -- (4.5,0);
    
    % Coupling
    \draw (0,3) -- (0,4) to[C, l=$C_C$] (3,4); % Collector to Tank
    \draw (-1.5, 1.85) -- (-2.5, 1.85) to[C, l=$C_{in}$] (-2.5, 0) -- (3,0); % Feedback to Base
\end{circuitikz}
\end{center}

\begin{itemize}
    \item \textbf{સર્કિટ કોમ્પોનેન્ટ્સ}: એમ્પ્લિફાયર (CE મોડ), ટેન્ક સર્કિટ ($L_1, L_2, C$)
    \item \textbf{ફ્રીક્વન્સી ફોર્મ્યુલા}: $f = \frac{1}{2\pi\sqrt{L_{eq}C}}$ જ્યાં $L_{eq} = L_1 + L_2$
    \item \textbf{ફીડબેક}: $L_2$ પર વોલ્ટેજ બેઝ પર પાછો આપવામાં આવે છે (ટેન્ક દ્વારા $180^\circ$ ફેઝ શિફ્ટ + એમ્પ્લિફાયર દ્વારા $180^\circ$ = $360^\circ$)
    \item \textbf{એપ્લિકેશન્સ}: RF સિગ્નલ જનરેટર, રેડિયો રિસીવર
\end{itemize}

\begin{mnemonicbox}
"TILC" - "Tapped Inductor with LC Circuit"
\end{mnemonicbox}
\end{solutionbox}

\vspace{0.5em}\centerline{\textbf{OR}}\questionmarks{2(a)}{3}{ટ્રાન્ઝિસ્ટરનું સ્વિચ તરીકે કાર્ય સમજાવો.}

\begin{solutionbox}
ટ્રાન્ઝિસ્ટર ડિજિટલ એપ્લિકેશન્સ માટે કટઓફ (OFF) અને સેચુરેશન (ON) રીજન્સ વચ્ચે સ્વિચ કરે છે.

\textbf{આકૃતિ:}

\begin{center}
\begin{tikzpicture}[gtu block]
    \node [draw, rectangle] (in) {ઇનપુટ};
    \node [draw, diamond, right=1cm of in] (trans) {ટ્રાન્ઝિસ્ટર};
    \node [draw, rectangle, right=1cm of trans] (out) {આઉટપુટ};
    
    \draw [gtu arrow] (in) -- node[above] {High} (trans);
    \draw [gtu arrow] (trans) -- node[above] {Low (ON)} (out);
    
    \node [below=0.5cm of trans] (state) {સ્ટેટ};
    \draw [dashed] (trans) -- (state);
\end{tikzpicture}
\end{center}

\begin{itemize}
    \item \textbf{કટઓફ રીજન}: $V_{BE} < 0.7V$, ઓપન સ્વિચ તરીકે કાર્ય કરે છે, $V_{CE} \approx V_{CC}$ (આઉટપુટ High)
    \item \textbf{સેચુરેશન રીજન}: $V_{BE} > 0.7V$, ક્લોઝ્ડ સ્વિચ તરીકે કાર્ય કરે છે, $V_{CE} \approx 0.2V$ (આઉટપુટ Low)
    \item \textbf{સ્વિચિંગ ટાઇમ}: જંક્શન કેપેસિટન્સ દ્વારા મર્યાદિત થાય છે
\end{itemize}

\begin{mnemonicbox}
"COPS" - "Cutoff-On-Produces Switching"
\end{mnemonicbox}
\end{solutionbox}

\questionmarks{2(b)}{4}{હીટ સિંક વ્યાખ્યાયિત કરો. હીટ સિંકના પ્રકારોની યાદી બનાવો અને તેની એપ્લિકેશન લખો.}

\begin{solutionbox}
હીટ સિંક એક થર્મલ કન્ડક્ટર છે જે ઓવરહિટીંગ અટકાવવા ઇલેક્ટ્રોનિક કોમ્પોનેન્ટ્સમાંથી ગરમી દૂર કરે છે.

\textbf{આકૃતિ:}

\begin{center}
\begin{tikzpicture}
    % Simple finned heat sink
    \draw[fill=gray!30] (0,0) rectangle (4, 0.5); % Base
    \foreach \x in {0.2, 0.8, ..., 3.8}
        \draw[fill=gray!30] (\x, 0.5) rectangle (\x+0.2, 1.5); % Fins
    \node at (2, -0.5) {Transistor};
    \draw (1.5, -0.5) rectangle (2.5, 0);
\end{tikzpicture}
\end{center}

\textbf{હીટ સિંકના પ્રકારો:}

\begin{center}
\captionof{table}{હીટ સિંક પ્રકાર અને એપ્લિકેશન}
\begin{tabulary}{\textwidth}{|L|L|L|}
\hline
\textbf{પ્રકાર} & \textbf{વર્ણન} & \textbf{એપ્લિકેશન} \\
\hline
પેસિવ & કોઈ ચલિત ભાગો નહીં, કુદરતી કન્વેક્શન & ઓછી પાવર ડિવાઇસીસ \\
\hline
એક્ટિવ & ફેન અથવા પંપ સાથે & હાઈ પાવર એમ્પ્લિફાયર, CPU \\
\hline
લિક્વિડ-કૂલ્ડ & હીટ ટ્રાન્સફર માટે પ્રવાહી વાપરે છે & સુપરકોમ્પ્યુટર્સ \\
\hline
ફિન્ડ & મલ્ટીપલ ફિન્સ સરફેસ એરિયા વધારે છે & પાવર ટ્રાન્ઝિસ્ટર \\
\hline
\end{tabulary}
\end{center}

\begin{mnemonicbox}
"COOL" - "Conducting Out Of Local heat"
\end{mnemonicbox}
\end{solutionbox}

\questionmarks{2(c)}{7}{એમ્પ્લીફાયરમાં નેગેટીવ ફીડબેક ના ફાયદા અને ગેરફાયદાને વિગતવાર સમજાવો.}

\begin{solutionbox}
નેગેટિવ ફીડબેક આઉટપુટ સિગ્નલના એક ભાગને વિરુદ્ધ ફેઝમાં ઇનપુટ પર પાછો મોકલે છે જેથી સ્ટેબિલિટી સુધરે.

\textbf{ફીડબેક બ્લોક ડાયાગ્રામ:}

\begin{center}
\begin{tikzpicture}[gtu block]
    \node [draw, circle] (sum) {$\Sigma$};
    \node [draw, rectangle, right=1cm of sum] (amp) {$A$};
    \node [draw, rectangle, below=1cm of amp] (beta) {$\beta$};
    \node [left=1cm of sum] (in) {$V_{in}$};
    \node [right=1cm of amp] (out) {$V_{out}$};

    \draw [gtu arrow] (in) -- node[above] {+} (sum);
    \draw [gtu arrow] (sum) -- (amp);
    \draw [gtu arrow] (amp) -- (out);
    \draw [gtu arrow] (amp.east) -- ++(0.5,0) |- (beta.east);
    \draw [gtu arrow] (beta.west) -| node[left] {-} (sum.south);
\end{tikzpicture}
\end{center}

\begin{center}
\captionof{table}{નેગેટિવ ફીડબેકના ફાયદા અને ગેરફાયદા}
\begin{tabulary}{\textwidth}{|L|L|}
\hline
\textbf{ફાયદા} & \textbf{ગેરફાયદા} \\
\hline
ગેઇન સ્ટેબિલાઇઝ કરે છે & સમગ્ર ગેઇન ઘટાડે છે \\
\hline
બેન્ડવિડ્થ વધારે છે & વધુ કોમ્પોનેન્ટ્સની જરૂર પડે છે \\
\hline
નોન-લિનિયર ડિસ્ટોર્શન ઘટાડે છે & ફીડબેક સર્કિટમાં પાવર વપરાશ \\
\hline
નોઇઝ ઘટાડે છે & જટિલ ડિઝાઇન \\
\hline
ઇનપુટ/આઉટપુટ ઇમ્પીડન્સ નિયંત્રિત કરે છે & અયોગ્ય ડિઝાઇન હોય તો ઓસિલેશનની શક્યતા \\
\hline
\end{tabulary}
\end{center}

\begin{mnemonicbox}
"STABLE" - "Stabilized Transmission And Bandwidth with Less Error"
\end{mnemonicbox}
\end{solutionbox}
\questionmarks{3(a)}{3}{SCR નો સિમ્બોલ દોરો અને SCR નું કાર્ય સમજાવો.}

\begin{solutionbox}
સિલિકોન કંટ્રોલ્ડ રેક્ટિફાયર (SCR) એ ત્રણ ટર્મિનલ વાળું PNPN ચાર-લેયર ડિવાઇસ છે.

\textbf{સિમ્બોલ:}

\begin{center}
\begin{circuitikz}[american, scale=1.2, transform shape]
    \draw (0,0) to[Thyristor, name=SCR] (0, 2);
    \node[above] at (0,2) {એનોડ (A)};
    \node[below] at (0,0) {કેથોડ (K)};
    \node[right] at (SCR.gate) {ગેટ (G)};
\end{circuitikz}
\end{center}

\begin{itemize}
    \item \textbf{સ્ટ્રક્ચર}: P-N-P-N ચાર-લેયર સેમિકન્ડક્ટર ડિવાઇસ
    \item \textbf{ઓપરેશન}: ગેટ ટ્રિગર ન થાય ત્યાં સુધી OFF રહે છે, ત્યારબાદ કરંટ હોલ્ડિંગ વેલ્યુથી નીચે ન જાય ત્યાં સુધી કન્ડક્ટ કરે છે
    \item \textbf{ટર્મિનલ્સ}: એનોડ (A), કેથોડ (K), ગેટ (G)
\end{itemize}

\begin{mnemonicbox}
"AGK" - "Anode-Gate controls Kathode current"
\end{mnemonicbox}
\end{solutionbox}

\questionmarks{3(b)}{4}{સર્કિટ ડાયાગ્રામ સાથે SCR ની ટુ ટ્રાન્ઝિસ્ટર એનાલોજી સમજાવો.}

\begin{solutionbox}
SCRને સેમિકન્ડક્ટર લેયર્સ શેર કરતા ઇન્ટરકનેક્ટેડ PNP અને NPN ટ્રાન્ઝિસ્ટર તરીકે રજૂ કરી શકાય છે.

\textbf{સર્કિટ ડાયાગ્રામ:}

\begin{center}
\begin{circuitikz}[american, scale=1.0, transform shape]
    % PNP Transistor (Upper)
    \draw (2,4) node[above] {એનોડ} to[Tpnp, n=Q1] (2,2);
    
    % NPN Transistor (Lower)
    \draw (1,0) node[below] {કેથોડ} to[Tnpn, mirror, n=Q2] (1,2);
    
    % Connections
    \draw (Q1.C) -- (1,2) -- (Q2.B); % PNP Collector to NPN Base
    \draw (Q2.C) -- (2,2) -- (Q1.B); % NPN Collector to PNP Base
    
    % Gate
    \draw (Q2.B) -- (0, 0.85) node[left] {ગેટ};
    
    % Nodes
    \node[right] at (Q1.E) {Q1 (PNP)};
    \node[left] at (Q2.E) {Q2 (NPN)};
\end{circuitikz}
\end{center}

\begin{itemize}
    \item \textbf{PNP સેક્શન}: ઉપરનો ટ્રાન્ઝિસ્ટર જેનો કલેક્ટર NPN બેઝ સાથે જોડાયેલો છે
    \item \textbf{NPN સેક્શન}: નીચેનો ટ્રાન્ઝિસ્ટર જેનો કલેક્ટર PNP બેઝ સાથે જોડાયેલો છે
    \item \textbf{રિજનરેટિવ ફીડબેક}: Q1 નો કલેક્ટર કરંટ Q2 ના બેઝને આપે છે; Q2 નો કલેક્ટર કરંટ Q1 ના બેઝને આપે છે. એકવાર ટ્રિગર થયા પછી, લૂપ કન્ડક્શન જાળવી રાખે છે.
\end{itemize}

\begin{mnemonicbox}
"PNPN" - "Positive-Negative-Positive-Negative layers"
\end{mnemonicbox}
\end{solutionbox}

\questionmarks{3(c)}{7}{સર્કિટ ડાયાગ્રામ સાથે TRIAC આધારિત ફેન રેગ્યુલેટરનું કાર્ય સમજાવો.}

\begin{solutionbox}
TRIAC-આધારિત ફેન રેગ્યુલેટર ફેઝ કંટ્રોલ દ્વારા લોડને અપાતા AC પાવરને નિયંત્રિત કરે છે.

\textbf{સર્કિટ ડાયાગ્રામ:}

\begin{center}
\begin{circuitikz}[american, scale=1.0, transform shape]
    \draw (0,4) node[left] {AC Live} to[short, o-] (1,4) -- (3,4) -- (5,4);
    
    % Load
    \draw (5,4) to[lamp, l=Fan/Load] (5,2) to[Triac, n=T1, mirror] (5,0) -- (0,0) node[left] {AC Neutral} node[short, o] {};
    
    % Trigger Circuit
    \draw (1,4) to[R, l=$R_1$] (1,2.5) to[vR, l=$R_2 (Speed)$] (1,1) -- (3,1);
    \draw (3,1) to[C, l=$C_1$] (3,0) node[ground] {};
    \draw (3,0) -- (5,0); % Connect to neutral line
    
    % DIAC
    \draw (3,1) -- (3,2) to[generic, l=DIAC] (T1.gate);
    
    \node[right] at (T1) {TRIAC};
\end{circuitikz}
\end{center}

\begin{itemize}
    \item \textbf{ફેઝ કંટ્રોલ}: અસરકારક RMS વોલ્ટેજ નિયંત્રિત કરવા માટે TRIAC ના ફાયરિંગ એંગલ બદલે છે
    \item \textbf{ડાયક}: TRIAC માટે બાયડાયરેક્શનલ ટ્રિગરિંગ પલ્સ આપે છે
    \item \textbf{RC ટાઇમિંગ સર્કિટ}: $R_1, R_2, C_1$ ડિલે એંગલ નક્કી કરે છે; $R_2$ બદલવાથી સ્પીડ બદલાય છે
    \item \textbf{ઓપરેશન}: જ્યારે કેપેસિટર વોલ્ટેજ DIAC બ્રેકઓવર સુધી પહોંચે છે, ત્યારે TRIAC ફાયર થાય છે
\end{itemize}

\begin{mnemonicbox}
"TRIAC" - "Triggered Response In AC Circuits"
\end{mnemonicbox}
\end{solutionbox}

\vspace{0.5em}\centerline{\textbf{OR}}\questionmarks{3(a)}{3}{DIAC અને TRIAC ની V-I લાક્ષણિકતાઓ દોરો.}

\begin{solutionbox}
DIACs અને TRIACs સિમેટ્રિકલ ફોરવર્ડ અને રિવર્સ લાક્ષણિકતાઓ ધરાવતા બાયડાયરેક્શનલ ડિવાઇસ છે.

\textbf{DIAC ખાસિયતો:}
\begin{center}
\begin{tikzpicture}[scale=0.6]
    \draw[->] (-3,0) -- (3,0) node[right] {$V$};
    \draw[->] (0,-3) -- (0,3) node[above] {$I$};
    \draw[thick, blue] plot [smooth] coordinates {(0,0) (1,0.1) (2,0.2) (1.5,1.5) (2.5,2.5)};
    \draw[thick, blue] plot [smooth] coordinates {(0,0) (-1,-0.1) (-2,-0.2) (-1.5,-1.5) (-2.5,-2.5)};
    \node at (2, 1.5) {કન્ડક્શન};
    \node at (-2, -1.5) {કન્ડક્શન};
    \node[below] at (2,0) {$V_{BO}$};
    \node[above] at (-2,0) {$-V_{BO}$};
\end{tikzpicture}
\end{center}

\textbf{TRIAC ખાસિયતો:}
\begin{center}
\begin{tikzpicture}[scale=0.6]
    \draw[->] (-3,0) -- (3,0) node[right] {$V_{MT2-MT1}$};
    \draw[->] (0,-3) -- (0,3) node[above] {$I_T$};
    % Quadrant 1
    \draw[thick, red] plot coordinates {(0,0) (2,0.2) (2.2,0.3) (1,1.5) (2,2.5)};
    % Quadrant 3
    \draw[thick, red] plot coordinates {(0,0) (-2,-0.2) (-2.2,-0.3) (-1,-1.5) (-2,-2.5)};
    
    \node at (2.5, 2) {Mode I};
    \node at (-2.5, -2) {Mode III};
    \node[right] at (1.5, 0.5) {Gate Triggered};
\end{tikzpicture}
\end{center}

\begin{mnemonicbox}
"BIBO" - "Bidirectional In, Bidirectional Out"
\end{mnemonicbox}
\end{solutionbox}

\questionmarks{3(b)}{4}{SCR ની ગેટ ટ્રિગરિંગ પદ્ધતિ સમજાવો.}

\begin{solutionbox}
ગેટ ટ્રિગરિંગ ચોક્કસ સમયે SCR ચાલુ કરવા માટે ગેટ ટર્મિનલ પર કંટ્રોલ સિગ્નલ લાગુ કરે છે.

\textbf{આકૃતિ:}

\begin{center}
\begin{circuitikz}[american, scale=1.0, transform shape]
    \draw (0,4) to[battery1, l=$V_{AK}$] (0,0);
    \draw (0,4) to[lamp, l=Load] (3,4) to[Thyristor, name=SCR] (3,0) -- (0,0);
    
    % Gate Circuit
    \draw (-2, 0.85) to[battery1, l=$V_G$] (-2, 2) to[R, l=$R_G$] (0.5, 2) -- (SCR.gate);
    \draw (-2, 0.85) -- (3, 0.85); % Common cathode ground level connection
    
    \node[right] at (SCR.gate) {$I_G$};
\end{circuitikz}
\end{center}

\begin{itemize}
    \item \textbf{ગેટ પલ્સ}: ગેટ અને કેથોડ વચ્ચે પોઝિટિવ કરંટ લાગુ કરવામાં આવે છે
    \item \textbf{ટાઇમિંગ}: AC સર્કિટ્સમાં ફાયરિંગ એંગલ ($\alpha$) ને નિયંત્રિત કરે છે
    \item \textbf{જરૂરિયાત}: એનોડ કરંટ લેચિંગ કરંટ સુધી પહોંચે ત્યાં સુધી ગેટ કરંટ ચાલુ રહેવો જોઈએ
\end{itemize}

\begin{mnemonicbox}
"GATE" - "Gain Activation Through Electron flow"
\end{mnemonicbox}
\end{solutionbox}

\questionmarks{3(c)}{7}{ડીસી પાવર કંટ્રોલ માટે SCRની એપ્લિકેશન સમજાવો.}

\begin{solutionbox}
SCR PWM અથવા ફેઝ કંટ્રોલ તકનીકોનો ઉપયોગ કરીને સપ્લાય વોલ્ટેજને ચોપિંગ (Chopping) કરીને DC પાવરને નિયંત્રિત કરે છે.

\textbf{સર્કિટ ડાયાગ્રામ:}

\begin{center}
\begin{circuitikz}[american, scale=1.0, transform shape]
    \draw (0,0) to[battery1, l=$V_{DC}$] (0,3) -- (2,3);
    \draw (2,3) to[Thyristor, name=SCR] (4,3);
    \draw (4,3) to[R, l=$R_{Load}$] (4,0) -- (0,0);
    
    % Trigger
    \draw (SCR.gate) -- (2.8, 2.2) node[below] {Trigger/PWM};
    
    \draw (4,0) to[open, v^>=$V_{out}$] (4,3);
\end{circuitikz}
\end{center}

\begin{itemize}
    \item \textbf{ચોપિંગ}: SCR સ્વીચ તરીકે કાર્ય કરે છે, જે ઝડપથી ON અને OFF થાય છે
    \item \textbf{પાવર કંટ્રોલ}: સરેરાશ આઉટપુટ વોલ્ટેજ $V_{avg} = V_{in} \times \text{Duty Cycle}$
    \item \textbf{કોમ્યુટેશન}: DC સર્કિટ્સમાં, SCR ને OFF કરવા માટે ફોર્સ્ડ કોમ્યુટેશન સર્કિટની જરૂર પડે છે
\end{itemize}

\begin{mnemonicbox}
"POWER" - "Pulse Operation With Electronic Regulation"
\end{mnemonicbox}
\end{solutionbox}
\questionmarks{4(a)}{3}{Ideal OP-AMP ની લાક્ષણિકતાઓની સૂચિ બનાવો.}

\begin{solutionbox}
આદર્શ ઓપરેશનલ એમ્પ્લિફાયર્સ સંપૂર્ણ લાક્ષણિકતાઓ ધરાવતા સૈદ્ધાંતિક ઉપકરણો છે.

\begin{center}
\captionof{table}{આદર્શ Op-Amp લાક્ષણિકતાઓ}
\begin{tabulary}{\textwidth}{|L|L|}
\hline
\textbf{લાક્ષણિકતા} & \textbf{આદર્શ મૂલ્ય} \\
\hline
ઓપન લૂપ ગેઇન ($A_{OL}$) & અનંત ($\infty$) \\
\hline
ઇનપુટ ઇમ્પીડન્સ ($Z_{in}$) & અનંત ($\infty$) \\
\hline
આઉટપુટ ઇમ્પીડન્સ ($Z_{out}$) & શૂન્ય ($0\Omega$) \\
\hline
બેન્ડવિડ્થ ($BW$) & અનંત ($\infty$) \\
\hline
CMRR & અનંત ($\infty$) \\
\hline
સ્લ્યુ રેટ & અનંત ($\infty$) \\
\hline
ઓફસેટ વોલ્ટેજ & શૂન્ય ($0V$) \\
\hline
\end{tabulary}
\end{center}

\begin{mnemonicbox}
"IBOCSS" - "Infinite Bandwidth, Open-loop gain, CMRR, Slew rate, and Sensitivity"
\end{mnemonicbox}
\end{solutionbox}

\questionmarks{4(b)}{4}{સર્કિટ ડાયાગ્રામ સાથે OP-AMP નો ઉપયોગ કરીને ડીફરન્સીઅલ એમ્પ્લીફાયરનું કાર્ય સમજાવો.}

\begin{solutionbox}
ડિફરેન્શિયલ એમ્પ્લિફાયર કોમન સિગ્નલ્સને રિજેક્ટ કરતી વખતે બે ઇનપુટ્સ વચ્ચેના વોલ્ટેજ તફાવતને એમ્પ્લિફાય કરે છે.

\textbf{સર્કિટ ડાયાગ્રામ:}

\begin{center}
\begin{circuitikz}[american, scale=1.0, transform shape]
    \draw (0,0) node[op amp] (opamp) {};
    
    % Inverting Input
    \draw (opamp.-) -- (-2, 0.5) to[R, l=$R_1$] (-4, 0.5) node[left] {$V_1$};
    \draw (opamp.-) -- (-1.5, 0.5) -- (-1.5, 2) to[R, l=$R_2$] (1.5, 2) -| (opamp.out);
    
    % Non-Inverting Input
    \draw (opamp.+) -- (-1.5, -0.5) to[R, l=$R_3 (R_1)$] (-4, -0.5) node[left] {$V_2$};
    \draw (-1.5, -0.5) to[R, l=$R_4 (R_2)$] (-1.5, -2) node[ground] {};
    
    % Output
    \draw (opamp.out) to[short, -o] (2,0) node[right] {$V_{out}$};
\end{circuitikz}
\end{center}

\begin{itemize}
    \item \textbf{ગેઇન ફોર્મ્યુલા}: $V_{out} = \frac{R_2}{R_1}(V_2 - V_1)$ (જ્યારે $R_3=R_1, R_4=R_2$)
    \item \textbf{કોમન મોડ રિજેક્શન}: બંને ઇનપુટ્સમાં સામાન્ય હોય તેવા સિગ્નલ્સ (નોઇઝ) ને દૂર કરે છે
    \item \textbf{એપ્લિકેશન્સ}: ઇન્સ્ટ્રુમેન્ટેશન, મેડિકલ ઇક્વિપમેન્ટ
\end{itemize}

\begin{mnemonicbox}
"DIFF" - "Dual Input For Feedback"
\end{mnemonicbox}
\end{solutionbox}

\questionmarks{4(c)}{7}{OP-AMP ને ઇન્વર્ટિંગ એમ્પ્લીફાયર (ક્લોઝ્ડ લૂપ) તરીકે સમજાવો અને વોલ્ટેજ ગેઇન નું સમીકરણ મેળવો.}

\begin{solutionbox}
ઇન્વર્ટિંગ એમ્પ્લિફાયર ઇનપુટ સિગ્નલને ઇન્વર્ટ (180° ફેઝ શિફ્ટ) અને એમ્પ્લિફાય કરે છે.

\textbf{સર્કિટ ડાયાગ્રામ:}

\begin{center}
\begin{circuitikz}[american, scale=1.0, transform shape]
    \draw (0,0) node[op amp] (opamp) {};
    \draw (opamp.+) -- (-1, -0.5) node[ground] {};
    \draw (opamp.-) -- (-1, 0.5) to[R, l=$R_i$] (-3, 0.5) node[left] {$V_{in}$};
    \draw (opamp.-) -- (-1, 0.5) -- (-1, 2) to[R, l=$R_f$] (1, 2) -| (opamp.out);
    \draw (opamp.out) to[short, -o] (2,0) node[right] {$V_{out}$};
\end{circuitikz}
\end{center}

\textbf{ગેઇન ડેરિવેશન:}
\begin{itemize}
    \item ઇન્વર્ટિંગ ઇનપુટ (વર્ચ્યુઅલ ગ્રાઉન્ડ નોડ $V^-$) પર KCL લાગુ કરો:
    \[ I_1 + I_2 = 0 \]
    \item ઓપ-એમ્પ ઇનપુટ ઇમ્પીડન્સ અનંત હોવાથી, અંદર જતો કરંટ શૂન્ય છે.
    \[ I_1 = \frac{V_{in} - V^-}{R_i}, \quad I_2 = \frac{V_{out} - V^-}{R_f} \]
    \item વર્ચ્યુઅલ ગ્રાઉન્ડ કન્સેપ્ટ ($V^+ = 0$) મુજબ, $V^- \approx 0$.
    \[ \frac{V_{in}}{R_i} + \frac{V_{out}}{R_f} = 0 \]
    \[ \frac{V_{out}}{R_f} = -\frac{V_{in}}{R_i} \]
    \[ A_v = \frac{V_{out}}{V_{in}} = -\frac{R_f}{R_i} \]
\end{itemize}

\begin{mnemonicbox}
"VAIN" - "Virtual ground Amplification Inverts Negative"
\end{mnemonicbox}
\end{solutionbox}

\vspace{0.5em}\centerline{\textbf{OR}}\questionmarks{4(a)}{3}{OPAMP ના નીચેના પેરામીટર્સ વ્યાખ્યાયિત કરો: 1) CMRR, 2) સ્લૂ રેટ, 3) ગેઇન બેન્ડવિડ્થ પ્રોડક્ટ}

\begin{solutionbox}
આ પેરામીટર્સ ઓપરેશનલ એમ્પ્લિફાયર્સની કીપરફોર્મન્સ લાક્ષણિકતાઓ નક્કી કરે છે.

\begin{center}
\captionof{table}{Op-Amp પેરામીટર્સ}
\begin{tabulary}{\textwidth}{|L|L|L|}
\hline
\textbf{પેરામીટર} & \textbf{વ્યાખ્યા} & \textbf{મહત્વ} \\
\hline
CMRR & ડિફરેન્શિયલ ગેઇનનો કોમન-મોડ ગેઇન સાથેનો ગુણોત્તર ($A_d/A_{cm}$) & ઊંચું CMRR નોઇઝ રિજેક્શન માટે સારું \\
\hline
સ્લ્યુ રેટ & આઉટપુટ વોલ્ટેજમાં થતા ફેરફારનો મહત્તમ દર ($dV/dt$) & હાઈ-ફ્રીક્વન્સી પરફોર્મન્સ નક્કી કરે છે \\
\hline
ગેઇન-બેન્ડવિડ્થ & ઓપન લૂપ ગેઇન અને ફ્રીક્વન્સીનો ગુણાકાર & આપેલ ઓપ-એમ્પ માટે અચળ રહે છે \\
\hline
\end{tabulary}
\end{center}

\begin{mnemonicbox}
"CSG" - "Common-mode rejection, Speed, and Gain"
\end{mnemonicbox}
\end{solutionbox}

\questionmarks{4(b)}{4}{OPAMP નો ઉપયોગ કરી સમિંગ એમ્પ્લીફાયર દોરો અને સમજાવો.}

\begin{solutionbox}
સમિંગ એમ્પ્લિફાયર ઇનપુટ વોલ્ટેજના વેઇટેડ સરવાળાના પ્રમાણમાં આઉટપુટ ઉત્પન્ન કરે છે.

\textbf{સર્કિટ ડાયાગ્રામ:}

\begin{center}
\begin{circuitikz}[american, scale=1.0, transform shape]
    \draw (0,0) node[op amp] (opamp) {};
    \draw (opamp.+) -- (-1, -0.5) node[ground] {};
    
    % Summing Inputs
    \draw (opamp.-) -- (-1, 0.5) -- (-1, 2) to[R, l=$R_f$] (1, 2) -| (opamp.out);
    
    \draw (-1, 0.5) -- (-2, 0.5);
    \draw (-2, 0.5) to[R, l=$R_1$] (-4, 0.5) node[left] {$V_1$};
    \draw (-2, 0.5) -- (-2, 0) to[R, l=$R_2$] (-4, 0) node[left] {$V_2$};
    \draw (-2, 0) -- (-2, -0.5) to[R, l=$R_3$] (-4, -0.5) node[left] {$V_3$};
    
    \draw (opamp.out) to[short, -o] (2,0) node[right] {$V_{out}$};
\end{circuitikz}
\end{center}

\begin{itemize}
    \item \textbf{આઉટપુટ ફોર્મ્યુલા}: $V_{out} = -R_f \left( \frac{V_1}{R_1} + \frac{V_2}{R_2} + \frac{V_3}{R_3} \right)$
    \item \textbf{એવરેજિંગ}: જો $R_1=R_2=R_3=R$ અને $R_f=R/3$, તો આઉટપુટ નેગેટિવ એવરેજ છે
    \item \textbf{એપ્લિકેશન્સ}: ઓડિયો મિક્સર, એનાલોગ એડિશન
\end{itemize}

\begin{mnemonicbox}
"SUM" - "Several Unified Multipliers"
\end{mnemonicbox}
\end{solutionbox}

\questionmarks{4(c)}{7}{IC 555 નો પિન ડાયાગ્રામ દોરો અને વેવફોર્મ સાથે IC555 નો ઉપયોગ કરીને મોનોસ્ટેબલ મલ્ટિવાઇબ્રેટર સમજાવો.}

\begin{solutionbox}
IC 555 ટાઇમર મોનોસ્ટેબલ મોડમાં ટ્રિગર થાય ત્યારે ફિક્સ્ડ સમયગાળા ($T$) નો સિંગલ પલ્સ ઉત્પન્ન કરે છે.

\textbf{પિન ડાયાગ્રામ:}

\begin{center}
\begin{tikzpicture}
    \draw (0,0) rectangle (3,4);
    \node at (1.5, 2) {555};
    \foreach \y/\t in {3.5/8 (VCC), 2.5/7 (DIS), 1.5/6 (THR), 0.5/5 (CV)}
        \draw (3,\y) -- (3.5,\y) node[right] {\t};
    \foreach \y/\t in {3.5/1 (GND), 2.5/2 (TR), 1.5/3 (OUT), 0.5/4 (RST)}
        \draw (0,\y) -- (-0.5,\y) node[left] {\t};
\end{tikzpicture}
\end{center}

\textbf{મોનોસ્ટેબલ સર્કિટ અને વેવફોર્મ્સ:}

\begin{center}
\begin{circuitikz}[american, scale=0.8, transform shape]
    % 555 Block inside
    \draw (0,0) rectangle (4,4);
    \node at (2,3.5) {IC 555};
    
    % External R-C
    \draw (2,4) -- (2,5) node[vcc] {$V_{CC}$};
    \draw (0,3) node[left] {Discharge (7)} -- (-1,3) to[R, l=$R$] (-1,5) -- (2,5);
    \draw (0,2) node[left] {Threshold (6)} -- (-1,2) -- (-1,3); 
    \draw (-1,2) to[C, l=$C$] (-1,0) node[ground] {};
    
    % Trigger
    \draw (0,1) node[left] {Trigger (2)} -- (-2,1) node[left] {Trig Input};
    
    % Output
    \draw (4,2) node[right] {Output (3)} -- (5,2) node[right] {$V_{out}$};
    
    % Ground
    \draw (2,0) -- (2,-0.5) node[ground] {};
    
    % Waveforms
    \begin{scope}[shift={(6,0)}]
        \draw[->] (0,0) -- (4,0) node[right] {$t$};
        \draw[->] (0,2.5) -- (0,3.5) node[above] {$V$};
        
        % Trigger
        \draw[blue] (0,3) -- (0.5,3) -- (0.5,1) -- (1,1) -- (1,3) -- (4,3);
        \node[blue, right] at (4,3) {Trigger};
        
        % Capacitor
        \draw[green!60!black] (0,0) -- (1,0) to[out=45, in=210] (3,2) -- (3,0) -- (4,0);
        \node[green!60!black, right] at (4,1) {$V_C$};
        
        % Output
        \draw[red] (0,0) -- (1,0) -- (1,2) -- (3,2) -- (3,0) -- (4,0);
        \node[red, right] at (4,2) {$V_{out}$};
        
        \draw[<->] (1,-0.5) -- (3,-0.5) node[midway, below] {$T = 1.1 RC$};
    \end{scope}
\end{circuitikz}
\end{center}

\begin{itemize}
    \item \textbf{ઓપરેશન}: ટ્રિગર (પિન 2) લો પલ્સ ટાઇમિંગ સાયકલ શરૂ કરે છે. આઉટપુટ હાઈ થાય છે. કેપેસિટર R મારફતે ચાર્જ થાય છે.
    \item \textbf{સાયકલનો અંત}: જ્યારે $V_C = 2/3 V_{CC}$ થાય, ત્યારે આઉટપુટ લો થાય છે, કેપેસિટર ડિસ્ચાર્જ થાય છે.
    \item \textbf{પલ્સ સમય}: $T = 1.1 \times R \times C$ સેકંડ.
\end{itemize}

\begin{mnemonicbox}
"TIMER" - "Triggered Input Makes Extended Response"
\end{mnemonicbox}
\end{solutionbox}
\questionmarks{5(a)}{3}{SMPS નો બ્લોક ડાયાગ્રામ દોરો અને તેની એપ્લીકેશન લખો.}

\begin{solutionbox}
સ્વિચ મોડ પાવર સપ્લાય (SMPS) કાર્યક્ષમ પાવર રૂપાંતરણ માટે સ્વિચિંગ એલિમેન્ટ્સનો ઉપયોગ કરે છે.

\textbf{બ્લોક ડાયાગ્રામ:}

\begin{center}
\begin{tikzpicture}[gtu block]
    \node [draw, rectangle] (ac) {AC ઇનપુટ};
    \node [draw, rectangle, right=0.5cm of ac] (rect1) {રેક્ટિફાયર \& ફિલ્ટર};
    \node [draw, rectangle, below=1cm of rect1] (switch) {સ્વિચિંગ સર્કિટ};
    \node [draw, rectangle, right=0.5cm of switch] (trans) {ટ્રાન્સફોર્મર};
    \node [draw, rectangle, right=0.5cm of trans] (rect2) {રેક્ટિફાયર \& ફિલ્ટર};
    \node [draw, rectangle, above=1cm of rect2] (out) {આઉટપુટ};
    
    \draw [gtu arrow] (ac) -- (rect1);
    \draw [gtu arrow] (rect1) -- (switch);
    \draw [gtu arrow] (switch) -- (trans);
    \draw [gtu arrow] (trans) -- (rect2);
    \draw [gtu arrow] (rect2) -- (out);
    
    % Feedback
    \node [draw, rectangle, below=1cm of switch] (pwm) {PWM કંટ્રોલ};
    \draw [gtu arrow] (out.east) -- ++(0.5,0) |- (pwm.east);
    \draw [gtu arrow] (pwm) -- (switch);
\end{tikzpicture}
\end{center}

\textbf{એપ્લિકેશન્સ:}
\begin{itemize}
    \item કોમ્પ્યુટર પાવર સપ્લાય (ATX)
    \item મોબાઇલ ફોન ચાર્જર
    \item TV પાવર સપ્લાય
    \item LED લાઇટિંગ ડ્રાઇવર્સ
\end{itemize}

\begin{mnemonicbox}
"SAFE" - "Switching Achieves Filtered Energy"
\end{mnemonicbox}
\end{solutionbox}

\questionmarks{5(b)}{4}{ડાયાગ્રામ સાથે રેગ્યુલેટેડ પાવર સ્પ્લાયનું કાર્ય સમજાવો.}

\begin{solutionbox}
રેગ્યુલેટેડ પાવર સપ્લાય ઇનપુટ અથવા લોડમાં ફેરફાર થવા છતાં સ્થિર આઉટપુટ જાળવે છે.

\textbf{બ્લોક ડાયાગ્રામ:}

\begin{center}
\begin{tikzpicture}[gtu block]
    \node [draw, rectangle] (ac) {AC ઇનપુટ};
    \node [draw, rectangle, right=0.5cm of ac] (trans) {ટ્રાન્સફોર્મર};
    \node [draw, rectangle, right=0.5cm of trans] (rect) {રેક્ટિફાયર};
    \node [draw, rectangle, below=1cm of rect] (filt) {ફિલ્ટર};
    \node [draw, rectangle, left=0.5cm of filt] (reg) {રેગ્યુલેટર};
    \node [draw, rectangle, left=0.5cm of reg] (load) {લોડ};
    
    \draw [gtu arrow] (ac) -- (trans);
    \draw [gtu arrow] (trans) -- (rect);
    \draw [gtu arrow] (rect) -- (filt);
    \draw [gtu arrow] (filt) -- (reg);
    \draw [gtu arrow] (reg) -- (load);
\end{tikzpicture}
\end{center}

\begin{itemize}
    \item \textbf{ટ્રાન્સફોર્મર}: AC વોલ્ટેજને જરૂરી લેવલ સુધી ઘટાડે છે
    \item \textbf{રેક્ટિફાયર}: AC ને પલ્સેટિંગ DC માં રૂપાંતરિત કરે છે (ડાયોડ બ્રિજ)
    \item \textbf{ફિલ્ટર}: કેપેસિટર્સ સાથે DC રિપલને સ્મૂથ કરે છે
    \item \textbf{રેગ્યુલેટર}: સ્થિર DC વોલ્ટેજ જાળવે છે (દા.ત., 7805, LM317)
\end{itemize}

\begin{mnemonicbox}
"TRFRO" - "Transform, Rectify, Filter, Regulate, Output"
\end{mnemonicbox}
\end{solutionbox}

\questionmarks{5(c)}{7}{OP-AMP નો મૂળભૂત બ્લોક ડાયાગ્રામ દોરી સમજાવો.}

\begin{solutionbox}
ઓપરેશનલ એમ્પ્લિફાયર ચોક્કસ કાર્યો કરતા ચાર કેસ્કેડેડ તબક્કાઓ ધરાવે છે.

\textbf{બ્લોક ડાયાગ્રામ:}

\begin{center}
\begin{tikzpicture}[gtu block]
    \node [draw, rectangle, align=center] (in) {ડિફરેન્શિયલ\\ઇનપુટ સ્ટેજ};
    \node [draw, rectangle, align=center, right=0.5cm of in] (mid) {ઇન્ટરમીડિયેટ\\સ્ટેજ};
    \node [draw, rectangle, align=center, right=0.5cm of mid] (level) {લેવલ\\શિફ્ટર};
    \node [draw, rectangle, align=center, right=0.5cm of level] (out) {આઉટપુટ\\સ્ટેજ};
    
    \draw [gtu arrow] (in) -- (mid);
    \draw [gtu arrow] (mid) -- (level);
    \draw [gtu arrow] (level) -- (out);
    
    \node [below=0.5cm of mid] (const) {કોન્સ્ટન્ટ કરંટ સોર્સ};
    \draw [dashed] (const) -- (in);
    \draw [dashed] (const) -- (mid);
\end{tikzpicture}
\end{center}

\begin{itemize}
    \item \textbf{ડિફરેન્શિયલ ઇનપુટ સ્ટેજ}: ઉચ્ચ ઇનપુટ ઇમ્પીડન્સ અને CMRR પ્રદાન કરે છે (ડ્યુઅલ ઇનપુટ બેલેન્સ્ડ આઉટપુટ)
    \item \textbf{ઇન્ટરમીડિયેટ સ્ટેજ}: ઉચ્ચ વોલ્ટેજ ગેઇન પ્રદાન કરે છે (ડ્યુઅલ ઇનપુટ અનબેલેન્સ્ડ આઉટપુટ)
    \item \textbf{લેવલ શિફ્ટર}: DC લેવલને શૂન્ય વોલ્ટ્સ પર શિફ્ટ કરે છે (એમીટર ફોલોઅર)
    \item \textbf{આઉટપુટ સ્ટેજ}: લો આઉટપુટ ઇમ્પીડન્સ અને કરંટ ડ્રાઇવ આપે છે (પુશ-પુલ એમ્પ્લિફાયર)
\end{itemize}

\begin{mnemonicbox}
"DILO" - "Differential Input, Level shift, Output"
\end{mnemonicbox}
\end{solutionbox}

\vspace{0.5em}\centerline{\textbf{OR}}\questionmarks{5(a)}{3}{LM317 નો ઉપયોગ કરીને એડજસ્ટેબલ વોલ્ટેજ રેગ્યુલેટર ડાયાગ્રામ સાથે સમજાવો.}

\begin{solutionbox}
LM317 એક વેરિયેબલ પોઝિટિવ વોલ્ટેજ રેગ્યુલેટર છે જે 1.25V થી 37V સુધી આઉટપુટ એડજસ્ટ કરે છે.

\textbf{સર્કિટ ડાયાગ્રામ:}

\begin{center}
\begin{circuitikz}[american, scale=1.0, transform shape]
    \draw (0,0) rectangle (2,1.5);
    \node at (1, 0.75) {LM317};
    \draw (-1, 1.25) to[short, o-] (0, 1.25) node[right] {IN};
    \draw (2, 1.25) to[short, -o] (4, 1.25) node[right] {$V_{out}$};
    \draw (2, 0.25) node[left] {ADJ} -- (2.5, 0.25) -- (2.5, -2) node[ground] {};
    
    \draw (2.5, 1.25) to[R, l=$R_1$] (2.5, 0.25);
    \draw (2.5, 0.25) to[vR, l=$R_2$] (2.5, -2);
    
    \draw (-1, 1.25) to[C, l=$C_{in}$] (-1, 0) node[ground] {};
    \draw (4, 1.25) to[C, l=$C_{out}$] (4, 0) node[ground] {};
\end{circuitikz}
\end{center}

\begin{itemize}
    \item \textbf{ફોર્મ્યુલા}: $V_{out} = 1.25V \times (1 + \frac{R_2}{R_1}) + I_{adj}R_2$ જ્યાં $1.25V$ રેફરન્સ વોલ્ટેજ ($V_{ref}$) છે
    \item \textbf{રેસિસ્ટર્સ}: $R_1$ રેફરન્સ કરંટ સેટ કરે છે, $R_2$ આઉટપુટ વોલ્ટેજ એડજસ્ટ કરે છે
    \item \textbf{પ્રોટેક્શન}: આંતરિક થર્મલ ઓવરલોડ અને શોર્ટ સર્કિટ પ્રોટેક્શન
\end{itemize}

\begin{mnemonicbox}
"AVR" - "Adjustable Voltage Regulation"
\end{mnemonicbox}
\end{solutionbox}

\questionmarks{5(b)}{4}{ફિક્સ્ડ વોલ્ટેજ રેગ્યુલેટર IC અને વેરિયેબલ વોલ્ટેજ રેગ્યુલેટર IC વચ્ચેનો તફાવત આપો.}

\begin{solutionbox}
ફિક્સ્ડ આઉટપુટ રેગ્યુલેટર્સ (જેમ કે 78XX) અને એડજસ્ટેબલ રેગ્યુલેટર્સ (જેમ કે LM317) વચ્ચેની સરખામણી.

\begin{center}
\captionof{table}{ફિક્સ્ડ vs વેરિયેબલ રેગ્યુલેટર્સ}
\begin{tabulary}{\textwidth}{|L|L|L|}
\hline
\textbf{પેરામીટર} & \textbf{ફિક્સ્ડ રેગ્યુલેટર} & \textbf{વેરિયેબલ રેગ્યુલેટર} \\
\hline
આઉટપુટ વોલ્ટેજ & પ્રીસેટ (દા.ત., 5V, 12V) & એડજસ્ટેબલ રેન્જ (દા.ત., 1.2V-37V) \\
\hline
કોમ્પોનેન્ટ્સ & ઓછા (2 કેપેસિટર્સ) & વધુ (રેસિસ્ટર્સ + કેપેસિટર્સ) \\
\hline
સુગમતા (Flexibility) & ઓછી (નિશ્ચિત એપ્લિકેશન) & ઉચ્ચ (સાર્વત્રિક એપ્લિકેશન) \\
\hline
ઉદાહરણો & 7805, 7812, 7905 & LM317, LM337, LM723 \\
\hline
કિંમત & સામાન્ય રીતે સસ્તા & થોડી વધારે \\
\hline
પિન કોન્ફિગ & In, Ground, Out & In, Adjust, Out \\
\hline
\end{tabulary}
\end{center}

\begin{mnemonicbox}
"FOCUS" - "Fixed Output Compared to User-Set"
\end{mnemonicbox}
\end{solutionbox}

\questionmarks{5(c)}{7}{OP-AMP ની એપ્લીકેશનની યાદી બનાવો. OP-AMP નો ઉપયોગ કરીને D to A કન્વર્ટરનું કાર્ય સર્કિટ ડાયાગ્રામ સાથ સમજાવો.}

\begin{solutionbox}
DAC ડિજિટલ બાઈનરી ઇનપુટનું સમકક્ષ એનાલોગ વોલ્ટેજમાં રૂપાંતર કરે છે.

\textbf{OP-AMP ની એપ્લિકેશન્સ:}
\begin{enumerate}
    \item એમ્પ્લિફાયર (ઇન્વર્ટિંગ, નોન-ઇન્વર્ટિંગ, ડિફરેન્શિયલ)
    \item ફિલ્ટર્સ (એક્ટિવ લો પાસ, હાઈ પાસ)
    \item ઓસિલેટર (વેઈન બ્રિજ, ફેઝ શિફ્ટ)
    \item કમ્પેરેટર અને શ્મિટ ટ્રિગર
    \item ગાણિતિક ક્રિયાઓ (સમિંગ, ઇન્ટિગ્રેટર, ડિફરન્શિએટર)
\end{enumerate}

\textbf{R-2R Ladder DAC સર્કિટ:}

\begin{center}
\begin{circuitikz}[american, scale=0.9, transform shape]
    \draw (0,0) node[op amp] (opamp) {};
    \draw (opamp.+) -- (-1, -0.5) node[ground] {};
    \draw (opamp.-) -- (-1, 0.5) -- (-1, 2) to[R, l=$3R$] (1, 2) -| (opamp.out);
    \draw (opamp.out) to[short, -o] (2,0) node[right] {$V_{out}$};

    % Ladder R-2R (4-bit)
    \draw (-1, 0.5) -- (-2, 0.5);
    \draw (-2, 0.5) to[R, l=$2R$] (-2, -1.5) node[below] {$b_3 (MSB)$};
    
    \draw (-2, 0.5) to[R, l=$R$] (-4, 0.5);
    \draw (-4, 0.5) to[R, l=$2R$] (-4, -1.5) node[below] {$b_2$};
    
    \draw (-4, 0.5) to[R, l=$R$] (-6, 0.5);
    \draw (-6, 0.5) to[R, l=$2R$] (-6, -1.5) node[below] {$b_1$};
    
    \draw (-6, 0.5) to[R, l=$R$] (-8, 0.5);
    \draw (-8, 0.5) to[R, l=$2R$] (-8, -1.5) node[below] {$b_0 (LSB)$};
    \draw (-8, 0.5) to[R, l=$2R$] (-10, 0.5) node[ground] {}; % Termination
    
\end{circuitikz}
\end{center}

\begin{itemize}
    \item \textbf{સિદ્ધાંત}: લેડર નેટવર્ક બાઈનરી વેઇટેડ કરંટ બનાવે છે
    \item \textbf{આઉટપુટ}: $V_{out} \propto -(b_3 2^{-1} + b_2 2^{-2} + b_1 2^{-3} + b_0 2^{-4}) V_{ref}$
    \item \textbf{ફાયદા}: માત્ર બે રેસિસ્ટર મૂલ્યો ($R, 2R$) ની જરૂર છે, સ્કેલેબલ છે
\end{itemize}

\begin{mnemonicbox}
"DART" - "Digital to Analog Resistor Translation"
\end{mnemonicbox}


\end{solutionbox}
\end{document}
