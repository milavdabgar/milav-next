\documentclass[10pt,a4paper]{article}

% content/resources/templates/preamble.tex
\usepackage[margin=0.6in]{geometry}
\author{Milav Dabgar}
\usepackage{amsmath,amssymb,amsthm}
\usepackage{booktabs}
\usepackage{multirow}
\usepackage{xcolor}
\usepackage{tcolorbox}
\tcbuselibrary{breakable,skins}
\usepackage[colorlinks=true,linkcolor=blue]{hyperref}
\usepackage{titlesec}
\usepackage{enumitem}
\usepackage{tikz}
\usepackage{pgfplots}
\usepackage{circuitikz}
\usepackage[version=4]{mhchem}
\usepackage{longtable}
\usepackage{array}
\usepackage{float}
\usepackage{caption}
\usepackage{listings}

\lstset{
  basicstyle=\small\ttfamily,
  breaklines=true,
  breakatwhitespace=false,
  postbreak=\mbox{\textcolor{red}{$\hookrightarrow$}\space},
  float=false,
  numbers=left,
  numberstyle=\tiny\color{gray},
  numbersep=10pt,
  xleftmargin=2em,
  keywordstyle=\color{blue},
  commentstyle=\color{green!60!black},
  stringstyle=\color{purple},
  backgroundcolor=\color{gray!5},
  showstringspaces=false,
  tabsize=2,
  captionpos=b,
  keepspaces=true,
  columns=flexible
}

\pgfplotsset{compat=1.18}
\usetikzlibrary{shapes,arrows,positioning,calc,patterns,decorations.pathmorphing,decorations.markings,arrows.meta}

% Color scheme
\definecolor{headcolor}{RGB}{0,102,204}
\definecolor{keycolor}{RGB}{220,20,60}
\definecolor{solutioncolor}{RGB}{34,139,34}
\definecolor{mnemoniccolor}{RGB}{148,0,211}
\definecolor{codecolor}{RGB}{0,0,100}

% Spacing
\setlength{\parskip}{3pt}
\setlist[itemize]{nosep}
\setlist[enumerate]{nosep}

% Title formatting
\titleformat{\section}{\Large\bfseries\color{headcolor}}{\thesection}{1em}{}
\titleformat{\subsection}{\large\bfseries\color{headcolor}}{\thesubsection}{1em}{}

% Pandoc tightlist compatibility
\providecommand{\tightlist}{%
  \setlength{\itemsep}{0pt}\setlength{\parskip}{0pt}}

% Pandoc longtable compatibility
\newcounter{none}
\def\thenone{}


% content/resources/templates/gujarati-boxes.tex
\usepackage{fontspec}
\usepackage{polyglossia}

% Set Gujarati as main language (document is primarily in Gujarati)
% Note: gloss-gujarati.ldf doesn't exist in polyglossia, but it will use hyphenation patterns
\setdefaultlanguage{gujarati}
\setotherlanguage{english}

% Configure Gujarati font properly
% Use Language=Default to prevent polyglossia from trying to add language-specific features
% that don't exist for Gujarati, which causes "empty feature" warnings
\newfontfamily\gujaratifont[Script=Gujarati,AutoFakeBold=2.5,AutoFakeSlant=0.3]{Noto Sans Gujarati}
\setmainfont[Script=Gujarati,AutoFakeBold=2.5,AutoFakeSlant=0.3]{Noto Sans Gujarati}
% Use Noto Sans Gujarati for monospace to support Gujarati in text
\setmonofont[Scale=0.9]{Noto Sans Gujarati}

% Configure English to use the same font
\newfontfamily\englishfont[Script=Gujarati,AutoFakeBold=2.5,AutoFakeSlant=0.3]{Noto Sans Gujarati}

% Translations for polyglossia
\gappto\captionsgujarati{
  \renewcommand{\tablename}{કોષ્ટક}
  \renewcommand{\figurename}{આકૃતિ}
}

% Helper for TikZ nodes to ensure Gujarati font
\newcommand{\gu}[1]{{\gujaratifont #1}}

% Custom environments
\newtcolorbox{solutionbox}{
    breakable,
    enhanced,
    colback=solutioncolor!5!white,
    colframe=solutioncolor!75!black,
    fonttitle=\bfseries,
    title=જવાબ
}

\newtcolorbox{solutionboxnobreak}{
 colback=solutioncolor!5!white,
 colframe=solutioncolor!75!black,
 fonttitle=\bfseries,
 title=જવાબ
}

\newtcolorbox{keyformula}{
 breakable,
 enhanced,
 colback=keycolor!5!white,
 colframe=keycolor!75!black,
 fonttitle=\bfseries,
 title=રાસાયણિક સમીકરણ/સૂત્ર
}

\newtcolorbox{mnemonicbox}{
 breakable,
 enhanced,
 colback=mnemoniccolor!5!white,
 colframe=mnemoniccolor!75!black,
 fonttitle=\bfseries,
 title=મેમરી ટ્રીક
}


\begin{document}

\begin{center}
{\Huge\bfseries\color{headcolor} Subject Name (Gujarati)}\\[5pt]
{\LARGE 1323202 -- Summer 2023}\\[3pt]
{\large Semester 1 Study Material}\\[3pt]
{\normalsize\textit{Detailed Solutions and Explanations}}
\end{center}

\vspace{10pt}

\subsection*{પ્રશ્ન 1(અ) [3
ગુણ]}\label{uxaaauxab0uxab6uxaa8-1uxa85-3-uxa97uxaa3}

\textbf{સંજ્ઞા દોરો(૧)એસ.સી.આર(૨)ડાયેક(૩)ટ્રાયેક}

\begin{solutionbox}

\textbf{આકૃતિ:}

\begin{verbatim}
SCR સંજ્ઞા:         DIAC સંજ્ઞા:         TRIAC સંજ્ઞા:
   A                   A1                  MT2
   |                    |                   |
   ▼                    ▼                   ▼
  ┌─┐                  ┌─┐                 ┌─┐
  │ │                  │ │                 │ │
──┤ ├──              ──┤ ├──             ──┤ ├──
  │ │                  │ │                 │ │
  └─┘                  └─┘                 └─┘
   ▲                    ▲                   ▲
   |                    |                   |
   K                   A2                  MT1
  /                                        /
 /                                        /
G                                        G
\end{verbatim}

\begin{itemize}
\tightlist
\item
  \textbf{SCR (સિલિકોન કંટ્રોલ્ડ રેક્ટિફાયર)}: ત્રણ-ટર્મિનલવાળું ઉપકરણ - એનોડ,
  કેથોડ અને ગેટ
\item
  \textbf{DIAC (ડાયોડ AC સ્વિચ)}: બે-ટર્મિનલવાળું બાયડાયરેક્શનલ ઉપકરણ - A1 અને
  A2
\item
  \textbf{TRIAC (ટ્રાયોડ AC સ્વિચ)}: ત્રણ-ટર્મિનલવાળું બાયડાયરેક્શનલ ઉપકરણ -
  MT1, MT2 અને ગેટ
\end{itemize}

\end{solutionbox}
\begin{mnemonicbox}
``AGK for SCR, AA for DIAC, MMG for TRIAC''

\end{mnemonicbox}
\subsection*{પ્રશ્ન 1(બ) [4
ગુણ]}\label{uxaaauxab0uxab6uxaa8-1uxaac-4-uxa97uxaa3}

\textbf{પદો સમજાવો(૧)સી.એમ.આર.આર.(૨)સ્લૂરેટ્.}

\begin{solutionbox}


{\def\LTcaptype{none} % do not increment counter
\vspace{-5pt}
\captionof{table}{ઓપ-એમ્પ પેરામીટર્સ}
\vspace{-10pt}
\begin{longtable}[]{@{}
  >{\raggedright\arraybackslash}p{(\linewidth - 4\tabcolsep) * \real{0.2973}}
  >{\raggedright\arraybackslash}p{(\linewidth - 4\tabcolsep) * \real{0.3243}}
  >{\raggedright\arraybackslash}p{(\linewidth - 4\tabcolsep) * \real{0.3784}}@{}}
\toprule\noalign{}
\begin{minipage}[b]{\linewidth}\raggedright
પેરામીટર
\end{minipage} & \begin{minipage}[b]{\linewidth}\raggedright
વ્યાખ્યા
\end{minipage} & \begin{minipage}[b]{\linewidth}\raggedright
મહત્વ
\end{minipage} \\
\midrule\noalign{}
\endhead
\bottomrule\noalign{}
\endlastfoot
\textbf{CMRR (કોમન મોડ રિજેક્શન રેશિયો)} & ડિફરેન્શિઅલ ગેઈન અને કોમન મોડ ગેઈનનો
ગુણોત્તર dB માં & ઊંચો CMRR એટલે કોમન ઇનપુટ સિગ્નલ્સનો વધુ સારો રિજેક્શન \\
\textbf{Slew Rate (સ્લૂ રેટ)} & આઉટપુટ વોલ્ટેજનો મહત્તમ પરિવર્તન દર (V/μs) &
ઓપ-એમ્પ ઝડપથી બદલાતા ઇનપુટ્સને કેવી ઝડપે પ્રતિસાદ આપી શકે છે તે નક્કી કરે છે \\
\end{longtable}
}

\begin{itemize}
\tightlist
\item
  \textbf{CMRR ફોર્મ્યુલા}: CMRR = 20 log_{1}_{0}(Ad/Acm) dB
\item
  \textbf{Slew Rate મહત્વ}: ઊંચી ફ્રીક્વન્સી પરફોર્મન્સને અસર કરે છે અને વિકૃતિ
  અટકાવે છે
\end{itemize}

\end{solutionbox}
\begin{mnemonicbox}
``Common Mode Rejected Rapidly, Slew shows Signal
Speed''

\end{mnemonicbox}
\subsection*{પ્રશ્ન 1(ક) [7
ગુણ]}\label{uxaaauxab0uxab6uxaa8-1uxa95-7-uxa97uxaa3}

\textbf{સમીન્ગ એમ્પલીફાયર દોરો અને સમજાવો.}

\begin{solutionbox}

\textbf{આકૃતિ:}

\begin{center}
\textbf{Mermaid Diagram (Code)}
\begin{verbatim}
{Shaded}
{Highlighting}[]
graph LR
    V1 {-{-} R1 {-}{-}{} A}
    V2 {-{-} R2 {-}{-}{} A}
    V3 {-{-} R3 {-}{-}{} A}
    A {-{-} Rf {-}{-}{} B[Op{-}Amp]}
    B {-{-}{} Vout}
    B {-{-} {}{-} {-}{-}{} A}
    A {-{-} {}+ {-}{-}{} Ground}
{Highlighting}
{Shaded}
\end{verbatim}
\end{center}

\textbf{સમિંગ એમ્પ્લિફાયરની કાર્યપ્રણાલી:}

\begin{itemize}
\item
  \textbf{સર્કિટ કાર્ય}: મલ્ટિપલ ઇનપુટ વોલ્ટેજને સ્કેલિંગ સાથે જોડે છે
\item
  \textbf{આઉટપુટ સમીકરણ}: Vout = -(Rf/R1 \times V1 + Rf/R2 \times V2 + Rf/R3 \times V3)
\item
  \textbf{ઇન્વર્ટિંગ કન્ફિગરેશન}: ઇનપુટ સિગ્નલ્સ 180^\circ ફેઝ શિફ્ટ અનુભવે છે
\item
  \textbf{ગેઈન કંટ્રોલ}: Rf/Rn દરેક ઇનપુટ સિગ્નલનું વજન નક્કી કરે છે
\item
  \textbf{ઉપયોગો}: ઓડિયો મિક્સિંગ, એનાલોગ કમ્પ્યુટેશન, સિગ્નલ પ્રોસેસિંગ
\item
  \textbf{મુખ્ય વિશેષતા}: ઇન્વર્ટિંગ ઇનપુટ પર વર્ચ્યુઅલ ગ્રાઉન્ડ વિશ્લેષણને સરળ બનાવે છે
\end{itemize}

\end{solutionbox}
\begin{mnemonicbox}
``Sum with Weights: Vout = -Rf(V1/R1 + V2/R2 +
V3/R3)''

\end{mnemonicbox}
\subsection*{પ્રશ્ન 1(ક OR) [7
ગુણ]}\label{uxaaauxab0uxab6uxaa8-1uxa95-or-7-uxa97uxaa3}

\textbf{ડીએ કન્વટ્ટર દોરો અને સમજાવો.}

\begin{solutionbox}

\textbf{આકૃતિ:}

\begin{center}
\textbf{Mermaid Diagram (Code)}
\begin{verbatim}
{Shaded}
{Highlighting}[]
graph LR
    D0 {-{-} 2^{0}R {-}{-}{} S1}
    D1 {-{-} 2^{1}R {-}{-}{} S2}
    D2 {-{-} 2^{2}R {-}{-}{} S3}
    D3 {-{-} 2^{3}R {-}{-}{} S4}
    S1 \& S2 \& S3 \& S4 {-{-}{} A[Summing Amp]}
    A {-{-}{} Vout}
{Highlighting}
{Shaded}
\end{verbatim}
\end{center}

\textbf{R-2R લેડર DAC કાર્યપ્રણાલી:}

\begin{itemize}
\item
  \textbf{કાર્ય}: ડિજિટલ બાઇનરી ઇનપુટને એનાલોગ આઉટપુટ વોલ્ટેજમાં રૂપાંતરિત કરે છે
\item
  \textbf{કાર્યસિદ્ધાંત}: વેઇટેડ રેસિસ્ટર નેટવર્ક સ્કેલ્ડ કરંટ બનાવે છે
\item
  \textbf{બાઇનરી વેઇટિંગ}: દરેક બિટ તેના સ્થાન (2^{n}) ના પ્રમાણમાં વોલ્ટેજમાં યોગદાન
  આપે છે
\item
  \textbf{રિઝોલ્યુશન}: બિટ્સની સંખ્યા (N) દ્વારા 1/2ᴺ ફુલ સ્કેલ તરીકે નક્કી થાય છે
\item
  \textbf{ફાયદા}: સરળ ડિઝાઇન, સારી ચોકસાઈ, ઝડપી રૂપાંતરણ
\item
  \textbf{ઉપયોગો}: ઓડિયો ઉપકરણો, સિગ્નલ જનરેશન, કંટ્રોલ સિસ્ટમ્સ
\end{itemize}

\end{solutionbox}
\begin{mnemonicbox}
``Digital Bits to Analog Steps - R-2R makes the
magic''

\end{mnemonicbox}
\subsection*{પ્રશ્ન 2(અ) [3
ગુણ]}\label{uxaaauxab0uxab6uxaa8-2uxa85-3-uxa97uxaa3}

\textbf{ટ્રાન્જીસ્ટર નુ થર્મલ રન અવે વર્ણવો.}

\begin{solutionbox}

\textbf{થર્મલ રનઅવે પ્રક્રિયા:}

\begin{center}
\textbf{Mermaid Diagram (Code)}
\begin{verbatim}
{Shaded}
{Highlighting}[]
graph LR
    A[વધેલું તાપમાન] {-{-}{} B[વધેલો કલેક્ટર કરંટ]}
    B {-{-}{} C[વધુ પાવર ડિસિપેશન]}
    C {-{-}{} A}
{Highlighting}
{Shaded}
\end{verbatim}
\end{center}

\begin{itemize}
\tightlist
\item
  \textbf{વ્યાખ્યા}: સ્વ-ત્વરણની પ્રક્રિયા જ્યાં ટ્રાન્ઝિસ્ટર ગરમ થાય છે અને વધુ કરંટ
  ખેંચે છે
\item
  \textbf{કારણ}: બેઝ-એમિટર વોલ્ટેજનો નેગેટિવ તાપમાન કોએફિશિયન્ટ
\item
  \textbf{નિવારણ}: યોગ્ય હીટ સિંક અને સ્ટેબિલાઈઝેશન સર્કિટનો ઉપયોગ
\end{itemize}

\end{solutionbox}
\begin{mnemonicbox}
``Heat feeds Current feeds Heat - a dangerous loop''

\end{mnemonicbox}
\subsection*{પ્રશ્ન 2(બ) [4
ગુણ]}\label{uxaaauxab0uxab6uxaa8-2uxaac-4-uxa97uxaa3}

\textbf{વૉલ્ટેજ સીરીજ નેગેટીવ ફીડબેક દોરો અને સમજાવો.}

\begin{solutionbox}

\textbf{આકૃતિ:}

\begin{center}
\textbf{Mermaid Diagram (Code)}
\begin{verbatim}
{Shaded}
{Highlighting}[]
graph LR
    Vin {-{-}{} A[Amplifier]}
    A {-{-}{} Vout}
    Vout {-{-} Feedback Network {-}{-}{} B[Subtractor]}
    B {-{-}{} A}
{Highlighting}
{Shaded}
\end{verbatim}
\end{center}

\textbf{વોલ્ટેજ સીરીઝ નેગેટિવ ફીડબેક:}

{\def\LTcaptype{none} % do not increment counter
\begin{longtable}[]{@{}ll@{}}
\toprule\noalign{}
પેરામીટર & નેગેટિવ ફીડબેકની અસર \\
\midrule\noalign{}
\endhead
\bottomrule\noalign{}
\endlastfoot
\textbf{ગેઈન સ્ટેબિલિટી} & સુધારો, એમ્પ્લિફાયર પેરામીટર્સ પર ઓછો આધાર \\
\textbf{બેન્ડવિડ્થ} & ફીડબેક ફેક્ટરના પ્રમાણમાં વધારો \\
\textbf{ડિસ્ટોર્શન} & નોંધપાત્ર રીતે ઘટાડો \\
\textbf{ઇનપુટ ઇમ્પેડન્સ} & વધારો \\
\end{longtable}
}

\begin{itemize}
\tightlist
\item
  \textbf{કાર્યસિદ્ધાંત}: આઉટપુટ વોલ્ટેજ સેમ્પલ કરીને ઇનપુટમાં પાછો ફીડ કરવામાં આવે છે
\item
  \textbf{ગેઈન ફોર્મ્યુલા}: ક્લોઝ્ડ-લૂપ ગેઈન = ઓપન-લૂપ ગેઈન/(1 + βA)
\end{itemize}

\end{solutionbox}
\begin{mnemonicbox}
``Series says Sample Voltage, Stabilize Gain''

\end{mnemonicbox}
\subsection*{પ્રશ્ન 2(ક) [7
ગુણ]}\label{uxaaauxab0uxab6uxaa8-2uxa95-7-uxa97uxaa3}

\textbf{કોમન એમીટર એમ્પલીફાયર માટે ડીસી લોડ લાઈન દોરો અને સમજાવો.}

\begin{solutionbox}

\textbf{આકૃતિ:}

\begin{center}
\textbf{Mermaid Diagram (Code)}
\begin{verbatim}
{Shaded}
{Highlighting}[]
graph TD
    subgraph DC Load Line
    direction LR
    A[Point A: IC=0, VCE=VCC] {-{-}{} B[Operating Point Q]}
    B {-{-}{} C[Point B: IC=VCC/RC, VCE=0]}
    end
{Highlighting}
{Shaded}
\end{verbatim}
\end{center}

\textbf{DC લોડ લાઈનની વિશેષતાઓ:}

\begin{itemize}
\tightlist
\item
  \textbf{વ્યાખ્યા}: બધા સંભવિત ઓપરેટિંગ પોઇન્ટ્સની ગ્રાફિકલ રજૂઆત
\item
  \textbf{સમીકરણ}: IC = VCC/RC - VCE/RC
\item
  \textbf{ચાવીરૂપ બિંદુઓ}:

  \begin{itemize}
  \tightlist
  \item
    સેચ્યુરેશન પોઇન્ટ (VCE \approx 0V, IC = VCC/RC)
  \item
    કટ-ઓફ પોઇન્ટ (IC \approx 0mA, VCE = VCC)
  \item
    Q-પોઇન્ટ (એમ્પ્લિફિકેશન માટે પસંદ કરેલ ઓપરેટિંગ પોઇન્ટ)
  \end{itemize}
\item
  \textbf{મહત્વ}: બાયસિંગ સ્ટેબિલિટી અને આઉટપુટ સિગ્નલની મર્યાદા નક્કી કરે છે
\item
  \textbf{સંબંધ}: DC લોડ લાઈન સર્કિટ કોમ્પોનન્ટ્સ (VCC અને RC) દ્વારા નિશ્ચિત થાય
  છે
\end{itemize}

\end{solutionbox}
\begin{mnemonicbox}
``Connect Cutoff to Saturation for DC Load Line''

\end{mnemonicbox}
\subsection*{પ્રશ્ન 2(અ OR) [3
ગુણ]}\label{uxaaauxab0uxab6uxaa8-2uxa85-or-3-uxa97uxaa3}

\textbf{ટ્રાન્જીસ્ટર મા ઓપરેટીન્ગ પોઈન્ટ(ક્યુ પોઈન્ટ) સમજાવો.}

\begin{solutionbox}

\textbf{Q-પોઇન્ટ (ઓપરેટિંગ પોઇન્ટ):}

\begin{verbatim}
      |
  Ic  |      DC Load Line
      |          /
      |         /
      |        /
      |       * Q-Point
      |      /
      |     /
      |    /
      |___/____________
          Vce
\end{verbatim}

\begin{itemize}
\tightlist
\item
  \textbf{વ્યાખ્યા}: એકટિવ રીજનમાં ટ્રાન્ઝિસ્ટર ઓપરેટ કરે તે માટેનો ચોક્કસ DC બાયસ
  પોઇન્ટ
\item
  \textbf{મહત્વ}: વિકૃતિ વિના આઉટપુટ સિગ્નલની રેન્જ નક્કી કરે છે
\item
  \textbf{પસંદગીના માપદંડ}: મહત્તમ સ્વિંગ માટે લોડ લાઈનનું મધ્ય બિંદુ
\end{itemize}

\end{solutionbox}
\begin{mnemonicbox}
``Quality amplification needs Quiet bias at
Q-point''

\end{mnemonicbox}
\subsection*{પ્રશ્ન 2(બ OR) [4
ગુણ]}\label{uxaaauxab0uxab6uxaa8-2uxaac-or-4-uxa97uxaa3}

\textbf{હાટટલે ઓસ્સીલેટર દોરો અને સમજાવો.}

\begin{solutionbox}

\textbf{આકૃતિ:}

\begin{center}
\textbf{Mermaid Diagram (Code)}
\begin{verbatim}
{Shaded}
{Highlighting}[]
graph LR
    A[Transistor] {-{-} Feedback {-}{-}{} B[LC Tank Circuit]}
    B {-{-}{} A}
    B {-{-} L1, L2, C {-}{-}{} Output}
{Highlighting}
{Shaded}
\end{verbatim}
\end{center}

\textbf{હાર્ટલે ઓસિલેટર:}

\begin{itemize}
\tightlist
\item
  \textbf{કન્ફિગરેશન}: ટેપ્ડ ઇન્ડક્ટર ફીડબેક સાથે કોમન એમિટર
\item
  \textbf{ફ્રીક્વન્સી ફોર્મ્યુલા}: f = 1/[2π\sqrt(C\times(L1+L2))]
\item
  \textbf{ફેઝ શિફ્ટ}: ઓસિલેશન માટે 360^\circ કુલ ફેઝ શિફ્ટની ખાતરી કરે છે
\item
  \textbf{ફીડબેક}: ઇન્ડક્ટિવ વોલ્ટેજ ડિવાઇડર પોઝિટિવ ફીડબેક પ્રદાન કરે છે
\end{itemize}

\end{solutionbox}
\begin{mnemonicbox}
``Hartley Has two coils with inductance for LC
oscillation''

\end{mnemonicbox}
\subsection*{પ્રશ્ન 2(ક OR) [7
ગુણ]}\label{uxaaauxab0uxab6uxaa8-2uxa95-or-7-uxa97uxaa3}

\textbf{કોમન એમીટર એમ્પલીફાયર માટે એસી લોડ લાઈન દોરો અને સમજાવો.}

\begin{solutionbox}

\textbf{આકૃતિ:}

\begin{center}
\textbf{Mermaid Diagram (Code)}
\begin{verbatim}
{Shaded}
{Highlighting}[]
graph TD
    subgraph AC and DC Load Lines
    direction LR
    A[DC Load Line] {-{-}{} B[Q{-}Point]}
    B {-{-}{} C[AC Load Line {-} Steeper]}
    end
{Highlighting}
{Shaded}
\end{verbatim}
\end{center}

\textbf{AC લોડ લાઈનની વિશેષતાઓ:}

\begin{itemize}
\tightlist
\item
  \textbf{વ્યાખ્યા}: સિગ્નલ એમ્પ્લિફિકેશન દરમિયાન ડાયનેમિક ઓપરેશનનું પ્રતિનિધિત્વ કરે
  છે
\item
  \textbf{સમીકરણ}: ic = (VCC-VCEQ)/R'c - vce/R'c જ્યાં R'c =
  RC\textbar\textbar RL
\item
  \textbf{DC લોડ લાઈન સાથે તુલના}:

  \begin{itemize}
  \tightlist
  \item
    AC લોડ લાઈન DC લોડ લાઈન કરતા વધુ તીવ્ર ઢાળવાળી હોય છે
  \item
    Q-પોઇન્ટ પરથી પસાર થાય છે
  \item
    વોલ્ટેજ અને કરંટ સિગ્નલ સ્વિંગ નક્કી કરે છે
  \end{itemize}
\item
  \textbf{મહત્વ}: વિકૃતિ વગરનો મહત્તમ આઉટપુટ સિગ્નલ વ્યાખ્યાયિત કરે છે
\item
  \textbf{મર્યાદા પરિબળ}: સેચ્યુરેશન અને કટ-ઓફ ક્ષેત્રોને ટાળવું
\end{itemize}

\end{solutionbox}
\begin{mnemonicbox}
``AC Amplitude Controlled by Load line Angle''

\end{mnemonicbox}
\subsection*{પ્રશ્ન 3(અ) [3
ગુણ]}\label{uxaaauxab0uxab6uxaa8-3uxa85-3-uxa97uxaa3}

\textbf{ફીક્સડ બાયાસ સર્કટટ દોરો અને તેનું કાયટ સમજાવો.}

\begin{solutionbox}

\textbf{આકૃતિ:}

\begin{verbatim}
      Vcc
       |
       R
       |
       |C
       |----Output
       |
      /|
     / |
    /--|
   /   |
  |    |
  B    E
  |    |
  Rb   |
  |    |
  |____|
  |
  Vin
\end{verbatim}

\begin{itemize}
\tightlist
\item
  \textbf{સ્ટ્રક્ચર}: VCC સાથે જોડાયેલ બેઝ રેઝિસ્ટર, લોડ માટે કલેક્ટર રેઝિસ્ટર
\item
  \textbf{ઓપરેશન}: ફિક્સ્ડ બેઝ કરંટ ટ્રાન્ઝિસ્ટરને બાયસ કરે છે
\item
  \textbf{ગેરફાયદો}: તાપમાન પરિવર્તન સામે નબળી સ્થિરતા
\end{itemize}

\end{solutionbox}
\begin{mnemonicbox}
``Fixed Bias Feeds Base from power supply''

\end{mnemonicbox}
\subsection*{પ્રશ્ન 3(બ) [4
ગુણ]}\label{uxaaauxab0uxab6uxaa8-3uxaac-4-uxa97uxaa3}

\textbf{હાટલે ઓસ્સીલેટરમા L1=5mH, L2=10mH, C=0.01µF. ઓસ્સીલેશન ની ફ્રીક્વન્સીની
ગણતરી કરો.}

\begin{solutionbox}

\textbf{ઉકેલ:}

\begin{itemize}
\tightlist
\item
  \textbf{આપેલું}: L1=5mH, L2=10mH, C=0.01µF
\item
  \textbf{ફ્રીક્વન્સી ફોર્મ્યુલા}: f = 1/[2π\sqrt(C\times(L1+L2))]
\item
  \textbf{ગણતરી}:

  \begin{itemize}
  \tightlist
  \item
    કુલ ઈન્ડક્ટન્સ LT = L1 + L2 = 5mH + 10mH = 15mH = 15\times10^{-}^{3} H
  \item
    C = 0.01µF = 1\times10^{-}^{8} F
  \item
    f = 1/[2π\sqrt(15\times10^{-}^{3} \times 1\times10^{-}^{8})]
  \item
    f = 1/[2π\sqrt(15\times10^{-}^{1}^{1})]
  \item
    f = 1/[2π\times3.873\times10^{-}^{6}]
  \item
    f = 1/[24.33\times10^{-}^{6}]
  \item
    f = 41,101 Hz \approx 41.1 kHz
  \end{itemize}
\end{itemize}

\end{solutionbox}
\begin{mnemonicbox}
``For Hartley's frequency, add coils then take
square root''

\end{mnemonicbox}
\subsection*{પ્રશ્ન 3(ક) [7
ગુણ]}\label{uxaaauxab0uxab6uxaa8-3uxa95-7-uxa97uxaa3}

\textbf{બે સ્ટેજ આર.સી. કપલ્ડ એમ્પલીફાયરનો ફ્રીક્વન્સી રીસપોન્સ કવટ દોરો અને
સમજાવો.}

\begin{solutionbox}

\textbf{આકૃતિ:}

\begin{center}
\textbf{Mermaid Diagram (Code)}
\begin{verbatim}
{Shaded}
{Highlighting}[]
graph TD
    subgraph Frequency Response
    direction LR
    A[Low Frequency] {-{-}{} B[Mid Frequency]}
    B {-{-}{} C[High Frequency]}
    end
{Highlighting}
{Shaded}
\end{verbatim}
\end{center}

\textbf{બે-સ્ટેજ RC કપલ્ડ એમ્પ્લિફાયર ફ્રીક્વન્સી રિસ્પોન્સ:}

\begin{itemize}
\tightlist
\item
  \textbf{લો-ફ્રીક્વન્સી રીજન}: ફ્રીક્વન્સી સાથે ગેઈન વધે છે (\textless{} 50Hz)

  \begin{itemize}
  \tightlist
  \item
    કપલિંગ અને બાયપાસ કેપેસિટર્સથી મર્યાદિત
  \end{itemize}
\item
  \textbf{મિડ-ફ્રીક્વન્સી રીજન}: સતત મહત્તમ ગેઈન (50Hz-20kHz)

  \begin{itemize}
  \tightlist
  \item
    ફ્લેટ રિસ્પોન્સ, આદર્શ ઓપરેટિંગ રીજન
  \end{itemize}
\item
  \textbf{હાઈ-ફ્રીક્વન્સી રીજન}: ફ્રીક્વન્સી સાથે ગેઈન ઘટે છે (\textgreater{}
  20kHz)

  \begin{itemize}
  \tightlist
  \item
    ટ્રાન્ઝિસ્ટર કેપેસિટન્સ અને મિલર ઇફેક્ટથી મર્યાદિત
  \end{itemize}
\item
  \textbf{બેન્ડવિડ્થ}: મહત્તમ ગેઈનના \geq 70.7\% ગેઈન સાથેની ફ્રીક્વન્સીની રેન્જ
\item
  \textbf{કટ-ઓફ ફ્રીક્વન્સી}: એ બિંદુઓ જ્યાં ગેઈન 3dB (0.707 ગણો મહત્તમ ગેઈન) ઘટે છે
\end{itemize}

\end{solutionbox}
\begin{mnemonicbox}
``Low-flat-high: capacitors block, amplify well,
then roll off''

\end{mnemonicbox}
\subsection*{પ્રશ્ન 3(અ OR) [3
ગુણ]}\label{uxaaauxab0uxab6uxaa8-3uxa85-or-3-uxa97uxaa3}

\textbf{ઓસ્સીલેશન માટેનો બાખૌસેન ક્રાઈટીરીયા વિગતવાર સમજાવો.}

\begin{solutionbox}

\textbf{બાર્ખોસેન ક્રાઈટેરિયન:}

{\def\LTcaptype{none} % do not increment counter
\begin{longtable}[]{@{}ll@{}}
\toprule\noalign{}
શરત & આવશ્યકતા \\
\midrule\noalign{}
\endhead
\bottomrule\noalign{}
\endlastfoot
\textbf{લૂપ ગેઈન} & ચોક્કસ 1 (Aβ = 1) હોવો જરૂરી \\
\textbf{ફેઝ શિફ્ટ} & લૂપની આસપાસ 0^\circ અથવા 360^\circ હોવો જરૂરી \\
\end{longtable}
}

\begin{itemize}
\tightlist
\item
  \textbf{હેતુ}: ડેમ્પિંગ વિના સતત ઓસિલેશન સુનિશ્ચિત કરે છે
\item
  \textbf{પરિણામો}:

  \begin{itemize}
  \tightlist
  \item
    જો Aβ \textless{} 1: ઓસિલેશન ધીમે ધીમે ઓછા થાય છે
  \item
    જો Aβ \textgreater{} 1: ઓસિલેશન વધતા રહે છે, નોન-લિનિયારિટી દ્વારા
    મર્યાદિત થાય ત્યાં સુધી
  \item
    જો Aβ = 1: સ્થિર ઓસિલેશન જાળવી રાખવામાં આવે છે
  \end{itemize}
\end{itemize}

\end{solutionbox}
\begin{mnemonicbox}
``Barkhausen's Balance: Loop Gain=1, Phase=360^\circ''

\end{mnemonicbox}
\subsection*{પ્રશ્ન 3(બ OR) [4
ગુણ]}\label{uxaaauxab0uxab6uxaa8-3uxaac-or-4-uxa97uxaa3}

\textbf{એમ્પલીફાયરના ગેઈન પર નેગેટીવ ફીડબેકની અસર સમજાવો.}

\begin{solutionbox}

\textbf{એમ્પ્લિફાયર ગેઈન પર નેગેટિવ ફીડબેકની અસર:}

{\def\LTcaptype{none} % do not increment counter
\begin{longtable}[]{@{}lll@{}}
\toprule\noalign{}
પેરામીટર & ફીડબેક વિના & ફીડબેક સાથે \\
\midrule\noalign{}
\endhead
\bottomrule\noalign{}
\endlastfoot
\textbf{વોલ્ટેજ ગેઈન} & A & A/(1+Aβ) \\
\textbf{સ્ટેબિલિટી} & ઓછી સ્થિર & વધુ સ્થિર \\
\textbf{બેન્ડવિડ્થ} & નીચી & ઉંચી \\
\textbf{ડિસ્ટોર્શન} & વધારે & ઓછું \\
\end{longtable}
}

\begin{itemize}
\tightlist
\item
  \textbf{ગેઈન ઘટાડો}: ગેઈન (1+Aβ) ફેક્ટર દ્વારા ઘટે છે
\item
  \textbf{ગેઈન-બેન્ડવિડ્થ ટ્રેડઓફ}: ગેઈન ઘટતાં બેન્ડવિડ્થ વધે છે
\item
  \textbf{ગેઈન સ્ટેબિલાઈઝેશન}: તાપમાન અને કોમ્પોનન્ટ વેરિએશન દ્વારા ઓછી અસરગ્રસ્ત
\end{itemize}

\end{solutionbox}
\begin{mnemonicbox}
``Negative Feedback: Less Gain, More Stability''

\end{mnemonicbox}
\subsection*{પ્રશ્ન 3(ક OR) [7
ગુણ]}\label{uxaaauxab0uxab6uxaa8-3uxa95-or-7-uxa97uxaa3}

\textbf{ફેન રેગ્યુલેટરની સરકીટ દોરો અને તે ફેનની સ્પીડ કેવી રીતે કંટ્રોલ કરે છે તે
સમજાવો}

\begin{solutionbox}

\textbf{આકૃતિ:}

\begin{center}
\textbf{Mermaid Diagram (Code)}
\begin{verbatim}
{Shaded}
{Highlighting}[]
graph LR
    A[AC Supply] {-{-}{} B[DIAC]}
    B {-{-}{} C[TRIAC]}
    C {-{-}{} D[Fan]}
    E[Variable Resistor] {-{-}{} F[RC Network]}
    F {-{-}{} B}
{Highlighting}
{Shaded}
\end{verbatim}
\end{center}

\textbf{ફેન રેગ્યુલેટર ઓપરેશન:}

\begin{itemize}
\tightlist
\item
  \textbf{કંટ્રોલ પદ્ધતિ}: TRIAC અને DIAC વાપરીને ફેઝ એંગલ કંટ્રોલ
\item
  \textbf{કાર્યસિદ્ધાંત}: RC નેટવર્ક વેરિએબલ ફેઝ શિફ્ટ બનાવે છે
\item
  \textbf{સ્પીડ કંટ્રોલ}: વેરિએબલ રેઝિસ્ટર RC ટાઈમ કોન્સ્ટન્ટ એડજસ્ટ કરે છે
\item
  \textbf{ઓપરેશન સિક્વન્સ}:

  \begin{itemize}
  \tightlist
  \item
    RC નેટવર્ક DIAC ફાયરિંગમાં વિલંબ કરે છે
  \item
    DIAC ટ્રાયકને AC સાઇકલમાં એડજસ્ટેબલ પોઇન્ટ પર ટ્રિગર કરે છે
  \item
    TRIAC AC હાફ-સાઇકલના બાકીના ભાગ માટે કન્ડક્ટ કરે છે
  \item
    ઓછો કન્ડક્શન સમય = ફેન પર ઓછી પાવર = ધીમી ગતિ
  \end{itemize}
\item
  \textbf{ફાયદા}: સરળ ડિઝાઇન, સુંવાળું નિયંત્રણ, ઊર્જા કાર્યક્ષમ
\item
  \textbf{ઉપયોગો}: સિલિંગ ફેન, એક્ઝોસ્ટ ફેન, કૂલિંગ સિસ્ટમ્સ
\end{itemize}

\end{solutionbox}
\begin{mnemonicbox}
``Delay the TRIAC firing, control fan's speed''

\end{mnemonicbox}
\subsection*{પ્રશ્ન 4(અ) [3
ગુણ]}\label{uxaaauxab0uxab6uxaa8-4uxa85-3-uxa97uxaa3}

\textbf{નેચરલ કોમ્યુટેશન પર ટૂંક નોંધ લખો.}

\begin{solutionbox}

\textbf{નેચરલ કોમ્યુટેશન:}

\begin{itemize}
\tightlist
\item
  \textbf{વ્યાખ્યા}: SCR જ્યારે કરંટ હોલ્ડિંગ કરંટ કરતાં નીચે પડે ત્યારે આપોઆપ બંધ થાય
  છે
\item
  \textbf{પ્રક્રિયા}: AC સર્કિટમાં દરેક ઝીરો-ક્રોસિંગ પોઇન્ટ પર થાય છે
\item
  \textbf{જરૂરિયાતો}: કોઈ બાહ્ય ઘટકોની જરૂર નથી, AC ઓપરેશનમાં સ્વાભાવિક છે
\end{itemize}

\end{solutionbox}
\begin{mnemonicbox}
``Natural Commutation: Zero Current Crossings Turn
Off Thyristors''

\end{mnemonicbox}
\subsection*{પ્રશ્ન 4(બ) [4
ગુણ]}\label{uxaaauxab0uxab6uxaa8-4uxaac-4-uxa97uxaa3}

\textbf{એમ્પલીફાયરના પેરામીટર ગેઈન અને બેન્ડવીડ્થ સમજાવો.}

\begin{solutionbox}

\textbf{એમ્પ્લિફાયર પેરામીટર્સ:}

{\def\LTcaptype{none} % do not increment counter
\begin{longtable}[]{@{}lll@{}}
\toprule\noalign{}
પેરામીટર & વ્યાખ્યા & ફોર્મ્યુલા \\
\midrule\noalign{}
\endhead
\bottomrule\noalign{}
\endlastfoot
\textbf{ગેઈન (A)} & આઉટપુટનો ઇનપુટ સિગ્નલ સાથેનો ગુણોત્તર & A = Vout/Vin \\
\textbf{બેન્ડવિડ્થ (BW)} & ફ્રીક્વન્સી રેન્જ જ્યાં ગેઈન \geq 70.7\% મહત્તમ & BW = fH -
fL \\
\end{longtable}
}

\begin{itemize}
\tightlist
\item
  \textbf{ગેઈન-બેન્ડવિડ્થ પ્રોડક્ટ}: અચળ રહે છે (GBP = ગેઈન \times બેન્ડવિડ્થ)
\item
  \textbf{કટ-ઓફ ફ્રીક્વન્સી}: લોઅર (fL) અને હાયર (fH) ફ્રીક્વન્સી જ્યાં ગેઈન 3dB
  ઘટે છે
\item
  \textbf{મહત્વ}: એમ્પ્લિફાયરની વિવિધ ફ્રીક્વન્સી સંભાળવાની ક્ષમતા નક્કી કરે છે
\end{itemize}

\end{solutionbox}
\begin{mnemonicbox}
``Good Amplifiers Balance Width and Magnitude''

\end{mnemonicbox}
\subsection*{પ્રશ્ન 4(ક) [7
ગુણ]}\label{uxaaauxab0uxab6uxaa8-4uxa95-7-uxa97uxaa3}

\textbf{ટ્રાયેકનું કન્સ્ટ્રકશન અને લાક્ષણિકતા દોરો તેનું કાર્ય સમજાવો. ટ્રાયેકના ઉપયોગો
લખો.}

\begin{solutionbox}

\textbf{TRIAC કન્સ્ટ્રક્શન અને લાક્ષણિકતા:}

\begin{verbatim}
           MT2
            |
      ------+------
     /      |      \
    /  P    |    N  \
   +--------+--------+
   |        |        |
   |    N   |    P   |
   +--------+--------+
   |        |        |
   |    P   |    N   |
   +--------+--------+
    \       |       /
     \      |      /
      ------+------
            |
           MT1
            |
            G
\end{verbatim}

\textbf{I-V લાક્ષણિકતા:}

\begin{verbatim}
    I
    ^
    |      /|
    |     / |
    |    /  |
    |---+---|----> V
    |   /   |
    |  /    |
    | /     |
\end{verbatim}

\textbf{TRIAC ઓપરેશન:}

\begin{itemize}
\tightlist
\item
  \textbf{સ્ટ્રક્ચર}: પાંચ-લેયર PNPN બાયડાયરેક્શનલ ડિવાઇસ
\item
  \textbf{સ્વિચિંગ}: ટ્રિગર થયા પછી બંને દિશામાં કન્ડક્ટ કરે છે
\item
  \textbf{ટ્રિગરિંગ મોડ્સ}: ફોર ક્વોડ્રન્ટ ઓપરેશન શક્ય
\item
  \textbf{ટર્ન-ઓફ}: કરંટ ઝીરો-ક્રોસિંગ પર નેચરલ કોમ્યુટેશન
\end{itemize}

\textbf{ઉપયોગો:}

\begin{itemize}
\tightlist
\item
  \textbf{લાઇટ ડિમર્સ}
\item
  \textbf{ફેન સ્પીડ કંટ્રોલર્સ}
\item
  \textbf{હીટર કંટ્રોલ્સ}
\item
  \textbf{મોટર સ્પીડ રેગ્યુલેશન}
\item
  \textbf{AC પાવર સ્વિચિંગ}
\end{itemize}

\end{solutionbox}
\begin{mnemonicbox}
``TRIAC Takes AC Control in Both Directions''

\end{mnemonicbox}
\subsection*{પ્રશ્ન 4(અ OR) [3
ગુણ]}\label{uxaaauxab0uxab6uxaa8-4uxa85-or-3-uxa97uxaa3}

\textbf{એસ.સી.આર ના કોઈપણ ત્રણ ઉપયોગો લખો}

\begin{solutionbox}

\textbf{SCR ના ઉપયોગો:}

{\def\LTcaptype{none} % do not increment counter
\begin{longtable}[]{@{}ll@{}}
\toprule\noalign{}
ઉપયોગ & કાર્ય \\
\midrule\noalign{}
\endhead
\bottomrule\noalign{}
\endlastfoot
\textbf{DC મોટર સ્પીડ કંટ્રોલ} & મોટર્સને વેરિએબલ DC પ્રદાન કરે છે \\
\textbf{બેટરી ચાર્જર્સ} & ચાર્જિંગ કરંટને નિયંત્રિત કરે છે \\
\textbf{પાવર ઈન્વર્ટર્સ} & DC ને AC માં કાર્યક્ષમતાથી રૂપાંતરિત કરે છે \\
\end{longtable}
}

\begin{itemize}
\tightlist
\item
  \textbf{ફાયદા}: ઉચ્ચ પાવર હેન્ડલિંગ, કાર્યક્ષમ નિયંત્રણ, મજબૂત ઓપરેશન
\item
  \textbf{મર્યાદાઓ}: DC સર્કિટ્સમાં ફોર્સ્ડ કોમ્યુટેશનની જરૂર પડે છે
\end{itemize}

\end{solutionbox}
\begin{mnemonicbox}
``SCR Controls DC - Motors, Batteries, Inverters''

\end{mnemonicbox}
\subsection*{પ્રશ્ન 4(બ OR) [4
ગુણ]}\label{uxaaauxab0uxab6uxaa8-4uxaac-or-4-uxa97uxaa3}

\textbf{એસ.સી.આર ના સંદર્ભમાં હોલ્ડીંગ કરન્ટ અને લેચીંગ કરન્ટ સમજાવો}

\begin{solutionbox}

\textbf{SCR કરંટ પેરામીટર્સ:}

{\def\LTcaptype{none} % do not increment counter
\begin{longtable}[]{@{}lll@{}}
\toprule\noalign{}
પેરામીટર & વ્યાખ્યા & સામાન્ય મૂલ્યો \\
\midrule\noalign{}
\endhead
\bottomrule\noalign{}
\endlastfoot
\textbf{હોલ્ડિંગ કરંટ (IH)} & કન્ડક્શન જાળવવા માટેનો લઘુત્તમ કરંટ & 5-40 mA \\
\textbf{લેચિંગ કરંટ (IL)} & કન્ડક્શન સ્થાપિત કરવા માટેનો લઘુત્તમ કરંટ & 10-100
mA \\
\end{longtable}
}

\begin{itemize}
\tightlist
\item
  \textbf{લેચિંગ કરંટ}: SCR લેચ થાય તે માટે ટ્રિગરિંગ પછી ટૂંક સમય માટે આટલો કરંટ
  વહેવો જોઈએ
\item
  \textbf{હોલ્ડિંગ કરંટ}: SCR ને કન્ડક્શનમાં રાખવા માટે જાળવવો જોઈએ
\item
  \textbf{સંબંધ}: સામાન્ય રીતે IL \textgreater{} IH
\item
  \textbf{મહત્વ}: વિશ્વસનીય સ્વિચિંગ ઓપરેશન માટે મહત્વપૂર્ણ
\end{itemize}

\end{solutionbox}
\begin{mnemonicbox}
``Latch with more, Hold with less, both keep SCR
conducting''

\end{mnemonicbox}
\subsection*{પ્રશ્ન 4(ક OR) [7
ગુણ]}\label{uxaaauxab0uxab6uxaa8-4uxa95-or-7-uxa97uxaa3}

\textbf{ઓપરેશનલ એમ્પલીફાયરનો બ્લોક ડાયગ્રામ દોરો અને વિગતવાર સમજાવો}

\begin{solutionbox}

\textbf{ઓપરેશનલ એમ્પ્લિફાયર બ્લોક ડાયાગ્રામ:}

\begin{center}
\textbf{Mermaid Diagram (Code)}
\begin{verbatim}
{Shaded}
{Highlighting}[]
graph LR
    A[Input Differential Stage] {-{-}{} B[Intermediate Stage]}
    B {-{-}{} C[Output Stage]}
    D[Bias Circuit] {-{-}{} A \& B \& C}
    E[Frequency Compensation] {-{-}{} B}
{Highlighting}
{Shaded}
\end{verbatim}
\end{center}

\textbf{ઓપ-એમ્પ બ્લોક્સ અને ફંક્શન્સ:}

\begin{itemize}
\tightlist
\item
  \textbf{ઇનપુટ ડિફરેન્શિયલ સ્ટેજ}:

  \begin{itemize}
  \tightlist
  \item
    ઉચ્ચ ઇનપુટ ઇમ્પેડન્સ
  \item
    કોમન-મોડ સિગ્નલ્સને રિજેક્ટ કરે છે
  \item
    ડિફરેન્શિયલ વોલ્ટેજ ગેઈન પ્રદાન કરે છે
  \end{itemize}
\item
  \textbf{ઇન્ટરમીડિએટ સ્ટેજ}:

  \begin{itemize}
  \tightlist
  \item
    વધારાનો વોલ્ટેજ ગેઈન
  \item
    લેવલ શિફ્ટિંગ
  \item
    ફ્રીક્વન્સી કોમ્પેન્સેશન
  \end{itemize}
\item
  \textbf{આઉટપુટ સ્ટેજ}:

  \begin{itemize}
  \tightlist
  \item
    ઓછી આઉટપુટ ઇમ્પેડન્સ
  \item
    કરંટ એમ્પ્લિફિકેશન
  \item
    લોડ્સ ચલાવવા માટે પાવર કેપેબિલિટી
  \end{itemize}
\item
  \textbf{બાયસ સર્કિટ}:

  \begin{itemize}
  \tightlist
  \item
    યોગ્ય ઓપરેટિંગ પોઇન્ટ્સ સ્થાપિત કરે છે
  \item
    તાપમાન સ્થિરતા
  \end{itemize}
\item
  \textbf{ફ્રીક્વન્સી કોમ્પેન્સેશન}:

  \begin{itemize}
  \tightlist
  \item
    ઓસિલેશન અટકાવે છે
  \item
    ફ્રીક્વન્સી રિસ્પોન્સ નિયંત્રિત કરે છે
  \end{itemize}
\end{itemize}

\end{solutionbox}
\begin{mnemonicbox}
``Differential Input, Gain in Middle, Power at
Output''

\end{mnemonicbox}
\subsection*{પ્રશ્ન 5(અ) [3
ગુણ]}\label{uxaaauxab0uxab6uxaa8-5uxa85-3-uxa97uxaa3}

\textbf{ઇનવરટિંગ એમ્પલીફાયર દોરો અને ટૂંકમાં સમજાવો}

\begin{solutionbox}

\textbf{ઇન્વર્ટિંગ એમ્પ્લિફાયર સર્કિટ:}

\begin{verbatim}
          Rf
          ___
    Vin---| |-----+
          ---     |
                  |
                 _|_
    +------+    /   \
    |      |---+     +---Vout
    |      |    \___/
Vin-+      |      |
    |Op-Amp|      |
    +------+      |
                  |
                 ---
                 ///
\end{verbatim}

\begin{itemize}
\tightlist
\item
  \textbf{ગેઈન ફોર્મ્યુલા}: Vout = -(Rf/Rin) \times Vin
\item
  \textbf{ઓપરેશન}: ઇનપુટ સિગ્નલ એમ્પ્લિફિકેશન સાથે ઇન્વર્ટ થાય છે
\item
  \textbf{વર્ચ્યુઅલ ગ્રાઉન્ડ}: ઇન્વર્ટિંગ ઇનપુટ 0V પર જાળવવામાં આવે છે
\end{itemize}

\end{solutionbox}
\begin{mnemonicbox}
``Inverting means Negative Gain equals -Rf/Rin''

\end{mnemonicbox}
\subsection*{પ્રશ્ન 5(બ) [4
ગુણ]}\label{uxaaauxab0uxab6uxaa8-5uxaac-4-uxa97uxaa3}

\textbf{રેગ્યુલેટેડ પાવર સપ્લાયનો બ્લોક ડાયગ્રામ દોરો અને સમજાવો}

\begin{solutionbox}

\textbf{રેગ્યુલેટેડ પાવર સપ્લાય બ્લોક ડાયાગ્રામ:}

\begin{center}
\textbf{Mermaid Diagram (Code)}
\begin{verbatim}
{Shaded}
{Highlighting}[]
graph LR
    A[Transformer] {-{-}{} B[Rectifier]}
    B {-{-}{} C[Filter]}
    C {-{-}{} D[Regulator]}
    D {-{-}{} E[Output]}
    F[Reference] {-{-}{} D}
    G[Feedback] {-{-}{} D}
{Highlighting}
{Shaded}
\end{verbatim}
\end{center}

\textbf{રેગ્યુલેટેડ પાવર સપ્લાય સ્ટેજેસ:}

\begin{itemize}
\tightlist
\item
  \textbf{ટ્રાન્સફોર્મર}: AC વોલ્ટેજને જરૂરી લેવલ સુધી નીચે લાવે છે
\item
  \textbf{રેક્ટિફાયર}: AC ને પલ્સેટિંગ DC માં રૂપાંતરિત કરે છે (ડાયોડ બ્રિજ)
\item
  \textbf{ફિલ્ટર}: પલ્સેટિંગ DC ને સુંવાળો બનાવે છે (કેપેસિટર્સ)
\item
  \textbf{રેગ્યુલેટર}: વેરિએશન હોવા છતાં સ્થિર આઉટપુટ જાળવે છે
\item
  \textbf{રેફરન્સ}: સ્થિર તુલના વોલ્ટેજ પ્રદાન કરે છે
\item
  \textbf{ફીડબેક}: આઉટપુટનું મોનિટરિંગ કરે છે અને રેગ્યુલેશન એડજસ્ટ કરે છે
\end{itemize}

\end{solutionbox}
\begin{mnemonicbox}
``Transform, Rectify, Filter, Regulate for Stable
DC''

\end{mnemonicbox}
\subsection*{પ્રશ્ન 5(ક) [7
ગુણ]}\label{uxaaauxab0uxab6uxaa8-5uxa95-7-uxa97uxaa3}

\textbf{એસ્ટેબલ મલ્ટીવાયબ્રેટર દોરો અને સમજાવો}

\begin{solutionbox}

\textbf{555 ટાઇમર વાપરીને એસ્ટેબલ મલ્ટીવાયબ્રેટર:}

\begin{center}
\textbf{Mermaid Diagram (Code)}
\begin{verbatim}
{Shaded}
{Highlighting}[]
graph TD
    subgraph 555 Timer
    A[Threshold] {-{-}{} B[Flip{-}Flop]}
    C[Trigger] {-{-}{} B}
    B {-{-}{} D[Output]}
    end
    E[R1] \& F[R2] \& G[C] {-{-}{} A \& C}
{Highlighting}
{Shaded}
\end{verbatim}
\end{center}

\textbf{એસ્ટેબલ મલ્ટીવાયબ્રેટરનું ઓપરેશન:}

\begin{itemize}
\item
  \textbf{કન્ફિગરેશન}: ફ્રી-રનિંગ ઓસિલેટર જેમાં કોઈ સ્ટેબલ સ્ટેટ્સ નથી
\item
  \textbf{ટાઇમિંગ કોમ્પોનન્ટ્સ}: બાહ્ય R1, R2, અને C
\item
  \textbf{ઓસિલેશન પ્રક્રિયા}:

  \begin{itemize}
  \tightlist
  \item
    કેપેસિટર R1+R2 દ્વારા ચાર્જ થાય છે
  \item
    કેપેસિટર R2 દ્વારા ડિસ્ચાર્જ થાય છે
  \item
    સતત ચાર્જિંગ/ડિસ્ચાર્જિંગ સાયકલ
  \end{itemize}
\item
  \textbf{આઉટપુટ વેવફોર્મ}: R1/R2 રેશિયો પર આધારિત ડ્યુટી સાયકલ સાથે રેક્ટેંગ્યુલર
\item
  \textbf{ફ્રીક્વન્સી ફોર્મ્યુલા}: f = 1.44/((R1+2R2)\timesC)
\item
  \textbf{ઉપયોગો}: ક્લોક જનરેશન, LED ફ્લેશર્સ, ટોન જનરેટર્સ
\item
  \textbf{ફાયદા}: સરળ ડિઝાઇન, સ્ટેબલ ફ્રીક્વન્સી, એડજસ્ટેબલ ડ્યુટી સાયકલ
\end{itemize}

\end{solutionbox}
\begin{mnemonicbox}
``Always Switching, Time set by RC, Both states
Least stable''

\end{mnemonicbox}
\subsection*{પ્રશ્ન 5(અ OR) [3
ગુણ]}\label{uxaaauxab0uxab6uxaa8-5uxa85-or-3-uxa97uxaa3}

\textbf{ઓપી. એએમપી. નોનઇનવરટિંગ એમ્પલીફાયરમા R1=2kΩ અને Rf=200kΩ છે.
નોનઇનવરટિંગ એમ્પલીફાયરનો ગેઈન શોધો.}

\begin{solutionbox}

\textbf{ઉકેલ:}

\begin{itemize}
\tightlist
\item
  \textbf{આપેલું}: R1 = 2kΩ, Rf = 200kΩ
\item
  \textbf{નોન-ઇન્વર્ટિંગ એમ્પ્લિફાયર ગેઈન ફોર્મ્યુલા}: A = 1 + (Rf/R1)
\item
  \textbf{ગણતરી}:

  \begin{itemize}
  \tightlist
  \item
    A = 1 + (200kΩ/2kΩ)
  \item
    A = 1 + 100
  \item
    A = 101
  \end{itemize}
\item
  \textbf{પરિણામ}: નોન-ઇન્વર્ટિંગ એમ્પ્લિફાયરનો વોલ્ટેજ ગેઈન 101 છે
\item
  \textbf{મહત્વ}: આઉટપુટ વોલ્ટેજ ઇનપુટ વોલ્ટેજના 101 ગણો હશે
\end{itemize}

\end{solutionbox}
\begin{mnemonicbox}
``Non-inverting amplifier gain: One plus Feedback
over Ground''

\end{mnemonicbox}
\subsection*{પ્રશ્ન 5(બ OR) [4
ગુણ]}\label{uxaaauxab0uxab6uxaa8-5uxaac-or-4-uxa97uxaa3}

\textbf{-5V રેગ્યુલેટેડ ડીસી આઉટપુટ વૉલ્ટેજ મેળવવા માટેની સરકીટ દોરો અને ટૂંકમાં
સમજાવો.}

\begin{solutionbox}

\textbf{નેગેટિવ વોલ્ટેજ રેગ્યુલેટર સર્કિટ:}

\begin{verbatim}
     +--------+
     |        |
Vin--+        +---Vout (-5V)
     | 7905   |
     |        |
     +--------+
         |
        ---
        ///
\end{verbatim}

\textbf{સર્કિટ ઓપરેશન:}

\begin{itemize}
\tightlist
\item
  \textbf{મુખ્ય ઘટક}: 7905 નેગેટિવ વોલ્ટેજ રેગ્યુલેટર IC
\item
  \textbf{ઇનપુટ આવશ્યકતા}: નેગેટિવ DC વોલ્ટેજ (સામાન્ય રીતે -7V થી -25V)
\item
  \textbf{ફિલ્ટરિંગ}: સ્થિરતા માટે ઇનપુટ અને આઉટપુટ કેપેસિટર્સ
\item
  \textbf{રેગ્યુલેશન પદ્ધતિ}: ફીડબેક કંટ્રોલ સાથે સીરીઝ પાસ એલિમેન્ટ
\item
  \textbf{આઉટપુટ લાક્ષણિકતાઓ}: 1A સુધીના કરંટ સાથે ફિક્સ્ડ -5V
\end{itemize}

\end{solutionbox}
\begin{mnemonicbox}
``79XX for Negative, 78XX for Positive regulated
voltage''

\end{mnemonicbox}
\subsection*{પ્રશ્ન 5(ક OR) [7
ગુણ]}\label{uxaaauxab0uxab6uxaa8-5uxa95-or-7-uxa97uxaa3}

\textbf{એસ.એમ.પી.એસ. નો બ્લોક ડાયગ્રામ દોરો અને સમજાવો}

\begin{solutionbox}

\textbf{SMPS બ્લોક ડાયાગ્રામ:}

\begin{center}
\textbf{Mermaid Diagram (Code)}
\begin{verbatim}
{Shaded}
{Highlighting}[]
graph LR
    A[AC Input] {-{-}{} B[EMI Filter]}
    B {-{-}{} C[Rectifier \& Filter]}
    C {-{-}{} D[High{-}Frequency Inverter]}
    D {-{-}{} E[Transformer]}
    E {-{-}{} F[Output Rectifier]}
    F {-{-}{} G[Output Filter]}
    G {-{-}{} H[DC Output]}
    I[Feedback \& Control] {-{-}{} D}
    H {-{-}{} I}
{Highlighting}
{Shaded}
\end{verbatim}
\end{center}

\textbf{SMPS ઓપરેશન:}

\begin{itemize}
\tightlist
\item
  \textbf{ઇનપુટ સ્ટેજ}: EMI ફિલ્ટર કરે છે, AC ને હાઈ-વોલ્ટેજ DC માં રેક્ટિફાય કરે છે
\item
  \textbf{સ્વિચિંગ સ્ટેજ}: DC ને હાઈ-ફ્રીક્વન્સી AC માં રૂપાંતરિત કરે છે (20-100 kHz)
\item
  \textbf{ટ્રાન્સફોર્મર}: આઇસોલેશન અને વોલ્ટેજ ટ્રાન્સફોર્મેશન પ્રદાન કરે છે
\item
  \textbf{આઉટપુટ સ્ટેજ}: ક્લીન DC ઉત્પન્ન કરવા માટે રેક્ટિફાય અને ફિલ્ટર કરે છે
\item
  \textbf{ફીડબેક કંટ્રોલ}: સ્વિચિંગ ડ્યુટી સાયકલ એડજસ્ટ કરીને આઉટપુટ રેગ્યુલેટ કરે છે
\end{itemize}

\textbf{SMPS ના ફાયદા:}

\begin{itemize}
\tightlist
\item
  \textbf{ઉચ્ચ કાર્યક્ષમતા} (80-90\%) સ્વિચિંગ ઓપરેશનને કારણે
\item
  \textbf{નાનું કદ અને વજન} હાઈ-ફ્રીક્વન્સી ટ્રાન્સફોર્મરથી
\item
  \textbf{વિસ્તૃત ઇનપુટ વોલ્ટેજ રેન્જ} સ્થિર આઉટપુટ સાથે
\item
  \textbf{સિંગલ ટ્રાન્સફોર્મરથી મલ્ટિપલ આઉટપુટ વોલ્ટેજ} શક્ય
\end{itemize}

\textbf{ઉપયોગો:}

\begin{itemize}
\tightlist
\item
  કમ્પ્યુટર પાવર સપ્લાય
\item
  ઇલેક્ટ્રોનિક ડિવાઇસ ચાર્જર્સ
\item
  ઔદ્યોગિક પાવર સિસ્ટમ્સ
\end{itemize}

\end{solutionbox}
\begin{mnemonicbox}
``Switch More Power Smartly: High frequency saves
size and energy''

\end{mnemonicbox}

\end{document}
