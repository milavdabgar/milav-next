\documentclass{article}

% content/resources/templates/preamble.tex
\usepackage[margin=0.6in]{geometry}
\author{Milav Dabgar}
\usepackage{amsmath,amssymb,amsthm}
\usepackage{booktabs}
\usepackage{multirow}
\usepackage{xcolor}
\usepackage{tcolorbox}
\tcbuselibrary{breakable,skins}
\usepackage[colorlinks=true,linkcolor=blue]{hyperref}
\usepackage{titlesec}
\usepackage{enumitem}
\usepackage{tikz}
\usepackage{pgfplots}
\usepackage{circuitikz}
\usepackage[version=4]{mhchem}
\usepackage{longtable}
\usepackage{array}
\usepackage{float}
\usepackage{caption}
\usepackage{listings}

\lstset{
  basicstyle=\small\ttfamily,
  breaklines=true,
  breakatwhitespace=false,
  postbreak=\mbox{\textcolor{red}{$\hookrightarrow$}\space},
  float=false,
  numbers=left,
  numberstyle=\tiny\color{gray},
  numbersep=10pt,
  xleftmargin=2em,
  keywordstyle=\color{blue},
  commentstyle=\color{green!60!black},
  stringstyle=\color{purple},
  backgroundcolor=\color{gray!5},
  showstringspaces=false,
  tabsize=2,
  captionpos=b,
  keepspaces=true,
  columns=flexible
}

\pgfplotsset{compat=1.18}
\usetikzlibrary{shapes,arrows,positioning,calc,patterns,decorations.pathmorphing,decorations.markings,arrows.meta}

% Color scheme
\definecolor{headcolor}{RGB}{0,102,204}
\definecolor{keycolor}{RGB}{220,20,60}
\definecolor{solutioncolor}{RGB}{34,139,34}
\definecolor{mnemoniccolor}{RGB}{148,0,211}
\definecolor{codecolor}{RGB}{0,0,100}

% Spacing
\setlength{\parskip}{3pt}
\setlist[itemize]{nosep}
\setlist[enumerate]{nosep}

% Title formatting
\titleformat{\section}{\Large\bfseries\color{headcolor}}{\thesection}{1em}{}
\titleformat{\subsection}{\large\bfseries\color{headcolor}}{\thesubsection}{1em}{}

% Pandoc tightlist compatibility
\providecommand{\tightlist}{%
  \setlength{\itemsep}{0pt}\setlength{\parskip}{0pt}}

% Pandoc longtable compatibility
\newcounter{none}
\def\thenone{}


% content/resources/templates/gujarati-boxes.tex
\usepackage{fontspec}
\usepackage{polyglossia}

% Set Gujarati as main language (document is primarily in Gujarati)
% Note: gloss-gujarati.ldf doesn't exist in polyglossia, but it will use hyphenation patterns
\setdefaultlanguage{gujarati}
\setotherlanguage{english}

% Configure Gujarati font properly
% Use Language=Default to prevent polyglossia from trying to add language-specific features
% that don't exist for Gujarati, which causes "empty feature" warnings
\newfontfamily\gujaratifont[Script=Gujarati,AutoFakeBold=2.5,AutoFakeSlant=0.3]{Noto Sans Gujarati}
\setmainfont[Script=Gujarati,AutoFakeBold=2.5,AutoFakeSlant=0.3]{Noto Sans Gujarati}
% Use Noto Sans Gujarati for monospace to support Gujarati in text
\setmonofont[Scale=0.9]{Noto Sans Gujarati}

% Configure English to use the same font
\newfontfamily\englishfont[Script=Gujarati,AutoFakeBold=2.5,AutoFakeSlant=0.3]{Noto Sans Gujarati}

% Translations for polyglossia
\gappto\captionsgujarati{
  \renewcommand{\tablename}{કોષ્ટક}
  \renewcommand{\figurename}{આકૃતિ}
}

% Helper for TikZ nodes to ensure Gujarati font
\newcommand{\gu}[1]{{\gujaratifont #1}}

% Custom environments
\newtcolorbox{solutionbox}{
    breakable,
    enhanced,
    colback=solutioncolor!5!white,
    colframe=solutioncolor!75!black,
    fonttitle=\bfseries,
    title=જવાબ
}

\newtcolorbox{solutionboxnobreak}{
 colback=solutioncolor!5!white,
 colframe=solutioncolor!75!black,
 fonttitle=\bfseries,
 title=જવાબ
}

\newtcolorbox{keyformula}{
 breakable,
 enhanced,
 colback=keycolor!5!white,
 colframe=keycolor!75!black,
 fonttitle=\bfseries,
 title=રાસાયણિક સમીકરણ/સૂત્ર
}

\newtcolorbox{mnemonicbox}{
 breakable,
 enhanced,
 colback=mnemoniccolor!5!white,
 colframe=mnemoniccolor!75!black,
 fonttitle=\bfseries,
 title=મેમરી ટ્રીક
}


% Custom commands for GTU solutions
% This file defines semantic commands for consistent formatting

% Question command with automatic formatting
\newcommand{\question}[2]{%
  \section*{Question #1}%
  \textbf{#2}%
}

% OR question variant
\newcommand{\questionor}[2]{%
  \section*{Question #1 OR}%
  \textbf{#2}%
}

% Proper table environment with caption
\newenvironment{answertable}[1]{%
  \begin{table}[htbp]
  \centering
  \caption{#1}
}{%
  \end{table}
}

% Proper figure environment for diagrams
\newenvironment{answerdiagram}[1]{%
  \begin{figure}[htbp]
  \centering
  \caption{#1}
}{%
  \end{figure}
}

% Semantic markup for key terms
\newcommand{\keyword}[1]{\textbf{#1}}
\newcommand{\code}[1]{\texttt{#1}}
\newcommand{\classname}[1]{\texttt{#1}}
\newcommand{\methodname}[1]{\texttt{#1}}

% Proper quotation marks
\newcommand{\mnemonic}[1]{``#1''}


\title{Electronics Devices \& Circuits (1323202) - Summer 2024 Solution}
\date{June 21, 2024}

\begin{document}
\maketitle

\questionmarks{1(a)}{3}{હીટ સિંક શું છે. તેના પ્રકારોની યાદી આપો.}

\begin{solutionbox}
હીટ સિંક એ એક પેસિવ ડિવાઈસ છે જે ઇલેક્ટ્રોનિક કોમ્પોનન્ટ્સમાંથી ગરમી શોષે અને ફેલાવે છે જેથી ઓવરહીટિંગ અટકાવી શકાય.

\begin{center}
\captionof{table}{હીટ સિંકના પ્રકારો}
\begin{tabulary}{\linewidth}{|L|L|}
\hline
\textbf{પ્રકાર} & \textbf{વર્ણન} \\
\hline
\textbf{પેસિવ} & બાહ્ય પાવર વિના નૈસર્ગિક કન્વેક્શનનો ઉપયોગ કરે છે \\
\hline
\textbf{એક્ટિવ} & ફેન અથવા લિક્વિડ કૂલિંગનો સમાવેશ કરે છે \\
\hline
\textbf{રેડિયલ} & સેન્ટરથી રેડિયલ પેટર્નમાં ગોઠવાયેલા ફિન્સ \\
\hline
\textbf{પિન-ફિન} & વધુ સપાટી ક્ષેત્રફળ માટે પિન અથવા રોડનો ઉપયોગ કરે છે \\
\hline
\textbf{એક્સટ્રુડેડ} & આકારવાળા ડાય દ્વારા એલ્યુમિનિયમને ફોર્સ કરીને બનાવવામાં આવે છે \\
\hline
\end{tabulary}
\end{center}

\begin{mnemonicbox}
"PAPER" (Passive, Active, Pin-fin, Extruded, Radial)
\end{mnemonicbox}
\end{solutionbox}

\questionmarks{1(b)}{4}{નીચેનાને વ્યાખ્યાયિત કરો: 1. થર્મલ રનઅવે 2. થર્મલ સ્ટેબીલિટી.}

\begin{solutionbox}
\textbf{થર્મલ રનઅવે}:
સ્વ-ત્વરિત વિનાશક પ્રક્રિયા જ્યાં વધતા તાપમાન કરંટ પ્રવાહમાં વધારો કરે છે, જે વધુ તાપમાન વધારે છે, જે ટ્રાન્ઝિસ્ટરને નુકસાન પહોંચાડી શકે છે.

\textbf{થર્મલ સ્ટેબીલિટી}:
તાપમાન ફેરફારો છતાં સ્થિર ઓપરેશન જાળવવા માટે ટ્રાન્ઝિસ્ટર સર્કિટની ક્ષમતા, જે થર્મલ રનઅવેને અટકાવે છે.

\begin{center}
\begin{tikzpicture}[gtu block]
    \node (temp) [draw, rectangle] {તાપમાન વધે છે};
    \node (ic) [draw, rectangle, right=1cm of temp] {કલેક્ટર કરંટ વધે છે};
    \node (pd) [draw, rectangle, right=1cm of ic] {પાવર ડિસીપેશન વધે છે};
    \node (dest) [draw, rectangle, below=1cm of temp] {ડિવાઇસ નાશ};
    
    \draw [gtu arrow] (temp) -- (ic);
    \draw [gtu arrow] (ic) -- (pd);
    \draw [gtu arrow] (pd.south) -- ++(0,-0.5) -| (temp.south);
    \draw [gtu arrow] (temp) -- (dest);
\end{tikzpicture}
\end{center}

\begin{mnemonicbox}
"RISE" (Runaway Is Self-Escalating)
\end{mnemonicbox}
\end{solutionbox}

\questionmarks{1(c)}{7}{વોલ્ટેજ ડિવાઈડર બાયસને વિગતવાર સમજાવો.}

\begin{solutionbox}
વોલ્ટેજ ડિવાઈડર બાયસ એ એક સામાન્ય ટ્રાન્ઝિસ્ટર બાયસિંગ ટેકનિક છે જે સ્થિર ઓપરેશન પ્રદાન કરે છે.

\begin{center}
\begin{circuitikz}[american]
    \draw (0,0) node[ground] {} to[R, l=$R_E$] (0,2) -- (0,2.5) node[npn, anchor=E] (Q) {};
    \draw (Q.C) -- (0,4.5) to[R, l=$R_C$] (0,6.5) -- (0,7) node[vcc] {$+V_{CC}$};
    \draw (Q.B) -- (-1.5, 3.25);
    \draw (-1.5, 3.25) to[R, l=$R_2$, *-] (-1.5, 0) node[ground] {};
    \draw (-1.5, 3.25) to[R, l=$R_1$] (-1.5, 7) node[vcc] {$+V_{CC}$};
\end{circuitikz}
\end{center}

\begin{itemize}
    \item \textbf{વોલ્ટેજ ડિવાઈડર નેટવર્ક}: $R_1$ અને $R_2$ એક નિશ્ચિત બેઝ વોલ્ટેજ સ્થાપિત કરે છે
    \item \textbf{સ્થિર Q-પોઈન્ટ}: તાપમાન વેરિએશન છતાં ઓપરેટિંગ પોઈન્ટને જાળવે છે
    \item \textbf{વધુ સારી સ્થિરતા}: ફિક્સ્ડ બાયસની તુલનામાં ઉચ્ચ સ્થિરતા ફેક્ટર
    \item \textbf{સ્વ-એડજસ્ટિંગ}: બેઝ કરંટ આપોઆપ તાપમાન ફેરફારોનો સામનો કરવા માટે એડજસ્ટ થાય છે
\end{itemize}

\begin{mnemonicbox}
"VSST" (Voltage divider, Stable, Self-adjusting, Temperature resistant)
\end{mnemonicbox}
\end{solutionbox}

\vspace{0.5em}\centerline{\textbf{OR}}\questionmarks{1(c)}{7}{ડી.સી. લોડ લાઈનને વિગતવાર સમજાવો.}

\begin{solutionbox}
DC લોડ લાઈન એ ટ્રાન્ઝિસ્ટર બાયસ કંડીશન્સના વિશ્લેષણ માટેની ગ્રાફિકલ પદ્ધતિ છે.

\begin{center}
\begin{tikzpicture}[scale=0.8]
    % Axes
    \draw[->] (0,0) -- (6,0) node[right] {$V_{CE}$};
    \draw[->] (0,0) -- (0,6) node[above] {$I_C$};
    
    % Load Line
    \draw [thick] (0,5) node[left] {$\frac{V_{CC}}{R_C}$} -- (5,0) node[below] {$V_{CC}$};
    
    % Output Characterisitics (Simplified)
    \draw [gray, domain=0:5.5] plot (\x, {4*(1-exp(-\x))*0.8});
    \draw [gray, domain=0:5.5] plot (\x, {4*(1-exp(-\x))*0.6});
    \draw [gray, domain=0:5.5] plot (\x, {4*(1-exp(-\x))*0.4});
    
    % Q-point
    \filldraw (2.1, 2.9) circle (2pt) node[above right] {Q-point};
    \draw [dashed] (2.1, 0) node[below] {$V_{CEQ}$} -- (2.1, 2.9) -- (0, 2.9) node[left] {$I_{CQ}$};
\end{tikzpicture}
\end{center}

\begin{itemize}
    \item \textbf{વ્યાખ્યા}: આપેલી સર્કિટ માટે તમામ સંભવિત ઓપરેટિંગ પોઇન્ટ્સ દર્શાવતી ગ્રાફિકલ લાઈન
    \item \textbf{એન્ડપોઈન્ટ}: $(0, V_{CC}/R_C)$ અને $(V_{CC}, 0)$ $I_C$-$V_{CE}$ પ્લેન પર
    \item \textbf{Q-પોઈન્ટ}: લોડ લાઈન અને ટ્રાન્ઝિસ્ટર કેરેક્ટરિસ્ટિક કર્વના છેદબિંદુ
    \item \textbf{સમીકરણ}: $I_C = (V_{CC} - V_{CE})/R_C$
\end{itemize}

\begin{mnemonicbox}
"QECC" (Q-point Exists where Collector Current meets characteristics)
\end{mnemonicbox}
\end{solutionbox}

\questionmarks{2(a)}{3}{ટ્રાન્ઝિસ્ટર સ્વીચ તરીકે કેવી રીતે કામ કરે છે તે સમજાવો.}

\begin{solutionbox}
ટ્રાન્ઝિસ્ટર સ્વિચ સેચુરેશન (ON) અથવા કટ-ઓફ (OFF) રીજનમાં કામ કરે છે.

\begin{center}
\captionof{table}{ટ્રાન્ઝિસ્ટર સ્વિચ ઓપરેશન}
\begin{tabulary}{\linewidth}{|L|L|L|L|L|}
\hline
\textbf{સ્થિતિ} & \textbf{રીજન} & \textbf{બેઝ કરંટ} & \textbf{કલેક્ટર કરંટ} & \textbf{VCE} \\
\hline
\textbf{OFF} & કટ-ઓફ & $I_B \approx 0$ & $I_C \approx 0$ & $V_{CE} \approx V_{CC}$ \\
\hline
\textbf{ON} & સેચુરેશન & $I_B > I_{B(sat)}$ & $I_C \approx I_{C(sat)}$ & $V_{CE} \approx 0.2V$ \\
\hline
\end{tabulary}
\end{center}

\begin{mnemonicbox}
"COS" (Cutoff Off, Saturation on)
\end{mnemonicbox}
\end{solutionbox}

\questionmarks{2(b)}{4}{કોલપીટ ઓસીલેટર દોરો અને સમજાવો.}

\begin{solutionbox}
કોલપીટ ઓસીલેટર એ LC ઓસીલેટર છે જે ફીડબેક માટે કેપેસિટિવ વોલ્ટેજ ડિવાઈડરનો ઉપયોગ કરે છે.

\begin{center}
\begin{circuitikz}[american, scale=0.9, transform shape]
    \node [npn] (Q) at (0,0) {};
    \draw (Q.E) to[R, l=$R_E$] ++(0,-2) node[ground]{};
    \draw (Q.C) to[R, l=$R_C$] ++(0,2) node[vcc]{$+V_{CC}$};
    \draw (Q.B) to[R, l=$R_2$] ++(0,-2) node[ground]{};
    \draw (Q.B) to[R, l=$R_1$] ++(0,2) node[vcc]{$+V_{CC}$};
    
    % Tank
    \draw (Q.C) to[C, l=$C_{out}$] (3,0.77);
    \draw (3,0.77) to[L, l=$L$] (3,-2) node[ground]{};
    \draw (3,0.77) -- (5,0.77) to[C, l=$C_1$] (5,-0.6) to[C, l=$C_2$] (5,-2) node[ground]{};
    \draw (5,-0.6) -- (5,-0.6) to[short] (5,-1) -- (-2,-1) -- (-2,0) -- (Q.B); % Feedback
\end{circuitikz}
\end{center}

\begin{itemize}
    \item \textbf{ફીડબેક}: કેપેસિટિવ વોલ્ટેજ ડિવાઈડર ($C_1$, $C_2$) દ્વારા પ્રદાન કરવામાં આવે છે
    \item \textbf{રેઝોનન્ટ ફ્રિક્વન્સી}: $f = 1/(2\pi\sqrt{L \times C})$, જ્યાં $C = (C_1 \times C_2)/(C_1+C_2)$
    \item \textbf{ઓસિલેશન}: રિજનરેટિવ ફીડબેક દ્વારા જાળવી રાખે છે
    \item \textbf{ફેઝ શિફ્ટ}: લૂપની આસપાસ 360\degree
\end{itemize}

\begin{mnemonicbox}
"CFPO" (Capacitive Feedback Produces Oscillations)
\end{mnemonicbox}
\end{solutionbox}

\questionmarks{2(c)}{7}{ટુ સ્ટેજ RC કપલ્ડ એમ્પ્લીફાયરનો ફ્રિક્વન્સી રિસ્પોન્સ સર્કિટ ડાયાગ્રામ સાથે સમજાવો.}

\begin{solutionbox}
બે-સ્ટેજ RC કપલ્ડ એમ્પ્લિફાયર બે એમ્પ્લિફાયર સ્ટેજને RC કપલિંગ સાથે જોડે છે.

\begin{center}
\begin{circuitikz}[american, scale=0.8, transform shape]
    % Stage 1
    \node [npn] (Q1) at (0,0) {$Q_1$};
    \draw (Q1.E) to[R, l=$R_{E1}$] (0,-2) node[ground]{};
    \draw (Q1.C) to[R, l=$R_{C1}$] (0,2) node[vcc](VCC){$+V_{CC}$};
    \draw (Q1.B) to[short] (-1,0); 
    \draw (-1,0) to[R, l=$R_{2}$] (-1,-2) node[ground]{};
    \draw (-1,0) to[R, l=$R_{1}$] (-1,2) node[vcc]{};
    \draw (-1,0) to[C, l=$C_{in}$] (-3,0) node[left]{$V_{in}$};
    
    % Coupling
    \draw (Q1.C) to[C, l=$C_C$] (2,0);
    
    % Stage 2
    \node [npn] (Q2) at (3,0) {$Q_2$};
    \draw (Q2.E) to[R, l=$R_{E2}$] (3,-2) node[ground]{};
    \draw (Q2.C) to[R, l=$R_{C2}$] (3,2) node[vcc]{};
    \draw (Q2.B) to[short] (2,0);
    \draw (2,0) to[R, l=$R_{4}$] (2,-2) node[ground]{};
    \draw (2,0) to[R, l=$R_{3}$] (2,2) node[vcc]{};
    \draw (Q2.C) to[C, l=$C_{out}$] (5,0) node[right]{$V_{out}$};
\end{circuitikz}
\end{center}

\textbf{ફ્રિક્વન્સી રિસ્પોન્સ:}
\begin{center}
\begin{tikzpicture}[scale=0.7]
    \draw[->] (0,0) -- (6,0) node[right] {Frequency ($f$)};
    \draw[->] (0,0) -- (0,4) node[above] {Gain ($A_v$)};
    
    \draw[thick] (0.5,1) to[out=80,in=180] (2,3) to[short] (4,3) to[out=0,in=100] (5.5,1);
    
    \draw[dashed] (2,0) -- (2,3);
    \draw[dashed] (4,0) -- (4,3);
    
    \node at (1,1.5) {Low Freq};
    \node at (3,2.5) {Mid Freq};
    \node at (5,1.5) {High Freq};
\end{tikzpicture}
\end{center}

\begin{itemize}
    \item \textbf{લો ફ્રિક્વન્સી}: કપલિંગ કેપેસિટર ઇમ્પિડન્સને કારણે ગેઇન ઘટે છે
    \item \textbf{મિડ ફ્રિક્વન્સી}: મહત્તમ ફ્લેટ ગેઇન રીજિયન (બેન્ડવિડ્થ)
    \item \textbf{હાઇ ફ્રિક્વન્સી}: ટ્રાન્ઝિસ્ટર કેપેસિટન્સ ઇફેક્ટ્સને કારણે ગેઇન ઘટે છે
    \item \textbf{ઓવરઓલ ગેઇન}: વ્યક્તિગત સ્ટેજ ગેઇનનો ગુણાકાર
\end{itemize}

\begin{mnemonicbox}
"LMH" (Low drops, Mid flat, High drops)
\end{mnemonicbox}
\end{solutionbox}

\vspace{0.5em}\centerline{\textbf{OR}}\questionmarks{2(a)}{3}{હાર્ટલી ઓસિલેટરનું સર્કિટ ડાયાગ્રામ દોરો.}

\begin{solutionbox}
\begin{center}
\captionof{figure}{હાર્ટલી ઓસિલેટર}
\begin{circuitikz}[american, scale=0.9, transform shape]
    \node [npn] (Q) at (0,0) {};
    \draw (Q.E) to[R, l=$R_E$] ++(0,-2) node[ground]{};
    \draw (Q.C) to[R, l=$R_C$] ++(0,2) node[vcc]{$+V_{CC}$};
    \draw (Q.B) to[R, l=$R_2$] ++(0,-2) node[ground]{};
    \draw (Q.B) to[R, l=$R_1$] ++(0,2) node[vcc]{};
    
    % Tank
    \draw (Q.C) to[C, l=$C_{out}$] (3,0.77);
    \draw (3,0.77) to[C, l=$C$] (3,-2) node[ground]{};
    \draw (3,0.77) -- (5,0.77) to[L, l=$L_1$] (5,-0.6) to[L, l=$L_2$] (5,-2) node[ground]{};
    \draw (5,-0.6) -- (5,-0.6) to[short] (5,-1) -- (-2,-1) -- (-2,0) -- (Q.B); % Feedback
\end{circuitikz}
\end{center}

\begin{mnemonicbox}
"ITLC" (Inductor Tapped for LC Circuit)
\end{mnemonicbox}
\end{solutionbox}

\questionmarks{2(b)}{4}{વિવિધ પ્રકારના નેગેટીવ ફીડબેકનું લિસ્ટ બનાવો.}

\begin{solutionbox}
\begin{center}
\captionof{table}{નેગેટિવ ફીડબેકના પ્રકારો}
\begin{tabulary}{\linewidth}{|L|L|L|}
\hline
\textbf{પ્રકાર} & \textbf{કન્ફિગરેશન} & \textbf{પેરામીટર્સ પર અસર} \\
\hline
\textbf{વોલ્ટેજ સીરીઝ} & આઉટપુટ વોલ્ટેજ ઇનપુટમાં સીરીઝમાં ફીડ થાય છે & ઇનપુટ ઇમ્પેડન્સમાં વધારો, ડિસ્ટોર્શનમાં ઘટાડો \\
\hline
\textbf{વોલ્ટેજ શન્ટ} & આઉટપુટ વોલ્ટેજ ઇનપુટમાં પેરેલલમાં ફીડ થાય છે & ઇનપુટ ઇમ્પેડન્સમાં ઘટાડો, બેન્ડવિડ્થમાં વધારો \\
\hline
\textbf{કરંટ સીરીઝ} & આઉટપુટ કરંટ ઇનપુટમાં સીરીઝમાં ફીડ થાય છે & આઉટપુટ ઇમ્પેડન્સમાં વધારો, કરંટ ગેઇનને સ્થિર કરે છે \\
\hline
\textbf{કરંટ શન્ટ} & આઉટપુટ કરંટ ઇનપુટમાં પેરેલલમાં ફીડ થાય છે & આઉટપુટ ઇમ્પેડન્સમાં ઘટાડો, વોલ્ટેજ ગેઇનને સ્થિર કરે છે \\
\hline
\end{tabulary}
\end{center}

\begin{mnemonicbox}
"VSCS" (Voltage Series, Current Shunt)
\end{mnemonicbox}
\end{solutionbox}

\questionmarks{2(c)}{7}{નેગેટિવ ફીડબેક એમ્પ્લીફાયરના ફાયદાઓની યાદી બનાવો અને વોલ્ટેજ સીરીઝ નેગેટિવ ફીડબેકને વિગતવાર સમજાવો.}

\begin{solutionbox}
\textbf{નેગેટિવ ફીડબેકના ફાયદાઓ:}
\begin{itemize}
    \item કોમ્પોનન્ટ વેરિએશન સામે ગેઇન સ્થિર કરે છે
    \item ડિસ્ટોર્શન અને નોઇઝમાં ઘટાડો
    \item બેન્ડવિડ્થમાં વધારો
    \item ઇનપુટ/આઉટપુટ ઇમ્પેડન્સમાં ફેરફાર કરે છે
    \item લિનિયારિટીમાં સુધારો
\end{itemize}

\textbf{વોલ્ટેજ સીરીઝ નેગેટિવ ફીડબેક:}

\begin{center}
\begin{tikzpicture}[gtu block]
    \node (sum) [draw, circle] {$\Sigma$};
    \node (amp) [draw, rectangle, right=1cm of sum] {Amplifier A};
    \node (fb) [draw, rectangle, below=1cm of amp] {Feedback Network $\beta$};
    
    \draw [gtu arrow] (sum) -- (amp);
    \draw [gtu arrow] (amp) -- (4,0) node[right] {$V_{out}$};
    \draw [gtu arrow] (3.5,0) |- (fb);
    \draw [gtu arrow] (fb) -| (sum);
    \draw [gtu arrow] (-1,0) node[left] {$V_{in}$} -- (sum);
\end{tikzpicture}
\end{center}

\begin{itemize}
    \item \textbf{કન્ફિગરેશન}: આઉટપુટ વોલ્ટેજ સેમ્પલ કરવામાં આવે છે, ઇનપુટમાં સીરીઝમાં ફીડ બેક કરવામાં આવે છે
    \item \textbf{ક્લોઝ્ડ-લૂપ ગેઇન}: $A_{CL} = A/(1+A\beta)$, જ્યાં A ઓપન-લૂપ ગેઇન છે અને $\beta$ ફીડબેક ફ્રેક્શન છે
    \item \textbf{ઇનપુટ ઇમ્પેડન્સ}: ફેક્ટર $(1+A\beta)$ દ્વારા વધે છે
    \item \textbf{આઉટપુટ ઇમ્પેડન્સ}: ફેક્ટર $(1+A\beta)$ દ્વારા ઘટે છે
\end{itemize}

\begin{mnemonicbox}
"SIGO" (Stable gain, Increased input impedance, Gain reduction, Output impedance reduction)
\end{mnemonicbox}
\end{solutionbox}
\questionmarks{3(a)}{3}{બે ટ્રાન્ઝિસ્ટર એનેલોજીનો ઉપયોગ કરીને SCRની સર્કિટ દોરો.}

\begin{solutionbox}
\textbf{SCRનું બે ટ્રાન્ઝિસ્ટર એનેલોજી:}

\begin{center}
\begin{circuitikz}[american, scale=0.8, transform shape]
    % PNP Transistor (Q1)
    \node [pnp, rotate=0] (Q1) at (0,2) {$Q_1$};
    % NPN Transistor (Q2)
    \node [npn, rotate=0] (Q2) at (2,0) {$Q_2$};
    
    % Connections
    \draw (Q1.E) to[short] (0,4) node[above] {Anode (A)};
    \draw (Q2.E) to[short] (2,-2) node[below] {Cathode (K)};
    
    % Collector Q1 to Base Q2
    \draw (Q1.C) -- (0,0) -- (Q2.B);
    
    % Base Q1 to Collector Q2
    \draw (Q1.B) -- (2,2) -- (Q2.C);
    
    \draw (Q2.B) to[short, -o] (-1,0) node[left] {Gate (G)};
\end{circuitikz}
\end{center}

\begin{mnemonicbox}
"PNPNPN" (PNP and NPN structure)
\end{mnemonicbox}
\end{solutionbox}

\questionmarks{3(b)}{4}{SCR ના નેચરલ કમ્યુટેશન સર્કિટ દોરી ને સમજાવો.}

\begin{solutionbox}
નેચરલ કમ્યુટેશન ત્યારે થાય છે જ્યારે SCR કરંટ કુદરતી રીતે હોલ્ડિંગ કરંટથી નીચે પડે છે.

\begin{center}
\begin{circuitikz}[american]
    \draw (0,0) to[sV, l=AC Source] (0,2) -- (2,2) to[thyristor, l=SCR] (4,2) -- (4,0) to[R, l=Load] (2,0) -- (0,0);
\end{circuitikz}
\end{center}

\textbf{કરંટ વેવફોર્મ:}
\begin{center}
\begin{tikzpicture}
    \draw[->] (0,0) -- (6,0) node[right] {Time};
    \draw[->] (0,-1.5) -- (0,1.5) node[above] {Current};
    \draw[thick] plot[domain=0:6, samples=100] (\x, {sin(\x*180)});
    % Highlight conduction
    \draw[thick, red] plot[domain=0.5:3.14, samples=50] (\x, {sin(\x*180)});
    \node at (2,1.2) {SCR ON};
    \node at (4,-1.2) {SCR OFF};
    \node at (1.57, -0.5) {Natural Zero Crossing};
\end{tikzpicture}
\end{center}

\begin{itemize}
    \item \textbf{વ્યાખ્યા}: કરંટ હોલ્ડિંગ કરંટથી નીચે પડે ત્યારે SCR આપોઆપ બંધ થાય છે
    \item \textbf{AC સર્કિટ}: દરેક પોઝિટિવ હાફ-સાયકલના અંતે કુદરતી રીતે થાય છે
    \item \textbf{ઝીરો ક્રોસિંગ}: AC વોલ્ટેજ શૂન્ય ક્રોસ કરે ત્યારે SCR બંધ થાય છે
    \item \textbf{કોઈ બાહ્ય સર્કિટ નથી}: ટર્ન-ઓફ માટે કોઈ વધારાના કોમ્પોનન્ટની જરૂર નથી
\end{itemize}

\begin{mnemonicbox}
"NAZC" (Natural At Zero Crossing)
\end{mnemonicbox}
\end{solutionbox}

\questionmarks{3(c)}{7}{ટ્રાયાકનો ઉપયોગ પંખાના રેગ્યુલેટર તરીકે અને એસી પાવર માટે ઓન-ઓફ કંટ્રોલ તરીકે કેવી રીતે થઈ શકે છે તે સમજાવો.}

\begin{solutionbox}
TRIAC એ બાયડાયરેક્શનલ ડિવાઇસ છે જે AC પાવર કંટ્રોલ એપ્લિકેશન માટે આદર્શ છે.

\textbf{TRIAC ફેન રેગ્યુલેટર સર્કિટ:}
\begin{center}
\begin{circuitikz}[american]
    \draw (0,0) to[sV, l=AC Source] (0,3) -- (2,3) to[L, l=Fan Motor] (4,3);
    \draw (4,3) to[triac, n=T, mirror] (4,0) -- (0,0);
    
    % Triggering
    \draw (2,3) -- (2,1.5) to[vR, l=$R$] (2,0.5) to[C, l=$C$] (4,0.5) -- (4,0);
    \draw (2,0.5) to[generic, l=DIAC] (T.G);
\end{circuitikz}
\end{center}

\textbf{TRIAC ઓન-ઓફ કંટ્રોલ:}
\begin{center}
\begin{circuitikz}[american]
    \draw (0,0) to[sV, l=AC] (0,3) -- (2,3) to[R, l=Load] (4,3);
    \draw (4,3) to[triac, n=T, mirror] (4,0) -- (0,0);
    \draw (2,3) -- (2,1.5) to[R] (2,1) to[switch, l=Select] (T.G);
\end{circuitikz}
\end{center}

\begin{itemize}
    \item \textbf{ફેન રેગ્યુલેશન}: ફેઝ કંટ્રોલ ટેકનિક ફેનમાં પાવર વેરી કરે છે
    \item \textbf{પોટેન્શિયોમીટર}: TRIACનો ફાયરિંગ એંગલ એડજસ્ટ કરે છે
    \item \textbf{ઓન-ઓફ કંટ્રોલ}: સરળ સ્વિચ TRIAC ગેટને ટ્રિગર કરે છે
    \item \textbf{બાયડાયરેક્શનલ}: બંને હાફ-સાયકલમાં કરંટ કંટ્રોલ કરે છે
\end{itemize}

\begin{mnemonicbox}
"FPOB" (Fan Power is controlled by Phase angle in both directions)
\end{mnemonicbox}
\end{solutionbox}

\vspace{0.5em}\centerline{\textbf{OR}}\questionmarks{3(a)}{3}{એસ.સી.આર, ડાયાક અને ટ્રાયાક ના સિમ્બોલ દોરો.}

\begin{solutionbox}
\begin{center}
\begin{circuitikz}[american, scale=1.2, transform shape]
    % SCR
    \draw (0,2) to[thyristor, n=scr] (0,0);
    \node at (0, 2.5) {\textbf{SCR}};
    \node at (0, 2) [left] {A};
    \node at (0, 0) [left] {K};
    \node at (scr.gate) [right] {G};

    % DIAC
    \node at (4, 2.5) {\textbf{DIAC}};
    \draw (4, 2) -- (4, 1.6);
    \draw (4, 0.4) -- (4, 0);
    \draw (3.6, 1.6) -- (4.4, 1.6) -- (4, 0.8) -- (3.6, 1.6); % Down
    \draw (3.6, 0.8) -- (4.4, 0.8) -- (4, 1.6) -- (3.6, 0.8); % Up
    \draw (3.6, 0.8) -- (4.4, 0.8); % Bottom line of Up triangle
    \node at (4, 2) [left] {A1};
    \node at (4, 0) [left] {A2};

    % TRIAC
    \draw (8,2) to[triac, n=triac] (8,0);
    \node at (8, 2.5) {\textbf{TRIAC}};
    \node at (8, 2) [left] {MT2};
    \node at (8, 0) [left] {MT1};
    \node at (triac.gate) [right] {G};
\end{circuitikz}
\end{center}

\begin{mnemonicbox}
"SDT" (SCR has gate on one side, DIAC has none, TRIAC has gate in middle)
\end{mnemonicbox}
\end{solutionbox}

\questionmarks{3(b)}{4}{એસ.સી.આર નુ ગેટ ટ્રીગરીંગ સર્કિટ દોરી ને સમજાવો.}

\begin{solutionbox}
ગેટ ટ્રિગરિંગ એ SCRને ચાલુ કરવાની સૌથી સામાન્ય પદ્ધતિ છે.

\begin{center}
\begin{circuitikz}[american]
    \draw (0,0) to[sV, l=AC] (0,3) -- (2,3) to[thyristor, l=SCR, n=scr] (4,3) -- (4,0) to[R, l=Load] (2,0) -- (0,0);
    
    % Gate trigger circuit
    \draw (scr.gate) -- (2.5, 2.25) to[switch, l=SW] (2.5, 1.5) to[battery1, l=DC] (2.5, 0.5) -- (2.5, 0) node[ground]{};
\end{circuitikz}
\end{center}

\begin{itemize}
    \item \textbf{સિદ્ધાંત}: ગેટ અને કેથોડ વચ્ચે પોઝિટિવ વોલ્ટેજ એપ્લાય કરવું
    \item \textbf{કરંટ જરૂરિયાત}: નાનો ગેટ કરંટ મોટા એનોડ કરંટને ટ્રિગર કરે છે
    \item \textbf{લેચિંગ}: એકવાર ટ્રિગર થયા પછી, ગેટ સિગ્નલ દૂર કરવામાં આવે તો પણ SCR ચાલુ રહે છે
    \item \textbf{ટર્ન-ઓફ}: એનોડ કરંટને હોલ્ડિંગ કરંટથી નીચે ઘટાડવાની જરૂર પડે છે
\end{itemize}

\begin{mnemonicbox}
"GPLT" (Gate Pulse Latches Thyristor)
\end{mnemonicbox}
\end{solutionbox}

\questionmarks{3(c)}{7}{SCRનું કંસ્ટ્રકશન અને V-I લાક્ષણિકતા દોરો અને V-I લાક્ષણિકતા સમજાવો.}

\begin{solutionbox}
SCR (સિલિકોન કંટ્રોલ્ડ રેક્ટિફાયર) એ ચાર-લેયર PNPN સેમિકન્ડક્ટર ડિવાઇસ છે.

\textbf{SCR કંસ્ટ્રકશન:}
\begin{center}
\begin{tikzpicture}
    % Layers
    \draw (0,0) rectangle (4,1) node[midway] {N (Cathode)};
    \draw (0,1) rectangle (4,2) node[midway] {P (Gate)};
    \draw (0,2) rectangle (4,3) node[midway] {N};
    \draw (0,3) rectangle (4,4) node[midway] {P (Anode)};
    
    % Terminals
    \draw (2,4) -- (2,4.5) node[above] {Anode};
    \draw (2,0) -- (2,-0.5) node[below] {Cathode};
    \draw (4,1.5) -- (4.5,1.5) node[right] {Gate};
\end{tikzpicture}
\end{center}

\textbf{V-I લાક્ષણિકતા:}
\begin{center}
\begin{tikzpicture}[scale=0.8]
    \draw[->] (-4,0) -- (4,0) node[right] {$V_{AK}$};
    \draw[->] (0,-4) -- (0,4) node[above] {$I_A$};
    
    % Forward Blocking
    \draw[thick, blue] (0,0) -- (2.5,0.2) node[right, font=\tiny] {Forward Blocking};
    % Breakover
    \draw[thick, blue] (2.5,0.2) -- (2,3.5);
    % Conduction
    \draw[thick, blue] (2,3.5) -- (0.5,3.5) -- (0.5,0) -- (0,0);
    \draw [blue, thick] (0,0) to[out=10, in=260] (3,1) -- (1,3) -- (1,4); 
    \node at (3.2, 1) [right] {$V_{BO}$};
    
    % Reverse
    \draw [red, thick] (0,0) -- (-3, -0.2) -- (-3, -3);
    \node at (-3, -0.2) [left] {$V_{BR}$};
    
    \node at (1, 2) {ON State};
\end{tikzpicture}
\end{center}

\begin{itemize}
    \item \textbf{ફોરવર્ડ બ્લોકિંગ રીજન}: બ્રેકઓવર વોલ્ટેજ સુધી SCR મિનિમલ કરંટ કન્ડક્ટ કરે છે
    \item \textbf{ફોરવર્ડ કન્ડક્શન રીજન}: ટ્રિગરિંગ પછી લો-રેઝિસ્ટન્સ સ્ટેટ
    \item \textbf{રિવર્સ બ્લોકિંગ રીજન}: રિવર્સ દિશામાં કરંટને બ્લોક કરે છે
    \item \textbf{ગેટ ટ્રિગરિંગ}: બ્રેકઓવર વોલ્ટેજને ઘટાડે છે, ટર્ન-ઓનને સરળ બનાવે છે
\end{itemize}

\begin{mnemonicbox}
"FBRH" (Forward Blocking, Reverse blocking, Holding current)
\end{mnemonicbox}
\end{solutionbox}
\questionmarks{4(a)}{3}{OP-AMP ને સમિંગ એમ્પ્લીફાયર તરીકે સમજાવો.}

\begin{solutionbox}
સમિંગ એમ્પ્લિફાયર વેઇટેડ ગેઇન સાથે મલ્ટિપલ ઇનપુટ સિગ્નલ્સ એડ કરે છે.

\begin{center}
\begin{circuitikz}[american, scale=1.0, transform shape]
    \node [op amp] (A) at (2,0) {};
    \draw (A.-) -- (0.5, 0.5) -- (0.5, 2);
    
    % Inputs
    \draw (0.5, 2) to[R, l=$R_1$] (-2, 2) node[left] {$V_1$};
    \draw (0.5, 1) to[R, l=$R_2$, *-] (-2, 1) node[left] {$V_2$};
    \draw (0.5, 0.5) to[short, *-] (0.5, 1);
    \draw (0.5, 0.5) -- (0.5, 0) to[R, l=$R_3$] (-2, 0) node[left] {$V_3$};
    
    \draw (A.+) -- (0.5, -0.5) to[short] (0.5,-1) node[ground] {};
    
    % Feedback
    \draw (A.-) -- (0.5, 0.5) -- (0.5, 3) to[R, l=$R_f$] (3.5, 3) -- (3.5, 0) -- (A.out);
    \draw (A.out) to[short, -o] (4,0) node[right] {$V_{out}$};
\end{circuitikz}
\end{center}

\begin{itemize}
    \item \textbf{ફંક્શન}: ઇનપુટ વોલ્ટેજનો વેઇટેડ સમ આઉટપુટ કરે છે
    \item \textbf{આઉટપુટ સમીકરણ}: $V_{out} = -(V_1 \cdot \frac{R_f}{R_1} + V_2 \cdot \frac{R_f}{R_2} + V_3 \cdot \frac{R_f}{R_3})$
    \item \textbf{સમાન ભાર}: જ્યારે $R_1 = R_2 = R_3$, આઉટપુટ સરળ સમ ગુણાકાર $-R_f/R$ છે
    \item \textbf{વર્ચ્યુઅલ ગ્રાઉન્ડ}: ઈન્વર્ટિંગ ઇનપુટ 0V પોટેન્શિયલ જાળવે છે
\end{itemize}

\begin{mnemonicbox}
"SWAP" (Sum Weighted And Proportional)
\end{mnemonicbox}
\end{solutionbox}

\questionmarks{4(b)}{4}{નીચેના OP-AMP પેરામીટરને વ્યાખ્યાયિત કરો: 1. ઇનપુટ બાયસ કરંટ 2. CMRR}

\begin{solutionbox}
\textbf{ઇનપુટ બાયસ કરંટ}:
જ્યારે આઉટપુટ શૂન્ય હોય ત્યારે બે ઇનપુટ ટર્મિનલમાં પ્રવાહિત થતા કરંટની સરેરાશ.

\textbf{CMRR (કોમન મોડ રિજેક્શન રેશિયો)}:
ડિફરેન્શિયલ ગેઇનનો કોમન-મોડ ગેઇન સાથેનો ગુણોત્તર, જે દર્શાવે છે કે ઓપ-એમ્પ બંને ઇનપુટ માટે સામાન્ય સિગ્નલને કેટલી સારી રીતે રિજેક્ટ કરે છે.

\begin{center}
\captionof{table}{ઓપ-એમ્પ પેરામીટર્સ}
\begin{tabulary}{\linewidth}{|L|L|L|}
\hline
\textbf{પેરામીટર} & \textbf{સામાન્ય મૂલ્ય} & \textbf{મહત્વ} \\
\hline
\textbf{ઇનપુટ બાયસ કરંટ} & 20-200 nA & હાઈ ઇમ્પિડન્સ સર્કિટ માટે ઓછું વધુ સારું \\
\hline
\textbf{CMRR} & 80-120 dB & નોઇઝ રિજેક્શન માટે વધુ સારું \\
\hline
\end{tabulary}
\end{center}

\begin{mnemonicbox}
"BIC-CMR" (Bias Is Current, Common Mode Rejection)
\end{mnemonicbox}
\end{solutionbox}

\questionmarks{4(c)}{7}{555 ટાઈમરનો ઉપયોગ કરીને મોનોસ્ટેબલ મલ્ટિવાઇબ્રેટર દોરો અને સમજાવો.}

\begin{solutionbox}
મોનોસ્ટેબલ મલ્ટીવાઇબ્રેટર ટ્રિગર થતાં પૂર્વનિર્ધારિત અવધિનો એક પલ્સ જનરેટ કરે છે.

\begin{center}
\begin{circuitikz}[american, scale=0.9, transform shape]
    % 555 Timer Block
    \draw (0,0) rectangle (4,4);
    \node at (2,2) {555 Timer};
    
    % Pins
    \node at (0,3.5) [right] {8 (VCC)}; \draw (0,3.5) -- (-1,3.5) node[vcc] {$V_{CC}$};
    \node at (0,0.5) [right] {1 (GND)}; \draw (0,0.5) -- (-1,0.5) node[ground] {};
    
    \node at (4,2.5) [left] {3 (OUT)}; \draw (4,2.5) -- (5,2.5) node[right] {$V_{out}$};
    
    \node at (0,2.5) [right] {2 (TRIG)}; \draw (0,2.5) -- (-1,2.5) node[left] {Trigger};
    \node at (0,1.5) [right] {4 (RST)}; \draw (0,1.5) -- (-1,1.5) node[heading] {to VCC};
    
    % External components
    \draw (4,3.5) -- (5,3.5); % Pin 7
    \draw (4,1.5) [left] node {6 (THR)}; \draw (4,1.5) -- (5,1.5) -- (5,3.5); % Pin 6 connected to 7
    \draw (5,3.5) to[R, l=$R$] (5,5.5) node[vcc] {$V_{CC}$};
    \draw (5,1.5) to[C, l=$C$] (5,-1) node[ground] {};
\end{circuitikz}
\end{center}

\textbf{આઉટપુટ વેવફોર્મ:}
\begin{center}
\begin{tikzpicture}[scale=0.8]
    \draw[->] (0,0) -- (6,0) node[right] {Time};
    
    % Trigger
    \draw[blue] (0,3) -- (1,3) -- (1,1) -- (1.5,1) -- (1.5,3) -- (6,3); 
    \node[blue] at (-0.5, 3) {Trig};
    
    % Output
    \draw[red, thick] (0,0) -- (1,0) -- (1,2) -- (4,2) -- (4,0) -- (6,0);
    \node[red] at (-0.5, 1) {Out};
    
    % Dimension
    \draw[<->] (1, -0.5) -- (4, -0.5) node[midway, below] {$T = 1.1 RC$};
\end{tikzpicture}
\end{center}

\begin{itemize}
    \item \textbf{ઓપરેશન}: સિંગલ સ્ટેબલ સ્ટેટ (આઉટપુટ LOW), ટ્રિગર થતાં અસ્થાયી રૂપે HIGH
    \item \textbf{પલ્સ વિડ્થ}: $T = 1.1 \times R \times C$ (સેકન્ડ)
    \item \textbf{ટ્રિગરિંગ}: TRIG પિન (પિન 2) પર ફોલિંગ એજ
    \item \textbf{ટાઇમિંગ કોમ્પોનન્ટ્સ}: $R$ અને $C$ પલ્સ અવધિ નક્કી કરે છે
\end{itemize}

\begin{mnemonicbox}
"POST" (Pulse Output, Single Trigger)
\end{mnemonicbox}
\end{solutionbox}

\vspace{0.5em}\centerline{\textbf{OR}}\questionmarks{4(a)}{3}{OP-AMP ના ઇન્વર્ટિંગ એમ્પ્લીફાયરનો સર્કિટ ડાયાગ્રામને દોરો.}

\begin{solutionbox}
\begin{center}
\begin{circuitikz}[american, scale=1.0, transform shape]
    \node [op amp] (A) at (0,0) {};
    \draw (A.+) -- (-1, -0.5) node[ground] {};
    \draw (A.-) -- (-1, 0.5) to[R, l=$R_1$] (-3, 0.5) node[left] {$V_{in}$};
    \draw (A.-) -- (-1, 0.5) -- (-1, 1.5) to[R, l=$R_f$] (1, 1.5) -- (1, 0) -- (A.out);
    \draw (A.out) to[short, -o] (2,0) node[right] {$V_{out}$};
\end{circuitikz}
\end{center}

\begin{mnemonicbox}
"IRON" (Inverting Requires One Negative input)
\end{mnemonicbox}
\end{solutionbox}

\questionmarks{4(b)}{4}{નીચેના OP-AMP પેરામીટરને વ્યાખ્યાયિત કરો: 1. ઇનપુટ ઓફસેટ કરંટ 2. સ્લ્યુ રેટ}

\begin{solutionbox}
\textbf{ઇનપુટ ઓફસેટ કરંટ}:
બે ઇનપુટ ટર્મિનલમાં પ્રવાહિત થતા કરંટ વચ્ચેનો તફાવત.

\textbf{સ્લ્યુ રેટ}:
આઉટપુટ વોલ્ટેજનો સમય પ્રતિ એકમ મહત્તમ ફેરફાર દર, સામાન્ય રીતે $V/\mu s$ માં માપવામાં આવે છે.

\begin{center}
\captionof{table}{ઓપ-એમ્પ પેરામીટર્સ}
\begin{tabulary}{\linewidth}{|L|L|L|}
\hline
\textbf{પેરામીટર} & \textbf{સામાન્ય મૂલ્ય} & \textbf{મહત્વ} \\
\hline
\textbf{ઇનપુટ ઓફસેટ કરંટ} & 2-50 nA & પ્રિસિઝન એપ્લિકેશન માટે ઓછું વધુ સારું \\
\hline
\textbf{સ્લ્યુ રેટ} & 0.5-20 V/$\mu$s & હાઈ-ફ્રિક્વન્સી ઓપરેશન માટે વધુ સારું \\
\hline
\end{tabulary}
\end{center}

\begin{mnemonicbox}
"IOSR" (Input Offset and Slew Rate)
\end{mnemonicbox}
\end{solutionbox}

\questionmarks{4(c)}{7}{ઑપ-એમ્પને ઇન્વર્ટિંગ એમ્પ્લીફાયર તરીકે સમજાવો અને તેના વોલ્ટેજ ગેઇનનું સમીકરણ મેળવો.}

\begin{solutionbox}
ઇન્વર્ટિંગ એમ્પ્લિફાયર એક ઇન્વર્ટેડ અને એમ્પ્લિફાઇડ આઉટપુટ સિગ્નલ ઉત્પન્ન કરે છે.

\begin{center}
\begin{circuitikz}[american, scale=1.0, transform shape]
    \node [op amp] (A) at (0,0) {};
    \draw (A.+) -- (-1, -0.5) node[ground] {};
    \draw (A.-) -- (-1, 0.5) to[R, l=$R_1$] (-3, 0.5) node[left] {$V_{in}$};
    \draw (A.-) -- (-1, 0.5) -- (-1, 1.5) to[R, l=$R_f$] (1, 1.5) -- (1, 0) -- (A.out);
    \draw (A.out) to[short, -o] (2,0) node[right] {$V_{out}$};
\end{circuitikz}
\end{center}

\textbf{વોલ્ટેજ ગેઇન ડેરિવેશન:}

નોડ N (ઇન્વર્ટિંગ ઇનપુટ) પર:
$$ I_1 + I_f = 0 \quad (\text{કિરકોફનો કરંટ લો દ્વારા}) $$
$$ \frac{V_{in} - V_N}{R_1} + \frac{V_{out} - V_N}{R_f} = 0 $$

જ્યારે $V_N \approx 0$ (વર્ચ્યુઅલ ગ્રાઉન્ડ):
$$ \frac{V_{in}}{R_1} + \frac{V_{out}}{R_f} = 0 $$
$$ \frac{V_{out}}{V_{in}} = -\frac{R_f}{R_1} $$

\begin{itemize}
    \item \textbf{ગેઇન સમીકરણ}: $V_{out}/V_{in} = -R_f/R_1$
    \item \textbf{વર્ચ્યુઅલ ગ્રાઉન્ડ}: ઇન્વર્ટિંગ ટર્મિનલ 0V પર જાળવવામાં આવે છે
    \item \textbf{ઇનપુટ ઇમ્પિડન્સ}: $R_1$ ને સમાન
    \item \textbf{નેગેટિવ ફીડબેક}: સ્થિરતા અને લિનિયારિટી પ્રદાન કરે છે
\end{itemize}

\begin{mnemonicbox}
"GIVN" (Gain Is Negative, Virtual ground)
\end{mnemonicbox}
\end{solutionbox}
\questionmarks{5(a)}{3}{IC 555 નો બ્લોક ડાયાગ્રામ દોરો.}

\begin{solutionbox}
\textbf{IC 555નો બ્લોક ડાયાગ્રામ:}

\begin{center}
\begin{tikzpicture}[gtu block]
    % Components:
    % Voltage Divider R-R-R
    \node (vcc) at (0,6) {$V_{CC} (8)$};
    \node [draw, rectangle] (r1) at (0,4.5) {5k$\Omega$};
    \node [draw, rectangle] (r2) at (0,3) {5k$\Omega$};
    \node [draw, rectangle] (r3) at (0,1.5) {5k$\Omega$};
    \node (gnd) at (0,0) {GND (1)};
    
    \draw (vcc) -- (r1) -- (r2) -- (r3) -- (gnd);
    
    % Comparators
    \node [draw, regular polygon, regular polygon sides=3, rotate=-90, inner sep=1pt] (comp1) at (3,4) {UC}; % Upper Comparator
    \node [draw, regular polygon, regular polygon sides=3, rotate=-90, inner sep=1pt] (comp2) at (3,2) {LC}; % Lower Comparator
    
    % Connections to Comparators
    \draw (0,4) -- (comp1.south) node[above left] {-};
    \draw (0,2) -- (comp2.north) node[below left] {+};
    
    % Pins
    \node (thr) at (-2, 4) {Threshold (6)};
    \draw (thr) -- (comp1.north) node[above left] {+};
    
    \node (trig) at (-2, 2) {Trigger (2)};
    \draw (trig) -- (comp2.south) node[below left] {-};
    
    \node (ctrl) at (2, 5) {Control (5)};
    \draw (ctrl) |- (0, 4);
    
    % Flip Flop
    \node [draw, rectangle, minimum width=2cm, minimum height=1.5cm] (ff) at (6,3) {SR Flip-Flop};
    \draw (comp1.east) -- (ff.140); % Upper input (R)
    \draw (comp2.east) -- (ff.220); % Lower input (S)
    
    % Output Stage
    \node [draw, triangular amplifier] (inv) at (9,3) {Buffer};
    \draw (ff.east) -- (inv.west);
    \draw (inv.east) -- (11,3) node[right] {Output (3)};
    
    % Discharge Transistor
    \node [npn, rotate=0] (Q1) at (6, 0.5) {};
    \draw (Q1.E) -- (6,0) node[ground] {};
    \draw (Q1.C) -- (8,0.5) node[right] {Discharge (7)};
    \draw (Q1.B) -| (ff.300);
    
    % Reset
    \node (rst) at (6, 5) {Reset (4)};
    \draw (rst) -- (ff.north);
\end{tikzpicture}
\end{center}

\begin{mnemonicbox}
"CVOT" (Comparators, Voltage divider, Output stage, Timer)
\end{mnemonicbox}
\end{solutionbox}

\questionmarks{5(b)}{4}{વેઈન બ્રિજ ઓસીલેટર તરીકે OP-AMPનો સર્કિટ ડાયાગ્રામ દોરો.}

\begin{solutionbox}
\textbf{વેઈન બ્રિજ ઓસીલેટર સર્કિટ:}

\begin{center}
\begin{circuitikz}[american, scale=1.0, transform shape]
    \node [op amp] (A) at (0,0) {};
    
    \draw (A.out) to[short, -o] (2,0) node[right] {$V_{out}$};
    
    % Positive Feedback (Lead-Lag)
    \draw (2,0) -- (2, 2) -- (-2, 2) to[C, l=$C$] (-4, 2) to[R, l=$R$] (-4, 0) -- (A.+); % Series arm
    \draw (-4, 0) to[R, l=$R$] (-4, -2) node[ground] {}; % Parallel R
    \draw (-2.5, 0) to[C, l=$C$] (-2.5, -2) node[ground] {}; % Parallel C
    \draw (-4,0) -- (-2.5,0);
    
    % Negative Feedback (Gain setting)
    \draw (A.-) -- (-1, 0.5) to[R, l=$R_3$] (-1, 2.5) -- (1, 2.5) -- (1,0); 
    \draw (A.-) -- (-1, 0.5) to[R, l=$R_4$] (-1, -1.5) node[ground] {};
    
\end{circuitikz}
\end{center}

\begin{mnemonicbox}
"WPRC" (Wein Produces Resonant Circuit)
\end{mnemonicbox}
\end{solutionbox}

\questionmarks{5(c)}{7}{વિવિધ પ્રકારના ફિક્સ્ડ અને વેરિયેબલ વોલ્ટેજ રેગ્યુલેટર IC ની કામગીરી સમજાવો.}

\begin{solutionbox}
વોલ્ટેજ રેગ્યુલેટર IC ઇનપુટ અથવા લોડ વેરિએશન છતાં સ્થિર આઉટપુટ વોલ્ટેજ જાળવે છે.

\textbf{ફિક્સ્ડ વોલ્ટેજ રેગ્યુલેટર:}
\begin{center}
\begin{circuitikz}[american, scale=0.9, transform shape]
    \draw (0,0) rectangle (3,2);
    \node at (1.5,1) {78XX};
    \draw (-1,1.5) to[short, o-] (0,1.5) node[above left] {IN};
    \draw (3,1.5) to[short, -o] (4,1.5) node[above right] {OUT};
    \draw (1.5,0) -- (1.5,-1) node[ground] {} node[right] {GND};
    
    \draw (-1,1.5) to[C, l=$C_{in}$] (-1,0) node[ground] {};
    \draw (4,1.5) to[C, l=$C_{out}$] (4,0) node[ground] {};
\end{circuitikz}
\end{center}

\textbf{વેરિએબલ વોલ્ટેજ રેગ્યુલેટર:}
\begin{center}
\begin{circuitikz}[american, scale=0.9, transform shape]
    \draw (0,0) rectangle (3,2);
    \node at (1.5,1) {LM317};
    \draw (-1,1.5) to[short, o-] (0,1.5) node[above left] {IN};
    \draw (3,1.5) to[short, -o] (4,1.5) node[above right] {OUT};
    \draw (1.5,0) -- (1.5,-0.5) node[right] {ADJ};
    
    \draw (-1,1.5) to[C, l=$C_{in}$] (-1,0) node[ground] {};
    
    \draw (3,1.5) -- (3.5,1.5) to[R, l=$R_1$] (3.5,-0.5) -- (1.5,-0.5);
    \draw (1.5,-0.5) to[R, l=$R_2$] (1.5,-2.5) node[ground] {};
    \draw (4,1.5) to[C, l=$C_{out}$] (4,0) node[ground] {};
\end{circuitikz}
\end{center}

\begin{itemize}
    \item \textbf{ફિક્સ્ડ રેગ્યુલેટર}: 78XX (પોઝિટિવ) અને 79XX (નેગેટિવ) સીરીઝ ચોક્કસ વોલ્ટેજ પ્રદાન કરે છે
    \item \textbf{વેરિએબલ રેગ્યુલેટર}: LM317 (પોઝિટિવ) અને LM337 (નેગેટિવ) એડજસ્ટેબલ આઉટપુટની મંજૂરી આપે છે
    \item \textbf{થ્રી-ટર્મિનલ ડિઝાઇન}: ઇનપુટ, આઉટપુટ અને ગ્રાઉન્ડ/એડજસ્ટ ટર્મિનલ
    \item \textbf{LM317 માટે આઉટપુટ સમીકરણ}: $V_{out} = 1.25V \times (1 + R_2/R_1)$
    \item \textbf{પ્રોટેક્શન ફીચર્સ}: શોર્ટ સર્કિટ, થર્મલ ઓવરલોડ અને સેફ એરિયા પ્રોટેક્શન
\end{itemize}

\begin{mnemonicbox}
"FAVOR" (Fixed And Variable Output Regulators)
\end{mnemonicbox}
\end{solutionbox}

\vspace{0.5em}\centerline{\textbf{OR}}\questionmarks{5(a)}{3}{555 ટાઈમરનો ઉપયોગ કરીને એસ્ટેબલ મલ્ટિવાઈબ્રેટરનો બ્લોક ડાયાગ્રામ દોરો.}

\begin{solutionbox}
\textbf{એસ્ટેબલ મલ્ટિવાઇબ્રેટર બ્લોક ડાયાગ્રામ:}

\begin{center}
\begin{tikzpicture}[gtu block]
    % 555 Box
    \draw [dashed] (-1,-1) rectangle (5,5);
    \node at (4.5,4.5) {555};
    
    \node [draw, rectangle, minimum width=2cm, minimum height=3cm] (ic) at (2,2) {555 Timer};
    
    % Pins
    \node (vcc) at (2,4.5) {8 (VCC)};
    \draw (ic.north) -- (vcc);
    
    \node (gnd) at (2,-0.5) {1 (GND)};
    \draw (ic.south) -- (gnd);
    
    \draw (ic.west) ++(0, 1) -- ++(-1,0) node[left] {7 (DIS)};
    \draw (ic.west) ++(0, 0) -- ++(-1,0) node[left] {6 (THR)};
    \draw (ic.west) ++(0, -1) -- ++(-1,0) node[left] {2 (TRIG)};
    
    \draw (ic.east) -- ++(1,0) node[right] {3 (OUT)};
    
    % External Circuit
    \draw (-3, 5) node[vcc] {$V_{CC}$} to[R, l=$R_1$] (-3, 3) -- (1, 3); % To DIS (7)
    \draw (-3, 3) to[R, l=$R_2$] (-3, 1) -- (1, 1); % To THR (6)
    \draw (-3, 1) -- (-3, 0) -- (1, 0); % To TRIG (2) shorted to 6
    \draw (-3, 0) to[C, l=$C$] (-3, -2) node[ground] {};
    
    \draw (1,3) -- (0.8,3);
    \draw (1,1) -- (0.8,1);
    \draw (1,0) -- (0.8,0);
\end{tikzpicture}
\end{center}

\begin{mnemonicbox}
"FOFT" (Free-running Oscillator From Timer)
\end{mnemonicbox}
\end{solutionbox}

\questionmarks{5(b)}{4}{સૌર આધારિત બેટરી ચાર્જર સર્કિટ દોરો અને સમજાવો.}

\begin{solutionbox}
સોલર બેટરી ચાર્જર સૂર્ય ઊર્જાને બેટરી ચાર્જ કરવા માટે રૂપાંતરિત કરે છે.

\begin{center}
\begin{circuitikz}[american, scale=1.0, transform shape]
    \draw (0,0) to[pvsource, l=Solar Panel] (0,3);
    \draw (0,3) to[D, l=Blocking Diode] (3,3) to[generic, l=Regulator IC] (5,3);
    \draw (5,3) to[battery, l=Battery] (5,0) -- (0,0);
    
    \draw (4,3) to[led, l=LED] (4,0) node[ground] {};
\end{circuitikz}
\end{center}

\begin{itemize}
    \item \textbf{સોલર પેનલ}: સૂર્યપ્રકાશને DC વીજળીમાં રૂપાંતરિત કરે છે
    \item \textbf{બ્લોકિંગ ડાયોડ}: રાત્રે પેનલ દ્વારા બેટરી ડિસ્ચાર્જને અટકાવે છે
    \item \textbf{રેગ્યુલેટર IC}: ચાર્જિંગ વોલ્ટેજ અને કરંટને નિયંત્રિત કરે છે
    \item \textbf{ચાર્જ ઇન્ડિકેટર}: ચાર્જિંગની સ્થિતિ દર્શાવે છે
    \item \textbf{પ્રોટેક્શન}: ઓવરચાર્જ અને રિવર્સ પોલારિટી પ્રોટેક્શન
\end{itemize}

\begin{mnemonicbox}
"SBRCP" (Solar, Blocking diode, Regulator, Charging, Protection)
\end{mnemonicbox}
\end{solutionbox}

\questionmarks{5(c)}{7}{SMPS ના બ્લોક ડાયાગ્રામ દોરો અને સમજાવો}

\begin{solutionbox}
SMPS (સ્વિચ મોડ પાવર સપ્લાય) સ્વિચિંગ રેગ્યુલેટર્સનો ઉપયોગ કરીને વીજળી શક્તિને કુશળતાથી રૂપાંતરિત કરે છે.

\begin{center}
\begin{tikzpicture}[gtu block]
    \node (ac) [draw, rectangle] {AC Input};
    \node (rect1) [draw, rectangle, right=0.5cm of ac] {Input Rectifier \& Filter};
    \node (switch) [draw, rectangle, right=0.5cm of rect1] {High Freq Switch};
    \node (trans) [draw, rectangle, right=0.5cm of switch] {Power Transformer};
    \node (rect2) [draw, rectangle, below=1cm of trans] {Output Rectifier \& Filter};
    \node (load) [draw, rectangle, left=0.5cm of rect2] {Load};
    \node (ctrl) [draw, rectangle, left=0.5cm of load] {Control Circuit};
    
    \draw [gtu arrow] (ac) -- (rect1);
    \draw [gtu arrow] (rect1) -- (switch);
    \draw [gtu arrow] (switch) -- (trans);
    \draw [gtu arrow] (trans) -- (rect2);
    \draw [gtu arrow] (rect2) -- (load);
    \draw [gtu arrow] (load) -- (ctrl);
    \draw [gtu arrow] (ctrl) -| (switch);
\end{tikzpicture}
\end{center}

\begin{itemize}
    \item \textbf{Input Stage}: AC મેઇન્સને હાઇ વોલ્ટેજ DC માં રેક્ટિફાય અને ફિલ્ટર કરે છે
    \item \textbf{Switching}: DC ને હાઇ-ફ્રિક્વન્સી AC પલ્સ ટ્રેનમાં ચોપ કરે છે
    \item \textbf{Transformer}: વોલ્ટેજ સ્ટેપ ડાઉન/અપ કરે છે અને આઇસોલેશન પ્રદાન કરે છે
    \item \textbf{Output Stage}: હાઇ-ફ્રિક્વન્સી AC ને ફ્લેટ DC માં રેક્ટિફાય અને ફિલ્ટર કરે છે
    \item \textbf{Feedback}: PWM કંટ્રોલ માટે આઉટપુટને રેફરન્સ સાથે સરખાવે છે
\end{itemize}

\begin{mnemonicbox}
"IRS-TOF" (Input, Rectifier, Switching, Transformer, Output, Feedback)
\end{mnemonicbox}
\end{solutionbox}

\end{document}
