\documentclass{article}

% content/resources/templates/preamble.tex
\usepackage[margin=0.6in]{geometry}
\author{Milav Dabgar}
\usepackage{amsmath,amssymb,amsthm}
\usepackage{booktabs}
\usepackage{multirow}
\usepackage{xcolor}
\usepackage{tcolorbox}
\tcbuselibrary{breakable,skins}
\usepackage[colorlinks=true,linkcolor=blue]{hyperref}
\usepackage{titlesec}
\usepackage{enumitem}
\usepackage{tikz}
\usepackage{pgfplots}
\usepackage{circuitikz}
\usepackage[version=4]{mhchem}
\usepackage{longtable}
\usepackage{array}
\usepackage{float}
\usepackage{caption}
\usepackage{listings}

\lstset{
  basicstyle=\small\ttfamily,
  breaklines=true,
  breakatwhitespace=false,
  postbreak=\mbox{\textcolor{red}{$\hookrightarrow$}\space},
  float=false,
  numbers=left,
  numberstyle=\tiny\color{gray},
  numbersep=10pt,
  xleftmargin=2em,
  keywordstyle=\color{blue},
  commentstyle=\color{green!60!black},
  stringstyle=\color{purple},
  backgroundcolor=\color{gray!5},
  showstringspaces=false,
  tabsize=2,
  captionpos=b,
  keepspaces=true,
  columns=flexible
}

\pgfplotsset{compat=1.18}
\usetikzlibrary{shapes,arrows,positioning,calc,patterns,decorations.pathmorphing,decorations.markings,arrows.meta}

% Color scheme
\definecolor{headcolor}{RGB}{0,102,204}
\definecolor{keycolor}{RGB}{220,20,60}
\definecolor{solutioncolor}{RGB}{34,139,34}
\definecolor{mnemoniccolor}{RGB}{148,0,211}
\definecolor{codecolor}{RGB}{0,0,100}

% Spacing
\setlength{\parskip}{3pt}
\setlist[itemize]{nosep}
\setlist[enumerate]{nosep}

% Title formatting
\titleformat{\section}{\Large\bfseries\color{headcolor}}{\thesection}{1em}{}
\titleformat{\subsection}{\large\bfseries\color{headcolor}}{\thesubsection}{1em}{}

% Pandoc tightlist compatibility
\providecommand{\tightlist}{%
  \setlength{\itemsep}{0pt}\setlength{\parskip}{0pt}}

% Pandoc longtable compatibility
\newcounter{none}
\def\thenone{}


% content/resources/templates/gujarati-boxes.tex
\usepackage{fontspec}
\usepackage{polyglossia}

% Set Gujarati as main language (document is primarily in Gujarati)
% Note: gloss-gujarati.ldf doesn't exist in polyglossia, but it will use hyphenation patterns
\setdefaultlanguage{gujarati}
\setotherlanguage{english}

% Configure Gujarati font properly
% Use Language=Default to prevent polyglossia from trying to add language-specific features
% that don't exist for Gujarati, which causes "empty feature" warnings
\newfontfamily\gujaratifont[Script=Gujarati,AutoFakeBold=2.5,AutoFakeSlant=0.3]{Noto Sans Gujarati}
\setmainfont[Script=Gujarati,AutoFakeBold=2.5,AutoFakeSlant=0.3]{Noto Sans Gujarati}
% Use Noto Sans Gujarati for monospace to support Gujarati in text
\setmonofont[Scale=0.9]{Noto Sans Gujarati}

% Configure English to use the same font
\newfontfamily\englishfont[Script=Gujarati,AutoFakeBold=2.5,AutoFakeSlant=0.3]{Noto Sans Gujarati}

% Translations for polyglossia
\gappto\captionsgujarati{
  \renewcommand{\tablename}{કોષ્ટક}
  \renewcommand{\figurename}{આકૃતિ}
}

% Helper for TikZ nodes to ensure Gujarati font
\newcommand{\gu}[1]{{\gujaratifont #1}}

% Custom environments
\newtcolorbox{solutionbox}{
    breakable,
    enhanced,
    colback=solutioncolor!5!white,
    colframe=solutioncolor!75!black,
    fonttitle=\bfseries,
    title=જવાબ
}

\newtcolorbox{solutionboxnobreak}{
 colback=solutioncolor!5!white,
 colframe=solutioncolor!75!black,
 fonttitle=\bfseries,
 title=જવાબ
}

\newtcolorbox{keyformula}{
 breakable,
 enhanced,
 colback=keycolor!5!white,
 colframe=keycolor!75!black,
 fonttitle=\bfseries,
 title=રાસાયણિક સમીકરણ/સૂત્ર
}

\newtcolorbox{mnemonicbox}{
 breakable,
 enhanced,
 colback=mnemoniccolor!5!white,
 colframe=mnemoniccolor!75!black,
 fonttitle=\bfseries,
 title=મેમરી ટ્રીક
}


% Custom commands for GTU solutions
% This file defines semantic commands for consistent formatting

% Question command with automatic formatting
\newcommand{\question}[2]{%
  \section*{Question #1}%
  \textbf{#2}%
}

% OR question variant
\newcommand{\questionor}[2]{%
  \section*{Question #1 OR}%
  \textbf{#2}%
}

% Proper table environment with caption
\newenvironment{answertable}[1]{%
  \begin{table}[htbp]
  \centering
  \caption{#1}
}{%
  \end{table}
}

% Proper figure environment for diagrams
\newenvironment{answerdiagram}[1]{%
  \begin{figure}[htbp]
  \centering
  \caption{#1}
}{%
  \end{figure}
}

% Semantic markup for key terms
\newcommand{\keyword}[1]{\textbf{#1}}
\newcommand{\code}[1]{\texttt{#1}}
\newcommand{\classname}[1]{\texttt{#1}}
\newcommand{\methodname}[1]{\texttt{#1}}

% Proper quotation marks
\newcommand{\mnemonic}[1]{``#1''}


\title{Electronics Devices \& Circuits (1323202) - Winter 2024 Solution Gujarati}
\date{January 18, 2024}

\begin{document}
\maketitle

\questionmarks{1(a)}{3}{થર્મલ રનઅવે વિગતવાર સમજાવો.}

\begin{solutionbox}
થર્મલ રનઅવે એક વિનાશક પ્રક્રિયા છે જેમાં ટ્રાન્ઝિસ્ટર વધુને વધુ ગરમ થાય છે જ્યાં સુધી તે નિષ્ફળ ન જાય.

\begin{answerdiagram}{થર્મલ રનઅવે પ્રક્રિયા}
\begin{tikzpicture}[gtu flow]
    \node[gtu block] (A) {ગરમી વધે છે};
    \node[gtu block, right of=A, node distance=3.5cm] (B) {કલેક્ટર કરંટ વધે છે};
    \node[gtu block, right of=B, node distance=3.5cm] (C) {વધુ પાવર વ્યય};
    \node[gtu block, below of=B, node distance=1.5cm] (D) {વધુ ગરમી ઉત્પન્ન થાય};

    \path [gtu arrow] (A) -- (B);
    \path [gtu arrow] (B) -- (C);
    \path [gtu arrow] (C) |- (D);
    \path [gtu arrow] (D) -| (A);
\end{tikzpicture}
\end{answerdiagram}

\begin{itemize}
    \item \keyword{કારણ}: તાપમાન વધવાથી બેઝ-એમિટર વોલ્ટેજ ઘટે છે
    \item \keyword{અસર}: તાપમાન વધવાથી કલેક્ટર કરંટ વધે છે
    \item \keyword{પરિણામ}: સ્વ-મજબૂત થતી ગરમીની સાયકલ વિનાશ તરફ દોરી જાય છે
\end{itemize}

\begin{mnemonicbox}
"ગરમી વધે, કરંટ ચડે, ટ્રાન્ઝિસ્ટર મરે"
\end{mnemonicbox}
\end{solutionbox}

\questionmarks{1(b)}{4}{ફિક્સડ બાયસ પદ્ધતિ દોરો અને સમજાવો.}

\begin{solutionbox}
ફિક્સડ બાયસ માટે બેઝને વોલ્ટેજ સપ્લાય સાથે જોડવા માટે એક જ રેસિસ્ટરનો ઉપયોગ થાય છે.

\begin{answerdiagram}{ફિક્સડ બાયસ સર્કિટ}
\begin{center}
\begin{circuitikz}[american, scale=0.8, transform shape]
    \draw (0,0) node[ground] {} to[npn, name=Q1] (0,2);
    \draw (Q1.E) -- (0,0);
    \draw (Q1.C) to[R, l=$R_C$] (0,4) node[vcc] {+VCC};
    \draw (Q1.B) -- (-2,0.85) to[R, l=$R_B$] (-2,4) -- (0,4);
    
    \node at (Q1.B) [left=2mm] {B};
    \node at (Q1.C) [right=2mm] {C};
    \node at (Q1.E) [right=2mm] {E};
\end{circuitikz}
\end{center}
\end{answerdiagram}

\begin{itemize}
    \item \keyword{કાર્યપદ્ધતિ}: બેઝ કરંટ ($I_B$) = ($V_{CC} - V_{BE}$)/$R_B$
    \item \keyword{લક્ષણો}: સરળ સર્કિટ પરંતુ ઓછી સ્થિરતા
    \item \keyword{ગેરલાભ}: તાપમાન ફેરફારો પ્રત્યે અતિસંવેદનશીલ
    \item \keyword{ઉપયોગ}: નાના સિગ્નલ સર્કિટ જ્યાં સ્થિરતા મહત્વની નથી
\end{itemize}

\begin{mnemonicbox}
"ફિક્સડ બાયસ: એક રેસિસ્ટર, ઓછી સ્થિરતા"
\end{mnemonicbox}
\end{solutionbox}

\questionmarks{1(c)}{7}{બાયસ પદ્ધતિઓની સૂચિ બનાવો. વોલ્ટેજ ડિવાઇડર પ્રકારની બાયસ પદ્ધતિની સર્કિટ દોરો અને સમજાવો.}

\begin{solutionbox}
ટ્રાન્ઝિસ્ટર માટે બાયસિંગ પદ્ધતિઓમાં યોગ્ય ઓપરેટિંગ પોઇન્ટ સ્થાપિત કરવા માટે કેટલીક તકનીકો શામેલ છે.

\begin{answertable}{ટ્રાન્ઝિસ્ટર બાયસિંગ પદ્ધતિઓ}
\begin{tabulary}{\linewidth}{|L|L|L|L|}
\hline
\textbf{પદ્ધતિ} & \textbf{સ્થિરતા} & \textbf{જટિલતા} & \textbf{તાપમાન સંવેદનશીલતા} \\
\hline
ફિક્સડ બાયસ & નબળી & સરળ & ઊંચી \\
\hline
કલેક્ટર-ટુ-બેઝ બાયસ & મધ્યમ & મધ્યમ & મધ્યમ \\
\hline
વોલ્ટેજ ડિવાઇડર બાયસ & ઉત્તમ & જટિલ & નીચી \\
\hline
એમિટર બાયસ & સારી & મધ્યમ & નીચી \\
\hline
\end{tabulary}
\end{answertable}

\begin{answerdiagram}{વોલ્ટેજ ડિવાઇડર બાયસ સર્કિટ}
\begin{center}
\begin{circuitikz}[american, scale=0.8, transform shape]
    \draw (0,0) node[ground] {} to[R, l=$R_E$] (0,1.5) to[npn, name=Q1] (0,3.5);
    \draw (Q1.E) -- (0,1.5);
    \draw (Q1.C) to[R, l=$R_C$] (0,5.5) node[vcc] {+VCC};
    
    \draw (Q1.B) -- (-1.5,2.35);
    \draw (-1.5,2.35) to[R, l=$R_2$] (-1.5,0) node[ground] {};
    \draw (-1.5,2.35) to[R, l=$R_1$] (-1.5,5.5) -- (0,5.5);
    
    \node at (Q1.B) [above right=1mm] {Base};
    \node at (Q1.C) [right=2mm] {Collector};
    \node at (Q1.E) [right=2mm] {Emitter};
\end{circuitikz}
\end{center}
\end{answerdiagram}

\begin{itemize}
    \item \keyword{કાર્યપદ્ધતિ}: $R_1$-$R_2$ ડિવાઇડર સ્થિર બેઝ વોલ્ટેજ બનાવે છે
    \item \keyword{ફાયદો}: $\beta$ વેરિએશન અને તાપમાનથી ઓછો પ્રભાવિત
    \item \keyword{મુખ્ય લક્ષણ}: RE નેગેટિવ ફીડબેક સ્થિરીકરણ પ્રદાન કરે છે
    \item \keyword{ઉપયોગ}: એમ્પલિફાયર સર્કિટમાં સૌથી વધુ વપરાય છે
\end{itemize}

\begin{mnemonicbox}
"વિભાજીત કરો અને સ્થિર બાયસ માટે રાજ કરો"
\end{mnemonicbox}
\end{solutionbox}

\orquestionmarks{1(c)}{7}{કોમન એમીટર એમ્પલીફાયર માટે ડીસી લોડ લાઈન દોરો અને સમજાવો.}

\begin{solutionbox}
ડીસી લોડ લાઈન ટ્રાન્ઝિસ્ટરના તમામ સંભવિત ઓપરેટિંગ પોઇન્ટ્સને દર્શાવે છે.

\begin{answerdiagram}{DC લોડ લાઈન}
\begin{tikzpicture}
    \begin{axis}[
        axis lines=left,
        xlabel={$V_{CE}$ (Volts)},
        ylabel={$I_C$ (mA)},
        xmin=0, xmax=12,
        ymin=0, ymax=12,
        xtick={0,10}, xticklabels={0,$V_{CC}$},
        ytick={0,10}, yticklabels={0,$V_{CC}/R_C$},
        grid=none
    ]
    \draw[thick, blue] (axis cs:0,10) -- (axis cs:10,0);
    \draw[fill=red] (axis cs:5,5) circle (2pt) node[above right] {Q-Point};
    \node at (axis cs:2,8) [rotate=-45] {DC Load Line};
    \end{axis}
\end{tikzpicture}
\end{answerdiagram}

\begin{answertable}{ઇક્વેશન કોષ્ટક}
\begin{tabulary}{\linewidth}{|L|L|L|}
\hline
\textbf{પેરામીટર} & \textbf{સમીકરણ} & \textbf{વર્ણન} \\
\hline
મહત્તમ $V_{CE}$ & $V_{CC}$ & જ્યારે $I_C = 0$ \\
\hline
મહત્તમ $I_C$ & $V_{CC}/R_C$ & જ્યારે $V_{CE} = 0$ \\
\hline
લોડ લાઈન સમીકરણ & $I_C = (V_{CC} - V_{CE})/R_C$ & બધા સંભવિત ઓપરેટિંગ પોઇન્ટ \\
\hline
Q-પોઇન્ટ & બાયસિંગ દ્વારા નિર્ધારિત & સ્થિર ઓપરેશન પોઇન્ટ \\
\hline
\end{tabulary}
\end{answertable}

\begin{itemize}
    \item \keyword{હેતુ}: $I_C$ અને $V_{CE}$ વચ્ચેના સંબંધને ગ્રાફિકલી બતાવે છે
    \item \keyword{મહત્વ}: ઓપરેટિંગ પોઇન્ટ (Q-પોઇન્ટ) નક્કી કરવામાં મદદ કરે છે
    \item \keyword{ઉપયોગ}: એમ્પલિફાયરની ડિઝાઇન અને વિશ્લેષણ માટે આવશ્યક
\end{itemize}

\begin{mnemonicbox}
"મહત્તમ કરંટ અથવા મહત્તમ વોલ્ટેજ, બંને ક્યારેય નહિં"
\end{mnemonicbox}
\end{solutionbox}

\questionmarks{2(a)}{3}{પદો સમજાવો (i) ગેઈન (ii) બેન્ડવિડ્થ.}

\begin{solutionbox}
આ એમ્પલિફાયર પરફોરમન્સને વર્ણવતા મુખ્ય પેરામીટર્સ છે.

\begin{answertable}{એમ્પલિફાયર પેરામીટર્સ}
\begin{tabulary}{\linewidth}{|L|L|L|L|}
\hline
\textbf{પેરામીટર} & \textbf{વ્યાખ્યા} & \textbf{એકમ} & \textbf{મહત્વ} \\
\hline
ગેઈન & આઉટપુટનો ઇનપુટ સિગ્નલ સાથેનો ગુણોત્તર & dB & એમ્પ્લિફિકેશન પાવર \\
\hline
બેન્ડવિડ્થ & ફ્રીક્વન્સીની રેન્જ જેમાં ગેઈન મહત્તમના 70.7\% કરતાં ઓછો ન હોય & Hz & ઉપયોગી ફ્રીક્વન્સી રેન્જ \\
\hline
\end{tabulary}
\end{answertable}

\begin{itemize}
    \item \keyword{ગેઈનના પ્રકાર}: વોલ્ટેજ ગેઈન ($A_v$), કરંટ ગેઈન ($A_i$), પાવર ગેઈન ($A_p$)
    \item \keyword{બેન્ડવિડ્થ ફોર્મ્યુલા}: $BW = f_H - f_L$ (ઉચ્ચ કટઓફ - નીચા કટઓફ)
    \item \keyword{સંબંધિત પેરામીટર}: ગેઈન-બેન્ડવિડ્થ પ્રોડક્ટ (ચોક્કસ એમ્પલિફાયર માટે અચળ)
\end{itemize}

\begin{mnemonicbox}
"ગેઈન મોટું બનાવે, બેન્ડવિડ્થ પહોળું બનાવે"
\end{mnemonicbox}
\end{solutionbox}

\questionmarks{2(b)}{4}{એમ્પલીફાયરમાં નેગેટીવ ફીડબેકના ફાયદા અને ગેરફાયદાની સૂચિ બનાવો.}

\begin{solutionbox}
નેગેટિવ ફીડબેક એમ્પલિફાયર પરફોરમન્સમાં નોંધપાત્ર સુધારો કરે છે પરંતુ ટ્રેડઓફ સાથે.

\begin{answertable}{નેગેટિવ ફીડબેક લક્ષણો}
\begin{tabulary}{\linewidth}{|L|L|}
\hline
\textbf{ફાયદા} & \textbf{ગેરફાયદા} \\
\hline
બેન્ડવિડ્થમાં વધારો & ગેઈનમાં ઘટાડો \\
\hline
ડિસ્ટોર્શનમાં ઘટાડો & વધુ ઇનપુટ સિગ્નલની જરૂર \\
\hline
સ્થિરતામાં સુધારો & વધુ જટિલ સર્કિટ \\
\hline
ઘોંઘાટ સામે વધુ ઈમ્યુનિટી & અયોગ્ય ડિઝાઇન થાય તો ઓસિલેશનની સંભાવના \\
\hline
ઇનપુટ/આઉટપુટ ઇમ્પીડન્સ નિયંત્રિત & વધુ પાવર વપરાશ \\
\hline
\end{tabulary}
\end{answertable}

\begin{mnemonicbox}
"સ્થિર, પહોળું અને ચોખ્ખું, માત્ર ગેઈન છોડો"
\end{mnemonicbox}
\end{solutionbox}

\questionmarks{2(c)}{7}{હાર્ટલી ઓસ્સીલેટર દોરો અને સમજાવો.}

\begin{solutionbox}
હાર્ટલી ઓસિલેટર ઇન્ડક્ટિવ ફીડબેકનો ઉપયોગ કરીને સાઇન વેવ્સ જનરેટ કરે છે.

\begin{answerdiagram}{હાર્ટલી ઓસિલેટર સર્કિટ}
\begin{center}
\begin{circuitikz}[american, scale=0.8, transform shape]
    \draw (0,0) node[ground] {} to[R, l=$R_E$] (0,1) to[npn, name=Q1] (0,3);
    \draw (Q1.E) -- (0,1);
    \draw (Q1.B) -- (-1.5,1.85);
    \draw (-1.5,0) node[ground] {} to[R, l=$R_{B2}$] (-1.5,1.85) to[R, l=$R_{B1}$] (-1.5,4.5) node[vcc] {+VCC};
    \draw (-1.5,4.5) -- (0,4.5) to[R, l=$R_C$] (0,3);
    \draw (0,3) -- (2,3) to[C, l=$C_{out}$] (3.5,3) node[right] {Output};
    \draw (2,3) -- (2,-1) to[L, l=$L_1$] (2,-2.5) coordinate (tap);
    \draw (tap) to[L, l=$L_2$] (2,-4) node[ground]{};
    \draw (2,3) -- (3.5,3) to[C, l=$C$] (3.5,-4) -- (2,-4);
    \draw (0,1) -- (2,1) -- (tap);
    \node[draw, dashed, fit={(2,3) (3.5,-4)}, label=below:Tank Circuit] {};
\end{circuitikz}
\end{center}
\end{answerdiagram}

\begin{itemize}
    \item \keyword{ફ્રીક્વન્સી નિર્ધારણ}: $L_1$, $L_2$ અને $C_1$ મૂલ્યો દ્વારા ($f = 1/2\pi\sqrt{L_{eq} C}$)
    \item \keyword{ફીડબેક મેકેનિઝમ}: ઇન્ડક્ટિવ વોલ્ટેજ ડિવાઇડર ($L_1$ અને $L_2$)
    \item \keyword{ઓળખ લક્ષણ}: ટેપ કરેલ ઇન્ડક્ટર અથવા શ્રેણીમાં બે ઇન્ડક્ટર્સ
    \item \keyword{ઉપયોગ}: RF સિગ્નલ જનરેશન, રેડિયો ટ્રાન્સમિટર્સ, કોમ્યુનિકેશન સિસ્ટમ્સ
\end{itemize}

\begin{mnemonicbox}
"હાર્ટલી હેલ્પફુલ ઇન્ડક્ટર્સ ધરાવે છે"
\end{mnemonicbox}
\end{solutionbox}

\orquestionmarks{2(a)}{3}{ઓસ્સીલેટર માટે બારખૌસન ક્રાઈટરીઆ (Barkhausen's criteria) જણાવો અને સમજાવો.}

\begin{solutionbox}
બારખૌસન ક્રાઈટેરિયા સતત ઓસિલેશન માટેની શરતો નિર્ધારિત કરે છે.

\begin{answerdiagram}{બારખૌસન ક્રાઈટેરિયા}
\begin{tikzpicture}[gtu flow]
    \node[gtu block] (A) {લૂપ ગેઈન $|A\beta| = 1$};
    \node[gtu block, below of=A] (B) {ફેઝ શિફ્ટ $\angle A\beta = 0^\circ$ or $360^\circ$};
    \node[gtu decision, right of=A, node distance=4cm, yshift=-1cm] (C) {સતત\\ઓસિલેશન};

    \path [gtu arrow] (A) -- (C);
    \path [gtu arrow] (B) -- (C);
\end{tikzpicture}
\end{answerdiagram}

\begin{itemize}
    \item \keyword{લૂપ ગેઈન કન્ડિશન}: $|A\beta| = 1$ (સતત ઓસિલેશન માટે ચોક્કસ 1)
    \item \keyword{ફેઝ શિફ્ટ કન્ડિશન}: $\angle A\beta = 0^\circ$ અથવા $360^\circ$ (સિગ્નલ રિઇન્ફોર્સમેન્ટ)
    \item \keyword{પ્રેક્ટિકલ ડિઝાઇન}: પ્રારંભિક $|A\beta| > 1$, અંતે $|A\beta| = 1$ પર સ્થિર થાય છે
\end{itemize}

\begin{mnemonicbox}
"ઓસિલેશન માટે: યુનિટ ગેઈન, ઝીરો ફેઝ"
\end{mnemonicbox}
\end{solutionbox}

\orquestionmarks{2(b)}{4}{નેગેટીવ અને પોસીટીવ ફીડબેક એમ્પલીફાયરને સરખાવો.}

\begin{solutionbox}
ફીડબેકનો પ્રકાર એમ્પલિફાયરના વર્તનને નાટકીય રીતે બદલે છે.

\begin{answertable}{તુલના કોષ્ટક}
\begin{tabulary}{\linewidth}{|L|L|L|}
\hline
\textbf{પેરામીટર} & \textbf{નેગેટિવ ફીડબેક} & \textbf{પોઝિટિવ ફીડબેક} \\
\hline
ગેઈન & ઘટે છે & વધે છે \\
\hline
બેન્ડવિડ્થ & વધે છે & ઘટે છે \\
\hline
ડિસ્ટોર્શન & ઘટાડે છે & વધારે છે \\
\hline
સ્થિરતા & સુધારે છે & ઘટાડે છે (ઓસિલેટ કરી શકે) \\
\hline
ઘોંઘાટ & ઘટાડે છે & વધારે છે \\
\hline
ઉપયોગ & સ્થિર એમ્પલિફાયર & ઓસિલેટર, ટ્રિગર સર્કિટ \\
\hline
ઇનપુટ/આઉટપુટ ઇમ્પીડન્સ & નિયંત્રિત & ઓછી અનુમાનિત \\
\hline
\end{tabulary}
\end{answertable}

\begin{mnemonicbox}
"નેગેટિવ સ્થિર કરે, પોઝિટિવ ઓસિલેટ કરે"
\end{mnemonicbox}
\end{solutionbox}

\orquestionmarks{2(c)}{7}{કોલપીટ્ટ્સ ઓસ્સીલેટર દોરો અને સમજાવો.}

\begin{solutionbox}
કોલપિટ્સ ઓસિલેટર ફીડબેક માટે કેપેસિટિવ વોલ્ટેજ ડિવાઇડરનો ઉપયોગ કરે છે.

\begin{answerdiagram}{કોલપિટ્સ ઓસિલેટર સર્કિટ}
\begin{center}
\begin{circuitikz}[american, scale=0.8, transform shape]
    \draw (0,0) node[ground] {} to[R, l=$R_E$] (0,1) to[npn, name=Q1] (0,3);
    \draw (Q1.E) -- (0,1);
    \draw (Q1.C) to[R, l=$R_C$] (0,5) node[vcc] {+VCC};
    \draw (Q1.B) -- (-1.5,1.85);
    \draw (-1.5,0) node[ground] {} to[R, l=$R_{B2}$] (-1.5,1.85) to[R, l=$R_{B1}$] (-1.5,5) -- (0,5);
    \draw (0,3) -- (2,3) to[C, l=$C_1$] (2,1) coordinate (tap);
    \draw (tap) to[C, l=$C_2$] (2,-1) node[ground]{};
    \draw (2,3) -- (3.5,3) to[L, l=$L$] (3.5,-1) -- (2,-1);
    \draw (0,1) -- (2,1);
    \draw (0,3) to[C, l=$C_{out}$] (-1,4) node[left] {Output};
\end{circuitikz}
\end{center}
\end{answerdiagram}

\begin{itemize}
    \item \keyword{ફ્રીક્વન્સી નિર્ધારણ}: $L$, $C_1$ અને $C_2$ મૂલ્યો દ્વારા ($f = 1/2\pi\sqrt{L C_{eq}}$)
    \item \keyword{ફીડબેક મેકેનિઝમ}: કેપેસિટિવ વોલ્ટેજ ડિવાઇડર ($C_1$ અને $C_2$)
    \item \keyword{ઓળખ લક્ષણ}: ઇન્ડક્ટર સામે શ્રેણીમાં બે કેપેસિટર
    \item \keyword{ફાયદો}: હાર્ટલી કરતાં વધુ સ્થિર ફ્રીક્વન્સી
\end{itemize}

\begin{mnemonicbox}
"કોલપિટ્સ કેપેસિટિવ કરંટ કેચ કરે છે"
\end{mnemonicbox}
\end{solutionbox}

\questionmarks{3(a)}{3}{ડાયક વિષે સમજાવો.}

\begin{solutionbox}
DIAC (Diode for Alternating Current) એ બાઇડિરેક્શનલ ટ્રિગર ડાયોડ છે.

\begin{answerdiagram}{DIAC સિમ્બોલ}
\begin{circuitikz}[circuitikz/diode scale=1.5]
    \draw (0,0) to[diode, l=DIAC] (2,0);
\end{circuitikz}
\end{answerdiagram}

\begin{itemize}
    \item \keyword{ઓપરેશન}: બ્રેકડાઉન વોલ્ટેજ પછી બંને દિશામાં વહન કરે છે
    \item \keyword{લક્ષણ}: બંને દિશામાં સિમેટ્રિકલ V-I કર્વ
    \item \keyword{કી પેરામીટર}: બ્રેકઓવર વોલ્ટેજ (સામાન્ય રીતે 30-40V)
    \item \keyword{મુખ્ય ઉપયોગ}: AC પાવર કંટ્રોલમાં TRIAC ટ્રિગરિંગ
\end{itemize}

\begin{mnemonicbox}
"DIAC: બેવડી દિશા બ્રેકડાઉન ડિવાઇસ"
\end{mnemonicbox}
\end{solutionbox}

\questionmarks{3(b)}{4}{SCRની ટ્રીગરિંગ પદ્ધતિઓ સમજાવો.}

\begin{solutionbox}
SCR વહન માટે ઘણી પદ્ધતિઓ દ્વારા ટ્રિગર થઈ શકે છે.

\begin{answertable}{SCR ટ્રિગરિંગ પદ્ધતિઓ}
\begin{tabulary}{\linewidth}{|L|L|L|L|}
\hline
\textbf{પદ્ધતિ} & \textbf{વર્ણન} & \textbf{ફાયદા} & \textbf{મર્યાદાઓ} \\
\hline
ગેટ ટ્રિગરિંગ & ગેટ પર કરંટ પલ્સ & સૌથી સામાન્ય, નિયંત્રિત & કંટ્રોલ સર્કિટની જરૂર \\
\hline
તાપમાન & ઉચ્ચ તાપમાન & કોઈ બાહ્ય સર્કિટ નહીં & અનિયંત્રિત, અવિશ્વસનીય \\
\hline
વોલ્ટેજ & બ્રેકઓવર વોલ્ટેજથી વધારે & કોઈ બાહ્ય સર્કિટ નહીં & ડિવાઇસ પર તણાવ, અનિયંત્રિત \\
\hline
dv/dt & ઝડપી વોલ્ટેજ વૃદ્ધિ & સરળ & અનિચ્છનીય ટ્રિગરિંગ થઈ શકે \\
\hline
પ્રકાશ & જંક્શન પર ફોટોન્સ & ઇલેક્ટ્રિકલ અલગતા & વિશેષ પેકેજિંગની જરૂર \\
\hline
\end{tabulary}
\end{answertable}

\begin{mnemonicbox}
"ગેટ વોલ્ટેજ તાપમાન રેટ લાઇટ"
\end{mnemonicbox}
\end{solutionbox}

\questionmarks{3(c)}{7}{SCRનો સિમ્બોલ અને કન્સ્ટ્રક્શન દોરો. ઉપરાંત SCRની V-I લાક્ષણિકતા દોરો અને સમજાવો.}

\begin{solutionbox}
SCR (Silicon Controlled Rectifier) એ ત્રણ ટર્મિનલવાળી ચાર-લેયર PNPN સેમિકન્ડક્ટર ડિવાઇસ છે.

\begin{answerdiagram}{SCR સિમ્બોલ અને કન્સ્ટ્રક્શન}
\begin{center}
\begin{tikzpicture}
    \draw (0,2) node[above] {A} to[Ty] (0,0) node[below] {K};
    \draw (-0.5,0.8) node[left] {G} -- (0,0.8);
    \node at (0,-1) {Symbol};
    
    \begin{scope}[xshift=4cm, yshift=-0.5cm]
        \draw (0,0) rectangle (2,3);
        \draw (0,0.75) -- (2,0.75);
        \draw (0,1.5) -- (2,1.5);
        \draw (0,2.25) -- (2,2.25);
        
        \node at (1,2.625) {P};
        \node at (1,1.875) {N};
        \node at (1,1.125) {P};
        \node at (1,0.375) {N};
        
        \draw (1,3) -- (1,3.5) node[above] {Anode};
        \draw (1,0) -- (1,-0.5) node[below] {Cathode};
        \draw (2,1.125) -- (2.5,1.125) node[right] {Gate};
        \node at (1,-1.5) {Construction};
    \end{scope}
\end{tikzpicture}
\end{center}
\end{answerdiagram}

\begin{answerdiagram}{SCR V-I લાક્ષણિકતા}
\begin{tikzpicture}
    \begin{axis}[
        axis lines=middle,
        xlabel={$V_{AK}$},
        ylabel={$I_A$},
        xmin=-10, xmax=10,
        ymin=-5, ymax=10,
        ticks=none
    ]
    \draw[blue, thick] (0,0) -- (8,1) node[right] {$V_{BO}$};
    \draw[green!60!black, thick] (1,1) -- (2,9) node[above] {On State};
    \draw[dashed] (8,1) -- (1,1);
    \draw[blue, thick] (0,0) -- (-8,-1);
    \draw[red, thick] (-8,-1) -- (-8,-5) node[below] {$V_{BR}$};
    \node at (4,6) {Forward Conduction};
    \node at (6,0.5) {Blocking};
    \node at (-4,-2) {Reverse Blocking};
    \end{axis}
\end{tikzpicture}
\end{answerdiagram}

\begin{itemize}
    \item \keyword{ફોરવર્ડ બ્લોકિંગ}: ટ્રિગરિંગ સુધી ઓછો કરંટ
    \item \keyword{ફોરવર્ડ કન્ડક્શન}: ટ્રિગરિંગ પછી ઉચ્ચ કરંટ (લેચડ)
    \item \keyword{હોલ્ડિંગ કરંટ}: કન્ડક્શન જાળવવા માટે ન્યૂનતમ કરંટ
    \item \keyword{લેચિંગ કરંટ}: લેચિંગ શરૂ કરવા માટે ન્યૂનતમ કરંટ
    \item \keyword{રિવર્સ બ્લોકિંગ}: રિવર્સ દિશામાં કરંટને અવરોધે છે
\end{itemize}

\begin{mnemonicbox}
"એક વાર ટ્રિગર, હંમેશા કન્ડક્ટ, જ્યાં સુધી કરંટ ન ઘટે"
\end{mnemonicbox}
\end{solutionbox}

\orquestionmarks{3(a)}{3}{SCRની નેચરલ કોમ્યુટેશન પદ્ધતિ વિષે સમજાવો.}

\begin{solutionbox}
નેચરલ કોમ્યુટેશન AC કરંટ કુદરતી રીતે શૂન્ય પર પહોંચે ત્યારે બાહ્ય સર્કિટ વિના SCRને બંધ કરે છે.

\begin{answerdiagram}{નેચરલ કોમ્યુટેશન પ્રક્રિયા}
\begin{tikzpicture}[gtu flow]
    \node[gtu block] (A) {AC સપ્લાય શૂન્ય ક્રોસ કરે છે};
    \node[gtu block, right of=A, node distance=3.5cm] (B) {કરંટ હોલ્ડિંગથી નીચે પડે છે};
    \node[gtu block, right of=B, node distance=3.5cm] (C) {SCR કુદરતી રીતે બંધ થાય છે};
    \path [gtu arrow] (A) -- (B);
    \path [gtu arrow] (B) -- (C);
\end{tikzpicture}
\end{answerdiagram}

\begin{itemize}
    \item \keyword{સિદ્ધાંત}: AC સપ્લાયના કુદરતી શૂન્ય-ક્રોસિંગનો ઉપયોગ કરે છે
    \item \keyword{ફાયદો}: કોઈ વધારાની કોમ્યુટેશન સર્કિટની જરૂર નથી
    \item \keyword{ઉપયોગ}: AC પાવર કંટ્રોલ સર્કિટ, લાઇટ ડિમર્સ
    \item \keyword{મર્યાદા}: માત્ર AC સપ્લાય સાથે કામ કરે છે, DC સાથે નહીં
\end{itemize}

\begin{mnemonicbox}
"નેચરલ કોમ્યુટેશન: શૂન્ય કરંટ, શૂન્ય પ્રયત્ન"
\end{mnemonicbox}
\end{solutionbox}

\orquestionmarks{3(b)}{4}{ઓપ્ટો-કપ્લર વિશે સમજાવો.}

\begin{solutionbox}
ઓપ્ટો-કપ્લર પ્રકાશ ટ્રાન્સમિશનનો ઉપયોગ કરીને ઇલેક્ટ્રિકલ આઈસોલેશન પ્રદાન કરે છે.

\begin{answerdiagram}{ઓપ્ટો-કપ્લર સંરચના}
\begin{tikzpicture}
    \draw[thick] (0,0) rectangle (3,2);
    \node at (0.5,1) (LED) {};
    \draw (0.5,0.5) to[leDo] (0.5,1.5);
    \node at (2.5,1) (Photo) {};
    \draw (2.5,0.5) to[photodiode] (2.5,1.5);
    \draw[dashed, ->, decorate, decoration={snake, amplitude=.4mm, segment length=2mm, post length=1mm}] (1,1) -- (2,1);
    \node at (1.5,1.5) {Light};
    \node at (0.5,-0.5) {Input};
    \node at (2.5,-0.5) {Output};
\end{tikzpicture}
\end{answerdiagram}

\begin{answertable}{ઓપ્ટો-કપ્લર પ્રકારો}
\begin{tabulary}{\linewidth}{|L|L|L|L|}
\hline
\textbf{પ્રકાર} & \textbf{ફોટોડિટેક્ટર} & \textbf{સ્પીડ} & \textbf{ઉપયોગો} \\
\hline
સ્ટાન્ડર્ડ & ફોટોટ્રાન્ઝિસ્ટર & મધ્યમ & સામાન્ય આઈસોલેશન \\
\hline
હાઈ-સ્પીડ & ફોટોડાયોડ & ઝડપી & ડિજિટલ કોમ્યુનિકેશન \\
\hline
TRIAC & ફોટો-TRIAC & ધીમું & AC પાવર કંટ્રોલ \\
\hline
લિનિયર & ફોટોડાર્લિંગટન & ધીમું & એનાલોગ સિગ્નલ્સ \\
\hline
\end{tabulary}
\end{answertable}

\begin{itemize}
    \item \keyword{CTR}: કરંટ ટ્રાન્સફર રેશિયો (આઉટપુટ/ઇનપુટ કરંટ)
    \item \keyword{મુખ્ય લક્ષણ}: સર્કિટ્સ વચ્ચે સંપૂર્ણ ઇલેક્ટ્રિકલ આઈસોલેશન
    \item \keyword{ફાયદા}: નોઈઝ ઈમ્યુનિટી, વોલ્ટેજ લેવલ શિફ્ટિંગ, સલામતી
\end{itemize}

\begin{mnemonicbox}
"પ્રકાશ કૂદે છે જ્યાં ઇલેક્ટ્રોન્સ નથી કૂદી શકતા"
\end{mnemonicbox}
\end{solutionbox}

\orquestionmarks{3(c)}{7}{TRIACનો સિમ્બોલ અને કન્સ્ટ્રક્શન દોરો. ઉપરાંત TRIACની V-I લાક્ષણિકતા દોરો અને સમજાવો.}

\begin{solutionbox}
TRIAC (Triode for Alternating Current) એ બાઇડિરેક્શનલ ત્રણ-ટર્મિનલવાળી સેમિકન્ડક્ટર ડિવાઇસ છે.

\begin{answerdiagram}{TRIAC સિમ્બોલ અને કન્સ્ટ્રક્શન}
\begin{center}
\begin{tikzpicture}
    \draw (0,0) to[triac] (0,2);
    \node at (0,2.2) {MT2};
    \node at (0,-0.2) {MT1};
    \node at (-1,0.5) {G};
    
    \begin{scope}[xshift=4cm]
        \draw (0,0) rectangle (3,3);
        \draw (0,1) -- (3,1);
        \draw (0,2) -- (3,2);
        \node at (1.5,0.5) {N2 - MT1};
        \node at (1.5,1.5) {P2 - Gate};
        \node at (1.5,2.5) {P1 - N1 - MT2};
    \end{scope}
\end{tikzpicture}
\end{center}
\end{answerdiagram}

\begin{answerdiagram}{TRIAC V-I લાક્ષણિકતા}
\begin{tikzpicture}
    \begin{axis}[
        axis lines=middle,
        xlabel={$V_{MT2-MT1}$},
        ylabel={$I_T$},
        xmin=-10, xmax=10,
        ymin=-10, ymax=10,
        ticks=none
    ]
    \draw[blue, thick] (0,0) -- (8,1) -- (2,8);
    \draw[blue, thick] (0,0) -- (-8,-1) -- (-2,-8);
    \node at (5,5) {Quadrant I};
    \node at (-5,-5) {Quadrant III};
    \end{axis}
\end{tikzpicture}
\end{answerdiagram}

\begin{itemize}
    \item \keyword{બાઇડિરેક્શનલ}: ટ્રિગરિંગ પછી બંને દિશામાં વહન કરે છે
    \item \keyword{ક્વોડ્રન્ટ ઓપરેશન}: પોલેરિટી પર આધારિત ચાર ટ્રિગરિંગ મોડ
    \item \keyword{ઉપયોગો}: AC પાવર કંટ્રોલ, લાઇટ ડિમર્સ, મોટર કંટ્રોલ
    \item \keyword{SCR કરતાં ફાયદો}: AC સાયકલના બંને અર્ધભાગોને નિયંત્રિત કરે છે
\end{itemize}

\begin{mnemonicbox}
"TRIAC: AC સર્કિટમાં બેવડી દિશાનો રસ્તો"
\end{mnemonicbox}
\end{solutionbox}

\questionmarks{4(a)}{3}{Ideal Op-Ampની લાક્ષણિકતા જણાવો.}

\begin{solutionbox}
આદર્શ Op-Amp એવી સંપૂર્ણ લાક્ષણિકતાઓ ધરાવે છે જેને વાસ્તવિક Op-Amps આશરે છે.

\begin{answertable}{આદર્શ Op-Amp લાક્ષણિકતાઓ}
\begin{tabulary}{\linewidth}{|L|L|L|}
\hline
\textbf{પેરામીટર} & \textbf{આદર્શ મૂલ્ય} & \textbf{અર્થ} \\
\hline
ઓપન-લૂપ ગેઈન & અનંત & નાનામાં નાના ઇનપુટ તફાવતને એમ્પ્લિફાય કરે છે \\
\hline
ઇનપુટ ઇમ્પીડન્સ & અનંત & સ્ત્રોતમાંથી કોઈ કરંટ લેતું નથી \\
\hline
આઉટપુટ ઇમ્પીડન્સ & શૂન્ય & કોઈપણ લોડને ડ્રાઇવ કરી શકે છે \\
\hline
બેન્ડવિડ્થ & અનંત & બધી ફ્રીક્વન્સી પર કામ કરે છે \\
\hline
CMRR & અનંત & કોમન-મોડ સિગ્નલ્સને નકારે છે \\
\hline
સ્લ્યૂ રેટ & અનંત & તાત્કાલિક આઉટપુટ ફેરફાર \\
\hline
ઓફસેટ વોલ્ટેજ & શૂન્ય & શૂન્ય ઇનપુટ સાથે કોઈ આઉટપુટ નહીં \\
\hline
\end{tabulary}
\end{answertable}

\begin{mnemonicbox}
"અનંત ગેઈન, ઇમ્પીડન્સ, બેન્ડવિડ્થ; શૂન્ય ઓફસેટ, આઉટપુટ Z"
\end{mnemonicbox}
\end{solutionbox}

\questionmarks{4(b)}{4}{555 ટાઈમર ICની મદદથી મોનોસ્ટેબલ મલ્ટીવાઇબ્રેટર દોરો અને સમજાવો.}

\begin{solutionbox}
મોનોસ્ટેબલ મલ્ટીવાઇબ્રેટર ટ્રિગર થાય ત્યારે નિશ્ચિત સમયગાળાનો એક પલ્સ ઉત્પન્ન કરે છે.

\begin{answerdiagram}{મોનોસ્ટેબલ 555 સર્કિટ}
\begin{center}
\begin{circuitikz}[american]
    \draw (0,0) node[ground]{} to[C, l=$C$] (0,2) -- (2,2) node[align=center]{Threshold\\Discharge};
    \draw (2,2) -- (2,3) to[R, l=$R$] (2,5) node[vcc]{+VCC};
    \draw (3,1) rectangle (6,4);
    \node at (4.5,2.5) {555 Timer};
    \draw (2,2) -- (3,3.5);
    \draw (3,1.5) -- (-1,1.5) node[left] {Trigger};
    \draw (6,2.5) -- (7,2.5) node[right] {Output};
\end{circuitikz}
\end{center}
\end{answerdiagram}

\begin{itemize}
    \item \keyword{ઓપરેશન}: નેગેટિવ ટ્રિગર $T = 1.1RC$ સમયગાળાનો આઉટપુટ પલ્સ ઉત્પન્ન કરે છે
    \item \keyword{સ્ટેબલ સ્ટેટ}: ટ્રિગર થાય ત્યાં સુધી આઉટપુટ LOW
    \item \keyword{ટાઇમિંગ કંટ્રોલ}: $R$ અને $C$ મૂલ્યો પલ્સ પહોળાઈ નક્કી કરે છે
    \item \keyword{રિટ્રિગરિંગ}: ટાઇમઆઉટ પછી ફરીથી ટ્રિગર થઈ શકે છે
\end{itemize}

\begin{mnemonicbox}
"વન શોટ વન્ડર: એક વાર ટ્રિગર, એક વાર પલ્સ"
\end{mnemonicbox}
\end{solutionbox}

\questionmarks{4(c)}{7}{741 ICની મદદથી ઇન્વર્ટિંગ એમ્પલીફાયર દોરો અને સમજાવો. ઉપરાંત તેના ઈનપુટ અને આઉટપુટ વેવફોર્મ્સ દોરો.}

\begin{solutionbox}
ઇન્વર્ટિંગ એમ્પલિફાયર ઇનપુટ સિગ્નલને એમ્પ્લિફાય કરતી વખતે પોલેરિટી ઉલટાવે છે.

\begin{answerdiagram}{ઇન્વર્ટિંગ એમ્પલિફાયર સર્કિટ}
\begin{center}
\begin{circuitikz}[american]
    \draw (0,0) node[op amp] (opamp) {};
    \draw (opamp.-) -- (-1.5,0.5) to[R, l=$R_{in}$] (-3,0.5) node[left] {$V_{in}$};
    \draw (opamp.+) -- (-1.5,-0.5) node[ground] {};
    \draw (opamp.-) -- (-1.5,0.5) -- (-1.5,1.5) to[R, l=$R_f$] (1.5,1.5) -- (1.5,0) -- (opamp.out);
    \draw (opamp.out) to[short,-o] (2,0) node[right] {$V_{out}$};
\end{circuitikz}
\end{center}
\end{answerdiagram}

\begin{answerdiagram}{ઇન્વર્ટિંગ વેવફોર્મ્સ}
\begin{tikzpicture}
    \draw[blue] (0,1) sin (1,1.5) cos (2,1) sin (3,0.5) cos (4,1);
    \node[left] at (0,1) {Input};
    \draw[red] (0,-1) sin (1,-2) cos (2,-1) sin (3,0) cos (4,-1);
    \node[left] at (0,-1) {Output};
\end{tikzpicture}
\end{answerdiagram}

\begin{itemize}
    \item \keyword{ગેઈન સમીકરણ}: $A_v = -R_f/R_{in}$ (નેગેટિવ ચિહ્ન ઇન્વર્ઝન સૂચવે છે)
    \item \keyword{ઇનપુટ ઇમ્પીડન્સ}: $R_{in}$ જેટલી
    \item \keyword{વર્ચ્યુઅલ ગ્રાઉન્ડ}: ઇન્વર્ટિંગ ઇનપુટ લગભગ 0V પર જળવાય છે
\end{itemize}

\begin{mnemonicbox}
"ઉલટાવે અને Rf/Rin વડે ગુણાકાર કરે છે"
\end{mnemonicbox}
\end{solutionbox}

\orquestionmarks{4(a)}{3}{IC 741નો સિમ્બોલ અને પીન ડાયગ્રામ દોરો.}

\begin{solutionbox}
741 એક લોકપ્રિય જનરલ-પરપસ ઓપરેશનલ એમ્પલિફાયર છે.

\begin{answerdiagram}{741 સિમ્બોલ અને પિન ડાયગ્રામ}
\begin{center}
\begin{tikzpicture}
    \draw (0,0) node[op amp] (opamp) {};
    \node at (0,-1.5) {Symbol};
    
    \begin{scope}[xshift=4cm]
        \draw (0,0) rectangle (2,3);
        \node at (1,1.5) {741};
        \foreach \i in {1,...,4} \draw (0,\i*0.6) -- (-0.5,\i*0.6) node[left] {\i};
        \foreach \i [evaluate=\i as \y using (\i-4)*0.6] in {5,...,8} \draw (2,\y) -- (2.5,\y) node[right] {\i};
        \node at (1,-1.5) {DIP-8};
    \end{scope}
\end{tikzpicture}
\end{center}
\end{answerdiagram}

\begin{itemize}
    \item \keyword{પિન ફંક્શન્સ}: 2:ઇન્વર્ટિંગ, 3:નોન-ઇન્વર્ટિંગ, 6:આઉટપુટ, 7:V+, 4:V-
    \item \keyword{ઓપ્શનલ પિન્સ}: 1,5:ઓફસેટ નલ, 8:NC
\end{itemize}

\begin{mnemonicbox}
"કદી ઉલટાવશો નહિં પ્લસ, વેરી આઉટપુટ નોટ કનેક્ટેડ"
\end{mnemonicbox}
\end{solutionbox}

\orquestionmarks{4(b)}{4}{પદો સમજાવો (i) સી.એમ.આર.આર (II) સ્લૂ રેટ.}

\begin{solutionbox}
આ પેરામીટર્સ ઓપરેશનલ એમ્પલિફાયરની કાર્યક્ષમતાની મર્યાદાઓ નિર્ધારિત કરે છે.

\begin{answertable}{મુખ્ય Op-Amp પેરામીટર્સ}
\begin{tabulary}{\linewidth}{|L|L|L|}
\hline
\textbf{પેરામીટર} & \textbf{સામાન્ય મૂલ્ય} & \textbf{મહત્વ} \\
\hline
CMRR & 90-120 dB & ઉચ્ચ હોય તે વધુ સારું \\
\hline
સ્લ્યૂ રેટ & 0.5-50 V/$\mu$s & ઝડપી સિગ્નલ્સ માટે ઉચ્ચ \\
\hline
\end{tabulary}
\end{answertable}

\begin{itemize}
    \item \keyword{CMRR}: ડિફરેન્શિયલ ગેઈનનો કોમન-મોડ ગેઈન સાથેનો ગુણોત્તર
    \item \keyword{સ્લ્યૂ રેટ}: આઉટપુટ વોલ્ટેજના ફેરફારનો મહત્તમ દર
\end{itemize}
\end{solutionbox}

\orquestionmarks{4(c)}{7}{555 ટાઈમર ICની મદદથી આસ્ટેબલ મલ્ટીવાઇબ્રેટર દોરો અને સમજાવો.}

\begin{solutionbox}
આસ્ટેબલ મલ્ટીવાઇબ્રેટર બાહ્ય ટ્રિગર વિના સતત સ્ક્વેર વેવ્સ ઉત્પન્ન કરે છે.

\begin{answerdiagram}{આસ્ટેબલ 555 સર્કિટ}
\begin{center}
\begin{circuitikz}[american]
    \draw (0,0) node[ground] {} to[C, l=$C$] (0,2);
    \draw (0,2) to[R, l=$R_B$] (0,4) to[R, l=$R_A$] (0,6) node[vcc] {+VCC};
    \node at (2,4) {555 Timer};
    \draw (1,3) rectangle (3,5);
    \draw (0,2) -- (1,3.5);
    \draw (0,4) -- (1,4.5);
\end{circuitikz}
\end{center}
\end{answerdiagram}

\begin{itemize}
    \item \keyword{ટાઇમિંગ}: $T_1 = 0.693(R_A+R_B)C$, $T_2 = 0.693(R_B)C$
    \item \keyword{ફ્રીક્વન્સી}: $f = 1.44/((R_A+2R_B)C)$
\end{itemize}
\end{solutionbox}

\questionmarks{5(a)}{3}{રેગ્યુલેટેડ પાવર સપ્લાયનો બેઝીક બ્લોક ડાયગ્રામ દોરો અને તેને સમજાવો.}

\begin{solutionbox}
રેગ્યુલેટેડ પાવર સપ્લાય AC ને સ્થિર DC વોલ્ટેજમાં રૂપાંતરિત કરે છે.

\begin{answerdiagram}{પાવર સપ્લાય બ્લોક ડાયગ્રામ}
\begin{tikzpicture}[gtu flow]
    \node[gtu block] (A) {ટ્રાન્સફોર્મર};
    \node[gtu block, right of=A, node distance=2.5cm] (B) {રેક્ટિફાયર};
    \node[gtu block, right of=B, node distance=2.5cm] (C) {ફિલ્ટર};
    \node[gtu block, right of=C, node distance=2.5cm] (D) {રેગ્યુલેટર};
    \node[gtu output, right of=D, node distance=2.5cm] (E) {લોડ};

    \path [gtu arrow] (A) -- (B);
    \path [gtu arrow] (B) -- (C);
    \path [gtu arrow] (C) -- (D);
    \path [gtu arrow] (D) -- (E);
\end{tikzpicture}
\end{answerdiagram}

\begin{mnemonicbox}
"ટ્રાન્સફોર્મર રેક્ટિફાય ફિલ્ટર રેગ્યુલેટ"
\end{mnemonicbox}
\end{solutionbox}

\questionmarks{5(b)}{4}{Op-ampની મદદથી સમિંગ એમ્પલીફાયર દોરો અને સમજાવો.}

\begin{solutionbox}
સમિંગ એમ્પલિફાયર વજનદાર અનુપાત સાથે બહુવિધ ઇનપુટ સિગ્નલ્સને ઉમેરે છે.

\begin{answerdiagram}{સમિંગ એમ્પલિફાયર}
\begin{center}
\begin{circuitikz}[american]
    \draw (0,0) node[op amp] (opamp) {};
    \draw (opamp.-) -- (-1,0.5) coordinate (sum);
    \draw (sum) to[R, l=$R_1$] (-3,1.5) node[left] {$V_1$};
    \draw (sum) to[R, l=$R_2$] (-3,0.5) node[left] {$V_2$};
    \draw (sum) to[R, l=$R_3$] (-3,-0.5) node[left] {$V_3$};
    \draw (sum) -- (-1,2) to[R, l=$R_f$] (1,2) -- (opamp.out);
    \draw (opamp.+) -- (-1,-0.5) node[ground] {};
    \draw (opamp.out) to[short,-o] (2,0) node[right] {$V_{out}$};
\end{circuitikz}
\end{center}
\end{answerdiagram}

\begin{itemize}
    \item \keyword{આઉટપુટ સમીકરણ}: $V_{out} = -R_f(V_1/R_1 + V_2/R_2 + V_3/R_3)$
\end{itemize}
\end{solutionbox}

\questionmarks{5(c)}{7}{IC LM317ની મદદથી 3 ટર્મિનલવાળા એડજસ્ટેબલ આઉટપુટ વોલ્ટેજ રેગ્યુલેટરનો સર્કિટ ડાયગ્રામ દોરો અને સમજાવો.}

\begin{solutionbox}
LM317 એ 1.25V થી 37V સુધીની આઉટપુટ રેન્જ સાથે વર્સેટાઇલ એડજસ્ટેબલ વોલ્ટેજ રેગ્યુલેટર છે.

\begin{answerdiagram}{LM317 સર્કિટ}
\begin{center}
\begin{circuitikz}[american]
    \draw (0,0) node[ground] {} to[C, l=$C_1$] (0,2) -- (2,2);
    \draw (0,2) -- (-1,2) node[left] {$V_{in}$};
    \draw (2,1) rectangle (4,3); \node at (3,2) {LM317};
    \draw (4,2) -- (6,2) node[right] {$V_{out}$};
    \draw (6,2) to[C, l=$C_2$] (6,0) node[ground] {};
    \draw (3,1) -- (3,0.5) -- (4,0.5) to[R, l=$R_1$] (6,0.5) -- (6,2);
    \draw (3,0.5) to[R, l=$R_2$, v^<=$Adjust$] (3,-1.5) node[ground] {};
\end{circuitikz}
\end{center}
\end{answerdiagram}

\begin{itemize}
    \item \keyword{આઉટપુટ વોલ્ટેજ}: $V_{out} = 1.25(1 + R_2/R_1)$
\end{itemize}

\begin{mnemonicbox}
"R2 વડે એડજસ્ટ કરો, રેફરન્સ 1.25 પર રહે છે"
\end{mnemonicbox}
\end{solutionbox}

\orquestionmarks{5(a)}{3}{એસ.એમ.પી.એસના કાર્યો જણાવો.}

\begin{solutionbox}
SMPS એટલે Switch Mode Power Supply.

\begin{answertable}{SMPS ઉપયોગો}
\begin{tabulary}{\linewidth}{|L|L|}
\hline
\textbf{ઉપયોગ} & \textbf{SMPS પ્રકાર} \\
\hline
કમ્પ્યુટર પાવર સપ્લાય & ATX \\
\hline
મોબાઇલ ફોન ચાર્જર & ફ્લાયબૅક \\
\hline
LED ડ્રાઇવર & બક \\
\hline
\end{tabulary}
\end{answertable}

\begin{mnemonicbox}
"સ્વિચ મોડ નાના ઉપકરણોને પાવર આપે છે"
\end{mnemonicbox}
\end{solutionbox}

\orquestionmarks{5(b)}{4}{Op-ampની મદદથી ડિફ્રન્સીએટર દોરો અને સમજાવો.}

\begin{solutionbox}
ડિફરન્શિએટર ઇનપુટના ફેરફારના દરના સમપ્રમાણમાં આઉટપુટ ઉત્પન્ન કરે છે.

\begin{answerdiagram}{ડિફરન્શિએટર સર્કિટ}
\begin{center}
\begin{circuitikz}[american]
    \draw (0,0) node[op amp] (opamp) {};
    \draw (opamp.-) -- (-1.5,0.5) to[C, l=$C$] (-3,0.5) node[left] {$V_{in}$};
    \draw (opamp.+) -- (-1.5,-0.5) node[ground] {};
    \draw (opamp.-) -- (-1.5,1.5) to[R, l=$R_f$] (1.5,1.5) -- (opamp.out);
    \draw (opamp.out) to[short,-o] (2,0) node[right] {$V_{out}$};
\end{circuitikz}
\end{center}
\end{answerdiagram}

\begin{itemize}
    \item \keyword{સમીકરણ}: $V_{out} = -RC ({dV_{in}}/{dt})$
    \item \keyword{ઉપયોગો}: વેવશેપિંગ, ફેરફાર-દરની શોધ
\end{itemize}

\begin{mnemonicbox}
"ફેરફારનો દર અંદર જાય, એમ્પલિટ્યુડ બહાર આવે"
\end{mnemonicbox}
\end{solutionbox}

\orquestionmarks{5(c)}{7}{-12 V રેગ્યુલેટેડ પાવર સપ્લાયનો સર્કિટ ડાયગ્રામ દોરો અને સમજાવો.}

\begin{solutionbox}
-12V રેગ્યુલેટેડ સપ્લાય એનાલોગ સર્કિટ્સ માટે સ્થિર નેગેટિવ વોલ્ટેજ પ્રદાન કરે છે.

\begin{answerdiagram}{-12V પાવર સપ્લાય}
\begin{center}
\begin{circuitikz}[american]
    \draw (0,0) to[V, l=AC] (0,2);
    \draw (2,1) node[align=center] {Bridge\\Rectifier};
    \draw (3,2) to[C, l=$C_{filt}$] (3,0);
    \draw (4,1) rectangle (6,3); \node at (5,2) {7912};
    \draw (7,2) to[C, l=$C_{out}$] (7,0);
    \draw (8,2) node[right] {-12V};
    \draw (0,0) -- (8,0);
\end{circuitikz}
\end{center}
\end{answerdiagram}

\begin{itemize}
    \item \keyword{મુખ્ય ઘટક}: 7912 રેગ્યુલેટર નેગેટિવ વોલ્ટેજ માટે
\end{itemize}

\begin{mnemonicbox}
"ફુલ બ્રિજ, મોટો કેપેસિટર, 7912 નેગેટિવ રેગ્યુલેટ કરે છે"
\end{mnemonicbox}
\end{solutionbox}

\end{document}
