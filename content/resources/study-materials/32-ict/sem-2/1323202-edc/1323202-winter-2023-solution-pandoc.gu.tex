\documentclass[10pt,a4paper]{article}

% content/resources/templates/preamble.tex
\usepackage[margin=0.6in]{geometry}
\author{Milav Dabgar}
\usepackage{amsmath,amssymb,amsthm}
\usepackage{booktabs}
\usepackage{multirow}
\usepackage{xcolor}
\usepackage{tcolorbox}
\tcbuselibrary{breakable,skins}
\usepackage[colorlinks=true,linkcolor=blue]{hyperref}
\usepackage{titlesec}
\usepackage{enumitem}
\usepackage{tikz}
\usepackage{pgfplots}
\usepackage{circuitikz}
\usepackage[version=4]{mhchem}
\usepackage{longtable}
\usepackage{array}
\usepackage{float}
\usepackage{caption}
\usepackage{listings}

\lstset{
  basicstyle=\small\ttfamily,
  breaklines=true,
  breakatwhitespace=false,
  postbreak=\mbox{\textcolor{red}{$\hookrightarrow$}\space},
  float=false,
  numbers=left,
  numberstyle=\tiny\color{gray},
  numbersep=10pt,
  xleftmargin=2em,
  keywordstyle=\color{blue},
  commentstyle=\color{green!60!black},
  stringstyle=\color{purple},
  backgroundcolor=\color{gray!5},
  showstringspaces=false,
  tabsize=2,
  captionpos=b,
  keepspaces=true,
  columns=flexible
}

\pgfplotsset{compat=1.18}
\usetikzlibrary{shapes,arrows,positioning,calc,patterns,decorations.pathmorphing,decorations.markings,arrows.meta}

% Color scheme
\definecolor{headcolor}{RGB}{0,102,204}
\definecolor{keycolor}{RGB}{220,20,60}
\definecolor{solutioncolor}{RGB}{34,139,34}
\definecolor{mnemoniccolor}{RGB}{148,0,211}
\definecolor{codecolor}{RGB}{0,0,100}

% Spacing
\setlength{\parskip}{3pt}
\setlist[itemize]{nosep}
\setlist[enumerate]{nosep}

% Title formatting
\titleformat{\section}{\Large\bfseries\color{headcolor}}{\thesection}{1em}{}
\titleformat{\subsection}{\large\bfseries\color{headcolor}}{\thesubsection}{1em}{}

% Pandoc tightlist compatibility
\providecommand{\tightlist}{%
  \setlength{\itemsep}{0pt}\setlength{\parskip}{0pt}}

% Pandoc longtable compatibility
\newcounter{none}
\def\thenone{}


% content/resources/templates/gujarati-boxes.tex
\usepackage{fontspec}
\usepackage{polyglossia}

% Set Gujarati as main language (document is primarily in Gujarati)
% Note: gloss-gujarati.ldf doesn't exist in polyglossia, but it will use hyphenation patterns
\setdefaultlanguage{gujarati}
\setotherlanguage{english}

% Configure Gujarati font properly
% Use Language=Default to prevent polyglossia from trying to add language-specific features
% that don't exist for Gujarati, which causes "empty feature" warnings
\newfontfamily\gujaratifont[Script=Gujarati,AutoFakeBold=2.5,AutoFakeSlant=0.3]{Noto Sans Gujarati}
\setmainfont[Script=Gujarati,AutoFakeBold=2.5,AutoFakeSlant=0.3]{Noto Sans Gujarati}
% Use Noto Sans Gujarati for monospace to support Gujarati in text
\setmonofont[Scale=0.9]{Noto Sans Gujarati}

% Configure English to use the same font
\newfontfamily\englishfont[Script=Gujarati,AutoFakeBold=2.5,AutoFakeSlant=0.3]{Noto Sans Gujarati}

% Translations for polyglossia
\gappto\captionsgujarati{
  \renewcommand{\tablename}{કોષ્ટક}
  \renewcommand{\figurename}{આકૃતિ}
}

% Helper for TikZ nodes to ensure Gujarati font
\newcommand{\gu}[1]{{\gujaratifont #1}}

% Custom environments
\newtcolorbox{solutionbox}{
    breakable,
    enhanced,
    colback=solutioncolor!5!white,
    colframe=solutioncolor!75!black,
    fonttitle=\bfseries,
    title=જવાબ
}

\newtcolorbox{solutionboxnobreak}{
 colback=solutioncolor!5!white,
 colframe=solutioncolor!75!black,
 fonttitle=\bfseries,
 title=જવાબ
}

\newtcolorbox{keyformula}{
 breakable,
 enhanced,
 colback=keycolor!5!white,
 colframe=keycolor!75!black,
 fonttitle=\bfseries,
 title=રાસાયણિક સમીકરણ/સૂત્ર
}

\newtcolorbox{mnemonicbox}{
 breakable,
 enhanced,
 colback=mnemoniccolor!5!white,
 colframe=mnemoniccolor!75!black,
 fonttitle=\bfseries,
 title=મેમરી ટ્રીક
}


\begin{document}

\begin{center}
{\Huge\bfseries\color{headcolor} Subject Name (Gujarati)}\\[5pt]
{\LARGE 1323202 -- Winter 2023}\\[3pt]
{\large Semester 1 Study Material}\\[3pt]
{\normalsize\textit{Detailed Solutions and Explanations}}
\end{center}

\vspace{10pt}

\subsection*{પ્રશ્ન 1(અ) [3
ગુણ]}\label{uxaaauxab0uxab6uxaa8-1uxa85-3-uxa97uxaa3}

\textbf{સ્વચ્છ આકૃતિ સાથે ડીસી લોડ લાઈન વિષે સમજાવો.}

\begin{solutionbox}
DC લોડ લાઈન ટ્રાન્ઝિસ્ટરના આઉટપુટ ખાસિયતો પર એક સીધી રેખા છે જે
બધા સંભવિત ઓપરેટિંગ પોઇન્ટ્સ બતાવે છે.

\textbf{આકૃતિ:}

\begin{center}
\textbf{Mermaid Diagram (Code)}
\begin{verbatim}
{Shaded}
{Highlighting}[]
graph LR
    style O fill:\#fff,stroke:\#000
    style Vcesat fill:\#fff,stroke:\#000
    style Icsat fill:\#fff,stroke:\#000
    style Vcc fill:\#fff,stroke:\#000
    O((O)) {-{-}{-} Icsat((Icsat))}
    O {-{-}{-} Vcc((Vcc))}
    Icsat {-{-}{-} Vcesat((Vcesat))}
    Vcesat {-{-}{-} Vcc}
{Highlighting}
{Shaded}
\end{verbatim}
\end{center}

\begin{itemize}
\tightlist
\item
  \textbf{કલેક્ટર સેચુરેશન કરંટ}: જ્યારે VCE = 0, ત્યારે IC = VCC/RC
\item
  \textbf{કટઓફ વોલ્ટેજ}: જ્યારે IC = 0, ત્યારે VCE = VCC
\item
  \textbf{Q-પોઇન્ટ}: લોડ લાઈન પર ઓપરેટિંગ પોઇન્ટ
\end{itemize}

\end{solutionbox}
\begin{mnemonicbox}
``LEVEL'' - ``Load line દરેક લોડ સ્થિતિ માટે વોલ્ટેજ અને
કરંટ સ્થાપિત કરે છે''

\end{mnemonicbox}
\subsection*{પ્રશ્ન 1(બ) [4
ગુણ]}\label{uxaaauxab0uxab6uxaa8-1uxaac-4-uxa97uxaa3}

\textbf{થર્મલ રનઅવે વિગતવાર સમજાવો.}

\begin{solutionbox}
થર્મલ રનઅવે એક એવી સ્થિતિ છે જ્યાં ગરમી ટ્રાન્ઝિસ્ટરના કલેક્ટર કરંટમાં
વધારો કરે છે, જે વધુ ગરમી ઉત્પન્ન કરે છે, જે ટ્રાન્ઝિસ્ટરને નુકસાન તરફ દોરી જાય છે.

\textbf{આકૃતિ:}

\begin{verbatim}
flowchart LR
    A[તાપમાન વધે છે] {-{-} B[લીકેજ કરંટ વધે છે]}
    B {-{-} C[કલેક્ટર કરંટ વધે છે]}
    C {-{-} D[વધુ પાવર વપરાશ]}
    D {-{-} E[વધુ તાપમાન વધારો]}
    E {-{-} A}
\end{verbatim}

\begin{itemize}
\tightlist
\item
  \textbf{ગરમી ઉત્પાદન}: પાવર વપરાશ = VCE \times IC
\item
  \textbf{મહત્વપૂર્ણ અસર}: વધારેલ જંક્શન તાપમાન VBE ઘટાડે છે
\item
  \textbf{નિવારણ}: હીટ સિંક, થર્મલ સ્ટેબલાઇઝેશન સર્કિટ્સ, યોગ્ય બાયસિંગ
\item
  \textbf{ખતરો}: નિયંત્રિત ન કરવામાં આવે તો ટ્રાન્ઝિસ્ટરને નષ્ટ કરી શકે છે
\end{itemize}

\end{solutionbox}
\begin{mnemonicbox}
``HEAT'' - ``વધુ ઉત્સર્જન તાપમાનમાં વધારો કરે છે''

\end{mnemonicbox}
\subsection*{પ્રશ્ન 1(ક) [7
ગુણ]}\label{uxaaauxab0uxab6uxaa8-1uxa95-7-uxa97uxaa3}

\textbf{ટુ સ્ટેજ R-C કપલ્ડ એમ્પ્લીફાયરનો સર્કિટ ડાયાગ્રામ અને ફ્રીક્વન્શી રિસ્પોન્સ
દોરો. દરેક કમ્પોનન્ટનું મહત્વ સમજાવો.}

\begin{solutionbox}
R-C કપલ્ડ એમ્પ્લીફાયર મલ્ટીપલ ટ્રાન્ઝિસ્ટર સ્ટેજ્સને જોડવા માટે
કેપેસિટર્સનો ઉપયોગ કરે છે જેથી ઉચ્ચ ગેઇન મેળવી શકાય.

\textbf{આકૃતિ:}

\begin{verbatim}
+{-{-}{-}{-}{-}{-}+            +{-}{-}{-}{-}{-}{-}+}
|      |            |      |
|  Q1  |            |  Q2  |
|      |            |      |
+{-{-}{-}{-}{-}{-}+            +{-}{-}{-}{-}{-}{-}+}
   |                   |
   |                   |
   R1       C2         R2
   |        ||         |
   +{-{-}{-}||{-}{-}{-}+{-}{-}{-}{-}{-}{-}{-}{-}{-}{-}+}
        C1             

Vin o{-{-}{-}{-}||{-}{-}{-}+        +{-}{-}{-}{-}{-}{-}o Vout}
             |         |
             R3        R4
             |         |
             +         +
\end{verbatim}

\textbf{ફ્રીક્વન્સી રિસ્પોન્સ:}

\begin{center}
\textbf{Mermaid Diagram (Code)}
\begin{verbatim}
{Shaded}
{Highlighting}[]
graph LR
    title["ફ્રીક્વન્સી રિસ્પોન્સ: ગેઇન (dB) vs ફ્રીક્વન્સી"]

    F10["10 Hz{nGain: 10 dB"] {-}{-}{} F100["100 Hz{}nGain: 30 dB"]}
    F100 {-{-}{} F1k["1 kHz{}nGain: 40 dB"]}
    F1k {-{-}{} F10k["10 kHz{}nGain: 40 dB"]}
    F10k {-{-}{} F100k["100 kHz{}nGain: 30 dB"]}
    F100k {-{-}{} F1M["1 MHz{}nGain: 10 dB"]}
{Highlighting}
{Shaded}
\end{verbatim}
\end{center}

\begin{itemize}
\tightlist
\item
  \textbf{કપલિંગ કેપેસિટર્સ}: DC બ્લોક કરે છે, સ્ટેજ્સ વચ્ચે AC સિગ્નલ ટ્રાન્સફર કરે છે
\item
  \textbf{બાયસિંગ રેસિસ્ટર્સ}: ટ્રાન્ઝિસ્ટર ઓપરેશન માટે યોગ્ય Q-પોઇન્ટ સ્થાપિત કરે છે
\item
  \textbf{બાયપાસ કેપેસિટર્સ}: નેગેટિવ ફીડબેકથી ગેઇન ઘટાડો રોકે છે
\item
  \textbf{બેન્ડવિડ્થ}: લો અને હાઈ કટઓફ ફ્રીક્વન્સી વચ્ચેનો રેન્જ
\end{itemize}

\end{solutionbox}
\begin{mnemonicbox}
``CARS'' - ``કપલિંગ કેપેસિટર્સ રેસિસ્ટન્સ સેપરેશન માટે મદદ કરે
છે''

\end{mnemonicbox}
\subsection*{અથવા}\label{uxa85uxaa5uxab5}

\subsection*{પ્રશ્ન 1(ક) [7
ગુણ]}\label{uxaaauxab0uxab6uxaa8-1uxa95-7-uxa97uxaa3-1}

\textbf{એમ્પ્લીફાયરમાં નેગેટીવ અને પોઝીટીવ ફીડબેક સરખાવો.}

\begin{solutionbox}
ફીડબેક સિસ્ટમ્સ આઉટપુટના એક ભાગને ઇનપુટ પર પાછો મોકલે છે જેમાં
ધ્રુવીયતાના આધારે અલગ અસરો થાય છે.


{\def\LTcaptype{none} % do not increment counter
\begin{longtable}[]{@{}lll@{}}
\toprule\noalign{}
પેરામીટર & નેગેટિવ ફીડબેક & પોઝિટિવ ફીડબેક \\
\midrule\noalign{}
\endhead
\bottomrule\noalign{}
\endlastfoot
ગેઇન & ઘટાડે છે & વધારે છે \\
બેન્ડવિડ્થ & વધારે છે & ઘટાડે છે \\
સ્ટેબિલિટી & સુધારે છે & ઘટાડે છે \\
ડિસ્ટોર્શન & ઘટાડે છે & વધારે છે \\
નોઇઝ & ઘટાડે છે & વધારે છે \\
ઇનપુટ/આઉટપુટ ઇમ્પીડન્સ & નિયંત્રિત કરી શકાય છે & અનિશ્ચિત \\
એપ્લિકેશન્સ & એમ્પ્લિફાયર, રેગ્યુલેટર & ઓસિલેટર, શ્મિટ ટ્રિગર \\
\end{longtable}
}

\begin{itemize}
\tightlist
\item
  \textbf{નેગેટિવ ફીડબેક}: આઉટપુટ ઇનપુટથી 180^\circ શિફ્ટ હોય છે
\item
  \textbf{પોઝિટિવ ફીડબેક}: આઉટપુટ ઇનપુટથી 0^\circ શિફ્ટ હોય છે
\item
  \textbf{બાર્ખાઉસન ક્રાઇટેરિયા}: યુનિટી ગેઇન સાથે પોઝિટિવ ફીડબેક ઓસિલેશન ઉત્પન્ન
  કરે છે
\end{itemize}

\end{solutionbox}
\begin{mnemonicbox}
``SIGN'' - ``સ્ટેબિલિટી ગેઇન નિગેશન સાથે વધે છે''

\end{mnemonicbox}
\subsection*{પ્રશ્ન 2(અ) [3
ગુણ]}\label{uxaaauxab0uxab6uxaa8-2uxa85-3-uxa97uxaa3}

\textbf{ઓસિલેશન માટે બારખૌસન ક્રાઈટરીઆ (Barkhausen's criteria) જણાવો અને
સમજાવો.}

\begin{solutionbox}
બાર્ખાઉસન ક્રાઇટેરિયા ફીડબેક સિસ્ટમમાં સતત ઓસિલેશન માટેની શરતો
નિર્ધારિત કરે છે.

\textbf{આકૃતિ:}

\begin{verbatim}
flowchart TD
    A[એમ્પ્લિફાયર] {-{-} B[ફીડબેક નેટવર્ક]}
    B {-{-} A}
    A {-{-} "લૂપ ગેઇન = 1" {-}{-} C[સતત ઓસિલેશન]}
    A {-{-} "લૂપ ગેઇન  1" {-}{-} D[ડેમ્પ્ડ ઓસિલેશન]}
    A {-{-} "લૂપ ગેઇન  1" {-}{-} E[વધતા ઓસિલેશન]}
\end{verbatim}

\begin{itemize}
\tightlist
\item
  \textbf{ગેઇન શરત}: લૂપ ગેઇન (A\timesβ) 1 (યુનિટી) હોવી જોઈએ
\item
  \textbf{ફેઝ શરત}: કુલ ફેઝ શિફ્ટ 0^\circ અથવા 360^\circ હોવી જોઈએ
\item
  \textbf{વ્યવહારિક અમલીકરણ}: પ્રારંભિક લૂપ ગેઇન \textgreater{} 1, પછી 1 પર
  સ્થિર થાય છે
\end{itemize}

\end{solutionbox}
\begin{mnemonicbox}
``LOOP'' - ``લૂપની સમગ્ર આઉટપુટ ફેઝ''

\end{mnemonicbox}
\subsection*{પ્રશ્ન 2(બ) [4
ગુણ]}\label{uxaaauxab0uxab6uxaa8-2uxaac-4-uxa97uxaa3}

\textbf{ફિક્સ્ડ બાયસ, કલેક્ટર ટુ બેઝ બાયસ અને વોલ્ટેજ ડિવાઈડર બાયસ પદ્ધતિઓની
સરખામણી કરો.}

\begin{solutionbox}
વિવિધ બાયસિંગ તકનીકો સ્થિરતા અને તાપમાન ક્ષતિપૂર્તિના વિવિધ
સ્તરો પ્રદાન કરે છે.


{\def\LTcaptype{none} % do not increment counter
\begin{longtable}[]{@{}llll@{}}
\toprule\noalign{}
પેરામીટર & ફિક્સ્ડ બાયસ & કલેક્ટર-બેઝ બાયસ & વોલ્ટેજ ડિવાઇડર બાયસ \\
\midrule\noalign{}
\endhead
\bottomrule\noalign{}
\endlastfoot
સ્ટેબિલિટી & નબળી & વધુ સારી & ઉત્તમ \\
સર્કિટ જટિલતા & સરળ & મધ્યમ & જટિલ \\
તાપમાન સ્ટેબિલિટી & નબળી & મધ્યમ & સારી \\
કોમ્પોનેન્ટ્સ & 1 રેસિસ્ટર & 1 રેસિસ્ટર & 3-4 રેસિસ્ટર \\
સ્ટેબિલિટી ફેક્ટર (S) & ઉચ્ચ & મધ્યમ & નીચો \\
\end{longtable}
}

\begin{itemize}
\tightlist
\item
  \textbf{ફિક્સ્ડ બાયસ}: બેઝથી VCC સુધી એક રેસિસ્ટર
\item
  \textbf{કલેક્ટર-બેઝ બાયસ}: કલેક્ટરથી બેઝ સુધી ફીડબેક રેસિસ્ટર
\item
  \textbf{વોલ્ટેજ ડિવાઇડર}: બે રેસિસ્ટર સ્થિર રેફરન્સ વોલ્ટેજ બનાવે છે
\end{itemize}

\end{solutionbox}
\begin{mnemonicbox}
``STORM'' - ``સ્ટેબિલિટી રેસિસ્ટર મેથડ્સ દ્વારા ઓપ્ટિમાઇઝ
થાય છે''

\end{mnemonicbox}
\subsection*{પ્રશ્ન 2(ક) [7
ગુણ]}\label{uxaaauxab0uxab6uxaa8-2uxa95-7-uxa97uxaa3}

\textbf{હાર્ટલી ઓસીલેટર પર ટૂંક નોંધ લખો.}

\begin{solutionbox}
હાર્ટલી ઓસિલેટર એક LC ઓસિલેટર છે જેમાં ફીડબેક માટે એક ટેપ્ડ ઇન્ડક્ટર
હોય છે.

\textbf{આકૃતિ:}

\begin{center}
\textbf{Mermaid Diagram (Code)}
\begin{verbatim}
{Shaded}
{Highlighting}[]
graph LR
    A[એમ્પ્લિફાયર] {-{-}{-} B[ફીડબેક નેટવર્ક]}
    B {-{-}{-} A}
    subgraph "ફીડબેક નેટવર્ક"
    direction LR
    L1[L1] {-{-}{-} L2[L2]}
    L1 {-{-}{-} C1[C]}
    L2 {-{-}{-} C1}
    end
{Highlighting}
{Shaded}
\end{verbatim}
\end{center}

\begin{itemize}
\tightlist
\item
  \textbf{સર્કિટ કોમ્પોનેન્ટ્સ}: એમ્પ્લિફાયર, ટેપ્ડ ઇન્ડક્ટર (L1+L2), કેપેસિટર C
\item
  \textbf{ફ્રીક્વન્સી ફોર્મ્યુલા}: f = 1/[2π\sqrt(LC)] જ્યાં L = L1+L2
\item
  \textbf{લાભ}: સરળ ડિઝાઇન, સારી ફ્રીક્વન્સી સ્ટેબિલિટી
\item
  \textbf{નુકસાન}: ઇન્ડક્ટર્સનું કદ, મર્યાદિત ફ્રીક્વન્સી રેન્જ
\item
  \textbf{એપ્લિકેશન્સ}: RF સિગ્નલ જનરેટર, રેડિયો રિસીવર, કોમ્યુનિકેશન
\end{itemize}

\end{solutionbox}
\begin{mnemonicbox}
``TILC'' - ``ટેપ્ડ ઇન્ડક્ટર LC સર્કિટ સાથે''

\end{mnemonicbox}
\subsection*{અથવા}\label{uxa85uxaa5uxab5-1}

\subsection*{પ્રશ્ન 2(અ) [3
ગુણ]}\label{uxaaauxab0uxab6uxaa8-2uxa85-3-uxa97uxaa3-1}

\textbf{ટ્રાન્ઝિસ્ટરનું સ્વિચ તરીકે કાર્ય સમજાવો.}

\begin{solutionbox}
ટ્રાન્ઝિસ્ટર કટઓફ (OFF) અને સેચુરેશન (ON) રીજન્સ વચ્ચે ડિજિટલ
એપ્લિકેશન્સ માટે સ્વિચ તરીકે કામ કરે છે.

\textbf{આકૃતિ:}

\begin{verbatim}
flowchart LR
    A[ઇનપુટ] {-{-} B\{ટ્રાન્ઝિસ્ટર\}}
    B {-{-} "સેચુરેશન (ON)" {-}{-} C[આઉટપુટ LOW]}
    B {-{-} "કટઓફ (OFF)" {-}{-} D[આઉટપુટ HIGH]}
\end{verbatim}

\begin{itemize}
\tightlist
\item
  \textbf{કટઓફ રીજન}: VBE \textless{} 0.7V, ઓપન સ્વિચ તરીકે કાર્ય કરે છે, VCE
  \approx VCC
\item
  \textbf{સેચુરેશન રીજન}: VBE \textgreater{} 0.7V, ક્લોઝ્ડ સ્વિચ તરીકે કાર્ય કરે
  છે, VCE \approx 0.2V
\item
  \textbf{સ્વિચિંગ ટાઇમ}: જંક્શન કેપેસિટન્સ દ્વારા મર્યાદિત
\end{itemize}

\end{solutionbox}
\begin{mnemonicbox}
``COPS'' - ``કટઓફ-સેચુરેશન-સ્વિચિંગ ઉત્પન્ન કરે છે''

\end{mnemonicbox}
\subsection*{પ્રશ્ન 2(બ) [4
ગુણ]}\label{uxaaauxab0uxab6uxaa8-2uxaac-4-uxa97uxaa3-1}

\textbf{હીટ સિંક વ્યાખ્યાયિત કરો. હીટ સિંકના પ્રકારોની યાદી બનાવો અને તેની
એપ્લિકેશન લખો.}

\begin{solutionbox}
હીટ સિંક એક થર્મલ કન્ડક્ટર છે જે ઇલેક્ટ્રોનિક કોમ્પોનેન્ટ્સમાંથી ગરમી
દૂર કરે છે.

\textbf{આકૃતિ:}

\begin{verbatim}
     ||||||||
    /||||||||{ હીટ સિંક}
   /||||||||||{}
  /||||||||||||{}
 /|||||||||||||{}
+{-{-}{-}{-}{-}{-}{-}{-}{-}{-}{-}{-}{-}{-}+}
| ટ્રાન્ઝિસ્ટર  |
+{-{-}{-}{-}{-}{-}{-}{-}{-}{-}{-}{-}{-}{-}+}
\end{verbatim}

\textbf{હીટ સિંકના પ્રકારો:}

{\def\LTcaptype{none} % do not increment counter
\begin{longtable}[]{@{}lll@{}}
\toprule\noalign{}
પ્રકાર & વર્ણન & એપ્લિકેશન \\
\midrule\noalign{}
\endhead
\bottomrule\noalign{}
\endlastfoot
પેસિવ & કોઈ ચલિત ભાગો નહીં, કુદરતી કન્વેક્શન & ઓછી પાવર ડિવાઇસીસ \\
એક્ટિવ & ફેન અથવા પંપ સાથે & હાઈ પાવર એમ્પ્લિફાયર \\
લિક્વિડ-કૂલ્ડ & હીટ ટ્રાન્સફર માટે પ્રવાહી વાપરે છે & કોમ્પ્યુટિંગ સિસ્ટમ \\
ફિન્ડ & મલ્ટીપલ ફિન્સ સરફેસ એરિયા વધારે છે & પાવર ટ્રાન્ઝિસ્ટર \\
\end{longtable}
}

\begin{itemize}
\tightlist
\item
  \textbf{હેતુ}: થર્મલ રનઅવે અને કોમ્પોનેન્ટ નિષ્ફળતા રોકે છે
\item
  \textbf{મટીરિયલ}: એલ્યુમિનિયમ, કોપર, અથવા હાઈ થર્મલ કન્ડક્ટિવિટી વાળા એલોય
\end{itemize}

\end{solutionbox}
\begin{mnemonicbox}
``COOL'' - ``કન્ડક્ટિંગ લોકલ હીટને બહાર લઈ જાય છે''

\end{mnemonicbox}
\subsection*{પ્રશ્ન 2(ક) [7
ગુણ]}\label{uxaaauxab0uxab6uxaa8-2uxa95-7-uxa97uxaa3-1}

\textbf{એમ્પ્લીફાયરમાં નેગેટીવ ફીડબેક ના ફાયદા અને ગેરફાયદાને વિગતવાર સમજાવો.}

\begin{solutionbox}
નેગેટિવ ફીડબેક આઉટપુટ સિગ્નલના એક ભાગને વિરુદ્ધ ફેઝમાં ઇનપુટ પર પાછો
મોકલે છે.


{\def\LTcaptype{none} % do not increment counter
\begin{longtable}[]{@{}ll@{}}
\toprule\noalign{}
ફાયદા & ગેરફાયદા \\
\midrule\noalign{}
\endhead
\bottomrule\noalign{}
\endlastfoot
ગેઇન સ્ટેબિલાઇઝ કરે છે & સમગ્ર ગેઇન ઘટાડે છે \\
બેન્ડવિડ્થ વધારે છે & વધુ કોમ્પોનેન્ટ્સની જરૂર પડે છે \\
ડિસ્ટોર્શન ઘટાડે છે & વધુ પાવરનો વપરાશ \\
નોઇઝ ઘટાડે છે & જટિલ સર્કિટ ડિઝાઇન \\
ઇનપુટ/આઉટપુટ ઇમ્પીડન્સ નિયંત્રિત કરે છે & અયોગ્ય ડિઝાઇન થાય તો સંભવિત ઓસિલેશન \\
લિનિયરિટી સુધારે છે & ફીડબેક નેટવર્કમાં સિગ્નલ લોસ \\
\end{longtable}
}

\textbf{આકૃતિ:}

\begin{center}
\textbf{Mermaid Diagram (Code)}
\begin{verbatim}
{Shaded}
{Highlighting}[]
graph LR
    A[ઇનપુટ] {-{-}{} B[એમ્પ્લિફાયર]}
    B {-{-}{} C[આઉટપુટ]}
    C {-{-} "ફીડબેક નેટવર્ક" {-}{-}{} D[સબટ્રેક્ટર]}
    D {-{-}{} B}
{Highlighting}
{Shaded}
\end{verbatim}
\end{center}

\begin{itemize}
\tightlist
\item
  \textbf{ગેઇન સ્ટેબિલાઇઝેશન}: ગેઇનને પેસિવ કોમ્પોનેન્ટ્સ પર આધારિત બનાવે છે
\item
  \textbf{બેન્ડવિડ્થ એક્સટેન્શન}: ગેઇન ઘટાડા ફેક્ટર જેટલી વધે છે
\item
  \textbf{ફીડબેક ફેક્ટર}: β સુધારાની માત્રા નક્કી કરે છે
\end{itemize}

\end{solutionbox}
\begin{mnemonicbox}
``STABLE'' - ``સ્ટેબિલાઇઝ્ડ ટ્રાન્સમિશન એન્ડ બેન્ડવિડ્થ વિથ
લેસ એરર''

\end{mnemonicbox}
\subsection*{પ્રશ્ન 3(અ) [3
ગુણ]}\label{uxaaauxab0uxab6uxaa8-3uxa85-3-uxa97uxaa3}

\textbf{SCR નો સિમ્બોલ દોરો અને SCR નું કાર્ય સમજાવો.}

\begin{solutionbox}
સિલિકોન કંટ્રોલ્ડ રેક્ટિફાયર (SCR) એ ત્રણ ટર્મિનલ વાળું PNPN
ચાર-લેયર ડિવાઇસ છે.

\textbf{સિમ્બોલ:}

\begin{verbatim}
      A(એનોડ)
       |
       |
       v
    +{-{-}{-}{-}{-}+}
    |     |
G{-{-}|     |}
    |     |
    +{-{-}{-}{-}{-}+}
       \^{}
       |
       |
      K(કેથોડ)
\end{verbatim}

\begin{itemize}
\tightlist
\item
  \textbf{સ્ટ્રક્ચર}: P-N-P-N ચાર-લેયર સેમિકન્ડક્ટર ડિવાઇસ
\item
  \textbf{ઓપરેશન}: ગેટ ટ્રિગર ન થાય ત્યાં સુધી OFF રહે છે, ત્યારબાદ કરંટ હોલ્ડિંગ
  વેલ્યુથી નીચે ન જાય ત્યાં સુધી કન્ડક્ટ કરે છે
\item
  \textbf{ટર્મિનલ્સ}: એનોડ, કેથોડ, ગેટ
\end{itemize}

\end{solutionbox}
\begin{mnemonicbox}
``AGK'' - ``એનોડ-ગેટ કેથોડ કરંટને નિયંત્રિત કરે છે''

\end{mnemonicbox}
\subsection*{પ્રશ્ન 3(બ) [4
ગુણ]}\label{uxaaauxab0uxab6uxaa8-3uxaac-4-uxa97uxaa3}

\textbf{સર્કિટ ડાયાગ્રામ સાથે SCR ની ટુ ટ્રાન્ઝિસ્ટર એનાલોજી સમજાવો}

\begin{solutionbox}
SCRને જંક્શન શેર કરતા ઇન્ટરકનેક્ટેડ PNP અને NPN ટ્રાન્ઝિસ્ટર તરીકે રજૂ
કરી શકાય છે.

\textbf{આકૃતિ:}

\begin{verbatim}
       એનોડ
         |
    +{-{-}{-}{-}|{-}{-}{-}{-}+}
    |    v    |
    |  +{-{-}{-}  |}
    |  | PNP  |
    |  +{-{-}{-}{-}+ |}
    |       | |
ગેટ |       v |
 {-{-}{-}{-}|{-}{-}{-}+  +{-}{-}}
    |   |  | NPN
    |   +{-{-}+{-}{-}{-}{-}+}
    |          |
    +{-{-}{-}{-}{-}{-}{-}{-}{-}{-}|{-}{-}}
              |
              v
            કેથોડ
\end{verbatim}

\begin{itemize}
\tightlist
\item
  \textbf{PNP સેક્શન}: ઉપરનો ટ્રાન્ઝિસ્ટર જેનો કલેક્ટર NPN બેઝ સાથે જોડાયેલો છે
\item
  \textbf{NPN સેક્શન}: નીચેનો ટ્રાન્ઝિસ્ટર જેનો કલેક્ટર PNP બેઝ સાથે જોડાયેલો છે
\item
  \textbf{ટ્રિગરિંગ}: નાનો ગેટ કરંટ NPN ચાલુ કરે છે, જે PNP ચાલુ કરે છે
\item
  \textbf{રિજનરેટિવ એક્શન}: દરેક ટ્રાન્ઝિસ્ટર બીજાને બેઝ કરંટ આપે છે
\end{itemize}

\end{solutionbox}
\begin{mnemonicbox}
``PNPN'' - ``પોઝિટિવ-નેગેટિવ-પોઝિટિવ-નેગેટિવ લેયર્સ''

\end{mnemonicbox}
\subsection*{પ્રશ્ન 3(ક) [7
ગુણ]}\label{uxaaauxab0uxab6uxaa8-3uxa95-7-uxa97uxaa3}

\textbf{સર્કિટ ડાયાગ્રામ સાથે TRIAC આધારિત ફેન રેગ્યુલેટરનું કાર્ય સમજાવો.}

\begin{solutionbox}
TRIAC-આધારિત ફેન રેગ્યુલેટર ફેઝ કંટ્રોલ દ્વારા AC પાવર નિયંત્રિત કરે
છે.

\textbf{સર્કિટ ડાયાગ્રામ:}

\begin{verbatim}
        +{-{-}{-}+  R1}
AC o{-{-}{-}{-}+   +{-}{-}//{-}{-}+}
        |           |
        | C1        |
        +{-{-}{-}||{-}{-}{-}{-}{-}{-}+{-}{-}{-}{-}+}
                    |    |
                    Z   MT1
                    |    |
                   G|    |
                    |   \_V\_
                    +{-{-}|   |{-}{-}+{-}{-}o ફેન}
                       |\_\_\_|  |
                        MT2   |
                              |
                              |
AC o{-{-}{-}{-}{-}{-}{-}{-}{-}{-}{-}{-}{-}{-}{-}{-}{-}{-}{-}{-}{-}{-}{-}{-}{-}{-}|}
\end{verbatim}

\begin{itemize}
\tightlist
\item
  \textbf{ફેઝ કંટ્રોલ}: TRIAC નો ફાયરિંગ એંગલ બદલીને પાવર કંટ્રોલ કરે છે
\item
  \textbf{ડાયક}: TRIAC માટે બાયડાયરેક્શનલ ટ્રિગરિંગ આપે છે
\item
  \textbf{RC ટાઇમિંગ સર્કિટ}: R1 અને C1 ફેઝ ડિલે સેટ કરે છે
\item
  \textbf{વેરિયેબલ રેસિસ્ટર}: સ્પીડ કંટ્રોલ માટે ફેઝ ડિલે એડજસ્ટ કરે છે
\item
  \textbf{પ્રોટેક્શન}: RC સ્નબર ખોટા ટ્રિગરિંગને રોકે છે
\end{itemize}

\end{solutionbox}
\begin{mnemonicbox}
``TRIAC'' - ``ટ્રિગર્ડ રિસ્પોન્સ ઇન AC સર્કિટ્સ''

\end{mnemonicbox}
\subsection*{અથવા}\label{uxa85uxaa5uxab5-2}

\subsection*{પ્રશ્ન 3(અ) [3
ગુણ]}\label{uxaaauxab0uxab6uxaa8-3uxa85-3-uxa97uxaa3-1}

\textbf{DIAC અને TRIAC ની V-I લાક્ષણિકતાઓ દોરો.}

\begin{solutionbox}
DIACs અને TRIACs બાયડાયરેક્શનલ ડિવાઇસીસ છે જેમાં સિમેટ્રિકલ
લાક્ષણિકતાઓ હોય છે.

\textbf{DIAC ખાસિયતો:}

\begin{verbatim}
xychart{-beta}
    title "DIAC V{-I લાક્ષણિકતાઓ"}
    x{-axis [{-}40, {-}30, {-}20, {-}10, 0, 10, 20, 30, 40]}
    y{-axis "કરંટ (mA)" {-}30 {-}{-} 30}
    line [30, 5, 0, 0, 0, 0, 0, 5, 30]
\end{verbatim}

\textbf{TRIAC ખાસિયતો:}

\begin{verbatim}
xychart{-beta}
    title "TRIAC V{-I લાક્ષણિકતાઓ"}
    x{-axis [{-}40, {-}30, {-}20, {-}10, 0, 10, 20, 30, 40]}
    y{-axis "કરંટ (mA)" {-}40 {-}{-} 40}
    line [40, 40, 40, 5, 0, 5, 40, 40, 40]
\end{verbatim}

\begin{itemize}
\tightlist
\item
  \textbf{DIAC}: બાયડાયરેક્શનલ ડાયોડ જે બ્રેકઓવર વોલ્ટેજ પછી કન્ડક્ટ કરે છે
\item
  \textbf{TRIAC}: ત્રણ-ટર્મિનલ ડિવાઇસ જે ટ્રિગર થાય ત્યારે બંને દિશામાં કન્ડક્ટ કરે
  છે
\end{itemize}

\end{solutionbox}
\begin{mnemonicbox}
``BIBO'' - ``બાયડાયરેક્શનલ ઇન, બાયડાયરેક્શનલ આઉટ''

\end{mnemonicbox}
\subsection*{પ્રશ્ન 3(બ) [4
ગુણ]}\label{uxaaauxab0uxab6uxaa8-3uxaac-4-uxa97uxaa3-1}

\textbf{SCR ની ગેટ ટ્રિગરિંગ પદ્ધતિ સમજાવો}

\begin{solutionbox}
ગેટ ટ્રિગરિંગ SCRને સક્રિય કરવાની સૌથી સામાન્ય પદ્ધતિ છે.

\textbf{આકૃતિ:}

\begin{verbatim}
        A
        |
     +{-{-}{-}{-}{-}+}
     |     |
     |     |
     |  +{-|}
RC {-{-}|{-}{-}+  |}
     |     |
     +{-{-}{-}{-}{-}+}
        |
        K
\end{verbatim}

\begin{itemize}
\tightlist
\item
  \textbf{ગેટ પલ્સ}: ગેટ અને કેથોડ વચ્ચે નાનો કરંટ લાગુ કરવામાં આવે છે
\item
  \textbf{ટ્રિગરિંગ મેથડ્સ}: DC, AC, અથવા પલ્સ સિગ્નલ્સ
\item
  \textbf{કરંટ જરૂરિયાતો}: સામાન્ય રીતે 5-20mA ગેટ કરંટ
\item
  \textbf{ફાયદા}: હાઈ-પાવર સર્કિટ્સનું લો પાવર કંટ્રોલ
\end{itemize}

\end{solutionbox}
\begin{mnemonicbox}
``GATE'' - ``ગેઇન એક્ટિવેશન થ્રુ ઇલેક્ટ્રોન ફ્લો''

\end{mnemonicbox}
\subsection*{પ્રશ્ન 3(ક) [7
ગુણ]}\label{uxaaauxab0uxab6uxaa8-3uxa95-7-uxa97uxaa3-1}

\textbf{ડીસી પાવર કંટ્રોલ માટે SCRની એપ્લિકેશન સમજાવો.}

\begin{solutionbox}
SCR વેરિયેબલ ડ્યુટી સાયકલ્સ પર સપ્લાય વોલ્ટેજને ચોપિંગ કરીને DC પાવર
નિયંત્રિત કરે છે.

\textbf{સર્કિટ:}

\begin{verbatim}
    +{-{-}{-}{-}{-}{-}{-}+       SCR}
    |       |       / |
DC{-{-}|{-}{-}{-}{-}{-}{-}{-}|{-}{-}{-}{-}{-}{-}/{-}{-}|{-}{-}+{-}{-}{-}o આઉટપુટ}
    |       |          |   |
    | PWM   |          |   |
    | Ctrl  |{-{-}{-}{-}.     |   |}
    |       |    |     |   |
    +{-{-}{-}{-}{-}{-}{-}+    +{-}{-}{-}{-}{-}|{-}{-}{-}+}
                       |
                       |
    +{-{-}{-}{-}{-}{-}{-}{-}{-}{-}{-}{-}{-}{-}{-}{-}{-}{-}|}
    |                  |
    GND{-{-}{-}{-}{-}{-}{-}{-}{-}{-}{-}{-}{-}{-}{-}{-}+}
\end{verbatim}

\begin{itemize}
\tightlist
\item
  \textbf{ફેઝ કંટ્રોલ}: સરેરાશ પાવર નિયંત્રિત કરવા માટે ફાયરિંગ એંગલ બદલે છે
\item
  \textbf{PWM કંટ્રોલ}: કાર્યક્ષમ નિયંત્રણ માટે પલ્સ વિડ્થ મોડ્યુલેશન
\item
  \textbf{એપ્લિકેશન્સ}: DC મોટર સ્પીડ કંટ્રોલ, ડિમિંગ, હીટિંગ
\item
  \textbf{ફાયદા}: હાઈ એફિશિયન્સી, કોઈ મૂવિંગ પાર્ટ્સ નહીં, વિશ્વસનીય
\item
  \textbf{મર્યાદાઓ}: યુનિડાયરેક્શનલ કરંટ ફ્લો, કોમ્યુટેશનની જરૂર પડે છે
\end{itemize}

\end{solutionbox}
\begin{mnemonicbox}
``POWER'' - ``પલ્સ ઓપરેશન વિથ ઇલેક્ટ્રોનિક રેગ્યુલેશન''

\end{mnemonicbox}
\subsection*{પ્રશ્ન 4(અ) [3
ગુણ]}\label{uxaaauxab0uxab6uxaa8-4uxa85-3-uxa97uxaa3}

\textbf{Ideal OP-AMP ની લાક્ષણિકતાઓની સૂચિ બનાવો.}

\begin{solutionbox}
આદર્શ ઓપરેશનલ એમ્પ્લિફાયર્સ સંપૂર્ણ લાક્ષણિકતાઓ ધરાવે છે જેને વાસ્તવિક
ઉપકરણો અનુમાનિત કરે છે.


{\def\LTcaptype{none} % do not increment counter
\begin{longtable}[]{@{}ll@{}}
\toprule\noalign{}
લાક્ષણિકતા & આદર્શ મૂલ્ય \\
\midrule\noalign{}
\endhead
\bottomrule\noalign{}
\endlastfoot
ઓપન લૂપ ગેઇન & અનંત \\
ઇનપુટ ઇમ્પીડન્સ & અનંત \\
આઉટપુટ ઇમ્પીડન્સ & શૂન્ય \\
બેન્ડવિડ્થ & અનંત \\
CMRR & અનંત \\
સ્લ્યુ રેટ & અનંત \\
ઓફસેટ વોલ્ટેજ & શૂન્ય \\
\end{longtable}
}

\begin{itemize}
\tightlist
\item
  \textbf{પ્રેક્ટિકલ વેલ્યુ}: વાસ્તવિક ઓપ-એમ્પ્સની મર્યાદાઓ હોય છે
\item
  \textbf{નિહિતાર્થ}: સર્કિટ ડિઝાઇનમાં વાસ્તવિક મર્યાદાઓને ધ્યાનમાં લેવી જોઈએ
\end{itemize}

\end{solutionbox}
\begin{mnemonicbox}
``IBOCSS'' - ``અનંત બેન્ડવિડ્થ, ઓપન-લૂપ ગેઇન, CMRR, સ્લ્યુ
રેટ, અને સેન્સિટિવિટી''

\end{mnemonicbox}
\subsection*{પ્રશ્ન 4(બ) [4
ગુણ]}\label{uxaaauxab0uxab6uxaa8-4uxaac-4-uxa97uxaa3}

\textbf{સર્કિટ ડાયાગ્રામ સાથે OP-AMP નો ઉપયોગ કરીને ડીફરન્સીઅલ એમ્પ્લીફાયરનું
કાર્ય સમજાવો.}

\begin{solutionbox}
ડિફરેન્શિયલ એમ્પ્લિફાયર બે ઇનપુટ્સ વચ્ચેના વોલ્ટેજ તફાવતને એમ્પ્લિફાય
કરે છે.

\textbf{સર્કિટ:}

\begin{verbatim}
              R2
       +{-{-}{-}{-}{-}{-}//{-}{-}{-}{-}{-}{-}+}
       |                |
       |           +{-{-}{-}{-}+}
       |           |    |
       |    R1     |    |
  V1 o{-+{-}{-}{-}//{-}{-}{-}{-}+    +{-}{-}{-}{-}o Vout}
                  \_|+   |
                 /      |
                /       |
               /\_\_\_\_\_\_  |
                  {-|    |}
       |    R1     |    |
  V2 o{-+{-}{-}{-}//{-}{-}{-}{-}+    |}
       |                |
       |                |
       +{-{-}{-}{-}{-}{-}//{-}{-}{-}{-}{-}{-}+}
              R2
\end{verbatim}

\begin{itemize}
\tightlist
\item
  \textbf{ગેઇન ફોર્મ્યુલા}: Vout = (V1-V2) \times (R2/R1)
\item
  \textbf{કોમન મોડ રિજેક્શન}: બંને ઇનપુટ્સ માટે સામાન્ય સિગ્નલ્સને દબાવે છે
\item
  \textbf{એપ્લિકેશન્સ}: ઇન્સ્ટ્રુમેન્ટેશન, મેડિકલ ઇક્વિપમેન્ટ, ઓડિયો
\end{itemize}

\end{solutionbox}
\begin{mnemonicbox}
``DIFF'' - ``ડ્યુઅલ ઇનપુટ ફોર ફીડબેક''

\end{mnemonicbox}
\subsection*{પ્રશ્ન 4(ક) [7
ગુણ]}\label{uxaaauxab0uxab6uxaa8-4uxa95-7-uxa97uxaa3}

\textbf{OP-AMP ને ઇન્વર્ટિંગ એમ્પ્લીફાયર (ક્લોઝ્ડ લૂપ) તરીકે સમજાવો અને વોલ્ટેજ ગેઇન નું
સમીકરણ મેળવો.}

\begin{solutionbox}
ઇન્વર્ટિંગ એમ્પ્લિફાયર ઇનપુટનું ઇન્વર્ટેડ અને એમ્પ્લિફાઇડ વર્ઝન આઉટપુટ
તરીકે આપે છે.

\textbf{સર્કિટ:}

\begin{verbatim}
          Rf
     +{-{-}{-}{-}//{-}{-}{-}{-}+}
     |            |
     |            |
     |    +{-{-}{-}{-}{-}{-}{-}+{-}{-}{-}{-}o Vout}
     |    |       |
     |    |   +{-{-}{-}+}
Vin o+{-{-}{-}{-}+{-}{-}{-}|+  |}
     |        |   |
     |  Ri    |   |
     +{-{-}//{-}{-}+{-}{-}{-}+}
                {-|}
                 |
                 |
                 |
     +{-{-}{-}{-}{-}{-}{-}{-}{-}{-}{-}+}
     |
    GND
\end{verbatim}

\textbf{ગેઇન ડેરિવેશન:}

\begin{itemize}
\item
  ઇન્વર્ટિંગ ઇનપુટ પર KCL લાગુ કરો: I_{1} + I_{2} = 0
\item
  I_{1} = (Vin - V^{-})/Ri અને I_{2} = (Vout - V^{-})/Rf
\item
  વર્ચ્યુઅલ ગ્રાઉન્ડ પર, V^{-} \approx 0
\item
  તેથી: Vin/Ri + Vout/Rf = 0
\item
  Vout/Vin માટે સોલ્વિંગ: Av = -Rf/Ri
\item
  \textbf{લાક્ષણિકતાઓ}: આઉટપુટ ઇનપુટથી 180^\circ ફેઝમાં હોય છે
\item
  \textbf{ફીડબેક}: ઇન્વર્ટિંગ ઇનપુટ પર વર્ચ્યુઅલ ગ્રાઉન્ડ બનાવે છે
\item
  \textbf{ક્લોઝ્ડ લૂપ ગેઇન}: બાહ્ય રેસિસ્ટર્સ દ્વારા નિયંત્રિત
\end{itemize}

\end{solutionbox}
\begin{mnemonicbox}
``VAIN'' - ``વર્ચ્યુઅલ ગ્રાઉન્ડ એમ્પ્લિફિકેશન ઇન્વર્ટ્સ નેગેટિવ''

\end{mnemonicbox}
\subsection*{અથવા}\label{uxa85uxaa5uxab5-3}

\subsection*{પ્રશ્ન 4(અ) [3
ગુણ]}\label{uxaaauxab0uxab6uxaa8-4uxa85-3-uxa97uxaa3-1}

\textbf{OPAMP ના નીચેના પેરામીટર્સ વ્યાખ્યાયિત કરો.} \textbf{1)
સી.એમ.આર.આર.(CMRR) 2) સ્લૂ રેટ(Slew rate) 3) ગેઇન બેન્ડવિડ્થ પ્રોડક્ટ}

\begin{solutionbox}
આ પેરામીટર્સ ઓપરેશનલ એમ્પ્લિફાયર્સની કીપરફોર્મન્સ લાક્ષણિકતાઓ નક્કી
કરે છે.


{\def\LTcaptype{none} % do not increment counter
\begin{longtable}[]{@{}
  >{\raggedright\arraybackslash}p{(\linewidth - 4\tabcolsep) * \real{0.3143}}
  >{\raggedright\arraybackslash}p{(\linewidth - 4\tabcolsep) * \real{0.3429}}
  >{\raggedright\arraybackslash}p{(\linewidth - 4\tabcolsep) * \real{0.3429}}@{}}
\toprule\noalign{}
\begin{minipage}[b]{\linewidth}\raggedright
પેરામીટર
\end{minipage} & \begin{minipage}[b]{\linewidth}\raggedright
વ્યાખ્યા
\end{minipage} & \begin{minipage}[b]{\linewidth}\raggedright
મહત્વ
\end{minipage} \\
\midrule\noalign{}
\endhead
\bottomrule\noalign{}
\endlastfoot
CMRR & ડિફરેન્શિયલ ગેઇનનો કોમન-મોડ ગેઇન સાથેનો ગુણોત્તર & ઊંચું હોય તે નોઇઝ રિજેક્શન
માટે વધુ સારું \\
સ્લ્યુ રેટ & આઉટપુટ વોલ્ટેજ ચેન્જનો મહત્તમ દર (V/μs) & લાર્જ-સિગ્નલ બેન્ડવિડ્થ નક્કી કરે
છે \\
ગેઇન-બેન્ડવિડ્થ પ્રોડક્ટ & ગેઇન અને ફ્રીક્વન્સીનો ગુણાકાર (MHz) & હાઈ-ફ્રીક્વન્સી
પરફોર્મન્સ માપે છે \\
\end{longtable}
}

\begin{itemize}
\tightlist
\item
  \textbf{CMRR}: ગુણવત્તાપૂર્ણ ઓપ-એમ્પ્સમાં સામાન્ય રીતે 80-120dB
\item
  \textbf{સ્લ્યુ રેટ}: હાઈ-ફ્રીક્વન્સી, હાઈ-એમ્પ્લિટ્યુડ સિગ્નલ્સ માટે આઉટપુટને મર્યાદિત
  કરે છે
\item
  \textbf{GBP}: ફ્રીક્વન્સી વધતાં કોન્સ્ટન્ટ રહે છે
\end{itemize}

\end{solutionbox}
\begin{mnemonicbox}
``CSG'' - ``કોમન-મોડ રિજેક્શન, સ્પીડ, અને ગેઇન''

\end{mnemonicbox}
\subsection*{પ્રશ્ન 4(બ) [4
ગુણ]}\label{uxaaauxab0uxab6uxaa8-4uxaac-4-uxa97uxaa3-1}

\textbf{OPAMP નો ઉપયોગ કરી સમિંગ એમ્પ્લીફાયર દોરો અને સમજાવો.}

\begin{solutionbox}
સમિંગ એમ્પ્લિફાયર ઇનપુટ વોલ્ટેજના વેઇટેડ સમના પ્રમાણમાં આઉટપુટ ઉત્પન્ન
કરે છે.

\textbf{સર્કિટ:}

\begin{verbatim}
              Rf
       +{-{-}{-}{-}{-}{-}//{-}{-}{-}{-}{-}{-}+}
       |                |
       |           +{-{-}{-}{-}+}
       |           |    |
       |    R1     |    |
  V1 o{-+{-}{-}{-}//{-}{-}{-}{-}+    +{-}{-}{-}{-}o Vout}
       |           |+   |
       |    R2    /     |
  V2 o{-+{-}{-}{-}//{-}{-}{-}+     |}
       |          {\_\_\_\_\_|}
       |    R3     |{-   |}
  V3 o{-+{-}{-}{-}//{-}{-}{-}{-}+    |}
       |                |
       |                |
     {-{-}{-}{-}{-}              |}
      {-{-}{-}               |}
       {-                |}
\end{verbatim}

\begin{itemize}
\tightlist
\item
  \textbf{આઉટપુટ ફોર્મ્યુલા}: Vout = -Rf(V_{1}/R_{1} + V_{2}/R_{2} + V_{3}/R_{3})
\item
  \textbf{એપ્લિકેશન્સ}: ઓડિયો મિક્સર, એનાલોગ કોમ્પ્યુટર, સિગ્નલ પ્રોસેસિંગ
\item
  \textbf{ફાયદા}: મલ્ટીપલ ઇનપુટ્સ એક સાથે પ્રોસેસ થઈ શકે છે
\end{itemize}

\end{solutionbox}
\begin{mnemonicbox}
``SUM'' - ``સેવરલ યુનિફાઇડ મલ્ટિપ્લાયર્સ''

\end{mnemonicbox}
\subsection*{પ્રશ્ન 4(ક) [7
ગુણ]}\label{uxaaauxab0uxab6uxaa8-4uxa95-7-uxa97uxaa3-1}

\textbf{IC 555 નો પિન ડાયાગ્રામ દોરો અને વેવફોર્મ સાથે IC555 નો ઉપયોગ કરીને
મોનોસ્ટેબલ મલ્ટિવાઇબ્રેટર સમજાવો.}

\begin{solutionbox}
IC 555 ટાઇમર મોનોસ્ટેબલ મોડમાં ટ્રિગર થાય ત્યારે ફિક્સ્ડ અવધિનો
સિંગલ પલ્સ ઉત્પન્ન કરે છે.

\textbf{પિન ડાયાગ્રામ:}

\begin{verbatim}
    +{-{-}{-}{-}{-}{-}{-}+}
  1 |o      | 8
    |       |
  2 |o      | 7
    |  555  |
  3 |o      | 6
    |       |
  4 |o      | 5
    +{-{-}{-}{-}{-}{-}{-}+}

1: GND     5: કંટ્રોલ
2: ટ્રિગર  6: થ્રેશોલ્ડ
3: આઉટપુટ   7: ડિસ્ચાર્જ
4: રીસેટ    8: VCC
\end{verbatim}

\textbf{સર્કિટ અને વેવફોર્મ:}

\begin{verbatim}
graph TB
    subgraph "મોનોસ્ટેબલ સર્કિટ"
    VCC {-{-}{-} R1 {-}{-}{-} A}
    A {-{-}{-} C1 {-}{-}{-} GND}
    A {-{-}{-} Pin6 \& Pin7}
    Pin2 {-{-}{-} Trigger}
    Pin3 {-{-}{-} Output}
    Pin4 {-{-}{-} Reset}
    Pin8 {-{-}{-} VCC}
    Pin1 {-{-}{-} GND}
    end
    
    subgraph "વેવફોર્મ્સ"
    direction TB
    Trig[ટ્રિગર] {-{-} O1[આઉટપુટ]}
    end
\end{verbatim}

\begin{itemize}
\tightlist
\item
  \textbf{ઓપરેશન}: નેગેટિવ ટ્રિગર ટાઇમિંગ સાયકલ શરૂ કરે છે
\item
  \textbf{ટાઇમ પીરિયડ}: T = 1.1 \times R \times C
\item
  \textbf{એપ્લિકેશન્સ}: ટાઇમર્સ, પલ્સ જનરેશન, ડિબાઉન્સિંગ
\item
  \textbf{ફાયદા}: સરળ, વિશ્વસનીય, વ્યાપકપણે ઉપલબ્ધ
\end{itemize}

\end{solutionbox}
\begin{mnemonicbox}
``TIMER'' - ``ટ્રિગર્ડ ઇનપુટ મેક્સ એક્સટેન્ડેડ રિસ્પોન્સ''

\end{mnemonicbox}
\subsection*{પ્રશ્ન 5(અ) [3
ગુણ]}\label{uxaaauxab0uxab6uxaa8-5uxa85-3-uxa97uxaa3}

\textbf{SMPS નો બ્લોક ડાયાગ્રામ દોરો અને તેની એપ્લીકેશન લખો.}

\begin{solutionbox}
સ્વિચ મોડ પાવર સપ્લાય (SMPS) કાર્યક્ષમ પાવર રૂપાંતરણ માટે સ્વિચિંગ
એલિમેન્ટ્સનો ઉપયોગ કરે છે.

\textbf{બ્લોક ડાયાગ્રામ:}

\begin{verbatim}
flowchart LR
    A[AC ઇનપુટ] {-{-} B[EMI ફિલ્ટર]}
    B {-{-} C[રેક્ટિફાયર]}
    C {-{-} D[ફિલ્ટર]}
    D {-{-} E[સ્વિચિંગ સર્કિટ]}
    E {-{-} F[ટ્રાન્સફોર્મર]}
    F {-{-} G[આઉટપુટ રેક્ટિફાયર]}
    G {-{-} H[આઉટપુટ ફિલ્ટર]}
    H {-{-} I[આઉટપુટ]}
    J[ફીડબેક કંટ્રોલ] {-{-} E}
    I {-{-} J}
\end{verbatim}

\textbf{એપ્લિકેશન્સ:}

\begin{itemize}
\item
  કોમ્પ્યુટર પાવર સપ્લાય
\item
  મોબાઇલ ફોન ચાર્જર
\item
  TV પાવર સપ્લાય
\item
  ઔદ્યોગિક પાવર સિસ્ટમ્સ
\item
  LED લાઇટિંગ ડ્રાઇવર્સ
\item
  \textbf{ફાયદા}: ઉચ્ચ કાર્યક્ષમતા, નાનું કદ, હલકું વજન
\item
  \textbf{પ્રકારો}: બક, બૂસ્ટ, બક-બૂસ્ટ, ફ્લાયબેક કન્વર્ટર્સ
\end{itemize}

\end{solutionbox}
\begin{mnemonicbox}
``SAFE'' - ``સ્વિચિંગ એચિવ્સ ફિલ્ટર્ડ એનર્જી''

\end{mnemonicbox}
\subsection*{પ્રશ્ન 5(બ) [4
ગુણ]}\label{uxaaauxab0uxab6uxaa8-5uxaac-4-uxa97uxaa3}

\textbf{ડાયાગ્રામ સાથે રેગ્યુલેટેડ પાવર સ્પ્લાયનું કાર્ય સમજાવો.}

\begin{solutionbox}
રેગ્યુલેટેડ પાવર સપ્લાય ઇનપુટ અથવા લોડમાં ફેરફાર થવા છતાં સ્થિર
આઉટપુટ જાળવે છે.

\textbf{બ્લોક ડાયાગ્રામ:}

\begin{verbatim}
flowchart LR
    A[AC ઇનપુટ] {-{-} B[ટ્રાન્સફોર્મર]}
    B {-{-} C[રેક્ટિફાયર]}
    C {-{-} D[ફિલ્ટર]}
    D {-{-} E[રેગ્યુલેટર]}
    E {-{-} F[આઉટપુટ]}
    G[ફીડબેક] {-{-} E}
    F {-{-} G}
\end{verbatim}

\begin{itemize}
\tightlist
\item
  \textbf{ટ્રાન્સફોર્મર}: AC વોલ્ટેજને જરૂરી લેવલ સુધી ઘટાડે છે
\item
  \textbf{રેક્ટિફાયર}: AC ને પલ્સેટિંગ DC માં રૂપાંતરિત કરે છે (ડાયોડ બ્રિજ)
\item
  \textbf{ફિલ્ટર}: કેપેસિટર્સ સાથે DC ને સ્મૂથ કરે છે
\item
  \textbf{રેગ્યુલેટર}: સ્થિર આઉટપુટ વોલ્ટેજ જાળવે છે
\item
  \textbf{ફીડબેક}: ઇનપુટ/લોડ વેરિએશન માટે ક્ષતિપૂર્તિ કરે છે
\end{itemize}

\end{solutionbox}
\begin{mnemonicbox}
``TRFRO'' - ``ટ્રાન્સફોર્મ, રેક્ટિફાય, ફિલ્ટર, રેગ્યુલેટ,
આઉટપુટ''

\end{mnemonicbox}
\subsection*{પ્રશ્ન 5(ક) [7
ગુણ]}\label{uxaaauxab0uxab6uxaa8-5uxa95-7-uxa97uxaa3}

\textbf{OP-AMP નો મૂળભૂત બ્લોક ડાયાગ્રામ દોરી સમજાવો.}

\begin{solutionbox}
ઓપરેશનલ એમ્પ્લિફાયરનું આંતરિક માળખું ચોક્કસ કાર્યો કરતા ઘણા
તબક્કાઓમાંથી બનેલું છે.

\textbf{બ્લોક ડાયાગ્રામ:}

\begin{verbatim}
flowchart LR
    A[ડિફરેન્શિયલ ઇનપુટ સ્ટેજ] {-{-} B[ઇન્ટરમીડિયેટ સ્ટેજ]}
    B {-{-} C[લેવલ શિફ્ટર]}
    C {-{-} D[આઉટપુટ સ્ટેજ]}
    E[બાયસ સર્કિટ] {-{-} A \& B \& C \& D}
\end{verbatim}

\begin{itemize}
\tightlist
\item
  \textbf{ડિફરેન્શિયલ ઇનપુટ સ્ટેજ}: હાઈ ઇમ્પીડન્સ, તફાવતને એમ્પ્લિફાય કરે છે
\item
  \textbf{ઇન્ટરમીડિયેટ સ્ટેજ}: વધારાનો ગેઇન પ્રદાન કરે છે
\item
  \textbf{લેવલ શિફ્ટર}: સ્ટેજ્સ વચ્ચે DC લેવલ એડજસ્ટ કરે છે
\item
  \textbf{આઉટપુટ સ્ટેજ}: લો ઇમ્પીડન્સ, કરંટ એમ્પ્લિફિકેશન
\item
  \textbf{બાયસ સર્કિટ}: બધા સ્ટેજ્સ માટે ઓપરેટિંગ પોઇન્ટ સ્થાપિત કરે છે
\item
  \textbf{કોમ્પેનસેશન}: સ્ટેબિલિટી માટે આંતરિક કેપેસિટર
\end{itemize}

\end{solutionbox}
\begin{mnemonicbox}
``DILO'' - ``ડિફરેન્શિયલ ઇનપુટ, લેવલ શિફ્ટ, આઉટપુટ''

\end{mnemonicbox}
\subsection*{અથવા}\label{uxa85uxaa5uxab5-4}

\subsection*{પ્રશ્ન 5(અ) [3
ગુણ]}\label{uxaaauxab0uxab6uxaa8-5uxa85-3-uxa97uxaa3-1}

\textbf{ડાયાગ્રામ સાથે LM317 નો ઉપયોગ કરીને એડજસ્ટેબલ વોલ્ટેજ રેગ્યુલેટર સમજાવો.}

\begin{solutionbox}
LM317 એક બહુવિધ એડજસ્ટેબલ પોઝિટિવ વોલ્ટેજ રેગ્યુલેટર છે જેની આઉટપુટ
રેન્જ 1.25V થી 37V છે.

\textbf{સર્કિટ:}

\begin{verbatim}
    Vin             LM317            Vout
     o{-{-}{-}{-}{-}+{-}{-}{-}{-}{-}{-}{-}{-}+{-}{-}{-}{-}{-}{-}{-}+{-}{-}{-}{-}{-}{-}{-}{-}o}
           |       Vin     |
           |        |      |
           |      +{-{-}{-}+    |}
           |      |317|    |
           |      +{-{-}{-}+    |  C2}
           |     Adj|Out   +{-{-}||{-}{-}+}
           |        |         |   |
           |        +{-{-}{-}{-}{-}{-}{-}{-}{-}+   |}
           |                  |   |
     C1    |       R1         |   |
     ||    +{-{-}{-}{-}{-}{-}//{-}{-}{-}{-}{-}{-}{-}{-}+   |}
     ||    |                  |   |
     ||    |                  |   |
     ++{-{-}{-}{-}+       R2         |   |}
      |            /{/{-}{-}{-}{-}{-}{-}{-}+   |}
      |            |          |   |
     GND          GND        GND GND
\end{verbatim}

\begin{itemize}
\tightlist
\item
  \textbf{ફોર્મ્યુલા}: Vout = 1.25(1 + R2/R1)
\item
  \textbf{ફાયદા}: સરળ એડજસ્ટમેન્ટ, બિલ્ટ-ઇન પ્રોટેક્શન
\item
  \textbf{એપ્લિકેશન્સ}: વેરિયેબલ પાવર સપ્લાય, બેટરી ચાર્જર્સ
\end{itemize}

\end{solutionbox}
\begin{mnemonicbox}
``AVR'' - ``એડજસ્ટેબલ વોલ્ટેજ રેગ્યુલેશન''

\end{mnemonicbox}
\subsection*{પ્રશ્ન 5(બ) [4
ગુણ]}\label{uxaaauxab0uxab6uxaa8-5uxaac-4-uxa97uxaa3-1}

\textbf{ફિક્સ્ડ વોલ્ટેજ રેગ્યુલેટર IC અને વેરીએબલ વોલ્ટેજ રેગ્યુલેટર IC વચ્ચેનો તફાવત
આપો.}

\begin{solutionbox}
વોલ્ટેજ રેગ્યુલેટર IC તેમની કોન્ફિગર કરવાની ક્ષમતા અને એપ્લિકેશન
જરૂરિયાતોમાં ભિન્ન હોય છે.


{\def\LTcaptype{none} % do not increment counter
\begin{longtable}[]{@{}
  >{\raggedright\arraybackslash}p{(\linewidth - 4\tabcolsep) * \real{0.1719}}
  >{\raggedright\arraybackslash}p{(\linewidth - 4\tabcolsep) * \real{0.3906}}
  >{\raggedright\arraybackslash}p{(\linewidth - 4\tabcolsep) * \real{0.4375}}@{}}
\toprule\noalign{}
\begin{minipage}[b]{\linewidth}\raggedright
પેરામીટર
\end{minipage} & \begin{minipage}[b]{\linewidth}\raggedright
ફિક્સ્ડ વોલ્ટેજ રેગ્યુલેટર
\end{minipage} & \begin{minipage}[b]{\linewidth}\raggedright
વેરિયેબલ વોલ્ટેજ રેગ્યુલેટર
\end{minipage} \\
\midrule\noalign{}
\endhead
\bottomrule\noalign{}
\endlastfoot
આઉટપુટ વોલ્ટેજ & પૂર્વનિર્ધારિત (દા.ત., 5V, 12V) & રેન્જ પર એડજસ્ટેબલ \\
બાહ્ય કોમ્પોનેન્ટ્સ & મિનિમલ (માત્ર કેપેસિટર્સ) & સેટિંગ માટે રેસિસ્ટર્સની જરૂર \\
સીરીઝ & 78xx (પોઝિટિવ), 79xx (નેગેટિવ) & LM317 (પોઝિટિવ), LM337 (નેગેટિવ) \\
એપ્લિકેશન્સ & સ્ટાન્ડર્ડ ઇક્વિપમેન્ટ & કસ્ટમ ડિઝાઇન, લેબોરેટરી સપ્લાય \\
ફ્લેક્સિબિલિટી & ફિક્સ્ડ મૂલ્યો સુધી મર્યાદિત & અત્યંત એડાપ્ટેબલ \\
પિન કાઉન્ટ & સામાન્ય રીતે 3 પિન & 3 અથવા વધુ પિન \\
\end{longtable}
}

\begin{itemize}
\tightlist
\item
  \textbf{ફિક્સ્ડ રેગ્યુલેટર્સ}: ઉપયોગમાં સરળ, મર્યાદિત એડજસ્ટમેન્ટ
\item
  \textbf{વેરિયેબલ રેગ્યુલેટર્સ}: વધુ બહુમુખી, ગણતરીની જરૂર પડે છે
\end{itemize}

\end{solutionbox}
\begin{mnemonicbox}
``FOCUS'' - ``ફિક્સ્ડ આઉટપુટ કમ્પેર્ડ ટુ યુઝર-સેટ''

\end{mnemonicbox}
\subsection*{પ્રશ્ન 5(ક) [7
ગુણ]}\label{uxaaauxab0uxab6uxaa8-5uxa95-7-uxa97uxaa3-1}

\textbf{OP-AMP ની એપ્લિકેશન લખો. OP-AMP નો ઉપયોગ કરી સર્કિટ ડાયાગ્રામ સાથે D
ટુ A (ડીજીટલ ટુ એનાલોગ) કન્વટર્રનું કાર્ય સમજાવો.}

\begin{solutionbox}
ઓપ-એમ્પ્સની ઘણી એપ્લિકેશન્સ છે; D/A કન્વર્ટર્સ ડિજિટલ સિગ્નલ્સને
એનાલોગમાં રૂપાંતરિત કરે છે.

\textbf{OP-AMP ની એપ્લિકેશન્સ:}

\begin{itemize}
\tightlist
\item
  એમ્પ્લિફાયર્સ (ઇન્વર્ટિંગ, નોન-ઇન્વર્ટિંગ)
\item
  ફિલ્ટર્સ (એક્ટિવ ફિલ્ટર્સ)
\item
  ઓસિલેટર્સ
\item
  કમ્પેરેટર્સ
\item
  ઇન્ટિગ્રેટર્સ અને ડિફરેનશિયેટર્સ
\item
  વોલ્ટેજ ફોલોવર્સ
\item
  ઇન્સ્ટ્રુમેન્ટેશન સર્કિટ્સ
\end{itemize}

\textbf{R-2R લેડર DAC સર્કિટ:}

\begin{verbatim}
    D3   D2   D1   D0
     |    |    |    |
     v    v    v    v
     SW   SW   SW   SW
     |    |    |    |
   2R|   2R|  2R|  2R|
     |    |    |    |
     +{-{-}{-}{-}+{-}{-}{-}{-}+{-}{-}{-}{-}+}
     |    |    |    |
     R    R    R    R
     |    |    |    |
     +{-{-}{-}{-}+{-}{-}{-}{-}+{-}{-}{-}{-}+{-}{-}{-}+}
                      \_|+
                     /
                    /
                   /\_\_\_
                      {-|}
                       |
              Rf       |
              /{/{-}{-}{-}{-}{-}+{-}{-}{-}{-}o Vout}
              |        |
              |        |
             GND      GND
\end{verbatim}

\begin{itemize}
\tightlist
\item
  \textbf{કાર્ય સિદ્ધાંત}: ડિજિટલ ઇનપુટ્સ રેસિસ્ટર નેટવર્ક દ્વારા કરંટને વેઇટ કરે છે
\item
  \textbf{રેસિસ્ટન્સ વેલ્યુ}: બાઇનરી-વેઇટેડ અથવા R-2R લેડર નેટવર્ક
\item
  \textbf{રૂપાંતરણ}: આઉટપુટ વોલ્ટેજ ડિજિટલ ઇનપુટ વેલ્યુના પ્રમાણમાં
\item
  \textbf{રેઝોલ્યુશન}: બિટ્સની સંખ્યા દ્વારા નિર્ધારિત (2^{n} લેવલ્સ)
\end{itemize}

\end{solutionbox}
\begin{mnemonicbox}
``DART'' - ``ડિજિટલ ટુ એનાલોગ રેસિસ્ટર ટ્રાન્સલેશન''

\end{mnemonicbox}

\end{document}
