\documentclass[10pt,a4paper]{article}

% content/resources/templates/preamble.tex
\usepackage[margin=0.6in]{geometry}
\author{Milav Dabgar}
\usepackage{amsmath,amssymb,amsthm}
\usepackage{booktabs}
\usepackage{multirow}
\usepackage{xcolor}
\usepackage{tcolorbox}
\tcbuselibrary{breakable,skins}
\usepackage[colorlinks=true,linkcolor=blue]{hyperref}
\usepackage{titlesec}
\usepackage{enumitem}
\usepackage{tikz}
\usepackage{pgfplots}
\usepackage{circuitikz}
\usepackage[version=4]{mhchem}
\usepackage{longtable}
\usepackage{array}
\usepackage{float}
\usepackage{caption}
\usepackage{listings}

\lstset{
  basicstyle=\small\ttfamily,
  breaklines=true,
  breakatwhitespace=false,
  postbreak=\mbox{\textcolor{red}{$\hookrightarrow$}\space},
  float=false,
  numbers=left,
  numberstyle=\tiny\color{gray},
  numbersep=10pt,
  xleftmargin=2em,
  keywordstyle=\color{blue},
  commentstyle=\color{green!60!black},
  stringstyle=\color{purple},
  backgroundcolor=\color{gray!5},
  showstringspaces=false,
  tabsize=2,
  captionpos=b,
  keepspaces=true,
  columns=flexible
}

\pgfplotsset{compat=1.18}
\usetikzlibrary{shapes,arrows,positioning,calc,patterns,decorations.pathmorphing,decorations.markings,arrows.meta}

% Color scheme
\definecolor{headcolor}{RGB}{0,102,204}
\definecolor{keycolor}{RGB}{220,20,60}
\definecolor{solutioncolor}{RGB}{34,139,34}
\definecolor{mnemoniccolor}{RGB}{148,0,211}
\definecolor{codecolor}{RGB}{0,0,100}

% Spacing
\setlength{\parskip}{3pt}
\setlist[itemize]{nosep}
\setlist[enumerate]{nosep}

% Title formatting
\titleformat{\section}{\Large\bfseries\color{headcolor}}{\thesection}{1em}{}
\titleformat{\subsection}{\large\bfseries\color{headcolor}}{\thesubsection}{1em}{}

% Pandoc tightlist compatibility
\providecommand{\tightlist}{%
  \setlength{\itemsep}{0pt}\setlength{\parskip}{0pt}}

% Pandoc longtable compatibility
\newcounter{none}
\def\thenone{}


% content/resources/templates/gujarati-boxes.tex
\usepackage{fontspec}
\usepackage{polyglossia}

% Set Gujarati as main language (document is primarily in Gujarati)
% Note: gloss-gujarati.ldf doesn't exist in polyglossia, but it will use hyphenation patterns
\setdefaultlanguage{gujarati}
\setotherlanguage{english}

% Configure Gujarati font properly
% Use Language=Default to prevent polyglossia from trying to add language-specific features
% that don't exist for Gujarati, which causes "empty feature" warnings
\newfontfamily\gujaratifont[Script=Gujarati,AutoFakeBold=2.5,AutoFakeSlant=0.3]{Noto Sans Gujarati}
\setmainfont[Script=Gujarati,AutoFakeBold=2.5,AutoFakeSlant=0.3]{Noto Sans Gujarati}
% Use Noto Sans Gujarati for monospace to support Gujarati in text
\setmonofont[Scale=0.9]{Noto Sans Gujarati}

% Configure English to use the same font
\newfontfamily\englishfont[Script=Gujarati,AutoFakeBold=2.5,AutoFakeSlant=0.3]{Noto Sans Gujarati}

% Translations for polyglossia
\gappto\captionsgujarati{
  \renewcommand{\tablename}{કોષ્ટક}
  \renewcommand{\figurename}{આકૃતિ}
}

% Helper for TikZ nodes to ensure Gujarati font
\newcommand{\gu}[1]{{\gujaratifont #1}}

% Custom environments
\newtcolorbox{solutionbox}{
    breakable,
    enhanced,
    colback=solutioncolor!5!white,
    colframe=solutioncolor!75!black,
    fonttitle=\bfseries,
    title=જવાબ
}

\newtcolorbox{solutionboxnobreak}{
 colback=solutioncolor!5!white,
 colframe=solutioncolor!75!black,
 fonttitle=\bfseries,
 title=જવાબ
}

\newtcolorbox{keyformula}{
 breakable,
 enhanced,
 colback=keycolor!5!white,
 colframe=keycolor!75!black,
 fonttitle=\bfseries,
 title=રાસાયણિક સમીકરણ/સૂત્ર
}

\newtcolorbox{mnemonicbox}{
 breakable,
 enhanced,
 colback=mnemoniccolor!5!white,
 colframe=mnemoniccolor!75!black,
 fonttitle=\bfseries,
 title=મેમરી ટ્રીક
}


\begin{document}

\begin{center}
{\Huge\bfseries\color{headcolor} Subject Name (Gujarati)}\\[5pt]
{\LARGE 1323202 -- Winter 2024}\\[3pt]
{\large Semester 1 Study Material}\\[3pt]
{\normalsize\textit{Detailed Solutions and Explanations}}
\end{center}

\vspace{10pt}

\subsection*{પ્રશ્ન 1(a) [3
marks]}\label{q1a}

\textbf{થર્મલ રનઅવે વિગતવાર સમજાવો.}

\begin{solutionbox}
થર્મલ રનઅવે એક વિનાશક પ્રક્રિયા છે જેમાં ટ્રાન્ઝિસ્ટર વધુને વધુ ગરમ
થાય છે જ્યાં સુધી તે નિષ્ફળ ન જાય.

\textbf{આકૃતિ:}

\begin{verbatim}
flowchart LR
    A[ગરમી વધે છે] {-{-} B[કલેક્ટર કરંટ વધે છે]}
    B {-{-} C[વધુ પાવર વ્યય]}
    C {-{-} D[વધુ ગરમી ઉત્પન્ન થાય]}
    D {-{-} A}
\end{verbatim}

\begin{itemize}
\tightlist
\item
  \textbf{કારણ}: તાપમાન વધવાથી બેઝ-એમિટર વોલ્ટેજ ઘટે છે
\item
  \textbf{અસર}: તાપમાન વધવાથી કલેક્ટર કરંટ વધે છે
\item
  \textbf{પરિણામ}: સ્વ-મજબૂત થતી ગરમીની સાયકલ વિનાશ તરફ દોરી જાય છે
\end{itemize}

\end{solutionbox}
\begin{mnemonicbox}
``ગરમી વધે, કરંટ ચડે, ટ્રાન્ઝિસ્ટર મરે''

\end{mnemonicbox}
\subsection*{પ્રશ્ન 1(b) [4
marks]}\label{q1b}

\textbf{ફિક્સડ બાયસ પદ્ધતિ દોરો અને સમજાવો.}

\begin{solutionbox}
ફિક્સડ બાયસ માટે બેઝને વોલ્ટેજ સપ્લાય સાથે જોડવા માટે એક જ
રેસિસ્ટરનો ઉપયોગ થાય છે.

\textbf{સર્કિટ આકૃતિ:}

\begin{center}
\textbf{Mermaid Diagram (Code)}
\begin{verbatim}
{Shaded}
{Highlighting}[]
graph LR
    VCC((+VCC)) {-{-}{-} RB[RB]}
    RB {-{-}{-} B[B]}
    B {-{-}{-} BE[BE Junction]}
    BE {-{-}{-} E[E]}
    E {-{-}{-} GND((GND))}
    B {-{-}{-} BC[BC Junction]}
    BC {-{-}{-} C[C]}
    C {-{-}{-} RC[RC]}
    RC {-{-}{-} VCC}
{Highlighting}
{Shaded}
\end{verbatim}
\end{center}

\begin{itemize}
\tightlist
\item
  \textbf{કાર્યપદ્ધતિ}: બેઝ કરંટ (IB) = (VCC - VBE)/RB
\item
  \textbf{લક્ષણો}: સરળ સર્કિટ પરંતુ ઓછી સ્થિરતા
\item
  \textbf{ગેરલાભ}: તાપમાન ફેરફારો પ્રત્યે અતિસંવેદનશીલ
\item
  \textbf{ઉપયોગ}: નાના સિગ્નલ સર્કિટ જ્યાં સ્થિરતા મહત્વની નથી
\end{itemize}

\end{solutionbox}
\begin{mnemonicbox}
``ફિક્સડ બાયસ: એક રેસિસ્ટર, ઓછી સ્થિરતા''

\end{mnemonicbox}
\subsection*{પ્રશ્ન 1(c) [7
marks]}\label{q1c}

\textbf{બાયસ પદ્ધતિઓની સૂચિ બનાવો. વોલ્ટેજ ડિવાઇડર પ્રકારની બાયસ પદ્ધતિની
સર્કિટ દોરો અને સમજાવો.}

\begin{solutionbox}
ટ્રાન્ઝિસ્ટર માટે બાયસિંગ પદ્ધતિઓમાં યોગ્ય ઓપરેટિંગ પોઇન્ટ સ્થાપિત
કરવા માટે કેટલીક તકનીકો શામેલ છે.


{\def\LTcaptype{none} % do not increment counter
\vspace{-5pt}
\captionof{table}{ટ્રાન્ઝિસ્ટર બાયસિંગ પદ્ધતિઓ}
\vspace{-10pt}
\begin{longtable}[]{@{}llll@{}}
\toprule\noalign{}
પદ્ધતિ & સ્થિરતા & જટિલતા & તાપમાન સંવેદનશીલતા \\
\midrule\noalign{}
\endhead
\bottomrule\noalign{}
\endlastfoot
ફિક્સડ બાયસ & નબળી & સરળ & ઊંચી \\
કલેક્ટર-ટુ-બેઝ બાયસ & મધ્યમ & મધ્યમ & મધ્યમ \\
વોલ્ટેજ ડિવાઇડર બાયસ & ઉત્તમ & જટિલ & નીચી \\
એમિટર બાયસ & સારી & મધ્યમ & નીચી \\
\end{longtable}
}

\textbf{સર્કિટ આકૃતિ:}

\begin{center}
\textbf{Mermaid Diagram (Code)}
\begin{verbatim}
{Shaded}
{Highlighting}[]
graph LR
    VCC((+VCC)) {-{-}{-} R1[R1]}
    VCC {-{-}{-} RC[RC]}
    R1 {-{-}{-} N1((Node))}
    N1 {-{-}{-} R2[R2]}
    N1 {-{-}{-} B[Base]}
    B {-{-}{-} BE[BE Junction]}
    BE {-{-}{-} E[Emitter]}
    E {-{-}{-} RE[RE]}
    RE {-{-}{-} GND((GND))}
    B {-{-}{-} BC[BC Junction]}
    BC {-{-}{-} C[Collector]}
    C {-{-}{-} RC}
    R2 {-{-}{-} GND}
{Highlighting}
{Shaded}
\end{verbatim}
\end{center}

\begin{itemize}
\tightlist
\item
  \textbf{કાર્યપદ્ધતિ}: R1-R2 ડિવાઇડર સ્થિર બેઝ વોલ્ટેજ બનાવે છે
\item
  \textbf{ફાયદો}: β વેરિએશન અને તાપમાનથી ઓછો પ્રભાવિત
\item
  \textbf{મુખ્ય લક્ષણ}: RE નેગેટિવ ફીડબેક સ્થિરીકરણ પ્રદાન કરે છે
\item
  \textbf{ઉપયોગ}: એમ્પલિફાયર સર્કિટમાં સૌથી વધુ વપરાય છે
\end{itemize}

\end{solutionbox}
\begin{mnemonicbox}
``વિભાજીત કરો અને સ્થિર બાયસ માટે રાજ કરો''

\end{mnemonicbox}
\subsection*{પ્રશ્ન 1(c OR) [7
marks]}\label{uxaaauxab0uxab6uxaa8-1c-or-7-marks}

\textbf{કોમન એમીટર એમ્પલીફાયર માટે ડીસી લોડ લાઈન દોરો અને સમજાવો.}

\begin{solutionbox}
ડીસી લોડ લાઈન ટ્રાન્ઝિસ્ટરના તમામ સંભવિત ઓપરેટિંગ પોઇન્ટ્સને દર્શાવે
છે.

\textbf{ગ્રાફ:}

\begin{center}
\textbf{Mermaid Diagram (Code)}
\begin{verbatim}
{Shaded}
{Highlighting}[]
graph TD
    subgraph DC Load Line
    A["VCE=VCC (IC=0)"] {-{-}{-} B["IC=VCC/RC (VCE=0)"]}
    Q["Q{-Point (Operating Point)"] }
    end
    style Q fill:\#f00,stroke:\#333,stroke{-width:2px}
{Highlighting}
{Shaded}
\end{verbatim}
\end{center}

\textbf{ઇક્વેશન કોષ્ટક:}

{\def\LTcaptype{none} % do not increment counter
\begin{longtable}[]{@{}lll@{}}
\toprule\noalign{}
પેરામીટર & સમીકરણ & વર્ણન \\
\midrule\noalign{}
\endhead
\bottomrule\noalign{}
\endlastfoot
મહત્તમ VCE & VCC & જ્યારે IC = 0 \\
મહત્તમ IC & VCC/RC & જ્યારે VCE = 0 \\
લોડ લાઈન સમીકરણ & IC = (VCC - VCE)/RC & બધા સંભવિત ઓપરેટિંગ પોઇન્ટ \\
Q-પોઇન્ટ & બાયસિંગ દ્વારા નિર્ધારિત & સ્થિર ઓપરેશન પોઇન્ટ \\
\end{longtable}
}

\begin{itemize}
\tightlist
\item
  \textbf{હેતુ}: IC અને VCE વચ્ચેના સંબંધને ગ્રાફિકલી બતાવે છે
\item
  \textbf{મહત્વ}: ઓપરેટિંગ પોઇન્ટ (Q-પોઇન્ટ) નક્કી કરવામાં મદદ કરે છે
\item
  \textbf{ઉપયોગ}: એમ્પલિફાયરની ડિઝાઇન અને વિશ્લેષણ માટે આવશ્યક
\end{itemize}

\end{solutionbox}
\begin{mnemonicbox}
``મહત્તમ કરંટ અથવા મહત્તમ વોલ્ટેજ, બંને ક્યારેય નહિં''

\end{mnemonicbox}
\subsection*{પ્રશ્ન 2(a) [3
marks]}\label{q2a}

\textbf{પદો સમજાવો (i) ગેઈન (ii) બેન્ડવિડ્થ.}

\begin{solutionbox}
આ એમ્પલિફાયર પરફોરમન્સને વર્ણવતા મુખ્ય પેરામીટર્સ છે.


{\def\LTcaptype{none} % do not increment counter
\vspace{-5pt}
\captionof{table}{એમ્પલિફાયર પેરામીટર્સ}
\vspace{-10pt}
\begin{longtable}[]{@{}
  >{\raggedright\arraybackslash}p{(\linewidth - 6\tabcolsep) * \real{0.2619}}
  >{\raggedright\arraybackslash}p{(\linewidth - 6\tabcolsep) * \real{0.2857}}
  >{\raggedright\arraybackslash}p{(\linewidth - 6\tabcolsep) * \real{0.1429}}
  >{\raggedright\arraybackslash}p{(\linewidth - 6\tabcolsep) * \real{0.3095}}@{}}
\toprule\noalign{}
\begin{minipage}[b]{\linewidth}\raggedright
પેરામીટર
\end{minipage} & \begin{minipage}[b]{\linewidth}\raggedright
વ્યાખ્યા
\end{minipage} & \begin{minipage}[b]{\linewidth}\raggedright
એકમ
\end{minipage} & \begin{minipage}[b]{\linewidth}\raggedright
મહત્વ
\end{minipage} \\
\midrule\noalign{}
\endhead
\bottomrule\noalign{}
\endlastfoot
ગેઈન & આઉટપુટનો ઇનપુટ સિગ્નલ સાથેનો ગુણોત્તર & dB & એમ્પ્લિફિકેશન પાવર \\
બેન્ડવિડ્થ & ફ્રીક્વન્સીની રેન્જ જેમાં ગેઈન મહત્તમના 70.7\% કરતાં ઓછો ન હોય & Hz &
ઉપયોગી ફ્રીક્વન્સી રેન્જ \\
\end{longtable}
}

\begin{itemize}
\tightlist
\item
  \textbf{ગેઈનના પ્રકાર}: વોલ્ટેજ ગેઈન (Av), કરંટ ગેઈન (Ai), પાવર ગેઈન (Ap)
\item
  \textbf{બેન્ડવિડ્થ ફોર્મ્યુલા}: BW = fH - fL (ઉચ્ચ કટઓફ - નીચા કટઓફ)
\item
  \textbf{સંબંધિત પેરામીટર}: ગેઈન-બેન્ડવિડ્થ પ્રોડક્ટ (ચોક્કસ એમ્પલિફાયર માટે અચળ)
\end{itemize}

\end{solutionbox}
\begin{mnemonicbox}
``ગેઈન મોટું બનાવે, બેન્ડવિડ્થ પહોળું બનાવે''

\end{mnemonicbox}
\subsection*{પ્રશ્ન 2(b) [4
marks]}\label{q2b}

\textbf{એમ્પલીફાયરમાં નેગેટીવ ફીડબેકના ફાયદા અને ગેરફાયદાની સૂચિ બનાવો.}

\begin{solutionbox}
નેગેટિવ ફીડબેક એમ્પલિફાયર પરફોરમન્સમાં નોંધપાત્ર સુધારો કરે છે પરંતુ
ટ્રેડઓફ સાથે.


{\def\LTcaptype{none} % do not increment counter
\vspace{-5pt}
\captionof{table}{નેગેટિવ ફીડબેક લક્ષણો}
\vspace{-10pt}
\begin{longtable}[]{@{}ll@{}}
\toprule\noalign{}
ફાયદા & ગેરફાયદા \\
\midrule\noalign{}
\endhead
\bottomrule\noalign{}
\endlastfoot
બેન્ડવિડ્થમાં વધારો & ગેઈનમાં ઘટાડો \\
ડિસ્ટોર્શનમાં ઘટાડો & વધુ ઇનપુટ સિગ્નલની જરૂર \\
સ્થિરતામાં સુધારો & વધુ જટિલ સર્કિટ \\
ઘોંઘાટ સામે વધુ ઈમ્યુનિટી & અયોગ્ય ડિઝાઇન થાય તો ઓસિલેશનની સંભાવના \\
ઇનપુટ/આઉટપુટ ઇમ્પીડન્સ નિયંત્રિત & વધુ પાવર વપરાશ \\
\end{longtable}
}

\end{solutionbox}
\begin{mnemonicbox}
``સ્થિર, પહોળું અને ચોખ્ખું, માત્ર ગેઈન છોડો''

\end{mnemonicbox}
\subsection*{પ્રશ્ન 2(c) [7
marks]}\label{q2c}

\textbf{હાર્ટલી ઓસ્સીલેટર દોરો અને સમજાવો.}

\begin{solutionbox}
હાર્ટલી ઓસિલેટર ઇન્ડક્ટિવ ફીડબેકનો ઉપયોગ કરીને સાઇન વેવ્સ જનરેટ કરે
છે.

\textbf{સર્કિટ આકૃતિ:}

\begin{center}
\textbf{Mermaid Diagram (Code)}
\begin{verbatim}
{Shaded}
{Highlighting}[]
graph LR
    VCC((+VCC)) {-{-}{-} RC[RC]}
    RC {-{-}{-} C[Collector]}
    C {-{-}{-} C1[C1]}
    C1 {-{-}{-} B[Base]}
    B {-{-}{-} RB1[RB1]}
    RB1 {-{-}{-} VCC}
    B {-{-}{-} RB2[RB2]}
    RB2 {-{-}{-} GND((GND))}
    C {-{-}{-} OUT((Output))}
    E[Emitter] {-{-}{-} L2[L2]}
    L2 {-{-}{-} GND}
    C1 {-{-}{-} L1[L1]}
    L1 {-{-}{-} L2}
    E {-{-}{-} BE[BE Junction]}
    BE {-{-}{-} B}
    E {-{-}{-} CE[CE]}
    CE {-{-}{-} GND}
{Highlighting}
{Shaded}
\end{verbatim}
\end{center}

\begin{itemize}
\tightlist
\item
  \textbf{ફ્રીક્વન્સી નિર્ધારણ}: L1, L2 અને C1 મૂલ્યો દ્વારા (f = 1/2π\sqrt(L \times C))
\item
  \textbf{ફીડબેક મેકેનિઝમ}: ઇન્ડક્ટિવ વોલ્ટેજ ડિવાઇડર (L1 અને L2)
\item
  \textbf{ઓળખ લક્ષણ}: ટેપ કરેલ ઇન્ડક્ટર અથવા શ્રેણીમાં બે ઇન્ડક્ટર્સ
\item
  \textbf{ઉપયોગ}: RF સિગ્નલ જનરેશન, રેડિયો ટ્રાન્સમિટર્સ, કોમ્યુનિકેશન સિસ્ટમ્સ
\end{itemize}

\end{solutionbox}
\begin{mnemonicbox}
``હાર્ટલી હેલ્પફુલ ઇન્ડક્ટર્સ ધરાવે છે''

\end{mnemonicbox}
\subsection*{પ્રશ્ન 2(a OR) [3
marks]}\label{uxaaauxab0uxab6uxaa8-2a-or-3-marks}

\textbf{ઓસ્સીલેટર માટે બારખૌસન ક્રાઈટરીઆ (Barkhausen's criteria) જણાવો અને
સમજાવો.}

\begin{solutionbox}
બારખૌસન ક્રાઈટેરિયા સતત ઓસિલેશન માટેની શરતો નિર્ધારિત કરે છે.

\textbf{બે મુખ્ય માપદંડ:}

\begin{center}
\textbf{Mermaid Diagram (Code)}
\begin{verbatim}
{Shaded}
{Highlighting}[]
graph LR
    A["લૂપ ગેઈન = 1"] {-{-}{} C["સતત ઓસિલેશન"]}
    B["ફેઝ શિફ્ટ = 360^"] {-{-}{} C}
{Highlighting}
{Shaded}
\end{verbatim}
\end{center}

\begin{itemize}
\tightlist
\item
  \textbf{લૂપ ગેઈન કન્ડિશન}: \textbar Aβ\textbar{} = 1 (સતત ઓસિલેશન માટે
  ચોક્કસ 1)
\item
  \textbf{ફેઝ શિફ્ટ કન્ડિશન}: ∠Aβ = 0^\circ અથવા 360^\circ (સિગ્નલ રિઇન્ફોર્સમેન્ટ)
\item
  \textbf{પ્રેક્ટિકલ ડિઝાઇન}: પ્રારંભિક \textbar Aβ\textbar{} \textgreater{}
  1, અંતે \textbar Aβ\textbar{} = 1 પર સ્થિર થાય છે
\end{itemize}

\end{solutionbox}
\begin{mnemonicbox}
``ઓસિલેશન માટે: યુનિટ ગેઈન, ઝીરો ફેઝ''

\end{mnemonicbox}
\subsection*{પ્રશ્ન 2(b OR) [4
marks]}\label{uxaaauxab0uxab6uxaa8-2b-or-4-marks}

\textbf{નેગેટીવ અને પોસીટીવ ફીડબેક એમ્પલીફાયરને સરખાવો.}

\begin{solutionbox}
ફીડબેકનો પ્રકાર એમ્પલિફાયરના વર્તનને નાટકીય રીતે બદલે છે.

\textbf{તુલના કોષ્ટક:}

{\def\LTcaptype{none} % do not increment counter
\begin{longtable}[]{@{}lll@{}}
\toprule\noalign{}
પેરામીટર & નેગેટિવ ફીડબેક & પોઝિટિવ ફીડબેક \\
\midrule\noalign{}
\endhead
\bottomrule\noalign{}
\endlastfoot
ગેઈન & ઘટે છે & વધે છે \\
બેન્ડવિડ્થ & વધે છે & ઘટે છે \\
ડિસ્ટોર્શન & ઘટાડે છે & વધારે છે \\
સ્થિરતા & સુધારે છે & ઘટાડે છે (ઓસિલેટ કરી શકે) \\
ઘોંઘાટ & ઘટાડે છે & વધારે છે \\
ઉપયોગ & સ્થિર એમ્પલિફાયર & ઓસિલેટર, ટ્રિગર સર્કિટ \\
ઇનપુટ/આઉટપુટ ઇમ્પીડન્સ & નિયંત્રિત & ઓછી અનુમાનિત \\
\end{longtable}
}

\end{solutionbox}
\begin{mnemonicbox}
``નેગેટિવ સ્થિર કરે, પોઝિટિવ ઓસિલેટ કરે''

\end{mnemonicbox}
\subsection*{પ્રશ્ન 2(c OR) [7
marks]}\label{uxaaauxab0uxab6uxaa8-2c-or-7-marks}

\textbf{કોલપીટ્ટ્સ ઓસ્સીલેટર દોરો અને સમજાવો.}

\begin{solutionbox}
કોલપિટ્સ ઓસિલેટર ફીડબેક માટે કેપેસિટિવ વોલ્ટેજ ડિવાઇડરનો ઉપયોગ કરે
છે.

\textbf{સર્કિટ આકૃતિ:}

\begin{center}
\textbf{Mermaid Diagram (Code)}
\begin{verbatim}
{Shaded}
{Highlighting}[]
graph LR
    VCC((+VCC)) {-{-}{-} RC[RC]}
    RC {-{-}{-} C[Collector]}
    C {-{-}{-} L[L]}
    L {-{-}{-} N((Node))}
    N {-{-}{-} C1[C1]}
    N {-{-}{-} C2[C2]}
    C2 {-{-}{-} GND((GND))}
    C1 {-{-}{-} B[Base]}
    B {-{-}{-} RB1[RB1]}
    RB1 {-{-}{-} VCC}
    B {-{-}{-} RB2[RB2]}
    RB2 {-{-}{-} GND}
    C {-{-}{-} CB[Coupling Capacitor]}
    CB {-{-}{-} OUT((Output))}
    E[Emitter] {-{-}{-} RE[RE]}
    RE {-{-}{-} GND}
    C1 {-{-}{-} E}
    E {-{-}{-} BE[BE Junction]}
    BE {-{-}{-} B}
{Highlighting}
{Shaded}
\end{verbatim}
\end{center}

\begin{itemize}
\tightlist
\item
  \textbf{ફ્રીક્વન્સી નિર્ધારણ}: L, C1 અને C2 મૂલ્યો દ્વારા (f = 1/2π\sqrt(L \times Ceq))
\item
  \textbf{ફીડબેક મેકેનિઝમ}: કેપેસિટિવ વોલ્ટેજ ડિવાઇડર (C1 અને C2)
\item
  \textbf{ઓળખ લક્ષણ}: ઇન્ડક્ટર સામે શ્રેણીમાં બે કેપેસિટર
\item
  \textbf{ફાયદો}: હાર્ટલી કરતાં વધુ સ્થિર ફ્રીક્વન્સી
\end{itemize}

\end{solutionbox}
\begin{mnemonicbox}
``કોલપિટ્સ કેપેસિટિવ કરંટ કેચ કરે છે''

\end{mnemonicbox}
\subsection*{પ્રશ્ન 3(a) [3
marks]}\label{q3a}

\textbf{ડાયક વિષે સમજાવો.}

\begin{solutionbox}
DIAC (Diode for Alternating Current) એ બાઇડિરેક્શનલ ટ્રિગર
ડાયોડ છે.

\textbf{સિમ્બોલ અને સંરચના:}

\begin{verbatim}
    A       K
    |       |
    +{-{-}{-}{-}{-}{-}{-}+}
    |       |
    +{-{-}{-}{-}{-}{-}{-}+}
    |       |
    K       A
\end{verbatim}

\begin{itemize}
\tightlist
\item
  \textbf{ઓપરેશન}: બ્રેકડાઉન વોલ્ટેજ પછી બંને દિશામાં વહન કરે છે
\item
  \textbf{લક્ષણ}: બંને દિશામાં સિમેટ્રિકલ V-I કર્વ
\item
  \textbf{કી પેરામીટર}: બ્રેકઓવર વોલ્ટેજ (સામાન્ય રીતે 30-40V)
\item
  \textbf{મુખ્ય ઉપયોગ}: AC પાવર કંટ્રોલમાં TRIAC ટ્રિગરિંગ
\end{itemize}

\end{solutionbox}
\begin{mnemonicbox}
``DIAC: બેવડી દિશા બ્રેકડાઉન ડિવાઇસ''

\end{mnemonicbox}
\subsection*{પ્રશ્ન 3(b) [4
marks]}\label{q3b}

\textbf{SCRની ટ્રીગરિંગ પદ્ધતિઓ સમજાવો.}

\begin{solutionbox}
SCR વહન માટે ઘણી પદ્ધતિઓ દ્વારા ટ્રિગર થઈ શકે છે.


{\def\LTcaptype{none} % do not increment counter
\vspace{-5pt}
\captionof{table}{SCR ટ્રિગરિંગ પદ્ધતિઓ}
\vspace{-10pt}
\begin{longtable}[]{@{}
  >{\raggedright\arraybackslash}p{(\linewidth - 6\tabcolsep) * \real{0.1739}}
  >{\raggedright\arraybackslash}p{(\linewidth - 6\tabcolsep) * \real{0.2826}}
  >{\raggedright\arraybackslash}p{(\linewidth - 6\tabcolsep) * \real{0.2609}}
  >{\raggedright\arraybackslash}p{(\linewidth - 6\tabcolsep) * \real{0.2826}}@{}}
\toprule\noalign{}
\begin{minipage}[b]{\linewidth}\raggedright
પદ્ધતિ
\end{minipage} & \begin{minipage}[b]{\linewidth}\raggedright
વર્ણન
\end{minipage} & \begin{minipage}[b]{\linewidth}\raggedright
ફાયદા
\end{minipage} & \begin{minipage}[b]{\linewidth}\raggedright
મર્યાદાઓ
\end{minipage} \\
\midrule\noalign{}
\endhead
\bottomrule\noalign{}
\endlastfoot
ગેટ ટ્રિગરિંગ & ગેટ પર કરંટ પલ્સ & સૌથી સામાન્ય, નિયંત્રિત & કંટ્રોલ સર્કિટની જરૂર \\
તાપમાન & ઉચ્ચ તાપમાન & કોઈ બાહ્ય સર્કિટ નહીં & અનિયંત્રિત, અવિશ્વસનીય \\
વોલ્ટેજ & બ્રેકઓવર વોલ્ટેજથી વધારે & કોઈ બાહ્ય સર્કિટ નહીં & ડિવાઇસ પર તણાવ,
અનિયંત્રિત \\
dv/dt & ઝડપી વોલ્ટેજ વૃદ્ધિ & સરળ & અનિચ્છનીય ટ્રિગરિંગ થઈ શકે \\
પ્રકાશ & જંક્શન પર ફોટોન્સ & ઇલેક્ટ્રિકલ અલગતા & વિશેષ પેકેજિંગની જરૂર \\
\end{longtable}
}

\end{solutionbox}
\begin{mnemonicbox}
``ગેટ વોલ્ટેજ તાપમાન રેટ લાઇટ''

\end{mnemonicbox}
\subsection*{પ્રશ્ન 3(c) [7
marks]}\label{q3c}

\textbf{SCRનો સિમ્બોલ અને કન્સ્ટ્રક્શન દોરો. ઉપરાંત SCRની V-I લાક્ષણિકતા દોરો અને
સમજાવો.}

\begin{solutionbox}
SCR (Silicon Controlled Rectifier) એ ત્રણ ટર્મિનલવાળી
ચાર-લેયર PNPN સેમિકન્ડક્ટર ડિવાઇસ છે.

\textbf{સિમ્બોલ:}

\begin{verbatim}
      A (Anode)
      |
      |
      v
    {-{-}{-}{-}{-}}
    |   |
G {-{-}|   |}
    |   |
    {-{-}{-}{-}{-}}
      \^{}
      |
      |
      K (Cathode)
\end{verbatim}

\textbf{કન્સ્ટ્રક્શન:}

\begin{center}
\textbf{Mermaid Diagram (Code)}
\begin{verbatim}
{Shaded}
{Highlighting}[]
graph LR
    A[Anode: P+] {-{-}{-} J3[Junction J3]}
    J3 {-{-}{-} N[N{-}layer]}
    N {-{-}{-} J2[Junction J2]}
    J2 {-{-}{-} P[P{-}layer]}
    P {-{-}{-} G[Gate]}
    P {-{-}{-} J1[Junction J1]}
    J1 {-{-}{-} K[Cathode: N+]}
{Highlighting}
{Shaded}
\end{verbatim}
\end{center}

\textbf{V-I લાક્ષણિકતા:}

\begin{center}
\textbf{Mermaid Diagram (Code)}
\begin{verbatim}
{Shaded}
{Highlighting}[]
graph TD
    subgraph V{-I Characteristic}
    A["Forward Blocking{br /{}(OFF State)"] {-}{-}{} B["Forward Conduction{}br /{}(ON State)"]}
    C["Reverse Blocking"] {-{-}{} D["Reverse Breakdown"]}
    end
{Highlighting}
{Shaded}
\end{verbatim}
\end{center}

\begin{itemize}
\tightlist
\item
  \textbf{ફોરવર્ડ બ્લોકિંગ}: ટ્રિગરિંગ સુધી ઓછો કરંટ
\item
  \textbf{ફોરવર્ડ કન્ડક્શન}: ટ્રિગરિંગ પછી ઉચ્ચ કરંટ (લેચડ)
\item
  \textbf{હોલ્ડિંગ કરંટ}: કન્ડક્શન જાળવવા માટે ન્યૂનતમ કરંટ
\item
  \textbf{લેચિંગ કરંટ}: લેચિંગ શરૂ કરવા માટે ન્યૂનતમ કરંટ
\item
  \textbf{રિવર્સ બ્લોકિંગ}: રિવર્સ દિશામાં કરંટને અવરોધે છે
\end{itemize}

\end{solutionbox}
\begin{mnemonicbox}
``એક વાર ટ્રિગર, હંમેશા કન્ડક્ટ, જ્યાં સુધી કરંટ ન ઘટે''

\end{mnemonicbox}
\subsection*{પ્રશ્ન 3(a OR) [3
marks]}\label{uxaaauxab0uxab6uxaa8-3a-or-3-marks}

\textbf{SCRની નેચરલ કોમ્યુટેશન પદ્ધતિ વિષે સમજાવો.}

\begin{solutionbox}
નેચરલ કોમ્યુટેશન AC કરંટ કુદરતી રીતે શૂન્ય પર પહોંચે ત્યારે બાહ્ય સર્કિટ
વિના SCRને બંધ કરે છે.

\textbf{પ્રક્રિયા આકૃતિ:}

\begin{center}
\textbf{Mermaid Diagram (Code)}
\begin{verbatim}
{Shaded}
{Highlighting}[]
graph LR
    A["AC સપ્લાય{br /{}શૂન્ય ક્રોસ કરે છે"] {-}{-}{} B["કરંટ હોલ્ડિંગથી{}br /{}નીચે પડે છે"]}
    B {-{-}{} C["SCR કુદરતી રીતે{}br /{}બંધ થાય છે"]}
    C {-{-}{} D["આગલા ટ્રિગર{}br /{}સુધી બંધ રહે છે"]}
{Highlighting}
{Shaded}
\end{verbatim}
\end{center}

\begin{itemize}
\tightlist
\item
  \textbf{સિદ્ધાંત}: AC સપ્લાયના કુદરતી શૂન્ય-ક્રોસિંગનો ઉપયોગ કરે છે
\item
  \textbf{ફાયદો}: કોઈ વધારાની કોમ્યુટેશન સર્કિટની જરૂર નથી
\item
  \textbf{ઉપયોગ}: AC પાવર કંટ્રોલ સર્કિટ, લાઇટ ડિમર્સ
\item
  \textbf{મર્યાદા}: માત્ર AC સપ્લાય સાથે કામ કરે છે, DC સાથે નહીં
\end{itemize}

\end{solutionbox}
\begin{mnemonicbox}
``નેચરલ કોમ્યુટેશન: શૂન્ય કરંટ, શૂન્ય પ્રયત્ન''

\end{mnemonicbox}
\subsection*{પ્રશ્ન 3(b OR) [4
marks]}\label{uxaaauxab0uxab6uxaa8-3b-or-4-marks}

\textbf{ઓપ્ટો-કપ્લર વિશે સમજાવો.}

\begin{solutionbox}
ઓપ્ટો-કપ્લર પ્રકાશ ટ્રાન્સમિશનનો ઉપયોગ કરીને ઇલેક્ટ્રિકલ આઈસોલેશન
પ્રદાન કરે છે.

\textbf{સંરચના:}

\begin{verbatim}
  .{-{-}{-}{-}{-}{-}{-}{-}{-}.}
  |  LED    |{}
  |         | {}
  {{-}{-}{-}{-}{-}{-}{-}{-}{-}  }
                {}
  .{-{-}{-}{-}{-}{-}{-}{-}{-}.  //}
  |PhotoDet | //
  |         |//
  {{-}{-}{-}{-}{-}{-}{-}{-}{-}}
\end{verbatim}


{\def\LTcaptype{none} % do not increment counter
\vspace{-5pt}
\captionof{table}{ઓપ્ટો-કપ્લર પ્રકારો}
\vspace{-10pt}
\begin{longtable}[]{@{}lllll@{}}
\toprule\noalign{}
પ્રકાર & ફોટોડિટેક્ટર & સ્પીડ & CTR & ઉપયોગો \\
\midrule\noalign{}
\endhead
\bottomrule\noalign{}
\endlastfoot
સ્ટાન્ડર્ડ & ફોટોટ્રાન્ઝિસ્ટર & મધ્યમ & 20-100\% & સામાન્ય આઈસોલેશન \\
હાઈ-સ્પીડ & ફોટોડાયોડ & ઝડપી & 10-50\% & ડિજિટલ કોમ્યુનિકેશન \\
TRIAC & ફોટો-TRIAC & ધીમું & N/A & AC પાવર કંટ્રોલ \\
લિનિયર & ફોટોડાર્લિંગટન & ધીમું & 100-1000\% & એનાલોગ સિગ્નલ્સ \\
\end{longtable}
}

\begin{itemize}
\tightlist
\item
  \textbf{CTR}: કરંટ ટ્રાન્સફર રેશિયો (આઉટપુટ/ઇનપુટ કરંટ)
\item
  \textbf{મુખ્ય લક્ષણ}: સર્કિટ્સ વચ્ચે સંપૂર્ણ ઇલેક્ટ્રિકલ આઈસોલેશન
\item
  \textbf{ફાયદા}: નોઈઝ ઈમ્યુનિટી, વોલ્ટેજ લેવલ શિફ્ટિંગ, સલામતી
\end{itemize}

\end{solutionbox}
\begin{mnemonicbox}
``પ્રકાશ કૂદે છે જ્યાં ઇલેક્ટ્રોન્સ નથી કૂદી શકતા''

\end{mnemonicbox}
\subsection*{પ્રશ્ન 3(c OR) [7
marks]}\label{uxaaauxab0uxab6uxaa8-3c-or-7-marks}

\textbf{TRIACનો સિમ્બોલ અને કન્સ્ટ્રક્શન દોરો. ઉપરાંત TRIACની V-I લાક્ષણિકતા
દોરો અને સમજાવો.}

\begin{solutionbox}
TRIAC (Triode for Alternating Current) એ બાઇડિરેક્શનલ
ત્રણ-ટર્મિનલવાળી સેમિકન્ડક્ટર ડિવાઇસ છે.

\textbf{સિમ્બોલ:}

\begin{verbatim}
    MT2
     |
     |
   {-{-}{-}{-}{-}}
   |   |
G{-{-}|   |}
   |   |
   {-{-}{-}{-}{-}}
     |
     |
    MT1
\end{verbatim}

\textbf{કન્સ્ટ્રક્શન:}

\begin{center}
\textbf{Mermaid Diagram (Code)}
\begin{verbatim}
{Shaded}
{Highlighting}[]
graph LR
    MT2[Main Terminal 2] {-{-}{-} P1[P{-}layer]}
    P1 {-{-}{-} N1[N{-}layer]}
    N1 {-{-}{-} P2[P{-}layer]}
    P2 {-{-}{-} N2[N{-}layer]}
    P2 {-{-}{-} G[Gate]}
    N2 {-{-}{-} MT1[Main Terminal 1]}
{Highlighting}
{Shaded}
\end{verbatim}
\end{center}

\textbf{V-I લાક્ષણિકતા:}

\begin{center}
\textbf{Mermaid Diagram (Code)}
\begin{verbatim}
{Shaded}
{Highlighting}[]
graph TD
    subgraph Quadrant I
    A1["MT2+, MT1{-{}br /{}Forward Blocking"] {-}{-}{} B1["MT2+, MT1{-}{}br /{}Forward Conducting"]}
    end
    subgraph Quadrant III
    A2["MT2{-, MT1+{}br /{}Reverse Blocking"] {-}{-}{} B2["MT2{-}, MT1+{}br /{}Reverse Conducting"]}
    end
{Highlighting}
{Shaded}
\end{verbatim}
\end{center}

\begin{itemize}
\tightlist
\item
  \textbf{બાઇડિરેક્શનલ}: ટ્રિગરિંગ પછી બંને દિશામાં વહન કરે છે
\item
  \textbf{ક્વોડ્રન્ટ ઓપરેશન}: પોલેરિટી પર આધારિત ચાર ટ્રિગરિંગ મોડ
\item
  \textbf{ઉપયોગો}: AC પાવર કંટ્રોલ, લાઇટ ડિમર્સ, મોટર કંટ્રોલ
\item
  \textbf{SCR કરતાં ફાયદો}: AC સાયકલના બંને અર્ધભાગોને નિયંત્રિત કરે છે
\end{itemize}

\end{solutionbox}
\begin{mnemonicbox}
``TRIAC: AC સર્કિટમાં બેવડી દિશાનો રસ્તો''

\end{mnemonicbox}
\subsection*{પ્રશ્ન 4(a) [3
marks]}\label{q4a}

\textbf{Ideal Op-Ampની લાક્ષણિકતા જણાવો.}

\begin{solutionbox}
આદર્શ Op-Amp એવી સંપૂર્ણ લાક્ષણિકતાઓ ધરાવે છે જેને વાસ્તવિક Op-Amps
આશરે છે.


{\def\LTcaptype{none} % do not increment counter
\vspace{-5pt}
\captionof{table}{આદર્શ Op-Amp લાક્ષણિકતાઓ}
\vspace{-10pt}
\begin{longtable}[]{@{}lll@{}}
\toprule\noalign{}
પેરામીટર & આદર્શ મૂલ્ય & અર્થ \\
\midrule\noalign{}
\endhead
\bottomrule\noalign{}
\endlastfoot
ઓપન-લૂપ ગેઈન & અનંત & નાનામાં નાના ઇનપુટ તફાવતને એમ્પ્લિફાય કરે છે \\
ઇનપુટ ઇમ્પીડન્સ & અનંત & સ્ત્રોતમાંથી કોઈ કરંટ લેતું નથી \\
આઉટપુટ ઇમ્પીડન્સ & શૂન્ય & કોઈપણ લોડને ડ્રાઇવ કરી શકે છે \\
બેન્ડવિડ્થ & અનંત & બધી ફ્રીક્વન્સી પર કામ કરે છે \\
CMRR & અનંત & કોમન-મોડ સિગ્નલ્સને નકારે છે \\
સ્લ્યૂ રેટ & અનંત & તાત્કાલિક આઉટપુટ ફેરફાર \\
ઓફસેટ વોલ્ટેજ & શૂન્ય & શૂન્ય ઇનપુટ સાથે કોઈ આઉટપુટ નહીં \\
\end{longtable}
}

\end{solutionbox}
\begin{mnemonicbox}
``અનંત ગેઈન, ઇમ્પીડન્સ, બેન્ડવિડ્થ; શૂન્ય ઓફસેટ, આઉટપુટ Z''

\end{mnemonicbox}
\subsection*{પ્રશ્ન 4(b) [4
marks]}\label{q4b}

\textbf{555 ટાઈમર ICની મદદથી મોનોસ્ટેબલ મલ્ટીવાઇબ્રેટર દોરો અને સમજાવો.}

\begin{solutionbox}
મોનોસ્ટેબલ મલ્ટીવાઇબ્રેટર ટ્રિગર થાય ત્યારે નિશ્ચિત સમયગાળાનો એક
પલ્સ ઉત્પન્ન કરે છે.

\textbf{સર્કિટ:}

\begin{center}
\textbf{Mermaid Diagram (Code)}
\begin{verbatim}
{Shaded}
{Highlighting}[]
graph LR
    VCC((+VCC)) {-{-}{-} R[R]}
    R {-{-}{-} DIS[7:DIS]}
    R {-{-}{-} RST[4:RST]}
    R {-{-}{-} VCC\_PIN[8:VCC]}
    TRG[2:TRIG] {-{-}{-} GND((GND))}
    THR[6:THRES] {-{-}{-} C[C]}
    C {-{-}{-} GND}
    TRG {-{-}{-} SW[Trigger Switch]}
    SW {-{-}{-} GND}
    DIS {-{-}{-} THR}
    VCC\_PIN {-{-}{-} IC[555 Timer]}
    RST {-{-}{-} IC}
    TRG {-{-}{-} IC}
    THR {-{-}{-} IC}
    IC {-{-}{-} OUT[3:OUT]}
    IC {-{-}{-} CTRL[5:CTRL]}
    CTRL {-{-}{-} CC[0.01µF]}
    CC {-{-}{-} GND}
    GND {-{-}{-} GND\_PIN[1:GND]}
    GND\_PIN {-{-}{-} IC}
    OUT {-{-}{-} Output((Output))}
{Highlighting}
{Shaded}
\end{verbatim}
\end{center}

\begin{itemize}
\tightlist
\item
  \textbf{ઓપરેશન}: નેગેટિવ ટ્રિગર T = 1.1RC સમયગાળાનો આઉટપુટ પલ્સ ઉત્પન્ન કરે છે
\item
  \textbf{સ્ટેબલ સ્ટેટ}: ટ્રિગર થાય ત્યાં સુધી આઉટપુટ LOW
\item
  \textbf{ટાઇમિંગ કંટ્રોલ}: R અને C મૂલ્યો પલ્સ પહોળાઈ નક્કી કરે છે
\item
  \textbf{રિટ્રિગરિંગ}: ટાઇમઆઉટ પછી ફરીથી ટ્રિગર થઈ શકે છે
\end{itemize}

\end{solutionbox}
\begin{mnemonicbox}
``વન શોટ વન્ડર: એક વાર ટ્રિગર, એક વાર પલ્સ''

\end{mnemonicbox}
\subsection*{પ્રશ્ન 4(c) [7
marks]}\label{q4c}

\textbf{741 ICની મદદથી ઇન્વર્ટિંગ એમ્પલીફાયર દોરો અને સમજાવો. ઉપરાંત તેના ઈનપુટ
અને આઉટપુટ વેવફોર્મ્સ દોરો.}

\begin{solutionbox}
ઇન્વર્ટિંગ એમ્પલિફાયર ઇનપુટ સિગ્નલને એમ્પ્લિફાય કરતી વખતે પોલેરિટી
ઉલટાવે છે.

\textbf{સર્કિટ:}

\begin{center}
\textbf{Mermaid Diagram (Code)}
\begin{verbatim}
{Shaded}
{Highlighting}[]
graph LR
    IN((Input)) {-{-}{-} Rin[Rin]}
    Rin {-{-}{-} INV[2:Inv]}
    INV {-{-}{-} FB[Feedback]}
    FB {-{-}{-} Rf[Rf]}
    Rf {-{-}{-} OUT((Output))}
    NINV[3:Non{-Inv] {-}{-}{-} GND((GND))}
    INV {-{-}{-} IC[741]}
    NINV {-{-}{-} IC}
    IC {-{-}{-} OUT}
    IC {-{-}{-} VCC[7:+VCC]}
    IC {-{-}{-} VEE[4:{-}VEE]}
{Highlighting}
{Shaded}
\end{verbatim}
\end{center}

\textbf{વેવફોર્મ્સ:}

\begin{verbatim}
Input:     /{-      /{-}}
          /   {    /   }
     \_\_\_\_/     {\_\_/     \_\_\_\_}

Output:   {    /    /}
           {  /    /}
     \_\_\_\_\_\_{/\_\_\_\_/\_\_\_\_\_\_\_\_}
            
            180^ ફેઝ શિફ્ટ
\end{verbatim}

\begin{itemize}
\tightlist
\item
  \textbf{ગેઈન સમીકરણ}: Av = -Rf/Rin (નેગેટિવ ચિહ્ન ઇન્વર્ઝન સૂચવે છે)
\item
  \textbf{ઇનપુટ ઇમ્પીડન્સ}: Rin જેટલી
\item
  \textbf{વર્ચ્યુઅલ ગ્રાઉન્ડ}: ઇન્વર્ટિંગ ઇનપુટ લગભગ 0V પર જળવાય છે
\item
  \textbf{બેન્ડવિડ્થ}: ગેઈન પર આધારિત (ઉચ્ચ ગેઈન = ઓછી બેન્ડવિડ્થ)
\item
  \textbf{ઉપયોગો}: સિગ્નલ કન્ડિશનિંગ, ઓડિયો એમ્પલિફાયર
\end{itemize}

\end{solutionbox}
\begin{mnemonicbox}
``ઉલટાવે અને Rf/Rin વડે ગુણાકાર કરે છે''

\end{mnemonicbox}
\subsection*{પ્રશ્ન 4(a OR) [3
marks]}\label{uxaaauxab0uxab6uxaa8-4a-or-3-marks}

\textbf{IC 741નો સિમ્બોલ અને પીન ડાયગ્રામ દોરો.}

\begin{solutionbox}
741 એક લોકપ્રિય જનરલ-પરપસ ઓપરેશનલ એમ્પલિફાયર છે.

\textbf{સિમ્બોલ:}

\begin{verbatim}
        |{ }
        | {}
Input {-{-}|+ }
        |   {}
        |    |{-{-}{-}{-} Output}
        |   /
Input {-{-}|− /}
        | /
        |/
\end{verbatim}

\textbf{8-Pin DIP પેકેજ:}

\begin{verbatim}
       \_\_\_\_\_\_\_
      |       |
NC 1{-{-}|       |{-}{-}8 Vcc+}
      |       |
{-IN 2{-}{-}|  741  |{-}{-}7 Output}
      |       |
+IN 3{-{-}|       |{-}{-}6 NC}
      |       |
Vcc{- 4{-}{-}|\_\_\_\_\_\_\_|{-}{-}5 Offset Null}
\end{verbatim}

\begin{itemize}
\tightlist
\item
  \textbf{પિન ફંક્શન્સ}: ઇન્વર્ટિંગ ઇનપુટ, નોન-ઇન્વર્ટિંગ ઇનપુટ, આઉટપુટ, પાવર સપ્લાય
\item
  \textbf{ઓપ્શનલ પિન્સ}: ઓફસેટ નલ, નો કનેક્શન
\item
  \textbf{પાવર સપ્લાય}: સામાન્ય રીતે \pm15V અથવા \pm12V ડ્યુઅલ સપ્લાય
\end{itemize}

\end{solutionbox}
\begin{mnemonicbox}
``કદી ઉલટાવશો નહિં પ્લસ, વેરી આઉટપુટ નોટ કનેક્ટેડ''

\end{mnemonicbox}
\subsection*{પ્રશ્ન 4(b OR) [4
marks]}\label{uxaaauxab0uxab6uxaa8-4b-or-4-marks}

\textbf{પદો સમજાવો (i) સી.એમ.આર.આર (II) સ્લૂ રેટ.}

\begin{solutionbox}
આ પેરામીટર્સ ઓપરેશનલ એમ્પલિફાયરની કાર્યક્ષમતાની મર્યાદાઓ
નિર્ધારિત કરે છે.


{\def\LTcaptype{none} % do not increment counter
\vspace{-5pt}
\captionof{table}{મુખ્ય Op-Amp પેરામીટર્સ}
\vspace{-10pt}
\begin{longtable}[]{@{}
  >{\raggedright\arraybackslash}p{(\linewidth - 6\tabcolsep) * \real{0.2157}}
  >{\raggedright\arraybackslash}p{(\linewidth - 6\tabcolsep) * \real{0.2353}}
  >{\raggedright\arraybackslash}p{(\linewidth - 6\tabcolsep) * \real{0.2941}}
  >{\raggedright\arraybackslash}p{(\linewidth - 6\tabcolsep) * \real{0.2549}}@{}}
\toprule\noalign{}
\begin{minipage}[b]{\linewidth}\raggedright
પેરામીટર
\end{minipage} & \begin{minipage}[b]{\linewidth}\raggedright
વ્યાખ્યા
\end{minipage} & \begin{minipage}[b]{\linewidth}\raggedright
સામાન્ય મૂલ્ય
\end{minipage} & \begin{minipage}[b]{\linewidth}\raggedright
મહત્વ
\end{minipage} \\
\midrule\noalign{}
\endhead
\bottomrule\noalign{}
\endlastfoot
CMRR (Common Mode Rejection Ratio) & ડિફરેન્શિયલ ગેઈનનો કોમન-મોડ ગેઈન સાથેનો
ગુણોત્તર & 90-120 dB & ઉચ્ચ હોય તે વધુ સારું \\
સ્લ્યૂ રેટ & આઉટપુટ વોલ્ટેજના ફેરફારનો મહત્તમ દર & 0.5-50 V/μs & ઝડપી સિગ્નલ્સ માટે
ઉચ્ચ \\
\end{longtable}
}

\begin{itemize}
\tightlist
\item
  \textbf{CMRR ફોર્મ્યુલા}: CMRR = 20 log_{1}_{0}(Ad/Acm) dB
\item
  \textbf{CMRR મહત્વ}: બંને ઇનપુટ પર સામાન્ય ઘોંઘાટને નકારે છે
\item
  \textbf{સ્લ્યૂ રેટ ફોર્મ્યુલા}: SR = dVo/dt (max)
\item
  \textbf{સ્લ્યૂ રેટ મર્યાદા}: ઉચ્ચ ફ્રીક્વન્સી પર ડિસ્ટોર્શન કરે છે
\end{itemize}

\end{solutionbox}
\begin{mnemonicbox}
``CMRR કોમન નોઈઝને ક્રશ કરે છે, સ્લ્યૂ રેટ સ્પીડ બતાવે છે''

\end{mnemonicbox}
\subsection*{પ્રશ્ન 4(c OR) [7
marks]}\label{uxaaauxab0uxab6uxaa8-4c-or-7-marks}

\textbf{555 ટાઈમર ICની મદદથી આસ્ટેબલ મલ્ટીવાઇબ્રેટર દોરો અને સમજાવો.}

\begin{solutionbox}
આસ્ટેબલ મલ્ટીવાઇબ્રેટર બાહ્ય ટ્રિગર વિના સતત સ્ક્વેર વેવ્સ ઉત્પન્ન કરે
છે.

\textbf{સર્કિટ:}

\begin{center}
\textbf{Mermaid Diagram (Code)}
\begin{verbatim}
{Shaded}
{Highlighting}[]
graph LR
    VCC((+VCC)) {-{-}{-} RA[RA]}
    RA {-{-}{-} RB[RB]}
    RB {-{-}{-} DIS[7:DIS]}
    RA {-{-}{-} RST[4:RST]}
    RA {-{-}{-} VCC\_PIN[8:VCC]}
    TRG[2:TRIG] {-{-}{-} C[C]}
    THR[6:THRES] {-{-}{-} C}
    C {-{-}{-} GND((GND))}
    TRG {-{-}{-} THR}
    VCC\_PIN {-{-}{-} IC[555 Timer]}
    RST {-{-}{-} IC}
    TRG {-{-}{-} IC}
    THR {-{-}{-} IC}
    IC {-{-}{-} OUT[3:OUT]}
    IC {-{-}{-} CTRL[5:CTRL]}
    CTRL {-{-}{-} CC[0.01µF]}
    CC {-{-}{-} GND}
    GND {-{-}{-} GND\_PIN[1:GND]}
    GND\_PIN {-{-}{-} IC}
    OUT {-{-}{-} Output((Output))}
    DIS {-{-}{-} THR}
{Highlighting}
{Shaded}
\end{verbatim}
\end{center}

\textbf{આઉટપુટ વેવફોર્મ:}

\begin{verbatim}
   HIGH  \_\_\_\_      \_\_\_\_      \_\_\_\_
        |    |    |    |    |    |
        |    |    |    |    |    |
   LOW  |\_\_\_\_|    |\_\_\_\_|    |\_\_\_\_|
        
        | T1 | T2 | T1 | T2 | T1 |
\end{verbatim}

\begin{itemize}
\tightlist
\item
  \textbf{ટાઇમિંગ}: T1 = 0.693(RA+RB)C, T2 = 0.693(RB)C
\item
  \textbf{ફ્રીક્વન્સી}: f = 1.44/((RA+2RB)C)
\item
  \textbf{ડ્યુટી સાયકલ}: RA અને RB દ્વારા એડજસ્ટ થઈ શકે છે
\item
  \textbf{ઉપયોગો}: ક્લોક જનરેટર, LED ફ્લેશર, ટોન જનરેટર
\end{itemize}

\end{solutionbox}
\begin{mnemonicbox}
``હંમેશા ઓસિલેટિંગ, ક્યારેય સ્ટોપિંગ નહીં''

\end{mnemonicbox}
\subsection*{પ્રશ્ન 5(a) [3
marks]}\label{q5a}

\textbf{રેગ્યુલેટેડ પાવર સપ્લાયનો બેઝીક બ્લોક ડાયગ્રામ દોરો અને તેને સમજાવો.}

\begin{solutionbox}
રેગ્યુલેટેડ પાવર સપ્લાય AC ને સ્થિર DC વોલ્ટેજમાં રૂપાંતરિત કરે છે.

\textbf{બ્લોક ડાયગ્રામ:}

\begin{verbatim}
flowchart LR
    A[AC Input] {-{-} B[Transformer]}
    B {-{-} C[Rectifier]}
    C {-{-} D[Filter]}
    D {-{-} E[Regulator]}
    E {-{-} F[DC Output]}
\end{verbatim}

\begin{itemize}
\tightlist
\item
  \textbf{ટ્રાન્સફોર્મર}: AC વોલ્ટેજને જરૂરી લેવલ સુધી ઘટાડે છે
\item
  \textbf{રેક્ટિફાયર}: AC ને પલ્સેટિંગ DC માં રૂપાંતરિત કરે છે (ડાયોડ બ્રિજ)
\item
  \textbf{ફિલ્ટર}: પલ્સેટિંગ DC ને સ્મૂધ કરે છે (કેપેસિટર્સ)
\item
  \textbf{રેગ્યુલેટર}: ફેરફારો છતાં સતત આઉટપુટ જાળવે છે
\item
  \textbf{આઉટપુટ}: ઇલેક્ટ્રોનિક સર્કિટ્સ માટે સ્થિર DC વોલ્ટેજ
\end{itemize}

\end{solutionbox}
\begin{mnemonicbox}
``ટ્રાન્સફોર્મર રેક્ટિફાય ફિલ્ટર રેગ્યુલેટ''

\end{mnemonicbox}
\subsection*{પ્રશ્ન 5(b) [4
marks]}\label{q5b}

\textbf{Op-ampની મદદથી સમિંગ એમ્પલીફાયર દોરો અને સમજાવો.}

\begin{solutionbox}
સમિંગ એમ્પલિફાયર વજનદાર અનુપાત સાથે બહુવિધ ઇનપુટ સિગ્નલ્સને ઉમેરે છે.

\textbf{સર્કિટ:}

\begin{center}
\textbf{Mermaid Diagram (Code)}
\begin{verbatim}
{Shaded}
{Highlighting}[]
graph LR
    IN1((V1)) {-{-}{-} R1[R1]}
    IN2((V2)) {-{-}{-} R2[R2]}
    IN3((V3)) {-{-}{-} R3[R3]}
    R1 {-{-}{-} SUM((Summing Point))}
    R2 {-{-}{-} SUM}
    R3 {-{-}{-} SUM}
    SUM {-{-}{-} INV[Inv Input]}
    INV {-{-}{-} IC[Op{-}Amp]}
    IC {-{-}{-} OUT((Output))}
    OUT {-{-}{-} Rf[Rf]}
    Rf {-{-}{-} SUM}
    NINV[Non{-Inv Input] {-}{-}{-} GND((GND))}
    NINV {-{-}{-} IC}
{Highlighting}
{Shaded}
\end{verbatim}
\end{center}

\begin{itemize}
\tightlist
\item
  \textbf{આઉટપુટ સમીકરણ}: Vout = -Rf(V1/R1 + V2/R2 + V3/R3)
\item
  \textbf{વિશેષ કેસ}: જ્યારે બધા રેસિસ્ટર સમાન હોય, Vout = -Rf/R \times (V1 + V2 +
  V3)
\item
  \textbf{ઉપયોગો}: ઓડિયો મિક્સિંગ, એનાલોગ કમ્પ્યુટર, સિગ્નલ એવરેજિંગ
\item
  \textbf{વેરિએશન્સ}: ઇન્વર્ટિંગ અને નોન-ઇન્વર્ટિંગ કોન્ફિગરેશન ઉપલબ્ધ
\end{itemize}

\end{solutionbox}
\begin{mnemonicbox}
``મલ્ટિપલ ઇનપુટ, વન આઉટપુટ, વેઇટેડ એડિશન''

\end{mnemonicbox}
\subsection*{પ્રશ્ન 5(c) [7
marks]}\label{q5c}

\textbf{IC LM317ની મદદથી 3 ટર્મિનલવાળા એડજસ્ટેબલ આઉટપુટ વોલ્ટેજ રેગ્યુલેટરનો સર્કિટ
ડાયગ્રામ દોરો અને સમજાવો.}

\begin{solutionbox}
LM317 એ 1.25V થી 37V સુધીની આઉટપુટ રેન્જ સાથે વર્સેટાઇલ એડજસ્ટેબલ
વોલ્ટેજ રેગ્યુલેટર છે.

\textbf{સર્કિટ:}

\begin{center}
\textbf{Mermaid Diagram (Code)}
\begin{verbatim}
{Shaded}
{Highlighting}[]
graph LR
    VIN((Vin)) {-{-}{-} C1[C1]}
    C1 {-{-}{-} IN[Input]}
    IN {-{-}{-} LM317[LM317]}
    LM317 {-{-}{-} OUT[Output]}
    OUT {-{-}{-} C2[C2]}
    C2 {-{-}{-} VOUT((Vout))}
    OUT {-{-}{-} R1[R1=240Ω]}
    R1 {-{-}{-} ADJ[Adjust]}
    ADJ {-{-}{-} R2[R2]}
    R2 {-{-}{-} GND((GND))}
    ADJ {-{-}{-} LM317}
    C2 {-{-}{-} GND}
    C1 {-{-}{-} GND}
{Highlighting}
{Shaded}
\end{verbatim}
\end{center}

\begin{itemize}
\tightlist
\item
  \textbf{આઉટપુટ વોલ્ટેજ}: VOUT = 1.25V(1 + R2/R1)
\item
  \textbf{ફિક્સ્ડ કમ્પોનન્ટ્સ}: R1 = 240Ω, રેફરન્સ વોલ્ટેજ = 1.25V
\item
  \textbf{એડજસ્ટેબિલિટી}: R2 બદલવાથી ઇચ્છિત આઉટપુટ વોલ્ટેજ સેટ થાય છે
\item
  \textbf{પ્રોટેક્શન ફીચર્સ}: કરંટ લિમિટિંગ, થર્મલ શટડાઉન
\item
  \textbf{ઉપયોગો}: વેરિએબલ પાવર સપ્લાય, બેટરી ચાર્જર
\item
  \textbf{ફાયદા}: ઓછા બાહ્ય ઘટકો, મજબૂત સુરક્ષા
\end{itemize}

\end{solutionbox}
\begin{mnemonicbox}
``R2 વડે એડજસ્ટ કરો, રેફરન્સ 1.25 પર રહે છે''

\end{mnemonicbox}
\subsection*{પ્રશ્ન 5(a OR) [3
marks]}\label{uxaaauxab0uxab6uxaa8-5a-or-3-marks}

\textbf{એસ.એમ.પી.એસનું સંપૂર્ણ ફોર્મ જણાવો. ઉપરાંત એસ.એમ.પી.એસના કાર્યો જણાવો.}

\begin{solutionbox}
SMPS એટલે Switch Mode Power Supply, એક આધુનિક કાર્યક્ષમ પાવર
રૂપાંતરણ ટેકનોલોજી.

\textbf{ઉપયોગ કોષ્ટક:}

{\def\LTcaptype{none} % do not increment counter
\begin{longtable}[]{@{}lll@{}}
\toprule\noalign{}
ઉપયોગ & SMPS પ્રકાર & ફાયદા \\
\midrule\noalign{}
\endhead
\bottomrule\noalign{}
\endlastfoot
કમ્પ્યુટર પાવર સપ્લાય & ATX & ઉચ્ચ કાર્યક્ષમતા, મલ્ટિપલ આઉટપુટ \\
મોબાઇલ ફોન ચાર્જર & ફ્લાયબૅક & કોમ્પેક્ટ સાઇઝ, હળવું વજન \\
LED ડ્રાઇવર & બક & કાર્યક્ષમ ડિમિંગ ક્ષમતા \\
TV પાવર સપ્લાય & ફોરવર્ડ & સારી રેગ્યુલેશન, મલ્ટિપલ આઉટપુટ \\
ઔદ્યોગિક કંટ્રોલ & પુશ-પુલ & ઉચ્ચ પાવર ક્ષમતા \\
બેટરી ચાર્જર & બૂસ્ટ & એડજસ્ટેબલ ચાર્જિંગ પ્રોફાઇલ \\
\end{longtable}
}

\begin{itemize}
\tightlist
\item
  \textbf{મુખ્ય ફાયદા}: ઉચ્ચ કાર્યક્ષમતા (80-95\%), નાનો આકાર, હળવું
\item
  \textbf{નુકસાન}: EMI ઉત્પાદન, વધુ જટિલ સર્કિટ
\end{itemize}

\end{solutionbox}
\begin{mnemonicbox}
``સ્વિચ મોડ નાના ઉપકરણોને પાવર આપે છે''

\end{mnemonicbox}
\subsection*{પ્રશ્ન 5(b OR) [4
marks]}\label{uxaaauxab0uxab6uxaa8-5b-or-4-marks}

\textbf{Op-ampની મદદથી ડિફ્રન્સીએટર દોરો અને સમજાવો.}

\begin{solutionbox}
ડિફરન્શિએટર ઇનપુટના ફેરફારના દરના સમપ્રમાણમાં આઉટપુટ ઉત્પન્ન કરે
છે.

\textbf{સર્કિટ:}

\begin{center}
\textbf{Mermaid Diagram (Code)}
\begin{verbatim}
{Shaded}
{Highlighting}[]
graph LR
    IN((Input)) {-{-}{-} C[C]}
    C {-{-}{-} INV[Inv Input]}
    INV {-{-}{-} IC[Op{-}Amp]}
    IC {-{-}{-} OUT((Output))}
    OUT {-{-}{-} Rf[Rf]}
    Rf {-{-}{-} INV}
    NINV[Non{-Inv Input] {-}{-}{-} GND((GND))}
    NINV {-{-}{-} IC}
{Highlighting}
{Shaded}
\end{verbatim}
\end{center}

\textbf{ઇનપુટ/આઉટપુટ વેવફોર્મ્સ:}

\begin{verbatim}
Input:      \_\_\_
           /   {}
          /     {\_\_\_}
         /
\_\_\_\_\_\_\_\_/

Output:   |
          |
     \_\_\_\_\_|\_\_\_\_\_
         / {}
        /   {}
\end{verbatim}

\begin{itemize}
\tightlist
\item
  \textbf{સમીકરણ}: Vout = -RC \times d(Vin)/dt
\item
  \textbf{ફંક્શન}: સ્ક્વેર વેવને સ્પાઇક્સમાં, ટ્રાયેંગલને સ્ક્વેરમાં રૂપાંતરિત કરે છે
\item
  \textbf{પ્રેક્ટિકલ સમસ્યા}: ઉચ્ચ નોઈઝ સેન્સિટિવિટી
\item
  \textbf{મોડિફિકેશન}: ઉચ્ચ-ફ્રીક્વન્સી ગેઈન મર્યાદિત કરવા માટે C સાથે શ્રેણીમાં
  નાનો રેસિસ્ટર
\item
  \textbf{ઉપયોગો}: વેવશેપિંગ, ફેરફાર-દરની શોધ
\end{itemize}

\end{solutionbox}
\begin{mnemonicbox}
``ફેરફારનો દર અંદર જાય, એમ્પલિટ્યુડ બહાર આવે''

\end{mnemonicbox}
\subsection*{પ્રશ્ન 5(c OR) [7
marks]}\label{uxaaauxab0uxab6uxaa8-5c-or-7-marks}

\textbf{-12 V રેગ્યુલેટેડ પાવર સપ્લાયનો સર્કિટ ડાયગ્રામ દોરો અને સમજાવો.}

\begin{solutionbox}
-12V રેગ્યુલેટેડ સપ્લાય એનાલોગ સર્કિટ્સ માટે સ્થિર નેગેટિવ વોલ્ટેજ
પ્રદાન કરે છે.

\textbf{સર્કિટ ડાયગ્રામ:}

\begin{center}
\textbf{Mermaid Diagram (Code)}
\begin{verbatim}
{Shaded}
{Highlighting}[]
graph LR
    AC((AC Input)) {-{-}{-} TRANS[Transformer]}
    TRANS {-{-}{-} D1[D1]}
    D1 {-{-}{-} D2[D2]}
    D2 {-{-}{-} C1[Filter Cap]}
    C1 {-{-}{-} IC[7912 IC]}
    IC {-{-}{-} C2[0.1µF]}
    C2 {-{-}{-} OUT[({}{-}12V Output)]}
    C1 {-{-}{-} GND((GND))}
    IC {-{-}{-} GND}
    C2 {-{-}{-} GND}
    D3[D3] {-{-}{-} D4[D4]}
    D3 {-{-}{-} TRANS}
    D4 {-{-}{-} C1}
{Highlighting}
{Shaded}
\end{verbatim}
\end{center}

\begin{itemize}
\tightlist
\item
  \textbf{કાર્યસિદ્ધાંત}: ફુલ-વેવ રેક્ટિફાયર નેગેટિવ વોલ્ટેજ બનાવે છે
\item
  \textbf{ઘટકો}: ટ્રાન્સફોર્મર, બ્રિજ રેક્ટિફાયર, ફિલ્ટર કેપેસિટર, 7912 રેગ્યુલેટર
\item
  \textbf{રેગ્યુલેટર IC}: 7912 આંતરિક સુરક્ષા સાથે ફિક્સ્ડ -12V આઉટપુટ પ્રદાન કરે છે
\item
  \textbf{ફિલ્ટર કેપેસિટર}: ઇનપુટ કેપેસિટર રિપલ ફિલ્ટર કરે છે, આઉટપુટ કેપેસિટર
  ટ્રાન્ઝિયન્ટ રિસ્પોન્સ સુધારે છે
\item
  \textbf{ઉપયોગો}: Op-amp નેગેટિવ રેલ, એનાલોગ સર્કિટ્સ, ઓડિયો ઇક્વિપમેન્ટ
\end{itemize}

\end{solutionbox}
\begin{mnemonicbox}
``ફુલ બ્રિજ, મોટો કેપેસિટર, 7912 નેગેટિવ રેગ્યુલેટ કરે છે''

આ સાથે ઇલેક્ટ્રોનિક્સ ડિવાઇસીસ એન્ડ સર્કિટ્સ વિન્ટર 2024 પરીક્ષા પેપરના બધા પ્રશ્નોના
ઉકેલ, બધા OR પ્રશ્નો સહિત પૂર્ણ થાય છે.

\end{mnemonicbox}

\end{document}
