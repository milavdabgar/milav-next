\documentclass[10pt,a4paper]{article}

% content/resources/templates/preamble.tex
\usepackage[margin=0.6in]{geometry}
\author{Milav Dabgar}
\usepackage{amsmath,amssymb,amsthm}
\usepackage{booktabs}
\usepackage{multirow}
\usepackage{xcolor}
\usepackage{tcolorbox}
\tcbuselibrary{breakable,skins}
\usepackage[colorlinks=true,linkcolor=blue]{hyperref}
\usepackage{titlesec}
\usepackage{enumitem}
\usepackage{tikz}
\usepackage{pgfplots}
\usepackage{circuitikz}
\usepackage[version=4]{mhchem}
\usepackage{longtable}
\usepackage{array}
\usepackage{float}
\usepackage{caption}
\usepackage{listings}

\lstset{
  basicstyle=\small\ttfamily,
  breaklines=true,
  breakatwhitespace=false,
  postbreak=\mbox{\textcolor{red}{$\hookrightarrow$}\space},
  float=false,
  numbers=left,
  numberstyle=\tiny\color{gray},
  numbersep=10pt,
  xleftmargin=2em,
  keywordstyle=\color{blue},
  commentstyle=\color{green!60!black},
  stringstyle=\color{purple},
  backgroundcolor=\color{gray!5},
  showstringspaces=false,
  tabsize=2,
  captionpos=b,
  keepspaces=true,
  columns=flexible
}

\pgfplotsset{compat=1.18}
\usetikzlibrary{shapes,arrows,positioning,calc,patterns,decorations.pathmorphing,decorations.markings,arrows.meta}

% Color scheme
\definecolor{headcolor}{RGB}{0,102,204}
\definecolor{keycolor}{RGB}{220,20,60}
\definecolor{solutioncolor}{RGB}{34,139,34}
\definecolor{mnemoniccolor}{RGB}{148,0,211}
\definecolor{codecolor}{RGB}{0,0,100}

% Spacing
\setlength{\parskip}{3pt}
\setlist[itemize]{nosep}
\setlist[enumerate]{nosep}

% Title formatting
\titleformat{\section}{\Large\bfseries\color{headcolor}}{\thesection}{1em}{}
\titleformat{\subsection}{\large\bfseries\color{headcolor}}{\thesubsection}{1em}{}

% Pandoc tightlist compatibility
\providecommand{\tightlist}{%
  \setlength{\itemsep}{0pt}\setlength{\parskip}{0pt}}

% Pandoc longtable compatibility
\newcounter{none}
\def\thenone{}


% content/resources/templates/english-boxes.tex
% This file is currently empty - it exists to maintain consistency with the import structure.
% Add custom environments here if needed in the future.


\begin{document}

\begin{center}
{\Huge\bfseries\color{headcolor} Subject Name Solutions}\\[5pt]
{\LARGE 1323202 -- Summer 2023}\\[3pt]
{\large Semester 1 Study Material}\\[3pt]
{\normalsize\textit{Detailed Solutions and Explanations}}
\end{center}

\vspace{10pt}

\subsection*{Question 1(a) [3 marks]}\label{q1a}

\textbf{Draw the symbol of (1)SCR (2)Diac(3)Triac}

\begin{solutionbox}

\textbf{Diagram:}

\begin{lstlisting}
SCR Symbol:        DIAC Symbol:       TRIAC Symbol:
   A                   A1                  MT2
   |                    |                   |
   ▼                    ▼                   ▼
  ┌─┐                  ┌─┐                 ┌─┐
  │ │                  │ │                 │ │
──┤ ├──              ──┤ ├──             ──┤ ├──
  │ │                  │ │                 │ │
  └─┘                  └─┘                 └─┘
   ▲                    ▲                   ▲
   |                    |                   |
   K                   A2                  MT1
  /                                        /
 /                                        /
G                                        G
\end{lstlisting}

\begin{itemize}
\tightlist
\item
  \textbf{SCR (Silicon Controlled Rectifier)}: Three-terminal device
  with Anode, Cathode, and Gate
\item
  \textbf{DIAC (Diode AC switch)}: Two-terminal bidirectional device
  with terminals A1 and A2
\item
  \textbf{TRIAC (Triode AC switch)}: Three-terminal bidirectional device
  with MT1, MT2, and Gate
\end{itemize}

\end{solutionbox}
\begin{mnemonicbox}
``AGK for SCR, AA for DIAC, MMG for TRIAC''

\end{mnemonicbox}
\subsection*{Question 1(b) [4 marks]}\label{q1b}

\textbf{Explain the term(1) CMRR (2) Slew rate}

\begin{solutionbox}


{\def\LTcaptype{none} % do not increment counter
\vspace{-5pt}
\captionof{table}{Op-Amp Parameters}
\vspace{-10pt}
\begin{longtable}[]{@{}
  >{\raggedright\arraybackslash}p{(\linewidth - 4\tabcolsep) * \real{0.2973}}
  >{\raggedright\arraybackslash}p{(\linewidth - 4\tabcolsep) * \real{0.3243}}
  >{\raggedright\arraybackslash}p{(\linewidth - 4\tabcolsep) * \real{0.3784}}@{}}
\toprule\noalign{}
\begin{minipage}[b]{\linewidth}\raggedright
Parameter
\end{minipage} & \begin{minipage}[b]{\linewidth}\raggedright
Definition
\end{minipage} & \begin{minipage}[b]{\linewidth}\raggedright
Significance
\end{minipage} \\
\midrule\noalign{}
\endhead
\bottomrule\noalign{}
\endlastfoot
\textbf{CMRR (Common Mode Rejection Ratio)} & Ratio of differential gain
to common mode gain expressed in dB & Higher CMRR means better rejection
of common input signals \\
\textbf{Slew Rate} & Maximum rate of change of output voltage (V/μs) &
Determines how fast op-amp responds to rapidly changing inputs \\
\end{longtable}
}

\begin{itemize}
\tightlist
\item
  \textbf{CMRR formula}: CMRR = 20 log_{1}_{0}(Ad/Acm) dB
\item
  \textbf{Slew Rate importance}: Affects high-frequency performance and
  prevents distortion
\end{itemize}

\end{solutionbox}
\begin{mnemonicbox}
``Common Mode Rejected Rapidly, Slew shows Signal
Speed''

\end{mnemonicbox}
\subsection*{Question 1(c) [7 marks]}\label{q1c}

\textbf{Draw and explain summing amplifier.}

\begin{solutionbox}

\textbf{Diagram:}

\includegraphics[width=1\linewidth,height=\textheight,keepaspectratio]{mermaid-ec42ae1e.pdf}

\textbf{Operation of Summing Amplifier:}

\begin{itemize}
\item
  \textbf{Circuit function}: Adds multiple input voltages with scaling
\item
  \textbf{Output equation}: Vout = -(Rf/R1 \times V1 + Rf/R2 \times V2 + Rf/R3 \times
  V3)
\item
  \textbf{Inverting configuration}: Input signals undergo 180^\circ phase
  shift
\item
  \textbf{Gain control}: Rf/Rn determines weight of each input signal
\item
  \textbf{Application}: Audio mixing, analog computation, signal
  processing
\item
  \textbf{Key feature}: Virtual ground at inverting input simplifies
  analysis
\end{itemize}

\end{solutionbox}
\begin{mnemonicbox}
``Sum with Weights: Vout = -Rf(V1/R1 + V2/R2 +
V3/R3)''

\end{mnemonicbox}
\subsection*{Question 1(c OR) [7
marks]}\label{question-1c-or-7-marks}

\textbf{Draw and explain DA converter}

\begin{solutionbox}

\textbf{Diagram:}

\includegraphics[width=1\linewidth,height=\textheight,keepaspectratio]{mermaid-75bcefbd.pdf}

\textbf{R-2R Ladder DAC Operation:}

\begin{itemize}
\item
  \textbf{Function}: Converts digital binary input to analog output
  voltage
\item
  \textbf{Working principle}: Weighted resistor network creates scaled
  currents
\item
  \textbf{Binary weighting}: Each bit contributes voltage proportional
  to its position (2^{n})
\item
  \textbf{Resolution}: Determined by number of bits (N) as 1/2ᴺ of full
  scale
\item
  \textbf{Advantages}: Simple design, good accuracy, fast conversion
\item
  \textbf{Applications}: Audio equipment, signal generation, control
  systems
\end{itemize}

\end{solutionbox}
\begin{mnemonicbox}
``Digital Bits to Analog Steps - R-2R makes the
magic''

\end{mnemonicbox}
\subsection*{Question 2(a) [3 marks]}\label{q2a}

\textbf{Describe thermal run away of transistor.}

\begin{solutionbox}

\textbf{Thermal Runaway Process:}

\includegraphics[width=1\linewidth,height=\textheight,keepaspectratio]{mermaid-96516664.pdf}

\begin{itemize}
\tightlist
\item
  \textbf{Definition}: Self-accelerating process where transistor heats
  up and draws more current
\item
  \textbf{Cause}: Negative temperature coefficient of base-emitter
  voltage
\item
  \textbf{Prevention}: Use proper heat sink and stabilization circuits
\end{itemize}

\end{solutionbox}
\begin{mnemonicbox}
``Heat feeds Current feeds Heat - a dangerous loop''

\end{mnemonicbox}
\subsection*{Question 2(b) [4 marks]}\label{q2b}

\textbf{Draw and explain voltage series negative feedback.}

\begin{solutionbox}

\textbf{Diagram:}

\includegraphics[width=1\linewidth,height=\textheight,keepaspectratio]{mermaid-b99a252d.pdf}

\textbf{Voltage Series Negative Feedback:}

{\def\LTcaptype{none} % do not increment counter
\begin{longtable}[]{@{}
  >{\raggedright\arraybackslash}p{(\linewidth - 2\tabcolsep) * \real{0.2821}}
  >{\raggedright\arraybackslash}p{(\linewidth - 2\tabcolsep) * \real{0.7179}}@{}}
\toprule\noalign{}
\begin{minipage}[b]{\linewidth}\raggedright
Parameter
\end{minipage} & \begin{minipage}[b]{\linewidth}\raggedright
Effect of Negative Feedback
\end{minipage} \\
\midrule\noalign{}
\endhead
\bottomrule\noalign{}
\endlastfoot
\textbf{Gain stability} & Improved, less dependent on amplifier
parameters \\
\textbf{Bandwidth} & Increased proportional to feedback factor \\
\textbf{Distortion} & Reduced significantly \\
\textbf{Input impedance} & Increased \\
\end{longtable}
}

\begin{itemize}
\tightlist
\item
  \textbf{Working principle}: Output voltage is sampled and fed back to
  input
\item
  \textbf{Gain formula}: Closed-loop gain = Open-loop gain/(1 + βA)
\end{itemize}

\end{solutionbox}
\begin{mnemonicbox}
``Series says Sample Voltage, Stabilize Gain''

\end{mnemonicbox}
\subsection*{Question 2(c) [7 marks]}\label{q2c}

\textbf{Draw and explain DC load line for common emitter amplifier.}

\begin{solutionbox}

\textbf{Diagram:}

\includegraphics[width=1\linewidth,height=\textheight,keepaspectratio]{mermaid-0975236a.pdf}

\textbf{DC Load Line Characteristics:}

\begin{itemize}
\tightlist
\item
  \textbf{Definition}: Graphical representation of all possible
  operating points
\item
  \textbf{Equation}: IC = VCC/RC - VCE/RC
\item
  \textbf{Key points}:

  \begin{itemize}
  \tightlist
  \item
    Saturation point (VCE \approx 0V, IC = VCC/RC)
  \item
    Cutoff point (IC \approx 0mA, VCE = VCC)
  \item
    Q-point (selected operating point for amplification)
  \end{itemize}
\item
  \textbf{Significance}: Determines biasing stability and output signal
  limits
\item
  \textbf{Relationship}: DC load line is fixed by circuit components
  (VCC and RC)
\end{itemize}

\end{solutionbox}
\begin{mnemonicbox}
``Connect Cutoff to Saturation for DC Load Line''

\end{mnemonicbox}
\subsection*{Question 2(a OR) [3
marks]}\label{question-2a-or-3-marks}

\textbf{Explain operating point(Q-point) in transistor}

\begin{solutionbox}

\textbf{Q-Point (Operating Point):}

\begin{lstlisting}
      |
  Ic  |      DC Load Line
      |          /
      |         /
      |        /
      |       * Q-Point
      |      /
      |     /
      |    /
      |___/____________
          Vce
\end{lstlisting}

\begin{itemize}
\tightlist
\item
  \textbf{Definition}: Specific DC bias point where transistor operates
  in active region
\item
  \textbf{Importance}: Determines output signal range without distortion
\item
  \textbf{Selection criteria}: Center of load line for maximum swing
\end{itemize}

\end{solutionbox}
\begin{mnemonicbox}
``Quality amplification needs Quiet bias at Q-point''

\end{mnemonicbox}
\subsection*{Question 2(b OR) [4
marks]}\label{question-2b-or-4-marks}

\textbf{Draw and explain hartley oscillator.}

\begin{solutionbox}

\textbf{Diagram:}

\includegraphics[width=1\linewidth,height=\textheight,keepaspectratio]{mermaid-b9d76e65.pdf}

\textbf{Hartley Oscillator:}

\begin{itemize}
\tightlist
\item
  \textbf{Configuration}: Common emitter with tapped inductor feedback
\item
  \textbf{Frequency formula}: f = 1/[2π\sqrt(C\times(L1+L2))]
\item
  \textbf{Phase shift}: Ensures 360^\circ total phase shift for oscillation
\item
  \textbf{Feedback}: Inductive voltage divider provides positive
  feedback
\end{itemize}

\end{solutionbox}
\begin{mnemonicbox}
``Hartley Has two coils with inductance for LC
oscillation''

\end{mnemonicbox}
\subsection*{Question 2(c OR) [7
marks]}\label{question-2c-or-7-marks}

\textbf{Draw and explain AC load line for common emitter amplifier.}

\begin{solutionbox}

\textbf{Diagram:}

\includegraphics[width=1\linewidth,height=\textheight,keepaspectratio]{mermaid-09ee3c17.pdf}

\textbf{AC Load Line Characteristics:}

\begin{itemize}
\tightlist
\item
  \textbf{Definition}: Represents dynamic operation during signal
  amplification
\item
  \textbf{Equation}: ic = (VCC-VCEQ)/R'c - vce/R'c where R'c =
  RC\textbar\textbar RL
\item
  \textbf{Comparison with DC load line}:

  \begin{itemize}
  \tightlist
  \item
    AC load line is steeper than DC load line
  \item
    Passes through Q-point
  \item
    Determines voltage and current signal swings
  \end{itemize}
\item
  \textbf{Significance}: Defines maximum undistorted output signal
\item
  \textbf{Limiting factor}: Avoiding saturation and cutoff regions
\end{itemize}

\end{solutionbox}
\begin{mnemonicbox}
``AC Amplitude Controlled by Load line Angle''

\end{mnemonicbox}
\subsection*{Question 3(a) [3 marks]}\label{q3a}

\textbf{Draw the fixed bias circuit and explain working of it}

\begin{solutionbox}

\textbf{Diagram:}

\begin{lstlisting}
      Vcc
       |
       R
       |
       |C
       |----Output
       |
      /|
     / |
    /--|
   /   |
  |    |
  B    E
  |    |
  Rb   |
  |    |
  |____|
  |
  Vin
\end{lstlisting}

\begin{itemize}
\tightlist
\item
  \textbf{Structure}: Base resistor connected to VCC, collector resistor
  for load
\item
  \textbf{Operation}: Fixed base current biases transistor
\item
  \textbf{Disadvantage}: Poor stability against temperature changes
\end{itemize}

\end{solutionbox}
\begin{mnemonicbox}
``Fixed Bias Feeds Base from power supply''

\end{mnemonicbox}
\subsection*{Question 3(b) [4 marks]}\label{q3b}

\textbf{In hartley oscillator L1=5mH, L2=10mH, C=0.01µF. Calculate
frequency of oscillations.}

\begin{solutionbox}

\textbf{Solution:}

\begin{itemize}
\tightlist
\item
  \textbf{Given}: L1=5mH, L2=10mH, C=0.01µF
\item
  \textbf{Frequency formula}: f = 1/[2π\sqrt(C\times(L1+L2))]
\item
  \textbf{Calculation}:

  \begin{itemize}
  \tightlist
  \item
    Total inductance LT = L1 + L2 = 5mH + 10mH = 15mH = 15\times10^{-}^{3} H
  \item
    C = 0.01µF = 1\times10^{-}^{8} F
  \item
    f = 1/[2π\sqrt(15\times10^{-}^{3} \times 1\times10^{-}^{8})]
  \item
    f = 1/[2π\sqrt(15\times10^{-}^{1}^{1})]
  \item
    f = 1/[2π\times3.873\times10^{-}^{6}]
  \item
    f = 1/[24.33\times10^{-}^{6}]
  \item
    f = 41,101 Hz \approx 41.1 kHz
  \end{itemize}
\end{itemize}

\end{solutionbox}
\begin{mnemonicbox}
``For Hartley's frequency, add coils then take square
root''

\end{mnemonicbox}
\subsection*{Question 3(c) [7 marks]}\label{q3c}

\textbf{Draw and explain the frequency response curve of two stage RC
coupled amplifier.}

\begin{solutionbox}

\textbf{Diagram:}

\includegraphics[width=1\linewidth,height=\textheight,keepaspectratio]{mermaid-4ac37bc7.pdf}

\textbf{Two-Stage RC Coupled Amplifier Frequency Response:}

\begin{itemize}
\tightlist
\item
  \textbf{Low-frequency region}: Gain rises with frequency (\textless{}
  50Hz)

  \begin{itemize}
  \tightlist
  \item
    Limited by coupling and bypass capacitors
  \end{itemize}
\item
  \textbf{Mid-frequency region}: Constant maximum gain (50Hz-20kHz)

  \begin{itemize}
  \tightlist
  \item
    Flat response, ideal operating region
  \end{itemize}
\item
  \textbf{High-frequency region}: Gain drops with frequency
  (\textgreater{} 20kHz)

  \begin{itemize}
  \tightlist
  \item
    Limited by transistor capacitances and Miller effect
  \end{itemize}
\item
  \textbf{Bandwidth}: Range of frequencies with gain \geq 70.7\% of maximum
  gain
\item
  \textbf{Cutoff frequencies}: Points where gain drops by 3dB (0.707
  times max gain)
\end{itemize}

\end{solutionbox}
\begin{mnemonicbox}
``Low-flat-high: capacitors block, amplify well, then
roll off''

\end{mnemonicbox}
\subsection*{Question 3(a OR) [3
marks]}\label{question-3a-or-3-marks}

\textbf{Explain in detail barkhausen criterion for oscillation.}

\begin{solutionbox}

\textbf{Barkhausen Criterion:}

{\def\LTcaptype{none} % do not increment counter
\begin{longtable}[]{@{}ll@{}}
\toprule\noalign{}
Condition & Requirement \\
\midrule\noalign{}
\endhead
\bottomrule\noalign{}
\endlastfoot
\textbf{Loop Gain} & Must equal exactly 1 (Aβ = 1) \\
\textbf{Phase Shift} & Must be 0^\circ or 360^\circ around loop \\
\end{longtable}
}

\begin{itemize}
\tightlist
\item
  \textbf{Purpose}: Ensures sustained oscillations without damping
\item
  \textbf{Consequences}:

  \begin{itemize}
  \tightlist
  \item
    If Aβ \textless{} 1: Oscillations die out
  \item
    If Aβ \textgreater{} 1: Oscillations grow until limited by
    nonlinearity
  \item
    If Aβ = 1: Stable oscillations maintained
  \end{itemize}
\end{itemize}

\end{solutionbox}
\begin{mnemonicbox}
``Barkhausen's Balance: Loop Gain=1, Phase=360^\circ''

\end{mnemonicbox}
\subsection*{Question 3(b OR) [4
marks]}\label{question-3b-or-4-marks}

\textbf{Explain the effect of negative feedback on the gain of
amplifier}

\begin{solutionbox}

\textbf{Effect of Negative Feedback on Amplifier Gain:}

{\def\LTcaptype{none} % do not increment counter
\begin{longtable}[]{@{}lll@{}}
\toprule\noalign{}
Parameter & Without Feedback & With Feedback \\
\midrule\noalign{}
\endhead
\bottomrule\noalign{}
\endlastfoot
\textbf{Voltage Gain} & A & A/(1+Aβ) \\
\textbf{Stability} & Less stable & More stable \\
\textbf{Bandwidth} & Lower & Higher \\
\textbf{Distortion} & Higher & Lower \\
\end{longtable}
}

\begin{itemize}
\tightlist
\item
  \textbf{Gain reduction}: Gain decreases by factor (1+Aβ)
\item
  \textbf{Gain-bandwidth tradeoff}: Bandwidth increases as gain
  decreases
\item
  \textbf{Gain stabilization}: Less affected by temperature and
  component variations
\end{itemize}

\end{solutionbox}
\begin{mnemonicbox}
``Negative Feedback: Less Gain, More Stability''

\end{mnemonicbox}
\subsection*{Question 3(c OR) [7
marks]}\label{question-3c-or-7-marks}

\textbf{Draw fan regulator circuit and explain how it will control the
speed of fan.}

\begin{solutionbox}

\textbf{Diagram:}

\includegraphics[width=1\linewidth,height=\textheight,keepaspectratio]{mermaid-99810a6c.pdf}

\textbf{Fan Regulator Operation:}

\begin{itemize}
\tightlist
\item
  \textbf{Control method}: Phase angle control using TRIAC and DIAC
\item
  \textbf{Working principle}: RC network creates variable phase shift
\item
  \textbf{Speed control}: Variable resistor adjusts RC time constant
\item
  \textbf{Operation sequence}:

  \begin{itemize}
  \tightlist
  \item
    RC network delays DIAC firing
  \item
    DIAC triggers TRIAC at adjustable point in AC cycle
  \item
    TRIAC conducts for remaining portion of AC half-cycle
  \item
    Less conduction time = lower power to fan = slower speed
  \end{itemize}
\item
  \textbf{Advantages}: Simple design, smooth control, energy efficient
\item
  \textbf{Applications}: Ceiling fans, exhaust fans, cooling systems
\end{itemize}

\end{solutionbox}
\begin{mnemonicbox}
``Delay the TRIAC firing, control fan's speed''

\end{mnemonicbox}
\subsection*{Question 4(a) [3 marks]}\label{q4a}

\textbf{Write short note on natural commutation}

\begin{solutionbox}

\textbf{Natural Commutation:}

\begin{itemize}
\tightlist
\item
  \textbf{Definition}: SCR turns off automatically when current falls
  below holding current
\item
  \textbf{Process}: Occurs in AC circuits at each zero-crossing point
\item
  \textbf{Requirements}: No external components needed, inherent to AC
  operation
\end{itemize}

\end{solutionbox}
\begin{mnemonicbox}
``Natural Commutation: Zero Current Crossings Turn
Off Thyristors''

\end{mnemonicbox}
\subsection*{Question 4(b) [4 marks]}\label{q4b}

\textbf{Explain the parameters gain and bandwidth of amplifier.}

\begin{solutionbox}

\textbf{Amplifier Parameters:}

{\def\LTcaptype{none} % do not increment counter
\begin{longtable}[]{@{}
  >{\raggedright\arraybackslash}p{(\linewidth - 4\tabcolsep) * \real{0.3438}}
  >{\raggedright\arraybackslash}p{(\linewidth - 4\tabcolsep) * \real{0.3750}}
  >{\raggedright\arraybackslash}p{(\linewidth - 4\tabcolsep) * \real{0.2812}}@{}}
\toprule\noalign{}
\begin{minipage}[b]{\linewidth}\raggedright
Parameter
\end{minipage} & \begin{minipage}[b]{\linewidth}\raggedright
Definition
\end{minipage} & \begin{minipage}[b]{\linewidth}\raggedright
Formula
\end{minipage} \\
\midrule\noalign{}
\endhead
\bottomrule\noalign{}
\endlastfoot
\textbf{Gain (A)} & Ratio of output to input signal & A = Vout/Vin \\
\textbf{Bandwidth (BW)} & Frequency range with gain \geq 70.7\% of maximum
& BW = fH - fL \\
\end{longtable}
}

\begin{itemize}
\tightlist
\item
  \textbf{Gain-bandwidth product}: Remains constant (GBP = Gain \times
  Bandwidth)
\item
  \textbf{Cutoff frequencies}: Lower (fL) and higher (fH) frequencies
  where gain drops by 3dB
\item
  \textbf{Significance}: Determines amplifier's ability to handle
  different frequencies
\end{itemize}

\end{solutionbox}
\begin{mnemonicbox}
``Good Amplifiers Balance Width and Magnitude''

\end{mnemonicbox}
\subsection*{Question 4(c) [7 marks]}\label{q4c}

\textbf{Draw the construction and characteristics of triac and describe
working of it, also write the application of triac.}

\begin{solutionbox}

\textbf{TRIAC Construction and Characteristics:}

\begin{lstlisting}
           MT2
            |
      ------+------
     /      |      \
    /  P    |    N  \
   +--------+--------+
   |        |        |
   |    N   |    P   |
   +--------+--------+
   |        |        |
   |    P   |    N   |
   +--------+--------+
    \       |       /
     \      |      /
      ------+------
            |
           MT1
            |
            G
\end{lstlisting}

\textbf{I-V Characteristics:}

\begin{lstlisting}
    I
    ^
    |      /|
    |     / |
    |    /  |
    |---+---|----> V
    |   /   |
    |  /    |
    | /     |
\end{lstlisting}

\textbf{TRIAC Operation:}

\begin{itemize}
\tightlist
\item
  \textbf{Structure}: Five-layer PNPN bidirectional device
\item
  \textbf{Switching}: Conducts in both directions when triggered
\item
  \textbf{Triggering modes}: Four quadrant operation possible
\item
  \textbf{Turn-off}: Natural commutation at current zero-crossing
\end{itemize}

\textbf{Applications:}

\begin{itemize}
\tightlist
\item
  \textbf{Light dimmers}
\item
  \textbf{Fan speed controllers}
\item
  \textbf{Heater controls}
\item
  \textbf{Motor speed regulation}
\item
  \textbf{AC power switching}
\end{itemize}

\end{solutionbox}
\begin{mnemonicbox}
``TRIAC Takes AC Control in Both Directions''

\end{mnemonicbox}
\subsection*{Question 4(a OR) [3
marks]}\label{question-4a-or-3-marks}

\textbf{Write any three application of SCR.}

\begin{solutionbox}

\textbf{Applications of SCR:}

{\def\LTcaptype{none} % do not increment counter
\begin{longtable}[]{@{}ll@{}}
\toprule\noalign{}
Application & Function \\
\midrule\noalign{}
\endhead
\bottomrule\noalign{}
\endlastfoot
\textbf{DC Motor Speed Control} & Provides variable DC to motors \\
\textbf{Battery Chargers} & Regulates charging current \\
\textbf{Power Inverters} & Converts DC to AC efficiently \\
\end{longtable}
}

\begin{itemize}
\tightlist
\item
  \textbf{Advantages}: High power handling, efficient control, robust
  operation
\item
  \textbf{Limitations}: Requires forced commutation in DC circuits
\end{itemize}

\end{solutionbox}
\begin{mnemonicbox}
``SCR Controls DC - Motors, Batteries, Inverters''

\end{mnemonicbox}
\subsection*{Question 4(b OR) [4
marks]}\label{question-4b-or-4-marks}

\textbf{Explain holding current and latching current with reference to
SCR}

\begin{solutionbox}

\textbf{SCR Current Parameters:}

{\def\LTcaptype{none} % do not increment counter
\begin{longtable}[]{@{}
  >{\raggedright\arraybackslash}p{(\linewidth - 4\tabcolsep) * \real{0.2821}}
  >{\raggedright\arraybackslash}p{(\linewidth - 4\tabcolsep) * \real{0.3077}}
  >{\raggedright\arraybackslash}p{(\linewidth - 4\tabcolsep) * \real{0.4103}}@{}}
\toprule\noalign{}
\begin{minipage}[b]{\linewidth}\raggedright
Parameter
\end{minipage} & \begin{minipage}[b]{\linewidth}\raggedright
Definition
\end{minipage} & \begin{minipage}[b]{\linewidth}\raggedright
Typical Values
\end{minipage} \\
\midrule\noalign{}
\endhead
\bottomrule\noalign{}
\endlastfoot
\textbf{Holding Current (IH)} & Minimum current to maintain conduction &
5-40 mA \\
\textbf{Latching Current (IL)} & Minimum current to establish conduction
& 10-100 mA \\
\end{longtable}
}

\begin{itemize}
\tightlist
\item
  \textbf{Latching current}: Must be exceeded briefly after triggering
  for SCR to latch
\item
  \textbf{Holding current}: Must be maintained to keep SCR in conduction
\item
  \textbf{Relationship}: Usually IL \textgreater{} IH
\item
  \textbf{Significance}: Critical for reliable switching operation
\end{itemize}

\end{solutionbox}
\begin{mnemonicbox}
``Latch with more, Hold with less, both keep SCR
conducting''

\end{mnemonicbox}
\subsection*{Question 4(c OR) [7
marks]}\label{question-4c-or-7-marks}

\textbf{Draw and explain in detail block diagram of operational
amplifier.}

\begin{solutionbox}

\textbf{Operational Amplifier Block Diagram:}

\includegraphics[width=1\linewidth,height=\textheight,keepaspectratio]{mermaid-52bb28cf.pdf}

\textbf{Op-Amp Blocks and Functions:}

\begin{itemize}
\tightlist
\item
  \textbf{Input differential stage}:

  \begin{itemize}
  \tightlist
  \item
    High input impedance
  \item
    Rejects common-mode signals
  \item
    Provides differential voltage gain
  \end{itemize}
\item
  \textbf{Intermediate stage}:

  \begin{itemize}
  \tightlist
  \item
    Additional voltage gain
  \item
    Level shifting
  \item
    Frequency compensation
  \end{itemize}
\item
  \textbf{Output stage}:

  \begin{itemize}
  \tightlist
  \item
    Low output impedance
  \item
    Current amplification
  \item
    Power capability for driving loads
  \end{itemize}
\item
  \textbf{Bias circuit}:

  \begin{itemize}
  \tightlist
  \item
    Establishes proper operating points
  \item
    Temperature stability
  \end{itemize}
\item
  \textbf{Frequency compensation}:

  \begin{itemize}
  \tightlist
  \item
    Prevents oscillation
  \item
    Controls frequency response
  \end{itemize}
\end{itemize}

\end{solutionbox}
\begin{mnemonicbox}
``Differential Input, Gain in Middle, Power at
Output''

\end{mnemonicbox}
\subsection*{Question 5(a) [3 marks]}\label{q5a}

\textbf{Draw and explain in brief inverting amplifier.}

\begin{solutionbox}

\textbf{Inverting Amplifier Circuit:}

\begin{lstlisting}
          Rf
          ___
    Vin---| |-----+
          ---     |
                  |
                 _|_
    +------+    /   \
    |      |---+     +---Vout
    |      |    \___/
Vin-+      |      |
    |Op-Amp|      |
    +------+      |
                  |
                 ---
                 ///
\end{lstlisting}

\begin{itemize}
\tightlist
\item
  \textbf{Gain formula}: Vout = -(Rf/Rin) \times Vin
\item
  \textbf{Operation}: Input signal inverted with amplification
\item
  \textbf{Virtual ground}: Inverting input maintained at 0V
\end{itemize}

\end{solutionbox}
\begin{mnemonicbox}
``Inverting means Negative Gain equals -Rf/Rin''

\end{mnemonicbox}
\subsection*{Question 5(b) [4 marks]}\label{q5b}

\textbf{Draw and explain the block diagram of regulated power supply.}

\begin{solutionbox}

\textbf{Regulated Power Supply Block Diagram:}

\includegraphics[width=1\linewidth,height=\textheight,keepaspectratio]{mermaid-73b46e2f.pdf}

\textbf{Regulated Power Supply Stages:}

\begin{itemize}
\tightlist
\item
  \textbf{Transformer}: Steps down AC voltage to required level
\item
  \textbf{Rectifier}: Converts AC to pulsating DC (diode bridge)
\item
  \textbf{Filter}: Smooths pulsating DC (capacitors)
\item
  \textbf{Regulator}: Maintains constant output despite variations
\item
  \textbf{Reference}: Provides stable comparison voltage
\item
  \textbf{Feedback}: Monitors output and adjusts regulation
\end{itemize}

\end{solutionbox}
\begin{mnemonicbox}
``Transform, Rectify, Filter, Regulate for Stable
DC''

\end{mnemonicbox}
\subsection*{Question 5(c) [7 marks]}\label{q5c}

\textbf{Draw and explain astable multivibrator.}

\begin{solutionbox}

\textbf{Astable Multivibrator Using 555 Timer:}

\includegraphics[width=1\linewidth,height=\textheight,keepaspectratio]{mermaid-45ce866a.pdf}

\textbf{Operation of Astable Multivibrator:}

\begin{itemize}
\item
  \textbf{Configuration}: Free-running oscillator with no stable states
\item
  \textbf{Timing components}: External R1, R2, and C
\item
  \textbf{Oscillation process}:

  \begin{itemize}
  \tightlist
  \item
    Capacitor charges through R1+R2
  \item
    Capacitor discharges through R2
  \item
    Continuous charging/discharging cycle
  \end{itemize}
\item
  \textbf{Output waveform}: Rectangular with duty cycle based on R1/R2
  ratio
\item
  \textbf{Frequency formula}: f = 1.44/((R1+2R2)\timesC)
\item
  \textbf{Applications}: Clock generation, LED flashers, tone generators
\item
  \textbf{Advantages}: Simple design, stable frequency, adjustable duty
  cycle
\end{itemize}

\end{solutionbox}
\begin{mnemonicbox}
``Always Switching, Time set by RC, Both states Least
stable''

\end{mnemonicbox}
\subsection*{Question 5(a OR) [3
marks]}\label{question-5a-or-3-marks}

\textbf{In an op amp non-inverting amplifier R1=2kΩ and Rf=200kΩ. Find
the voltage gain of non-inverting amplifier.}

\begin{solutionbox}

\textbf{Solution:}

\begin{itemize}
\tightlist
\item
  \textbf{Given}: R1 = 2kΩ, Rf = 200kΩ
\item
  \textbf{Non-inverting amplifier gain formula}: A = 1 + (Rf/R1)
\item
  \textbf{Calculation}:

  \begin{itemize}
  \tightlist
  \item
    A = 1 + (200kΩ/2kΩ)
  \item
    A = 1 + 100
  \item
    A = 101
  \end{itemize}
\item
  \textbf{Result}: Voltage gain of non-inverting amplifier is 101
\item
  \textbf{Significance}: Output voltage will be 101 times the input
  voltage
\end{itemize}

\end{solutionbox}
\begin{mnemonicbox}
``Non-inverting amplifier gain: One plus Feedback
over Ground''

\end{mnemonicbox}
\subsection*{Question 5(b OR) [4
marks]}\label{question-5b-or-4-marks}

\textbf{Draw and explain in brief circuit to get -5V regulated dc output
voltage.}

\begin{solutionbox}

\textbf{Negative Voltage Regulator Circuit:}

\begin{lstlisting}
     +--------+
     |        |
Vin--+        +---Vout (-5V)
     | 7905   |
     |        |
     +--------+
         |
        ---
        ///
\end{lstlisting}

\textbf{Circuit Operation:}

\begin{itemize}
\tightlist
\item
  \textbf{Key component}: 7905 negative voltage regulator IC
\item
  \textbf{Input requirement}: Negative DC voltage (typically -7V to
  -25V)
\item
  \textbf{Filtering}: Input and output capacitors for stability
\item
  \textbf{Regulation method}: Series pass element with feedback control
\item
  \textbf{Output characteristics}: Fixed -5V with current up to 1A
\end{itemize}

\end{solutionbox}
\begin{mnemonicbox}
``79XX for Negative, 78XX for Positive regulated
voltage''

\end{mnemonicbox}
\subsection*{Question 5(c OR) [7
marks]}\label{question-5c-or-7-marks}

\textbf{Draw and explain the block diagram of SMPS.}

\begin{solutionbox}

\textbf{SMPS Block Diagram:}

\includegraphics[width=1\linewidth,height=\textheight,keepaspectratio]{mermaid-e964b7a6.pdf}

\textbf{SMPS Operation:}

\begin{itemize}
\tightlist
\item
  \textbf{Input stage}: Filters EMI, rectifies AC to high-voltage DC
\item
  \textbf{Switching stage}: Converts DC to high-frequency AC (20-100
  kHz)
\item
  \textbf{Transformer}: Provides isolation and voltage transformation
\item
  \textbf{Output stage}: Rectifies and filters to produce clean DC
\item
  \textbf{Feedback control}: Regulates output by adjusting switching
  duty cycle
\end{itemize}

\textbf{Advantages of SMPS:}

\begin{itemize}
\tightlist
\item
  \textbf{High efficiency} (80-90\%) due to switching operation
\item
  \textbf{Small size and weight} from high-frequency transformer
\item
  \textbf{Wide input voltage range} with stable output
\item
  \textbf{Multiple output voltages} possible from single transformer
\end{itemize}

\textbf{Applications:}

\begin{itemize}
\tightlist
\item
  Computer power supplies
\item
  Electronic device chargers
\item
  Industrial power systems
\end{itemize}

\end{solutionbox}
\begin{mnemonicbox}
``Switch More Power Smartly: High frequency saves
size and energy''

\end{mnemonicbox}

\end{document}
