\documentclass[10pt,a4paper]{article}

% content/resources/templates/preamble.tex
\usepackage[margin=0.6in]{geometry}
\author{Milav Dabgar}
\usepackage{amsmath,amssymb,amsthm}
\usepackage{booktabs}
\usepackage{multirow}
\usepackage{xcolor}
\usepackage{tcolorbox}
\tcbuselibrary{breakable,skins}
\usepackage[colorlinks=true,linkcolor=blue]{hyperref}
\usepackage{titlesec}
\usepackage{enumitem}
\usepackage{tikz}
\usepackage{pgfplots}
\usepackage{circuitikz}
\usepackage[version=4]{mhchem}
\usepackage{longtable}
\usepackage{array}
\usepackage{float}
\usepackage{caption}
\usepackage{listings}

\lstset{
  basicstyle=\small\ttfamily,
  breaklines=true,
  breakatwhitespace=false,
  postbreak=\mbox{\textcolor{red}{$\hookrightarrow$}\space},
  float=false,
  numbers=left,
  numberstyle=\tiny\color{gray},
  numbersep=10pt,
  xleftmargin=2em,
  keywordstyle=\color{blue},
  commentstyle=\color{green!60!black},
  stringstyle=\color{purple},
  backgroundcolor=\color{gray!5},
  showstringspaces=false,
  tabsize=2,
  captionpos=b,
  keepspaces=true,
  columns=flexible
}

\pgfplotsset{compat=1.18}
\usetikzlibrary{shapes,arrows,positioning,calc,patterns,decorations.pathmorphing,decorations.markings,arrows.meta}

% Color scheme
\definecolor{headcolor}{RGB}{0,102,204}
\definecolor{keycolor}{RGB}{220,20,60}
\definecolor{solutioncolor}{RGB}{34,139,34}
\definecolor{mnemoniccolor}{RGB}{148,0,211}
\definecolor{codecolor}{RGB}{0,0,100}

% Spacing
\setlength{\parskip}{3pt}
\setlist[itemize]{nosep}
\setlist[enumerate]{nosep}

% Title formatting
\titleformat{\section}{\Large\bfseries\color{headcolor}}{\thesection}{1em}{}
\titleformat{\subsection}{\large\bfseries\color{headcolor}}{\thesubsection}{1em}{}

% Pandoc tightlist compatibility
\providecommand{\tightlist}{%
  \setlength{\itemsep}{0pt}\setlength{\parskip}{0pt}}

% Pandoc longtable compatibility
\newcounter{none}
\def\thenone{}


% content/resources/templates/english-boxes.tex
% This file is currently empty - it exists to maintain consistency with the import structure.
% Add custom environments here if needed in the future.


\begin{document}

\begin{center}
{\Huge\bfseries\color{headcolor} Subject Name Solutions}\\[5pt]
{\LARGE 1323202 -- Winter 2024}\\[3pt]
{\large Semester 1 Study Material}\\[3pt]
{\normalsize\textit{Detailed Solutions and Explanations}}
\end{center}

\vspace{10pt}

\subsection*{Question 1(a) [3 marks]}\label{q1a}

\textbf{Explain thermal runaway in detail.}

\begin{solutionbox}
Thermal runaway is a destructive process where a
transistor gets increasingly hotter until it fails.

\textbf{Diagram:}

\includegraphics[width=1\linewidth,height=\textheight,keepaspectratio]{mermaid-9281eee9.pdf}

\begin{itemize}
\tightlist
\item
  \textbf{Cause}: Increased temperature decreases base-emitter voltage
\item
  \textbf{Effect}: Collector current increases with temperature
\item
  \textbf{Result}: Self-reinforcing cycle of heating leads to
  destruction
\end{itemize}

\end{solutionbox}
\begin{mnemonicbox}
``Heat Rises, Current Climbs, Transistor Dies''

\end{mnemonicbox}
\subsection*{Question 1(b) [4 marks]}\label{q1b}

\textbf{Draw and explain fixed bias method.}

\begin{solutionbox}
Fixed bias uses a single resistor from base to voltage
supply for biasing.

\textbf{Circuit Diagram:}

\includegraphics[width=1\linewidth,height=\textheight,keepaspectratio]{mermaid-e4542a13.pdf}

\begin{itemize}
\tightlist
\item
  \textbf{Working}: Base current (IB) = (VCC - VBE)/RB
\item
  \textbf{Characteristics}: Simple circuit but poor stability
\item
  \textbf{Disadvantage}: Highly sensitive to temperature variations
\item
  \textbf{Application}: Used in small signal circuits where stability
  isn't critical
\end{itemize}

\end{solutionbox}
\begin{mnemonicbox}
``Fixed Bias: One Resistor, Poor Stability''

\end{mnemonicbox}
\subsection*{Question 1(c) [7 marks]}\label{q1c}

\textbf{List the biasing methods. Draw the circuit of voltage divider
type bias method and explain it.}

\begin{solutionbox}
The biasing methods for transistors include several
techniques for establishing proper operating points.


{\def\LTcaptype{none} % do not increment counter
\vspace{-5pt}
\captionof{table}{Transistor Biasing Methods}
\vspace{-10pt}
\begin{longtable}[]{@{}llll@{}}
\toprule\noalign{}
Method & Stability & Complexity & Temperature Sensitivity \\
\midrule\noalign{}
\endhead
\bottomrule\noalign{}
\endlastfoot
Fixed Bias & Poor & Simple & High \\
Collector-to-Base Bias & Medium & Medium & Medium \\
Voltage Divider Bias & Excellent & Complex & Low \\
Emitter Bias & Good & Medium & Low \\
\end{longtable}
}

\textbf{Circuit Diagram:}

\includegraphics[width=1\linewidth,height=\textheight,keepaspectratio]{mermaid-e09e0486.pdf}

\begin{itemize}
\tightlist
\item
  \textbf{Working}: R1-R2 divider creates stable base voltage
\item
  \textbf{Advantage}: Less affected by β variations and temperature
\item
  \textbf{Key feature}: RE provides negative feedback stabilization
\item
  \textbf{Application}: Most widely used in amplifier circuits
\end{itemize}

\end{solutionbox}
\begin{mnemonicbox}
``Divide and Rule for Stable Bias''

\end{mnemonicbox}
\subsection*{Question 1(c OR) [7
marks]}\label{question-1c-or-7-marks}

\textbf{Draw and explain DC load line for common emitter amplifier.}

\begin{solutionbox}
DC load line represents all possible operating points
of a transistor.

\textbf{Graph:}

\includegraphics[width=1\linewidth,height=\textheight,keepaspectratio]{mermaid-4887bcd1.pdf}

\textbf{Equation Table:}

{\def\LTcaptype{none} % do not increment counter
\begin{longtable}[]{@{}
  >{\raggedright\arraybackslash}p{(\linewidth - 4\tabcolsep) * \real{0.3235}}
  >{\raggedright\arraybackslash}p{(\linewidth - 4\tabcolsep) * \real{0.2941}}
  >{\raggedright\arraybackslash}p{(\linewidth - 4\tabcolsep) * \real{0.3824}}@{}}
\toprule\noalign{}
\begin{minipage}[b]{\linewidth}\raggedright
Parameter
\end{minipage} & \begin{minipage}[b]{\linewidth}\raggedright
Equation
\end{minipage} & \begin{minipage}[b]{\linewidth}\raggedright
Description
\end{minipage} \\
\midrule\noalign{}
\endhead
\bottomrule\noalign{}
\endlastfoot
Maximum VCE & VCC & When IC = 0 \\
Maximum IC & VCC/RC & When VCE = 0 \\
Load Line Equation & IC = (VCC - VCE)/RC & All possible operating
points \\
Q-point & Set by biasing & Stable operation point \\
\end{longtable}
}

\begin{itemize}
\tightlist
\item
  \textbf{Purpose}: Graphically shows relationship between IC and VCE
\item
  \textbf{Significance}: Helps determine operating point (Q-point)
\item
  \textbf{Application}: Essential for amplifier design and analysis
\end{itemize}

\end{solutionbox}
\begin{mnemonicbox}
``Maximum Current or Maximum Voltage, Never Both''

\end{mnemonicbox}
\subsection*{Question 2(a) [3 marks]}\label{q2a}

\textbf{Explain term (i) Gain (ii) Bandwidth.}

\begin{solutionbox}
These are key parameters that describe amplifier
performance.


{\def\LTcaptype{none} % do not increment counter
\vspace{-5pt}
\captionof{table}{Amplifier Parameters}
\vspace{-10pt}
\begin{longtable}[]{@{}
  >{\raggedright\arraybackslash}p{(\linewidth - 6\tabcolsep) * \real{0.2619}}
  >{\raggedright\arraybackslash}p{(\linewidth - 6\tabcolsep) * \real{0.2857}}
  >{\raggedright\arraybackslash}p{(\linewidth - 6\tabcolsep) * \real{0.1429}}
  >{\raggedright\arraybackslash}p{(\linewidth - 6\tabcolsep) * \real{0.3095}}@{}}
\toprule\noalign{}
\begin{minipage}[b]{\linewidth}\raggedright
Parameter
\end{minipage} & \begin{minipage}[b]{\linewidth}\raggedright
Definition
\end{minipage} & \begin{minipage}[b]{\linewidth}\raggedright
Unit
\end{minipage} & \begin{minipage}[b]{\linewidth}\raggedright
Significance
\end{minipage} \\
\midrule\noalign{}
\endhead
\bottomrule\noalign{}
\endlastfoot
Gain & Ratio of output to input signal & dB & Amplification power \\
Bandwidth & Range of frequencies with gain not less than 70.7\% of
maximum & Hz & Useful frequency range \\
\end{longtable}
}

\begin{itemize}
\tightlist
\item
  \textbf{Gain Types}: Voltage gain (Av), Current gain (Ai), Power gain
  (Ap)
\item
  \textbf{Bandwidth Formula}: BW = fH - fL (Higher cutoff - Lower
  cutoff)
\item
  \textbf{Related Parameter}: Gain-Bandwidth Product (constant for a
  specific amplifier)
\end{itemize}

\end{solutionbox}
\begin{mnemonicbox}
``Gain Makes Bigger, Bandwidth Makes Broader''

\end{mnemonicbox}
\subsection*{Question 2(b) [4 marks]}\label{q2b}

\textbf{List advantages and disadvantages of negative feedback in
amplifier.}

\begin{solutionbox}
Negative feedback significantly improves amplifier
performance but with tradeoffs.


{\def\LTcaptype{none} % do not increment counter
\vspace{-5pt}
\captionof{table}{Negative Feedback Characteristics}
\vspace{-10pt}
\begin{longtable}[]{@{}ll@{}}
\toprule\noalign{}
Advantages & Disadvantages \\
\midrule\noalign{}
\endhead
\bottomrule\noalign{}
\endlastfoot
Increased bandwidth & Reduced gain \\
Reduced distortion & More input signal required \\
Improved stability & More complex circuit \\
Better noise immunity & Potential oscillation if improperly designed \\
Controlled input/output impedances & Higher power consumption \\
\end{longtable}
}

\end{solutionbox}
\begin{mnemonicbox}
``Stabilize Wide And Clean, Just Give Up Gain''

\end{mnemonicbox}
\subsection*{Question 2(c) [7 marks]}\label{q2c}

\textbf{Draw and explain Hartley oscillator.}

\begin{solutionbox}
Hartley oscillator generates sine waves using inductive
feedback.

\textbf{Circuit Diagram:}

\includegraphics[width=1\linewidth,height=\textheight,keepaspectratio]{mermaid-79f81173.pdf}

\begin{itemize}
\tightlist
\item
  \textbf{Frequency Determination}: By L1, L2 and C1 values (f = 1/2π\sqrt(L
  \times C))
\item
  \textbf{Feedback Mechanism}: Inductive voltage divider (L1 and L2)
\item
  \textbf{Identifying Feature}: Tapped inductor or two inductors in
  series
\item
  \textbf{Applications}: RF signal generation, radio transmitters,
  communication systems
\end{itemize}

\end{solutionbox}
\begin{mnemonicbox}
``Hartley Has Helpful Inductors''

\end{mnemonicbox}
\subsection*{Question 2(a OR) [3
marks]}\label{question-2a-or-3-marks}

\textbf{State and explain Barkhausen criterion of oscillation.}

\begin{solutionbox}
Barkhausen criteria define conditions for sustained
oscillations.

\textbf{The Two Main Criteria:}

\includegraphics[width=1\linewidth,height=\textheight,keepaspectratio]{mermaid-a536ec72.pdf}

\begin{itemize}
\tightlist
\item
  \textbf{Loop Gain Condition}: \textbar Aβ\textbar{} = 1 (exactly 1 for
  sustained oscillation)
\item
  \textbf{Phase Shift Condition}: ∠Aβ = 0^\circ or 360^\circ (signal
  reinforcement)
\item
  \textbf{Practical Design}: Initial \textbar Aβ\textbar{}
  \textgreater{} 1, eventually stabilizes at \textbar Aβ\textbar{} = 1
\end{itemize}

\end{solutionbox}
\begin{mnemonicbox}
``For Oscillation: Unit Gain, Zero Phase''

\end{mnemonicbox}
\subsection*{Question 2(b OR) [4
marks]}\label{question-2b-or-4-marks}

\textbf{Compare negative and positive feedback amplifier.}

\begin{solutionbox}
Feedback type dramatically changes amplifier behavior.

\textbf{Comparison Table:}

{\def\LTcaptype{none} % do not increment counter
\begin{longtable}[]{@{}lll@{}}
\toprule\noalign{}
Parameter & Negative Feedback & Positive Feedback \\
\midrule\noalign{}
\endhead
\bottomrule\noalign{}
\endlastfoot
Gain & Decreases & Increases \\
Bandwidth & Increases & Decreases \\
Distortion & Reduces & Increases \\
Stability & Improves & Reduced (may oscillate) \\
Noise & Reduces & Amplifies \\
Applications & Stable amplifiers & Oscillators, triggers \\
Input/Output impedance & Controllable & Less predictable \\
\end{longtable}
}

\end{solutionbox}
\begin{mnemonicbox}
``Negative Stabilizes, Positive Oscillates''

\end{mnemonicbox}
\subsection*{Question 2(c OR) [7
marks]}\label{question-2c-or-7-marks}

\textbf{Draw and explain colpitt's oscillator.}

\begin{solutionbox}
Colpitt's oscillator uses capacitive voltage divider
for feedback.

\textbf{Circuit Diagram:}

\includegraphics[width=1\linewidth,height=\textheight,keepaspectratio]{mermaid-9e53aea5.pdf}

\begin{itemize}
\tightlist
\item
  \textbf{Frequency Determination}: By L, C1 and C2 values (f = 1/2π\sqrt(L
  \times Ceq))
\item
  \textbf{Feedback Mechanism}: Capacitive voltage divider (C1 and C2)
\item
  \textbf{Identifying Feature}: Two capacitors in series across inductor
\item
  \textbf{Advantage}: More stable frequency than Hartley
\end{itemize}

\end{solutionbox}
\begin{mnemonicbox}
``Colpitts Catches Capacitive Current''

\end{mnemonicbox}
\subsection*{Question 3(a) [3 marks]}\label{q3a}

\textbf{Explain about DIAC.}

\begin{solutionbox}
DIAC (Diode for Alternating Current) is a bidirectional
trigger diode.

\textbf{Symbol and Structure:}

\begin{lstlisting}
    A       K
    |       |
    +-------+
    |       |
    +-------+
    |       |
    K       A
\end{lstlisting}

\begin{itemize}
\tightlist
\item
  \textbf{Operation}: Conducts in both directions after breakdown
  voltage
\item
  \textbf{Characteristic}: Symmetrical V-I curve in both directions
\item
  \textbf{Key Parameter}: Breakover voltage (typically 30-40V)
\item
  \textbf{Main Application}: Triggering TRIACs in AC power control
\end{itemize}

\end{solutionbox}
\begin{mnemonicbox}
``DIAC: Double Direction Breakdown Device''

\end{mnemonicbox}
\subsection*{Question 3(b) [4 marks]}\label{q3b}

\textbf{Explain triggering methods of SCR.}

\begin{solutionbox}
SCR can be triggered to conduct by several methods.


{\def\LTcaptype{none} % do not increment counter
\vspace{-5pt}
\captionof{table}{SCR Triggering Methods}
\vspace{-10pt}
\begin{longtable}[]{@{}
  >{\raggedright\arraybackslash}p{(\linewidth - 6\tabcolsep) * \real{0.1739}}
  >{\raggedright\arraybackslash}p{(\linewidth - 6\tabcolsep) * \real{0.2826}}
  >{\raggedright\arraybackslash}p{(\linewidth - 6\tabcolsep) * \real{0.2609}}
  >{\raggedright\arraybackslash}p{(\linewidth - 6\tabcolsep) * \real{0.2826}}@{}}
\toprule\noalign{}
\begin{minipage}[b]{\linewidth}\raggedright
Method
\end{minipage} & \begin{minipage}[b]{\linewidth}\raggedright
Description
\end{minipage} & \begin{minipage}[b]{\linewidth}\raggedright
Advantages
\end{minipage} & \begin{minipage}[b]{\linewidth}\raggedright
Limitations
\end{minipage} \\
\midrule\noalign{}
\endhead
\bottomrule\noalign{}
\endlastfoot
Gate Triggering & Current pulse at gate & Most common, controllable &
Requires control circuit \\
Temperature & High temperature & No external circuit & Uncontrolled,
unreliable \\
Voltage & Exceeding breakover voltage & No external circuit & Stresses
device, uncontrolled \\
dv/dt & Rapid voltage rise & Simple & Can cause unwanted triggering \\
Light & Photons hitting junction & Electrical isolation & Requires
special packaging \\
\end{longtable}
}

\end{solutionbox}
\begin{mnemonicbox}
``Gate Voltage Temperature Rate Light''

\end{mnemonicbox}
\subsection*{Question 3(c) [7 marks]}\label{q3c}

\textbf{Draw symbol and construction of SCR. Also draw and explain V-I
characteristic of SCR.}

\begin{solutionbox}
SCR (Silicon Controlled Rectifier) is a four-layer PNPN
semiconductor device with three terminals.

\textbf{Symbol:}

\begin{lstlisting}
      A (Anode)
      |
      |
      v
    -----
    |   |
G -->|   |
    |   |
    -----
      ^
      |
      |
      K (Cathode)
\end{lstlisting}

\textbf{Construction:}

\includegraphics[width=1\linewidth,height=\textheight,keepaspectratio]{mermaid-ef41d858.pdf}

\textbf{V-I Characteristic:}

\includegraphics[width=1\linewidth,height=\textheight,keepaspectratio]{mermaid-9283f8a0.pdf}

\begin{itemize}
\tightlist
\item
  \textbf{Forward Blocking}: Low current until triggering
\item
  \textbf{Forward Conduction}: High current after triggering (latched)
\item
  \textbf{Holding Current}: Minimum current to maintain conduction
\item
  \textbf{Latching Current}: Minimum current to start latching
\item
  \textbf{Reverse Blocking}: Blocks current in reverse direction
\end{itemize}

\end{solutionbox}
\begin{mnemonicbox}
``Trigger Once, Conducts Forever, Until Current
Falls''

\end{mnemonicbox}
\subsection*{Question 3(a OR) [3
marks]}\label{question-3a-or-3-marks}

\textbf{Explain about natural commutation technique of SCR.}

\begin{solutionbox}
Natural commutation turns off SCR without external
circuit when AC current naturally reaches zero.

\textbf{Process Diagram:}

\includegraphics[width=1\linewidth,height=\textheight,keepaspectratio]{mermaid-96078e09.pdf}

\begin{itemize}
\tightlist
\item
  \textbf{Principle}: Uses natural zero-crossing of AC supply
\item
  \textbf{Advantage}: No additional commutation circuit required
\item
  \textbf{Application}: AC power control circuits, light dimmers
\item
  \textbf{Limitation}: Only works with AC supplies, not DC
\end{itemize}

\end{solutionbox}
\begin{mnemonicbox}
``Natural Commutation: Zero Current, Zero Effort''

\end{mnemonicbox}
\subsection*{Question 3(b OR) [4
marks]}\label{question-3b-or-4-marks}

\textbf{Explain about Opto-couplers.}

\begin{solutionbox}
Opto-couplers provide electrical isolation using light
transmission.

\textbf{Structure:}

\begin{lstlisting}
  .---------.
  |  LED    |\\
  |         | \\
  '---------'  \\
                >
  .---------.  //
  |PhotoDet | //
  |         |//
  '---------'
\end{lstlisting}


{\def\LTcaptype{none} % do not increment counter
\vspace{-5pt}
\captionof{table}{Opto-coupler Types}
\vspace{-10pt}
\begin{longtable}[]{@{}lllll@{}}
\toprule\noalign{}
Type & Photodetector & Speed & CTR & Applications \\
\midrule\noalign{}
\endhead
\bottomrule\noalign{}
\endlastfoot
Standard & Phototransistor & Medium & 20-100\% & General isolation \\
High-speed & Photodiode & Fast & 10-50\% & Digital communication \\
TRIAC & Photo-TRIAC & Slow & N/A & AC power control \\
Linear & Photodarlington & Slow & 100-1000\% & Analog signals \\
\end{longtable}
}

\begin{itemize}
\tightlist
\item
  \textbf{CTR}: Current Transfer Ratio (output/input current)
\item
  \textbf{Key Feature}: Complete electrical isolation between circuits
\item
  \textbf{Benefits}: Noise immunity, voltage level shifting, safety
\end{itemize}

\end{solutionbox}
\begin{mnemonicbox}
``Light Leaps gaps Electrons Can't''

\end{mnemonicbox}
\subsection*{Question 3(c OR) [7
marks]}\label{question-3c-or-7-marks}

\textbf{Draw symbol and construction of TRIAC. Also draw and explain V-I
characteristic of TRIAC.}

\begin{solutionbox}
TRIAC (Triode for Alternating Current) is a
bidirectional three-terminal semiconductor device.

\textbf{Symbol:}

\begin{lstlisting}
    MT2
     |
     |
   -----
   |   |
G--|   |
   |   |
   -----
     |
     |
    MT1
\end{lstlisting}

\textbf{Construction:}

\includegraphics[width=1\linewidth,height=\textheight,keepaspectratio]{mermaid-40759ded.pdf}

\textbf{V-I Characteristic:}

\includegraphics[width=1\linewidth,height=\textheight,keepaspectratio]{mermaid-a21c200a.pdf}

\begin{itemize}
\tightlist
\item
  \textbf{Bidirectional}: Conducts in both directions after triggering
\item
  \textbf{Quadrant Operation}: Four triggering modes based on polarities
\item
  \textbf{Applications}: AC power control, light dimmers, motor control
\item
  \textbf{Advantage over SCR}: Controls both halves of AC cycle
\end{itemize}

\end{solutionbox}
\begin{mnemonicbox}
``TRIAC: Two-way Road In AC Circuits''

\end{mnemonicbox}
\subsection*{Question 4(a) [3 marks]}\label{q4a}

\textbf{State characteristics of ideal Op-Amp.}

\begin{solutionbox}
An ideal Op-Amp has perfect characteristics that real
Op-Amps approximate.


{\def\LTcaptype{none} % do not increment counter
\vspace{-5pt}
\captionof{table}{Ideal Op-Amp Characteristics}
\vspace{-10pt}
\begin{longtable}[]{@{}lll@{}}
\toprule\noalign{}
Parameter & Ideal Value & Meaning \\
\midrule\noalign{}
\endhead
\bottomrule\noalign{}
\endlastfoot
Open-loop gain & Infinite & Amplifies smallest input difference \\
Input impedance & Infinite & Draws no current from source \\
Output impedance & Zero & Can drive any load \\
Bandwidth & Infinite & Works at all frequencies \\
CMRR & Infinite & Rejects common-mode signals \\
Slew rate & Infinite & Instantaneous output change \\
Offset voltage & Zero & No output with zero input \\
\end{longtable}
}

\end{solutionbox}
\begin{mnemonicbox}
``Infinite Gain, Impedance, Bandwidth; Zero Offset,
Output Z''

\end{mnemonicbox}
\subsection*{Question 4(b) [4 marks]}\label{q4b}

\textbf{Draw and explain monostable multivibrator using 555 timer IC.}

\begin{solutionbox}
Monostable multivibrator produces single pulse of fixed
duration when triggered.

\textbf{Circuit:}

\includegraphics[width=1\linewidth,height=\textheight,keepaspectratio]{mermaid-1cd0adc4.pdf}

\begin{itemize}
\tightlist
\item
  \textbf{Operation}: Negative trigger produces output pulse with
  duration T = 1.1RC
\item
  \textbf{Stable State}: Output LOW until triggered
\item
  \textbf{Timing Control}: R and C values determine pulse width
\item
  \textbf{Retriggering}: Can be retriggered after timeout
\end{itemize}

\end{solutionbox}
\begin{mnemonicbox}
``One Shot Wonder: Trigger Once, Pulse Once''

\end{mnemonicbox}
\subsection*{Question 4(c) [7 marks]}\label{q4c}

\textbf{Draw and explain Inverting amplifier using IC 741. Also draw
input and output waveforms.}

\begin{solutionbox}
Inverting amplifier reverses polarity while amplifying
input signal.

\textbf{Circuit:}

\includegraphics[width=1\linewidth,height=\textheight,keepaspectratio]{mermaid-0744ffd2.pdf}

\textbf{Waveforms:}

\begin{lstlisting}
Input:     /-\      /-\
          /   \    /   \
     ____/     \__/     \____

Output:   \    /\    /
           \  /  \  /
     ______\/____\/________
            
            180^\circ phase shift
\end{lstlisting}

\begin{itemize}
\tightlist
\item
  \textbf{Gain Equation}: Av = -Rf/Rin (negative sign indicates
  inversion)
\item
  \textbf{Input Impedance}: Equal to Rin
\item
  \textbf{Virtual Ground}: Inverting input maintained near 0V
\item
  \textbf{Bandwidth}: Depends on gain (higher gain = lower bandwidth)
\item
  \textbf{Applications}: Signal conditioning, audio amplifiers
\end{itemize}

\end{solutionbox}
\begin{mnemonicbox}
``Flips and Multiplies by Rf/Rin''

\end{mnemonicbox}
\subsection*{Question 4(a OR) [3
marks]}\label{question-4a-or-3-marks}

\textbf{Draw symbol and pin diagram of IC 741.}

\begin{solutionbox}
The 741 is a popular general-purpose operational
amplifier.

\textbf{Symbol:}

\begin{lstlisting}
        |\ 
        | \
Input --|+ \
        |   \
        |    |---- Output
        |   /
Input --|− /
        | /
        |/
\end{lstlisting}

\textbf{8-Pin DIP Package:}

\begin{lstlisting}
       _______
      |       |
NC 1--|       |--8 Vcc+
      |       |
-IN 2--|  741  |--7 Output
      |       |
+IN 3--|       |--6 NC
      |       |
Vcc- 4--|_______|--5 Offset Null
\end{lstlisting}

\begin{itemize}
\tightlist
\item
  \textbf{Pin Functions}: Inverting input, non-inverting input, output,
  power supplies
\item
  \textbf{Optional Pins}: Offset null, no connection
\item
  \textbf{Power Supply}: Typically \pm15V or \pm12V dual supply
\end{itemize}

\end{solutionbox}
\begin{mnemonicbox}
``Never Invert Plus, Very Output Not Connected''

\end{mnemonicbox}
\subsection*{Question 4(b OR) [4
marks]}\label{question-4b-or-4-marks}

\textbf{Explain term (i) CMRR (II) Slew Rate.}

\begin{solutionbox}
These parameters define operational amplifier
performance limits.


{\def\LTcaptype{none} % do not increment counter
\vspace{-5pt}
\captionof{table}{Key Op-Amp Parameters}
\vspace{-10pt}
\begin{longtable}[]{@{}
  >{\raggedright\arraybackslash}p{(\linewidth - 6\tabcolsep) * \real{0.2157}}
  >{\raggedright\arraybackslash}p{(\linewidth - 6\tabcolsep) * \real{0.2353}}
  >{\raggedright\arraybackslash}p{(\linewidth - 6\tabcolsep) * \real{0.2941}}
  >{\raggedright\arraybackslash}p{(\linewidth - 6\tabcolsep) * \real{0.2549}}@{}}
\toprule\noalign{}
\begin{minipage}[b]{\linewidth}\raggedright
Parameter
\end{minipage} & \begin{minipage}[b]{\linewidth}\raggedright
Definition
\end{minipage} & \begin{minipage}[b]{\linewidth}\raggedright
Typical Value
\end{minipage} & \begin{minipage}[b]{\linewidth}\raggedright
Significance
\end{minipage} \\
\midrule\noalign{}
\endhead
\bottomrule\noalign{}
\endlastfoot
CMRR (Common Mode Rejection Ratio) & Ratio of differential gain to
common-mode gain & 90-120 dB & Higher is better \\
Slew Rate & Maximum rate of output voltage change & 0.5-50 V/μs & Higher
for faster signals \\
\end{longtable}
}

\begin{itemize}
\tightlist
\item
  \textbf{CMRR Formula}: CMRR = 20 log_{1}_{0}(Ad/Acm) dB
\item
  \textbf{CMRR Importance}: Rejects noise common to both inputs
\item
  \textbf{Slew Rate Formula}: SR = dVo/dt (max)
\item
  \textbf{Slew Rate Limitation}: Causes distortion at high frequencies
\end{itemize}

\end{solutionbox}
\begin{mnemonicbox}
``CMRR Crushes Common Noise, Slew Rate Shows Speed''

\end{mnemonicbox}
\subsection*{Question 4(c OR) [7
marks]}\label{question-4c-or-7-marks}

\textbf{Draw and explain Astable multivibrator using 555 timer IC.}

\begin{solutionbox}
Astable multivibrator generates continuous square waves
without external trigger.

\textbf{Circuit:}

\includegraphics[width=1\linewidth,height=\textheight,keepaspectratio]{mermaid-3bbea268.pdf}

\textbf{Output Waveform:}

\begin{lstlisting}
   HIGH  ____      ____      ____
        |    |    |    |    |    |
        |    |    |    |    |    |
   LOW  |____|    |____|    |____|
        
        | T1 | T2 | T1 | T2 | T1 |
\end{lstlisting}

\begin{itemize}
\tightlist
\item
  \textbf{Timing}: T1 = 0.693(RA+RB)C, T2 = 0.693(RB)C
\item
  \textbf{Frequency}: f = 1.44/((RA+2RB)C)
\item
  \textbf{Duty Cycle}: Can be adjusted by RA and RB
\item
  \textbf{Applications}: Clock generators, LED flashers, tone generators
\end{itemize}

\end{solutionbox}
\begin{mnemonicbox}
``Always Oscillating, Never Stopping''

\end{mnemonicbox}
\subsection*{Question 5(a) [3 marks]}\label{q5a}

\textbf{Draw basic block diagram of regulated power supply and explain
it.}

\begin{solutionbox}
A regulated power supply converts AC to stable DC
voltage.

\textbf{Block Diagram:}

\includegraphics[width=1\linewidth,height=\textheight,keepaspectratio]{mermaid-6d1d2420.pdf}

\begin{itemize}
\tightlist
\item
  \textbf{Transformer}: Steps down AC voltage to required level
\item
  \textbf{Rectifier}: Converts AC to pulsating DC (diode bridge)
\item
  \textbf{Filter}: Smooths pulsating DC (capacitors)
\item
  \textbf{Regulator}: Maintains constant output despite variations
\item
  \textbf{Output}: Stable DC voltage for electronic circuits
\end{itemize}

\end{solutionbox}
\begin{mnemonicbox}
``Transformer Rectifies Filters Regulates''

\end{mnemonicbox}
\subsection*{Question 5(b) [4 marks]}\label{q5b}

\textbf{Draw and explain summing amplifier using Op-amp.}

\begin{solutionbox}
Summing amplifier adds multiple input signals with
weighted proportions.

\textbf{Circuit:}

\includegraphics[width=1\linewidth,height=\textheight,keepaspectratio]{mermaid-728bdddf.pdf}

\begin{itemize}
\tightlist
\item
  \textbf{Output Equation}: Vout = -Rf(V1/R1 + V2/R2 + V3/R3)
\item
  \textbf{Special Case}: When all resistors equal, Vout = -Rf/R \times (V1 +
  V2 + V3)
\item
  \textbf{Applications}: Audio mixing, analog computers, signal
  averaging
\item
  \textbf{Variations}: Inverting and non-inverting configurations
  available
\end{itemize}

\end{solutionbox}
\begin{mnemonicbox}
``Multiple Inputs, One Output, Weighted Addition''

\end{mnemonicbox}
\subsection*{Question 5(c) [7 marks]}\label{q5c}

\textbf{Draw and explain the circuit diagram of 3 terminal voltage
regulator using IC LM317 with adjustable output voltage.}

\begin{solutionbox}
LM317 is a versatile adjustable voltage regulator with
output range of 1.25V to 37V.

\textbf{Circuit:}

\includegraphics[width=1\linewidth,height=\textheight,keepaspectratio]{mermaid-a5e4f199.pdf}

\begin{itemize}
\tightlist
\item
  \textbf{Output Voltage}: VOUT = 1.25V(1 + R2/R1)
\item
  \textbf{Fixed Components}: R1 = 240Ω, reference voltage = 1.25V
\item
  \textbf{Adjustability}: Changing R2 sets desired output voltage
\item
  \textbf{Protection Features}: Current limiting, thermal shutdown
\item
  \textbf{Applications}: Variable power supplies, battery chargers
\item
  \textbf{Advantages}: Few external components, robust protection
\end{itemize}

\end{solutionbox}
\begin{mnemonicbox}
``Adjust with R2, Reference Stays at 1.25''

\end{mnemonicbox}
\subsection*{Question 5(a OR) [3
marks]}\label{question-5a-or-3-marks}

\textbf{State full form of SMPS. Also state applications of SMPS.}

\begin{solutionbox}
SMPS stands for Switch Mode Power Supply, a modern
efficient power conversion technology.

\textbf{Applications Table:}

{\def\LTcaptype{none} % do not increment counter
\begin{longtable}[]{@{}lll@{}}
\toprule\noalign{}
Application & SMPS Type & Advantages \\
\midrule\noalign{}
\endhead
\bottomrule\noalign{}
\endlastfoot
Computer Power Supply & ATX & High efficiency, multiple outputs \\
Mobile Phone Chargers & Flyback & Compact size, lightweight \\
LED Drivers & Buck & Efficient dimming capability \\
TV Power Supply & Forward & Good regulation, multiple outputs \\
Industrial Controls & Push-Pull & High power capability \\
Battery Chargers & Boost & Adjustable charging profiles \\
\end{longtable}
}

\begin{itemize}
\tightlist
\item
  \textbf{Key Benefits}: High efficiency (80-95\%), small size,
  lightweight
\item
  \textbf{Drawbacks}: EMI generation, more complex circuits
\end{itemize}

\end{solutionbox}
\begin{mnemonicbox}
``Switch Mode Powers Small devices''

\end{mnemonicbox}
\subsection*{Question 5(b OR) [4
marks]}\label{question-5b-or-4-marks}

\textbf{Draw and explain differentiator using Op-amp.}

\begin{solutionbox}
Differentiator produces output proportional to rate of
change of input.

\textbf{Circuit:}

\includegraphics[width=1\linewidth,height=\textheight,keepaspectratio]{mermaid-14457c1b.pdf}

\textbf{Input/Output Waveforms:}

\begin{lstlisting}
Input:      ___
           /   \
          /     \___
         /
________/

Output:   |
          |
     _____|_____
         / \
        /   \
\end{lstlisting}

\begin{itemize}
\tightlist
\item
  \textbf{Equation}: Vout = -RC \times d(Vin)/dt
\item
  \textbf{Function}: Converts square wave to spikes, triangle to square
\item
  \textbf{Practical Issue}: High noise sensitivity
\item
  \textbf{Modification}: Small resistor in series with C to limit
  high-frequency gain
\item
  \textbf{Applications}: Waveshaping, rate-of-change detection
\end{itemize}

\end{solutionbox}
\begin{mnemonicbox}
``Rate of Change Goes In, Amplitude Comes Out''

\end{mnemonicbox}
\subsection*{Question 5(c OR) [7
marks]}\label{question-5c-or-7-marks}

\textbf{Draw and explain the circuit diagram of -12 V regulated dc power
supply.}

\begin{solutionbox}
A -12V regulated supply provides stable negative
voltage for analog circuits.

\textbf{Circuit Diagram:}

\includegraphics[width=1\linewidth,height=\textheight,keepaspectratio]{mermaid-e0a18e6d.pdf}

\begin{itemize}
\tightlist
\item
  \textbf{Working Principle}: Full-wave rectifier creates negative
  voltage
\item
  \textbf{Components}: Transformer, bridge rectifier, filter capacitors,
  7912 regulator
\item
  \textbf{Regulator IC}: 7912 provides fixed -12V output with internal
  protection
\item
  \textbf{Filter Capacitors}: Input capacitor filters ripple, output
  capacitor improves transient response
\item
  \textbf{Applications}: Op-amp negative rail, analog circuits, audio
  equipment
\end{itemize}

\end{solutionbox}
\begin{mnemonicbox}
``Full Bridge, Big Capacitor, 7912 Regulates
Negative''

\end{mnemonicbox}

\end{document}
