\documentclass[10pt,a4paper]{article}

% content/resources/templates/preamble.tex
\usepackage[margin=0.6in]{geometry}
\author{Milav Dabgar}
\usepackage{amsmath,amssymb,amsthm}
\usepackage{booktabs}
\usepackage{multirow}
\usepackage{xcolor}
\usepackage{tcolorbox}
\tcbuselibrary{breakable,skins}
\usepackage[colorlinks=true,linkcolor=blue]{hyperref}
\usepackage{titlesec}
\usepackage{enumitem}
\usepackage{tikz}
\usepackage{pgfplots}
\usepackage{circuitikz}
\usepackage[version=4]{mhchem}
\usepackage{longtable}
\usepackage{array}
\usepackage{float}
\usepackage{caption}
\usepackage{listings}

\lstset{
  basicstyle=\small\ttfamily,
  breaklines=true,
  breakatwhitespace=false,
  postbreak=\mbox{\textcolor{red}{$\hookrightarrow$}\space},
  float=false,
  numbers=left,
  numberstyle=\tiny\color{gray},
  numbersep=10pt,
  xleftmargin=2em,
  keywordstyle=\color{blue},
  commentstyle=\color{green!60!black},
  stringstyle=\color{purple},
  backgroundcolor=\color{gray!5},
  showstringspaces=false,
  tabsize=2,
  captionpos=b,
  keepspaces=true,
  columns=flexible
}

\pgfplotsset{compat=1.18}
\usetikzlibrary{shapes,arrows,positioning,calc,patterns,decorations.pathmorphing,decorations.markings,arrows.meta}

% Color scheme
\definecolor{headcolor}{RGB}{0,102,204}
\definecolor{keycolor}{RGB}{220,20,60}
\definecolor{solutioncolor}{RGB}{34,139,34}
\definecolor{mnemoniccolor}{RGB}{148,0,211}
\definecolor{codecolor}{RGB}{0,0,100}

% Spacing
\setlength{\parskip}{3pt}
\setlist[itemize]{nosep}
\setlist[enumerate]{nosep}

% Title formatting
\titleformat{\section}{\Large\bfseries\color{headcolor}}{\thesection}{1em}{}
\titleformat{\subsection}{\large\bfseries\color{headcolor}}{\thesubsection}{1em}{}

% Pandoc tightlist compatibility
\providecommand{\tightlist}{%
  \setlength{\itemsep}{0pt}\setlength{\parskip}{0pt}}

% Pandoc longtable compatibility
\newcounter{none}
\def\thenone{}


% content/resources/templates/english-boxes.tex
% This file is currently empty - it exists to maintain consistency with the import structure.
% Add custom environments here if needed in the future.


\begin{document}

\begin{center}
{\Huge\bfseries\color{headcolor} Subject Name Solutions}\\[5pt]
{\LARGE 1323202 -- Winter 2023}\\[3pt]
{\large Semester 1 Study Material}\\[3pt]
{\normalsize\textit{Detailed Solutions and Explanations}}
\end{center}

\vspace{10pt}

\subsection*{Question 1(a) [3 marks]}\label{q1a}

\textbf{Explain the concept of dc load line with the help of neat
diagram.}

\begin{solutionbox}
DC load line is a straight line on output
characteristics that shows all possible operating points of a
transistor.

\textbf{Diagram:}

\includegraphics[width=1\linewidth,height=\textheight,keepaspectratio]{mermaid-95309ce0.pdf}

\begin{itemize}
\tightlist
\item
  \textbf{Collector saturation current}: When VCE = 0, IC = VCC/RC
\item
  \textbf{Cutoff voltage}: When IC = 0, VCE = VCC
\item
  \textbf{Q-point}: Operating point along load line
\end{itemize}

\end{solutionbox}
\begin{mnemonicbox}
``LEVEL'' - ``Load line Establishes Voltage and
current for Every Load condition''

\end{mnemonicbox}
\subsection*{Question 1(b) [4 marks]}\label{q1b}

\textbf{Explain thermal runaway in detail.}

\begin{solutionbox}
Thermal runaway is a condition where heat causes
transistor's collector current to increase, which generates more heat,
leading to destruction.

\textbf{Diagram:}

\includegraphics[width=1\linewidth,height=\textheight,keepaspectratio]{mermaid-2402ab51.pdf}

\begin{itemize}
\tightlist
\item
  \textbf{Heat generation}: Power dissipation = VCE \times IC
\item
  \textbf{Critical effect}: Increased junction temperature decreases VBE
\item
  \textbf{Prevention}: Heat sinks, thermal stabilization circuits,
  proper biasing
\item
  \textbf{Danger}: Can destroy transistor if not controlled
\end{itemize}

\end{solutionbox}
\begin{mnemonicbox}
``HEAT'' - ``Higher Emission Amplifies Temperature''

\end{mnemonicbox}
\subsection*{Question 1(c) [7 marks]}\label{q1c}

\textbf{Draw the circuit diagram and frequency response of a two stage
R-C coupled amplifier. Explain the importance of each component.}

\begin{solutionbox}
R-C coupled amplifier uses capacitors to connect
multiple transistor stages for higher gain.

\textbf{Diagram:}

\begin{lstlisting}
+------+            +------+
|      |            |      |
|  Q1  |            |  Q2  |
|      |            |      |
+------+            +------+
   |                   |
   |                   |
   R1       C2         R2
   |        ||         |
   +---||---+----------+
        C1             

Vin o----||---+        +------o Vout
             |         |
             R3        R4
             |         |
             +         +
\end{lstlisting}

\textbf{Frequency Response:}

\includegraphics[width=1\linewidth,height=\textheight,keepaspectratio]{mermaid-5e6313e4.pdf}

\begin{itemize}
\tightlist
\item
  \textbf{Coupling capacitors}: Block DC, allow AC signal transfer
  between stages
\item
  \textbf{Biasing resistors}: Establish proper Q-point for transistor
  operation
\item
  \textbf{Bypass capacitors}: Prevent gain reduction from negative
  feedback
\item
  \textbf{Bandwidth}: Range between low and high cutoff frequencies
\end{itemize}

\end{solutionbox}
\begin{mnemonicbox}
``CARS'' - ``Coupling capacitors Allow Resistance
Separation''

\end{mnemonicbox}
\subsection*{OR}\label{or}

\subsection*{Question 1(c) [7 marks]}\label{q1c}

\textbf{Compare negative and positive feedback in amplifier.}

\begin{solutionbox}
Feedback systems return a portion of output to the
input with different effects based on polarity.


{\def\LTcaptype{none} % do not increment counter
\begin{longtable}[]{@{}
  >{\raggedright\arraybackslash}p{(\linewidth - 4\tabcolsep) * \real{0.2245}}
  >{\raggedright\arraybackslash}p{(\linewidth - 4\tabcolsep) * \real{0.3878}}
  >{\raggedright\arraybackslash}p{(\linewidth - 4\tabcolsep) * \real{0.3878}}@{}}
\toprule\noalign{}
\begin{minipage}[b]{\linewidth}\raggedright
Parameter
\end{minipage} & \begin{minipage}[b]{\linewidth}\raggedright
Negative Feedback
\end{minipage} & \begin{minipage}[b]{\linewidth}\raggedright
Positive Feedback
\end{minipage} \\
\midrule\noalign{}
\endhead
\bottomrule\noalign{}
\endlastfoot
Gain & Decreases & Increases \\
Bandwidth & Increases & Decreases \\
Stability & Improves & Decreases \\
Distortion & Reduces & Increases \\
Noise & Reduces & Amplifies \\
Input/Output impedance & Can be controlled & Unpredictable \\
Applications & Amplifiers, regulators & Oscillators, Schmitt triggers \\
\end{longtable}
}

\begin{itemize}
\tightlist
\item
  \textbf{Negative feedback}: Output is out of phase with input (180^\circ
  shifted)
\item
  \textbf{Positive feedback}: Output is in phase with input (0^\circ shifted)
\item
  \textbf{Barkhausen criteria}: Positive feedback with unity gain
  creates oscillation
\end{itemize}

\end{solutionbox}
\begin{mnemonicbox}
``SIGN'' - ``Stability Increases with Gain Negation''

\end{mnemonicbox}
\subsection*{Question 2(a) [3 marks]}\label{q2a}

\textbf{State and explain Barkhausen's criteria for oscillations.}

\begin{solutionbox}
Barkhausen's criteria define conditions for sustained
oscillations in a feedback system.

\textbf{Diagram:}

\includegraphics[width=1\linewidth,height=\textheight,keepaspectratio]{mermaid-d10fa2d3.pdf}

\begin{itemize}
\tightlist
\item
  \textbf{Gain condition}: Loop gain (A\timesβ) must equal 1 (unity)
\item
  \textbf{Phase condition}: Total phase shift must be 0^\circ or 360^\circ
\item
  \textbf{Practical implementation}: Initial loop gain \textgreater{} 1,
  then stabilizes at 1
\end{itemize}

\end{solutionbox}
\begin{mnemonicbox}
``LOOP'' - ``Loop's Overall Output Phase''

\end{mnemonicbox}
\subsection*{Question 2(b) [4 marks]}\label{q2b}

\textbf{Compare Fixed bias, Collector to base bias \& Voltage divider
bias methods.}

\begin{solutionbox}
Different biasing techniques provide varying degrees of
stability and temperature compensation.


{\def\LTcaptype{none} % do not increment counter
\begin{longtable}[]{@{}llll@{}}
\toprule\noalign{}
Parameter & Fixed Bias & Collector-Base Bias & Voltage Divider Bias \\
\midrule\noalign{}
\endhead
\bottomrule\noalign{}
\endlastfoot
Stability & Poor & Better & Excellent \\
Circuit complexity & Simple & Medium & Complex \\
Temperature stability & Poor & Medium & Good \\
Components & 1 Resistor & 1 Resistor & 3-4 Resistors \\
Stability factor (S) & High & Medium & Low \\
\end{longtable}
}

\begin{itemize}
\tightlist
\item
  \textbf{Fixed bias}: Single resistor from base to VCC
\item
  \textbf{Collector-base bias}: Feedback resistor from collector to base
\item
  \textbf{Voltage divider}: Two resistors create stable reference
  voltage
\end{itemize}

\end{solutionbox}
\begin{mnemonicbox}
``STORM'' - ``Stability Through Optimized Resistor
Methods''

\end{mnemonicbox}
\subsection*{Question 2(c) [7 marks]}\label{q2c}

\textbf{Write short note on Hartley oscillator.}

\begin{solutionbox}
Hartley oscillator is an LC oscillator with a tapped
inductor for feedback.

\textbf{Diagram:}

\includegraphics[width=1\linewidth,height=\textheight,keepaspectratio]{mermaid-2b1fea2b.pdf}

\begin{itemize}
\tightlist
\item
  \textbf{Circuit components}: Amplifier, tapped inductor (L1+L2),
  capacitor C
\item
  \textbf{Frequency formula}: f = 1/[2π\sqrt(LC)] where L = L1+L2
\item
  \textbf{Advantages}: Simple design, good frequency stability
\item
  \textbf{Drawbacks}: Size of inductors, limited frequency range
\item
  \textbf{Applications}: RF signal generators, radio receivers,
  communication
\end{itemize}

\end{solutionbox}
\begin{mnemonicbox}
``TILC'' - ``Tapped Inductor with LC Circuit''

\end{mnemonicbox}
\subsection*{OR}\label{or-1}

\subsection*{Question 2(a) [3 marks]}\label{q2a}

\textbf{Explain working of transistor as a switch.}

\begin{solutionbox}
Transistor switches between cutoff (OFF) and saturation
(ON) regions for digital applications.

\textbf{Diagram:}

\includegraphics[width=1\linewidth,height=\textheight,keepaspectratio]{mermaid-c7cd92cf.pdf}

\begin{itemize}
\tightlist
\item
  \textbf{Cutoff region}: VBE \textless{} 0.7V, acts as open switch, VCE
  \approx VCC
\item
  \textbf{Saturation region}: VBE \textgreater{} 0.7V, acts as closed
  switch, VCE \approx 0.2V
\item
  \textbf{Switching time}: Limited by junction capacitance
\end{itemize}

\end{solutionbox}
\begin{mnemonicbox}
``COPS'' - ``Cutoff-On-Produces Switching''

\end{mnemonicbox}
\subsection*{Question 2(b) [4 marks]}\label{q2b}

\textbf{Define heat sink. List types of heat sink and give its
applications.}

\begin{solutionbox}
Heat sink is a thermal conductor that transfers heat
away from electronic components.

\textbf{Diagram:}

\begin{lstlisting}
     ||||||||
    /||||||||\ Heat Sink
   /||||||||||\
  /||||||||||||\
 /|||||||||||||\
+--------------+
|  Transistor  |
+--------------+
\end{lstlisting}

\textbf{Types of Heat Sinks:}

{\def\LTcaptype{none} % do not increment counter
\begin{longtable}[]{@{}lll@{}}
\toprule\noalign{}
Type & Description & Application \\
\midrule\noalign{}
\endhead
\bottomrule\noalign{}
\endlastfoot
Passive & No moving parts, natural convection & Low-power devices \\
Active & With fans or pumps & High-power amplifiers \\
Liquid-cooled & Uses fluid for heat transfer & Computing systems \\
Finned & Multiple fins increase surface area & Power transistors \\
\end{longtable}
}

\begin{itemize}
\tightlist
\item
  \textbf{Purpose}: Prevents thermal runaway and component failure
\item
  \textbf{Materials}: Aluminum, copper, or alloys with high thermal
  conductivity
\end{itemize}

\end{solutionbox}
\begin{mnemonicbox}
``COOL'' - ``Conducting Out Of Local heat''

\end{mnemonicbox}
\subsection*{Question 2(c) [7 marks]}\label{q2c}

\textbf{Explain advantages and disadvantages of negative feedback in
amplifiers in detail.}

\begin{solutionbox}
Negative feedback returns a portion of output signal to
input with opposite phase.


{\def\LTcaptype{none} % do not increment counter
\begin{longtable}[]{@{}
  >{\raggedright\arraybackslash}p{(\linewidth - 2\tabcolsep) * \real{0.4444}}
  >{\raggedright\arraybackslash}p{(\linewidth - 2\tabcolsep) * \real{0.5556}}@{}}
\toprule\noalign{}
\begin{minipage}[b]{\linewidth}\raggedright
Advantages
\end{minipage} & \begin{minipage}[b]{\linewidth}\raggedright
Disadvantages
\end{minipage} \\
\midrule\noalign{}
\endhead
\bottomrule\noalign{}
\endlastfoot
Stabilizes gain & Reduces overall gain \\
Increases bandwidth & More components needed \\
Reduces distortion & More power consumption \\
Decreases noise & Complex circuit design \\
Controls input/output impedance & Potential oscillation if improperly
designed \\
Improves linearity & Signal loss in feedback network \\
\end{longtable}
}

\textbf{Diagram:}

\includegraphics[width=1\linewidth,height=\textheight,keepaspectratio]{mermaid-a5589afc.pdf}

\begin{itemize}
\tightlist
\item
  \textbf{Gain stabilization}: Makes gain dependent on passive
  components
\item
  \textbf{Bandwidth extension}: Increases by factor equal to gain
  reduction
\item
  \textbf{Feedback factor}: β determines amount of improvement
\end{itemize}

\end{solutionbox}
\begin{mnemonicbox}
``STABLE'' - ``Stabilized Transmission And Bandwidth
with Less Error''

\end{mnemonicbox}
\subsection*{Question 3(a) [3 marks]}\label{q3a}

\textbf{Draw symbol of SCR and explain working of SCR.}

\begin{solutionbox}
Silicon Controlled Rectifier (SCR) is a four-layer PNPN
device with three terminals.

\textbf{Symbol:}

\begin{lstlisting}
      A(Anode)
       |
       |
       v
    +-----+
    |     |
G-->|     |
    |     |
    +-----+
       ^
       |
       |
      K(Cathode)
\end{lstlisting}

\begin{itemize}
\tightlist
\item
  \textbf{Structure}: P-N-P-N four-layer semiconductor device
\item
  \textbf{Operation}: Remains OFF until gate triggered, then conducts
  until current falls below holding value
\item
  \textbf{Terminals}: Anode, Cathode, Gate
\end{itemize}

\end{solutionbox}
\begin{mnemonicbox}
``AGK'' - ``Anode-Gate controls Kathode current''

\end{mnemonicbox}
\subsection*{Question 3(b) [4 marks]}\label{q3b}

\textbf{Explain two transistor analogy of SCR with circuit diagram.}

\begin{solutionbox}
SCR can be represented as interconnected PNP and NPN
transistors sharing junctions.

\textbf{Diagram:}

\begin{lstlisting}
       Anode
         |
    +----|----+
    |    v    |
    |  +--->  |
    |  | PNP  |
    |  +----+ |
    |       | |
Gate |       v |
 ----|---+  +-->
    |   |  | NPN
    |   +--+----+
    |          |
    +----------|--
              |
              v
            Cathode
\end{lstlisting}

\begin{itemize}
\tightlist
\item
  \textbf{PNP section}: Upper transistor with collector connected to NPN
  base
\item
  \textbf{NPN section}: Lower transistor with collector connected to PNP
  base
\item
  \textbf{Triggering}: Small gate current turns on NPN, which turns on
  PNP
\item
  \textbf{Regenerative action}: Each transistor supplies base current to
  other
\end{itemize}

\end{solutionbox}
\begin{mnemonicbox}
``PNPN'' - ``Positive-Negative-Positive-Negative
layers''

\end{mnemonicbox}
\subsection*{Question 3(c) [7 marks]}\label{q3c}

\textbf{Explain the working of TRIAC based fan regulator with circuit
diagram.}

\begin{solutionbox}
TRIAC-based fan regulator controls AC power through
phase control.

\textbf{Circuit Diagram:}

\begin{lstlisting}
        +---+  R1
AC o----+   +--/\/\--+
        |           |
        | C1        |
        +---||------+----+
                    |    |
                    Z   MT1
                    |    |
                   G|    |
                    |   _V_
                    +--|   |--+--o Fan
                       |___|  |
                        MT2   |
                              |
                              |
AC o--------------------------|
\end{lstlisting}

\begin{itemize}
\tightlist
\item
  \textbf{Phase control}: Varies firing angle of TRIAC to control power
\item
  \textbf{Diac}: Provides bidirectional triggering for TRIAC
\item
  \textbf{RC timing circuit}: R1 and C1 set phase delay
\item
  \textbf{Variable resistor}: Adjusts phase delay for speed control
\item
  \textbf{Protection}: RC snubber prevents false triggering
\end{itemize}

\end{solutionbox}
\begin{mnemonicbox}
``TRIAC'' - ``Triggered Response In AC Circuits''

\end{mnemonicbox}
\subsection*{OR}\label{or-2}

\subsection*{Question 3(a) [3 marks]}\label{q3a}

\textbf{Draw V-I characteristics of DIAC and TRIAC.}

\begin{solutionbox}
DIACs and TRIACs are bidirectional devices with
symmetrical characteristics.

\textbf{DIAC Characteristics:}

\includegraphics[width=1\linewidth,height=\textheight,keepaspectratio]{mermaid-2a410943.pdf}

\textbf{TRIAC Characteristics:}

\includegraphics[width=1\linewidth,height=\textheight,keepaspectratio]{mermaid-48853bf4.pdf}

\begin{itemize}
\tightlist
\item
  \textbf{DIAC}: Bidirectional diode that conducts after breakover
  voltage
\item
  \textbf{TRIAC}: Three-terminal device that conducts in both directions
  when triggered
\end{itemize}

\end{solutionbox}
\begin{mnemonicbox}
``BIBO'' - ``Bidirectional In, Bidirectional Out''

\end{mnemonicbox}
\subsection*{Question 3(b) [4 marks]}\label{q3b}

\textbf{Explain the Gate triggering method of SCR.}

\begin{solutionbox}
Gate triggering is the most common method to activate
an SCR.

\textbf{Diagram:}

\begin{lstlisting}
        A
        |
     +-----+
     |     |
     |     |
     |  +->|
RC --|--+  |
     |     |
     +-----+
        |
        K
\end{lstlisting}

\begin{itemize}
\tightlist
\item
  \textbf{Gate pulse}: Small current applied between gate and cathode
\item
  \textbf{Triggering methods}: DC, AC, or pulse signals
\item
  \textbf{Current requirements}: Typically 5-20mA gate current
\item
  \textbf{Advantages}: Low power control of high-power circuits
\end{itemize}

\end{solutionbox}
\begin{mnemonicbox}
``GATE'' - ``Gain Activation Through Electron flow''

\end{mnemonicbox}
\subsection*{Question 3(c) [7 marks]}\label{q3c}

\textbf{Explain SCR application for DC power control.}

\begin{solutionbox}
SCR controls DC power by chopping the supply voltage at
variable duty cycles.

\textbf{Circuit:}

\begin{lstlisting}
    +-------+       SCR
    |       |       / |
DC--|-------|------/--|--+---o Output
    |       |          |   |
    | PWM   |          |   |
    | Ctrl  |----.     |   |
    |       |    |     |   |
    +-------+    +-----|---+
                       |
                       |
    +------------------|
    |                  |
    GND----------------+
\end{lstlisting}

\begin{itemize}
\tightlist
\item
  \textbf{Phase control}: Varies firing angle to control average power
\item
  \textbf{PWM control}: Pulse width modulation for efficient control
\item
  \textbf{Applications}: DC motor speed control, dimming, heating
\item
  \textbf{Advantages}: High efficiency, no moving parts, reliable
\item
  \textbf{Limitations}: Unidirectional current flow, needs commutation
\end{itemize}

\end{solutionbox}
\begin{mnemonicbox}
``POWER'' - ``Pulse Operation With Electronic
Regulation''

\end{mnemonicbox}
\subsection*{Question 4(a) [3 marks]}\label{q4a}

\textbf{List characteristics of Ideal OP-AMP.}

\begin{solutionbox}
Ideal operational amplifiers have perfect
characteristics that real devices approximate.


{\def\LTcaptype{none} % do not increment counter
\begin{longtable}[]{@{}ll@{}}
\toprule\noalign{}
Characteristic & Ideal Value \\
\midrule\noalign{}
\endhead
\bottomrule\noalign{}
\endlastfoot
Open loop gain & Infinite \\
Input impedance & Infinite \\
Output impedance & Zero \\
Bandwidth & Infinite \\
CMRR & Infinite \\
Slew rate & Infinite \\
Offset voltage & Zero \\
\end{longtable}
}

\begin{itemize}
\tightlist
\item
  \textbf{Practical values}: Actual op-amps have limitations
\item
  \textbf{Implications}: Circuit design must account for real
  limitations
\end{itemize}

\end{solutionbox}
\begin{mnemonicbox}
``IBOCSS'' - ``Infinite Bandwidth, Open-loop gain,
CMRR, Slew rate, and Sensitivity''

\end{mnemonicbox}
\subsection*{Question 4(b) [4 marks]}\label{q4b}

\textbf{Explain working of differential amplifier using OP-AMP with
circuit diagram.}

\begin{solutionbox}
Differential amplifier amplifies the voltage difference
between two inputs.

\textbf{Circuit:}

\begin{lstlisting}
              R2
       +------/\/\------+
       |                |
       |           +----+
       |           |    |
       |    R1     |    |
  V1 o-+---/\/\----+    +----o Vout
                  _|+   |
                 /      |
                /       |
               /______  |
                  -|    |
       |    R1     |    |
  V2 o-+---/\/\----+    |
       |                |
       |                |
       +------/\/\------+
              R2
\end{lstlisting}

\begin{itemize}
\tightlist
\item
  \textbf{Gain formula}: Vout = (V1-V2) \times (R2/R1)
\item
  \textbf{Common mode rejection}: Suppresses signals common to both
  inputs
\item
  \textbf{Applications}: Instrumentation, medical equipment, audio
\end{itemize}

\end{solutionbox}
\begin{mnemonicbox}
``DIFF'' - ``Dual Input For Feedback''

\end{mnemonicbox}
\subsection*{Question 4(c) [7 marks]}\label{q4c}

\textbf{Explain OP-AMP as an inverting amplifier (Closed loop) and
derive the formula of voltage gain.}

\begin{solutionbox}
Inverting amplifier produces output that is inverted
and amplified version of input.

\textbf{Circuit:}

\begin{lstlisting}
          Rf
     +----/\/\----+
     |            |
     |            |
     |    +-------+----o Vout
     |    |       |
     |    |   +---+
Vin o+----+---|+  |
     |        |   |
     |  Ri    |   |
     +--/\/\--+---+
                -|
                 |
                 |
                 |
     +-----------+
     |
    GND
\end{lstlisting}

\textbf{Gain Derivation:}

\begin{itemize}
\item
  Apply KCL at inverting input: I_{1} + I_{2} = 0
\item
  I_{1} = (Vin - V^{-})/Ri and I_{2} = (Vout - V^{-})/Rf
\item
  At virtual ground, V^{-} \approx 0
\item
  Therefore: Vin/Ri + Vout/Rf = 0
\item
  Solving for Vout/Vin: Av = -Rf/Ri
\item
  \textbf{Characteristics}: Output 180^\circ out of phase with input
\item
  \textbf{Feedback}: Creates virtual ground at inverting input
\item
  \textbf{Closed loop gain}: Controlled by external resistors
\end{itemize}

\end{solutionbox}
\begin{mnemonicbox}
``VAIN'' - ``Virtual ground Amplification Inverts
Negative''

\end{mnemonicbox}
\subsection*{OR}\label{or-3}

\subsection*{Question 4(a) [3 marks]}\label{q4a}

\textbf{Define the following parameters of OPAMP:} \textbf{1) CMRR\\
2) Slew rate\\
3) Gain Bandwidth Product}

\begin{solutionbox}
These parameters define key performance characteristics
of operational amplifiers.


{\def\LTcaptype{none} % do not increment counter
\begin{longtable}[]{@{}
  >{\raggedright\arraybackslash}p{(\linewidth - 4\tabcolsep) * \real{0.3143}}
  >{\raggedright\arraybackslash}p{(\linewidth - 4\tabcolsep) * \real{0.3429}}
  >{\raggedright\arraybackslash}p{(\linewidth - 4\tabcolsep) * \real{0.3429}}@{}}
\toprule\noalign{}
\begin{minipage}[b]{\linewidth}\raggedright
Parameter
\end{minipage} & \begin{minipage}[b]{\linewidth}\raggedright
Definition
\end{minipage} & \begin{minipage}[b]{\linewidth}\raggedright
Importance
\end{minipage} \\
\midrule\noalign{}
\endhead
\bottomrule\noalign{}
\endlastfoot
CMRR & Ratio of differential gain to common-mode gain & Higher is better
for rejecting noise \\
Slew Rate & Maximum rate of output voltage change (V/μs) & Determines
large-signal bandwidth \\
Gain-Bandwidth Product & Product of gain and frequency (MHz) & Measures
high-frequency performance \\
\end{longtable}
}

\begin{itemize}
\tightlist
\item
  \textbf{CMRR}: Typically 80-120dB in quality op-amps
\item
  \textbf{Slew Rate}: Limits output for high-frequency, high-amplitude
  signals
\item
  \textbf{GBP}: Remains constant as frequency increases
\end{itemize}

\end{solutionbox}
\begin{mnemonicbox}
``CSG'' - ``Common-mode rejection, Speed, and Gain''

\end{mnemonicbox}
\subsection*{Question 4(b) [4 marks]}\label{q4b}

\textbf{Draw and explain summing amplifier using OP-AMP.}

\begin{solutionbox}
Summing amplifier produces output proportional to
weighted sum of input voltages.

\textbf{Circuit:}

\begin{lstlisting}
              Rf
       +------/\/\------+
       |                |
       |           +----+
       |           |    |
       |    R1     |    |
  V1 o-+---/\/\----+    +----o Vout
       |           |+   |
       |    R2    /     |
  V2 o-+---/\/\---+     |
       |          \_____|
       |    R3     |-   |
  V3 o-+---/\/\----+    |
       |                |
       |                |
     -----              |
      ---               |
       -                |
\end{lstlisting}

\begin{itemize}
\tightlist
\item
  \textbf{Output formula}: Vout = -Rf(V_{1}/R_{1} + V_{2}/R_{2} + V_{3}/R_{3})
\item
  \textbf{Applications}: Audio mixer, analog computers, signal
  processing
\item
  \textbf{Advantage}: Multiple inputs can be processed simultaneously
\end{itemize}

\end{solutionbox}
\begin{mnemonicbox}
``SUM'' - ``Several Unified Multipliers''

\end{mnemonicbox}
\subsection*{Question 4(c) [7 marks]}\label{q4c}

\textbf{Draw the pin diagram of IC 555 and explain Monostable
multivibrator using IC555 with waveform.}

\begin{solutionbox}
IC 555 timer in monostable mode produces a single pulse
of fixed duration when triggered.

\textbf{Pin Diagram:}

\begin{lstlisting}
    +-------+
  1 |o      | 8
    |       |
  2 |o      | 7
    |  555  |
  3 |o      | 6
    |       |
  4 |o      | 5
    +-------+

1: GND     5: Control
2: Trigger  6: Threshold
3: Output   7: Discharge
4: Reset    8: VCC
\end{lstlisting}

\textbf{Circuit and Waveform:}

\includegraphics[width=1\linewidth,height=\textheight,keepaspectratio]{mermaid-1189a9ad.pdf}

\begin{itemize}
\tightlist
\item
  \textbf{Operation}: Negative trigger starts timing cycle
\item
  \textbf{Time period}: T = 1.1 \times R \times C
\item
  \textbf{Applications}: Timers, pulse generation, debouncing
\item
  \textbf{Advantages}: Simple, reliable, widely available
\end{itemize}

\end{solutionbox}
\begin{mnemonicbox}
``TIMER'' - ``Triggered Input Makes Extended
Response''

\end{mnemonicbox}
\subsection*{Question 5(a) [3 marks]}\label{q5a}

\textbf{Draw block diagram of SMPS and give its applications.}

\begin{solutionbox}
Switch Mode Power Supply (SMPS) uses switching elements
for efficient power conversion.

\textbf{Block Diagram:}

\includegraphics[width=1\linewidth,height=\textheight,keepaspectratio]{mermaid-9295385c.pdf}

\textbf{Applications:}

\begin{itemize}
\item
  Computer power supplies
\item
  Mobile phone chargers
\item
  TV power supplies
\item
  Industrial power systems
\item
  LED lighting drivers
\item
  \textbf{Advantages}: High efficiency, small size, lightweight
\item
  \textbf{Types}: Buck, boost, buck-boost, flyback converters
\end{itemize}

\end{solutionbox}
\begin{mnemonicbox}
``SAFE'' - ``Switching Achieves Filtered Energy''

\end{mnemonicbox}
\subsection*{Question 5(b) [4 marks]}\label{q5b}

\textbf{Explain working of Regulated Power Supply with diagram.}

\begin{solutionbox}
Regulated power supply maintains constant output
despite input or load variations.

\textbf{Block Diagram:}

\includegraphics[width=1\linewidth,height=\textheight,keepaspectratio]{mermaid-cc1bc154.pdf}

\begin{itemize}
\tightlist
\item
  \textbf{Transformer}: Steps down AC voltage to required level
\item
  \textbf{Rectifier}: Converts AC to pulsating DC (diode bridge)
\item
  \textbf{Filter}: Smooths DC with capacitors
\item
  \textbf{Regulator}: Maintains constant output voltage
\item
  \textbf{Feedback}: Compensates for input/load variations
\end{itemize}

\end{solutionbox}
\begin{mnemonicbox}
``TRFRO'' - ``Transform, Rectify, Filter, Regulate,
Output''

\end{mnemonicbox}
\subsection*{Question 5(c) [7 marks]}\label{q5c}

\textbf{Explain basic block diagram of OP-AMP with diagram.}

\begin{solutionbox}
Operational amplifier's internal structure consists of
several stages performing specific functions.

\textbf{Block Diagram:}

\includegraphics[width=1\linewidth,height=\textheight,keepaspectratio]{mermaid-5070e595.pdf}

\begin{itemize}
\tightlist
\item
  \textbf{Differential input stage}: High impedance, amplifies
  difference
\item
  \textbf{Intermediate stage}: Provides additional gain
\item
  \textbf{Level shifter}: Adjusts DC level between stages
\item
  \textbf{Output stage}: Low impedance, current amplification
\item
  \textbf{Bias circuit}: Establishes operating points for all stages
\item
  \textbf{Compensation}: Internal capacitor for stability
\end{itemize}

\end{solutionbox}
\begin{mnemonicbox}
``DILO'' - ``Differential Input, Level shift,
Output''

\end{mnemonicbox}
\subsection*{OR}\label{or-4}

\subsection*{Question 5(a) [3 marks]}\label{q5a}

\textbf{Explain adjustable voltage regulator using LM317 with diagram.}

\begin{solutionbox}
LM317 is a versatile adjustable positive voltage
regulator with output range of 1.25V to 37V.

\textbf{Circuit:}

\begin{lstlisting}
    Vin             LM317            Vout
     o-----+--------+-------+--------o
           |       Vin     |
           |        |      |
           |      +---+    |
           |      |317|    |
           |      +---+    |  C2
           |     Adj|Out   +--||--+
           |        |         |   |
           |        +---------+   |
           |                  |   |
     C1    |       R1         |   |
     ||    +------/\/\--------+   |
     ||    |                  |   |
     ||    |                  |   |
     ++----+       R2         |   |
      |            /\/\-------+   |
      |            |          |   |
     GND          GND        GND GND
\end{lstlisting}

\begin{itemize}
\tightlist
\item
  \textbf{Formula}: Vout = 1.25(1 + R2/R1)
\item
  \textbf{Advantages}: Simple adjustment, built-in protection
\item
  \textbf{Applications}: Variable power supplies, battery chargers
\end{itemize}

\end{solutionbox}
\begin{mnemonicbox}
``AVR'' - ``Adjustable Voltage Regulation''

\end{mnemonicbox}
\subsection*{Question 5(b) [4 marks]}\label{q5b}

\textbf{Give the difference between Fixed voltage regulator IC and
Variable voltage regulator IC.}

\begin{solutionbox}
Voltage regulator ICs differ in their configurability
and application requirements.


{\def\LTcaptype{none} % do not increment counter
\begin{longtable}[]{@{}
  >{\raggedright\arraybackslash}p{(\linewidth - 4\tabcolsep) * \real{0.1719}}
  >{\raggedright\arraybackslash}p{(\linewidth - 4\tabcolsep) * \real{0.3906}}
  >{\raggedright\arraybackslash}p{(\linewidth - 4\tabcolsep) * \real{0.4375}}@{}}
\toprule\noalign{}
\begin{minipage}[b]{\linewidth}\raggedright
Parameter
\end{minipage} & \begin{minipage}[b]{\linewidth}\raggedright
Fixed Voltage Regulator
\end{minipage} & \begin{minipage}[b]{\linewidth}\raggedright
Variable Voltage Regulator
\end{minipage} \\
\midrule\noalign{}
\endhead
\bottomrule\noalign{}
\endlastfoot
Output voltage & Predetermined (e.g., 5V, 12V) & Adjustable over a
range \\
External components & Minimal (capacitors only) & Requires resistors for
setting \\
Series & 78xx (positive), 79xx (negative) & LM317 (positive), LM337
(negative) \\
Applications & Standard equipment & Custom designs, laboratory
supplies \\
Flexibility & Limited to fixed values & Highly adaptable \\
Pin count & Typically 3 pins & 3 or more pins \\
\end{longtable}
}

\begin{itemize}
\tightlist
\item
  \textbf{Fixed regulators}: Simple to use, limited adjustment
\item
  \textbf{Variable regulators}: More versatile, require calculation
\end{itemize}

\end{solutionbox}
\begin{mnemonicbox}
``FOCUS'' - ``Fixed Output Compared to User-Set''

\end{mnemonicbox}
\subsection*{Question 5(c) [7 marks]}\label{q5c}

\textbf{List applications of OP-AMP. Explain working operation of D to A
converter with circuit diagram using OP-AMP.}

\begin{solutionbox}
Op-amps have numerous applications; D/A converters
transform digital signals to analog.

\textbf{Applications of OP-AMP:}

\begin{itemize}
\tightlist
\item
  Amplifiers (inverting, non-inverting)
\item
  Filters (active filters)
\item
  Oscillators
\item
  Comparators
\item
  Integrators and differentiators
\item
  Voltage followers
\item
  Instrumentation circuits
\end{itemize}

\textbf{R-2R Ladder DAC Circuit:}

\begin{lstlisting}
    D3   D2   D1   D0
     |    |    |    |
     v    v    v    v
     SW   SW   SW   SW
     |    |    |    |
   2R|   2R|  2R|  2R|
     |    |    |    |
     +----+----+----+
     |    |    |    |
     R    R    R    R
     |    |    |    |
     +----+----+----+---+
                      _|+
                     /
                    /
                   /___
                      -|
                       |
              Rf       |
              /\/\-----+----o Vout
              |        |
              |        |
             GND      GND
\end{lstlisting}

\begin{itemize}
\tightlist
\item
  \textbf{Working principle}: Digital inputs weight currents through
  resistor network
\item
  \textbf{Resistance values}: Binary-weighted or R-2R ladder network
\item
  \textbf{Conversion}: Output voltage proportional to digital input
  value
\item
  \textbf{Resolution}: Determined by number of bits (2^{n} levels)
\end{itemize}

\end{solutionbox}
\begin{mnemonicbox}
``DART'' - ``Digital to Analog Resistor Translation''

\end{mnemonicbox}

\end{document}
