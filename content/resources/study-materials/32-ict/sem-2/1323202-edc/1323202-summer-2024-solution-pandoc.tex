\documentclass[10pt,a4paper]{article}

% content/resources/templates/preamble.tex
\usepackage[margin=0.6in]{geometry}
\author{Milav Dabgar}
\usepackage{amsmath,amssymb,amsthm}
\usepackage{booktabs}
\usepackage{multirow}
\usepackage{xcolor}
\usepackage{tcolorbox}
\tcbuselibrary{breakable,skins}
\usepackage[colorlinks=true,linkcolor=blue]{hyperref}
\usepackage{titlesec}
\usepackage{enumitem}
\usepackage{tikz}
\usepackage{pgfplots}
\usepackage{circuitikz}
\usepackage[version=4]{mhchem}
\usepackage{longtable}
\usepackage{array}
\usepackage{float}
\usepackage{caption}
\usepackage{listings}

\lstset{
  basicstyle=\small\ttfamily,
  breaklines=true,
  breakatwhitespace=false,
  postbreak=\mbox{\textcolor{red}{$\hookrightarrow$}\space},
  float=false,
  numbers=left,
  numberstyle=\tiny\color{gray},
  numbersep=10pt,
  xleftmargin=2em,
  keywordstyle=\color{blue},
  commentstyle=\color{green!60!black},
  stringstyle=\color{purple},
  backgroundcolor=\color{gray!5},
  showstringspaces=false,
  tabsize=2,
  captionpos=b,
  keepspaces=true,
  columns=flexible
}

\pgfplotsset{compat=1.18}
\usetikzlibrary{shapes,arrows,positioning,calc,patterns,decorations.pathmorphing,decorations.markings,arrows.meta}

% Color scheme
\definecolor{headcolor}{RGB}{0,102,204}
\definecolor{keycolor}{RGB}{220,20,60}
\definecolor{solutioncolor}{RGB}{34,139,34}
\definecolor{mnemoniccolor}{RGB}{148,0,211}
\definecolor{codecolor}{RGB}{0,0,100}

% Spacing
\setlength{\parskip}{3pt}
\setlist[itemize]{nosep}
\setlist[enumerate]{nosep}

% Title formatting
\titleformat{\section}{\Large\bfseries\color{headcolor}}{\thesection}{1em}{}
\titleformat{\subsection}{\large\bfseries\color{headcolor}}{\thesubsection}{1em}{}

% Pandoc tightlist compatibility
\providecommand{\tightlist}{%
  \setlength{\itemsep}{0pt}\setlength{\parskip}{0pt}}

% Pandoc longtable compatibility
\newcounter{none}
\def\thenone{}


% content/resources/templates/english-boxes.tex
% This file is currently empty - it exists to maintain consistency with the import structure.
% Add custom environments here if needed in the future.


\begin{document}

\begin{center}
{\Huge\bfseries\color{headcolor} Subject Name Solutions}\\[5pt]
{\LARGE 1323202 -- Summer 2024}\\[3pt]
{\large Semester 1 Study Material}\\[3pt]
{\normalsize\textit{Detailed Solutions and Explanations}}
\end{center}

\vspace{10pt}

\subsection*{Question 1(a) [3 marks]}\label{q1a}

\textbf{What is heat sink. lists its types}

\begin{solutionbox}
A heat sink is a passive device that absorbs and
dissipates heat from electronic components to prevent overheating.


{\def\LTcaptype{none} % do not increment counter
\vspace{-5pt}
\captionof{table}{Types of Heat Sinks}
\vspace{-10pt}
\begin{longtable}[]{@{}ll@{}}
\toprule\noalign{}
Type & Description \\
\midrule\noalign{}
\endhead
\bottomrule\noalign{}
\endlastfoot
\textbf{Passive} & Uses natural convection without external power \\
\textbf{Active} & Incorporates fans or liquid cooling \\
\textbf{Radial} & Fins arranged in radial pattern from center \\
\textbf{Pin-fin} & Uses pins or rods for increased surface area \\
\textbf{Extruded} & Made by forcing aluminum through shaped die \\
\end{longtable}
}

\end{solutionbox}
\begin{mnemonicbox}
``PAPER'' (Passive, Active, Pin-fin, Extruded,
Radial)

\end{mnemonicbox}
\subsection*{Question 1(b) [4 marks]}\label{q1b}

\textbf{Define the Following: 1. Thermal Runaway 2. Thermal Stability}

\begin{solutionbox}

\textbf{Thermal Runaway}: The self-accelerating destructive process
where increased temperature causes increased current flow, which further
increases temperature, potentially destroying the transistor.

\textbf{Thermal Stability}: The ability of a transistor circuit to
maintain stable operation despite temperature changes, preventing
thermal runaway.

\textbf{Diagram: Thermal Runaway Process}

\includegraphics[width=1\linewidth,height=\textheight,keepaspectratio]{mermaid-46437b30.pdf}

\end{solutionbox}
\begin{mnemonicbox}
``RISE'' (Runaway Is Self-Escalating)

\end{mnemonicbox}
\subsection*{Question 1(c) [7 marks]}\label{q1c}

\textbf{Explain voltage divider bias in details.}

\begin{solutionbox}
Voltage divider bias is a common transistor biasing
technique that provides stable operation.

\textbf{Circuit Diagram:}

\includegraphics[width=1\linewidth,height=\textheight,keepaspectratio]{mermaid-66e1db85.pdf}

\begin{itemize}
\tightlist
\item
  \textbf{Voltage divider network}: R1 and R2 establish a fixed base
  voltage
\item
  \textbf{Stable Q-point}: Maintains operating point despite temperature
  variations
\item
  \textbf{Better stability}: Higher stability factor compared to fixed
  bias
\item
  \textbf{Self-adjusting}: Base current automatically adjusts to counter
  temperature changes
\end{itemize}

\end{solutionbox}
\begin{mnemonicbox}
``VSST'' (Voltage divider, Stable, Self-adjusting,
Temperature resistant)

\end{mnemonicbox}
\subsection*{Question 1(c) OR [7
marks]}\label{q1c}

\textbf{Explain D.C. Load Line in details.}

\begin{solutionbox}
DC Load Line is a graphical method for analyzing
transistor bias conditions.

\textbf{Diagram: DC Load Line on Transistor Characteristic Curves}

\includegraphics[width=1\linewidth,height=\textheight,keepaspectratio]{mermaid-af2982a8.pdf}

\begin{itemize}
\tightlist
\item
  \textbf{Definition}: Graphical line showing all possible operating
  points for a given circuit
\item
  \textbf{Endpoints}: (0, VCC/RC) and (VCC, 0) on IC-VCE plane
\item
  \textbf{Q-point}: Intersection of load line with transistor
  characteristic curve
\item
  \textbf{Equation}: IC = (VCC - VCE)/RC
\end{itemize}

\end{solutionbox}
\begin{mnemonicbox}
``QECC'' (Q-point Exists where Collector Current
meets characteristics)

\end{mnemonicbox}
\subsection*{Question 2(a) [3 marks]}\label{q2a}

\textbf{Explain how transistor works as a switch.}

\begin{solutionbox}
A transistor switch operates in either saturation (ON)
or cutoff (OFF) regions.


{\def\LTcaptype{none} % do not increment counter
\vspace{-5pt}
\captionof{table}{Transistor Switch Operation}
\vspace{-10pt}
\begin{longtable}[]{@{}lllll@{}}
\toprule\noalign{}
State & Region & Base Current & Collector Current & VCE \\
\midrule\noalign{}
\endhead
\bottomrule\noalign{}
\endlastfoot
OFF & Cutoff & IB \approx 0 & IC \approx 0 & VCE \approx VCC \\
ON & Saturation & IB \textgreater{} IB(sat) & IC \approx IC(sat) & VCE \approx
0.2V \\
\end{longtable}
}

\end{solutionbox}
\begin{mnemonicbox}
``COS'' (Cutoff Off, Saturation on)

\end{mnemonicbox}
\subsection*{Question 2(b) [4 marks]}\label{q2b}

\textbf{Draw and explain colpitt oscillator.}

\begin{solutionbox}
Colpitt oscillator is an LC oscillator using a
capacitive voltage divider for feedback.

\textbf{Circuit Diagram:}

\includegraphics[width=1\linewidth,height=\textheight,keepaspectratio]{mermaid-7b75536c.pdf}

\begin{itemize}
\tightlist
\item
  \textbf{Feedback}: Provided by capacitive voltage divider (C1, C2)
\item
  \textbf{Resonant frequency}: f = 1/(2π\sqrt(L\timesC)), where C =
  (C1\timesC2)/(C1+C2)
\item
  \textbf{Oscillation}: Maintains through regenerative feedback
\item
  \textbf{Phase shift}: 360^\circ around the loop
\end{itemize}

\end{solutionbox}
\begin{mnemonicbox}
``CFPO'' (Capacitive Feedback Produces Oscillations)

\end{mnemonicbox}
\subsection*{Question 2(c) [7 marks]}\label{q2c}

\textbf{Explain Frequency Response Two Stage RC Coupled Amplifier with
circuit diagram.}

\begin{solutionbox}
Two-stage RC coupled amplifier combines two amplifier
stages with RC coupling.

\textbf{Circuit Diagram:}

\includegraphics[width=1\linewidth,height=\textheight,keepaspectratio]{mermaid-7a80a75f.pdf}

\textbf{Frequency Response:}

\includegraphics[width=1\linewidth,height=\textheight,keepaspectratio]{mermaid-cc4eb31f.pdf}

\begin{itemize}
\tightlist
\item
  \textbf{Low frequency}: Gain drops due to coupling capacitor impedance
\item
  \textbf{Mid frequency}: Maximum flat gain region (bandwidth)
\item
  \textbf{High frequency}: Gain drops due to transistor capacitance
  effects
\item
  \textbf{Overall gain}: Product of individual stage gains
\end{itemize}

\end{solutionbox}
\begin{mnemonicbox}
``LMH'' (Low drops, Mid flat, High drops)

\end{mnemonicbox}
\subsection*{Question 2(a) OR [3
marks]}\label{q2a}

\textbf{Draw circuit diagram of Hartley oscillator.}

\begin{solutionbox}

\textbf{Circuit Diagram of Hartley Oscillator:}

\includegraphics[width=1\linewidth,height=\textheight,keepaspectratio]{mermaid-f2698584.pdf}

\end{solutionbox}
\begin{mnemonicbox}
``ITLC'' (Inductor Tapped for LC Circuit)

\end{mnemonicbox}
\subsection*{Question 2(b) OR [4
marks]}\label{q2b}

\textbf{List different types of negative feedback.}

\begin{solutionbox}


{\def\LTcaptype{none} % do not increment counter
\vspace{-5pt}
\captionof{table}{Types of Negative Feedback}
\vspace{-10pt}
\begin{longtable}[]{@{}
  >{\raggedright\arraybackslash}p{(\linewidth - 4\tabcolsep) * \real{0.1429}}
  >{\raggedright\arraybackslash}p{(\linewidth - 4\tabcolsep) * \real{0.3571}}
  >{\raggedright\arraybackslash}p{(\linewidth - 4\tabcolsep) * \real{0.5000}}@{}}
\toprule\noalign{}
\begin{minipage}[b]{\linewidth}\raggedright
Type
\end{minipage} & \begin{minipage}[b]{\linewidth}\raggedright
Configuration
\end{minipage} & \begin{minipage}[b]{\linewidth}\raggedright
Effect on Parameters
\end{minipage} \\
\midrule\noalign{}
\endhead
\bottomrule\noalign{}
\endlastfoot
\textbf{Voltage Series} & Output voltage fed to input in series &
Increases input impedance, reduces distortion \\
\textbf{Voltage Shunt} & Output voltage fed to input in parallel &
Decreases input impedance, increases bandwidth \\
\textbf{Current Series} & Output current fed to input in series &
Increases output impedance, stabilizes current gain \\
\textbf{Current Shunt} & Output current fed to input in parallel &
Decreases output impedance, stabilizes voltage gain \\
\end{longtable}
}

\end{solutionbox}
\begin{mnemonicbox}
``VSCS'' (Voltage Series, Current Shunt)

\end{mnemonicbox}
\subsection*{Question 2(c) OR [7
marks]}\label{q2c}

\textbf{List advantages of Negative feedback amplifier and Explain
voltage series negative feedback in details.}

\begin{solutionbox}

\textbf{Advantages of Negative Feedback:}

\begin{itemize}
\tightlist
\item
  Stabilizes gain against component variations
\item
  Reduces distortion and noise
\item
  Increases bandwidth
\item
  Modifies input/output impedance
\item
  Improves linearity
\end{itemize}

\textbf{Voltage Series Negative Feedback:}

\includegraphics[width=1\linewidth,height=\textheight,keepaspectratio]{mermaid-15c46489.pdf}

\begin{itemize}
\tightlist
\item
  \textbf{Configuration}: Output voltage sampled, fed back in series
  with input
\item
  \textbf{Closed-loop gain}: ACL = A/(1+Aβ), where A is open-loop gain
  and β is feedback fraction
\item
  \textbf{Input impedance}: Increases by factor (1+Aβ)
\item
  \textbf{Output impedance}: Decreases by factor (1+Aβ)
\end{itemize}

\end{solutionbox}
\begin{mnemonicbox}
``SIGO'' (Stable gain, Increased input impedance,
Gain reduction, Output impedance reduction)

\end{mnemonicbox}
\subsection*{Question 3(a) [3 marks]}\label{q3a}

\textbf{Draw circuit of SCR using two transistor analogy.}

\begin{solutionbox}

\textbf{Two Transistor Analogy of SCR:}

\includegraphics[width=1\linewidth,height=\textheight,keepaspectratio]{mermaid-d0620b6c.pdf}

\end{solutionbox}
\begin{mnemonicbox}
``PNPNPN'' (PNP and NPN structure)

\end{mnemonicbox}
\subsection*{Question 3(b) [4 marks]}\label{q3b}

\textbf{Draw and explain Natural Commutation of SCR.}

\begin{solutionbox}
Natural commutation occurs when the SCR current
naturally falls below the holding current.

\textbf{Circuit Diagram:}

\includegraphics[width=1\linewidth,height=\textheight,keepaspectratio]{mermaid-19cb68a7.pdf}

\textbf{Current Waveform:}

\begin{lstlisting}
       ┌───┐     ┌───┐
       │   │     │   │
───────┘   └─────┘   └─────
  SCR OFF    SCR OFF
       SCR ON    SCR ON
\end{lstlisting}

\begin{itemize}
\tightlist
\item
  \textbf{Definition}: SCR turns off automatically when current falls
  below holding current
\item
  \textbf{AC circuit}: Occurs naturally at end of each positive
  half-cycle
\item
  \textbf{Zero crossing}: SCR turns off when AC voltage crosses zero
\item
  \textbf{No external circuit}: No additional components needed for
  turn-off
\end{itemize}

\end{solutionbox}
\begin{mnemonicbox}
``NAZC'' (Natural At Zero Crossing)

\end{mnemonicbox}
\subsection*{Question 3(c) [7 marks]}\label{q3c}

\textbf{Explain how TRIAC can be used as fan regulator and on-off
control for ac power.}

\begin{solutionbox}
TRIAC is a bidirectional device ideal for AC power
control applications.

\textbf{TRIAC Fan Regulator Circuit:}

\includegraphics[width=1\linewidth,height=\textheight,keepaspectratio]{mermaid-2e80ee38.pdf}

\textbf{TRIAC On-Off Control:}

\includegraphics[width=1\linewidth,height=\textheight,keepaspectratio]{mermaid-eb8823b5.pdf}

\begin{itemize}
\tightlist
\item
  \textbf{Fan Regulation}: Phase control technique varies power to fan
\item
  \textbf{Potentiometer}: Adjusts firing angle of TRIAC
\item
  \textbf{On-Off Control}: Simple switch triggers TRIAC gate
\item
  \textbf{Bidirectional}: Controls current in both half-cycles
\end{itemize}

\end{solutionbox}
\begin{mnemonicbox}
``FPOB'' (Fan Power is controlled by Phase angle in
both directions)

\end{mnemonicbox}
\subsection*{Question 3(a) OR [3
marks]}\label{q3a}

\textbf{Draw symbol of SCR, DIAC and TRIAC.}

\begin{solutionbox}

\textbf{Symbols of Thyristors:}

\begin{lstlisting}
    SCR            DIAC           TRIAC
    
    A              
    |              
   ┌┴┐            ┌─┐            ┌─┐
   │ │            │ │            │ │
   └┬┘            └─┘            └─┘
    │              |              |
    │              |              |
    G─┐            |              G─┐
      │            |                │
    K |            |                |
\end{lstlisting}

\end{solutionbox}
\begin{mnemonicbox}
``SDT'' (SCR has gate on one side, DIAC has none,
TRIAC has gate in middle)

\end{mnemonicbox}
\subsection*{Question 3(b) OR [4
marks]}\label{q3b}

\textbf{Draw and explain Gate triggering of SCR.}

\begin{solutionbox}
Gate triggering is the most common method to turn on an
SCR.

\textbf{Circuit Diagram:}

\includegraphics[width=1\linewidth,height=\textheight,keepaspectratio]{mermaid-4a2a1c7e.pdf}

\begin{itemize}
\tightlist
\item
  \textbf{Principle}: Applying positive voltage between gate and cathode
\item
  \textbf{Current requirement}: Small gate current triggers much larger
  anode current
\item
  \textbf{Latching}: Once triggered, SCR remains ON even if gate signal
  is removed
\item
  \textbf{Turn-off}: Requires reducing anode current below holding
  current
\end{itemize}

\end{solutionbox}
\begin{mnemonicbox}
``GPLT'' (Gate Pulse Latches Thyristor)

\end{mnemonicbox}
\subsection*{Question 3(c) OR [7
marks]}\label{q3c}

\textbf{Draw Construction and Voltage Vs Current characteristic of SCR
and explain V-I characteristic.}

\begin{solutionbox}
SCR (Silicon Controlled Rectifier) is a four-layer PNPN
semiconductor device.

\textbf{SCR Construction:}

\includegraphics[width=1\linewidth,height=\textheight,keepaspectratio]{mermaid-c6c82156.pdf}

\textbf{V-I Characteristic:}

\begin{lstlisting}
          I
          ↑
          │        ON State
          │       ┌────────
          │       │
          │       │
  Holding │       │
  current ├───────┤
          │       │
          │Forward│
          │breakover
          │voltage│
          │       │
          └───────┴──────\rightarrow V
                   Reverse
                   breakdown
                   voltage
\end{lstlisting}

\begin{itemize}
\tightlist
\item
  \textbf{Forward blocking region}: SCR conducts minimal current until
  breakover voltage
\item
  \textbf{Forward conduction region}: Low resistance state after
  triggering
\item
  \textbf{Reverse blocking region}: Blocks current in reverse direction
\item
  \textbf{Gate triggering}: Reduces breakover voltage, facilitating
  turn-on
\end{itemize}

\end{solutionbox}
\begin{mnemonicbox}
``FBRH'' (Forward Blocking, Reverse blocking, Holding
current)

\end{mnemonicbox}
\subsection*{Question 4(a) [3 marks]}\label{q4a}

\textbf{Explain OP-AMP as a summing amplifier.}

\begin{solutionbox}
Summing amplifier adds multiple input signals with
weighted gains.

\textbf{Circuit Diagram:}

\includegraphics[width=1\linewidth,height=\textheight,keepaspectratio]{mermaid-c5f7b0b9.pdf}

\begin{itemize}
\tightlist
\item
  \textbf{Function}: Outputs weighted sum of input voltages
\item
  \textbf{Output equation}: Vout = -(V1\timesRf/R1 + V2\timesRf/R2 + V3\timesRf/R3)
\item
  \textbf{Equal weights}: When R1 = R2 = R3, output is simple sum
  multiplied by -Rf/R
\item
  \textbf{Virtual ground}: Inverting input maintains 0V potential
\end{itemize}

\end{solutionbox}
\begin{mnemonicbox}
``SWAP'' (Sum Weighted And Proportional)

\end{mnemonicbox}
\subsection*{Question 4(b) [4 marks]}\label{q4b}

\textbf{Define the following OP-AMP parameters: 1. input bias current 2.
CMRR}

\begin{solutionbox}

\textbf{Input Bias Current}: The average of the currents flowing into
the two input terminals of an op-amp when the output is at zero.

\textbf{CMRR (Common Mode Rejection Ratio)}: The ratio of differential
gain to common-mode gain, indicating how well an op-amp rejects signals
common to both inputs.


{\def\LTcaptype{none} % do not increment counter
\vspace{-5pt}
\captionof{table}{Op-Amp Parameters}
\vspace{-10pt}
\begin{longtable}[]{@{}
  >{\raggedright\arraybackslash}p{(\linewidth - 4\tabcolsep) * \real{0.2895}}
  >{\raggedright\arraybackslash}p{(\linewidth - 4\tabcolsep) * \real{0.3947}}
  >{\raggedright\arraybackslash}p{(\linewidth - 4\tabcolsep) * \real{0.3158}}@{}}
\toprule\noalign{}
\begin{minipage}[b]{\linewidth}\raggedright
Parameter
\end{minipage} & \begin{minipage}[b]{\linewidth}\raggedright
Typical Value
\end{minipage} & \begin{minipage}[b]{\linewidth}\raggedright
Importance
\end{minipage} \\
\midrule\noalign{}
\endhead
\bottomrule\noalign{}
\endlastfoot
Input Bias Current & 20-200 nA & Lower is better for high impedance
circuits \\
CMRR & 80-120 dB & Higher is better for noise rejection \\
\end{longtable}
}

\end{solutionbox}
\begin{mnemonicbox}
``BIC-CMR'' (Bias Is Current, Common Mode Rejection)

\end{mnemonicbox}
\subsection*{Question 4(c) [7 marks]}\label{q4c}

\textbf{Draw and explain monostable multivibrator using 555 Timer.}

\begin{solutionbox}
Monostable multivibrator generates a single pulse of
predetermined duration when triggered.

\textbf{Circuit Diagram:}

\includegraphics[width=1\linewidth,height=\textheight,keepaspectratio]{mermaid-c83e0a29.pdf}

\textbf{Output Waveform:}

\begin{lstlisting}
Trigger  ___┐      ____________
             │______│
             
Output   ____┌──────┐__________
              │      │
              T = 1.1RC
\end{lstlisting}

\begin{itemize}
\tightlist
\item
  \textbf{Operation}: Single stable state (output LOW), temporarily HIGH
  when triggered
\item
  \textbf{Pulse width}: T = 1.1 \times R \times C (seconds)
\item
  \textbf{Triggering}: Falling edge on TRIG pin (pin 2)
\item
  \textbf{Timing components}: R and C determine pulse duration
\end{itemize}

\end{solutionbox}
\begin{mnemonicbox}
``POST'' (Pulse Output, Single Trigger)

\end{mnemonicbox}
\subsection*{Question 4(a) OR [3
marks]}\label{q4a}

\textbf{Draw the circuit diagram of OP-AMP as a inverting amplifier.}

\begin{solutionbox}

\textbf{Inverting Amplifier Circuit:}

\includegraphics[width=1\linewidth,height=\textheight,keepaspectratio]{mermaid-421cae7c.pdf}

\end{solutionbox}
\begin{mnemonicbox}
``IRON'' (Inverting Requires One Negative input)

\end{mnemonicbox}
\subsection*{Question 4(b) OR [4
marks]}\label{q4b}

\textbf{Define the following OP-AMP parameters: 1. input offset current
2. slew rate}

\begin{solutionbox}

\textbf{Input Offset Current}: The difference between the currents
flowing into the two input terminals of an op-amp.

\textbf{Slew Rate}: The maximum rate of change of output voltage per
unit of time, typically measured in V/μs.


{\def\LTcaptype{none} % do not increment counter
\vspace{-5pt}
\captionof{table}{Op-Amp Parameters}
\vspace{-10pt}
\begin{longtable}[]{@{}
  >{\raggedright\arraybackslash}p{(\linewidth - 4\tabcolsep) * \real{0.2895}}
  >{\raggedright\arraybackslash}p{(\linewidth - 4\tabcolsep) * \real{0.3947}}
  >{\raggedright\arraybackslash}p{(\linewidth - 4\tabcolsep) * \real{0.3158}}@{}}
\toprule\noalign{}
\begin{minipage}[b]{\linewidth}\raggedright
Parameter
\end{minipage} & \begin{minipage}[b]{\linewidth}\raggedright
Typical Value
\end{minipage} & \begin{minipage}[b]{\linewidth}\raggedright
Importance
\end{minipage} \\
\midrule\noalign{}
\endhead
\bottomrule\noalign{}
\endlastfoot
Input Offset Current & 2-50 nA & Lower is better for precision
applications \\
Slew Rate & 0.5-20 V/μs & Higher is better for high-frequency
operation \\
\end{longtable}
}

\end{solutionbox}
\begin{mnemonicbox}
``IOSR'' (Input Offset and Slew Rate)

\end{mnemonicbox}
\subsection*{Question 4(c) OR [7
marks]}\label{q4c}

\textbf{Explain op-amp as Inverting amplifier and obtain equation of its
Voltage gain.}

\begin{solutionbox}
Inverting amplifier produces an output signal that is
inverted and amplified.

\textbf{Circuit Diagram:}

\includegraphics[width=1\linewidth,height=\textheight,keepaspectratio]{mermaid-421cae7c.pdf}

\textbf{Voltage Gain Derivation:}

\begin{lstlisting}
At node N (inverting input):
I1 + If = 0  (By Kirchhoff's Current Law)
(Vin - VN)/R1 + (Vout - VN)/Rf = 0

Since VN \approx 0 (virtual ground):
Vin/R1 + Vout/Rf = 0
Vout/Vin = -Rf/R1
\end{lstlisting}

\begin{itemize}
\tightlist
\item
  \textbf{Gain equation}: Vout/Vin = -Rf/R1
\item
  \textbf{Virtual ground}: Inverting terminal maintained at 0V
\item
  \textbf{Input impedance}: Equal to R1
\item
  \textbf{Negative feedback}: Provides stability and linearity
\end{itemize}

\end{solutionbox}
\begin{mnemonicbox}
``GIVN'' (Gain Is Negative, Virtual ground)

\end{mnemonicbox}
\subsection*{Question 5(a) [3 marks]}\label{q5a}

\textbf{Draw the block diagram of IC 555.}

\begin{solutionbox}

\textbf{Block Diagram of IC 555:}

\includegraphics[width=1\linewidth,height=\textheight,keepaspectratio]{mermaid-c933de86.pdf}

\end{solutionbox}
\begin{mnemonicbox}
``CVOT'' (Comparators, Voltage divider, Output stage,
Timer)

\end{mnemonicbox}
\subsection*{Question 5(b) [4 marks]}\label{q5b}

\textbf{Draw the circuit diagram of OP-AMP as a wein bridge oscillator.}

\begin{solutionbox}

\textbf{Wein Bridge Oscillator Circuit:}

\includegraphics[width=1\linewidth,height=\textheight,keepaspectratio]{mermaid-1f9166e5.pdf}

\end{solutionbox}
\begin{mnemonicbox}
``WPRC'' (Wein Produces Resonant Circuit)

\end{mnemonicbox}
\subsection*{Question 5(c) [7 marks]}\label{q5c}

\textbf{Explain working of different types of Fixed and variable voltage
regulator IC.}

\begin{solutionbox}
Voltage regulator ICs maintain stable output voltage
despite input or load variations.

\textbf{Fixed Voltage Regulators:}

\includegraphics[width=1\linewidth,height=\textheight,keepaspectratio]{mermaid-019a3ed5.pdf}

\textbf{Variable Voltage Regulator:}

\includegraphics[width=1\linewidth,height=\textheight,keepaspectratio]{mermaid-cae7b6bc.pdf}

\begin{itemize}
\tightlist
\item
  \textbf{Fixed regulators}: 78XX (positive) and 79XX (negative) series
  provide specific voltages
\item
  \textbf{Variable regulators}: LM317 (positive) and LM337 (negative)
  allow adjustable output
\item
  \textbf{Three-terminal design}: Input, output, and ground/adjust
  terminals
\item
  \textbf{Output equation for LM317}: Vout = 1.25V \times (1 + R2/R1)
\item
  \textbf{Protection features}: Short circuit, thermal overload, and
  safe area protection
\end{itemize}

\end{solutionbox}
\begin{mnemonicbox}
``FAVOR'' (Fixed And Variable Output Regulators)

\end{mnemonicbox}
\subsection*{Question 5(a) OR [3
marks]}\label{q5a}

\textbf{Draw the block diagram of astable multivibrator using 555
timer.}

\begin{solutionbox}

\textbf{Astable Multivibrator Block Diagram:}

\includegraphics[width=1\linewidth,height=\textheight,keepaspectratio]{mermaid-05ac2a1d.pdf}

\end{solutionbox}
\begin{mnemonicbox}
``FOFT'' (Free-running Oscillator From Timer)

\end{mnemonicbox}
\subsection*{Question 5(b) OR [4
marks]}\label{q5b}

\textbf{Draw and explain solar based battery charger circuits.}

\begin{solutionbox}
Solar battery charger converts solar energy to charge
batteries.

\textbf{Circuit Diagram:}

\includegraphics[width=1\linewidth,height=\textheight,keepaspectratio]{mermaid-09c6da6e.pdf}

\begin{itemize}
\tightlist
\item
  \textbf{Solar panel}: Converts sunlight to DC electricity
\item
  \textbf{Blocking diode}: Prevents battery discharge through panel at
  night
\item
  \textbf{Regulator IC}: Controls charging voltage and current
\item
  \textbf{Charge indicator}: Shows charging status
\item
  \textbf{Protection}: Overcharge and reverse polarity protection
\end{itemize}

\end{solutionbox}
\begin{mnemonicbox}
``SBRCP'' (Solar, Blocking diode, Regulator,
Charging, Protection)

\end{mnemonicbox}
\subsection*{Question 5(c) OR [7
marks]}\label{q5c}

\textbf{Draw and explain the block diagram of SMPS.}

\begin{solutionbox}
SMPS (Switch Mode Power Supply) converts electrical
power efficiently using switching regulators.

\textbf{Block Diagram:}

\includegraphics[width=1\linewidth,height=\textheight,keepaspectratio]{mermaid-c8481bc0.pdf}

\begin{itemize}
\tightlist
\item
  \textbf{EMI filter}: Removes noise from AC input
\item
  \textbf{Rectifier}: Converts AC to unregulated DC
\item
  \textbf{Switching circuit}: Chops DC at high frequency (20-100 kHz)
\item
  \textbf{Transformer}: Provides isolation and voltage conversion
\item
  \textbf{Output rectifier}: Converts high-frequency AC back to DC
\item
  \textbf{Output filter}: Smooths DC output
\item
  \textbf{Feedback circuit}: Monitors output for regulation
\item
  \textbf{Control circuit}: Adjusts switching based on feedback
\end{itemize}

\end{solutionbox}
\begin{mnemonicbox}
``ERST-FOFC'' (EMI filter, Rectifier, Switching,
Transformer, Feedback, Output rectifier, Filter, Control)

\end{mnemonicbox}

\end{document}
