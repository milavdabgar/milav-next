\documentclass[10pt,a4paper]{article}

% content/resources/templates/preamble.tex
\usepackage[margin=0.6in]{geometry}
\author{Milav Dabgar}
\usepackage{amsmath,amssymb,amsthm}
\usepackage{booktabs}
\usepackage{multirow}
\usepackage{xcolor}
\usepackage{tcolorbox}
\tcbuselibrary{breakable,skins}
\usepackage[colorlinks=true,linkcolor=blue]{hyperref}
\usepackage{titlesec}
\usepackage{enumitem}
\usepackage{tikz}
\usepackage{pgfplots}
\usepackage{circuitikz}
\usepackage[version=4]{mhchem}
\usepackage{longtable}
\usepackage{array}
\usepackage{float}
\usepackage{caption}
\usepackage{listings}

\lstset{
  basicstyle=\small\ttfamily,
  breaklines=true,
  breakatwhitespace=false,
  postbreak=\mbox{\textcolor{red}{$\hookrightarrow$}\space},
  float=false,
  numbers=left,
  numberstyle=\tiny\color{gray},
  numbersep=10pt,
  xleftmargin=2em,
  keywordstyle=\color{blue},
  commentstyle=\color{green!60!black},
  stringstyle=\color{purple},
  backgroundcolor=\color{gray!5},
  showstringspaces=false,
  tabsize=2,
  captionpos=b,
  keepspaces=true,
  columns=flexible
}

\pgfplotsset{compat=1.18}
\usetikzlibrary{shapes,arrows,positioning,calc,patterns,decorations.pathmorphing,decorations.markings,arrows.meta}

% Color scheme
\definecolor{headcolor}{RGB}{0,102,204}
\definecolor{keycolor}{RGB}{220,20,60}
\definecolor{solutioncolor}{RGB}{34,139,34}
\definecolor{mnemoniccolor}{RGB}{148,0,211}
\definecolor{codecolor}{RGB}{0,0,100}

% Spacing
\setlength{\parskip}{3pt}
\setlist[itemize]{nosep}
\setlist[enumerate]{nosep}

% Title formatting
\titleformat{\section}{\Large\bfseries\color{headcolor}}{\thesection}{1em}{}
\titleformat{\subsection}{\large\bfseries\color{headcolor}}{\thesubsection}{1em}{}

% Pandoc tightlist compatibility
\providecommand{\tightlist}{%
  \setlength{\itemsep}{0pt}\setlength{\parskip}{0pt}}

% Pandoc longtable compatibility
\newcounter{none}
\def\thenone{}


% content/resources/templates/gujarati-boxes.tex
\usepackage{fontspec}
\usepackage{polyglossia}

% Set Gujarati as main language (document is primarily in Gujarati)
% Note: gloss-gujarati.ldf doesn't exist in polyglossia, but it will use hyphenation patterns
\setdefaultlanguage{gujarati}
\setotherlanguage{english}

% Configure Gujarati font properly
% Use Language=Default to prevent polyglossia from trying to add language-specific features
% that don't exist for Gujarati, which causes "empty feature" warnings
\newfontfamily\gujaratifont[Script=Gujarati,AutoFakeBold=2.5,AutoFakeSlant=0.3]{Noto Sans Gujarati}
\setmainfont[Script=Gujarati,AutoFakeBold=2.5,AutoFakeSlant=0.3]{Noto Sans Gujarati}
% Use Noto Sans Gujarati for monospace to support Gujarati in text
\setmonofont[Scale=0.9]{Noto Sans Gujarati}

% Configure English to use the same font
\newfontfamily\englishfont[Script=Gujarati,AutoFakeBold=2.5,AutoFakeSlant=0.3]{Noto Sans Gujarati}

% Translations for polyglossia
\gappto\captionsgujarati{
  \renewcommand{\tablename}{કોષ્ટક}
  \renewcommand{\figurename}{આકૃતિ}
}

% Helper for TikZ nodes to ensure Gujarati font
\newcommand{\gu}[1]{{\gujaratifont #1}}

% Custom environments
\newtcolorbox{solutionbox}{
    breakable,
    enhanced,
    colback=solutioncolor!5!white,
    colframe=solutioncolor!75!black,
    fonttitle=\bfseries,
    title=જવાબ
}

\newtcolorbox{solutionboxnobreak}{
 colback=solutioncolor!5!white,
 colframe=solutioncolor!75!black,
 fonttitle=\bfseries,
 title=જવાબ
}

\newtcolorbox{keyformula}{
 breakable,
 enhanced,
 colback=keycolor!5!white,
 colframe=keycolor!75!black,
 fonttitle=\bfseries,
 title=રાસાયણિક સમીકરણ/સૂત્ર
}

\newtcolorbox{mnemonicbox}{
 breakable,
 enhanced,
 colback=mnemoniccolor!5!white,
 colframe=mnemoniccolor!75!black,
 fonttitle=\bfseries,
 title=મેમરી ટ્રીક
}


\begin{document}

\begin{center}
{\Huge\bfseries\color{headcolor} Subject Name (Gujarati)}\\[5pt]
{\LARGE 1323202 -- Summer 2024}\\[3pt]
{\large Semester 1 Study Material}\\[3pt]
{\normalsize\textit{Detailed Solutions and Explanations}}
\end{center}

\vspace{10pt}

\subsection*{પ્રશ્ન 1(અ) [3
ગુણ]}\label{uxaaauxab0uxab6uxaa8-1uxa85-3-uxa97uxaa3}

\textbf{હીટ સિંક શું છે. તેના પ્રકારોની યાદી આપો.}

\begin{solutionbox}
હીટ સિંક એ એક પેસિવ ડિવાઈસ છે જે ઇલેક્ટ્રોનિક કોમ્પોનન્ટ્સમાંથી ગરમી
શોષે અને ફેલાવે છે જેથી ઓવરહીટિંગ અટકાવી શકાય.


{\def\LTcaptype{none} % do not increment counter
\vspace{-5pt}
\captionof{table}{હીટ સિંકના પ્રકારો}
\vspace{-10pt}
\begin{longtable}[]{@{}ll@{}}
\toprule\noalign{}
પ્રકાર & વર્ણન \\
\midrule\noalign{}
\endhead
\bottomrule\noalign{}
\endlastfoot
\textbf{પેસિવ} & બાહ્ય પાવર વિના નૈસર્ગિક કન્વેક્શનનો ઉપયોગ કરે છે \\
\textbf{એક્ટિવ} & ફેન અથવા લિક્વિડ કૂલિંગનો સમાવેશ કરે છે \\
\textbf{રેડિયલ} & સેન્ટરથી રેડિયલ પેટર્નમાં ગોઠવાયેલા ફિન્સ \\
\textbf{પિન-ફિન} & વધુ સપાટી ક્ષેત્રફળ માટે પિન અથવા રોડનો ઉપયોગ કરે છે \\
\textbf{એક્સટ્રુડેડ} & આકારવાળા ડાય દ્વારા એલ્યુમિનિયમને ફોર્સ કરીને બનાવવામાં આવે
છે \\
\end{longtable}
}

\end{solutionbox}
\begin{mnemonicbox}
``PAPER'' (Passive, Active, Pin-fin, Extruded,
Radial)

\end{mnemonicbox}
\subsection*{પ્રશ્ન 1(બ) [4
ગુણ]}\label{uxaaauxab0uxab6uxaa8-1uxaac-4-uxa97uxaa3}

\textbf{નીચેનાને વ્યાખ્યાયિત કરો: 1. થર્મલ રનઅવે 2. થર્મલ સ્ટેબીલિટી.}

\begin{solutionbox}

\textbf{થર્મલ રનઅવે}: સ્વ-ત્વરિત વિનાશક પ્રક્રિયા જ્યાં વધતા તાપમાન કરંટ પ્રવાહમાં
વધારો કરે છે, જે વધુ તાપમાન વધારે છે, જે ટ્રાન્ઝિસ્ટરને નુકસાન પહોંચાડી શકે છે.

\textbf{થર્મલ સ્ટેબીલિટી}: તાપમાન ફેરફારો છતાં સ્થિર ઓપરેશન જાળવવા માટે
ટ્રાન્ઝિસ્ટર સર્કિટની ક્ષમતા, જે થર્મલ રનઅવેને અટકાવે છે.

\textbf{આકૃતિ: થર્મલ રનઅવે પ્રક્રિયા}

\includegraphics[width=1\linewidth,height=\textheight,keepaspectratio]{mermaid-32299943.pdf}

\end{solutionbox}
\begin{mnemonicbox}
``RISE'' (Runaway Is Self-Escalating)

\end{mnemonicbox}
\subsection*{પ્રશ્ન 1(ક) [7
ગુણ]}\label{uxaaauxab0uxab6uxaa8-1uxa95-7-uxa97uxaa3}

\textbf{વોલ્ટેજ ડિવાઈડર બાયસને વિગતવાર સમજાવો.}

\begin{solutionbox}
વોલ્ટેજ ડિવાઈડર બાયસ એ એક સામાન્ય ટ્રાન્ઝિસ્ટર બાયસિંગ ટેકનિક છે જે
સ્થિર ઓપરેશન પ્રદાન કરે છે.

\textbf{સર્કિટ ડાયાગ્રામ:}

\includegraphics[width=1\linewidth,height=\textheight,keepaspectratio]{mermaid-66e1db85.pdf}

\begin{itemize}
\tightlist
\item
  \textbf{વોલ્ટેજ ડિવાઈડર નેટવર્ક}: R1 અને R2 એક નિશ્ચિત બેઝ વોલ્ટેજ સ્થાપિત કરે છે
\item
  \textbf{સ્થિર Q-પોઈન્ટ}: તાપમાન વેરિએશન છતાં ઓપરેટિંગ પોઈન્ટને જાળવે છે
\item
  \textbf{વધુ સારી સ્થિરતા}: ફિક્સ્ડ બાયસની તુલનામાં ઉચ્ચ સ્થિરતા ફેક્ટર
\item
  \textbf{સ્વ-એડજસ્ટિંગ}: બેઝ કરંટ આપોઆપ તાપમાન ફેરફારોનો સામનો કરવા માટે એડજસ્ટ
  થાય છે
\end{itemize}

\end{solutionbox}
\begin{mnemonicbox}
``VSST'' (Voltage divider, Stable, Self-adjusting,
Temperature resistant)

\end{mnemonicbox}
\subsection*{પ્રશ્ન 1(ક) OR [7
ગુણ]}\label{uxaaauxab0uxab6uxaa8-1uxa95-or-7-uxa97uxaa3}

\textbf{ડી.સી. લોડ લાઈનને વિગતવાર સમજાવો.}

\begin{solutionbox}
DC લોડ લાઈન એ ટ્રાન્ઝિસ્ટર બાયસ કંડીશન્સના વિશ્લેષણ માટેની
ગ્રાફિકલ પદ્ધતિ છે.

\textbf{આકૃતિ: ટ્રાન્ઝિસ્ટર કેરેક્ટરિસ્ટિક કર્વ પર DC લોડ લાઈન}

\includegraphics[width=1\linewidth,height=\textheight,keepaspectratio]{mermaid-af2982a8.pdf}

\begin{itemize}
\tightlist
\item
  \textbf{વ્યાખ્યા}: આપેલી સર્કિટ માટે તમામ સંભવિત ઓપરેટિંગ પોઇન્ટ્સ દર્શાવતી
  ગ્રાફિકલ લાઈન
\item
  \textbf{એન્ડપોઈન્ટ}: (0, VCC/RC) અને (VCC, 0) IC-VCE પ્લેન પર
\item
  \textbf{Q-પોઈન્ટ}: લોડ લાઈન અને ટ્રાન્ઝિસ્ટર કેરેક્ટરિસ્ટિક કર્વના છેદબિંદુ
\item
  \textbf{સમીકરણ}: IC = (VCC - VCE)/RC
\end{itemize}

\end{solutionbox}
\begin{mnemonicbox}
``QECC'' (Q-point Exists where Collector Current
meets characteristics)

\end{mnemonicbox}
\subsection*{પ્રશ્ન 2(અ) [3
ગુણ]}\label{uxaaauxab0uxab6uxaa8-2uxa85-3-uxa97uxaa3}

\textbf{ટ્રાન્ઝિસ્ટર સ્વીચ તરીકે કેવી રીતે કામ કરે છે તે સમજાવો.}

\begin{solutionbox}
ટ્રાન્ઝિસ્ટર સ્વિચ સેચુરેશન (ON) અથવા કટ-ઓફ (OFF) રીજનમાં કામ કરે
છે.


{\def\LTcaptype{none} % do not increment counter
\vspace{-5pt}
\captionof{table}{ટ્રાન્ઝિસ્ટર સ્વિચ ઓપરેશન}
\vspace{-10pt}
\begin{longtable}[]{@{}lllll@{}}
\toprule\noalign{}
સ્થિતિ & રીજન & બેઝ કરંટ & કલેક્ટર કરંટ & VCE \\
\midrule\noalign{}
\endhead
\bottomrule\noalign{}
\endlastfoot
OFF & કટ-ઓફ & IB \approx 0 & IC \approx 0 & VCE \approx VCC \\
ON & સેચુરેશન & IB \textgreater{} IB(sat) & IC \approx IC(sat) & VCE \approx 0.2V \\
\end{longtable}
}

\end{solutionbox}
\begin{mnemonicbox}
``COS'' (Cutoff Off, Saturation on)

\end{mnemonicbox}
\subsection*{પ્રશ્ન 2(બ) [4
ગુણ]}\label{uxaaauxab0uxab6uxaa8-2uxaac-4-uxa97uxaa3}

\textbf{કોલપીટ ઓસીલેટર દોરો અને સમજાવો.}

\begin{solutionbox}
કોલપીટ ઓસીલેટર એ LC ઓસીલેટર છે જે ફીડબેક માટે કેપેસિટિવ વોલ્ટેજ
ડિવાઈડરનો ઉપયોગ કરે છે.

\textbf{સર્કિટ ડાયાગ્રામ:}

\includegraphics[width=1\linewidth,height=\textheight,keepaspectratio]{mermaid-1dd86190.pdf}

\begin{itemize}
\tightlist
\item
  \textbf{ફીડબેક}: કેપેસિટિવ વોલ્ટેજ ડિવાઈડર (C1, C2) દ્વારા પ્રદાન કરવામાં આવે છે
\item
  \textbf{રેઝોનન્ટ ફ્રિક્વન્સી}: f = 1/(2π\sqrt(L\timesC)), જ્યાં C = (C1\timesC2)/(C1+C2)
\item
  \textbf{ઓસિલેશન}: રિજનરેટિવ ફીડબેક દ્વારા જાળવી રાખે છે
\item
  \textbf{ફેઝ શિફ્ટ}: લૂપની આસપાસ 360^\circ
\end{itemize}

\end{solutionbox}
\begin{mnemonicbox}
``CFPO'' (Capacitive Feedback Produces Oscillations)

\end{mnemonicbox}
\subsection*{પ્રશ્ન 2(ક) [7
ગુણ]}\label{uxaaauxab0uxab6uxaa8-2uxa95-7-uxa97uxaa3}

\textbf{ટુ સ્ટેજ RC કપલ્ડ એમ્પ્લીફાયરનો ફ્રિક્વન્સી રિસ્પોન્સ સર્કિટ ડાયાગ્રામ સાથે
સમજાવો.}

\begin{solutionbox}
બે-સ્ટેજ RC કપલ્ડ એમ્પ્લિફાયર બે એમ્પ્લિફાયર સ્ટેજને RC કપલિંગ સાથે જોડે
છે.

\textbf{સર્કિટ ડાયાગ્રામ:}

\includegraphics[width=1\linewidth,height=\textheight,keepaspectratio]{mermaid-7a80a75f.pdf}

\textbf{ફ્રિક્વન્સી રિસ્પોન્સ:}

\includegraphics[width=1\linewidth,height=\textheight,keepaspectratio]{mermaid-cc4eb31f.pdf}

\begin{itemize}
\tightlist
\item
  \textbf{લો ફ્રિક્વન્સી}: કપલિંગ કેપેસિટર ઇમ્પિડન્સને કારણે ગેઇન ઘટે છે
\item
  \textbf{મિડ ફ્રિક્વન્સી}: મહત્તમ ફ્લેટ ગેઇન રીજિયન (બેન્ડવિડ્થ)
\item
  \textbf{હાઇ ફ્રિક્વન્સી}: ટ્રાન્ઝિસ્ટર કેપેસિટન્સ ઇફેક્ટ્સને કારણે ગેઇન ઘટે છે
\item
  \textbf{ઓવરઓલ ગેઇન}: વ્યક્તિગત સ્ટેજ ગેઇનનો ગુણાકાર
\end{itemize}

\end{solutionbox}
\begin{mnemonicbox}
``LMH'' (Low drops, Mid flat, High drops)

\end{mnemonicbox}
\subsection*{પ્રશ્ન 2(અ) OR [3
ગુણ]}\label{uxaaauxab0uxab6uxaa8-2uxa85-or-3-uxa97uxaa3}

\textbf{હાર્ટલી ઓસિલેટરનું સર્કિટ ડાયાગ્રામ દોરો.}

\begin{solutionbox}

\textbf{હાર્ટલી ઓસિલેટરનું સર્કિટ ડાયાગ્રામ:}

\includegraphics[width=1\linewidth,height=\textheight,keepaspectratio]{mermaid-e3a6dd60.pdf}

\end{solutionbox}
\begin{mnemonicbox}
``ITLC'' (Inductor Tapped for LC Circuit)

\end{mnemonicbox}
\subsection*{પ્રશ્ન 2(બ) OR [4
ગુણ]}\label{uxaaauxab0uxab6uxaa8-2uxaac-or-4-uxa97uxaa3}

\textbf{વિવિધ પ્રકારના નેગેટીવ ફીડબેકનું લિસ્ટ બનાવો.}

\begin{solutionbox}


{\def\LTcaptype{none} % do not increment counter
\vspace{-5pt}
\captionof{table}{નેગેટિવ ફીડબેકના પ્રકારો}
\vspace{-10pt}
\begin{longtable}[]{@{}
  >{\raggedright\arraybackslash}p{(\linewidth - 4\tabcolsep) * \real{0.1429}}
  >{\raggedright\arraybackslash}p{(\linewidth - 4\tabcolsep) * \real{0.3571}}
  >{\raggedright\arraybackslash}p{(\linewidth - 4\tabcolsep) * \real{0.5000}}@{}}
\toprule\noalign{}
\begin{minipage}[b]{\linewidth}\raggedright
પ્રકાર
\end{minipage} & \begin{minipage}[b]{\linewidth}\raggedright
કન્ફિગરેશન
\end{minipage} & \begin{minipage}[b]{\linewidth}\raggedright
પેરામીટર્સ પર અસર
\end{minipage} \\
\midrule\noalign{}
\endhead
\bottomrule\noalign{}
\endlastfoot
\textbf{વોલ્ટેજ સીરીઝ} & આઉટપુટ વોલ્ટેજ ઇનપુટમાં સીરીઝમાં ફીડ થાય છે & ઇનપુટ
ઇમ્પેડન્સમાં વધારો, ડિસ્ટોર્શનમાં ઘટાડો \\
\textbf{વોલ્ટેજ શન્ટ} & આઉટપુટ વોલ્ટેજ ઇનપુટમાં પેરેલલમાં ફીડ થાય છે & ઇનપુટ ઇમ્પેડન્સમાં
ઘટાડો, બેન્ડવિડ્થમાં વધારો \\
\textbf{કરંટ સીરીઝ} & આઉટપુટ કરંટ ઇનપુટમાં સીરીઝમાં ફીડ થાય છે & આઉટપુટ ઇમ્પેડન્સમાં
વધારો, કરંટ ગેઇનને સ્થિર કરે છે \\
\textbf{કરંટ શન્ટ} & આઉટપુટ કરંટ ઇનપુટમાં પેરેલલમાં ફીડ થાય છે & આઉટપુટ ઇમ્પેડન્સમાં
ઘટાડો, વોલ્ટેજ ગેઇનને સ્થિર કરે છે \\
\end{longtable}
}

\end{solutionbox}
\begin{mnemonicbox}
``VSCS'' (Voltage Series, Current Shunt)

\end{mnemonicbox}
\subsection*{પ્રશ્ન 2(ક) OR [7
ગુણ]}\label{uxaaauxab0uxab6uxaa8-2uxa95-or-7-uxa97uxaa3}

\textbf{નેગેટિવ ફીડબેક એમ્પ્લીફાયરના ફાયદાઓની યાદી બનાવો અને વોલ્ટેજ સીરીઝ નેગેટિવ
ફીડબેકને વિગતવાર સમજાવો.}

\begin{solutionbox}

\textbf{નેગેટિવ ફીડબેકના ફાયદાઓ:} - કોમ્પોનન્ટ વેરિએશન સામે ગેઇન સ્થિર કરે છે -
ડિસ્ટોર્શન અને નોઇઝમાં ઘટાડો - બેન્ડવિડ્થમાં વધારો - ઇનપુટ/આઉટપુટ ઇમ્પેડન્સમાં ફેરફાર
કરે છે - લિનિયારિટીમાં સુધારો

\textbf{વોલ્ટેજ સીરીઝ નેગેટિવ ફીડબેક:}

\includegraphics[width=1\linewidth,height=\textheight,keepaspectratio]{mermaid-15c46489.pdf}

\begin{itemize}
\tightlist
\item
  \textbf{કન્ફિગરેશન}: આઉટપુટ વોલ્ટેજ સેમ્પલ કરવામાં આવે છે, ઇનપુટમાં સીરીઝમાં ફીડ બેક
  કરવામાં આવે છે
\item
  \textbf{ક્લોઝ્ડ-લૂપ ગેઇન}: ACL = A/(1+Aβ), જ્યાં A ઓપન-લૂપ ગેઇન છે અને β ફીડબેક
  ફ્રેક્શન છે
\item
  \textbf{ઇનપુટ ઇમ્પેડન્સ}: ફેક્ટર (1+Aβ) દ્વારા વધે છે
\item
  \textbf{આઉટપુટ ઇમ્પેડન્સ}: ફેક્ટર (1+Aβ) દ્વારા ઘટે છે
\end{itemize}

\end{solutionbox}
\begin{mnemonicbox}
``SIGO'' (Stable gain, Increased input impedance,
Gain reduction, Output impedance reduction)

\end{mnemonicbox}
\subsection*{પ્રશ્ન 3(અ) [3
ગુણ]}\label{uxaaauxab0uxab6uxaa8-3uxa85-3-uxa97uxaa3}

\textbf{બે ટ્રાન્ઝિસ્ટર એનેલોજીનો ઉપયોગ કરીને SCRની સર્કિટ દોરો.}

\begin{solutionbox}

\textbf{SCRનું બે ટ્રાન્ઝિસ્ટર એનેલોજી:}

\includegraphics[width=1\linewidth,height=\textheight,keepaspectratio]{mermaid-42adc0a9.pdf}

\end{solutionbox}
\begin{mnemonicbox}
``PNPNPN'' (PNP and NPN structure)

\end{mnemonicbox}
\subsection*{પ્રશ્ન 3(બ) [4
ગુણ]}\label{uxaaauxab0uxab6uxaa8-3uxaac-4-uxa97uxaa3}

\textbf{SCR ના નેચરલ કમ્યુટેશન સર્કિટ દોરી ને સમજાવો.}

\begin{solutionbox}
નેચરલ કમ્યુટેશન ત્યારે થાય છે જ્યારે SCR કરંટ કુદરતી રીતે હોલ્ડિંગ
કરંટથી નીચે પડે છે.

\textbf{સર્કિટ ડાયાગ્રામ:}

\includegraphics[width=1\linewidth,height=\textheight,keepaspectratio]{mermaid-19cb68a7.pdf}

\textbf{કરંટ વેવફોર્મ:}

\begin{lstlisting}
       ┌───┐     ┌───┐
       │   │     │   │
───────┘   └─────┘   └─────
  SCR OFF    SCR OFF
       SCR ON    SCR ON
\end{lstlisting}

\begin{itemize}
\tightlist
\item
  \textbf{વ્યાખ્યા}: કરંટ હોલ્ડિંગ કરંટથી નીચે પડે ત્યારે SCR આપોઆપ બંધ થાય છે
\item
  \textbf{AC સર્કિટ}: દરેક પોઝિટિવ હાફ-સાયકલના અંતે કુદરતી રીતે થાય છે
\item
  \textbf{ઝીરો ક્રોસિંગ}: AC વોલ્ટેજ શૂન્ય ક્રોસ કરે ત્યારે SCR બંધ થાય છે
\item
  \textbf{કોઈ બાહ્ય સર્કિટ નથી}: ટર્ન-ઓફ માટે કોઈ વધારાના કોમ્પોનન્ટની જરૂર નથી
\end{itemize}

\end{solutionbox}
\begin{mnemonicbox}
``NAZC'' (Natural At Zero Crossing)

\end{mnemonicbox}
\subsection*{પ્રશ્ન 3(ક) [7
ગુણ]}\label{uxaaauxab0uxab6uxaa8-3uxa95-7-uxa97uxaa3}

\textbf{ટ્રાયાકનો ઉપયોગ પંખાના રેગ્યુલેટર તરીકે અને એસી પાવર માટે ઓન-ઓફ કંટ્રોલ
તરીકે કેવી રીતે થઈ શકે છે તે સમજાવો.}

\begin{solutionbox}
TRIAC એ બાયડાયરેક્શનલ ડિવાઇસ છે જે AC પાવર કંટ્રોલ એપ્લિકેશન માટે
આદર્શ છે.

\textbf{TRIAC ફેન રેગ્યુલેટર સર્કિટ:}

\includegraphics[width=1\linewidth,height=\textheight,keepaspectratio]{mermaid-2e80ee38.pdf}

\textbf{TRIAC ઓન-ઓફ કંટ્રોલ:}

\includegraphics[width=1\linewidth,height=\textheight,keepaspectratio]{mermaid-eb8823b5.pdf}

\begin{itemize}
\tightlist
\item
  \textbf{ફેન રેગ્યુલેશન}: ફેઝ કંટ્રોલ ટેકનિક ફેનમાં પાવર વેરી કરે છે
\item
  \textbf{પોટેન્શિયોમીટર}: TRIACનો ફાયરિંગ એંગલ એડજસ્ટ કરે છે
\item
  \textbf{ઓન-ઓફ કંટ્રોલ}: સરળ સ્વિચ TRIAC ગેટને ટ્રિગર કરે છે
\item
  \textbf{બાયડાયરેક્શનલ}: બંને હાફ-સાયકલમાં કરંટ કંટ્રોલ કરે છે
\end{itemize}

\end{solutionbox}
\begin{mnemonicbox}
``FPOB'' (Fan Power is controlled by Phase angle in
both directions)

\end{mnemonicbox}
\subsection*{પ્રશ્ન 3(અ) OR [3
ગુણ]}\label{uxaaauxab0uxab6uxaa8-3uxa85-or-3-uxa97uxaa3}

\textbf{એસ.સી.આર, ડાયાક અને ટ્રાયાક ના સિમ્બોલ દોરો.}

\begin{solutionbox}

\textbf{થાઇરિસ્ટરના સિમ્બોલ:}

\begin{lstlisting}
    SCR            DIAC           TRIAC
    
    A              
    |              
   ┌┴┐            ┌─┐            ┌─┐
   │ │            │ │            │ │
   └┬┘            └─┘            └─┘
    │              |              |
    │              |              |
    G─┐            |              G─┐
      │            |                │
    K |            |                |
\end{lstlisting}

\end{solutionbox}
\begin{mnemonicbox}
``SDT'' (SCR has gate on one side, DIAC has none,
TRIAC has gate in middle)

\end{mnemonicbox}
\subsection*{પ્રશ્ન 3(બ) OR [4
ગુણ]}\label{uxaaauxab0uxab6uxaa8-3uxaac-or-4-uxa97uxaa3}

\textbf{એસ.સી.આર નુ ગેટ ટ્રીગરીંગ સર્કિટ દોરી ને સમજાવો.}

\begin{solutionbox}
ગેટ ટ્રિગરિંગ એ SCRને ચાલુ કરવાની સૌથી સામાન્ય પદ્ધતિ છે.

\textbf{સર્કિટ ડાયાગ્રામ:}

\includegraphics[width=1\linewidth,height=\textheight,keepaspectratio]{mermaid-4a2a1c7e.pdf}

\begin{itemize}
\tightlist
\item
  \textbf{સિદ્ધાંત}: ગેટ અને કેથોડ વચ્ચે પોઝિટિવ વોલ્ટેજ એપ્લાય કરવું
\item
  \textbf{કરંટ જરૂરિયાત}: નાનો ગેટ કરંટ મોટા એનોડ કરંટને ટ્રિગર કરે છે
\item
  \textbf{લેચિંગ}: એકવાર ટ્રિગર થયા પછી, ગેટ સિગ્નલ દૂર કરવામાં આવે તો પણ SCR
  ચાલુ રહે છે
\item
  \textbf{ટર્ન-ઓફ}: એનોડ કરંટને હોલ્ડિંગ કરંટથી નીચે ઘટાડવાની જરૂર પડે છે
\end{itemize}

\end{solutionbox}
\begin{mnemonicbox}
``GPLT'' (Gate Pulse Latches Thyristor)

\end{mnemonicbox}
\subsection*{પ્રશ્ન 3(ક) OR [7
ગુણ]}\label{uxaaauxab0uxab6uxaa8-3uxa95-or-7-uxa97uxaa3}

\textbf{SCRનું કંસ્ટ્રકશન અને V-I લાક્ષણિકતા દોરો અને V-I લાક્ષણિકતા સમજાવો.}

\begin{solutionbox}
SCR (સિલિકોન કંટ્રોલ્ડ રેક્ટિફાયર) એ ચાર-લેયર PNPN સેમિકન્ડક્ટર
ડિવાઇસ છે.

\textbf{SCR કંસ્ટ્રકશન:}

\includegraphics[width=1\linewidth,height=\textheight,keepaspectratio]{mermaid-1dc697b9.pdf}

\textbf{V-I લાક્ષણિકતા:}

\begin{lstlisting}
          I
          ↑
          │        ON State
          │       ┌────────
          │       │
          │       │
  Holding │       │
  current ├───────┤
          │       │
          │Forward│
          │breakover
          │voltage│
          │       │
          └───────┴──────\rightarrow V
                   Reverse
                   breakdown
                   voltage
\end{lstlisting}

\begin{itemize}
\tightlist
\item
  \textbf{ફોરવર્ડ બ્લોકિંગ રીજન}: બ્રેકઓવર વોલ્ટેજ સુધી SCR મિનિમલ કરંટ કન્ડક્ટ કરે
  છે
\item
  \textbf{ફોરવર્ડ કન્ડક્શન રીજન}: ટ્રિગરિંગ પછી લો-રેઝિસ્ટન્સ સ્ટેટ
\item
  \textbf{રિવર્સ બ્લોકિંગ રીજન}: રિવર્સ દિશામાં કરંટને બ્લોક કરે છે
\item
  \textbf{ગેટ ટ્રિગરિંગ}: બ્રેકઓવર વોલ્ટેજને ઘટાડે છે, ટર્ન-ઓનને સરળ બનાવે છે
\end{itemize}

\end{solutionbox}
\begin{mnemonicbox}
``FBRH'' (Forward Blocking, Reverse blocking,
Holding current)

\end{mnemonicbox}
\subsection*{પ્રશ્ન 4(અ) [3
ગુણ]}\label{uxaaauxab0uxab6uxaa8-4uxa85-3-uxa97uxaa3}

\textbf{OP-AMP ને સમિંગ એમ્પ્લીફાયર તરીકે સમજાવો.}

\begin{solutionbox}
સમિંગ એમ્પ્લિફાયર વેઇટેડ ગેઇન સાથે મલ્ટિપલ ઇનપુટ સિગ્નલ્સ એડ કરે છે.

\textbf{સર્કિટ ડાયાગ્રામ:}

\includegraphics[width=1\linewidth,height=\textheight,keepaspectratio]{mermaid-9f06110a.pdf}

\begin{itemize}
\tightlist
\item
  \textbf{ફંક્શન}: ઇનપુટ વોલ્ટેજનો વેઇટેડ સમ આઉટપુટ કરે છે
\item
  \textbf{આઉટપુટ સમીકરણ}: Vout = -(V1\timesRf/R1 + V2\timesRf/R2 + V3\timesRf/R3)
\item
  \textbf{સમાન ભાર}: જ્યારે R1 = R2 = R3, આઉટપુટ સરળ સમ ગુણાકાર -Rf/R છે
\item
  \textbf{વર્ચ્યુઅલ ગ્રાઉન્ડ}: ઈન્વર્ટિંગ ઇનપુટ 0V પોટેન્શિયલ જાળવે છે
\end{itemize}

\end{solutionbox}
\begin{mnemonicbox}
``SWAP'' (Sum Weighted And Proportional)

\end{mnemonicbox}
\subsection*{પ્રશ્ન 4(બ) [4
ગુણ]}\label{uxaaauxab0uxab6uxaa8-4uxaac-4-uxa97uxaa3}

\textbf{નીચેના OP-AMP પેરામીટરને વ્યાખ્યાયિત કરો: 1. ઇનપુટ બાયસ કરંટ 2. CMRR}

\begin{solutionbox}

\textbf{ઇનપુટ બાયસ કરંટ}: જ્યારે આઉટપુટ શૂન્ય હોય ત્યારે બે ઇનપુટ ટર્મિનલમાં પ્રવાહિત
થતા કરંટની સરેરાશ.

\textbf{CMRR (કોમન મોડ રિજેક્શન રેશિયો)}: ડિફરેન્શિયલ ગેઇનનો કોમન-મોડ ગેઇન
સાથેનો ગુણોત્તર, જે દર્શાવે છે કે ઓપ-એમ્પ બંને ઇનપુટ માટે સામાન્ય સિગ્નલને કેટલી સારી રીતે
રિજેક્ટ કરે છે.


{\def\LTcaptype{none} % do not increment counter
\vspace{-5pt}
\captionof{table}{ઓપ-એમ્પ પેરામીટર્સ}
\vspace{-10pt}
\begin{longtable}[]{@{}lll@{}}
\toprule\noalign{}
પેરામીટર & સામાન્ય મૂલ્ય & મહત્વ \\
\midrule\noalign{}
\endhead
\bottomrule\noalign{}
\endlastfoot
ઇનપુટ બાયસ કરંટ & 20-200 nA & હાઈ ઇમ્પિડન્સ સર્કિટ માટે ઓછું વધુ સારું \\
CMRR & 80-120 dB & નોઇઝ રિજેક્શન માટે વધુ સારું \\
\end{longtable}
}

\end{solutionbox}
\begin{mnemonicbox}
``BIC-CMR'' (Bias Is Current, Common Mode Rejection)

\end{mnemonicbox}
\subsection*{પ્રશ્ન 4(ક) [7
ગુણ]}\label{uxaaauxab0uxab6uxaa8-4uxa95-7-uxa97uxaa3}

\textbf{555 ટાઈમરનો ઉપયોગ કરીને મોનોસ્ટેબલ મલ્ટિવાઇબ્રેટર દોરો અને સમજાવો.}

\begin{solutionbox}
મોનોસ્ટેબલ મલ્ટીવાઇબ્રેટર ટ્રિગર થતાં પૂર્વનિર્ધારિત અવધિનો એક પલ્સ
જનરેટ કરે છે.

\textbf{સર્કિટ ડાયાગ્રામ:}

\includegraphics[width=1\linewidth,height=\textheight,keepaspectratio]{mermaid-c83e0a29.pdf}

\textbf{આઉટપુટ વેવફોર્મ:}

\begin{lstlisting}
Trigger  ___┐      ____________
             │______│
             
Output   ____┌──────┐__________
              │      │
              T = 1.1RC
\end{lstlisting}

\begin{itemize}
\tightlist
\item
  \textbf{ઓપરેશન}: સિંગલ સ્ટેબલ સ્ટેટ (આઉટપુટ LOW), ટ્રિગર થતાં અસ્થાયી રૂપે HIGH
\item
  \textbf{પલ્સ વિડ્થ}: T = 1.1 \times R \times C (સેકન્ડ)
\item
  \textbf{ટ્રિગરિંગ}: TRIG પિન (પિન 2) પર ફોલિંગ એજ
\item
  \textbf{ટાઇમિંગ કોમ્પોનન્ટ્સ}: R અને C પલ્સ અવધિ નક્કી કરે છે
\end{itemize}

\end{solutionbox}
\begin{mnemonicbox}
``POST'' (Pulse Output, Single Trigger)

\end{mnemonicbox}
\subsection*{પ્રશ્ન 4(અ) OR [3
ગુણ]}\label{uxaaauxab0uxab6uxaa8-4uxa85-or-3-uxa97uxaa3}

\textbf{OP-AMP ના ઇન્વર્ટિંગ એમ્પ્લીફાયરનો સર્કિટ ડાયાગ્રામને દોરો.}

\begin{solutionbox}

\textbf{ઇન્વર્ટિંગ એમ્પ્લિફાયર સર્કિટ:}

\includegraphics[width=1\linewidth,height=\textheight,keepaspectratio]{mermaid-421cae7c.pdf}

\end{solutionbox}
\begin{mnemonicbox}
``IRON'' (Inverting Requires One Negative input)

\end{mnemonicbox}
\subsection*{પ્રશ્ન 4(બ) OR [4
ગુણ]}\label{uxaaauxab0uxab6uxaa8-4uxaac-or-4-uxa97uxaa3}

\textbf{નીચેના OP-AMP પેરામીટરને વ્યાખ્યાયિત કરો: 1. ઇનપુટ ઓફસેટ કરંટ 2. સ્લ્યુ રેટ}

\begin{solutionbox}

\textbf{ઇનપુટ ઓફસેટ કરંટ}: બે ઇનપુટ ટર્મિનલમાં પ્રવાહિત થતા કરંટ વચ્ચેનો તફાવત.

\textbf{સ્લ્યુ રેટ}: આઉટપુટ વોલ્ટેજનો સમય પ્રતિ એકમ મહત્તમ ફેરફાર દર, સામાન્ય રીતે
V/μs માં માપવામાં આવે છે.


{\def\LTcaptype{none} % do not increment counter
\vspace{-5pt}
\captionof{table}{ઓપ-એમ્પ પેરામીટર્સ}
\vspace{-10pt}
\begin{longtable}[]{@{}lll@{}}
\toprule\noalign{}
પેરામીટર & સામાન્ય મૂલ્ય & મહત્વ \\
\midrule\noalign{}
\endhead
\bottomrule\noalign{}
\endlastfoot
ઇનપુટ ઓફસેટ કરંટ & 2-50 nA & પ્રિસિઝન એપ્લિકેશન માટે ઓછું વધુ સારું \\
સ્લ્યુ રેટ & 0.5-20 V/μs & હાઈ-ફ્રિક્વન્સી ઓપરેશન માટે વધુ સારું \\
\end{longtable}
}

\end{solutionbox}
\begin{mnemonicbox}
``IOSR'' (Input Offset and Slew Rate)

\end{mnemonicbox}
\subsection*{પ્રશ્ન 4(ક) OR [7
ગુણ]}\label{uxaaauxab0uxab6uxaa8-4uxa95-or-7-uxa97uxaa3}

\textbf{ઑપ-એમ્પને ઇન્વર્ટિંગ એમ્પ્લીફાયર તરીકે સમજાવો અને તેના વોલ્ટેજ ગેઇનનું સમીકરણ
મેળવો.}

\begin{solutionbox}
ઇન્વર્ટિંગ એમ્પ્લિફાયર એક ઇન્વર્ટેડ અને એમ્પ્લિફાઇડ આઉટપુટ સિગ્નલ
ઉત્પન્ન કરે છે.

\textbf{સર્કિટ ડાયાગ્રામ:}

\includegraphics[width=1\linewidth,height=\textheight,keepaspectratio]{mermaid-421cae7c.pdf}

\textbf{વોલ્ટેજ ગેઇન ડેરિવેશન:}

\begin{lstlisting}
નોડ N (ઇન્વર્ટિંગ ઇનપુટ) પર:
I1 + If = 0  (કિરકોફનો કરંટ લો દ્વારા)
(Vin - VN)/R1 + (Vout - VN)/Rf = 0

જ્યારે VN \approx 0 (વર્ચ્યુઅલ ગ્રાઉન્ડ):
Vin/R1 + Vout/Rf = 0
Vout/Vin = -Rf/R1
\end{lstlisting}

\begin{itemize}
\tightlist
\item
  \textbf{ગેઇન સમીકરણ}: Vout/Vin = -Rf/R1
\item
  \textbf{વર્ચ્યુઅલ ગ્રાઉન્ડ}: ઇન્વર્ટિંગ ટર્મિનલ 0V પર જાળવવામાં આવે છે
\item
  \textbf{ઇનપુટ ઇમ્પિડન્સ}: R1 ને સમાન
\item
  \textbf{નેગેટિવ ફીડબેક}: સ્થિરતા અને લિનિયારિટી પ્રદાન કરે છે
\end{itemize}

\end{solutionbox}
\begin{mnemonicbox}
``GIVN'' (Gain Is Negative, Virtual ground)

\end{mnemonicbox}
\subsection*{પ્રશ્ન 5(અ) [3
ગુણ]}\label{uxaaauxab0uxab6uxaa8-5uxa85-3-uxa97uxaa3}

\textbf{IC 555 નો બ્લોક ડાયાગ્રામ દોરો.}

\begin{solutionbox}

\textbf{IC 555નો બ્લોક ડાયાગ્રામ:}

\includegraphics[width=1\linewidth,height=\textheight,keepaspectratio]{mermaid-c933de86.pdf}

\end{solutionbox}
\begin{mnemonicbox}
``CVOT'' (Comparators, Voltage divider, Output
stage, Timer)

\end{mnemonicbox}
\subsection*{પ્રશ્ન 5(બ) [4
ગુણ]}\label{uxaaauxab0uxab6uxaa8-5uxaac-4-uxa97uxaa3}

\textbf{વેઈન બ્રિજ ઓસીલેટર તરીકે OP-AMPનો સર્કિટ ડાયાગ્રામ દોરો.}

\begin{solutionbox}

\textbf{વેઈન બ્રિજ ઓસીલેટર સર્કિટ:}

\includegraphics[width=1\linewidth,height=\textheight,keepaspectratio]{mermaid-9492b311.pdf}

\end{solutionbox}
\begin{mnemonicbox}
``WPRC'' (Wein Produces Resonant Circuit)

\end{mnemonicbox}
\subsection*{પ્રશ્ન 5(ક) [7
ગુણ]}\label{uxaaauxab0uxab6uxaa8-5uxa95-7-uxa97uxaa3}

\textbf{વિવિધ પ્રકારના ફિક્સ્ડ અને વેરિયેબલ વોલ્ટેજ રેગ્યુલેટર IC ની કામગીરી
સમજાવો.}

\begin{solutionbox}
વોલ્ટેજ રેગ્યુલેટર IC ઇનપુટ અથવા લોડ વેરિએશન છતાં સ્થિર આઉટપુટ વોલ્ટેજ
જાળવે છે.

\textbf{ફિક્સ્ડ વોલ્ટેજ રેગ્યુલેટર:}

\includegraphics[width=1\linewidth,height=\textheight,keepaspectratio]{mermaid-019a3ed5.pdf}

\textbf{વેરિએબલ વોલ્ટેજ રેગ્યુલેટર:}

\includegraphics[width=1\linewidth,height=\textheight,keepaspectratio]{mermaid-cae7b6bc.pdf}

\begin{itemize}
\tightlist
\item
  \textbf{ફિક્સ્ડ રેગ્યુલેટર}: 78XX (પોઝિટિવ) અને 79XX (નેગેટિવ) સીરીઝ ચોક્કસ
  વોલ્ટેજ પ્રદાન કરે છે
\item
  \textbf{વેરિએબલ રેગ્યુલેટર}: LM317 (પોઝિટિવ) અને LM337 (નેગેટિવ) એડજસ્ટેબલ
  આઉટપુટની મંજૂરી આપે છે
\item
  \textbf{થ્રી-ટર્મિનલ ડિઝાઇન}: ઇનપુટ, આઉટપુટ અને ગ્રાઉન્ડ/એડજસ્ટ ટર્મિનલ
\item
  \textbf{LM317 માટે આઉટપુટ સમીકરણ}: Vout = 1.25V \times (1 + R2/R1)
\item
  \textbf{પ્રોટેક્શન ફીચર્સ}: શોર્ટ સર્કિટ, થર્મલ ઓવરલોડ અને સેફ એરિયા પ્રોટેક્શન
\end{itemize}

\end{solutionbox}
\begin{mnemonicbox}
``FAVOR'' (Fixed And Variable Output Regulators)

\end{mnemonicbox}
\subsection*{પ્રશ્ન 5(અ) OR [3
ગુણ]}\label{uxaaauxab0uxab6uxaa8-5uxa85-or-3-uxa97uxaa3}

\textbf{555 ટાઈમરનો ઉપયોગ કરીને એસ્ટેબલ મલ્ટિવાઈબ્રેટરનો બ્લોક ડાયાગ્રામ દોરો.}

\begin{solutionbox}

\textbf{એસ્ટેબલ મલ્ટિવાઇબ્રેટર બ્લોક ડાયાગ્રામ:}

\includegraphics[width=1\linewidth,height=\textheight,keepaspectratio]{mermaid-05ac2a1d.pdf}

\end{solutionbox}
\begin{mnemonicbox}
``FOFT'' (Free-running Oscillator From Timer)

\end{mnemonicbox}
\subsection*{પ્રશ્ન 5(બ) OR [4
ગુણ]}\label{uxaaauxab0uxab6uxaa8-5uxaac-or-4-uxa97uxaa3}

\textbf{સૌર આધારિત બેટરી ચાર્જર સર્કિટ દોરો અને સમજાવો.}

\begin{solutionbox}
સોલર બેટરી ચાર્જર સૂર્ય ઊર્જાને બેટરી ચાર્જ કરવા માટે રૂપાંતરિત કરે
છે.

\textbf{સર્કિટ ડાયાગ્રામ:}

\includegraphics[width=1\linewidth,height=\textheight,keepaspectratio]{mermaid-09c6da6e.pdf}

\begin{itemize}
\tightlist
\item
  \textbf{સોલર પેનલ}: સૂર્યપ્રકાશને DC વીજળીમાં રૂપાંતરિત કરે છે
\item
  \textbf{બ્લોકિંગ ડાયોડ}: રાત્રે પેનલ દ્વારા બેટરી ડિસ્ચાર્જને અટકાવે છે
\item
  \textbf{રેગ્યુલેટર IC}: ચાર્જિંગ વોલ્ટેજ અને કરંટને નિયંત્રિત કરે છે
\item
  \textbf{ચાર્જ ઇન્ડિકેટર}: ચાર્જિંગની સ્થિતિ દર્શાવે છે
\item
  \textbf{પ્રોટેક્શન}: ઓવરચાર્જ અને રિવર્સ પોલારિટી પ્રોટેક્શન
\end{itemize}

\end{solutionbox}
\begin{mnemonicbox}
``SBRCP'' (Solar, Blocking diode, Regulator,
Charging, Protection)

\end{mnemonicbox}
\subsection*{પ્રશ્ન 5(ક) OR [7
ગુણ]}\label{uxaaauxab0uxab6uxaa8-5uxa95-or-7-uxa97uxaa3}

\textbf{SMPS ના બ્લોક ડાયાગ્રામ દોરો અને સમજાવો}

\begin{solutionbox}
SMPS (સ્વિચ મોડ પાવર સપ્લાય) સ્વિચિંગ રેગ્યુલેટર્સનો ઉપયોગ કરીને
વીજળી શક્તિને કુશળતાથી રૂપાંતરિત કરે છે.

\textbf{બ્લોક ડાયાગ્રામ:}

\includegraphics[width=1\linewidth,height=\textheight,keepaspectratio]{mermaid-c8481bc0.pdf}

\begin{itemize}
\tightlist
\item
  \textbf{EMI ફિલ્ટર}: AC ઇનપુટમાંથી નોઇઝ દૂર કરે છે
\item
  \textbf{રેક્ટિફાયર}: AC ને અનરેગ્યુલેટેડ DC માં રૂપાંતરિત કરે છે
\item
  \textbf{સ્વિચિંગ સર્કિટ}: DC ને ઉચ્ચ ફ્રિક્વન્સી (20-100 kHz) પર ચોપ કરે છે
\item
  \textbf{ટ્રાન્સફોર્મર}: આઇસોલેશન અને વોલ્ટેજ રૂપાંતરણ પ્રદાન કરે છે
\item
  \textbf{આઉટપુટ રેક્ટિફાયર}: હાઈ-ફ્રિક્વન્સી AC ને ફરીથી DC માં કન્વર્ટ કરે છે
\item
  \textbf{આઉટપુટ ફિલ્ટર}: DC આઉટપુટને સ્મૂથ કરે છે
\item
  \textbf{ફીડબેક સર્કિટ}: રેગ્યુલેશન માટે આઉટપુટનું મોનિટરિંગ કરે છે
\item
  \textbf{કંટ્રોલ સર્કિટ}: ફીડબેકના આધારે સ્વિચિંગ એડજસ્ટ કરે છે
\end{itemize}

\end{solutionbox}
\begin{mnemonicbox}
``ERST-FOFC'' (EMI filter, Rectifier, Switching,
Transformer, Feedback, Output rectifier, Filter, Control)

\end{mnemonicbox}

\end{document}
