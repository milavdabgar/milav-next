\documentclass{article}

% content/resources/templates/preamble.tex
\usepackage[margin=0.6in]{geometry}
\author{Milav Dabgar}
\usepackage{amsmath,amssymb,amsthm}
\usepackage{booktabs}
\usepackage{multirow}
\usepackage{xcolor}
\usepackage{tcolorbox}
\tcbuselibrary{breakable,skins}
\usepackage[colorlinks=true,linkcolor=blue]{hyperref}
\usepackage{titlesec}
\usepackage{enumitem}
\usepackage{tikz}
\usepackage{pgfplots}
\usepackage{circuitikz}
\usepackage[version=4]{mhchem}
\usepackage{longtable}
\usepackage{array}
\usepackage{float}
\usepackage{caption}
\usepackage{listings}

\lstset{
  basicstyle=\small\ttfamily,
  breaklines=true,
  breakatwhitespace=false,
  postbreak=\mbox{\textcolor{red}{$\hookrightarrow$}\space},
  float=false,
  numbers=left,
  numberstyle=\tiny\color{gray},
  numbersep=10pt,
  xleftmargin=2em,
  keywordstyle=\color{blue},
  commentstyle=\color{green!60!black},
  stringstyle=\color{purple},
  backgroundcolor=\color{gray!5},
  showstringspaces=false,
  tabsize=2,
  captionpos=b,
  keepspaces=true,
  columns=flexible
}

\pgfplotsset{compat=1.18}
\usetikzlibrary{shapes,arrows,positioning,calc,patterns,decorations.pathmorphing,decorations.markings,arrows.meta}

% Color scheme
\definecolor{headcolor}{RGB}{0,102,204}
\definecolor{keycolor}{RGB}{220,20,60}
\definecolor{solutioncolor}{RGB}{34,139,34}
\definecolor{mnemoniccolor}{RGB}{148,0,211}
\definecolor{codecolor}{RGB}{0,0,100}

% Spacing
\setlength{\parskip}{3pt}
\setlist[itemize]{nosep}
\setlist[enumerate]{nosep}

% Title formatting
\titleformat{\section}{\Large\bfseries\color{headcolor}}{\thesection}{1em}{}
\titleformat{\subsection}{\large\bfseries\color{headcolor}}{\thesubsection}{1em}{}

% Pandoc tightlist compatibility
\providecommand{\tightlist}{%
  \setlength{\itemsep}{0pt}\setlength{\parskip}{0pt}}

% Pandoc longtable compatibility
\newcounter{none}
\def\thenone{}


% content/resources/templates/english-boxes.tex

% Custom environments
\newtcolorbox{solutionbox}{
 breakable,
 enhanced,
 colback=solutioncolor!5!white,
 colframe=solutioncolor!75!black,
 fonttitle=\bfseries,
 title=Solution
}

\newtcolorbox{solutionboxnobreak}{
 colback=solutioncolor!5!white,
 colframe=solutioncolor!75!black,
 fonttitle=\bfseries,
 title=Solution
}

\newtcolorbox{keyformula}{
 breakable,
 enhanced,
 colback=keycolor!5!white,
 colframe=keycolor!75!black,
 fonttitle=\bfseries,
 title=Key Formula
}

\newtcolorbox{mnemonicboxenv}{
 breakable,
 enhanced,
 colback=mnemoniccolor!5!white,
 colframe=mnemoniccolor!75!black,
 fonttitle=\bfseries,
 title=Mnemonic
}

\newcommand{\mnemonicbox}[1]{%
  \begin{mnemonicboxenv}
    #1
  \end{mnemonicboxenv}
}


% Custom commands for GTU solutions
% This file defines semantic commands for consistent formatting

% Question command with automatic formatting
\newcommand{\question}[2]{%
  \section*{Question #1}%
  \textbf{#2}%
}

% OR question variant
\newcommand{\questionor}[2]{%
  \section*{Question #1 OR}%
  \textbf{#2}%
}

% Proper table environment with caption
\newenvironment{answertable}[1]{%
  \begin{table}[htbp]
  \centering
  \caption{#1}
}{%
  \end{table}
}

% Proper figure environment for diagrams
\newenvironment{answerdiagram}[1]{%
  \begin{figure}[htbp]
  \centering
  \caption{#1}
}{%
  \end{figure}
}

% Semantic markup for key terms
\newcommand{\keyword}[1]{\textbf{#1}}
\newcommand{\code}[1]{\texttt{#1}}
\newcommand{\classname}[1]{\texttt{#1}}
\newcommand{\methodname}[1]{\texttt{#1}}

% Proper quotation marks
\newcommand{\mnemonic}[1]{``#1''}


\title{Python Programming (1323203) - Summer 2024 Solution}
\date{June 18, 2024}

\begin{document}
\maketitle

\questionmarks{1(a)}{3}{Lists the Importance of flowchart and algorithm}

\begin{solutionbox}
\begin{answertable}{Importance of Flowchart and Algorithm}
\begin{tabulary}{\linewidth}{|L|L|}
\hline
\textbf{Importance of Flowchart} & \textbf{Importance of Algorithm} \\
\hline
Visual representation of program logic & Step-by-step procedure to solve a problem \\
\hline
Easier to debug and identify errors & Language-independent solution approach \\
\hline
Helps in understanding complex processes & Serves as a foundation for programming \\
\hline
Improves communication among team members & Defines logic before coding begins \\
\hline
\end{tabulary}
\end{answertable}
\end{solutionbox}

\begin{mnemonicbox}
\mnemonic{VASE Decisions} - Visualize, Analyze, Sequence, Execute
\end{mnemonicbox}

\questionmarks{1(b)}{4}{Draw a flowchart to find the entered number is even or odd.}

\begin{solutionbox}
\begin{answerdiagram}{Flowchart to check Even or Odd number}
\begin{tikzpicture}[gtu flow]
    \node[gtu start] (start) {Start};
    \node[gtu input, below of=start] (input) {Input Number n};
    \node[gtu decision, below of=input] (dec) {n \% 2 == 0?};
    \node[gtu process, below left of=dec, xshift=-1cm, yshift=-1cm] (even) {Print Even Number};
    \node[gtu process, below right of=dec, xshift=1cm, yshift=-1cm] (odd) {Print Odd Number};
    \node[gtu stop, below of=dec, yshift=-3cm] (end) {End};

    \draw[gtu arrow] (start) -- (input);
    \draw[gtu arrow] (input) -- (dec);
    \draw[gtu arrow] (dec) -| node[above left] {Yes} (even);
    \draw[gtu arrow] (dec) -| node[above right] {No} (odd);
    \draw[gtu arrow] (even) |- (end);
    \draw[gtu arrow] (odd) |- (end);
\end{tikzpicture}
\end{answerdiagram}

\textbf{Key Steps:}
\begin{itemize}
    \item \keyword{Input collection}: Get number from user
    \item \keyword{Modulo operation}: Divide by 2 and check remainder
    \item \keyword{Conditional output}: Display result based on remainder
\end{itemize}
\end{solutionbox}

\begin{mnemonicbox}
\mnemonic{MODE} - Modulo Operation Determines Evenness
\end{mnemonicbox}

\questionmarks{1(c)}{7}{List out all Logical operators and explain each by giving python code example.}

\begin{solutionbox}
\begin{answertable}{Logical Operators}
\begin{tabulary}{\linewidth}{|l|L|L|l|}
\hline
\textbf{Operator} & \textbf{Description} & \textbf{Example} & \textbf{Output} \\
\hline
\code{and} & Returns True if both statements are true & \code{x = 5; print(x > 3 and x < 10)} & \code{True} \\
\hline
\code{or} & Returns True if one of the statements is true & \code{x = 5; print(x > 10 or x == 5)} & \code{True} \\
\hline
\code{not} & Reverse the result, returns False if result is true & \code{x = 5; print(not(x > 3))} & \code{False} \\
\hline
\end{tabulary}
\end{answertable}

\textbf{Code Example:}
\begin{lstlisting}[language=Python]
# Logical AND example
age = 25
income = 50000
print("Loan eligibility:", age > 18 and income > 30000)  # True

# Logical OR example
has_credit_card = False
has_cash = True
print("Can purchase:", has_credit_card or has_cash)  # True

# Logical NOT example
is_holiday = False
print("Should work today:", not is_holiday)  # True
\end{lstlisting}
\end{solutionbox}

\begin{mnemonicbox}
\mnemonic{AON Clarity} - And, Or, Not for logical clarity
\end{mnemonicbox}

\orquestionmarks{1(c)}{7}{Develop a Program that can calculate simple interest and compound interest on given data.}

\begin{solutionbox}
\begin{lstlisting}[language=Python]
# Program to calculate Simple and Compound Interest

# Input values
principal = float(input("Enter principal amount: "))
rate = float(input("Enter rate of interest (in %): "))
time = float(input("Enter time period (in years): "))

# Calculate Simple Interest
simple_interest = (principal * rate * time) / 100

# Calculate Compound Interest
compound_interest = principal * ((1 + rate/100) ** time - 1)

# Display results
print("Simple Interest:", round(simple_interest, 2))
print("Compound Interest:", round(compound_interest, 2))
\end{lstlisting}

\begin{keyformula}
\begin{itemize}
    \item \textbf{Simple Interest (SI)}: Principal $\times$ Rate $\times$ Time / 100
    \item \textbf{Compound Interest (CI)}: Principal $\times$ ((1 + Rate/100)\textsuperscript{Time} - 1)
\end{itemize}
\end{keyformula}
\end{solutionbox}

\begin{mnemonicbox}
\mnemonic{PRT Money Grows} - Principal, Rate, Time make money grow
\end{mnemonicbox}

\questionmarks{2(a)}{3}{Create a Program to find a minimum number among the given three numbers.}

\begin{solutionbox}
\begin{lstlisting}[language=Python]
# Program to find minimum of three numbers

# Input three numbers
num1 = float(input("Enter first number: "))
num2 = float(input("Enter second number: "))
num3 = float(input("Enter third number: "))

# Find minimum using built-in min() function
minimum = min(num1, num2, num3)

# Display result
print("Minimum number is:", minimum)
\end{lstlisting}
\end{solutionbox}

\begin{mnemonicbox}
\mnemonic{MIN Finds Least} - Minimum Is Numerically Found with Least
\end{mnemonicbox}

\questionmarks{2(b)}{4}{Define pseudocode. Write pseudocode to find Largest of three numbers x, y and z.}

\begin{solutionbox}
\begin{answertable}{Pseudocode Definition}
\begin{tabulary}{\linewidth}{|L|}
\hline
\textbf{Definition} \\
\hline
A detailed yet readable description of what a computer program must do, expressed in a formally-styled natural language rather than in a programming language. \\
\hline
\end{tabulary}
\end{answertable}

\textbf{Pseudocode for finding largest of three numbers:}
\begin{lstlisting}
BEGIN
    INPUT x, y, z
    SET largest = x
    
    IF y > largest THEN
        SET largest = y
    END IF
    
    IF z > largest THEN
        SET largest = z
    END IF
    
    OUTPUT "Largest number is: ", largest
END
\end{lstlisting}
\end{solutionbox}

\begin{mnemonicbox}
\mnemonic{PIE Writing} - Program Ideas Expressed in simple writing
\end{mnemonicbox}

\questionmarks{2(c)}{7}{Explain While loop in python with its syntax, flowchart and with python code example.}

\begin{solutionbox}
\textbf{Syntax:}
\begin{lstlisting}[language=Python]
while condition:
    # code to be executed
\end{lstlisting}

\begin{answerdiagram}{Flowchart of While Loop}
\begin{tikzpicture}[gtu flow]
    \node[gtu start] (start) {Start};
    \node[gtu process, below of=start] (init) {Initialize Variables};
    \node[gtu decision, below of=init] (cond) {Condition True?};
    \node[gtu process, right of=cond, xshift=2cm] (exec) {Execute Statements};
    \node[gtu stop, below of=cond, yshift=-1cm] (end) {End};

    \draw[gtu arrow] (start) -- (init);
    \draw[gtu arrow] (init) -- (cond);
    \draw[gtu arrow] (cond) -- node[above] {Yes} (exec);
    \draw[gtu arrow] (exec) |- (init); % Loop back to init/check - actually usually checks condition again so loop back to condition
    
    % Correcting loop back
    \draw[gtu arrow] (exec) |- (0.5,-3.5) -- (cond); % Drawing a line back to top of condition?
    % Simpler loop back
    \draw[gtu arrow] (exec) |- (cond); 
    
    % Adjusting Diagram for standard while loop representation
\end{tikzpicture}
% Retrying simpler Manual construction for While loop
\begin{tikzpicture}[gtu flow]
    \node[gtu start] (start) {Start};
    \node[gtu process, below of=start] (init) {Initialize Variables};
    \node[gtu decision, below of=init, yshift=-0.5cm] (cond) {Condition True?};
    \node[gtu process, right of=cond, xshift=2.5cm] (body) {Execute Statements};
    \node[gtu stop, below of=cond, yshift=-2cm] (end) {End};

    \draw[gtu arrow] (start) -- (init);
    \draw[gtu arrow] (init) -- (cond);
    \draw[gtu arrow] (cond) -- node[above] {Yes} (body);
    \draw[gtu arrow] (body) |- ($(cond.north)+(0,0.5)$) -- (cond.north);
    \draw[gtu arrow] (cond) -- node[right] {No} (end);
\end{tikzpicture}
\end{answerdiagram}

\textbf{Code Example:}
\begin{lstlisting}[language=Python]
# Print first 5 natural numbers using while loop
count = 1

while count <= 5:
    print(count)
    count += 1  # Increment counter

# Output:
# 1
# 2
# 3
# 4
# 5
\end{lstlisting}

\textbf{Key Characteristics:}
\begin{itemize}
    \item \keyword{Entry controlled}: Condition checked before loop execution
    \item \keyword{Initialization}: Variables set before the loop
    \item \keyword{Updation}: Variables updated inside the loop
    \item \keyword{Termination}: Loop exits when condition becomes False
\end{itemize}
\end{solutionbox}

\begin{mnemonicbox}
\mnemonic{IUTE Loop} - Initialize, Update, Test for Exit
\end{mnemonicbox}

\orquestionmarks{2(a)}{3}{Describe continue statement in python in brief.}

\begin{solutionbox}
\begin{answertable}{Continue Statement in Python}
\begin{tabulary}{\linewidth}{|L|}
\hline
\textbf{Description} \\
\hline
The continue statement skips the current iteration of a loop and continues with the next iteration \\
\hline
When encountered, the code inside the loop following the continue statement is skipped \\
\hline
Useful for skipping specific conditions while keeping the loop running \\
\hline
\end{tabulary}
\end{answertable}

\textbf{Code Example:}
\begin{lstlisting}[language=Python]
# Skip printing even numbers
for i in range(1, 6):
    if i % 2 == 0:
        continue
    print(i)  # Prints only 1, 3, 5
\end{lstlisting}
\end{solutionbox}

\begin{mnemonicbox}
\mnemonic{SKIP Ahead} - Skip Keeping Iteration Process
\end{mnemonicbox}

\orquestionmarks{2(b)}{4}{What is the output of the following code:}

\begin{solutionbox}
\textbf{Code:}
\begin{lstlisting}[language=Python]
x=8
y=2
print (x*y)
print (x ** y)
print (x % y)
print(x>y)
\end{lstlisting}

\begin{answertable}{Output Analysis}
\begin{tabulary}{\linewidth}{|l|l|L|}
\hline
\textbf{Operation} & \textbf{Result} & \textbf{Explanation} \\
\hline
\code{x*y} & \code{16} & Multiplication: 8 $\times$ 2 = 16 \\
\hline
\code{x**y} & \code{64} & Exponentiation: 8\textsuperscript{2} = 64 \\
\hline
\code{x\%y} & \code{0} & Modulo (remainder): 8 $\div$ 2 = 4 with remainder 0 \\
\hline
\code{x>y} & \code{True} & Comparison: 8 > 2 is True \\
\hline
\end{tabulary}
\end{answertable}
\end{solutionbox}

\begin{mnemonicbox}
\mnemonic{MEMO} - Multiply, Exponent, Modulo, Operator comparison
\end{mnemonicbox}

\orquestionmarks{2(c)}{7}{Explain if-elif-else Ladder in python with its syntax, flowchart and with python code example.}

\begin{solutionbox}
\textbf{Syntax:}
\begin{lstlisting}[language=Python]
if condition1:
    # code block 1
elif condition2:
    # code block 2
elif condition3:
    # code block 3
else:
    # code block 4
\end{lstlisting}

\begin{answerdiagram}{Flowchart of If-Elif-Else Ladder}
\begin{tikzpicture}[gtu flow]
    \node[gtu start] (start) {Start};
    \node[gtu decision, below of=start] (dec1) {condition1?};
    \node[gtu process, right of=dec1, xshift=2cm] (proc1) {Execute block 1};
    
    \node[gtu decision, below of=dec1, yshift=-1cm] (dec2) {condition2?};
    \node[gtu process, right of=dec2, xshift=2cm] (proc2) {Execute block 2};
    
    \node[gtu decision, below of=dec2, yshift=-1cm] (dec3) {condition3?};
    \node[gtu process, right of=dec3, xshift=2cm] (proc3) {Execute block 3};
    
    \node[gtu process, below of=dec3, yshift=-1cm] (proc4) {Execute block 4}; % Else block
    
    \node[gtu stop, below of=proc4] (end) {End};

    \draw[gtu arrow] (start) -- (dec1);
    
    \draw[gtu arrow] (dec1) -- node[above] {True} (proc1);
    \draw[gtu arrow] (dec1) -- node[left] {False} (dec2);
    
    \draw[gtu arrow] (dec2) -- node[above] {True} (proc2);
    \draw[gtu arrow] (dec2) -- node[left] {False} (dec3);
    
    \draw[gtu arrow] (dec3) -- node[above] {True} (proc3);
    \draw[gtu arrow] (dec3) -- node[left] {False} (proc4);
    
    \draw[gtu arrow] (proc1) -| (4, -9) |- (end);
    \draw[gtu arrow] (proc2) -| (4, -9) |- (end);
    \draw[gtu arrow] (proc3) -| (4, -9) |- (end);
    \draw[gtu arrow] (proc4) -- (end);
    
\end{tikzpicture}
\end{answerdiagram}

\textbf{Code Example:}
\begin{lstlisting}[language=Python]
# Grade calculation based on marks
marks = 75

if marks >= 90:
    grade = "A+"
elif marks >= 80:
    grade = "A"
elif marks >= 70:
    grade = "B"
elif marks >= 60:
    grade = "C"
else:
    grade = "D"

print("Grade:", grade)  # Output: Grade: B
\end{lstlisting}

\textbf{Key Characteristics:}
\begin{itemize}
    \item \keyword{Sequential evaluation}: Conditions checked from top to bottom
    \item \keyword{Exclusive execution}: Only one block executes
    \item \keyword{Default action}: Else block executes if no conditions are True
\end{itemize}
\end{solutionbox}

\begin{mnemonicbox}
\mnemonic{SEEP Logic} - Sequential Evaluation with Exclusive Path
\end{mnemonicbox}

\questionmarks{3(a)}{3}{Write a Python program to print odd numbers between 1 to 20 using loops.}

\begin{solutionbox}
\begin{lstlisting}[language=Python]
# Program to print odd numbers between 1 to 20

# Using for loop with range and step
for number in range(1, 21, 2):
    print(number, end=" ")

# Output: 1 3 5 7 9 11 13 15 17 19
\end{lstlisting}

\textbf{Alternate approach:}
\begin{lstlisting}[language=Python]
# Using for loop with if condition
for number in range(1, 21):
    if number % 2 != 0:
        print(number, end=" ")
\end{lstlisting}
\end{solutionbox}

\begin{mnemonicbox}
\mnemonic{STEO} - Skip Two, Extract Odds
\end{mnemonicbox}

\questionmarks{3(b)}{4}{Explain Nested if statement in brief.}

\begin{solutionbox}
\begin{answertable}{Nested if Statement}
\begin{tabulary}{\linewidth}{|L|}
\hline
\textbf{Description} \\
\hline
An if statement inside another if statement \\
\hline
Allows for more complex conditional logic \\
\hline
Inner if only evaluated when outer if is True \\
\hline
Can have multiple levels of nesting \\
\hline
\end{tabulary}
\end{answertable}

\textbf{Code Example:}
\begin{lstlisting}[language=Python]
age = 25
income = 50000

if age > 18:
    print("Adult")
    if income > 30000:
        print("Eligible for credit card")
    else:
        print("Not eligible for credit card")
else:
    print("Minor")
\end{lstlisting}
\end{solutionbox}

\begin{mnemonicbox}
\mnemonic{LION} - Layered If-statements Operating Nested
\end{mnemonicbox}

\questionmarks{3(c)}{7}{Using a user-defined function write a Program to check entered number is an 'Armstrong number' or a palindrome in which number is passed as argument in calling function.}

\begin{solutionbox}
\begin{lstlisting}[language=Python]
# Program to check Armstrong number or palindrome

def check_number(num):
    # Check if Armstrong number
    # An Armstrong number is one where sum of each digit raised to power of
    # total digits equals the original number
    temp = num
    digits = len(str(num))
    sum = 0
    
    while temp > 0:
        digit = temp % 10
        sum += digit ** digits
        temp //= 10
    
    is_armstrong = (sum == num)
    
    # Check if palindrome
    # A palindrome reads the same backward as forward
    is_palindrome = (str(num) == str(num)[::-1])
    
    # Return results
    return is_armstrong, is_palindrome

# Get input from user
number = int(input("Enter a number: "))

# Call function and display results
armstrong, palindrome = check_number(number)

if armstrong:
    print(number, "is an Armstrong number")
else:
    print(number, "is not an Armstrong number")
    
if palindrome:
    print(number, "is a Palindrome")
else:
    print(number, "is not a Palindrome")
\end{lstlisting}

\textbf{Examples:}
\begin{itemize}
    \item \textbf{Armstrong}: 153 ($1^3 + 5^3 + 3^3 = 1 + 125 + 27 = 153$)
    \item \textbf{Palindrome}: 121 (Same forward and backward)
\end{itemize}
\end{solutionbox}

\begin{mnemonicbox}
\mnemonic{APTEST} - Armstrong Palindrome Test Equal Sum Test
\end{mnemonicbox}

\orquestionmarks{3(a)}{3}{Write a python program to find sum of 1 to 100.}

\begin{solutionbox}
\begin{lstlisting}[language=Python]
# Program to find sum of numbers from 1 to 100

# Method 1: Using loop
total = 0
for num in range(1, 101):
    total += num
print("Sum using loop:", total)

# Method 2: Using formula n(n+1)/2
n = 100
sum_formula = n * (n + 1) // 2
print("Sum using formula:", sum_formula)

# Output: 
# Sum using loop: 5050
# Sum using formula: 5050
\end{lstlisting}
\end{solutionbox}

\begin{mnemonicbox}
\mnemonic{SUM Formula} - Sum Using Mathematical Formula
\end{mnemonicbox}

\orquestionmarks{3(b)}{4}{Write a python program to print the following pattern.}

\begin{solutionbox}
\textbf{Pattern:}
\begin{verbatim}
1
2 3
4 5 6
7 8 9 10
\end{verbatim}

\textbf{Code:}
\begin{lstlisting}[language=Python]
# Program to print the number pattern

num = 1
for i in range(1, 5):  # 4 rows
    for j in range(i):  # columns equal to row number
        print(num, end=" ")
        num += 1
    print()  # New line after each row
\end{lstlisting}

\textbf{Pattern Logic:}
\begin{itemize}
    \item \textbf{Row 1}: 1 number (1)
    \item \textbf{Row 2}: 2 numbers (2, 3)
    \item \textbf{Row 3}: 3 numbers (4, 5, 6)
    \item \textbf{Row 4}: 4 numbers (7, 8, 9, 10)
\end{itemize}
\end{solutionbox}

\begin{mnemonicbox}
\mnemonic{CNIR} - Counter Number Increases with Rows
\end{mnemonicbox}

\orquestionmarks{3(c)}{7}{Write a Program using the function that reverses the entered value.}

\begin{solutionbox}
\begin{lstlisting}[language=Python]
# Program to reverse entered value using functions

def reverse_number(num):
    """Function to reverse an integer number"""
    return int(str(num)[::-1])

def reverse_string(text):
    """Function to reverse a string"""
    return text[::-1]

# Main program
def main():
    choice = input("What do you want to reverse? (n for number, s for string): ")
    
    if choice.lower() == 'n':
        num = int(input("Enter a number: "))
        print("Reversed number:", reverse_number(num))
    elif choice.lower() == 's':
        text = input("Enter a string: ")
        print("Reversed string:", reverse_string(text))
    else:
        print("Invalid choice!")

# Call the main function
main()
\end{lstlisting}

\textbf{Alternate Method for Number Reversal:}
\begin{lstlisting}[language=Python]
def reverse_number_algorithm(num):
    reversed_num = 0
    while num > 0:
        digit = num % 10
        reversed_num = reversed_num * 10 + digit
        num //= 10
    return reversed_num
\end{lstlisting}
\end{solutionbox}

\begin{mnemonicbox}
\mnemonic{FLIP Digits} - Function Logic Inverts Position of Digits
\end{mnemonicbox}

\questionmarks{4(a)}{3}{Describe python math module with proper python code example.}

\begin{solutionbox}
\begin{answertable}{Python Math Module}
\begin{tabulary}{\linewidth}{|L|}
\hline
\textbf{Features} \\
\hline
Provides mathematical functions and constants \\
\hline
Includes trigonometric, logarithmic, and other functions \\
\hline
Contains mathematical constants like pi and e \\
\hline
Requires import before use \\
\hline
\end{tabulary}
\end{answertable}

\textbf{Code Example:}
\begin{lstlisting}[language=Python]
import math

# Constants
print("Value of pi:", math.pi)  # 3.141592653589793
print("Value of e:", math.e)    # 2.718281828459045

# Basic math functions
print("Square root of 16:", math.sqrt(16))  # 4.0
print("5 raised to power 3:", math.pow(5, 3))  # 125.0

# Trigonometric functions (radians)
print("Sine of 90°:", math.sin(math.pi/2))  # 1.0
print("Cosine of 0°:", math.cos(0))  # 1.0

# Logarithmic functions
print("Log base 10 of 100:", math.log10(100))  # 2.0
print("Natural log of e:", math.log(math.e))  # 1.0
\end{lstlisting}
\end{solutionbox}

\begin{mnemonicbox}
\mnemonic{CALM Operations} - Constants And Logarithmic Mathematical Operations
\end{mnemonicbox}

\questionmarks{4(b)}{4}{Write a python program that explains scope of variable.}

\begin{solutionbox}
\begin{lstlisting}[language=Python]
# Program to demonstrate variable scope in Python

# Global variable
global_var = "I am global"

def demonstration():
    # Local variable
    local_var = "I am local"
    
    # Accessing global variable
    print("Inside function - Global variable:", global_var)
    
    # Accessing local variable
    print("Inside function - Local variable:", local_var)
    
    # Creating a variable with same name as global
    global_var = "I am local with global name"
    print("Inside function - Shadowed global:", global_var)

# Function call
demonstration()

# Accessing global variable
print("Outside function - Global variable:", global_var)

# Trying to access local variable would cause error
# print("Outside function - Local variable:", local_var)  # Error!
\end{lstlisting}
\end{solutionbox}

\begin{mnemonicbox}
\mnemonic{GLOVES} - Global Local Variable Encapsulation System
\end{mnemonicbox}

\questionmarks{4(c)}{7}{Explain List Methods and its built-in Functions}

\begin{solutionbox}
\begin{answertable}{List Methods and Functions}
\begin{tabulary}{\linewidth}{|l|L|L|l|}
\hline
\textbf{Method} & \textbf{Description} & \textbf{Example} & \textbf{Output} \\
\hline
\code{append()} & Adds an element at the end & \code{l=['a']; l.append('b')} & \code{['a', 'b']} \\
\hline
\code{insert()} & Adds element at specified position & \code{l=[1,3]; l.insert(1,2)} & \code{[1, 2, 3]} \\
\hline
\code{remove()} & Removes specified item & \code{l=['r','b']; l.remove('r')} & \code{['b']} \\
\hline
\code{pop()} & Removes item at specified index & \code{l=['a','b']; l.pop(1)} & \code{'b'} \\
\hline
\code{clear()} & Removes all elements & \code{l=[1,2]; l.clear()} & \code{[]} \\
\hline
\code{len()} & Returns number of elements & \code{len([1, 2, 3])} & \code{3} \\
\hline
\code{sorted()} & Returns sorted list & \code{sorted([3, 1, 2])} & \code{[1, 2, 3]} \\
\hline
\code{max()} & Returns max value & \code{max([5, 10, 3])} & \code{10} \\
\hline
\end{tabulary}
\end{answertable}

\textbf{Code Example:}
\begin{lstlisting}[language=Python]
# Create a list
my_list = [3, 1, 4, 1, 5]
my_list.append(9)          # Add to end
my_list.insert(2, 7)       # Add at index 2
my_list.remove(1)          # Remove first occurrence of 1
popped = my_list.pop()     # Remove last element

print("Length:", len(my_list))
print("Sorted:", sorted(my_list))
print("Sum:", sum(my_list))
print("Count of 1:", my_list.count(1))
\end{lstlisting}
\end{solutionbox}

\begin{mnemonicbox}
\mnemonic{LISP Operations} - List Insert Sort Pop Operations
\end{mnemonicbox}

\orquestionmarks{4(a)}{3}{List out Python standard library mathematical functions.}

\begin{solutionbox}
\begin{answertable}{Standard Mathematical Functions}
\begin{tabulary}{\linewidth}{|l|L|L|}
\hline
\textbf{Function} & \textbf{Description} & \textbf{Example} \\
\hline
\code{abs()} & Returns absolute value & \code{abs(-5)} $\to$ \code{5} \\
\hline
\code{round()} & Rounds to nearest integer & \code{round(3.7)} $\to$ \code{4} \\
\hline
\code{max()} & Returns largest item & \code{max(1, 5)} $\to$ \code{5} \\
\hline
\code{min()} & Returns smallest item & \code{min(1, 5)} $\to$ \code{1} \\
\hline
\code{sum()} & Adds items of iterable & \code{sum([1, 2])} $\to$ \code{3} \\
\hline
\code{pow()} & Returns x to power y & \code{pow(2, 3)} $\to$ \code{8} \\
\hline
\end{tabulary}
\end{answertable}

\textbf{Additional from math module:}
\begin{itemize}
    \item \code{math.sqrt()}: Square root
    \item \code{math.floor()}: Rounds down
    \item \code{math.ceil()}: Rounds up
    \item \code{math.factorial()}: Factorial of a number
    \item \code{math.gcd()}: Greatest common divisor
\end{itemize}
\end{solutionbox}

\begin{mnemonicbox}
\mnemonic{SMART Calculations} - Standard Mathematical Arithmetic Routines and Tools
\end{mnemonicbox}

\orquestionmarks{4(b)}{4}{Explain built in function in python.}

\begin{solutionbox}
\begin{answertable}{Built-in Functions}
\begin{tabulary}{\linewidth}{|L|}
\hline
\textbf{Description} \\
\hline
Pre-defined functions available in Python without importing any module \\
\hline
Called directly without any prefix \\
\hline
Designed to perform common operations \\
\hline
Examples include \code{print()}, \code{len()}, \code{type()}, \code{input()}, \code{range()} \\
\hline
\end{tabulary}
\end{answertable}

\textbf{Categories with Examples:}
\begin{lstlisting}[language=Python]
# Type conversion
print(int("10"))       # 10
print(str(10))         # "10"

# Math functions
print(abs(-7))         # 7
print(max(5, 10, 3))   # 10

# Collection processing
print(len("hello"))    # 5
print(sorted([3,1,2])) # [1, 2, 3]
\end{lstlisting}
\end{solutionbox}

\begin{mnemonicbox}
\mnemonic{EPIC Functions} - Embedded Python Integrated Core Functions
\end{mnemonicbox}

\orquestionmarks{4(c)}{7}{Write a Python Program to count and display the number of vowels, consonants, uppercase, lowercase characters in a string.}

\begin{solutionbox}
\begin{lstlisting}[language=Python]
# Program to count vowels, consonants, uppercase and lowercase characters

def analyze_string(text):
    # Initialize counters
    vowels = 0
    consonants = 0
    uppercase = 0
    lowercase = 0
    
    # Define vowels
    vowel_set = {'a', 'e', 'i', 'o', 'u'}
    
    # Analyze each character
    for char in text:
        # Check if alphabetic
        if char.isalpha():
            # Check case
            if char.isupper():
                uppercase += 1
            else:
                lowercase += 1
                
            # Check if vowel (case-insensitive)
            if char.lower() in vowel_set:
                vowels += 1
            else:
                consonants += 1
    
    # Return results
    return vowels, consonants, uppercase, lowercase

# Get input
text = input("Enter a string: ")

# Get counts
vowels, consonants, uppercase, lowercase = analyze_string(text)

# Display results
print("Number of vowels:", vowels)
print("Number of consonants:", consonants)
print("Number of uppercase characters:", uppercase)
print("Number of lowercase characters:", lowercase)
\end{lstlisting}
\end{solutionbox}

\begin{mnemonicbox}
\mnemonic{VOCAL Analysis} - Vowels Or Consonants And Letter case
\end{mnemonicbox}

\questionmarks{5(a)}{3}{Write a python code to swap given two elements in a list.}

\begin{solutionbox}
\begin{lstlisting}[language=Python]
# Program to swap two elements in a list

def swap_elements(lst, pos1, pos2):
    """Function to swap two elements in a list"""
    lst[pos1], lst[pos2] = lst[pos2], lst[pos1]
    return lst

# Example usage
my_list = [10, 20, 30, 40, 50]
print("Original list:", my_list)

# Swap elements at positions 1 and 3
result = swap_elements(my_list, 1, 3)
print("After swapping elements at positions 1 and 3:", result)

# Output:
# Original list: [10, 20, 30, 40, 50]
# After swapping elements at positions 1 and 3: [10, 40, 30, 20, 50]
\end{lstlisting}
\end{solutionbox}

\begin{mnemonicbox}
\mnemonic{STEP Logic} - Swap Two Elements with Python Logic
\end{mnemonicbox}

\questionmarks{5(b)}{4}{Write a python Program to check if a substring is present in a given string.}

\begin{solutionbox}
\begin{lstlisting}[language=Python]
# Program to check if a substring is present in a string

def check_substring(main_string, sub_string):
    """Function to check if a substring exists in a string"""
    if sub_string in main_string:
        return True
    else:
        return False

# Get input from user
main_string = input("Enter the main string: ")
sub_string = input("Enter the substring to find: ")

# Check and display result
if check_substring(main_string, sub_string):
    print(f"'{sub_string}' is present in '{main_string}'")
else:
    print(f"'{sub_string}' is not present in '{main_string}'")
\end{lstlisting}
\end{solutionbox}

\begin{mnemonicbox}
\mnemonic{FIND Method} - Find IN Directly with Methods
\end{mnemonicbox}

\questionmarks{5(c)}{7}{Explain tuple Operations, Functions and Methods}

\begin{solutionbox}
\begin{answertable}{Tuple Operations}
\begin{tabulary}{\linewidth}{|l|L|L|l|}
\hline
\textbf{Op/Func} & \textbf{Description} & \textbf{Example} & \textbf{Result} \\
\hline
Creation & Create with parentheses & \code{t=(1,2)} & \code{(1, 2)} \\
\hline
Indexing & Access elements & \code{t[1]} & \code{2} \\
\hline
Slicing & Get subset & \code{t[0:1]} & \code{(1,)} \\
\hline
Concatenation & Join tuples & \code{(1)+(2)} & \code{(1, 2)} \\
\hline
Repetition & Repeat elements & \code{(1)*2} & \code{(1, 1)} \\
\hline
Membership & Check existence & \code{1 in t} & \code{True} \\
\hline
\code{len()} & Number of items & \code{len(t)} & \code{2} \\
\hline
\code{count()} & Count value & \code{t.count(1)} & \code{1} \\
\hline
\code{index()} & Find position & \code{t.index(2)} & \code{1} \\
\hline
\end{tabulary}
\end{answertable}

\textbf{Code Example:}
\begin{lstlisting}[language=Python]
my_tuple = (3, 1, 4, 1, 5, 9)
print("First:", my_tuple[0])
print("Slice:", my_tuple[1:4])
print("Count of 1:", my_tuple.count(1))
print("Index of 4:", my_tuple.index(4))
a, b, c, *rest = my_tuple # Unpacking
\end{lstlisting}
\end{solutionbox}

\begin{mnemonicbox}
\mnemonic{ICONS} - Immutable Collection Operations, Numbering, and Searching
\end{mnemonicbox}

\orquestionmarks{5(a)}{3}{Write a python program find the sum of elements in a list.}

\begin{solutionbox}
\begin{lstlisting}[language=Python]
# Program to find sum of elements in a list

def sum_of_list(numbers):
    """Function to find sum of all elements in a list"""
    total = 0
    for num in numbers:
        total += num
    return total

# Example with user input
num_elements = int(input("Enter the number of elements: "))
my_list = []

# Get elements from user
for i in range(num_elements):
    element = float(input(f"Enter element {i+1}: "))
    my_list.append(element)

# Calculate sum using function
result1 = sum_of_list(my_list)
print("Sum using custom function:", result1)

# Calculate sum using built-in sum() function
result2 = sum(my_list)
print("Sum using built-in function:", result2)
\end{lstlisting}
\end{solutionbox}

\begin{mnemonicbox}
\mnemonic{SALT} - Sum All List Together
\end{mnemonicbox}

\orquestionmarks{5(b)}{4}{Write a Program to demonstrate the set functions and operations.}

\begin{solutionbox}
\begin{lstlisting}[language=Python]
# Program to demonstrate set functions and operations

set1 = {1, 2, 3, 4, 5}
set2 = {4, 5, 6, 7, 8}

# Set operations
print("Union:", set1 | set2)
print("Intersection:", set1 & set2)
print("Difference:", set1 - set2)
print("Symmetric Difference:", set1 ^ set2)

# Set methods
set3 = set1.copy()
set3.add(6)
set3.remove(1)
set3.discard(10) # No error if not found
popped = set3.pop()
set3.clear()
\end{lstlisting}
\end{solutionbox}

\begin{mnemonicbox}
\mnemonic{COSI Methods} - Create, Operate, Search, Investigate with Set Methods
\end{mnemonicbox}

\orquestionmarks{5(c)}{7}{Write a Program to demonstrate the dictionaries functions and operations.}

\begin{solutionbox}
\begin{lstlisting}[language=Python]
# Program to demonstrate dictionary functions and operations

# Creating a dictionary
student = {
    'name': 'John',
    'roll_no': 101,
    'marks': 85
}

# Accessing elements
print("Name:", student['name'])
print("Roll No:", student.get('roll_no'))

# Modifying and Adding
student['marks'] = 90
student['address'] = 'New York'

# Removing items
removed = student.pop('address')
last_item = student.popitem()

# Dictionary methods
print("Keys:", list(student.keys()))
print("Values:", list(student.values()))
print("Items:", list(student.items()))

# Clearing
student.clear()
\end{lstlisting}

\textbf{Key Operations:}
\begin{itemize}
    \item \keyword{Access}: Using key or get() method
    \item \keyword{Modify}: Assign new value to existing key
    \item \keyword{Add}: Assign value to new key
    \item \keyword{Remove}: Using pop(), popitem(), or del
\end{itemize}
\end{solutionbox}

\begin{mnemonicbox}
\mnemonic{ACME Dictionary} - Access, Create, Modify, Extract from Dictionary
\end{mnemonicbox}

\end{document}
