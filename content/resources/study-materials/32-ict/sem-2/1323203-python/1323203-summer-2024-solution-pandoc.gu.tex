\documentclass[10pt,a4paper]{article}

% content/resources/templates/preamble.tex
\usepackage[margin=0.6in]{geometry}
\author{Milav Dabgar}
\usepackage{amsmath,amssymb,amsthm}
\usepackage{booktabs}
\usepackage{multirow}
\usepackage{xcolor}
\usepackage{tcolorbox}
\tcbuselibrary{breakable,skins}
\usepackage[colorlinks=true,linkcolor=blue]{hyperref}
\usepackage{titlesec}
\usepackage{enumitem}
\usepackage{tikz}
\usepackage{pgfplots}
\usepackage{circuitikz}
\usepackage[version=4]{mhchem}
\usepackage{longtable}
\usepackage{array}
\usepackage{float}
\usepackage{caption}
\usepackage{listings}

\lstset{
  basicstyle=\small\ttfamily,
  breaklines=true,
  breakatwhitespace=false,
  postbreak=\mbox{\textcolor{red}{$\hookrightarrow$}\space},
  float=false,
  numbers=left,
  numberstyle=\tiny\color{gray},
  numbersep=10pt,
  xleftmargin=2em,
  keywordstyle=\color{blue},
  commentstyle=\color{green!60!black},
  stringstyle=\color{purple},
  backgroundcolor=\color{gray!5},
  showstringspaces=false,
  tabsize=2,
  captionpos=b,
  keepspaces=true,
  columns=flexible
}

\pgfplotsset{compat=1.18}
\usetikzlibrary{shapes,arrows,positioning,calc,patterns,decorations.pathmorphing,decorations.markings,arrows.meta}

% Color scheme
\definecolor{headcolor}{RGB}{0,102,204}
\definecolor{keycolor}{RGB}{220,20,60}
\definecolor{solutioncolor}{RGB}{34,139,34}
\definecolor{mnemoniccolor}{RGB}{148,0,211}
\definecolor{codecolor}{RGB}{0,0,100}

% Spacing
\setlength{\parskip}{3pt}
\setlist[itemize]{nosep}
\setlist[enumerate]{nosep}

% Title formatting
\titleformat{\section}{\Large\bfseries\color{headcolor}}{\thesection}{1em}{}
\titleformat{\subsection}{\large\bfseries\color{headcolor}}{\thesubsection}{1em}{}

% Pandoc tightlist compatibility
\providecommand{\tightlist}{%
  \setlength{\itemsep}{0pt}\setlength{\parskip}{0pt}}

% Pandoc longtable compatibility
\newcounter{none}
\def\thenone{}


% content/resources/templates/gujarati-boxes.tex
\usepackage{fontspec}
\usepackage{polyglossia}

% Set Gujarati as main language (document is primarily in Gujarati)
% Note: gloss-gujarati.ldf doesn't exist in polyglossia, but it will use hyphenation patterns
\setdefaultlanguage{gujarati}
\setotherlanguage{english}

% Configure Gujarati font properly
% Use Language=Default to prevent polyglossia from trying to add language-specific features
% that don't exist for Gujarati, which causes "empty feature" warnings
\newfontfamily\gujaratifont[Script=Gujarati,AutoFakeBold=2.5,AutoFakeSlant=0.3]{Noto Sans Gujarati}
\setmainfont[Script=Gujarati,AutoFakeBold=2.5,AutoFakeSlant=0.3]{Noto Sans Gujarati}
% Use Noto Sans Gujarati for monospace to support Gujarati in text
\setmonofont[Scale=0.9]{Noto Sans Gujarati}

% Configure English to use the same font
\newfontfamily\englishfont[Script=Gujarati,AutoFakeBold=2.5,AutoFakeSlant=0.3]{Noto Sans Gujarati}

% Translations for polyglossia
\gappto\captionsgujarati{
  \renewcommand{\tablename}{કોષ્ટક}
  \renewcommand{\figurename}{આકૃતિ}
}

% Helper for TikZ nodes to ensure Gujarati font
\newcommand{\gu}[1]{{\gujaratifont #1}}

% Custom environments
\newtcolorbox{solutionbox}{
    breakable,
    enhanced,
    colback=solutioncolor!5!white,
    colframe=solutioncolor!75!black,
    fonttitle=\bfseries,
    title=જવાબ
}

\newtcolorbox{solutionboxnobreak}{
 colback=solutioncolor!5!white,
 colframe=solutioncolor!75!black,
 fonttitle=\bfseries,
 title=જવાબ
}

\newtcolorbox{keyformula}{
 breakable,
 enhanced,
 colback=keycolor!5!white,
 colframe=keycolor!75!black,
 fonttitle=\bfseries,
 title=રાસાયણિક સમીકરણ/સૂત્ર
}

\newtcolorbox{mnemonicbox}{
 breakable,
 enhanced,
 colback=mnemoniccolor!5!white,
 colframe=mnemoniccolor!75!black,
 fonttitle=\bfseries,
 title=મેમરી ટ્રીક
}


\begin{document}

\begin{center}
{\Huge\bfseries\color{headcolor} Subject Name (Gujarati)}\\[5pt]
{\LARGE 1323203 -- Summer 2024}\\[3pt]
{\large Semester 1 Study Material}\\[3pt]
{\normalsize\textit{Detailed Solutions and Explanations}}
\end{center}

\vspace{10pt}

\subsection*{પ્રશ્ન 1(અ) [3
માર્ક્સ]}\label{uxaaauxab0uxab6uxaa8-1uxa85-3-uxaaeuxab0uxa95uxab8}

\textbf{ફ્લોચાર્ટ અને અલ્ગોરિધમના મહત્વની યાદી આપો.}

\begin{solutionbox}

{\def\LTcaptype{none} % do not increment counter
\begin{longtable}[]{@{}
  >{\raggedright\arraybackslash}p{(\linewidth - 2\tabcolsep) * \real{0.5000}}
  >{\raggedright\arraybackslash}p{(\linewidth - 2\tabcolsep) * \real{0.5000}}@{}}
\toprule\noalign{}
\begin{minipage}[b]{\linewidth}\raggedright
ફ્લોચાર્ટનું મહત્વ
\end{minipage} & \begin{minipage}[b]{\linewidth}\raggedright
અલ્ગોરિધમનું મહત્વ
\end{minipage} \\
\midrule\noalign{}
\endhead
\bottomrule\noalign{}
\endlastfoot
પ્રોગ્રામ લોજિકનું દૃશ્ય નિરૂપણ & સમસ્યાને ઉકેલવા માટેનું પગલાંવાર પ્રક્રિયા \\
ભૂલોને સરળતાથી શોધવા અને સુધારવા & ભાષાથી સ્વતંત્ર ઉકેલ અભિગમ \\
જટિલ પ્રક્રિયાઓને સમજવામાં મદદ & પ્રોગ્રામિંગની પાયારૂપ શરૂઆત \\
ટીમના સભ્યો વચ્ચે સંદેશાવ્યવહાર સુધારે & કોડિંગ શરૂ કરતા પહેલા લોજિક નિર્ધારિત કરે \\
\end{longtable}
}

\end{solutionbox}
\begin{mnemonicbox}
``VASE નિર્ણયો'' - Visualize, Analyze, Sequence,
Execute

\end{mnemonicbox}
\subsection*{પ્રશ્ન 1(બ) [4
માર્ક્સ]}\label{uxaaauxab0uxab6uxaa8-1uxaac-4-uxaaeuxab0uxa95uxab8}

\textbf{દાખલ કરેલ સંખ્યા ઈવન કે ઓડ છે તે શોધવા માટે ફ્લોચાર્ટ દોરો.}

\begin{solutionbox}

\includegraphics[width=1\linewidth,height=\textheight,keepaspectratio]{mermaid-9e00195f.pdf}

\textbf{મુખ્ય પગલાં:}

\begin{itemize}
\tightlist
\item
  \textbf{ડેટા એકત્રીકરણ}: વપરાશકર્તા પાસેથી નંબર મેળવો
\item
  \textbf{મોડ્યુલો ઓપરેશન}: 2 વડે ભાગીને શેષ તપાસો
\item
  \textbf{શરતી આઉટપુટ}: શેષના આધારે પરિણામ દર્શાવો
\end{itemize}

\end{solutionbox}
\begin{mnemonicbox}
``MODE'' - Modulo Operation Determines Evenness

\end{mnemonicbox}
\subsection*{પ્રશ્ન 1(ક) [7
માર્ક્સ]}\label{uxaaauxab0uxab6uxaa8-1uxa95-7-uxaaeuxab0uxa95uxab8}

\textbf{બધા લોજિકલ ઓપરેટરોની યાદી બનાવો અને પાયથોન કોડનું ઉદાહરણ આપીને દરેકને
સમજાવો.}

\begin{solutionbox}

{\def\LTcaptype{none} % do not increment counter
\begin{longtable}[]{@{}
  >{\raggedright\arraybackslash}p{(\linewidth - 6\tabcolsep) * \real{0.2500}}
  >{\raggedright\arraybackslash}p{(\linewidth - 6\tabcolsep) * \real{0.3250}}
  >{\raggedright\arraybackslash}p{(\linewidth - 6\tabcolsep) * \real{0.2250}}
  >{\raggedright\arraybackslash}p{(\linewidth - 6\tabcolsep) * \real{0.2000}}@{}}
\toprule\noalign{}
\begin{minipage}[b]{\linewidth}\raggedright
ઓપરેટર
\end{minipage} & \begin{minipage}[b]{\linewidth}\raggedright
વર્ણન
\end{minipage} & \begin{minipage}[b]{\linewidth}\raggedright
ઉદાહરણ
\end{minipage} & \begin{minipage}[b]{\linewidth}\raggedright
આઉટપુટ
\end{minipage} \\
\midrule\noalign{}
\endhead
\bottomrule\noalign{}
\endlastfoot
\passthrough{\lstinline!and!} & બંને સ્ટેટમેન્ટ સાચા હોય તો True રિટર્ન કરે &
\passthrough{\lstinline!x = 5; print(x > 3 and x < 10)!} &
\passthrough{\lstinline!True!} \\
\passthrough{\lstinline!or!} & બે સ્ટેટમેન્ટમાંથી એક સાચું હોય તો True રિટર્ન કરે
& \passthrough{\lstinline!x = 5; print(x > 10 or

x == 5)!} &

\passthrough{\lstinline!True!} \\
\passthrough{\lstinline!not!} & પરિણામને ઉલટાવે, જો પરિણામ સાચું હોય તો
False રિટર્ન કરે & \passthrough{\lstinline!x = 5; print(not(x > 3))!} &
\passthrough{\lstinline!False!} \\
\end{longtable}
}

\textbf{કોડ ઉદાહરણ:}

\begin{lstlisting}[language=Python]
# લોજિકલ AND ઉદાહરણ
age = 25
income = 50000
print("લોન પાત્રતા:", age > 18 and income > 30000)  # True

# લોજિકલ OR ઉદાહરણ
has_credit_card = False
has_cash = True
print("ખરીદી કરી શકે છે:", has_credit_card or has_cash)  # True

# લોજિકલ NOT ઉદાહરણ
is_holiday = False
print("આજે કામ કરવું જોઈએ:", not is_holiday)  # True
\end{lstlisting}

\end{solutionbox}
\begin{mnemonicbox}
``AON સ્પષ્ટતા'' - And, Or, Not લોજિકલ સ્પષ્ટતા માટે

\end{mnemonicbox}
\subsection*{પ્રશ્ન 1(ક) OR [7
માર્ક્સ]}\label{uxaaauxab0uxab6uxaa8-1uxa95-or-7-uxaaeuxab0uxa95uxab8}

\textbf{એક પાયથોન પ્રોગ્રામ લખો કરો જે આપેલ ડેટા પર સાદા વ્યાજ અને ચક્રવૃદ્ધિ
વ્યાજની ગણતરી કરી શકે.}

\begin{solutionbox}

\begin{lstlisting}[language=Python]
# સાદા અને ચક્રવૃદ્ધિ વ્યાજની ગણતરી માટેનો પ્રોગ્રામ

# ઇનપુટ મૂલ્યો
principal = float(input("મૂળ રકમ દાખલ કરો: "))
rate = float(input("વ્યાજ દર દાખલ કરો (% માં): "))
time = float(input("સમય અવધિ દાખલ કરો (વર્ષોમાં): "))

# સાદા વ્યાજની ગણતરી
simple_interest = (principal * rate * time) / 100

# ચક્રવૃદ્ધિ વ્યાજની ગણતરી
compound_interest = principal * ((1 + rate/100) ** time - 1)

# પરિણામો દર્શાવો
print("સાદું વ્યાજ:", round(simple_interest, 2))
print("ચક્રવૃદ્ધિ વ્યાજ:", round(compound_interest, 2))
\end{lstlisting}

\textbf{મુખ્ય સૂત્રો:}

\begin{itemize}
\tightlist
\item
  \textbf{સાદું વ્યાજ (SI)}: મૂળ રકમ \times દર \times સમય / 100
\item
  \textbf{ચક્રવૃદ્ધિ વ્યાજ (CI)}: મૂળ રકમ \times ((1 + દર/100)\^{}સમય - 1)
\end{itemize}

\end{solutionbox}
\begin{mnemonicbox}
``PRT નાણાં વૃદ્ધિ'' - Principal, Rate, Time નાણાંની
વૃદ્ધિ

\end{mnemonicbox}
\subsection*{પ્રશ્ન 2(અ) [3
માર્ક્સ]}\label{uxaaauxab0uxab6uxaa8-2uxa85-3-uxaaeuxab0uxa95uxab8}

\textbf{આપેલ ત્રણ નંબરોમાંથી ન્યૂનતમ સંખ્યા શોધવા માટે પાયથોન પ્રોગ્રામ બનાવો.}

\begin{solutionbox}

\begin{lstlisting}[language=Python]
# ત્રણ નંબરોમાંથી ન્યૂનતમ શોધવાનો પ્રોગ્રામ

# ત્રણ નંબર ઇનપુટ લો
num1 = float(input("પ્રથમ નંબર દાખલ કરો: "))
num2 = float(input("બીજો નંબર દાખલ કરો: "))
num3 = float(input("ત્રીજો નંબર દાખલ કરો: "))

# બિલ્ટ-ઇન min() ફંક્શનનો ઉપયોગ કરીને ન્યૂનતમ શોધો
minimum = min(num1, num2, num3)

# પરિણામ દર્શાવો
print("ન્યૂનતમ નંબર છે:", minimum)
\end{lstlisting}

\end{solutionbox}
\begin{mnemonicbox}
``MIN શોધે ન્યૂનતમ'' - Minimum Is Numerically શોધાય
ન્યૂનતમ સાથે

\end{mnemonicbox}
\subsection*{પ્રશ્ન 2(બ) [4
માર્ક્સ]}\label{uxaaauxab0uxab6uxaa8-2uxaac-4-uxaaeuxab0uxa95uxab8}

\textbf{સ્યુડોકોડ વ્યાખ્યાયિત કરો. x, y અને z ત્રણમાંથી સૌથી મોટી સંખ્યા શોધવા
માટે સ્યુડોકોડ લખો.}

\begin{solutionbox}

{\def\LTcaptype{none} % do not increment counter
\begin{longtable}[]{@{}
  >{\raggedright\arraybackslash}p{(\linewidth - 0\tabcolsep) * \real{1.0000}}@{}}
\toprule\noalign{}
\begin{minipage}[b]{\linewidth}\raggedright
સ્યુડોકોડની વ્યાખ્યા
\end{minipage} \\
\midrule\noalign{}
\endhead
\bottomrule\noalign{}
\endlastfoot
કમ્પ્યુટર પ્રોગ્રામે શું કરવું જોઈએ તેનું વિગતવાર અને વાંચી શકાય તેવું વર્ણન, જે પ્રોગ્રામિંગ
ભાષાને બદલે ઔપચારિક શૈલીમાં લખાયેલી કુદરતી ભાષામાં વ્યક્ત કરવામાં આવે છે. \\
\end{longtable}
}

\textbf{ત્રણ નંબરોમાંથી સૌથી મોટો શોધવા માટે સ્યુડોકોડ:}

\begin{lstlisting}
BEGIN
    INPUT x, y, z
    SET largest = x
    
    IF y > largest THEN
        SET largest = y
    END IF
    
    IF z > largest THEN
        SET largest = z
    END IF
    
    OUTPUT "સૌથી મોટો નંબર છે: ", largest
END
\end{lstlisting}

\end{solutionbox}
\begin{mnemonicbox}
``PIE લખાણ'' - Program Ideas Expressed સરળ લખાણમાં

\end{mnemonicbox}
\subsection*{પ્રશ્ન 2(ક) [7
માર્ક્સ]}\label{uxaaauxab0uxab6uxaa8-2uxa95-7-uxaaeuxab0uxa95uxab8}

\textbf{પાયથોનમાં વાઈલ લૂપને તેના સિન્ટેક્સ, ફ્લોચાર્ટ અને પાયથોન કોડના ઉદાહરણ સાથે
સમજાવો.}

\begin{solutionbox}

\textbf{સિન્ટેક્સ:}

\begin{lstlisting}[language=Python]
while શરત:
    # કોડ જે એક્ઝિક્યુટ કરવાનો છે
\end{lstlisting}

\textbf{ફ્લોચાર્ટ:}

\includegraphics[width=1\linewidth,height=\textheight,keepaspectratio]{mermaid-38863490.pdf}

\textbf{કોડ ઉદાહરણ:}

\begin{lstlisting}[language=Python]
# પ્રથમ 5 કુદરતી સંખ્યાઓ while લૂપનો ઉપયોગ કરીને પ્રિન્ટ કરો
count = 1

while count <= 5:
    print(count)
    count += 1  # કાઉન્ટર વધારો

# આઉટપુટ:
# 1
# 2
# 3
# 4
# 5
\end{lstlisting}

\textbf{મુખ્ય લક્ષણો:}

\begin{itemize}
\tightlist
\item
  \textbf{એન્ટ્રી કંટ્રોલ}: લૂપ એક્ઝિક્યુશન પહેલાં શરત ચકાસવામાં આવે છે
\item
  \textbf{ઇનિશિયલાઇઝેશન}: લૂપ પહેલાં વેરિએબલ્સ સેટ કરવામાં આવે છે
\item
  \textbf{અપડેશન}: લૂપની અંદર વેરિએબલ્સ અપડેટ કરવામાં આવે છે
\item
  \textbf{ટર્મિનેશન}: શરત ખોટી થાય ત્યારે લૂપ બહાર નીકળે છે
\end{itemize}

\end{solutionbox}
\begin{mnemonicbox}
``IUTE લૂપ'' - Initialize, Update, Test for Exit

\end{mnemonicbox}
\subsection*{પ્રશ્ન 2(અ) OR [3
માર્ક્સ]}\label{uxaaauxab0uxab6uxaa8-2uxa85-or-3-uxaaeuxab0uxa95uxab8}

\textbf{પાયથોનમાં કન્ટિન્યુ સ્ટેટમેન્ટનું ટૂંકમાં વર્ણન કરો.}

\begin{solutionbox}

{\def\LTcaptype{none} % do not increment counter
\begin{longtable}[]{@{}l@{}}
\toprule\noalign{}
પાયથોનમાં કન્ટિન્યુ સ્ટેટમેન્ટ \\
\midrule\noalign{}
\endhead
\bottomrule\noalign{}
\endlastfoot
કન્ટિન્યુ સ્ટેટમેન્ટ લૂપના વર્તમાન ઇટરેશનને છોડી દે છે અને આગલા ઇટરેશનથી ચાલુ રાખે છે \\
જ્યારે એનકાઉન્ટર થાય, ત્યારે કન્ટિન્યુ સ્ટેટમેન્ટ પછીનો લૂપનો કોડ છોડી દેવામાં આવે છે \\
ચોક્કસ શરતોને છોડીને લૂપને ચાલુ રાખવા માટે ઉપયોગી છે \\
\end{longtable}
}

\textbf{કોડ ઉદાહરણ:}

\begin{lstlisting}[language=Python]
# બેકી સંખ્યાઓ પ્રિન્ટ કરવાનું છોડી દો
for i in range(1, 6):
    if i % 2 == 0:
        continue
    print(i)  # માત્ર 1, 3, 5 પ્રિન્ટ થાય
\end{lstlisting}

\end{solutionbox}
\begin{mnemonicbox}
``SKIP આગળ'' - Skip Keeping Iteration Process

\end{mnemonicbox}
\subsection*{પ્રશ્ન 2(બ) OR [4
માર્ક્સ]}\label{uxaaauxab0uxab6uxaa8-2uxaac-or-4-uxaaeuxab0uxa95uxab8}

\textbf{નીચેના કોડનું આઉટપુટ શું હશે?}

\begin{lstlisting}[language=Python]
x=8
y=2
print (x*y)
print (x ** y)
print (x % y)
print(x>y)
\end{lstlisting}

\begin{solutionbox}

{\def\LTcaptype{none} % do not increment counter
\begin{longtable}[]{@{}lll@{}}
\toprule\noalign{}
ઓપરેશન & પરિણામ & સમજૂતી \\
\midrule\noalign{}
\endhead
\bottomrule\noalign{}
\endlastfoot
\passthrough{\lstinline!x*y!} & \passthrough{\lstinline!16!} & ગુણાકાર: 8
\times 2 = 16 \\
\passthrough{\lstinline!x**y!} & \passthrough{\lstinline!64!} & પાવર: 8^{2}
= 64 \\
\passthrough{\lstinline!x\%y!} & \passthrough{\lstinline!0!} & મોડ્યુલો
(શેષ): 8 \div 2 = 4 શેષ 0 \\
\passthrough{\lstinline!x>y!} & \passthrough{\lstinline!True!} & તુલના: 8
\textgreater{} 2 સાચું છે \\
\end{longtable}
}

\end{solutionbox}
\begin{mnemonicbox}
``MEMO'' - Multiply, Exponent, Modulo, Operator
comparison

\end{mnemonicbox}
\subsection*{પ્રશ્ન 2(ક) OR [7
માર્ક્સ]}\label{uxaaauxab0uxab6uxaa8-2uxa95-or-7-uxaaeuxab0uxa95uxab8}

\textbf{પાયથોનમાં ઈફ-ઈએલઈએફ-એલ્સ લેડરને તેના સિન્ટેક્સ, ફ્લોચાર્ટ અને પાયથોન કોડના
ઉદાહરણ સાથે સમજાવો.}

\begin{solutionbox}

\textbf{સિન્ટેક્સ:}

\begin{lstlisting}[language=Python]
if શરત1:
    # કોડ બ્લોક 1
elif શરત2:
    # કોડ બ્લોક 2
elif શરત3:
    # કોડ બ્લોક 3
else:
    # કોડ બ્લોક 4
\end{lstlisting}

\textbf{ફ્લોચાર્ટ:}

\includegraphics[width=1\linewidth,height=\textheight,keepaspectratio]{mermaid-8694f4be.pdf}

\textbf{કોડ ઉદાહરણ:}

\begin{lstlisting}[language=Python]
# માર્ક્સના આધારે ગ્રેડની ગણતરી
marks = 75

if marks >= 90:
    grade = "A+"
elif marks >= 80:
    grade = "A"
elif marks >= 70:
    grade = "B"
elif marks >= 60:
    grade = "C"
else:
    grade = "D"

print("ગ્રેડ:", grade)  # આઉટપુટ: ગ્રેડ: B
\end{lstlisting}

\textbf{મુખ્ય લક્ષણો:}

\begin{itemize}
\tightlist
\item
  \textbf{અનુક્રમિક મૂલ્યાંકન}: શરતો ઉપરથી નીચે તપાસવામાં આવે છે
\item
  \textbf{અનન્ય એક્ઝિક્યુશન}: માત્ર એક બ્લોક એક્ઝિક્યુટ થાય છે
\item
  \textbf{ડિફોલ્ટ એક્શન}: જો કોઈ શરત સાચી ન હોય તો else બ્લોક એક્ઝિક્યુટ થાય છે
\end{itemize}

\end{solutionbox}
\begin{mnemonicbox}
``SEEP લોજિક'' - Sequential Evaluation with
Exclusive Path

\end{mnemonicbox}
\subsection*{પ્રશ્ન 3(અ) [3
માર્ક્સ]}\label{uxaaauxab0uxab6uxaa8-3uxa85-3-uxaaeuxab0uxa95uxab8}

\textbf{લૂપ્સનો ઉપયોગ કરીને 1 થી 20 વચ્ચેની એકી સંખ્યાઓ છાપવા માટે પાયથોન
પ્રોગ્રામ લખો.}

\begin{solutionbox}

\begin{lstlisting}[language=Python]
# 1 થી 20 વચ્ચેની એકી સંખ્યાઓ છાપવાનો પ્રોગ્રામ

# range અને step સાથે for લૂપનો ઉપયોગ
for number in range(1, 21, 2):
    print(number, end=" ")

# આઉટપુટ: 1 3 5 7 9 11 13 15 17 19
\end{lstlisting}

\textbf{વૈકલ્પિક અભિગમ:}

\begin{lstlisting}[language=Python]
# if શરત સાથે for લૂપનો ઉપયોગ
for number in range(1, 21):
    if number % 2 != 0:
        print(number, end=" ")
\end{lstlisting}

\end{solutionbox}
\begin{mnemonicbox}
``STEO'' - Skip Two, Extract Odds

\end{mnemonicbox}
\subsection*{પ્રશ્ન 3(બ) [4
માર્ક્સ]}\label{uxaaauxab0uxab6uxaa8-3uxaac-4-uxaaeuxab0uxa95uxab8}

\textbf{નેસ્ટેડ ઈફ સ્ટેટમેન્ટને સંક્ષિપ્તમાં સમજાવો.}

\begin{solutionbox}

{\def\LTcaptype{none} % do not increment counter
\begin{longtable}[]{@{}l@{}}
\toprule\noalign{}
નેસ્ટેડ ઈફ સ્ટેટમેન્ટ \\
\midrule\noalign{}
\endhead
\bottomrule\noalign{}
\endlastfoot
બીજા if સ્ટેટમેન્ટની અંદર એક if સ્ટેટમેન્ટ \\
વધુ જટિલ શરતી લોજિકની મંજૂરી આપે છે \\
બાહ્ય if સાચું હોય ત્યારે જ આંતરિક if મૂલ્યાંકન કરવામાં આવે છે \\
નેસ્ટિંગના ઘણા સ્તરો હોઈ શકે છે \\
\end{longtable}
}

\textbf{કોડ ઉદાહરણ:}

\begin{lstlisting}[language=Python]
age = 25
income = 50000

if age > 18:
    print("પુખ્ત")
    if income > 30000:
        print("ક્રેડિટ કાર્ડ માટે પાત્ર")
    else:
        print("ક્રેડિટ કાર્ડ માટે અપાત્ર")
else:
    print("સગીર")
\end{lstlisting}

\end{solutionbox}
\begin{mnemonicbox}
``LION'' - Layered If-statements Operating Nested

\end{mnemonicbox}
\subsection*{પ્રશ્ન 3(ક) [7
માર્ક્સ]}\label{uxaaauxab0uxab6uxaa8-3uxa95-7-uxaaeuxab0uxa95uxab8}

\textbf{યુઝર ડિફાઈન ફંક્શનનો ઉપયોગ કરીને દાખલ કરેલ નંબર `આર્મસ્ટ્રોંગ નંબર' અથવા
પેલિન્ડ્રોમ છે તે તપાસવા માટે પ્રોગ્રામ લખો એ જેમાં કૉલિંગ ફંક્શનમાં આર્ગ્યુમેંટ તરીકે નંબર
આપવામા આવે છે.}

\begin{solutionbox}

\begin{lstlisting}[language=Python]
# આર્મસ્ટ્રોંગ નંબર અથવા પેલિન્ડ્રોમ તપાસવાનો પ્રોગ્રામ

def check_number(num):
    # આર્મસ્ટ્રોંગ નંબર તપાસો
    temp = num
    digits = len(str(num))
    sum = 0
    
    while temp > 0:
        digit = temp % 10
        sum += digit ** digits
        temp //= 10
    
    is_armstrong = (sum == num)
    
    # પેલિન્ડ્રોમ તપાસો
    is_palindrome = (str(num) == str(num)[::-1])
    
    # પરિણામો પાછા આપો
    return is_armstrong, is_palindrome

# વપરાશકર્તા પાસેથી ઇનપુટ લો
number = int(input("એક નંબર દાખલ કરો: "))

# ફંક્શન કૉલ કરો અને પરિણામો દર્શાવો
armstrong, palindrome = check_number(number)

if armstrong:
    print(number, "એક આર્મસ્ટ્રોંગ નંબર છે")
else:
    print(number, "આર્મસ્ટ્રોંગ નંબર નથી")
    
if palindrome:
    print(number, "એક પેલિન્ડ્રોમ છે")
else:
    print(number, "પેલિન્ડ્રોમ નથી")
\end{lstlisting}

\textbf{આર્મસ્ટ્રોંગ ઉદાહરણો:}

\begin{itemize}
\tightlist
\item
  153: 1^{3} + 5^{3} + 3^{3} = 1 + 125 + 27 = 153 ✓
\item
  370: 3^{3} + 7^{3} + 0^{3} = 27 + 343 + 0 = 370 ✓
\end{itemize}

\end{solutionbox}
\begin{mnemonicbox}
``APTEST'' - Armstrong Palindrome Test Equal Sum
Test

\end{mnemonicbox}
\subsection*{પ્રશ્ન 3(અ) OR [3
માર્ક્સ]}\label{uxaaauxab0uxab6uxaa8-3uxa85-or-3-uxaaeuxab0uxa95uxab8}

\textbf{૧ થી ૧૦૦ સુધી નો સરવાળો શોધવા માટે પાયથોન પ્રોગ્રામ લખો.}

\begin{solutionbox}

\begin{lstlisting}[language=Python]
# 1 થી 100 સુધીની સંખ્યાઓનો સરવાળો શોધવાનો પ્રોગ્રામ

# પદ્ધતિ 1: લૂપનો ઉપયોગ
total = 0
for num in range(1, 101):
    total += num
print("લૂપનો ઉપયોગ કરીને સરવાળો:", total)

# પદ્ધતિ 2: સૂત્ર n(n+1)/2 નો ઉપયોગ
n = 100
sum_formula = n * (n + 1) // 2
print("સૂત્રનો ઉપયોગ કરીને સરવાળો:", sum_formula)

# આઉટપુટ: 
# લૂપનો ઉપયોગ કરીને સરવાળો: 5050
# સૂત્રનો ઉપયોગ કરીને સરવાળો: 5050
\end{lstlisting}

\end{solutionbox}
\begin{mnemonicbox}
``SUM સૂત્ર'' - Sum Using Mathematical સૂત્ર

\end{mnemonicbox}
\subsection*{પ્રશ્ન 3(બ) OR [4
માર્ક્સ]}\label{uxaaauxab0uxab6uxaa8-3uxaac-or-4-uxaaeuxab0uxa95uxab8}

\textbf{નીચેની પેટર્ન છાપવા માટે પાયથોન પ્રોગ્રામ લખો.}

\begin{lstlisting}
1
2 3
4 5 6
7 8 9 10
\end{lstlisting}

\begin{solutionbox}

\begin{lstlisting}[language=Python]
# સંખ્યા પેટર્ન છાપવાનો પ્રોગ્રામ

num = 1
for i in range(1, 5):  # 4 પંક્તિઓ
    for j in range(i):  # પંક્તિ નંબર જેટલા કોલમ
        print(num, end=" ")
        num += 1
    print()  # દરેક પંક્તિ પછી નવી લાઈન
\end{lstlisting}

\textbf{પેટર્ન લોજિક:}

\begin{itemize}
\tightlist
\item
  \textbf{પંક્તિ 1}: 1 સંખ્યા (1)
\item
  \textbf{પંક્તિ 2}: 2 સંખ્યાઓ (2, 3)
\item
  \textbf{પંક્તિ 3}: 3 સંખ્યાઓ (4, 5, 6)
\item
  \textbf{પંક્તિ 4}: 4 સંખ્યાઓ (7, 8, 9, 10)
\end{itemize}

\end{solutionbox}
\begin{mnemonicbox}
``CNIR'' - Counter Number Increases with Rows

\end{mnemonicbox}
\subsection*{પ્રશ્ન 3(ક) OR [7
માર્ક્સ]}\label{uxaaauxab0uxab6uxaa8-3uxa95-or-7-uxaaeuxab0uxa95uxab8}

\textbf{ફંક્શનનો ઉપયોગ કરીને પ્રોગ્રામ લખો જે દાખલ કરેલ નંબરને ઉલટાવે}

\begin{solutionbox}

\begin{lstlisting}[language=Python]
# દાખલ કરેલ મૂલ્યને ઉલટાવવા માટે ફંક્શન ઉપયોગ કરતો પ્રોગ્રામ

def reverse_number(num):
    """સંખ્યાને ઉલટાવવા માટેનું ફંક્શન"""
    return int(str(num)[::-1])

def reverse_string(text):
    """સ્ટ્રિંગને ઉલટાવવા માટેનું ફંક્શન"""
    return text[::-1]

# મુખ્ય પ્રોગ્રામ
def main():
    choice = input("તમે શું ઉલટાવવા માંગો છો? (n માટે નંબર, s માટે સ્ટ્રિંગ): ")
    
    if choice.lower() == 'n':
        num = int(input("એક નંબર દાખલ કરો: "))
        print("ઉલટાવેલ નંબર:", reverse_number(num))
    elif choice.lower() == 's':
        text = input("એક સ્ટ્રિંગ દાખલ કરો: ")
        print("ઉલટાવેલ સ્ટ્રિંગ:", reverse_string(text))
    else:
        print("અમાન્ય પસંદગી!")

# મુખ્ય ફંક્શન કૉલ કરો
main()
\end{lstlisting}

\textbf{નંબર ઉલટાવવા માટે વૈકલ્પિક પદ્ધતિ:}

\begin{lstlisting}[language=Python]
def reverse_number_algorithm(num):
    reversed_num = 0
    while num > 0:
        digit = num % 10
        reversed_num = reversed_num * 10 + digit
        num //= 10
    return reversed_num
\end{lstlisting}

\end{solutionbox}
\begin{mnemonicbox}
``FLIP અંકો'' - Function Logic Inverts Position of
અંકો

\end{mnemonicbox}
\subsection*{પ્રશ્ન 4(અ) [3
માર્ક્સ]}\label{uxaaauxab0uxab6uxaa8-4uxa85-3-uxaaeuxab0uxa95uxab8}

\textbf{યોગ્ય પાયથોન કોડ ઉદાહરણ સાથે પાયથોન મેથ મોડ્યુલનું વર્ણન કરો.}

\begin{solutionbox}

{\def\LTcaptype{none} % do not increment counter
\begin{longtable}[]{@{}l@{}}
\toprule\noalign{}
પાયથોન મેથ મોડ્યુલની વિશેષતાઓ \\
\midrule\noalign{}
\endhead
\bottomrule\noalign{}
\endlastfoot
ગાણિતિક ફંક્શન્સ અને સ્થિરાંકો પ્રદાન કરે છે \\
ત્રિકોણમિતિય, લોગરિધમિક અને અન્ય ફંક્શન્સ શામેલ છે \\
pi અને e જેવા ગાણિતિક સ્થિરાંકો ધરાવે છે \\
ઉપયોગ કરતા પહેલા import કરવું જરૂરી છે \\
\end{longtable}
}

\textbf{કોડ ઉદાહરણ:}

\begin{lstlisting}[language=Python]
import math

# સ્થિરાંકો
print("pi નું મૂલ્ય:", math.pi)  # 3.141592653589793
print("e નું મૂલ્ય:", math.e)    # 2.718281828459045

# મૂળભૂત ગાણિતિક ફંક્શન્સ
print("16 નો વર્ગમૂળ:", math.sqrt(16))  # 4.0
print("5 ની ઘાત 3:", math.pow(5, 3))  # 125.0

# ત્રિકોણમિતિય ફંક્શન્સ (રેડિયન)
print("90^\circ નો સાઇન:", math.sin(math.pi/2))  # 1.0
print("0^\circ નો કોસાઇન:", math.cos(0))  # 1.0

# લોગરિધમિક ફંક્શન્સ
print("100 નો આધાર 10 લોગ:", math.log10(100))  # 2.0
print("e નો નેચરલ લોગ:", math.log(math.e))  # 1.0
\end{lstlisting}

\end{solutionbox}
\begin{mnemonicbox}
``CALM ઓપરેશન્સ'' - Constants And Logarithmic
Mathematical ઓપરેશન્સ

\end{mnemonicbox}
\subsection*{પ્રશ્ન 4(બ) [4
માર્ક્સ]}\label{uxaaauxab0uxab6uxaa8-4uxaac-4-uxaaeuxab0uxa95uxab8}

\textbf{વેરીએબલના સ્કોપને સમજાવતો પાયથોન પ્રોગ્રામ લખો.}

\begin{solutionbox}

\begin{lstlisting}[language=Python]
# પાયથોનમાં વેરીએબલ સ્કોપ દર્શાવતો પ્રોગ્રામ

# ગ્લોબલ વેરીએબલ
global_var = "હું ગ્લોબલ છું"

def demonstration():
    # લોકલ વેરીએબલ
    local_var = "હું લોકલ છું"
    
    # ગ્લોબલ વેરીએબલ એક્સેસ કરવું
    print("ફંક્શનની અંદર - ગ્લોબલ વેરીએબલ:", global_var)
    
    # લોકલ વેરીએબલ એક્સેસ કરવું
    print("ફંક્શનની અંદર - લોકલ વેરીએબલ:", local_var)
    
    # ગ્લોબલ નામ ધરાવતું લોકલ વેરીએબલ બનાવવું
    global_var = "હું ગ્લોબલ નામવાળો લોકલ છું"
    print("ફંક્શનની અંદર - શેડોડ ગ્લોબલ:", global_var)

# ફંક્શન કૉલ
demonstration()

# ગ્લોબલ વેરીએબલ એક્સેસ કરવું
print("ફંક્શનની બહાર - ગ્લોબલ વેરીએબલ:", global_var)

# લોકલ વેરીએબલ એક્સેસ કરવાનો પ્રયાસ ભૂલ ઉત્પન્ન કરશે
# print("ફંક્શનની બહાર - લોકલ વેરીએબલ:", local_var)  # ભૂલ!
\end{lstlisting}

\textbf{આઉટપુટ:}

\begin{lstlisting}
ફંક્શનની અંદર - ગ્લોબલ વેરીએબલ: હું ગ્લોબલ છું
ફંક્શનની અંદર - લોકલ વેરીએબલ: હું લોકલ છું
ફંક્શનની અંદર - શેડોડ ગ્લોબલ: હું ગ્લોબલ નામવાળો લોકલ છું
ફંક્શનની બહાર - ગ્લોબલ વેરીએબલ: હું ગ્લોબલ છું
\end{lstlisting}

\end{solutionbox}
\begin{mnemonicbox}
``GLOVES'' - Global Local Variable Encapsulation
System

\end{mnemonicbox}
\subsection*{પ્રશ્ન 4(ક) [7
માર્ક્સ]}\label{uxaaauxab0uxab6uxaa8-4uxa95-7-uxaaeuxab0uxa95uxab8}

\textbf{લિસ્ટ પદ્ધતિઓ અને તેના બિલ્ટ-ઇન કાયો સમજાવો}

\begin{solutionbox}

{\def\LTcaptype{none} % do not increment counter
\begin{longtable}[]{@{}
  >{\raggedright\arraybackslash}p{(\linewidth - 6\tabcolsep) * \real{0.3617}}
  >{\raggedright\arraybackslash}p{(\linewidth - 6\tabcolsep) * \real{0.2766}}
  >{\raggedright\arraybackslash}p{(\linewidth - 6\tabcolsep) * \real{0.1915}}
  >{\raggedright\arraybackslash}p{(\linewidth - 6\tabcolsep) * \real{0.1702}}@{}}
\toprule\noalign{}
\begin{minipage}[b]{\linewidth}\raggedright
પદ્ધતિ/ફંક્શન
\end{minipage} & \begin{minipage}[b]{\linewidth}\raggedright
વર્ણન
\end{minipage} & \begin{minipage}[b]{\linewidth}\raggedright
ઉદાહરણ
\end{minipage} & \begin{minipage}[b]{\linewidth}\raggedright
આઉટપુટ
\end{minipage} \\
\midrule\noalign{}
\endhead
\bottomrule\noalign{}
\endlastfoot
\passthrough{\lstinline!append()!} & અંતે એલિમેન્ટ ઉમેરે છે &
\passthrough{\lstinline!fruits = ['apple']; fruits.append('banana'); print(fruits)!}
& \passthrough{\lstinline!['apple', 'banana']!} \\
\passthrough{\lstinline!insert()!} & ચોક્કસ પોઝિશન પર એલિમેન્ટ ઉમેરે &
\passthrough{\lstinline!nums = [1, 3]; nums.insert(1, 2); print(nums)!}
& \passthrough{\lstinline![1, 2, 3]!} \\
\passthrough{\lstinline!remove()!} & ચોક્કસ આઈટમ દૂર કરે &
\passthrough{\lstinline!colors = ['red', 'blue']; colors.remove('red'); print(colors)!}
& \passthrough{\lstinline!['blue']!} \\
\passthrough{\lstinline!pop()!} & ચોક્કસ ઇન્ડેક્સ પર આઈટમ દૂર કરે &
\passthrough{\lstinline!letters = ['a', 'b', 'c']; x = letters.pop(1); print(x, letters)!}
& \passthrough{\lstinline!b ['a', 'c']!} \\
\passthrough{\lstinline!clear()!} & બધા એલિમેન્ટ્સ દૂર કરે &
\passthrough{\lstinline!items = [1, 2]; items.clear(); print(items)!} &
\passthrough{\lstinline![]!} \\
\passthrough{\lstinline!len()!} & એલિમેન્ટ્સની સંખ્યા પાછી આપે &
\passthrough{\lstinline!print(len([1, 2, 3]))!} &
\passthrough{\lstinline!3!} \\
\passthrough{\lstinline!sorted()!} & સૉર્ટેડ લિસ્ટ પાછી આપે &
\passthrough{\lstinline!print(sorted([3, 1, 2]))!} &
\passthrough{\lstinline![1, 2, 3]!} \\
\passthrough{\lstinline!max()/min()!} & મહત્તમ/લઘુત્તમ મૂલ્ય પાછું આપે &
\passthrough{\lstinline!print(max([5, 10, 3]), min([5, 10, 3]))!} &
\passthrough{\lstinline!10 3!} \\
\end{longtable}
}

\textbf{કોડ ઉદાહરણ:}

\begin{lstlisting}[language=Python]
# લિસ્ટ બનાવવી
my_list = [3, 1, 4, 1, 5]
print("મૂળ:", my_list)

# એલિમેન્ટ્સ ઉમેરવા
my_list.append(9)
print("append પછી:", my_list)

my_list.insert(2, 7)
print("insert પછી:", my_list)

# એલિમેન્ટ્સ દૂર કરવા
my_list.remove(1)  # પ્રથમ 1 દૂર કરે છે
print("remove પછી:", my_list)

popped = my_list.pop()  # છેલ્લું એલિમેન્ટ દૂર કરે અને પાછું આપે
print("pop કરેલું મૂલ્ય:", popped)
print("pop પછી:", my_list)

# અન્ય ઓપરેશન્સ
print("લંબાઈ:", len(my_list))
print("સૉર્ટેડ:", sorted(my_list))
print("સરવાળો:", sum(my_list))
print("1 ની સંખ્યા:", my_list.count(1))
\end{lstlisting}

\end{solutionbox}
\begin{mnemonicbox}
``LISP ઓપરેશન્સ'' - List Insert Sort Pop ઓપરેશન્સ

\end{mnemonicbox}
\subsection*{પ્રશ્ન 4(અ) OR [3
માર્ક્સ]}\label{uxaaauxab0uxab6uxaa8-4uxa85-or-3-uxaaeuxab0uxa95uxab8}

\textbf{પાયથોન સ્ટાન્ડર્ડ લાઇબ્રેરી ગાણિતિક કાયોની સૂચિ બનાવો.}

\begin{solutionbox}

{\def\LTcaptype{none} % do not increment counter
\begin{longtable}[]{@{}lll@{}}
\toprule\noalign{}
ગાણિતિક ફંક્શન & વર્ણન & ઉદાહરણ \\
\midrule\noalign{}
\endhead
\bottomrule\noalign{}
\endlastfoot
\passthrough{\lstinline!abs()!} & નિરપેક્ષ મૂલ્ય પાછું આપે &
\passthrough{\lstinline!abs(-5)!} \rightarrow \passthrough{\lstinline!5!} \\
\passthrough{\lstinline!round()!} & નજીકના પૂર્ણાંક સુધી ગોળ કરે &
\passthrough{\lstinline!round(3.7)!} \rightarrow \passthrough{\lstinline!4!} \\
\passthrough{\lstinline!max()!} & સૌથી મોટી આઈટમ પાછી આપે &
\passthrough{\lstinline!max(1, 5, 3)!} \rightarrow \passthrough{\lstinline!5!} \\
\passthrough{\lstinline!min()!} & સૌથી નાની આઈટમ પાછી આપે &
\passthrough{\lstinline!min(1, 5, 3)!} \rightarrow \passthrough{\lstinline!1!} \\
\passthrough{\lstinline!sum()!} & ઇટરેબલની આઈટમ્સનો સરવાળો કરે &
\passthrough{\lstinline!sum([1, 2, 3])!} \rightarrow
\passthrough{\lstinline!6!} \\
\passthrough{\lstinline!pow()!} & x ને y ની ઘાત પાછી આપે &
\passthrough{\lstinline!pow(2, 3)!} \rightarrow \passthrough{\lstinline!8!} \\
\passthrough{\lstinline!divmod()!} & ભાગફળ અને શેષ પાછા આપે &
\passthrough{\lstinline!divmod(7, 2)!} \rightarrow
\passthrough{\lstinline!(3, 1)!} \\
\end{longtable}
}

\textbf{math મોડ્યુલમાંથી વધારાના:}

\begin{itemize}
\tightlist
\item
  \passthrough{\lstinline!math.sqrt()!}: વર્ગમૂળ
\item
  \passthrough{\lstinline!math.floor()!}: નીચે ગોળ કરે
\item
  \passthrough{\lstinline!math.ceil()!}: ઉપર ગોળ કરે
\item
  \passthrough{\lstinline!math.factorial()!}: ફેક્ટોરિયલ
\item
  \passthrough{\lstinline!math.gcd()!}: મહત્તમ સામાન્ય અવયવ
\end{itemize}

\end{solutionbox}
\begin{mnemonicbox}
``SMART ગણતરી'' - Standard Mathematical Arithmetic
Routines and Tools

\end{mnemonicbox}
\subsection*{પ્રશ્ન 4(બ) OR [4
માર્ક્સ]}\label{uxaaauxab0uxab6uxaa8-4uxaac-or-4-uxaaeuxab0uxa95uxab8}

\textbf{પાયથોનમાં બિલ્ટ ઇન ફંક્શન સમજાવો.}

\begin{solutionbox}

{\def\LTcaptype{none} % do not increment counter
\begin{longtable}[]{@{}l@{}}
\toprule\noalign{}
પાયથોનમાં બિલ્ટ-ઇન ફંક્શન્સ \\
\midrule\noalign{}
\endhead
\bottomrule\noalign{}
\endlastfoot
કોઈપણ મોડ્યુલ ઇમ્પોર્ટ કર્યા વિના પાયથોનમાં ઉપલબ્ધ પ્રી-ડિફાઇન્ડ ફંક્શન્સ \\
કોઈપણ પ્રીફિક્સ વિના સીધા જ કૉલ કરી શકાય છે \\
સામાન્ય ઓપરેશન્સ કરવા માટે ડિઝાઇન કરેલ છે \\
ઉદાહરણોમાં print(), len(), type(), input(), range() શામેલ છે \\
\end{longtable}
}

\textbf{કેટેગરીઓ સાથે ઉદાહરણો:}

\begin{lstlisting}[language=Python]
# ટાઇપ કન્વર્ઝન ફંક્શન્સ
print(int("10"))       # 10
print(float("10.5"))   # 10.5
print(str(10))         # "10"
print(list("abc"))     # ['a', 'b', 'c']

# ગાણિતિક ફંક્શન્સ
print(abs(-7))         # 7
print(round(3.7))      # 4
print(max(5, 10, 3))   # 10

# કલેક્શન પ્રોસેસિંગ
print(len("hello"))    # 5
print(sorted([3,1,2])) # [1, 2, 3]
print(sum([1, 2, 3]))  # 6
\end{lstlisting}

\end{solutionbox}
\begin{mnemonicbox}
``EPIC ફંક્શન્સ'' - Embedded Python Integrated Core
ફંક્શન્સ

\end{mnemonicbox}
\subsection*{પ્રશ્ન 4(ક) OR [7
માર્ક્સ]}\label{uxaaauxab0uxab6uxaa8-4uxa95-or-7-uxaaeuxab0uxa95uxab8}

\textbf{વાક્યમાં રહેલ સ્વરો, વ્યંજન, અપરકેસ, લોઅરકેસ અક્ષરોની સંખ્યા ગણવા અને
દર્શાવવા માટે પાયથોન પ્રોગ્રામ લખો.}

\begin{solutionbox}

\begin{lstlisting}[language=Python]
# સ્ટ્રિંગમાં સ્વરો, વ્યંજન, અપરકેસ, લોઅરકેસ ગણતરી માટેનો પ્રોગ્રામ

def analyze_string(text):
    # કાઉન્ટર્સ ઇનિશિયલાઇઝ કરો
    vowels = 0
    consonants = 0
    uppercase = 0
    lowercase = 0
    
    # સ્વરો ડિફાઇન કરો
    vowel_set = {'a', 'e', 'i', 'o', 'u'}
    
    # દરેક અક્ષરનું વિશ્લેષણ
    for char in text:
        # તપાસો કે શું અક્ષર છે
        if char.isalpha():
            # કેસ તપાસો
            if char.isupper():
                uppercase += 1
            else:
                lowercase += 1
                
            # તપાસો કે સ્વર છે (કેસ-સેન્સિટિવ)
            if char.lower() in vowel_set:
                vowels += 1
            else:
                consonants += 1
    
    # પરિણામો પાછા આપો
    return vowels, consonants, uppercase, lowercase

# ઇનપુટ લો
text = input("એક સ્ટ્રિંગ દાખલ કરો: ")

# ગણતરી મેળવો
vowels, consonants, uppercase, lowercase = analyze_string(text)

# પરિણામો દર્શાવો
print("સ્વરોની સંખ્યા:", vowels)
print("વ્યંજનોની સંખ્યા:", consonants)
print("અપરકેસ અક્ષરોની સંખ્યા:", uppercase)
print("લોઅરકેસ અક્ષરોની સંખ્યા:", lowercase)
\end{lstlisting}

\textbf{ઉદાહરણ:}

\begin{itemize}
\tightlist
\item
  ઇનપુટ: ``Hello World!''
\item
  આઉટપુટ:

  \begin{itemize}
  \tightlist
  \item
    સ્વરો: 3 (e, o, o)
  \item
    વ્યંજનો: 7 (H, l, l, W, r, l, d)
  \item
    અપરકેસ: 2 (H, W)
  \item
    લોઅરકેસ: 8 (e, l, l, o, o, r, l, d)
  \end{itemize}
\end{itemize}

\end{solutionbox}
\begin{mnemonicbox}
``VOCAL વિશ્લેષણ'' - Vowels Or Consonants And Letter
case

\end{mnemonicbox}
\subsection*{પ્રશ્ન 5(અ) [3
માર્ક્સ]}\label{uxaaauxab0uxab6uxaa8-5uxa85-3-uxaaeuxab0uxa95uxab8}

\textbf{લિસ્ટ મા આપેલ બે એલીમેંટ ને સ્વેપ કરવા માટે પાયથોન કોડ લખો.}

\begin{solutionbox}

\begin{lstlisting}[language=Python]
# લિસ્ટમાં બે એલિમેન્ટ્સ સ્વેપ કરવાનો પ્રોગ્રામ

def swap_elements(lst, pos1, pos2):
    """લિસ્ટમાં બે એલિમેન્ટ્સ સ્વેપ કરવા માટેનું ફંક્શન"""
    lst[pos1], lst[pos2] = lst[pos2], lst[pos1]
    return lst

# ઉદાહરણ ઉપયોગ
my_list = [10, 20, 30, 40, 50]
print("મૂળ લિસ્ટ:", my_list)

# પોઝિશન 1 અને 3 પરના એલિમેન્ટ્સ સ્વેપ કરો
result = swap_elements(my_list, 1, 3)
print("પોઝિશન 1 અને 3 પરના એલિમેન્ટ્સ સ્વેપ કર્યા પછી:", result)

# આઉટપુટ:
# મૂળ લિસ્ટ: [10, 20, 30, 40, 50]
# પોઝિશન 1 અને 3 પરના એલિમેન્ટ્સ સ્વેપ કર્યા પછી: [10, 40, 30, 20, 50]
\end{lstlisting}

\end{solutionbox}
\begin{mnemonicbox}
``STEP લોજિક'' - Swap Two Elements with Python લોજિક

\end{mnemonicbox}
\subsection*{પ્રશ્ન 5(બ) [4
માર્ક્સ]}\label{uxaaauxab0uxab6uxaa8-5uxaac-4-uxaaeuxab0uxa95uxab8}

\textbf{આપેલ સ્ટ્રિંગમાં સબસ્ટ્રિંગ હાજર છે કે કેમ તે તપાસવા માટે પાયથોન પ્રોગ્રામ
લખો.}

\begin{solutionbox}

\begin{lstlisting}[language=Python]
# સ્ટ્રિંગમાં સબસ્ટ્રિંગની હાજરી તપાસવાનો પ્રોગ્રામ

def check_substring(main_string, sub_string):
    """સ્ટ્રિંગમાં સબસ્ટ્રિંગની હાજરી તપાસવા માટેનું ફંક્શન"""
    if sub_string in main_string:
        return True
    else:
        return False

# વપરાશકર્તા પાસેથી ઇનપુટ લો
main_string = input("મુખ્ય સ્ટ્રિંગ દાખલ કરો: ")
sub_string = input("શોધવાની સબસ્ટ્રિંગ દાખલ કરો: ")

# તપાસો અને પરિણામ દર્શાવો
if check_substring(main_string, sub_string):
    print(f"'{sub_string}' '{main_string}' માં હાજર છે")
else:
    print(f"'{sub_string}' '{main_string}' માં હાજર નથી")
\end{lstlisting}

\textbf{find() પદ્ધતિનો ઉપયોગ કરીને વૈકલ્પિક રીત:}

\begin{lstlisting}[language=Python]
def check_substring_find(main_string, sub_string):
    """સબસ્ટ્રિંગ તપાસવા માટે find પદ્ધતિનો ઉપયોગ"""
    position = main_string.find(sub_string)
    return position != -1  # જો સબસ્ટ્રિંગ મળી હોય તો True પાછું આપે
\end{lstlisting}

\end{solutionbox}
\begin{mnemonicbox}
``FIND પદ્ધતિ'' - Find IN Directly with પદ્ધતિઓ

\end{mnemonicbox}
\subsection*{પ્રશ્ન 5(ક) [7
માર્ક્સ]}\label{uxaaauxab0uxab6uxaa8-5uxa95-7-uxaaeuxab0uxa95uxab8}

\textbf{ટપલ ઓપરેશન, ફંકશન અને મેથડ સમજાવો.}

\begin{solutionbox}

{\def\LTcaptype{none} % do not increment counter
\begin{longtable}[]{@{}
  >{\raggedright\arraybackslash}p{(\linewidth - 6\tabcolsep) * \real{0.4643}}
  >{\raggedright\arraybackslash}p{(\linewidth - 6\tabcolsep) * \real{0.2321}}
  >{\raggedright\arraybackslash}p{(\linewidth - 6\tabcolsep) * \real{0.1607}}
  >{\raggedright\arraybackslash}p{(\linewidth - 6\tabcolsep) * \real{0.1429}}@{}}
\toprule\noalign{}
\begin{minipage}[b]{\linewidth}\raggedright
ઓપરેશન/ફંક્શન/મેથડ
\end{minipage} & \begin{minipage}[b]{\linewidth}\raggedright
વર્ણન
\end{minipage} & \begin{minipage}[b]{\linewidth}\raggedright
ઉદાહરણ
\end{minipage} & \begin{minipage}[b]{\linewidth}\raggedright
આઉટપુટ
\end{minipage} \\
\midrule\noalign{}
\endhead
\bottomrule\noalign{}
\endlastfoot
\textbf{બનાવટ} & કૌંસ સાથે ટપલ બનાવવું &
\passthrough{\lstinline!t = (1, 2, 3)!} &
\passthrough{\lstinline!(1, 2, 3)!} \\
\textbf{ઇન્ડેક્સિંગ} & ટપલ એલિમેન્ટ્સ એક્સેસ કરવા &
\passthrough{\lstinline!t[1]!} & \passthrough{\lstinline!2!} \\
\textbf{સ્લાઇસિંગ} & ટપલનો સબસેટ મેળવવો & \passthrough{\lstinline!t[1:3]!} &
\passthrough{\lstinline!(2, 3)!} \\
\textbf{કેટેનેશન} & બે ટપલ જોડવા & \passthrough{\lstinline!(1, 2) + (3, 4)!}
& \passthrough{\lstinline!(1, 2, 3, 4)!} \\
\textbf{રિપિટેશન} & ટપલ એલિમેન્ટ્સ રિપીટ કરવા &
\passthrough{\lstinline!(1, 2) * 2!} &
\passthrough{\lstinline!(1, 2, 1, 2)!} \\
\textbf{મેમ્બરશિપ} & એલિમેન્ટ છે કે નહીં તે તપાસવું &
\passthrough{\lstinline!3 in (1, 2, 3)!} &
\passthrough{\lstinline!True!} \\
\textbf{len()} & આઇટમ્સની સંખ્યા મેળવવી &
\passthrough{\lstinline!len((1, 2, 3))!} &
\passthrough{\lstinline!3!} \\
\textbf{min()/max()} & લઘુત્તમ/મહત્તમ મૂલ્ય શોધવું &
\passthrough{\lstinline!min((3, 1, 2))!} &
\passthrough{\lstinline!1!} \\
\textbf{count()} & મૂલ્યની સંખ્યા ગણવી &
\passthrough{\lstinline!(1, 2, 1).count(1)!} &
\passthrough{\lstinline!2!} \\
\textbf{index()} & મૂલ્યની પોઝિશન શોધવી &
\passthrough{\lstinline!(1, 2, 3).index(2)!} &
\passthrough{\lstinline!1!} \\
\textbf{sorted()} & ટપલમાંથી સૉર્ટેડ લિસ્ટ પાછી આપે &
\passthrough{\lstinline!sorted((3, 1, 2))!} &
\passthrough{\lstinline![1, 2, 3]!} \\
\end{longtable}
}

\textbf{કોડ ઉદાહરણ:}

\begin{lstlisting}[language=Python]
# ટપલ બનાવવું
my_tuple = (3, 1, 4, 1, 5, 9)
print("મૂળ ટપલ:", my_tuple)

# એલિમેન્ટ્સ એક્સેસ કરવા
print("પ્રથમ એલિમેન્ટ:", my_tuple[0])
print("છેલ્લું એલિમેન્ટ:", my_tuple[-1])
print("સ્લાઇસ (1:4):", my_tuple[1:4])

# ઓપરેશન્સ
tuple2 = (2, 7)
combined = my_tuple + tuple2
print("જોડાયેલું:", combined)

repeated = tuple2 * 3
print("રિપીટ કરેલું:", repeated)

# ફંક્શન્સ અને મેથડ્સ
print("લંબાઈ:", len(my_tuple))
print("1 ની સંખ્યા:", my_tuple.count(1))
print("4 નો ઇન્ડેક્સ:", my_tuple.index(4))
print("ન્યૂનતમ મૂલ્ય:", min(my_tuple))
print("મહત્તમ મૂલ્ય:", max(my_tuple))
print("સૉર્ટેડ:", sorted(my_tuple))  # લિસ્ટ પાછી આપે

# અનપેકિંગ
a, b, c, *rest = my_tuple
print("અનપેક કરેલું:", a, b, c, rest)
\end{lstlisting}

\end{solutionbox}
\begin{mnemonicbox}
``ICONS'' - Immutable Collection Operations,
Numbering, and Searching

\end{mnemonicbox}
\subsection*{પ્રશ્ન 5(અ) OR [3
માર્ક્સ]}\label{uxaaauxab0uxab6uxaa8-5uxa85-or-3-uxaaeuxab0uxa95uxab8}

\textbf{લિસ્ટ મા આપેલ એલીમેંટ નો સરવાળો શોધવા માટે પાયથોન પ્રોગ્રામ લખો.}

\begin{solutionbox}

\begin{lstlisting}[language=Python]
# લિસ્ટના એલિમેન્ટ્સનો સરવાળો શોધવાનો પ્રોગ્રામ

def sum_of_list(numbers):
    """લિસ્ટના બધા એલિમેન્ટ્સનો સરવાળો શોધવા માટેનું ફંક્શન"""
    total = 0
    for num in numbers:
        total += num
    return total

# વપરાશકર્તા ઇનપુટ સાથે ઉદાહરણ
num_elements = int(input("એલિમેન્ટ્સની સંખ્યા દાખલ કરો: "))
my_list = []

# વપરાશકર્તા પાસેથી એલિમેન્ટ્સ લો
for i in range(num_elements):
    element = float(input(f"એલિમેન્ટ {i+1} દાખલ કરો: "))
    my_list.append(element)

# ફંક્શનનો ઉપયોગ કરીને સરવાળો ગણો
result1 = sum_of_list(my_list)
print("કસ્ટમ ફંક્શનનો ઉપયોગ કરીને સરવાળો:", result1)

# બિલ્ટ-ઇન sum() ફંક્શનનો ઉપયોગ કરીને સરવાળો ગણો
result2 = sum(my_list)
print("બિલ્ટ-ઇન ફંક્શનનો ઉપયોગ કરીને સરવાળો:", result2)
\end{lstlisting}

\end{solutionbox}
\begin{mnemonicbox}
``SALT'' - Sum All List Together

\end{mnemonicbox}
\subsection*{પ્રશ્ન 5(બ) OR [4
માર્ક્સ]}\label{uxaaauxab0uxab6uxaa8-5uxaac-or-4-uxaaeuxab0uxa95uxab8}

\textbf{સેટ ફંકશન અને ઓપરેશન દર્શાવવા માટે એક પ્રોગ્રામ લખો.}

\begin{solutionbox}

\begin{lstlisting}[language=Python]
# સેટ ફંક્શન અને ઓપરેશન્સ દર્શાવતો પ્રોગ્રામ

# સેટ બનાવવા
set1 = {1, 2, 3, 4, 5}
set2 = {4, 5, 6, 7, 8}

print("સેટ 1:", set1)
print("સેટ 2:", set2)

# સેટ ઓપરેશન્સ
print("\nસેટ ઓપરેશન્સ:")
print("યુનિયન:", set1 | set2)  # વૈકલ્પિક: set1.union(set2)
print("ઇન્ટરસેક્શન:", set1 & set2)  # વૈકલ્પિક: set1.intersection(set2)
print("ડિફરન્સ (set1-set2):", set1 - set2)  # વૈકલ્પિક: set1.difference(set2)
print("સિમેટ્રિક ડિફરન્સ:", set1 ^ set2)  # વૈકલ્પિક: set1.symmetric_difference(set2)

# સેટ મેથડ્સ
print("\nસેટ મેથડ્સ:")
set3 = set1.copy()
print("સેટ1ની કૉપી:", set3)

set3.add(6)
print("6 ઉમેર્યા પછી:", set3)

set3.remove(1)
print("1 દૂર કર્યા પછી:", set3)

set3.discard(10)  # જો એલિમેન્ટ ન હોય તો કોઈ ભૂલ નહીં
print("10 ડિસ્કાર્ડ કર્યા પછી:", set3)

popped = set3.pop()
print("પોપ કરેલું એલિમેન્ટ:", popped)
print("પોપ પછી:", set3)

set3.clear()
print("ક્લિયર કર્યા પછી:", set3)
\end{lstlisting}

\end{solutionbox}
\begin{mnemonicbox}
``COSI મેથડ્સ'' - Create, Operate, Search, Investigate
with સેટ મેથડ્સ

\end{mnemonicbox}
\subsection*{પ્રશ્ન 5(ક) OR [7
માર્ક્સ]}\label{uxaaauxab0uxab6uxaa8-5uxa95-or-7-uxaaeuxab0uxa95uxab8}

\textbf{ડિક્શનેરી ફંક્શન અને ઓપરેશન સમજાવવા માટે પાયથોન પ્રોગામ લખો.}

\begin{solutionbox}

\begin{lstlisting}[language=Python]
# ડિક્શનેરી ફંક્શન અને ઓપરેશન્સ દર્શાવતો પ્રોગ્રામ

# ડિક્શનેરી બનાવવી
student = {
    'name': 'John',
    'roll_no': 101,
    'marks': 85,
    'subjects': ['Python', 'Math', 'English']
}

print("મૂળ ડિક્શનેરી:", student)

# એલિમેન્ટ્સ એક્સેસ કરવા
print("\nએલિમેન્ટ્સ એક્સેસ કરવા:")
print("નામ:", student['name'])
print("માર્ક્સ:", student['marks'])

# get() વાપરવું - સુરક્ષિત એક્સેસ પદ્ધતિ
print("રોલ નંબર (get વાપરીને):", student.get('roll_no'))
print("સરનામું (get વાપરીને):", student.get('address', 'ઉપલબ્ધ નથી'))  # જો કી ન મળે તો ડિફોલ્ટ વેલ્યુ

# મૂલ્યો સુધારવા
print("\nડિક્શનેરી સુધારવી:")
student['marks'] = 90
print("માર્ક્સ અપડેટ કર્યા પછી:", student)

# નવી કી-વેલ્યુ જોડી ઉમેરવી
student['address'] = 'New York'
print("સરનામું ઉમેર્યા પછી:", student)

# આઇટમ્સ દૂર કરવી
print("\nઆઇટમ્સ દૂર કરવી:")
removed_value = student.pop('address')
print("દૂર કરેલું મૂલ્ય:", removed_value)
print("pop() પછી:", student)

# છેલ્લે ઉમેરાયેલી આઇટમ દૂર કરવી
last_item = student.popitem()
print("છેલ્લે દૂર કરેલી આઇટમ:", last_item)
print("popitem() પછી:", student)

# ડિક્શનેરી મેથડ્સ
print("\nડિક્શનેરી મેથડ્સ:")
print("કીઝ:", list(student.keys()))
print("વેલ્યુઝ:", list(student.values()))
print("આઇટમ્સ:", list(student.items()))
\end{lstlisting}

\textbf{મુખ્ય ઓપરેશન્સ:}

\begin{itemize}
\tightlist
\item
  \textbf{એક્સેસ}: કી અથવા get() મેથડનો ઉપયોગ કરીને
\item
  \textbf{મોડિફાય}: અસ્તિત્વમાં રહેલી કીને નવું મૂલ્ય આપવું
\item
  \textbf{એડ}: નવી કીને મૂલ્ય આપવું
\item
  \textbf{રિમૂવ}: pop(), popitem(), અથવા del સ્ટેટમેન્ટનો ઉપયોગ કરીને
\item
  \textbf{ઇટરેટ}: કીઝ, વેલ્યુઝ, અથવા આઇટમ્સ દ્વારા
\end{itemize}

\end{solutionbox}
\begin{mnemonicbox}
``ACME ડિક્શનેરી'' - Access, Create, Modify, Extract
from ડિક્શનેરી

\end{mnemonicbox}

\end{document}
