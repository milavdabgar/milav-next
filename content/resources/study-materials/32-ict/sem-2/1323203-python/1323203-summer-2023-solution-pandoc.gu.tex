\documentclass[10pt,a4paper]{article}

% content/resources/templates/preamble.tex
\usepackage[margin=0.6in]{geometry}
\author{Milav Dabgar}
\usepackage{amsmath,amssymb,amsthm}
\usepackage{booktabs}
\usepackage{multirow}
\usepackage{xcolor}
\usepackage{tcolorbox}
\tcbuselibrary{breakable,skins}
\usepackage[colorlinks=true,linkcolor=blue]{hyperref}
\usepackage{titlesec}
\usepackage{enumitem}
\usepackage{tikz}
\usepackage{pgfplots}
\usepackage{circuitikz}
\usepackage[version=4]{mhchem}
\usepackage{longtable}
\usepackage{array}
\usepackage{float}
\usepackage{caption}
\usepackage{listings}

\lstset{
  basicstyle=\small\ttfamily,
  breaklines=true,
  breakatwhitespace=false,
  postbreak=\mbox{\textcolor{red}{$\hookrightarrow$}\space},
  float=false,
  numbers=left,
  numberstyle=\tiny\color{gray},
  numbersep=10pt,
  xleftmargin=2em,
  keywordstyle=\color{blue},
  commentstyle=\color{green!60!black},
  stringstyle=\color{purple},
  backgroundcolor=\color{gray!5},
  showstringspaces=false,
  tabsize=2,
  captionpos=b,
  keepspaces=true,
  columns=flexible
}

\pgfplotsset{compat=1.18}
\usetikzlibrary{shapes,arrows,positioning,calc,patterns,decorations.pathmorphing,decorations.markings,arrows.meta}

% Color scheme
\definecolor{headcolor}{RGB}{0,102,204}
\definecolor{keycolor}{RGB}{220,20,60}
\definecolor{solutioncolor}{RGB}{34,139,34}
\definecolor{mnemoniccolor}{RGB}{148,0,211}
\definecolor{codecolor}{RGB}{0,0,100}

% Spacing
\setlength{\parskip}{3pt}
\setlist[itemize]{nosep}
\setlist[enumerate]{nosep}

% Title formatting
\titleformat{\section}{\Large\bfseries\color{headcolor}}{\thesection}{1em}{}
\titleformat{\subsection}{\large\bfseries\color{headcolor}}{\thesubsection}{1em}{}

% Pandoc tightlist compatibility
\providecommand{\tightlist}{%
  \setlength{\itemsep}{0pt}\setlength{\parskip}{0pt}}

% Pandoc longtable compatibility
\newcounter{none}
\def\thenone{}


% content/resources/templates/gujarati-boxes.tex
\usepackage{fontspec}
\usepackage{polyglossia}

% Set Gujarati as main language (document is primarily in Gujarati)
% Note: gloss-gujarati.ldf doesn't exist in polyglossia, but it will use hyphenation patterns
\setdefaultlanguage{gujarati}
\setotherlanguage{english}

% Configure Gujarati font properly
% Use Language=Default to prevent polyglossia from trying to add language-specific features
% that don't exist for Gujarati, which causes "empty feature" warnings
\newfontfamily\gujaratifont[Script=Gujarati,AutoFakeBold=2.5,AutoFakeSlant=0.3]{Noto Sans Gujarati}
\setmainfont[Script=Gujarati,AutoFakeBold=2.5,AutoFakeSlant=0.3]{Noto Sans Gujarati}
% Use Noto Sans Gujarati for monospace to support Gujarati in text
\setmonofont[Scale=0.9]{Noto Sans Gujarati}

% Configure English to use the same font
\newfontfamily\englishfont[Script=Gujarati,AutoFakeBold=2.5,AutoFakeSlant=0.3]{Noto Sans Gujarati}

% Translations for polyglossia
\gappto\captionsgujarati{
  \renewcommand{\tablename}{કોષ્ટક}
  \renewcommand{\figurename}{આકૃતિ}
}

% Helper for TikZ nodes to ensure Gujarati font
\newcommand{\gu}[1]{{\gujaratifont #1}}

% Custom environments
\newtcolorbox{solutionbox}{
    breakable,
    enhanced,
    colback=solutioncolor!5!white,
    colframe=solutioncolor!75!black,
    fonttitle=\bfseries,
    title=જવાબ
}

\newtcolorbox{solutionboxnobreak}{
 colback=solutioncolor!5!white,
 colframe=solutioncolor!75!black,
 fonttitle=\bfseries,
 title=જવાબ
}

\newtcolorbox{keyformula}{
 breakable,
 enhanced,
 colback=keycolor!5!white,
 colframe=keycolor!75!black,
 fonttitle=\bfseries,
 title=રાસાયણિક સમીકરણ/સૂત્ર
}

\newtcolorbox{mnemonicbox}{
 breakable,
 enhanced,
 colback=mnemoniccolor!5!white,
 colframe=mnemoniccolor!75!black,
 fonttitle=\bfseries,
 title=મેમરી ટ્રીક
}


\begin{document}

\begin{center}
{\Huge\bfseries\color{headcolor} Subject Name (Gujarati)}\\[5pt]
{\LARGE 1323203 -- Summer 2023}\\[3pt]
{\large Semester 1 Study Material}\\[3pt]
{\normalsize\textit{Detailed Solutions and Explanations}}
\end{center}

\vspace{10pt}

\subsection*{પ્રશ્ન 1(અ) [3
ગુણ]}\label{uxaaauxab0uxab6uxaa8-1uxa85-3-uxa97uxaa3}

\textbf{અલગોરીધમ વ્યાખ્યાયિત કરો. અલગોરીધમનાં ફાયદા શું છે?}

\begin{solutionbox}
અલગોરીધમ એ ચોક્કસ સમસ્યાને ઉકેલવા માટે પગલાઓના ક્રમબદ્ધ સમૂહ અથવા
નિયમોનો સેટ છે.

\textbf{અલગોરીધમના ફાયદા:}

\begin{itemize}
\tightlist
\item
  \textbf{સ્પષ્ટતા}: સ્પષ્ટ, અસંદિગ્ધ સૂચનાઓ પ્રદાન કરે છે
\item
  \textbf{કાર્યક્ષમતા}: સમય અને સંસાધનોને અનુકૂળ બનાવવામાં મદદ કરે છે
\item
  \textbf{પુન:ઉપયોગ}: સમાન સમસ્યાઓ માટે વારંવાર ઉપયોગ કરી શકાય છે
\item
  \textbf{ચકાસણી}: અમલીકરણ પહેલાં પરીક્ષણ અને ડિબગ કરવું સરળ
\item
  \textbf{સંદેશાવ્યવહાર}: ઉકેલને સંદેશાવ્યવહાર કરવા માટે બ્લુપ્રિન્ટ તરીકે કામ કરે છે
\end{itemize}

\end{solutionbox}
\begin{mnemonicbox}
``CERVC'' (Clarity, Efficiency, Reusability,
Verification, Communication)

\end{mnemonicbox}
\subsection*{પ્રશ્ન 1(બ) [4
ગુણ]}\label{uxaaauxab0uxab6uxaa8-1uxaac-4-uxa97uxaa3}

\textbf{ફલોચાર્ટનો ઉપયોગ કરીને સમસ્યા ઉકેલવાના નિયમો શું છે? સાદું વ્યાજ શોધવા
માટેનો ફલોચાર્ટ ડીઝાઈન કરો.}

\begin{solutionbox}
ફલોચાર્ટનો ઉપયોગ કરીને સમસ્યા ઉકેલવાના નિયમો:

\begin{itemize}
\tightlist
\item
  \textbf{યોગ્ય સિમ્બોલ}: વિવિધ ઓપરેશન માટે માનક સિમ્બોલનો ઉપયોગ કરવો
\item
  \textbf{દિશાનો પ્રવાહ}: હંમેશા ઉપરથી નીચે, ડાબેથી જમણે સ્પષ્ટ પ્રવાહ જાળવવો
\item
  \textbf{એક એન્ટ્રી/એક્ઝિટ}: સ્પષ્ટ શરૂઆત અને અંત બિંદુ હોવા જોઈએ
\item
  \textbf{સ્પષ્ટતા}: પગલાં સ્પષ્ટ અને સંક્ષિપ્ત રાખવા
\item
  \textbf{સુસંગતતા}: વિગતોનું સુસંગત સ્તર જાળવવું
\end{itemize}

\textbf{સાદું વ્યાજ ગણતરી માટેનો ફલોચાર્ટ:}

\begin{verbatim}
flowchart LR
    A([શરૂઆત]) {-{-} B[/મૂળ રકમ P, વ્યાજ દર R, સમય T ઇનપુટ કરો/]}
    B {-{-} C[SI = P * R * T / 100]}
    C {-{-} D[/SI આઉટપુટ કરો/]}
    D {-{-} E([અંત])}
\end{verbatim}

\end{solutionbox}
\begin{mnemonicbox}
``PDRSC'' (Proper symbols, Direction flow, Required
entry/exit, Simplicity, Consistency)

\end{mnemonicbox}
\subsection*{પ્રશ્ન 1(ક) [7
ગુણ]}\label{uxaaauxab0uxab6uxaa8-1uxa95-7-uxa97uxaa3}

\textbf{પાયથોનનાં અસાઇમેંટ ઓપરટેરની યાદી બનાવો અને કોઈપણ ત્રણ અસાઇમેંટ ઓપરટેરોની
કામગીરી દશાર્વવા માટે પાયથોન કોડ બનાવો.}

\begin{solutionbox}
પાયથોન અસાઇમેંટ ઓપરેટર્સ:

{\def\LTcaptype{none} % do not increment counter
\begin{longtable}[]{@{}lll@{}}
\toprule\noalign{}
ઓપરેટર & ઉદાહરણ & સમકક્ષ \\
\midrule\noalign{}
\endhead
\bottomrule\noalign{}
\endlastfoot
= &

x = 5 &

x = 5 \\

+= & x += 5 &

x = x + 5 \\

-= & x -= 5 &

x = x - 5 \\

*= & x *= 5 &

x = x * 5 \\

/= & x /= 5 &

x = x / 5 \\

\%= & x \%= 5 & x = x \% 5 \\
//= & x //= 5 &

x = x // 5 \\

**= & x **= 5 &

x = x ** 5 \\

\&= & x \&= 5 & x = x \& 5 \\
\textbar= & x \textbar= 5 & x = x \textbar{} 5 \\
\^{}= & x \^{}= 5 & x = x \^{} 5 \\
\textgreater\textgreater= & x \textgreater\textgreater= 5 & x = x
\textgreater\textgreater{} 5 \\
\textless\textless= & x \textless\textless= 5 & x = x
\textless\textless{} 5 \\
\end{longtable}
}

\textbf{અસાઇમેંટ ઓપરેટર્સ દર્શાવતો કોડ:}

\begin{verbatim}
\# અસાઇમેંટ ઓપરેટર્સનું પ્રદર્શન
num = 10
print("પ્રારંભિક મૂલ્ય:", num)

\# += ઓપરેટરનો ઉપયોગ
num += 5
print("+= 5 પછી:", num)  \# આઉટપુટ: 15

\# {-= ઓપરેટરનો ઉપયોગ}
num {-=} 3
print("{-= 3 પછી:"}, num)  \# આઉટપુટ: 12

\# *= ઓપરેટરનો ઉપયોગ
num *= 2
print("*= 2 પછી:", num)  \# આઉટપુટ: 24
\end{verbatim}

\end{solutionbox}
\begin{mnemonicbox}
``VALUE'' (Variable Assignment is Like Updating
Existing values)

\end{mnemonicbox}
\subsection*{પ્રશ્ન 1(ક) OR [7
ગુણ]}\label{uxaaauxab0uxab6uxaa8-1uxa95-or-7-uxa97uxaa3}

\textbf{પાયથોનનાં ડેટા ટાઇપ્સની યાદી બનાવો અને કોઈપણ ત્રણ ડેટા ટાઇપ્સને ઓળખવા
માટેનો પાયથોન કોડ બનાવો.}

\begin{solutionbox}
પાયથોન ડેટા ટાઇપ્સ:

{\def\LTcaptype{none} % do not increment counter
\begin{longtable}[]{@{}lll@{}}
\toprule\noalign{}
ડેટા ટાઇપ & વર્ણન & ઉદાહરણ \\
\midrule\noalign{}
\endhead
\bottomrule\noalign{}
\endlastfoot
int & ઇન્ટીજર (પૂર્ણાંક સંખ્યાઓ) & 42 \\
float & ફ્લોટિંગ પોઇન્ટ (દશાંશ) & 3.14 \\
str & સ્ટ્રિંગ (ટેક્સ્ટ) & ``Hello'' \\
bool & બૂલિયન (True/False) & True \\
list & ક્રમિક, પરિવર્તનશીલ સંગ્રહ & [1, 2, 3] \\
tuple & ક્રમિક, અપરિવર્તનીય સંગ્રહ & (1, 2, 3) \\
set & અક્રમિક સંગ્રહ & \{1, 2, 3\} \\
dict & કી-વેલ્યુ જોડી & \{``name'': ``John''\} \\
complex & કોમ્પ્લેક્સ નંબર & 2+3j \\
NoneType & None દર્શાવે છે & None \\
\end{longtable}
}

\textbf{ત્રણ ડેટા ટાઇપ્સ ઓળખવા માટેનો કોડ:}

\begin{verbatim}
\# ડેટા ટાઇપ્સ ઓળખવાનો પ્રોગ્રામ
def identify\_data\_type(value):
    data\_type = type(value).\_\_name\_\_
    print(f"મૂલ્ય: \{value\}")
    print(f"ડેટા ટાઇપ: \{data\_type\}")
    print("{-"} * 20)

\# 3 અલગ{-અલગ ડેટા ટાઇપ્સ સાથે ટેસ્ટિંગ}
identify\_data\_type(42)            \# Integer
identify\_data\_type(3.14)          \# Float
identify\_data\_type("Hello World") \# String

\# આઉટપુટ:
\# મૂલ્ય: 42
\# ડેટા ટાઇપ: int
\# {-{-}{-}{-}{-}{-}{-}{-}{-}{-}{-}{-}{-}{-}{-}{-}{-}{-}{-}{-}}
\# મૂલ્ય: 3.14
\# ડેટા ટાઇપ: float
\# {-{-}{-}{-}{-}{-}{-}{-}{-}{-}{-}{-}{-}{-}{-}{-}{-}{-}{-}{-}}
\# મૂલ્ય: Hello World
\# ડેટા ટાઇપ: str
\# {-{-}{-}{-}{-}{-}{-}{-}{-}{-}{-}{-}{-}{-}{-}{-}{-}{-}{-}{-}}
\end{verbatim}

\end{solutionbox}
\begin{mnemonicbox}
``TYPE-ID'' (Tell Your Python Elements - Identify
Data)

\end{mnemonicbox}
\subsection*{પ્રશ્ન 2(અ) [3
ગુણ]}\label{uxaaauxab0uxab6uxaa8-2uxa85-3-uxa97uxaa3}

\textbf{સ્યુડોકોડ વ્યાખ્યાયિત કરો. કોઈપણ બે સંખ્યા માંથી સૌથી નાની સંખ્યા શોધવા
માટે સ્યુડોકોડ લખો.}

\begin{solutionbox}
સ્યુડોકોડ એ એલ્ગોરિધમનું ઉચ્ચ-સ્તરીય વર્ણન છે જે પ્રોગ્રામિંગ ભાષાના
માળખાકીય સંકેતોનો ઉપયોગ કરે છે પરંતુ મશીન વાંચન કરતાં માનવ વાંચન માટે ડિઝાઇન કરેલ
છે.

\textbf{બે સંખ્યાઓમાંથી સૌથી નાની શોધવા માટે સ્યુડોકોડ:}

\begin{verbatim}
BEGIN
    INPUT first_number, second_number
    IF first_number < second_number THEN
        smallest = first_number
    ELSE
        smallest = second_number
    END IF
    OUTPUT smallest
END
\end{verbatim}

\end{solutionbox}
\begin{mnemonicbox}
``RISE'' (Read Input, Select smallest, Echo result)

\end{mnemonicbox}
\subsection*{પ્રશ્ન 2(બ) [4
ગુણ]}\label{uxaaauxab0uxab6uxaa8-2uxaac-4-uxa97uxaa3}

\textbf{યુઝર્સ પાસેથી ત્રણ ઇનપુટ વાંચો અને સંખ્યાઓની સરેરાશ શોધવા માટેનો પાયથોન કોડ
વિકસાવો.}

\begin{solutionbox}

\begin{verbatim}
\# ત્રણ સંખ્યાઓની સરેરાશ ગણવા માટેનો પ્રોગ્રામ
\# વપરાશકર્તા પાસેથી ત્રણ સંખ્યાઓ લો
num1 = float(input("પ્રથમ સંખ્યા દાખલ કરો: "))
num2 = float(input("બીજી સંખ્યા દાખલ કરો: "))
num3 = float(input("ત્રીજી સંખ્યા દાખલ કરો: "))

\# સરેરાશની ગણતરી
average = (num1 + num2 + num3) / 3

\# પરિણામ દર્શાવો
print(f"\{num1\}, \{num2\}, અને \{num3\}ની સરેરાશ: \{average\}")
\end{verbatim}

\textbf{આકૃતિ:}

\begin{verbatim}
flowchart LR
    A([શરૂઆત]) {-{-} B[/num1, num2, num3 ઇનપુટ કરો/]}
    B {-{-} C["average = (num1 + num2 + num3) / 3"]}
    C {-{-} D[/average આઉટપુટ કરો/]}
    D {-{-} E([અંત])}
\end{verbatim}

\end{solutionbox}
\begin{mnemonicbox}
``I-ADD-D'' (Input three, ADD them up, Divide by 3)

\end{mnemonicbox}
\subsection*{પ્રશ્ન 2(ક) [7
ગુણ]}\label{uxaaauxab0uxab6uxaa8-2uxa95-7-uxa97uxaa3}

\textbf{દાખલ કરેલ સંખ્યા prime છે કે નહીં તે બતાવવા પાયથોન કોડ લખો.}

\begin{solutionbox}

\begin{verbatim}
\# સંખ્યા પ્રાઇમ છે કે નહીં તે તપાસવાનો પ્રોગ્રામ
\# વપરાશકર્તા પાસેથી સંખ્યા લો
num = int(input("એક સંખ્યા દાખલ કરો: "))

\# 2થી ઓછી સંખ્યા છે કે નહીં તપાસો
if num {} 2:
    print(f"\{num\} એક પ્રાઇમ સંખ્યા નથી")
else:
    \# is\_prime ને True તરીકે આરંભો
    is\_prime = True
    
    \# 2 થી sqrt(num) સુધી તપાસો
    for i in range(2, int(num**0.5) + 1):
if num \%

i == 0:

            is\_prime = False
            break
    
    \# પરિણામ દર્શાવો
    if is\_prime:
        print(f"\{num\} એક પ્રાઇમ સંખ્યા છે")
    else:
        print(f"\{num\} એક પ્રાઇમ સંખ્યા નથી")
\end{verbatim}

\textbf{આકૃતિ:}

\begin{verbatim}
flowchart LR
    A("શરૂાત") {-{-} B("num ઇનપુટ કરો")}
    B {-{-} C\{"num  2?"\}}
    C {-{-}|હા| D("num પ્રાઇમ નથી")}
    C {-{-}|ના| E("is\_prime = True")}
    E {-{-} F("i = 2")}
    F {-{-} G\{"(i * i) = num?"\}}
    G {-{-}|ના| J("is\_prime = True")}
G {-{-}|હા| H\{"num \%

i == 0?"\}}

    H {-{-}|હા| I("is\_prime = False")}
    I {-{-} J}
    H {-{-}|ના| K("i = i + 1")}
    K {-{-} F}
    J {-{-}|હા| L("num પ્રાઇમ છે")}
    J {-{-}|ના| D}
    L {-{-} M("અંત")}
    D {-{-} M}
\end{verbatim}

\end{solutionbox}
\begin{mnemonicbox}
``PRIME'' (Positive number, Range check from 2 to
\sqrtn, If divisible it's Multiple, Else it's prime)

\end{mnemonicbox}
\subsection*{પ્રશ્ન 2(અ) OR [3
ગુણ]}\label{uxaaauxab0uxab6uxaa8-2uxa85-or-3-uxa97uxaa3}

\textbf{ફલોચાર્ટ અને એલ્ગોરિધમ વચ્ચેનો તફાવત લખો.}

\begin{solutionbox}

{\def\LTcaptype{none} % do not increment counter
\begin{longtable}[]{@{}
  >{\raggedright\arraybackslash}p{(\linewidth - 2\tabcolsep) * \real{0.5217}}
  >{\raggedright\arraybackslash}p{(\linewidth - 2\tabcolsep) * \real{0.4783}}@{}}
\toprule\noalign{}
\begin{minipage}[b]{\linewidth}\raggedright
ફલોચાર્ટ
\end{minipage} & \begin{minipage}[b]{\linewidth}\raggedright
એલ્ગોરિધમ
\end{minipage} \\
\midrule\noalign{}
\endhead
\bottomrule\noalign{}
\endlastfoot
માનક સિમ્બોલ અને આકારોનો ઉપયોગ કરીને \textbf{દૃશ્ય પ્રતિનિધિત્વ} & \textbf{લેખિત
વર્ણન} માળખાકીય ભાષાનો ઉપયોગ કરીને \\
ગ્રાફિકલ પ્રકૃતિને કારણે \textbf{સમજવું સરળ} & સિન્ટેક્સ અને શબ્દાવલીનું જ્ઞાન જરૂરી \\
\textbf{તાર્કિક પ્રવાહ} અને સંબંધોને સ્પષ્ટ રીતે દર્શાવે & ક્રમિક ક્રમમાં
\textbf{વિગતવાર પગલાં} પ્રદાન કરે \\
બનાવવા માટે \textbf{સમય-લેતી} પરંતુ સમજવા માટે સરળ & \textbf{ઝડપથી ડ્રાફ્ટ} પરંતુ
સમજવામાં મુશ્કેલ હોઈ શકે \\
ફેરફાર કરવા કે અપડેટ કરવા વધુ મુશ્કેલ & ફેરફાર કરવા કે અપડેટ કરવા વધુ સરળ \\
\end{longtable}
}

\end{solutionbox}
\begin{mnemonicbox}
``VITAL'' (Visual vs Textual, Interpretation ease,
Time to create, Alteration flexibility, Logical representation)

\end{mnemonicbox}
\subsection*{પ્રશ્ન 2(બ) OR [4
ગુણ]}\label{uxaaauxab0uxab6uxaa8-2uxaac-or-4-uxa97uxaa3}

\textbf{નીચેનાં કોડનું આઉટપુટ શું છે?}

\begin{verbatim}
x=10
y=2
print (x*y)
print (x ** y)
print (x//y)
print (x \% y)
\end{verbatim}

\begin{solutionbox}

{\def\LTcaptype{none} % do not increment counter
\begin{longtable}[]{@{}lll@{}}
\toprule\noalign{}
ઓપરેશન & સમજૂતી & આઉટપુટ \\
\midrule\noalign{}
\endhead
\bottomrule\noalign{}
\endlastfoot
x*y & ગુણાકાર: 10 \times 2 & 20 \\
x**y & ઘાતાંક: 10^{2} & 100 \\
x//y & પૂર્ણાંક ભાગાકાર: 10 \div 2 & 5 \\
x\%y & મોડ્યુલસ (શેષ): 10 \div 2 & 0 \\
\end{longtable}
}

\end{solutionbox}
\begin{mnemonicbox}
``MEMO'' (Multiply, Exponent, Modulo, Operations)

\end{mnemonicbox}
\subsection*{પ્રશ્ન 2(ક) OR [7
ગુણ]}\label{uxaaauxab0uxab6uxaa8-2uxa95-or-7-uxa97uxaa3}

\textbf{નીચેની પેટર્ન દર્શાવવા પાયથોન કોડ લખો:}

\begin{verbatim}
A)                    B)
1                    * * * *
1 2                  * * *
1 2 3                * *
1 2 3 4              *
\end{verbatim}

\begin{solutionbox}

\begin{verbatim}
\# પેટર્ન A: સંખ્યા પેટર્ન
print("પેટર્ન A:")
for i in range(1, 5):
    for j in range(1, i + 1):
        print(j, end=" ")
    print()

\# પેટર્ન B: તારા પેટર્ન
print("{n}પેટર્ન B:")
for i in range(4, 0, {-}1):
    for j in range(i):
        print("*", end=" ")
    print()
\end{verbatim}

\textbf{આકૃતિ:}

\begin{verbatim}
flowchart TD
    A([શરૂઆત]) {-{-} B[પેટર્ન A]}
    B {-{-} C[i = 1 થી 4]}
    C {-{-} D[j = 1 થી i]}
    D {-{-} E[j પ્રિન્ટ કરો]}
    E {-{-} F[ઇનર લૂપ પછી નવી લાઇન]}
    F {-{-} G[પેટર્ન B]}
    G {-{-} H[i = 4 થી 1]}
    H {-{-} I[j = 0 થી i{-}1]}
    I {-{-} J["* પ્રિન્ટ કરો"]}
    J {-{-} K[ઇનર લૂપ પછી નવી લાઇન]}
    K {-{-} L([અંત])}
\end{verbatim}

\end{solutionbox}
\begin{mnemonicbox}
``LOOP-NED'' (Loop Outer, Order Pattern, Nested
loops, End with newline, Display)

\end{mnemonicbox}
\subsection*{પ્રશ્ન 3(અ) [3
ગુણ]}\label{uxaaauxab0uxab6uxaa8-3uxa85-3-uxa97uxaa3}

\textbf{જરૂરી ઉદાહરણો સાથે break statementનાં ઉપયોગનું વર્ણન કરો.}

\begin{solutionbox}
break સ્ટેટમેન્ટનો ઉપયોગ લૂપને વચ્ચેથી સમાપ્ત કરવા માટે થાય છે, જ્યારે
કોઈ ચોક્કસ શરત પૂરી થાય.

\textbf{ઉદાહરણ:}

\begin{verbatim}
\# લિસ્ટમાં પ્રથમ વિષમ સંખ્યા શોધવી
numbers = [2, 4, 6, 7, 8, 10]
for num in numbers:
    if num \% 2 != 0:
        print(f"વિષમ સંખ્યા મળી: \{num\}")
        break
    print(f"\{num\} તપાસી રહ્યા છીએ")
\end{verbatim}

\textbf{આઉટપુટ:}

\begin{verbatim}
2 તપાસી રહ્યા છીએ
4 તપાસી રહ્યા છીએ
6 તપાસી રહ્યા છીએ
વિષમ સંખ્યા મળી: 7
\end{verbatim}

\end{solutionbox}
\begin{mnemonicbox}
``EXIT'' (EXecute until condition, Immediately
Terminate)

\end{mnemonicbox}
\subsection*{પ્રશ્ન 3(બ) [4
ગુણ]}\label{uxaaauxab0uxab6uxaa8-3uxaac-4-uxa97uxaa3}

\textbf{યોગ્ય ઉદાહરણ સાથે if\ldots else statement સમજાવો.}

\begin{solutionbox}
if\ldots else સ્ટેટમેન્ટ એ એક કન્ડિશનલ સ્ટેટમેન્ટ છે જે નિર્દિષ્ટ શરત
True કે False હોવાના આધારે અલગ-અલગ કોડ બ્લોક્સ એક્ઝિક્યુટ કરે છે.

\textbf{સિન્ટેક્સ:}

\begin{verbatim}
if શરત:
    \# જો શરત True હોય તો આ કોડ એક્ઝિક્યુટ થશે
else:
    \# જો શરત False હોય તો આ કોડ એક્ઝિક્યુટ થશે
\end{verbatim}

\textbf{ઉદાહરણ:}

\begin{verbatim}
\# સંખ્યા સમ છે કે વિષમ તે તપાસવું
number = int(input("એક સંખ્યા દાખલ કરો: "))

if number \% 2 == 0:
    print(f"\{number\} એક સમ સંખ્યા છે")
else:
    print(f"\{number\} એક વિષમ સંખ્યા છે")
\end{verbatim}

\textbf{આકૃતિ:}

\begin{verbatim}
flowchart LR
    A([શરૂઆત]) {-{-} B[/સંખ્યા ઇનપુટ કરો/]}
    B {-{-} C\{number \% 2 == 0?\}}
    C {-{-}|હા| D[/"number સમ છે" પ્રિન્ટ કરો/]}
    C {-{-}|ના| E[/"number વિષમ છે" પ્રિન્ટ કરો/]}
    D {-{-} F([અંત])}
    E {-{-} F}
\end{verbatim}

\end{solutionbox}
\begin{mnemonicbox}
``CITE'' (Check condition, If True Execute this,
Else execute that)

\end{mnemonicbox}
\subsection*{પ્રશ્ન 3(ક) [7
ગુણ]}\label{uxaaauxab0uxab6uxaa8-3uxa95-7-uxa97uxaa3}

\textbf{0 થી N સંખ્યા સુધીની ફીબોનાકી શ્રેણી પ્રિન્ટ કરવા માટે યુઝર ડેફાઇન ફંકશન
બનાવો જેમાં N એ પૂર્ણાંક સંખ્યા છે અને આરગયુમેન્ટ તરીકે પાસ થાય છે.}

\begin{solutionbox}

\begin{verbatim}
\# ફીબોનાકી શ્રેણી પ્રિન્ટ કરવા માટેનું ફંકશન
def print\_fibonacci(n):
    """
    0 થી n સુધીની ફીબોનાકી શ્રેણી પ્રિન્ટ કરે છે
    આરગયુમેન્ટ:
        n: ઉપરની લિમિટ (inclusive)
    """
    \# પ્રથમ બે પદો ઇનિશિયલાઇઝ કરો
    a, b = 0, 1
    
    \# n માન્ય છે કે નહીં તે તપાસો
    if n {} 0:
        print("કૃપા કરીને એક હકારાત્મક સંખ્યા દાખલ કરો")
        return
    
    \# ફીબોનાકી શ્રેણી પ્રિન્ટ કરો
    print(n, "સુધીની ફીબોનાકી શ્રેણી:")
    
    if n {=} 0:
        print(a, end=" ")  \# પ્રથમ પદ પ્રિન્ટ કરો
    
    if n {=} 1:
        print(b, end=" ")  \# બીજો પદ પ્રિન્ટ કરો
    
    \# બાકીની શ્રેણી બનાવો અને પ્રિન્ટ કરો
    while a + b {=} n:
        next\_term = a + b
        print(next\_term, end=" ")
        a, b = b, next\_term

\# ફંકશનનું ટેસ્ટિંગ
print\_fibonacci(55)
\end{verbatim}

\textbf{આકૃતિ:}

\begin{verbatim}
flowchart LR
A([શરૂઆત]) {-{-} B[a=0,

b=1 ઇનિશિયલાઇઝ કરો]}

    B {-{-} C\{n  0?\}}
    C {-{-}|હા| D[એરર મેસેજ પ્રિન્ટ કરો]}
    D {-{-} K([અંત])}
    C {-{-}|ના| E\{n = 0?\}}
    E {-{-}|હા| F[a પ્રિન્ટ કરો]}
    E {-{-}|ના| G\{n = 1?\}}
    F {-{-} G}
    G {-{-}|હા| H[b પ્રિન્ટ કરો]}
    G {-{-}|ના| I\{a + b = n?\}}
    H {-{-} I}
I {-{-}|હા| J[next\_term = a + bn next\_term પ્રિન્ટ કરોn

a = bn

b = next\_term]}

    J {-{-} I}
    I {-{-}|ના| K}
\end{verbatim}

\end{solutionbox}
\begin{mnemonicbox}
``FIBER'' (First terms set, Initialize variables,
Build next term, Echo results, Repeat until limit)

\end{mnemonicbox}
\subsection*{પ્રશ્ન 3(અ) OR [3
ગુણ]}\label{uxaaauxab0uxab6uxaa8-3uxa85-or-3-uxa97uxaa3}

\textbf{જરૂરી ઉદાહરણો સાથે continue statementનાં ઉપયોગનું વર્ણન કરો.}

\begin{solutionbox}
continue સ્ટેટમેન્ટનો ઉપયોગ લૂપની વર્તમાન ઇટરેશન છોડીને આગળની
ઇટરેશન પર જવા માટે થાય છે.

\textbf{ઉદાહરણ:}

\begin{verbatim}
\# 1 થી 10 સુધીની માત્ર વિષમ સંખ્યાઓ પ્રિન્ટ કરવી
for i in range(1, 11):
    if i \% 2 == 0:
        continue  \# સમ સંખ્યાઓ છોડી દો
    print(i)
\end{verbatim}

\textbf{આઉટપુટ:}

\begin{verbatim}
1
3
5
7
9
\end{verbatim}

\end{solutionbox}
\begin{mnemonicbox}
``SKIP'' (Skip current iteration, Keep looping,
Ignore remaining statements, Proceed to next iteration)

\end{mnemonicbox}
\subsection*{પ્રશ્ન 3(બ) OR [4
ગુણ]}\label{uxaaauxab0uxab6uxaa8-3uxaac-or-4-uxa97uxaa3}

\textbf{ઉદાહરણ સાથે For loop statement સમજાવો.}

\begin{solutionbox}
For લૂપનો ઉપયોગ કોઈ સિક્વન્સ (જેમ કે લિસ્ટ, ટપલ, સ્ટ્રિંગ) કે અન્ય
ઇટરેબલ ઓબ્જેક્ટ પર ઇટરેશન કરવા અને દરેક આઇટમ માટે કોડનો બ્લોક એક્ઝિક્યુટ કરવા માટે
થાય છે.

\textbf{સિન્ટેક્સ:}

\begin{verbatim}
for વેરિએબલ in સિક્વન્સ:
    \# દરેક આઇટમ માટે એક્ઝિક્યુટ થનાર કોડ
\end{verbatim}

\textbf{ઉદાહરણ:}

\begin{verbatim}
\# 1 થી 5 સુધીની સંખ્યાઓના વર્ગ પ્રિન્ટ કરવા
for num in range(1, 6):
    square = num ** 2
    print(f"\{num\}નો વર્ગ \{square\} છે")
\end{verbatim}

\textbf{આઉટપુટ:}

\begin{verbatim}
1નો વર્ગ 1 છે
2નો વર્ગ 4 છે
3નો વર્ગ 9 છે
4નો વર્ગ 16 છે
5નો વર્ગ 25 છે
\end{verbatim}

\textbf{આકૃતિ:}

\begin{verbatim}
flowchart LR
    A([શરૂઆત]) {-{-} B[num = 1 થી 5 સુધી]}
    B {-{-} C[square = num ** 2]}
    C {-{-} D[પરિણામ પ્રિન્ટ કરો]}
    D {-{-} E\{વધુ આઇટમ્સ?\}}
    E {-{-}|હા| B}
    E {-{-}|ના| F([અંત])}
\end{verbatim}

\end{solutionbox}
\begin{mnemonicbox}
``FIRE'' (For each Item, Run commands, Execute until
end)

\end{mnemonicbox}
\subsection*{પ્રશ્ન 3(ક) OR [7
ગુણ]}\label{uxaaauxab0uxab6uxaa8-3uxa95-or-7-uxa97uxaa3}

\textbf{યુઝર ડિફાઇન ફંકશનની મદદથી આપેલ નંબર આર્મસ્ટ્રોંગ નંબર છે કે પેલિન્ડ્રોમ તે
નિર્ધારિત કરવા પાયથોન કોડ લખો.}

\begin{solutionbox}

\begin{verbatim}
\# સંખ્યા આર્મસ્ટ્રોંગ નંબર છે કે નહીં તે તપાસવાનું ફંકશન
def is\_armstrong(num):
    \# ડિજિટ્સની સંખ્યા ગણવા માટે સ્ટ્રિંગમાં રૂપાંતરિત કરો
    num\_str = str(num)
    n = len(num\_str)
    
    \# દરેક ડિજિટને n ઘાત પર ઊંચકી તેનો સરવાળો કરો
    sum\_of\_powers = sum(int(digit) ** n for digit in num\_str)
    
    \# સરવાળો મૂળ સંખ્યા સાથે સરખાવો
    return sum\_of\_powers == num

\# સંખ્યા પેલિન્ડ્રોમ છે કે નહીં તે તપાસવાનું ફંકશન
def is\_palindrome(num):
    \# સ્ટ્રિંગમાં રૂપાંતરિત કરો
    num\_str = str(num)
    
    \# સ્ટ્રિંગ તેના રિવર્સ સાથે સરખાવો
    return num\_str == num\_str[::{-}1]

\# બંને સ્થિતિઓ તપાસવા માટેનું મુખ્ય ફંકશન
def check\_number(num):
    if is\_armstrong(num):
        print(f"\{num\} એક આર્મસ્ટ્રોંગ નંબર છે")
    else:
        print(f"\{num\} એક આર્મસ્ટ્રોંગ નંબર નથી")
    
    if is\_palindrome(num):
        print(f"\{num\} એક પેલિન્ડ્રોમ છે")
    else:
        print(f"\{num\} એક પેલિન્ડ્રોમ નથી")

\# ફંકશનનું ટેસ્ટિંગ
number = int(input("એક સંખ્યા દાખલ કરો: "))
check\_number(number)
\end{verbatim}

\textbf{આકૃતિ:}

\begin{verbatim}
flowchart LR
    A([શરૂઆત]) {-{-} B[/સંખ્યા ઇનપુટ કરો/]}
    B {-{-} C[check\_number ફંકશન કૉલ કરો]}
    C {-{-} D[is\_armstrong ફંકશન કૉલ કરો]}
    D {-{-} E\{sum\_of\_powers == num?\}}
    E {-{-}|હા| F[/"આર્મસ્ટ્રોંગ છે" પ્રિન્ટ કરો/]}
    E {-{-}|ના| G[/"આર્મસ્ટ્રોંગ નથી" પ્રિન્ટ કરો/]}
    C {-{-} H[is\_palindrome ફંકશન કૉલ કરો]}
    H {-{-} I\{num == reversed num?\}}
    I {-{-}|હા| J[/"પેલિન્ડ્રોમ છે" પ્રિન્ટ કરો/]}
    I {-{-}|ના| K[/"પેલિન્ડ્રોમ નથી" પ્રિન્ટ કરો/]}
    F {-{-} L([અંત])}
    G {-{-} L}
    J {-{-} L}
    K {-{-} L}
\end{verbatim}

\end{solutionbox}
\begin{mnemonicbox}
``APC'' (Armstrong check: Power sum of digits,
Palindrome check: Compare with reverse)

\end{mnemonicbox}
\subsection*{પ્રશ્ન 4(અ) [3
ગુણ]}\label{uxaaauxab0uxab6uxaa8-4uxa85-3-uxa97uxaa3}

\textbf{સ્કેન કરેલ નંબર even છે કે odd તે શોધવા પાયથોન કોડ વિકસાવો અને યોગ્ય મેસેજ
પ્રિન્ટ કરો.}

\begin{solutionbox}

\begin{verbatim}
\# સંખ્યા સમ છે કે વિષમ તે તપાસવાનો પ્રોગ્રામ
\# વપરાશકર્તા પાસેથી સંખ્યા લો
number = int(input("એક સંખ્યા દાખલ કરો: "))

\# સંખ્યા સમ છે કે વિષમ તે તપાસો
if number \% 2 == 0:
    print(f"\{number\} એક સમ સંખ્યા છે")
else:
    print(f"\{number\} એક વિષમ સંખ્યા છે")
\end{verbatim}

\textbf{આકૃતિ:}

\begin{verbatim}
flowchart LR
    A([શરૂઆત]) {-{-} B[/સંખ્યા ઇનપુટ કરો/]}
    B {-{-} C\{number \% 2 == 0?\}}
    C {-{-}|હા| D[/"number સમ છે" પ્રિન્ટ કરો/]}
    C {-{-}|ના| E[/"number વિષમ છે" પ્રિન્ટ કરો/]}
    D {-{-} F([અંત])}
    E {-{-} F}
\end{verbatim}

\end{solutionbox}
\begin{mnemonicbox}
``MODE'' (Modulo Operation Determines Even-odd)

\end{mnemonicbox}
\subsection*{પ્રશ્ન 4(બ) [4
ગુણ]}\label{uxaaauxab0uxab6uxaa8-4uxaac-4-uxa97uxaa3}

\textbf{ફંકશનની વ્યાખ્યા આપો. યુઝર ડિફાઇન ફંકશન યોગ્ય ઉદાહરણ આપી સમજાવો.}

\begin{solutionbox}
ફંકશન એ કોડનો એવો બ્લોક છે જે ચોક્કસ કાર્ય કરવા માટે વ્યવસ્થિત અને
ફરીથી ઉપયોગ કરી શકાય છે. યુઝર-ડિફાઇન ફંકશન એ પ્રોગ્રામર દ્વારા બનાવવામાં આવેલા
ફંકશન છે જે કસ્ટમ ઓપરેશન કરે છે.

\textbf{યુઝર-ડિફાઇન ફંકશનના ઘટકો:}

\begin{itemize}
\tightlist
\item
  \textbf{def કીવર્ડ}: ફંકશન વ્યાખ્યાની શરૂઆત દર્શાવે છે
\item
  \textbf{ફંકશન નામ}: ફંકશન માટે ઓળખકર્તા
\item
  \textbf{પેરામીટર્સ}: ઇનપુટ વેલ્યુઝ (વૈકલ્પિક)
\item
  \textbf{ડોકસ્ટ્રિંગ}: ફંકશનનું વર્ણન (વૈકલ્પિક)
\item
  \textbf{ફંકશન બોડી}: એક્ઝિક્યુટ થનાર કોડ
\item
  \textbf{રિટર્ન સ્ટેટમેન્ટ}: આઉટપુટ વેલ્યુ (વૈકલ્પિક)
\end{itemize}

\textbf{ઉદાહરણ:}

\begin{verbatim}
\# લંબચોરસનું ક્ષેત્રફળ ગણવા માટેનું યુઝર{-ડિફાઇન ફંકશન}
def calculate\_area(length, width):
    """
    લંબચોરસનું ક્ષેત્રફળ ગણે છે
    આરગ્યુમેન્ટ્સ:
        length: લંબચોરસની લંબાઈ
        width: લંબચોરસની પહોળાઈ
    રિટર્ન:
        લંબચોરસનું ક્ષેત્રફળ
    """
    area = length * width
    return area

\# ફંકશન કૉલ કરો
result = calculate\_area(5, 3)
print(f"લંબચોરસનું ક્ષેત્રફળ: \{result\}")
\end{verbatim}

\end{solutionbox}
\begin{mnemonicbox}
``DRAPE'' (Define function, Receive parameters,
Acquire result, Process data, End with return)

\end{mnemonicbox}
\subsection*{પ્રશ્ન 4(ક) [7
ગુણ]}\label{uxaaauxab0uxab6uxaa8-4uxa95-7-uxa97uxaa3}

\textbf{વિવિધ સ્ટ્રિંગ ઓપરેશનની યાદી બનાવો અને કોઈપણ ત્રણ ઉદાહરણનો ઉપયોગ કરીને
સમજાવો.}

\begin{solutionbox}
પાયથોનમાં સ્ટ્રિંગ ઓપરેશન્સ:

{\def\LTcaptype{none} % do not increment counter
\begin{longtable}[]{@{}ll@{}}
\toprule\noalign{}
ઓપરેશન & વર્ણન \\
\midrule\noalign{}
\endhead
\bottomrule\noalign{}
\endlastfoot
Concatenation & + નો ઉપયોગ કરીને સ્ટ્રિંગ્સ જોડવી \\
Repetition & * નો ઉપયોગ કરીને સ્ટ્રિંગ રિપીટ કરવી \\
Indexing & પોઝિશન દ્વારા કેરેક્ટર એક્સેસ કરવા \\
Slicing & સ્ટ્રિંગનો ભાગ એક્સટ્રેક્ટ કરવો \\
Methods (len, upper, lower, વગેરે) & સ્ટ્રિંગ મેનિપ્યુલેશન માટે બિલ્ટ-ઇન ફંકશન્સ \\
Membership Testing & સ્ટ્રિંગમાં સબસ્ટ્રિંગ છે કે નહીં તે તપાસવું \\
Formatting & ફોર્મેટેડ સ્ટ્રિંગ્સ બનાવવી \\
Escape Sequences & ~થી શરૂ થતા સ્પેશિયલ કેરેક્ટર્સ \\
\end{longtable}
}

\textbf{ત્રણ સ્ટ્રિંગ ઓપરેશન્સ વિથ ઉદાહરણ:}

\begin{enumerate}
\tightlist
\item
  \textbf{સ્ટ્રિંગ Concatenation:}
\end{enumerate}

\begin{verbatim}
first\_name = "John"
last\_name = "Doe"
full\_name = first\_name + " " + last\_name
print(full\_name)  \# આઉટપુટ: John Doe
\end{verbatim}

\begin{enumerate}
\tightlist
\item
  \textbf{સ્ટ્રિંગ Slicing:}
\end{enumerate}

\begin{verbatim}
message = "Python Programming"
print(message[0:6])    \# આઉટપુટ: Python
print(message[7:])     \# આઉટપુટ: Programming
print(message[{-}11:])   \# આઉટપુટ: Programming
\end{verbatim}

\begin{enumerate}
\tightlist
\item
  \textbf{સ્ટ્રિંગ Methods:}
\end{enumerate}

\begin{verbatim}
text = "python programming"
print(text.upper())    \# આઉટપુટ: PYTHON PROGRAMMING
print(text.capitalize())  \# આઉટપુટ: Python programming
print(text.replace("python", "Java"))  \# આઉટપુટ: Java programming
\end{verbatim}

\end{solutionbox}
\begin{mnemonicbox}
``CSM'' (Concatenate strings, Slice portions,
Manipulate with methods)

\end{mnemonicbox}
\subsection*{પ્રશ્ન 4(અ) OR [3
ગુણ]}\label{uxaaauxab0uxab6uxaa8-4uxa85-or-3-uxa97uxaa3}

\textbf{પોઝિટિવ અને નેગેટિવ નંબર તપાસવા પાયથોન કોડ બનાવો.}

\begin{solutionbox}

\begin{verbatim}
\# સંખ્યા પોઝિટિવ છે કે નેગેટિવ તે તપાસવાનો પ્રોગ્રામ
\# વપરાશકર્તા પાસેથી સંખ્યા લો
number = float(input("એક સંખ્યા દાખલ કરો: "))

\# સંખ્યા પોઝિટિવ, નેગેટિવ, કે શૂન્ય છે તે તપાસો
if number {} 0:
    print(f"\{number\} એક પોઝિટિવ સંખ્યા છે")
elif number {} 0:
    print(f"\{number\} એક નેગેટિવ સંખ્યા છે")
else:
    print("સંખ્યા શૂન્ય છે")
\end{verbatim}

\textbf{આકૃતિ:}

\begin{verbatim}
flowchart LR
    A([શરૂઆત]) {-{-} B[/સંખ્યા ઇનપુટ કરો/]}
    B {-{-} C\{number  0?\}}
    C {-{-}|હા| D[/"number પોઝિટિવ છે" પ્રિન્ટ કરો/]}
    C {-{-}|ના| E\{number  0?\}}
    E {-{-}|હા| F[/"number નેગેટિવ છે" પ્રિન્ટ કરો/]}
    E {-{-}|ના| G[/"સંખ્યા શૂન્ય છે" પ્રિન્ટ કરો/]}
    D {-{-} H([અંત])}
    F {-{-} H}
    G {-{-} H}
\end{verbatim}

\end{solutionbox}
\begin{mnemonicbox}
``SIGN'' (See If Greater than 0, Negative otherwise)

\end{mnemonicbox}
\subsection*{પ્રશ્ન 4(બ) OR [4
ગુણ]}\label{uxaaauxab0uxab6uxaa8-4uxaac-or-4-uxa97uxaa3}

\textbf{યોગ્ય ઉદાહરણો સાથે local અને global વેરિએબલ સમજાવો.}

\begin{solutionbox}
પાયથોનમાં વેરિએબલ્સના અલગ-અલગ સ્કોપ્સ હોઈ શકે છે:

{\def\LTcaptype{none} % do not increment counter
\begin{longtable}[]{@{}
  >{\raggedright\arraybackslash}p{(\linewidth - 2\tabcolsep) * \real{0.5357}}
  >{\raggedright\arraybackslash}p{(\linewidth - 2\tabcolsep) * \real{0.4643}}@{}}
\toprule\noalign{}
\begin{minipage}[b]{\linewidth}\raggedright
વેરિએબલ પ્રકાર
\end{minipage} & \begin{minipage}[b]{\linewidth}\raggedright
વર્ણન
\end{minipage} \\
\midrule\noalign{}
\endhead
\bottomrule\noalign{}
\endlastfoot
Local Variable & ફંકશનની અંદર વ્યાખ્યાયિત અને માત્ર તે ફંકશનની અંદર જ એક્સેસિબલ \\
Global Variable & ફંકશનની બહાર વ્યાખ્યાયિત અને પ્રોગ્રામના તમામ ભાગમાં
એક્સેસિબલ \\
\end{longtable}
}

\textbf{ઉદાહરણ:}

\begin{verbatim}
\# Global વેરિએબલ
count = 0  \# આ Global વેરિએબલ છે

def update\_count():
    \# Local વેરિએબલ
    local\_var = 5  \# આ Local વેરિએબલ છે
    
    \# ફંકશનની અંદર Global વેરિએબલ એક્સેસ કરવો
    global count
    count += 1
    
    print(f"Local વેરિએબલ: \{local\_var\}")
    print(f"Global વેરિએબલ (ફંકશનની અંદર): \{count\}")
    
\# ફંકશન કૉલ કરો
update\_count()

\# ફંકશનની બહાર વેરિએબલ એક્સેસ કરવા
print(f"Global વેરિએબલ (ફંકશનની બહાર): \{count\}")

\# આ અનકમેન્ટ કરવાથી એરર આવશે
\# print(local\_var)  \# NameError: name {local\_var is not defined}
\end{verbatim}

\textbf{આઉટપુટ:}

\begin{verbatim}
Local વેરિએબલ: 5
Global વેરિએબલ (ફંકશનની અંદર): 1
Global વેરિએબલ (ફંકશનની બહાર): 1
\end{verbatim}

\end{solutionbox}
\begin{mnemonicbox}
``SCOPE'' (Some variables Confined to function Only,
Program-wide Exposure for others)

\end{mnemonicbox}
\subsection*{પ્રશ્ન 4(ક) OR [7
ગુણ]}\label{uxaaauxab0uxab6uxaa8-4uxa95-or-7-uxa97uxaa3}

\textbf{વિવિધ લિસ્ટ ઓપરેશનની યાદી બનાવો અને કોઈપણ ત્રણ ઉદાહરણનો ઉપયોગ કરીને
સમજાવો.}

\begin{solutionbox}
પાયથોનમાં લિસ્ટ ઓપરેશન્સ:

{\def\LTcaptype{none} % do not increment counter
\begin{longtable}[]{@{}ll@{}}
\toprule\noalign{}
ઓપરેશન & વર્ણન \\
\midrule\noalign{}
\endhead
\bottomrule\noalign{}
\endlastfoot
લિસ્ટ બનાવવી & સ્ક્વેર બ્રેકેટ્સ [] નો ઉપયોગ \\
ઇન્ડેક્સિંગ & પોઝિશન દ્વારા એલિમેન્ટ એક્સેસ કરવા \\
સ્લાઇસિંગ & લિસ્ટના ભાગો એક્સટ્રેક્ટ કરવા \\
એપેન્ડ & છેલ્લે એલિમેન્ટ ઉમેરવા \\
ઇન્સર્ટ & ચોક્કસ પોઝિશન પર એલિમેન્ટ ઉમેરવા \\
રિમૂવ & ચોક્કસ એલિમેન્ટ દૂર કરવા \\
પોપ & એલિમેન્ટ દૂર કરવું અને પાછું મેળવવું \\
સોર્ટ & લિસ્ટ એલિમેન્ટ્સ ઓર્ડર કરવા \\
રિવર્સ & લિસ્ટનો ક્રમ ઊલટાવવો \\
એક્સ્ટેન્ડ & લિસ્ટ્સ જોડવી \\
લિસ્ટ કોમ્પ્રિહેન્શન્સ & એક્સપ્રેશન્સનો ઉપયોગ કરીને લિસ્ટ બનાવવી \\
\end{longtable}
}

\textbf{ત્રણ લિસ્ટ ઓપરેશન્સ વિથ ઉદાહરણ:}

\begin{enumerate}
\tightlist
\item
  \textbf{લિસ્ટ ઇન્ડેક્સિંગ અને સ્લાઇસિંગ:}
\end{enumerate}

\begin{verbatim}
fruits = ["apple", "banana", "cherry", "orange", "kiwi"]
print(fruits[1])        \# આઉટપુટ: banana
print(fruits[{-}1])       \# આઉટપુટ: kiwi
print(fruits[1:4])      \# આઉટપુટ: [{banana, cherry, orange]}
\end{verbatim}

\begin{enumerate}
\tightlist
\item
  \textbf{લિસ્ટ મેથડ્સ (append, insert, remove):}
\end{enumerate}

\begin{verbatim}
numbers = [1, 2, 3]
numbers.append(4)       \# છેલ્લે 4 ઉમેરો
print(numbers)          \# આઉટપુટ: [1, 2, 3, 4]

numbers.insert(0, 0)    \# પોઝિશન 0 પર 0 ઇન્સર્ટ કરો
print(numbers)          \# આઉટપુટ: [0, 1, 2, 3, 4]

numbers.remove(2)       \# 2 વેલ્યુ ધરાવતો એલિમેન્ટ દૂર કરો
print(numbers)          \# આઉટપુટ: [0, 1, 3, 4]
\end{verbatim}

\begin{enumerate}
\tightlist
\item
  \textbf{લિસ્ટ કોમ્પ્રિહેન્શન્સ:}
\end{enumerate}

\begin{verbatim}
\# સ્ક્વેર્સની લિસ્ટ બનાવવી
squares = [x**2 for x in range(1, 6)]
print(squares)  \# આઉટપુટ: [1, 4, 9, 16, 25]

\# સમ સંખ્યાઓ ફિલ્ટર કરવી
numbers = [1, 2, 3, 4, 5, 6, 7, 8, 9, 10]
evens = [x for x in numbers if x \% 2 == 0]
print(evens)    \# આઉટપુટ: [2, 4, 6, 8, 10]
\end{verbatim}

\end{solutionbox}
\begin{mnemonicbox}
``AIM'' (Access with index, Insert/modify elements,
Make using comprehensions)

\end{mnemonicbox}
\subsection*{પ્રશ્ન 5(અ) [3
ગુણ]}\label{uxaaauxab0uxab6uxaa8-5uxa85-3-uxa97uxaa3}

\textbf{લિસ્ટમાં આપેલ બે એલિમેન્ટ્સને સ્વેપ કરવા પાયથોન કોડ લખો.}

\begin{solutionbox}

\begin{verbatim}
\# લિસ્ટમાં બે એલિમેન્ટ્સને સ્વેપ કરવાનો પ્રોગ્રામ
def swap\_elements(my\_list, pos1, pos2):
    """
    લિસ્ટમાં pos1 અને pos2 પોઝિશન પરના એલિમેન્ટ્સને સ્વેપ કરે છે
    """
    \# પોઝિશન્સ માન્ય છે કે નહીં તે તપાસો
    if 0 {=} pos1 {} len(my\_list) and 0 {=} pos2 {} len(my\_list):
        \# એલિમેન્ટ્સ સ્વેપ કરો
        my\_list[pos1], my\_list[pos2] = my\_list[pos2], my\_list[pos1]
        return True
    else:
        return False

\# ઉદાહરણ
numbers = [10, 20, 30, 40, 50]
print("મૂળ લિસ્ટ:", numbers)

\# પોઝિશન 1 અને 3 પરના એલિમેન્ટ્સ સ્વેપ કરો
if swap\_elements(numbers, 1, 3):
    print("સ્વેપ પછી:", numbers)
else:
    print("અમાન્ય પોઝિશન્સ")
\end{verbatim}

\textbf{આઉટપુટ:}

\begin{verbatim}
મૂળ લિસ્ટ: [10, 20, 30, 40, 50]
સ્વેપ પછી: [10, 40, 30, 20, 50]
\end{verbatim}

\end{solutionbox}
\begin{mnemonicbox}
``SWAP'' (Select positions, Watch boundaries, Assign
simultaneously, Print result)

\end{mnemonicbox}
\subsection*{પ્રશ્ન 5(બ) [4
ગુણ]}\label{uxaaauxab0uxab6uxaa8-5uxaac-4-uxa97uxaa3}

\textbf{પાયથોનનાં Math મોડ્યુલ અને random મોડ્યુલ ઉદાહરણનાં ઉપયોગ કરીને સમજાવો.}

\begin{solutionbox}
Math અને random મોડ્યુલ મેથેમેટિકલ ઓપરેશન્સ અને રેન્ડમ નંબર જનરેશન
માટેના ફંકશન્સ પ્રદાન કરે છે.

\textbf{Math મોડ્યુલ:}

\begin{verbatim}
import math

\# કોન્સ્ટન્ટ્સ
print(math.pi)          \# આઉટપુટ: 3.141592653589793
print(math.e)           \# આઉટપુટ: 2.718281828459045

\# મેથેમેટિકલ ફંકશન્સ
print(math.sqrt(16))    \# આઉટપુટ: 4.0
print(math.ceil(4.2))   \# આઉટપુટ: 5
print(math.floor(4.8))  \# આઉટપુટ: 4
print(math.pow(2, 3))   \# આઉટપુટ: 8.0
\end{verbatim}

\textbf{Random મોડ્યુલ:}

\begin{verbatim}
import random

\# 0 અને 1 વચ્ચેની રેન્ડમ ફ્લોટ
print(random.random())       \# આઉટપુટ: 0.123... (રેન્ડમ)

\# રેન્જની અંદર રેન્ડમ ઇન્ટીજર
print(random.randint(1, 10)) \# આઉટપુટ: 7 (1 અને 10 વચ્ચે રેન્ડમ)

\# સીક્વન્સમાંથી રેન્ડમ પસંદગી
colors = ["red", "green", "blue"]
print(random.choice(colors)) \# આઉટપુટ: "green" (રેન્ડમ)

\# લિસ્ટને શફલ કરવી
numbers = [1, 2, 3, 4, 5]
random.shuffle(numbers)
print(numbers)               \# આઉટપુટ: [3, 1, 5, 2, 4] (રેન્ડમ)
\end{verbatim}

\end{solutionbox}
\begin{mnemonicbox}
``MR-CS'' (Math for Calculations, Random for Choice
and Shuffling)

\end{mnemonicbox}
\subsection*{પ્રશ્ન 5(ક) [7
ગુણ]}\label{uxaaauxab0uxab6uxaa8-5uxa95-7-uxa97uxaa3}

\textbf{Tuple ફંકશન અને ઓપરેશન દર્શાવવા પાયથોન કોડ લખો.}

\begin{solutionbox}

\begin{verbatim}
\# Tuple ફંકશન અને ઓપરેશનનું પ્રદર્શન

\# Tuples બનાવવા
empty\_tuple = ()
single\_item\_tuple = (1,)  \# અહીં કોમા અગત્યનો છે
mixed\_tuple = (1, "Hello", 3.14, True)
nested\_tuple = (1, 2, (3, 4))

\# Tuple એલિમેન્ટને એક્સેસ કરવા
print("એલિમેન્ટ એક્સેસ કરવા:")
print(mixed\_tuple[0])      \# આઉટપુટ: 1
print(mixed\_tuple[{-}1])     \# આઉટપુટ: True
print(nested\_tuple[2][0])  \# આઉટપુટ: 3

\# Tuple સ્લાઇસિંગ
print("{n}Tuple સ્લાઇસિંગ:")
print(mixed\_tuple[1:3])    \# આઉટપુટ: ("Hello", 3.14)

\# Tuple જોડવા (concatenation)
tuple1 = (1, 2, 3)
tuple2 = (4, 5, 6)
tuple3 = tuple1 + tuple2
print("{n}જોડાયેલ tuple:", tuple3)  \# આઉટપુટ: (1, 2, 3, 4, 5, 6)

\# Tuple રિપિટિશન
repeated\_tuple = tuple1 * 3
print("{n}રિપીટ થયેલ tuple:", repeated\_tuple)  \# આઉટપુટ: (1, 2, 3, 1, 2, 3, 1, 2, 3)

\# Tuple મેથડ્સ
numbers = (1, 2, 3, 2, 4, 2)
print("{n}2ની સંખ્યા:", numbers.count(2))  \# આઉટપુટ: 3
print("3નો ઇન્ડેક્સ:", numbers.index(3))    \# આઉટપુટ: 2

\# Tuple અનપૅકિંગ
print("{n}Tuple અનપૅકિંગ:")
x, y, z = (10, 20, 30)
print(f"x=\{x\},

y=\{y\},

z=\{z\}")  \# આઉટપુટ:

x=10,

y=20,

z=30


\# Tupleમાં એલિમેન્ટ છે કે નહીં તે તપાસવું
print("{n}મેમ્બરશિપ ટેસ્ટિંગ:")
print(3 in numbers)     \# આઉટપુટ: True
print(5 in numbers)     \# આઉટપુટ: False

\# લિસ્ટને tupleમાં અને tupleને લિસ્ટમાં રૂપાંતરિત કરવું
my\_list = [1, 2, 3]
my\_tuple = tuple(my\_list)
print("{n}લિસ્ટથી tuple:", my\_tuple)

back\_to\_list = list(my\_tuple)
print("tupleથી લિસ્ટ:", back\_to\_list)
\end{verbatim}

\textbf{આકૃતિ:}

\begin{verbatim}
flowchart LR
    A[Tuples બનાવવા] {-{-} B[એલિમેન્ટ એક્સેસ કરવા]}
    B {-{-} C[Tuples સ્લાઇસ કરવા]}
    C {-{-} D[Tuples જોડવા]}
    D {-{-} E[Tuples રિપીટ કરવા]}
    E {-{-} F[Tuple મેથડ્સનો ઉપયોગ કરવો]}
    F {-{-} G[Tuples અનપૅક કરવા]}
    G {-{-} H[મેમ્બરશિપ તપાસવી]}
    H {-{-} I[પ્રકાર રૂપાંતરિત કરવા]}
\end{verbatim}

\end{solutionbox}
\begin{mnemonicbox}
``CASC-RUMTC'' (Create, Access, Slice, Concatenate,
Repeat, Use methods, Membership test, Tuple conversion)

\end{mnemonicbox}
\subsection*{પ્રશ્ન 5(અ) OR [3
ગુણ]}\label{uxaaauxab0uxab6uxaa8-5uxa85-or-3-uxa97uxaa3}

\textbf{લિસ્ટમાં સામેલ એલિમેંટનો સરવાળો કરવા પાયથોન કોડ લખો.}

\begin{solutionbox}

\begin{verbatim}
\# લિસ્ટના એલિમેન્ટ્સનો સરવાળો કરવા માટેનો પ્રોગ્રામ
def sum\_of\_elements(numbers):
    """
    લિસ્ટના બધા એલિમેન્ટ્સનો સરવાળો કરે છે
    """
    total = 0
    for num in numbers:
        total += num
    return total

\# ઉદાહરણ
my\_list = [10, 20, 30, 40, 50]
print("લિસ્ટ:", my\_list)
print("એલિમેન્ટ્સનો સરવાળો:", sum\_of\_elements(my\_list))  \# આઉટપુટ: 150

\# બિલ્ટ{-ઇન sum() ફંકશનનો ઉપયોગ કરીને વૈકલ્પિક રીત}
print("બિલ્ટ{-ઇન ફંકશનનો ઉપયોગ કરીને સરવાળો:"}, sum(my\_list))  \# આઉટપુટ: 150
\end{verbatim}

\textbf{આકૃતિ:}

\begin{verbatim}
flowchart LR
        direction LR
    A([શરૂઆત]) {-{-} B[total = 0 થી શરૂ કરો]}
    B {-{-} C[લિસ્ટમાંના દરેક num માટે]}
    C {-{-} D[total += num]}
    D {-{-} E\{વધુ એલિમેન્ટ્સ?\}}
    E {-{-}|હા| C}
    E {-{-}|ના| F[total પરત કરો]}
    F {-{-} G([અંત])}
\end{verbatim}

\end{solutionbox}
\begin{mnemonicbox}
``SITE'' (Sum Initialized To zero, Elements added
one by one)

\end{mnemonicbox}
\subsection*{પ્રશ્ન 5(બ) OR [4
ગુણ]}\label{uxaaauxab0uxab6uxaa8-5uxaac-or-4-uxa97uxaa3}

\textbf{નીચે આપેલ built in functionsનો ઉપયોગ સમજાવો:} \textbf{૧) Print()
૨) Min() ૩) Sum() ૪) Input()}

\begin{solutionbox}

{\def\LTcaptype{none} % do not increment counter
\begin{longtable}[]{@{}
  >{\raggedright\arraybackslash}p{(\linewidth - 6\tabcolsep) * \real{0.2778}}
  >{\raggedright\arraybackslash}p{(\linewidth - 6\tabcolsep) * \real{0.2500}}
  >{\raggedright\arraybackslash}p{(\linewidth - 6\tabcolsep) * \real{0.2500}}
  >{\raggedright\arraybackslash}p{(\linewidth - 6\tabcolsep) * \real{0.2222}}@{}}
\toprule\noalign{}
\begin{minipage}[b]{\linewidth}\raggedright
ફંકશન
\end{minipage} & \begin{minipage}[b]{\linewidth}\raggedright
હેતુ
\end{minipage} & \begin{minipage}[b]{\linewidth}\raggedright
ઉદાહરણ
\end{minipage} & \begin{minipage}[b]{\linewidth}\raggedright
આઉટપુટ
\end{minipage} \\
\midrule\noalign{}
\endhead
\bottomrule\noalign{}
\endlastfoot
\textbf{print()} & કન્સોલ પર આઉટપુટ દર્શાવે છે &
\texttt{print("Hello\ World")} & Hello World \\
\textbf{min()} & iterableમાંથી સૌથી નાના આઇટમને પરત કરે છે &
\texttt{min([5,\ 3,\ 8,\ 1])} & 1 \\
\textbf{sum()} & iterableમાંના તમામ આઇટમ્સનો સરવાળો આપે છે &
\texttt{sum([1,\ 2,\ 3,\ 4])} & 10 \\
\textbf{input()} & વપરાશકર્તા પાસેથી ઇનપુટ વાંચે છે &
\texttt{name\ =\ input("નામ\ દાખલ\ કરો:\ ")} & (વપરાશકર્તાની ઇનપુટની રાહ
જુએ છે) \\
\end{longtable}
}

\textbf{ઉદાહરણ કોડ:}

\begin{verbatim}
\# print() ફંકશન
print("હેલો, પાયથોન!")  \# મૂળભૂત આઉટપુટ
print("a", "b", "c", sep="{-"})  \# સેપરેટર સાથે આઉટપુટ: a{-b{-}c}
print("નવી લાઇન નહીં", end=" ")  \# કસ્ટમ end કેરેક્ટર
print("એક જ લાઇનમાં")  \# આઉટપુટ: નવી લાઇન નહીં એક જ લાઇનમાં

\# min() ફંકશન
numbers = [15, 8, 23, 4, 42]
print("ન્યૂનતમ મૂલ્ય:", min(numbers))  \# આઉટપુટ: 4
print("5, 2, 9 માંથી ન્યૂનતમ:", min(5, 2, 9))  \# આઉટપુટ: 2
chars = "wxyz"
print("ન્યૂનતમ અક્ષર:", min(chars))  \# આઉટપુટ: w

\# sum() ફંકશન
print("સંખ્યાઓનો સરવાળો:", sum(numbers))  \# આઉટપુટ: 92
print("પ્રારંભિક મૂલ્ય સાથે સરવાળો:", sum(numbers, 10))  \# આઉટપુટ: 102

\# input() ફંકશન
user\_input = input("કંઈક દાખલ કરો: ")  \# વપરાશકર્તાને ઇનપુટ માટે પ્રોમ્પ્ટ કરે છે
print("તમે દાખલ કર્યું:", user\_input)  \# વપરાશકર્તાનો ઇનપુટ દર્શાવે છે
\end{verbatim}

\end{solutionbox}
\begin{mnemonicbox}
``PMSI'' (Print to display, Min for smallest, Sum
for total, Input for reading)

\end{mnemonicbox}
\subsection*{પ્રશ્ન 5(ક) OR [7
ગુણ]}\label{uxaaauxab0uxab6uxaa8-5uxa95-or-7-uxa97uxaa3}

\textbf{સેટ ફંકશન અને ઓપરેશન દર્શાવવા પાયથોન કોડ લખો.}

\begin{solutionbox}

\begin{verbatim}
\# સેટ ફંકશન અને ઓપરેશનનું પ્રદર્શન

\# સેટ બનાવવા
empty\_set = set()  \# ખાલી સેટ
numbers = \{1, 2, 3, 4, 5\}
duplicates = \{1, 2, 2, 3, 4, 4, 5\  }\# ડુપ્લિકેટ આપોઆપ દૂર થાય છે
print("મૂળ સેટ:", numbers)
print("ડુપ્લિકેટ સાથેનો સેટ:", duplicates)  \# આઉટપુટ: \{1, 2, 3, 4, 5\}

\# એલિમેન્ટ ઉમેરવા
numbers.add(6)
print("{n}6 ઉમેર્યા પછી:", numbers)  \# આઉટપુટ: \{1, 2, 3, 4, 5, 6\}

\# અનેક એલિમેન્ટ્સ સાથે અપડેટ કરવું
numbers.update([7, 8, 9])
print("અપડેટ કર્યા પછી:", numbers)  \# આઉટપુટ: \{1, 2, 3, 4, 5, 6, 7, 8, 9\}

\# એલિમેન્ટ્સ દૂર કરવા
numbers.remove(5)  \# એલિમેન્ટ ન મળે તો એરર આપે
print("{n}5 દૂર કર્યા પછી:", numbers)

numbers.discard(10)  \# એલિમેન્ટ ન મળે તો કોઈ એરર નહીં
print("10 discard કર્યા પછી:", numbers)  \# કોઈ ફેરફાર નહીં

popped = numbers.pop()  \# આપમેળે એક એલિમેન્ટ દૂર કરે અને પરત કરે
print("pop કરેલ એલિમેન્ટ:", popped)
print("pop પછી:", numbers)

\# સેટ ઓપરેશન્સ
set1 = \{1, 2, 3, 4, 5\}
set2 = \{4, 5, 6, 7, 8\}

\# યુનિયન
union\_set = set1 | set2  \# અથવા set1.union(set2)
print("{n}યુનિયન:", union\_set)  \# આઉટપુટ: \{1, 2, 3, 4, 5, 6, 7, 8\}

\# ઇન્ટરસેક્શન
intersection\_set = set1 \& set2  \# અથવા set1.intersection(set2)
print("ઇન્ટરસેક્શન:", intersection\_set)  \# આઉટપુટ: \{4, 5\}

\# ડિફરન્સ
difference\_set = set1 {-} set2  \# અથવા set1.difference(set2)
print("ડિફરન્સ (set1 {- set2):"}, difference\_set)  \# આઉટપુટ: \{1, 2, 3\}

\# સિમેટ્રિક ડિફરન્સ
symmetric\_diff = set1 \^{} set2  \# અથવા set1.symmetric\_difference(set2)
print("સિમેટ્રિક ડિફરન્સ:", symmetric\_diff)  \# આઉટપુટ: \{1, 2, 3, 6, 7, 8\}

\# સબસેટ અને સુપરસેટ
subset = \{1, 2\}
print("{n}શું \{1, 2\ એ set1નો સબસેટ છે?"}, subset.issubset(set1))  \# આઉટપુટ: True
print("શું set1 એ \{1, 2\નો સુપરસેટ છે?"}, set1.issuperset(subset))  \# આઉટપુટ: True
\end{verbatim}

\textbf{આકૃતિ:}

\begin{verbatim}
flowchart TD
    A[સેટ બનાવવા] {-{-} B[સેટ મોડિફાય કરવા]}
    B {-{-} C[એલિમેન્ટ ઉમેરવા]}
    B {-{-} D[એલિમેન્ટ દૂર કરવા]}
    A {-{-} E[સેટ ઓપરેશન્સ]}
    E {-{-} F[યુનિયન]}
    E {-{-} G[ઇન્ટરસેક્શન]}
    E {-{-} H[ડિફરન્સ]}
    E {-{-} I[સિમેટ્રિક ડિફરન્સ]}
    E {-{-} J[સબસેટ/સુપરસેટ]}
\end{verbatim}

\end{solutionbox}
\begin{mnemonicbox}
``CARDS-UI'' (Create, Add, Remove, Discard elements,
Set operations - Union, Intersection)

\end{mnemonicbox}

\end{document}
