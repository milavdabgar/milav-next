\documentclass[10pt,a4paper]{article}

% content/resources/templates/preamble.tex
\usepackage[margin=0.6in]{geometry}
\author{Milav Dabgar}
\usepackage{amsmath,amssymb,amsthm}
\usepackage{booktabs}
\usepackage{multirow}
\usepackage{xcolor}
\usepackage{tcolorbox}
\tcbuselibrary{breakable,skins}
\usepackage[colorlinks=true,linkcolor=blue]{hyperref}
\usepackage{titlesec}
\usepackage{enumitem}
\usepackage{tikz}
\usepackage{pgfplots}
\usepackage{circuitikz}
\usepackage[version=4]{mhchem}
\usepackage{longtable}
\usepackage{array}
\usepackage{float}
\usepackage{caption}
\usepackage{listings}

\lstset{
  basicstyle=\small\ttfamily,
  breaklines=true,
  breakatwhitespace=false,
  postbreak=\mbox{\textcolor{red}{$\hookrightarrow$}\space},
  float=false,
  numbers=left,
  numberstyle=\tiny\color{gray},
  numbersep=10pt,
  xleftmargin=2em,
  keywordstyle=\color{blue},
  commentstyle=\color{green!60!black},
  stringstyle=\color{purple},
  backgroundcolor=\color{gray!5},
  showstringspaces=false,
  tabsize=2,
  captionpos=b,
  keepspaces=true,
  columns=flexible
}

\pgfplotsset{compat=1.18}
\usetikzlibrary{shapes,arrows,positioning,calc,patterns,decorations.pathmorphing,decorations.markings,arrows.meta}

% Color scheme
\definecolor{headcolor}{RGB}{0,102,204}
\definecolor{keycolor}{RGB}{220,20,60}
\definecolor{solutioncolor}{RGB}{34,139,34}
\definecolor{mnemoniccolor}{RGB}{148,0,211}
\definecolor{codecolor}{RGB}{0,0,100}

% Spacing
\setlength{\parskip}{3pt}
\setlist[itemize]{nosep}
\setlist[enumerate]{nosep}

% Title formatting
\titleformat{\section}{\Large\bfseries\color{headcolor}}{\thesection}{1em}{}
\titleformat{\subsection}{\large\bfseries\color{headcolor}}{\thesubsection}{1em}{}

% Pandoc tightlist compatibility
\providecommand{\tightlist}{%
  \setlength{\itemsep}{0pt}\setlength{\parskip}{0pt}}

% Pandoc longtable compatibility
\newcounter{none}
\def\thenone{}


% content/resources/templates/english-boxes.tex
% This file is currently empty - it exists to maintain consistency with the import structure.
% Add custom environments here if needed in the future.


\begin{document}

\begin{center}
{\Huge\bfseries\color{headcolor} Subject Name Solutions}\\[5pt]
{\LARGE 1323203 -- Summer 2023}\\[3pt]
{\large Semester 1 Study Material}\\[3pt]
{\normalsize\textit{Detailed Solutions and Explanations}}
\end{center}

\vspace{10pt}

\subsection*{Question 1(a) [3 marks]}\label{q1a}

\textbf{Define algorithm. What are the advantages of Algorithm?}

\begin{solutionbox}
An algorithm is a step-by-step procedure or a set of
rules to solve a specific problem in a finite sequence of steps.

\textbf{Advantages of Algorithm:}

\begin{itemize}
\tightlist
\item
  \textbf{Clarity}: Provides clear, unambiguous instructions
\item
  \textbf{Efficiency}: Helps in optimizing time and resources
\item
  \textbf{Reusability}: Can be used repeatedly for similar problems
\item
  \textbf{Verification}: Easy to test and debug before implementation
\item
  \textbf{Communication}: Acts as a blueprint to communicate the
  solution
\end{itemize}

\end{solutionbox}
\begin{mnemonicbox}
``CERVC'' (Clarity, Efficiency, Reusability,
Verification, Communication)

\end{mnemonicbox}
\subsection*{Question 1(b) [4 marks]}\label{q1b}

\textbf{What are the rules for problem solving using flowchart? Design a
flowchart to find simple interest.}

\begin{solutionbox}
Rules for problem solving using flowchart:

\begin{itemize}
\tightlist
\item
  \textbf{Proper symbols}: Use standard symbols for different operations
\item
  \textbf{Direction flow}: Always maintain clear top-to-bottom,
  left-to-right flow
\item
  \textbf{Single entry/exit}: Have a clear start and end point
\item
  \textbf{Clarity}: Keep steps clear and concise
\item
  \textbf{Consistency}: Maintain consistent level of detail
\end{itemize}

\textbf{Flowchart for Simple Interest Calculation:}

\includegraphics[width=1\linewidth,height=\textheight,keepaspectratio]{mermaid-5d3c45ac.pdf}

\end{solutionbox}
\begin{mnemonicbox}
``PDRSC'' (Proper symbols, Direction flow, Required
entry/exit, Simplicity, Consistency)

\end{mnemonicbox}
\subsection*{Question 1(c) [7 marks]}\label{q1c}

\textbf{List out assignment operator in python and build a python code
to demonstrate an operation of any three assignment operators.}

\begin{solutionbox}
Python assignment operators:

{\def\LTcaptype{none} % do not increment counter
\begin{longtable}[]{@{}lll@{}}
\toprule\noalign{}
Operator & Example & Equivalent To \\
\midrule\noalign{}
\endhead
\bottomrule\noalign{}
\endlastfoot
= &

x = 5 &

x = 5 \\

+= & x += 5 &

x = x + 5 \\

-= & x -= 5 &

x = x - 5 \\

*= & x *= 5 &

x = x * 5 \\

/= & x /= 5 &

x = x / 5 \\

\%= & x \%= 5 & x = x \% 5 \\
//= & x //= 5 &

x = x // 5 \\

**= & x **= 5 &

x = x ** 5 \\

\&= & x \&= 5 & x = x \& 5 \\
\textbar= & x \textbar= 5 & x = x \textbar{} 5 \\
\^{}= & x \^{}= 5 & x = x \^{} 5 \\
\textgreater\textgreater= & x \textgreater\textgreater= 5 & x = x
\textgreater\textgreater{} 5 \\
\textless\textless= & x \textless\textless= 5 & x = x
\textless\textless{} 5 \\
\end{longtable}
}

\textbf{Code demonstrating assignment operators:}

\begin{lstlisting}[language=Python]
# Demonstrating Assignment Operators
num = 10
print("Initial value:", num)

# Using += operator
num += 5
print("After += 5:", num)  # Output: 15

# Using -= operator
num -= 3
print("After -= 3:", num)  # Output: 12

# Using *= operator
num *= 2
print("After *= 2:", num)  # Output: 24
\end{lstlisting}

\end{solutionbox}
\begin{mnemonicbox}
``VALUE'' (Variable Assignment is Like Updating
Existing values)

\end{mnemonicbox}
\subsection*{Question 1(c) OR [7
marks]}\label{q1c}

\textbf{List out data types in python and Develop a Program to identify
any three data types in python.}

\begin{solutionbox}
Python data types:

{\def\LTcaptype{none} % do not increment counter
\begin{longtable}[]{@{}lll@{}}
\toprule\noalign{}
Data Type & Description & Example \\
\midrule\noalign{}
\endhead
\bottomrule\noalign{}
\endlastfoot
int & Integer (whole numbers) & 42 \\
float & Floating point (decimal) & 3.14 \\
str & String (text) & ``Hello'' \\
bool & Boolean (True/False) & True \\
list & Ordered, mutable collection & [1, 2, 3] \\
tuple & Ordered, immutable collection & (1, 2, 3) \\
set & Unordered collection of unique items & \{1, 2, 3\} \\
dict & Key-value pairs & \{``name'': ``John''\} \\
complex & Complex numbers & 2+3j \\
NoneType & Represents None & None \\
\end{longtable}
}

\textbf{Code to identify three data types:}

\begin{lstlisting}[language=Python]
# Program to identify data types
def identify_data_type(value):
    data_type = type(value).__name__
    print(f"Value: {value}")
    print(f"Data Type: {data_type}")
    print("-" * 20)

# Testing with 3 different data types
identify_data_type(42)            # Integer
identify_data_type(3.14)          # Float
identify_data_type("Hello World") # String

# Output:
# Value: 42
# Data Type: int
# --------------------
# Value: 3.14
# Data Type: float
# --------------------
# Value: Hello World
# Data Type: str
# --------------------
\end{lstlisting}

\end{solutionbox}
\begin{mnemonicbox}
``TYPE-ID'' (Tell Your Python Elements - Identify
Data)

\end{mnemonicbox}
\subsection*{Question 2(a) [3 marks]}\label{q2a}

\textbf{Define pseudocode. Write pseudocode to find smallest of two
number.}

\begin{solutionbox}
Pseudocode is a high-level description of an algorithm
that uses structural conventions of a programming language but is
designed for human reading rather than machine reading.

\textbf{Pseudocode to find smallest of two numbers:}

\begin{lstlisting}
BEGIN
    INPUT first_number, second_number
    IF first_number < second_number THEN
        smallest = first_number
    ELSE
        smallest = second_number
    END IF
    OUTPUT smallest
END
\end{lstlisting}

\end{solutionbox}
\begin{mnemonicbox}
``RISE'' (Read Input, Select smallest, Echo result)

\end{mnemonicbox}
\subsection*{Question 2(b) [4 marks]}\label{q2b}

\textbf{Develop a python code to read three numbers from the user and
find the average of the numbers.}

\begin{solutionbox}

\begin{lstlisting}[language=Python]
# Program to calculate average of three numbers
# Input three numbers from user
num1 = float(input("Enter first number: "))
num2 = float(input("Enter second number: "))
num3 = float(input("Enter third number: "))

# Calculate the average
average = (num1 + num2 + num3) / 3

# Display the result
print(f"The average of {num1}, {num2}, and {num3} is: {average}")
\end{lstlisting}

\textbf{Diagram:}

\includegraphics[width=1\linewidth,height=\textheight,keepaspectratio]{mermaid-dfd12368.pdf}

\end{solutionbox}
\begin{mnemonicbox}
``I-ADD-D'' (Input three, ADD them up, Divide by 3)

\end{mnemonicbox}
\subsection*{Question 2(c) [7 marks]}\label{q2c}

\textbf{Write a python code to show whether the entered number is prime
or not.}

\begin{solutionbox}

\begin{lstlisting}[language=Python]
# Program to check if a number is prime
# Input number from user
num = int(input("Enter a number: "))

# Check if number is less than 2
if num < 2:
    print(f"{num} is not a prime number")
else:
    # Initialize is_prime as True
    is_prime = True
    
    # Check from 2 to sqrt(num)
    for i in range(2, int(num**0.5) + 1):
if num %

i == 0:

            is_prime = False
            break
    
    # Display result
    if is_prime:
        print(f"{num} is a prime number")
    else:
        print(f"{num} is not a prime number")
\end{lstlisting}

\textbf{Diagram:}

\includegraphics[width=1\linewidth,height=\textheight,keepaspectratio]{mermaid-50951e79.pdf}

\end{solutionbox}
\begin{mnemonicbox}
``PRIME'' (Positive number, Range check from 2 to \sqrtn,
If divisible it's Multiple, Else it's prime)

\end{mnemonicbox}
\subsection*{Question 2(a) OR [3
marks]}\label{q2a}

\textbf{Write down a difference between Flow chart and Algorithm.}

\begin{solutionbox}

{\def\LTcaptype{none} % do not increment counter
\begin{longtable}[]{@{}
  >{\raggedright\arraybackslash}p{(\linewidth - 2\tabcolsep) * \real{0.5217}}
  >{\raggedright\arraybackslash}p{(\linewidth - 2\tabcolsep) * \real{0.4783}}@{}}
\toprule\noalign{}
\begin{minipage}[b]{\linewidth}\raggedright
Flow Chart
\end{minipage} & \begin{minipage}[b]{\linewidth}\raggedright
Algorithm
\end{minipage} \\
\midrule\noalign{}
\endhead
\bottomrule\noalign{}
\endlastfoot
\textbf{Visual representation} using standard symbols and shapes &
\textbf{Textual description} using structured language \\
\textbf{Easier to understand} due to graphical nature & Requires
knowledge of syntax and terminology \\
Shows \textbf{logical flow} and relationships clearly & Provides
\textbf{detailed steps} in sequential order \\
\textbf{Time-consuming to create} but easier to follow & \textbf{Quicker
to draft} but may be harder to interpret \\
More difficult to modify or update & Easier to modify or update \\
\end{longtable}
}

\end{solutionbox}
\begin{mnemonicbox}
``VITAL'' (Visual vs Textual, Interpretation ease,
Time to create, Alteration flexibility, Logical representation)

\end{mnemonicbox}
\subsection*{Question 2(b) OR [4
marks]}\label{q2b}

\textbf{What is the output of the following code:}

\begin{lstlisting}[language=Python]
x=10
y=2
print (x*y)
print (x ** y)
print (x//y)
print (x % y)
\end{lstlisting}

\begin{solutionbox}

{\def\LTcaptype{none} % do not increment counter
\begin{longtable}[]{@{}lll@{}}
\toprule\noalign{}
Operation & Explanation & Output \\
\midrule\noalign{}
\endhead
\bottomrule\noalign{}
\endlastfoot
x*y & Multiplication: 10 \times 2 & 20 \\
x**y & Exponentiation: 10^{2} & 100 \\
x//y & Integer division: 10 \div 2 & 5 \\
x\%y & Modulus (remainder): 10 \div 2 & 0 \\
\end{longtable}
}

\end{solutionbox}
\begin{mnemonicbox}
``MEMO'' (Multiply, Exponent, Modulo, Operations)

\end{mnemonicbox}
\subsection*{Question 2(c) OR [7
marks]}\label{q2c}

\textbf{Write a python code to display the following patterns:}

\begin{lstlisting}
A)                    B)
1                    * * * *
1 2                  * * *
1 2 3                * *
1 2 3 4              *
\end{lstlisting}

\begin{solutionbox}

\begin{lstlisting}[language=Python]
# Pattern A: Number pattern
print("Pattern A:")
for i in range(1, 5):
    for j in range(1, i + 1):
        print(j, end=" ")
    print()

# Pattern B: Star pattern
print("\nPattern B:")
for i in range(4, 0, -1):
    for j in range(i):
        print("*", end=" ")
    print()
\end{lstlisting}

\textbf{Diagram:}

\includegraphics[width=1\linewidth,height=\textheight,keepaspectratio]{mermaid-5de9ab79.pdf}

\end{solutionbox}
\begin{mnemonicbox}
``LOOP-NED'' (Loop Outer, Order Pattern, Nested
loops, End with newline, Display)

\end{mnemonicbox}
\subsection*{Question 3(a) [3 marks]}\label{q3a}

\textbf{With the necessary examples describe the use of break
statement.}

\begin{solutionbox}
Break statement is used to exit or terminate a loop
prematurely when a specific condition is met.

\textbf{Example:}

\begin{lstlisting}[language=Python]
# Finding the first odd number in a list
numbers = [2, 4, 6, 7, 8, 10]
for num in numbers:
    if num % 2 != 0:
        print(f"Found odd number: {num}")
        break
    print(f"Checking {num}")
\end{lstlisting}

\textbf{Output:}

\begin{lstlisting}
Checking 2
Checking 4
Checking 6
Found odd number: 7
\end{lstlisting}

\end{solutionbox}
\begin{mnemonicbox}
``EXIT'' (EXecute until condition, Immediately
Terminate)

\end{mnemonicbox}
\subsection*{Question 3(b) [4 marks]}\label{q3b}

\textbf{Explain if\ldots else statement with suitable example.}

\begin{solutionbox}
The if\ldots else statement is a conditional statement
that executes different blocks of code based on whether a specified
condition evaluates to True or False.

\textbf{Syntax:}

\begin{lstlisting}[language=Python]
if condition:
    # Code to be executed if condition is True
else:
    # Code to be executed if condition is False
\end{lstlisting}

\textbf{Example:}

\begin{lstlisting}[language=Python]
# Check if a number is even or odd
number = int(input("Enter a number: "))

if number % 2 == 0:
    print(f"{number} is an even number")
else:
    print(f"{number} is an odd number")
\end{lstlisting}

\textbf{Diagram:}

\includegraphics[width=1\linewidth,height=\textheight,keepaspectratio]{mermaid-c37e6859.pdf}

\end{solutionbox}
\begin{mnemonicbox}
``CITE'' (Check condition, If True Execute this, Else
execute that)

\end{mnemonicbox}
\subsection*{Question 3(c) [7 marks]}\label{q3c}

\textbf{Create a User-defined function to print the Fibonacci series of
0 to N numbers where N is an integer number and passed as an argument.}

\begin{solutionbox}

\begin{lstlisting}[language=Python]
# Function to print Fibonacci series
def print_fibonacci(n):
    """
    Print Fibonacci series from 0 to n
    Args:
        n: Upper limit (inclusive)
    """
    # Initialize first two terms
    a, b = 0, 1
    
    # Check if n is valid
    if n < 0:
        print("Please enter a positive number")
        return
    
    # Print Fibonacci series
    print("Fibonacci series up to", n, ":")
    
    if n >= 0:
        print(a, end=" ")  # Print first term
    
    if n >= 1:
        print(b, end=" ")  # Print second term
    
    # Generate and print the rest of the series
    while a + b <= n:
        next_term = a + b
        print(next_term, end=" ")
        a, b = b, next_term

# Test the function
print_fibonacci(55)
\end{lstlisting}

\textbf{Diagram:}

\includegraphics[width=1\linewidth,height=\textheight,keepaspectratio]{mermaid-6537019e.pdf}

\end{solutionbox}
\begin{mnemonicbox}
``FIBER'' (First terms set, Initialize variables,
Build next term, Echo results, Repeat until limit)

\end{mnemonicbox}
\subsection*{Question 3(a) OR [3
marks]}\label{q3a}

\textbf{With the necessary examples describe the use of continue
statement.}

\begin{solutionbox}
Continue statement is used to skip the current
iteration of a loop and continue with the next iteration.

\textbf{Example:}

\begin{lstlisting}[language=Python]
# Print only odd numbers from 1 to 10
for i in range(1, 11):
    if i % 2 == 0:
        continue  # Skip even numbers
    print(i)
\end{lstlisting}

\textbf{Output:}

\begin{lstlisting}
1
3
5
7
9
\end{lstlisting}

\end{solutionbox}
\begin{mnemonicbox}
``SKIP'' (Skip current iteration, Keep looping,
Ignore remaining statements, Proceed to next iteration)

\end{mnemonicbox}
\subsection*{Question 3(b) OR [4
marks]}\label{q3b}

\textbf{Explain For loop statement with example.}

\begin{solutionbox}
For loop is used to iterate over a sequence (like list,
tuple, string) or other iterable objects and execute a block of code for
each item in the sequence.

\textbf{Syntax:}

\begin{lstlisting}[language=Python]
for variable in sequence:
    # Code to be executed for each item
\end{lstlisting}

\textbf{Example:}

\begin{lstlisting}[language=Python]
# Print squares of numbers from 1 to 5
for num in range(1, 6):
    square = num ** 2
    print(f"The square of {num} is {square}")
\end{lstlisting}

\textbf{Output:}

\begin{lstlisting}
The square of 1 is 1
The square of 2 is 4
The square of 3 is 9
The square of 4 is 16
The square of 5 is 25
\end{lstlisting}

\textbf{Diagram:}

\includegraphics[width=1\linewidth,height=\textheight,keepaspectratio]{mermaid-10af04d9.pdf}

\end{solutionbox}
\begin{mnemonicbox}
``FIRE'' (For each Item, Run commands, Execute until
end)

\end{mnemonicbox}
\subsection*{Question 3(c) OR [7
marks]}\label{q3c}

\textbf{Write a python code that determines whether a given number is an
`Armstrong number' or a palindrome using a user-defined function.}

\begin{solutionbox}

\begin{lstlisting}[language=Python]
# Function to check if a number is Armstrong number
def is_armstrong(num):
    # Convert to string to count digits
    num_str = str(num)
    n = len(num_str)
    
    # Calculate sum of each digit raised to power of total digits
    sum_of_powers = sum(int(digit) ** n for digit in num_str)
    
    # Check if sum equals the original number
    return sum_of_powers == num

# Function to check if a number is a palindrome
def is_palindrome(num):
    # Convert to string
    num_str = str(num)
    
    # Check if string equals its reverse
    return num_str == num_str[::-1]

# Main function to check both conditions
def check_number(num):
    if is_armstrong(num):
        print(f"{num} is an Armstrong number")
    else:
        print(f"{num} is not an Armstrong number")
    
    if is_palindrome(num):
        print(f"{num} is a palindrome")
    else:
        print(f"{num} is not a palindrome")

# Test the function
number = int(input("Enter a number: "))
check_number(number)
\end{lstlisting}

\textbf{Diagram:}

\includegraphics[width=1\linewidth,height=\textheight,keepaspectratio]{mermaid-5159b240.pdf}

\end{solutionbox}
\begin{mnemonicbox}
``APC'' (Armstrong check: Power sum of digits,
Palindrome check: Compare with reverse)

\end{mnemonicbox}
\subsection*{Question 4(a) [3 marks]}\label{q4a}

\textbf{Develop a python code to identify whether the scanned number is
even or odd and print an appropriate message.}

\begin{solutionbox}

\begin{lstlisting}[language=Python]
# Program to check if a number is even or odd
# Input number from user
number = int(input("Enter a number: "))

# Check if number is even or odd
if number % 2 == 0:
    print(f"{number} is an even number")
else:
    print(f"{number} is an odd number")
\end{lstlisting}

\textbf{Diagram:}

\includegraphics[width=1\linewidth,height=\textheight,keepaspectratio]{mermaid-c37e6859.pdf}

\end{solutionbox}
\begin{mnemonicbox}
``MODE'' (Modulo Operation Determines Even-odd)

\end{mnemonicbox}
\subsection*{Question 4(b) [4 marks]}\label{q4b}

\textbf{Define function. Explain user define function using suitable
example.}

\begin{solutionbox}
A function is a block of organized, reusable code that
performs a specific task. User-defined functions are functions created
by the programmer to perform custom operations.

\textbf{Components of a User-defined Function:}

\begin{itemize}
\tightlist
\item
  \textbf{def keyword}: Marks the start of function definition
\item
  \textbf{Function name}: Identifier for the function
\item
  \textbf{Parameters}: Input values (optional)
\item
  \textbf{Docstring}: Description of the function (optional)
\item
  \textbf{Function body}: Code to be executed
\item
  \textbf{Return statement}: Output value (optional)
\end{itemize}

\textbf{Example:}

\begin{lstlisting}[language=Python]
# User-defined function to calculate area of rectangle
def calculate_area(length, width):
    """
    Calculate area of rectangle
    Args:
        length: Length of rectangle
        width: Width of rectangle
    Returns:
        Area of rectangle
    """
    area = length * width
    return area

# Call the function
result = calculate_area(5, 3)
print(f"Area of rectangle: {result}")
\end{lstlisting}

\end{solutionbox}
\begin{mnemonicbox}
``DRAPE'' (Define function, Receive parameters,
Acquire result, Process data, End with return)

\end{mnemonicbox}
\subsection*{Question 4(c) [7 marks]}\label{q4c}

\textbf{List out various String operations and explain any three using
example.}

\begin{solutionbox}
String operations in Python:

{\def\LTcaptype{none} % do not increment counter
\begin{longtable}[]{@{}
  >{\raggedright\arraybackslash}p{(\linewidth - 2\tabcolsep) * \real{0.4583}}
  >{\raggedright\arraybackslash}p{(\linewidth - 2\tabcolsep) * \real{0.5417}}@{}}
\toprule\noalign{}
\begin{minipage}[b]{\linewidth}\raggedright
Operation
\end{minipage} & \begin{minipage}[b]{\linewidth}\raggedright
Description
\end{minipage} \\
\midrule\noalign{}
\endhead
\bottomrule\noalign{}
\endlastfoot
Concatenation & Joining strings together using + \\
Repetition & Repeating a string using * \\
Indexing & Accessing characters by position \\
Slicing & Extracting a portion of a string \\
Methods (len, upper, lower, etc.) & Built-in functions for string
manipulation \\
Membership Testing & Check if a substring exists in a string \\
Formatting & Create formatted strings \\
Escape Sequences & Special characters preceded by \textbackslash{} \\
\end{longtable}
}

\textbf{Three String Operations with Examples:}

\begin{enumerate}
\tightlist
\item
  \textbf{String Concatenation:}
\end{enumerate}

\begin{lstlisting}[language=Python]
first_name = "John"
last_name = "Doe"
full_name = first_name + " " + last_name
print(full_name)  # Output: John Doe
\end{lstlisting}

\begin{enumerate}
\tightlist
\item
  \textbf{String Slicing:}
\end{enumerate}

\begin{lstlisting}[language=Python]
message = "Python Programming"
print(message[0:6])    # Output: Python
print(message[7:])     # Output: Programming
print(message[-11:])   # Output: Programming
\end{lstlisting}

\begin{enumerate}
\tightlist
\item
  \textbf{String Methods:}
\end{enumerate}

\begin{lstlisting}[language=Python]
text = "python programming"
print(text.upper())    # Output: PYTHON PROGRAMMING
print(text.capitalize())  # Output: Python programming
print(text.replace("python", "Java"))  # Output: Java programming
\end{lstlisting}

\end{solutionbox}
\begin{mnemonicbox}
``CSM'' (Concatenate strings, Slice portions,
Manipulate with methods)

\end{mnemonicbox}
\subsection*{Question 4(a) OR [3
marks]}\label{q4a}

\textbf{Create a python code to check positive or negative number.}

\begin{solutionbox}

\begin{lstlisting}[language=Python]
# Program to check if a number is positive or negative
# Input number from user
number = float(input("Enter a number: "))

# Check if number is positive, negative, or zero
if number > 0:
    print(f"{number} is a positive number")
elif number < 0:
    print(f"{number} is a negative number")
else:
    print("The number is zero")
\end{lstlisting}

\textbf{Diagram:}

\includegraphics[width=1\linewidth,height=\textheight,keepaspectratio]{mermaid-a0a6f796.pdf}

\end{solutionbox}
\begin{mnemonicbox}
``SIGN'' (See If Greater than 0, Negative otherwise)

\end{mnemonicbox}
\subsection*{Question 4(b) OR [4
marks]}\label{q4b}

\textbf{Explain local and global variables using suitable examples.}

\begin{solutionbox}
In Python, variables can have different scopes:

{\def\LTcaptype{none} % do not increment counter
\begin{longtable}[]{@{}
  >{\raggedright\arraybackslash}p{(\linewidth - 2\tabcolsep) * \real{0.5357}}
  >{\raggedright\arraybackslash}p{(\linewidth - 2\tabcolsep) * \real{0.4643}}@{}}
\toprule\noalign{}
\begin{minipage}[b]{\linewidth}\raggedright
Variable Type
\end{minipage} & \begin{minipage}[b]{\linewidth}\raggedright
Description
\end{minipage} \\
\midrule\noalign{}
\endhead
\bottomrule\noalign{}
\endlastfoot
Local Variable & Defined within a function and accessible only inside
that function \\
Global Variable & Defined outside functions and accessible throughout
the program \\
\end{longtable}
}

\textbf{Example:}

\begin{lstlisting}[language=Python]
# Global variable
count = 0  # This is a global variable

def update_count():
    # Local variable
    local_var = 5  # This is a local variable
    
    # Accessing global variable inside function
    global count
    count += 1
    
    print(f"Local variable: {local_var}")
    print(f"Global variable (inside function): {count}")
    
# Call the function
update_count()

# Accessing variables outside function
print(f"Global variable (outside function): {count}")

# This would cause an error if uncommented
# print(local_var)  # NameError: name 'local_var' is not defined
\end{lstlisting}

\textbf{Output:}

\begin{lstlisting}
Local variable: 5
Global variable (inside function): 1
Global variable (outside function): 1
\end{lstlisting}

\end{solutionbox}
\begin{mnemonicbox}
``SCOPE'' (Some variables Confined to function Only,
Program-wide Exposure for others)

\end{mnemonicbox}
\subsection*{Question 4(c) OR [7
marks]}\label{q4c}

\textbf{List out various List operations and explain any three using
example.}

\begin{solutionbox}
List operations in Python:

{\def\LTcaptype{none} % do not increment counter
\begin{longtable}[]{@{}ll@{}}
\toprule\noalign{}
Operation & Description \\
\midrule\noalign{}
\endhead
\bottomrule\noalign{}
\endlastfoot
Creating Lists & Using square brackets [] \\
Indexing & Accessing elements by position \\
Slicing & Extracting portions of a list \\
Append & Adding elements to the end \\
Insert & Adding elements at specific positions \\
Remove & Removing specific elements \\
Pop & Removing and returning elements \\
Sort & Ordering list elements \\
Reverse & Reversing list order \\
Extend & Combining lists \\
List Comprehensions & Creating lists using expressions \\
\end{longtable}
}

\textbf{Three List Operations with Examples:}

\begin{enumerate}
\tightlist
\item
  \textbf{List Indexing and Slicing:}
\end{enumerate}

\begin{lstlisting}[language=Python]
fruits = ["apple", "banana", "cherry", "orange", "kiwi"]
print(fruits[1])        # Output: banana
print(fruits[-1])       # Output: kiwi
print(fruits[1:4])      # Output: ['banana', 'cherry', 'orange']
\end{lstlisting}

\begin{enumerate}
\tightlist
\item
  \textbf{List Methods (append, insert, remove):}
\end{enumerate}

\begin{lstlisting}[language=Python]
numbers = [1, 2, 3]
numbers.append(4)       # Add 4 to the end
print(numbers)          # Output: [1, 2, 3, 4]

numbers.insert(0, 0)    # Insert 0 at position 0
print(numbers)          # Output: [0, 1, 2, 3, 4]

numbers.remove(2)       # Remove element with value 2
print(numbers)          # Output: [0, 1, 3, 4]
\end{lstlisting}

\begin{enumerate}
\tightlist
\item
  \textbf{List Comprehensions:}
\end{enumerate}

\begin{lstlisting}[language=Python]
# Create a list of squares
squares = [x**2 for x in range(1, 6)]
print(squares)  # Output: [1, 4, 9, 16, 25]

# Filter even numbers
numbers = [1, 2, 3, 4, 5, 6, 7, 8, 9, 10]
evens = [x for x in numbers if x % 2 == 0]
print(evens)    # Output: [2, 4, 6, 8, 10]
\end{lstlisting}

\end{solutionbox}
\begin{mnemonicbox}
``AIM'' (Access with index, Insert/modify elements,
Make using comprehensions)

\end{mnemonicbox}
\subsection*{Question 5(a) [3 marks]}\label{q5a}

\textbf{Write python code to swap given two elements in a list.}

\begin{solutionbox}

\begin{lstlisting}[language=Python]
# Program to swap two elements in a list
def swap_elements(my_list, pos1, pos2):
    """
    Swap elements at positions pos1 and pos2 in the list
    """
    # Check if positions are valid
    if 0 <= pos1 < len(my_list) and 0 <= pos2 < len(my_list):
        # Swap elements
        my_list[pos1], my_list[pos2] = my_list[pos2], my_list[pos1]
        return True
    else:
        return False

# Example usage
numbers = [10, 20, 30, 40, 50]
print("Original list:", numbers)

# Swap elements at positions 1 and 3
if swap_elements(numbers, 1, 3):
    print("After swapping:", numbers)
else:
    print("Invalid positions")
\end{lstlisting}

\textbf{Output:}

\begin{lstlisting}
Original list: [10, 20, 30, 40, 50]
After swapping: [10, 40, 30, 20, 50]
\end{lstlisting}

\end{solutionbox}
\begin{mnemonicbox}
``SWAP'' (Select positions, Watch boundaries, Assign
simultaneously, Print result)

\end{mnemonicbox}
\subsection*{Question 5(b) [4 marks]}\label{q5b}

\textbf{Explain math module and random module in python using example.}

\begin{solutionbox}
Math and random modules provide functions for
mathematical operations and random number generation.

\textbf{Math Module:}

\begin{lstlisting}[language=Python]
import math

# Constants
print(math.pi)          # Output: 3.141592653589793
print(math.e)           # Output: 2.718281828459045

# Mathematical functions
print(math.sqrt(16))    # Output: 4.0
print(math.ceil(4.2))   # Output: 5
print(math.floor(4.8))  # Output: 4
print(math.pow(2, 3))   # Output: 8.0
\end{lstlisting}

\textbf{Random Module:}

\begin{lstlisting}[language=Python]
import random

# Random float between 0 and 1
print(random.random())       # Output: 0.123... (random)

# Random integer within range
print(random.randint(1, 10)) # Output: 7 (random between 1 and 10)

# Random choice from a sequence
colors = ["red", "green", "blue"]
print(random.choice(colors)) # Output: "green" (random)

# Shuffle a list
numbers = [1, 2, 3, 4, 5]
random.shuffle(numbers)
print(numbers)               # Output: [3, 1, 5, 2, 4] (random)
\end{lstlisting}

\end{solutionbox}
\begin{mnemonicbox}
``MR-CS'' (Math for Calculations, Random for Choice
and Shuffling)

\end{mnemonicbox}
\subsection*{Question 5(c) [7 marks]}\label{q5c}

\textbf{Write a python code to demonstrate tuples functions and
operations.}

\begin{solutionbox}

\begin{lstlisting}[language=Python]
# Demonstrating Tuple Functions and Operations

# Creating tuples
empty_tuple = ()
single_item_tuple = (1,)  # Note the comma
mixed_tuple = (1, "Hello", 3.14, True)
nested_tuple = (1, 2, (3, 4))

# Accessing tuple elements
print("Accessing elements:")
print(mixed_tuple[0])      # Output: 1
print(mixed_tuple[-1])     # Output: True
print(nested_tuple[2][0])  # Output: 3

# Tuple slicing
print("\nTuple slicing:")
print(mixed_tuple[1:3])    # Output: ("Hello", 3.14)

# Tuple concatenation
tuple1 = (1, 2, 3)
tuple2 = (4, 5, 6)
tuple3 = tuple1 + tuple2
print("\nConcatenated tuple:", tuple3)  # Output: (1, 2, 3, 4, 5, 6)

# Tuple repetition
repeated_tuple = tuple1 * 3
print("\nRepeated tuple:", repeated_tuple)  # Output: (1, 2, 3, 1, 2, 3, 1, 2, 3)

# Tuple methods
numbers = (1, 2, 3, 2, 4, 2)
print("\nCount of 2:", numbers.count(2))  # Output: 3
print("Index of 3:", numbers.index(3))    # Output: 2

# Tuple unpacking
print("\nTuple unpacking:")
x, y, z = (10, 20, 30)
print(f"x={x},

y={y},

z={z}")  # Output:

x=10,

y=20,

z=30


# Check if an element exists in a tuple
print("\nMembership testing:")
print(3 in numbers)     # Output: True
print(5 in numbers)     # Output: False

# Converting list to tuple and vice versa
my_list = [1, 2, 3]
my_tuple = tuple(my_list)
print("\nList to tuple:", my_tuple)

back_to_list = list(my_tuple)
print("Tuple to list:", back_to_list)
\end{lstlisting}

\textbf{Diagram:}

\includegraphics[width=1\linewidth,height=\textheight,keepaspectratio]{mermaid-24fc5861.pdf}

\end{solutionbox}
\begin{mnemonicbox}
``CASC-RUMTC'' (Create, Access, Slice, Concatenate,
Repeat, Use methods, Membership test, Tuple conversion)

\end{mnemonicbox}
\subsection*{Question 5(a) OR [3
marks]}\label{q5a}

\textbf{Write a python code to find the sum of elements in a list.}

\begin{solutionbox}

\begin{lstlisting}[language=Python]
# Program to find the sum of elements in a list
def sum_of_elements(numbers):
    """
    Calculate the sum of all elements in a list
    """
    total = 0
    for num in numbers:
        total += num
    return total

# Example usage
my_list = [10, 20, 30, 40, 50]
print("List:", my_list)
print("Sum of elements:", sum_of_elements(my_list))  # Output: 150

# Alternative using built-in sum() function
print("Sum using built-in function:", sum(my_list))  # Output: 150
\end{lstlisting}

\textbf{Diagram:}

\includegraphics[width=1\linewidth,height=\textheight,keepaspectratio]{mermaid-4e3583d1.pdf}

\end{solutionbox}
\begin{mnemonicbox}
``SITE'' (Sum Initialized To zero, Elements added one
by one)

\end{mnemonicbox}
\subsection*{Question 5(b) OR [4
marks]}\label{q5b}

\textbf{Explain the usage of following built in functions:} \textbf{1)
Print() 2) Min() 3) Sum() 4) Input()}

\begin{solutionbox}

{\def\LTcaptype{none} % do not increment counter
\begin{longtable}[]{@{}
  >{\raggedright\arraybackslash}p{(\linewidth - 6\tabcolsep) * \real{0.2778}}
  >{\raggedright\arraybackslash}p{(\linewidth - 6\tabcolsep) * \real{0.2500}}
  >{\raggedright\arraybackslash}p{(\linewidth - 6\tabcolsep) * \real{0.2500}}
  >{\raggedright\arraybackslash}p{(\linewidth - 6\tabcolsep) * \real{0.2222}}@{}}
\toprule\noalign{}
\begin{minipage}[b]{\linewidth}\raggedright
Function
\end{minipage} & \begin{minipage}[b]{\linewidth}\raggedright
Purpose
\end{minipage} & \begin{minipage}[b]{\linewidth}\raggedright
Example
\end{minipage} & \begin{minipage}[b]{\linewidth}\raggedright
Output
\end{minipage} \\
\midrule\noalign{}
\endhead
\bottomrule\noalign{}
\endlastfoot
\textbf{print()} & Displays output to the console &
\passthrough{\lstinline!print("Hello World")!} & Hello World \\
\textbf{min()} & Returns smallest item in an iterable &
\passthrough{\lstinline!min([5, 3, 8, 1])!} & 1 \\
\textbf{sum()} & Returns sum of all items in an iterable &
\passthrough{\lstinline!sum([1, 2, 3, 4])!} & 10 \\
\textbf{input()} & Reads input from the user &
\passthrough{\lstinline!name = input("Enter name: ")!} & (waits for user
input) \\
\end{longtable}
}

\textbf{Example Code:}

\begin{lstlisting}[language=Python]
# print() function
print("Hello, Python!")  # Basic output
print("a", "b", "c", sep="-")  # Output with separator: a-b-c
print("No newline", end=" ")  # Custom end character
print("on same line")  # Output: No newline on same line

# min() function
numbers = [15, 8, 23, 4, 42]
print("Minimum value:", min(numbers))  # Output: 4
print("Minimum of 5, 2, 9:", min(5, 2, 9))  # Output: 2
chars = "wxyz"
print("Minimum character:", min(chars))  # Output: w

# sum() function
print("Sum of numbers:", sum(numbers))  # Output: 92
print("Sum with start value:", sum(numbers, 10))  # Output: 102

# input() function
user_input = input("Enter something: ")  # Prompts user for input
print("You entered:", user_input)  # Displays user's input
\end{lstlisting}

\end{solutionbox}
\begin{mnemonicbox}
``PMSI'' (Print to display, Min for smallest, Sum for
total, Input for reading)

\end{mnemonicbox}
\subsection*{Question 5(c) OR [7
marks]}\label{q5c}

\textbf{Write a python code to demonstrate the set functions and
operations.}

\begin{solutionbox}

\begin{lstlisting}[language=Python]
# Demonstrating Set Functions and Operations

# Creating sets
empty_set = set()  # Empty set
numbers = {1, 2, 3, 4, 5}
duplicates = {1, 2, 2, 3, 4, 4, 5}  # Duplicates removed automatically
print("Original set:", numbers)
print("Set with duplicates:", duplicates)  # Output: {1, 2, 3, 4, 5}

# Adding elements
numbers.add(6)
print("\nAfter adding 6:", numbers)  # Output: {1, 2, 3, 4, 5, 6}

# Updating with multiple elements
numbers.update([7, 8, 9])
print("After updating:", numbers)  # Output: {1, 2, 3, 4, 5, 6, 7, 8, 9}

# Removing elements
numbers.remove(5)  # Raises error if element not found
print("\nAfter removing 5:", numbers)

numbers.discard(10)  # No error if element not found
print("After discarding 10:", numbers)  # No change

popped = numbers.pop()  # Removes and returns arbitrary element
print("Popped element:", popped)
print("After pop:", numbers)

# Set operations
set1 = {1, 2, 3, 4, 5}
set2 = {4, 5, 6, 7, 8}

# Union
union_set = set1 | set2  # or set1.union(set2)
print("\nUnion:", union_set)  # Output: {1, 2, 3, 4, 5, 6, 7, 8}

# Intersection
intersection_set = set1 & set2  # or set1.intersection(set2)
print("Intersection:", intersection_set)  # Output: {4, 5}

# Difference
difference_set = set1 - set2  # or set1.difference(set2)
print("Difference (set1 - set2):", difference_set)  # Output: {1, 2, 3}

# Symmetric Difference
symmetric_diff = set1 ^ set2  # or set1.symmetric_difference(set2)
print("Symmetric difference:", symmetric_diff)  # Output: {1, 2, 3, 6, 7, 8}

# Subset and Superset
subset = {1, 2}
print("\nIs {1, 2} subset of set1?", subset.issubset(set1))  # Output: True
print("Is set1 superset of {1, 2}?", set1.issuperset(subset))  # Output: True
\end{lstlisting}

\textbf{Diagram:}

\includegraphics[width=1\linewidth,height=\textheight,keepaspectratio]{mermaid-fbee1dc3.pdf}

\end{solutionbox}
\begin{mnemonicbox}
``CARDS-UI'' (Create, Add, Remove, Discard elements,
Set operations - Union, Intersection)

\end{mnemonicbox}

\end{document}
