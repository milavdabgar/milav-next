\documentclass{article}

% content/resources/templates/preamble.tex
\usepackage[margin=0.6in]{geometry}
\author{Milav Dabgar}
\usepackage{amsmath,amssymb,amsthm}
\usepackage{booktabs}
\usepackage{multirow}
\usepackage{xcolor}
\usepackage{tcolorbox}
\tcbuselibrary{breakable,skins}
\usepackage[colorlinks=true,linkcolor=blue]{hyperref}
\usepackage{titlesec}
\usepackage{enumitem}
\usepackage{tikz}
\usepackage{pgfplots}
\usepackage{circuitikz}
\usepackage[version=4]{mhchem}
\usepackage{longtable}
\usepackage{array}
\usepackage{float}
\usepackage{caption}
\usepackage{listings}

\lstset{
  basicstyle=\small\ttfamily,
  breaklines=true,
  breakatwhitespace=false,
  postbreak=\mbox{\textcolor{red}{$\hookrightarrow$}\space},
  float=false,
  numbers=left,
  numberstyle=\tiny\color{gray},
  numbersep=10pt,
  xleftmargin=2em,
  keywordstyle=\color{blue},
  commentstyle=\color{green!60!black},
  stringstyle=\color{purple},
  backgroundcolor=\color{gray!5},
  showstringspaces=false,
  tabsize=2,
  captionpos=b,
  keepspaces=true,
  columns=flexible
}

\pgfplotsset{compat=1.18}
\usetikzlibrary{shapes,arrows,positioning,calc,patterns,decorations.pathmorphing,decorations.markings,arrows.meta}

% Color scheme
\definecolor{headcolor}{RGB}{0,102,204}
\definecolor{keycolor}{RGB}{220,20,60}
\definecolor{solutioncolor}{RGB}{34,139,34}
\definecolor{mnemoniccolor}{RGB}{148,0,211}
\definecolor{codecolor}{RGB}{0,0,100}

% Spacing
\setlength{\parskip}{3pt}
\setlist[itemize]{nosep}
\setlist[enumerate]{nosep}

% Title formatting
\titleformat{\section}{\Large\bfseries\color{headcolor}}{\thesection}{1em}{}
\titleformat{\subsection}{\large\bfseries\color{headcolor}}{\thesubsection}{1em}{}

% Pandoc tightlist compatibility
\providecommand{\tightlist}{%
  \setlength{\itemsep}{0pt}\setlength{\parskip}{0pt}}

% Pandoc longtable compatibility
\newcounter{none}
\def\thenone{}


% content/resources/templates/gujarati-boxes.tex
\usepackage{fontspec}
\usepackage{polyglossia}

% Set Gujarati as main language (document is primarily in Gujarati)
% Note: gloss-gujarati.ldf doesn't exist in polyglossia, but it will use hyphenation patterns
\setdefaultlanguage{gujarati}
\setotherlanguage{english}

% Configure Gujarati font properly
% Use Language=Default to prevent polyglossia from trying to add language-specific features
% that don't exist for Gujarati, which causes "empty feature" warnings
\newfontfamily\gujaratifont[Script=Gujarati,AutoFakeBold=2.5,AutoFakeSlant=0.3]{Noto Sans Gujarati}
\setmainfont[Script=Gujarati,AutoFakeBold=2.5,AutoFakeSlant=0.3]{Noto Sans Gujarati}
% Use Noto Sans Gujarati for monospace to support Gujarati in text
\setmonofont[Scale=0.9]{Noto Sans Gujarati}

% Configure English to use the same font
\newfontfamily\englishfont[Script=Gujarati,AutoFakeBold=2.5,AutoFakeSlant=0.3]{Noto Sans Gujarati}

% Translations for polyglossia
\gappto\captionsgujarati{
  \renewcommand{\tablename}{કોષ્ટક}
  \renewcommand{\figurename}{આકૃતિ}
}

% Helper for TikZ nodes to ensure Gujarati font
\newcommand{\gu}[1]{{\gujaratifont #1}}

% Custom environments
\newtcolorbox{solutionbox}{
    breakable,
    enhanced,
    colback=solutioncolor!5!white,
    colframe=solutioncolor!75!black,
    fonttitle=\bfseries,
    title=જવાબ
}

\newtcolorbox{solutionboxnobreak}{
 colback=solutioncolor!5!white,
 colframe=solutioncolor!75!black,
 fonttitle=\bfseries,
 title=જવાબ
}

\newtcolorbox{keyformula}{
 breakable,
 enhanced,
 colback=keycolor!5!white,
 colframe=keycolor!75!black,
 fonttitle=\bfseries,
 title=રાસાયણિક સમીકરણ/સૂત્ર
}

\newtcolorbox{mnemonicbox}{
 breakable,
 enhanced,
 colback=mnemoniccolor!5!white,
 colframe=mnemoniccolor!75!black,
 fonttitle=\bfseries,
 title=મેમરી ટ્રીક
}


% Custom commands for GTU solutions
% This file defines semantic commands for consistent formatting

% Question command with automatic formatting
\newcommand{\question}[2]{%
  \section*{Question #1}%
  \textbf{#2}%
}

% OR question variant
\newcommand{\questionor}[2]{%
  \section*{Question #1 OR}%
  \textbf{#2}%
}

% Proper table environment with caption
\newenvironment{answertable}[1]{%
  \begin{table}[htbp]
  \centering
  \caption{#1}
}{%
  \end{table}
}

% Proper figure environment for diagrams
\newenvironment{answerdiagram}[1]{%
  \begin{figure}[htbp]
  \centering
  \caption{#1}
}{%
  \end{figure}
}

% Semantic markup for key terms
\newcommand{\keyword}[1]{\textbf{#1}}
\newcommand{\code}[1]{\texttt{#1}}
\newcommand{\classname}[1]{\texttt{#1}}
\newcommand{\methodname}[1]{\texttt{#1}}

% Proper quotation marks
\newcommand{\mnemonic}[1]{``#1''}


\title{પાયથોન પ્રોગ્રામિંગ (1323203) - ઉનાળુ 2024 સોલ્યુશન}
\date{જૂન 18, 2024}

\begin{document}
\maketitle

\questionmarks{1(અ)}{3}{ફ્લોચાર્ટ અને અલ્ગોરિધમના મહત્વની યાદી આપો.}

\begin{solutionbox}
\begin{answertable}{ફ્લોચાર્ટ અને અલ્ગોરિધમનું મહત્વ}
\begin{tabulary}{\linewidth}{|L|L|}
\hline
\textbf{ફ્લોચાર્ટનું મહત્વ} & \textbf{અલ્ગોરિધમનું મહત્વ} \\
\hline
પ્રોગ્રામ લોજિકનું દૃશ્ય નિરૂપણ & સમસ્યાને ઉકેલવા માટેનું પગલાંવાર પ્રક્રિયા \\
\hline
ભૂલોને સરળતાથી શોધવા અને સુધારવા & ભાષાથી સ્વતંત્ર ઉકેલ અભિગમ \\
\hline
જટિલ પ્રક્રિયાઓને સમજવામાં મદદ & પ્રોગ્રામિંગની પાયારૂપ શરૂઆત \\
\hline
ટીમના સભ્યો વચ્ચે સંદેશાવ્યવહાર સુધારે & કોડિંગ શરૂ કરતા પહેલા લોજિક નિર્ધારિત કરે \\
\hline
\end{tabulary}
\end{answertable}
\end{solutionbox}

\begin{mnemonicbox}
\mnemonic{VASE Decisions} - Visualize, Analyze, Sequence, Execute
\end{mnemonicbox}

\questionmarks{1(બ)}{4}{દાખલ કરેલ સંખ્યા ઈવન કે ઓડ છે તે શોધવા માટે ફ્લોચાર્ટ દોરો.}

\begin{solutionbox}
\begin{answerdiagram}{ઈવન કે ઓડ સંખ્યા તપાસવા માટે ફ્લોચાર્ટ}
\begin{tikzpicture}[gtu flow]
    \node[gtu start] (start) {શરૂઆત};
    \node[gtu input, below of=start] (input) {નંબર n ઇનપુટ};
    \node[gtu decision, below of=input] (dec) {n \% 2 == 0?};
    \node[gtu process, below left of=dec, xshift=-1cm, yshift=-1cm] (even) {પ્રિન્ટ ઈવન નંબર};
    \node[gtu process, below right of=dec, xshift=1cm, yshift=-1cm] (odd) {પ્રિન્ટ ઓડ નંબર};
    \node[gtu stop, below of=dec, yshift=-3cm] (end) {અંત};

    \draw[gtu arrow] (start) -- (input);
    \draw[gtu arrow] (input) -- (dec);
    \draw[gtu arrow] (dec) -| node[above left] {હા} (even);
    \draw[gtu arrow] (dec) -| node[above right] {ના} (odd);
    \draw[gtu arrow] (even) |- (end);
    \draw[gtu arrow] (odd) |- (end);
\end{tikzpicture}
\end{answerdiagram}

\textbf{મુખ્ય પગલાં:}
\begin{itemize}
    \item \keyword{ડેટા એકત્રીકરણ}: વપરાશકર્તા પાસેથી નંબર મેળવો
    \item \keyword{મોડ્યુલો ઓપરેશન}: 2 વડે ભાગીને શેષ તપાસો
    \item \keyword{શરતી આઉટપુટ}: શેષના આધારે પરિણામ દર્શાવો
\end{itemize}
\end{solutionbox}

\begin{mnemonicbox}
\mnemonic{MODE} - Modulo Operation Determines Evenness
\end{mnemonicbox}

\questionmarks{1(ક)}{7}{બધા લોજિકલ ઓપરેટરોની યાદી બનાવો અને પાયથોન કોડનું ઉદાહરણ આપીને દરેકને સમજાવો.}

\begin{solutionbox}
\begin{answertable}{લોજિકલ ઓપરેટરો}
\begin{tabulary}{\linewidth}{|l|L|L|l|}
\hline
\textbf{ઓપરેટર} & \textbf{વર્ણન} & \textbf{ઉદાહરણ} & \textbf{આઉટપુટ} \\
\hline
\code{and} & બંને સ્ટેટમેન્ટ સાચા હોય તો True રિટર્ન કરે & \code{x = 5; print(x > 3 and x < 10)} & \code{True} \\
\hline
\code{or} & બે સ્ટેટમેન્ટમાંથી એક સાચું હોય તો True રિટર્ન કરે & \code{x = 5; print(x > 10 or x == 5)} & \code{True} \\
\hline
\code{not} & પરિણામને ઉલટાવે, જો પરિણામ સાચું હોય તો False રિટર્ન કરે & \code{x = 5; print(not(x > 3))} & \code{False} \\
\hline
\end{tabulary}
\end{answertable}

\textbf{કોડ ઉદાહરણ:}
\begin{lstlisting}[language=Python]
# લોજિકલ AND ઉદાહરણ
age = 25
income = 50000
print("Loan eligibility:", age > 18 and income > 30000)  # True

# લોજિકલ OR ઉદાહરણ
has_credit_card = False
has_cash = True
print("Can purchase:", has_credit_card or has_cash)  # True

# લોજિકલ NOT ઉદાહરણ
is_holiday = False
print("Should work today:", not is_holiday)  # True
\end{lstlisting}
\end{solutionbox}

\begin{mnemonicbox}
\mnemonic{AON Clarity} - And, Or, Not for logical clarity
\end{mnemonicbox}

\orquestionmarks{1(ક)}{7}{એક પાયથોન પ્રોગ્રામ લખો કરો જે આપેલ ડેટા પર સાદા વ્યાજ અને ચક્રવૃદ્ધિ વ્યાજની ગણતરી કરી શકે.}

\begin{solutionbox}
\begin{lstlisting}[language=Python]
# સાદા અને ચક્રવૃદ્ધિ વ્યાજની ગણતરી માટેનો પ્રોગ્રામ

# ઇનપુટ મૂલ્યો
principal = float(input("Enter principal amount: "))
rate = float(input("Enter rate of interest (in %): "))
time = float(input("Enter time period (in years): "))

# સાદા વ્યાજની ગણતરી
simple_interest = (principal * rate * time) / 100

# ચક્રવૃદ્ધિ વ્યાજની ગણતરી
compound_interest = principal * ((1 + rate/100) ** time - 1)

# પરિણામો દર્શાવો
print("Simple Interest:", round(simple_interest, 2))
print("Compound Interest:", round(compound_interest, 2))
\end{lstlisting}

\begin{keyformula}
\begin{itemize}
    \item \textbf{સાદું વ્યાજ (SI)}: મૂળ રકમ $\times$ દર $\times$ સમય / 100
    \item \textbf{ચક્રવૃદ્ધિ વ્યાજ (CI)}: મૂળ રકમ $\times$ ((1 + દર/100)\textsuperscript{સમય} - 1)
\end{itemize}
\end{keyformula}
\end{solutionbox}

\begin{mnemonicbox}
\mnemonic{PRT Money Grows} - Principal, Rate, Time make money grow
\end{mnemonicbox}

\questionmarks{2(અ)}{3}{આપેલ ત્રણ નંબરોમાંથી ન્યૂનતમ સંખ્યા શોધવા માટે પાયથોન પ્રોગ્રામ બનાવો.}

\begin{solutionbox}
\begin{lstlisting}[language=Python]
# ત્રણ નંબરોમાંથી ન્યૂનતમ શોધવાનો પ્રોગ્રામ

# ત્રણ નંબર ઇનપુટ લો
num1 = float(input("Enter first number: "))
num2 = float(input("Enter second number: "))
num3 = float(input("Enter third number: "))

# બિલ્ટ-ઇન min() ફંક્શનનો ઉપયોગ કરીને ન્યૂનતમ શોધો
minimum = min(num1, num2, num3)

# પરિણામ દર્શાવો
print("Minimum number is:", minimum)
\end{lstlisting}
\end{solutionbox}

\begin{mnemonicbox}
\mnemonic{MIN Finds Least} - Minimum Is Numerically Found with Least
\end{mnemonicbox}

\questionmarks{2(બ)}{4}{સ્યુડોકોડ વ્યાખ્યાયિત કરો. x, y અને z ત્રણમાંથી સૌથી મોટી સંખ્યા શોધવા માટે સ્યુડોકોડ લખો.}

\begin{solutionbox}
\begin{answertable}{સ્યુડોકોડની વ્યાખ્યા}
\begin{tabulary}{\linewidth}{|L|}
\hline
\textbf{વ્યાખ્યા} \\
\hline
કમ્પ્યુટર પ્રોગ્રામે શું કરવું જોઈએ તેનું વિગતવાર અને વાંચી શકાય તેવું વર્ણન, જે પ્રોગ્રામિંગ ભાષાને બદલે ઔપચારિક શૈલીમાં લખાયેલી કુદરતી ભાષામાં વ્યક્ત કરવામાં આવે છે. \\
\hline
\end{tabulary}
\end{answertable}

\textbf{ત્રણ નંબરોમાંથી સૌથી મોટો શોધવા માટે સ્યુડોકોડ:}
\begin{lstlisting}
BEGIN
    INPUT x, y, z
    SET largest = x
    
    IF y > largest THEN
        SET largest = y
    END IF
    
    IF z > largest THEN
        SET largest = z
    END IF
    
    OUTPUT "Largest number is: ", largest
END
\end{lstlisting}
\end{solutionbox}

\begin{mnemonicbox}
\mnemonic{PIE Writing} - Program Ideas Expressed in simple writing
\end{mnemonicbox}

\questionmarks{2(ક)}{7}{પાયથોનમાં વાઈલ લૂપને તેના સિન્ટેક્સ, ફ્લોચાર્ટ અને પાયથોન કોડના ઉદાહરણ સાથે સમજાવો.}

\begin{solutionbox}
\textbf{સિન્ટેક્સ:}
\begin{lstlisting}[language=Python]
while condition:
    # code to be executed
\end{lstlisting}

\begin{answerdiagram}{વાઈલ લૂપનો ફ્લોચાર્ટ}
\begin{tikzpicture}[gtu flow]
    \node[gtu start] (start) {શરૂઆત};
    \node[gtu process, below of=start] (init) {Variables Initialize કરો};
    \node[gtu decision, below of=init, yshift=-0.5cm] (cond) {શરત સાચી છે?};
    \node[gtu process, right of=cond, xshift=2.5cm] (body) {સ્ટેટમેન્ટ્સ એક્ઝિક્યુટ કરો};
    \node[gtu stop, below of=cond, yshift=-2cm] (end) {અંત};

    \draw[gtu arrow] (start) -- (init);
    \draw[gtu arrow] (init) -- (cond);
    \draw[gtu arrow] (cond) -- node[above] {હા} (body);
    \draw[gtu arrow] (body) |- ($(cond.north)+(0,0.5)$) -- (cond.north);
    \draw[gtu arrow] (cond) -- node[right] {ના} (end);
\end{tikzpicture}
\end{answerdiagram}

\textbf{કોડ ઉદાહરણ:}
\begin{lstlisting}[language=Python]
# પ્રથમ 5 કુદરતી સંખ્યાઓ while લૂપનો ઉપયોગ કરીને પ્રિન્ટ કરો
count = 1

while count <= 5:
    print(count)
    count += 1  # કાઉન્ટર વધારો

# આઉટપુટ:
# 1
# 2
# 3
# 4
# 5
\end{lstlisting}

\textbf{મુખ્ય લક્ષણો:}
\begin{itemize}
    \item \keyword{એન્ટ્રી કંટ્રોલ}: લૂપ એક્ઝિક્યુશન પહેલાં શરત ચકાસવામાં આવે છે
    \item \keyword{ઇનિશિયલાઇઝેશન}: લૂપ પહેલાં વેરિએબલ્સ સેટ કરવામાં આવે છે
    \item \keyword{અપડેશન}: લૂપની અંદર વેરિએબલ્સ અપડેટ કરવામાં આવે છે
    \item \keyword{ટર્મિનેશન}: શરત ખોટી થાય ત્યારે લૂપ બહાર નીકળે છે
\end{itemize}
\end{solutionbox}

\begin{mnemonicbox}
\mnemonic{IUTE Loop} - Initialize, Update, Test for Exit
\end{mnemonicbox}

\orquestionmarks{2(અ)}{3}{પાયથોનમાં કન્ટિન્યુ સ્ટેટમેન્ટનું ટૂંકમાં વર્ણન કરો.}

\begin{solutionbox}
\begin{answertable}{કન્ટિન્યુ સ્ટેટમેન્ટ}
\begin{tabulary}{\linewidth}{|L|}
\hline
\textbf{વર્ણન} \\
\hline
કન્ટિન્યુ સ્ટેટમેન્ટ લૂપના વર્તમાન ઇટરેશનને છોડી દે છે અને આગલા ઇટરેશનથી ચાલુ રાખે છે \\
\hline
જ્યારે એનકાઉન્ટર થાય, ત્યારે કન્ટિન્યુ સ્ટેટમેન્ટ પછીનો લૂપનો કોડ છોડી દેવામાં આવે છે \\
\hline
ચોક્કસ શરતોને છોડીને લૂપને ચાલુ રાખવા માટે ઉપયોગી છે \\
\hline
\end{tabulary}
\end{answertable}

\textbf{કોડ ઉદાહરણ:}
\begin{lstlisting}[language=Python]
# બેકી સંખ્યાઓ પ્રિન્ટ કરવાનું છોડી દો
for i in range(1, 6):
    if i % 2 == 0:
        continue
    print(i)  # માત્ર 1, 3, 5 પ્રિન્ટ થાય
\end{lstlisting}
\end{solutionbox}

\begin{mnemonicbox}
\mnemonic{SKIP Ahead} - Skip Keeping Iteration Process
\end{mnemonicbox}

\orquestionmarks{2(બ)}{4}{નીચેના કોડનું આઉટપુટ શું હશે?}

\begin{solutionbox}
\textbf{કોડ:}
\begin{lstlisting}[language=Python]
x=8
y=2
print (x*y)
print (x ** y)
print (x % y)
print(x>y)
\end{lstlisting}

\begin{answertable}{આઉટપુટ વિશ્લેષણ}
\begin{tabulary}{\linewidth}{|l|l|L|}
\hline
\textbf{ઓપરેશન} & \textbf{પરિણામ} & \textbf{સમજૂતી} \\
\hline
\code{x*y} & \code{16} & ગુણાકાર: 8 $\times$ 2 = 16 \\
\hline
\code{x**y} & \code{64} & પાવર: 8\textsuperscript{2} = 64 \\
\hline
\code{x\%y} & \code{0} & મોડ્યુલો (શેષ): 8 $\div$ 2 = 4 શેષ 0 \\
\hline
\code{x>y} & \code{True} & તુલના: 8 > 2 સાચું છે \\
\hline
\end{tabulary}
\end{answertable}
\end{solutionbox}

\begin{mnemonicbox}
\mnemonic{MEMO} - Multiply, Exponent, Modulo, Operator comparison
\end{mnemonicbox}

\orquestionmarks{2(ક)}{7}{પાયથોનમાં ઈફ-ઈએલઈએફ-એલ્સ લેડરને તેના સિન્ટેક્સ, ફ્લોચાર્ટ અને પાયથોન કોડના ઉદાહરણ સાથે સમજાવો.}

\begin{solutionbox}
\textbf{સિન્ટેક્સ:}
\begin{lstlisting}[language=Python]
if condition1:
    # code block 1
elif condition2:
    # code block 2
elif condition3:
    # code block 3
else:
    # code block 4
\end{lstlisting}

\begin{answerdiagram}{ઈફ-ઈએલઈએફ-એલ્સ લેડરનો ફ્લોચાર્ટ}
\begin{tikzpicture}[gtu flow]
    \node[gtu start] (start) {શરૂઆત};
    \node[gtu decision, below of=start] (dec1) {શરત1?};
    \node[gtu process, right of=dec1, xshift=2cm] (proc1) {કોડ બ્લોક 1};
    
    \node[gtu decision, below of=dec1, yshift=-1cm] (dec2) {શરત2?};
    \node[gtu process, right of=dec2, xshift=2cm] (proc2) {કોડ બ્લોક 2};
    
    \node[gtu decision, below of=dec2, yshift=-1cm] (dec3) {શરત3?};
    \node[gtu process, right of=dec3, xshift=2cm] (proc3) {કોડ બ્લોક 3};
    
    \node[gtu process, below of=dec3, yshift=-1cm] (proc4) {કોડ બ્લોક 4}; % Else block
    
    \node[gtu stop, below of=proc4] (end) {અંત};

    \draw[gtu arrow] (start) -- (dec1);
    
    \draw[gtu arrow] (dec1) -- node[above] {સાચું} (proc1);
    \draw[gtu arrow] (dec1) -- node[left] {ખોટું} (dec2);
    
    \draw[gtu arrow] (dec2) -- node[above] {સાચું} (proc2);
    \draw[gtu arrow] (dec2) -- node[left] {ખોટું} (dec3);
    
    \draw[gtu arrow] (dec3) -- node[above] {સાચું} (proc3);
    \draw[gtu arrow] (dec3) -- node[left] {ખોટું} (proc4);
    
    \draw[gtu arrow] (proc1) -| (4, -9) |- (end);
    \draw[gtu arrow] (proc2) -| (4, -9) |- (end);
    \draw[gtu arrow] (proc3) -| (4, -9) |- (end);
    \draw[gtu arrow] (proc4) -- (end);
\end{tikzpicture}
\end{answerdiagram}

\textbf{કોડ ઉદાહરણ:}
\begin{lstlisting}[language=Python]
# માર્ક્સના આધારે ગ્રેડની ગણતરી
marks = 75

if marks >= 90:
    grade = "A+"
elif marks >= 80:
    grade = "A"
elif marks >= 70:
    grade = "B"
elif marks >= 60:
    grade = "C"
else:
    grade = "D"

print("Grade:", grade)  # આઉટપુટ: Grade: B
\end{lstlisting}

\textbf{મુખ્ય લક્ષણો:}
\begin{itemize}
    \item \keyword{અનુક્રમિક મૂલ્યાંકન}: શરતો ઉપરથી નીચે તપાસવામાં આવે છે
    \item \keyword{અનન્ય એક્ઝિક્યુશન}: માત્ર એક બ્લોક એક્ઝિક્યુટ થાય છે
    \item \keyword{ડિફોલ્ટ એક્શન}: જો કોઈ શરત સાચી ન હોય તો else બ્લોક એક્ઝિક્યુટ થાય છે
\end{itemize}
\end{solutionbox}

\begin{mnemonicbox}
\mnemonic{SEEP Logic} - Sequential Evaluation with Exclusive Path
\end{mnemonicbox}

\questionmarks{3(અ)}{3}{લૂપ્સનો ઉપયોગ કરીને 1 થી 20 વચ્ચેની એકી સંખ્યાઓ છાપવા માટે પાયથોન પ્રોગ્રામ લખો.}

\begin{solutionbox}
\begin{lstlisting}[language=Python]
# 1 થી 20 વચ્ચેની એકી સંખ્યાઓ છાપવાનો પ્રોગ્રામ

# range અને step સાથે for લૂપનો ઉપયોગ
for number in range(1, 21, 2):
    print(number, end=" ")

# આઉટપુટ: 1 3 5 7 9 11 13 15 17 19
\end{lstlisting}

\textbf{વૈકલ્પિક અભિગમ:}
\begin{lstlisting}[language=Python]
# if શરત સાથે for લૂપનો ઉપયોગ
for number in range(1, 21):
    if number % 2 != 0:
        print(number, end=" ")
\end{lstlisting}
\end{solutionbox}

\begin{mnemonicbox}
\mnemonic{STEO} - Skip Two, Extract Odds
\end{mnemonicbox}

\questionmarks{3(બ)}{4}{નેસ્ટેડ ઈફ સ્ટેટમેન્ટને સંક્ષિપ્તમાં સમજાવો.}

\begin{solutionbox}
\begin{answertable}{નેસ્ટેડ ઈફ સ્ટેટમેન્ટ}
\begin{tabulary}{\linewidth}{|L|}
\hline
\textbf{વર્ણન} \\
\hline
બીજા if સ્ટેટમેન્ટની અંદર એક if સ્ટેટમેન્ટ \\
\hline
વધુ જટિલ શરતી લોજિકની મંજૂરી આપે છે \\
\hline
બાહ્ય if સાચું હોય ત્યારે જ આંતરિક if મૂલ્યાંકન કરવામાં આવે છે \\
\hline
નેસ્ટિંગના ઘણા સ્તરો હોઈ શકે છે \\
\hline
\end{tabulary}
\end{answertable}

\textbf{કોડ ઉદાહરણ:}
\begin{lstlisting}[language=Python]
age = 25
income = 50000

if age > 18:
    print("પુખ્ત")
    if income > 30000:
        print("ક્રેડિટ કાર્ડ માટે પાત્ર")
    else:
        print("ક્રેડિટ કાર્ડ માટે અપાત્ર")
else:
    print("સગીર")
\end{lstlisting}
\end{solutionbox}

\begin{mnemonicbox}
\mnemonic{LION} - Layered If-statements Operating Nested
\end{mnemonicbox}

\questionmarks{3(ક)}{7}{યુઝર ડિફાઈન ફંક્શનનો ઉપયોગ કરીને પ્રોગ્રામ લખો જે દાખલ કરેલ નંબર 'આર્મસ્ટ્રોંગ નંબર' અથવા પેલિન્ડ્રોમ છે તે તપાસવા માટે પ્રોગ્રામ લખો એ જેમાં કૉલિંગ ફંક્શનમાં આર્ગ્યુમેંટ તરીકે નંબર આપવામા આવે છે.}

\begin{solutionbox}
\begin{lstlisting}[language=Python]
# આર્મસ્ટ્રોંગ નંબર અથવા પેલિન્ડ્રોમ તપાસવાનો પ્રોગ્રામ

def check_number(num):
    # આર્મસ્ટ્રોંગ નંબર તપાસો
    temp = num
    digits = len(str(num))
    sum = 0
    
    while temp > 0:
        digit = temp % 10
        sum += digit ** digits
        temp //= 10
    
    is_armstrong = (sum == num)
    
    # પેલિન્ડ્રોમ તપાસો
    is_palindrome = (str(num) == str(num)[::-1])
    
    # પરિણામો પાછા આપો
    return is_armstrong, is_palindrome

# વપરાશકર્તા પાસેથી ઇનપુટ લો
number = int(input("એક નંબર દાખલ કરો: "))

# ફંક્શન કૉલ કરો અને પરિણામો દર્શાવો
armstrong, palindrome = check_number(number)

if armstrong:
    print(number, "is an Armstrong number")
else:
    print(number, "is not an Armstrong number")
    
if palindrome:
    print(number, "is a Palindrome")
else:
    print(number, "is not a Palindrome")
\end{lstlisting}

\textbf{ઉદાહરણો:}
\begin{itemize}
    \item \textbf{આર્મસ્ટ્રોંગ}: 153 ($1^3 + 5^3 + 3^3 = 1 + 125 + 27 = 153$)
    \item \textbf{પેલિન્ડ્રોમ}: 121 (આગળ અને પાછળ સમાન)
\end{itemize}
\end{solutionbox}

\begin{mnemonicbox}
\mnemonic{APTEST} - Armstrong Palindrome Test Equal Sum Test
\end{mnemonicbox}

\orquestionmarks{3(અ)}{3}{૧ થી ૧૦૦ સુધી નો સરવાળો શોધવા માટે પાયથોન પ્રોગ્રામ લખો.}

\begin{solutionbox}
\begin{lstlisting}[language=Python]
# 1 થી 100 સુધીની સંખ્યાઓનો સરવાળો શોધવાનો પ્રોગ્રામ

# પદ્ધતિ 1: લૂપનો ઉપયોગ
total = 0
for num in range(1, 101):
    total += num
print("Sum using loop:", total)

# પદ્ધતિ 2: સૂત્ર n(n+1)/2 નો ઉપયોગ
n = 100
sum_formula = n * (n + 1) // 2
print("Sum using formula:", sum_formula)

# આઉટપુટ: 
# Sum using loop: 5050
# Sum using formula: 5050
\end{lstlisting}
\end{solutionbox}

\begin{mnemonicbox}
\mnemonic{SUM Formula} - Sum Using Mathematical Formula
\end{mnemonicbox}

\orquestionmarks{3(બ)}{4}{નીચેની પેટર્ન છાપવા માટે પાયથોન પ્રોગ્રામ લખો.}

\begin{solutionbox}
\textbf{પેટર્ન:}
\begin{verbatim}
1
2 3
4 5 6
7 8 9 10
\end{verbatim}

\textbf{કોડ:}
\begin{lstlisting}[language=Python]
# સંખ્યા પેટર્ન છાપવાનો પ્રોગ્રામ

num = 1
for i in range(1, 5):  # 4 પંક્તિઓ
    for j in range(i):  # પંક્તિ નંબર જેટલા કોલમ
        print(num, end=" ")
        num += 1
    print()  # દરેક પંક્તિ પછી નવી લાઈન
\end{lstlisting}

\textbf{પેટર્ન લોજિક:}
\begin{itemize}
    \item \textbf{પંક્તિ 1}: 1 સંખ્યા (1)
    \item \textbf{પંક્તિ 2}: 2 સંખ્યાઓ (2, 3)
    \item \textbf{પંક્તિ 3}: 3 સંખ્યાઓ (4, 5, 6)
    \item \textbf{પંક્તિ 4}: 4 સંખ્યાઓ (7, 8, 9, 10)
\end{itemize}
\end{solutionbox}

\begin{mnemonicbox}
\mnemonic{CNIR} - Counter Number Increases with Rows
\end{mnemonicbox}

\orquestionmarks{3(ક)}{7}{ફંક્શનનો ઉપયોગ કરીને પ્રોગ્રામ લખો જે દાખલ કરેલ નંબરને ઉલટાવે}

\begin{solutionbox}
\begin{lstlisting}[language=Python]
# દાખલ કરેલ મૂલ્યને ઉલટાવવા માટે ફંક્શન ઉપયોગ કરતો પ્રોગ્રામ

def reverse_number(num):
    """સંખ્યાને ઉલટાવવા માટેનું ફંક્શન"""
    return int(str(num)[::-1])

def reverse_string(text):
    """સ્ટ્રિંગને ઉલટાવવા માટેનું ફંક્શન"""
    return text[::-1]

# મુખ્ય પ્રોગ્રામ
def main():
    choice = input("What do you want to reverse? (n for number, s for string): ")
    
    if choice.lower() == 'n':
        num = int(input("Enter a number: "))
        print("Reversed number:", reverse_number(num))
    elif choice.lower() == 's':
        text = input("Enter a string: ")
        print("Reversed string:", reverse_string(text))
    else:
        print("Invalid choice!")

# મુખ્ય ફંક્શન કૉલ કરો
main()
\end{lstlisting}

\textbf{નંબર ઉલટાવવા માટે વૈકલ્પિક પદ્ધતિ:}
\begin{lstlisting}[language=Python]
def reverse_number_algorithm(num):
    reversed_num = 0
    while num > 0:
        digit = num % 10
        reversed_num = reversed_num * 10 + digit
        num //= 10
    return reversed_num
\end{lstlisting}
\end{solutionbox}

\begin{mnemonicbox}
\mnemonic{FLIP Digits} - Function Logic Inverts Position of Digits
\end{mnemonicbox}

\questionmarks{4(અ)}{3}{યોગ્ય પાયથોન કોડ ઉદાહરણ સાથે પાયથોન મેથ મોડ્યુલનું વર્ણન કરો.}

\begin{solutionbox}
\begin{answertable}{પાયથોન મેથ મોડ્યુલ}
\begin{tabulary}{\linewidth}{|L|}
\hline
\textbf{વિશેષતાઓ} \\
\hline
ગાણિતિક ફંક્શન્સ અને સ્થિરાંકો પ્રદાન કરે છે \\
\hline
ત્રિકોણમિતિય, લોગરિધમિક અને અન્ય ફંક્શન્સ શામેલ છે \\
\hline
pi અને e જેવા ગાણિતિક સ્થિરાંકો ધરાવે છે \\
\hline
ઉપયોગ કરતા પહેલા import કરવું જરૂરી છે \\
\hline
\end{tabulary}
\end{answertable}

\textbf{કોડ ઉદાહરણ:}
\begin{lstlisting}[language=Python]
import math

# સ્થિરાંકો
print("Value of pi:", math.pi)  # 3.141592653589793
print("Value of e:", math.e)    # 2.718281828459045

# મૂળભૂત ગાણિતિક ફંક્શન્સ
print("16 નો વર્ગમૂળ:", math.sqrt(16))  # 4.0
print("5 ની ઘાત 3:", math.pow(5, 3))  # 125.0

# ત્રિકોણમિતિય ફંક્શન્સ (રેડિયન)
print("90° નો સાઇન:", math.sin(math.pi/2))  # 1.0
print("0° નો કોસાઇન:", math.cos(0))  # 1.0

# લોગરિધમિક ફંક્શન્સ
print("100 નો આધાર 10 લોગ:", math.log10(100))  # 2.0
print("e નો નેચરલ લોગ:", math.log(math.e))  # 1.0
\end{lstlisting}
\end{solutionbox}

\begin{mnemonicbox}
\mnemonic{CALM Operations} - Constants And Logarithmic Mathematical Operations
\end{mnemonicbox}

\questionmarks{4(બ)}{4}{વેરીએબલના સ્કોપને સમજાવતો પાયથોન પ્રોગ્રામ લખો.}

\begin{solutionbox}
\begin{lstlisting}[language=Python]
# પાયથોનમાં વેરીએબલ સ્કોપ દર્શાવતો પ્રોગ્રામ

# ગ્લોબલ વેરીએબલ
global_var = "I am global"

def demonstration():
    # લોકલ વેરીએબલ
    local_var = "I am local"
    
    # ગ્લોબલ વેરીએબલ એક્સેસ કરવું
    print("Inside function - Global variable:", global_var)
    
    # લોકલ વેરીએબલ એક્સેસ કરવું
    print("Inside function - Local variable:", local_var)
    
    # ગ્લોબલ નામ ધરાવતું લોકલ વેરીએબલ બનાવવું
    global_var = "I am local with global name"
    print("Inside function - Shadowed global:", global_var)

# ફંક્શન કૉલ
demonstration()

# ગ્લોબલ વેરીએબલ એક્સેસ કરવું
print("Outside function - Global variable:", global_var)

# લોકલ વેરીએબલ એક્સેસ કરવાનો પ્રયાસ ભૂલ ઉત્પન્ન કરશે
# print("Outside function - Local variable:", local_var)  # Error!
\end{lstlisting}
\end{solutionbox}

\begin{mnemonicbox}
\mnemonic{GLOVES} - Global Local Variable Encapsulation System
\end{mnemonicbox}

\questionmarks{4(ક)}{7}{લિસ્ટ પદ્ધતિઓ અને તેના બિલ્ટ-ઇન કાયો સમજાવો}

\begin{solutionbox}
\begin{answertable}{લિસ્ટ પદ્ધતિઓ અને ફંક્શન્સ}
\begin{tabulary}{\linewidth}{|l|L|L|l|}
\hline
\textbf{પદ્ધતિ} & \textbf{વર્ણન} & \textbf{ઉદાહરણ} & \textbf{આઉટપુટ} \\
\hline
\code{append()} & અંતે એલિમેન્ટ ઉમેરે છે & \code{l=['a']; l.append('b')} & \code{['a', 'b']} \\
\hline
\code{insert()} & ચોક્કસ પોઝિશન પર એલિમેન્ટ ઉમેરે & \code{l=[1,3]; l.insert(1,2)} & \code{[1, 2, 3]} \\
\hline
\code{remove()} & ચોક્કસ આઈટમ દૂર કરે & \code{l=['r','b']; l.remove('r')} & \code{['b']} \\
\hline
\code{pop()} & ચોક્કસ ઇન્ડેક્સ પર આઈટમ દૂર કરે & \code{l=['a','b']; l.pop(1)} & \code{'b'} \\
\hline
\code{clear()} & બધા એલિમેન્ટ્સ દૂર કરે & \code{l=[1,2]; l.clear()} & \code{[]} \\
\hline
\code{len()} & એલિમેન્ટ્સની સંખ્યા પાછી આપે & \code{len([1, 2, 3])} & \code{3} \\
\hline
\code{sorted()} & સૉર્ટેડ લિસ્ટ પાછી આપે & \code{sorted([3, 1, 2])} & \code{[1, 2, 3]} \\
\hline
\code{max()} & મહત્તમ મૂલ્ય પાછું આપે & \code{max([5, 10, 3])} & \code{10} \\
\hline
\end{tabulary}
\end{answertable}

\textbf{કોડ ઉદાહરણ:}
\begin{lstlisting}[language=Python]
# લિસ્ટ બનાવવી
my_list = [3, 1, 4, 1, 5]
my_list.append(9)          # અંતે ઉમેરો
my_list.insert(2, 7)       # ઇન્ડેક્સ 2 પર ઉમેરો
my_list.remove(1)          # પ્રથમ 1 દૂર કરો
popped = my_list.pop()     # છેલ્લું એલિમેન્ટ દૂર કરો

print("લંબાઈ:", len(my_list))
print("સૉર્ટેડ:", sorted(my_list))
print("સરવાળો:", sum(my_list))
print("1 ની સંખ્યા:", my_list.count(1))
\end{lstlisting}
\end{solutionbox}

\begin{mnemonicbox}
\mnemonic{LISP Operations} - List Insert Sort Pop Operations
\end{mnemonicbox}

\orquestionmarks{4(અ)}{3}{પાયથોન સ્ટાન્ડર્ડ લાઇબ્રેરી ગાણિતિક કાયોની સૂચિ બનાવો.}

\begin{solutionbox}
\begin{answertable}{ગાણિતિક ફંક્શન્સ}
\begin{tabulary}{\linewidth}{|l|L|L|}
\hline
\textbf{ગાણિતિક ફંક્શન} & \textbf{વર્ણન} & \textbf{ઉદાહરણ} \\
\hline
\code{abs()} & નિરપેક્ષ મૂલ્ય પાછું આપે & \code{abs(-5)} $\to$ \code{5} \\
\hline
\code{round()} & નજીકના પૂર્ણાંક સુધી ગોળ કરે & \code{round(3.7)} $\to$ \code{4} \\
\hline
\code{max()} & સૌથી મોટી આઈટમ પાછી આપે & \code{max(1, 5)} $\to$ \code{5} \\
\hline
\code{min()} & સૌથી નાની આઈટમ પાછી આપે & \code{min(1, 5)} $\to$ \code{1} \\
\hline
\code{sum()} & ઇટરેબલની આઈટમ્સનો સરવાળો કરે & \code{sum([1, 2])} $\to$ \code{3} \\
\hline
\code{pow()} & x ને y ની ઘાત પાછી આપે & \code{pow(2, 3)} $\to$ \code{8} \\
\hline
\end{tabulary}
\end{answertable}

\textbf{math મોડ્યુલમાંથી વધારાના:}
\begin{itemize}
    \item \code{math.sqrt()}: વર્ગમૂળ
    \item \code{math.floor()}: નીચે ગોળ કરે
    \item \code{math.ceil()}: ઉપર ગોળ કરે
    \item \code{math.factorial()}: ફેક્ટોરિયલ
    \item \code{math.gcd()}: મહત્તમ સામાન્ય અવયવ
\end{itemize}
\end{solutionbox}

\begin{mnemonicbox}
\mnemonic{SMART Calculations} - Standard Mathematical Arithmetic Routines and Tools
\end{mnemonicbox}

\orquestionmarks{4(બ)}{4}{પાયથોનમાં બિલ્ટ ઇન ફંક્શન સમજાવો.}

\begin{solutionbox}
\begin{answertable}{બિલ્ટ-ઇન ફંક્શન્સ}
\begin{tabulary}{\linewidth}{|L|}
\hline
\textbf{વર્ણન} \\
\hline
કોઈપણ મોડ્યુલ ઇમ્પોર્ટ કર્યા વિના પાયથોનમાં ઉપલબ્ધ પ્રી-ડિફાઇન્ડ ફંક્શન્સ \\
\hline
કોઈપણ પ્રીફિક્સ વિના સીધા જ કૉલ કરી શકાય છે \\
\hline
સામાન્ય ઓપરેશન્સ કરવા માટે ડિઝાઇન કરેલ છે \\
\hline
ઉદાહરણોમાં \code{print()}, \code{len()}, \code{type()}, \code{input()}, \code{range()} શામેલ છે \\
\hline
\end{tabulary}
\end{answertable}

\textbf{કેટેગરીઓ સાથે ઉદાહરણો:}
\begin{lstlisting}[language=Python]
# ટાઇપ કન્વર્ઝન
print(int("10"))       # 10
print(str(10))         # "10"

# ગાણિતિક ફંક્શન્સ
print(abs(-7))         # 7
print(max(5, 10, 3))   # 10

# કલેક્શન પ્રોસેસિંગ
print(len("hello"))    # 5
print(sorted([3,1,2])) # [1, 2, 3]
\end{lstlisting}
\end{solutionbox}

\begin{mnemonicbox}
\mnemonic{EPIC Functions} - Embedded Python Integrated Core Functions
\end{mnemonicbox}

\orquestionmarks{4(ક)}{7}{વાક્યમાં રહેલ સ્વરો, વ્યંજન, અપરકેસ, લોઅરકેસ અક્ષરોની સંખ્યા ગણવા અને દર્શાવવા માટે પાયથોન પ્રોગ્રામ લખો.}

\begin{solutionbox}
\begin{lstlisting}[language=Python]
# સ્ટ્રિંગમાં સ્વરો, વ્યંજન, અપરકેસ, લોઅરકેસ ગણતરી માટેનો પ્રોગ્રામ

def analyze_string(text):
    # કાઉન્ટર્સ ઇનિશિયલાઇઝ કરો
    vowels = 0
    consonants = 0
    uppercase = 0
    lowercase = 0
    
    # સ્વરો ડિફાઇન કરો
    vowel_set = {'a', 'e', 'i', 'o', 'u'}
    
    # દરેક અક્ષરનું વિશ્લેષણ
    for char in text:
        # તપાસો કે શું અક્ષર છે
        if char.isalpha():
            # કેસ તપાસો
            if char.isupper():
                uppercase += 1
            else:
                lowercase += 1
                
            # તપાસો કે સ્વર છે (કેસ-સેન્સિટિવ)
            if char.lower() in vowel_set:
                vowels += 1
            else:
                consonants += 1
    
    # પરિણામો પાછા આપો
    return vowels, consonants, uppercase, lowercase

# ઇનપુટ લો
text = input("Enter a string: ")

# ગણતરી મેળવો
vowels, consonants, uppercase, lowercase = analyze_string(text)

# પરિણામો દર્શાવો
print("Number of vowels:", vowels)
print("Number of consonants:", consonants)
print("Number of uppercase characters:", uppercase)
print("Number of lowercase characters:", lowercase)
\end{lstlisting}
\end{solutionbox}

\begin{mnemonicbox}
\mnemonic{VOCAL Analysis} - Vowels Or Consonants And Letter case
\end{mnemonicbox}

\questionmarks{5(અ)}{3}{લિસ્ટ મા આપેલ બે એલીમેંટ ને સ્વેપ કરવા માટે પાયથોન કોડ લખો.}

\begin{solutionbox}
\begin{lstlisting}[language=Python]
# લિસ્ટમાં બે એલિમેન્ટ્સ સ્વેપ કરવાનો પ્રોગ્રામ

def swap_elements(lst, pos1, pos2):
    """લિસ્ટમાં બે એલિમેન્ટ્સ સ્વેપ કરવા માટેનું ફંક્શન"""
    lst[pos1], lst[pos2] = lst[pos2], lst[pos1]
    return lst

# ઉદાહરણ ઉપયોગ
my_list = [10, 20, 30, 40, 50]
print("મૂળ લિસ્ટ:", my_list)

# પોઝિશન 1 અને 3 પરના એલિમેન્ટ્સ સ્વેપ કરો
result = swap_elements(my_list, 1, 3)
print("પોઝિશન 1 અને 3 પરના એલિમેન્ટ્સ સ્વેપ કર્યા પછી:", result)

# આઉટપુટ:
# મૂળ લિસ્ટ: [10, 20, 30, 40, 50]
# પોઝિશન 1 અને 3 પરના એલિમેન્ટ્સ સ્વેપ કર્યા પછી: [10, 40, 30, 20, 50]
\end{lstlisting}
\end{solutionbox}

\begin{mnemonicbox}
\mnemonic{STEP Logic} - Swap Two Elements with Python Logic
\end{mnemonicbox}

\questionmarks{5(બ)}{4}{આપેલ સ્ટ્રિંગમાં સબસ્ટ્રિંગ હાજર છે કે કેમ તે તપાસવા માટે પાયથોન પ્રોગ્રામ લખો.}

\begin{solutionbox}
\begin{lstlisting}[language=Python]
# સ્ટ્રિંગમાં સબસ્ટ્રિંગની હાજરી તપાસવાનો પ્રોગ્રામ

def check_substring(main_string, sub_string):
    """સ્ટ્રિંગમાં સબસ્ટ્રિંગની હાજરી તપાસવા માટેનું ફંક્શન"""
    if sub_string in main_string:
        return True
    else:
        return False

# વપરાશકર્તા પાસેથી ઇનપુટ લો
main_string = input("Enter main string: ")
sub_string = input("Enter substring: ")

# તપાસો અને પરિણામ દર્શાવો
if check_substring(main_string, sub_string):
    print(f"'{sub_string}' is present in '{main_string}'")
else:
    print(f"'{sub_string}' is not present in '{main_string}'")
\end{lstlisting}
\end{solutionbox}

\begin{mnemonicbox}
\mnemonic{FIND Method} - Find IN Directly with Methods
\end{mnemonicbox}

\questionmarks{5(ક)}{7}{ટપલ ઓપરેશન, ફંકશન અને મેથડ સમજાવો.}

\begin{solutionbox}
\begin{answertable}{ટપલ ઓપરેશન્સ}
\begin{tabulary}{\linewidth}{|l|L|L|l|}
\hline
\textbf{ઓપરેશન} & \textbf{વર્ણન} & \textbf{ઉદાહરણ} & \textbf{પરિણામ} \\
\hline
બનાવટ & કૌંસ સાથે બનાવવું & \code{t=(1,2)} & \code{(1, 2)} \\
\hline
ઇન્ડેક્સિંગ & એલિમેન્ટ્સ એક્સેસ કરવા & \code{t[1]} & \code{2} \\
\hline
સ્લાઇસિંગ & સબસેટ મેળવવો & \code{t[0:1]} & \code{(1,)} \\
\hline
કેટેનેશન & બે ટપલ જોડવા & \code{(1)+(2)} & \code{(1, 2)} \\
\hline
રિપિટેશન & રિપીટ કરવા & \code{(1)*2} & \code{(1, 1)} \\
\hline
મેમ્બરશિપ & અસ્તિત્વ તપાસવું & \code{1 in t} & \code{True} \\
\hline
\code{len()} & આઇટમ્સની સંખ્યા & \code{len(t)} & \code{2} \\
\hline
\code{count()} & સંખ્યા ગણવી & \code{t.count(1)} & \code{1} \\
\hline
\code{index()} & પોઝિશન શોધવી & \code{t.index(2)} & \code{1} \\
\hline
\end{tabulary}
\end{answertable}

\textbf{કોડ ઉદાહરણ:}
\begin{lstlisting}[language=Python]
my_tuple = (3, 1, 4, 1, 5, 9)
print("પ્રથમ:", my_tuple[0])
print("સ્લાઇસ:", my_tuple[1:4])
print("1 ની સંખ્યા:", my_tuple.count(1))
print("4 નો ઇન્ડેક્સ:", my_tuple.index(4))
a, b, c, *rest = my_tuple # અનપેકિંગ
\end{lstlisting}
\end{solutionbox}

\begin{mnemonicbox}
\mnemonic{ICONS} - Immutable Collection Operations, Numbering, and Searching
\end{mnemonicbox}

\orquestionmarks{5(અ)}{3}{લિસ્ટ મા આપેલ એલીમેંટ નો સરવાળો શોધવા માટે પાયથોન પ્રોગ્રામ લખો.}

\begin{solutionbox}
\begin{lstlisting}[language=Python]
# લિસ્ટના એલિમેન્ટ્સનો સરવાળો શોધવાનો પ્રોગ્રામ

def sum_of_list(numbers):
    """લિસ્ટના બધા એલિમેન્ટ્સનો સરવાળો શોધવા માટેનું ફંક્શન"""
    total = 0
    for num in numbers:
        total += num
    return total

# ઉદાહરણ
num_elements = int(input("Enter number of elements: "))
my_list = []

# એલિમેન્ટ્સ મેળવો
for i in range(num_elements):
    element = float(input(f"Enter element {i+1}: "))
    my_list.append(element)

# સરવાળો ગણો
result1 = sum_of_list(my_list)
print("Sum using custom function:", result1)

# બિલ્ટ-ઇન sum()
result2 = sum(my_list)
print("Sum using built-in function:", result2)
\end{lstlisting}
\end{solutionbox}

\begin{mnemonicbox}
\mnemonic{SALT} - Sum All List Together
\end{mnemonicbox}

\orquestionmarks{5(બ)}{4}{સેટ ફંકશન અને ઓપરેશન દર્શાવવા માટે એક પ્રોગ્રામ લખો.}

\begin{solutionbox}
\begin{lstlisting}[language=Python]
# સેટ ફંક્શન અને ઓપરેશન્સ દર્શાવતો પ્રોગ્રામ

set1 = {1, 2, 3, 4, 5}
set2 = {4, 5, 6, 7, 8}

# સેટ ઓપરેશન્સ
print("યુનિયન:", set1 | set2)
print("ઇન્ટરસેક્શન:", set1 & set2)
print("ડિફરન્સ:", set1 - set2)
print("સિમેટ્રિક ડિફરન્સ:", set1 ^ set2)

# સેટ મેથડ્સ
set3 = set1.copy()
set3.add(6)
set3.remove(1)
set3.discard(10) # ભૂલ નહીં
popped = set3.pop()
set3.clear()
\end{lstlisting}
\end{solutionbox}

\begin{mnemonicbox}
\mnemonic{COSI Methods} - Create, Operate, Search, Investigate with Set Methods
\end{mnemonicbox}

\orquestionmarks{5(ક)}{7}{ડિક્શનેરી ફંક્શન અને ઓપરેશન સમજાવવા માટે પાયથોન પ્રોગામ લખો.}

\begin{solutionbox}
\begin{lstlisting}[language=Python]
# ડિક્શનેરી ફંક્શન અને ઓપરેશન્સ દર્શાવતો પ્રોગ્રામ

# ડિક્શનેરી બનાવવી
student = {
    'name': 'John',
    'roll_no': 101,
    'marks': 85
}

# એક્સેસ
print("Name:", student['name'])
print("Roll No:", student.get('roll_no'))

# સુધારવું અને ઉમેરવું
student['marks'] = 90
student['address'] = 'New York'

# દૂર કરવું
removed = student.pop('address')
last_item = student.popitem()

# મેથડ્સ
print("Keys:", list(student.keys()))
print("Values:", list(student.values()))
print("Items:", list(student.items()))

# ક્લિયર
student.clear()
\end{lstlisting}

\textbf{મુખ્ય ઓપરેશન્સ:}
\begin{itemize}
    \item \keyword{એક્સેસ}: કી અથવા get() મેથડનો ઉપયોગ કરીને
    \item \keyword{મોડિફાય}: અસ્તિત્વમાં રહેલી કીને નવું મૂલ્ય આપવું
    \item \keyword{એડ}: નવી કીને મૂલ્ય આપવું
    \item \keyword{રિમૂવ}: pop(), popitem(), અથવા del
\end{itemize}
\end{solutionbox}

\begin{mnemonicbox}
\mnemonic{ACME Dictionary} - Access, Create, Modify, Extract from Dictionary
\end{mnemonicbox}

\end{document}
