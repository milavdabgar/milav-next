\documentclass{article}

% content/resources/templates/preamble.tex
\usepackage[margin=0.6in]{geometry}
\author{Milav Dabgar}
\usepackage{amsmath,amssymb,amsthm}
\usepackage{booktabs}
\usepackage{multirow}
\usepackage{xcolor}
\usepackage{tcolorbox}
\tcbuselibrary{breakable,skins}
\usepackage[colorlinks=true,linkcolor=blue]{hyperref}
\usepackage{titlesec}
\usepackage{enumitem}
\usepackage{tikz}
\usepackage{pgfplots}
\usepackage{circuitikz}
\usepackage[version=4]{mhchem}
\usepackage{longtable}
\usepackage{array}
\usepackage{float}
\usepackage{caption}
\usepackage{listings}

\lstset{
  basicstyle=\small\ttfamily,
  breaklines=true,
  breakatwhitespace=false,
  postbreak=\mbox{\textcolor{red}{$\hookrightarrow$}\space},
  float=false,
  numbers=left,
  numberstyle=\tiny\color{gray},
  numbersep=10pt,
  xleftmargin=2em,
  keywordstyle=\color{blue},
  commentstyle=\color{green!60!black},
  stringstyle=\color{purple},
  backgroundcolor=\color{gray!5},
  showstringspaces=false,
  tabsize=2,
  captionpos=b,
  keepspaces=true,
  columns=flexible
}

\pgfplotsset{compat=1.18}
\usetikzlibrary{shapes,arrows,positioning,calc,patterns,decorations.pathmorphing,decorations.markings,arrows.meta}

% Color scheme
\definecolor{headcolor}{RGB}{0,102,204}
\definecolor{keycolor}{RGB}{220,20,60}
\definecolor{solutioncolor}{RGB}{34,139,34}
\definecolor{mnemoniccolor}{RGB}{148,0,211}
\definecolor{codecolor}{RGB}{0,0,100}

% Spacing
\setlength{\parskip}{3pt}
\setlist[itemize]{nosep}
\setlist[enumerate]{nosep}

% Title formatting
\titleformat{\section}{\Large\bfseries\color{headcolor}}{\thesection}{1em}{}
\titleformat{\subsection}{\large\bfseries\color{headcolor}}{\thesubsection}{1em}{}

% Pandoc tightlist compatibility
\providecommand{\tightlist}{%
  \setlength{\itemsep}{0pt}\setlength{\parskip}{0pt}}

% Pandoc longtable compatibility
\newcounter{none}
\def\thenone{}


% content/resources/templates/gujarati-boxes.tex
\usepackage{fontspec}
\usepackage{polyglossia}

% Set Gujarati as main language (document is primarily in Gujarati)
% Note: gloss-gujarati.ldf doesn't exist in polyglossia, but it will use hyphenation patterns
\setdefaultlanguage{gujarati}
\setotherlanguage{english}

% Configure Gujarati font properly
% Use Language=Default to prevent polyglossia from trying to add language-specific features
% that don't exist for Gujarati, which causes "empty feature" warnings
\newfontfamily\gujaratifont[Script=Gujarati,AutoFakeBold=2.5,AutoFakeSlant=0.3]{Noto Sans Gujarati}
\setmainfont[Script=Gujarati,AutoFakeBold=2.5,AutoFakeSlant=0.3]{Noto Sans Gujarati}
% Use Noto Sans Gujarati for monospace to support Gujarati in text
\setmonofont[Scale=0.9]{Noto Sans Gujarati}

% Configure English to use the same font
\newfontfamily\englishfont[Script=Gujarati,AutoFakeBold=2.5,AutoFakeSlant=0.3]{Noto Sans Gujarati}

% Translations for polyglossia
\gappto\captionsgujarati{
  \renewcommand{\tablename}{કોષ્ટક}
  \renewcommand{\figurename}{આકૃતિ}
}

% Helper for TikZ nodes to ensure Gujarati font
\newcommand{\gu}[1]{{\gujaratifont #1}}

% Custom environments
\newtcolorbox{solutionbox}{
    breakable,
    enhanced,
    colback=solutioncolor!5!white,
    colframe=solutioncolor!75!black,
    fonttitle=\bfseries,
    title=જવાબ
}

\newtcolorbox{solutionboxnobreak}{
 colback=solutioncolor!5!white,
 colframe=solutioncolor!75!black,
 fonttitle=\bfseries,
 title=જવાબ
}

\newtcolorbox{keyformula}{
 breakable,
 enhanced,
 colback=keycolor!5!white,
 colframe=keycolor!75!black,
 fonttitle=\bfseries,
 title=રાસાયણિક સમીકરણ/સૂત્ર
}

\newtcolorbox{mnemonicbox}{
 breakable,
 enhanced,
 colback=mnemoniccolor!5!white,
 colframe=mnemoniccolor!75!black,
 fonttitle=\bfseries,
 title=મેમરી ટ્રીક
}


% Custom commands for GTU solutions
% This file defines semantic commands for consistent formatting

% Question command with automatic formatting
\newcommand{\question}[2]{%
  \section*{Question #1}%
  \textbf{#2}%
}

% OR question variant
\newcommand{\questionor}[2]{%
  \section*{Question #1 OR}%
  \textbf{#2}%
}

% Proper table environment with caption
\newenvironment{answertable}[1]{%
  \begin{table}[htbp]
  \centering
  \caption{#1}
}{%
  \end{table}
}

% Proper figure environment for diagrams
\newenvironment{answerdiagram}[1]{%
  \begin{figure}[htbp]
  \centering
  \caption{#1}
}{%
  \end{figure}
}

% Semantic markup for key terms
\newcommand{\keyword}[1]{\textbf{#1}}
\newcommand{\code}[1]{\texttt{#1}}
\newcommand{\classname}[1]{\texttt{#1}}
\newcommand{\methodname}[1]{\texttt{#1}}

% Proper quotation marks
\newcommand{\mnemonic}[1]{``#1''}


\title{પાયથોન પ્રોગ્રામિંગ (1323203) - સમર 2023 સોલ્યુશન}
\date{૦૯ ઓગસ્ટ, ૨૦૨૩}

\begin{document}
\maketitle

\questionmarks{1(અ)}{3}{અલગોરીધમ વ્યાખ્યાયિત કરો. અલગોરીધમનાં ફાયદા શું છે?}

\begin{solutionbox}
અલગોરીધમ એ ચોક્કસ સમસ્યાને ઉકેલવા માટે પગલાઓના ક્રમબદ્ધ સમૂહ અથવા નિયમોનો સેટ છે.

\textbf{અલગોરીધમના ફાયદા:}
\begin{itemize}
    \item \keyword{સ્પષ્ટતા} (Clarity): સ્પષ્ટ, અસંદિગ્ધ સૂચનાઓ પ્રદાન કરે છે
    \item \keyword{કાર્યક્ષમતા} (Efficiency): સમય અને સંસાધનોને અનુકૂળ બનાવવામાં મદદ કરે છે
    \item \keyword{પુન:ઉપયોગ} (Reusability): સમાન સમસ્યાઓ માટે વારંવાર ઉપયોગ કરી શકાય છે
    \item \keyword{ચકાસણી} (Verification): અમલીકરણ પહેલાં પરીક્ષણ અને ડિબગ કરવું સરળ
    \item \keyword{સંદેશાવ્યવહાર} (Communication): ઉકેલને સંદેશાવ્યવહાર કરવા માટે બ્લુપ્રિન્ટ તરીકે કામ કરે છે
\end{itemize}

\begin{mnemonicbox}
"CERVC" (Clarity, Efficiency, Reusability, Verification, Communication)
\end{mnemonicbox}
\end{solutionbox}

\questionmarks{1(બ)}{4}{ફલોચાર્ટનો ઉપયોગ કરીને સમસ્યા ઉકેલવાના નિયમો શું છે? સાદું વ્યાજ શોધવા માટેનો ફલોચાર્ટ ડીઝાઈન કરો.}

\begin{solutionbox}
ફલોચાર્ટનો ઉપયોગ કરીને સમસ્યા ઉકેલવાના નિયમો:
\begin{itemize}
    \item \textbf{યોગ્ય સિમ્બોલ}: વિવિધ ઓપરેશન માટે માનક સિમ્બોલનો ઉપયોગ કરવો
    \item \textbf{દિશાનો પ્રવાહ}: હંમેશા ઉપરથી નીચે, ડાબેથી જમણે સ્પષ્ટ પ્રવાહ જાળવવો
    \item \textbf{એક એન્ટ્રી/એક્ઝિટ}: સ્પષ્ટ શરૂઆત અને અંત બિંદુ હોવા જોઈએ
    \item \textbf{સ્પષ્ટતા}: પગલાં સ્પષ્ટ અને સંક્ષિપ્ત રાખવા
    \item \textbf{સુસંગતતા}: વિગતોનું સુસંગત સ્તર જાળવવું
\end{itemize}

\begin{answerdiagram}{સાદું વ્યાજ ગણતરી માટેનો ફલોચાર્ટ}
\begin{center}
\begin{tikzpicture}[gtu flow]
    \node[gtu start] (start) {શરૂઆત};
    \node[gtu input, below of=start] (input) {મૂળ રકમ P, વ્યાજ દર R, સમય T ઇનપુટ કરો};
    \node[gtu process, below of=input] (calc) {SI = P * R * T / 100};
    \node[gtu output, below of=calc] (output) {SI આઉટપુટ કરો};
    \node[gtu stop, below of=output] (stop) {અંત};

    \draw[gtu arrow] (start) -- (input);
    \draw[gtu arrow] (input) -- (calc);
    \draw[gtu arrow] (calc) -- (output);
    \draw[gtu arrow] (output) -- (stop);
\end{tikzpicture}
\end{center}
\end{answerdiagram}

\begin{mnemonicbox}
"PDRSC" (Proper symbols, Direction flow, Required entry/exit, Simplicity, Consistency)
\end{mnemonicbox}
\end{solutionbox}

\questionmarks{1(ક)}{7}{પાયથોનનાં અસાઇમેંટ ઓપરટેરની યાદી બનાવો અને કોઈપણ ત્રણ અસાઇમેંટ ઓપરટેરોની કામગીરી દશાર્વવા માટે પાયથોન કોડ બનાવો.}

\begin{solutionbox}
પાયથોન અસાઇમેંટ ઓપરેટર્સ:

\begin{answertable}{અસાઇમેંટ ઓપરેટર્સ}
\begin{tabulary}{\linewidth}{|L|L|L|}
\hline
\textbf{ઓપરેટર} & \textbf{ઉદાહરણ} & \textbf{સમકક્ષ} \\
\hline
= & x = 5 & x = 5 \\
+= & x += 5 & x = x + 5 \\
-= & x -= 5 & x = x - 5 \\
*= & x *= 5 & x = x * 5 \\
/= & x /= 5 & x = x / 5 \\
\%= & x \%= 5 & x = x \% 5 \\
//= & x //= 5 & x = x // 5 \\
**= & x **= 5 & x = x ** 5 \\
\&= & x \&= 5 & x = x \& 5 \\
|= & x |= 5 & x = x | 5 \\
\^{}= & x \^{}= 5 & x = x \^{} 5 \\
>>= & x >>= 5 & x = x >> 5 \\
<<= & x <<= 5 & x = x << 5 \\
\hline
\end{tabulary}
\end{answertable}

\textbf{અસાઇમેંટ ઓપરેટર્સ દર્શાવતો કોડ:}
\begin{lstlisting}[language=Python]
# અસાઇમેંટ ઓપરેટર્સનું પ્રદર્શન
num = 10
print("પ્રારંભિક મૂલ્ય:", num)

# += ઓપરેટરનો ઉપયોગ
num += 5
print("+= 5 પછી:", num)  # આઉટપુટ: 15

# -= ઓપરેટરનો ઉપયોગ
num -= 3
print("-= 3 પછી:", num)  # આઉટપુટ: 12

# *= ઓપરેટરનો ઉપયોગ
num *= 2
print("*= 2 પછી:", num)  # આઉટપુટ: 24
\end{lstlisting}

\begin{mnemonicbox}
"VALUE" (Variable Assignment is Like Updating Existing values)
\end{mnemonicbox}
\end{solutionbox}

\orquestionmarks{1(ક)}{7}{પાયથોનનાં ડેટા ટાઇપ્સની યાદી બનાવો અને કોઈપણ ત્રણ ડેટા ટાઇપ્સને ઓળખવા માટેનો પાયથોન કોડ બનાવો.}

\begin{solutionbox}
પાયથોન ડેટા ટાઇપ્સ:

\begin{answertable}{પાયથોન ડેટા ટાઇપ્સ}
\begin{tabulary}{\linewidth}{|L|L|L|}
\hline
\textbf{ડેટા ટાઇપ} & \textbf{વર્ણન} & \textbf{ઉદાહરણ} \\
\hline
int & ઇન્ટીજર (પૂર્ણાંક સંખ્યાઓ) & 42 \\
float & ફ્લોટિંગ પોઇન્ટ (દશાંશ) & 3.14 \\
str & સ્ટ્રિંગ (ટેક્સ્ટ) & "Hello" \\
bool & બૂલિયન (True/False) & True \\
list & ક્રમિક, પરિવર્તનશીલ સંગ્રહ & [1, 2, 3] \\
tuple & ક્રમિક, અપરિવર્તનીય સંગ્રહ & (1, 2, 3) \\
set & અક્રમિક સંગ્રહ & \{1, 2, 3\} \\
dict & કી-વેલ્યુ જોડી & \{"name": "John"\} \\
complex & કોમ્પ્લેક્સ નંબર & 2+3j \\
NoneType & None દર્શાવે છે & None \\
\hline
\end{tabulary}
\end{answertable}

\textbf{ત્રણ ડેટા ટાઇપ્સ ઓળખવા માટેનો કોડ:}
\begin{lstlisting}[language=Python]
# ડેટા ટાઇપ્સ ઓળખવાનો પ્રોગ્રામ
def identify_data_type(value):
    data_type = type(value).__name__
    print(f"મૂલ્ય: {value}")
    print(f"ડેટા ટાઇપ: {data_type}")
    print("-" * 20)

# 3 અલગ-અલગ ડેટા ટાઇપ્સ સાથે ટેસ્ટિંગ
identify_data_type(42)            # Integer
identify_data_type(3.14)          # Float
identify_data_type("Hello World") # String

# આઉટપુટ:
# મૂલ્ય: 42
# ડેટા ટાઇપ: int
# --------------------
# મૂલ્ય: 3.14
# ડેટા ટાઇપ: float
# --------------------
# મૂલ્ય: Hello World
# ડેટા ટાઇપ: str
# --------------------
\end{lstlisting}

\begin{mnemonicbox}
"TYPE-ID" (Tell Your Python Elements - Identify Data)
\end{mnemonicbox}
\end{solutionbox}

\questionmarks{2(અ)}{3}{સ્યુડોકોડ વ્યાખ્યાયિત કરો. કોઈપણ બે સંખ્યા માંથી સૌથી નાની સંખ્યા શોધવા માટે સ્યુડોકોડ લખો.}

\begin{solutionbox}
સ્યુડોકોડ એ એલ્ગોરિધમનું ઉચ્ચ-સ્તરીય વર્ણન છે જે પ્રોગ્રામિંગ ભાષાના માળખાકીય સંકેતોનો ઉપયોગ કરે છે પરંતુ મશીન વાંચન કરતાં માનવ વાંચન માટે ડિઝાઇન કરેલ છે.

\textbf{બે સંખ્યાઓમાંથી સૌથી નાની શોધવા માટે સ્યુડોકોડ:}
\begin{lstlisting}
BEGIN
    INPUT first_number, second_number
    IF first_number < second_number THEN
        smallest = first_number
    ELSE
        smallest = second_number
    END IF
    OUTPUT smallest
END
\end{lstlisting}

\begin{mnemonicbox}
"RISE" (Read Input, Select smallest, Echo result)
\end{mnemonicbox}
\end{solutionbox}

\questionmarks{2(બ)}{4}{યુઝર્સ પાસેથી ત્રણ ઇનપુટ વાંચો અને સંખ્યાઓની સરેરાશ શોધવા માટેનો પાયથોન કોડ વિકસાવો.}

\begin{solutionbox}
\begin{lstlisting}[language=Python]
# ત્રણ સંખ્યાઓની સરેરાશ ગણવા માટેનો પ્રોગ્રામ
num1 = float(input("પ્રથમ સંખ્યા દાખલ કરો: "))
num2 = float(input("બીજી સંખ્યા દાખલ કરો: "))
num3 = float(input("ત્રીજી સંખ્યા દાખલ કરો: "))

# સરેરાશની ગણતરી
average = (num1 + num2 + num3) / 3

# પરિણામ દર્શાવો
print(f"{num1}, {num2}, અને {num3}ની સરેરાશ: {average}")
\end{lstlisting}

\begin{answerdiagram}{સરેરાશ ગણતરી માટેનો ફલોચાર્ટ}
\begin{center}
\begin{tikzpicture}[gtu flow]
    \node[gtu start] (start) {શરૂઆત};
    \node[gtu input, below of=start] (input) {num1, num2, num3 ઇનપુટ કરો};
    \node[gtu process, below of=input] (process) {average = (num1 + num2 + num3) / 3};
    \node[gtu output, below of=process] (output) {average આઉટપુટ કરો};
    \node[gtu stop, below of=output] (stop) {અંત};

    \draw[gtu arrow] (start) -- (input);
    \draw[gtu arrow] (input) -- (process);
    \draw[gtu arrow] (process) -- (output);
    \draw[gtu arrow] (output) -- (stop);
\end{tikzpicture}
\end{center}
\end{answerdiagram}

\begin{mnemonicbox}
"I-ADD-D" (Input three, ADD them up, Divide by 3)
\end{mnemonicbox}
\end{solutionbox}

\questionmarks{2(ક)}{7}{દાખલ કરેલ સંખ્યા prime છે કે નહીં તે બતાવવા પાયથોન કોડ લખો.}

\begin{solutionbox}
\begin{lstlisting}[language=Python]
# સંખ્યા પ્રાઇમ છે કે નહીં તે તપાસવાનો પ્રોગ્રામ
num = int(input("એક સંખ્યા દાખલ કરો: "))

# 2થી ઓછી સંખ્યા છે કે નહીં તપાસો
if num < 2:
    print(f"{num} એક પ્રાઇમ સંખ્યા નથી")
else:
    # is_prime ને True તરીકે આરંભો
    is_prime = True
    
    # 2 થી sqrt(num) સુધી તપાસો
    for i in range(2, int(num**0.5) + 1):
        if num % i == 0:
            is_prime = False
            break
    
    # પરિણામ દર્શાવો
    if is_prime:
        print(f"{num} એક પ્રાઇમ સંખ્યા છે")
    else:
        print(f"{num} એક પ્રાઇમ સંખ્યા નથી")
\end{lstlisting}

\begin{answerdiagram}{પ્રાઇમ ચેક માટે ફલોચાર્ટ}
\begin{center}
\begin{tikzpicture}[gtu flow]
    \node[gtu start] (start) {શરૂઆત};
    \node[gtu input, below of=start] (input) {num ઇનપુટ કરો};
    \node[gtu decision, below of=input] (less2) {num < 2?};
    \node[gtu output, right=3cm of less2] (notprime) {num પ્રાઇમ નથી};
    \node[gtu process, below of=less2] (init) {is\_prime = True};
    \node[gtu process, below of=init] (loopinit) {i = 2};
    \node[gtu decision, below of=loopinit] (loopcond) {i * i <= num?};
    \node[gtu decision, right=3cm of loopcond] (isdiv) {num \% i == 0?};
    \node[gtu decision, below of=loopcond] (checkprime) {is\_prime?};
    \node[gtu output, right of=checkprime, xshift=2cm] (isprime) {num પ્રાઇમ છે};
    \node[gtu process, below of=isdiv] (setfalse) {is\_prime = False};
    \node[gtu process, left of=loopcond, xshift=-2cm] (increment) {i = i + 1};
    \node[gtu stop, below of=checkprime] (stop) {અંત};

    \draw[gtu arrow] (start) -- (input);
    \draw[gtu arrow] (input) -- (less2);
    \draw[gtu arrow] (less2) -- node[above] {હા} (notprime);
    \draw[gtu arrow] (less2) -- node[right] {ના} (init);
    \draw[gtu arrow] (init) -- (loopinit);
    \draw[gtu arrow] (loopinit) -- (loopcond);
    \draw[gtu arrow] (loopcond) -- node[above] {હા} (isdiv);
    \draw[gtu arrow] (loopcond) -- node[right] {ના} (checkprime);
    \draw[gtu arrow] (isdiv) -- node[right] {હા} (setfalse);
    \draw[gtu arrow] (isdiv) -- node[right] {ના} (increment);
    \draw[gtu arrow] (setfalse) -- (checkprime);
    \draw[gtu arrow] (increment) |- (loopcond);
    \draw[gtu arrow] (checkprime) -- node[above] {હા} (isprime);
    \draw[gtu arrow] (checkprime) -- node[right] {ના} (stop -| notprime) -- (notprime);
    \draw[gtu arrow] (isprime) |- (stop);
    \draw[gtu arrow] (notprime) |- (stop);
\end{tikzpicture}
\end{center}
\end{answerdiagram}

\begin{mnemonicbox}
"PRIME" (Positive number, Range check from 2 to $\sqrt{n}$, If divisible it's Multiple, Else it's prime)
\end{mnemonicbox}
\end{solutionbox}

\orquestionmarks{2(અ)}{3}{ફલોચાર્ટ અને એલ્ગોરિધમ વચ્ચેનો તફાવત લખો.}

\begin{solutionbox}
\begin{answertable}{ફલોચાર્ટ અને એલ્ગોરિધમ વચ્ચેનો તફાવત}
\begin{tabulary}{\linewidth}{|L|L|}
\hline
\textbf{ફલોચાર્ટ} & \textbf{એલ્ગોરિધમ} \\
\hline
માનક સિમ્બોલ અને આકારોનો ઉપયોગ કરીને \textbf{દૃશ્ય પ્રતિનિધિત્વ} & \textbf{લેખિત વર્ણન} માળખાકીય ભાષાનો ઉપયોગ કરીને \\
\hline
ગ્રાફિકલ પ્રકૃતિને કારણે \textbf{સમજવું સરળ} & સિન્ટેક્સ અને શબ્દાવલીનું જ્ઞાન જરૂરી \\
\hline
\textbf{તાર્કિક પ્રવાહ} અને સંબંધોને સ્પષ્ટ રીતે દર્શાવે & ક્રમિક ક્રમમાં \textbf{વિગતવાર પગલાં} પ્રદાન કરે \\
\hline
બનાવવા માટે \textbf{સમય-લેતી} પરંતુ સમજવા માટે સરળ & \textbf{ઝડપથી ડ્રાફ્ટ} પરંતુ સમજવામાં મુશ્કેલ હોઈ શકે \\
\hline
ફેરફાર કરવા કે અપડેટ કરવા વધુ મુશ્કેલ & ફેરફાર કરવા કે અપડેટ કરવા વધુ સરળ \\
\hline
\end{tabulary}
\end{answertable}

\begin{mnemonicbox}
"VITAL" (Visual vs Textual, Interpretation ease, Time to create, Alteration flexibility, Logical representation)
\end{mnemonicbox}
\end{solutionbox}

\orquestionmarks{2(બ)}{4}{નીચેનાં કોડનું આઉટપુટ શું છે?}

\begin{solutionbox}
\begin{lstlisting}[language=Python]
x=10
y=2
print (x*y)
print (x ** y)
print (x//y)
print (x % y)
\end{lstlisting}

\textbf{જવાબ:}

\begin{answertable}{આઉટપુટ સમજૂતી}
\begin{tabulary}{\linewidth}{|L|L|L|}
\hline
\textbf{ઓપરેશન} & \textbf{સમજૂતી} & \textbf{આઉટપુટ} \\
\hline
x*y & ગુણાકાર: 10 $\times$ 2 & 20 \\
\hline
x**y & ઘાતાંક: $10^2$ & 100 \\
\hline
x//y & પૂર્ણાંક ભાગાકાર: 10 $\div$ 2 & 5 \\
\hline
x\%y & મોડ્યુલસ (શેષ): 10 $\div$ 2 & 0 \\
\hline
\end{tabulary}
\end{answertable}

\begin{mnemonicbox}
"MEMO" (Multiply, Exponent, Modulo, Operations)
\end{mnemonicbox}
\end{solutionbox}

\orquestionmarks{2(ક)}{7}{નીચેની પેટર્ન દર્શાવવા પાયથોન કોડ લખો:}

\begin{solutionbox}
\begin{lstlisting}
A)                    B)
1                    * * * *
1 2                  * * *
1 2 3                * *
1 2 3 4              *
\end{lstlisting}

\begin{lstlisting}[language=Python]
# પેટર્ન A: સંખ્યા પેટર્ન
print("પેટર્ન A:")
for i in range(1, 5):
    for j in range(1, i + 1):
        print(j, end=" ")
    print()

# પેટર્ન B: તારા પેટર્ન
print("\nપેટર્ન B:")
for i in range(4, 0, -1):
    for j in range(i):
        print("*", end=" ")
    print()
\end{lstlisting}

\begin{answerdiagram}{પેટર્ન માટે ફલોચાર્ટ}
\begin{center}
\begin{tikzpicture}[gtu flow]
    \node[gtu start] (start) {શરૂઆત};
    \node[gtu process, below of=start] (pattA) {પેટર્ન A લોજિક};
    \node[gtu process, below of=pattA] (loopA) {i = 1 થી 4};
    \node[gtu process, below of=loopA] (innerA) {j = 1 થી i};
    \node[gtu output, below of=innerA] (printA) {j પ્રિન્ટ કરો};
    \node[gtu process, below of=printA] (newlineA) {નવી લાઇન};
    
    \node[gtu process, right=4cm of pattA] (pattB) {પેટર્ન B લોજિક};
    \node[gtu process, below of=pattB] (loopB) {i = 4 થી 1};
    \node[gtu process, below of=loopB] (innerB) {j = 0 થી i-1};
    \node[gtu output, below of=innerB] (printB) {* પ્રિન્ટ કરો};
    \node[gtu process, below of=printB] (newlineB) {નવી લાઇન};
    \node[gtu stop, below of=newlineB] (stop) {અંત};

    \draw[gtu arrow] (start) -- (pattA);
    \draw[gtu arrow] (pattA) -- (loopA);
    \draw[gtu arrow] (loopA) -- (innerA);
    \draw[gtu arrow] (innerA) -- (printA);
    \draw[gtu arrow] (printA) -- (newlineA);
    \draw[gtu arrow] (newlineA) -| (pattB);
    
    \draw[gtu arrow] (pattB) -- (loopB);
    \draw[gtu arrow] (loopB) -- (innerB);
    \draw[gtu arrow] (innerB) -- (printB);
    \draw[gtu arrow] (printB) -- (newlineB);
    \draw[gtu arrow] (newlineB) -- (stop);
\end{tikzpicture}
\end{center}
\end{answerdiagram}

\begin{mnemonicbox}
"LOOP-NED" (Loop Outer, Order Pattern, Nested loops, End with newline, Display)
\end{mnemonicbox}
\end{solutionbox}

\questionmarks{3(અ)}{3}{જૂરી ઉદાહરણો સાથે break statementનાં ઉપયોગનું વર્ણન કરો.}

\begin{solutionbox}
break સ્ટેટમેન્ટનો ઉપયોગ લૂપને વચ્ચેથી સમાપ્ત કરવા માટે થાય છે, જ્યારે કોઈ ચોક્કસ શરત પૂરી થાય.

\textbf{ઉદાહરણ:}
\begin{lstlisting}[language=Python]
# લિસ્ટમાં પ્રથમ વિષમ સંખ્યા શોધવી
numbers = [2, 4, 6, 7, 8, 10]
for num in numbers:
    if num % 2 != 0:
        print(f"વિષમ સંખ્યા મળી: {num}")
        break
    print(f"{num} તપાસી રહ્યા છીએ")
\end{lstlisting}

\textbf{આઉટપુટ:}
\begin{lstlisting}
2 તપાસી રહ્યા છીએ
4 તપાસી રહ્યા છીએ
6 તપાસી રહ્યા છીએ
વિષમ સંખ્યા મળી: 7
\end{lstlisting}

\begin{mnemonicbox}
"EXIT" (EXecute until condition, Immediately Terminate)
\end{mnemonicbox}
\end{solutionbox}

\questionmarks{3(બ)}{4}{યોગ્ય ઉદાહરણ સાથે if...else statement સમજાવો.}

\begin{solutionbox}
if...else સ્ટેટમેન્ટ એ એક કન્ડિશનલ સ્ટેટમેન્ટ છે જે નિર્દિષ્ટ શરત True કે False હોવાના આધારે અલગ-અલગ કોડ બ્લોક્સ એક્ઝિક્યુટ કરે છે.

\textbf{સિન્ટેક્સ:}
\begin{lstlisting}[language=Python]
if શરત:
    # જો શરત True હોય તો આ કોડ એક્ઝિક્યુટ થશે
else:
    # જો શરત False હોય તો આ કોડ એક્ઝિક્યુટ થશે
\end{lstlisting}

\textbf{ઉદાહરણ:}
\begin{lstlisting}[language=Python]
# સંખ્યા સમ છે કે વિષમ તે તપાસવું
number = int(input("એક સંખ્યા દાખલ કરો: "))

if number % 2 == 0:
    print(f"{number} એક સમ સંખ્યા છે")
else:
    print(f"{number} એક વિષમ સંખ્યા છે")
\end{lstlisting}

\begin{answerdiagram}{If-Else માટે ફલોચાર્ટ}
\begin{center}
\begin{tikzpicture}[gtu flow]
    \node[gtu start] (start) {શરૂઆત};
    \node[gtu input, below of=start] (input) {સંખ્યા ઇનપુટ કરો};
    \node[gtu decision, below of=input] (cond) {number \% 2 == 0?};
    \node[gtu output, below left of=cond, xshift=-2cm] (even) {"સમ" પ્રિન્ટ કરો};
    \node[gtu output, below right of=cond, xshift=2cm] (odd) {"વિષમ" પ્રિન્ટ કરો};
    \node[gtu stop, below of=cond, yshift=-4cm] (stop) {અંત};

    \draw[gtu arrow] (start) -- (input);
    \draw[gtu arrow] (input) -- (cond);
    \draw[gtu arrow] (cond) -| node[above] {હા} (even);
    \draw[gtu arrow] (cond) -| node[above] {ના} (odd);
    \draw[gtu arrow] (even) |- (stop);
    \draw[gtu arrow] (odd) |- (stop);
\end{tikzpicture}
\end{center}
\end{answerdiagram}

\begin{mnemonicbox}
"CITE" (Check condition, If True Execute this, Else execute that)
\end{mnemonicbox}
\end{solutionbox}

\questionmarks{3(ક)}{7}{0 થી N સંખ્યા સુધીની ફીબોનાકી શ્રેણી પ્રિન્ટ કરવા માટે યુઝર ડેફાઇન ફંકશન બનાવો.}

\begin{solutionbox}
\begin{lstlisting}[language=Python]
# ફીબોનાકી શ્રેણી પ્રિન્ટ કરવા માટેનું ફંકશન
def print_fibonacci(n):
    # પ્રથમ બે પદો ઇનિશિયલાઇઝ કરો
    a, b = 0, 1
    
    # n માન્ય છે કે નહીં તે તપાસો
    if n < 0:
        print("કૃપા કરીને એક હકારાત્મક સંખ્યા દાખલ કરો")
        return
    
    # ફીબોનાકી શ્રેણી પ્રિન્ટ કરો
    print(n, "સુધીની ફીબોનાકી શ્રેણી:")
    
    if n >= 0:
        print(a, end=" ")  # પ્રથમ પદ પ્રિન્ટ કરો
    
    if n >= 1:
        print(b, end=" ")  # બીજો પદ પ્રિન્ટ કરો
    
    # બાકીની શ્રેણી બનાવો અને પ્રિન્ટ કરો
    while a + b <= n:
        next_term = a + b
        print(next_term, end=" ")
        a, b = b, next_term

# ફંકશનનું ટેસ્ટિંગ
print_fibonacci(55)
\end{lstlisting}

\begin{answerdiagram}{ફીબોનાકી શ્રેણી માટે ફલોચાર્ટ}
\begin{center}
\begin{tikzpicture}[gtu flow]
    \node[gtu start] (start) {શરૂઆત};
    \node[gtu process, below of=start] (init) {a=0, b=1 ઇનિશિયલાઇઝ કરો};
    \node[gtu decision, below of=init] (checkneg) {n < 0?};
    \node[gtu output, right=3cm of checkneg] (error) {એરર મેસેજ પ્રિન્ટ કરો};
    \node[gtu decision, below of=checkneg] (check0) {n >= 0?};
    \node[gtu output, left=3cm of check0] (printa) {a પ્રિન્ટ કરો};
    \node[gtu decision, below of=check0] (check1) {n >= 1?};
    \node[gtu output, left=3cm of check1] (printb) {b પ્રિન્ટ કરો};
    \node[gtu decision, below of=check1] (checkloop) {a + b <= n?};
    \node[gtu process, right=3cm of checkloop] (update) {next = a + b\\next પ્રિન્ટ કરો\\a=b, b=next};
    \node[gtu stop, below of=checkloop] (stop) {અંત};

    \draw[gtu arrow] (start) -- (init);
    \draw[gtu arrow] (init) -- (checkneg);
    \draw[gtu arrow] (checkneg) -- node[above] {હા} (error);
    \draw[gtu arrow] (checkneg) -- node[right] {ના} (check0);
    \draw[gtu arrow] (error) |- (stop);
    
    \draw[gtu arrow] (check0) -- node[above] {હા} (printa);
    \draw[gtu arrow] (check0) -- node[right] {ના} (check1);
    \draw[gtu arrow] (printa) -- (check1);
    
    \draw[gtu arrow] (check1) -- node[above] {હા} (printb);
    \draw[gtu arrow] (check1) -- node[right] {ના} (checkloop);
    \draw[gtu arrow] (printb) -- (checkloop);
    
    \draw[gtu arrow] (checkloop) -- node[above] {હા} (update);
    \draw[gtu arrow] (checkloop) -- node[right] {ના} (stop);
    \draw[gtu arrow] (update) |- (checkloop);
\end{tikzpicture}
\end{center}
\end{answerdiagram}

\begin{mnemonicbox}
"FIBER" (First terms set, Initialize variables, Build next term, Echo results, Repeat until limit)
\end{mnemonicbox}
\end{solutionbox}

\orquestionmarks{3(અ)}{3}{જરૂરી ઉદાહરણો સાથે continue statementનાં ઉપયોગનું વર્ણન કરો.}

\begin{solutionbox}
continue સ્ટેટમેન્ટનો ઉપયોગ લૂપની વર્તમાન ઇટરેશન છોડીને આગળની ઇટરેશન પર જવા માટે થાય છે.

\textbf{ઉદાહરણ:}
\begin{lstlisting}[language=Python]
# 1 થી 10 સુધીની માત્ર વિષમ સંખ્યાઓ પ્રિન્ટ કરવી
for i in range(1, 11):
    if i % 2 == 0:
        continue  # સમ સંખ્યાઓ છોડી દો
    print(i)
\end{lstlisting}

\textbf{આઉટપુટ:}
\begin{lstlisting}
1
3
5
7
9
\end{lstlisting}

\begin{mnemonicbox}
"SKIP" (Skip current iteration, Keep looping, Ignore remaining statements, Proceed to next iteration)
\end{mnemonicbox}
\end{solutionbox}

\orquestionmarks{3(બ)}{4}{ઉદાહરણ સાથે For loop statement સમજાવો.}

\begin{solutionbox}
For લૂપનો ઉપયોગ કોઈ સિક્વન્સ (જેમ કે લિસ્ટ, ટપલ, સ્ટ્રિંગ) પર ઇટરેશન કરવા માટે થાય છે.

\textbf{સિન્ટેક્સ:}
\begin{lstlisting}[language=Python]
for વેરિએબલ in સિક્વન્સ:
    # દરેક આઇટમ માટે એક્ઝિક્યુટ થનાર કોડ
\end{lstlisting}

\textbf{ઉદાહરણ:}
\begin{lstlisting}[language=Python]
# 1 થી 5 સુધીની સંખ્યાઓના વર્ગ પ્રિન્ટ કરવા
for num in range(1, 6):
    square = num ** 2
    print(f"{num}નો વર્ગ {square} છે")
\end{lstlisting}

\textbf{આઉટપુટ:}
\begin{lstlisting}
1નો વર્ગ 1 છે
2નો વર્ગ 4 છે
3નો વર્ગ 9 છે
4નો વર્ગ 16 છે
5નો વર્ગ 25 છે
\end{lstlisting}

\begin{answerdiagram}{For Loop માટે ફલોચાર્ટ}
\begin{center}
\begin{tikzpicture}[gtu flow]
    \node[gtu start] (start) {શરૂઆત};
    \node[gtu decision, below of=start] (cond) {સિક્વન્સમાં આઇટમ છે?};
    \node[gtu process, below of=cond] (body) {કોડ બ્લોક એક્ઝિક્યુટ કરો};
    \node[gtu stop, right=3cm of cond] (stop) {અંત};

    \draw[gtu arrow] (start) -- (cond);
    \draw[gtu arrow] (cond) -- node[right] {હા} (body);
    \draw[gtu arrow] (cond) -- node[above] {ના} (stop);
    \draw[gtu arrow] (body) -- ++(-2,0) |- (cond);
\end{tikzpicture}
\end{center}
\end{answerdiagram}

\begin{mnemonicbox}
"FIRE" (For each Item, Run commands, Execute until end)
\end{mnemonicbox}
\end{solutionbox}

\orquestionmarks{3(ક)}{7}{આપેલ નંબર આર્મસ્ટ્રોંગ નંબર છે કે પેલિન્ડ્રોમ તે નિર્ધારિત કરવા પાયથોન કોડ લખો.}

\begin{solutionbox}
\begin{lstlisting}[language=Python]
# સંખ્યા આર્મસ્ટ્રોંગ નંબર છે કે નહીં તે તપાસવાનું ફંકશન
def is_armstrong(num):
    # ડિજિટ્સની સંખ્યા ગણવા
    num_str = str(num)
    n = len(num_str)
    
    # દરેક ડિજિટને n ઘાત પર ઊંચકી તેનો સરવાળો કરો
    sum_of_powers = sum(int(digit) ** n for digit in num_str)
    
    # સરવાળો મૂળ સંખ્યા સાથે સરખાવો
    return sum_of_powers == num

# સંખ્યા પેલિન્ડ્રોમ છે કે નહીં તે તપાસવાનું ફંકશન
def is_palindrome(num):
    num_str = str(num)
    # સ્ટ્રિંગ તેના રિવર્સ સાથે સરખાવો
    return num_str == num_str[::-1]

# મુખ્ય ફંકશન
def check_number(num):
    if is_armstrong(num):
        print(f"{num} એક આર્મસ્ટ્રોંગ નંબર છે")
    else:
        print(f"{num} એક આર્મસ્ટ્રોંગ નંબર નથી")
    
    if is_palindrome(num):
        print(f"{num} એક પેલિન્ડ્રોમ છે")
    else:
        print(f"{num} એક પેલિન્ડ્રોમ નથી")

# ફંકશનનું ટેસ્ટિંગ
number = int(input("એક સંખ્યા દાખલ કરો: "))
check_number(number)
\end{lstlisting}

\begin{answerdiagram}{આર્મસ્ટ્રોંગ અને પેલિન્ડ્રોમ ચેક માટે ફલોચાર્ટ}
\begin{center}
\begin{tikzpicture}[gtu flow]
    \node[gtu start] (start) {શરૂઆત};
    \node[gtu input, below of=start] (input) {સંખ્યા ઇનપુટ કરો};
    \node[gtu process, below of=input] (call) {check\_number કૉલ કરો};
    \node[gtu process, below of=call] (arm) {is\_armstrong કૉલ કરો};
    \node[gtu decision, below of=arm] (isarm) {પરિણામ True?};
    \node[gtu output, left=2cm of isarm] (pyes) {"આર્મસ્ટ્રોંગ" પ્રિન્ટ કરો};
    \node[gtu output, right=2cm of isarm] (pno) {"આર્મસ્ટ્રોંગ નથી" પ્રિન્ટ કરો};
    \node[gtu process, below of=isarm, yshift=-1cm] (pal) {is\_palindrome કૉલ કરો};
    \node[gtu decision, below of=pal] (ispal) {પરિણામ True?};
    \node[gtu output, left=2cm of ispal] (ppyes) {"પેલિન્ડ્રોમ" પ્રિન્ટ કરો};
    \node[gtu output, right=2cm of ispal] (ppno) {"પેલિન્ડ્રોમ નથી" પ્રિન્ટ કરો};
    \node[gtu stop, below of=ispal] (stop) {અંત};

    \draw[gtu arrow] (start) -- (input);
    \draw[gtu arrow] (input) -- (call);
    \draw[gtu arrow] (call) -- (arm);
    \draw[gtu arrow] (arm) -- (isarm);
    \draw[gtu arrow] (isarm) -- node[above] {હા} (pyes);
    \draw[gtu arrow] (isarm) -- node[above] {ના} (pno);
    \draw[gtu arrow] (pyes) |- (pal);
    \draw[gtu arrow] (pno) |- (pal);
    \draw[gtu arrow] (pal) -- (ispal);
    \draw[gtu arrow] (ispal) -- node[above] {હા} (ppyes);
    \draw[gtu arrow] (ispal) -- node[above] {ના} (ppno);
    \draw[gtu arrow] (ppyes) |- (stop);
    \draw[gtu arrow] (ppno) |- (stop);
\end{tikzpicture}
\end{center}
\end{answerdiagram}

\begin{mnemonicbox}
"APC" (Armstrong check: Power sum of digits, Palindrome check: Compare with reverse)
\end{mnemonicbox}
\end{solutionbox}

\questionmarks{4(અ)}{3}{સ્કેન કરેલ નંબર even છે કે odd તે શોધવા પાયથોન કોડ વિકસાવો અને યોગ્ય મેસેજ પ્રિન્ટ કરો.}

\begin{solutionbox}
\begin{lstlisting}[language=Python]
# સંખ્યા સમ છે કે વિષમ તે તપાસવાનો પ્રોગ્રામ
number = int(input("એક સંખ્યા દાખલ કરો: "))

if number % 2 == 0:
    print(f"{number} એક સમ સંખ્યા છે")
else:
    print(f"{number} એક વિષમ સંખ્યા છે")
\end{lstlisting}

\begin{answerdiagram}{સમ/વિષમ માટે ફલોચાર્ટ}
\begin{center}
\begin{tikzpicture}[gtu flow]
    \node[gtu start] (start) {શરૂઆત};
    \node[gtu input, below of=start] (input) {સંખ્યા ઇનપુટ કરો};
    \node[gtu decision, below of=input] (cond) {number \% 2 == 0?};
    \node[gtu output, below left of=cond, xshift=-1cm] (even) {"સમ" પ્રિન્ટ કરો};
    \node[gtu output, below right of=cond, xshift=1cm] (odd) {"વિષમ" પ્રિન્ટ કરો};
    \node[gtu stop, below of=cond, yshift=-3cm] (stop) {અંત};

    \draw[gtu arrow] (start) -- (input);
    \draw[gtu arrow] (input) -- (cond);
    \draw[gtu arrow] (cond) -| node[above] {હા} (even);
    \draw[gtu arrow] (cond) -| node[above] {ના} (odd);
    \draw[gtu arrow] (even) |- (stop);
    \draw[gtu arrow] (odd) |- (stop);
\end{tikzpicture}
\end{center}
\end{answerdiagram}

\begin{mnemonicbox}
"MODE" (Modulo Operation Determines Even-odd)
\end{mnemonicbox}
\end{solutionbox}

\questionmarks{4(બ)}{4}{ફંકશનની વ્યાખ્યા આપો. યુઝર ડિફાઇન ફંકશન યોગ્ય ઉદાહરણ આપી સમજાવો.}

\begin{solutionbox}
ફંકશન એ કોડનો એવો બ્લોક છે જે ચોક્કસ કાર્ય કરવા માટે વ્યવસ્થિત અને ફરીથી ઉપયોગ કરી શકાય છે.

\textbf{યુઝર-ડિફાઇન ફંકશનના ઘટકો:}
\begin{itemize}
    \item \textbf{def કીવર્ડ}: ફંકશન વ્યાખ્યાની શરૂઆત દર્શાવે છે
    \item \textbf{ફંકશન નામ}: ફંકશન માટે ઓળખકર્તા
    \item \textbf{પેરામીટર્સ}: ઇનપુટ વેલ્યુઝ (વૈકલ્પિક)
    \item \textbf{ડોકસ્ટ્રિંગ}: ફંકશનનું વર્ણન (વૈકલ્પિક)
    \item \textbf{ફંકશન બોડી}: એક્ઝિક્યુટ થનાર કોડ
    \item \textbf{રિટર્ન સ્ટેટમેન્ટ}: આઉટપુટ વેલ્યુ (વૈકલ્પિક)
\end{itemize}

\textbf{ઉદાહરણ:}
\begin{lstlisting}[language=Python]
# લંબચોરસનું ક્ષેત્રફળ ગણવા માટેનું યુઝર-ડિફાઇન ફંકશન
def calculate_area(length, width):
    """
    લંબચોરસનું ક્ષેત્રફળ ગણે છે
    """
    area = length * width
    return area

# ફંકશન કૉલ કરો
result = calculate_area(5, 3)
print(f"લંબચોરસનું ક્ષેત્રફળ: {result}")
\end{lstlisting}

\begin{mnemonicbox}
"DRAPE" (Define function, Receive parameters, Acquire result, Process data, End with return)
\end{mnemonicbox}
\end{solutionbox}

\questionmarks{4(ક)}{7}{વિવિધ સ્ટ્રિંગ ઓપરેશનની યાદી બનાવો અને કોઈપણ ત્રણ ઉદાહરણનો ઉપયોગ કરીને સમજાવો.}

\begin{solutionbox}
પાયથોનમાં સ્ટ્રિંગ ઓપરેશન્સ:

\begin{answertable}{સ્ટ્રિંગ ઓપરેશન્સ}
\begin{tabulary}{\linewidth}{|L|L|}
\hline
\textbf{ઓપરેશન} & \textbf{વર્ણન} \\
\hline
Concatenation & + નો ઉપયોગ કરીને સ્ટ્રિંગ્સ જોડવી \\
\hline
Repetition & * નો ઉપયોગ કરીને સ્ટ્રિંગ રિપીટ કરવી \\
\hline
Indexing & પોઝિશન દ્વારા કેરેક્ટર એક્સેસ કરવા \\
\hline
Slicing & સ્ટ્રિંગનો ભાગ એક્સટ્રેક્ટ કરવો \\
\hline
Methods & બિલ્ટ-ઇન ફંકશન્સ (len, upper, lower, વગેરે) \\
\hline
Membership Testing & સ્ટ્રિંગમાં સબસ્ટ્રિંગ છે કે નહીં તે તપાસવું \\
\hline
Formatting & ફોર્મેટેડ સ્ટ્રિંગ્સ બનાવવી \\
\hline
Escape Sequences & \textbackslash થી શરૂ થતા સ્પેશિયલ કેરેક્ટર્સ \\
\hline
\end{tabulary}
\end{answertable}

\textbf{ઉદાહરણ:}

\textbf{1. સ્ટ્રિંગ Concatenation:}
\begin{lstlisting}[language=Python]
first_name = "John"
last_name = "Doe"
full_name = first_name + " " + last_name
print(full_name)  # આઉટપુટ: John Doe
\end{lstlisting}

\textbf{2. સ્ટ્રિંગ Slicing:}
\begin{lstlisting}[language=Python]
message = "Python Programming"
print(message[0:6])    # આઉટપુટ: Python
print(message[7:])     # આઉટપુટ: Programming
print(message[-11:])   # આઉટપુટ: Programming
\end{lstlisting}

\textbf{3. સ્ટ્રિંગ Methods:}
\begin{lstlisting}[language=Python]
text = "python programming"
print(text.upper())    # આઉટપુટ: PYTHON PROGRAMMING
print(text.capitalize())  # આઉટપુટ: Python programming
print(text.replace("python", "Java"))  # આઉટપુટ: Java programming
\end{lstlisting}

\begin{mnemonicbox}
"CSM" (Concatenate strings, Slice portions, Manipulate with methods)
\end{mnemonicbox}
\end{solutionbox}

\orquestionmarks{4(અ)}{3}{પોઝિટિવ અને નેગેટિવ નંબર તપાસવા પાયથોન કોડ બનાવો.}

\begin{solutionbox}
\begin{lstlisting}[language=Python]
# સંખ્યા પોઝિટિવ છે કે નેગેટિવ તે તપાસવાનો પ્રોગ્રામ
number = float(input("એક સંખ્યા દાખલ કરો: "))

if number > 0:
    print(f"{number} એક પોઝિટિવ સંખ્યા છે")
elif number < 0:
    print(f"{number} એક નેગેટિવ સંખ્યા છે")
else:
    print("સંખ્યા શૂન્ય છે")
\end{lstlisting}

\begin{answerdiagram}{પોઝિટિવ/નેગેટિવ માટે ફલોચાર્ટ}
\begin{center}
\begin{tikzpicture}[gtu flow]
    \node[gtu start] (start) {શરૂઆત};
    \node[gtu input, below of=start] (input) {સંખ્યા ઇનપુટ કરો};
    \node[gtu decision, below of=input] (pos) {number > 0?};
    \node[gtu output, left=2cm of pos] (ppos) {"પોઝિટિવ" પ્રિન્ટ કરો};
    \node[gtu decision, below of=pos] (neg) {number < 0?};
    \node[gtu output, left=2cm of neg] (pneg) {"નેગેટિવ" પ્રિન્ટ કરો};
    \node[gtu output, right=2cm of neg] (pzero) {"શૂન્ય" પ્રિન્ટ કરો};
    \node[gtu stop, below of=neg] (stop) {અંત};

    \draw[gtu arrow] (start) -- (input);
    \draw[gtu arrow] (input) -- (pos);
    \draw[gtu arrow] (pos) -- node[above] {હા} (ppos);
    \draw[gtu arrow] (pos) -- node[right] {ના} (neg);
    \draw[gtu arrow] (neg) -- node[above] {હા} (pneg);
    \draw[gtu arrow] (neg) -- node[above] {ના} (pzero);
    \draw[gtu arrow] (ppos) |- (stop);
    \draw[gtu arrow] (pneg) |- (stop);
    \draw[gtu arrow] (pzero) |- (stop);
\end{tikzpicture}
\end{center}
\end{answerdiagram}

\begin{mnemonicbox}
"SIGN" (See If Greater than 0, Negative otherwise)
\end{mnemonicbox}
\end{solutionbox}

\orquestionmarks{4(બ)}{4}{યોગ્ય ઉદાહરણો સાથે local અને global વેરિએબલ સમજાવો.}

\begin{solutionbox}
પાયથોનમાં વેરિએબલ્સના અલગ-અલગ સ્કોપ્સ હોઈ શકે છે:

\begin{answertable}{વેરિએબલ સ્કોપ્સ}
\begin{tabulary}{\linewidth}{|L|L|}
\hline
\textbf{વેરિએબલ પ્રકાર} & \textbf{વર્ણન} \\
\hline
Local Variable & ફંકશનની અંદર વ્યાખ્યાયિત અને માત્ર તે ફંકશનની અંદર જ એક્સેસિબલ \\
\hline
Global Variable & ફંકશનની બહાર વ્યાખ્યાયિત અને પ્રોગ્રામના તમામ ભાગમાં એક્સેસિબલ \\
\hline
\end{tabulary}
\end{answertable}

\textbf{ઉદાહરણ:}
\begin{lstlisting}[language=Python]
# Global વેરિએબલ
count = 0 

def update_count():
    # Local વેરિએબલ
    local_var = 5 
    
    # ફંકશનની અંદર Global વેરિએબલ એક્સેસ કરવો
    global count
    count += 1
    
    print(f"Local: {local_var}")
    print(f"Global (inside): {count}")
    
# ફંકશન કૉલ કરો
update_count()

# ફંકશનની બહાર વેરિએબલ એક્સેસ કરવા
print(f"Global (outside): {count}")
\end{lstlisting}

\begin{mnemonicbox}
"SCOPE" (Some variables Confined to function Only, Program-wide Exposure for others)
\end{mnemonicbox}
\end{solutionbox}

\orquestionmarks{4(ક)}{7}{વિવિધ લિસ્ટ ઓપરેશનની યાદી બનાવો અને કોઈપણ ત્રણ ઉદાહરણનો ઉપયોગ કરીને સમજાવો.}

\begin{solutionbox}
પાયથોનમાં લિસ્ટ ઓપરેશન્સ:

\begin{answertable}{લિસ્ટ ઓપરેશન્સ}
\begin{tabulary}{\linewidth}{|L|L|}
\hline
\textbf{ઓપરેશન} & \textbf{વર્ણન} \\
\hline
લિસ્ટ બનાવવી & સ્ક્વેર બ્રેકેટ્સ [] નો ઉપયોગ \\
\hline
ઇન્ડેક્સિંગ & પોઝિશન દ્વારા એલિમેન્ટ એક્સેસ કરવા \\
\hline
સ્લાઇસિંગ & લિસ્ટના ભાગો એક્સટ્રેક્ટ કરવા \\
\hline
એપેન્ડ & છેલ્લે એલિમેન્ટ ઉમેરવા \\
\hline
ઇન્સર્ટ & ચોક્કસ પોઝિશન પર એલિમેન્ટ ઉમેરવા \\
\hline
રિમૂવ & ચોક્કસ એલિમેન્ટ દૂર કરવા \\
\hline
પોપ & એલિમેન્ટ દૂર કરવું અને પાછું મેળવવું \\
\hline
સોર્ટ & લિસ્ટ એલિમેન્ટ્સ ઓર્ડર કરવા \\
\hline
રિવર્સ & લિસ્ટનો ક્રમ ઊલટાવવો \\
\hline
એક્સ્ટેન્ડ & લિસ્ટ્સ જોડવી \\
\hline
\end{tabulary}
\end{answertable}

\textbf{ઉદાહરણ:}

\textbf{1. લિસ્ટ ઇન્ડેક્સિંગ અને સ્લાઇસિંગ:}
\begin{lstlisting}[language=Python]
fruits = ["apple", "banana", "cherry", "orange", "kiwi"]
print(fruits[1])        # આઉટપુટ: banana
print(fruits[-1])       # આઉટપુટ: kiwi
print(fruits[1:4])      # આઉટપુટ: ['banana', 'cherry', 'orange']
\end{lstlisting}

\textbf{2. લિસ્ટ મેથડ્સ (append, insert, remove):}
\begin{lstlisting}[language=Python]
numbers = [1, 2, 3]
numbers.append(4)       # છેલ્લે 4 ઉમેરો
numbers.insert(0, 0)    # પોઝિશન 0 પર 0 ઇન્સર્ટ કરો
numbers.remove(2)       # 2 વેલ્યુ ધરાવતો એલિમેન્ટ દૂર કરો
print(numbers)          # આઉટપુટ: [0, 1, 3, 4]
\end{lstlisting}

\textbf{3. લિસ્ટ કોમ્પ્રિહેન્શન્સ:}
\begin{lstlisting}[language=Python]
# સ્ક્વેર્સની લિસ્ટ બનાવવી
squares = [x**2 for x in range(1, 6)]
print(squares)  # આઉટપુટ: [1, 4, 9, 16, 25]
\end{lstlisting}

\begin{mnemonicbox}
"AIM" (Access with index, Insert/modify elements, Make using comprehensions)
\end{mnemonicbox}
\end{solutionbox}

\questionmarks{5(અ)}{3}{લિસ્ટમાં આપેલ બે એલિમેન્ટ્સને સ્વેપ કરવા પાયથોન કોડ લખો.}

\begin{solutionbox}
\begin{lstlisting}[language=Python]
# લિસ્ટમાં બે એલિમેન્ટ્સને સ્વેપ કરવાનો પ્રોગ્રામ
def swap_elements(my_list, pos1, pos2):
    # પોઝિશન્સ માન્ય છે કે નહીં તે તપાસો
    if 0 <= pos1 < len(my_list) and 0 <= pos2 < len(my_list):
        # એલિમેન્ટ્સ સ્વેપ કરો
        my_list[pos1], my_list[pos2] = my_list[pos2], my_list[pos1]
        return True
    else:
        return False

# ઉદાહરણ
numbers = [10, 20, 30, 40, 50]
print("મૂળ લિસ્ટ:", numbers)

# પોઝિશન 1 અને 3 પરના એલિમેન્ટ્સ સ્વેપ કરો
if swap_elements(numbers, 1, 3):
    print("સ્વેપ પછી:", numbers)
else:
    print("અમાન્ય પોઝિશન્સ")
\end{lstlisting}

\begin{mnemonicbox}
"SWAP" (Select positions, Watch boundaries, Assign simultaneously, Print result)
\end{mnemonicbox}
\end{solutionbox}

\questionmarks{5(બ)}{4}{પાયથોનનાં Math મોડ્યુલ અને random મોડ્યુલ ઉદાહરણનાં ઉપયોગ કરીને સમજાવો.}

\begin{solutionbox}
Math અને random મોડ્યુલ મેથેમેટિકલ ઓપરેશન્સ અને રેન્ડમ નંબર જનરેશન માટેના ફંકશન્સ પ્રદાન કરે છે.

\textbf{Math મોડ્યુલ:}
\begin{lstlisting}[language=Python]
import math
print(math.pi)          # આઉટપુટ: 3.14159...
print(math.sqrt(16))    # આઉટપુટ: 4.0
print(math.ceil(4.2))   # આઉટપુટ: 5
\end{lstlisting}

\textbf{Random મોડ્યુલ:}
\begin{lstlisting}[language=Python]
import random
print(random.random())       # રેન્ડમ ફ્લોટ
print(random.randint(1, 10)) # રેન્ડમ ઇન્ટીજર
colors = ["red", "green"]
print(random.choice(colors)) # રેન્ડમ પસંદગી
\end{lstlisting}

\begin{mnemonicbox}
"MR-CS" (Math for Calculations, Random for Choice and Shuffling)
\end{mnemonicbox}
\end{solutionbox}

\questionmarks{5(ક)}{7}{Tuple ફંકશન અને ઓપરેશન દર્શાવવા પાયથોન કોડ લખો.}

\begin{solutionbox}
\begin{lstlisting}[language=Python]
# Tuples બનાવવા
mixed_tuple = (1, "Hello", 3.14, True)

# એલિમેન્ટ એક્સેસ કરવા
print(mixed_tuple[0])      # આઉટપુટ: 1

# Tuple સ્લાઇસિંગ
print(mixed_tuple[1:3])    # આઉટપુટ: ("Hello", 3.14)

# Tuple જોડવા
tuple1 = (1, 2)
tuple2 = (3, 4)
print(tuple1 + tuple2)     # આઉટપુટ: (1, 2, 3, 4)

# Tuple મેથડ્સ
numbers = (1, 2, 2)
print(numbers.count(2))    # આઉટપુટ: 2

# મેમ્બરશિપ ટેસ્ટિંગ
print(1 in numbers)        # આઉટપુટ: True
\end{lstlisting}

\begin{answerdiagram}{Tuple ઓપરેશન્સ}
\begin{center}
\begin{tikzpicture}[gtu tree]
    \node[gtu root] {Tuple ઓપરેશન્સ}
    child {node[gtu child] {બનાવવી\\ \texttt{()}}}
    child {node[gtu child] {એક્સેસિંગ\\ \texttt{[]}}}
    child {node[gtu child] {સ્લાઇસિંગ\\ \texttt{[:]}}}
    child {node[gtu child] {જોડવું\\ \texttt{+}}}
    child {node[gtu child] {મેથડ્સ\\ \texttt{count()}}};
\end{tikzpicture}
\end{center}
\end{answerdiagram}

\begin{mnemonicbox}
"CASC-RUMTC" (Create, Access, Slice, Concatenate, Repeat, Use methods, Membership test, Tuple conversion)
\end{mnemonicbox}
\end{solutionbox}

\orquestionmarks{5(અ)}{3}{લિસ્ટમાં સામેલ એલિમેંટનો સરવાળો કરવા પાયથોન કોડ લખો.}

\begin{solutionbox}
\begin{lstlisting}[language=Python]
# લિસ્ટના એલિમેન્ટ્સનો સરવાળો કરવા માટેનો પ્રોગ્રામ
def sum_of_elements(numbers):
    total = 0
    for num in numbers:
        total += num
    return total

# ઉદાહરણ
my_list = [10, 20, 30, 40, 50]
print("સરવાળો:", sum_of_elements(my_list))
\end{lstlisting}

\begin{answerdiagram}{સરવાળો માટે ફલોચાર્ટ}
\begin{center}
\begin{tikzpicture}[gtu flow]
    \node[gtu start] (start) {શરૂઆત};
    \node[gtu process, below of=start] (init) {total = 0 શરૂ કરો};
    \node[gtu decision, below of=init] (loop) {વધુ એલિમેન્ટ્સ?};
    \node[gtu process, below of=loop] (add) {total += num};
    \node[gtu output, right=3cm of loop] (ret) {total પરત કરો};
    \node[gtu stop, below of=ret] (stop) {અંત};

    \draw[gtu arrow] (start) -- (init);
    \draw[gtu arrow] (init) -- (loop);
    \draw[gtu arrow] (loop) -- node[right] {હા} (add);
    \draw[gtu arrow] (loop) -- node[above] {ના} (ret);
    \draw[gtu arrow] (add) -- ++(-2,0) |- (loop);
    \draw[gtu arrow] (ret) -- (stop);
\end{tikzpicture}
\end{center}
\end{answerdiagram}

\begin{mnemonicbox}
"SITE" (Sum Initialized To zero, Elements added one by one)
\end{mnemonicbox}
\end{solutionbox}

\orquestionmarks{5(બ)}{4}{નીચે આપેલ built in functionsનો ઉપયોગ સમજાવો: 1) Print() 2) Min() 3) Sum() 4) Input()}

\begin{solutionbox}
\begin{answertable}{Built-in ફંકશન્સ}
\begin{tabulary}{\linewidth}{|l|L|l|}
\hline
\textbf{ફંકશન} & \textbf{હેતુ} & \textbf{ઉદાહરણ} \\
\hline
\textbf{print()} & કન્સોલ પર આઉટપુટ દર્શાવે છે & \texttt{print("Hi")} \\
\hline
\textbf{min()} & સૌથી નાના આઇટમને પરત કરે છે & \texttt{min([5, 1])}\\
\hline
\textbf{sum()} & તમામ આઇટમ્સનો સરવાળો આપે છે & \texttt{sum([1, 2])} \\
\hline
\textbf{input()} & વપરાશકર્તા પાસેથી ઇનપુટ વાંચે છે & \texttt{input("Naam:")} \\
\hline
\end{tabulary}
\end{answertable}

\begin{lstlisting}[language=Python]
print("Hello")
print(min([5, 3, 8]))
print(sum([1, 2, 3]))
name = input("Enter name: ")
\end{lstlisting}

\begin{mnemonicbox}
"PMSI" (Print to display, Min for smallest, Sum for total, Input for reading)
\end{mnemonicbox}
\end{solutionbox}

\orquestionmarks{5(ક)}{7}{સેટ ફંકશન અને ઓપરેશન દર્શાવવા પાયથોન કોડ લખો.}

\begin{solutionbox}
\begin{lstlisting}[language=Python]
# સેટ બનાવવા
numbers = {1, 2, 3}

# એલિમેન્ટ ઉમેરવા
numbers.add(4)

# અપડેટ
numbers.update([5, 6])

# દૂર કરવા
numbers.remove(3)

# સેટ ઓપરેશન્સ
set1 = {1, 2, 3}
set2 = {3, 4, 5}

# યુનિયન
print(set1 | set2)      # આઉટપુટ: {1, 2, 3, 4, 5}

# ઇન્ટરસેક્શન
print(set1 & set2)      # આઉટપુટ: {3}

# ડિફરન્સ
print(set1 - set2)      # આઉટપુટ: {1, 2}
\end{lstlisting}

\begin{answerdiagram}{સેટ ઓપરેશન્સ}
\begin{center}
\begin{tikzpicture}[gtu tree]
    \node[gtu root] {સેટ ઓપરેશન્સ}
    child {node[gtu child] {મોડિફાય\\(Add/Remove)}}
    child {node[gtu child] {યુનિયન\\($|$ )}}
    child {node[gtu child] {ઇન્ટરસેક્શન\\($\& $)}}
    child {node[gtu child] {ડિફરન્સ\\($-$)}};
\end{tikzpicture}
\end{center}
\end{answerdiagram}

\begin{mnemonicbox}
"CARDS-UI" (Create, Add, Remove, Discard elements, Set operations - Union, Intersection)
\end{mnemonicbox}
\end{solutionbox}

\end{document}
