\documentclass{article}

% content/resources/templates/preamble.tex
\usepackage[margin=0.6in]{geometry}
\author{Milav Dabgar}
\usepackage{amsmath,amssymb,amsthm}
\usepackage{booktabs}
\usepackage{multirow}
\usepackage{xcolor}
\usepackage{tcolorbox}
\tcbuselibrary{breakable,skins}
\usepackage[colorlinks=true,linkcolor=blue]{hyperref}
\usepackage{titlesec}
\usepackage{enumitem}
\usepackage{tikz}
\usepackage{pgfplots}
\usepackage{circuitikz}
\usepackage[version=4]{mhchem}
\usepackage{longtable}
\usepackage{array}
\usepackage{float}
\usepackage{caption}
\usepackage{listings}

\lstset{
  basicstyle=\small\ttfamily,
  breaklines=true,
  breakatwhitespace=false,
  postbreak=\mbox{\textcolor{red}{$\hookrightarrow$}\space},
  float=false,
  numbers=left,
  numberstyle=\tiny\color{gray},
  numbersep=10pt,
  xleftmargin=2em,
  keywordstyle=\color{blue},
  commentstyle=\color{green!60!black},
  stringstyle=\color{purple},
  backgroundcolor=\color{gray!5},
  showstringspaces=false,
  tabsize=2,
  captionpos=b,
  keepspaces=true,
  columns=flexible
}

\pgfplotsset{compat=1.18}
\usetikzlibrary{shapes,arrows,positioning,calc,patterns,decorations.pathmorphing,decorations.markings,arrows.meta}

% Color scheme
\definecolor{headcolor}{RGB}{0,102,204}
\definecolor{keycolor}{RGB}{220,20,60}
\definecolor{solutioncolor}{RGB}{34,139,34}
\definecolor{mnemoniccolor}{RGB}{148,0,211}
\definecolor{codecolor}{RGB}{0,0,100}

% Spacing
\setlength{\parskip}{3pt}
\setlist[itemize]{nosep}
\setlist[enumerate]{nosep}

% Title formatting
\titleformat{\section}{\Large\bfseries\color{headcolor}}{\thesection}{1em}{}
\titleformat{\subsection}{\large\bfseries\color{headcolor}}{\thesubsection}{1em}{}

% Pandoc tightlist compatibility
\providecommand{\tightlist}{%
  \setlength{\itemsep}{0pt}\setlength{\parskip}{0pt}}

% Pandoc longtable compatibility
\newcounter{none}
\def\thenone{}


% content/resources/templates/english-boxes.tex

% Custom environments
\newtcolorbox{solutionbox}{
 breakable,
 enhanced,
 colback=solutioncolor!5!white,
 colframe=solutioncolor!75!black,
 fonttitle=\bfseries,
 title=Solution
}

\newtcolorbox{solutionboxnobreak}{
 colback=solutioncolor!5!white,
 colframe=solutioncolor!75!black,
 fonttitle=\bfseries,
 title=Solution
}

\newtcolorbox{keyformula}{
 breakable,
 enhanced,
 colback=keycolor!5!white,
 colframe=keycolor!75!black,
 fonttitle=\bfseries,
 title=Key Formula
}

\newtcolorbox{mnemonicboxenv}{
 breakable,
 enhanced,
 colback=mnemoniccolor!5!white,
 colframe=mnemoniccolor!75!black,
 fonttitle=\bfseries,
 title=Mnemonic
}

\newcommand{\mnemonicbox}[1]{%
  \begin{mnemonicboxenv}
    #1
  \end{mnemonicboxenv}
}


% Custom commands for GTU solutions
% This file defines semantic commands for consistent formatting

% Question command with automatic formatting
\newcommand{\question}[2]{%
  \section*{Question #1}%
  \textbf{#2}%
}

% OR question variant
\newcommand{\questionor}[2]{%
  \section*{Question #1 OR}%
  \textbf{#2}%
}

% Proper table environment with caption
\newenvironment{answertable}[1]{%
  \begin{table}[htbp]
  \centering
  \caption{#1}
}{%
  \end{table}
}

% Proper figure environment for diagrams
\newenvironment{answerdiagram}[1]{%
  \begin{figure}[htbp]
  \centering
  \caption{#1}
}{%
  \end{figure}
}

% Semantic markup for key terms
\newcommand{\keyword}[1]{\textbf{#1}}
\newcommand{\code}[1]{\texttt{#1}}
\newcommand{\classname}[1]{\texttt{#1}}
\newcommand{\methodname}[1]{\texttt{#1}}

% Proper quotation marks
\newcommand{\mnemonic}[1]{``#1''}


\title{Python Programming (1323203) - Winter 2023 Solution}
\date{January 20, 2024}

\begin{document}
\maketitle

\questionmarks{1(a)}{3}{Write a pseudocode to check the given number is positive or negative.}

\begin{solutionbox}
\begin{lstlisting}[caption={Pseudocode for Number Check}]
BEGIN
    Input number
    IF number > 0 THEN
        Display "Number is positive"
    ELSE IF number < 0 THEN
        Display "Number is negative"
    ELSE
        Display "Number is zero"
    END IF
END
\end{lstlisting}
\end{solutionbox}

\begin{mnemonicbox}
\mnemonic{Compare Zero}
\end{mnemonicbox}

\questionmarks{1(b)}{4}{Define Algorithm and Design it for Finding maximum from given three Numbers.}

\begin{solutionbox}
\textbf{Algorithm Definition}: An algorithm is a step-by-step procedure or set of rules designed to solve a specific problem or perform a computation.

\textbf{Algorithm for Finding Maximum of Three Numbers}:

\begin{lstlisting}[caption={Algorithm for Maximum of Three Numbers}]
BEGIN
    Input num1, num2, num3
    Set max = num1
    IF num2 > max THEN
        Set max = num2
    END IF
    IF num3 > max THEN
        Set max = num3
    END IF
    Display max
END
\end{lstlisting}

\begin{center}
\begin{tikzpicture}[node distance=1.5cm, auto]
    \node [gtu state] (start) {Start};
    \node [gtu block, below=of start] (input) {Input num1, num2, num3};
    \node [gtu block, below=of input] (init) {Set max = num1};
    \node [gtu decision, below=of init] (cond1) {Is num2 $>$ max?};
    \node [gtu block, right=of cond1] (set2) {Set max = num2};
    \node [gtu decision, below=of cond1] (cond2) {Is num3 $>$ max?};
    \node [gtu block, right=of cond2] (set3) {Set max = num3};
    \node [gtu block, below=of cond2] (disp) {Display max};
    \node [gtu state, below=of disp] (end) {End};

    \path [gtu arrow] (start) -- (input);
    \path [gtu arrow] (input) -- (init);
    \path [gtu arrow] (init) -- (cond1);
    \path [gtu arrow] (cond1) -- node {Yes} (set2);
    \path [gtu arrow] (cond1) -- node {No} (cond2);
    \path [gtu arrow] (set2) |- (cond2);
    \path [gtu arrow] (cond2) -- node {Yes} (set3);
    \path [gtu arrow] (cond2) -- node {No} (disp);
    \path [gtu arrow] (set3) |- (disp);
    \path [gtu arrow] (disp) -- (end);
\end{tikzpicture}
\captionof{figure}{Flowchart to Find Maximum of Three Numbers}
\end{center}
\end{solutionbox}

\begin{mnemonicbox}
\mnemonic{Compare and Replace}
\end{mnemonicbox}

\questionmarks{1(c)}{7}{Develop a Python code to convert Temperature parameter from Celsius to Fahrenheit.}

\begin{solutionbox}
\begin{lstlisting}[language=Python, caption={Celsius to Fahrenheit Conversion}]
# Program to convert Celsius to Fahrenheit

# Get the Celsius temperature from user
celsius = float(input("Enter temperature in Celsius: "))

# Convert to Fahrenheit using the formula: F = (C * 9/5) + 32
fahrenheit = (celsius * 9/5) + 32

# Display the result
print(f"{celsius}\u00B0C is equal to {fahrenheit}\u00B0F")
\end{lstlisting}

\begin{center}
\captionof{table}{Temperature Conversion}
\begin{tabulary}{\linewidth}{|L|L|}
\hline
\textbf{Component} & \textbf{Description} \\ \hline
\textbf{Input} & Temperature in Celsius \\ \hline
\textbf{Formula} & F = (C $\times$ 9/5) + 32 \\ \hline
\textbf{Output} & Temperature in Fahrenheit \\ \hline
\end{tabulary}
\end{center}
\end{solutionbox}

\begin{mnemonicbox}
\mnemonic{Multiply by 9, divide by 5, add 32}
\end{mnemonicbox}

\questionmarks{1(c OR)}{7}{List out all comparison operators and explain each by giving python code example.}

\begin{solutionbox}
\begin{center}
\captionof{table}{Python Comparison Operators}
\begin{tabulary}{\linewidth}{|C|L|L|C|}
\hline
\textbf{Operator} & \textbf{Description} & \textbf{Example} & \textbf{Result} \\ \hline
\textbf{==} & Equal to & \code{5 == 5} & \code{True} \\ \hline
\textbf{!=} & Not equal to & \code{5 != 6} & \code{True} \\ \hline
\textbf{>} & Greater than & \code{6 > 3} & \code{True} \\ \hline
\textbf{<} & Less than & \code{3 < 6} & \code{True} \\ \hline
\textbf{>=} & Greater than or equal to & \code{5 >= 5} & \code{True} \\ \hline
\textbf{<=} & Less than or equal to & \code{5 <= 5} & \code{True} \\ \hline
\end{tabulary}
\end{center}

\textbf{Code Example}:

\begin{lstlisting}[language=Python, caption={Comparison Operators Example}]
# Python comparison operators example
a = 10
b = 5

# Equal to
print(f"{a} == {b}: {a == b}")  # False

# Not equal to
print(f"{a} != {b}: {a != b}")  # True

# Greater than
print(f"{a} > {b}: {a > b}")    # True

# Less than
print(f"{a} < {b}: {a < b}")    # False

# Greater than or equal to
print(f"{a} >= {b}: {a >= b}")  # True

# Less than or equal to
print(f"{a} <= {b}: {a <= b}")  # False
\end{lstlisting}
\end{solutionbox}

\begin{mnemonicbox}
\mnemonic{CLEAN: Compare, Less than, Equal to, Above, Not equal}
\end{mnemonicbox}

\questionmarks{2(a)}{3}{Describe data types in python with its examples.}

\begin{solutionbox}
\begin{center}
\captionof{table}{Python Data Types}
\begin{tabulary}{\linewidth}{|L|L|L|}
\hline
\textbf{Data Type} & \textbf{Description} & \textbf{Example} \\ \hline
\textbf{int} & Integer values & \code{x = 10} \\ \hline
\textbf{float} & Decimal point values & \code{y = 10.5} \\ \hline
\textbf{str} & Text or character values & \code{name = "Python"} \\ \hline
\textbf{bool} & Logical values (True/False) & \code{is\_valid = True} \\ \hline
\textbf{list} & Ordered, mutable collection & \code{nums = [1, 2, 3]} \\ \hline
\textbf{tuple} & Ordered, immutable collection & \code{point = (5, 10)} \\ \hline
\textbf{dict} & Key-value pairs & \code{student = \{"name": "John"\}} \\ \hline
\end{tabulary}
\end{center}
\end{solutionbox}

\begin{mnemonicbox}
\mnemonic{NIFTY SLD: Numbers, Integers, Floats, Text, Yes/No, Sequences, Lists, Dictionaries}
\end{mnemonicbox}

\questionmarks{2(b)}{4}{Explain Nested if in python with python code example.}

\begin{solutionbox}
\textbf{Nested if}: A conditional statement inside another conditional statement is called a nested if. It allows checking for multiple conditions in sequence.

\begin{lstlisting}[language=Python, caption={Nested If Example}]
# Nested if example to check if a number is positive, negative, or zero
# And if positive, check if it's even or odd

num = int(input("Enter a number: "))

if num > 0:
    print("Positive number")
    # Nested if to check if the positive number is even or odd
    if num % 2 == 0:
        print("Even number")
    else:
        print("Odd number")
elif num < 0:
    print("Negative number")
else:
    print("Zero")
\end{lstlisting}

\begin{center}
\begin{tikzpicture}[node distance=1.5cm, auto]
    \node [gtu state] (start) {Start};
    \node [gtu block, below=of start] (input) {Input num};
    \node [gtu decision, below=of input] (cond1) {Is num $>$ 0?};
    \node [gtu block, left=of cond1] (neg) {Print Negative \\ number};
    \node [gtu decision, below=of neg] (cond3) {Is num $<$ 0?};
    \node [gtu block, below=of cond3] (zero) {Print Zero};
    
    \node [gtu block, right=of cond1] (pos) {Print Positive \\ number};
    \node [gtu decision, below=of pos] (cond2) {Is num \% 2 == 0?};
    \node [gtu block, right=of cond2] (even) {Print Even \\ number};
    \node [gtu block, left=of cond2] (odd) {Print Odd \\ number};
    \node [gtu state, below=of zero, yshift=-2cm] (end) {End};

    \path [gtu arrow] (start) -- (input);
    \path [gtu arrow] (input) -- (cond1);
    
    % Positive branch
    \path [gtu arrow] (cond1) -- node {Yes} (pos);
    \path [gtu arrow] (pos) -- (cond2);
    \path [gtu arrow] (cond2) -- node {Yes} (even);
    \path [gtu arrow] (cond2) -- node {No} (odd);
    
    % Negative/Zero branch
    \path [gtu arrow] (cond1) -- node {No} (cond3);
    \path [gtu arrow] (cond3) -- node {Yes} (neg);
    \path [gtu arrow] (cond3) -- node {No} (zero);
    
    \path [gtu arrow] (even) |- (end);
    \path [gtu arrow] (odd) |- (end);
    \path [gtu arrow] (neg) |- (end);
    \path [gtu arrow] (zero) -- (end);
\end{tikzpicture}
\captionof{figure}{Nested If Flowchart}
\end{center}
\end{solutionbox}

\begin{mnemonicbox}
\mnemonic{Check Inside Check}
\end{mnemonicbox}

\questionmarks{2(c)}{7}{Write use of different types of selection / decision making flow of control structures with example.}

\begin{solutionbox}
\begin{center}
\captionof{table}{Selection Control Structures in Python}
\begin{tabulary}{\linewidth}{|L|L|L|}
\hline
\textbf{Structure} & \textbf{Purpose} & \textbf{Use Case} \\ \hline
\textbf{if} & Execute code when condition is true & Simple condition check \\ \hline
\textbf{if-else} & Execute one code for true condition, another for false & Binary decision making \\ \hline
\textbf{if-elif-else} & Multiple condition checking & Multiple possible outcomes \\ \hline
\textbf{Nested if} & Condition checking inside another condition & Complex hierarchical decisions \\ \hline
\textbf{Ternary operator} & One-line if-else & Simple conditional assignment \\ \hline
\end{tabulary}
\end{center}

\textbf{Code Example}:

\begin{lstlisting}[language=Python, caption={Selection Structures Example}]
# Example of different selection structures
score = int(input("Enter your score: "))

# Simple if
if score >= 90:
    print("Excellent!")

# if-else
if score >= 60:
    print("You passed.")
else:
    print("You failed.")

# if-elif-else
if score >= 90:
    grade = "A"
elif score >= 80:
    grade = "B"
elif score >= 70:
    grade = "C"
elif score >= 60:
    grade = "D"
else:
    grade = "F"
print(f"Your grade is {grade}")

# Ternary operator
result = "Pass" if score >= 60 else "Fail"
print(result)
\end{lstlisting}
\end{solutionbox}

\begin{mnemonicbox}
\mnemonic{SCENE: Simple if, Conditions with else, Elif for multiple, Nested for complex, Express with ternary}
\end{mnemonicbox}

\questionmarks{2(a OR)}{3}{List out rules for defining variables in python.}

\begin{solutionbox}
\begin{center}
\captionof{table}{Rules for Defining Variables in Python}
\begin{tabulary}{\linewidth}{|L|L|L|}
\hline
\textbf{Rule} & \textbf{Description} & \textbf{Example} \\ \hline
\textbf{Start with letter or underscore} & First character must be a letter or underscore & \code{name = "John"}, \code{\_count = 10} \\ \hline
\textbf{No special characters} & Only letters, numbers, and underscores allowed & \code{user\_name} (valid), \code{user-name} (invalid) \\ \hline
\textbf{Case sensitive} & Uppercase and lowercase are different & \code{age} and \code{Age} are different variables \\ \hline
\textbf{No reserved keywords} & Cannot use Python keywords as variable names & Cannot use \code{if}, \code{for}, \code{while}, etc. \\ \hline
\textbf{No spaces} & Use underscores instead of spaces & \code{first\_name} instead of \code{first name} \\ \hline
\end{tabulary}
\end{center}
\end{solutionbox}

\begin{mnemonicbox}
\mnemonic{SILKS: Start properly, Ignore special chars, Look at case, Keywords avoided, Spaces not allowed}
\end{mnemonicbox}

\questionmarks{2(b OR)}{4}{Explain For loop in python with necessary python code example.}

\begin{solutionbox}
\textbf{For Loop in Python}: A for loop is used to iterate over a sequence (list, tuple, string) or other iterable objects. It executes a block of code for each item in the sequence.

\begin{lstlisting}[language=Python, caption={For Loop Example}]
# Example of for loop in Python
# Printing each element in a list
fruits = ["apple", "banana", "cherry"]
for fruit in fruits:
    print(fruit)

# Using range function with for loop
print("Numbers from 1 to 5:")
for i in range(1, 6):
    print(i)

# Using for loop with string
name = "Python"
for char in name:
    print(char)
\end{lstlisting}

\begin{center}
\begin{tikzpicture}[node distance=1.5cm, auto]
    \node [gtu state] (start) {Start};
    \node [gtu block, below=of start] (init) {Initialize sequence};
    \node [gtu block, below=of init] (get) {Get next item};
    \node [gtu decision, below=of get] (check) {More items?};
    \node [gtu block, right=of check] (exec) {Execute code block};
    \node [gtu state, below=of check] (end) {End};

    \path [gtu arrow] (start) -- (init);
    \path [gtu arrow] (init) -- (get);
    \path [gtu arrow] (get) -- (check);
    \path [gtu arrow] (check) -- node {Yes} (exec);
    \path [gtu arrow] (exec) |- (get);
    \path [gtu arrow] (check) -- node {No} (end);
\end{tikzpicture}
\captionof{figure}{For Loop Flowchart}
\end{center}
\end{solutionbox}

\begin{mnemonicbox}
\mnemonic{ITEM: Iterate Through Each Member}
\end{mnemonicbox}

\questionmarks{2(c OR)}{7}{Describe Break and continue statement in python in brief.}

\begin{solutionbox}
\begin{center}
\captionof{table}{Break and Continue Statements}
\begin{tabulary}{\linewidth}{|L|L|L|}
\hline
\textbf{Statement} & \textbf{Purpose} & \textbf{Effect} \\ \hline
\textbf{break} & Exit the loop immediately & Terminates the current loop and transfers control to the statement following the loop \\ \hline
\textbf{continue} & Skip the current iteration & Jumps to the next iteration of the loop, skipping any code after the continue statement \\ \hline
\end{tabulary}
\end{center}

\textbf{Code Example}:

\begin{lstlisting}[language=Python, caption={Break and Continue Example}]
# Break statement example
print("Break example:")
for i in range(1, 11):
    if i == 6:
        print("Breaking the loop at i =", i)
        break
    print(i, end=" ")
print("\nLoop ended")

# Continue statement example
print("\nContinue example:")
for i in range(1, 11):
    if i % 2 == 0:
        continue
    print(i, end=" ")
print("\nOnly odd numbers were printed")
\end{lstlisting}

\begin{center}
\begin{tikzpicture}[node distance=1.5cm, auto]
    \node [gtu state] (start) {Start Loop};
    \node [gtu decision, below=of start] (cond_break) {Condition for break?};
    \node [gtu block, right=of cond_break] (exit) {Exit Loop};
    \node [gtu decision, below=of cond_break] (cond_cont) {Condition for continue?};
    \node [gtu block, right=of cond_cont] (skip) {Skip to next iteration};
    \node [gtu block, below=of cond_cont] (exec) {Execute remaining code};
    \node [gtu block, below=of exit] (after) {Continue execution};
    
    \path [gtu arrow] (start) -- (cond_break);
    \path [gtu arrow] (cond_break) -- node {Yes} (exit);
    \path [gtu arrow] (cond_break) -- node {No} (cond_cont);
    \path [gtu arrow] (cond_cont) -- node {Yes} (skip);
    \path [gtu arrow] (cond_cont) -- node {No} (exec);
    
    \path [gtu arrow] (skip) |- (start);
    \path [gtu arrow] (exec) -- ++(0,-1) -| (start);
    \path [gtu arrow] (exit) -- (after);
\end{tikzpicture}
\captionof{figure}{Break and Continue Flowchart}
\end{center}
\end{solutionbox}

\begin{mnemonicbox}
\mnemonic{EXIT SKIP: EXIT with break, SKIP with continue}
\end{mnemonicbox}

\questionmarks{3(a)}{3}{Develop a python program to print 1 to 10 numbers using loops.}

\begin{solutionbox}
\begin{lstlisting}[language=Python, caption={Printing 1 to 10}]
# Using for loop to print numbers from 1 to 10
print("Using for loop:")
for i in range(1, 11):
    print(i, end=" ")

print("\n\nUsing while loop:")
# Using while loop to print numbers from 1 to 10
counter = 1
while counter <= 10:
    print(counter, end=" ")
    counter += 1
\end{lstlisting}

\begin{center}
\captionof{table}{Loop Approaches}
\begin{tabulary}{\linewidth}{|L|L|}
\hline
\textbf{Approach} & \textbf{Advantage} \\ \hline
\textbf{For loop with range} & Simple, concise, automatically manages counter \\ \hline
\textbf{While loop} & More flexible for complex conditions \\ \hline
\end{tabulary}
\end{center}
\end{solutionbox}

\begin{mnemonicbox}
\mnemonic{COUNT UP: Counter Updates in each iteration}
\end{mnemonicbox}

\questionmarks{3(b)}{4}{Develop a python program to print following pattern using loop: \newline \texttt{* \newline ** \newline *** \newline **** \newline *****}}

\begin{solutionbox}
\begin{lstlisting}[language=Python, caption={Star Pattern Program}]
# Print star pattern using for loop
rows = 5

for i in range(1, rows + 1):
    # Print i stars in each row
    print("*" * i)
\end{lstlisting}

\textbf{Alternative solution with nested loops}:

\begin{lstlisting}[language=Python, caption={Star Pattern Nested Loop}]
# Print star pattern using nested loops
rows = 5

for i in range(1, rows + 1):
    for j in range(1, i + 1):
        print("*", end="")
    print()  # New line after each row
\end{lstlisting}

\begin{center}
\begin{tikzpicture}[node distance=1.5cm, auto]
    \node [gtu state] (start) {Start};
    \node [gtu block, below=of start] (set) {Set rows = 5};
    \node [gtu block, below=of set] (init) {Initialize i = 1};
    \node [gtu decision, below=of init] (cond) {Is i $\le$ rows?};
    \node [gtu block, right=of cond] (print) {Print "*" * i};
    \node [gtu block, below=of print] (inc) {Increment i};
    \node [gtu state, below=of cond, yshift=-1.5cm] (end) {End};

    \path [gtu arrow] (start) -- (set);
    \path [gtu arrow] (set) -- (init);
    \path [gtu arrow] (init) -- (cond);
    \path [gtu arrow] (cond) -- node {Yes} (print);
    \path [gtu arrow] (print) -- (inc);
    \path [gtu arrow] (inc) |- (cond);
    \path [gtu arrow] (cond) -- node {No} (end);
\end{tikzpicture}
\captionof{figure}{Pattern Printing Flowchart}
\end{center}
\end{solutionbox}

\begin{mnemonicbox}
\mnemonic{RISE UP: Row Increases, Stars Expand Upward Progressively}
\end{mnemonicbox}

\questionmarks{3(c)}{7}{Create a user define function to find factorial of the given number.}

\begin{solutionbox}
\begin{lstlisting}[language=Python, caption={Factorial Function}]
# Function to find factorial of a given number
def factorial(n):
    # Check if input is valid
    if not isinstance(n, int) or n < 0:
        return "Invalid input. Please enter a non-negative integer."
    
    # Base case: factorial of 0 or 1 is 1
    if n == 0 or n == 1:
        return 1
    
    # Calculate factorial using iteration
    result = 1
    for i in range(2, n + 1):
        result *= i
    
    return result

# Test the function
number = int(input("Enter a number to find its factorial: "))
print(f"Factorial of {number} is {factorial(number)}")
\end{lstlisting}

\begin{center}
\begin{tikzpicture}[node distance=1.5cm, auto]
    \node [gtu state] (start) {Start};
    \node [gtu block, below=of start] (def) {Define factorial(n)};
    \node [gtu decision, below=of def] (check) {Is n valid?};
    \node [gtu block, right=of check] (err) {Return Error};
    \node [gtu decision, below=of check] (base) {Is n == 0 or 1?};
    \node [gtu block, right=of base] (ret1) {Return 1};
    \node [gtu block, below=of base] (init) {Set result = 1};
    \node [gtu block, below=of init] (loop) {Loop 2 to n};
    \node [gtu block, below=of loop] (calc) {result *= i};
    \node [gtu block, below=of calc] (ret) {Return result};
    \node [gtu state, below=of ret] (end) {End};

    \path [gtu arrow] (start) -- (def);
    \path [gtu arrow] (def) -- (check);
    \path [gtu arrow] (check) -- node {Invalid} (err);
    \path [gtu arrow] (check) -- node {Valid} (base);
    \path [gtu arrow] (base) -- node {Yes} (ret1);
    \path [gtu arrow] (base) -- node {No} (init);
    \path [gtu arrow] (init) -- (loop);
    \path [gtu arrow] (loop) -- (calc);
    \path [gtu arrow] (calc) -- (ret);
    \path [gtu arrow] (ret) -- (end);
    \path [gtu arrow] (ret1) |- (end);
\end{tikzpicture}
\captionof{figure}{Factorial Function Flowchart}
\end{center}

\begin{center}
\captionof{table}{Factorial Examples}
\begin{tabulary}{\linewidth}{|L|L|L|}
\hline
\textbf{Number} & \textbf{Calculation} & \textbf{Factorial} \\ \hline
0 & 0! = 1 & 1 \\ \hline
1 & 1! = 1 & 1 \\ \hline
3 & 3! = 3 $\times$ 2 $\times$ 1 & 6 \\ \hline
5 & 5! = 5 $\times$ 4 $\times$ 3 $\times$ 2 $\times$ 1 & 120 \\ \hline
\end{tabulary}
\end{center}
\end{solutionbox}

\begin{mnemonicbox}
\mnemonic{Multiply Down To One: Multiply all integers down to 1}
\end{mnemonicbox}

\questionmarks{3(a OR)}{3}{Develop a python code to find odd and even numbers from 1 to N using loops.}

\begin{solutionbox}
\begin{lstlisting}[language=Python, caption={Odd and Even Numbers Loop}]
# Program to find odd and even numbers from 1 to N

# Get input from user
N = int(input("Enter the value of N: "))

print("Even numbers from 1 to", N, "are:")
for i in range(1, N + 1):
    if i % 2 == 0:
        print(i, end=" ")

print("\nOdd numbers from 1 to", N, "are:")
for i in range(1, N + 1):
    if i % 2 != 0:
        print(i, end=" ")
\end{lstlisting}

\begin{center}
\captionof{table}{Even and Odd Check}
\begin{tabulary}{\linewidth}{|L|L|L|}
\hline
\textbf{Number} & \textbf{Check} & \textbf{Type} \\ \hline
Even numbers & \code{number \% 2 == 0} & 2, 4, 6, ... \\ \hline
Odd numbers & \code{number \% 2 != 0} & 1, 3, 5, ... \\ \hline
\end{tabulary}
\end{center}
\end{solutionbox}

\begin{mnemonicbox}
\mnemonic{MOD-2: Modulo 2 determines odd or even}
\end{mnemonicbox}

\questionmarks{3(b OR)}{4}{Develop a code to create nested list and display elements.}

\begin{solutionbox}
\begin{lstlisting}[language=Python, caption={Nested List Example}]
# Program to create and display nested list

# Create a nested list
nested_list = [
    [1, 2, 3],
    [4, 5, 6],
    [7, 8, 9]
]

# Display the nested list
print("Nested List:", nested_list)

# Display each element using nested loops
print("\nElements of the nested list:")
for i in range(len(nested_list)):
    for j in range(len(nested_list[i])):
        print(f"nested_list[{i}][{j}] = {nested_list[i][j]}")

# Alternative way to display using enumerate
print("\nUsing enumerate:")
for i, inner_list in enumerate(nested_list):
    for j, value in enumerate(inner_list):
        print(f"Position ({i}, {j}): {value}")
\end{lstlisting}

\begin{center}
\begin{tikzpicture}[node distance=1.5cm, auto]
    \node [gtu state] (root) {Nested List};
    \node [gtu block, below left=of root] (row0) {Row 0};
    \node [gtu block, below=of root] (row1) {Row 1};
    \node [gtu block, below right=of root] (row2) {Row 2};
    
    \node [gtu block, below=of row0, xshift=-0.5cm] (r0i1) {1};
    \node [gtu block, right=0.2cm of r0i1] (r0i2) {2};
    \node [gtu block, right=0.2cm of r0i2] (r0i3) {3};
    
    \node [gtu block, below=of row1, xshift=-0.5cm] (r1i1) {4};
    \node [gtu block, right=0.2cm of r1i1] (r1i2) {5};
    \node [gtu block, right=0.2cm of r1i2] (r1i3) {6};
    
    \node [gtu block, below=of row2, xshift=-0.5cm] (r2i1) {7};
    \node [gtu block, right=0.2cm of r2i1] (r2i2) {8};
    \node [gtu block, right=0.2cm of r2i2] (r2i3) {9};
    
    \path [gtu arrow] (root) -- (row0);
    \path [gtu arrow] (root) -- (row1);
    \path [gtu arrow] (root) -- (row2);
    
    \path [gtu arrow] (row0) -- (r0i2);
    \path [gtu arrow] (row1) -- (r1i2);
    \path [gtu arrow] (row2) -- (r2i2);
\end{tikzpicture}
\captionof{figure}{Nested List Structure}
\end{center}
\end{solutionbox}

\begin{mnemonicbox}
\mnemonic{ROWS COLS: Rows and Columns form the structure}
\end{mnemonicbox}

\questionmarks{3(c OR)}{7}{Explain local and global variables using examples.}

\begin{solutionbox}
\begin{center}
\captionof{table}{Local vs Global Variables}
\begin{tabulary}{\linewidth}{|L|L|L|L|}
\hline
\textbf{Type} & \textbf{Scope} & \textbf{Accessibility} & \textbf{Declaration} \\ \hline
\textbf{Local Variables} & Only within the function where declared & Only inside declaring function & Inside a function \\ \hline
\textbf{Global Variables} & Throughout the program & All functions can access & Outside any function \\ \hline
\end{tabulary}
\end{center}

\begin{lstlisting}[language=Python, caption={Global vs Local Variables}]
# Global variable
total = 0

def add_numbers(a, b):
    # Local variables
    sum_result = a + b
    print(f"Local variable sum_result: {sum_result}")
    
    # Accessing global variable
    print(f"Global variable total before modification: {total}")
    
    # To modify global variable within function
    global total
    total = sum_result
    print(f"Global variable total after modification: {total}")
    
    return sum_result

# Main program
x = 5  # Local to main program
y = 10  # Local to main program

result = add_numbers(x, y)
print(f"Result: {result}")
print(f"Updated global total: {total}")
\end{lstlisting}

\begin{center}
\begin{tikzpicture}[node distance=2cm, auto]
    \node [gtu state] (prog) {Program Scope};
    \node [gtu block, below=of prog] (glob) {Global Variables: total};
    \node [gtu block, left=of glob, xshift=-1cm] (func) {Function Scope: \code{add_numbers}};
    \node [gtu block, below=of func] (local) {Local Variables: \code{sum_result}, \code{a}, \code{b}};
    \node [gtu block, right=of glob, xshift=1cm] (main) {Main Program Vars: \code{x}, \code{y}, \code{result}};
    
    \path [gtu arrow] (prog) -- (glob);
    \path [gtu arrow] (prog) -- (func);
    \path [gtu arrow] (prog) -- (main);
    \path [gtu arrow] (func) -- (local);
    
    \path [dashed] (glob) edge[bend right] node[above] {Access} (func);
\end{tikzpicture}
\captionof{figure}{Variable Scoops}
\end{center}
\end{solutionbox}

\begin{mnemonicbox}
\mnemonic{GLOBAL SEES ALL: Global variables are visible everywhere}
\end{mnemonicbox}

\questionmarks{4(a)}{3}{List out Python standard library mathematical functions.}

\begin{solutionbox}
\begin{center}
\captionof{table}{Python Math Module Functions}
\begin{tabulary}{\linewidth}{|L|L|L|}
\hline
\textbf{Function} & \textbf{Description} & \textbf{Example} \\ \hline
\textbf{abs()} & Returns absolute value & \code{abs(-5)} $\rightarrow$ \code{5} \\ \hline
\textbf{pow()} & Returns x to power y & \code{pow(2, 3)} $\rightarrow$ \code{8} \\ \hline
\textbf{max()} & Returns largest value & \code{max(5, 10, 15)} $\rightarrow$ \code{15} \\ \hline
\textbf{min()} & Returns smallest value & \code{min(5, 10, 15)} $\rightarrow$ \code{5} \\ \hline
\textbf{round()} & Rounds to nearest integer & \code{round(4.6)} $\rightarrow$ \code{5} \\ \hline
\textbf{math.sqrt()} & Square root & \code{math.sqrt(16)} $\rightarrow$ \code{4.0} \\ \hline
\textbf{math.sin()} & Sine function & \code{math.sin(math.pi/2)} $\rightarrow$ \code{1.0} \\ \hline
\end{tabulary}
\end{center}
\end{solutionbox}

\begin{mnemonicbox}
\mnemonic{PEARS Math: Power, Exponents, Arithmetic, Roots, Sine functions in Math}
\end{mnemonicbox}

\questionmarks{4(b)}{4}{Explain Module in python with example python code of it.}

\begin{solutionbox}
\textbf{Module}: A module in Python is a file containing Python definitions and statements. The file name is the module name with the suffix .py added.

\begin{lstlisting}[language=Python, caption={Module Usage Example}]
# Example of using math module
import math

# Using mathematical functions from math module
radius = 5
area = math.pi * math.pow(radius, 2)
print(f"Area of circle with radius {radius} is {area:.2f}")

# Using different import techniques
from math import sqrt, sin
angle = math.pi / 4
print(f"Square root of 25 is {sqrt(25)}")
print(f"Sine of {angle} radians is {sin(angle):.4f}")

# Importing with alias
import random as rnd
random_number = rnd.randint(1, 100)
print(f"Random number between 1 and 100: {random_number}")
\end{lstlisting}

\begin{center}
\captionof{table}{Module Import Techniques}
\begin{tabulary}{\linewidth}{|L|L|L|}
\hline
\textbf{Method} & \textbf{Syntax} & \textbf{Example} \\ \hline
\textbf{Import entire module} & \code{import module\_name} & \code{import math} \\ \hline
\textbf{Import specific items} & \code{from module\_name import item1, item2} & \code{from math import sqrt, sin} \\ \hline
\textbf{Import with alias} & \code{import module\_name as alias} & \code{import random as rnd} \\ \hline
\end{tabulary}
\end{center}
\end{solutionbox}

\begin{mnemonicbox}
\mnemonic{CODE-LIB: Code Libraries for reuse}
\end{mnemonicbox}

\questionmarks{4(c)}{7}{Write a Program that determines whether a given number is an 'Armstrong number' or a palindrome using a user-defined function.}

\begin{solutionbox}
\begin{lstlisting}[language=Python, caption={Armstrong and Palindrome Check}]
# Function to check if a number is an Armstrong number
def is_armstrong(num):
    # Convert number to string to count digits
    num_str = str(num)
    n = len(num_str)
    
    # Calculate sum of each digit raised to power of number of digits
    armstrong_sum = 0
    for digit in num_str:
        armstrong_sum += int(digit) ** n
    
    # Check if sum equals the original number
    return armstrong_sum == num

# Function to check if a number is a palindrome
def is_palindrome(num):
    # Convert number to string and check if it reads the same forwards and backwards
    num_str = str(num)
    return num_str == num_str[::-1]

# Main program
number = int(input("Enter a number: "))

# Check if the number is an Armstrong number
if is_armstrong(number):
    print(f"{number} is an Armstrong number")
else:
    print(f"{number} is not an Armstrong number")

# Check if the number is a palindrome
if is_palindrome(number):
    print(f"{number} is a palindrome")
else:
    print(f"{number} is not a palindrome")
\end{lstlisting}

\begin{center}
\captionof{table}{Examples}
\begin{tabulary}{\linewidth}{|L|L|L|}
\hline
\textbf{Number} & \textbf{Armstrong Check} & \textbf{Palindrome Check} \\ \hline
153 & $1^3 + 5^3 + 3^3 = 1 + 125 + 27 = 153$ \checkmark & $153 \neq 351$ \textbf{x} \\ \hline
121 & $1^3 + 2^3 + 1^3 = 1 + 8 + 1 = 10 \neq 121$ \textbf{x} & $121 = 121$ \checkmark \\ \hline
1634 & $1^4 + 6^4 + 3^4 + 4^4 = 1 + 1296 + 81 + 256 = 1634$ \checkmark & $1634 \neq 4361$ \textbf{x} \\ \hline
\end{tabulary}
\end{center}

\begin{center}
\begin{tikzpicture}[node distance=1.5cm, auto]
    \node [gtu state] (start) {Start};
    \node [gtu block, below=of start] (input) {Input number};
    \node [gtu decision, below=of input] (checkA) {Check Armstrong};
    \node [gtu decision, below=of checkA] (checkP) {Check Palindrome};
    
    \node [gtu block, right=of checkA] (isA) {Print Is Armstrong};
    \node [gtu block, left=of checkA] (notA) {Print Not Armstrong};
    
    \node [gtu block, right=of checkP] (isP) {Print Is Palindrome};
    \node [gtu block, left=of checkP] (notP) {Print Not Palindrome};
    
    \node [gtu state, below=of checkP, yshift=-1.5cm] (end) {End};

    \path [gtu arrow] (start) -- (input);
    \path [gtu arrow] (input) -- (checkA);
    \path [gtu arrow] (checkA) -- node {Yes} (isA);
    \path [gtu arrow] (checkA) -- node {No} (notA);
    \path [gtu arrow] (isA) |- (checkP);
    \path [gtu arrow] (notA) |- (checkP);
    
    \path [gtu arrow] (checkP) -- node {Yes} (isP);
    \path [gtu arrow] (checkP) -- node {No} (notP);
    
    \path [gtu arrow] (isP) |- (end);
    \path [gtu arrow] (notP) |- (end);
\end{tikzpicture}
\captionof{figure}{Validation Flowchart}
\end{center}
\end{solutionbox}

\begin{mnemonicbox}
\mnemonic{SAME SUM: SAME forwards and backwards for palindrome, SUM of powered digits for Armstrong}
\end{mnemonicbox}

\questionmarks{4(a OR)}{3}{Explain built in functions in python.}

\begin{solutionbox}
\textbf{Built-in Functions}: These are functions that are part of Python's standard library and available without importing any module.

\begin{center}
\captionof{table}{Common Python Built-in Functions}
\begin{tabulary}{\linewidth}{|L|L|L|}
\hline
\textbf{Function} & \textbf{Purpose} & \textbf{Example} \\ \hline
\textbf{print()} & Display output & \code{print("Hello")} \\ \hline
\textbf{input()} & Get user input & \code{name = input("Name: ")} \\ \hline
\textbf{len()} & Return object length & \code{len([1, 2, 3])} $\rightarrow$ \code{3} \\ \hline
\textbf{type()} & Return object type & \code{type(5)} $\rightarrow$ \code{<class 'int'>} \\ \hline
\textbf{int(), float(), str()} & Convert to specific type & \code{int("5")} $\rightarrow$ \code{5} \\ \hline
\textbf{range()} & Generate sequence & \code{list(range(3))} $\rightarrow$ \code{[0, 1, 2]} \\ \hline
\textbf{sum()} & Calculate sum & \code{sum([1, 2, 3])} $\rightarrow$ \code{6} \\ \hline
\end{tabulary}
\end{center}
\end{solutionbox}

\begin{mnemonicbox}
\mnemonic{PITS LCR: Print, Input, Type, Sum, Len, Convert, Range}
\end{mnemonicbox}

\questionmarks{4(b OR)}{4}{Describe python math module by giving one python code example.}

\begin{solutionbox}
\textbf{Python Math Module}: The math module provides access to mathematical functions defined by the C standard.

\begin{lstlisting}[language=Python, caption={Math Module Example}]
# Example using math module
import math

# Basic constants
print(f"Value of pi: {math.pi}")
print(f"Value of e: {math.e}")

# Trigonometric functions (argument in radians)
angle = math.pi / 3  # 60 degrees
print(f"Sine of {angle:.2f} radians: {math.sin(angle):.4f}")
print(f"Cosine of {angle:.2f} radians: {math.cos(angle):.4f}")
print(f"Tangent of {angle:.2f} radians: {math.tan(angle):.4f}")

# Logarithmic and exponential functions
x = 10
print(f"Natural logarithm of {x}: {math.log(x):.4f}")
print(f"Logarithm base 10 of {x}: {math.log10(x):.4f}")
print(f"e raised to power {x}: {math.exp(x):.4f}")

# Other functions
print(f"Square root of 25: {math.sqrt(25)}")
print(f"Ceiling of 4.3: {math.ceil(4.3)}")
print(f"Floor of 4.7: {math.floor(4.7)}")
\end{lstlisting}

\begin{center}
\captionof{table}{Math Module Categories}
\begin{tabulary}{\linewidth}{|L|L|}
\hline
\textbf{Category} & \textbf{Functions} \\ \hline
\textbf{Constants} & \code{math.pi}, \code{math.e} \\ \hline
\textbf{Trigonometric} & \code{sin()}, \code{cos()}, \code{tan()} \\ \hline
\textbf{Logarithmic} & \code{log()}, \code{log10()}, \code{exp()} \\ \hline
\textbf{Numeric} & \code{sqrt()}, \code{ceil()}, \code{floor()} \\ \hline
\end{tabulary}
\end{center}
\end{solutionbox}

\begin{mnemonicbox}
\mnemonic{PENT: Pi/constants, Exponents, Numbers, Trigonometry}
\end{mnemonicbox}

\questionmarks{4(c OR)}{7}{Explain concept of scope of variable in Python and Apply global and local variable concepts in python program.}

\begin{solutionbox}
\textbf{Scope of Variables in Python}: The scope of a variable determines where in the program a variable is accessible or visible.

\begin{center}
\captionof{table}{Variable Scope Types}
\begin{tabulary}{\linewidth}{|L|L|L|}
\hline
\textbf{Scope} & \textbf{Description} & \textbf{Access} \\ \hline
\textbf{Local} & Variables defined inside a function & Only within the function \\ \hline
\textbf{Global} & Variables defined at the top level & Throughout the program \\ \hline
\textbf{Enclosing} & Variables in outer function of nested functions & In the outer and inner function \\ \hline
\textbf{Built-in} & Pre-defined variables in Python & Throughout the program \\ \hline
\end{tabulary}
\end{center}

\begin{lstlisting}[language=Python, caption={Variable Scope Example}]
# Variable scope demonstration

# Global variable
count = 0

def outer_function():
    # Enclosing scope variable
    name = "Python"
    
    def inner_function():
        # Local variable
        age = 30
        # Accessing global variable
        global count
        count += 1
        # Accessing enclosing variable
        print(f"Inside inner_function: name is {name}")
        print(f"Inside inner_function: age is {age}")
        print(f"Inside inner_function: count is {count}")
    
    # Local variable to outer_function
    language = "Programming"
    print(f"Inside outer_function: name is {name}")
    print(f"Inside outer_function: language is {language}")
    print(f"Inside outer_function: count is {count}")
    
    # Call inner function
    inner_function()

# Main program
print(f"Global scope: count is {count}")
outer_function()
print(f"Global scope after function call: count is {count}")
\end{lstlisting}

\begin{center}
\begin{tikzpicture}[node distance=2cm, auto]
    \node [gtu state] (glob) {Global Scope: count};
    \node [gtu block, below=of glob] (outer) {Outer Function};
    \node [gtu decision, below=of outer] (enc) {Enclosing: name};
    \node [gtu block, below=of enc] (inner) {Inner Function};
    \node [gtu decision, below=of inner] (local) {Local: age};
    
    \path [gtu arrow] (glob) -- (outer);
    \path [gtu arrow] (outer) -- (enc);
    \path [gtu arrow] (enc) -- (inner);
    \path [gtu arrow] (inner) -- (local);
    
    \path [dashed] (glob) edge[bend right=60] node[left] {Access} (inner);
    \path [dashed] (enc) edge[bend right=40] node[left] {Access} (inner);
\end{tikzpicture}
\captionof{figure}{Rule of LEGB}
\end{center}
\end{solutionbox}

\begin{mnemonicbox}
\mnemonic{LEGB: Local, Enclosing, Global, Built-in - order of scope lookup}
\end{mnemonicbox}

\questionmarks{5(a)}{3}{Develop a python program to swap two elements in given list}

\begin{solutionbox}
\begin{lstlisting}[language=Python, caption={Swapping List Elements}]
# Program to swap two elements in a list

# Create a list
my_list = [10, 20, 30, 40, 50]
print("Original list:", my_list)

# Get positions to swap
pos1 = int(input("Enter first position (index starts from 0): "))
pos2 = int(input("Enter second position (index starts from 0): "))

# Swap elements using a temporary variable
if 0 <= pos1 < len(my_list) and 0 <= pos2 < len(my_list):
    # Swapping
    temp = my_list[pos1]
    my_list[pos1] = my_list[pos2]
    my_list[pos2] = temp
    
    print(f"List after swapping elements at positions {pos1} and {pos2}:", my_list)
else:
    print("Invalid positions! Positions should be within list range.")
\end{lstlisting}

\textbf{Alternative method}:

\begin{lstlisting}[language=Python, caption={Pythonic Swap}]
# Swap using Python's tuple unpacking (more pythonic)
if 0 <= pos1 < len(my_list) and 0 <= pos2 < len(my_list):
    my_list[pos1], my_list[pos2] = my_list[pos2], my_list[pos1]
    print(f"List after swapping elements at positions {pos1} and {pos2}:", my_list)
\end{lstlisting}

\begin{center}
\captionof{table}{Swapping Methods}
\begin{tabulary}{\linewidth}{|L|L|}
\hline
\textbf{Method} & \textbf{Code} \\ \hline
\textbf{Using temp variable} & \code{temp = a; a = b; b = temp} \\ \hline
\textbf{Python tuple unpacking} & \code{a, b = b, a} \\ \hline
\end{tabulary}
\end{center}
\end{solutionbox}

\begin{mnemonicbox}
\mnemonic{TEMP SWAP: Temporary variable helps safe swapping}
\end{mnemonicbox}

\questionmarks{5(b)}{4}{Explain nested list by giving example.}

\begin{solutionbox}
\textbf{Nested List}: A nested list is a list that contains other lists as its elements, creating a multi-dimensional data structure.

\begin{lstlisting}[language=Python, caption={Nested List Operations}]
# Creating a nested list (3x3 matrix)
matrix = [
    [1, 2, 3],
    [4, 5, 6],
    [7, 8, 9]
]

# Accessing elements
print("Complete matrix:", matrix)
print("First row:", matrix[0])
print("Element at row 1, column 2:", matrix[0][1])  # Output: 2

# Modifying elements
matrix[1][1] = 50
print("Matrix after modification:", matrix)

# Iterating through a nested list
print("\nPrinting the matrix:")
for row in matrix:
    for element in row:
        print(element, end=" ")
    print()  # New line after each row
\end{lstlisting}

\begin{center}
\begin{tikzpicture}[node distance=1.5cm, auto]
    \node [gtu state] (matrix) {matrix};
    \node [gtu block, below left=of matrix] (row0) {Row 0};
    \node [gtu block, below=of matrix] (row1) {Row 1};
    \node [gtu block, below right=of matrix] (row2) {Row 2};
    
    \node [gtu block, below=of row1, yshift=-0.5cm] (val) {Values: 4, 50, 6};
    \path [gtu arrow] (matrix) -- (row0);
    \path [gtu arrow] (matrix) -- (row1);
    \path [gtu arrow] (matrix) -- (row2);
    \path [gtu arrow] (row1) -- (val);
\end{tikzpicture}
\captionof{figure}{Nested List Structure}
\end{center}

\begin{center}
\captionof{table}{Nested List Operations}
\begin{tabulary}{\linewidth}{|L|L|L|}
\hline
\textbf{Operation} & \textbf{Syntax} & \textbf{Example} \\ \hline
\textbf{Access element} & \code{list[row][col]} & \code{matrix[0][1]} \\ \hline
\textbf{Modify element} & \code{list[row][col] = new\_value} & \code{matrix[1][1] = 50} \\ \hline
\textbf{Add new row} & \code{list.append([...])} & \code{matrix.append([10, 11, 12])} \\ \hline
\end{tabulary}
\end{center}
\end{solutionbox}

\begin{mnemonicbox}
\mnemonic{MARS: Matrix Access with Row and column Structure}
\end{mnemonicbox}

\questionmarks{5(c)}{7}{Explain string operations with examples.}

\begin{solutionbox}
\begin{center}
\captionof{table}{String Operations in Python}
\begin{tabulary}{\linewidth}{|L|L|L|}
\hline
\textbf{Operation} & \textbf{Description} & \textbf{Example} \\ \hline
\textbf{Concatenation} & Joining strings & \code{"Hello" + " World"} $\rightarrow$ \code{"Hello World"} \\ \hline
\textbf{Repetition} & Repeating strings & \code{"Python" * 3} $\rightarrow$ \code{"PythonPythonPython"} \\ \hline
\textbf{Slicing} & Extract substring & \code{"Python"[1:4]} $\rightarrow$ \code{"yth"} \\ \hline
\textbf{Indexing} & Access character & \code{"Python"[0]} $\rightarrow$ \code{"P"} \\ \hline
\textbf{Length} & Count characters & \code{len("Python")} $\rightarrow$ \code{6} \\ \hline
\textbf{Membership} & Check if present & \code{"P" in "Python"} $\rightarrow$ \code{True} \\ \hline
\textbf{Comparison} & Compare strings & \code{"apple" < "banana"} $\rightarrow$ \code{True} \\ \hline
\end{tabulary}
\end{center}

\begin{lstlisting}[language=Python, caption={String Operations Example}]
# String operations demonstration
text = "Python Programming"

# Indexing
print("First character:", text[0])
print("Last character:", text[-1])

# Slicing
print("First word:", text[:6])
print("Second word:", text[7:])
print("Middle characters:", text[3:10])
print("Reverse:", text[::-1])

# String methods
print("Uppercase:", text.upper())
print("Lowercase:", text.lower())
print("Replace 'P' with 'J':", text.replace("P", "J"))
print("Split by space:", text.split())
print("Count 'm':", text.count('m'))
print("Find 'gram':", text.find("gram"))

# Check operations
print("Is alphanumeric?", text.isalnum())
print("Starts with 'Py'?", text.startswith("Py"))
print("Ends with 'ing'?", text.endswith("ing"))
\end{lstlisting}

\begin{center}
\begin{tikzpicture}[node distance=1.5cm, auto]
    \node [gtu state] (str) {"Python Programming"};
    \node [gtu block, below left=of str] (idx) {Indexing};
    \node [gtu block, below=of str] (slice) {Slicing};
    \node [gtu block, below right=of str] (meth) {Methods};
    
    \node [gtu block, below=of idx] (idx_ex) {P (0), g (-1)};
    \node [gtu block, below=of slice] (slice_ex) {Python (0:6)};
    \node [gtu block, below=of meth] (meth_ex) {upper(), split()};
    
    \path [gtu arrow] (str) -- (idx);
    \path [gtu arrow] (str) -- (slice);
    \path [gtu arrow] (str) -- (meth);
    \path [gtu arrow] (idx) -- (idx_ex);
    \path [gtu arrow] (slice) -- (slice_ex);
    \path [gtu arrow] (meth) -- (meth_ex);
\end{tikzpicture}
\captionof{figure}{String Operations}
\end{center}
\end{solutionbox}

\begin{mnemonicbox}
\mnemonic{SCREAM: Slice, Concat, Replace, Extract, Access, Methods}
\end{mnemonicbox}

\questionmarks{5(a OR)}{3}{Develop a python program to find sum of all elements in given list}

\begin{solutionbox}
\begin{lstlisting}[language=Python, caption={Sum of List Elements}]
# Program to find sum of all elements in a list

# Method 1: Using built-in sum() function
def sum_list_builtin(numbers):
    return sum(numbers)

# Method 2: Using a loop
def sum_list_loop(numbers):
    total = 0
    for num in numbers:
        total += num
    return total

# Create a sample list
my_list = [10, 20, 30, 40, 50]
print("List:", my_list)

# Calculate sum using built-in function
print("Sum using built-in function:", sum_list_builtin(my_list))

# Calculate sum using loop
print("Sum using loop:", sum_list_loop(my_list))
\end{lstlisting}

\begin{center}
\captionof{table}{Sum Methods Comparison}
\begin{tabulary}{\linewidth}{|L|L|}
\hline
\textbf{Method} & \textbf{Advantage} \\ \hline
\textbf{Built-in sum()} & Simple, efficient, fast \\ \hline
\textbf{Loop approach} & Works for custom summing logic \\ \hline
\end{tabulary}
\end{center}
\end{solutionbox}

\begin{mnemonicbox}
\mnemonic{ADD ALL: Add All elements in sequence}
\end{mnemonicbox}

\questionmarks{5(b OR)}{4}{Explain indexing and slicing operations in python list}

\begin{solutionbox}
\begin{center}
\captionof{table}{Indexing and Slicing Operations}
\begin{tabulary}{\linewidth}{|L|L|L|L|}
\hline
\textbf{Operation} & \textbf{Syntax} & \textbf{Description} & \textbf{Example} \\ \hline
\textbf{Positive Indexing} & \code{list[i]} & Access item at position i & \code{fruits[0]} \\ \hline
\textbf{Negative Indexing} & \code{list[-i]} & Access item from end & \code{fruits[-1]} \\ \hline
\textbf{Basic Slicing} & \code{list[s:e]} & Items from start to end-1 & \code{fruits[1:3]} \\ \hline
\textbf{Slice with Step} & \code{list[s:e:st]} & Items with interval of step & \code{nums[1:6:2]} \\ \hline
\textbf{Reverse} & \code{list[::-1]} & Reverse the list & \code{fruits[::-1]} \\ \hline
\end{tabulary}
\end{center}

\begin{lstlisting}[language=Python, caption={Indexing and Slicing Demo}]
# Indexing and slicing demonstration
fruits = ["apple", "banana", "cherry", "date", "elderberry", "fig"]
print("Original list:", fruits)

# Indexing
print("\nIndexing examples:")
print("First item:", fruits[0])  # apple
print("Last item:", fruits[-1])  # fig

# Slicing
print("\nSlicing examples:")
print("First three items:", fruits[:3])
print("Last three items:", fruits[-3:])
print("Middle items:", fruits[2:4])
print("Every second item:", fruits[::2])
print("Reversed list:", fruits[::-1])
\end{lstlisting}

\begin{center}
\begin{tikzpicture}[node distance=1.5cm, auto]
    \node [gtu state] (list) {List: fruits};
    \node [gtu block, below left=of list] (idx) {Indexing};
    \node [gtu block, below right=of list] (slice) {Slicing};
    
    \node [gtu block, below=of idx] (idx_ex) {fruits[0], fruits[-1]};
    \node [gtu block, below=of slice] (slice_ex) {fruits[1:3], fruits[::-1]};
    
    \path [gtu arrow] (list) -- (idx);
    \path [gtu arrow] (list) -- (slice);
    \path [gtu arrow] (idx) -- (idx_ex);
    \path [gtu arrow] (slice) -- (slice_ex);
\end{tikzpicture}
\captionof{figure}{Indexing and Slicing}
\end{center}
\end{solutionbox}

\begin{mnemonicbox}
\mnemonic{START-END-STEP: Slicing syntax: [start:end:step]}
\end{mnemonicbox}

\questionmarks{5(c OR)}{7}{Explain tuple in brief with necessary example.}

\begin{solutionbox}
\textbf{Tuple}: A tuple is an ordered, immutable collection of elements. Once created, the elements cannot be changed.

\begin{center}
\captionof{table}{Tuple vs List}
\begin{tabulary}{\linewidth}{|L|L|L|}
\hline
\textbf{Feature} & \textbf{Tuple} & \textbf{List} \\ \hline
\textbf{Syntax} & \code{(item1, item2)} & \code{[item1, item2]} \\ \hline
\textbf{Mutability} & Immutable (cannot change) & Mutable (can change) \\ \hline
\textbf{Performance} & Faster & Slower \\ \hline
\textbf{Use Case} & Fixed data, keys & Dynamic data \\ \hline
\end{tabulary}
\end{center}

\begin{lstlisting}[language=Python, caption={Tuple Example}]
# Creating tuples
empty_tuple = ()
single_item_tuple = (1,)  # Comma is necessary for single item
mixed_tuple = (1, "Hello", 3.14, True)
nested_tuple = (1, 2, (3, 4), 5)

# Accessing tuple elements
print("First item:", mixed_tuple[0])  # 1
print("Nested tuple element:", nested_tuple[2][0])  # 3

# Tuple operations
combined_tuple = mixed_tuple + nested_tuple
print("Combined tuple:", combined_tuple)

# This will cause error as tuples are immutable
# mixed_tuple[0] = 100  # TypeError
\end{lstlisting}

\begin{center}
\begin{tikzpicture}[node distance=1.5cm, auto]
    \node [gtu state] (tuple) {Tuple: (1, 2, 3)};
    \node [gtu block, below left=of tuple] (access) {Access};
    \node [gtu block, below=of tuple] (props) {Properties};
    \node [gtu block, below right=of tuple] (ops) {Operations};
    
    \node [gtu block, below=of access] (acc_ex) {tuple[0] $\rightarrow$ 1};
    \node [gtu block, below=of props] (prop_ex) {Immutable};
    \node [gtu block, below=of ops] (ops_ex) {Concatenation (+)};
    
    \path [gtu arrow] (tuple) -- (access);
    \path [gtu arrow] (tuple) -- (props);
    \path [gtu arrow] (tuple) -- (ops);
    
    \path [gtu arrow] (access) -- (acc_ex);
    \path [gtu arrow] (props) -- (prop_ex);
    \path [gtu arrow] (ops) -- (ops_ex);
\end{tikzpicture}
\captionof{figure}{Tuple Concepts}
\end{center}
\end{solutionbox}

\begin{mnemonicbox}
\mnemonic{IPAC: Immutable, Parentheses, Access only, Cannot modify}
\end{mnemonicbox}

\end{document}
