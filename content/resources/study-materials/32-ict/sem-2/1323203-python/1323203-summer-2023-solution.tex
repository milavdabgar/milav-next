\documentclass{article}

% content/resources/templates/preamble.tex
\usepackage[margin=0.6in]{geometry}
\author{Milav Dabgar}
\usepackage{amsmath,amssymb,amsthm}
\usepackage{booktabs}
\usepackage{multirow}
\usepackage{xcolor}
\usepackage{tcolorbox}
\tcbuselibrary{breakable,skins}
\usepackage[colorlinks=true,linkcolor=blue]{hyperref}
\usepackage{titlesec}
\usepackage{enumitem}
\usepackage{tikz}
\usepackage{pgfplots}
\usepackage{circuitikz}
\usepackage[version=4]{mhchem}
\usepackage{longtable}
\usepackage{array}
\usepackage{float}
\usepackage{caption}
\usepackage{listings}

\lstset{
  basicstyle=\small\ttfamily,
  breaklines=true,
  breakatwhitespace=false,
  postbreak=\mbox{\textcolor{red}{$\hookrightarrow$}\space},
  float=false,
  numbers=left,
  numberstyle=\tiny\color{gray},
  numbersep=10pt,
  xleftmargin=2em,
  keywordstyle=\color{blue},
  commentstyle=\color{green!60!black},
  stringstyle=\color{purple},
  backgroundcolor=\color{gray!5},
  showstringspaces=false,
  tabsize=2,
  captionpos=b,
  keepspaces=true,
  columns=flexible
}

\pgfplotsset{compat=1.18}
\usetikzlibrary{shapes,arrows,positioning,calc,patterns,decorations.pathmorphing,decorations.markings,arrows.meta}

% Color scheme
\definecolor{headcolor}{RGB}{0,102,204}
\definecolor{keycolor}{RGB}{220,20,60}
\definecolor{solutioncolor}{RGB}{34,139,34}
\definecolor{mnemoniccolor}{RGB}{148,0,211}
\definecolor{codecolor}{RGB}{0,0,100}

% Spacing
\setlength{\parskip}{3pt}
\setlist[itemize]{nosep}
\setlist[enumerate]{nosep}

% Title formatting
\titleformat{\section}{\Large\bfseries\color{headcolor}}{\thesection}{1em}{}
\titleformat{\subsection}{\large\bfseries\color{headcolor}}{\thesubsection}{1em}{}

% Pandoc tightlist compatibility
\providecommand{\tightlist}{%
  \setlength{\itemsep}{0pt}\setlength{\parskip}{0pt}}

% Pandoc longtable compatibility
\newcounter{none}
\def\thenone{}


% content/resources/templates/english-boxes.tex

% Custom environments
\newtcolorbox{solutionbox}{
 breakable,
 enhanced,
 colback=solutioncolor!5!white,
 colframe=solutioncolor!75!black,
 fonttitle=\bfseries,
 title=Solution
}

\newtcolorbox{solutionboxnobreak}{
 colback=solutioncolor!5!white,
 colframe=solutioncolor!75!black,
 fonttitle=\bfseries,
 title=Solution
}

\newtcolorbox{keyformula}{
 breakable,
 enhanced,
 colback=keycolor!5!white,
 colframe=keycolor!75!black,
 fonttitle=\bfseries,
 title=Key Formula
}

\newtcolorbox{mnemonicboxenv}{
 breakable,
 enhanced,
 colback=mnemoniccolor!5!white,
 colframe=mnemoniccolor!75!black,
 fonttitle=\bfseries,
 title=Mnemonic
}

\newcommand{\mnemonicbox}[1]{%
  \begin{mnemonicboxenv}
    #1
  \end{mnemonicboxenv}
}


% Custom commands for GTU solutions
% This file defines semantic commands for consistent formatting

% Question command with automatic formatting
\newcommand{\question}[2]{%
  \section*{Question #1}%
  \textbf{#2}%
}

% OR question variant
\newcommand{\questionor}[2]{%
  \section*{Question #1 OR}%
  \textbf{#2}%
}

% Proper table environment with caption
\newenvironment{answertable}[1]{%
  \begin{table}[htbp]
  \centering
  \caption{#1}
}{%
  \end{table}
}

% Proper figure environment for diagrams
\newenvironment{answerdiagram}[1]{%
  \begin{figure}[htbp]
  \centering
  \caption{#1}
}{%
  \end{figure}
}

% Semantic markup for key terms
\newcommand{\keyword}[1]{\textbf{#1}}
\newcommand{\code}[1]{\texttt{#1}}
\newcommand{\classname}[1]{\texttt{#1}}
\newcommand{\methodname}[1]{\texttt{#1}}

% Proper quotation marks
\newcommand{\mnemonic}[1]{``#1''}


\title{Python Programming (1323203) - Summer 2023 Solution}
\date{August 09, 2023}

\begin{document}
\maketitle

\questionmarks{1(a)}{3}{Define algorithm. What are the advantages of Algorithm?}

\begin{solutionbox}
An algorithm is a step-by-step procedure or a set of rules to solve a specific problem in a finite sequence of steps.

\textbf{Advantages of Algorithm:}
\begin{itemize}
    \item \keyword{Clarity}: Provides clear, unambiguous instructions
    \item \keyword{Efficiency}: Helps in optimizing time and resources
    \item \keyword{Reusability}: Can be used repeatedly for similar problems
    \item \keyword{Verification}: Easy to test and debug before implementation
    \item \keyword{Communication}: Acts as a blueprint to communicate the solution
\end{itemize}

\begin{mnemonicbox}
"CERVC" (Clarity, Efficiency, Reusability, Verification, Communication)
\end{mnemonicbox}
\end{solutionbox}

\questionmarks{1(b)}{4}{What are the rules for problem solving using flowchart? Design a flowchart to find simple interest.}

\begin{solutionbox}
Rules for problem solving using flowchart:
\begin{itemize}
    \item \textbf{Proper symbols}: Use standard symbols for different operations
    \item \textbf{Direction flow}: Always maintain clear top-to-bottom, left-to-right flow
    \item \textbf{Single entry/exit}: Have a clear start and end point
    \item \textbf{Clarity}: Keep steps clear and concise
    \item \textbf{Consistency}: Maintain consistent level of detail
\end{itemize}

\begin{answerdiagram}{Flowchart for Simple Interest Calculation}
\begin{center}
\begin{tikzpicture}[gtu flow]
    \node[gtu start] (start) {Start};
    \node[gtu input, below of=start] (input) {Input Principal P, Rate R, Time T};
    \node[gtu process, below of=input] (calc) {SI = P * R * T / 100};
    \node[gtu output, below of=calc] (output) {Output SI};
    \node[gtu stop, below of=output] (stop) {End};

    \draw[gtu arrow] (start) -- (input);
    \draw[gtu arrow] (input) -- (calc);
    \draw[gtu arrow] (calc) -- (output);
    \draw[gtu arrow] (output) -- (stop);
\end{tikzpicture}
\end{center}
\end{answerdiagram}

\begin{mnemonicbox}
"PDRSC" (Proper symbols, Direction flow, Required entry/exit, Simplicity, Consistency)
\end{mnemonicbox}
\end{solutionbox}

\questionmarks{1(c)}{7}{List out assignment operator in python and build a python code to demonstrate an operation of any three assignment operators.}

\begin{solutionbox}
Python assignment operators:

\begin{answertable}{Assignment Operators}
\begin{tabulary}{\linewidth}{|L|L|L|}
\hline
\textbf{Operator} & \textbf{Example} & \textbf{Equivalent To} \\
\hline
= & x = 5 & x = 5 \\
+= & x += 5 & x = x + 5 \\
-= & x -= 5 & x = x - 5 \\
*= & x *= 5 & x = x * 5 \\
/= & x /= 5 & x = x / 5 \\
\%= & x \%= 5 & x = x \% 5 \\
//= & x //= 5 & x = x // 5 \\
**= & x **= 5 & x = x ** 5 \\
\&= & x \&= 5 & x = x \& 5 \\
|= & x |= 5 & x = x | 5 \\
\^{}= & x \^{}= 5 & x = x \^{} 5 \\
>>= & x >>= 5 & x = x >> 5 \\
<<= & x <<= 5 & x = x << 5 \\
\hline
\end{tabulary}
\end{answertable}

\textbf{Code demonstrating assignment operators:}
\begin{lstlisting}[language=Python]
# Demonstrating Assignment Operators
num = 10
print("Initial value:", num)

# Using += operator
num += 5
print("After += 5:", num)  # Output: 15

# Using -= operator
num -= 3
print("After -= 3:", num)  # Output: 12

# Using *= operator
num *= 2
print("After *= 2:", num)  # Output: 24
\end{lstlisting}

\begin{mnemonicbox}
"VALUE" (Variable Assignment is Like Updating Existing values)
\end{mnemonicbox}
\end{solutionbox}

\orquestionmarks{1(c)}{7}{List out data types in python and Develop a Program to identify any three data types in python.}

\begin{solutionbox}
Python data types:

\begin{answertable}{Python Data Types}
\begin{tabulary}{\linewidth}{|L|L|L|}
\hline
\textbf{Data Type} & \textbf{Description} & \textbf{Example} \\
\hline
int & Integer (whole numbers) & 42 \\
float & Floating point (decimal) & 3.14 \\
str & String (text) & "Hello" \\
bool & Boolean (True/False) & True \\
list & Ordered, mutable collection & [1, 2, 3] \\
tuple & Ordered, immutable collection & (1, 2, 3) \\
set & Unordered collection of unique items & \{1, 2, 3\} \\
dict & Key-value pairs & \{"name": "John"\} \\
complex & Complex numbers & 2+3j \\
NoneType & Represents None & None \\
\hline
\end{tabulary}
\end{answertable}

\textbf{Code to identify three data types:}
\begin{lstlisting}[language=Python]
# Program to identify data types
def identify_data_type(value):
    data_type = type(value).__name__
    print(f"Value: {value}")
    print(f"Data Type: {data_type}")
    print("-" * 20)

# Testing with 3 different data types
identify_data_type(42)            # Integer
identify_data_type(3.14)          # Float
identify_data_type("Hello World") # String

# Output:
# Value: 42
# Data Type: int
# --------------------
# Value: 3.14
# Data Type: float
# --------------------
# Value: Hello World
# Data Type: str
# --------------------
\end{lstlisting}

\begin{mnemonicbox}
"TYPE-ID" (Tell Your Python Elements - Identify Data)
\end{mnemonicbox}
\end{solutionbox}

\questionmarks{2(a)}{3}{Define pseudocode. Write pseudocode to find smallest of two number.}

\begin{solutionbox}
Pseudocode is a high-level description of an algorithm that uses structural conventions of a programming language but is designed for human reading rather than machine reading.

\textbf{Pseudocode to find smallest of two numbers:}
\begin{lstlisting}
BEGIN
    INPUT first_number, second_number
    IF first_number < second_number THEN
        smallest = first_number
    ELSE
        smallest = second_number
    END IF
    OUTPUT smallest
END
\end{lstlisting}

\begin{mnemonicbox}
"RISE" (Read Input, Select smallest, Echo result)
\end{mnemonicbox}
\end{solutionbox}

\questionmarks{2(b)}{4}{Develop a python code to read three numbers from the user and find the average of the numbers.}

\begin{solutionbox}
\begin{lstlisting}[language=Python]
# Program to calculate average of three numbers
# Input three numbers from user
num1 = float(input("Enter first number: "))
num2 = float(input("Enter second number: "))
num3 = float(input("Enter third number: "))

# Calculate the average
average = (num1 + num2 + num3) / 3

# Display the result
print(f"The average of {num1}, {num2}, and {num3} is: {average}")
\end{lstlisting}

\begin{answerdiagram}{Flowchart for Average Calculation}
\begin{center}
\begin{tikzpicture}[gtu flow]
    \node[gtu start] (start) {Start};
    \node[gtu input, below of=start] (input) {Input num1, num2, num3};
    \node[gtu process, below of=input] (process) {average = (num1 + num2 + num3) / 3};
    \node[gtu output, below of=process] (output) {Output average};
    \node[gtu stop, below of=output] (stop) {End};

    \draw[gtu arrow] (start) -- (input);
    \draw[gtu arrow] (input) -- (process);
    \draw[gtu arrow] (process) -- (output);
    \draw[gtu arrow] (output) -- (stop);
\end{tikzpicture}
\end{center}
\end{answerdiagram}

\begin{mnemonicbox}
"I-ADD-D" (Input three, ADD them up, Divide by 3)
\end{mnemonicbox}
\end{solutionbox}

\questionmarks{2(c)}{7}{Write a python code to show whether the entered number is prime or not.}

\begin{solutionbox}
\begin{lstlisting}[language=Python]
# Program to check if a number is prime
# Input number from user
num = int(input("Enter a number: "))

# Check if number is less than 2
if num < 2:
    print(f"{num} is not a prime number")
else:
    # Initialize is_prime as True
    is_prime = True
    
    # Check from 2 to sqrt(num)
    for i in range(2, int(num**0.5) + 1):
        if num % i == 0:
            is_prime = False
            break
    
    # Display result
    if is_prime:
        print(f"{num} is a prime number")
    else:
        print(f"{num} is not a prime number")
\end{lstlisting}

\begin{answerdiagram}{Flowchart for Prime Check}
\begin{center}
\begin{tikzpicture}[gtu flow]
    \node[gtu start] (start) {Start};
    \node[gtu input, below of=start] (input) {Input num};
    \node[gtu decision, below of=input] (less2) {num < 2?};
    \node[gtu output, right=3cm of less2] (notprime) {num is not prime};
    \node[gtu process, below of=less2] (init) {is\_prime = True};
    \node[gtu process, below of=init] (loopinit) {i = 2};
    \node[gtu decision, below of=loopinit] (loopcond) {i * i <= num?};
    \node[gtu decision, right=3cm of loopcond] (isdiv) {num \% i == 0?};
    \node[gtu decision, below of=loopcond] (checkprime) {is\_prime?};
    \node[gtu output, right of=checkprime, xshift=2cm] (isprime) {num is prime};
    \node[gtu process, below of=isdiv] (setfalse) {is\_prime = False};
    \node[gtu process, left of=loopcond, xshift=-2cm] (increment) {i = i + 1};
    \node[gtu stop, below of=checkprime] (stop) {End};

    \draw[gtu arrow] (start) -- (input);
    \draw[gtu arrow] (input) -- (less2);
    \draw[gtu arrow] (less2) -- node[above] {Yes} (notprime);
    \draw[gtu arrow] (less2) -- node[right] {No} (init);
    \draw[gtu arrow] (init) -- (loopinit);
    \draw[gtu arrow] (loopinit) -- (loopcond);
    \draw[gtu arrow] (loopcond) -- node[above] {Yes} (isdiv);
    \draw[gtu arrow] (loopcond) -- node[right] {No} (checkprime);
    \draw[gtu arrow] (isdiv) -- node[right] {Yes} (setfalse);
    \draw[gtu arrow] (isdiv) -- node[right] {No} (increment);
    \draw[gtu arrow] (setfalse) -- (checkprime); % Break logic simplified visually
    \draw[gtu arrow] (increment) |- (loopcond);
    \draw[gtu arrow] (checkprime) -- node[above] {Yes} (isprime);
    \draw[gtu arrow] (checkprime) -- node[right] {No} (stop -| notprime) -- (notprime); % Link to not prime output logic
    \draw[gtu arrow] (isprime) |- (stop);
    \draw[gtu arrow] (notprime) |- (stop);
\end{tikzpicture}
\end{center}
\end{answerdiagram}

\begin{mnemonicbox}
"PRIME" (Positive number, Range check from 2 to $\sqrt{n}$, If divisible it's Multiple, Else it's prime)
\end{mnemonicbox}
\end{solutionbox}

\orquestionmarks{2(a)}{3}{Write down a difference between Flow chart and Algorithm.}

\begin{solutionbox}
\begin{answertable}{Difference between Flowchart and Algorithm}
\begin{tabulary}{\linewidth}{|L|L|}
\hline
\textbf{Flow Chart} & \textbf{Algorithm} \\
\hline
\textbf{Visual representation} using standard symbols and shapes & \textbf{Textual description} using structured language \\
\hline
\textbf{Easier to understand} due to graphical nature & Requires knowledge of syntax and terminology \\
\hline
Shows \textbf{logical flow} and relationships clearly & Provides \textbf{detailed steps} in sequential order \\
\hline
\textbf{Time-consuming to create} but easier to follow & \textbf{Quicker to draft} but may be harder to interpret \\
\hline
More difficult to modify or update & Easier to modify or update \\
\hline
\end{tabulary}
\end{answertable}

\begin{mnemonicbox}
"VITAL" (Visual vs Textual, Interpretation ease, Time to create, Alteration flexibility, Logical representation)
\end{mnemonicbox}
\end{solutionbox}

\orquestionmarks{2(b)}{4}{What is the output of the following code:}

\begin{solutionbox}
\begin{lstlisting}[language=Python]
x=10
y=2
print (x*y)
print (x ** y)
print (x//y)
print (x % y)
\end{lstlisting}

\textbf{Answer:}

\begin{answertable}{Output Explanation}
\begin{tabulary}{\linewidth}{|L|L|L|}
\hline
\textbf{Operation} & \textbf{Explanation} & \textbf{Output} \\
\hline
x*y & Multiplication: 10 $\times$ 2 & 20 \\
\hline
x**y & Exponentiation: $10^2$ & 100 \\
\hline
x//y & Integer division: 10 $\div$ 2 & 5 \\
\hline
x\%y & Modulus (remainder): 10 $\div$ 2 & 0 \\
\hline
\end{tabulary}
\end{answertable}

\begin{mnemonicbox}
"MEMO" (Multiply, Exponent, Modulo, Operations)
\end{mnemonicbox}
\end{solutionbox}

\orquestionmarks{2(c)}{7}{Write a python code to display the following patterns:}

\begin{solutionbox}
\begin{lstlisting}
A)                    B)
1                    * * * *
1 2                  * * *
1 2 3                * *
1 2 3 4              *
\end{lstlisting}

\begin{lstlisting}[language=Python]
# Pattern A: Number pattern
print("Pattern A:")
for i in range(1, 5):
    for j in range(1, i + 1):
        print(j, end=" ")
    print()

# Pattern B: Star pattern
print("\nPattern B:")
for i in range(4, 0, -1):
    for j in range(i):
        print("*", end=" ")
    print()
\end{lstlisting}

\begin{answerdiagram}{Flowchart for Patterns}
\begin{center}
\begin{tikzpicture}[gtu flow]
    \node[gtu start] (start) {Start};
    \node[gtu process, below of=start] (pattA) {Pattern A Logic};
    \node[gtu process, below of=pattA] (loopA) {i = 1 to 4};
    \node[gtu process, below of=loopA] (innerA) {j = 1 to i};
    \node[gtu output, below of=innerA] (printA) {Print j};
    \node[gtu process, below of=printA] (newlineA) {New line};
    
    \node[gtu process, right=4cm of pattA] (pattB) {Pattern B Logic};
    \node[gtu process, below of=pattB] (loopB) {i = 4 to 1};
    \node[gtu process, below of=loopB] (innerB) {j = 0 to i-1};
    \node[gtu output, below of=innerB] (printB) {Print *};
    \node[gtu process, below of=printB] (newlineB) {New line};
    \node[gtu stop, below of=newlineB] (stop) {End};

    \draw[gtu arrow] (start) -- (pattA);
    \draw[gtu arrow] (pattA) -- (loopA);
    \draw[gtu arrow] (loopA) -- (innerA);
    \draw[gtu arrow] (innerA) -- (printA);
    \draw[gtu arrow] (printA) -- (newlineA);
    \draw[gtu arrow] (newlineA) -| (pattB);
    
    \draw[gtu arrow] (pattB) -- (loopB);
    \draw[gtu arrow] (loopB) -- (innerB);
    \draw[gtu arrow] (innerB) -- (printB);
    \draw[gtu arrow] (printB) -- (newlineB);
    \draw[gtu arrow] (newlineB) -- (stop);
\end{tikzpicture}
\end{center}
\end{answerdiagram}

\begin{mnemonicbox}
"LOOP-NED" (Loop Outer, Order Pattern, Nested loops, End with newline, Display)
\end{mnemonicbox}
\end{solutionbox}

\questionmarks{3(a)}{3}{With the necessary examples describe the use of break statement.}

\begin{solutionbox}
Break statement is used to exit or terminate a loop prematurely when a specific condition is met.

\textbf{Example:}
\begin{lstlisting}[language=Python]
# Finding the first odd number in a list
numbers = [2, 4, 6, 7, 8, 10]
for num in numbers:
    if num % 2 != 0:
        print(f"Found odd number: {num}")
        break
    print(f"Checking {num}")
\end{lstlisting}

\textbf{Output:}
\begin{lstlisting}
Checking 2
Checking 4
Checking 6
Found odd number: 7
\end{lstlisting}

\begin{mnemonicbox}
"EXIT" (EXecute until condition, Immediately Terminate)
\end{mnemonicbox}
\end{solutionbox}

\questionmarks{3(b)}{4}{Explain if...else statement with suitable example.}

\begin{solutionbox}
The if...else statement is a conditional statement that executes different blocks of code based on whether a specified condition evaluates to True or False.

\textbf{Syntax:}
\begin{lstlisting}[language=Python]
if condition:
    # Code to be executed if condition is True
else:
    # Code to be executed if condition is False
\end{lstlisting}

\textbf{Example:}
\begin{lstlisting}[language=Python]
# Check if a number is even or odd
number = int(input("Enter a number: "))

if number % 2 == 0:
    print(f"{number} is an even number")
else:
    print(f"{number} is an odd number")
\end{lstlisting}

\begin{answerdiagram}{Flowchart for If-Else}
\begin{center}
\begin{tikzpicture}[gtu flow]
    \node[gtu start] (start) {Start};
    \node[gtu input, below of=start] (input) {Input number};
    \node[gtu decision, below of=input] (cond) {number \% 2 == 0?};
    \node[gtu output, below left of=cond, xshift=-2cm] (even) {Print "even"};
    \node[gtu output, below right of=cond, xshift=2cm] (odd) {Print "odd"};
    \node[gtu stop, below of=cond, yshift=-4cm] (stop) {End};

    \draw[gtu arrow] (start) -- (input);
    \draw[gtu arrow] (input) -- (cond);
    \draw[gtu arrow] (cond) -| node[above] {Yes} (even);
    \draw[gtu arrow] (cond) -| node[above] {No} (odd);
    \draw[gtu arrow] (even) |- (stop);
    \draw[gtu arrow] (odd) |- (stop);
\end{tikzpicture}
\end{center}
\end{answerdiagram}

\begin{mnemonicbox}
"CITE" (Check condition, If True Execute this, Else execute that)
\end{mnemonicbox}
\end{solutionbox}

\questionmarks{3(c)}{7}{Create a User-defined function to print the Fibonacci series of 0 to N numbers where N is an integer number and passed as an argument.}

\begin{solutionbox}
\begin{lstlisting}[language=Python]
# Function to print Fibonacci series
def print_fibonacci(n):
    # Initialize first two terms
    a, b = 0, 1
    
    # Check if n is valid
    if n < 0:
        print("Please enter a positive number")
        return
    
    # Print Fibonacci series
    print("Fibonacci series up to", n, ":")
    
    if n >= 0:
        print(a, end=" ")  # Print first term
    
    if n >= 1:
        print(b, end=" ")  # Print second term
    
    # Generate and print the rest of the series
    while a + b <= n:
        next_term = a + b
        print(next_term, end=" ")
        a, b = b, next_term

# Test the function
print_fibonacci(55)
\end{lstlisting}

\begin{answerdiagram}{Flowchart for Fibonacci Series}
\begin{center}
\begin{tikzpicture}[gtu flow]
    \node[gtu start] (start) {Start};
    \node[gtu process, below of=start] (init) {Initialize a=0, b=1};
    \node[gtu decision, below of=init] (checkneg) {n < 0?};
    \node[gtu output, right=3cm of checkneg] (error) {Print error};
    \node[gtu decision, below of=checkneg] (check0) {n >= 0?};
    \node[gtu output, left=3cm of check0] (printa) {Print a};
    \node[gtu decision, below of=check0] (check1) {n >= 1?};
    \node[gtu output, left=3cm of check1] (printb) {Print b};
    \node[gtu decision, below of=check1] (checkloop) {a + b <= n?};
    \node[gtu process, right=3cm of checkloop] (update) {next = a + b\\Print next\\a=b, b=next};
    \node[gtu stop, below of=checkloop] (stop) {End};

    \draw[gtu arrow] (start) -- (init);
    \draw[gtu arrow] (init) -- (checkneg);
    \draw[gtu arrow] (checkneg) -- node[above] {Yes} (error);
    \draw[gtu arrow] (checkneg) -- node[right] {No} (check0);
    \draw[gtu arrow] (error) |- (stop);
    
    \draw[gtu arrow] (check0) -- node[above] {Yes} (printa);
    \draw[gtu arrow] (check0) -- node[right] {No} (check1);
    \draw[gtu arrow] (printa) -- (check1);
    
    \draw[gtu arrow] (check1) -- node[above] {Yes} (printb);
    \draw[gtu arrow] (check1) -- node[right] {No} (checkloop);
    \draw[gtu arrow] (printb) -- (checkloop);
    
    \draw[gtu arrow] (checkloop) -- node[above] {Yes} (update);
    \draw[gtu arrow] (checkloop) -- node[right] {No} (stop);
    \draw[gtu arrow] (update) |- (checkloop);
\end{tikzpicture}
\end{center}
\end{answerdiagram}

\begin{mnemonicbox}
"FIBER" (First terms set, Initialize variables, Build next term, Echo results, Repeat until limit)
\end{mnemonicbox}
\end{solutionbox}

\orquestionmarks{3(a)}{3}{With the necessary examples describe the use of continue statement.}

\begin{solutionbox}
Continue statement is used to skip the current iteration of a loop and continue with the next iteration.

\textbf{Example:}
\begin{lstlisting}[language=Python]
# Print only odd numbers from 1 to 10
for i in range(1, 11):
    if i % 2 == 0:
        continue  # Skip even numbers
    print(i)
\end{lstlisting}

\textbf{Output:}
\begin{lstlisting}
1
3
5
7
9
\end{lstlisting}

\begin{mnemonicbox}
"SKIP" (Skip current iteration, Keep looping, Ignore remaining statements, Proceed to next iteration)
\end{mnemonicbox}
\end{solutionbox}

\orquestionmarks{3(b)}{4}{Explain For loop statement with example.}

\begin{solutionbox}
For loop is used to iterate over a sequence (like list, tuple, string) or other iterable objects and execute a block of code for each item in the sequence.

\textbf{Syntax:}
\begin{lstlisting}[language=Python]
for variable in sequence:
    # Code to be executed for each item
\end{lstlisting}

\textbf{Example:}
\begin{lstlisting}[language=Python]
# Print squares of numbers from 1 to 5
for num in range(1, 6):
    square = num ** 2
    print(f"The square of {num} is {square}")
\end{lstlisting}

\textbf{Output:}
\begin{lstlisting}
The square of 1 is 1
The square of 2 is 4
The square of 3 is 9
The square of 4 is 16
The square of 5 is 25
\end{lstlisting}

\begin{answerdiagram}{Flowchart for For Loop}
\begin{center}
\begin{tikzpicture}[gtu flow]
    \node[gtu start] (start) {Start};
    \node[gtu decision, below of=start] (cond) {Item in sequence?};
    \node[gtu process, below of=cond] (body) {Execute code block};
    \node[gtu stop, right=3cm of cond] (stop) {End};

    \draw[gtu arrow] (start) -- (cond);
    \draw[gtu arrow] (cond) -- node[right] {Yes} (body);
    \draw[gtu arrow] (cond) -- node[above] {No} (stop);
    \draw[gtu arrow] (body) -- ++(-2,0) |- (cond);
\end{tikzpicture}
\end{center}
\end{answerdiagram}

\begin{mnemonicbox}
"FIRE" (For each Item, Run commands, Execute until end)
\end{mnemonicbox}
\end{solutionbox}

\orquestionmarks{3(c)}{7}{Write a python code that determines whether a given number is an 'Armstrong number' or a palindrome using a user-defined function.}

\begin{solutionbox}
\begin{lstlisting}[language=Python]
# Function to check if a number is Armstrong number
def is_armstrong(num):
    # Convert to string to count digits
    num_str = str(num)
    n = len(num_str)
    
    # Calculate sum of each digit raised to power of total digits
    sum_of_powers = sum(int(digit) ** n for digit in num_str)
    
    # Check if sum equals the original number
    return sum_of_powers == num

# Function to check if a number is a palindrome
def is_palindrome(num):
    # Convert to string
    num_str = str(num)
    
    # Check if string equals its reverse
    return num_str == num_str[::-1]

# Main function to check both conditions
def check_number(num):
    if is_armstrong(num):
        print(f"{num} is an Armstrong number")
    else:
        print(f"{num} is not an Armstrong number")
    
    if is_palindrome(num):
        print(f"{num} is a palindrome")
    else:
        print(f"{num} is not a palindrome")

# Test the function
number = int(input("Enter a number: "))
check_number(number)
\end{lstlisting}

\begin{answerdiagram}{Flowchart for Armstrong and Palindrome Check}
\begin{center}
\begin{tikzpicture}[gtu flow]
    \node[gtu start] (start) {Start};
    \node[gtu input, below of=start] (input) {Input number};
    \node[gtu process, below of=input] (call) {Call check\_number};
    \node[gtu process, below of=call] (arm) {Call is\_armstrong};
    \node[gtu decision, below of=arm] (isarm) {Result True?};
    \node[gtu output, left=2cm of isarm] (pyes) {Print "Armstrong"};
    \node[gtu output, right=2cm of isarm] (pno) {Print "Not Armstrong"};
    \node[gtu process, below of=isarm, yshift=-1cm] (pal) {Call is\_palindrome};
    \node[gtu decision, below of=pal] (ispal) {Result True?};
    \node[gtu output, left=2cm of ispal] (ppyes) {Print "Palindrome"};
    \node[gtu output, right=2cm of ispal] (ppno) {Print "Not Palindrome"};
    \node[gtu stop, below of=ispal] (stop) {End};

    \draw[gtu arrow] (start) -- (input);
    \draw[gtu arrow] (input) -- (call);
    \draw[gtu arrow] (call) -- (arm);
    \draw[gtu arrow] (arm) -- (isarm);
    \draw[gtu arrow] (isarm) -- node[above] {Yes} (pyes);
    \draw[gtu arrow] (isarm) -- node[above] {No} (pno);
    \draw[gtu arrow] (pyes) |- (pal);
    \draw[gtu arrow] (pno) |- (pal);
    \draw[gtu arrow] (pal) -- (ispal);
    \draw[gtu arrow] (ispal) -- node[above] {Yes} (ppyes);
    \draw[gtu arrow] (ispal) -- node[above] {No} (ppno);
    \draw[gtu arrow] (ppyes) |- (stop);
    \draw[gtu arrow] (ppno) |- (stop);
\end{tikzpicture}
\end{center}
\end{answerdiagram}

\begin{mnemonicbox}
"APC" (Armstrong check: Power sum of digits, Palindrome check: Compare with reverse)
\end{mnemonicbox}
\end{solutionbox}

\questionmarks{4(a)}{3}{Develop a python code to identify whether the scanned number is even or odd and print an appropriate message.}

\begin{solutionbox}
\begin{lstlisting}[language=Python]
# Program to check if a number is even or odd
number = int(input("Enter a number: "))

if number % 2 == 0:
    print(f"{number} is an even number")
else:
    print(f"{number} is an odd number")
\end{lstlisting}

\begin{answerdiagram}{Flowchart for Even/Odd}
\begin{center}
\begin{tikzpicture}[gtu flow]
    \node[gtu start] (start) {Start};
    \node[gtu input, below of=start] (input) {Input number};
    \node[gtu decision, below of=input] (cond) {number \% 2 == 0?};
    \node[gtu output, below left of=cond, xshift=-1cm] (even) {Print "even"};
    \node[gtu output, below right of=cond, xshift=1cm] (odd) {Print "odd"};
    \node[gtu stop, below of=cond, yshift=-3cm] (stop) {End};

    \draw[gtu arrow] (start) -- (input);
    \draw[gtu arrow] (input) -- (cond);
    \draw[gtu arrow] (cond) -| node[above] {Yes} (even);
    \draw[gtu arrow] (cond) -| node[above] {No} (odd);
    \draw[gtu arrow] (even) |- (stop);
    \draw[gtu arrow] (odd) |- (stop);
\end{tikzpicture}
\end{center}
\end{answerdiagram}

\begin{mnemonicbox}
"MODE" (Modulo Operation Determines Even-odd)
\end{mnemonicbox}
\end{solutionbox}

\questionmarks{4(b)}{4}{Define function. Explain user define function using suitable example.}

\begin{solutionbox}
A function is a block of organized, reusable code that performs a specific task. User-defined functions are functions created by the programmer to perform custom operations.

\textbf{Components of a User-defined Function:}
\begin{itemize}
    \item \textbf{def keyword}: Marks the start of function definition
    \item \textbf{Function name}: Identifier for the function
    \item \textbf{Parameters}: Input values (optional)
    \item \textbf{Docstring}: Description of the function (optional)
    \item \textbf{Function body}: Code to be executed
    \item \textbf{Return statement}: Output value (optional)
\end{itemize}

\textbf{Example:}
\begin{lstlisting}[language=Python]
# User-defined function to calculate area of rectangle
def calculate_area(length, width):
    """
    Calculate area of rectangle
    """
    area = length * width
    return area

# Call the function
result = calculate_area(5, 3)
print(f"Area of rectangle: {result}")
\end{lstlisting}

\begin{mnemonicbox}
"DRAPE" (Define function, Receive parameters, Acquire result, Process data, End with return)
\end{mnemonicbox}
\end{solutionbox}

\questionmarks{4(c)}{7}{List out various String operations and explain any three using example.}

\begin{solutionbox}
String operations in Python:

\begin{answertable}{String Operations}
\begin{tabulary}{\linewidth}{|L|L|}
\hline
\textbf{Operation} & \textbf{Description} \\
\hline
Concatenation & Joining strings together using + \\
\hline
Repetition & Repeating a string using * \\
\hline
Indexing & Accessing characters by position \\
\hline
Slicing & Extracting a portion of a string \\
\hline
Methods & Built-in functions (len, upper, lower, etc.) \\
\hline
Membership Testing & Check if a substring exists in a string \\
\hline
Formatting & Create formatted strings \\
\hline
Escape Sequences & Special characters preceded by \textbackslash \\
\hline
\end{tabulary}
\end{answertable}

\textbf{Three String Operations with Examples:}

\textbf{1. String Concatenation:}
\begin{lstlisting}[language=Python]
first_name = "John"
last_name = "Doe"
full_name = first_name + " " + last_name
print(full_name)  # Output: John Doe
\end{lstlisting}

\textbf{2. String Slicing:}
\begin{lstlisting}[language=Python]
message = "Python Programming"
print(message[0:6])    # Output: Python
print(message[7:])     # Output: Programming
print(message[-11:])   # Output: Programming
\end{lstlisting}

\textbf{3. String Methods:}
\begin{lstlisting}[language=Python]
text = "python programming"
print(text.upper())    # Output: PYTHON PROGRAMMING
print(text.capitalize())  # Output: Python programming
print(text.replace("python", "Java"))  # Output: Java programming
\end{lstlisting}

\begin{mnemonicbox}
"CSM" (Concatenate strings, Slice portions, Manipulate with methods)
\end{mnemonicbox}
\end{solutionbox}

\orquestionmarks{4(a)}{3}{Create a python code to check positive or negative number.}

\begin{solutionbox}
\begin{lstlisting}[language=Python]
# Program to check if a number is positive or negative
number = float(input("Enter a number: "))

if number > 0:
    print(f"{number} is a positive number")
elif number < 0:
    print(f"{number} is a negative number")
else:
    print("The number is zero")
\end{lstlisting}

\begin{answerdiagram}{Flowchart for Positive/Negative}
\begin{center}
\begin{tikzpicture}[gtu flow]
    \node[gtu start] (start) {Start};
    \node[gtu input, below of=start] (input) {Input number};
    \node[gtu decision, below of=input] (pos) {number > 0?};
    \node[gtu output, left=2cm of pos] (ppos) {Print "Positive"};
    \node[gtu decision, below of=pos] (neg) {number < 0?};
    \node[gtu output, left=2cm of neg] (pneg) {Print "Negative"};
    \node[gtu output, right=2cm of neg] (pzero) {Print "Zero"};
    \node[gtu stop, below of=neg] (stop) {End};

    \draw[gtu arrow] (start) -- (input);
    \draw[gtu arrow] (input) -- (pos);
    \draw[gtu arrow] (pos) -- node[above] {Yes} (ppos);
    \draw[gtu arrow] (pos) -- node[right] {No} (neg);
    \draw[gtu arrow] (neg) -- node[above] {Yes} (pneg);
    \draw[gtu arrow] (neg) -- node[above] {No} (pzero);
    \draw[gtu arrow] (ppos) |- (stop);
    \draw[gtu arrow] (pneg) |- (stop);
    \draw[gtu arrow] (pzero) |- (stop);
\end{tikzpicture}
\end{center}
\end{answerdiagram}

\begin{mnemonicbox}
"SIGN" (See If Greater than 0, Negative otherwise)
\end{mnemonicbox}
\end{solutionbox}

\orquestionmarks{4(b)}{4}{Explain local and global variables using suitable examples.}

\begin{solutionbox}
In Python, variables can have different scopes:

\begin{answertable}{Variable Scopes}
\begin{tabulary}{\linewidth}{|L|L|}
\hline
\textbf{Variable Type} & \textbf{Description} \\
\hline
Local Variable & Defined within a function and accessible only inside that function \\
\hline
Global Variable & Defined outside functions and accessible throughout the program \\
\hline
\end{tabulary}
\end{answertable}

\textbf{Example:}
\begin{lstlisting}[language=Python]
# Global variable
count = 0 

def update_count():
    # Local variable
    local_var = 5 
    
    # Accessing global variable inside function
    global count
    count += 1
    
    print(f"Local variable: {local_var}")
    print(f"Global variable (inside function): {count}")
    
# Call the function
update_count()

# Accessing variables outside function
print(f"Global variable (outside function): {count}")
\end{lstlisting}

\begin{mnemonicbox}
"SCOPE" (Some variables Confined to function Only, Program-wide Exposure for others)
\end{mnemonicbox}
\end{solutionbox}

\orquestionmarks{4(c)}{7}{List out various List operations and explain any three using example.}

\begin{solutionbox}
List operations in Python:

\begin{answertable}{List Operations}
\begin{tabulary}{\linewidth}{|L|L|}
\hline
\textbf{Operation} & \textbf{Description} \\
\hline
Creating Lists & Using square brackets [] \\
\hline
Indexing & Accessing elements by position \\
\hline
Slicing & Extracting portions of a list \\
\hline
Append & Adding elements to the end \\
\hline
Insert & Adding elements at specific positions \\
\hline
Remove & Removing specific elements \\
\hline
Pop & Removing and returning elements \\
\hline
Sort & Ordering list elements \\
\hline
Reverse & Reversing list order \\
\hline
Extend & Combining lists \\
\hline
\end{tabulary}
\end{answertable}

\textbf{Three List Operations with Examples:}

\textbf{1. List Indexing and Slicing:}
\begin{lstlisting}[language=Python]
fruits = ["apple", "banana", "cherry", "orange", "kiwi"]
print(fruits[1])        # Output: banana
print(fruits[-1])       # Output: kiwi
print(fruits[1:4])      # Output: ['banana', 'cherry', 'orange']
\end{lstlisting}

\textbf{2. List Methods (append, insert, remove):}
\begin{lstlisting}[language=Python]
numbers = [1, 2, 3]
numbers.append(4)       # Add 4 to the end
numbers.insert(0, 0)    # Insert 0 at position 0
numbers.remove(2)       # Remove element with value 2
print(numbers)          # Output: [0, 1, 3, 4]
\end{lstlisting}

\textbf{3. List Comprehensions:}
\begin{lstlisting}[language=Python]
# Create a list of squares
squares = [x**2 for x in range(1, 6)]
print(squares)  # Output: [1, 4, 9, 16, 25]
\end{lstlisting}

\begin{mnemonicbox}
"AIM" (Access with index, Insert/modify elements, Make using comprehensions)
\end{mnemonicbox}
\end{solutionbox}

\questionmarks{5(a)}{3}{Write python code to swap given two elements in a list.}

\begin{solutionbox}
\begin{lstlisting}[language=Python]
# Program to swap two elements in a list
def swap_elements(my_list, pos1, pos2):
    # Check if positions are valid
    if 0 <= pos1 < len(my_list) and 0 <= pos2 < len(my_list):
        # Swap elements
        my_list[pos1], my_list[pos2] = my_list[pos2], my_list[pos1]
        return True
    else:
        return False

# Example usage
numbers = [10, 20, 30, 40, 50]
print("Original list:", numbers)

# Swap elements at positions 1 and 3
if swap_elements(numbers, 1, 3):
    print("After swapping:", numbers)
else:
    print("Invalid positions")
\end{lstlisting}

\begin{mnemonicbox}
"SWAP" (Select positions, Watch boundaries, Assign simultaneously, Print result)
\end{mnemonicbox}
\end{solutionbox}

\questionmarks{5(b)}{4}{Explain math module and random module in python using example.}

\begin{solutionbox}
Math and random modules provide functions for mathematical operations and random number generation.

\textbf{Math Module:}
\begin{lstlisting}[language=Python]
import math
print(math.pi)          # Output: 3.14159...
print(math.sqrt(16))    # Output: 4.0
print(math.ceil(4.2))   # Output: 5
\end{lstlisting}

\textbf{Random Module:}
\begin{lstlisting}[language=Python]
import random
print(random.random())       # Random float
print(random.randint(1, 10)) # Random integer
colors = ["red", "green"]
print(random.choice(colors)) # Random choice
\end{lstlisting}

\begin{mnemonicbox}
"MR-CS" (Math for Calculations, Random for Choice and Shuffling)
\end{mnemonicbox}
\end{solutionbox}

\questionmarks{5(c)}{7}{Write a python code to demonstrate tuples functions and operations.}

\begin{solutionbox}
\begin{lstlisting}[language=Python]
# Creating tuples
mixed_tuple = (1, "Hello", 3.14, True)

# Accessing tuple elements
print(mixed_tuple[0])      # Output: 1

# Tuple slicing
print(mixed_tuple[1:3])    # Output: ("Hello", 3.14)

# Tuple concatenation
tuple1 = (1, 2)
tuple2 = (3, 4)
print(tuple1 + tuple2)     # Output: (1, 2, 3, 4)

# Tuple methods
numbers = (1, 2, 2)
print(numbers.count(2))    # Output: 2

# Membership testing
print(1 in numbers)        # Output: True
\end{lstlisting}

\begin{answerdiagram}{Tuple Operations Visualized}
\begin{center}
\begin{tikzpicture}[gtu tree]
    \node[gtu root] {Tuple Operations}
    child {node[gtu child] {Creation\\ \texttt{()}}}
    child {node[gtu child] {Accessing\\ \texttt{[]}}}
    child {node[gtu child] {Slicing\\ \texttt{[:]}}}
    child {node[gtu child] {Concatenation\\ \texttt{+}}}
    child {node[gtu child] {Methods\\ \texttt{count()}}};
\end{tikzpicture}
\end{center}
\end{answerdiagram}

\begin{mnemonicbox}
"CASC-RUMTC" (Create, Access, Slice, Concatenate, Repeat, Use methods, Membership test, Tuple conversion)
\end{mnemonicbox}
\end{solutionbox}

\orquestionmarks{5(a)}{3}{Write a python code to find the sum of elements in a list.}

\begin{solutionbox}
\begin{lstlisting}[language=Python]
# Program to find the sum of elements in a list
def sum_of_elements(numbers):
    total = 0
    for num in numbers:
        total += num
    return total

# Example usage
my_list = [10, 20, 30, 40, 50]
print("Sum of elements:", sum_of_elements(my_list))
\end{lstlisting}

\begin{answerdiagram}{Flowchart for Sum of Elements}
\begin{center}
\begin{tikzpicture}[gtu flow]
    \node[gtu start] (start) {Start};
    \node[gtu process, below of=start] (init) {Initialize total = 0};
    \node[gtu decision, below of=init] (loop) {More elements?};
    \node[gtu process, below of=loop] (add) {total += num};
    \node[gtu output, right=3cm of loop] (ret) {Return total};
    \node[gtu stop, below of=ret] (stop) {End};

    \draw[gtu arrow] (start) -- (init);
    \draw[gtu arrow] (init) -- (loop);
    \draw[gtu arrow] (loop) -- node[right] {Yes} (add);
    \draw[gtu arrow] (loop) -- node[above] {No} (ret);
    \draw[gtu arrow] (add) -- ++(-2,0) |- (loop);
    \draw[gtu arrow] (ret) -- (stop);
\end{tikzpicture}
\end{center}
\end{answerdiagram}

\begin{mnemonicbox}
"SITE" (Sum Initialized To zero, Elements added one by one)
\end{mnemonicbox}
\end{solutionbox}

\orquestionmarks{5(b)}{4}{Explain the usage of following built in functions: 1) Print() 2) Min() 3) Sum() 4) Input()}

\begin{solutionbox}
\begin{answertable}{Built-in Functions}
\begin{tabulary}{\linewidth}{|l|L|l|}
\hline
\textbf{Function} & \textbf{Purpose} & \textbf{Example} \\
\hline
\textbf{print()} & Displays output to the console & \texttt{print("Hi")} \\
\hline
\textbf{min()} & Returns smallest item & \texttt{min([5, 1])}\\
\hline
\textbf{sum()} & Returns sum of all items & \texttt{sum([1, 2])} \\
\hline
\textbf{input()} & Reads input from the user & \texttt{input("Name:")} \\
\hline
\end{tabulary}
\end{answertable}

\begin{lstlisting}[language=Python]
print("Hello")
print(min([5, 3, 8]))
print(sum([1, 2, 3]))
name = input("Enter name: ")
\end{lstlisting}

\begin{mnemonicbox}
"PMSI" (Print to display, Min for smallest, Sum for total, Input for reading)
\end{mnemonicbox}
\end{solutionbox}

\orquestionmarks{5(c)}{7}{Write a python code to demonstrate the set functions and operations.}

\begin{solutionbox}
\begin{lstlisting}[language=Python]
# Creating sets
numbers = {1, 2, 3}

# Adding elements
numbers.add(4)

# Update
numbers.update([5, 6])

# Remove
numbers.remove(3)

# Set operations
set1 = {1, 2, 3}
set2 = {3, 4, 5}

# Union
print(set1 | set2)      # Output: {1, 2, 3, 4, 5}

# Intersection
print(set1 & set2)      # Output: {3}

# Difference
print(set1 - set2)      # Output: {1, 2}
\end{lstlisting}

\begin{answerdiagram}{Set Operations}
\begin{center}
\begin{tikzpicture}[gtu tree]
    \node[gtu root] {Set Operations}
    child {node[gtu child] {Modify\\(Add/Remove)}}
    child {node[gtu child] {Union\\($|$ )}}
    child {node[gtu child] {Intersection\\($\& $)}}
    child {node[gtu child] {Difference\\($-$)}};
\end{tikzpicture}
\end{center}
\end{answerdiagram}

\begin{mnemonicbox}
"CARDS-UI" (Create, Add, Remove, Discard elements, Set operations - Union, Intersection)
\end{mnemonicbox}
\end{solutionbox}

\end{document}
