\documentclass[10pt,a4paper]{article}

% content/resources/templates/preamble.tex
\usepackage[margin=0.6in]{geometry}
\author{Milav Dabgar}
\usepackage{amsmath,amssymb,amsthm}
\usepackage{booktabs}
\usepackage{multirow}
\usepackage{xcolor}
\usepackage{tcolorbox}
\tcbuselibrary{breakable,skins}
\usepackage[colorlinks=true,linkcolor=blue]{hyperref}
\usepackage{titlesec}
\usepackage{enumitem}
\usepackage{tikz}
\usepackage{pgfplots}
\usepackage{circuitikz}
\usepackage[version=4]{mhchem}
\usepackage{longtable}
\usepackage{array}
\usepackage{float}
\usepackage{caption}
\usepackage{listings}

\lstset{
  basicstyle=\small\ttfamily,
  breaklines=true,
  breakatwhitespace=false,
  postbreak=\mbox{\textcolor{red}{$\hookrightarrow$}\space},
  float=false,
  numbers=left,
  numberstyle=\tiny\color{gray},
  numbersep=10pt,
  xleftmargin=2em,
  keywordstyle=\color{blue},
  commentstyle=\color{green!60!black},
  stringstyle=\color{purple},
  backgroundcolor=\color{gray!5},
  showstringspaces=false,
  tabsize=2,
  captionpos=b,
  keepspaces=true,
  columns=flexible
}

\pgfplotsset{compat=1.18}
\usetikzlibrary{shapes,arrows,positioning,calc,patterns,decorations.pathmorphing,decorations.markings,arrows.meta}

% Color scheme
\definecolor{headcolor}{RGB}{0,102,204}
\definecolor{keycolor}{RGB}{220,20,60}
\definecolor{solutioncolor}{RGB}{34,139,34}
\definecolor{mnemoniccolor}{RGB}{148,0,211}
\definecolor{codecolor}{RGB}{0,0,100}

% Spacing
\setlength{\parskip}{3pt}
\setlist[itemize]{nosep}
\setlist[enumerate]{nosep}

% Title formatting
\titleformat{\section}{\Large\bfseries\color{headcolor}}{\thesection}{1em}{}
\titleformat{\subsection}{\large\bfseries\color{headcolor}}{\thesubsection}{1em}{}

% Pandoc tightlist compatibility
\providecommand{\tightlist}{%
  \setlength{\itemsep}{0pt}\setlength{\parskip}{0pt}}

% Pandoc longtable compatibility
\newcounter{none}
\def\thenone{}


% content/resources/templates/english-boxes.tex
% This file is currently empty - it exists to maintain consistency with the import structure.
% Add custom environments here if needed in the future.


\begin{document}

\begin{center}
{\Huge\bfseries\color{headcolor} Subject Name Solutions}\\[5pt]
{\LARGE 1323203 -- Winter 2024}\\[3pt]
{\large Semester 1 Study Material}\\[3pt]
{\normalsize\textit{Detailed Solutions and Explanations}}
\end{center}

\vspace{10pt}

\subsection*{Question 1(a) [3 marks]}\label{q1a}

\textbf{Define flowchart and list out the any four symbols of
flowchart.}

\begin{solutionbox}
A flowchart is a diagrammatic representation that uses
standard symbols to illustrate the sequence of steps in a process,
algorithm, or program.

\textbf{Common Flowchart Symbols:}

{\def\LTcaptype{none} % do not increment counter
\begin{longtable}[]{@{}
  >{\raggedright\arraybackslash}p{(\linewidth - 4\tabcolsep) * \real{0.3333}}
  >{\raggedright\arraybackslash}p{(\linewidth - 4\tabcolsep) * \real{0.3333}}
  >{\raggedright\arraybackslash}p{(\linewidth - 4\tabcolsep) * \real{0.3333}}@{}}
\toprule\noalign{}
\begin{minipage}[b]{\linewidth}\raggedright
Symbol
\end{minipage} & \begin{minipage}[b]{\linewidth}\raggedright
Name
\end{minipage} & \begin{minipage}[b]{\linewidth}\raggedright
Purpose
\end{minipage} \\
\midrule\noalign{}
\endhead
\bottomrule\noalign{}
\endlastfoot
Oval/Rounded Rectangle & Terminal/Start/End & Indicates start or end of
a process \\
Rectangle & Process & Represents computation or data processing \\
Diamond & Decision & Shows conditional branching point \\
Parallelogram & Input/Output & Represents data input or output \\
\end{longtable}
}

\end{solutionbox}
\begin{mnemonicbox}
``TP-DI'' (Terminal-Process-Decision-Input/Output)

\end{mnemonicbox}
\subsection*{Question 1(b) [4 marks]}\label{q1b}

\textbf{List out various data types in python. Explain any three data
types with example.}

\begin{solutionbox}
Python data types categorize different types of data
values.

{\def\LTcaptype{none} % do not increment counter
\begin{longtable}[]{@{}lll@{}}
\toprule\noalign{}
Data Type & Description & Example \\
\midrule\noalign{}
\endhead
\bottomrule\noalign{}
\endlastfoot
Integer & Whole numbers without decimals &
\passthrough{\lstinline!x = 10!} \\
Float & Numbers with decimal points &
\passthrough{\lstinline!y = 3.14!} \\
String & Sequence of characters &
\passthrough{\lstinline!name = "Python"!} \\
Boolean & True or False values &
\passthrough{\lstinline!is\_valid = True!} \\
List & Ordered, mutable collection &
\passthrough{\lstinline!colors = ["red", "green"]!} \\
Tuple & Ordered, immutable collection &
\passthrough{\lstinline!point = (5, 10)!} \\
Dictionary & Key-value pairs &
\passthrough{\lstinline!person = \{"name": "John"\}!} \\
Set & Unordered collection of unique items &
\passthrough{\lstinline!unique = \{1, 2, 3\}!} \\
\end{longtable}
}

\textbf{Integer:} Represents whole numbers without decimal points.

\begin{lstlisting}[language=Python]
age = 25
count = -10
\end{lstlisting}

\textbf{String:} Represents sequence of characters enclosed in quotes.

\begin{lstlisting}[language=Python]
name = "Python"
message = 'Hello World'
\end{lstlisting}

\textbf{List:} Ordered, mutable collection of items that can be of
different types.

\begin{lstlisting}[language=Python]
numbers = [1, 2, 3, 4]
mixed = [1, "Python", True, 3.14]
\end{lstlisting}

\end{solutionbox}
\begin{mnemonicbox}
``FIBS-LTDS''
(Float-Integer-Boolean-String-List-Tuple-Dictionary-Set)

\end{mnemonicbox}
\subsection*{Question 1(c) [7 marks]}\label{q1c}

\textbf{Design a flowchart to calculate the sum of first twenty even
natural numbers.}

\begin{solutionbox}

\includegraphics[width=1\linewidth,height=\textheight,keepaspectratio]{mermaid-a2ff1128.pdf}

\textbf{Explanation:}

\begin{itemize}
\tightlist
\item
  \textbf{Initialize variables}: Set sum=0, count=0 (to track even
  numbers found), num=2 (first even number)
\item
  \textbf{Loop condition}: Continue until we've found 20 even numbers
\item
  \textbf{Process}: Add current even number to sum
\item
  \textbf{Update}: Increase counter and move to next even number
\item
  \textbf{Output}: Print the final sum when loop completes
\end{itemize}

\end{solutionbox}
\begin{mnemonicbox}
``SCNL-20'' (Sum-Count-Number-Loop until 20)

\end{mnemonicbox}
\subsection*{Question 1(c) OR [7
marks]}\label{q1c}

\textbf{Create an algorithm to print odd numbers between 1 to 20.}

\begin{solutionbox}

\textbf{Algorithm:}

\begin{enumerate}
\tightlist
\item
  Initialize a variable num = 1 (starting with first odd number)
\item
  While num \leq 20, do steps 3-5
\item
  Print the value of num
\item
  Increment num by 2 (to get next odd number)
\item
  Repeat from step 2
\item
  End
\end{enumerate}

\textbf{Diagram:}

\includegraphics[width=1\linewidth,height=\textheight,keepaspectratio]{mermaid-7c95ca87.pdf}

\textbf{Code Implementation:}

\begin{lstlisting}[language=Python]
# Print odd numbers between 1 to 20
num = 1
while num <= 20:
    print(num)
    num += 2
\end{lstlisting}

\end{solutionbox}
\begin{mnemonicbox}
``SOLO-20'' (Start Odd Loop Output until 20)

\end{mnemonicbox}
\subsection*{Question 2(a) [3 marks]}\label{q2a}

\textbf{Discuss the membership operator of python.}

\begin{solutionbox}
Membership operators in Python are used to test if a
value or variable exists in a sequence.

\textbf{Table of Membership Operators:}

{\def\LTcaptype{none} % do not increment counter
\begin{longtable}[]{@{}
  >{\raggedright\arraybackslash}p{(\linewidth - 6\tabcolsep) * \real{0.2500}}
  >{\raggedright\arraybackslash}p{(\linewidth - 6\tabcolsep) * \real{0.2500}}
  >{\raggedright\arraybackslash}p{(\linewidth - 6\tabcolsep) * \real{0.2500}}
  >{\raggedright\arraybackslash}p{(\linewidth - 6\tabcolsep) * \real{0.2500}}@{}}
\toprule\noalign{}
\begin{minipage}[b]{\linewidth}\raggedright
Operator
\end{minipage} & \begin{minipage}[b]{\linewidth}\raggedright
Description
\end{minipage} & \begin{minipage}[b]{\linewidth}\raggedright
Example
\end{minipage} & \begin{minipage}[b]{\linewidth}\raggedright
Output
\end{minipage} \\
\midrule\noalign{}
\endhead
\bottomrule\noalign{}
\endlastfoot
\passthrough{\lstinline!in!} & Returns True if a value exists in
sequence & \passthrough{\lstinline!5 in [1,2,5]!} &
\passthrough{\lstinline!True!} \\
\passthrough{\lstinline!not in!} & Returns True if a value doesn't exist
& \passthrough{\lstinline!4 not in [1,2,5]!} &
\passthrough{\lstinline!True!} \\
\end{longtable}
}

\textbf{Common Usage:}

\begin{itemize}
\tightlist
\item
  Checking if an element exists in a list:
  \passthrough{\lstinline!if item in my\_list:!}
\item
  Checking if a key exists in dictionary:
  \passthrough{\lstinline!if key in my\_dict:!}
\item
  Checking if a substring exists:
  \passthrough{\lstinline!if "py" in "python":!}
\end{itemize}

\end{solutionbox}
\begin{mnemonicbox}
``IM-NOT'' (In Membership - NOT in Membership)

\end{mnemonicbox}
\subsection*{Question 2(b) [4 marks]}\label{q2b}

\textbf{Explain the need for continue and break statements.}

\begin{solutionbox}

{\def\LTcaptype{none} % do not increment counter
\begin{longtable}[]{@{}
  >{\raggedright\arraybackslash}p{(\linewidth - 6\tabcolsep) * \real{0.2500}}
  >{\raggedright\arraybackslash}p{(\linewidth - 6\tabcolsep) * \real{0.2500}}
  >{\raggedright\arraybackslash}p{(\linewidth - 6\tabcolsep) * \real{0.2500}}
  >{\raggedright\arraybackslash}p{(\linewidth - 6\tabcolsep) * \real{0.2500}}@{}}
\toprule\noalign{}
\begin{minipage}[b]{\linewidth}\raggedright
Statement
\end{minipage} & \begin{minipage}[b]{\linewidth}\raggedright
Purpose
\end{minipage} & \begin{minipage}[b]{\linewidth}\raggedright
Use Case
\end{minipage} & \begin{minipage}[b]{\linewidth}\raggedright
Example
\end{minipage} \\
\midrule\noalign{}
\endhead
\bottomrule\noalign{}
\endlastfoot
\passthrough{\lstinline!break!} & Terminates the loop immediately & Exit
loop when a condition is met & Finding an element \\
\passthrough{\lstinline!continue!} & Skips current iteration and jumps
to next & Skip processing for certain values & Filtering values \\
\end{longtable}
}

\textbf{Break Statement:}

\begin{itemize}
\tightlist
\item
  \textbf{Purpose}: Immediately exits the loop
\item
  \textbf{When to use}: When the required condition is achieved and
  further processing is unnecessary
\item
  \textbf{Example}: Finding a specific element in a list
\end{itemize}

\begin{lstlisting}[language=Python]
for num in range(1, 10):
    if num == 5:
        print("Found 5!")
        break
    print(num)
\end{lstlisting}

\textbf{Continue Statement:}

\begin{itemize}
\tightlist
\item
  \textbf{Purpose}: Skips the current iteration and proceeds to the next
\item
  \textbf{When to use}: When certain values should be skipped but the
  loop should continue
\item
  \textbf{Example}: Skipping even numbers in a loop
\end{itemize}

\begin{lstlisting}[language=Python]
for num in range(1, 10):
    if num % 2 == 0:
        continue
    print(num)  # Prints only odd numbers
\end{lstlisting}

\end{solutionbox}
\begin{mnemonicbox}
``BS-CE'' (Break Stops, Continue Excepts)

\end{mnemonicbox}
\subsection*{Question 2(c) [7 marks]}\label{q2c}

\textbf{Create a program to calculate total and average marks based on
four subject marks taken as input from user.}

\begin{solutionbox}

\begin{lstlisting}[language=Python]
# Program to calculate total and average marks
# Input marks for four subjects
subject1 = float(input("Enter marks for subject 1: "))
subject2 = float(input("Enter marks for subject 2: "))
subject3 = float(input("Enter marks for subject 3: "))
subject4 = float(input("Enter marks for subject 4: "))

# Calculate total and average
total_marks = subject1 + subject2 + subject3 + subject4
average_marks = total_marks / 4

# Display results
print(f"Total marks: {total_marks}")
print(f"Average marks: {average_marks}")
\end{lstlisting}

\textbf{Diagram:}

\includegraphics[width=1\linewidth,height=\textheight,keepaspectratio]{mermaid-f8e3c57d.pdf}

\textbf{Explanation:}

\begin{itemize}
\tightlist
\item
  \textbf{Input}: Get marks for four subjects from user
\item
  \textbf{Process}: Calculate total by adding all subject marks and
  average by dividing total by number of subjects
\item
  \textbf{Output}: Display total and average marks
\end{itemize}

\end{solutionbox}
\begin{mnemonicbox}
``IAPO'' (Input-Add-Process-Output)

\end{mnemonicbox}
\subsection*{Question 2(a) OR [3
marks]}\label{q2a}

\textbf{Write a short note on assignment operator.}

\begin{solutionbox}
The assignment operator in Python is used to assign
values to variables.

{\def\LTcaptype{none} % do not increment counter
\begin{longtable}[]{@{}
  >{\raggedright\arraybackslash}p{(\linewidth - 6\tabcolsep) * \real{0.2500}}
  >{\raggedright\arraybackslash}p{(\linewidth - 6\tabcolsep) * \real{0.2500}}
  >{\raggedright\arraybackslash}p{(\linewidth - 6\tabcolsep) * \real{0.2500}}
  >{\raggedright\arraybackslash}p{(\linewidth - 6\tabcolsep) * \real{0.2500}}@{}}
\toprule\noalign{}
\begin{minipage}[b]{\linewidth}\raggedright
Operator
\end{minipage} & \begin{minipage}[b]{\linewidth}\raggedright
Name
\end{minipage} & \begin{minipage}[b]{\linewidth}\raggedright
Description
\end{minipage} & \begin{minipage}[b]{\linewidth}\raggedright
Example
\end{minipage} \\
\midrule\noalign{}
\endhead
\bottomrule\noalign{}
\endlastfoot
\passthrough{\lstinline!=!} & Simple Assignment & Assigns right operand
value to left operand & \passthrough{\lstinline!x = 10!} \\
\passthrough{\lstinline!+=!} & Add AND & Adds right operand to left and
assigns result & \passthrough{\lstinline!x += 5!} (same as
\passthrough{\lstinline!x = x + 5!}) \\
\passthrough{\lstinline!-=!} & Subtract AND & Subtracts right operand
from left and assigns & \passthrough{\lstinline!x -= 3!} (same as
\passthrough{\lstinline!x = x - 3!}) \\
\passthrough{\lstinline!*=!} & Multiply AND & Multiplies left by right
and assigns result & \passthrough{\lstinline!x *= 2!} (same as
\passthrough{\lstinline!x = x * 2!}) \\
\passthrough{\lstinline!/=!} & Divide AND & Divides left by right and
assigns result & \passthrough{\lstinline!x /= 4!} (same as
\passthrough{\lstinline!x = x / 4!}) \\
\end{longtable}
}

\textbf{Compound assignment operators} combine an arithmetic operation
with assignment, making code more concise and readable.

\end{solutionbox}
\begin{mnemonicbox}
``SAME'' (Simple Assignment Makes Easy)

\end{mnemonicbox}
\subsection*{Question 2(b) OR [4
marks]}\label{q2b}

\textbf{Explain the use of for loop by giving syntax, flowchart and
example.}

\begin{solutionbox}

\textbf{Syntax of For Loop:}

\begin{lstlisting}[language=Python]
for variable in sequence:
    # code block to be executed
\end{lstlisting}

\textbf{Flowchart:}

\includegraphics[width=1\linewidth,height=\textheight,keepaspectratio]{mermaid-886d514a.pdf}

\textbf{Example:}

\begin{lstlisting}[language=Python]
# Print squares of numbers from 1 to 5
for num in range(1, 6):
    square = num ** 2
    print(f"{num} squared = {square}")
\end{lstlisting}

The \passthrough{\lstinline!for!} loop in Python is used for definite
iteration over a sequence (list, tuple, string, etc.) or other iterable
objects. It's particularly useful when you know the number of iterations
in advance.

\end{solutionbox}
\begin{mnemonicbox}
``SIFE'' (Sequence Iteration For Each item)

\end{mnemonicbox}
\subsection*{Question 2(c) OR [7
marks]}\label{q2c}

\textbf{Develop a code to find the square and cube of a given number
from user.}

\begin{solutionbox}

\begin{lstlisting}[language=Python]
# Program to find square and cube of a number
# Input number from user
num = float(input("Enter a number: "))

# Calculate square and cube
square = num ** 2
cube = num ** 3

# Display results
print(f"The number entered is: {num}")
print(f"Square of {num} is: {square}")
print(f"Cube of {num} is: {cube}")
\end{lstlisting}

\textbf{Diagram:}

\includegraphics[width=1\linewidth,height=\textheight,keepaspectratio]{mermaid-057b15fd.pdf}

\textbf{Explanation:}

\begin{itemize}
\tightlist
\item
  \textbf{Input}: Get a number from user
\item
  \textbf{Process}: Calculate square by raising to power 2, cube by
  raising to power 3
\item
  \textbf{Output}: Display the input number, its square and cube
\end{itemize}

\end{solutionbox}
\begin{mnemonicbox}
``ISCO'' (Input-Square-Cube-Output)

\end{mnemonicbox}
\subsection*{Question 3(a) [3 marks]}\label{q3a}

\textbf{Explain if-elif-else statement with flowchart and suitable
example.}

\begin{solutionbox}
The if-elif-else statement in Python allows for
conditional execution where multiple expressions are evaluated.

\textbf{Flowchart:}

\includegraphics[width=1\linewidth,height=\textheight,keepaspectratio]{mermaid-3c7d25b2.pdf}

\textbf{Example:}

\begin{lstlisting}[language=Python]
# Grade assignment based on marks
marks = 75

if marks >= 90:
    grade = "A"
elif marks >= 80:
    grade = "B"
elif marks >= 70:
    grade = "C"
elif marks >= 60:
    grade = "D"
else:
    grade = "F"

print(f"Your grade is: {grade}")
\end{lstlisting}

\end{solutionbox}
\begin{mnemonicbox}
``CITE'' (Check If Then Else)

\end{mnemonicbox}
\subsection*{Question 3(b) [4 marks]}\label{q3b}

\textbf{Explain how to define and call user defined function by giving
suitable example.}

\begin{solutionbox}

\textbf{Function Definition and Calling:}

{\def\LTcaptype{none} % do not increment counter
\begin{longtable}[]{@{}
  >{\raggedright\arraybackslash}p{(\linewidth - 4\tabcolsep) * \real{0.3333}}
  >{\raggedright\arraybackslash}p{(\linewidth - 4\tabcolsep) * \real{0.3333}}
  >{\raggedright\arraybackslash}p{(\linewidth - 4\tabcolsep) * \real{0.3333}}@{}}
\toprule\noalign{}
\begin{minipage}[b]{\linewidth}\raggedright
Aspect
\end{minipage} & \begin{minipage}[b]{\linewidth}\raggedright
Syntax
\end{minipage} & \begin{minipage}[b]{\linewidth}\raggedright
Purpose
\end{minipage} \\
\midrule\noalign{}
\endhead
\bottomrule\noalign{}
\endlastfoot
Definition & \passthrough{\lstinline!def function\_name(parameters):!} &
Creates a reusable block of code \\
Function Body & Indented code block & Contains the function's logic \\
Return Statement & \passthrough{\lstinline!return [expression]!} & Sends
a value back to the caller \\
Function Call & \passthrough{\lstinline!function\_name(arguments)!} &
Executes the function code \\
\end{longtable}
}

\textbf{Example of Defining and Calling a Function:}

\begin{lstlisting}[language=Python]
# Define a function to calculate area of rectangle
def calculate_area(length, width):
    """Calculate area of a rectangle with given length and width"""
    area = length * width
    return area

# Call the function
result = calculate_area(5, 3)
print(f"Area of rectangle: {result}")
\end{lstlisting}

\textbf{Explanation:}

\begin{itemize}
\tightlist
\item
  \textbf{Function Definition}: Use \passthrough{\lstinline!def!}
  keyword followed by function name and parameters
\item
  \textbf{Documentation}: Optional docstring describing the function
\item
  \textbf{Function Body}: Code that performs the task
\item
  \textbf{Return Statement}: Sends result back to caller
\item
  \textbf{Function Call}: Pass arguments to execute the function
\end{itemize}

\end{solutionbox}
\begin{mnemonicbox}
``DBRCA'' (Define-Body-Return-Call-Arguments)

\end{mnemonicbox}
\subsection*{Question 3(c) [7 marks]}\label{q3c}

\textbf{Develop a code to find the factorial of a given number.}

\begin{solutionbox}

\begin{lstlisting}[language=Python]
# Program to find factorial of a number
# Input number from user
num = int(input("Enter a positive integer: "))

# Initialize factorial
factorial = 1

# Check if number is negative, zero or positive
if num < 0:
    print("Factorial doesn't exist for negative numbers")
elif num == 0:
    print("Factorial of 0 is 1")
else:
    # Calculate factorial
    for i in range(1, num + 1):
        factorial *= i
    print(f"Factorial of {num} is {factorial}")
\end{lstlisting}

\textbf{Diagram:}

\includegraphics[width=1\linewidth,height=\textheight,keepaspectratio]{mermaid-29297ef0.pdf}

\textbf{Explanation:}

\begin{itemize}
\tightlist
\item
  \textbf{Input}: Get a number from user
\item
  \textbf{Check}: Validate if number is negative (factorial not
  defined), zero (factorial is 1), or positive
\item
  \textbf{Process}: For positive numbers, multiply factorial by each
  number from 1 to num
\item
  \textbf{Output}: Display the factorial result
\end{itemize}

\end{solutionbox}
\begin{mnemonicbox}
``MICE'' (Multiply Incrementally, Check Edge-cases)

\end{mnemonicbox}
\subsection*{Question 3(a) OR [3
marks]}\label{q3a}

\textbf{Explain nested loop using suitable example.}

\begin{solutionbox}
A nested loop is a loop inside another loop. The inner
loop completes all its iterations for each iteration of the outer loop.

\textbf{Diagram:}

\includegraphics[width=1\linewidth,height=\textheight,keepaspectratio]{mermaid-655ec370.pdf}

\textbf{Example:}

\begin{lstlisting}[language=Python]
# Print multiplication table from 1 to 3
for i in range(1, 4):  # Outer loop: 1 to 3
    print(f"Multiplication table for {i}:")
    for j in range(1, 6):  # Inner loop: 1 to 5
        print(f"{i} x {j} = {i*j}")
    print()  # Empty line after each table
\end{lstlisting}

\end{solutionbox}
\begin{mnemonicbox}
``LOFI'' (Loop Outside, Finish Inside)

\end{mnemonicbox}
\subsection*{Question 3(b) OR [4
marks]}\label{q3b}

\textbf{Explain return statement in function handling.}

\begin{solutionbox}

{\def\LTcaptype{none} % do not increment counter
\begin{longtable}[]{@{}
  >{\raggedright\arraybackslash}p{(\linewidth - 4\tabcolsep) * \real{0.3333}}
  >{\raggedright\arraybackslash}p{(\linewidth - 4\tabcolsep) * \real{0.3333}}
  >{\raggedright\arraybackslash}p{(\linewidth - 4\tabcolsep) * \real{0.3333}}@{}}
\toprule\noalign{}
\begin{minipage}[b]{\linewidth}\raggedright
Aspect
\end{minipage} & \begin{minipage}[b]{\linewidth}\raggedright
Description
\end{minipage} & \begin{minipage}[b]{\linewidth}\raggedright
Example
\end{minipage} \\
\midrule\noalign{}
\endhead
\bottomrule\noalign{}
\endlastfoot
Purpose & Send value back to caller &
\passthrough{\lstinline!return result!} \\
Multiple Returns & Return multiple values as tuple &
\passthrough{\lstinline!return x, y, z!} \\
Early Exit & Exit function before end &
\passthrough{\lstinline!if error: return None!} \\
No Return & Function returns None by default &
\passthrough{\lstinline!def show(): print("Hi")!} \\
\end{longtable}
}

The \passthrough{\lstinline!return!} statement in Python functions:

\begin{enumerate}
\tightlist
\item
  Terminates the function execution
\item
  Passes a value back to the function caller
\item
  Can return multiple values (as tuple)
\item
  Is optional (if omitted, function returns None)
\end{enumerate}

\textbf{Example:}

\begin{lstlisting}[language=Python]
def calculate_circle(radius):
    """Calculate area and circumference of a circle"""
    if radius < 0:
        return None  # Early exit for invalid input
    
    area = 3.14 * radius ** 2
    circumference = 2 * 3.14 * radius
    
    return area, circumference  # Return multiple values
    
# Function call
result = calculate_circle(5)
print(f"Area and circumference: {result}")
\end{lstlisting}

\end{solutionbox}
\begin{mnemonicbox}
``TERM'' (Terminate Execution, Return Multiple
values)

\end{mnemonicbox}
\subsection*{Question 3(c) OR [7
marks]}\label{q3c}

\textbf{Create a program to display the following patterns using loop
concept}

\begin{lstlisting}
A
AB
ABC
ABCD
ABCDE
\end{lstlisting}

\begin{solutionbox}

\begin{lstlisting}[language=Python]
# Program to print character pattern
# First pattern: A to E in triangle form

# Loop through rows (1 to 5)
for i in range(1, 6):
    # For each row, print characters from 'A' to required letter
    for j in range(i):
        # ASCII value of 'A' is 65, add j to get successive letters
        print(chr(65 + j), end="")
    # Move to next line after each row
    print()
\end{lstlisting}

\textbf{Diagram:}

\includegraphics[width=1\linewidth,height=\textheight,keepaspectratio]{mermaid-a3266fa6.pdf}

\textbf{Explanation:}

\begin{itemize}
\tightlist
\item
  \textbf{Outer loop}: Controls the number of rows (1 to 5)
\item
  \textbf{Inner loop}: For each row i, prints i characters starting from
  `A'
\item
  \textbf{Character generation}: Using ASCII value conversion (chr(65+j)
  gives `A', `B', etc.)
\item
  \textbf{Output formatting}: Using end=``\,'' to print characters in
  same line for each row
\end{itemize}

\end{solutionbox}
\begin{mnemonicbox}
``OICE'' (Outer-Inner-Character-Endline)

\end{mnemonicbox}
\subsection*{Question 4(a) [3 marks]}\label{q4a}

\textbf{Describe following built-in functions with suitable example.}
\textbf{i) max() ii) input() iii) pow()}

\begin{solutionbox}

{\def\LTcaptype{none} % do not increment counter
\begin{longtable}[]{@{}
  >{\raggedright\arraybackslash}p{(\linewidth - 6\tabcolsep) * \real{0.2500}}
  >{\raggedright\arraybackslash}p{(\linewidth - 6\tabcolsep) * \real{0.2500}}
  >{\raggedright\arraybackslash}p{(\linewidth - 6\tabcolsep) * \real{0.2500}}
  >{\raggedright\arraybackslash}p{(\linewidth - 6\tabcolsep) * \real{0.2500}}@{}}
\toprule\noalign{}
\begin{minipage}[b]{\linewidth}\raggedright
Function
\end{minipage} & \begin{minipage}[b]{\linewidth}\raggedright
Purpose
\end{minipage} & \begin{minipage}[b]{\linewidth}\raggedright
Syntax
\end{minipage} & \begin{minipage}[b]{\linewidth}\raggedright
Example
\end{minipage} \\
\midrule\noalign{}
\endhead
\bottomrule\noalign{}
\endlastfoot
\passthrough{\lstinline!max()!} & Returns largest item in an iterable or
largest of two or more arguments &
\passthrough{\lstinline!max(iterable)!} or
\passthrough{\lstinline!max(arg1, arg2, ...)!} &
\passthrough{\lstinline!max([1, 5, 3])!} returns
\passthrough{\lstinline!5!} \\
\passthrough{\lstinline!input()!} & Reads a line from input and returns
as string & \passthrough{\lstinline!input([prompt])!} &
\passthrough{\lstinline!input("Enter name: ")!} \\
\passthrough{\lstinline!pow()!} & Returns x to power y &
\passthrough{\lstinline!pow(x, y)!} &
\passthrough{\lstinline!pow(2, 3)!} returns
\passthrough{\lstinline!8!} \\
\end{longtable}
}

\textbf{Examples in code:}

\begin{lstlisting}[language=Python]
# max() function example
numbers = [10, 5, 20, 15]
maximum = max(numbers)
print(f"Maximum value: {maximum}")  # Output: Maximum value: 20

# input() function example
name = input("Enter your name: ")
print(f"Hello, {name}!")

# pow() function example
result = pow(2, 4)
print(f"2 raised to power 4 is: {result}")  # Output: 2 raised to power 4 is: 16
\end{lstlisting}

\end{solutionbox}
\begin{mnemonicbox}
``MIP'' (Max-Input-Power)

\end{mnemonicbox}
\subsection*{Question 4(b) [4 marks]}\label{q4b}

\textbf{Explain slicing of string by giving suitable example.}

\begin{solutionbox}

String slicing in Python is used to extract a substring from a string.

\textbf{Syntax:} \passthrough{\lstinline!string[start:end:step]!}

{\def\LTcaptype{none} % do not increment counter
\begin{longtable}[]{@{}
  >{\raggedright\arraybackslash}p{(\linewidth - 6\tabcolsep) * \real{0.2500}}
  >{\raggedright\arraybackslash}p{(\linewidth - 6\tabcolsep) * \real{0.2500}}
  >{\raggedright\arraybackslash}p{(\linewidth - 6\tabcolsep) * \real{0.2500}}
  >{\raggedright\arraybackslash}p{(\linewidth - 6\tabcolsep) * \real{0.2500}}@{}}
\toprule\noalign{}
\begin{minipage}[b]{\linewidth}\raggedright
Parameter
\end{minipage} & \begin{minipage}[b]{\linewidth}\raggedright
Description
\end{minipage} & \begin{minipage}[b]{\linewidth}\raggedright
Default
\end{minipage} & \begin{minipage}[b]{\linewidth}\raggedright
Example
\end{minipage} \\
\midrule\noalign{}
\endhead
\bottomrule\noalign{}
\endlastfoot
\passthrough{\lstinline!start!} & Starting index (inclusive) & 0 &
\passthrough{\lstinline!"Python"[1:]!} \rightarrow
\passthrough{\lstinline!"ython"!} \\
\passthrough{\lstinline!end!} & Ending index (exclusive) & Length of
string & \passthrough{\lstinline!"Python"[:3]!} \rightarrow
\passthrough{\lstinline!"Pyt"!} \\
\passthrough{\lstinline!step!} & Increment between characters & 1 &
\passthrough{\lstinline!"Python"[::2]!} \rightarrow
\passthrough{\lstinline!"Pto"!} \\
\end{longtable}
}

\textbf{Examples:}

\begin{lstlisting}[language=Python]
text = "Python Programming"

# Basic slicing
print(text[0:6])     # Output: "Python"
print(text[7:])      # Output: "Programming"
print(text[:6])      # Output: "Python"

# With step
print(text[::2])     # Output: "Pto rgamn"
print(text[0:10:2])  # Output: "Pto r"

# Negative indices (count from end)
print(text[-11:])    # Output: "Programming"
print(text[:-12])    # Output: "Python"

# Reverse a string
print(text[::-1])    # Output: "gnimmargorP nohtyP"
\end{lstlisting}

\end{solutionbox}
\begin{mnemonicbox}
``SES'' (Start-End-Step)

\end{mnemonicbox}
\subsection*{Question 4(c) [7 marks]}\label{q4c}

\textbf{Create a user defined function which prints cube of all the odd
numbers between 1 to 7.}

\begin{solutionbox}

\begin{lstlisting}[language=Python]
# Function to print cube of odd numbers in a range
def print_odd_cubes(start, end):
    """
    Print cube of all odd numbers between start and end (inclusive)
    """
    print(f"Cubes of odd numbers between {start} and {end}:")
    
    # Loop through the range
    for num in range(start, end + 1):
        # Check if number is odd
        if num % 2 != 0:
            # Calculate and print cube
            cube = num ** 3
            print(f"Cube of {num} is {cube}")

# Call the function to print odd cubes from 1 to 7
print_odd_cubes(1, 7)
\end{lstlisting}

\textbf{Diagram:}

\includegraphics[width=1\linewidth,height=\textheight,keepaspectratio]{mermaid-a36dbbc7.pdf}

\textbf{Explanation:}

\begin{itemize}
\tightlist
\item
  \textbf{Function Definition}: Create a function to process odd numbers
  in a range
\item
  \textbf{Loop}: Iterate through numbers from start to end
\item
  \textbf{Condition}: Check if number is odd using modulo operator
\item
  \textbf{Processing}: Calculate cube of odd numbers
\item
  \textbf{Output}: Display each odd number and its cube
\end{itemize}

\end{solutionbox}
\begin{mnemonicbox}
``FLOOP'' (Function-Loop-Odd-Output-Power)

\end{mnemonicbox}
\subsection*{Question 4(a) OR [3
marks]}\label{q4a}

\textbf{Explain random module with various functions.}

\begin{solutionbox}

The random module in Python provides functions for generating random
numbers and making random selections.

{\def\LTcaptype{none} % do not increment counter
\begin{longtable}[]{@{}
  >{\raggedright\arraybackslash}p{(\linewidth - 6\tabcolsep) * \real{0.2500}}
  >{\raggedright\arraybackslash}p{(\linewidth - 6\tabcolsep) * \real{0.2500}}
  >{\raggedright\arraybackslash}p{(\linewidth - 6\tabcolsep) * \real{0.2500}}
  >{\raggedright\arraybackslash}p{(\linewidth - 6\tabcolsep) * \real{0.2500}}@{}}
\toprule\noalign{}
\begin{minipage}[b]{\linewidth}\raggedright
Function
\end{minipage} & \begin{minipage}[b]{\linewidth}\raggedright
Description
\end{minipage} & \begin{minipage}[b]{\linewidth}\raggedright
Example
\end{minipage} & \begin{minipage}[b]{\linewidth}\raggedright
Result
\end{minipage} \\
\midrule\noalign{}
\endhead
\bottomrule\noalign{}
\endlastfoot
\passthrough{\lstinline!random()!} & Returns random float between 0 and
1 & \passthrough{\lstinline!random.random()!} &
\passthrough{\lstinline!0.7134346335849448!} \\
\passthrough{\lstinline!randint(a, b)!} & Returns random integer between
a and b (inclusive) & \passthrough{\lstinline!random.randint(1, 10)!} &
\passthrough{\lstinline!7!} \\
\passthrough{\lstinline!choice(seq)!} & Returns random element from
sequence &
\passthrough{\lstinline!random.choice(['red', 'green', 'blue'])!} &
\passthrough{\lstinline!'green'!} \\
\passthrough{\lstinline!shuffle(seq)!} & Shuffles a sequence in-place &
\passthrough{\lstinline!random.shuffle(my\_list)!} & No return value \\
\passthrough{\lstinline!sample(seq, k)!} & Returns k unique random
elements from sequence &
\passthrough{\lstinline!random.sample(range(1, 30), 5)!} &
\passthrough{\lstinline![3, 12, 21, 7, 25]!} \\
\end{longtable}
}

\textbf{Example:}

\begin{lstlisting}[language=Python]
import random

# Generate random float between 0 and 1
print(random.random())  

# Generate random integer between 1 and 10
print(random.randint(1, 10))  

# Select random element from list
colors = ["red", "green", "blue", "yellow"]
print(random.choice(colors))  

# Shuffle a list in-place
random.shuffle(colors)
print(colors)  

# Select 2 unique random elements
print(random.sample(colors, 2))  
\end{lstlisting}

\end{solutionbox}
\begin{mnemonicbox}
``RICES'' (Random-Integer-Choice-Elements-Shuffle)

\end{mnemonicbox}
\subsection*{Question 4(b) OR [4
marks]}\label{q4b}

\textbf{Discuss the following list functions.} \textbf{i. len() ii.
sum() iii. sort() iv. index()}

\begin{solutionbox}

{\def\LTcaptype{none} % do not increment counter
\begin{longtable}[]{@{}
  >{\raggedright\arraybackslash}p{(\linewidth - 8\tabcolsep) * \real{0.2000}}
  >{\raggedright\arraybackslash}p{(\linewidth - 8\tabcolsep) * \real{0.2000}}
  >{\raggedright\arraybackslash}p{(\linewidth - 8\tabcolsep) * \real{0.2000}}
  >{\raggedright\arraybackslash}p{(\linewidth - 8\tabcolsep) * \real{0.2000}}
  >{\raggedright\arraybackslash}p{(\linewidth - 8\tabcolsep) * \real{0.2000}}@{}}
\toprule\noalign{}
\begin{minipage}[b]{\linewidth}\raggedright
Function
\end{minipage} & \begin{minipage}[b]{\linewidth}\raggedright
Purpose
\end{minipage} & \begin{minipage}[b]{\linewidth}\raggedright
Syntax
\end{minipage} & \begin{minipage}[b]{\linewidth}\raggedright
Example
\end{minipage} & \begin{minipage}[b]{\linewidth}\raggedright
Output
\end{minipage} \\
\midrule\noalign{}
\endhead
\bottomrule\noalign{}
\endlastfoot
\passthrough{\lstinline!len()!} & Returns number of items in list &
\passthrough{\lstinline!len(list)!} &
\passthrough{\lstinline!len([1, 2, 3])!} &
\passthrough{\lstinline!3!} \\
\passthrough{\lstinline!sum()!} & Returns sum of all items in list &
\passthrough{\lstinline!sum(list)!} &
\passthrough{\lstinline!sum([1, 2, 3])!} &
\passthrough{\lstinline!6!} \\
\passthrough{\lstinline!sort()!} & Sorts list in-place &
\passthrough{\lstinline!list.sort()!} &
\passthrough{\lstinline![3, 1, 2].sort()!} & None (modifies original) \\
\passthrough{\lstinline!index()!} & Returns index of first occurrence &
\passthrough{\lstinline!list.index(value)!} &
\passthrough{\lstinline![10, 20, 30].index(20)!} &
\passthrough{\lstinline!1!} \\
\end{longtable}
}

\textbf{Examples:}

\begin{lstlisting}[language=Python]
# len() function
numbers = [5, 10, 15, 20, 25]
print(f"Length of list: {len(numbers)}")  # Output: 5

# sum() function
print(f"Sum of all items: {sum(numbers)}")  # Output: 75

# sort() function
mixed = [3, 1, 4, 2]
mixed.sort()  # Sorts in-place
print(f"Sorted list: {mixed}")  # Output: [1, 2, 3, 4]
mixed.sort(reverse=True)
print(f"Reverse sorted: {mixed}")  # Output: [4, 3, 2, 1]

# index() function
fruits = ["apple", "banana", "cherry", "apple"]
print(f"Index of 'banana': {fruits.index('banana')}")  # Output: 1
\end{lstlisting}

\end{solutionbox}
\begin{mnemonicbox}
``LSSI'' (Length-Sum-Sort-Index)

\end{mnemonicbox}
\subsection*{Question 4(c) OR [7
marks]}\label{q4c}

\textbf{Create a user-defined function to print the Fibonacci series of
0 to N numbers. (Where N is an integer number and passed as an
argument)}

\begin{solutionbox}

\begin{lstlisting}[language=Python]
# Function to print Fibonacci series up to N
def print_fibonacci(n):
    """
    Print Fibonacci series up to n terms
    Where 0th term is 0 and 1st term is 1
    """
    # Check if input is valid
    if n < 0:
        print("Please enter a positive integer")
        return
    
    # Initialize first two terms
    a, b = 0, 1
    count = 0
    
    print(f"Fibonacci series up to {n} terms:")
    
    # Print Fibonacci series
    while count < n:
        print(a, end=" ")
        # Update variables for next iteration
        next_term = a + b
        a = b
        b = next_term
        count += 1
\end{lstlisting}

\textbf{Diagram:}

\includegraphics[width=1\linewidth,height=\textheight,keepaspectratio]{mermaid-f5202f85.pdf}

\textbf{Explanation:}

\begin{itemize}
\tightlist
\item
  \textbf{Input Validation}: Check if N is a valid positive integer
\item
  \textbf{Initialize Variables}: Set first two Fibonacci terms
\item
  \textbf{Print Series}: Loop to print Fibonacci numbers
\item
  \textbf{Update Terms}: Calculate next term and shift values for next
  iteration
\item
  \textbf{Termination}: Stop when count reaches N
\end{itemize}

\end{solutionbox}
\begin{mnemonicbox}
``FIST'' (Fibonacci-Initialize-Shift-Terminate)

\end{mnemonicbox}
\subsection*{Question 5(a) [3 marks]}\label{q5a}

\textbf{Explain given string methods:} \textbf{i. count() ii. upper()
iii. replace()}

\begin{solutionbox}

{\def\LTcaptype{none} % do not increment counter
\begin{longtable}[]{@{}
  >{\raggedright\arraybackslash}p{(\linewidth - 8\tabcolsep) * \real{0.2000}}
  >{\raggedright\arraybackslash}p{(\linewidth - 8\tabcolsep) * \real{0.2000}}
  >{\raggedright\arraybackslash}p{(\linewidth - 8\tabcolsep) * \real{0.2000}}
  >{\raggedright\arraybackslash}p{(\linewidth - 8\tabcolsep) * \real{0.2000}}
  >{\raggedright\arraybackslash}p{(\linewidth - 8\tabcolsep) * \real{0.2000}}@{}}
\toprule\noalign{}
\begin{minipage}[b]{\linewidth}\raggedright
Method
\end{minipage} & \begin{minipage}[b]{\linewidth}\raggedright
Purpose
\end{minipage} & \begin{minipage}[b]{\linewidth}\raggedright
Syntax
\end{minipage} & \begin{minipage}[b]{\linewidth}\raggedright
Example
\end{minipage} & \begin{minipage}[b]{\linewidth}\raggedright
Output
\end{minipage} \\
\midrule\noalign{}
\endhead
\bottomrule\noalign{}
\endlastfoot
\passthrough{\lstinline!count()!} & Counts occurrences of substring &
\passthrough{\lstinline!str.count(substring)!} &
\passthrough{\lstinline!"hello".count("l")!} &
\passthrough{\lstinline!2!} \\
\passthrough{\lstinline!upper()!} & Converts string to uppercase &
\passthrough{\lstinline!str.upper()!} &
\passthrough{\lstinline!"hello".upper()!} &
\passthrough{\lstinline!"HELLO"!} \\
\passthrough{\lstinline!replace()!} & Replaces all occurrences of a
substring & \passthrough{\lstinline!str.replace(old, new)!} &
\passthrough{\lstinline!"hello".replace("l", "r")!} &
\passthrough{\lstinline!"herro"!} \\
\end{longtable}
}

\textbf{Examples:}

\begin{lstlisting}[language=Python]
text = "Python programming is fun and Python is easy to learn"

# count() method
print(f"Count of 'Python': {text.count('Python')}")  # Output: 2
print(f"Count of 'is': {text.count('is')}")  # Output: 2

# upper() method
print(f"Uppercase: {text.upper()}")  # Output: "PYTHON PROGRAMMING IS FUN AND PYTHON IS EASY TO LEARN"

# replace() method
print(f"Replace 'Python' with 'Java': {text.replace('Python', 'Java')}")
# Output: "Java programming is fun and Java is easy to learn"
\end{lstlisting}

\end{solutionbox}
\begin{mnemonicbox}
``CUR'' (Count-Upper-Replace)

\end{mnemonicbox}
\subsection*{Question 5(b) [4 marks]}\label{q5b}

\textbf{Explain tuple operation with example.}

\begin{solutionbox}

Tuples in Python are ordered, immutable collections enclosed in
parentheses.

{\def\LTcaptype{none} % do not increment counter
\begin{longtable}[]{@{}
  >{\raggedright\arraybackslash}p{(\linewidth - 6\tabcolsep) * \real{0.2500}}
  >{\raggedright\arraybackslash}p{(\linewidth - 6\tabcolsep) * \real{0.2500}}
  >{\raggedright\arraybackslash}p{(\linewidth - 6\tabcolsep) * \real{0.2500}}
  >{\raggedright\arraybackslash}p{(\linewidth - 6\tabcolsep) * \real{0.2500}}@{}}
\toprule\noalign{}
\begin{minipage}[b]{\linewidth}\raggedright
Operation
\end{minipage} & \begin{minipage}[b]{\linewidth}\raggedright
Description
\end{minipage} & \begin{minipage}[b]{\linewidth}\raggedright
Example
\end{minipage} & \begin{minipage}[b]{\linewidth}\raggedright
Result
\end{minipage} \\
\midrule\noalign{}
\endhead
\bottomrule\noalign{}
\endlastfoot
Creation & Define tuple with values &
\passthrough{\lstinline!t = (1, 2, 3)!} & Tuple with 3 items \\
Indexing & Access item by position & \passthrough{\lstinline!t[0]!} &
\passthrough{\lstinline!1!} \\
Slicing & Extract portion of tuple & \passthrough{\lstinline!t[1:3]!} &
\passthrough{\lstinline!(2, 3)!} \\
Concatenation & Join two tuples & \passthrough{\lstinline!t1 + t2!} &
Combined tuple \\
Repetition & Repeat tuple elements & \passthrough{\lstinline!t * 2!} &
Duplicated elements \\
\end{longtable}
}

\textbf{Examples:}

\begin{lstlisting}[language=Python]
# Create a tuple
fruits = ("apple", "banana", "cherry")
print(f"Fruits tuple: {fruits}")

# Access tuple items
print(f"First fruit: {fruits[0]}")  # Output: "apple"
print(f"Last fruit: {fruits[-1]}")  # Output: "cherry"

# Tuple slicing
print(f"First two fruits: {fruits[:2]}")  # Output: ("apple", "banana")

# Tuple concatenation
more_fruits = ("orange", "kiwi")
all_fruits = fruits + more_fruits
print(f"All fruits: {all_fruits}")  # Output: ("apple", "banana", "cherry", "orange", "kiwi")

# Tuple repetition
duplicated = fruits * 2
print(f"Duplicated: {duplicated}")  # Output: ("apple", "banana", "cherry", "apple", "banana", "cherry")

# Tuple functions
print(f"Length: {len(fruits)}")  # Output: 3
print(f"Max: {max(fruits)}")  # Output: "cherry" (alphabetical comparison)
print(f"Min: {min(fruits)}")  # Output: "apple" (alphabetical comparison)
\end{lstlisting}

\end{solutionbox}
\begin{mnemonicbox}
``ICSM'' (Immutable-Create-Slice-Merge)

\end{mnemonicbox}
\subsection*{Question 5(c) [7 marks]}\label{q5c}

\textbf{Develop a code to create two set and perform given operations
with those created set:} \textbf{i) Union Operation on Sets} \textbf{ii)
Intersection Operation on Sets} \textbf{iii) Difference Operation on
Sets} \textbf{iv) Symmetric Difference of Two Sets}

\begin{solutionbox}

\begin{lstlisting}[language=Python]
# Program to demonstrate set operations

# Create two sets
set_A = {1, 2, 3, 4, 5}
set_B = {4, 5, 6, 7, 8}

print(f"Set A: {set_A}")
print(f"Set B: {set_B}")

# i) Union Operation (A \cup B)
# Elements present in either A or B or both
union_result = set_A.union(set_B)  # OR set_A | set_B
print(f"\ni) Union of A and B (A \cup B): {union_result}")

# ii) Intersection Operation (A \cap B)
# Elements present in both A and B
intersection_result = set_A.intersection(set_B)  # OR set_A & set_B
print(f"ii) Intersection of A and B (A \cap B): {intersection_result}")

# iii) Difference Operation (A - B)
# Elements present in A but not in B
difference_result = set_A.difference(set_B)  # OR set_A - set_B
print(f"iii) Difference (A - B): {difference_result}")

# Alternative difference (B - A)
difference_alt = set_B.difference(set_A)  # OR set_B - set_A
print(f"    Difference (B - A): {difference_alt}")

# iv) Symmetric Difference (A △ B)
# Elements present in A or B but not in both
symmetric_difference = set_A.symmetric_difference(set_B)  # OR set_A ^ set_B
print(f"iv) Symmetric Difference (A △ B): {symmetric_difference}")
\end{lstlisting}

\textbf{Diagram:}

\includegraphics[width=1\linewidth,height=\textheight,keepaspectratio]{mermaid-147f610d.pdf}

\textbf{Explanation:}

\begin{itemize}
\tightlist
\item
  \textbf{Union}: All elements from both sets without duplicates (1, 2,
  3, 4, 5, 6, 7, 8)
\item
  \textbf{Intersection}: Common elements in both sets (4, 5)
\item
  \textbf{Difference (A-B)}: Elements in A but not in B (1, 2, 3)
\item
  \textbf{Difference (B-A)}: Elements in B but not in A (6, 7, 8)
\item
  \textbf{Symmetric Difference}: Elements in either A or B but not in
  both (1, 2, 3, 6, 7, 8)
\end{itemize}

\end{solutionbox}
\begin{mnemonicbox}
``UIDS'' (Union-Intersection-Difference-Symmetric)

\end{mnemonicbox}
\subsection*{Question 5(a) OR [3
marks]}\label{q5a}

\textbf{Define list and how it is created in python?}

\begin{solutionbox}
A list in Python is an ordered, mutable collection of
items that can be of different data types, enclosed in square brackets.

\textbf{Table of List Creation Methods:}

{\def\LTcaptype{none} % do not increment counter
\begin{longtable}[]{@{}
  >{\raggedright\arraybackslash}p{(\linewidth - 4\tabcolsep) * \real{0.3333}}
  >{\raggedright\arraybackslash}p{(\linewidth - 4\tabcolsep) * \real{0.3333}}
  >{\raggedright\arraybackslash}p{(\linewidth - 4\tabcolsep) * \real{0.3333}}@{}}
\toprule\noalign{}
\begin{minipage}[b]{\linewidth}\raggedright
Method
\end{minipage} & \begin{minipage}[b]{\linewidth}\raggedright
Description
\end{minipage} & \begin{minipage}[b]{\linewidth}\raggedright
Example
\end{minipage} \\
\midrule\noalign{}
\endhead
\bottomrule\noalign{}
\endlastfoot
Literal & Create using square brackets &
\passthrough{\lstinline!my\_list = [1, 2, 3]!} \\
Constructor & Create using list() function &
\passthrough{\lstinline!my\_list = list((1, 2, 3))!} \\
Comprehension & Create using a single line expression &
\passthrough{\lstinline!my\_list = [x for x in range(5)]!} \\
From iterable & Convert other iterables to list &
\passthrough{\lstinline!my\_list = list("abc")!} \\
Empty list & Create empty list and append later &
\passthrough{\lstinline!my\_list = []!} \\
\end{longtable}
}

\textbf{Examples:}

\begin{lstlisting}[language=Python]
# Create list using literals
numbers = [1, 2, 3, 4, 5]
mixed = [1, "hello", 3.14, True]

# Create using list() constructor
tuple_to_list = list((10, 20, 30))
string_to_list = list("Python")

# Create using list comprehension
squares = [x**2 for x in range(1, 6)]

# Create empty list and add values
empty_list = []
empty_list.append("first")
empty_list.append("second")

print(f"Numbers: {numbers}")
print(f"Mixed: {mixed}")
print(f"From tuple: {tuple_to_list}")
print(f"From string: {string_to_list}")
print(f"Squares: {squares}")
print(f"Built list: {empty_list}")
\end{lstlisting}

\end{solutionbox}
\begin{mnemonicbox}
``LCMIE'' (Literal-Constructor-Mixed-Iterable-Empty)

\end{mnemonicbox}
\subsection*{Question 5(b) OR [4
marks]}\label{q5b}

\textbf{Explain dictionary built-in function and methods.}

\begin{solutionbox}

Dictionary is a collection of key-value pairs enclosed in curly braces
\passthrough{\lstinline!\{\}!}.

{\def\LTcaptype{none} % do not increment counter
\begin{longtable}[]{@{}
  >{\raggedright\arraybackslash}p{(\linewidth - 6\tabcolsep) * \real{0.2500}}
  >{\raggedright\arraybackslash}p{(\linewidth - 6\tabcolsep) * \real{0.2500}}
  >{\raggedright\arraybackslash}p{(\linewidth - 6\tabcolsep) * \real{0.2500}}
  >{\raggedright\arraybackslash}p{(\linewidth - 6\tabcolsep) * \real{0.2500}}@{}}
\toprule\noalign{}
\begin{minipage}[b]{\linewidth}\raggedright
Function/Method
\end{minipage} & \begin{minipage}[b]{\linewidth}\raggedright
Description
\end{minipage} & \begin{minipage}[b]{\linewidth}\raggedright
Example
\end{minipage} & \begin{minipage}[b]{\linewidth}\raggedright
Result
\end{minipage} \\
\midrule\noalign{}
\endhead
\bottomrule\noalign{}
\endlastfoot
\passthrough{\lstinline!dict()!} & Creates a dictionary &
\passthrough{\lstinline!dict(name='John', age=25)!} &
\passthrough{\lstinline!\{'name': 'John', 'age': 25\}!} \\
\passthrough{\lstinline!len()!} & Returns number of items &
\passthrough{\lstinline!len(my\_dict)!} & Integer count \\
\passthrough{\lstinline!keys()!} & Returns view of all keys &
\passthrough{\lstinline!my\_dict.keys()!} & Dictionary view object \\
\passthrough{\lstinline!values()!} & Returns view of all values &
\passthrough{\lstinline!my\_dict.values()!} & Dictionary view object \\
\passthrough{\lstinline!items()!} & Returns view of (key, value) pairs &
\passthrough{\lstinline!my\_dict.items()!} & Dictionary view object \\
\passthrough{\lstinline!get()!} & Returns value for key, or default &
\passthrough{\lstinline!my\_dict.get('key', 'default')!} & Value or
default \\
\passthrough{\lstinline!update()!} & Updates dict with keys/values from
another & \passthrough{\lstinline!my\_dict.update(other\_dict)!} & None
(updates in-place) \\
\passthrough{\lstinline!pop()!} & Removes item with key and returns
value & \passthrough{\lstinline!my\_dict.pop('key')!} & Value of removed
item \\
\end{longtable}
}

\textbf{Examples:}

\begin{lstlisting}[language=Python]
# Create a dictionary
student = {
    'name': 'John',
    'age': 20,
    'courses': ['Math', 'Science']
}

# Built-in functions
print(f"Length: {len(student)}")  # Output: 3

# Dictionary methods
print(f"Keys: {student.keys()}")
print(f"Values: {student.values()}")
print(f"Items: {student.items()}")

# Get method with default
print(f"Get grade (with default): {student.get('grade', 'N/A')}")

# Update dictionary
student.update({'grade': 'A', 'age': 21})
print(f"After update: {student}")

# Pop method
removed_item = student.pop('age')
print(f"Removed item: {removed_item}")
print(f"After pop: {student}")
\end{lstlisting}

\end{solutionbox}
\begin{mnemonicbox}
``LKVIGUP'' (Length-Keys-Values-Items-Get-Update-Pop)

\end{mnemonicbox}
\subsection*{Question 5(c) OR [7
marks]}\label{q5c}

\textbf{Develop python code to create list of prime and non-prime
numbers in range 1 to 50.}

\begin{solutionbox}

\begin{lstlisting}[language=Python]
# Program to create list of prime and non-prime numbers from 1 to 50

def is_prime(num):
    """
    Check if a number is prime
    Returns True if prime, False otherwise
    """
    # 1 is not a prime number
    if num <= 1:
        return False
    
    # 2 is a prime number
    if num == 2:
        return True
    
    # Even numbers greater than 2 are not prime
    if num % 2 == 0:
        return False
    
    # Check odd divisors up to square root of num
    # (optimization: we only need to check up to sqrt(num))
    for i in range(3, int(num**0.5) + 1, 2):
if num %

i == 0:

            return False
    
    return True

# Initialize empty lists for prime and non-prime numbers
prime_numbers = []
non_prime_numbers = []

# Check each number from 1 to 50
for num in range(1, 51):
    if is_prime(num):
        prime_numbers.append(num)
    else:
        non_prime_numbers.append(num)

# Display results
print(f"Prime numbers from 1 to 50: {prime_numbers}")
print(f"Non-prime numbers from 1 to 50: {non_prime_numbers}")
\end{lstlisting}

\textbf{Diagram:}

\includegraphics[width=1\linewidth,height=\textheight,keepaspectratio]{mermaid-d0617f19.pdf}

\textbf{Explanation:}

\begin{itemize}
\tightlist
\item
  \textbf{Helper Function}: \passthrough{\lstinline!is\_prime()!}
  efficiently checks if a number is prime
\item
  \textbf{Optimization}: Only checks divisibility up to square root of
  number
\item
  \textbf{Classification}: Sort numbers into prime or non-prime lists
\item
  \textbf{Output}: Display both lists at the end
\end{itemize}

\textbf{Prime numbers (from 1 to 50):} 2, 3, 5, 7, 11, 13, 17, 19, 23,
29, 31, 37, 41, 43, 47 \textbf{Non-prime numbers (from 1 to 50):} 1, 4,
6, 8, 9, 10, 12, 14, 15, 16, 18, 20, 21, 22, 24, 25, 26, 27, 28, 30, 32,
33, 34, 35, 36, 38, 39, 40, 42, 44, 45, 46, 48, 49, 50

\end{solutionbox}
\begin{mnemonicbox}
``POEMS''
(Prime-Optimization-Efficient-Modulo-Sorting)

\end{mnemonicbox}

\end{document}
