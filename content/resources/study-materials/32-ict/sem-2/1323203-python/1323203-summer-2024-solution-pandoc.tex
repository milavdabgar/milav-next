\documentclass[10pt,a4paper]{article}

% content/resources/templates/preamble.tex
\usepackage[margin=0.6in]{geometry}
\author{Milav Dabgar}
\usepackage{amsmath,amssymb,amsthm}
\usepackage{booktabs}
\usepackage{multirow}
\usepackage{xcolor}
\usepackage{tcolorbox}
\tcbuselibrary{breakable,skins}
\usepackage[colorlinks=true,linkcolor=blue]{hyperref}
\usepackage{titlesec}
\usepackage{enumitem}
\usepackage{tikz}
\usepackage{pgfplots}
\usepackage{circuitikz}
\usepackage[version=4]{mhchem}
\usepackage{longtable}
\usepackage{array}
\usepackage{float}
\usepackage{caption}
\usepackage{listings}

\lstset{
  basicstyle=\small\ttfamily,
  breaklines=true,
  breakatwhitespace=false,
  postbreak=\mbox{\textcolor{red}{$\hookrightarrow$}\space},
  float=false,
  numbers=left,
  numberstyle=\tiny\color{gray},
  numbersep=10pt,
  xleftmargin=2em,
  keywordstyle=\color{blue},
  commentstyle=\color{green!60!black},
  stringstyle=\color{purple},
  backgroundcolor=\color{gray!5},
  showstringspaces=false,
  tabsize=2,
  captionpos=b,
  keepspaces=true,
  columns=flexible
}

\pgfplotsset{compat=1.18}
\usetikzlibrary{shapes,arrows,positioning,calc,patterns,decorations.pathmorphing,decorations.markings,arrows.meta}

% Color scheme
\definecolor{headcolor}{RGB}{0,102,204}
\definecolor{keycolor}{RGB}{220,20,60}
\definecolor{solutioncolor}{RGB}{34,139,34}
\definecolor{mnemoniccolor}{RGB}{148,0,211}
\definecolor{codecolor}{RGB}{0,0,100}

% Spacing
\setlength{\parskip}{3pt}
\setlist[itemize]{nosep}
\setlist[enumerate]{nosep}

% Title formatting
\titleformat{\section}{\Large\bfseries\color{headcolor}}{\thesection}{1em}{}
\titleformat{\subsection}{\large\bfseries\color{headcolor}}{\thesubsection}{1em}{}

% Pandoc tightlist compatibility
\providecommand{\tightlist}{%
  \setlength{\itemsep}{0pt}\setlength{\parskip}{0pt}}

% Pandoc longtable compatibility
\newcounter{none}
\def\thenone{}


% content/resources/templates/english-boxes.tex
% This file is currently empty - it exists to maintain consistency with the import structure.
% Add custom environments here if needed in the future.


\begin{document}

\begin{center}
{\Huge\bfseries\color{headcolor} Subject Name Solutions}\\[5pt]
{\LARGE 1323203 -- Summer 2024}\\[3pt]
{\large Semester 1 Study Material}\\[3pt]
{\normalsize\textit{Detailed Solutions and Explanations}}
\end{center}

\vspace{10pt}

\subsection*{Question 1(a) [3 marks]}\label{q1a}

\textbf{Lists the Importance of flowchart and algorithm}

\begin{solutionbox}

{\def\LTcaptype{none} % do not increment counter
\begin{longtable}[]{@{}
  >{\raggedright\arraybackslash}p{(\linewidth - 2\tabcolsep) * \real{0.5000}}
  >{\raggedright\arraybackslash}p{(\linewidth - 2\tabcolsep) * \real{0.5000}}@{}}
\toprule\noalign{}
\begin{minipage}[b]{\linewidth}\raggedright
Importance of Flowchart
\end{minipage} & \begin{minipage}[b]{\linewidth}\raggedright
Importance of Algorithm
\end{minipage} \\
\midrule\noalign{}
\endhead
\bottomrule\noalign{}
\endlastfoot
Visual representation of program logic & Step-by-step procedure to solve
a problem \\
Easier to debug and identify errors & Language-independent solution
approach \\
Helps in understanding complex processes & Serves as a foundation for
programming \\
Improves communication among team members & Defines logic before coding
begins \\
\end{longtable}
}

\end{solutionbox}
\begin{mnemonicbox}
``VASE Decisions'' - Visualize, Analyze, Sequence,
Execute

\end{mnemonicbox}
\subsection*{Question 1(b) [4 marks]}\label{q1b}

\textbf{Draw a flowchart to find the entered number is even or odd.}

\begin{solutionbox}

\includegraphics[width=1\linewidth,height=\textheight,keepaspectratio]{mermaid-9e00195f.pdf}

\textbf{Key Steps:}

\begin{itemize}
\tightlist
\item
  \textbf{Input collection}: Get number from user
\item
  \textbf{Modulo operation}: Divide by 2 and check remainder
\item
  \textbf{Conditional output}: Display result based on remainder
\end{itemize}

\end{solutionbox}
\begin{mnemonicbox}
``MODE'' - Modulo Operation Determines Evenness

\end{mnemonicbox}
\subsection*{Question 1(c) [7 marks]}\label{q1c}

\textbf{List out all Logical operators and explain each by giving python
code example.}

\begin{solutionbox}

{\def\LTcaptype{none} % do not increment counter
\begin{longtable}[]{@{}
  >{\raggedright\arraybackslash}p{(\linewidth - 6\tabcolsep) * \real{0.2500}}
  >{\raggedright\arraybackslash}p{(\linewidth - 6\tabcolsep) * \real{0.3250}}
  >{\raggedright\arraybackslash}p{(\linewidth - 6\tabcolsep) * \real{0.2250}}
  >{\raggedright\arraybackslash}p{(\linewidth - 6\tabcolsep) * \real{0.2000}}@{}}
\toprule\noalign{}
\begin{minipage}[b]{\linewidth}\raggedright
Operator
\end{minipage} & \begin{minipage}[b]{\linewidth}\raggedright
Description
\end{minipage} & \begin{minipage}[b]{\linewidth}\raggedright
Example
\end{minipage} & \begin{minipage}[b]{\linewidth}\raggedright
Output
\end{minipage} \\
\midrule\noalign{}
\endhead
\bottomrule\noalign{}
\endlastfoot
\passthrough{\lstinline!and!} & Returns True if both statements are true
& \passthrough{\lstinline!x = 5; print(x > 3 and x < 10)!} &
\passthrough{\lstinline!True!} \\
\passthrough{\lstinline!or!} & Returns True if one of the statements is
true & \passthrough{\lstinline!x = 5; print(x > 10 or

x == 5)!} &

\passthrough{\lstinline!True!} \\
\passthrough{\lstinline!not!} & Reverse the result, returns False if
result is true & \passthrough{\lstinline!x = 5; print(not(x > 3))!} &
\passthrough{\lstinline!False!} \\
\end{longtable}
}

\textbf{Code Example:}

\begin{lstlisting}[language=Python]
# Logical AND example
age = 25
income = 50000
print("Loan eligibility:", age > 18 and income > 30000)  # True

# Logical OR example
has_credit_card = False
has_cash = True
print("Can purchase:", has_credit_card or has_cash)  # True

# Logical NOT example
is_holiday = False
print("Should work today:", not is_holiday)  # True
\end{lstlisting}

\end{solutionbox}
\begin{mnemonicbox}
``AON Clarity'' - And, Or, Not for logical clarity

\end{mnemonicbox}
\subsection*{Question 1(c) OR [7
marks]}\label{q1c}

\textbf{Develop a Program that can calculate simple interest and
compound interest on given data.}

\begin{solutionbox}

\begin{lstlisting}[language=Python]
# Program to calculate Simple and Compound Interest

# Input values
principal = float(input("Enter principal amount: "))
rate = float(input("Enter rate of interest (in %): "))
time = float(input("Enter time period (in years): "))

# Calculate Simple Interest
simple_interest = (principal * rate * time) / 100

# Calculate Compound Interest
compound_interest = principal * ((1 + rate/100) ** time - 1)

# Display results
print("Simple Interest:", round(simple_interest, 2))
print("Compound Interest:", round(compound_interest, 2))
\end{lstlisting}

\textbf{Key Formulas:}

\begin{itemize}
\tightlist
\item
  \textbf{Simple Interest (SI)}: Principal \times Rate \times Time / 100
\item
  \textbf{Compound Interest (CI)}: Principal \times ((1 + Rate/100)\^{}Time -
  1)
\end{itemize}

\end{solutionbox}
\begin{mnemonicbox}
``PRT Money Grows'' - Principal, Rate, Time make
money grow

\end{mnemonicbox}
\subsection*{Question 2(a) [3 marks]}\label{q2a}

\textbf{Create a Program to find a minimum number among the given three
numbers.}

\begin{solutionbox}

\begin{lstlisting}[language=Python]
# Program to find minimum of three numbers

# Input three numbers
num1 = float(input("Enter first number: "))
num2 = float(input("Enter second number: "))
num3 = float(input("Enter third number: "))

# Find minimum using built-in min() function
minimum = min(num1, num2, num3)

# Display result
print("Minimum number is:", minimum)
\end{lstlisting}

\end{solutionbox}
\begin{mnemonicbox}
``MIN Finds Least'' - Minimum Is Numerically Found
with Least

\end{mnemonicbox}
\subsection*{Question 2(b) [4 marks]}\label{q2b}

\textbf{Define pseudocode. Write pseudocode to find Largest of three
numbers x, y and z.}

\begin{solutionbox}

{\def\LTcaptype{none} % do not increment counter
\begin{longtable}[]{@{}
  >{\raggedright\arraybackslash}p{(\linewidth - 0\tabcolsep) * \real{1.0000}}@{}}
\toprule\noalign{}
\begin{minipage}[b]{\linewidth}\raggedright
Pseudocode Definition
\end{minipage} \\
\midrule\noalign{}
\endhead
\bottomrule\noalign{}
\endlastfoot
A detailed yet readable description of what a computer program must do,
expressed in a formally-styled natural language rather than in a
programming language. \\
\end{longtable}
}

\textbf{Pseudocode for finding largest of three numbers:}

\begin{lstlisting}
BEGIN
    INPUT x, y, z
    SET largest = x
    
    IF y > largest THEN
        SET largest = y
    END IF
    
    IF z > largest THEN
        SET largest = z
    END IF
    
    OUTPUT "Largest number is: ", largest
END
\end{lstlisting}

\end{solutionbox}
\begin{mnemonicbox}
``PIE Writing'' - Program Ideas Expressed in simple
writing

\end{mnemonicbox}
\subsection*{Question 2(c) [7 marks]}\label{q2c}

\textbf{Explain While loop in python with its syntax, flowchart and with
python code example.}

\begin{solutionbox}

\textbf{Syntax:}

\begin{lstlisting}[language=Python]
while condition:
    # code to be executed
\end{lstlisting}

\textbf{Flowchart:}

\includegraphics[width=1\linewidth,height=\textheight,keepaspectratio]{mermaid-c149e615.pdf}

\textbf{Code Example:}

\begin{lstlisting}[language=Python]
# Print first 5 natural numbers using while loop
count = 1

while count <= 5:
    print(count)
    count += 1  # Increment counter

# Output:
# 1
# 2
# 3
# 4
# 5
\end{lstlisting}

\textbf{Key Characteristics:}

\begin{itemize}
\tightlist
\item
  \textbf{Entry controlled}: Condition checked before loop execution
\item
  \textbf{Initialization}: Variables set before the loop
\item
  \textbf{Updation}: Variables updated inside the loop
\item
  \textbf{Termination}: Loop exits when condition becomes False
\end{itemize}

\end{solutionbox}
\begin{mnemonicbox}
``IUTE Loop'' - Initialize, Update, Test for Exit

\end{mnemonicbox}
\subsection*{Question 2(a) OR [3
marks]}\label{q2a}

\textbf{Describe continue statement in python in brief.}

\begin{solutionbox}

{\def\LTcaptype{none} % do not increment counter
\begin{longtable}[]{@{}
  >{\raggedright\arraybackslash}p{(\linewidth - 0\tabcolsep) * \real{1.0000}}@{}}
\toprule\noalign{}
\begin{minipage}[b]{\linewidth}\raggedright
Continue Statement in Python
\end{minipage} \\
\midrule\noalign{}
\endhead
\bottomrule\noalign{}
\endlastfoot
The continue statement skips the current iteration of a loop and
continues with the next iteration \\
When encountered, the code inside the loop following the continue
statement is skipped \\
Useful for skipping specific conditions while keeping the loop
running \\
\end{longtable}
}

\textbf{Code Example:}

\begin{lstlisting}[language=Python]
# Skip printing even numbers
for i in range(1, 6):
    if i % 2 == 0:
        continue
    print(i)  # Prints only 1, 3, 5
\end{lstlisting}

\end{solutionbox}
\begin{mnemonicbox}
``SKIP Ahead'' - Skip Keeping Iteration Process

\end{mnemonicbox}
\subsection*{Question 2(b) OR [4
marks]}\label{q2b}

\textbf{What is the output of the following code:}

\begin{lstlisting}[language=Python]
x=8
y=2
print (x*y)
print (x ** y)
print (x % y)
print(x>y)
\end{lstlisting}

\begin{solutionbox}

{\def\LTcaptype{none} % do not increment counter
\begin{longtable}[]{@{}lll@{}}
\toprule\noalign{}
Operation & Result & Explanation \\
\midrule\noalign{}
\endhead
\bottomrule\noalign{}
\endlastfoot
\passthrough{\lstinline!x*y!} & \passthrough{\lstinline!16!} &
Multiplication: 8 \times 2 = 16 \\
\passthrough{\lstinline!x**y!} & \passthrough{\lstinline!64!} &
Exponentiation: 8^{2} = 64 \\
\passthrough{\lstinline!x\%y!} & \passthrough{\lstinline!0!} & Modulo
(remainder): 8 \div 2 = 4 with remainder 0 \\
\passthrough{\lstinline!x>y!} & \passthrough{\lstinline!True!} &
Comparison: 8 \textgreater{} 2 is True \\
\end{longtable}
}

\end{solutionbox}
\begin{mnemonicbox}
``MEMO'' - Multiply, Exponent, Modulo, Operator
comparison

\end{mnemonicbox}
\subsection*{Question 2(c) OR [7
marks]}\label{q2c}

\textbf{Explain if-elif-else Ladder in python with its syntax, flowchart
and with python code example.}

\begin{solutionbox}

\textbf{Syntax:}

\begin{lstlisting}[language=Python]
if condition1:
    # code block 1
elif condition2:
    # code block 2
elif condition3:
    # code block 3
else:
    # code block 4
\end{lstlisting}

\textbf{Flowchart:}

\includegraphics[width=1\linewidth,height=\textheight,keepaspectratio]{mermaid-5d5d80dc.pdf}

\textbf{Code Example:}

\begin{lstlisting}[language=Python]
# Grade calculation based on marks
marks = 75

if marks >= 90:
    grade = "A+"
elif marks >= 80:
    grade = "A"
elif marks >= 70:
    grade = "B"
elif marks >= 60:
    grade = "C"
else:
    grade = "D"

print("Grade:", grade)  # Output: Grade: B
\end{lstlisting}

\textbf{Key Characteristics:}

\begin{itemize}
\tightlist
\item
  \textbf{Sequential evaluation}: Conditions checked from top to bottom
\item
  \textbf{Exclusive execution}: Only one block executes
\item
  \textbf{Default action}: Else block executes if no conditions are True
\end{itemize}

\end{solutionbox}
\begin{mnemonicbox}
``SEEP Logic'' - Sequential Evaluation with Exclusive
Path

\end{mnemonicbox}
\subsection*{Question 3(a) [3 marks]}\label{q3a}

\textbf{Write a Python program to print odd numbers between 1 to 20
using loops.}

\begin{solutionbox}

\begin{lstlisting}[language=Python]
# Program to print odd numbers between 1 to 20

# Using for loop with range and step
for number in range(1, 21, 2):
    print(number, end=" ")

# Output: 1 3 5 7 9 11 13 15 17 19
\end{lstlisting}

\textbf{Alternate approach:}

\begin{lstlisting}[language=Python]
# Using for loop with if condition
for number in range(1, 21):
    if number % 2 != 0:
        print(number, end=" ")
\end{lstlisting}

\end{solutionbox}
\begin{mnemonicbox}
``STEO'' - Skip Two, Extract Odds

\end{mnemonicbox}
\subsection*{Question 3(b) [4 marks]}\label{q3b}

\textbf{Explain Nested if statement in brief.}

\begin{solutionbox}

{\def\LTcaptype{none} % do not increment counter
\begin{longtable}[]{@{}l@{}}
\toprule\noalign{}
Nested if Statement \\
\midrule\noalign{}
\endhead
\bottomrule\noalign{}
\endlastfoot
An if statement inside another if statement \\
Allows for more complex conditional logic \\
Inner if only evaluated when outer if is True \\
Can have multiple levels of nesting \\
\end{longtable}
}

\textbf{Code Example:}

\begin{lstlisting}[language=Python]
age = 25
income = 50000

if age > 18:
    print("Adult")
    if income > 30000:
        print("Eligible for credit card")
    else:
        print("Not eligible for credit card")
else:
    print("Minor")
\end{lstlisting}

\end{solutionbox}
\begin{mnemonicbox}
``LION'' - Layered If-statements Operating Nested

\end{mnemonicbox}
\subsection*{Question 3(c) [7 marks]}\label{q3c}

\textbf{Using a user-defined function write a Program to check entered
number is an `Armstrong number' or a palindrome in which number is
passed as argument in calling function.}

\begin{solutionbox}

\begin{lstlisting}[language=Python]
# Program to check Armstrong number or palindrome

def check_number(num):
    # Check if Armstrong number
    # An Armstrong number is one where sum of each digit raised to power of
    # total digits equals the original number
    temp = num
    digits = len(str(num))
    sum = 0
    
    while temp > 0:
        digit = temp % 10
        sum += digit ** digits
        temp //= 10
    
    is_armstrong = (sum == num)
    
    # Check if palindrome
    # A palindrome reads the same backward as forward
    is_palindrome = (str(num) == str(num)[::-1])
    
    # Return results
    return is_armstrong, is_palindrome

# Get input from user
number = int(input("Enter a number: "))

# Call function and display results
armstrong, palindrome = check_number(number)

if armstrong:
    print(number, "is an Armstrong number")
else:
    print(number, "is not an Armstrong number")
    
if palindrome:
    print(number, "is a Palindrome")
else:
    print(number, "is not a Palindrome")
\end{lstlisting}

\textbf{Armstrong Examples:}

\begin{itemize}
\tightlist
\item
  153: 1^{3} + 5^{3} + 3^{3} = 1 + 125 + 27 = 153 ✓
\item
  370: 3^{3} + 7^{3} + 0^{3} = 27 + 343 + 0 = 370 ✓
\end{itemize}

\textbf{Palindrome Examples:}

\begin{itemize}
\tightlist
\item
  121: Same forward and backward ✓
\item
  123: Not same backward (321) ✗
\end{itemize}

\end{solutionbox}
\begin{mnemonicbox}
``APTEST'' - Armstrong Palindrome Test Equal Sum Test

\end{mnemonicbox}
\subsection*{Question 3(a) OR [3
marks]}\label{q3a}

\textbf{Write a python program to find sum of 1 to 100.}

\begin{solutionbox}

\begin{lstlisting}[language=Python]
# Program to find sum of numbers from 1 to 100

# Method 1: Using loop
total = 0
for num in range(1, 101):
    total += num
print("Sum using loop:", total)

# Method 2: Using formula n(n+1)/2
n = 100
sum_formula = n * (n + 1) // 2
print("Sum using formula:", sum_formula)

# Output: 
# Sum using loop: 5050
# Sum using formula: 5050
\end{lstlisting}

\end{solutionbox}
\begin{mnemonicbox}
``SUM Formula'' - Sum Using Mathematical Formula

\end{mnemonicbox}
\subsection*{Question 3(b) OR [4
marks]}\label{q3b}

\textbf{Write a python program to print the following pattern.}

\begin{lstlisting}
1
2 3
4 5 6
7 8 9 10
\end{lstlisting}

\begin{solutionbox}

\begin{lstlisting}[language=Python]
# Program to print the number pattern

num = 1
for i in range(1, 5):  # 4 rows
    for j in range(i):  # columns equal to row number
        print(num, end=" ")
        num += 1
    print()  # New line after each row
\end{lstlisting}

\textbf{Pattern Logic:}

\begin{itemize}
\tightlist
\item
  \textbf{Row 1}: 1 number (1)
\item
  \textbf{Row 2}: 2 numbers (2, 3)
\item
  \textbf{Row 3}: 3 numbers (4, 5, 6)
\item
  \textbf{Row 4}: 4 numbers (7, 8, 9, 10)
\end{itemize}

\end{solutionbox}
\begin{mnemonicbox}
``CNIR'' - Counter Number Increases with Rows

\end{mnemonicbox}
\subsection*{Question 3(c) OR [7
marks]}\label{q3c}

\textbf{Write a Program using the function that reverses the entered
value.}

\begin{solutionbox}

\begin{lstlisting}[language=Python]
# Program to reverse entered value using functions

def reverse_number(num):
    """Function to reverse an integer number"""
    return int(str(num)[::-1])

def reverse_string(text):
    """Function to reverse a string"""
    return text[::-1]

# Main program
def main():
    choice = input("What do you want to reverse? (n for number, s for string): ")
    
    if choice.lower() == 'n':
        num = int(input("Enter a number: "))
        print("Reversed number:", reverse_number(num))
    elif choice.lower() == 's':
        text = input("Enter a string: ")
        print("Reversed string:", reverse_string(text))
    else:
        print("Invalid choice!")

# Call the main function
main()
\end{lstlisting}

\textbf{Alternate Method for Number Reversal:}

\begin{lstlisting}[language=Python]
def reverse_number_algorithm(num):
    reversed_num = 0
    while num > 0:
        digit = num % 10
        reversed_num = reversed_num * 10 + digit
        num //= 10
    return reversed_num
\end{lstlisting}

\end{solutionbox}
\begin{mnemonicbox}
``FLIP Digits'' - Function Logic Inverts Position of
Digits

\end{mnemonicbox}
\subsection*{Question 4(a) [3 marks]}\label{q4a}

\textbf{Describe python math module with proper python code example.}

\begin{solutionbox}

{\def\LTcaptype{none} % do not increment counter
\begin{longtable}[]{@{}l@{}}
\toprule\noalign{}
Python Math Module Features \\
\midrule\noalign{}
\endhead
\bottomrule\noalign{}
\endlastfoot
Provides mathematical functions and constants \\
Includes trigonometric, logarithmic, and other functions \\
Contains mathematical constants like pi and e \\
Requires import before use \\
\end{longtable}
}

\textbf{Code Example:}

\begin{lstlisting}[language=Python]
import math

# Constants
print("Value of pi:", math.pi)  # 3.141592653589793
print("Value of e:", math.e)    # 2.718281828459045

# Basic math functions
print("Square root of 16:", math.sqrt(16))  # 4.0
print("5 raised to power 3:", math.pow(5, 3))  # 125.0

# Trigonometric functions (radians)
print("Sine of 90^\circ:", math.sin(math.pi/2))  # 1.0
print("Cosine of 0^\circ:", math.cos(0))  # 1.0

# Logarithmic functions
print("Log base 10 of 100:", math.log10(100))  # 2.0
print("Natural log of e:", math.log(math.e))  # 1.0
\end{lstlisting}

\end{solutionbox}
\begin{mnemonicbox}
``CALM Operations'' - Constants And Logarithmic
Mathematical Operations

\end{mnemonicbox}
\subsection*{Question 4(b) [4 marks]}\label{q4b}

\textbf{Write a python program that explains scope of variable.}

\begin{solutionbox}

\begin{lstlisting}[language=Python]
# Program to demonstrate variable scope in Python

# Global variable
global_var = "I am global"

def demonstration():
    # Local variable
    local_var = "I am local"
    
    # Accessing global variable
    print("Inside function - Global variable:", global_var)
    
    # Accessing local variable
    print("Inside function - Local variable:", local_var)
    
    # Creating a variable with same name as global
    global_var = "I am local with global name"
    print("Inside function - Shadowed global:", global_var)

# Function call
demonstration()

# Accessing global variable
print("Outside function - Global variable:", global_var)

# Trying to access local variable would cause error
# print("Outside function - Local variable:", local_var)  # Error!
\end{lstlisting}

\textbf{Output:}

\begin{lstlisting}
Inside function - Global variable: I am global
Inside function - Local variable: I am local
Inside function - Shadowed global: I am local with global name
Outside function - Global variable: I am global
\end{lstlisting}

\end{solutionbox}
\begin{mnemonicbox}
``GLOVES'' - Global Local Variable Encapsulation
System

\end{mnemonicbox}
\subsection*{Question 4(c) [7 marks]}\label{q4c}

\textbf{Explain List Methods and its built-in Functions}

\begin{solutionbox}

{\def\LTcaptype{none} % do not increment counter
\begin{longtable}[]{@{}
  >{\raggedright\arraybackslash}p{(\linewidth - 6\tabcolsep) * \real{0.3617}}
  >{\raggedright\arraybackslash}p{(\linewidth - 6\tabcolsep) * \real{0.2766}}
  >{\raggedright\arraybackslash}p{(\linewidth - 6\tabcolsep) * \real{0.1915}}
  >{\raggedright\arraybackslash}p{(\linewidth - 6\tabcolsep) * \real{0.1702}}@{}}
\toprule\noalign{}
\begin{minipage}[b]{\linewidth}\raggedright
Method/Function
\end{minipage} & \begin{minipage}[b]{\linewidth}\raggedright
Description
\end{minipage} & \begin{minipage}[b]{\linewidth}\raggedright
Example
\end{minipage} & \begin{minipage}[b]{\linewidth}\raggedright
Output
\end{minipage} \\
\midrule\noalign{}
\endhead
\bottomrule\noalign{}
\endlastfoot
\passthrough{\lstinline!append()!} & Adds an element at the end &
\passthrough{\lstinline!fruits = ['apple']; fruits.append('banana'); print(fruits)!}
& \passthrough{\lstinline!['apple', 'banana']!} \\
\passthrough{\lstinline!insert()!} & Adds element at specified position
&
\passthrough{\lstinline!nums = [1, 3]; nums.insert(1, 2); print(nums)!}
& \passthrough{\lstinline![1, 2, 3]!} \\
\passthrough{\lstinline!remove()!} & Removes specified item &
\passthrough{\lstinline!colors = ['red', 'blue']; colors.remove('red'); print(colors)!}
& \passthrough{\lstinline!['blue']!} \\
\passthrough{\lstinline!pop()!} & Removes item at specified index &
\passthrough{\lstinline!letters = ['a', 'b', 'c']; x = letters.pop(1); print(x, letters)!}
& \passthrough{\lstinline!b ['a', 'c']!} \\
\passthrough{\lstinline!clear()!} & Removes all elements &
\passthrough{\lstinline!items = [1, 2]; items.clear(); print(items)!} &
\passthrough{\lstinline![]!} \\
\passthrough{\lstinline!len()!} & Returns number of elements &
\passthrough{\lstinline!print(len([1, 2, 3]))!} &
\passthrough{\lstinline!3!} \\
\passthrough{\lstinline!sorted()!} & Returns sorted list &
\passthrough{\lstinline!print(sorted([3, 1, 2]))!} &
\passthrough{\lstinline![1, 2, 3]!} \\
\passthrough{\lstinline!max()/min()!} & Returns max/min value &
\passthrough{\lstinline!print(max([5, 10, 3]), min([5, 10, 3]))!} &
\passthrough{\lstinline!10 3!} \\
\end{longtable}
}

\textbf{Code Example:}

\begin{lstlisting}[language=Python]
# Create a list
my_list = [3, 1, 4, 1, 5]
print("Original:", my_list)

# Add elements
my_list.append(9)
print("After append:", my_list)

my_list.insert(2, 7)
print("After insert:", my_list)

# Remove elements
my_list.remove(1)  # Removes first occurrence of 1
print("After remove:", my_list)

popped = my_list.pop()  # Removes & returns last element
print("Popped value:", popped)
print("After pop:", my_list)

# Other operations
print("Length:", len(my_list))
print("Sorted:", sorted(my_list))
print("Sum:", sum(my_list))
print("Count of 1:", my_list.count(1))
\end{lstlisting}

\end{solutionbox}
\begin{mnemonicbox}
``LISP Operations'' - List Insert Sort Pop Operations

\end{mnemonicbox}
\subsection*{Question 4(a) OR [3
marks]}\label{q4a}

\textbf{List out Python standard library mathematical functions.}

\begin{solutionbox}

{\def\LTcaptype{none} % do not increment counter
\begin{longtable}[]{@{}
  >{\raggedright\arraybackslash}p{(\linewidth - 4\tabcolsep) * \real{0.5000}}
  >{\raggedright\arraybackslash}p{(\linewidth - 4\tabcolsep) * \real{0.2955}}
  >{\raggedright\arraybackslash}p{(\linewidth - 4\tabcolsep) * \real{0.2045}}@{}}
\toprule\noalign{}
\begin{minipage}[b]{\linewidth}\raggedright
Mathematical Function
\end{minipage} & \begin{minipage}[b]{\linewidth}\raggedright
Description
\end{minipage} & \begin{minipage}[b]{\linewidth}\raggedright
Example
\end{minipage} \\
\midrule\noalign{}
\endhead
\bottomrule\noalign{}
\endlastfoot
\passthrough{\lstinline!abs()!} & Returns absolute value &
\passthrough{\lstinline!abs(-5)!} \rightarrow \passthrough{\lstinline!5!} \\
\passthrough{\lstinline!round()!} & Rounds to nearest integer &
\passthrough{\lstinline!round(3.7)!} \rightarrow \passthrough{\lstinline!4!} \\
\passthrough{\lstinline!max()!} & Returns largest item &
\passthrough{\lstinline!max(1, 5, 3)!} \rightarrow \passthrough{\lstinline!5!} \\
\passthrough{\lstinline!min()!} & Returns smallest item &
\passthrough{\lstinline!min(1, 5, 3)!} \rightarrow \passthrough{\lstinline!1!} \\
\passthrough{\lstinline!sum()!} & Adds items of iterable &
\passthrough{\lstinline!sum([1, 2, 3])!} \rightarrow
\passthrough{\lstinline!6!} \\
\passthrough{\lstinline!pow()!} & Returns x to power y &
\passthrough{\lstinline!pow(2, 3)!} \rightarrow \passthrough{\lstinline!8!} \\
\passthrough{\lstinline!divmod()!} & Returns quotient and remainder &
\passthrough{\lstinline!divmod(7, 2)!} \rightarrow
\passthrough{\lstinline!(3, 1)!} \\
\end{longtable}
}

\textbf{Additional from math module:}

\begin{itemize}
\tightlist
\item
  \passthrough{\lstinline!math.sqrt()!}: Square root
\item
  \passthrough{\lstinline!math.floor()!}: Rounds down
\item
  \passthrough{\lstinline!math.ceil()!}: Rounds up
\item
  \passthrough{\lstinline!math.factorial()!}: Factorial of a number
\item
  \passthrough{\lstinline!math.gcd()!}: Greatest common divisor
\end{itemize}

\end{solutionbox}
\begin{mnemonicbox}
``SMART Calculations'' - Standard Mathematical
Arithmetic Routines and Tools

\end{mnemonicbox}
\subsection*{Question 4(b) OR [4
marks]}\label{q4b}

\textbf{Explain built in function in python.}

\begin{solutionbox}

{\def\LTcaptype{none} % do not increment counter
\begin{longtable}[]{@{}
  >{\raggedright\arraybackslash}p{(\linewidth - 0\tabcolsep) * \real{1.0000}}@{}}
\toprule\noalign{}
\begin{minipage}[b]{\linewidth}\raggedright
Built-in Functions in Python
\end{minipage} \\
\midrule\noalign{}
\endhead
\bottomrule\noalign{}
\endlastfoot
Pre-defined functions available in Python without importing any
module \\
Called directly without any prefix \\
Designed to perform common operations \\
Examples include print(), len(), type(), input(), range() \\
\end{longtable}
}

\textbf{Categories with Examples:}

\begin{lstlisting}[language=Python]
# Type conversion functions
print(int("10"))       # 10
print(float("10.5"))   # 10.5
print(str(10))         # "10"
print(list("abc"))     # ['a', 'b', 'c']

# Math functions
print(abs(-7))         # 7
print(round(3.7))      # 4
print(max(5, 10, 3))   # 10

# Collection processing
print(len("hello"))    # 5
print(sorted([3,1,2])) # [1, 2, 3]
print(sum([1, 2, 3]))  # 6
\end{lstlisting}

\end{solutionbox}
\begin{mnemonicbox}
``EPIC Functions'' - Embedded Python Integrated Core
Functions

\end{mnemonicbox}
\subsection*{Question 4(c) OR [7
marks]}\label{q4c}

\textbf{Write a Python Program to count and display the number of
vowels, consonants, uppercase, lowercase characters in a string.}

\begin{solutionbox}

\begin{lstlisting}[language=Python]
# Program to count vowels, consonants, uppercase and lowercase characters

def analyze_string(text):
    # Initialize counters
    vowels = 0
    consonants = 0
    uppercase = 0
    lowercase = 0
    
    # Define vowels
    vowel_set = {'a', 'e', 'i', 'o', 'u'}
    
    # Analyze each character
    for char in text:
        # Check if alphabetic
        if char.isalpha():
            # Check case
            if char.isupper():
                uppercase += 1
            else:
                lowercase += 1
                
            # Check if vowel (case-insensitive)
            if char.lower() in vowel_set:
                vowels += 1
            else:
                consonants += 1
    
    # Return results
    return vowels, consonants, uppercase, lowercase

# Get input
text = input("Enter a string: ")

# Get counts
vowels, consonants, uppercase, lowercase = analyze_string(text)

# Display results
print("Number of vowels:", vowels)
print("Number of consonants:", consonants)
print("Number of uppercase characters:", uppercase)
print("Number of lowercase characters:", lowercase)
\end{lstlisting}

\textbf{Example:}

\begin{itemize}
\tightlist
\item
  Input: ``Hello World!''
\item
  Output:

  \begin{itemize}
  \tightlist
  \item
    Vowels: 3 (e, o, o)
  \item
    Consonants: 7 (H, l, l, W, r, l, d)
  \item
    Uppercase: 2 (H, W)
  \item
    Lowercase: 8 (e, l, l, o, o, r, l, d)
  \end{itemize}
\end{itemize}

\end{solutionbox}
\begin{mnemonicbox}
``VOCAL Analysis'' - Vowels Or Consonants And Letter
case

\end{mnemonicbox}
\subsection*{Question 5(a) [3 marks]}\label{q5a}

\textbf{Write a python code to swap given two elements in a list.}

\begin{solutionbox}

\begin{lstlisting}[language=Python]
# Program to swap two elements in a list

def swap_elements(lst, pos1, pos2):
    """Function to swap two elements in a list"""
    lst[pos1], lst[pos2] = lst[pos2], lst[pos1]
    return lst

# Example usage
my_list = [10, 20, 30, 40, 50]
print("Original list:", my_list)

# Swap elements at positions 1 and 3
result = swap_elements(my_list, 1, 3)
print("After swapping elements at positions 1 and 3:", result)

# Output:
# Original list: [10, 20, 30, 40, 50]
# After swapping elements at positions 1 and 3: [10, 40, 30, 20, 50]
\end{lstlisting}

\end{solutionbox}
\begin{mnemonicbox}
``STEP Logic'' - Swap Two Elements with Python Logic

\end{mnemonicbox}
\subsection*{Question 5(b) [4 marks]}\label{q5b}

\textbf{Write a python Program to check if a substring is present in a
given string.}

\begin{solutionbox}

\begin{lstlisting}[language=Python]
# Program to check if a substring is present in a string

def check_substring(main_string, sub_string):
    """Function to check if a substring exists in a string"""
    if sub_string in main_string:
        return True
    else:
        return False

# Get input from user
main_string = input("Enter the main string: ")
sub_string = input("Enter the substring to find: ")

# Check and display result
if check_substring(main_string, sub_string):
    print(f"'{sub_string}' is present in '{main_string}'")
else:
    print(f"'{sub_string}' is not present in '{main_string}'")
\end{lstlisting}

\textbf{Alternate method using find():}

\begin{lstlisting}[language=Python]
def check_substring_find(main_string, sub_string):
    """Using find method to check substring"""
    position = main_string.find(sub_string)
    return position != -1  # Returns True if substring found
\end{lstlisting}

\end{solutionbox}
\begin{mnemonicbox}
``FIND Method'' - Find IN Directly with Methods

\end{mnemonicbox}
\subsection*{Question 5(c) [7 marks]}\label{q5c}

\textbf{Explain tuple Operations, Functions and Methods}

\begin{solutionbox}

{\def\LTcaptype{none} % do not increment counter
\begin{longtable}[]{@{}
  >{\raggedright\arraybackslash}p{(\linewidth - 6\tabcolsep) * \real{0.4643}}
  >{\raggedright\arraybackslash}p{(\linewidth - 6\tabcolsep) * \real{0.2321}}
  >{\raggedright\arraybackslash}p{(\linewidth - 6\tabcolsep) * \real{0.1607}}
  >{\raggedright\arraybackslash}p{(\linewidth - 6\tabcolsep) * \real{0.1429}}@{}}
\toprule\noalign{}
\begin{minipage}[b]{\linewidth}\raggedright
Operation/Function/Method
\end{minipage} & \begin{minipage}[b]{\linewidth}\raggedright
Description
\end{minipage} & \begin{minipage}[b]{\linewidth}\raggedright
Example
\end{minipage} & \begin{minipage}[b]{\linewidth}\raggedright
Output
\end{minipage} \\
\midrule\noalign{}
\endhead
\bottomrule\noalign{}
\endlastfoot
\textbf{Creation} & Create tuples with parentheses &
\passthrough{\lstinline!t = (1, 2, 3)!} &
\passthrough{\lstinline!(1, 2, 3)!} \\
\textbf{Indexing} & Access tuple elements &
\passthrough{\lstinline!t[1]!} & \passthrough{\lstinline!2!} \\
\textbf{Slicing} & Get subset of tuple &
\passthrough{\lstinline!t[1:3]!} & \passthrough{\lstinline!(2, 3)!} \\
\textbf{Concatenation} & Join two tuples &
\passthrough{\lstinline!(1, 2) + (3, 4)!} &
\passthrough{\lstinline!(1, 2, 3, 4)!} \\
\textbf{Repetition} & Repeat tuple elements &
\passthrough{\lstinline!(1, 2) * 2!} &
\passthrough{\lstinline!(1, 2, 1, 2)!} \\
\textbf{Membership} & Check if element exists &
\passthrough{\lstinline!3 in (1, 2, 3)!} &
\passthrough{\lstinline!True!} \\
\textbf{len()} & Get number of items &
\passthrough{\lstinline!len((1, 2, 3))!} &
\passthrough{\lstinline!3!} \\
\textbf{min()/max()} & Find min/max value &
\passthrough{\lstinline!min((3, 1, 2))!} &
\passthrough{\lstinline!1!} \\
\textbf{count()} & Count occurrences of value &
\passthrough{\lstinline!(1, 2, 1).count(1)!} &
\passthrough{\lstinline!2!} \\
\textbf{index()} & Find position of value &
\passthrough{\lstinline!(1, 2, 3).index(2)!} &
\passthrough{\lstinline!1!} \\
\textbf{sorted()} & Return sorted list from tuple &
\passthrough{\lstinline!sorted((3, 1, 2))!} &
\passthrough{\lstinline![1, 2, 3]!} \\
\end{longtable}
}

\textbf{Code Example:}

\begin{lstlisting}[language=Python]
# Create a tuple
my_tuple = (3, 1, 4, 1, 5, 9)
print("Original tuple:", my_tuple)

# Accessing elements
print("First element:", my_tuple[0])
print("Last element:", my_tuple[-1])
print("Slice (1:4):", my_tuple[1:4])

# Operations
tuple2 = (2, 7)
combined = my_tuple + tuple2
print("Concatenated:", combined)

repeated = tuple2 * 3
print("Repeated:", repeated)

# Functions and methods
print("Length:", len(my_tuple))
print("Count of 1:", my_tuple.count(1))
print("Index of 4:", my_tuple.index(4))
print("Min value:", min(my_tuple))
print("Max value:", max(my_tuple))
print("Sorted:", sorted(my_tuple))  # Returns a list

# Unpacking
a, b, c, *rest = my_tuple
print("Unpacked:", a, b, c, rest)
\end{lstlisting}

\end{solutionbox}
\begin{mnemonicbox}
``ICONS'' - Immutable Collection Operations,
Numbering, and Searching

\end{mnemonicbox}
\subsection*{Question 5(a) OR [3
marks]}\label{q5a}

\textbf{Write a python program find the sum of elements in a list.}

\begin{solutionbox}

\begin{lstlisting}[language=Python]
# Program to find sum of elements in a list

def sum_of_list(numbers):
    """Function to find sum of all elements in a list"""
    total = 0
    for num in numbers:
        total += num
    return total

# Example with user input
num_elements = int(input("Enter the number of elements: "))
my_list = []

# Get elements from user
for i in range(num_elements):
    element = float(input(f"Enter element {i+1}: "))
    my_list.append(element)

# Calculate sum using function
result1 = sum_of_list(my_list)
print("Sum using custom function:", result1)

# Calculate sum using built-in sum() function
result2 = sum(my_list)
print("Sum using built-in function:", result2)
\end{lstlisting}

\end{solutionbox}
\begin{mnemonicbox}
``SALT'' - Sum All List Together

\end{mnemonicbox}
\subsection*{Question 5(b) OR [4
marks]}\label{q5b}

\textbf{Write a Program to demonstrate the set functions and
operations.}

\begin{solutionbox}

\begin{lstlisting}[language=Python]
# Program to demonstrate set functions and operations

# Creating sets
set1 = {1, 2, 3, 4, 5}
set2 = {4, 5, 6, 7, 8}

print("Set 1:", set1)
print("Set 2:", set2)

# Set operations
print("\nSet Operations:")
print("Union:", set1 | set2)  # Alternative: set1.union(set2)
print("Intersection:", set1 & set2)  # Alternative: set1.intersection(set2)
print("Difference (set1-set2):", set1 - set2)  # Alternative: set1.difference(set2)
print("Symmetric Difference:", set1 ^ set2)  # Alternative: set1.symmetric_difference(set2)

# Set methods
print("\nSet Methods:")
set3 = set1.copy()
print("Copy of set1:", set3)

set3.add(6)
print("After adding 6:", set3)

set3.remove(1)
print("After removing 1:", set3)

set3.discard(10)  # No error if element doesn't exist
print("After discarding 10:", set3)

popped = set3.pop()
print("Popped element:", popped)
print("After pop:", set3)

set3.clear()
print("After clear:", set3)

# Check subset/superset
print("\nSubset/Superset:")
subset = {4, 5}
print(f"Is {subset} subset of {set1}?", subset.issubset(set1))
print(f"Is {set1} superset of {subset}?", set1.issuperset(subset))
\end{lstlisting}

\end{solutionbox}
\begin{mnemonicbox}
``COSI Methods'' - Create, Operate, Search,
Investigate with Set Methods

\end{mnemonicbox}
\subsection*{Question 5(c) OR [7
marks]}\label{q5c}

\textbf{Write a Program to demonstrate the dictionaries functions and
operations.}

\begin{solutionbox}

\begin{lstlisting}[language=Python]
# Program to demonstrate dictionary functions and operations

# Creating a dictionary
student = {
    'name': 'John',
    'roll_no': 101,
    'marks': 85,
    'subjects': ['Python', 'Math', 'English']
}

print("Original Dictionary:", student)

# Accessing elements
print("\nAccessing Elements:")
print("Name:", student['name'])
print("Marks:", student['marks'])

# Using get() - safer access method
print("Roll Number (using get):", student.get('roll_no'))
print("Address (using get):", student.get('address', 'Not available'))  # Default value if key not found

# Modifying values
print("\nModifying Dictionary:")
student['marks'] = 90
print("After updating marks:", student)

# Adding new key-value pairs
student['address'] = 'New York'
print("After adding address:", student)

# Removing items
print("\nRemoving Items:")
removed_value = student.pop('address')
print("Removed value:", removed_value)
print("After pop():", student)

# Removing last inserted item
last_item = student.popitem()
print("Last removed item:", last_item)
print("After popitem():", student)

# Dictionary methods
print("\nDictionary Methods:")
print("Keys:", list(student.keys()))
print("Values:", list(student.values()))
print("Items:", list(student.items()))

# Creating a copy
student_copy = student.copy()
print("\nCopy of dictionary:", student_copy)

# Clearing the dictionary
student.clear()
print("After clear():", student)

# Creating dictionary with dict() constructor
new_dict = dict(name='Alice', age=20, city='Boston')
print("\nCreated with dict() constructor:", new_dict)

# Dictionary comprehension example
squares = {x: x**2 for x in range(1, 6)}
print("\nDictionary comprehension result:", squares)
\end{lstlisting}

\textbf{Key Operations:}

\begin{itemize}
\tightlist
\item
  \textbf{Access}: Using key or get() method
\item
  \textbf{Modify}: Assign new value to existing key
\item
  \textbf{Add}: Assign value to new key
\item
  \textbf{Remove}: Using pop(), popitem(), or del statement
\item
  \textbf{Iterate}: Through keys, values, or items
\end{itemize}

\end{solutionbox}
\begin{mnemonicbox}
``ACME Dictionary'' - Access, Create, Modify, Extract
from Dictionary

\end{mnemonicbox}

\end{document}
