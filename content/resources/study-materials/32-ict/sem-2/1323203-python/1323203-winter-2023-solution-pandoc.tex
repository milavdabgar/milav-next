\documentclass[10pt,a4paper]{article}

% content/resources/templates/preamble.tex
\usepackage[margin=0.6in]{geometry}
\author{Milav Dabgar}
\usepackage{amsmath,amssymb,amsthm}
\usepackage{booktabs}
\usepackage{multirow}
\usepackage{xcolor}
\usepackage{tcolorbox}
\tcbuselibrary{breakable,skins}
\usepackage[colorlinks=true,linkcolor=blue]{hyperref}
\usepackage{titlesec}
\usepackage{enumitem}
\usepackage{tikz}
\usepackage{pgfplots}
\usepackage{circuitikz}
\usepackage[version=4]{mhchem}
\usepackage{longtable}
\usepackage{array}
\usepackage{float}
\usepackage{caption}
\usepackage{listings}

\lstset{
  basicstyle=\small\ttfamily,
  breaklines=true,
  breakatwhitespace=false,
  postbreak=\mbox{\textcolor{red}{$\hookrightarrow$}\space},
  float=false,
  numbers=left,
  numberstyle=\tiny\color{gray},
  numbersep=10pt,
  xleftmargin=2em,
  keywordstyle=\color{blue},
  commentstyle=\color{green!60!black},
  stringstyle=\color{purple},
  backgroundcolor=\color{gray!5},
  showstringspaces=false,
  tabsize=2,
  captionpos=b,
  keepspaces=true,
  columns=flexible
}

\pgfplotsset{compat=1.18}
\usetikzlibrary{shapes,arrows,positioning,calc,patterns,decorations.pathmorphing,decorations.markings,arrows.meta}

% Color scheme
\definecolor{headcolor}{RGB}{0,102,204}
\definecolor{keycolor}{RGB}{220,20,60}
\definecolor{solutioncolor}{RGB}{34,139,34}
\definecolor{mnemoniccolor}{RGB}{148,0,211}
\definecolor{codecolor}{RGB}{0,0,100}

% Spacing
\setlength{\parskip}{3pt}
\setlist[itemize]{nosep}
\setlist[enumerate]{nosep}

% Title formatting
\titleformat{\section}{\Large\bfseries\color{headcolor}}{\thesection}{1em}{}
\titleformat{\subsection}{\large\bfseries\color{headcolor}}{\thesubsection}{1em}{}

% Pandoc tightlist compatibility
\providecommand{\tightlist}{%
  \setlength{\itemsep}{0pt}\setlength{\parskip}{0pt}}

% Pandoc longtable compatibility
\newcounter{none}
\def\thenone{}


% content/resources/templates/english-boxes.tex
% This file is currently empty - it exists to maintain consistency with the import structure.
% Add custom environments here if needed in the future.


\begin{document}

\begin{center}
{\Huge\bfseries\color{headcolor} Subject Name Solutions}\\[5pt]
{\LARGE 1323203 -- Winter 2023}\\[3pt]
{\large Semester 1 Study Material}\\[3pt]
{\normalsize\textit{Detailed Solutions and Explanations}}
\end{center}

\vspace{10pt}

\subsection*{Question 1(a) [3 marks]}\label{q1a}

\textbf{Write a pseudocode to check the given number is positive or
negative.}

\begin{solutionbox}

\begin{lstlisting}
BEGIN
    Input number
    IF number > 0 THEN
        Display "Number is positive"
    ELSE IF number < 0 THEN
        Display "Number is negative"
    ELSE
        Display "Number is zero"
    END IF
END
\end{lstlisting}

\end{solutionbox}
\begin{mnemonicbox}
``Compare Zero''

\end{mnemonicbox}
\subsection*{Question 1(b) [4 marks]}\label{q1b}

\textbf{Define Algorithm and Design it for Finding maximum from given
three Numbers.}

\begin{solutionbox}

\textbf{Algorithm Definition}: An algorithm is a step-by-step procedure
or set of rules designed to solve a specific problem or perform a
computation.

\textbf{Algorithm for Finding Maximum of Three Numbers}:

\begin{lstlisting}
BEGIN
    Input num1, num2, num3
    Set max = num1
    IF num2 > max THEN
        Set max = num2
    END IF
    IF num3 > max THEN
        Set max = num3
    END IF
    Display max
END
\end{lstlisting}

\textbf{Diagram}:

\includegraphics[width=1\linewidth,height=\textheight,keepaspectratio]{mermaid-207ea0be.pdf}

\end{solutionbox}
\begin{mnemonicbox}
``Compare and Replace''

\end{mnemonicbox}
\subsection*{Question 1(c) [7 marks]}\label{q1c}

\textbf{Develop a Python code to convert Temperature parameter from
Celsius to Fahrenheit.}

\begin{solutionbox}

\begin{lstlisting}[language=Python]
# Program to convert Celsius to Fahrenheit

# Get the Celsius temperature from user
celsius = float(input("Enter temperature in Celsius: "))

# Convert to Fahrenheit using the formula: F = (C * 9/5) + 32
fahrenheit = (celsius * 9/5) + 32

# Display the result
print(f"{celsius}^\circC is equal to {fahrenheit}^\circF")
\end{lstlisting}


{\def\LTcaptype{none} % do not increment counter
\vspace{-5pt}
\captionof{table}{Temperature Conversion}
\vspace{-10pt}
\begin{longtable}[]{@{}ll@{}}
\toprule\noalign{}
Component & Description \\
\midrule\noalign{}
\endhead
\bottomrule\noalign{}
\endlastfoot
\textbf{Input} & Temperature in Celsius \\
\textbf{Formula} & F = (C \times 9/5) + 32 \\
\textbf{Output} & Temperature in Fahrenheit \\
\end{longtable}
}

\end{solutionbox}
\begin{mnemonicbox}
``Multiply by 9, divide by 5, add 32''

\end{mnemonicbox}
\subsection*{Question 1(c OR) [7
marks]}\label{question-1c-or-7-marks}

\textbf{List out all comparison operators and explain each by giving
python code example.}

\begin{solutionbox}


{\def\LTcaptype{none} % do not increment counter
\vspace{-5pt}
\captionof{table}{Python Comparison Operators}
\vspace{-10pt}
\begin{longtable}[]{@{}llll@{}}
\toprule\noalign{}
Operator & Description & Example & Result \\
\midrule\noalign{}
\endhead
\bottomrule\noalign{}
\endlastfoot
\textbf{==} & Equal to & \passthrough{\lstinline!5 == 5!} &
\passthrough{\lstinline!True!} \\
\textbf{!=} & Not equal to & \passthrough{\lstinline"5 != 6"} &
\passthrough{\lstinline!True!} \\
\textbf{\textgreater{}} & Greater than & \passthrough{\lstinline!6 > 3!}
& \passthrough{\lstinline!True!} \\
\textbf{\textless{}} & Less than & \passthrough{\lstinline!3 < 6!} &
\passthrough{\lstinline!True!} \\
\textbf{\textgreater=} & Greater than or equal to &
\passthrough{\lstinline!5 >= 5!} & \passthrough{\lstinline!True!} \\
\textbf{\textless=} & Less than or equal to &
\passthrough{\lstinline!5 <= 5!} & \passthrough{\lstinline!True!} \\
\end{longtable}
}

\textbf{Code Example}:

\begin{lstlisting}[language=Python]
# Python comparison operators example
a = 10
b = 5

# Equal to
print(f"{a} == {b}: {a == b}")  # False

# Not equal to
print(f"{a} != {b}: {a != b}")  # True

# Greater than
print(f"{a} > {b}: {a > b}")    # True

# Less than
print(f"{a} < {b}: {a < b}")    # False

# Greater than or equal to
print(f"{a} >= {b}: {a >= b}")  # True

# Less than or equal to
print(f"{a} <= {b}: {a <= b}")  # False
\end{lstlisting}

\end{solutionbox}
\begin{mnemonicbox}
``CLEAN'' (Compare, Less than, Equal to, Above, Not
equal)

\end{mnemonicbox}
\subsection*{Question 2(a) [3 marks]}\label{q2a}

\textbf{Describe data types in python with its examples.}

\begin{solutionbox}


{\def\LTcaptype{none} % do not increment counter
\vspace{-5pt}
\captionof{table}{Python Data Types}
\vspace{-10pt}
\begin{longtable}[]{@{}lll@{}}
\toprule\noalign{}
Data Type & Description & Example \\
\midrule\noalign{}
\endhead
\bottomrule\noalign{}
\endlastfoot
\textbf{int} & Integer values & \passthrough{\lstinline!x = 10!} \\
\textbf{float} & Decimal point values &
\passthrough{\lstinline!y = 10.5!} \\
\textbf{str} & Text or character values &
\passthrough{\lstinline!name = "Python"!} \\
\textbf{bool} & Logical values (True/False) &
\passthrough{\lstinline!is\_valid = True!} \\
\textbf{list} & Ordered, mutable collection &
\passthrough{\lstinline!nums = [1, 2, 3]!} \\
\textbf{tuple} & Ordered, immutable collection &
\passthrough{\lstinline!point = (5, 10)!} \\
\textbf{dict} & Key-value pairs &
\passthrough{\lstinline!student = \{"name": "John"\}!} \\
\end{longtable}
}

\end{solutionbox}
\begin{mnemonicbox}
``NIFTY SLD'' (Numbers, Integers, Floats, Text,
Yes/No, Sequences, Lists, Dictionaries)

\end{mnemonicbox}
\subsection*{Question 2(b) [4 marks]}\label{q2b}

\textbf{Explain Nested if in python with python code example.}

\begin{solutionbox}

\textbf{Nested if}: A conditional statement inside another conditional
statement is called a nested if. It allows checking for multiple
conditions in sequence.

\begin{lstlisting}[language=Python]
# Nested if example to check if a number is positive, negative, or zero
# And if positive, check if it's even or odd

num = int(input("Enter a number: "))

if num > 0:
    print("Positive number")
    # Nested if to check if the positive number is even or odd
    if num % 2 == 0:
        print("Even number")
    else:
        print("Odd number")
elif num < 0:
    print("Negative number")
else:
    print("Zero")
\end{lstlisting}

\textbf{Diagram}:

\includegraphics[width=1\linewidth,height=\textheight,keepaspectratio]{mermaid-3c057486.pdf}

\end{solutionbox}
\begin{mnemonicbox}
``Check Inside Check''

\end{mnemonicbox}
\subsection*{Question 2(c) [7 marks]}\label{q2c}

\textbf{Write use of different types of selection / decision making flow
of control structures with example.}

\begin{solutionbox}


{\def\LTcaptype{none} % do not increment counter
\vspace{-5pt}
\captionof{table}{Selection Control Structures in Python}
\vspace{-10pt}
\begin{longtable}[]{@{}
  >{\raggedright\arraybackslash}p{(\linewidth - 4\tabcolsep) * \real{0.3667}}
  >{\raggedright\arraybackslash}p{(\linewidth - 4\tabcolsep) * \real{0.3000}}
  >{\raggedright\arraybackslash}p{(\linewidth - 4\tabcolsep) * \real{0.3333}}@{}}
\toprule\noalign{}
\begin{minipage}[b]{\linewidth}\raggedright
Structure
\end{minipage} & \begin{minipage}[b]{\linewidth}\raggedright
Purpose
\end{minipage} & \begin{minipage}[b]{\linewidth}\raggedright
Use Case
\end{minipage} \\
\midrule\noalign{}
\endhead
\bottomrule\noalign{}
\endlastfoot
\textbf{if} & Execute code when condition is true & Simple condition
check \\
\textbf{if-else} & Execute one code for true condition, another for
false & Binary decision making \\
\textbf{if-elif-else} & Multiple condition checking & Multiple possible
outcomes \\
\textbf{Nested if} & Condition checking inside another condition &
Complex hierarchical decisions \\
\textbf{Ternary operator} & One-line if-else & Simple conditional
assignment \\
\end{longtable}
}

\textbf{Code Example}:

\begin{lstlisting}[language=Python]
# Example of different selection structures
score = int(input("Enter your score: "))

# Simple if
if score >= 90:
    print("Excellent!")

# if-else
if score >= 60:
    print("You passed.")
else:
    print("You failed.")

# if-elif-else
if score >= 90:
    grade = "A"
elif score >= 80:
    grade = "B"
elif score >= 70:
    grade = "C"
elif score >= 60:
    grade = "D"
else:
    grade = "F"
print(f"Your grade is {grade}")

# Ternary operator
result = "Pass" if score >= 60 else "Fail"
print(result)
\end{lstlisting}

\end{solutionbox}
\begin{mnemonicbox}
``SCENE'' (Simple if, Conditions with else, Elif for
multiple, Nested for complex, Express with ternary)

\end{mnemonicbox}
\subsection*{Question 2(a) [3 marks] - OR
Option}\label{q2a}

\textbf{List out rules for defining variables in python.}

\begin{solutionbox}


{\def\LTcaptype{none} % do not increment counter
\vspace{-5pt}
\captionof{table}{Rules for Defining Variables in Python}
\vspace{-10pt}
\begin{longtable}[]{@{}
  >{\raggedright\arraybackslash}p{(\linewidth - 4\tabcolsep) * \real{0.2143}}
  >{\raggedright\arraybackslash}p{(\linewidth - 4\tabcolsep) * \real{0.4643}}
  >{\raggedright\arraybackslash}p{(\linewidth - 4\tabcolsep) * \real{0.3214}}@{}}
\toprule\noalign{}
\begin{minipage}[b]{\linewidth}\raggedright
Rule
\end{minipage} & \begin{minipage}[b]{\linewidth}\raggedright
Description
\end{minipage} & \begin{minipage}[b]{\linewidth}\raggedright
Example
\end{minipage} \\
\midrule\noalign{}
\endhead
\bottomrule\noalign{}
\endlastfoot
\textbf{Start with letter or underscore} & First character must be a
letter or underscore & \passthrough{\lstinline!name = "John"!},
\passthrough{\lstinline!\_count = 10!} \\
\textbf{No special characters} & Only letters, numbers, and underscores
allowed & \passthrough{\lstinline!user\_name!} (valid),
\passthrough{\lstinline!user-name!} (invalid) \\
\textbf{Case sensitive} & Uppercase and lowercase are different &
\passthrough{\lstinline!age!} and \passthrough{\lstinline!Age!} are
different variables \\
\textbf{No reserved keywords} & Cannot use Python keywords as variable
names & Cannot use \passthrough{\lstinline!if!},
\passthrough{\lstinline!for!}, \passthrough{\lstinline!while!}, etc. \\
\textbf{No spaces} & Use underscores instead of spaces &
\passthrough{\lstinline!first\_name!} instead of
\passthrough{\lstinline!first name!} \\
\end{longtable}
}

\end{solutionbox}
\begin{mnemonicbox}
``SILKS'' (Start properly, Ignore special chars, Look
at case, Keywords avoided, Spaces not allowed)

\end{mnemonicbox}
\subsection*{Question 2(b) [4 marks] - OR
Option}\label{q2b}

\textbf{Explain For loop in python with necessary python code example.}

\begin{solutionbox}

\textbf{For Loop in Python}: A for loop is used to iterate over a
sequence (list, tuple, string) or other iterable objects. It executes a
block of code for each item in the sequence.

\begin{lstlisting}[language=Python]
# Example of for loop in Python
# Printing each element in a list
fruits = ["apple", "banana", "cherry"]
for fruit in fruits:
    print(fruit)

# Using range function with for loop
print("Numbers from 1 to 5:")
for i in range(1, 6):
    print(i)

# Using for loop with string
name = "Python"
for char in name:
    print(char)
\end{lstlisting}

\textbf{Diagram}:

\includegraphics[width=1\linewidth,height=\textheight,keepaspectratio]{mermaid-0f0f277c.pdf}

\end{solutionbox}
\begin{mnemonicbox}
``ITEM'' (Iterate Through Each Member)

\end{mnemonicbox}
\subsection*{Question 2(c) [7 marks] - OR
Option}\label{q2c}

\textbf{Describe Break and continue statement in python in brief.}

\begin{solutionbox}


{\def\LTcaptype{none} % do not increment counter
\vspace{-5pt}
\captionof{table}{Break and Continue Statements}
\vspace{-10pt}
\begin{longtable}[]{@{}
  >{\raggedright\arraybackslash}p{(\linewidth - 4\tabcolsep) * \real{0.3929}}
  >{\raggedright\arraybackslash}p{(\linewidth - 4\tabcolsep) * \real{0.3214}}
  >{\raggedright\arraybackslash}p{(\linewidth - 4\tabcolsep) * \real{0.2857}}@{}}
\toprule\noalign{}
\begin{minipage}[b]{\linewidth}\raggedright
Statement
\end{minipage} & \begin{minipage}[b]{\linewidth}\raggedright
Purpose
\end{minipage} & \begin{minipage}[b]{\linewidth}\raggedright
Effect
\end{minipage} \\
\midrule\noalign{}
\endhead
\bottomrule\noalign{}
\endlastfoot
\textbf{break} & Exit the loop immediately & Terminates the current loop
and transfers control to the statement following the loop \\
\textbf{continue} & Skip the current iteration & Jumps to the next
iteration of the loop, skipping any code after the continue statement \\
\end{longtable}
}

\textbf{Code Example}:

\begin{lstlisting}[language=Python]
# Break statement example
print("Break example:")
for i in range(1, 11):
if

i == 6:

        print("Breaking the loop at i =", i)
        break
    print(i, end=" ")
print("\nLoop ended")

# Continue statement example
print("\nContinue example:")
for i in range(1, 11):
    if i % 2 == 0:
        continue
    print(i, end=" ")
print("\nOnly odd numbers were printed")
\end{lstlisting}

\textbf{Diagram}:

\includegraphics[width=1\linewidth,height=\textheight,keepaspectratio]{mermaid-1ab522f5.pdf}

\end{solutionbox}
\begin{mnemonicbox}
``EXIT SKIP'' (EXIT with break, SKIP with continue)

\end{mnemonicbox}
\subsection*{Question 3(a) [3 marks]}\label{q3a}

\textbf{Develop a python program to print 1 to 10 numbers using loops.}

\begin{solutionbox}

\begin{lstlisting}[language=Python]
# Using for loop to print numbers from 1 to 10
print("Using for loop:")
for i in range(1, 11):
    print(i, end=" ")

print("\n\nUsing while loop:")
# Using while loop to print numbers from 1 to 10
counter = 1
while counter <= 10:
    print(counter, end=" ")
    counter += 1
\end{lstlisting}


{\def\LTcaptype{none} % do not increment counter
\vspace{-5pt}
\captionof{table}{Loop Approaches}
\vspace{-10pt}
\begin{longtable}[]{@{}
  >{\raggedright\arraybackslash}p{(\linewidth - 2\tabcolsep) * \real{0.4762}}
  >{\raggedright\arraybackslash}p{(\linewidth - 2\tabcolsep) * \real{0.5238}}@{}}
\toprule\noalign{}
\begin{minipage}[b]{\linewidth}\raggedright
Approach
\end{minipage} & \begin{minipage}[b]{\linewidth}\raggedright
Advantage
\end{minipage} \\
\midrule\noalign{}
\endhead
\bottomrule\noalign{}
\endlastfoot
\textbf{For loop with range} & Simple, concise, automatically manages
counter \\
\textbf{While loop} & More flexible for complex conditions \\
\end{longtable}
}

\end{solutionbox}
\begin{mnemonicbox}
``COUNT UP'' (Counter Updates in each iteration)

\end{mnemonicbox}
\subsection*{Question 3(b) [4 marks]}\label{q3b}

\textbf{Develop a python program to print following pattern using loop.}

\begin{lstlisting}
*
**
***
****
*****
\end{lstlisting}

\begin{solutionbox}

\begin{lstlisting}[language=Python]
# Print star pattern using for loop
rows = 5

for i in range(1, rows + 1):
    # Print i stars in each row
    print("*" * i)
\end{lstlisting}

\textbf{Alternative solution with nested loops}:

\begin{lstlisting}[language=Python]
# Print star pattern using nested loops
rows = 5

for i in range(1, rows + 1):
    for j in range(1, i + 1):
        print("*", end="")
    print()  # New line after each row
\end{lstlisting}

\textbf{Diagram}:

\includegraphics[width=1\linewidth,height=\textheight,keepaspectratio]{mermaid-f47a0519.pdf}

\end{solutionbox}
\begin{mnemonicbox}
``RISE UP'' (Row Increases, Stars Expand Upward
Progressively)

\end{mnemonicbox}
\subsection*{Question 3(c) [7 marks]}\label{q3c}

\textbf{Create a user define function to find factorial of the given
number.}

\begin{solutionbox}

\begin{lstlisting}[language=Python]
# Function to find factorial of a given number
def factorial(n):
    # Check if input is valid
    if not isinstance(n, int) or n < 0:
        return "Invalid input. Please enter a non-negative integer."
    
    # Base case: factorial of 0 or 1 is 1
if

n == 0 or

n == 1:

        return 1
    
    # Calculate factorial using iteration
    result = 1
    for i in range(2, n + 1):
        result *= i
    
    return result

# Test the function
number = int(input("Enter a number to find its factorial: "))
print(f"Factorial of {number} is {factorial(number)}")
\end{lstlisting}

\textbf{Diagram}:

\includegraphics[width=1\linewidth,height=\textheight,keepaspectratio]{mermaid-0ad4b060.pdf}


{\def\LTcaptype{none} % do not increment counter
\vspace{-5pt}
\captionof{table}{Factorial Examples}
\vspace{-10pt}
\begin{longtable}[]{@{}lll@{}}
\toprule\noalign{}
Number & Calculation & Factorial \\
\midrule\noalign{}
\endhead
\bottomrule\noalign{}
\endlastfoot
0 & 0! = 1 & 1 \\
1 & 1! = 1 & 1 \\
3 & 3! = 3 \times 2 \times 1 & 6 \\
5 & 5! = 5 \times 4 \times 3 \times 2 \times 1 & 120 \\
\end{longtable}
}

\end{solutionbox}
\begin{mnemonicbox}
``Multiply Down To One'' (Multiply all integers down
to 1)

\end{mnemonicbox}
\subsection*{Question 3(a) [3 marks] - OR
Option}\label{q3a}

\textbf{Develop a python code to find odd and even numbers from 1 to N
using loops.}

\begin{solutionbox}

\begin{lstlisting}[language=Python]
# Program to find odd and even numbers from 1 to N

# Get input from user
N = int(input("Enter the value of N: "))

print("Even numbers from 1 to", N, "are:")
for i in range(1, N + 1):
    if i % 2 == 0:
        print(i, end=" ")

print("\nOdd numbers from 1 to", N, "are:")
for i in range(1, N + 1):
    if i % 2 != 0:
        print(i, end=" ")
\end{lstlisting}


{\def\LTcaptype{none} % do not increment counter
\vspace{-5pt}
\captionof{table}{Even and Odd Check}
\vspace{-10pt}
\begin{longtable}[]{@{}lll@{}}
\toprule\noalign{}
Number & Check & Type \\
\midrule\noalign{}
\endhead
\bottomrule\noalign{}
\endlastfoot
Even numbers & \passthrough{\lstinline!number \% 2 == 0!} & 2, 4, 6,
\ldots{} \\
Odd numbers & \passthrough{\lstinline"number \% 2 != 0"} & 1, 3, 5,
\ldots{} \\
\end{longtable}
}

\end{solutionbox}
\begin{mnemonicbox}
``MOD-2'' (Modulo 2 determines odd or even)

\end{mnemonicbox}
\subsection*{Question 3(b) [4 marks] - OR
Option}\label{q3b}

\textbf{Develop a code to create nested list and display elements.}

\begin{solutionbox}

\begin{lstlisting}[language=Python]
# Program to create and display nested list

# Create a nested list
nested_list = [
    [1, 2, 3],
    [4, 5, 6],
    [7, 8, 9]
]

# Display the nested list
print("Nested List:", nested_list)

# Display each element using nested loops
print("\nElements of the nested list:")
for i in range(len(nested_list)):
    for j in range(len(nested_list[i])):
        print(f"nested_list[{i}][{j}] = {nested_list[i][j]}")

# Alternative way to display using enumerate
print("\nUsing enumerate:")
for i, inner_list in enumerate(nested_list):
    for j, value in enumerate(inner_list):
        print(f"Position ({i}, {j}): {value}")
\end{lstlisting}

\textbf{Diagram}:

\includegraphics[width=1\linewidth,height=\textheight,keepaspectratio]{mermaid-70fd9b33.pdf}

\end{solutionbox}
\begin{mnemonicbox}
``ROWS COLS'' (Rows and Columns form the structure)

\end{mnemonicbox}
\subsection*{Question 3(c) [7 marks] - OR
Option}\label{q3c}

\textbf{Explain local and global variables using examples.}

\begin{solutionbox}


{\def\LTcaptype{none} % do not increment counter
\vspace{-5pt}
\captionof{table}{Local vs Global Variables}
\vspace{-10pt}
\begin{longtable}[]{@{}
  >{\raggedright\arraybackslash}p{(\linewidth - 6\tabcolsep) * \real{0.1463}}
  >{\raggedright\arraybackslash}p{(\linewidth - 6\tabcolsep) * \real{0.1707}}
  >{\raggedright\arraybackslash}p{(\linewidth - 6\tabcolsep) * \real{0.3659}}
  >{\raggedright\arraybackslash}p{(\linewidth - 6\tabcolsep) * \real{0.3171}}@{}}
\toprule\noalign{}
\begin{minipage}[b]{\linewidth}\raggedright
Type
\end{minipage} & \begin{minipage}[b]{\linewidth}\raggedright
Scope
\end{minipage} & \begin{minipage}[b]{\linewidth}\raggedright
Accessibility
\end{minipage} & \begin{minipage}[b]{\linewidth}\raggedright
Declaration
\end{minipage} \\
\midrule\noalign{}
\endhead
\bottomrule\noalign{}
\endlastfoot
\textbf{Local Variables} & Only within the function where declared &
Only inside declaring function & Inside a function \\
\textbf{Global Variables} & Throughout the program & All functions can
access & Outside any function \\
\end{longtable}
}

\textbf{Code Example}:

\begin{lstlisting}[language=Python]
# Global variable
total = 0

def add_numbers(a, b):
    # Local variables
    sum_result = a + b
    print(f"Local variable sum_result: {sum_result}")
    
    # Accessing global variable
    print(f"Global variable total before modification: {total}")
    
    # To modify global variable within function
    global total
    total = sum_result
    print(f"Global variable total after modification: {total}")
    
    return sum_result

# Main program
x = 5  # Local to main program
y = 10  # Local to main program

result = add_numbers(x, y)
print(f"Result: {result}")
print(f"Updated global total: {total}")

# This would cause an error because sum_result is local to add_numbers
# print(sum_result)  # NameError: name 'sum_result' is not defined
\end{lstlisting}

\textbf{Diagram}:

\includegraphics[width=1\linewidth,height=\textheight,keepaspectratio]{mermaid-a854a1e6.pdf}

\end{solutionbox}
\begin{mnemonicbox}
``GLOBAL SEES ALL'' (Global variables are visible
everywhere)

\end{mnemonicbox}
\subsection*{Question 4(a) [3 marks]}\label{q4a}

\textbf{List out Python standard library mathematical functions.}

\begin{solutionbox}


{\def\LTcaptype{none} % do not increment counter
\vspace{-5pt}
\captionof{table}{Python Math Module Functions}
\vspace{-10pt}
\begin{longtable}[]{@{}lll@{}}
\toprule\noalign{}
Function & Description & Example \\
\midrule\noalign{}
\endhead
\bottomrule\noalign{}
\endlastfoot
\textbf{abs()} & Returns absolute value &
\passthrough{\lstinline!abs(-5)!} \rightarrow \passthrough{\lstinline!5!} \\
\textbf{pow()} & Returns x to power y &
\passthrough{\lstinline!pow(2, 3)!} \rightarrow \passthrough{\lstinline!8!} \\
\textbf{max()} & Returns largest value &
\passthrough{\lstinline!max(5, 10, 15)!} \rightarrow
\passthrough{\lstinline!15!} \\
\textbf{min()} & Returns smallest value &
\passthrough{\lstinline!min(5, 10, 15)!} \rightarrow
\passthrough{\lstinline!5!} \\
\textbf{round()} & Rounds to nearest integer &
\passthrough{\lstinline!round(4.6)!} \rightarrow \passthrough{\lstinline!5!} \\
\textbf{math.sqrt()} & Square root &
\passthrough{\lstinline!math.sqrt(16)!} \rightarrow
\passthrough{\lstinline!4.0!} \\
\textbf{math.sin()} & Sine function &
\passthrough{\lstinline!math.sin(math.pi/2)!} \rightarrow
\passthrough{\lstinline!1.0!} \\
\end{longtable}
}

\end{solutionbox}
\begin{mnemonicbox}
``PEARS Math'' (Power, Exponents, Arithmetic, Roots,
Sine functions in Math)

\end{mnemonicbox}
\subsection*{Question 4(b) [4 marks]}\label{q4b}

\textbf{Explain Module in python with example python code of it.}

\begin{solutionbox}

\textbf{Module}: A module in Python is a file containing Python
definitions and statements. The file name is the module name with the
suffix .py added.

\begin{lstlisting}[language=Python]
# Example of using math module
import math

# Using mathematical functions from math module
radius = 5
area = math.pi * math.pow(radius, 2)
print(f"Area of circle with radius {radius} is {area:.2f}")

# Using different import techniques
from math import sqrt, sin
angle = math.pi / 4
print(f"Square root of 25 is {sqrt(25)}")
print(f"Sine of {angle} radians is {sin(angle):.4f}")

# Importing with alias
import random as rnd
random_number = rnd.randint(1, 100)
print(f"Random number between 1 and 100: {random_number}")
\end{lstlisting}


{\def\LTcaptype{none} % do not increment counter
\vspace{-5pt}
\captionof{table}{Module Import Techniques}
\vspace{-10pt}
\begin{longtable}[]{@{}
  >{\raggedright\arraybackslash}p{(\linewidth - 4\tabcolsep) * \real{0.3200}}
  >{\raggedright\arraybackslash}p{(\linewidth - 4\tabcolsep) * \real{0.3200}}
  >{\raggedright\arraybackslash}p{(\linewidth - 4\tabcolsep) * \real{0.3600}}@{}}
\toprule\noalign{}
\begin{minipage}[b]{\linewidth}\raggedright
Method
\end{minipage} & \begin{minipage}[b]{\linewidth}\raggedright
Syntax
\end{minipage} & \begin{minipage}[b]{\linewidth}\raggedright
Example
\end{minipage} \\
\midrule\noalign{}
\endhead
\bottomrule\noalign{}
\endlastfoot
\textbf{Import entire module} &
\passthrough{\lstinline!import module\_name!} &
\passthrough{\lstinline!import math!} \\
\textbf{Import specific items} &
\passthrough{\lstinline!from module\_name import item1, item2!} &
\passthrough{\lstinline!from math import sqrt, sin!} \\
\textbf{Import with alias} &
\passthrough{\lstinline!import module\_name as alias!} &
\passthrough{\lstinline!import random as rnd!} \\
\end{longtable}
}

\end{solutionbox}
\begin{mnemonicbox}
``CODE-LIB'' (Code Libraries for reuse)

\end{mnemonicbox}
\subsection*{Question 4(c) [7 marks]}\label{q4c}

\textbf{Write a Program that determines whether a given number is an
`Armstrong number' or a palindrome using a user-defined function.}

\begin{solutionbox}

\begin{lstlisting}[language=Python]
# Function to check if a number is an Armstrong number
def is_armstrong(num):
    # Convert number to string to count digits
    num_str = str(num)
    n = len(num_str)
    
    # Calculate sum of each digit raised to power of number of digits
    armstrong_sum = 0
    for digit in num_str:
        armstrong_sum += int(digit) ** n
    
    # Check if sum equals the original number
    return armstrong_sum == num

# Function to check if a number is a palindrome
def is_palindrome(num):
    # Convert number to string and check if it reads the same forwards and backwards
    num_str = str(num)
    return num_str == num_str[::-1]

# Main program
number = int(input("Enter a number: "))

# Check if the number is an Armstrong number
if is_armstrong(number):
    print(f"{number} is an Armstrong number")
else:
    print(f"{number} is not an Armstrong number")

# Check if the number is a palindrome
if is_palindrome(number):
    print(f"{number} is a palindrome")
else:
    print(f"{number} is not a palindrome")
\end{lstlisting}


{\def\LTcaptype{none} % do not increment counter
\vspace{-5pt}
\captionof{table}{Examples}
\vspace{-10pt}
\begin{longtable}[]{@{}
  >{\raggedright\arraybackslash}p{(\linewidth - 4\tabcolsep) * \real{0.1951}}
  >{\raggedright\arraybackslash}p{(\linewidth - 4\tabcolsep) * \real{0.3902}}
  >{\raggedright\arraybackslash}p{(\linewidth - 4\tabcolsep) * \real{0.4146}}@{}}
\toprule\noalign{}
\begin{minipage}[b]{\linewidth}\raggedright
Number
\end{minipage} & \begin{minipage}[b]{\linewidth}\raggedright
Armstrong Check
\end{minipage} & \begin{minipage}[b]{\linewidth}\raggedright
Palindrome Check
\end{minipage} \\
\midrule\noalign{}
\endhead
\bottomrule\noalign{}
\endlastfoot
153 & 1^{3} + 5^{3} + 3^{3} = 1 + 125 + 27 = 153 ✓ & 153 \neq 351 ✗ \\
121 & 1^{3} + 2^{3} + 1^{3} = 1 + 8 + 1 = 10 \neq 121 ✗ & 121 = 121 ✓ \\
1634 & 1^{4} + 6^{4} + 3^{4} + 4^{4} = 1 + 1296 + 81 + 256 = 1634 ✓ & 1634 \neq 4361
✗ \\
\end{longtable}
}

\textbf{Diagram}:

\includegraphics[width=1\linewidth,height=\textheight,keepaspectratio]{mermaid-b778269a.pdf}

\end{solutionbox}
\begin{mnemonicbox}
``SAME SUM'' (SAME forwards and backwards for
palindrome, SUM of powered digits for Armstrong)

\end{mnemonicbox}
\subsection*{Question 4(a) [3 marks] - OR
Option}\label{q4a}

\textbf{Explain built in functions in python.}

\begin{solutionbox}

\textbf{Built-in Functions}: These are functions that are part of
Python's standard library and available without importing any module.


{\def\LTcaptype{none} % do not increment counter
\vspace{-5pt}
\captionof{table}{Common Python Built-in Functions}
\vspace{-10pt}
\begin{longtable}[]{@{}
  >{\raggedright\arraybackslash}p{(\linewidth - 4\tabcolsep) * \real{0.3571}}
  >{\raggedright\arraybackslash}p{(\linewidth - 4\tabcolsep) * \real{0.3214}}
  >{\raggedright\arraybackslash}p{(\linewidth - 4\tabcolsep) * \real{0.3214}}@{}}
\toprule\noalign{}
\begin{minipage}[b]{\linewidth}\raggedright
Function
\end{minipage} & \begin{minipage}[b]{\linewidth}\raggedright
Purpose
\end{minipage} & \begin{minipage}[b]{\linewidth}\raggedright
Example
\end{minipage} \\
\midrule\noalign{}
\endhead
\bottomrule\noalign{}
\endlastfoot
\textbf{print()} & Display output &
\passthrough{\lstinline!print("Hello")!} \\
\textbf{input()} & Get user input &
\passthrough{\lstinline!name = input("Name: ")!} \\
\textbf{len()} & Return object length &
\passthrough{\lstinline!len([1, 2, 3])!} \rightarrow
\passthrough{\lstinline!3!} \\
\textbf{type()} & Return object type & \passthrough{\lstinline!type(5)!}
\rightarrow \passthrough{\lstinline!<class 'int'>!} \\
\textbf{int(), float(), str()} & Convert to specific type &
\passthrough{\lstinline!int("5")!} \rightarrow \passthrough{\lstinline!5!} \\
\textbf{range()} & Generate sequence &
\passthrough{\lstinline!list(range(3))!} \rightarrow
\passthrough{\lstinline![0, 1, 2]!} \\
\textbf{sum()} & Calculate sum &
\passthrough{\lstinline!sum([1, 2, 3])!} \rightarrow
\passthrough{\lstinline!6!} \\
\end{longtable}
}

\end{solutionbox}
\begin{mnemonicbox}
``PITS LCR'' (Print, Input, Type, Sum, Len, Convert,
Range)

\end{mnemonicbox}
\subsection*{Question 4(b) [4 marks] - OR
Option}\label{q4b}

\textbf{Describe python math module by giving one python code example.}

\begin{solutionbox}

\textbf{Python Math Module}: The math module provides access to
mathematical functions defined by the C standard.

\begin{lstlisting}[language=Python]
# Example using math module
import math

# Basic constants
print(f"Value of pi: {math.pi}")
print(f"Value of e: {math.e}")

# Trigonometric functions (argument in radians)
angle = math.pi / 3  # 60 degrees
print(f"Sine of {angle:.2f} radians: {math.sin(angle):.4f}")
print(f"Cosine of {angle:.2f} radians: {math.cos(angle):.4f}")
print(f"Tangent of {angle:.2f} radians: {math.tan(angle):.4f}")

# Logarithmic and exponential functions
x = 10
print(f"Natural logarithm of {x}: {math.log(x):.4f}")
print(f"Logarithm base 10 of {x}: {math.log10(x):.4f}")
print(f"e raised to power {x}: {math.exp(x):.4f}")

# Other functions
print(f"Square root of 25: {math.sqrt(25)}")
print(f"Ceiling of 4.3: {math.ceil(4.3)}")
print(f"Floor of 4.7: {math.floor(4.7)}")
\end{lstlisting}


{\def\LTcaptype{none} % do not increment counter
\vspace{-5pt}
\captionof{table}{Math Module Categories}
\vspace{-10pt}
\begin{longtable}[]{@{}ll@{}}
\toprule\noalign{}
Category & Functions \\
\midrule\noalign{}
\endhead
\bottomrule\noalign{}
\endlastfoot
\textbf{Constants} & \passthrough{\lstinline!math.pi!},
\passthrough{\lstinline!math.e!} \\
\textbf{Trigonometric} & \passthrough{\lstinline!sin()!},
\passthrough{\lstinline!cos()!}, \passthrough{\lstinline!tan()!} \\
\textbf{Logarithmic} & \passthrough{\lstinline!log()!},
\passthrough{\lstinline!log10()!}, \passthrough{\lstinline!exp()!} \\
\textbf{Numeric} & \passthrough{\lstinline!sqrt()!},
\passthrough{\lstinline!ceil()!}, \passthrough{\lstinline!floor()!} \\
\end{longtable}
}

\end{solutionbox}
\begin{mnemonicbox}
``PENT'' (Pi/constants, Exponents, Numbers,
Trigonometry)

\end{mnemonicbox}
\subsection*{Question 4(c) [7 marks] - OR
Option}\label{q4c}

\textbf{Explain concept of scope of variable in Python and Apply global
and local variable concepts in python program.}

\begin{solutionbox}

\textbf{Scope of Variables in Python}: The scope of a variable
determines where in the program a variable is accessible or visible.


{\def\LTcaptype{none} % do not increment counter
\vspace{-5pt}
\captionof{table}{Variable Scope Types}
\vspace{-10pt}
\begin{longtable}[]{@{}
  >{\raggedright\arraybackslash}p{(\linewidth - 4\tabcolsep) * \real{0.2500}}
  >{\raggedright\arraybackslash}p{(\linewidth - 4\tabcolsep) * \real{0.4643}}
  >{\raggedright\arraybackslash}p{(\linewidth - 4\tabcolsep) * \real{0.2857}}@{}}
\toprule\noalign{}
\begin{minipage}[b]{\linewidth}\raggedright
Scope
\end{minipage} & \begin{minipage}[b]{\linewidth}\raggedright
Description
\end{minipage} & \begin{minipage}[b]{\linewidth}\raggedright
Access
\end{minipage} \\
\midrule\noalign{}
\endhead
\bottomrule\noalign{}
\endlastfoot
\textbf{Local} & Variables defined inside a function & Only within the
function \\
\textbf{Global} & Variables defined at the top level & Throughout the
program \\
\textbf{Enclosing} & Variables in outer function of nested functions &
In the outer and inner function \\
\textbf{Built-in} & Pre-defined variables in Python & Throughout the
program \\
\end{longtable}
}

\textbf{Code Example}:

\begin{lstlisting}[language=Python]
# Variable scope demonstration

# Global variable
count = 0

def outer_function():
    # Enclosing scope variable
    name = "Python"
    
    def inner_function():
        # Local variable
        age = 30
        # Accessing global variable
        global count
        count += 1
        # Accessing enclosing variable
        print(f"Inside inner_function: name is {name}")
        print(f"Inside inner_function: age is {age}")
        print(f"Inside inner_function: count is {count}")
    
    # Local variable to outer_function
    language = "Programming"
    print(f"Inside outer_function: name is {name}")
    print(f"Inside outer_function: language is {language}")
    print(f"Inside outer_function: count is {count}")
    
    # Call inner function
    inner_function()
    
    # This would cause an error - age is local to inner_function
    # print(age)

# Main program
print(f"Global scope: count is {count}")
outer_function()
print(f"Global scope after function call: count is {count}")

# These would cause errors - they are local to functions
# print(name)
# print(language)
\end{lstlisting}

\textbf{Diagram}:

\includegraphics[width=1\linewidth,height=\textheight,keepaspectratio]{mermaid-fdba69e3.pdf}

\end{solutionbox}
\begin{mnemonicbox}
``LEGB'' (Local, Enclosing, Global, Built-in - order
of scope lookup)

\end{mnemonicbox}
\subsection*{Question 5(a) [3 marks]}\label{q5a}

\textbf{Develop a python program to swap two elements in given list}

\begin{solutionbox}

\begin{lstlisting}[language=Python]
# Program to swap two elements in a list

# Create a list
my_list = [10, 20, 30, 40, 50]
print("Original list:", my_list)

# Get positions to swap
pos1 = int(input("Enter first position (index starts from 0): "))
pos2 = int(input("Enter second position (index starts from 0): "))

# Swap elements using a temporary variable
if 0 <= pos1 < len(my_list) and 0 <= pos2 < len(my_list):
    # Swapping
    temp = my_list[pos1]
    my_list[pos1] = my_list[pos2]
    my_list[pos2] = temp
    
    print(f"List after swapping elements at positions {pos1} and {pos2}:", my_list)
else:
    print("Invalid positions! Positions should be within list range.")
\end{lstlisting}

\textbf{Alternative method}:

\begin{lstlisting}[language=Python]
# Swap using Python's tuple unpacking (more pythonic)
if 0 <= pos1 < len(my_list) and 0 <= pos2 < len(my_list):
    my_list[pos1], my_list[pos2] = my_list[pos2], my_list[pos1]
    print(f"List after swapping elements at positions {pos1} and {pos2}:", my_list)
\end{lstlisting}


{\def\LTcaptype{none} % do not increment counter
\vspace{-5pt}
\captionof{table}{Swapping Methods}
\vspace{-10pt}
\begin{longtable}[]{@{}ll@{}}
\toprule\noalign{}
Method & Code \\
\midrule\noalign{}
\endhead
\bottomrule\noalign{}
\endlastfoot
\textbf{Using temp variable} &
\passthrough{\lstinline!temp = a; a = b; b = temp!} \\
\textbf{Python tuple unpacking} &
\passthrough{\lstinline!a, b = b, a!} \\
\end{longtable}
}

\end{solutionbox}
\begin{mnemonicbox}
``TEMP SWAP'' (Temporary variable helps safe
swapping)

\end{mnemonicbox}
\subsection*{Question 5(b) [4 marks]}\label{q5b}

\textbf{Explain nested list by giving example.}

\begin{solutionbox}

\textbf{Nested List}: A nested list is a list that contains other lists
as its elements, creating a multi-dimensional data structure.

\begin{lstlisting}[language=Python]
# Creating a nested list (3x3 matrix)
matrix = [
    [1, 2, 3],
    [4, 5, 6],
    [7, 8, 9]
]

# Accessing elements
print("Complete matrix:", matrix)
print("First row:", matrix[0])
print("Element at row 1, column 2:", matrix[0][1])  # Output: 2

# Modifying elements
matrix[1][1] = 50
print("Matrix after modification:", matrix)

# Iterating through a nested list
print("\nPrinting the matrix:")
for row in matrix:
    for element in row:
        print(element, end=" ")
    print()  # New line after each row
\end{lstlisting}

\textbf{Diagram}:

\includegraphics[width=1\linewidth,height=\textheight,keepaspectratio]{mermaid-8af68ef5.pdf}


{\def\LTcaptype{none} % do not increment counter
\vspace{-5pt}
\captionof{table}{Nested List Operations}
\vspace{-10pt}
\begin{longtable}[]{@{}
  >{\raggedright\arraybackslash}p{(\linewidth - 4\tabcolsep) * \real{0.3929}}
  >{\raggedright\arraybackslash}p{(\linewidth - 4\tabcolsep) * \real{0.2857}}
  >{\raggedright\arraybackslash}p{(\linewidth - 4\tabcolsep) * \real{0.3214}}@{}}
\toprule\noalign{}
\begin{minipage}[b]{\linewidth}\raggedright
Operation
\end{minipage} & \begin{minipage}[b]{\linewidth}\raggedright
Syntax
\end{minipage} & \begin{minipage}[b]{\linewidth}\raggedright
Example
\end{minipage} \\
\midrule\noalign{}
\endhead
\bottomrule\noalign{}
\endlastfoot
\textbf{Access element} & \passthrough{\lstinline!list[row][col]!} &
\passthrough{\lstinline!matrix[0][1]!} \\
\textbf{Modify element} &
\passthrough{\lstinline!list[row][col] = new\_value!} &
\passthrough{\lstinline!matrix[1][1] = 50!} \\
\textbf{Add new row} & \passthrough{\lstinline!list.append([...])!} &
\passthrough{\lstinline!matrix.append([10, 11, 12])!} \\
\end{longtable}
}

\end{solutionbox}
\begin{mnemonicbox}
``MARS'' (Matrix Access with Row and column
Structure)

\end{mnemonicbox}
\subsection*{Question 5(c) [7 marks]}\label{q5c}

\textbf{Explain string operations with examples.}

\begin{solutionbox}


{\def\LTcaptype{none} % do not increment counter
\vspace{-5pt}
\captionof{table}{String Operations in Python}
\vspace{-10pt}
\begin{longtable}[]{@{}
  >{\raggedright\arraybackslash}p{(\linewidth - 4\tabcolsep) * \real{0.3333}}
  >{\raggedright\arraybackslash}p{(\linewidth - 4\tabcolsep) * \real{0.3939}}
  >{\raggedright\arraybackslash}p{(\linewidth - 4\tabcolsep) * \real{0.2727}}@{}}
\toprule\noalign{}
\begin{minipage}[b]{\linewidth}\raggedright
Operation
\end{minipage} & \begin{minipage}[b]{\linewidth}\raggedright
Description
\end{minipage} & \begin{minipage}[b]{\linewidth}\raggedright
Example
\end{minipage} \\
\midrule\noalign{}
\endhead
\bottomrule\noalign{}
\endlastfoot
\textbf{Concatenation} & Joining strings &
\passthrough{\lstinline!"Hello" + " World"!} \rightarrow
\passthrough{\lstinline!"Hello World"!} \\
\textbf{Repetition} & Repeating strings &
\passthrough{\lstinline!"Python" * 3!} \rightarrow
\passthrough{\lstinline!"PythonPythonPython"!} \\
\textbf{Slicing} & Extract substring &
\passthrough{\lstinline!"Python"[1:4]!} \rightarrow
\passthrough{\lstinline!"yth"!} \\
\textbf{Indexing} & Access character &
\passthrough{\lstinline!"Python"[0]!} \rightarrow \passthrough{\lstinline!"P"!} \\
\textbf{Length} & Count characters &
\passthrough{\lstinline!len("Python")!} \rightarrow \passthrough{\lstinline!6!} \\
\textbf{Membership} & Check if present &
\passthrough{\lstinline!"P" in "Python"!} \rightarrow
\passthrough{\lstinline!True!} \\
\textbf{Comparison} & Compare strings &
\passthrough{\lstinline!"apple" < "banana"!} \rightarrow
\passthrough{\lstinline!True!} \\
\end{longtable}
}

\textbf{Code Example}:

\begin{lstlisting}[language=Python]
# String operations demonstration
text = "Python Programming"

# Indexing
print("First character:", text[0])
print("Last character:", text[-1])

# Slicing
print("First word:", text[:6])
print("Second word:", text[7:])
print("Middle characters:", text[3:10])
print("Reverse:", text[::-1])

# String methods
print("Uppercase:", text.upper())
print("Lowercase:", text.lower())
print("Replace 'P' with 'J':", text.replace("P", "J"))
print("Split by space:", text.split())
print("Count 'm':", text.count('m'))
print("Find 'gram':", text.find("gram"))

# Check operations
print("Is alphanumeric?", text.isalnum())
print("Starts with 'Py'?", text.startswith("Py"))
print("Ends with 'ing'?", text.endswith("ing"))
\end{lstlisting}

\textbf{Diagram}:

\includegraphics[width=1\linewidth,height=\textheight,keepaspectratio]{mermaid-4ead0af2.pdf}

\end{solutionbox}
\begin{mnemonicbox}
``SCREAM'' (Slice, Concat, Replace, Extract, Access,
Methods)

\end{mnemonicbox}
\subsection*{Question 5(a) [3 marks] - OR
Option}\label{q5a}

\textbf{Develop a python program to find sum of all elements in given
list}

\begin{solutionbox}

\begin{lstlisting}[language=Python]
# Program to find sum of all elements in a list

# Method 1: Using built-in sum() function
def sum_list_builtin(numbers):
    return sum(numbers)

# Method 2: Using a loop
def sum_list_loop(numbers):
    total = 0
    for num in numbers:
        total += num
    return total

# Create a sample list
my_list = [10, 20, 30, 40, 50]
print("List:", my_list)

# Calculate sum using built-in function
print("Sum using built-in function:", sum_list_builtin(my_list))

# Calculate sum using loop
print("Sum using loop:", sum_list_loop(my_list))
\end{lstlisting}


{\def\LTcaptype{none} % do not increment counter
\vspace{-5pt}
\captionof{table}{Sum Methods Comparison}
\vspace{-10pt}
\begin{longtable}[]{@{}ll@{}}
\toprule\noalign{}
Method & Advantage \\
\midrule\noalign{}
\endhead
\bottomrule\noalign{}
\endlastfoot
\textbf{Built-in sum()} & Simple, efficient, fast \\
\textbf{Loop approach} & Works for custom summing logic \\
\end{longtable}
}

\end{solutionbox}
\begin{mnemonicbox}
``ADD ALL'' (Add All elements in sequence)

\end{mnemonicbox}
\subsection*{Question 5(b) [4 marks] - OR
Option}\label{q5b}

\textbf{Explain indexing and slicing operations in python list}

\begin{solutionbox}


{\def\LTcaptype{none} % do not increment counter
\vspace{-5pt}
\captionof{table}{Indexing and Slicing Operations}
\vspace{-10pt}
\begin{longtable}[]{@{}
  >{\raggedright\arraybackslash}p{(\linewidth - 6\tabcolsep) * \real{0.2683}}
  >{\raggedright\arraybackslash}p{(\linewidth - 6\tabcolsep) * \real{0.1951}}
  >{\raggedright\arraybackslash}p{(\linewidth - 6\tabcolsep) * \real{0.3171}}
  >{\raggedright\arraybackslash}p{(\linewidth - 6\tabcolsep) * \real{0.2195}}@{}}
\toprule\noalign{}
\begin{minipage}[b]{\linewidth}\raggedright
Operation
\end{minipage} & \begin{minipage}[b]{\linewidth}\raggedright
Syntax
\end{minipage} & \begin{minipage}[b]{\linewidth}\raggedright
Description
\end{minipage} & \begin{minipage}[b]{\linewidth}\raggedright
Example
\end{minipage} \\
\midrule\noalign{}
\endhead
\bottomrule\noalign{}
\endlastfoot
\textbf{Positive Indexing} & \passthrough{\lstinline!list[i]!} & Access
item at position i (0-based) & \passthrough{\lstinline!fruits[0]!} \rightarrow
first item \\
\textbf{Negative Indexing} & \passthrough{\lstinline!list[-i]!} & Access
item from end (-1 is last) & \passthrough{\lstinline!fruits[-1]!} \rightarrow last
item \\
\textbf{Basic Slicing} & \passthrough{\lstinline!list[start:end]!} &
Items from start to end-1 & \passthrough{\lstinline!fruits[1:3]!} \rightarrow
items at 1,2 \\
\textbf{Slice with Step} &
\passthrough{\lstinline!list[start:end:step]!} & Items with interval of
step & \passthrough{\lstinline!nums[1:6:2]!} \rightarrow items at 1,3,5 \\
\textbf{Omitting Indices} & \passthrough{\lstinline!list[:end]!},
\passthrough{\lstinline!list[start:]!} & From beginning or to end &
\passthrough{\lstinline!fruits[:3]!} \rightarrow first 3 items \\
\textbf{Negative Slicing} & \passthrough{\lstinline!list[-start:-end]!}
& Slice from end & \passthrough{\lstinline!fruits[-3:-1]!} \rightarrow 3rd and 2nd
last \\
\textbf{Reverse} & \passthrough{\lstinline!list[::-1]!} & Reverse the
list & \passthrough{\lstinline!fruits[::-1]!} \rightarrow list in reverse \\
\end{longtable}
}

\textbf{Code Example}:

\begin{lstlisting}[language=Python]
# Indexing and slicing demonstration
fruits = ["apple", "banana", "cherry", "date", "elderberry", "fig"]
print("Original list:", fruits)

# Indexing
print("\nIndexing examples:")
print("First item:", fruits[0])  # apple
print("Last item:", fruits[-1])  # fig
print("Third item:", fruits[2])  # cherry

# Slicing
print("\nSlicing examples:")
print("First three items:", fruits[:3])  # ['apple', 'banana', 'cherry']
print("Last three items:", fruits[-3:])  # ['date', 'elderberry', 'fig']
print("Middle items:", fruits[2:4])  # ['cherry', 'date']
print("Every second item:", fruits[::2])  # ['apple', 'cherry', 'elderberry']
print("Reversed list:", fruits[::-1])  # ['fig', 'elderberry', 'date', 'cherry', 'banana', 'apple']
\end{lstlisting}

\textbf{Diagram}:

\includegraphics[width=1\linewidth,height=\textheight,keepaspectratio]{mermaid-cf3ad4e7.pdf}

\end{solutionbox}
\begin{mnemonicbox}
``START-END-STEP'' (Slicing syntax:
[start:end:step])

\end{mnemonicbox}
\subsection*{Question 5(c) [7 marks] - OR
Option}\label{q5c}

\textbf{Explain tuple in brief with necessary example.}

\begin{solutionbox}

\textbf{Tuple}: A tuple is an ordered, immutable collection of elements.
Once created, the elements cannot be changed.


{\def\LTcaptype{none} % do not increment counter
\vspace{-5pt}
\captionof{table}{Tuple vs List}
\vspace{-10pt}
\begin{longtable}[]{@{}
  >{\raggedright\arraybackslash}p{(\linewidth - 4\tabcolsep) * \real{0.4091}}
  >{\raggedright\arraybackslash}p{(\linewidth - 4\tabcolsep) * \real{0.3182}}
  >{\raggedright\arraybackslash}p{(\linewidth - 4\tabcolsep) * \real{0.2727}}@{}}
\toprule\noalign{}
\begin{minipage}[b]{\linewidth}\raggedright
Feature
\end{minipage} & \begin{minipage}[b]{\linewidth}\raggedright
Tuple
\end{minipage} & \begin{minipage}[b]{\linewidth}\raggedright
List
\end{minipage} \\
\midrule\noalign{}
\endhead
\bottomrule\noalign{}
\endlastfoot
\textbf{Syntax} & \passthrough{\lstinline!(item1, item2)!} &
\passthrough{\lstinline![item1, item2]!} \\
\textbf{Mutability} & Immutable (cannot change) & Mutable (can
change) \\
\textbf{Performance} & Faster & Slower \\
\textbf{Use Case} & Fixed data, dictionary keys & Data that needs
modification \\
\textbf{Methods} & Few methods & Many methods \\
\end{longtable}
}

\textbf{Code Example}:

\begin{lstlisting}[language=Python]
# Creating tuples
empty_tuple = ()
single_item_tuple = (1,)  # Comma is necessary for single item
mixed_tuple = (1, "Hello", 3.14, True)
nested_tuple = (1, 2, (3, 4), 5)

# Accessing tuple elements
print("First item:", mixed_tuple[0])  # 1
print("Last item:", mixed_tuple[-1])  # True
print("Nested tuple element:", nested_tuple[2][0])  # 3

# Slicing tuple
print("First two items:", mixed_tuple[:2])  # (1, "Hello")

# Tuple unpacking
a, b, c, d = mixed_tuple
print("Unpacked values:", a, b, c, d)

# Tuple methods
print("Count of 1:", mixed_tuple.count(1))  # 1
print("Index of 'Hello':", mixed_tuple.index("Hello"))  # 1

# Tuple operations
combined_tuple = mixed_tuple + nested_tuple
repeated_tuple = mixed_tuple * 2
print("Combined tuple:", combined_tuple)
print("Repeated tuple:", repeated_tuple)

# This will cause error as tuples are immutable
# mixed_tuple[0] = 100  # TypeError: 'tuple' object does not support item assignment
\end{lstlisting}

\textbf{Diagram}:

\includegraphics[width=1\linewidth,height=\textheight,keepaspectratio]{mermaid-c8477666.pdf}

\end{solutionbox}
\begin{mnemonicbox}
``IPAC'' (Immutable, Parentheses, Access only, Cannot
modify)

\end{mnemonicbox}

\end{document}
