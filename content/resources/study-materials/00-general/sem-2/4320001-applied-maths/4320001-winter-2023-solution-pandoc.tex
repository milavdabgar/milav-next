\documentclass[10pt,a4paper]{article}

% content/resources/templates/preamble.tex
\usepackage[margin=0.6in]{geometry}
\author{Milav Dabgar}
\usepackage{amsmath,amssymb,amsthm}
\usepackage{booktabs}
\usepackage{multirow}
\usepackage{xcolor}
\usepackage{tcolorbox}
\tcbuselibrary{breakable,skins}
\usepackage[colorlinks=true,linkcolor=blue]{hyperref}
\usepackage{titlesec}
\usepackage{enumitem}
\usepackage{tikz}
\usepackage{pgfplots}
\usepackage{circuitikz}
\usepackage[version=4]{mhchem}
\usepackage{longtable}
\usepackage{array}
\usepackage{float}
\usepackage{caption}
\usepackage{listings}

\lstset{
  basicstyle=\small\ttfamily,
  breaklines=true,
  breakatwhitespace=false,
  postbreak=\mbox{\textcolor{red}{$\hookrightarrow$}\space},
  float=false,
  numbers=left,
  numberstyle=\tiny\color{gray},
  numbersep=10pt,
  xleftmargin=2em,
  keywordstyle=\color{blue},
  commentstyle=\color{green!60!black},
  stringstyle=\color{purple},
  backgroundcolor=\color{gray!5},
  showstringspaces=false,
  tabsize=2,
  captionpos=b,
  keepspaces=true,
  columns=flexible
}

\pgfplotsset{compat=1.18}
\usetikzlibrary{shapes,arrows,positioning,calc,patterns,decorations.pathmorphing,decorations.markings,arrows.meta}

% Color scheme
\definecolor{headcolor}{RGB}{0,102,204}
\definecolor{keycolor}{RGB}{220,20,60}
\definecolor{solutioncolor}{RGB}{34,139,34}
\definecolor{mnemoniccolor}{RGB}{148,0,211}
\definecolor{codecolor}{RGB}{0,0,100}

% Spacing
\setlength{\parskip}{3pt}
\setlist[itemize]{nosep}
\setlist[enumerate]{nosep}

% Title formatting
\titleformat{\section}{\Large\bfseries\color{headcolor}}{\thesection}{1em}{}
\titleformat{\subsection}{\large\bfseries\color{headcolor}}{\thesubsection}{1em}{}

% Pandoc tightlist compatibility
\providecommand{\tightlist}{%
  \setlength{\itemsep}{0pt}\setlength{\parskip}{0pt}}

% Pandoc longtable compatibility
\newcounter{none}
\def\thenone{}


% content/resources/templates/english-boxes.tex
% This file is currently empty - it exists to maintain consistency with the import structure.
% Add custom environments here if needed in the future.


\begin{document}

\begin{center}
{\Huge\bfseries\color{headcolor} Subject Name Solutions}\\[5pt]
{\LARGE 4320001 -- Winter 2023}\\[3pt]
{\large Semester 1 Study Material}\\[3pt]
{\normalsize\textit{Detailed Solutions and Explanations}}
\end{center}

\vspace{10pt}

\subsection*{Q.1 Fill in the blanks [14
marks]}\label{q.1-fill-in-the-blanks-14-marks}

\subsubsection{Q1.1 [1 mark]}\label{q1.1-1-mark}

\textbf{If A = [1 2; 3 -1] then 4A = \ldots{}}

\begin{solutionbox}
(b) [4 8; 12 -4]

\textbf{Solution}:
\(4A = 4 \begin{bmatrix} 1 & 2 \\ 3 & -1 \end{bmatrix} = \begin{bmatrix} 4 & 8 \\ 12 & -4 \end{bmatrix}\)

\end{solutionbox}
\subsubsection{Q1.2 [1 mark]}\label{q1.2-1-mark}

\textbf{Order of the matrix [1 1 2; -3 2 3] is \ldots{}}

\begin{solutionbox}
(a) 2 \times 3

\textbf{Solution}: Matrix has 2 rows and 3 columns, so order is 2 \times 3.

\end{solutionbox}
\subsubsection{Q1.3 [1 mark]}\label{q1.3-1-mark}

\textbf{If A = [1 1; 1 1] then A^{2} = \ldots{}}

\begin{solutionbox}
(d) [2 2; 2 2]

\textbf{Solution}:
\(A^2 = \begin{bmatrix} 1 & 1 \\ 1 & 1 \end{bmatrix} \begin{bmatrix} 1 & 1 \\ 1 & 1 \end{bmatrix} = \begin{bmatrix} 2 & 2 \\ 2 & 2 \end{bmatrix}\)

\end{solutionbox}
\subsubsection{Q1.4 [1 mark]}\label{q1.4-1-mark}

\textbf{If A = [2 -1; 3 4] then adjoint of A = \ldots{}}

\begin{solutionbox}
(c) [4 1; -3 2]

\textbf{Solution}: For matrix A = [a b; c d], adj(A) = [d -b; -c
a] adj(A) = [4 1; -3 2]

\end{solutionbox}
\subsubsection{Q1.5 [1 mark]}\label{q1.5-1-mark}

\textbf{d/dx(tan x) = \ldots{}}

\begin{solutionbox}
(d) sec^{2}x

\textbf{Solution}: \(\frac{d}{dx}(\tan x) = \sec^2 x\)

\end{solutionbox}
\subsubsection{Q1.6 [1 mark]}\label{q1.6-1-mark}

\textbf{d/dx(sin 5x) = \ldots{}}

\begin{solutionbox}
(b) 5cos5x

\textbf{Solution}: \(\frac{d}{dx}(\sin 5x) = 5\cos 5x\) (using chain
rule)

\end{solutionbox}
\subsubsection{Q1.7 [1 mark]}\label{q1.7-1-mark}

\textbf{If function y = f(x) is maximum at x = a then f'(a) = \ldots{}}

\begin{solutionbox}
(c) 0

\textbf{Solution}: At maximum point, first derivative equals zero: f'(a)
= 0

\end{solutionbox}
\subsubsection{Q1.8 [1 mark]}\label{q1.8-1-mark}

\textbf{\intsin x dx = \ldots{} + C}

\begin{solutionbox}
(a) -cos x

\textbf{Solution}: \(\int \sin x \, dx = -\cos x + C\)

\end{solutionbox}
\subsubsection{Q1.9 [1 mark]}\label{q1.9-1-mark}

\textbf{\int1/(x^{2}+4) dx = \ldots{} + C}

\begin{solutionbox}
(d) (1/2)tan^{-}^{1}(x/2)

\textbf{Solution}:
\(\int \frac{1}{x^2+4} dx = \frac{1}{2}\tan^{-1}\left(\frac{x}{2}\right) + C\)

\end{solutionbox}
\subsubsection{Q1.10 [1 mark]}\label{q1.10-1-mark}

\textbf{\int_{1}^{2} x^{2} dx = \ldots{}}

\begin{solutionbox}
(a) 7/3

\textbf{Solution}:
\(\int_1^2 x^2 dx = \left[\frac{x^3}{3}\right]_1^2 = \frac{8}{3} - \frac{1}{3} = \frac{7}{3}\)

\end{solutionbox}
\subsubsection{Q1.11 [1 mark]}\label{q1.11-1-mark}

\textbf{Order of differential equation (d^{3}y/dx^{3})^{4} + dy/dx + 5y = 0 is
\ldots{}}

\begin{solutionbox}
(c) 3

\textbf{Solution}: Order is the highest derivative present = 3

\end{solutionbox}
\subsubsection{Q1.12 [1 mark]}\label{q1.12-1-mark}

\textbf{Integrating factor of dy/dx + y/x = 1 is \ldots{}}

\begin{solutionbox}
(b) x

\textbf{Solution}: I.F. = \(e^{\int \frac{1}{x} dx} = e^{\ln x} = x\)

\end{solutionbox}
\subsubsection{Q1.13 [1 mark]}\label{q1.13-1-mark}

\textbf{Mean of 39,23,58,47,50,16,61 is \ldots{}}

\begin{solutionbox}
(b) 42

\textbf{Solution}: Mean =
\(\frac{39+23+58+47+50+16+61}{7} = \frac{294}{7} = 42\)

\end{solutionbox}
\subsubsection{Q1.14 [1 mark]}\label{q1.14-1-mark}

\textbf{Mean of first five natural numbers is \ldots{}}

\begin{solutionbox}
(a) 3

\textbf{Solution}: Mean = \(\frac{1+2+3+4+5}{5} = \frac{15}{5} = 3\)

\end{solutionbox}
\subsection*{Q.2 Attempt any two [14 marks
total]}\label{q.2-attempt-any-two-14-marks-total}

\subsubsection{Q2(A).1 [3 marks]}\label{q2a.1-3-marks}

\textbf{If A = [1 3 5; -1 0 2; 4 3 6], B = [3 4 5; 5 4 3; 3 5
4], C = [1 2 1; 3 3 3; 4 5 6], find 3A+2B-4C}

\textbf{Solution}:
\(3A = \begin{bmatrix} 3 & 9 & 15 \\ -3 & 0 & 6 \\ 12 & 9 & 18 \end{bmatrix}\)

\(2B = \begin{bmatrix} 6 & 8 & 10 \\ 10 & 8 & 6 \\ 6 & 10 & 8 \end{bmatrix}\)

\(4C = \begin{bmatrix} 4 & 8 & 4 \\ 12 & 12 & 12 \\ 16 & 20 & 24 \end{bmatrix}\)

\(3A + 2B - 4C = \begin{bmatrix} 5 & 9 & 21 \\ -5 & -4 & 0 \\ 2 & -1 & 2 \end{bmatrix}\)

\subsubsection{Q2(A).2 [3 marks]}\label{q2a.2-3-marks}

\textbf{If A = [7 5; -1 2], B = [1 -1; 3 2], show that (A+B)ᵀ =
Aᵀ + Bᵀ}

\textbf{Solution}:
\(A + B = \begin{bmatrix} 8 & 4 \\ 2 & 4 \end{bmatrix}\)

\((A + B)^T = \begin{bmatrix} 8 & 2 \\ 4 & 4 \end{bmatrix}\)

\(A^T = \begin{bmatrix} 7 & -1 \\ 5 & 2 \end{bmatrix}\),
\(B^T = \begin{bmatrix} 1 & 3 \\ -1 & 2 \end{bmatrix}\)

\(A^T + B^T = \begin{bmatrix} 8 & 2 \\ 4 & 4 \end{bmatrix}\)

Hence proved: \((A + B)^T = A^T + B^T\)

\subsubsection{Q2(A).3 [3 marks]}\label{q2a.3-3-marks}

\textbf{Solve the differential equation xy dy = (x+1)(y+1)dx}

\textbf{Solution}: Separating variables:
\(\frac{y}{y+1} dy = \frac{x+1}{x} dx\)

\(\left(1 - \frac{1}{y+1}\right) dy = \left(1 + \frac{1}{x}\right) dx\)

Integrating: \(y - \ln|y+1| = x + \ln|x| + C\)

\textbf{Final answer}: \(y - x = \ln|y+1| + \ln|x| + C\)

\subsubsection{Q2(B).1 [4 marks]}\label{q2b.1-4-marks}

\textbf{Find the inverse of matrix [3 1 2; 2 -3 -1; 1 2 1]}

\textbf{Solution}: Let
\(A = \begin{bmatrix} 3 & 1 & 2 \\ 2 & -3 & -1 \\ 1 & 2 & 1 \end{bmatrix}\)

\(|A| = 3(-3-(-2)) - 1(2-(-1)) + 2(4-(-3)) = 3(-1) - 1(3) + 2(7) = -3 - 3 + 14 = 8\)

\textbf{Cofactors}:

\begin{itemize}
\tightlist
\item
  C_{1}_{1} = -1, C_{1}_{2} = -3, C_{1}_{3} = 7
\item
  C_{2}_{1} = 3, C_{2}_{2} = 1, C_{2}_{3} = -5
\item
  C_{3}_{1} = 5, C_{3}_{2} = 7, C_{3}_{3} = -11
\end{itemize}

\(adj(A) = \begin{bmatrix} -1 & 3 & 5 \\ -3 & 1 & 7 \\ 7 & -5 & -11 \end{bmatrix}\)

\(A^{-1} = \frac{1}{8} \begin{bmatrix} -1 & 3 & 5 \\ -3 & 1 & 7 \\ 7 & -5 & -11 \end{bmatrix}\)

\subsubsection{Q2(B).2 [4 marks]}\label{q2b.2-4-marks}

\textbf{Solve 3x - 2y = 8, 5x + 4y = 6 using matrix method}

\textbf{Solution}:
\(\begin{bmatrix} 3 & -2 \\ 5 & 4 \end{bmatrix} \begin{bmatrix} x \\ y \end{bmatrix} = \begin{bmatrix} 8 \\ 6 \end{bmatrix}\)

\(|A| = 3(4) - (-2)(5) = 12 + 10 = 22\)

\(A^{-1} = \frac{1}{22} \begin{bmatrix} 4 & 2 \\ -5 & 3 \end{bmatrix}\)

\(\begin{bmatrix} x \\ y \end{bmatrix} = \frac{1}{22} \begin{bmatrix} 4 & 2 \\ -5 & 3 \end{bmatrix} \begin{bmatrix} 8 \\ 6 \end{bmatrix} = \frac{1}{22} \begin{bmatrix} 44 \\ -22 \end{bmatrix}\)

\begin{solutionbox}
x = 2,

y = -1


\end{solutionbox}
\subsubsection{Q2(B).3 [4 marks]}\label{q2b.3-4-marks}

\textbf{If A = [1 2 1; 2 3 1; 1 2 2], find A·adj(A)}

\textbf{Solution}: \(|A| = 1(6-2) - 2(4-1) + 1(4-3) = 4 - 6 + 1 = -1\)

For any matrix A: \(A \cdot adj(A) = |A| \cdot I\)

\(A \cdot adj(A) = (-1) \begin{bmatrix} 1 & 0 & 0 \\ 0 & 1 & 0 \\ 0 & 0 & 1 \end{bmatrix} = \begin{bmatrix} -1 & 0 & 0 \\ 0 & -1 & 0 \\ 0 & 0 & -1 \end{bmatrix}\)

\subsection*{Q.3 Attempt any two [14 marks
total]}\label{q.3-attempt-any-two-14-marks-total}

\subsubsection{Q3(A).1 [3 marks]}\label{q3a.1-3-marks}

\textbf{If y = log(sin x/(1+cos x)), find dy/dx}

\textbf{Solution}: \(y = \log(\sin x) - \log(1+\cos x)\)

\(\frac{dy}{dx} = \frac{1}{\sin x} \cdot \cos x - \frac{1}{1+\cos x} \cdot (-\sin x)\)

\(= \frac{\cos x}{\sin x} + \frac{\sin x}{1+\cos x}\)

\(= \cot x + \frac{\sin x}{1+\cos x}\)

Using identity: \(\frac{\sin x}{1+\cos x} = \tan(\frac{x}{2})\)

\begin{solutionbox}
\(\frac{dy}{dx} = \cot x + \tan(\frac{x}{2})\)

\end{solutionbox}
\subsubsection{Q3(A).2 [3 marks]}\label{q3a.2-3-marks}

\textbf{If y = sin(x+y), find dy/dx}

\textbf{Solution}: Differentiating both sides:
\(\frac{dy}{dx} = \cos(x+y) \cdot \left(1 + \frac{dy}{dx}\right)\)

\(\frac{dy}{dx} = \cos(x+y) + \cos(x+y) \cdot \frac{dy}{dx}\)

\(\frac{dy}{dx} - \cos(x+y) \cdot \frac{dy}{dx} = \cos(x+y)\)

\(\frac{dy}{dx}[1 - \cos(x+y)] = \cos(x+y)\)

\begin{solutionbox}
\(\frac{dy}{dx} = \frac{\cos(x+y)}{1-\cos(x+y)}\)

\end{solutionbox}
\subsubsection{Q3(A).3 [3 marks]}\label{q3a.3-3-marks}

\textbf{Obtain \intx^{2}log x dx}

\textbf{Solution}: Using integration by parts: \intu dv = uv - \intv du

Let

u = log x, dv = x^{2} dx Then du = (1/x) dx,

v = x^{3}/3


\(\int x^2 \log x \, dx = \log x \cdot \frac{x^3}{3} - \int \frac{x^3}{3} \cdot \frac{1}{x} dx\)

\(= \frac{x^3 \log x}{3} - \int \frac{x^2}{3} dx\)

\(= \frac{x^3 \log x}{3} - \frac{x^3}{9} + C\)

\begin{solutionbox}
\(\frac{x^3}{3}(\log x - \frac{1}{3}) + C\)

\end{solutionbox}
\subsubsection{Q3(B).1 [4 marks]}\label{q3b.1-4-marks}

\textbf{Motion equation s = 2t^{3} - 3t^{2} - 12t + 7. Find s and t when
acceleration is zero}

\textbf{Solution}: \(s = 2t^3 - 3t^2 - 12t + 7\)

Velocity: \(v = \frac{ds}{dt} = 6t^2 - 6t - 12\)

Acceleration: \(a = \frac{dv}{dt} = 12t - 6\)

When acceleration = 0: \(12t - 6 = 0\) \(t = \frac{1}{2}\)

At t = 1/2:
\(s = 2(\frac{1}{2})^3 - 3(\frac{1}{2})^2 - 12(\frac{1}{2}) + 7 = \frac{1}{4} - \frac{3}{4} - 6 + 7 = \frac{1}{2}\)

\begin{solutionbox}
t = 1/2,

s = 1/2


\end{solutionbox}
\subsubsection{Q3(B).2 [4 marks]}\label{q3b.2-4-marks}

\textbf{If y = 2e^{3}^{x} + 3e^{-}^{2}^{x}, prove d^{2}y/dx^{2} - dy/dx - 6y = 0}

\textbf{Solution}: \(y = 2e^{3x} + 3e^{-2x}\)

\(\frac{dy}{dx} = 6e^{3x} - 6e^{-2x}\)

\(\frac{d^2y}{dx^2} = 18e^{3x} + 12e^{-2x}\)

Now: \(\frac{d^2y}{dx^2} - \frac{dy}{dx} - 6y\)

\(= (18e^{3x} + 12e^{-2x}) - (6e^{3x} - 6e^{-2x}) - 6(2e^{3x} + 3e^{-2x})\)

\(= 18e^{3x} + 12e^{-2x} - 6e^{3x} + 6e^{-2x} - 12e^{3x} - 18e^{-2x}\)

\(= (18-6-12)e^{3x} + (12+6-18)e^{-2x} = 0\)

\textbf{Hence proved}

\subsubsection{Q3(B).3 [4 marks]}\label{q3b.3-4-marks}

\textbf{Find maximum and minimum values of f(x) = x^{3} - 3x + 11}

\textbf{Solution}: \(f(x) = x^3 - 3x + 11\)

\(f'(x) = 3x^2 - 3 = 3(x^2 - 1) = 3(x-1)(x+1)\)

Critical points:

x = 1,

x = -1


\(f''(x) = 6x\)

At

x = 1: f'\,`(1) = 6 \textgreater{} 0 \rightarrow Local minimum At

x = -1:

f'\,'(-1) = -6 \textless{} 0 \rightarrow Local maximum

\(f(1) = 1 - 3 + 11 = 9\) (minimum) \(f(-1) = -1 + 3 + 11 = 13\)
(maximum)

\begin{solutionbox}
Maximum = 13 at

x = -1, Minimum = 9 at

x = 1


\end{solutionbox}
\subsection*{Q.4 Attempt any two [14 marks
total]}\label{q.4-attempt-any-two-14-marks-total}

\subsubsection{Q4(A).1 [3 marks]}\label{q4a.1-3-marks}

\textbf{Obtain \intsin 5x sin 6x dx}

\textbf{Solution}: Using identity:
\(\sin A \sin B = \frac{1}{2}[\cos(A-B) - \cos(A+B)]\)

\(\sin 5x \sin 6x = \frac{1}{2}[\cos(5x-6x) - \cos(5x+6x)]\)

\(= \frac{1}{2}[\cos(-x) - \cos(11x)] = \frac{1}{2}[\cos x - \cos(11x)]\)

\(\int \sin 5x \sin 6x \, dx = \frac{1}{2} \int [\cos x - \cos(11x)] dx\)

\(= \frac{1}{2}[\sin x - \frac{\sin(11x)}{11}] + C\)

\begin{solutionbox}
\(\frac{1}{2}\sin x - \frac{\sin(11x)}{22} + C\)

\end{solutionbox}
\subsubsection{Q4(A).2 [3 marks]}\label{q4a.2-3-marks}

\textbf{Obtain \int(1+x)e^{x}/cos^{2}(xe^{x}) dx}

\textbf{Solution}: Let \(u = xe^x\), then \(du = (1+x)e^x dx\)

The integral becomes:
\(\int \frac{du}{\cos^2 u} = \int \sec^2 u \, du = \tan u + C\)

Substituting back: \(= \tan(xe^x) + C\)

\begin{solutionbox}
\(\tan(xe^x) + C\)

\end{solutionbox}
\subsubsection{Q4(A).3 [3 marks]}\label{q4a.3-3-marks}

\textbf{Find standard deviation for data: 6,7,10,12,13,4,8,12}

\textbf{Solution}: Data: 6, 7, 10, 12, 13, 4, 8, 12 n = 8

Mean = \(\frac{6+7+10+12+13+4+8+12}{8} = \frac{72}{8} = 9\)

\begin{longtable}[]{@{}lll@{}}
\toprule\noalign{}
x & x-9 & (x-9)^{2} \\
\midrule\noalign{}
\endhead
\bottomrule\noalign{}
\endlastfoot
6 & -3 & 9 \\
7 & -2 & 4 \\
10 & 1 & 1 \\
12 & 3 & 9 \\
13 & 4 & 16 \\
4 & -5 & 25 \\
8 & -1 & 1 \\
12 & 3 & 9 \\
\end{longtable}

Σ(x-9)^{2} = 74

Standard deviation =
\(\sqrt{\frac{\sum(x-\bar{x})^2}{n}} = \sqrt{\frac{74}{8}} = \sqrt{9.25} = 3.04\)

\begin{solutionbox}
σ = 3.04

\end{solutionbox}
\subsubsection{Q4(B).1 [4 marks]}\label{q4b.1-4-marks}

\textbf{Obtain \int(2x+1)/[(x+1)(x-3)] dx}

\textbf{Solution}: Using partial fractions:
\(\frac{2x+1}{(x+1)(x-3)} = \frac{A}{x+1} + \frac{B}{x-3}\)

\(2x+1 = A(x-3) + B(x+1)\)

When x = -1:
\(2(-1)+1 = A(-4) \Rightarrow -1 = -4A \Rightarrow A = \frac{1}{4}\)

When x = 3:
\(2(3)+1 = B(4) \Rightarrow 7 = 4B \Rightarrow B = \frac{7}{4}\)

\(\int \frac{2x+1}{(x+1)(x-3)} dx = \frac{1}{4}\int \frac{1}{x+1} dx + \frac{7}{4}\int \frac{1}{x-3} dx\)

\(= \frac{1}{4}\ln|x+1| + \frac{7}{4}\ln|x-3| + C\)

\begin{solutionbox}
\(\frac{1}{4}\ln|x+1| + \frac{7}{4}\ln|x-3| + C\)

\end{solutionbox}
\subsubsection{Q4(B).2 [4 marks]}\label{q4b.2-4-marks}

\textbf{Obtain \int_{0}\^{}(π/2) \sqrt(cot x)/(\sqrt(cot x) + \sqrt(tan x)) dx}

\textbf{Solution}: Let
\(I = \int_0^{\pi/2} \frac{\sqrt{\cot x}}{\sqrt{\cot x} + \sqrt{\tan x}} dx\)

Using property: \(\int_0^a f(x) dx = \int_0^a f(a-x) dx\)

\(I = \int_0^{\pi/2} \frac{\sqrt{\cot(\pi/2-x)}}{\sqrt{\cot(\pi/2-x)} + \sqrt{\tan(\pi/2-x)}} dx\)

Since \(\cot(\pi/2-x) = \tan x\) and \(\tan(\pi/2-x) = \cot x\):

\(I = \int_0^{\pi/2} \frac{\sqrt{\tan x}}{\sqrt{\tan x} + \sqrt{\cot x}} dx\)

Adding both expressions:
\(2I = \int_0^{\pi/2} \frac{\sqrt{\cot x} + \sqrt{\tan x}}{\sqrt{\cot x} + \sqrt{\tan x}} dx = \int_0^{\pi/2} 1 \, dx = \frac{\pi}{2}\)

\begin{solutionbox}
\(I = \frac{\pi}{4}\)

\end{solutionbox}
\subsubsection{Q4(B).3 [4 marks]}\label{q4b.3-4-marks}

\textbf{Find mean deviation for grouped data}

\begin{longtable}[]{@{}llllllll@{}}
\toprule\noalign{}
xᵢ & 4 & 8 & 11 & 17 & 20 & 24 & 32 \\
\midrule\noalign{}
\endhead
\bottomrule\noalign{}
\endlastfoot
fᵢ & 3 & 5 & 9 & 5 & 4 & 3 & 1 \\
\end{longtable}

\textbf{Solution}: N = Σfᵢ = 3+5+9+5+4+3+1 = 30

Mean =
\(\frac{\sum f_i x_i}{N} = \frac{3(4)+5(8)+9(11)+5(17)+4(20)+3(24)+1(32)}{30}\)

\(= \frac{12+40+99+85+80+72+32}{30} = \frac{420}{30} = 14\)

\begin{longtable}[]{@{}llll@{}}
\toprule\noalign{}
xᵢ & fᵢ & & xᵢ-14 \\
\midrule\noalign{}
\endhead
\bottomrule\noalign{}
\endlastfoot
4 & 3 & 10 & 30 \\
8 & 5 & 6 & 30 \\
11 & 9 & 3 & 27 \\
17 & 5 & 3 & 15 \\
20 & 4 & 6 & 24 \\
24 & 3 & 10 & 30 \\
32 & 1 & 18 & 18 \\
\end{longtable}

Σfᵢ\textbar xᵢ-14\textbar{} = 174

Mean deviation =
\(\frac{\sum f_i |x_i - \bar{x}|}{N} = \frac{174}{30} = 5.8\)

\begin{solutionbox}
Mean deviation = 5.8

\end{solutionbox}
\subsection*{Q.5 Attempt any two [14 marks
total]}\label{q.5-attempt-any-two-14-marks-total}

\subsubsection{Q5(A).1 [3 marks]}\label{q5a.1-3-marks}

\textbf{Find mean deviation for grouped data}

\begin{longtable}[]{@{}llllllll@{}}
\toprule\noalign{}
Class & 30-40 & 40-50 & 50-60 & 60-70 & 70-80 & 80-90 & 90-100 \\
\midrule\noalign{}
\endhead
\bottomrule\noalign{}
\endlastfoot
Freq & 3 & 7 & 12 & 15 & 8 & 3 & 2 \\
\end{longtable}

\textbf{Solution}:

\begin{longtable}[]{@{}llll@{}}
\toprule\noalign{}
Class & Mid-value & fᵢ & fᵢxᵢ \\
\midrule\noalign{}
\endhead
\bottomrule\noalign{}
\endlastfoot
30-40 & 35 & 3 & 105 \\
40-50 & 45 & 7 & 315 \\
50-60 & 55 & 12 & 660 \\
60-70 & 65 & 15 & 975 \\
70-80 & 75 & 8 & 600 \\
80-90 & 85 & 3 & 255 \\
90-100 & 95 & 2 & 190 \\
\end{longtable}

N = 50, Σfᵢxᵢ = 3100

Mean = 3100/50 = 62

\begin{longtable}[]{@{}lllll@{}}
\toprule\noalign{}
Class & xᵢ & fᵢ & & xᵢ-62 \\
\midrule\noalign{}
\endhead
\bottomrule\noalign{}
\endlastfoot
30-40 & 35 & 3 & 27 & 81 \\
40-50 & 45 & 7 & 17 & 119 \\
50-60 & 55 & 12 & 7 & 84 \\
60-70 & 65 & 15 & 3 & 45 \\
70-80 & 75 & 8 & 13 & 104 \\
80-90 & 85 & 3 & 23 & 69 \\
90-100 & 95 & 2 & 33 & 66 \\
\end{longtable}

Mean deviation = 568/50 = 11.36

\begin{solutionbox}
Mean deviation = 11.36

\end{solutionbox}
\subsubsection{Q5(A).2 [3 marks]}\label{q5a.2-3-marks}

\textbf{Find standard deviation for given data}

\begin{longtable}[]{@{}llllllllll@{}}
\toprule\noalign{}
Class & 60 & 61 & 62 & 63 & 64 & 65 & 66 & 67 & 68 \\
\midrule\noalign{}
\endhead
\bottomrule\noalign{}
\endlastfoot
Freq & 2 & 1 & 12 & 29 & 25 & 12 & 10 & 4 & 5 \\
\end{longtable}

\textbf{Solution}: N = 100, Mean = (2\times60 + 1\times61 + \ldots{} + 5\times68)/100 =
6380/100 = 63.8

\begin{longtable}[]{@{}lllll@{}}
\toprule\noalign{}
xᵢ & fᵢ & (xᵢ-63.8) & (xᵢ-63.8)^{2} & fᵢ(xᵢ-63.8)^{2} \\
\midrule\noalign{}
\endhead
\bottomrule\noalign{}
\endlastfoot
60 & 2 & -3.8 & 14.44 & 28.88 \\
61 & 1 & -2.8 & 7.84 & 7.84 \\
62 & 12 & -1.8 & 3.24 & 38.88 \\
63 & 29 & -0.8 & 0.64 & 18.56 \\
64 & 25 & 0.2 & 0.04 & 1.00 \\
65 & 12 & 1.2 & 1.44 & 17.28 \\
66 & 10 & 2.2 & 4.84 & 48.40 \\
67 & 4 & 3.2 & 10.24 & 40.96 \\
68 & 5 & 4.2 & 17.64 & 88.20 \\
\end{longtable}

Σfᵢ(xᵢ-x̄)^{2} = 290

Standard deviation = \sqrt(290/100) = \sqrt2.9 = 1.70

\begin{solutionbox}
σ = 1.70

\end{solutionbox}
\subsubsection{Q5(A).3 [3 marks]}\label{q5a.3-3-marks}

\textbf{Find mean for grouped data}

\begin{longtable}[]{@{}lllllll@{}}
\toprule\noalign{}
Class & 0-20 & 20-40 & 40-60 & 60-80 & 80-100 & 100-120 \\
\midrule\noalign{}
\endhead
\bottomrule\noalign{}
\endlastfoot
Freq & 26 & 31 & 35 & 42 & 82 & 71 \\
\end{longtable}

\textbf{Solution}:

\begin{longtable}[]{@{}llll@{}}
\toprule\noalign{}
Class & Mid-value & fᵢ & fᵢxᵢ \\
\midrule\noalign{}
\endhead
\bottomrule\noalign{}
\endlastfoot
0-20 & 10 & 26 & 260 \\
20-40 & 30 & 31 & 930 \\
40-60 & 50 & 35 & 1750 \\
60-80 & 70 & 42 & 2940 \\
80-100 & 90 & 82 & 7380 \\
100-120 & 110 & 71 & 7810 \\
\end{longtable}

N = 287, Σfᵢxᵢ = 21070

Mean = \(\frac{\sum f_i x_i}{N} = \frac{21070}{287} = 73.42\)

\begin{solutionbox}
Mean = 73.42

\end{solutionbox}
\subsubsection{Q5(B).1 [4 marks]}\label{q5b.1-4-marks}

\textbf{Solve differential equation (x + y + 1)^{2} dy/dx = 1}

\textbf{Solution}: Let z = x + y + 1, then dz/dx = 1 + dy/dx So dy/dx =
dz/dx - 1

Substituting: \(z^2(dz/dx - 1) = 1\) \(z^2 dz/dx - z^2 = 1\)
\(z^2 dz/dx = 1 + z^2\) \(\frac{z^2}{1 + z^2} dz = dx\)

Integrating: \(\int \frac{z^2}{1 + z^2} dz = \int dx\)

\(\int \left(1 - \frac{1}{1 + z^2}\right) dz = x + C\)

\(z - \tan^{-1}z = x + C\)

Substituting back z = x + y + 1:
\((x + y + 1) - \tan^{-1}(x + y + 1) = x + C\)

\begin{solutionbox}
\(y + 1 = \tan^{-1}(x + y + 1) + C\)

\end{solutionbox}
\subsubsection{Q5(B).2 [4 marks]}\label{q5b.2-4-marks}

\textbf{Solve dy/dx + y/x = e^{x}, y(0) = 2}

\textbf{Solution}: This is a linear differential equation of the form
dy/dx + P(x)y = Q(x)

Here P(x) = 1/x, Q(x) = e^{x}

Integrating factor:
\(I.F. = e^{\int \frac{1}{x} dx} = e^{\ln|x|} = |x| = x\) (for x
\textgreater{} 0)

Multiplying the equation by x: \(x \frac{dy}{dx} + y = xe^x\)

\(\frac{d}{dx}(xy) = xe^x\)

Integrating both sides: \(xy = \int xe^x dx\)

Using integration by parts for \intxe^{x} dx: Let

u = x, dv = e^{x} dx Then du =

dx, v = e^{x}

\(\int xe^x dx = xe^x - \int e^x dx = xe^x - e^x = e^x(x-1)\)

So: \(xy = e^x(x-1) + C\) \(y = \frac{e^x(x-1) + C}{x}\)

Using initial condition y(0) = 2: As x \rightarrow 0, we need to use L'Hôpital's
rule or series expansion.

From the original equation at

x = 0: dy/dx = e^{x} - y/x This suggests we

need to be more careful with the initial condition.

\textbf{Alternative approach}: Since the equation has a singularity at x
= 0, we solve in the neighborhood where x \neq 0.

\begin{solutionbox}
\(y = \frac{e^x(x-1) + C}{x}\) where C is determined by
boundary conditions.

\end{solutionbox}
\subsubsection{Q5(B).3 [4 marks]}\label{q5b.3-4-marks}

\textbf{Solve y dy/dx = \sqrt(1 + x^{2} + y^{2} + x^{2}y^{2})}

\textbf{Solution}: \(y \frac{dy}{dx} = \sqrt{1 + x^2 + y^2 + x^2y^2}\)

\(y \frac{dy}{dx} = \sqrt{(1 + x^2)(1 + y^2)}\)

\(\frac{y dy}{\sqrt{1 + y^2}} = \sqrt{1 + x^2} dx\)

Integrating both sides:
\(\int \frac{y dy}{\sqrt{1 + y^2}} = \int \sqrt{1 + x^2} dx\)

For the left side, let

u = 1 + y^{2}, then du = 2y dy:

\(\int \frac{y dy}{\sqrt{1 + y^2}} = \frac{1}{2} \int \frac{du}{\sqrt{u}} = \sqrt{u} = \sqrt{1 + y^2}\)

For the right side:
\(\int \sqrt{1 + x^2} dx = \frac{x\sqrt{1 + x^2}}{2} + \frac{1}{2}\ln|x + \sqrt{1 + x^2}| + C\)

Therefore:
\(\sqrt{1 + y^2} = \frac{x\sqrt{1 + x^2}}{2} + \frac{1}{2}\ln|x + \sqrt{1 + x^2}| + C\)

\begin{solutionbox}
\(\sqrt{1 + y^2} = \frac{x\sqrt{1 + x^2}}{2} + \frac{1}{2}\ln|x + \sqrt{1 + x^2}| + C\)

\end{solutionbox}
\begin{center}\rule{0.5\linewidth}{0.5pt}\end{center}

\subsection*{Formula Cheat Sheet}\label{formula-cheat-sheet}

\subsubsection{Matrix Operations}\label{matrix-operations}

\begin{itemize}
\tightlist
\item
  \((A + B)^T = A^T + B^T\)
\item
  \((AB)^T = B^T A^T\)
\item
  \(A \cdot adj(A) = |A| \cdot I\)
\item
  For 2\times2 matrix \([a \; b; c \; d]\): \(adj = [d \; -b; -c \; a]\)
\end{itemize}

\subsubsection{Differentiation Formulas}\label{differentiation-formulas}

\begin{itemize}
\tightlist
\item
  \(\frac{d}{dx}(\sin x) = \cos x\)
\item
  \(\frac{d}{dx}(\cos x) = -\sin x\)
\item
  \(\frac{d}{dx}(\tan x) = \sec^2 x\)
\item
  \(\frac{d}{dx}(\log x) = \frac{1}{x}\)
\item
  \(\frac{d}{dx}(e^x) = e^x\)
\item
  Chain rule: \(\frac{d}{dx}f(g(x)) = f'(g(x)) \cdot g'(x)\)
\end{itemize}

\subsubsection{Integration Formulas}\label{integration-formulas}

\begin{itemize}
\tightlist
\item
  \(\int \sin x \, dx = -\cos x + C\)
\item
  \(\int \cos x \, dx = \sin x + C\)
\item
  \(\int \sec^2 x \, dx = \tan x + C\)
\item
  \(\int \frac{1}{x} dx = \ln|x| + C\)
\item
  \(\int e^x dx = e^x + C\)
\item
  \(\int \frac{1}{x^2 + a^2} dx = \frac{1}{a}\tan^{-1}(\frac{x}{a}) + C\)
\end{itemize}

\subsubsection{Differential Equations}\label{differential-equations}

\begin{itemize}
\tightlist
\item
  \textbf{Linear DE}: \(\frac{dy}{dx} + P(x)y = Q(x)\)
\item
  \textbf{Integrating Factor}: \(I.F. = e^{\int P(x) dx}\)
\item
  \textbf{Variable Separable}:
  \(\frac{dy}{dx} = f(x)g(y) \Rightarrow \frac{dy}{g(y)} = f(x)dx\)
\end{itemize}

\subsubsection{Statistics}\label{statistics}

\begin{itemize}
\tightlist
\item
  \textbf{Mean}: \(\bar{x} = \frac{\sum x_i}{n}\) (ungrouped),
  \(\bar{x} = \frac{\sum f_i x_i}{\sum f_i}\) (grouped)
\item
  \textbf{Mean Deviation}: \(M.D. = \frac{\sum |x_i - \bar{x}|}{n}\)
\item
  \textbf{Standard Deviation}:
  \(\sigma = \sqrt{\frac{\sum (x_i - \bar{x})^2}{n}}\)
\end{itemize}

\subsection*{Problem-Solving
Strategies}\label{problem-solving-strategies}

\subsubsection{Matrix Problems}\label{matrix-problems}

\begin{enumerate}
\tightlist
\item
  \textbf{Always check dimensions} before operations
\item
  \textbf{For inverse}: Calculate determinant first, then adjoint
\item
  \textbf{For system of equations}: Use \(X = A^{-1}B\) where \(AX = B\)
\end{enumerate}

\subsubsection{Differentiation Problems}\label{differentiation-problems}

\begin{enumerate}
\tightlist
\item
  \textbf{Identify the type}: Chain rule, product rule, quotient rule
\item
  \textbf{For implicit differentiation}: Differentiate both sides,
  collect dy/dx terms
\item
  \textbf{For parametric}: Use \(\frac{dy}{dx} = \frac{dy/dt}{dx/dt}\)
\end{enumerate}

\subsubsection{Integration Problems}\label{integration-problems}

\begin{enumerate}
\tightlist
\item
  \textbf{Try substitution} if you see function and its derivative
\item
  \textbf{Use integration by parts} for products (LIATE rule)
\item
  \textbf{For definite integrals}: Check for symmetry properties
\end{enumerate}

\subsubsection{Differential Equations}\label{differential-equations-1}

\begin{enumerate}
\tightlist
\item
  \textbf{Identify type}: Separable, linear, exact
\item
  \textbf{For linear DE}: Find integrating factor first
\item
  \textbf{Always verify} your solution by substitution
\end{enumerate}

\subsubsection{Statistics Problems}\label{statistics-problems}

\begin{enumerate}
\tightlist
\item
  \textbf{Find mean first} for deviation calculations
\item
  \textbf{Use grouped data formulas} when data is in classes
\item
  \textbf{Create frequency table} to organize calculations
\end{enumerate}

\subsection*{Common Mistakes to Avoid}\label{common-mistakes-to-avoid}

\begin{enumerate}
\tightlist
\item
  \textbf{Matrix multiplication}: Order matters (AB \neq BA generally)
\item
  \textbf{Chain rule}: Don't forget to multiply by derivative of inner
  function
\item
  \textbf{Integration by parts}: Choose u and dv carefully using LIATE
\item
  \textbf{Differential equations}: Don't forget the constant of
  integration
\item
  \textbf{Statistics}: Use correct formula for grouped vs ungrouped data
\end{enumerate}

\subsection*{Exam Tips}\label{exam-tips}

\begin{enumerate}
\tightlist
\item
  \textbf{Read questions carefully} - especially for OR questions
\item
  \textbf{Show all steps} - partial marks are awarded
\item
  \textbf{Check units and signs} in your final answers
\item
  \textbf{Verify solutions} when possible by substitution
\item
  \textbf{Manage time wisely} - attempt questions you're confident about
  first
\item
  \textbf{Use standard formulas} - memorize the formula sheet content
\item
  \textbf{For fill-in-blanks}: Eliminate obviously wrong options first
\end{enumerate}


\end{document}
