\documentclass[10pt,a4paper]{article}

% content/resources/templates/preamble.tex
\usepackage[margin=0.6in]{geometry}
\author{Milav Dabgar}
\usepackage{amsmath,amssymb,amsthm}
\usepackage{booktabs}
\usepackage{multirow}
\usepackage{xcolor}
\usepackage{tcolorbox}
\tcbuselibrary{breakable,skins}
\usepackage[colorlinks=true,linkcolor=blue]{hyperref}
\usepackage{titlesec}
\usepackage{enumitem}
\usepackage{tikz}
\usepackage{pgfplots}
\usepackage{circuitikz}
\usepackage[version=4]{mhchem}
\usepackage{longtable}
\usepackage{array}
\usepackage{float}
\usepackage{caption}
\usepackage{listings}

\lstset{
  basicstyle=\small\ttfamily,
  breaklines=true,
  breakatwhitespace=false,
  postbreak=\mbox{\textcolor{red}{$\hookrightarrow$}\space},
  float=false,
  numbers=left,
  numberstyle=\tiny\color{gray},
  numbersep=10pt,
  xleftmargin=2em,
  keywordstyle=\color{blue},
  commentstyle=\color{green!60!black},
  stringstyle=\color{purple},
  backgroundcolor=\color{gray!5},
  showstringspaces=false,
  tabsize=2,
  captionpos=b,
  keepspaces=true,
  columns=flexible
}

\pgfplotsset{compat=1.18}
\usetikzlibrary{shapes,arrows,positioning,calc,patterns,decorations.pathmorphing,decorations.markings,arrows.meta}

% Color scheme
\definecolor{headcolor}{RGB}{0,102,204}
\definecolor{keycolor}{RGB}{220,20,60}
\definecolor{solutioncolor}{RGB}{34,139,34}
\definecolor{mnemoniccolor}{RGB}{148,0,211}
\definecolor{codecolor}{RGB}{0,0,100}

% Spacing
\setlength{\parskip}{3pt}
\setlist[itemize]{nosep}
\setlist[enumerate]{nosep}

% Title formatting
\titleformat{\section}{\Large\bfseries\color{headcolor}}{\thesection}{1em}{}
\titleformat{\subsection}{\large\bfseries\color{headcolor}}{\thesubsection}{1em}{}

% Pandoc tightlist compatibility
\providecommand{\tightlist}{%
  \setlength{\itemsep}{0pt}\setlength{\parskip}{0pt}}

% Pandoc longtable compatibility
\newcounter{none}
\def\thenone{}


% content/resources/templates/english-boxes.tex
% This file is currently empty - it exists to maintain consistency with the import structure.
% Add custom environments here if needed in the future.


\begin{document}

\begin{center}
{\Huge\bfseries\color{headcolor} Subject Name Solutions}\\[5pt]
{\LARGE 4320001 -- Winter 2022}\\[3pt]
{\large Semester 1 Study Material}\\[3pt]
{\normalsize\textit{Detailed Solutions and Explanations}}
\end{center}

\vspace{10pt}

\subsection*{Q.1 [14 marks]}\label{q.1-14-marks}

\textbf{Fill in the blanks using appropriate choice from the given
options.}

\subsubsection{Q1.1 [1 mark]}\label{q1.1-1-mark}

\textbf{Order of the matrix
\(\begin{bmatrix} 1 & 4 \\ 3 & 2 \end{bmatrix}\) is \_\_\_\_\_\_\_\_}

\begin{solutionbox}
b. 2 \times 2

\textbf{Solution}: Matrix has 2 rows and 2 columns, so order is 2 \times 2.

\end{solutionbox}
\subsubsection{Q1.2 [1 mark]}\label{q1.2-1-mark}

\textbf{If \(A = \begin{bmatrix} 1 & 2 \\ -1 & 1 \end{bmatrix}\) then
\(2A - 3I\) = \_\_\_\_\_\_}

\begin{solutionbox}
a. \(\begin{bmatrix} -1 & 4 \\ -2 & -1 \end{bmatrix}\)

\textbf{Solution}:
\(2A = 2\begin{bmatrix} 1 & 2 \\ -1 & 1 \end{bmatrix} = \begin{bmatrix} 2 & 4 \\ -2 & 2 \end{bmatrix}\)

\(3I = 3\begin{bmatrix} 1 & 0 \\ 0 & 1 \end{bmatrix} = \begin{bmatrix} 3 & 0 \\ 0 & 3 \end{bmatrix}\)

\(2A - 3I = \begin{bmatrix} 2 & 4 \\ -2 & 2 \end{bmatrix} - \begin{bmatrix} 3 & 0 \\ 0 & 3 \end{bmatrix} = \begin{bmatrix} -1 & 4 \\ -2 & -1 \end{bmatrix}\)

\end{solutionbox}
\subsubsection{Q1.3 [1 mark]}\label{q1.3-1-mark}

\textbf{If \(A_{2\times3}\) and \(B_{3\times4}\) are matrices then order of \(AB\)
is \_\_\_\_\_\_\_\_}

\begin{solutionbox}
b. 2 \times 4

\textbf{Solution}: For matrix multiplication \(AB\), if \(A\) is \(m\timesn\)
and \(B\) is \(n\timesp\), then \(AB\) is \(m\timesp\). Here:
\(A_{2\times3} \times B_{3\times4} = (AB)_{2\times4}\)

\end{solutionbox}
\subsubsection{Q1.4 [1 mark]}\label{q1.4-1-mark}

\textbf{If \(AB = I\) then matrix \(B\) = \ldots{}}

\begin{solutionbox}
c.~\(A^{-1}\)

\textbf{Solution}: If \(AB = I\), then \(B\) is the inverse of \(A\),
i.e., \(B = A^{-1}\)

\end{solutionbox}
\subsubsection{Q1.5 [1 mark]}\label{q1.5-1-mark}

\textbf{\(\frac{d}{dx}(x^3 + 3^x + 3^3)\) = \_\_\_\_\_\_\_\_}

\begin{solutionbox}
c.~\(3x^2 + 3^x \log 3\)

\textbf{Solution}:
\(\frac{d}{dx}(x^3 + 3^x + 3^3) = 3x^2 + 3^x \log 3 + 0 = 3x^2 + 3^x \log 3\)

\end{solutionbox}
\subsubsection{Q1.6 [1 mark]}\label{q1.6-1-mark}

\textbf{If \(f(x) = e^{3x}\) then \(f'(0)\) = \_\_\_\_\_\_\_\_}

\begin{solutionbox}
b. 3

\textbf{Solution}: \(f'(x) = 3e^{3x}\)
\(f'(0) = 3e^{3(0)} = 3e^0 = 3(1) = 3\)

\end{solutionbox}
\subsubsection{Q1.7 [1 mark]}\label{q1.7-1-mark}

\textbf{If \(y = e^x + 100x\) then \(\frac{d^2y}{dx^2}\) =
\_\_\_\_\_\_\_\_}

\begin{solutionbox}
a. \(e^x\)

\textbf{Solution}: \(\frac{dy}{dx} = e^x + 100\)
\(\frac{d^2y}{dx^2} = e^x + 0 = e^x\)

\end{solutionbox}
\subsubsection{Q1.8 [1 mark]}\label{q1.8-1-mark}

\textbf{\(\int \frac{1}{x^2} dx\) = \_\_\_\_\_\_\_\_ + c}

\begin{solutionbox}
b. \(-\frac{1}{x}\)

\textbf{Solution}:
\(\int x^{-2} dx = \frac{x^{-2+1}}{-2+1} = \frac{x^{-1}}{-1} = -\frac{1}{x} + c\)

\end{solutionbox}
\subsubsection{Q1.9 [1 mark]}\label{q1.9-1-mark}

\textbf{\(\int (\log a) dx\) = \_\_\_\_\_\_\_\_ + c}

\begin{solutionbox}
a. \(x\log a\)

\textbf{Solution}: Since \(\log a\) is a constant:
\(\int (\log a) dx = (\log a) \int dx = x\log a + c\)

\end{solutionbox}
\subsubsection{Q1.10 [1 mark]}\label{q1.10-1-mark}

\textbf{\(\int_0^1 e^x dx\) = \_\_\_\_\_\_\_\_}

\begin{solutionbox}
a. \(e - 1\)

\textbf{Solution}: \(\int_0^1 e^x dx = [e^x]_0^1 = e^1 - e^0 = e - 1\)

\end{solutionbox}
\subsubsection{Q1.11 [1 mark]}\label{q1.11-1-mark}

\textbf{The Order and degree of the differential equation
\(\frac{d^2y}{dx^2} - 5\frac{dy}{dx} + 6y = 0\) are respectively
\_\_\_\_\_\_\_\_ and \_\_\_\_\_\_\_\_}

\begin{solutionbox}
d.~2,1

\textbf{Solution}: Order = highest derivative = 2 Degree = power of
highest derivative = 1

\end{solutionbox}
\subsubsection{Q1.12 [1 mark]}\label{q1.12-1-mark}

\textbf{Integrating factor (I.F) of the differential equation
\(\frac{dy}{dx} + y = 3x\) is \_\_\_\_\_\_\_\_}

\begin{solutionbox}
c.~\(e^x\)

\textbf{Solution}: For equation \(\frac{dy}{dx} + Py = Q\) where
\(P = 1\): I.F. = \(e^{\int P dx} = e^{\int 1 dx} = e^x\)

\end{solutionbox}
\subsubsection{Q1.13 [1 mark]}\label{q1.13-1-mark}

\textbf{Mean of first five natural numbers is \_\_\_\_\_\_\_\_}

\begin{solutionbox}
c.~3

\textbf{Solution}: First five natural numbers: 1, 2, 3, 4, 5 Mean =
\(\frac{1+2+3+4+5}{5} = \frac{15}{5} = 3\)

\end{solutionbox}
\subsubsection{Q1.14 [1 mark]}\label{q1.14-1-mark}

\textbf{If the mean of the observations 11, x, 19, 21, y, 29 is 20 then
\(x + y\) = \_\_\_\_\_\_\_\_}

\begin{solutionbox}
a. 40

\textbf{Solution}: Mean = \(\frac{11+x+19+21+y+29}{6} = 20\)
\(\frac{80+x+y}{6} = 20\) \(80+x+y = 120\) \(x+y = 40\)

\end{solutionbox}
\subsection*{Q.2 (A) [6 marks]}\label{q.2-a-6-marks}

\textbf{Attempt any two}

\subsubsection{Q2.1 [3 marks]}\label{q2.1-3-marks}

\textbf{If \(A = \begin{bmatrix} 1 & 3 & 2 \\ 2 & 0 & 1 \end{bmatrix}\)
and \(B = \begin{bmatrix} 2 & 1 \\ -1 & 1 \\ 1 & -1 \end{bmatrix}\) then
find \((AB)^T\)}

\begin{solutionbox}

\textbf{Solution}: First find \(AB\):
\(AB = \begin{bmatrix} 1 & 3 & 2 \\ 2 & 0 & 1 \end{bmatrix} \begin{bmatrix} 2 & 1 \\ -1 & 1 \\ 1 & -1 \end{bmatrix}\)

\(AB = \begin{bmatrix} 1(2)+3(-1)+2(1) & 1(1)+3(1)+2(-1) \\ 2(2)+0(-1)+1(1) & 2(1)+0(1)+1(-1) \end{bmatrix}\)

\(AB = \begin{bmatrix} 2-3+2 & 1+3-2 \\ 4+0+1 & 2+0-1 \end{bmatrix} = \begin{bmatrix} 1 & 2 \\ 5 & 1 \end{bmatrix}\)

\((AB)^T = \begin{bmatrix} 1 & 5 \\ 2 & 1 \end{bmatrix}\)

\end{solutionbox}
\subsubsection{Q2.2 [3 marks]}\label{q2.2-3-marks}

\textbf{If \(1 + x + x^2 = 0\) and \(x^3 = 1\) then prove that
\(\begin{bmatrix} 1 & x^2 \\ x & x \end{bmatrix} \cdot \begin{bmatrix} x & x^2 \\ 1 & x \end{bmatrix} = \begin{bmatrix} -1 & -1 \\ -1 & 2 \end{bmatrix}\)}

\textbf{Solution}: Given: \(1 + x + x^2 = 0\) and \(x^3 = 1\)

From \(1 + x + x^2 = 0\), we get \(x^2 = -1 - x\)

Let's compute the matrix product:
\(\begin{bmatrix} 1 & x^2 \\ x & x \end{bmatrix} \cdot \begin{bmatrix} x & x^2 \\ 1 & x \end{bmatrix}\)

\(= \begin{bmatrix} 1(x)+x^2(1) & 1(x^2)+x^2(x) \\ x(x)+x(1) & x(x^2)+x(x) \end{bmatrix}\)

\(= \begin{bmatrix} x+x^2 & x^2+x^3 \\ x^2+x & x^3+x^2 \end{bmatrix}\)

Since \(x^3 = 1\) and \(x+x^2 = -1\):
\(= \begin{bmatrix} -1 & x^2+1 \\ -1 & 1+x^2 \end{bmatrix}\)

Since \(x^2 = -1-x\), we have \(x^2+1 = -x\) and \(1+x^2 = -x\)

From \(1+x+x^2 = 0\), if \(x\) is a cube root of unity, then
\(x^2+1 = -x = -1\)

\(= \begin{bmatrix} -1 & -1 \\ -1 & 2 \end{bmatrix}\) (verified)

\subsubsection{Q2.3 [3 marks]}\label{q2.3-3-marks}

\textbf{Solve \(\frac{dy}{dx} + x^2e^{-y} = 0\)}

\textbf{Solution}: \(\frac{dy}{dx} = -x^2e^{-y}\)

Separating variables: \(e^y dy = -x^2 dx\)

Integrating both sides: \(\int e^y dy = \int -x^2 dx\)

\(e^y = -\frac{x^3}{3} + C\)

\(y = \ln\left(-\frac{x^3}{3} + C\right)\)

\subsection*{Q.2 (B) [8 marks]}\label{q.2-b-8-marks}

\textbf{Attempt any two}

\subsubsection{Q2.4 [4 marks]}\label{q2.4-4-marks}

\textbf{If
\(A = \begin{bmatrix} 1 & 2 & 2 \\ 2 & 1 & 2 \\ 2 & 2 & 1 \end{bmatrix}\)
then prove that \(A^2 - 4A - 5I_3 = O\)}

\textbf{Solution}: First calculate \(A^2\):
\(A^2 = \begin{bmatrix} 1 & 2 & 2 \\ 2 & 1 & 2 \\ 2 & 2 & 1 \end{bmatrix} \begin{bmatrix} 1 & 2 & 2 \\ 2 & 1 & 2 \\ 2 & 2 & 1 \end{bmatrix}\)

\(A^2 = \begin{bmatrix} 1+4+4 & 2+2+4 & 2+4+2 \\ 2+2+4 & 4+1+4 & 4+2+2 \\ 2+4+2 & 4+2+2 & 4+4+1 \end{bmatrix} = \begin{bmatrix} 9 & 8 & 8 \\ 8 & 9 & 8 \\ 8 & 8 & 9 \end{bmatrix}\)

Now calculate \(A^2 - 4A - 5I_3\):
\(4A = \begin{bmatrix} 4 & 8 & 8 \\ 8 & 4 & 8 \\ 8 & 8 & 4 \end{bmatrix}\)

\(5I_3 = \begin{bmatrix} 5 & 0 & 0 \\ 0 & 5 & 0 \\ 0 & 0 & 5 \end{bmatrix}\)

\(A^2 - 4A - 5I_3 = \begin{bmatrix} 9 & 8 & 8 \\ 8 & 9 & 8 \\ 8 & 8 & 9 \end{bmatrix} - \begin{bmatrix} 4 & 8 & 8 \\ 8 & 4 & 8 \\ 8 & 8 & 4 \end{bmatrix} - \begin{bmatrix} 5 & 0 & 0 \\ 0 & 5 & 0 \\ 0 & 0 & 5 \end{bmatrix}\)

\(= \begin{bmatrix} 0 & 0 & 0 \\ 0 & 0 & 0 \\ 0 & 0 & 0 \end{bmatrix} = O\)

\subsubsection{Q2.5 [4 marks]}\label{q2.5-4-marks}

\textbf{For which values of x, the matrix
\(\begin{bmatrix} 3-x & 2 & 2 \\ 1 & 4-x & 1 \\ -2 & -4 & -1-x \end{bmatrix}\)
is singular matrix?}

\textbf{Solution}: A matrix is singular when its determinant equals
zero.

\(\det(A) = (3-x)\begin{vmatrix} 4-x & 1 \\ -4 & -1-x \end{vmatrix} - 2\begin{vmatrix} 1 & 1 \\ -2 & -1-x \end{vmatrix} + 2\begin{vmatrix} 1 & 4-x \\ -2 & -4 \end{vmatrix}\)

\(= (3-x)[(4-x)(-1-x) - (1)(-4)] - 2[1(-1-x) - 1(-2)] + 2[1(-4) - (4-x)(-2)]\)

\(= (3-x)[-(4-x)(1+x) + 4] - 2[-1-x+2] + 2[-4 + 2(4-x)]\)

\(= (3-x)[-4-4x+x+x^2+4] - 2[1-x] + 2[-4+8-2x]\)

\(= (3-x)[x^2-3x] - 2(1-x) + 2(4-2x)\)

\(= (3-x)x(x-3) - 2 + 2x + 8 - 4x\)

\(= -(3-x)x(3-x) + 6 - 2x\)

\(= -x(3-x)^2 + 6 - 2x\)

Setting equal to zero: \(-x(3-x)^2 + 6 - 2x = 0\)

This gives us \(x = 1,

x = 2,

x = 3\)


\subsubsection{Q2.6 [4 marks]}\label{q2.6-4-marks}

\textbf{Solve by using matrix method: \(2y + 5x = 4\), \(7x + 3y = 5\)}

\textbf{Solution}: Write in matrix form \(AX = B\):
\(\begin{bmatrix} 5 & 2 \\ 7 & 3 \end{bmatrix} \begin{bmatrix} x \\ y \end{bmatrix} = \begin{bmatrix} 4 \\ 5 \end{bmatrix}\)

Find \(A^{-1}\): \(\det(A) = 5(3) - 2(7) = 15 - 14 = 1\)

\(A^{-1} = \frac{1}{1}\begin{bmatrix} 3 & -2 \\ -7 & 5 \end{bmatrix} = \begin{bmatrix} 3 & -2 \\ -7 & 5 \end{bmatrix}\)

\(X = A^{-1}B = \begin{bmatrix} 3 & -2 \\ -7 & 5 \end{bmatrix} \begin{bmatrix} 4 \\ 5 \end{bmatrix} = \begin{bmatrix} 12-10 \\ -28+25 \end{bmatrix} = \begin{bmatrix} 2 \\ -3 \end{bmatrix}\)

Therefore: \(x = 2,

y = -3\)


\subsection*{Q.3 (A) [6 marks]}\label{q.3-a-6-marks}

\textbf{Attempt any two}

\subsubsection{Q3.1 [3 marks]}\label{q3.1-3-marks}

\textbf{Find the derivative of function using definition
\(f(x) = \sqrt{x}\)}

\textbf{Solution}: Using definition:
\(f'(x) = \lim_{h \to 0} \frac{f(x+h) - f(x)}{h}\)

\(f'(x) = \lim_{h \to 0} \frac{\sqrt{x+h} - \sqrt{x}}{h}\)

Rationalize the numerator:
\(= \lim_{h \to 0} \frac{(\sqrt{x+h} - \sqrt{x})(\sqrt{x+h} + \sqrt{x})}{h(\sqrt{x+h} + \sqrt{x})}\)

\(= \lim_{h \to 0} \frac{(x+h) - x}{h(\sqrt{x+h} + \sqrt{x})}\)

\(= \lim_{h \to 0} \frac{h}{h(\sqrt{x+h} + \sqrt{x})}\)

\(= \lim_{h \to 0} \frac{1}{\sqrt{x+h} + \sqrt{x}}\)

\(= \frac{1}{\sqrt{x} + \sqrt{x}} = \frac{1}{2\sqrt{x}}\)

\subsubsection{Q3.2 [3 marks]}\label{q3.2-3-marks}

\textbf{Find \(\frac{dy}{dx}\) if \(x + y = \sin(xy)\)}

\textbf{Solution}: Differentiating both sides with respect to \(x\):
\(\frac{d}{dx}(x + y) = \frac{d}{dx}[\sin(xy)]\)

\(1 + \frac{dy}{dx} = \cos(xy) \cdot \frac{d}{dx}(xy)\)

\(1 + \frac{dy}{dx} = \cos(xy) \cdot \left(x\frac{dy}{dx} + y\right)\)

\(1 + \frac{dy}{dx} = \cos(xy) \cdot x\frac{dy}{dx} + y\cos(xy)\)

\(1 + \frac{dy}{dx} - x\cos(xy)\frac{dy}{dx} = y\cos(xy)\)

\(\frac{dy}{dx}(1 - x\cos(xy)) = y\cos(xy) - 1\)

\(\frac{dy}{dx} = \frac{y\cos(xy) - 1}{1 - x\cos(xy)}\)

\subsubsection{Q3.3 [3 marks]}\label{q3.3-3-marks}

\textbf{Evaluate: \(\int \frac{\sin^3x + \cos^3x}{\sin^2x \cos^2x} dx\)}

\textbf{Solution}:
\(\int \frac{\sin^3x + \cos^3x}{\sin^2x \cos^2x} dx = \int \frac{\sin^3x}{\sin^2x \cos^2x} dx + \int \frac{\cos^3x}{\sin^2x \cos^2x} dx\)

\(= \int \frac{\sin x}{\cos^2x} dx + \int \frac{\cos x}{\sin^2x} dx\)

\(= \int \sin x \sec^2x dx + \int \cos x \csc^2x dx\)

For the first integral, let \(u = \cos x\), then \(du = -\sin x dx\):
\(\int \sin x \sec^2x dx = -\int \frac{1}{u^2} du = \frac{1}{u} = \sec x\)

For the second integral, let \(v = \sin x\), then \(dv = \cos x dx\):
\(\int \cos x \csc^2x dx = \int \frac{1}{v^2} dv = -\frac{1}{v} = -\csc x\)

Therefore:
\(\int \frac{\sin^3x + \cos^3x}{\sin^2x \cos^2x} dx = \sec x - \csc x + C\)

\subsection*{Q.3 (B) [8 marks]}\label{q.3-b-8-marks}

\textbf{Attempt any two}

\subsubsection{Q3.4 [4 marks]}\label{q3.4-4-marks}

\textbf{If \(y = e^x \cdot \sin x\) then prove that
\(\frac{d^2y}{dx^2} - 2\frac{dy}{dx} + 2y = 0\)}

\textbf{Solution}: Given: \(y = e^x \sin x\)

Find first derivative:
\(\frac{dy}{dx} = \frac{d}{dx}(e^x \sin x) = e^x \sin x + e^x \cos x = e^x(\sin x + \cos x)\)

Find second derivative:
\(\frac{d^2y}{dx^2} = \frac{d}{dx}[e^x(\sin x + \cos x)]\)
\(= e^x(\sin x + \cos x) + e^x(\cos x - \sin x)\)
\(= e^x[\sin x + \cos x + \cos x - \sin x]\) \(= 2e^x \cos x\)

Now verify: \(\frac{d^2y}{dx^2} - 2\frac{dy}{dx} + 2y\)
\(= 2e^x \cos x - 2e^x(\sin x + \cos x) + 2e^x \sin x\)
\(= 2e^x \cos x - 2e^x \sin x - 2e^x \cos x + 2e^x \sin x\) \(= 0\)

Hence proved.

\subsubsection{Q3.5 [4 marks]}\label{q3.5-4-marks}

\textbf{Find maximum and minimum value of function
\(f(x) = x^3 - 4x^2 + 5x + 7\)}

\textbf{Solution}: Find critical points by setting \(f'(x) = 0\):
\(f'(x) = 3x^2 - 8x + 5 = 0\)

Using quadratic formula:
\(x = \frac{8 \pm \sqrt{64 - 60}}{6} = \frac{8 \pm 2}{6}\)

So \(x = \frac{5}{3}\) or \(x = 1\)

Find second derivative: \(f''(x) = 6x - 8\)

Test critical points: - At \(x = 1\): \(f''(1) = 6(1) - 8 = -2 < 0\) \rightarrow
Local maximum - At \(x = \frac{5}{3}\):
\(f''\left(\frac{5}{3}\right) = 6\left(\frac{5}{3}\right) - 8 = 10 - 8 = 2 > 0\)
\rightarrow Local minimum

Calculate function values: - \(f(1) = 1 - 4 + 5 + 7 = 9\) (local
maximum) -
\(f\left(\frac{5}{3}\right) = \left(\frac{5}{3}\right)^3 - 4\left(\frac{5}{3}\right)^2 + 5\left(\frac{5}{3}\right) + 7 = \frac{125}{27} - \frac{100}{9} + \frac{25}{3} + 7 = \frac{158}{27}\)
(local minimum)

\subsubsection{Q3.6 [4 marks]}\label{q3.6-4-marks}

\textbf{The equation of motion of particle is \(s = t^3 - 6t^2 + 9t\)
then} \textbf{(i) Find Velocity and acceleration at \(t = 3\) second.}
\textbf{(ii) Find ``t'' when acceleration is zero.}

\textbf{Solution}: Given: \(s = t^3 - 6t^2 + 9t\)

Velocity: \(v = \frac{ds}{dt} = 3t^2 - 12t + 9\)

Acceleration: \(a = \frac{dv}{dt} = 6t - 12\)

\textbf{(i) At \(t = 3\) seconds:} - Velocity:
\(v(3) = 3(9) - 12(3) + 9 = 27 - 36 + 9 = 0\) m/s - Acceleration:
\(a(3) = 6(3) - 12 = 18 - 12 = 6\) m/s^{2}

\textbf{(ii) When acceleration is zero:} \(6t - 12 = 0\) \(t = 2\)
seconds

\subsection*{Q.4 (A) [6 marks]}\label{q.4-a-6-marks}

\textbf{Attempt any two}

\subsubsection{Q4.1 [3 marks]}\label{q4.1-3-marks}

\textbf{Evaluate: \(\int \frac{x}{(x+1)(x+2)} dx\)}

\textbf{Solution}: Using partial fractions:
\(\frac{x}{(x+1)(x+2)} = \frac{A}{x+1} + \frac{B}{x+2}\)

\(x = A(x+2) + B(x+1)\)

Setting \(x = -1\): \(-1 = A(1) \Rightarrow

A = -1\) Setting \(x = -2\):

\(-2 = B(-1) \Rightarrow B = 2\)

\(\int \frac{x}{(x+1)(x+2)} dx = \int \left(\frac{-1}{x+1} + \frac{2}{x+2}\right) dx\)

\(= -\ln|x+1| + 2\ln|x+2| + C\)

\(= \ln\left|\frac{(x+2)^2}{x+1}\right| + C\)

\subsubsection{Q4.2 [3 marks]}\label{q4.2-3-marks}

\textbf{Evaluate: \(\int_0^{\pi/2} \frac{\sin x}{\sin x + \cos x} dx\)}

\textbf{Solution}: Let
\(I = \int_0^{\pi/2} \frac{\sin x}{\sin x + \cos x} dx\) \ldots{} (1)

Using property \(\int_0^a f(x) dx = \int_0^a f(a-x) dx\):

\(I = \int_0^{\pi/2} \frac{\sin(\pi/2 - x)}{\sin(\pi/2 - x) + \cos(\pi/2 - x)} dx\)

\(= \int_0^{\pi/2} \frac{\cos x}{\cos x + \sin x} dx\) \ldots{} (2)

Adding equations (1) and (2):
\(2I = \int_0^{\pi/2} \frac{\sin x + \cos x}{\sin x + \cos x} dx = \int_0^{\pi/2} 1 dx\)

\(2I = \left[x\right]_0^{\pi/2} = \frac{\pi}{2}\)

Therefore: \(I = \frac{\pi}{4}\)

\subsubsection{Q4.3 [3 marks]}\label{q4.3-3-marks}

\textbf{If mean of 15, 7, 6, a, 3 is 7 then find the value of ``a''.}

\textbf{Solution}: Mean =
\(\frac{\text{Sum of observations}}{\text{Number of observations}}\)

\(7 = \frac{15 + 7 + 6 + a + 3}{5}\)

\(7 = \frac{31 + a}{5}\)

\(35 = 31 + a\)

\(a = 4\)

\subsection*{Q.4 (B) [8 marks]}\label{q.4-b-8-marks}

\textbf{Attempt any two}

\subsubsection{Q4.4 [4 marks]}\label{q4.4-4-marks}

\textbf{Evaluate: \(\int x^2 e^x dx\)}

\textbf{Solution}: Using integration by parts twice:

Let \(u = x^2\), \(dv = e^x dx\) Then \(du = 2x dx\), \(v = e^x\)

\(\int x^2 e^x dx = x^2 e^x - \int 2x e^x dx\)

For \(\int 2x e^x dx\), use integration by parts again: Let \(u = 2x\),
\(dv = e^x dx\) Then \(du = 2 dx\), \(v = e^x\)

\(\int 2x e^x dx = 2x e^x - \int 2 e^x dx = 2x e^x - 2e^x\)

Therefore: \(\int x^2 e^x dx = x^2 e^x - (2x e^x - 2e^x) + C\)
\(= x^2 e^x - 2x e^x + 2e^x + C\) \(= e^x(x^2 - 2x + 2) + C\)

\subsubsection{Q4.5 [4 marks]}\label{q4.5-4-marks}

\textbf{Find the area of the region bounded by curve \(y = 2x^2\), lines
\(x = 1\), \(x = 3\) and X-axis.}

\textbf{Solution}: Area = \(\int_1^3 2x^2 dx\)

\(= 2\int_1^3 x^2 dx\)

\(= 2\left[\frac{x^3}{3}\right]_1^3\)

\(= \frac{2}{3}[x^3]_1^3\)

\(= \frac{2}{3}(27 - 1)\)

\(= \frac{2}{3} \times 26\)

\(= \frac{52}{3}\) square units

\subsubsection{Q4.6 [4 marks]}\label{q4.6-4-marks}

\textbf{Find the mean for the following grouped data using short
method:}

\begin{longtable}[]{@{}lllllll@{}}
\toprule\noalign{}
Marks & 21-25 & 26-30 & 31-35 & 36-40 & 41-45 & 46-50 \\
\midrule\noalign{}
\endhead
\bottomrule\noalign{}
\endlastfoot
No.~of Students & 8 & 10 & 24 & 30 & 12 & 16 \\
\end{longtable}

\textbf{Solution}: Using step deviation method:

\begin{longtable}[]{@{}lllll@{}}
\toprule\noalign{}
Class & \(x_i\) & \(f_i\) & \(d_i = \frac{x_i - A}{h}\) & \(f_i d_i\) \\
\midrule\noalign{}
\endhead
\bottomrule\noalign{}
\endlastfoot
21-25 & 23 & 8 & -3 & -24 \\
26-30 & 28 & 10 & -2 & -20 \\
31-35 & 33 & 24 & -1 & -24 \\
36-40 & 38 & 30 & 0 & 0 \\
41-45 & 43 & 12 & 1 & 12 \\
46-50 & 48 & 16 & 2 & 32 \\
Total & - & 100 & - & -24 \\
\end{longtable}

Assumed mean \(A = 38\), Class width \(h = 5\)

Mean = \(A + \frac{\sum f_i d_i}{\sum f_i} \times h\)

Mean = \(38 + \frac{-24}{100} \times 5 = 38 - 1.2 = 36.8\)

\subsection*{Q.5 (A) [6 marks]}\label{q.5-a-6-marks}

\textbf{Attempt any two}

\subsubsection{Q5.1 [3 marks]}\label{q5.1-3-marks}

\textbf{Find the mean for the following grouped data:}

\begin{longtable}[]{@{}lllllll@{}}
\toprule\noalign{}
\(x_i\) & 92 & 93 & 97 & 98 & 102 & 104 \\
\midrule\noalign{}
\endhead
\bottomrule\noalign{}
\endlastfoot
\(f_i\) & 3 & 2 & 3 & 2 & 6 & 4 \\
\end{longtable}

\textbf{Solution}: Mean = \(\frac{\sum f_i x_i}{\sum f_i}\)

\begin{longtable}[]{@{}lll@{}}
\toprule\noalign{}
\(x_i\) & \(f_i\) & \(f_i x_i\) \\
\midrule\noalign{}
\endhead
\bottomrule\noalign{}
\endlastfoot
92 & 3 & 276 \\
93 & 2 & 186 \\
97 & 3 & 291 \\
98 & 2 & 196 \\
102 & 6 & 612 \\
104 & 4 & 416 \\
Total & 20 & 1977 \\
\end{longtable}

Mean = \(\frac{1977}{20} = 98.85\)

\subsubsection{Q5.2 [3 marks]}\label{q5.2-3-marks}

\textbf{Find the mean deviation of 4, 6, 2, 4, 5, 4, 4, 5, 3, 4.}

\textbf{Solution}: First find the mean: Mean =
\(\frac{4+6+2+4+5+4+4+5+3+4}{10} = \frac{41}{10} = 4.1\)

Calculate deviations from mean:

\begin{longtable}[]{@{}ll@{}}
\toprule\noalign{}
\(x_i\) & \(|x_i - \bar{x}|\) \\
\midrule\noalign{}
\endhead
\bottomrule\noalign{}
\endlastfoot
4 & \(|4 - 4.1| = 0.1\) \\
6 & \(|6 - 4.1| = 1.9\) \\
2 & \(|2 - 4.1| = 2.1\) \\
4 & \(|4 - 4.1| = 0.1\) \\
5 & \(|5 - 4.1| = 0.9\) \\
4 & \(|4 - 4.1| = 0.1\) \\
4 & \(|4 - 4.1| = 0.1\) \\
5 & \(|5 - 4.1| = 0.9\) \\
3 & \(|3 - 4.1| = 1.1\) \\
4 & \(|4 - 4.1| = 0.1\) \\
Total & \\
\end{longtable}

Mean Deviation =
\(\frac{\sum |x_i - \bar{x}|}{n} = \frac{7.4}{10} = 0.74\)

\subsubsection{Q5.3 [3 marks]}\label{q5.3-3-marks}

\textbf{Find the standard deviation for the following discrete grouped
data:}

\begin{longtable}[]{@{}llllllll@{}}
\toprule\noalign{}
\(x_i\) & 4 & 8 & 11 & 17 & 20 & 24 & 32 \\
\midrule\noalign{}
\endhead
\bottomrule\noalign{}
\endlastfoot
\(f_i\) & 3 & 5 & 9 & 5 & 4 & 3 & 1 \\
\end{longtable}

\textbf{Solution}: First find the mean:

\begin{longtable}[]{@{}lll@{}}
\toprule\noalign{}
\(x_i\) & \(f_i\) & \(f_i x_i\) \\
\midrule\noalign{}
\endhead
\bottomrule\noalign{}
\endlastfoot
4 & 3 & 12 \\
8 & 5 & 40 \\
11 & 9 & 99 \\
17 & 5 & 85 \\
20 & 4 & 80 \\
24 & 3 & 72 \\
32 & 1 & 32 \\
Total & 30 & 420 \\
\end{longtable}

Mean = \(\frac{420}{30} = 14\)

Now calculate standard deviation:

\begin{longtable}[]{@{}
  >{\raggedright\arraybackslash}p{(\linewidth - 8\tabcolsep) * \real{0.0921}}
  >{\raggedright\arraybackslash}p{(\linewidth - 8\tabcolsep) * \real{0.0921}}
  >{\raggedright\arraybackslash}p{(\linewidth - 8\tabcolsep) * \real{0.2237}}
  >{\raggedright\arraybackslash}p{(\linewidth - 8\tabcolsep) * \real{0.2763}}
  >{\raggedright\arraybackslash}p{(\linewidth - 8\tabcolsep) * \real{0.3158}}@{}}
\toprule\noalign{}
\begin{minipage}[b]{\linewidth}\raggedright
\(x_i\)
\end{minipage} & \begin{minipage}[b]{\linewidth}\raggedright
\(f_i\)
\end{minipage} & \begin{minipage}[b]{\linewidth}\raggedright
\(x_i - \bar{x}\)
\end{minipage} & \begin{minipage}[b]{\linewidth}\raggedright
\((x_i - \bar{x})^2\)
\end{minipage} & \begin{minipage}[b]{\linewidth}\raggedright
\(f_i(x_i - \bar{x})^2\)
\end{minipage} \\
\midrule\noalign{}
\endhead
\bottomrule\noalign{}
\endlastfoot
4 & 3 & -10 & 100 & 300 \\
8 & 5 & -6 & 36 & 180 \\
11 & 9 & -3 & 9 & 81 \\
17 & 5 & 3 & 9 & 45 \\
20 & 4 & 6 & 36 & 144 \\
24 & 3 & 10 & 100 & 300 \\
32 & 1 & 18 & 324 & 324 \\
Total & 30 & - & - & 1374 \\
\end{longtable}

Standard Deviation =
\(\sqrt{\frac{\sum f_i(x_i - \bar{x})^2}{n}} = \sqrt{\frac{1374}{30}} = \sqrt{45.8} = 6.77\)

\subsection*{Q.5 (B) [8 marks]}\label{q.5-b-8-marks}

\textbf{Attempt any two}

\subsubsection{Q5.4 [4 marks]}\label{q5.4-4-marks}

\textbf{Solve:
\(\frac{dy}{dx} + \frac{4x}{1+x^2}y = \frac{1}{(1+x^2)^2}\)}

\textbf{Solution}: This is a linear differential equation of the form
\(\frac{dy}{dx} + Py = Q\)

Where \(P = \frac{4x}{1+x^2}\) and \(Q = \frac{1}{(1+x^2)^2}\)

Find integrating factor:
\(\text{I.F.} = e^{\int P dx} = e^{\int \frac{4x}{1+x^2} dx}\)

Let \(u = 1+x^2\), then \(du = 2x dx\)
\(\int \frac{4x}{1+x^2} dx = 2\int \frac{du}{u} = 2\ln|u| = 2\ln(1+x^2)\)

\(\text{I.F.} = e^{2\ln(1+x^2)} = (1+x^2)^2\)

The solution is:
\(y \cdot (1+x^2)^2 = \int \frac{1}{(1+x^2)^2} \cdot (1+x^2)^2 dx\)

\(y(1+x^2)^2 = \int 1 dx = x + C\)

\(y = \frac{x + C}{(1+x^2)^2}\)

\subsubsection{Q5.5 [4 marks]}\label{q5.5-4-marks}

\textbf{Solve: \((x + y + 1)^2 \frac{dy}{dx} = 1\)}

\textbf{Solution}: \((x + y + 1)^2 \frac{dy}{dx} = 1\)

\(\frac{dy}{dx} = \frac{1}{(x + y + 1)^2}\)

Let \(v = x + y + 1\), then \(\frac{dv}{dx} = 1 + \frac{dy}{dx}\)

So \(\frac{dy}{dx} = \frac{dv}{dx} - 1\)

Substituting: \(\frac{dv}{dx} - 1 = \frac{1}{v^2}\)

\(\frac{dv}{dx} = 1 + \frac{1}{v^2} = \frac{v^2 + 1}{v^2}\)

Separating variables: \(\frac{v^2}{v^2 + 1} dv = dx\)

\(\left(1 - \frac{1}{v^2 + 1}\right) dv = dx\)

Integrating both sides:
\(\int \left(1 - \frac{1}{v^2 + 1}\right) dv = \int dx\)

\(v - \arctan(v) = x + C\)

Substituting back \(v = x + y + 1\):
\((x + y + 1) - \arctan(x + y + 1) = x + C\)

\(y + 1 - \arctan(x + y + 1) = C\)

\(y = \arctan(x + y + 1) + C - 1\)

\subsubsection{Q5.6 [4 marks]}\label{q5.6-4-marks}

\textbf{Solve: \(\frac{dy}{dx} + y = e^x\), \(y(0) = 1\)}

\textbf{Solution}: This is a linear differential equation with \(P = 1\)
and \(Q = e^x\)

Integrating factor: \(\text{I.F.} = e^{\int 1 dx} = e^x\)

The solution is:
\(y \cdot e^x = \int e^x \cdot e^x dx = \int e^{2x} dx\)

\(ye^x = \frac{e^{2x}}{2} + C\)

\(y = \frac{e^x}{2} + Ce^{-x}\)

Using initial condition \(y(0) = 1\):
\(1 = \frac{e^0}{2} + Ce^0 = \frac{1}{2} + C\)

\(C = 1 - \frac{1}{2} = \frac{1}{2}\)

Therefore:
\(y = \frac{e^x}{2} + \frac{1}{2}e^{-x} = \frac{1}{2}(e^x + e^{-x})\)

\begin{center}\rule{0.5\linewidth}{0.5pt}\end{center}

\subsection*{Formula Cheat Sheet}\label{formula-cheat-sheet}

\subsubsection{\texorpdfstring{\textbf{Matrix
Operations}}{Matrix Operations}}\label{matrix-operations}

\begin{itemize}
\tightlist
\item
  \textbf{Matrix Multiplication}: \((AB)_{ij} = \sum_{k} A_{ik}B_{kj}\)
\item
  \textbf{Transpose}: \((A^T)_{ij} = A_{ji}\)
\item
  \textbf{Inverse}: \(A^{-1} = \frac{1}{|A|} \text{adj}(A)\)
\item
  \textbf{Determinant 2\times2}:
  \(\begin{vmatrix} a & b \\ c & d \end{vmatrix} = ad - bc\)
\end{itemize}

\subsubsection{\texorpdfstring{\textbf{Differentiation}}{Differentiation}}\label{differentiation}

\begin{itemize}
\tightlist
\item
  \textbf{Basic Rules}: \(\frac{d}{dx}(x^n) = nx^{n-1}\),
  \(\frac{d}{dx}(e^x) = e^x\), \(\frac{d}{dx}(\ln x) = \frac{1}{x}\)
\item
  \textbf{Chain Rule}: \(\frac{d}{dx}[f(g(x))] = f'(g(x)) \cdot g'(x)\)
\item
  \textbf{Product Rule}: \(\frac{d}{dx}[uv] = u'v + uv'\)
\item
  \textbf{Implicit Differentiation}: Differentiate both sides, treat
  \(y\) as function of \(x\)
\end{itemize}

\subsubsection{\texorpdfstring{\textbf{Integration}}{Integration}}\label{integration}

\begin{itemize}
\tightlist
\item
  \textbf{Basic Integrals}: \(\int x^n dx = \frac{x^{n+1}}{n+1} + C\) (n
  \neq -1)
\item
  \textbf{Integration by Parts}: \(\int u dv = uv - \int v du\)
\item
  \textbf{Definite Integral}: \(\int_a^b f(x) dx = F(b) - F(a)\)
\end{itemize}

\subsubsection{\texorpdfstring{\textbf{Differential
Equations}}{Differential Equations}}\label{differential-equations}

\begin{itemize}
\tightlist
\item
  \textbf{Linear DE}: \(\frac{dy}{dx} + Py = Q\), Solution:
  \(y \cdot \text{I.F.} = \int Q \cdot \text{I.F.} dx\)
\item
  \textbf{Integrating Factor}: \(\text{I.F.} = e^{\int P dx}\)
\item
  \textbf{Variable Separable}: \(\frac{dy}{dx} = f(x)g(y)\) \rightarrow
  \(\frac{dy}{g(y)} = f(x) dx\)
\end{itemize}

\subsubsection{\texorpdfstring{\textbf{Statistics}}{Statistics}}\label{statistics}

\begin{itemize}
\tightlist
\item
  \textbf{Mean}: \(\bar{x} = \frac{\sum f_i x_i}{\sum f_i}\)
\item
  \textbf{Mean Deviation}:
  \(\text{M.D.} = \frac{\sum |x_i - \bar{x}|}{n}\)
\item
  \textbf{Standard Deviation}:
  \(\sigma = \sqrt{\frac{\sum (x_i - \bar{x})^2}{n}}\)
\end{itemize}

\subsection*{Problem-Solving
Strategies}\label{problem-solving-strategies}

\subsubsection{\texorpdfstring{\textbf{Matrix
Problems}}{Matrix Problems}}\label{matrix-problems}

\begin{enumerate}
\tightlist
\item
  Check dimensions for multiplication compatibility
\item
  Use properties like \((AB)^T = B^T A^T\)
\item
  For inverse, find determinant first (must be non-zero)
\end{enumerate}

\subsubsection{\texorpdfstring{\textbf{Calculus
Problems}}{Calculus Problems}}\label{calculus-problems}

\begin{enumerate}
\tightlist
\item
  Identify the type of function before differentiating
\item
  Use appropriate rules (chain, product, quotient)
\item
  For integration, look for substitution opportunities
\item
  Check if integration by parts is needed
\end{enumerate}

\subsubsection{\texorpdfstring{\textbf{Differential
Equations}}{Differential Equations}}\label{differential-equations-1}

\begin{enumerate}
\tightlist
\item
  Identify the type (linear, separable, exact)
\item
  Find integrating factor for linear equations
\item
  Always check initial conditions
\end{enumerate}

\subsubsection{\texorpdfstring{\textbf{Statistics}}{Statistics}}\label{statistics-1}

\begin{enumerate}
\tightlist
\item
  Organize data in frequency tables
\item
  Use appropriate formulas for grouped/ungrouped data
\item
  Apply step deviation method for large numbers
\end{enumerate}

\subsection*{Common Mistakes to Avoid}\label{common-mistakes-to-avoid}

\begin{enumerate}
\tightlist
\item
  \textbf{Matrix multiplication}: Remember order matters, \(AB \neq BA\)
\item
  \textbf{Chain rule}: Don't forget to multiply by derivative of inner
  function
\item
  \textbf{Integration}: Always add constant of integration for
  indefinite integrals
\item
  \textbf{Differential equations}: Apply initial conditions to find
  particular solution
\item
  \textbf{Statistics}: Use correct formulas for grouped vs ungrouped
  data
\end{enumerate}

\subsection*{Exam Tips}\label{exam-tips}

\begin{enumerate}
\tightlist
\item
  \textbf{Time Management}: Allocate time based on marks (1 mark = 2
  minutes)
\item
  \textbf{Show Work}: Write all steps clearly for partial credit
\item
  \textbf{Check Units}: Ensure answers have appropriate units when
  applicable
\item
  \textbf{Verify}: Substitute back into original equation when possible
\item
  \textbf{Practice}: Focus on computational accuracy and speed
\end{enumerate}


\end{document}
