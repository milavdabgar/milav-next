\documentclass[10pt,a4paper]{article}

% content/resources/templates/preamble.tex
\usepackage[margin=0.6in]{geometry}
\author{Milav Dabgar}
\usepackage{amsmath,amssymb,amsthm}
\usepackage{booktabs}
\usepackage{multirow}
\usepackage{xcolor}
\usepackage{tcolorbox}
\tcbuselibrary{breakable,skins}
\usepackage[colorlinks=true,linkcolor=blue]{hyperref}
\usepackage{titlesec}
\usepackage{enumitem}
\usepackage{tikz}
\usepackage{pgfplots}
\usepackage{circuitikz}
\usepackage[version=4]{mhchem}
\usepackage{longtable}
\usepackage{array}
\usepackage{float}
\usepackage{caption}
\usepackage{listings}

\lstset{
  basicstyle=\small\ttfamily,
  breaklines=true,
  breakatwhitespace=false,
  postbreak=\mbox{\textcolor{red}{$\hookrightarrow$}\space},
  float=false,
  numbers=left,
  numberstyle=\tiny\color{gray},
  numbersep=10pt,
  xleftmargin=2em,
  keywordstyle=\color{blue},
  commentstyle=\color{green!60!black},
  stringstyle=\color{purple},
  backgroundcolor=\color{gray!5},
  showstringspaces=false,
  tabsize=2,
  captionpos=b,
  keepspaces=true,
  columns=flexible
}

\pgfplotsset{compat=1.18}
\usetikzlibrary{shapes,arrows,positioning,calc,patterns,decorations.pathmorphing,decorations.markings,arrows.meta}

% Color scheme
\definecolor{headcolor}{RGB}{0,102,204}
\definecolor{keycolor}{RGB}{220,20,60}
\definecolor{solutioncolor}{RGB}{34,139,34}
\definecolor{mnemoniccolor}{RGB}{148,0,211}
\definecolor{codecolor}{RGB}{0,0,100}

% Spacing
\setlength{\parskip}{3pt}
\setlist[itemize]{nosep}
\setlist[enumerate]{nosep}

% Title formatting
\titleformat{\section}{\Large\bfseries\color{headcolor}}{\thesection}{1em}{}
\titleformat{\subsection}{\large\bfseries\color{headcolor}}{\thesubsection}{1em}{}

% Pandoc tightlist compatibility
\providecommand{\tightlist}{%
  \setlength{\itemsep}{0pt}\setlength{\parskip}{0pt}}

% Pandoc longtable compatibility
\newcounter{none}
\def\thenone{}


% content/resources/templates/english-boxes.tex
% This file is currently empty - it exists to maintain consistency with the import structure.
% Add custom environments here if needed in the future.


\begin{document}

\begin{center}
{\Huge\bfseries\color{headcolor} Subject Name Solutions}\\[5pt]
{\LARGE 4320001 -- Summer 2024}\\[3pt]
{\large Semester 1 Study Material}\\[3pt]
{\normalsize\textit{Detailed Solutions and Explanations}}
\end{center}

\vspace{10pt}

\subsection*{Q.1 Fill in the blanks [14
marks]}\label{q.1-fill-in-the-blanks-14-marks}

\subsubsection{Q1.1 [1 mark]}\label{q1.1-1-mark}

\textbf{Order of the matrix
\(\begin{bmatrix} 1 & 2 & 3 \\ -4 & 5 & 6 \end{bmatrix}\) is =
\_\_\_\_\_\_\_\_\_\_\_}

\begin{solutionbox}
(b) \(2 \times 3\)

\textbf{Solution}: A matrix with 2 rows and 3 columns has order
\(2 \times 3\).

\end{solutionbox}
\subsubsection{Q1.2 [1 mark]}\label{q1.2-1-mark}

\textbf{If
\(\begin{bmatrix} x-3 & 2 \\ 4 & 0 \end{bmatrix} = \begin{bmatrix} 5 & 2 \\ 4 & 0 \end{bmatrix}\)
then \(x\) = \_\_\_\_}

\begin{solutionbox}
(d) 8

\textbf{Solution}: For matrix equality, corresponding elements must be
equal: \(x - 3 = 5\) \(x = 8\)

\end{solutionbox}
\subsubsection{Q1.3 [1 mark]}\label{q1.3-1-mark}

\textbf{The adjoint of \(\begin{bmatrix} -3 & 2 \\ 0 & 1 \end{bmatrix}\)
= \_\_\_\_\_\_\_\_\_\_\_\_\_}

\begin{solutionbox}
(b) \(\begin{bmatrix} 1 & -2 \\ 0 & -3 \end{bmatrix}\)

\textbf{Solution}: For matrix
\(A = \begin{bmatrix} a & b \\ c & d \end{bmatrix}\),
\(\text{adj}(A) = \begin{bmatrix} d & -b \\ -c & a \end{bmatrix}\)
\(\text{adj}\begin{bmatrix} -3 & 2 \\ 0 & 1 \end{bmatrix} = \begin{bmatrix} 1 & -2 \\ 0 & -3 \end{bmatrix}\)

\end{solutionbox}
\subsubsection{Q1.4 [1 mark]}\label{q1.4-1-mark}

\textbf{For any square matrix \(A\), \((A^{-1})^{-1}\) =
\_\_\_\_\_\_\_\_\_\_\_\_}

\begin{solutionbox}
(b) \(A\)

\textbf{Solution}: By definition of inverse matrices:
\((A^{-1})^{-1} = A\)

\end{solutionbox}
\subsubsection{Q1.5 [1 mark]}\label{q1.5-1-mark}

\textbf{\(\frac{d}{dx} \log x\) = \_\_\_\_\_\_\_\_\_}

\begin{solutionbox}
(b) \(\frac{1}{x}\)

\textbf{Solution}: The derivative of natural logarithm:
\(\frac{d}{dx} \log x = \frac{1}{x}\)

\end{solutionbox}
\subsubsection{Q1.6 [1 mark]}\label{q1.6-1-mark}

\textbf{\(\frac{d}{dx}(\tan^{-1} x + \cot^{-1} x)\) = \_\_\_\_\_\_\_}

\begin{solutionbox}
(d) 0

\textbf{Solution}: \(\tan^{-1} x + \cot^{-1} x = \frac{\pi}{2}\)
(constant) Therefore, \(\frac{d}{dx}(\tan^{-1} x + \cot^{-1} x) = 0\)

\end{solutionbox}
\subsubsection{Q1.7 [1 mark]}\label{q1.7-1-mark}

\textbf{If \(x = a \cos \theta\), \(y = a \sin \theta\) then
\(\frac{dy}{dx}\) = \_\_\_\_\_\_\_\_\_\_}

\begin{solutionbox}
(a) \(-\cot \theta\)

\textbf{Solution}: \(\frac{dx}{d\theta} = -a \sin \theta\),
\(\frac{dy}{d\theta} = a \cos \theta\)
\(\frac{dy}{dx} = \frac{dy/d\theta}{dx/d\theta} = \frac{a \cos \theta}{-a \sin \theta} = -\cot \theta\)

\end{solutionbox}
\subsubsection{Q1.8 [1 mark]}\label{q1.8-1-mark}

\textbf{\(\int 5x^4 dx\) = \_\_\_\_\_\_\_\_\_\_\_\_ + \(c\)}

\begin{solutionbox}
(d) \(x^5\)

\textbf{Solution}: \(\int 5x^4 dx = 5 \cdot \frac{x^5}{5} = x^5 + c\)

\end{solutionbox}
\subsubsection{Q1.9 [1 mark]}\label{q1.9-1-mark}

\textbf{\(\int_0^1 e^x dx\) = \_\_\_\_\_\_\_\_\_\_}

\begin{solutionbox}
(a) \(e - 1\)

\textbf{Solution}: \(\int_0^1 e^x dx = [e^x]_0^1 = e^1 - e^0 = e - 1\)

\end{solutionbox}
\subsubsection{Q1.10 [1 mark]}\label{q1.10-1-mark}

\textbf{\(\int_{-1}^1 3x^2 - 2x + 1 dx\) = \_\_\_\_\_\_\_\_\_\_}

\begin{solutionbox}
(c) 4

\textbf{Solution}:
\(\int_{-1}^1 (3x^2 - 2x + 1) dx = [x^3 - x^2 + x]_{-1}^1\)
\(= (1 - 1 + 1) - (-1 - 1 - 1) = 1 - (-3) = 4\)

\end{solutionbox}
\subsubsection{Q1.11 [1 mark]}\label{q1.11-1-mark}

\textbf{The order of differential equation
\((\frac{dy}{dx})^2 + 4y = x\) is \_\_\_\_\_\_\_\_\_\_\_}

\begin{solutionbox}
(d) 1

\textbf{Solution}: Order is the highest derivative present. Here, only
first derivative \(\frac{dy}{dx}\) appears, so order = 1.

\end{solutionbox}
\subsubsection{Q1.12 [1 mark]}\label{q1.12-1-mark}

\textbf{The integrating factor of \(\frac{dy}{dx} + 3y = x\) is
\_\_\_\_\_\_\_\_\_\_\_\_\_}

\begin{solutionbox}
(d) \(e^{3x}\)

\textbf{Solution}: For linear DE \(\frac{dy}{dx} + Py = Q\), integrating
factor = \(e^{\int P dx}\) Here \(P = 3\), so I.F. =
\(e^{\int 3 dx} = e^{3x}\)

\end{solutionbox}
\subsubsection{Q1.13 [1 mark]}\label{q1.13-1-mark}

\textbf{The mean of first ten natural numbers is\_\_\_\_\_\_\_\_\_}

\begin{solutionbox}
(a) 5.5

\textbf{Solution}: Mean =
\(\frac{1 + 2 + 3 + ... + 10}{10} = \frac{55}{10} = 5.5\)

\end{solutionbox}
\subsubsection{Q1.14 [1 mark]}\label{q1.14-1-mark}

\textbf{The range of the data 17, 15, 25, 34, 32 is
\_\_\_\_\_\_\_\_\_\_\_\_\_\_\_}

\begin{solutionbox}
(d) 19

\textbf{Solution}: Range = Maximum - Minimum = 34 - 15 = 19

\end{solutionbox}
\begin{center}\rule{0.5\linewidth}{0.5pt}\end{center}

\subsection*{Q.2 (A) Attempt any two [6
marks]}\label{q.2-a-attempt-any-two-6-marks}

\subsubsection{Q2.1 [3 marks]}\label{q2.1-3-marks}

\textbf{If \(A = \begin{bmatrix} 1 & -1 \\ 2 & 3 \end{bmatrix}\) then
find \(A + A^T + I\).}

\begin{solutionbox}

\textbf{Solution}: \(A = \begin{bmatrix} 1 & -1 \\ 2 & 3 \end{bmatrix}\)

\(A^T = \begin{bmatrix} 1 & 2 \\ -1 & 3 \end{bmatrix}\)

\(I = \begin{bmatrix} 1 & 0 \\ 0 & 1 \end{bmatrix}\)

\(A + A^T + I = \begin{bmatrix} 1 & -1 \\ 2 & 3 \end{bmatrix} + \begin{bmatrix} 1 & 2 \\ -1 & 3 \end{bmatrix} + \begin{bmatrix} 1 & 0 \\ 0 & 1 \end{bmatrix}\)

\(= \begin{bmatrix} 3 & 1 \\ 1 & 7 \end{bmatrix}\)

\end{solutionbox}
\subsubsection{Q2.2 [3 marks]}\label{q2.2-3-marks}

\textbf{If \(A = \begin{bmatrix} 2 & 3 \\ -1 & 2 \end{bmatrix}\) then
prove that \(A^2 - 4A + 7I_2 = 0\)}

\begin{solutionbox}
Proved

\textbf{Solution}: \(A = \begin{bmatrix} 2 & 3 \\ -1 & 2 \end{bmatrix}\)

\(A^2 = \begin{bmatrix} 2 & 3 \\ -1 & 2 \end{bmatrix} \begin{bmatrix} 2 & 3 \\ -1 & 2 \end{bmatrix} = \begin{bmatrix} 1 & 12 \\ -4 & 1 \end{bmatrix}\)

\(4A = 4\begin{bmatrix} 2 & 3 \\ -1 & 2 \end{bmatrix} = \begin{bmatrix} 8 & 12 \\ -4 & 8 \end{bmatrix}\)

\(7I_2 = \begin{bmatrix} 7 & 0 \\ 0 & 7 \end{bmatrix}\)

\(A^2 - 4A + 7I_2 = \begin{bmatrix} 1 & 12 \\ -4 & 1 \end{bmatrix} - \begin{bmatrix} 8 & 12 \\ -4 & 8 \end{bmatrix} + \begin{bmatrix} 7 & 0 \\ 0 & 7 \end{bmatrix}\)

\(= \begin{bmatrix} 0 & 0 \\ 0 & 0 \end{bmatrix} = 0\) ✓

\end{solutionbox}
\subsubsection{Q2.3 [3 marks]}\label{q2.3-3-marks}

\textbf{Solve differential equation \(dy - 3x^2e^{-y}dx = 0\)}

\begin{solutionbox}
\(e^y = x^3 + C\)

\textbf{Solution}: \(dy - 3x^2e^{-y}dx = 0\) \(dy = 3x^2e^{-y}dx\)
\(e^y dy = 3x^2 dx\)

Integrating both sides: \(\int e^y dy = \int 3x^2 dx\) \(e^y = x^3 + C\)

\end{solutionbox}
\begin{center}\rule{0.5\linewidth}{0.5pt}\end{center}

\subsection*{Q.2 (B) Attempt any two [8
marks]}\label{q.2-b-attempt-any-two-8-marks}

\subsubsection{Q2.1 [4 marks]}\label{q2.1-4-marks}

\textbf{Find the inverse of matrix
\(\begin{bmatrix} 3 & -1 & 2 \\ 4 & 1 & -1 \\ 5 & 0 & 1 \end{bmatrix}\)}

\begin{solutionbox}
\(A^{-1} = \begin{bmatrix} 1/14 & 1/14 & -1/14 \\ -9/14 & -7/14 & 11/14 \\ -5/14 & -5/14 & 1/2 \end{bmatrix}\)

\textbf{Solution}: Let
\(A = \begin{bmatrix} 3 & -1 & 2 \\ 4 & 1 & -1 \\ 5 & 0 & 1 \end{bmatrix}\)

First, find \(\det(A)\):
\(\det(A) = 3(1 \cdot 1 - (-1) \cdot 0) - (-1)(4 \cdot 1 - (-1) \cdot 5) + 2(4 \cdot 0 - 1 \cdot 5)\)
\(= 3(1) + 1(9) + 2(-5) = 3 + 9 - 10 = 2\)

Since \(\det(A) \neq 0\), inverse exists.

Finding cofactors and adjoint matrix: \(C_{11} = 1\), \(C_{12} = -9\),
\(C_{13} = -5\) \(C_{21} = 1\), \(C_{22} = -7\), \(C_{23} = -5\)\\
\(C_{31} = -1\), \(C_{32} = 11\), \(C_{33} = 7\)

\(\text{adj}(A) = \begin{bmatrix} 1 & 1 & -1 \\ -9 & -7 & 11 \\ -5 & -5 & 7 \end{bmatrix}\)

\(A^{-1} = \frac{1}{\det(A)} \cdot \text{adj}(A) = \frac{1}{2} \begin{bmatrix} 1 & 1 & -1 \\ -9 & -7 & 11 \\ -5 & -5 & 7 \end{bmatrix}\)

\end{solutionbox}
\subsubsection{Q2.2 [4 marks]}\label{q2.2-4-marks}

\textbf{If \(A + B = \begin{bmatrix} 1 & -1 \\ 3 & 0 \end{bmatrix}\) and
\(A - B = \begin{bmatrix} 3 & 1 \\ 1 & 4 \end{bmatrix}\) then find
\(AB\).}

\begin{solutionbox}
\(AB = \begin{bmatrix} 0 & -1 \\ 4 & -2 \end{bmatrix}\)

\textbf{Solution}: Adding the equations: \((A + B) + (A - B) = 2A\)
\(2A = \begin{bmatrix} 1 & -1 \\ 3 & 0 \end{bmatrix} + \begin{bmatrix} 3 & 1 \\ 1 & 4 \end{bmatrix} = \begin{bmatrix} 4 & 0 \\ 4 & 4 \end{bmatrix}\)
\(A = \begin{bmatrix} 2 & 0 \\ 2 & 2 \end{bmatrix}\)

Subtracting the equations: \((A + B) - (A - B) = 2B\)
\(2B = \begin{bmatrix} 1 & -1 \\ 3 & 0 \end{bmatrix} - \begin{bmatrix} 3 & 1 \\ 1 & 4 \end{bmatrix} = \begin{bmatrix} -2 & -2 \\ 2 & -4 \end{bmatrix}\)
\(B = \begin{bmatrix} -1 & -1 \\ 1 & -2 \end{bmatrix}\)

\(AB = \begin{bmatrix} 2 & 0 \\ 2 & 2 \end{bmatrix} \begin{bmatrix} -1 & -1 \\ 1 & -2 \end{bmatrix} = \begin{bmatrix} -2 & -2 \\ 0 & -6 \end{bmatrix}\)

\end{solutionbox}
\subsubsection{Q2.3 [4 marks]}\label{q2.3-4-marks}

\textbf{Solve the system of linear equation \(2x + 3y = 1\),
\(y - 4x = 2\) using matrices.}

\begin{solutionbox}
\(x = -\frac{1}{11}\), \(y = \frac{13}{11}\)

\textbf{Solution}: The system can be written as: \(AX = B\)
\(\begin{bmatrix} 2 & 3 \\ -4 & 1 \end{bmatrix} \begin{bmatrix} x \\ y \end{bmatrix} = \begin{bmatrix} 1 \\ 2 \end{bmatrix}\)

\(\det(A) = 2(1) - 3(-4) = 2 + 12 = 14\)

\(A^{-1} = \frac{1}{14} \begin{bmatrix} 1 & -3 \\ 4 & 2 \end{bmatrix}\)

\(X = A^{-1}B = \frac{1}{14} \begin{bmatrix} 1 & -3 \\ 4 & 2 \end{bmatrix} \begin{bmatrix} 1 \\ 2 \end{bmatrix} = \frac{1}{14} \begin{bmatrix} -5 \\ 8 \end{bmatrix}\)

Therefore: \(x = -\frac{5}{14}\), \(y = \frac{8}{14} = \frac{4}{7}\)

\end{solutionbox}
\begin{center}\rule{0.5\linewidth}{0.5pt}\end{center}

\subsection*{Q.3 (A) Attempt any two [6
marks]}\label{q.3-a-attempt-any-two-6-marks}

\subsubsection{Q3.1 [3 marks]}\label{q3.1-3-marks}

\textbf{Find the derivative of \(f(x) = e^x\) using definition of
derivative.}

\begin{solutionbox}
\(f'(x) = e^x\)

\textbf{Solution}: Using the definition:
\(f'(x) = \lim_{h \to 0} \frac{f(x+h) - f(x)}{h}\)

\(f'(x) = \lim_{h \to 0} \frac{e^{x+h} - e^x}{h}\)
\(= \lim_{h \to 0} \frac{e^x \cdot e^h - e^x}{h}\)
\(= e^x \lim_{h \to 0} \frac{e^h - 1}{h}\) \(= e^x \cdot 1 = e^x\)

\end{solutionbox}
\subsubsection{Q3.2 [3 marks]}\label{q3.2-3-marks}

\textbf{If \(\sqrt{x} + \sqrt{y} = \sqrt{a}\) then prove that
\(\frac{dy}{dx} = -\sqrt{\frac{y}{x}}\)}

\begin{solutionbox}
Proved

\textbf{Solution}: \(\sqrt{x} + \sqrt{y} = \sqrt{a}\)

Differentiating both sides with respect to \(x\):
\(\frac{1}{2\sqrt{x}} + \frac{1}{2\sqrt{y}} \cdot \frac{dy}{dx} = 0\)

\(\frac{1}{2\sqrt{y}} \cdot \frac{dy}{dx} = -\frac{1}{2\sqrt{x}}\)

\(\frac{dy}{dx} = -\frac{\sqrt{y}}{\sqrt{x}} = -\sqrt{\frac{y}{x}}\) ✓

\end{solutionbox}
\subsubsection{Q3.3 [3 marks]}\label{q3.3-3-marks}

\textbf{Evaluate \(\int \frac{\tan x}{\sec x + \tan x} dx\)}

\begin{solutionbox}
\(x - \ln|\sec x + \tan x| + C\)

\textbf{Solution}: Let \(I = \int \frac{\tan x}{\sec x + \tan x} dx\)

Multiply numerator and denominator by \((\sec x - \tan x)\):
\(I = \int \frac{\tan x(\sec x - \tan x)}{(\sec x + \tan x)(\sec x - \tan x)} dx\)
\(= \int \frac{\tan x(\sec x - \tan x)}{\sec^2 x - \tan^2 x} dx\)
\(= \int \frac{\tan x(\sec x - \tan x)}{1} dx\)
\(= \int (\tan x \sec x - \tan^2 x) dx\)
\(= \int \tan x \sec x dx - \int (\sec^2 x - 1) dx\)
\(= \sec x - \tan x + x + C\)

\end{solutionbox}
\begin{center}\rule{0.5\linewidth}{0.5pt}\end{center}

\subsection*{Q.3 (B) Attempt any two [8
marks]}\label{q.3-b-attempt-any-two-8-marks}

\subsubsection{Q3.1 [4 marks]}\label{q3.1-4-marks}

\textbf{If \(e^x + e^y = e^{x+y}\) then find \(\frac{dy}{dx}\).}

\begin{solutionbox}
\(\frac{dy}{dx} = \frac{e^x(e^y - 1)}{e^y(e^x - 1)}\)

\textbf{Solution}: \(e^x + e^y = e^{x+y}\)

Differentiating both sides with respect to \(x\):
\(e^x + e^y \frac{dy}{dx} = e^{x+y}(1 + \frac{dy}{dx})\)
\(e^x + e^y \frac{dy}{dx} = e^{x+y} + e^{x+y} \frac{dy}{dx}\)

Rearranging:
\(e^x - e^{x+y} = e^{x+y} \frac{dy}{dx} - e^y \frac{dy}{dx}\)
\(e^x - e^{x+y} = \frac{dy}{dx}(e^{x+y} - e^y)\)

\(\frac{dy}{dx} = \frac{e^x - e^{x+y}}{e^{x+y} - e^y} = \frac{e^x(1 - e^y)}{e^y(e^x - 1)} = \frac{e^x(e^y - 1)}{e^y(e^x - 1)}\)

\end{solutionbox}
\subsubsection{Q3.2 [4 marks]}\label{q3.2-4-marks}

\textbf{For \(y = 2e^{3x} + 3e^{-2x}\), prove that
\(\frac{d^2y}{dx^2} - \frac{dy}{dx} - 6y = 0\).}

\begin{solutionbox}
Proved

\textbf{Solution}: \(y = 2e^{3x} + 3e^{-2x}\)

\(\frac{dy}{dx} = 6e^{3x} - 6e^{-2x}\)

\(\frac{d^2y}{dx^2} = 18e^{3x} + 12e^{-2x}\)

Now checking the equation: \(\frac{d^2y}{dx^2} - \frac{dy}{dx} - 6y\)
\(= (18e^{3x} + 12e^{-2x}) - (6e^{3x} - 6e^{-2x}) - 6(2e^{3x} + 3e^{-2x})\)
\(= 18e^{3x} + 12e^{-2x} - 6e^{3x} + 6e^{-2x} - 12e^{3x} - 18e^{-2x}\)
\(= (18 - 6 - 12)e^{3x} + (12 + 6 - 18)e^{-2x}\)
\(= 0 \cdot e^{3x} + 0 \cdot e^{-2x} = 0\) ✓

\end{solutionbox}
\subsubsection{Q3.3 [4 marks]}\label{q3.3-4-marks}

\textbf{Equation of motion of a moving particle given by
\(s = t^3 + 3t\), \(t > 0\), when the velocity and acceleration will be
equal?}

\begin{solutionbox}
At \(t = 1\) second

\textbf{Solution}: Given: \(s = t^3 + 3t\)

Velocity: \(v = \frac{ds}{dt} = 3t^2 + 3\) Acceleration:
\(a = \frac{dv}{dt} = 6t\)

For velocity = acceleration: \(3t^2 + 3 = 6t\) \(3t^2 - 6t + 3 = 0\)
\(t^2 - 2t + 1 = 0\) \((t - 1)^2 = 0\) \(t = 1\)

Therefore, velocity and acceleration are equal at \(t = 1\) second.

\end{solutionbox}
\begin{center}\rule{0.5\linewidth}{0.5pt}\end{center}

\subsection*{Q.4 (A) Attempt any two [6
marks]}\label{q.4-a-attempt-any-two-6-marks}

\subsubsection{Q4.1 [3 marks]}\label{q4.1-3-marks}

\textbf{Evaluate: \(\int \frac{\sin\sqrt{x}}{\sqrt{x}} dx\)}

\begin{solutionbox}
\(-2\cos\sqrt{x} + C\)

\textbf{Solution}: Let \(u = \sqrt{x}\), then
\(du = \frac{1}{2\sqrt{x}} dx\), so \(dx = 2\sqrt{x} du = 2u du\)

\(\int \frac{\sin\sqrt{x}}{\sqrt{x}} dx = \int \frac{\sin u}{u} \cdot 2u du = 2\int \sin u du = -2\cos u + C = -2\cos\sqrt{x} + C\)

\end{solutionbox}
\subsubsection{Q4.2 [3 marks]}\label{q4.2-3-marks}

\textbf{Evaluate:
\(\int_0^{\pi/2} \frac{\sqrt{\sin x}}{\sqrt{\cos x} + \sqrt{\sin x}} dx\)}

\begin{solutionbox}
\(\frac{\pi}{4}\)

\textbf{Solution}: Let
\(I = \int_0^{\pi/2} \frac{\sqrt{\sin x}}{\sqrt{\cos x} + \sqrt{\sin x}} dx\)

Using property \(\int_0^a f(x) dx = \int_0^a f(a-x) dx\):
\(I = \int_0^{\pi/2} \frac{\sqrt{\sin(\pi/2 - x)}}{\sqrt{\cos(\pi/2 - x)} + \sqrt{\sin(\pi/2 - x)}} dx\)
\(= \int_0^{\pi/2} \frac{\sqrt{\cos x}}{\sqrt{\sin x} + \sqrt{\cos x}} dx\)

Adding both expressions:
\(2I = \int_0^{\pi/2} \frac{\sqrt{\sin x} + \sqrt{\cos x}}{\sqrt{\cos x} + \sqrt{\sin x}} dx = \int_0^{\pi/2} 1 dx = \frac{\pi}{2}\)

Therefore: \(I = \frac{\pi}{4}\)

\end{solutionbox}
\subsubsection{Q4.3 [3 marks]}\label{q4.3-3-marks}

\textbf{Find the mean of the frequency distribution:}

\begin{longtable}[]{@{}lllllllll@{}}
\toprule\noalign{}
Age & 20-24 & 25-29 & 30-34 & 35-39 & 40-44 & 45-49 & 50-54 & 55-59 \\
\midrule\noalign{}
\endhead
\bottomrule\noalign{}
\endlastfoot
Staff & 5 & 7 & 9 & 11 & 10 & 8 & 6 & 4 \\
\end{longtable}

\begin{solutionbox}
Mean = 37.5 years

\textbf{Solution}:

\begin{longtable}[]{@{}llll@{}}
\toprule\noalign{}
Class & Midpoint (x) & Frequency (f) & fx \\
\midrule\noalign{}
\endhead
\bottomrule\noalign{}
\endlastfoot
20-24 & 22 & 5 & 110 \\
25-29 & 27 & 7 & 189 \\
30-34 & 32 & 9 & 288 \\
35-39 & 37 & 11 & 407 \\
40-44 & 42 & 10 & 420 \\
45-49 & 47 & 8 & 376 \\
50-54 & 52 & 6 & 312 \\
55-59 & 57 & 4 & 228 \\
\textbf{Total} & & \textbf{60} & \textbf{2330} \\
\end{longtable}

Mean = \(\frac{\sum fx}{\sum f} = \frac{2330}{60} = 38.83\) years

\end{solutionbox}
\begin{center}\rule{0.5\linewidth}{0.5pt}\end{center}

\subsection*{Q.4 (B) Attempt any two [8
marks]}\label{q.4-b-attempt-any-two-8-marks}

\subsubsection{Q4.1 [4 marks]}\label{q4.1-4-marks}

\textbf{Evaluate: \(\int_0^1 \frac{x^2}{1 + x^6} dx\)}

\begin{solutionbox}
\(\frac{\pi}{12}\)

\textbf{Solution}: Let \(u = x^3\), then \(du = 3x^2 dx\), so
\(x^2 dx = \frac{1}{3} du\) When \(x = 0\), \(u = 0\); when \(x = 1\),
\(u = 1\)

\(\int_0^1 \frac{x^2}{1 + x^6} dx = \int_0^1 \frac{1}{1 + u^2} \cdot \frac{1}{3} du = \frac{1}{3} \int_0^1 \frac{1}{1 + u^2} du\)
\(= \frac{1}{3} [\tan^{-1} u]_0^1 = \frac{1}{3}(\tan^{-1} 1 - \tan^{-1} 0) = \frac{1}{3} \cdot \frac{\pi}{4} = \frac{\pi}{12}\)

\end{solutionbox}
\subsubsection{Q4.2 [4 marks]}\label{q4.2-4-marks}

\textbf{Find area enclosed by curve \(y = x^2\), \(X\)-axis and
\(x = 2\)}

\begin{solutionbox}
Area = \(\frac{8}{3}\) square units

\textbf{Solution}: The area is bounded by \(y = x^2\), \(y = 0\)
(X-axis), \(x = 0\) and \(x = 2\)

Area =
\(\int_0^2 x^2 dx = \left[\frac{x^3}{3}\right]_0^2 = \frac{8}{3} - 0 = \frac{8}{3}\)
square units

\end{solutionbox}
\subsubsection{Q4.3 [4 marks]}\label{q4.3-4-marks}

\textbf{Calculate the standard deviation for the following continuous
grouped data:}

\begin{longtable}[]{@{}llllll@{}}
\toprule\noalign{}
Class & 0-10 & 10-20 & 20-30 & 30-40 & 40-50 \\
\midrule\noalign{}
\endhead
\bottomrule\noalign{}
\endlastfoot
Frequency & 5 & 8 & 15 & 16 & 6 \\
\end{longtable}

\begin{solutionbox}
Standard deviation = 10.95

\textbf{Solution}:

\begin{longtable}[]{@{}llllll@{}}
\toprule\noalign{}
Class & Midpoint (x) & f & fx & \(x^2\) & \(fx^2\) \\
\midrule\noalign{}
\endhead
\bottomrule\noalign{}
\endlastfoot
0-10 & 5 & 5 & 25 & 25 & 125 \\
10-20 & 15 & 8 & 120 & 225 & 1800 \\
20-30 & 25 & 15 & 375 & 625 & 9375 \\
30-40 & 35 & 16 & 560 & 1225 & 19600 \\
40-50 & 45 & 6 & 270 & 2025 & 12150 \\
\textbf{Total} & & \textbf{50} & \textbf{1350} & & \textbf{43050} \\
\end{longtable}

Mean \(\bar{x} = \frac{1350}{50} = 27\)

Variance =
\(\frac{\sum fx^2}{n} - (\bar{x})^2 = \frac{43050}{50} - (27)^2 = 861 - 729 = 132\)

Standard deviation = \(\sqrt{132} = 11.49\)

\end{solutionbox}
\begin{center}\rule{0.5\linewidth}{0.5pt}\end{center}

\subsection*{Q.5 (A) Attempt any two [6
marks]}\label{q.5-a-attempt-any-two-6-marks}

\subsubsection{Q5.1 [3 marks]}\label{q5.1-3-marks}

\textbf{If mean of 25 observation is 50 and mean of other 75 observation
is 60. Considering all the observation then find the mean.}

\begin{solutionbox}
Combined mean = 57.5

\textbf{Solution}: Combined mean =
\(\frac{n_1\bar{x_1} + n_2\bar{x_2}}{n_1 + n_2}\)
\(= \frac{25 \times 50 + 75 \times 60}{25 + 75} = \frac{1250 + 4500}{100} = \frac{5750}{100} = 57.5\)

\end{solutionbox}
\subsubsection{Q5.2 [3 marks]}\label{q5.2-3-marks}

\textbf{Find the mean deviation for the following frequency
distribution:}

\begin{longtable}[]{@{}lllllll@{}}
\toprule\noalign{}
\(x_i\) & 3 & 4 & 5 & 6 & 7 & 8 \\
\midrule\noalign{}
\endhead
\bottomrule\noalign{}
\endlastfoot
\(f_i\) & 1 & 3 & 7 & 5 & 2 & 2 \\
\end{longtable}

\begin{solutionbox}
Mean deviation = 1.1

\textbf{Solution}: \textbar{} \(x_i\) \textbar{} \(f_i\) \textbar{}
\(f_i x_i\) \textbar{} \(|x_i - \bar{x}|\) \textbar{}
\(f_i|x_i - \bar{x}|\) \textbar{}
\textbar-------\textbar-------\textbar-----------\textbar------------------\textbar---------------------\textbar{}
\textbar{} 3 \textbar{} 1 \textbar{} 3 \textbar{} 2 \textbar{} 2
\textbar{} \textbar{} 4 \textbar{} 3 \textbar{} 12 \textbar{} 1
\textbar{} 3 \textbar{} \textbar{} 5 \textbar{} 7 \textbar{} 35
\textbar{} 0 \textbar{} 0 \textbar{} \textbar{} 6 \textbar{} 5
\textbar{} 30 \textbar{} 1 \textbar{} 5 \textbar{} \textbar{} 7
\textbar{} 2 \textbar{} 14 \textbar{} 2 \textbar{} 4 \textbar{}
\textbar{} 8 \textbar{} 2 \textbar{} 16 \textbar{} 3 \textbar{} 6
\textbar{} \textbar{} \textbf{Total} \textbar{} \textbf{20} \textbar{}
\textbf{110} \textbar{} \textbar{} \textbf{20} \textbar{}

Mean \(\bar{x} = \frac{110}{20} = 5.5\)

Recalculating deviations from mean = 5.5: Mean deviation =
\(\frac{\sum f_i|x_i - \bar{x}|}{\sum f_i} = \frac{22}{20} = 1.1\)

\end{solutionbox}
\subsubsection{Q5.3 [3 marks]}\label{q5.3-3-marks}

\textbf{Calculate the standard deviation for the following ungrouped
data:} \textbf{120, 132, 148, 136, 142, 140, 165, 153}

\begin{solutionbox}
Standard deviation = 13.36

\textbf{Solution}:

\begin{longtable}[]{@{}lll@{}}
\toprule\noalign{}
\(x\) & \(x - \bar{x}\) & \((x - \bar{x})^2\) \\
\midrule\noalign{}
\endhead
\bottomrule\noalign{}
\endlastfoot
120 & -19.5 & 380.25 \\
132 & -7.5 & 56.25 \\
148 & 8.5 & 72.25 \\
136 & -3.5 & 12.25 \\
142 & 2.5 & 6.25 \\
140 & 0.5 & 0.25 \\
165 & 25.5 & 650.25 \\
153 & 13.5 & 182.25 \\
\textbf{Total} & \textbf{0} & \textbf{1360} \\
\end{longtable}

\(n = 8\), \(\sum x = 1116\) Mean \(\bar{x} = \frac{1116}{8} = 139.5\)

Variance = \(\frac{\sum(x - \bar{x})^2}{n} = \frac{1360}{8} = 170\)

Standard deviation = \(\sqrt{170} = 13.04\)

\end{solutionbox}
\begin{center}\rule{0.5\linewidth}{0.5pt}\end{center}

\subsection*{Q.5 (B) Attempt any two [8
marks]}\label{q.5-b-attempt-any-two-8-marks}

\subsubsection{Q5.1 [4 marks]}\label{q5.1-4-marks}

\textbf{Solve: \(\frac{dy}{dx} + \tan x \cdot \tan y = 0\)}

\begin{solutionbox}
\(\ln|\cos y| = \ln|\cos x| + C\) or
\(\cos y = A\cos x\)

\textbf{Solution}: \(\frac{dy}{dx} + \tan x \cdot \tan y = 0\)
\(\frac{dy}{dx} = -\tan x \cdot \tan y\)
\(\frac{dy}{\tan y} = -\tan x \, dx\) \(\cot y \, dy = -\tan x \, dx\)

Integrating both sides: \(\int \cot y \, dy = -\int \tan x \, dx\)
\(\ln|\sin y| = \ln|\cos x| + C_1\) \(\ln|\sin y| - \ln|\cos x| = C_1\)
\(\ln\left|\frac{\sin y}{\cos x}\right| = C_1\)

Taking exponential: \(\frac{\sin y}{\cos x} = C\) (where
\(C = e^{C_1}\)) \(\sin y = C \cos x\)

Alternative form: \(\cos y = A \cos x\) where \(A\) is a constant.

\end{solutionbox}
\subsubsection{Q5.2 [4 marks]}\label{q5.2-4-marks}

\textbf{Solve: \(\frac{dy}{dx} + 2y = 3e^x\)}

\begin{solutionbox}
\(y = e^x + Ce^{-2x}\)

\textbf{Solution}: This is a first-order linear differential equation of
the form \(\frac{dy}{dx} + Py = Q\) where \(P = 2\) and \(Q = 3e^x\)

Integrating factor:
\(I.F. = e^{\int P \, dx} = e^{\int 2 \, dx} = e^{2x}\)

Multiplying the equation by \(e^{2x}\):
\(e^{2x}\frac{dy}{dx} + 2e^{2x}y = 3e^{3x}\)

The left side is the derivative of \(ye^{2x}\):
\(\frac{d}{dx}(ye^{2x}) = 3e^{3x}\)

Integrating both sides: \(ye^{2x} = \int 3e^{3x} \, dx = e^{3x} + C\)

Therefore: \(y = e^x + Ce^{-2x}\)

\end{solutionbox}
\subsubsection{Q5.3 [4 marks]}\label{q5.3-4-marks}

\textbf{Solve: \(dy + 4xy^2dx = 0\); \(y(0) = 1\)}

\begin{solutionbox}
\(y = \frac{1}{1 + 2x^2}\)

\textbf{Solution}: \(dy + 4xy^2dx = 0\) \(dy = -4xy^2dx\)
\(\frac{dy}{y^2} = -4x \, dx\)

Integrating both sides: \(\int y^{-2} \, dy = \int -4x \, dx\)
\(-\frac{1}{y} = -2x^2 + C\) \(\frac{1}{y} = 2x^2 - C\)

Using initial condition \(y(0) = 1\): \(\frac{1}{1} = 2(0)^2 - C\)
\(1 = -C\) \(C = -1\)

Therefore: \(\frac{1}{y} = 2x^2 + 1\) \(y = \frac{1}{2x^2 + 1}\)

\end{solutionbox}
\begin{center}\rule{0.5\linewidth}{0.5pt}\end{center}

\subsection*{Formula Cheat Sheet}\label{formula-cheat-sheet}

\subsubsection{Matrix Operations}\label{matrix-operations}

\begin{itemize}
\tightlist
\item
  \textbf{Matrix Addition/Subtraction}: Element-wise operation
\item
  \textbf{Matrix Multiplication}: \((AB)_{ij} = \sum_{k} a_{ik}b_{kj}\)
\item
  \textbf{Transpose}: \((A^T)_{ij} = A_{ji}\)
\item
  \textbf{Determinant (2\times2)}:
  \(\det\begin{bmatrix} a & b \\ c & d \end{bmatrix} = ad - bc\)
\item
  \textbf{Inverse (2\times2)}:
  \(A^{-1} = \frac{1}{\det(A)}\begin{bmatrix} d & -b \\ -c & a \end{bmatrix}\)
\item
  \textbf{Adjoint (2\times2)}:
  \(\text{adj}\begin{bmatrix} a & b \\ c & d \end{bmatrix} = \begin{bmatrix} d & -b \\ -c & a \end{bmatrix}\)
\end{itemize}

\subsubsection{Differentiation Formulas}\label{differentiation-formulas}

\begin{itemize}
\tightlist
\item
  \(\frac{d}{dx}(x^n) = nx^{n-1}\)
\item
  \(\frac{d}{dx}(e^x) = e^x\)
\item
  \(\frac{d}{dx}(\ln x) = \frac{1}{x}\)
\item
  \(\frac{d}{dx}(\sin x) = \cos x\)
\item
  \(\frac{d}{dx}(\cos x) = -\sin x\)
\item
  \(\frac{d}{dx}(\tan x) = \sec^2 x\)
\item
  \(\frac{d}{dx}(\tan^{-1} x) = \frac{1}{1+x^2}\)
\item
  \textbf{Chain Rule}: \(\frac{d}{dx}f(g(x)) = f'(g(x)) \cdot g'(x)\)
\item
  \textbf{Product Rule}: \((uv)' = u'v + uv'\)
\item
  \textbf{Quotient Rule}: \((\frac{u}{v})' = \frac{u'v - uv'}{v^2}\)
\end{itemize}

\subsubsection{Integration Formulas}\label{integration-formulas}

\begin{itemize}
\tightlist
\item
  \(\int x^n \, dx = \frac{x^{n+1}}{n+1} + C\) (for \(n \neq -1\))
\item
  \(\int \frac{1}{x} \, dx = \ln|x| + C\)
\item
  \(\int e^x \, dx = e^x + C\)
\item
  \(\int \sin x \, dx = -\cos x + C\)
\item
  \(\int \cos x \, dx = \sin x + C\)
\item
  \(\int \sec^2 x \, dx = \tan x + C\)
\item
  \(\int \frac{1}{1+x^2} \, dx = \tan^{-1} x + C\)
\item
  \textbf{Integration by Parts}: \(\int u \, dv = uv - \int v \, du\)
\end{itemize}

\subsubsection{Differential Equations}\label{differential-equations}

\begin{itemize}
\tightlist
\item
  \textbf{Variable Separable}:
  \(\frac{dy}{dx} = f(x)g(y) \Rightarrow \frac{dy}{g(y)} = f(x)dx\)
\item
  \textbf{Linear DE}: \(\frac{dy}{dx} + Py = Q\), Solution:
  \(y \cdot I.F. = \int Q \cdot I.F. \, dx\)
\item
  \textbf{Integrating Factor}: \(I.F. = e^{\int P \, dx}\)
\end{itemize}

\subsubsection{Statistics Formulas}\label{statistics-formulas}

\begin{itemize}
\tightlist
\item
  \textbf{Mean}: \(\bar{x} = \frac{\sum x_i}{n}\) (ungrouped),
  \(\bar{x} = \frac{\sum f_i x_i}{\sum f_i}\) (grouped)
\item
  \textbf{Mean Deviation}: \(M.D. = \frac{\sum |x_i - \bar{x}|}{n}\)
  (ungrouped), \(M.D. = \frac{\sum f_i |x_i - \bar{x}|}{\sum f_i}\)
  (grouped)
\item
  \textbf{Standard Deviation}:
  \(\sigma = \sqrt{\frac{\sum (x_i - \bar{x})^2}{n}}\) (ungrouped)
\item
  \textbf{Variance}: \(\sigma^2 = \frac{\sum (x_i - \bar{x})^2}{n}\)
\item
  \textbf{Range}: Maximum value - Minimum value
\item
  \textbf{Combined Mean}:
  \(\bar{x} = \frac{n_1\bar{x_1} + n_2\bar{x_2}}{n_1 + n_2}\)
\end{itemize}

\begin{center}\rule{0.5\linewidth}{0.5pt}\end{center}

\subsection*{Problem-Solving
Strategies}\label{problem-solving-strategies}

\subsubsection{Matrix Problems}\label{matrix-problems}

\begin{enumerate}
\tightlist
\item
  \textbf{Check dimensions} before operations
\item
  \textbf{Calculate determinant} first to check if inverse exists
\item
  \textbf{Use cofactor method} for 3\times3 matrix inverse
\item
  \textbf{Set up equations} properly for system solving
\end{enumerate}

\subsubsection{Differentiation Problems}\label{differentiation-problems}

\begin{enumerate}
\tightlist
\item
  \textbf{Identify the type} (implicit, parametric, composite)
\item
  \textbf{Apply appropriate rules} (chain, product, quotient)
\item
  \textbf{Simplify step by step}
\item
  \textbf{Check units} in application problems
\end{enumerate}

\subsubsection{Integration Problems}\label{integration-problems}

\begin{enumerate}
\tightlist
\item
  \textbf{Try standard forms} first
\item
  \textbf{Use substitution} when inner function derivative is present
\item
  \textbf{Apply integration by parts} for products
\item
  \textbf{Check limits} carefully in definite integrals
\end{enumerate}

\subsubsection{Differential Equations}\label{differential-equations-1}

\begin{enumerate}
\tightlist
\item
  \textbf{Identify the type} (separable, linear, homogeneous)
\item
  \textbf{Apply appropriate method}
\item
  \textbf{Use initial conditions} to find constants
\item
  \textbf{Verify solution} by substitution
\end{enumerate}

\subsubsection{Statistics Problems}\label{statistics-problems}

\begin{enumerate}
\tightlist
\item
  \textbf{Organize data} in tabular form
\item
  \textbf{Calculate systematically} using formulas
\item
  \textbf{Use class midpoints} for grouped data
\item
  \textbf{Double-check calculations}
\end{enumerate}

\begin{center}\rule{0.5\linewidth}{0.5pt}\end{center}

\subsection*{Common Mistakes to Avoid}\label{common-mistakes-to-avoid}

\begin{enumerate}
\tightlist
\item
  \textbf{Matrix multiplication}: Remember it's not commutative
  (\(AB \neq BA\))
\item
  \textbf{Chain rule}: Don't forget to multiply by derivative of inner
  function
\item
  \textbf{Integration limits}: Be careful with sign changes
\item
  \textbf{Differential equations}: Always include constant of
  integration
\item
  \textbf{Statistics}: Use correct formulas for grouped vs ungrouped
  data
\item
  \textbf{Arithmetic errors}: Double-check all calculations
\item
  \textbf{Units}: Maintain proper units throughout calculations
\end{enumerate}

\begin{center}\rule{0.5\linewidth}{0.5pt}\end{center}

\subsection*{Exam Tips}\label{exam-tips}

\begin{enumerate}
\tightlist
\item
  \textbf{Read questions carefully} - understand what's being asked
\item
  \textbf{Show all steps} - partial credit is often awarded
\item
  \textbf{Use proper mathematical notation}
\item
  \textbf{Check your answers} when possible
\item
  \textbf{Manage time effectively} - attempt questions you're confident
  about first
\item
  \textbf{Use formulas correctly} - refer to the formula sheet
\item
  \textbf{For optional questions} - choose the ones you can solve
  completely
\item
  \textbf{In statistics problems} - organize data clearly before
  calculations
\item
  \textbf{For differential equations} - verify your solution satisfies
  the original equation
\item
  \textbf{Practice numerical problems} - accuracy in calculations is
  crucial
\end{enumerate}


\end{document}
