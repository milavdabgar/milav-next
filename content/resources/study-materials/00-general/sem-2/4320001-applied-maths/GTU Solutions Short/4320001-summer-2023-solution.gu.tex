\documentclass{article}

% content/resources/templates/preamble.tex
\usepackage[margin=0.6in]{geometry}
\author{Milav Dabgar}
\usepackage{amsmath,amssymb,amsthm}
\usepackage{booktabs}
\usepackage{multirow}
\usepackage{xcolor}
\usepackage{tcolorbox}
\tcbuselibrary{breakable,skins}
\usepackage[colorlinks=true,linkcolor=blue]{hyperref}
\usepackage{titlesec}
\usepackage{enumitem}
\usepackage{tikz}
\usepackage{pgfplots}
\usepackage{circuitikz}
\usepackage[version=4]{mhchem}
\usepackage{longtable}
\usepackage{array}
\usepackage{float}
\usepackage{caption}
\usepackage{listings}

\lstset{
  basicstyle=\small\ttfamily,
  breaklines=true,
  breakatwhitespace=false,
  postbreak=\mbox{\textcolor{red}{$\hookrightarrow$}\space},
  float=false,
  numbers=left,
  numberstyle=\tiny\color{gray},
  numbersep=10pt,
  xleftmargin=2em,
  keywordstyle=\color{blue},
  commentstyle=\color{green!60!black},
  stringstyle=\color{purple},
  backgroundcolor=\color{gray!5},
  showstringspaces=false,
  tabsize=2,
  captionpos=b,
  keepspaces=true,
  columns=flexible
}

\pgfplotsset{compat=1.18}
\usetikzlibrary{shapes,arrows,positioning,calc,patterns,decorations.pathmorphing,decorations.markings,arrows.meta}

% Color scheme
\definecolor{headcolor}{RGB}{0,102,204}
\definecolor{keycolor}{RGB}{220,20,60}
\definecolor{solutioncolor}{RGB}{34,139,34}
\definecolor{mnemoniccolor}{RGB}{148,0,211}
\definecolor{codecolor}{RGB}{0,0,100}

% Spacing
\setlength{\parskip}{3pt}
\setlist[itemize]{nosep}
\setlist[enumerate]{nosep}

% Title formatting
\titleformat{\section}{\Large\bfseries\color{headcolor}}{\thesection}{1em}{}
\titleformat{\subsection}{\large\bfseries\color{headcolor}}{\thesubsection}{1em}{}

% Pandoc tightlist compatibility
\providecommand{\tightlist}{%
  \setlength{\itemsep}{0pt}\setlength{\parskip}{0pt}}

% Pandoc longtable compatibility
\newcounter{none}
\def\thenone{}


% content/resources/templates/gujarati-boxes.tex
\usepackage{fontspec}
\usepackage{polyglossia}

% Set Gujarati as main language (document is primarily in Gujarati)
% Note: gloss-gujarati.ldf doesn't exist in polyglossia, but it will use hyphenation patterns
\setdefaultlanguage{gujarati}
\setotherlanguage{english}

% Configure Gujarati font properly
% Use Language=Default to prevent polyglossia from trying to add language-specific features
% that don't exist for Gujarati, which causes "empty feature" warnings
\newfontfamily\gujaratifont[Script=Gujarati,AutoFakeBold=2.5,AutoFakeSlant=0.3]{Noto Sans Gujarati}
\setmainfont[Script=Gujarati,AutoFakeBold=2.5,AutoFakeSlant=0.3]{Noto Sans Gujarati}
% Use Noto Sans Gujarati for monospace to support Gujarati in text
\setmonofont[Scale=0.9]{Noto Sans Gujarati}

% Configure English to use the same font
\newfontfamily\englishfont[Script=Gujarati,AutoFakeBold=2.5,AutoFakeSlant=0.3]{Noto Sans Gujarati}

% Translations for polyglossia
\gappto\captionsgujarati{
  \renewcommand{\tablename}{કોષ્ટક}
  \renewcommand{\figurename}{આકૃતિ}
}

% Helper for TikZ nodes to ensure Gujarati font
\newcommand{\gu}[1]{{\gujaratifont #1}}

% Custom environments
\newtcolorbox{solutionbox}{
    breakable,
    enhanced,
    colback=solutioncolor!5!white,
    colframe=solutioncolor!75!black,
    fonttitle=\bfseries,
    title=જવાબ
}

\newtcolorbox{solutionboxnobreak}{
 colback=solutioncolor!5!white,
 colframe=solutioncolor!75!black,
 fonttitle=\bfseries,
 title=જવાબ
}

\newtcolorbox{keyformula}{
 breakable,
 enhanced,
 colback=keycolor!5!white,
 colframe=keycolor!75!black,
 fonttitle=\bfseries,
 title=રાસાયણિક સમીકરણ/સૂત્ર
}

\newtcolorbox{mnemonicbox}{
 breakable,
 enhanced,
 colback=mnemoniccolor!5!white,
 colframe=mnemoniccolor!75!black,
 fonttitle=\bfseries,
 title=મેમરી ટ્રીક
}


% Custom commands for GTU solutions
% This file defines semantic commands for consistent formatting

% Question command with automatic formatting
\newcommand{\question}[2]{%
  \section*{Question #1}%
  \textbf{#2}%
}

% OR question variant
\newcommand{\questionor}[2]{%
  \section*{Question #1 OR}%
  \textbf{#2}%
}

% Proper table environment with caption
\newenvironment{answertable}[1]{%
  \begin{table}[htbp]
  \centering
  \caption{#1}
}{%
  \end{table}
}

% Proper figure environment for diagrams
\newenvironment{answerdiagram}[1]{%
  \begin{figure}[htbp]
  \centering
  \caption{#1}
}{%
  \end{figure}
}

% Semantic markup for key terms
\newcommand{\keyword}[1]{\textbf{#1}}
\newcommand{\code}[1]{\texttt{#1}}
\newcommand{\classname}[1]{\texttt{#1}}
\newcommand{\methodname}[1]{\texttt{#1}}

% Proper quotation marks
\newcommand{\mnemonic}[1]{``#1''}


\title{Applied Mathematics (4320001) - Summer 2023 Solution}
\date{August 02, 2023}

\begin{document}
\maketitle

\questionmarks{1}{14}{નીચેના વિકલ્પોમાંથી યોગ્ય વિકલ્પ પસંદ કરી ખાલી જગ્યા પૂરો}

\questionmarks{1.1}{1}{જો $A = \begin{bmatrix} 1 & 2 \\ 3 & 1 \\ 4 & 2 \end{bmatrix}$, તો $A^T = $ \_\_\_\_\_\_\_\_}
\textbf{જવાબ}: b. $\begin{bmatrix} 1 & 3 & 4 \\ 2 & 1 & 2 \end{bmatrix}$

\begin{solutionbox}
શ્રેણિકના પરિવર્ત માટે, હાર એ સ્તંભ અને સ્તંભ એ હાર બને છે.
$A^T = \begin{bmatrix} 1 & 3 & 4 \\ 2 & 1 & 2 \end{bmatrix}$
\end{solutionbox}

\questionmarks{1.2}{1}{જો $\begin{bmatrix} x+y & 3 \\ -7 & x-y \end{bmatrix} = \begin{bmatrix} 8 & 3 \\ -7 & 2 \end{bmatrix}$, તો $(x,y) = $ \_\_\_\_\_\_\_\_}
\textbf{જવાબ}: c. $(5,3)$

\begin{solutionbox}
અનુરૂપ ઘટકોની સરખામણી કરતા:
$x + y = 8$ ... (1)
$x - y = 2$ ... (2)

સમીકરણ (1) અને (2) નો સરવાળો કરતા: $2x = 10 \implies x = 5$
સમીકરણ (1) માં કિંમત મૂકતા: $5 + y = 8 \implies y = 3$
\end{solutionbox}

\questionmarks{1.3}{1}{જો $\begin{bmatrix} x & 3 \\ y & 2 \end{bmatrix} \begin{bmatrix} 2 \\ 3 \end{bmatrix} = \begin{bmatrix} 15 \\ 12 \end{bmatrix}$, તો $y = $ \_\_\_\_\_\_\_\_}
\textbf{જવાબ}: c. 3

\begin{solutionbox}
શ્રેણિક ગુણાકાર કરતા:
$2x + 9 = 15 \implies 2x = 6 \implies x = 3$
$2y + 6 = 12 \implies 2y = 6 \implies y = 3$
\end{solutionbox}

\questionmarks{1.4}{1}{શ્રેણિક $\begin{bmatrix} 1 & -3 \\ -2 & 1 \\ 4 & 5 \end{bmatrix}$ ની કક્ષા \_\_\_\_\_\_\_\_ છે}
\textbf{જવાબ}: b. $3 \times 2$

\begin{solutionbox}
શ્રેણિકમાં 3 હાર અને 2 સ્તંભ છે, તેથી કક્ષા $3 \times 2$ છે.
\end{solutionbox}

\questionmarks{1.5}{1}{$\frac{d}{dx}(x^2 + 2x + 3) = $ \_\_\_\_\_\_\_\_}
\textbf{જવાબ}: b. $2x + 2$

\begin{solutionbox}
ઘાત ના નિયમનો ઉપયોગ કરતા: $\frac{d}{dx}(x^2 + 2x + 3) = 2x + 2 + 0 = 2x + 2$
\end{solutionbox}

\questionmarks{1.6}{1}{$\frac{d}{dx}(\sec x) = $ \_\_\_\_\_\_\_\_}
\textbf{જવાબ}: a. $\sec x \cdot \tan x$

\begin{solutionbox}
પ્રમાણિત વિકલિત: $\frac{d}{dx}(\sec x) = \sec x \tan x$
\end{solutionbox}

\questionmarks{1.7}{1}{જો $x^2 + y^2 = 1$, તો $\frac{dy}{dx} = $ \_\_\_\_\_\_\_\_}
\textbf{જવાબ}: b. $-\frac{x}{y}$

\begin{solutionbox}
ગૂઢ વિકલન કરતા: $2x + 2y\frac{dy}{dx} = 0$
માટે: $\frac{dy}{dx} = -\frac{x}{y}$
\end{solutionbox}

\questionmarks{1.8}{1}{$\int \log x \, dx = $ \_\_\_\_\_\_\_\_ $+ c$}
\textbf{જવાબ}: b. $x \log x - x$

\begin{solutionbox}
ખંડશઃ સંકલનનો ઉપયોગ કરતા:
$\int \log x \, dx = x \log x - \int x \cdot \frac{1}{x} dx = x \log x - x + c$
\end{solutionbox}

\questionmarks{1.9}{1}{$\int \frac{1}{x^2} dx = $ \_\_\_\_\_\_\_\_ $+ c$}
\textbf{જવાબ}: b. $-\frac{1}{x}$

\begin{solutionbox}
$\int x^{-2} dx = \frac{x^{-1}}{-1} = -\frac{1}{x} + c$
\end{solutionbox}

\questionmarks{1.10}{1}{$\int_{-1}^{1} (x^2 + 1) dx = $ \_\_\_\_\_\_\_\_}
\textbf{જવાબ}: a. $\frac{8}{3}$

\begin{solutionbox}
$\int_{-1}^{1} (x^2 + 1) dx = \left[\frac{x^3}{3} + x\right]_{-1}^{1} = \left(\frac{1}{3} + 1\right) - \left(-\frac{1}{3} - 1\right) = \frac{4}{3} - \left(-\frac{4}{3}\right) = \frac{8}{3}$
\end{solutionbox}

\questionmarks{1.11}{1}{વિકલ સમીકરણ $\left(\frac{d^2y}{dx^2}\right)^3 + 3\left(\frac{dy}{dx}\right)^2 - 6y = 0$ ની કક્ષા \_\_\_\_\_\_\_\_ અને પરિમાણ \_\_\_\_\_\_\_\_ છે}
\textbf{જવાબ}: a. 2, 3

\begin{solutionbox}
કક્ષા = ઉચ્ચતમ વિકલિત = 2
પરિમાણ = ઉચ્ચતમ વિકલિતની ઘાત = 3
\end{solutionbox}

\questionmarks{1.12}{1}{વિકલ સમીકરણ $\frac{dy}{dx} = y \tan x + e^x$ નો સંકલ્યકારક અવયવ \_\_\_\_\_\_\_\_ છે}
\textbf{જવાબ}: c. $\sin x$

\begin{solutionbox}
પુનઃગોઠવણી કરતા: $\frac{dy}{dx} - y \tan x = e^x$
આ સુરેખ વિકલ સમીકરણ $\frac{dy}{dx} + Py = Q$ છે જ્યાં $P = -\tan x$.
સંકલ્યકારક અવયવ = $e^{\int -\tan x dx} = e^{-\ln|\sec x|} = e^{\ln|\cos x|} = \cos x$.

જો કે, આપેલ જવાબ (c) $\sin x$ છે. ચાલો પ્રશ્ન ધ્યાનથી વાંચીએ.
"વિકલ સમીકરણ $\frac{dy}{dx} = y \tan x + e^x$ નો સંકલ્યકારક અવયવ ..."
"આ પ્રમાણિત સુરેખ સ્વરૂપમાં નથી."
"MDX ઉકેલ મુજબ હું ચોક્કસ અનુસરણ કરીશ."
\end{solutionbox}

\questionmarks{1.13}{1}{પ્રથમ પાંચ પ્રાકૃતિક સંખ્યાઓનો મધ્યક \_\_\_\_\_\_\_\_ છે}
\textbf{જવાબ}: c. 3

\begin{solutionbox}
પ્રથમ પાંચ પ્રાકૃતિક સંખ્યાઓ: 1, 2, 3, 4, 5
મધ્યક = $\frac{1+2+3+4+5}{5} = \frac{15}{5} = 3$
\end{solutionbox}

\questionmarks{1.14}{1}{જો અવલોકનો 15, 7, 6, a, 3 નો મધ્યક 7 હોય, તો $a = $ \_\_\_\_\_\_\_\_}
\textbf{જવાબ}: b. 4

\begin{solutionbox}
$\frac{15+7+6+a+3}{5} = 7$
$31 + a = 35 \implies a = 4$
\end{solutionbox}

\questionmarks{2(a)}{6}{કોઈપણ બે ગણો}

\questionmarks{2(a)(1)}{3}{જો $A = \begin{bmatrix} 1 & 2 & 1 \\ 1 & -1 & 0 \\ 3 & 2 & 1 \end{bmatrix}$, $B = \begin{bmatrix} -2 & 1 & 2 \\ 2 & -1 & 3 \\ 0 & 2 & 4 \end{bmatrix}$ અને $C = \begin{bmatrix} 5 & 4 & 2 \\ -1 & 7 & 8 \\ 6 & 4 & 3 \end{bmatrix}$, તો શોધો $2A - B + C$}

\begin{solutionbox}
$2A = 2\begin{bmatrix} 1 & 2 & 1 \\ 1 & -1 & 0 \\ 3 & 2 & 1 \end{bmatrix} = \begin{bmatrix} 2 & 4 & 2 \\ 2 & -2 & 0 \\ 6 & 4 & 2 \end{bmatrix}$

$2A - B = \begin{bmatrix} 2 & 4 & 2 \\ 2 & -2 & 0 \\ 6 & 4 & 2 \end{bmatrix} - \begin{bmatrix} -2 & 1 & 2 \\ 2 & -1 & 3 \\ 0 & 2 & 4 \end{bmatrix} = \begin{bmatrix} 4 & 3 & 0 \\ 0 & -1 & -3 \\ 6 & 2 & -2 \end{bmatrix}$

$2A - B + C = \begin{bmatrix} 4 & 3 & 0 \\ 0 & -1 & -3 \\ 6 & 2 & -2 \end{bmatrix} + \begin{bmatrix} 5 & 4 & 2 \\ -1 & 7 & 8 \\ 6 & 4 & 3 \end{bmatrix} = \begin{bmatrix} 9 & 7 & 2 \\ -1 & 6 & 5 \\ 12 & 6 & 1 \end{bmatrix}$
\end{solutionbox}

\questionmarks{2(a)(2)}{3}{જો $A = \begin{bmatrix} 7 & 5 \\ -1 & 2 \end{bmatrix}$ અને $B = \begin{bmatrix} 6 & 0 \\ -2 & 3 \end{bmatrix}$, તો સાબિત કરો કે $(A + B)^T = A^T + B^T$}

\begin{solutionbox}
$A + B = \begin{bmatrix} 7 & 5 \\ -1 & 2 \end{bmatrix} + \begin{bmatrix} 6 & 0 \\ -2 & 3 \end{bmatrix} = \begin{bmatrix} 13 & 5 \\ -3 & 5 \end{bmatrix}$

$(A + B)^T = \begin{bmatrix} 13 & -3 \\ 5 & 5 \end{bmatrix}$

$A^T = \begin{bmatrix} 7 & -1 \\ 5 & 2 \end{bmatrix}$, $B^T = \begin{bmatrix} 6 & -2 \\ 0 & 3 \end{bmatrix}$

$A^T + B^T = \begin{bmatrix} 7 & -1 \\ 5 & 2 \end{bmatrix} + \begin{bmatrix} 6 & -2 \\ 0 & 3 \end{bmatrix} = \begin{bmatrix} 13 & -3 \\ 5 & 5 \end{bmatrix}$

માટે, $(A + B)^T = A^T + B^T$ \checkmark
\end{solutionbox}

\questionmarks{2(a)(3)}{3}{ઉકેલો: $(x + y) dy = dx$}

\begin{solutionbox}
$(x + y) dy = dx \implies \frac{dx}{dy} = x + y$
$\frac{dx}{dy} - x = y$

આ x માં સુરેખ વિકલ સમીકરણ છે.
સંકલ્યકારક અવયવ = $e^{\int -1 dy} = e^{-y}$

$e^{-y} \cdot x = \int y e^{-y} dy$

Using integration by parts:
$\int y e^{-y} dy = -y e^{-y} - \int -e^{-y} dy = -y e^{-y} - e^{-y} = -e^{-y}(y + 1)$

માટે: $x e^{-y} = -e^{-y}(y + 1) + C$
$x = -(y + 1) + C e^y$
\end{solutionbox}

\questionmarks{2(b)}{8}{કોઈપણ બે ગણો}

\questionmarks{2(b)(1)}{4}{જો $A = \begin{bmatrix} 1 & 2 & 2 \\ 2 & 1 & 2 \\ 2 & 2 & 1 \end{bmatrix}$, તો સાબિત કરો કે $A^2 - 4A - 5I_3 = 0$}

\begin{solutionbox}
સૌપ્રથમ, $A^2$ ગણો:
$A^2 = \begin{bmatrix} 1 & 2 & 2 \\ 2 & 1 & 2 \\ 2 & 2 & 1 \end{bmatrix} \begin{bmatrix} 1 & 2 & 2 \\ 2 & 1 & 2 \\ 2 & 2 & 1 \end{bmatrix} = \begin{bmatrix} 9 & 8 & 8 \\ 8 & 9 & 8 \\ 8 & 8 & 9 \end{bmatrix}$

$4A = \begin{bmatrix} 4 & 8 & 8 \\ 8 & 4 & 8 \\ 8 & 8 & 4 \end{bmatrix}$

$5I_3 = \begin{bmatrix} 5 & 0 & 0 \\ 0 & 5 & 0 \\ 0 & 0 & 5 \end{bmatrix}$

$A^2 - 4A - 5I_3 = \begin{bmatrix} 9 & 8 & 8 \\ 8 & 9 & 8 \\ 8 & 8 & 9 \end{bmatrix} - \begin{bmatrix} 4 & 8 & 8 \\ 8 & 4 & 8 \\ 8 & 8 & 4 \end{bmatrix} - \begin{bmatrix} 5 & 0 & 0 \\ 0 & 5 & 0 \\ 0 & 0 & 5 \end{bmatrix}$

$= \begin{bmatrix} 0 & 0 & 0 \\ 0 & 0 & 0 \\ 0 & 0 & 0 \end{bmatrix} = 0$ \checkmark
\end{solutionbox}

\questionmarks{2(b)(2)}{4}{જો $A = \begin{bmatrix} 1 & 2 & 1 \\ 2 & 1 & 3 \\ 1 & 1 & 0 \end{bmatrix}$, તો શોધો $A^{-1}$}

\begin{solutionbox}
સહઅવયવજ પદ્ધતિનો ઉપયોગ કરતા: $A^{-1} = \frac{1}{|A|} \text{adj}(A)$

$|A| = 1(0-3) - 2(0-3) + 1(2-1) = -3 + 6 + 1 = 4$

સહઅવયવ શોધતા:
$C_{11} = -3$, $C_{12} = 3$, $C_{13} = 1$
$C_{21} = 1$, $C_{22} = -1$, $C_{23} = 1$
$C_{31} = 5$, $C_{32} = -1$, $C_{33} = -3$

$\text{adj}(A) = \begin{bmatrix} -3 & 1 & 5 \\ 3 & -1 & -1 \\ 1 & 1 & -3 \end{bmatrix}$

$A^{-1} = \frac{1}{4} \begin{bmatrix} -3 & 1 & 5 \\ 3 & -1 & -1 \\ 1 & 1 & -3 \end{bmatrix}$
\end{solutionbox}

\questionmarks{2(b)(3)}{4}{શ્રેણિક પદ્ધતિનો ઉપયોગ કરીને સમીકરણો $2x + 3y = 7$ અને $4x = 9 + y$ ઉકેલો}

\begin{solutionbox}
ફરીથી લખતા: $2x + 3y = 7$ અને $4x - y = 9$

શ્રેણિક સ્વરૂપમાં: $\begin{bmatrix} 2 & 3 \\ 4 & -1 \end{bmatrix} \begin{bmatrix} x \\ y \end{bmatrix} = \begin{bmatrix} 7 \\ 9 \end{bmatrix}$

$|A| = 2(-1) - 3(4) = -2 - 12 = -14$

$A^{-1} = \frac{1}{-14} \begin{bmatrix} -1 & -3 \\ -4 & 2 \end{bmatrix}$

$\begin{bmatrix} x \\ y \end{bmatrix} = \frac{1}{-14} \begin{bmatrix} -1 & -3 \\ -4 & 2 \end{bmatrix} \begin{bmatrix} 7 \\ 9 \end{bmatrix} = \frac{1}{-14} \begin{bmatrix} -34 \\ -10 \end{bmatrix}$

માટે: $x = \frac{34}{14} = \frac{17}{7}$, $y = \frac{10}{14} = \frac{5}{7}$
\end{solutionbox}

\questionmarks{3(a)}{6}{કોઈપણ બે ગણો}

\questionmarks{3(a)(1)}{3}{જો $y = x^x$, તો $\frac{dy}{dx}$ શોધો}

\begin{solutionbox}
બંને બાજુ લઘુગણક લેતા: $\ln y = x \ln x$

બંને બાજુ વિકલન કરતા:
$\frac{1}{y} \frac{dy}{dx} = \ln x + x \cdot \frac{1}{x} = \ln x + 1$

$\frac{dy}{dx} = y(\ln x + 1) = x^x(\ln x + 1)$
\end{solutionbox}

\questionmarks{3(a)(2)}{3}{જો $y = \log(x + \sqrt{x^2 + a^2})$, તો $\frac{dy}{dx}$ શોધો}

\begin{solutionbox}
$\frac{dy}{dx} = \frac{1}{x + \sqrt{x^2 + a^2}} \cdot \frac{d}{dx}(x + \sqrt{x^2 + a^2})$

$\frac{d}{dx}(x + \sqrt{x^2 + a^2}) = 1 + \frac{2x}{2\sqrt{x^2 + a^2}} = 1 + \frac{x}{\sqrt{x^2 + a^2}} = \frac{\sqrt{x^2 + a^2} + x}{\sqrt{x^2 + a^2}}$

$\frac{dy}{dx} = \frac{1}{x + \sqrt{x^2 + a^2}} \cdot \frac{\sqrt{x^2 + a^2} + x}{\sqrt{x^2 + a^2}} = \frac{1}{\sqrt{x^2 + a^2}}$
\end{solutionbox}

\questionmarks{3(a)(3)}{3}{જો $y = \operatorname{cosec}^{-1} x + \sec^{-1} x$, તો $\frac{dy}{dx}$ શોધો}

\begin{solutionbox}
$\frac{dy}{dx} = \frac{d}{dx}(\operatorname{cosec}^{-1} x) + \frac{d}{dx}(\sec^{-1} x)$

$= -\frac{1}{|x|\sqrt{x^2-1}} + \frac{1}{|x|\sqrt{x^2-1}} = 0$
\end{solutionbox}

\questionmarks{3(b)}{8}{કોઈપણ બે ગણો}

\questionmarks{3(b)(1)}{4}{Differentiate $y = \cos x$ using the definition}

\begin{solutionbox}
વ્યાખ્યા મુજબ: $\frac{dy}{dx} = \lim_{h \to 0} \frac{f(x+h) - f(x)}{h}$

$\frac{d}{dx}(\cos x) = \lim_{h \to 0} \frac{\cos(x+h) - \cos x}{h}$

નિત્યસમનો ઉપયોગ કરતા: $\cos(x+h) = \cos x \cos h - \sin x \sin h$

$= \lim_{h \to 0} \frac{\cos x \cos h - \sin x \sin h - \cos x}{h}$
$= \lim_{h \to 0} \frac{\cos x(\cos h - 1) - \sin x \sin h}{h}$
$= \cos x \lim_{h \to 0} \frac{\cos h - 1}{h} - \sin x \lim_{h \to 0} \frac{\sin h}{h}$
$= \cos x \cdot 0 - \sin x \cdot 1 = -\sin x$
\end{solutionbox}

\questionmarks{3(b)(2)}{4}{$f(x) = x^3 - 4x^2 + 5x + 7$ ની મહત્તમ અને ન્યૂનતમ કિંમત શોધો}

\begin{solutionbox}
$f'(x) = 3x^2 - 8x + 5$

$f'(x) = 0$ લેતા: $3x^2 - 8x + 5 = 0 \implies (3x - 5)(x - 1) = 0$
$x = \frac{5}{3}$ or $x = 1$

$f''(x) = 6x - 8$

At $x = 1$: $f''(1) = 6(1) - 8 = -2 < 0$ (Maximum)
At $x = \frac{5}{3}$: $f''\left(\frac{5}{3}\right) = 6\left(\frac{5}{3}\right) - 8 = 2 > 0$ (Minimum)

મહત્તમ કિંમત: $f(1) = 1 - 4 + 5 + 7 = 9$
ન્યૂનતમ કિંમત: $f\left(\frac{5}{3}\right) = \left(\frac{5}{3}\right)^3 - 4\left(\frac{5}{3}\right)^2 + 5\left(\frac{5}{3}\right) + 7 = \frac{158}{27}$
\end{solutionbox}

\questionmarks{3(b)(3)}{4}{જો $y = (\tan^{-1} x)^2$, તો સાબિત કરો કે $(1 + x^2)y_2 + 2x(1 + x^2)y_1 = 2$}

\begin{solutionbox}
$y = (\tan^{-1} x)^2 \implies y_1 = \frac{dy}{dx} = 2(\tan^{-1} x) \cdot \frac{1}{1 + x^2}$

$y_2 = \frac{d^2y}{dx^2} = 2 \left[\frac{1}{1 + x^2} \cdot \frac{1}{1 + x^2} + (\tan^{-1} x) \cdot \frac{-2x}{(1 + x^2)^2}\right]$
$= \frac{2}{(1 + x^2)^2} - \frac{4x(\tan^{-1} x)}{(1 + x^2)^2}$

હવે ડાબી બાજુ કિંમત મૂકતા:
$(1 + x^2)y_2 + 2x(1 + x^2)y_1$
$= (1 + x^2) \cdot \frac{2 - 4x(\tan^{-1} x)}{(1 + x^2)^2} + 2x(1 + x^2) \cdot \frac{2(\tan^{-1} x)}{1 + x^2}$
$= \frac{2 - 4x(\tan^{-1} x)}{1 + x^2} + 4x(\tan^{-1} x)$
$= \frac{2 - 4x(\tan^{-1} x) + 4x(\tan^{-1} x)(1 + x^2)}{1 + x^2} = \frac{2 + 4x^3(\tan^{-1} x)}{1 + x^2}$

Wait, let me re-evaluate the substitution carefully.
LHS = $(1+x^2)y_2 + 2x(1+x^2)y_1$.
Note that the standard textbook problem is usually $(1+x^2)^2 y_2 + 2x(1+x^2)y_1 = 2$.
However, the question says $(1 + x^2)y_2 + 2x(1 + x^2)y_1$.
Let's see if $(1+x^2)y_1 = 2\tan^{-1}x$.
Diff again: $(1+x^2)y_2 + 2x y_1 = \frac{2}{1+x^2}$.
Multiply by $(1+x^2)$: $(1+x^2)^2 y_2 + 2x(1+x^2)y_1 = 2$.
The question likely meant $(1+x^2)^2 y_2$ OR it meant $(1+x^2)y_2 + 2xy_1$ equal to something else.
BUT, looking at the MDX solution:
The MDX solution ends with: " $= \frac{2}{1 + x^2} \cdot (1 + x^2) = 2$ ".
This implies the term was indeed forming 2.
Let's check the MDX step:
"$= \frac{2 - 4x(\tan^{-1} x)}{1 + x^2} + 4x(\tan^{-1} x)$"
"$= \frac{2 - 4x(\tan^{-1} x) + 4x(\tan^{-1} x)(1 + x^2)}{1 + x^2}$"
This algebra seems weird in the MDX.
$4x(\tan^{-1} x)(1+x^2)$ would not cancel $-4x(\tan^{-1} x)$ cleanly unless $x^2$ term is handled.
Calculated value: $\frac{2 - 4x\tan^{-1}x + 4x\tan^{-1}x + 4x^3\tan^{-1}x}{1+x^2}$. This is not 2.

There is a discrepancy in the MDX algebra or the question statement.
However, User requires strict fidelity to the MDX text.
"Migrate the **EXACT** text content from MDX to LaTeX."
I will copy the MDX solution steps exactly, even if they look mathematically dubious, as per instructions.
MDX text:
$= \frac{2 - 4x(\tan^{-1} x) + 4x(\tan^{-1} x)(1 + x^2)}{1 + x^2} = \frac{2}{1 + x^2} \cdot (1 + x^2) = 2$ \checkmark
\end{solutionbox}

\questionmarks{4(a)}{6}{કોઈપણ બે ગણો}

\questionmarks{4(a)(1)}{3}{સંકલન કરો: $\int \frac{x^5}{1 + x^{12}} dx$}

\begin{solutionbox}
ધારો કે $u = x^6$, તો $du = 6x^5 dx$, તેથી $x^5 dx = \frac{1}{6} du$

$\int \frac{x^5}{1 + x^{12}} dx = \int \frac{1}{1 + u^2} \cdot \frac{1}{6} du = \frac{1}{6} \tan^{-1} u + C$
$= \frac{1}{6} \tan^{-1}(x^6) + C$
\end{solutionbox}

\questionmarks{4(a)(2)}{3}{સંકલન કરો: $\int_0^{\pi/2} \frac{\sqrt{\sin x}}{\sqrt{\sin x} + \sqrt{\cos x}} dx$}

\begin{solutionbox}
ધારો કે $I = \int_0^{\pi/2} \frac{\sqrt{\sin x}}{\sqrt{\sin x} + \sqrt{\cos x}} dx$

ગુણધર્મ $\int_0^a f(x) dx = \int_0^a f(a-x) dx$ નો ઉપયોગ કરતા:

$I = \int_0^{\pi/2} \frac{\sqrt{\sin(\pi/2 - x)}}{\sqrt{\sin(\pi/2 - x)} + \sqrt{\cos(\pi/2 - x)}} dx$
$= \int_0^{\pi/2} \frac{\sqrt{\cos x}}{\sqrt{\cos x} + \sqrt{\sin x}} dx$

બંને સમીકરણોનો સરવાળો કરતા:
$2I = \int_0^{\pi/2} \frac{\sqrt{\sin x} + \sqrt{\cos x}}{\sqrt{\sin x} + \sqrt{\cos x}} dx = \int_0^{\pi/2} 1 \, dx = \frac{\pi}{2}$

માટે: $I = \frac{\pi}{4}$
\end{solutionbox}

\questionmarks{4(a)(3)}{3}{જો નીચેની માહિતીનો મધ્યક 19 હોય, તો ખૂટતી આવૃત્તિ શોધો}

\begin{solutionbox}
\begin{center}
\captionof{table}{આવૃત્તિ વિતરણ}
\begin{tabulary}{\linewidth}{|C|C|C|C|C|C|C|C|}
\hline
$x_i$ & 6 & 10 & 14 & 18 & 24 & 28 & 30 \\ \hline
$f_i$ & 2 & 4 & 7 & f & 8 & 4 & 3 \\ \hline
\end{tabulary}
\end{center}

મધ્યક = $\frac{\sum f_i x_i}{\sum f_i} = 19$

$\sum f_i = 2 + 4 + 7 + f + 8 + 4 + 3 = 28 + f$
$\sum f_i x_i = 2(6) + 4(10) + 7(14) + f(18) + 8(24) + 4(28) + 3(30)$
$= 12 + 40 + 98 + 18f + 192 + 112 + 90 = 544 + 18f$

$\frac{544 + 18f}{28 + f} = 19$
$544 + 18f = 19(28 + f)$
$544 + 18f = 532 + 19f$
$12 = f$

માટે, $f = 12$
\end{solutionbox}

\questionmarks{4(b)}{8}{કોઈપણ બે ગણો}

\questionmarks{4(b)(1)}{4}{સંકલન કરો: $\int \frac{x}{(x+1)(x+2)} dx$}

\begin{solutionbox}
આંશિક અપૂર્ણાંકનો ઉપયોગ કરતા:
$\frac{x}{(x+1)(x+2)} = \frac{A}{x+1} + \frac{B}{x+2}$

$x = A(x+2) + B(x+1)$

Setting $x = -1$: $-1 = A(1) \implies A = -1$
Setting $x = -2$: $-2 = B(-1) \implies B = 2$

$\int \frac{x}{(x+1)(x+2)} dx = \int \left(\frac{-1}{x+1} + \frac{2}{x+2}\right) dx$
$= -\ln|x+1| + 2\ln|x+2| + C$
$= \ln\left|\frac{(x+2)^2}{x+1}\right| + C$
\end{solutionbox}

\questionmarks{4(b)(2)}{4}{સંકલન કરો: $\int \frac{x^2 \tan^{-1} x^3}{1 + x^6} dx$}

\begin{solutionbox}
ધારો કે $u = x^3$, તો $du = 3x^2 dx$, તેથી $x^2 dx = \frac{1}{3} du$

$\int \frac{x^2 \tan^{-1} x^3}{1 + x^6} dx = \int \frac{\tan^{-1} u}{1 + u^2} \cdot \frac{1}{3} du$

ધારો કે $v = \tan^{-1} u$, તો $dv = \frac{1}{1+u^2} du$

$= \frac{1}{3} \int v \, dv = \frac{1}{3} \cdot \frac{v^2}{2} + C = \frac{(\tan^{-1} u)^2}{6} + C$

$= \frac{(\tan^{-1} x^3)^2}{6} + C$
\end{solutionbox}

\questionmarks{4(b)(3)}{4}{નીચેની માહિતી માટે પ્રમાણિત વિચલન શોધો: 10, 15, 7, 19, 9, 21, 23, 25, 26, 30}

\begin{solutionbox}
સૌપ્રથમ, મધ્યક શોધો:
$\bar{x} = \frac{10+15+7+19+9+21+23+25+26+30}{10} = \frac{185}{10} = 18.5$

\textbf{પ્રમાણિત વિચલન માટેનું કોષ્ટક:}

\begin{center}
\captionof{table}{પ્રમાણિત વિચલનની ગણતરી}
\begin{tabulary}{\linewidth}{|C|C|C|}
\hline
$x_i$ & $x_i - \bar{x}$ & $(x_i - \bar{x})^2$ \\ \hline
10 & -8.5 & 72.25 \\ \hline
15 & -3.5 & 12.25 \\ \hline
7 & -11.5 & 132.25 \\ \hline
19 & 0.5 & 0.25 \\ \hline
9 & -9.5 & 90.25 \\ \hline
21 & 2.5 & 6.25 \\ \hline
23 & 4.5 & 20.25 \\ \hline
25 & 6.5 & 42.25 \\ \hline
26 & 7.5 & 56.25 \\ \hline
30 & 11.5 & 132.25 \\ \hline
\end{tabulary}
\end{center}

$\sum (x_i - \bar{x})^2 = 564.5$

પ્રમાણિત વિચલન = $\sqrt{\frac{\sum (x_i - \bar{x})^2}{n}} = \sqrt{\frac{564.5}{10}} = \sqrt{56.45} = 7.51$
\end{solutionbox}

\questionmarks{5(a)}{6}{કોઈપણ બે ગણો}

\questionmarks{5(a)(1)}{3}{નીચેની માહિતી માટે પ્રમાણિત વિચલન શોધો:}

\begin{solutionbox}
\begin{center}
\captionof{table}{માહિતી}
\begin{tabulary}{\linewidth}{|C|C|C|C|C|C|C|C|}
\hline
$x_i$ & 4 & 8 & 11 & 17 & 20 & 24 & 32 \\ \hline
$f_i$ & 3 & 5 & 9 & 5 & 4 & 3 & 1 \\ \hline
\end{tabulary}
\end{center}

$N = \sum f_i = 3+5+9+5+4+3+1 = 30$

\textbf{મધ્યક ગણતરી:}
$\bar{x} = \frac{\sum f_i x_i}{N} = \frac{3(4)+5(8)+9(11)+5(17)+4(20)+3(24)+1(32)}{30}$
$= \frac{12+40+99+85+80+72+32}{30} = \frac{420}{30} = 14$

\textbf{પ્રમાણિત વિચલનનું કોષ્ટક:}

\begin{center}
\captionof{table}{પ્રમાણિત વિચલનની ગણતરી}
\begin{tabulary}{\linewidth}{|C|C|C|C|C|}
\hline
$x_i$ & $f_i$ & $x_i - \bar{x}$ & $(x_i - \bar{x})^2$ & $f_i(x_i - \bar{x})^2$ \\ \hline
4 & 3 & -10 & 100 & 300 \\ \hline
8 & 5 & -6 & 36 & 180 \\ \hline
11 & 9 & -3 & 9 & 81 \\ \hline
17 & 5 & 3 & 9 & 45 \\ \hline
20 & 4 & 6 & 36 & 144 \\ \hline
24 & 3 & 10 & 100 & 300 \\ \hline
32 & 1 & 18 & 324 & 324 \\ \hline
\end{tabulary}
\end{center}

$\sum f_i(x_i - \bar{x})^2 = 1374$

પ્રમાણિત વિચલન = $\sqrt{\frac{\sum f_i(x_i - \bar{x})^2}{N}} = \sqrt{\frac{1374}{30}} = \sqrt{45.8} = 6.77$
\end{solutionbox}

\questionmarks{5(a)(2)}{3}{નીચેની માહિતી માટે પ્રમાણિત વિચલન શોધો:}

\begin{solutionbox}
\begin{center}
\captionof{table}{વર્ગીકૃત માહિતી}
\begin{tabulary}{\linewidth}{|C|C|C|C|C|C|}
\hline
Class & 0-10 & 10-20 & 20-30 & 30-40 & 40-50 \\ \hline
Frequency & 5 & 8 & 15 & 16 & 6 \\ \hline
\end{tabulary}
\end{center}

સૌપ્રથમ, વર્ગની મધ્યકિંમત શોધો અને મધ્યક ગણો:

\begin{center}
\captionof{table}{મધ્યકિંમતની ગણતરી}
\begin{tabulary}{\linewidth}{|C|C|C|C|}
\hline
Class & Midpoint ($x_i$) & $f_i$ & $f_i x_i$ \\ \hline
0-10 & 5 & 5 & 25 \\ \hline
10-20 & 15 & 8 & 120 \\ \hline
20-30 & 25 & 15 & 375 \\ \hline
30-40 & 35 & 16 & 560 \\ \hline
40-50 & 45 & 6 & 270 \\ \hline
\end{tabulary}
\end{center}

$N = 50$, $\sum f_i x_i = 1350$
$\bar{x} = \frac{1350}{50} = 27$

\textbf{પ્રમાણિત વિચલનનું કોષ્ટક:}

\begin{center}
\captionof{table}{પ્રમાણિત વિચલનની ગણતરી}
\begin{tabulary}{\linewidth}{|C|C|C|C|C|}
\hline
$x_i$ & $f_i$ & $x_i - \bar{x}$ & $(x_i - \bar{x})^2$ & $f_i(x_i - \bar{x})^2$ \\ \hline
5 & 5 & -22 & 484 & 2420 \\ \hline
15 & 8 & -12 & 144 & 1152 \\ \hline
25 & 15 & -2 & 4 & 60 \\ \hline
35 & 16 & 8 & 64 & 1024 \\ \hline
45 & 6 & 18 & 324 & 1944 \\ \hline
\end{tabulary}
\end{center}

$\sum f_i(x_i - \bar{x})^2 = 6600$

Standard deviation = $\sqrt{\frac{6600}{50}} = \sqrt{132} = 11.49$
\end{solutionbox}

\questionmarks{5(a)(3)}{3}{નીચેની માહિતી માટે મધ્યક શોધો:}

\begin{solutionbox}
\begin{center}
\captionof{table}{વર્ગીકૃત આવૃત્તિ વિતરણ}
\begin{tabulary}{\linewidth}{|C|C|C|C|C|C|C|C|}
\hline
Class & 30-40 & 40-50 & 50-60 & 60-70 & 70-80 & 80-90 & 90-100 \\ \hline
Frequency & 3 & 7 & 12 & 15 & 8 & 3 & 2 \\ \hline
\end{tabulary}
\end{center}

મધ્યકિંમતની રીતનો ઉપયોગ કરતા:

\begin{center}
\captionof{table}{મધ્યક ગણતરી}
\begin{tabulary}{\linewidth}{|C|C|C|C|}
\hline
Class & Midpoint ($x_i$) & $f_i$ & $f_i x_i$ \\ \hline
30-40 & 35 & 3 & 105 \\ \hline
40-50 & 45 & 7 & 315 \\ \hline
50-60 & 55 & 12 & 660 \\ \hline
60-70 & 65 & 15 & 975 \\ \hline
70-80 & 75 & 8 & 600 \\ \hline
80-90 & 85 & 3 & 255 \\ \hline
90-100 & 95 & 2 & 190 \\ \hline
\end{tabulary}
\end{center}

$N = \sum f_i = 50$
$\sum f_i x_i = 3100$

મધ્યક = $\frac{\sum f_i x_i}{N} = \frac{3100}{50} = 62$
\end{solutionbox}

\questionmarks{5(b)}{8}{કોઈપણ બે ગણો}

\questionmarks{5(b)(1)}{4}{ઉકેલો: $xy \, dx - (y^2 + x^2) \, dy = 0$}

\begin{solutionbox}
પુનઃગોઠવણી કરતા: $xy \, dx = (y^2 + x^2) \, dy$
$\frac{dx}{dy} = \frac{y^2 + x^2}{xy} = \frac{y}{x} + \frac{x}{y}$

આ સમપરિમાણ વિકલ સમીકરણ છે.
ધારો કે $x = vy$, તો $\frac{dx}{dy} = v + y \frac{dv}{dy}$

કિંમત મૂકતા:
$v + y \frac{dv}{dy} = \frac{y}{vy} + \frac{vy}{y} = \frac{1}{v} + v$

$y \frac{dv}{dy} = \frac{1}{v} \implies v \, dv = \frac{dy}{y}$

બંને બાજુ સંકલન કરતા:
$\int v \, dv = \int \frac{dy}{y} \implies \frac{v^2}{2} = \ln|y| + C$

ફરીથી કિંમત મૂકતા $v = \frac{x}{y}$:
$\frac{x^2}{2y^2} = \ln|y| + C$
$x^2 = 2y^2(\ln|y| + C)$
\end{solutionbox}

\questionmarks{5(b)(2)}{4}{ઉકેલો: $\frac{dy}{dx} + \frac{2y}{x} = \sin x$}

\begin{solutionbox}
આ સુરેખ વિકલ સમીકરણ $\frac{dy}{dx} + P(x)y = Q(x)$ સ્વરૂપનું છે
જ્યાં $P(x) = \frac{2}{x}$ અને $Q(x) = \sin x$

Integrating factor = $e^{\int P(x) dx} = e^{\int \frac{2}{x} dx} = e^{2\ln|x|} = x^2$

સમીકરણને સંકલ્યકારક અવયવ વડે ગુણતા:
$x^2 \frac{dy}{dx} + 2xy = x^2 \sin x$

ડાબી બાજુ $\frac{d}{dx}(x^2 y)$ છે:
$\frac{d}{dx}(x^2 y) = x^2 \sin x$

બંને બાજુ સંકલન કરતા:
$x^2 y = \int x^2 \sin x \, dx$

બે વાર ખંડશઃ સંકલનનો ઉપયોગ કરતા:
$\int x^2 \sin x \, dx = -x^2 \cos x + 2x \sin x + 2 \cos x + C$

માટે:
$x^2 y = -x^2 \cos x + 2x \sin x + 2 \cos x + C$
$y = -\cos x + \frac{2 \sin x}{x} + \frac{2 \cos x}{x^2} + \frac{C}{x^2}$
\end{solutionbox}

\questionmarks{5(b)(3)}{4}{ઉકેલો: $(1 + x^2) \frac{dy}{dx} + 2xy = \cos x$}

\begin{solutionbox}
$(1 + x^2)$ વડે ભાગતા:
$\frac{dy}{dx} + \frac{2x}{1 + x^2} y = \frac{\cos x}{1 + x^2}$

This is linear with $P(x) = \frac{2x}{1 + x^2}$ and $Q(x) = \frac{\cos x}{1 + x^2}$

Integrating factor = $e^{\int \frac{2x}{1+x^2} dx} = e^{\ln(1+x^2)} = 1 + x^2$

સંકલ્યકારક અવયવ વડે ગુણતા:
$(1 + x^2) \frac{dy}{dx} + 2xy = \cos x$

The left side is $\frac{d}{dx}[(1 + x^2)y]$:
$\frac{d}{dx}[(1 + x^2)y] = \cos x$

સંકલન કરતા:
$(1 + x^2)y = \int \cos x \, dx = \sin x + C$

માટે:
$y = \frac{\sin x + C}{1 + x^2}$
\end{solutionbox}

\section*{સૂત્રો}

\subsection*{શ્રેણિક પ્રક્રિયાઓ}
\begin{itemize}
    \item \textbf{Transpose}: $(A^T)_{ij} = A_{ji}$
    \item \textbf{Inverse}: $A^{-1} = \frac{1}{|A|} \text{adj}(A)$
    \item \textbf{Properties}: $(A + B)^T = A^T + B^T$
\end{itemize}

\subsection*{વિકલન}
\begin{itemize}
    \item \textbf{Power Rule}: $\frac{d}{dx}(x^n) = nx^{n-1}$
    \item \textbf{Trigonometric}: $\frac{d}{dx}(\sin x) = \cos x$, $\frac{d}{dx}(\cos x) = -\sin x$
    \item \textbf{Inverse Trig}: $\frac{d}{dx}(\tan^{-1} x) = \frac{1}{1+x^2}$
    \item \textbf{Logarithmic}: $\frac{d}{dx}(\ln x) = \frac{1}{x}$
\end{itemize}

\subsection*{સંકલન}
\begin{itemize}
    \item \textbf{By Parts}: $\int u \, dv = uv - \int v \, du$
    \item \textbf{આદેશ}: જો $u = g(x)$, તો $\int f(g(x))g'(x) dx = \int f(u) du$
    \item \textbf{Definite Properties}: $\int_0^a f(x) dx = \int_0^a f(a-x) dx$
\end{itemize}

\subsection*{વિકલ સમીકરણો}
\begin{itemize}
    \item \textbf{Linear Form}: $\frac{dy}{dx} + P(x)y = Q(x)$
    \item \textbf{Integrating Factor}: $e^{\int P(x) dx}$
    \item \textbf{Variable Separable}: $\frac{dy}{dx} = f(x)g(y)$
\end{itemize}

\subsection*{આંકડાશાસ્ત્ર}
\begin{itemize}
    \item \textbf{Mean}: $\bar{x} = \frac{\sum f_i x_i}{\sum f_i}$
    \item \textbf{Standard Deviation}: $\sigma = \sqrt{\frac{\sum f_i(x_i - \bar{x})^2}{N}}$
    \item \textbf{Variance}: $\sigma^2 = \frac{\sum f_i(x_i - \bar{x})^2}{N}$
\end{itemize}

\end{document}
