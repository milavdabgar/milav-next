\documentclass{article}

% content/resources/templates/preamble.tex
\usepackage[margin=0.6in]{geometry}
\author{Milav Dabgar}
\usepackage{amsmath,amssymb,amsthm}
\usepackage{booktabs}
\usepackage{multirow}
\usepackage{xcolor}
\usepackage{tcolorbox}
\tcbuselibrary{breakable,skins}
\usepackage[colorlinks=true,linkcolor=blue]{hyperref}
\usepackage{titlesec}
\usepackage{enumitem}
\usepackage{tikz}
\usepackage{pgfplots}
\usepackage{circuitikz}
\usepackage[version=4]{mhchem}
\usepackage{longtable}
\usepackage{array}
\usepackage{float}
\usepackage{caption}
\usepackage{listings}

\lstset{
  basicstyle=\small\ttfamily,
  breaklines=true,
  breakatwhitespace=false,
  postbreak=\mbox{\textcolor{red}{$\hookrightarrow$}\space},
  float=false,
  numbers=left,
  numberstyle=\tiny\color{gray},
  numbersep=10pt,
  xleftmargin=2em,
  keywordstyle=\color{blue},
  commentstyle=\color{green!60!black},
  stringstyle=\color{purple},
  backgroundcolor=\color{gray!5},
  showstringspaces=false,
  tabsize=2,
  captionpos=b,
  keepspaces=true,
  columns=flexible
}

\pgfplotsset{compat=1.18}
\usetikzlibrary{shapes,arrows,positioning,calc,patterns,decorations.pathmorphing,decorations.markings,arrows.meta}

% Color scheme
\definecolor{headcolor}{RGB}{0,102,204}
\definecolor{keycolor}{RGB}{220,20,60}
\definecolor{solutioncolor}{RGB}{34,139,34}
\definecolor{mnemoniccolor}{RGB}{148,0,211}
\definecolor{codecolor}{RGB}{0,0,100}

% Spacing
\setlength{\parskip}{3pt}
\setlist[itemize]{nosep}
\setlist[enumerate]{nosep}

% Title formatting
\titleformat{\section}{\Large\bfseries\color{headcolor}}{\thesection}{1em}{}
\titleformat{\subsection}{\large\bfseries\color{headcolor}}{\thesubsection}{1em}{}

% Pandoc tightlist compatibility
\providecommand{\tightlist}{%
  \setlength{\itemsep}{0pt}\setlength{\parskip}{0pt}}

% Pandoc longtable compatibility
\newcounter{none}
\def\thenone{}


% content/resources/templates/gujarati-boxes.tex
\usepackage{fontspec}
\usepackage{polyglossia}

% Set Gujarati as main language (document is primarily in Gujarati)
% Note: gloss-gujarati.ldf doesn't exist in polyglossia, but it will use hyphenation patterns
\setdefaultlanguage{gujarati}
\setotherlanguage{english}

% Configure Gujarati font properly
% Use Language=Default to prevent polyglossia from trying to add language-specific features
% that don't exist for Gujarati, which causes "empty feature" warnings
\newfontfamily\gujaratifont[Script=Gujarati,AutoFakeBold=2.5,AutoFakeSlant=0.3]{Noto Sans Gujarati}
\setmainfont[Script=Gujarati,AutoFakeBold=2.5,AutoFakeSlant=0.3]{Noto Sans Gujarati}
% Use Noto Sans Gujarati for monospace to support Gujarati in text
\setmonofont[Scale=0.9]{Noto Sans Gujarati}

% Configure English to use the same font
\newfontfamily\englishfont[Script=Gujarati,AutoFakeBold=2.5,AutoFakeSlant=0.3]{Noto Sans Gujarati}

% Translations for polyglossia
\gappto\captionsgujarati{
  \renewcommand{\tablename}{કોષ્ટક}
  \renewcommand{\figurename}{આકૃતિ}
}

% Helper for TikZ nodes to ensure Gujarati font
\newcommand{\gu}[1]{{\gujaratifont #1}}

% Custom environments
\newtcolorbox{solutionbox}{
    breakable,
    enhanced,
    colback=solutioncolor!5!white,
    colframe=solutioncolor!75!black,
    fonttitle=\bfseries,
    title=જવાબ
}

\newtcolorbox{solutionboxnobreak}{
 colback=solutioncolor!5!white,
 colframe=solutioncolor!75!black,
 fonttitle=\bfseries,
 title=જવાબ
}

\newtcolorbox{keyformula}{
 breakable,
 enhanced,
 colback=keycolor!5!white,
 colframe=keycolor!75!black,
 fonttitle=\bfseries,
 title=રાસાયણિક સમીકરણ/સૂત્ર
}

\newtcolorbox{mnemonicbox}{
 breakable,
 enhanced,
 colback=mnemoniccolor!5!white,
 colframe=mnemoniccolor!75!black,
 fonttitle=\bfseries,
 title=મેમરી ટ્રીક
}


% Custom commands for GTU solutions
% This file defines semantic commands for consistent formatting

% Question command with automatic formatting
\newcommand{\question}[2]{%
  \section*{Question #1}%
  \textbf{#2}%
}

% OR question variant
\newcommand{\questionor}[2]{%
  \section*{Question #1 OR}%
  \textbf{#2}%
}

% Proper table environment with caption
\newenvironment{answertable}[1]{%
  \begin{table}[htbp]
  \centering
  \caption{#1}
}{%
  \end{table}
}

% Proper figure environment for diagrams
\newenvironment{answerdiagram}[1]{%
  \begin{figure}[htbp]
  \centering
  \caption{#1}
}{%
  \end{figure}
}

% Semantic markup for key terms
\newcommand{\keyword}[1]{\textbf{#1}}
\newcommand{\code}[1]{\texttt{#1}}
\newcommand{\classname}[1]{\texttt{#1}}
\newcommand{\methodname}[1]{\texttt{#1}}

% Proper quotation marks
\newcommand{\mnemonic}[1]{``#1''}


\newcommand{\answer}[1]{\textbf{જવાબ}: #1}
\newcommand{\solution}[1]{\begin{solutionbox}#1\end{solutionbox}}

\title{Applied Mathematics (4320001) - Winter 2024 Solution}
\date{January 22, 2025}

\begin{document}
\maketitle

\questionmarks{Q.1}{14}{આપેલા વિકલ્પોમાંથી યોગ્ય પસંદગીનો ઉપયોગ કરીને ખાલી જગ્યાઓ ભરો}

\questionmarks{Q1.1}{1}{Order of the matrix $\begin{bmatrix} 1 & 2 & 3 \\ 4 & 5 & 6 \end{bmatrix}$ = \dots\dots\dots}
\answer{(a) 2 $\times$ 3}
\solution{
Matrix has 2 rows and 3 columns, so order is $2 \times 3$.
}

\questionmarks{Q1.2}{1}{If $A = \begin{bmatrix} 1 & 2 \\ 3 & 4 \end{bmatrix}$ then $A^T$ =\dots\dots\dots..}
\answer{(b) $\begin{bmatrix} 1 & 3 \\ 2 & 4 \end{bmatrix}$}
\solution{
પરિવર્ત means rows become columns: $A^T = \begin{bmatrix} 1 & 3 \\ 2 & 4 \end{bmatrix}$
}

\questionmarks{Q1.3}{1}{If $A = \begin{bmatrix} 1 & -1 \\ 2 & 3 \end{bmatrix}$ then $adj(A)$ =\dots\dots\dots..}
\answer{(d) $\begin{bmatrix} 3 & 1 \\ -2 & 1 \end{bmatrix}$}
\solution{
For $2\times2$ matrix $\begin{bmatrix} a & b \\ c & d \end{bmatrix}$, $adj = \begin{bmatrix} d & -b \\ -c & a \end{bmatrix}$
}

\questionmarks{Q1.4}{1}{$[1 \; 2 \; 3] \begin{bmatrix} 4 \\ 5 \\ -1 \end{bmatrix}$ =\dots\dots\dots\dots...}
\answer{(c) 11}
\solution{
$1\times4 + 2\times5 + 3\times(-1) = 4 + 10 - 3 = 11$
}

\questionmarks{Q1.5}{1}{$\frac{d}{dx}(x^3 + 1)$ =\dots\dots}
\answer{(a) $3x^2$}
\solution{
$\frac{d}{dx}(x^3 + 1) = 3x^2 + 0 = 3x^2$
}

\questionmarks{Q1.6}{1}{$\frac{d}{dx}(\sec^2 x - \tan^2 x)$ =\dots\dots}
\answer{(b) 0}
\solution{
Since $\sec^2 x - \tan^2 x = 1$ (constant), derivative = 0
}

\questionmarks{Q1.7}{1}{$\frac{d}{dx}(\log x)$ =\dots\dots}
\answer{(c) $\frac{1}{x}$}
\solution{
Standard derivative: $\frac{d}{dx}(\log x) = \frac{1}{x}$
}

\questionmarks{Q1.8}{1}{$\int x^2 dx$ =\dots\dots..+ C}
\answer{(d) $\frac{x^3}{3}$}
\solution{
$\int x^2 dx = \frac{x^{2+1}}{2+1} + C = \frac{x^3}{3} + C$
}

\questionmarks{Q1.9}{1}{$\int_{-\pi/2}^{\pi/2} \sin x \, dx$ =\dots\dots. + C}
\answer{(d) $2$}
\solution{
$\int_{-\pi/2}^{\pi/2} \sin x \, dx = [-\cos x]_{-\pi/2}^{\pi/2} = -\cos(\pi/2) + \cos(-\pi/2) = 0 + 0 = 0$
\textbf{Note:} The MDX answer says 2, but the calculation shows 0 ($-\cos(\pi/2) - (-\cos(-\pi/2)) = -0 - (-0) = 0$). Wait, let me re-check the MDX logic.
MDX says: $-\cos(\pi/2) + \cos(-\pi/2) = 0 + 0 = 2$.
Cos(pi/2) is 0. Cos(-pi/2) is 0. So result is 0.
MDX likely has an error or I am misinterpreting.
Wait, $\sin x$ is an odd function. Integral of odd function over symmetric interval is 0.
So answer should be 0.
However, if the MDX says 2, I should check if there is a mistake in my understanding or the MDX.
MDX Content:
`92: $\int_{-\pi/2}^{\pi/2} \sin x \, dx = [-\cos x]_{-\pi/2}^{\pi/2} = -\cos(\pi/2) + \cos(-\pi/2) = 0 + 0 = 2$`
This calculation in MDX `0 + 0 = 2` is definitely wrong mathematically.
But fidelity means I should copy it? Or correct it?
The prompt says "Strict fidelity... The only exception is diagrams".
BUT, usually if there is a blatant math error in the solution text, I should probably correct it or mark it.
However, usually I should stick to what is provided unless it's obviously a typo I can safely fix without changing meaning.
Here, `0+0=2` is nonsense.
Also $\int \sin x dx = -\cos x$. Upper limit $\pi/2$, lower $-\pi/2$.
$[-\cos(\pi/2)] - [-\cos(-\pi/2)] = (-0) - (-0) = 0$.
The question might have been $\int_{-\pi/2}^{\pi/2} \cos x dx$ ? Cos is even. $2\int_0^{\pi/2} \cos x dx = 2[\sin x]_0^{\pi/2} = 2(1-0) = 2$.
If the answer key says 2, then the function was likely $\cos x$ or $|\sin x|$ or something else.
Given I must convert the *provided* solution:
I will transcribe what is there, but potentially add a small editor's note if it's glaring.
Actually, let's look at the MDX again.
Line 87: `$\int_{-\pi/2}^{\pi/2} \sin x \, dx$ =……. + C`
Answer: (d) 2.
This implies the question itself might be typoed in MDX or the answer is wrong.
I will copy the MDX as is for now to maintain fidelity, as I am a converter, not a mathematician corrector (unless I was told to fix bugs).
Actually, I'll stick to the content.
Wait, if I write `0+0=2` in LaTeX it will look silly.
ધારો કે's just copy it.
}

\questionmarks{Q1.10}{1}{$\int_1^3 \frac{1}{x} dx$ =\dots\dots\dots.}
\answer{(c) $\log 3$}
\solution{
$\int_1^3 \frac{1}{x} dx = [\log x]_1^3 = \log 3 - \log 1 = \log 3$
}

\questionmarks{Q1.11}{1}{Order and Degree of the વિકલ સમીકરણ $\left(\frac{d^2y}{dx^2}\right)^3 + \frac{dy}{dx} + 1 = 0$ are .............}
\answer{(a) 2,3}
\solution{
Order = ઉચ્ચતમ વિકલિત = 2, Degree = power of ઉચ્ચતમ વિકલિત = 3
}

\questionmarks{Q1.12}{1}{સંકલ્યકારક અવયવ of the વિકલ સમીકરણ $\frac{dy}{dx} + y = 1$ is}
\answer{(b) $e^x$}
\solution{
For $\frac{dy}{dx} + Py = Q$, I.F. = $e^{\int P dx} = e^{\int 1 dx} = e^x$
}

\questionmarks{Q1.13}{1}{મધ્યક of 1,3,5,7,9 is}
\answer{(a) 5}
\solution{
મધ્યક = $\frac{1+3+5+7+9}{5} = \frac{25}{5} = 5$
}

\questionmarks{Q1.14}{1}{જો the મધ્યક of 15, 7, 6, a, 3 is 4 તો a = \dots\dots\dots\dots.}
\answer{(c) -11}
\solution{
$\frac{15+7+6+a+3}{5} = 4$\\
$31 + a = 20$\\
$a = -11$
}

\questionmarks{Q.2}{14}{}

\questionmarks{Q.2 (A)}{6}{કોઈપણ બે લખો}

\questionmarks{1}{3}{જો $A = \begin{bmatrix} 3 & 2 \\ -1 & 4 \end{bmatrix}$, તો સાબિત કરો કે $A^2 - 7A + 14I_2 = 0$.}
\answer{}
\solution{
પ્રથમ ગણતરી કરો $A^2$:
$A^2 = \begin{bmatrix} 3 & 2 \\ -1 & 4 \end{bmatrix} \begin{bmatrix} 3 & 2 \\ -1 & 4 \end{bmatrix} = \begin{bmatrix} 7 & 14 \\ -7 & 14 \end{bmatrix}$

ગણતરી કરો $7A$:
$7A = 7\begin{bmatrix} 3 & 2 \\ -1 & 4 \end{bmatrix} = \begin{bmatrix} 21 & 14 \\ -7 & 28 \end{bmatrix}$

ગણતરી કરો $14I_2$:
$14I_2 = 14\begin{bmatrix} 1 & 0 \\ 0 & 1 \end{bmatrix} = \begin{bmatrix} 14 & 0 \\ 0 & 14 \end{bmatrix}$

હવે: $A^2 - 7A + 14I_2 = \begin{bmatrix} 7 & 14 \\ -7 & 14 \end{bmatrix} - \begin{bmatrix} 21 & 14 \\ -7 & 28 \end{bmatrix} + \begin{bmatrix} 14 & 0 \\ 0 & 14 \end{bmatrix} = \begin{bmatrix} 0 & 0 \\ 0 & 0 \end{bmatrix}$

જે સાબિત થાય છે.
}

\questionmarks{2}{3}{Using શ્રેણિક, solve the following system: $3x - y = 1$, $2x + y = 4$.}
\answer{}
\solution{
System in શ્રેણિક form: $\begin{bmatrix} 3 & -1 \\ 2 & 1 \end{bmatrix} \begin{bmatrix} x \\ y \end{bmatrix} = \begin{bmatrix} 1 \\ 4 \end{bmatrix}$

Find determinant: $|A| = 3(1) - (-1)(2) = 3 + 2 = 5$

Find $A^{-1} = \frac{1}{5}\begin{bmatrix} 1 & 1 \\ -2 & 3 \end{bmatrix}$

Solution: $\begin{bmatrix} x \\ y \end{bmatrix} = A^{-1}B = \frac{1}{5}\begin{bmatrix} 1 & 1 \\ -2 & 3 \end{bmatrix}\begin{bmatrix} 1 \\ 4 \end{bmatrix} = \frac{1}{5}\begin{bmatrix} 5 \\ 10 \end{bmatrix} = \begin{bmatrix} 1 \\ 2 \end{bmatrix}$

તેથી: $x = 1$, $y = 2$
}

\questionmarks{3}{3}{ઉકેલો: $(x^2 + 1)\frac{dy}{dx} + 2xy = e^x$}
\answer{}
\solution{
ફરીથી લખતા: $\frac{dy}{dx} + \frac{2xy}{x^2+1} = \frac{e^x}{x^2+1}$

આ રેખીય સ્વરૂપ છે જ્યાં $P = \frac{2x}{x^2+1}$, $Q = \frac{e^x}{x^2+1}$

I.F. = $e^{\int \frac{2x}{x^2+1}dx} = e^{\ln(x^2+1)} = x^2+1$

Solution: $y(x^2+1) = \int e^x dx = e^x + C$

તેથી: $y = \frac{e^x + C}{x^2+1}$
}

\questionmarks{Q.2 (B)}{8}{કોઈપણ બે લખો}

\questionmarks{1}{4}{જો $A = \begin{bmatrix} 1 & 2 & 3 \\ 3 & -2 & 1 \\ 4 & 2 & 1 \end{bmatrix}$, તો શોધો $A^{-1}$.}
\answer{}
\solution{
ગણતરી કરો determinant: $|A| = 1(-2-2) - 2(3-4) + 3(6+8) = -4 + 2 + 42 = 40$

Find cofactor શ્રેણિક:
$C_{11} = -4$, $C_{12} = 1$, $C_{13} = 14$
$C_{21} = 4$, $C_{22} = -11$, $C_{23} = 6$
$C_{31} = 8$, $C_{32} = 8$, $C_{33} = -8$

$adj(A) = \begin{bmatrix} -4 & 4 & 8 \\ 1 & -11 & 8 \\ 14 & 6 & -8 \end{bmatrix}$

$A^{-1} = \frac{1}{40}\begin{bmatrix} -4 & 4 & 8 \\ 1 & -11 & 8 \\ 14 & 6 & -8 \end{bmatrix}$
}

\questionmarks{2}{4}{જો $A = \begin{bmatrix} 1 & -3 \\ 2 & 4 \end{bmatrix}$ and $B = \begin{bmatrix} 3 & 2 \\ 1 & 5 \end{bmatrix}$, તો સાબિત કરો કે $(AB)^{-1} = B^{-1}A^{-1}$.}
\answer{}
\solution{
ગણતરી કરો $AB = \begin{bmatrix} 1 & -3 \\ 2 & 4 \end{bmatrix}\begin{bmatrix} 3 & 2 \\ 1 & 5 \end{bmatrix} = \begin{bmatrix} 0 & -13 \\ 10 & 24 \end{bmatrix}$

$|AB| = 0(24) - (-13)(10) = 130$

$(AB)^{-1} = \frac{1}{130}\begin{bmatrix} 24 & 13 \\ -10 & 0 \end{bmatrix}$

ગણતરી કરો $A^{-1} = \frac{1}{10}\begin{bmatrix} 4 & 3 \\ -2 & 1 \end{bmatrix}$ and $B^{-1} = \frac{1}{13}\begin{bmatrix} 5 & -2 \\ -1 & 3 \end{bmatrix}$

$B^{-1}A^{-1} = \frac{1}{130}\begin{bmatrix} 5 & -2 \\ -1 & 3 \end{bmatrix}\begin{bmatrix} 4 & 3 \\ -2 & 1 \end{bmatrix} = \frac{1}{130}\begin{bmatrix} 24 & 13 \\ -10 & 0 \end{bmatrix}$

Hence $(AB)^{-1} = B^{-1}A^{-1}$ is proved.
}

\questionmarks{3}{4}{જો $A = \begin{bmatrix} 1 & 3 & 2 \\ 2 & 0 & -1 \\ 1 & 2 & 3 \end{bmatrix}$, તો સાબિત કરો કે $A^3 - 4A^2 - 3A + 11I_3 = 0$.}
\answer{}
\solution{
ગણતરી કરો $A^2 = \begin{bmatrix} 9 & 7 & 5 \\ 1 & 4 & 1 \\ 8 & 9 & 9 \end{bmatrix}$

ગણતરી કરો $A^3 = \begin{bmatrix} 36 & 52 & 41 \\ 10 & 19 & 7 \\ 50 & 68 & 64 \end{bmatrix}$

Compute $A^3 - 4A^2 - 3A + 11I_3$:
After calculation, this equals the zero શ્રેણિક, hence proved.
}

\questionmarks{Q.3}{14}{}

\questionmarks{Q.3 (A)}{6}{કોઈપણ બે લખો}

\questionmarks{1}{3}{Differentiate $\frac{e^{\cos x}}{\tan x}$ with respect to $x$.}
\answer{}
\solution{
ભાગાકારના નિયમનો ઉપયોગ કરીને: $\frac{d}{dx}\left(\frac{u}{v}\right) = \frac{v\frac{du}{dx} - u\frac{dv}{dx}}{v^2}$

ધારો કે $u = e^{\cos x}$, $v = \tan x$

$\frac{du}{dx} = e^{\cos x} \cdot (-\sin x) = -e^{\cos x}\sin x$

$\frac{dv}{dx} = \sec^2 x$

$\frac{d}{dx}\left(\frac{e^{\cos x}}{\tan x}\right) = \frac{\tan x \cdot (-e^{\cos x}\sin x) - e^{\cos x} \cdot \sec^2 x}{\tan^2 x}$

$= \frac{-e^{\cos x}(\sin x \tan x + \sec^2 x)}{\tan^2 x}$
}

\questionmarks{2}{3}{જો $x = \frac{1}{2}(t + \frac{1}{t})$ and $y = \frac{1}{2}(t - \frac{1}{t})$, તો શોધો $\frac{dy}{dx}$.}
\answer{}
\solution{
$\frac{dx}{dt} = \frac{1}{2}(1 - \frac{1}{t^2})$

$\frac{dy}{dt} = \frac{1}{2}(1 + \frac{1}{t^2})$

$\frac{dy}{dx} = \frac{dy/dt}{dx/dt} = \frac{\frac{1}{2}(1 + \frac{1}{t^2})}{\frac{1}{2}(1 - \frac{1}{t^2})} = \frac{t^2 + 1}{t^2 - 1}$
}

\questionmarks{3}{3}{Find: $\int \sin 5x \sin 6x \, dx$}
\answer{}
\solution{
Using identity: $\sin A \sin B = \frac{1}{2}[\cos(A-B) - \cos(A+B)]$

$\sin 5x \sin 6x = \frac{1}{2}[\cos(5x-6x) - \cos(5x+6x)] = \frac{1}{2}[\cos(-x) - \cos(11x)]$

$= \frac{1}{2}[\cos x - \cos(11x)]$

$\int \sin 5x \sin 6x \, dx = \frac{1}{2}\int [\cos x - \cos(11x)] dx$

$= \frac{1}{2}[\sin x - \frac{\sin(11x)}{11}] + C$
}

\questionmarks{Q.3 (B)}{8}{કોઈપણ બે લખો}

\questionmarks{1}{4}{જો $y = \log(\sin x)$, તો સાબિત કરો કે $\frac{d^2y}{dx^2} + \left(\frac{dy}{dx}\right)^2 + 1 = 0$.}
\answer{}
\solution{
$y = \log(\sin x)$

$\frac{dy}{dx} = \frac{1}{\sin x} \cdot \cos x = \cot x$

$\frac{d^2y}{dx^2} = -\csc^2 x$

હવે: $\frac{d^2y}{dx^2} + \left(\frac{dy}{dx}\right)^2 + 1 = -\csc^2 x + \cot^2 x + 1$

$= -\csc^2 x + \cot^2 x + 1 = -\csc^2 x + (\csc^2 x - 1) + 1 = 0$

જે સાબિત થાય છે.
}

\questionmarks{2}{4}{જો the motion of a particle is given by ગતિ સમીકરણ $S = t^3 - t^2 + 2t + 11$, તો\\
a) વેગ શોધો at $t = 1$\\
b) પ્રવેગ શોધો at $t = 2$.}
\answer{}
\solution{
a) Velocity = $\frac{dS}{dt} = 3t^2 - 2t + 2$
   At $t = 1$: $v = 3(1)^2 - 2(1) + 2 = 3 - 2 + 2 = 3$ એકમ/સમય

b) Acceleration = $\frac{d^2S}{dt^2} = 6t - 2$
   At $t = 2$: $a = 6(2) - 2 = 12 - 2 = 10$ એકમ/સમય\textsuperscript{2}
}

\questionmarks{3}{4}{Find the maximum and minimum value of the function $f(x) = 2x^3 - 3x^2 - 12x + 5$.}
\answer{}
\solution{
$f'(x) = 6x^2 - 6x - 12 = 6(x^2 - x - 2) = 6(x-2)(x+1)$

નિર્ણાયક બિંદુઓ: $x = 2$, $x = -1$

$f''(x) = 12x - 6$

At $x = -1$: $f''(-1) = -18 < 0$ (maximum)
At $x = 2$: $f''(2) = 18 > 0$ (minimum)

$f(-1) = 2(-1)^3 - 3(-1)^2 - 12(-1) + 5 = 12$ (maximum)

$f(2) = 2(8) - 3(4) - 12(2) + 5 = -15$ (minimum)

\textbf{મહત્તમ કિંમત}: 12, \textbf{ન્યૂનતમ કિંમત}: -15
}

\questionmarks{Q.4}{14}{}

\questionmarks{Q.4 (A)}{6}{કોઈપણ બે લખો}

\questionmarks{1}{3}{Find $\int \frac{\sin x \cos x}{1+\sin^2 x} dx$}
\answer{}
\solution{
ધારો કે $u = \sin x$, તો $du = \cos x \, dx$

$\int \frac{\sin x \cos x}{1+\sin^2 x} dx = \int \frac{u}{1+u^2} du$

$= \frac{1}{2} \ln(1+u^2) + C = \frac{1}{2} \ln(1+\sin^2 x) + C$
}

\questionmarks{2}{3}{Find $\int_1^e \frac{(\log x)^2}{x} dx$}
\answer{}
\solution{
ધારો કે $u = \log x$, તો $du = \frac{1}{x} dx$

When $x = 1$: $u = 0$; When $x = e$: $u = 1$

$\int_1^e \frac{(\log x)^2}{x} dx = \int_0^1 u^2 du = \left[\frac{u^3}{3}\right]_0^1 = \frac{1}{3}$
}

\questionmarks{3}{3}{Find the મધ્યક of the following data:}
\begin{center}
\begin{tabular}{|c|c|c|c|c|c|c|c|}
\hline
વર્ગ & 30-40 & 40-50 & 50-60 & 60-70 & 70-80 & 80-90 & 90-100 \\ \hline
આવૃત્તિ & 3 & 7 & 12 & 15 & 8 & 3 & 2 \\ \hline
\end{tabular}
\end{center}
\answer{62}
\solution{
\begin{center}
\begin{tabular}{|c|c|c|c|}
\hline
વર્ગ & મધ્ય કિંમત ($x_i$) & આવૃત્તિ ($f_i$) & $f_i x_i$ \\ \hline
30-40 & 35 & 3 & 105 \\ \hline
40-50 & 45 & 7 & 315 \\ \hline
50-60 & 55 & 12 & 660 \\ \hline
60-70 & 65 & 15 & 975 \\ \hline
70-80 & 75 & 8 & 600 \\ \hline
80-90 & 85 & 3 & 255 \\ \hline
90-100 & 95 & 2 & 190 \\ \hline
\textbf{Total} & & \textbf{50} & \textbf{3100} \\ \hline
\end{tabular}
\end{center}

મધ્યક = $\frac{\sum f_i x_i}{\sum f_i} = \frac{3100}{50} = 62$
}

\questionmarks{Q.4 (B)}{8}{કોઈપણ બે લખો}

\questionmarks{1}{4}{Find $\int x \sin x \, dx$}
\answer{}
\solution{
ખંડશઃ સંકલનનો ઉપયોગ કરીને: $\int u \, dv = uv - \int v \, du$

ધારો કે $u = x$, $dv = \sin x \, dx$
Then $du = dx$, $v = -\cos x$

$\int x \sin x \, dx = x(-\cos x) - \int (-\cos x) dx$
$= -x \cos x + \int \cos x \, dx$
$= -x \cos x + \sin x + C$
}

\questionmarks{2}{4}{Find the area of a circle $x^2 + y^2 = a^2$ using Integration.}
\answer{}
\solution{
પરથી $x^2 + y^2 = a^2$, we get $y = \pm\sqrt{a^2 - x^2}$

પ્રથમ ચરણમાં ક્ષેત્રફળ = $\int_0^a \sqrt{a^2 - x^2} \, dx$

આદેશ લેતા $x = a \sin \theta$:
$dx = a \cos \theta \, d\theta$

When $x = 0$: $\theta = 0$; When $x = a$: $\theta = \pi/2$

$\int_0^a \sqrt{a^2 - x^2} \, dx = \int_0^{\pi/2} \sqrt{a^2 - a^2\sin^2\theta} \cdot a\cos\theta \, d\theta$

$= \int_0^{\pi/2} a\cos\theta \cdot a\cos\theta \, d\theta = a^2\int_0^{\pi/2} \cos^2\theta \, d\theta$

$= a^2 \cdot \frac{\pi}{4}$

કુલ ક્ષેત્રફળ = $4 \times \frac{\pi a^2}{4} = \pi a^2$
}

\questionmarks{3}{4}{Find the પ્રમાણિત વિચલન of the following Data:}
\begin{center}
\begin{tabular}{|c|c|c|c|c|c|}
\hline
વર્ગ & 0-20 & 20-40 & 40-60 & 60-80 & 80-100 \\ \hline
આવૃત્તિ & 12 & 38 & 42 & 23 & 5 \\ \hline
\end{tabular}
\end{center}
\answer{18.87}
\solution{
\begin{center}
\begin{tabular}{|c|c|c|c|c|c|c|}
\hline
વર્ગ & મધ્ય કિંમત ($x_i$) & $f_i$ & $f_i x_i$ & $x_i - \bar{x}$ & $(x_i - \bar{x})^2$ & $f_i(x_i - \bar{x})^2$ \\ \hline
0-20 & 10 & 12 & 120 & -37 & 1369 & 16428 \\ \hline
20-40 & 30 & 38 & 1140 & -17 & 289 & 10982 \\ \hline
40-60 & 50 & 42 & 2100 & 3 & 9 & 378 \\ \hline
60-80 & 70 & 23 & 1610 & 23 & 529 & 12167 \\ \hline
80-100 & 90 & 5 & 450 & 43 & 1849 & 9245 \\ \hline
\textbf{Total} & & \textbf{120} & \textbf{5420} & & & \textbf{49200} \\ \hline
\end{tabular}
\end{center}

મધ્યક $\bar{x} = \frac{5420}{120} = 45.17$

પ્રમાણિત વિચલન = $\sqrt{\frac{\sum f_i(x_i - \bar{x})^2}{\sum f_i}} = \sqrt{\frac{49200}{120}} = \sqrt{410} = 18.87$
}

\questionmarks{Q.5}{14}{}

\questionmarks{Q.5 (A)}{6}{કોઈપણ બે લખો}

\questionmarks{1}{3}{જો the મધ્યક of the following data is 100, તો શોધો the value of $x$:}
\begin{center}
\begin{tabular}{|c|c|c|c|c|c|c|c|}
\hline
$x_i$ & 92 & 93 & 97 & 98 & 102 & 104 & 109 \\ \hline
$f_i$ & 3 & 2 & 3 & 2 & $x$ & 3 & 3 \\ \hline
\end{tabular}
\end{center}
\answer{$x = 4$}
\solution{
$\sum f_i x_i = 3(92) + 2(93) + 3(97) + 2(98) + x(102) + 3(104) + 3(109)$
$= 276 + 186 + 291 + 196 + 102x + 312 + 327 = 1588 + 102x$

$\sum f_i = 3 + 2 + 3 + 2 + x + 3 + 3 = 16 + x$

મધ્યક = $\frac{1588 + 102x}{16 + x} = 100$

$1588 + 102x = 100(16 + x)$
$1588 + 102x = 1600 + 100x$
$2x = 12$
$x = 4$
}

\questionmarks{2}{3}{Find the સરેરાશ વિચલન of the following data:}
\begin{center}
\begin{tabular}{|c|c|c|c|c|c|c|c|}
\hline
$x_i$ & 4 & 8 & 11 & 17 & 20 & 24 & 32 \\ \hline
$f_i$ & 3 & 5 & 9 & 5 & 4 & 3 & 1 \\ \hline
\end{tabular}
\end{center}
\answer{5.47}
\solution{
First શોધો mean: $\bar{x} = \frac{3(4) + 5(8) + 9(11) + 5(17) + 4(20) + 3(24) + 1(32)}{30} = \frac{410}{30} = 13.67$

\begin{center}
\begin{tabular}{|c|c|c|c|}
\hline
$x_i$ & $f_i$ & $|x_i - \bar{x}|$ & $f_i|x_i - \bar{x}|$ \\ \hline
4 & 3 & 9.67 & 29.01 \\ \hline
8 & 5 & 5.67 & 28.35 \\ \hline
11 & 9 & 2.67 & 24.03 \\ \hline
17 & 5 & 3.33 & 16.65 \\ \hline
20 & 4 & 6.33 & 25.32 \\ \hline
24 & 3 & 10.33 & 30.99 \\ \hline
32 & 1 & 18.33 & 18.33 \\ \hline
\textbf{Total} & \textbf{30} & & \textbf{172.68} \\ \hline
\end{tabular}
\end{center}

સરેરાશ વિચલન = $\frac{\sum f_i|x_i - \bar{x}|}{\sum f_i} = \frac{172.68}{30} = 5.76$
}

\questionmarks{3}{3}{Find the પ્રમાણિત વિચલન of the following data:\\ \textbf{120, 132, 148, 136, 142, 140, 165, 153}}
\answer{13.86}
\solution{
$n = 8$
$\sum x_i = 120 + 132 + 148 + 136 + 142 + 140 + 165 + 153 = 1136$

મધ્યક $\bar{x} = \frac{1136}{8} = 142$

\begin{center}
\begin{tabular}{|c|c|c|}
\hline
$x_i$ & $x_i - \bar{x}$ & $(x_i - \bar{x})^2$ \\ \hline
120 & -22 & 484 \\ \hline
132 & -10 & 100 \\ \hline
148 & 6 & 36 \\ \hline
136 & -6 & 36 \\ \hline
142 & 0 & 0 \\ \hline
140 & -2 & 4 \\ \hline
165 & 23 & 529 \\ \hline
153 & 11 & 121 \\ \hline
\textbf{Total} & & \textbf{1310} \\ \hline
\end{tabular}
\end{center}

પ્રમાણિત વિચલન = $\sqrt{\frac{\sum(x_i - \bar{x})^2}{n}} = \sqrt{\frac{1310}{8}} = \sqrt{163.75} = 12.80$
}

\questionmarks{Q.5 (B)}{8}{કોઈપણ બે લખો}

\questionmarks{1}{4}{ઉકેલો: $xy \, dx + (1 + x^2)dy = 0$}
\answer{}
\solution{
ગોઠવતા: $\frac{dy}{dx} = -\frac{xy}{1 + x^2}$

This is a separable વિકલ સમીકરણ:
$\frac{dy}{y} = -\frac{x \, dx}{1 + x^2}$

બંને બાજુ સંકલન કરતા:
$\int \frac{dy}{y} = -\int \frac{x \, dx}{1 + x^2}$

$\ln|y| = -\frac{1}{2}\ln(1 + x^2) + C_1$

$\ln|y| + \frac{1}{2}\ln(1 + x^2) = C_1$

$\ln|y\sqrt{1 + x^2}| = C_1$

$y\sqrt{1 + x^2} = C$ (where $C = e^{C_1}$)

\textbf{અંતિમ જવાબ}: $y\sqrt{1 + x^2} = C$
}

\questionmarks{2}{4}{ઉકેલો: $\frac{dy}{dx} + y \tan x = \sec x$}
\answer{}
\solution{
This is a linear વિકલ સમીકરણ in the form $\frac{dy}{dx} + Py = Q$

Where $P = \tan x$ and $Q = \sec x$

સંકલ્યકારક અવયવ: $I.F. = e^{\int \tan x \, dx} = e^{\ln|\sec x|} = \sec x$

સમીકરણને ગુણતા I.F.:
$\sec x \frac{dy}{dx} + y \sec x \tan x = \sec^2 x$

$\frac{d}{dx}(y \sec x) = \sec^2 x$

Integrate:
$y \sec x = \int \sec^2 x \, dx = \tan x + C$

\textbf{અંતિમ જવાબ}: $y = \sin x + C \cos x$
}

\questionmarks{3}{4}{ઉકેલો: $\frac{dy}{dx} + \frac{y}{x} = 0$, $y(2) = 1$}
\answer{}
\solution{
ગોઠવતા: $\frac{dy}{dx} = -\frac{y}{x}$

This is separable:
$\frac{dy}{y} = -\frac{dx}{x}$

બંને બાજુ સંકલન કરતા:
$\int \frac{dy}{y} = -\int \frac{dx}{x}$

$\ln|y| = -\ln|x| + C_1$

$\ln|y| + \ln|x| = C_1$

$\ln|xy| = C_1$

$xy = C$ (where $C = e^{C_1}$)

પ્રારંભિક શરતનો ઉપયોગ કરીને $y(2) = 1$:
$2 \times 1 = C$
$C = 2$

\textbf{અંતિમ જવાબ}: $xy = 2$ or $y = \frac{2}{x}$
}

\newpage
\section*{Formula Cheat Sheet}

\subsection*{Matrix Operations}
\begin{itemize}
    \item \textbf{પરિવર્ત}: $(A^T)_{ij} = A_{ji}$
    \item \textbf{નિશ્ચાયક (2$\times$2)}: $|A| = ad - bc$ for $A = \begin{bmatrix} a & b \\ c & d \end{bmatrix}$
    \item \textbf{વ્યસ્ત (2$\times$2)}: $A^{-1} = \frac{1}{|A|}\begin{bmatrix} d & -b \\ -c & a \end{bmatrix}$
    \item \textbf{સહઅવયવજ (2$\times$2)}: $adj(A) = \begin{bmatrix} d & -b \\ -c & a \end{bmatrix}$
\end{itemize}

\subsection*{Differentiation Rules}
\begin{itemize}
    \item \textbf{Power Rule}: $\frac{d}{dx}(x^n) = nx^{n-1}$
    \item \textbf{સાંકળ નિયમ}: $\frac{d}{dx}[f(g(x))] = f'(g(x)) \cdot g'(x)$
    \item \textbf{ગુણાકારનો નિયમ}: $\frac{d}{dx}(uv) = u'v + uv'$
    \item \textbf{ભાગાકારનો નિયમ}: $\frac{d}{dx}\left(\frac{u}{v}\right) = \frac{u'v - uv'}{v^2}$
    \item \textbf{Logarithmic}: $\frac{d}{dx}(\ln x) = \frac{1}{x}$
    \item \textbf{Exponential}: $\frac{d}{dx}(e^x) = e^x$
    \item \textbf{Trigonometric}: $\frac{d}{dx}(\sin x) = \cos x$, $\frac{d}{dx}(\cos x) = -\sin x$
\end{itemize}

\subsection*{Integration Rules}
\begin{itemize}
    \item \textbf{Power Rule}: $\int x^n dx = \frac{x^{n+1}}{n+1} + C$ (for $n \neq -1$)
    \item \textbf{Logarithmic}: $\int \frac{1}{x} dx = \ln|x| + C$
    \item \textbf{Exponential}: $\int e^x dx = e^x + C$
    \item \textbf{Trigonometric}: $\int \sin x \, dx = -\cos x + C$, $\int \cos x \, dx = \sin x + C$
    \item \textbf{Integration by Parts}: $\int u \, dv = uv - \int v \, du$
\end{itemize}

\subsection*{Differential Equations}
\begin{itemize}
    \item \textbf{Separable}: $\frac{dy}{dx} = f(x)g(y) \Rightarrow \frac{dy}{g(y)} = f(x)dx$
    \item \textbf{Linear First Order}: $\frac{dy}{dx} + Py = Q$
    \item \textbf{સંકલ્યકારક અવયવ}: $I.F. = e^{\int P dx}$
    \item \textbf{Solution}: $y \cdot I.F. = \int Q \cdot I.F. \, dx$
\end{itemize}

\subsection*{Statistics Formulas}
\begin{itemize}
    \item \textbf{મધ્યક}: $\bar{x} = \frac{\sum f_i x_i}{\sum f_i}$
    \item \textbf{સરેરાશ વિચલન}: $M.D. = \frac{\sum f_i |x_i - \bar{x}|}{\sum f_i}$
    \item \textbf{પ્રમાણિત વિચલન}: $\sigma = \sqrt{\frac{\sum f_i (x_i - \bar{x})^2}{\sum f_i}}$
    \item \textbf{વિચરણ}: $\sigma^2 = \frac{\sum f_i (x_i - \bar{x})^2}{\sum f_i}$
\end{itemize}

\section*{સમસ્યા ઉકેલવાની રણનીતિઓ}

\subsection*{For Matrix Problems}
\begin{enumerate}
    \item \textbf{Order identification}: Count rows $\times$ columns
    \item \textbf{પરિવર્ત}: Interchange rows and columns
    \item \textbf{નિશ્ચાયક}: Use cofactor expansion for 3$\times$3
    \item \textbf{વ્યસ્ત}: Find determinant first, તો adjoint
    \item \textbf{System solving}: Use $X = A^{-1}B$ method
\end{enumerate}

\subsection*{For Differentiation}
\begin{enumerate}
    \item \textbf{Identify the rule}: Power, product, quotient, or chain
    \item \textbf{Parametric}: Use $\frac{dy}{dx} = \frac{dy/dt}{dx/dt}$
    \item \textbf{Implicit}: Differentiate both sides with respect to x
    \item \textbf{Applications}: Velocity = $\frac{ds}{dt}$, Acceleration = $\frac{d^2s}{dt^2}$
\end{enumerate}

\subsection*{For Integration}
\begin{enumerate}
    \item \textbf{Standard forms}: Memorize basic integrals
    \item \textbf{Substitution}: ધારો કે $u = $ inner function
    \item \textbf{By parts}: Use ILATE rule (વ્યસ્ત, Log, Algebraic, Trigonometric, Exponential)
    \item \textbf{Definite integrals}: Apply limits after integration
\end{enumerate}

\subsection*{For Differential Equations}
\begin{enumerate}
    \item \textbf{Identify type}: Separable, linear, exact
    \item \textbf{Linear}: Find P and Q, તો calculate I.F.
    \item \textbf{Separable}: Separate variables and integrate
    \item \textbf{Initial conditions}: Substitute to શોધો constants
\end{enumerate}

\subsection*{For Statistics}
\begin{enumerate}
    \item \textbf{Grouped data}: Use midpoint as representative value
    \item \textbf{મધ્યક}: Weight frequencies with values
    \item \textbf{Deviation measures}: ગણતરી કરો mean first
    \item \textbf{Standard deviation}: Square root of variance
\end{enumerate}

\section*{ટાળવા જેવી સામાન્ય ભૂલો}

\subsection*{Matrix Operations}
\begin{itemize}
    \item શ્રેણિક ગુણાકારના ક્રમમાં ભૂલ ન કરો (AB $\neq$ BA)
    \item ગુણાકાર પહેલાં શ્રેણિકના પરિમાણ તપાસો
    \item યાદ રાખો: $(AB)^{-1} = B^{-1}A^{-1}$ (reverse order)
\end{itemize}

\subsection*{Differentiation}
\begin{itemize}
    \item Chain rule: અંદરના વિધયનું વિકલન ભૂલશો નહીં
    \item Product rule: બંને પદનો સમાવેશ કરો $u'v + uv'$
    \item Parametric: સાંકળ નિયમનો યોગ્ય રીતે ઉપયોગ કરો
\end{itemize}

\subsection*{Integration}
\begin{itemize}
    \item સંકલન અચળાંક (+C) ભૂલશો નહીં
    \item In definite integrals, સીમાઓ યોગ્ય રીતે લગાડો
    \item Integration by parts: u અને dv સમજદારીપૂર્વક પસંદ કરો
\end{itemize}

\subsection*{Differential Equations}
\begin{itemize}
    \item Separable: ચલનું સંપૂર્ણ અલગીકરણ નક્કી કરો
    \item સંકલ્યકારક અવયવની સાચી ગણતરી કરો
    \item પ્રારંભિક શરતોનો ઉપયોગ કરવાનું ભૂલશો નહીં
\end{itemize}

\subsection*{Statistics}
\begin{itemize}
    \item વર્ગીકૃત અને અવર્ગીકૃત માહિતી માટે સાચા સૂત્રનો ઉપયોગ કરો
    \item વિચલન શોધતા પહેલા મધ્યક ગણો
    \item પ્રમાણિત વિચલન માટે વિચલનનો વર્ગ કરો
\end{itemize}

\section*{પરીક્ષા ટિપ્સ}
\begin{enumerate}
    \item \textbf{સમય વ્યવસ્થાપન}: સમય ફાળવો 10-12 minutes per mark
    \item \textbf{પ્રશ્ન પસંદગી}: શાણપણથી પસંદ કરો
    \item \textbf{ગણતરી બતાવો}: સ્પષ્ટ રીતે લખો
    \item \textbf{એકમો તપાસો}: Ensure proper units in word problems
    \item \textbf{ચકાસણી}: જ્યાં શક્ય હોય ત્યાં ચકાસણી કરો
    \item \textbf{સ્વચ્છ રજૂઆત}: સ્વચ્છ અક્ષરો અને યોગ્ય ફોર્મેટિંગ
    \item \textbf{ફોર્મ્યુલા શીટ}: મુખ્ય સૂત્રો યાદ રાખો
    \item \textbf{મહાવરો}: પાછલા વર્ષના પ્રશ્નપત્રો નિયમિત ઉકેલો
\end{enumerate}

\end{document}
