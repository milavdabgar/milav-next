\documentclass{article}

% content/resources/templates/preamble.tex
\usepackage[margin=0.6in]{geometry}
\author{Milav Dabgar}
\usepackage{amsmath,amssymb,amsthm}
\usepackage{booktabs}
\usepackage{multirow}
\usepackage{xcolor}
\usepackage{tcolorbox}
\tcbuselibrary{breakable,skins}
\usepackage[colorlinks=true,linkcolor=blue]{hyperref}
\usepackage{titlesec}
\usepackage{enumitem}
\usepackage{tikz}
\usepackage{pgfplots}
\usepackage{circuitikz}
\usepackage[version=4]{mhchem}
\usepackage{longtable}
\usepackage{array}
\usepackage{float}
\usepackage{caption}
\usepackage{listings}

\lstset{
  basicstyle=\small\ttfamily,
  breaklines=true,
  breakatwhitespace=false,
  postbreak=\mbox{\textcolor{red}{$\hookrightarrow$}\space},
  float=false,
  numbers=left,
  numberstyle=\tiny\color{gray},
  numbersep=10pt,
  xleftmargin=2em,
  keywordstyle=\color{blue},
  commentstyle=\color{green!60!black},
  stringstyle=\color{purple},
  backgroundcolor=\color{gray!5},
  showstringspaces=false,
  tabsize=2,
  captionpos=b,
  keepspaces=true,
  columns=flexible
}

\pgfplotsset{compat=1.18}
\usetikzlibrary{shapes,arrows,positioning,calc,patterns,decorations.pathmorphing,decorations.markings,arrows.meta}

% Color scheme
\definecolor{headcolor}{RGB}{0,102,204}
\definecolor{keycolor}{RGB}{220,20,60}
\definecolor{solutioncolor}{RGB}{34,139,34}
\definecolor{mnemoniccolor}{RGB}{148,0,211}
\definecolor{codecolor}{RGB}{0,0,100}

% Spacing
\setlength{\parskip}{3pt}
\setlist[itemize]{nosep}
\setlist[enumerate]{nosep}

% Title formatting
\titleformat{\section}{\Large\bfseries\color{headcolor}}{\thesection}{1em}{}
\titleformat{\subsection}{\large\bfseries\color{headcolor}}{\thesubsection}{1em}{}

% Pandoc tightlist compatibility
\providecommand{\tightlist}{%
  \setlength{\itemsep}{0pt}\setlength{\parskip}{0pt}}

% Pandoc longtable compatibility
\newcounter{none}
\def\thenone{}


% content/resources/templates/gujarati-boxes.tex
\usepackage{fontspec}
\usepackage{polyglossia}

% Set Gujarati as main language (document is primarily in Gujarati)
% Note: gloss-gujarati.ldf doesn't exist in polyglossia, but it will use hyphenation patterns
\setdefaultlanguage{gujarati}
\setotherlanguage{english}

% Configure Gujarati font properly
% Use Language=Default to prevent polyglossia from trying to add language-specific features
% that don't exist for Gujarati, which causes "empty feature" warnings
\newfontfamily\gujaratifont[Script=Gujarati,AutoFakeBold=2.5,AutoFakeSlant=0.3]{Noto Sans Gujarati}
\setmainfont[Script=Gujarati,AutoFakeBold=2.5,AutoFakeSlant=0.3]{Noto Sans Gujarati}
% Use Noto Sans Gujarati for monospace to support Gujarati in text
\setmonofont[Scale=0.9]{Noto Sans Gujarati}

% Configure English to use the same font
\newfontfamily\englishfont[Script=Gujarati,AutoFakeBold=2.5,AutoFakeSlant=0.3]{Noto Sans Gujarati}

% Translations for polyglossia
\gappto\captionsgujarati{
  \renewcommand{\tablename}{કોષ્ટક}
  \renewcommand{\figurename}{આકૃતિ}
}

% Helper for TikZ nodes to ensure Gujarati font
\newcommand{\gu}[1]{{\gujaratifont #1}}

% Custom environments
\newtcolorbox{solutionbox}{
    breakable,
    enhanced,
    colback=solutioncolor!5!white,
    colframe=solutioncolor!75!black,
    fonttitle=\bfseries,
    title=જવાબ
}

\newtcolorbox{solutionboxnobreak}{
 colback=solutioncolor!5!white,
 colframe=solutioncolor!75!black,
 fonttitle=\bfseries,
 title=જવાબ
}

\newtcolorbox{keyformula}{
 breakable,
 enhanced,
 colback=keycolor!5!white,
 colframe=keycolor!75!black,
 fonttitle=\bfseries,
 title=રાસાયણિક સમીકરણ/સૂત્ર
}

\newtcolorbox{mnemonicbox}{
 breakable,
 enhanced,
 colback=mnemoniccolor!5!white,
 colframe=mnemoniccolor!75!black,
 fonttitle=\bfseries,
 title=મેમરી ટ્રીક
}


% Custom commands for GTU solutions
% This file defines semantic commands for consistent formatting

% Question command with automatic formatting
\newcommand{\question}[2]{%
  \section*{Question #1}%
  \textbf{#2}%
}

% OR question variant
\newcommand{\questionor}[2]{%
  \section*{Question #1 OR}%
  \textbf{#2}%
}

% Proper table environment with caption
\newenvironment{answertable}[1]{%
  \begin{table}[htbp]
  \centering
  \caption{#1}
}{%
  \end{table}
}

% Proper figure environment for diagrams
\newenvironment{answerdiagram}[1]{%
  \begin{figure}[htbp]
  \centering
  \caption{#1}
}{%
  \end{figure}
}

% Semantic markup for key terms
\newcommand{\keyword}[1]{\textbf{#1}}
\newcommand{\code}[1]{\texttt{#1}}
\newcommand{\classname}[1]{\texttt{#1}}
\newcommand{\methodname}[1]{\texttt{#1}}

% Proper quotation marks
\newcommand{\mnemonic}[1]{``#1''}


\title{Applied Mathematics (4320001) - સરવાળોmer 2022 Solution}
\date{August 23, 2022}

\begin{document}
\maketitle

\questionmarks{1}{14}{નીચેના વિકલ્પોમાંથી યોગ્ય વિકલ્પ પસંદ કરી ખાલી જગ્યા પૂરો.}

\questionmarks{1.1}{1}{જો $A = \begin{bmatrix} 1 & 2 \\ 3 & 4 \end{bmatrix}$ તો $A^2$ = .........}
\textbf{જવાબ}: (c) $\begin{bmatrix} 7 & 10 \\ 15 & 22 \end{bmatrix}$

\begin{solutionbox}
$A^2 = A \times A = \begin{bmatrix} 1 & 2 \\ 3 & 4 \end{bmatrix} \times \begin{bmatrix} 1 & 2 \\ 3 & 4 \end{bmatrix}$

$A^2 = \begin{bmatrix} 1(1)+2(3) & 1(2)+2(4) \\ 3(1)+4(3) & 3(2)+4(4) \end{bmatrix} = \begin{bmatrix} 7 & 10 \\ 15 & 22 \end{bmatrix}$
\end{solutionbox}

\questionmarks{1.2}{1}{જો $A = \begin{bmatrix} 1 & 3 \\ 4 & -2 \end{bmatrix}$ તો $2A - 2I$ = .........}
\textbf{જવાબ}: (a) $\begin{bmatrix} 0 & 6 \\ 8 & -6 \end{bmatrix}$

\begin{solutionbox}
$2A = 2\begin{bmatrix} 1 & 3 \\ 4 & -2 \end{bmatrix} = \begin{bmatrix} 2 & 6 \\ 8 & -4 \end{bmatrix}$

$2I = 2\begin{bmatrix} 1 & 0 \\ 0 & 1 \end{bmatrix} = \begin{bmatrix} 2 & 0 \\ 0 & 2 \end{bmatrix}$

$2A - 2I = \begin{bmatrix} 2 & 6 \\ 8 & -4 \end{bmatrix} - \begin{bmatrix} 2 & 0 \\ 0 & 2 \end{bmatrix} = \begin{bmatrix} 0 & 6 \\ 8 & -6 \end{bmatrix}$
\end{solutionbox}

\questionmarks{1.3}{1}{જો $A = \begin{bmatrix} -8 & -6 \\ 3 & 4 \end{bmatrix}$ તો $\text{Adj } A$ = .........}
\textbf{જવાબ}: (a) $\begin{bmatrix} 4 & 6 \\ -3 & -8 \end{bmatrix}$

\begin{solutionbox}
For a $2 \times 2$ matrix $\begin{bmatrix} a & b \\ c & d \end{bmatrix}$, $\text{Adj } A = \begin{bmatrix} d & -b \\ -c & a \end{bmatrix}$

$\text{Adj } A = \begin{bmatrix} 4 & 6 \\ -3 & -8 \end{bmatrix}$
\end{solutionbox}

\questionmarks{1.4}{1}{શ્રેણિકની કક્ષા $\begin{bmatrix} 5 & 2 & 20 & 41 & 0 \\ 15 & 4 & 30 & 40 & 1 \\ 25 & 6 & 40 & 39 & 2 \\ 35 & 8 & 50 & 38 & 3 \end{bmatrix}$ is .........}
\textbf{જવાબ}: (b) $4 \times 5$

\begin{solutionbox}
શ્રેણિકમાં 4 હાર અને 5 સ્તંભ છે, તેથી કક્ષા $4 \times 5$ છે.
\end{solutionbox}

\questionmarks{1.5}{1}{$\frac{d}{dx}(\cos^2 x + \sin^2 x)$ = .........}
\textbf{જવાબ}: (d) 0

\begin{solutionbox}
કારણ કે $\cos^2 x + \sin^2 x = 1$ (ત્રિકોણમિતીય નિત્યસમ)

$\frac{d}{dx}(1) = 0$
\end{solutionbox}

\questionmarks{1.6}{1}{જો $f(x) = \log x$ તો $f'(1)$ = .........}
\textbf{જવાબ}: (a) 1

\begin{solutionbox}
$f(x) = \log x \implies f'(x) = \frac{1}{x}$

$f'(1) = \frac{1}{1} = 1$
\end{solutionbox}

\questionmarks{1.7}{1}{જો $x^2 + y^2 = a^2$ તો $\frac{dy}{dx}$ = .........}
\textbf{જવાબ}: (b) $-\frac{x}{y}$

\begin{solutionbox}
બંને બાજુ $x$ ની સાપેક્ષે વિકલન કરતા:
$2x + 2y\frac{dy}{dx} = 0 \implies \frac{dy}{dx} = -\frac{x}{y}$
\end{solutionbox}

\questionmarks{1.8}{1}{$\int x^2 dx$ = ........}
\textbf{જવાબ}: (b) $\frac{x^3}{3}$

\begin{solutionbox}
$\int x^2 dx = \frac{x^{2+1}}{2+1} + c = \frac{x^3}{3} + c$
\end{solutionbox}

\questionmarks{1.9}{1}{$\int e^{x\log a} dx$ = ........}
\textbf{જવાબ}: (d) $\frac{a^x}{\log a}$

\begin{solutionbox}
$e^{x\log a} = a^x$

$\int a^x dx = \frac{a^x}{\log a} + c$
\end{solutionbox}

\questionmarks{1.10}{1}{$\int \cot x dx$ = ........}
\textbf{જવાબ}: (a) $\log|\sin x|$

\begin{solutionbox}
$\int \cot x dx = \int \frac{\cos x}{\sin x} dx$

ધારો કે $u = \sin x$, તો $du = \cos x dx$.
$\int \frac{du}{u} = \log|u| + c = \log|\sin x| + c$
\end{solutionbox}

\questionmarks{1.11}{1}{વિકલ સમીકરણની કક્ષા $\left(\frac{d^2y}{dx^2}\right)^4 + \left(\frac{d^2y}{dx^2}\right)^3 = 0$ is ........}
\textbf{જવાબ}: (b) 2

\begin{solutionbox}
અહીં ઉચ્ચતમ વિકલિત $\frac{d^2y}{dx^2}$ છે, જે દ્વિતીય વિકલિત છે.
માટે, the order is 2.
\end{solutionbox}

\questionmarks{1.12}{1}{સંકલ્યકારક અવયવ of વિકલ સમીકરણ $\frac{dy}{dx} + y = 3x$ is ........}
\textbf{જવાબ}: (c) $e^x$

\begin{solutionbox}
સુરેખ વિકલ સમીકરણ $\frac{dy}{dx} + Py = Q$ માટે, જ્યાં $P = 1$.
સંકલ્યકારક અવયવ = $e^{\int P dx} = e^{\int 1 dx} = e^x$
\end{solutionbox}

\questionmarks{1.13}{1}{જો given data is 6, 9, 7, 3, 8, 5, 4, 8, 7 and 8 તો mean is ........}
\textbf{જવાબ}: (b) 6.5

\begin{solutionbox}
મધ્યક = $\frac{\text{સરવાળો of all values}}{\text{Number of values}}$

સરવાળો = $6 + 9 + 7 + 3 + 8 + 5 + 4 + 8 + 7 + 8 = 65$

Number of values = 10. મધ્યક = $\frac{65}{10} = 6.5$
\end{solutionbox}

\questionmarks{1.14}{1}{The mean value of first eight natural numbers is ........}
\textbf{જવાબ}: (b) 4.5

\begin{solutionbox}
પ્રથમ આઠ પ્રાકૃતિક સંખ્યાઓ: 1, 2, 3, 4, 5, 6, 7, 8.
સરવાળો = $1 + 2 + 3 + 4 + 5 + 6 + 7 + 8 = 36$
મધ્યક = $\frac{36}{8} = 4.5$
\end{solutionbox}

\questionmarks{2(a)}{6}{કોઈપણ બે ગણો}

\questionmarks{2(a)(1)}{3}{જો $M = \begin{bmatrix} 2 & 3 \\ 1 & 0 \end{bmatrix}$, $N = \begin{bmatrix} 4 & 1 \\ 2 & -3 \end{bmatrix}$ તો સાબિત કરો કે $(M + N)^T = M^T + N^T$}

\begin{solutionbox}
$M + N = \begin{bmatrix} 2 & 3 \\ 1 & 0 \end{bmatrix} + \begin{bmatrix} 4 & 1 \\ 2 & -3 \end{bmatrix} = \begin{bmatrix} 6 & 4 \\ 3 & -3 \end{bmatrix}$

$(M + N)^T = \begin{bmatrix} 6 & 3 \\ 4 & -3 \end{bmatrix}$

$M^T = \begin{bmatrix} 2 & 1 \\ 3 & 0 \end{bmatrix}$, $N^T = \begin{bmatrix} 4 & 2 \\ 1 & -3 \end{bmatrix}$

$M^T + N^T = \begin{bmatrix} 2 & 1 \\ 3 & 0 \end{bmatrix} + \begin{bmatrix} 4 & 2 \\ 1 & -3 \end{bmatrix} = \begin{bmatrix} 6 & 3 \\ 4 & -3 \end{bmatrix}$

માટે, $(M + N)^T = M^T + N^T$. \textbf{સાબિત.}
\end{solutionbox}

\questionmarks{2(a)(2)}{3}{જો $A = \begin{bmatrix} 3 & 1 \\ -1 & 2 \end{bmatrix}$ તો સાબિત કરો કે $A^2 - 5A + 7I = 0$}

\begin{solutionbox}
$A^2 = \begin{bmatrix} 3 & 1 \\ -1 & 2 \end{bmatrix} \begin{bmatrix} 3 & 1 \\ -1 & 2 \end{bmatrix} = \begin{bmatrix} 8 & 5 \\ -5 & 3 \end{bmatrix}$

$5A = 5\begin{bmatrix} 3 & 1 \\ -1 & 2 \end{bmatrix} = \begin{bmatrix} 15 & 5 \\ -5 & 10 \end{bmatrix}$

$7I = 7\begin{bmatrix} 1 & 0 \\ 0 & 1 \end{bmatrix} = \begin{bmatrix} 7 & 0 \\ 0 & 7 \end{bmatrix}$

$A^2 - 5A + 7I = \begin{bmatrix} 8 & 5 \\ -5 & 3 \end{bmatrix} - \begin{bmatrix} 15 & 5 \\ -5 & 10 \end{bmatrix} + \begin{bmatrix} 7 & 0 \\ 0 & 7 \end{bmatrix}$

$= \begin{bmatrix} 8-15+7 & 5-5+0 \\ -5+5+0 & 3-10+7 \end{bmatrix} = \begin{bmatrix} 0 & 0 \\ 0 & 0 \end{bmatrix}$

માટે, $A^2 - 5A + 7I = 0$. \textbf{સાબિત.}
\end{solutionbox}

\questionmarks{2(a)(3)}{3}{ઉકેલો વિકલ સમીકરણ $\frac{dy}{dx} + x^2 e^{-y} = 0$}

\begin{solutionbox}
$\frac{dy}{dx} = -x^2 e^{-y} \implies e^y dy = -x^2 dx$

બંને બાજુ સંકલન કરતા:
$\int e^y dy = \int -x^2 dx$

$e^y = -\frac{x^3}{3} + C$

$y = \log\left(-\frac{x^3}{3} + C\right)$
\end{solutionbox}

\questionmarks{2(b)}{8}{કોઈપણ બે ગણો}

\questionmarks{2(b)(1)}{4}{ઉકેલો $-5y + 3x = 1$, $x + 2y - 4 = 0$ શ્રેણિકની મદદથી}

\begin{solutionbox}
સમીકરણોને ફરીથી લખતા:
$3x - 5y = 1$
$x + 2y = 4$

શ્રેણિક સ્વરૂપમાં: $\begin{bmatrix} 3 & -5 \\ 1 & 2 \end{bmatrix} \begin{bmatrix} x \\ y \end{bmatrix} = \begin{bmatrix} 1 \\ 4 \end{bmatrix}$

ધારો કે $A = \begin{bmatrix} 3 & -5 \\ 1 & 2 \end{bmatrix}$

$|A| = 3(2) - (-5)(1) = 6 + 5 = 11$

$A^{-1} = \frac{1}{11}\begin{bmatrix} 2 & 5 \\ -1 & 3 \end{bmatrix}$

$\begin{bmatrix} x \\ y \end{bmatrix} = A^{-1}\begin{bmatrix} 1 \\ 4 \end{bmatrix} = \frac{1}{11}\begin{bmatrix} 2 & 5 \\ -1 & 3 \end{bmatrix}\begin{bmatrix} 1 \\ 4 \end{bmatrix}$

$= \frac{1}{11}\begin{bmatrix} 2+20 \\ -1+12 \end{bmatrix} = \frac{1}{11}\begin{bmatrix} 22 \\ 11 \end{bmatrix} = \begin{bmatrix} 2 \\ 1 \end{bmatrix}$

માટે, $x = 2$, $y = 1$
\end{solutionbox}

\questionmarks{2(b)(2)}{4}{જો $A + B = \begin{bmatrix} 1 & -1 \\ 3 & 0 \end{bmatrix}$, $A - B = \begin{bmatrix} 3 & 1 \\ 1 & 4 \end{bmatrix}$ તો find $(AB)^{-1}$}

\begin{solutionbox}
સમીકરણોનો સરવાળો કરતા:
$2A = \begin{bmatrix} 1 & -1 \\ 3 & 0 \end{bmatrix} + \begin{bmatrix} 3 & 1 \\ 1 & 4 \end{bmatrix} = \begin{bmatrix} 4 & 0 \\ 4 & 4 \end{bmatrix} \implies A = \begin{bmatrix} 2 & 0 \\ 2 & 2 \end{bmatrix}$

બાદબાકી કરતા: $(A + B) - (A - B) = 2B$
$2B = \begin{bmatrix} 1 & -1 \\ 3 & 0 \end{bmatrix} - \begin{bmatrix} 3 & 1 \\ 1 & 4 \end{bmatrix} = \begin{bmatrix} -2 & -2 \\ 2 & -4 \end{bmatrix} \implies B = \begin{bmatrix} -1 & -1 \\ 1 & -2 \end{bmatrix}$

$AB = \begin{bmatrix} 2 & 0 \\ 2 & 2 \end{bmatrix}\begin{bmatrix} -1 & -1 \\ 1 & -2 \end{bmatrix} = \begin{bmatrix} -2 & -2 \\ 0 & -6 \end{bmatrix}$

$|AB| = (-2)(-6) - (-2)(0) = 12$

$(AB)^{-1} = \frac{1}{12}\begin{bmatrix} -6 & 2 \\ 0 & -2 \end{bmatrix} = \begin{bmatrix} -1/2 & 1/6 \\ 0 & -1/6 \end{bmatrix}$
\end{solutionbox}

\questionmarks{2(b)(3)}{4}{જો $B = \begin{bmatrix} -4 & -3 & -3 \\ 1 & 0 & 1 \\ 4 & 4 & 3 \end{bmatrix}$ તો સાબિત કરો કે $\text{adj } B = B$}

\begin{solutionbox}
$3 \times 3$ શ્રેણિક માટે, આપણે સહઅવયવ શ્રેણિક શોધીને તેનો પરિવર્ત કરવો પડશે.

$C_{11} = +\begin{vmatrix} 0 & 1 \\ 4 & 3 \end{vmatrix} = -4$, $C_{12} = -\begin{vmatrix} 1 & 1 \\ 4 & 3 \end{vmatrix} = 1$, $C_{13} = +\begin{vmatrix} 1 & 0 \\ 4 & 4 \end{vmatrix} = 4$

$C_{21} = -\begin{vmatrix} -3 & -3 \\ 4 & 3 \end{vmatrix} = -3$, $C_{22} = +\begin{vmatrix} -4 & -3 \\ 4 & 3 \end{vmatrix} = 0$, $C_{23} = -\begin{vmatrix} -4 & -3 \\ 4 & 4 \end{vmatrix} = 4$

$C_{31} = +\begin{vmatrix} -3 & -3 \\ 0 & 1 \end{vmatrix} = -3$, $C_{32} = -\begin{vmatrix} -4 & -3 \\ 1 & 1 \end{vmatrix} = 1$, $C_{33} = +\begin{vmatrix} -4 & -3 \\ 1 & 0 \end{vmatrix} = 3$

Cofactor matrix = $\begin{bmatrix} -4 & 1 & 4 \\ -3 & 0 & 4 \\ -3 & 1 & 3 \end{bmatrix}$

$\text{adj } B = (\text{Cofactor matrix})^T = \begin{bmatrix} -4 & -3 & -3 \\ 1 & 0 & 1 \\ 4 & 4 & 3 \end{bmatrix}$

Since $\text{adj } B = B$. \textbf{સાબિત.}
\end{solutionbox}

\questionmarks{3(a)}{6}{કોઈપણ બે ગણો}

\questionmarks{3(a)(1)}{3}{જો $y = \frac{1 + \tan x}{1 - \tan x}$ તો find $\frac{dy}{dx}$}

\begin{solutionbox}
ભાગાકારના નિયમનો ઉપયોગ કરતા: $\frac{d}{dx}\left(\frac{u}{v}\right) = \frac{v\frac{du}{dx} - u\frac{dv}{dx}}{v^2}$

ધારો કે $u = 1 + \tan x$, $v = 1 - \tan x$.
$\frac{du}{dx} = \sec^2 x$, $\frac{dv}{dx} = -\sec^2 x$

$\frac{dy}{dx} = \frac{(1-\tan x)(\sec^2 x) - (1+\tan x)(-\sec^2 x)}{(1-\tan x)^2}$

$= \frac{\sec^2 x - \tan x \sec^2 x + \sec^2 x + \tan x \sec^2 x}{(1-\tan x)^2}$

$= \frac{2\sec^2 x}{(1-\tan x)^2}$
\end{solutionbox}

\questionmarks{3(a)(2)}{3}{જો $x = a(t + \sin t)$, $y = a(1 - \cos t)$ તો find $\frac{dy}{dx}$}

\begin{solutionbox}
$\frac{dx}{dt} = a(1 + \cos t)$, $\frac{dy}{dt} = a \sin t$

$\frac{dy}{dx} = \frac{dy/dt}{dx/dt} = \frac{a \sin t}{a(1 + \cos t)} = \frac{\sin t}{1 + \cos t}$

Using the identity $\sin t = 2\sin(t/2)\cos(t/2)$ and $1 + \cos t = 2\cos^2(t/2)$:

$\frac{dy}{dx} = \frac{2\sin(t/2)\cos(t/2)}{2\cos^2(t/2)} = \frac{\sin(t/2)}{\cos(t/2)} = \tan(t/2)$
\end{solutionbox}

\questionmarks{3(a)(3)}{3}{કિંમત શોધો $\int_0^{\pi/2} \sin x \cos x \, dx$}

\begin{solutionbox}
રીત 1: આદેશની રીતનો ઉપયોગ કરતા
ધારો કે $u = \sin x$, તો $du = \cos x \, dx$.
When $x = 0$, $u = 0$; when $x = \pi/2$, $u = 1$.

$\int_0^{\pi/2} \sin x \cos x \, dx = \int_0^1 u \, du = \left[\frac{u^2}{2}\right]_0^1 = \frac{1}{2}$

રીત 2: દ્વિ-ખૂણાના નિત્યસમનો ઉપયોગ કરતા
$\sin x \cos x = \frac{1}{2}\sin 2x$

$\int_0^{\pi/2} \sin x \cos x \, dx = \frac{1}{2}\int_0^{\pi/2} \sin 2x \, dx = \frac{1}{2}\left[-\frac{\cos 2x}{2}\right]_0^{\pi/2}$

$= -\frac{1}{4}[\cos \pi - \cos 0] = -\frac{1}{4}[-1 - 1] = \frac{1}{2}$
\end{solutionbox}

\questionmarks{3(b)}{8}{કોઈપણ બે ગણો}

\questionmarks{3(b)(1)}{4}{જો $y = (\sin x)^{\tan x}$ તો find $\frac{dy}{dx}$}

\begin{solutionbox}
બંને બાજુ લઘુગણક લેતા:
$\ln y = \tan x \ln(\sin x)$

બંને બાજુ વિકલન કરતા:
$\frac{1}{y}\frac{dy}{dx} = \sec^2 x \ln(\sin x) + \tan x \cdot \frac{\cos x}{\sin x}$

$\frac{1}{y}\frac{dy}{dx} = \sec^2 x \ln(\sin x) + \tan x \cot x$

$\frac{1}{y}\frac{dy}{dx} = \sec^2 x \ln(\sin x) + 1$

$\frac{dy}{dx} = y[\sec^2 x \ln(\sin x) + 1]$

$\frac{dy}{dx} = (\sin x)^{\tan x}[\sec^2 x \ln(\sin x) + 1]$
\end{solutionbox}

\questionmarks{3(b)(2)}{4}{શોધો maximum and minimum value of $f(x) = 2x^3 - 3x^2 - 12x + 5$}

\begin{solutionbox}
$f'(x) = 6x^2 - 6x - 12 = 6(x^2 - x - 2) = 6(x-2)(x+1)$

નિર્ણાયક બિંદુઓ માટે: $f'(x) = 0 \implies x = 2$ or $x = -1$

$f''(x) = 12x - 6$

At $x = -1$: $f''(-1) = -12 - 6 = -18 < 0$ (Maximum)
At $x = 2$: $f''(2) = 24 - 6 = 18 > 0$ (Minimum)

$f(-1) = 2(-1)^3 - 3(-1)^2 - 12(-1) + 5 = -2 - 3 + 12 + 5 = 12$

$f(2) = 2(8) - 3(4) - 12(2) + 5 = 16 - 12 - 24 + 5 = -15$

\textbf{મહત્તમ કિંમત = 12 at $x = -1$}
\textbf{ન્યૂનતમ કિંમત = -15 at $x = 2$}
\end{solutionbox}

\questionmarks{3(b)(3)}{4}{The motion of a particle is given by $S = t^3 + 6t^2 + 3t + 5$. શોધો the velocity and acceleration at $t = 3$ sec.}

\begin{solutionbox}
સ્થાન: $S = t^3 + 6t^2 + 3t + 5$

વેગ: $v = \frac{dS}{dt} = 3t^2 + 12t + 3$

પ્રવેગ: $a = \frac{dv}{dt} = 6t + 12$

At $t = 3$:
વેગ: $v(3) = 3(9) + 12(3) + 3 = 27 + 36 + 3 = 66$ units/sec

પ્રવેગ: $a(3) = 6(3) + 12 = 18 + 12 = 30$ units/sec²
\end{solutionbox}

\questionmarks{4(a)}{6}{કોઈપણ બે ગણો}

\questionmarks{4(a)(1)}{3}{કિંમત શોધો $\int x^2 e^x dx$}

\begin{solutionbox}
ખંડશઃ સંકલનની રીતનો બે વાર ઉપયોગ કરતા:
ધારો કે $u = x^2$, $dv = e^x dx \implies du = 2x dx$, $v = e^x$

$\int x^2 e^x dx = x^2 e^x - \int 2x e^x dx$

For $\int 2x e^x dx$:
ધારો કે $u_1 = 2x$, $dv_1 = e^x dx \implies du_1 = 2 dx$, $v_1 = e^x$

$\int 2x e^x dx = 2x e^x - \int 2 e^x dx = 2x e^x - 2e^x$

માટે:
$\int x^2 e^x dx = x^2 e^x - (2x e^x - 2e^x) + C$
$= x^2 e^x - 2x e^x + 2e^x + C$
$= e^x(x^2 - 2x + 2) + C$
\end{solutionbox}

\questionmarks{4(a)(2)}{3}{કિંમત શોધો $\int \frac{2x + 3}{(x-1)(x+2)} dx$}

\begin{solutionbox}
આંશિક અપૂર્ણાંકની રીતનો ઉપયોગ કરતા:
$\frac{2x + 3}{(x-1)(x+2)} = \frac{A}{x-1} + \frac{B}{x+2}$

$2x + 3 = A(x+2) + B(x-1)$

Setting $x = 1$: $5 = 3A \implies A = \frac{5}{3}$
Setting $x = -2$: $-1 = -3B \implies B = \frac{1}{3}$

$\int \frac{2x + 3}{(x-1)(x+2)} dx = \int \left(\frac{5/3}{x-1} + \frac{1/3}{x+2}\right) dx$

$= \frac{5}{3}\ln|x-1| + \frac{1}{3}\ln|x+2| + C$
\end{solutionbox}

\questionmarks{4(a)(3)}{3}{શોધો mean using the given information}

\begin{solutionbox}
\begin{center}
\captionof{table}{આવૃત્તિ વિતરણ}
\begin{tabulary}{\linewidth}{|C|C|C|C|C|C|}
\hline
$x_i$ & 52 & 55 & 58 & 62 & 79 \\ \hline
$f_i$ & 5 & 3 & 2 & 3 & 6 \\ \hline
\end{tabulary}
\end{center}

મધ્યક = $\frac{\sum f_i x_i}{\sum f_i}$

$\sum f_i x_i = 52(5) + 55(3) + 58(2) + 62(3) + 79(6)$
$= 260 + 165 + 116 + 186 + 474 = 1201$

$\sum f_i = 5 + 3 + 2 + 3 + 6 = 19$

મધ્યક = $\frac{1201}{19} = 63.21$
\end{solutionbox}

\questionmarks{4(b)}{8}{કોઈપણ બે ગણો}

\questionmarks{4(b)(1)}{4}{કિંમત શોધો $\int_{-1}^{1} \frac{x^5 - 6x}{x - 4} dx$}

\begin{solutionbox}
સૌપ્રથમ બહુપદી ભાગાકાર કરતા:
$\frac{x^5 - 6x}{x - 4} = x^4 + 4x^3 + 16x^2 + 64x + 250 + \frac{1000}{x-4}$

$\int_{-1}^{1} \frac{x^5 - 6x}{x - 4} dx = \int_{-1}^{1} \left(x^4 + 4x^3 + 16x^2 + 64x + 250 + \frac{1000}{x-4}\right) dx$

$= \left[\frac{x^5}{5} + x^4 + \frac{16x^3}{3} + 32x^2 + 250x + 1000\ln|x-4|\right]_{-1}^{1}$

At $x = 1$: $\frac{1}{5} + 1 + \frac{16}{3} + 32 + 250 + 1000\ln 3$
At $x = -1$: $-\frac{1}{5} + 1 - \frac{16}{3} + 32 - 250 + 1000\ln 5$

$= \left(\frac{2}{5} + \frac{32}{3} + 500 + 1000\ln\frac{3}{5}\right)$

$= \frac{6 + 160 + 1500}{15} + 1000\ln\frac{3}{5} = \frac{1666}{15} + 1000\ln\frac{3}{5}$
\end{solutionbox}

\questionmarks{4(b)(2)}{4}{કિંમત શોધો $\int \sin 5x \sin 6x \, dx$}

\begin{solutionbox}
ગુણાકાર-સરવાળાના સૂત્રનો ઉપયોગ કરતા:
$\sin A \sin B = \frac{1}{2}[\cos(A-B) - \cos(A+B)]$

$\sin 5x \sin 6x = \frac{1}{2}[\cos(5x-6x) - \cos(5x+6x)]$
$= \frac{1}{2}[\cos(-x) - \cos(11x)] = \frac{1}{2}[\cos x - \cos(11x)]$

$\int \sin 5x \sin 6x \, dx = \frac{1}{2}\int [\cos x - \cos(11x)] dx$

$= \frac{1}{2}\left[\sin x - \frac{\sin(11x)}{11}\right] + C$

$= \frac{\sin x}{2} - \frac{\sin(11x)}{22} + C$
\end{solutionbox}

\questionmarks{4(b)(3)}{4}{Calculate the પ્રમાણિત વિચલન for the following data: 6, 7, 9, 11, 13, 15, 8, 10}

\begin{solutionbox}
Data: 6, 7, 8, 9, 10, 11, 13, 15 (arranged in order)
$n = 8$

\textbf{Step 1: ગણતરી મધ્યક}
$\bar{x} = \frac{6 + 7 + 8 + 9 + 10 + 11 + 13 + 15}{8} = \frac{79}{8} = 9.875$

\textbf{Step 2: વિચલન અને તેના વર્ગની ગણતરી}

\begin{center}
\captionof{table}{પ્રમાણિત વિચલન ગણતરી}
\begin{tabulary}{\linewidth}{|C|C|C|}
\hline
$x_i$ & $x_i - \bar{x}$ & $(x_i - \bar{x})^2$ \\ \hline
6 & -3.875 & 15.016 \\ \hline
7 & -2.875 & 8.266 \\ \hline
8 & -1.875 & 3.516 \\ \hline
9 & -0.875 & 0.766 \\ \hline
10 & 0.125 & 0.016 \\ \hline
11 & 1.125 & 1.266 \\ \hline
13 & 3.125 & 9.766 \\ \hline
15 & 5.125 & 26.266 \\ \hline
\end{tabulary}
\end{center}

$\sum (x_i - \bar{x})^2 = 64.878$

\textbf{Step 3: ગણતરી પ્રમાણિત વિચલન}
$\sigma = \sqrt{\frac{\sum (x_i - \bar{x})^2}{n}} = \sqrt{\frac{64.878}{8}} = \sqrt{8.11} = 2.85$

\textbf{પ્રમાણિત વિચલન = 2.85}
\end{solutionbox}

\questionmarks{5(a)}{6}{કોઈપણ બે ગણો}

\questionmarks{5(a)(1)}{3}{શોધો the mean for the following data:}

\begin{solutionbox}
\begin{center}
\captionof{table}{Data}
\begin{tabulary}{\linewidth}{|C|C|C|C|C|C|C|}
\hline
$X_i$ & 92 & 93 & 97 & 98 & 102 & 104 \\ \hline
$F_i$ & 3 & 2 & 2 & 3 & 6 & 4 \\ \hline
\end{tabulary}
\end{center}

મધ્યક = $\frac{\sum f_i x_i}{\sum f_i}$

$\sum f_i x_i = 92(3) + 93(2) + 97(2) + 98(3) + 102(6) + 104(4)$
$= 276 + 186 + 194 + 294 + 612 + 416 = 1978$

$\sum f_i = 3 + 2 + 2 + 3 + 6 + 4 = 20$

મધ્યક = $\frac{1978}{20} = 98.9$
\end{solutionbox}

\questionmarks{5(a)(2)}{3}{Calculate the પ્રમાણિત વિચલન for the following data: 5, 9, 8, 12, 6, 10, 6, 8}

\begin{solutionbox}
Data: 5, 6, 6, 8, 8, 9, 10, 12 (arranged in order)
$n = 8$

\textbf{Step 1: ગણતરી મધ્યક}
$\bar{x} = \frac{5 + 6 + 6 + 8 + 8 + 9 + 10 + 12}{8} = \frac{64}{8} = 8$

\textbf{Step 2: ગણતરી પ્રમાણિત વિચલન}
\begin{center}
\captionof{table}{Deviations}
\begin{tabulary}{\linewidth}{|C|C|C|}
\hline
$x_i$ & $x_i - \bar{x}$ & $(x_i - \bar{x})^2$ \\ \hline
5 & -3 & 9 \\ \hline
6 & -2 & 4 \\ \hline
6 & -2 & 4 \\ \hline
8 & 0 & 0 \\ \hline
8 & 0 & 0 \\ \hline
9 & 1 & 1 \\ \hline
10 & 2 & 4 \\ \hline
12 & 4 & 16 \\ \hline
\end{tabulary}
\end{center}

$\sum (x_i - \bar{x})^2 = 38$

$\sigma = \sqrt{\frac{38}{8}} = \sqrt{4.75} = 2.18$

\textbf{પ્રમાણિત વિચલન = 2.18}
\end{solutionbox}

\questionmarks{5(a)(3)}{3}{Calculate the મધ્યક for the following data: 5, 15, 25, 35, 45, 55, 65, 75, 85, 95, 75}

\begin{solutionbox}
$n = 11$

સરવાળો = $5 + 15 + 25 + 35 + 45 + 55 + 65 + 75 + 85 + 95 + 75 = 575$

મધ્યક = $\frac{575}{11} = 52.27$
\end{solutionbox}

\questionmarks{5(b)}{8}{કોઈપણ બે ગણો}

\questionmarks{5(b)(1)}{4}{ઉકેલો વિકલ સમીકરણ $\frac{dy}{dx} + \frac{y}{x} = e^x$, $y(0) = 2$}

\begin{solutionbox}
This is a first-order linear વિકલ સમીકરણ of the form $\frac{dy}{dx} + Py = Q$

Here, $P = \frac{1}{x}$ and $Q = e^x$

\textbf{Integrating Factor:} $\mu = e^{\int P dx} = e^{\int \frac{1}{x} dx} = e^{\ln x} = x$ (for $x > 0$)

Multiplying the equation by $\mu = x$:
$x\frac{dy}{dx} + y = xe^x \implies \frac{d}{dx}(xy) = xe^x$

બંને બાજુ સંકલન કરતા:
$xy = \int xe^x dx$

માટે ખંડશઃ સંકલનનો ઉપયોગ કરતા $\int xe^x dx$:
ધારો કે $u = x$, $dv = e^x dx \implies du = dx$, $v = e^x$

$\int xe^x dx = xe^x - \int e^x dx = xe^x - e^x = e^x(x-1)$

માટે: $xy = e^x(x-1) + C$

$y = \frac{e^x(x-1) + C}{x}$

\textbf{Final Answer:} $y = e^x + \frac{1}{x}$ (subject to domain restrictions)
\end{solutionbox}

\questionmarks{5(b)(2)}{4}{ઉકેલો વિકલ સમીકરણ $\frac{dy}{dx} + \frac{4x}{x^2 + 1}y = \frac{1}{(x^2 + 1)^2}$}

\begin{solutionbox}
This is a first-order linear વિકલ સમીકરણ.
$P = \frac{4x}{x^2 + 1}$, $Q = \frac{1}{(x^2 + 1)^2}$

\textbf{Integrating Factor:}
$\mu = e^{\int P dx} = e^{\int \frac{4x}{x^2 + 1} dx}$

ધારો કે $u = x^2 + 1$, તો $du = 2x dx$.
$\int \frac{4x}{x^2 + 1} dx = 2\int \frac{du}{u} = 2\ln u = 2\ln(x^2 + 1)$

$\mu = e^{2\ln(x^2 + 1)} = (x^2 + 1)^2$

Multiplying the equation by $\mu$:
$(x^2 + 1)^2 \frac{dy}{dx} + 4x(x^2 + 1)y = 1$

This can be written as: $\frac{d}{dx}[y(x^2 + 1)^2] = 1$

Integrating: $y(x^2 + 1)^2 = x + C$

$y = \frac{x + C}{(x^2 + 1)^2}$
\end{solutionbox}

\questionmarks{5(b)(3)}{4}{ઉકેલો વિકલ સમીકરણ $\frac{dy}{dx} = \sin(x + y)$}

\begin{solutionbox}
ધારો કે $v = x + y$, તો $\frac{dv}{dx} = 1 + \frac{dy}{dx} \implies \frac{dy}{dx} = \frac{dv}{dx} - 1$

મૂળ સમીકરણમાં કિંમત મૂકતા:
$\frac{dv}{dx} - 1 = \sin v \implies \frac{dv}{dx} = 1 + \sin v$

ચલ વિયોજન કરતા:
$\frac{dv}{1 + \sin v} = dx$

ડાબી બાજુનું સંકલન કરવા માટે, નિત્યસમનો ઉપયોગ કરતા:
$\frac{1}{1 + \sin v} = \frac{1 - \sin v}{(1 + \sin v)(1 - \sin v)} = \frac{1 - \sin v}{\cos^2 v}$

$\int \frac{dv}{1 + \sin v} = \int \frac{1 - \sin v}{\cos^2 v} dv = \int (\sec^2 v - \sec v \tan v) dv$

$= \tan v - \sec v + C_1$

માટે: $\tan(x + y) - \sec(x + y) = x + C$
\end{solutionbox}

\section*{સૂત્રો}

\begin{itemize}
    \item \textbf{શ્રેણિક પ્રક્રિયાઓ}: $(A + B)^T = A^T + B^T$, $(AB)^T = B^T A^T$, $(A^{-1})^T = (A^T)^{-1}$
    \item \textbf{Differentiation}: $\frac{d}{dx}[x^n] = nx^{n-1}$, $\frac{d}{dx}[\ln x] = \frac{1}{x}$, $\frac{d}{dx}[e^x] = e^x$, $\frac{d}{dx}[\sin x] = \cos x$
    \item \textbf{Integration}: $\int x^n dx = \frac{x^{n+1}}{n+1}$, $\int e^x dx = e^x$, $\int \sin x dx = -\cos x$
    \item \textbf{Differential Equations}: Linear DE $\frac{dy}{dx} + Py = Q$, IF $\mu = e^{\int P dx}$
    \item \textbf{આંકડાશાસ્ત્ર}: મધ્યક $\bar{x} = \frac{\sum f_i x_i}{\sum f_i}$, SD $\sigma = \sqrt{\frac{\sum (x_i - \bar{x})^2}{n}}$
\end{itemize}

\end{document}
