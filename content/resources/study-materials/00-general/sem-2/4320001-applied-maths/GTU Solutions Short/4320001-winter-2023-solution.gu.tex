\documentclass{article}

% content/resources/templates/preamble.tex
\usepackage[margin=0.6in]{geometry}
\author{Milav Dabgar}
\usepackage{amsmath,amssymb,amsthm}
\usepackage{booktabs}
\usepackage{multirow}
\usepackage{xcolor}
\usepackage{tcolorbox}
\tcbuselibrary{breakable,skins}
\usepackage[colorlinks=true,linkcolor=blue]{hyperref}
\usepackage{titlesec}
\usepackage{enumitem}
\usepackage{tikz}
\usepackage{pgfplots}
\usepackage{circuitikz}
\usepackage[version=4]{mhchem}
\usepackage{longtable}
\usepackage{array}
\usepackage{float}
\usepackage{caption}
\usepackage{listings}

\lstset{
  basicstyle=\small\ttfamily,
  breaklines=true,
  breakatwhitespace=false,
  postbreak=\mbox{\textcolor{red}{$\hookrightarrow$}\space},
  float=false,
  numbers=left,
  numberstyle=\tiny\color{gray},
  numbersep=10pt,
  xleftmargin=2em,
  keywordstyle=\color{blue},
  commentstyle=\color{green!60!black},
  stringstyle=\color{purple},
  backgroundcolor=\color{gray!5},
  showstringspaces=false,
  tabsize=2,
  captionpos=b,
  keepspaces=true,
  columns=flexible
}

\pgfplotsset{compat=1.18}
\usetikzlibrary{shapes,arrows,positioning,calc,patterns,decorations.pathmorphing,decorations.markings,arrows.meta}

% Color scheme
\definecolor{headcolor}{RGB}{0,102,204}
\definecolor{keycolor}{RGB}{220,20,60}
\definecolor{solutioncolor}{RGB}{34,139,34}
\definecolor{mnemoniccolor}{RGB}{148,0,211}
\definecolor{codecolor}{RGB}{0,0,100}

% Spacing
\setlength{\parskip}{3pt}
\setlist[itemize]{nosep}
\setlist[enumerate]{nosep}

% Title formatting
\titleformat{\section}{\Large\bfseries\color{headcolor}}{\thesection}{1em}{}
\titleformat{\subsection}{\large\bfseries\color{headcolor}}{\thesubsection}{1em}{}

% Pandoc tightlist compatibility
\providecommand{\tightlist}{%
  \setlength{\itemsep}{0pt}\setlength{\parskip}{0pt}}

% Pandoc longtable compatibility
\newcounter{none}
\def\thenone{}


% content/resources/templates/gujarati-boxes.tex
\usepackage{fontspec}
\usepackage{polyglossia}

% Set Gujarati as main language (document is primarily in Gujarati)
% Note: gloss-gujarati.ldf doesn't exist in polyglossia, but it will use hyphenation patterns
\setdefaultlanguage{gujarati}
\setotherlanguage{english}

% Configure Gujarati font properly
% Use Language=Default to prevent polyglossia from trying to add language-specific features
% that don't exist for Gujarati, which causes "empty feature" warnings
\newfontfamily\gujaratifont[Script=Gujarati,AutoFakeBold=2.5,AutoFakeSlant=0.3]{Noto Sans Gujarati}
\setmainfont[Script=Gujarati,AutoFakeBold=2.5,AutoFakeSlant=0.3]{Noto Sans Gujarati}
% Use Noto Sans Gujarati for monospace to support Gujarati in text
\setmonofont[Scale=0.9]{Noto Sans Gujarati}

% Configure English to use the same font
\newfontfamily\englishfont[Script=Gujarati,AutoFakeBold=2.5,AutoFakeSlant=0.3]{Noto Sans Gujarati}

% Translations for polyglossia
\gappto\captionsgujarati{
  \renewcommand{\tablename}{કોષ્ટક}
  \renewcommand{\figurename}{આકૃતિ}
}

% Helper for TikZ nodes to ensure Gujarati font
\newcommand{\gu}[1]{{\gujaratifont #1}}

% Custom environments
\newtcolorbox{solutionbox}{
    breakable,
    enhanced,
    colback=solutioncolor!5!white,
    colframe=solutioncolor!75!black,
    fonttitle=\bfseries,
    title=જવાબ
}

\newtcolorbox{solutionboxnobreak}{
 colback=solutioncolor!5!white,
 colframe=solutioncolor!75!black,
 fonttitle=\bfseries,
 title=જવાબ
}

\newtcolorbox{keyformula}{
 breakable,
 enhanced,
 colback=keycolor!5!white,
 colframe=keycolor!75!black,
 fonttitle=\bfseries,
 title=રાસાયણિક સમીકરણ/સૂત્ર
}

\newtcolorbox{mnemonicbox}{
 breakable,
 enhanced,
 colback=mnemoniccolor!5!white,
 colframe=mnemoniccolor!75!black,
 fonttitle=\bfseries,
 title=મેમરી ટ્રીક
}


% Custom commands for GTU solutions
% This file defines semantic commands for consistent formatting

% Question command with automatic formatting
\newcommand{\question}[2]{%
  \section*{Question #1}%
  \textbf{#2}%
}

% OR question variant
\newcommand{\questionor}[2]{%
  \section*{Question #1 OR}%
  \textbf{#2}%
}

% Proper table environment with caption
\newenvironment{answertable}[1]{%
  \begin{table}[htbp]
  \centering
  \caption{#1}
}{%
  \end{table}
}

% Proper figure environment for diagrams
\newenvironment{answerdiagram}[1]{%
  \begin{figure}[htbp]
  \centering
  \caption{#1}
}{%
  \end{figure}
}

% Semantic markup for key terms
\newcommand{\keyword}[1]{\textbf{#1}}
\newcommand{\code}[1]{\texttt{#1}}
\newcommand{\classname}[1]{\texttt{#1}}
\newcommand{\methodname}[1]{\texttt{#1}}

% Proper quotation marks
\newcommand{\mnemonic}[1]{``#1''}


\title{Applied Mathematics (4320001) - Winter 2023 Solution}
\date{January 30, 2024}

\begin{document}
\maketitle

\questionmarks{1}{14}{નીચેના વિકલ્પોમાંથી યોગ્ય વિકલ્પ પસંદ કરી ખાલી જગ્યા પૂરો.}

\questionmarks{1(1)}{1}{જો $A = \begin{bmatrix} 1 & 2 \\ 3 & -1 \end{bmatrix}$ હોય તો $4A$ = ...}
\textbf{જવાબ}: (b) $\begin{bmatrix} 4 & 8 \\ 12 & -4 \end{bmatrix}$

\begin{solutionbox}
$4A = 4 \begin{bmatrix} 1 & 2 \\ 3 & -1 \end{bmatrix} = \begin{bmatrix} 4 & 8 \\ 12 & -4 \end{bmatrix}$
\end{solutionbox}

\questionmarks{1(2)}{1}{$\begin{bmatrix} 1 & 1 & 2 \\ -3 & 2 & 3 \end{bmatrix}$ શ્રેણિકની કક્ષા \underline{\hspace{2cm}} છે.}
\textbf{જવાબ}: (a) 2 × 3

\begin{solutionbox}
શ્રેણિકને 2 હાર અને 3 સ્તંભ છે, તેથી કક્ષા 2 × 3 છે.
\end{solutionbox}

\questionmarks{1(3)}{1}{જો $A = \begin{bmatrix} 1 & 1 \\ 1 & 1 \end{bmatrix}$ હોય તો $A^2$ = ...}
\textbf{જવાબ}: (d) $\begin{bmatrix} 2 & 2 \\ 2 & 2 \end{bmatrix}$

\begin{solutionbox}
$A^2 = \begin{bmatrix} 1 & 1 \\ 1 & 1 \end{bmatrix} \begin{bmatrix} 1 & 1 \\ 1 & 1 \end{bmatrix} = \begin{bmatrix} 2 & 2 \\ 2 & 2 \end{bmatrix}$
\end{solutionbox}

\questionmarks{1(4)}{1}{જો $A = \begin{bmatrix} 2 & -1 \\ 3 & 4 \end{bmatrix}$ હોય તો $A$ નો સહ-અવયવજ (adjoint) શ્રેણિક = ...}
\textbf{જવાબ}: (c) $\begin{bmatrix} 4 & 1 \\ -3 & 2 \end{bmatrix}$

\begin{solutionbox}
શ્રેણિક $A = \begin{bmatrix} a & b \\ c & d \end{bmatrix}$ માટે, $adj(A) = \begin{bmatrix} d & -b \\ -c & a \end{bmatrix}$
$adj(A) = \begin{bmatrix} 4 & 1 \\ -3 & 2 \end{bmatrix}$
\end{solutionbox}

\questionmarks{1(5)}{1}{$\frac{d}{dx}(\tan x)$ = ...}
\textbf{જવાબ}: (d) $\sec^2x$

\begin{solutionbox}
$\frac{d}{dx}(\tan x) = \sec^2 x$
\end{solutionbox}

\questionmarks{1(6)}{1}{$\frac{d}{dx}(\sin 5x)$ = ...}
\textbf{જવાબ}: (b) $5\cos 5x$

\begin{solutionbox}
$\frac{d}{dx}(\sin 5x) = 5\cos 5x$ (ચેઇન રૂલનો ઉપયોગ કરીને)
\end{solutionbox}

\questionmarks{1(7)}{1}{જો વિધેય $y = f(x)$ એ $x = a$ આગળ મહત્તમ હોય તો $f'(a)$ = ...}
\textbf{જવાબ}: (c) 0

\begin{solutionbox}
મહત્તમ બિંદુએ, પ્રથમ વિકલિત શૂન્ય થાય: $f'(a) = 0$
\end{solutionbox}

\questionmarks{1(8)}{1}{$\int \sin x dx$ = ... + C}
\textbf{જવાબ}: (a) $-\cos x$

\begin{solutionbox}
$\int \sin x \, dx = -\cos x + C$
\end{solutionbox}

\questionmarks{1(9)}{1}{$\int \frac{1}{x^2+4} dx$ = ... + C}
\textbf{જવાબ}: (d) $\frac{1}{2}\tan^{-1}(\frac{x}{2})$

\begin{solutionbox}
$\int \frac{1}{x^2+4} dx = \frac{1}{2}\tan^{-1}\left(\frac{x}{2}\right) + C$
\end{solutionbox}

\questionmarks{1(10)}{1}{$\int_1^2 x^2 dx$ = ...}
\textbf{જવાબ}: (a) 7/3

\begin{solutionbox}
$\int_1^2 x^2 dx = \left[\frac{x^3}{3}\right]_1^2 = \frac{8}{3} - \frac{1}{3} = \frac{7}{3}$
\end{solutionbox}

\questionmarks{1(11)}{1}{વિકલ સમીકરણ $\left(\frac{d^3y}{dx^3}\right)^4 + \frac{dy}{dx} + 5y = 0$ ની કક્ષા (order) ... છે.}
\textbf{જવાબ}: (c) 3

\begin{solutionbox}
કક્ષા એ મહત્તમ વિકલન છે = 3
\end{solutionbox}

\questionmarks{1(12)}{1}{વિકલ સમીકરણ $\frac{dy}{dx} + \frac{y}{x} = 1$ નો સંકલ્યકારક અવયવ (I.F) ... છે.}
\textbf{જવાબ}: (b) x

\begin{solutionbox}
I.F. = $e^{\int \frac{1}{x} dx} = e^{\ln x} = x$
\end{solutionbox}

\questionmarks{1(13)}{1}{39, 23, 58, 47, 50, 16, 61 નો મધ્યક ... છે.}
\textbf{જવાબ}: (b) 42

\begin{solutionbox}
મધ્યક = $\frac{39+23+58+47+50+16+61}{7} = \frac{294}{7} = 42$
\end{solutionbox}

\questionmarks{1(14)}{1}{પ્રથમ પાંચ પ્રાકૃતિક સંખ્યાઓનો મધ્યક ... છે.}
\textbf{જવાબ}: (a) 3

\begin{solutionbox}
મધ્યક = $\frac{1+2+3+4+5}{5} = \frac{15}{5} = 3$
\end{solutionbox}

\questionmarks{2}{14}{કોઈપણ બે ગણો}

\questionmarks{2(a)}{6}{}

\questionmarks{2(a)(1)}{3}{જો $A = \begin{bmatrix} 1 & 3 & 5 \\ -1 & 0 & 2 \\ 4 & 3 & 6 \end{bmatrix}$, $B = \begin{bmatrix} 3 & 4 & 5 \\ 5 & 4 & 3 \\ 3 & 5 & 4 \end{bmatrix}$, $C = \begin{bmatrix} 1 & 2 & 1 \\ 3 & 3 & 3 \\ 4 & 5 & 6 \end{bmatrix}$ હોય, તો $3A+2B-4C$ શોધો.}

\begin{solutionbox}
$3A = \begin{bmatrix} 3 & 9 & 15 \\ -3 & 0 & 6 \\ 12 & 9 & 18 \end{bmatrix}$

$2B = \begin{bmatrix} 6 & 8 & 10 \\ 10 & 8 & 6 \\ 6 & 10 & 8 \end{bmatrix}$

$4C = \begin{bmatrix} 4 & 8 & 4 \\ 12 & 12 & 12 \\ 16 & 20 & 24 \end{bmatrix}$

$3A + 2B - 4C = \begin{bmatrix} 5 & 9 & 21 \\ -5 & -4 & 0 \\ 2 & -1 & 2 \end{bmatrix}$
\end{solutionbox}

\questionmarks{2(a)(2)}{3}{જો $A = \begin{bmatrix} 7 & 5 \\ -1 & 2 \end{bmatrix}$, $B = \begin{bmatrix} 1 & -1 \\ 3 & 2 \end{bmatrix}$ હોય, તો સાબિત કરો કે $(A+B)^T = A^T + B^T$}

\begin{solutionbox}
$A + B = \begin{bmatrix} 8 & 4 \\ 2 & 4 \end{bmatrix}$

$(A + B)^T = \begin{bmatrix} 8 & 2 \\ 4 & 4 \end{bmatrix}$

$A^T = \begin{bmatrix} 7 & -1 \\ 5 & 2 \end{bmatrix}$, $B^T = \begin{bmatrix} 1 & 3 \\ -1 & 2 \end{bmatrix}$

$A^T + B^T = \begin{bmatrix} 8 & 2 \\ 4 & 4 \end{bmatrix}$

સાબિત થાય છે: $(A + B)^T = A^T + B^T$
\end{solutionbox}

\questionmarks{2(a)(3)}{3}{વિકલ સમીકરણ ઉકેલો: $xy dy = (x+1)(y+1)dx$}

\begin{solutionbox}
ચલને અલગ કરતા (Separating variables):
$\frac{y}{y+1} dy = \frac{x+1}{x} dx$

$\left(1 - \frac{1}{y+1}\right) dy = \left(1 + \frac{1}{x}\right) dx$

બંને બાજુ સંકલન કરતા:
$y - \ln|y+1| = x + \ln|x| + C$

\textbf{અંતિમ જવાબ}: $y - x = \ln|y+1| + \ln|x| + C$
\end{solutionbox}

\questionmarks{2(b)}{8}{}

\questionmarks{2(b)(1)}{4}{શ્રેણિક $\begin{bmatrix} 3 & 1 & 2 \\ 2 & -3 & -1 \\ 1 & 2 & 1 \end{bmatrix}$ નો વ્યસ્ત શ્રેણિક શોધો.}

\begin{solutionbox}
ધારો કે $A = \begin{bmatrix} 3 & 1 & 2 \\ 2 & -3 & -1 \\ 1 & 2 & 1 \end{bmatrix}$

$|A| = 3(-3-(-2)) - 1(2-(-1)) + 2(4-(-3)) = 3(-1) - 1(3) + 2(7) = -3 - 3 + 14 = 8$

\textbf{સહ-અવયવ (Cofactors)}:
\begin{itemize}
\item $C_{11} = -1, C_{12} = -3, C_{13} = 7$
\item $C_{21} = 3, C_{22} = 1, C_{23} = -5$
\item $C_{31} = 5, C_{32} = 7, C_{33} = -11$
\end{itemize}

$adj(A) = \begin{bmatrix} -1 & 3 & 5 \\ -3 & 1 & 7 \\ 7 & -5 & -11 \end{bmatrix}$

$A^{-1} = \frac{1}{8} \begin{bmatrix} -1 & 3 & 5 \\ -3 & 1 & 7 \\ 7 & -5 & -11 \end{bmatrix}$
\end{solutionbox}

\questionmarks{2(b)(2)}{4}{શ્રેણિક પદ્ધતિથી ઉકેલો: $3x - 2y = 8, 5x + 4y = 6$}

\begin{solutionbox}
$\begin{bmatrix} 3 & -2 \\ 5 & 4 \end{bmatrix} \begin{bmatrix} x \\ y \end{bmatrix} = \begin{bmatrix} 8 \\ 6 \end{bmatrix}$

$|A| = 3(4) - (-2)(5) = 12 + 10 = 22$

$A^{-1} = \frac{1}{22} \begin{bmatrix} 4 & 2 \\ -5 & 3 \end{bmatrix}$

$\begin{bmatrix} x \\ y \end{bmatrix} = \frac{1}{22} \begin{bmatrix} 4 & 2 \\ -5 & 3 \end{bmatrix} \begin{bmatrix} 8 \\ 6 \end{bmatrix} = \frac{1}{22} \begin{bmatrix} 44 \\ -22 \end{bmatrix}$

\textbf{જવાબ}: $x = 2, y = -1$
\end{solutionbox}

\questionmarks{2(b)(3)}{4}{જો $A = \begin{bmatrix} 1 & 2 & 1 \\ 2 & 3 & 1 \\ 1 & 2 & 2 \end{bmatrix}$ હોય, તો $A \cdot adj(A)$ શોધો.}

\begin{solutionbox}
$|A| = 1(6-2) - 2(4-1) + 1(4-3) = 4 - 6 + 1 = -1$

કોઈપણ શ્રેણિક A માટે: $A \cdot adj(A) = |A| \cdot I$

$A \cdot adj(A) = (-1) \begin{bmatrix} 1 & 0 & 0 \\ 0 & 1 & 0 \\ 0 & 0 & 1 \end{bmatrix} = \begin{bmatrix} -1 & 0 & 0 \\ 0 & -1 & 0 \\ 0 & 0 & -1 \end{bmatrix}$
\end{solutionbox}

\questionmarks{3}{14}{કોઈપણ બે ગણો}

\questionmarks{3(a)}{6}{}

\questionmarks{3(a)(1)}{3}{જો $y = \log(\frac{\sin x}{1+\cos x})$ હોય, તો $\frac{dy}{dx}$ શોધો.}

\begin{solutionbox}
$y = \log(\sin x) - \log(1+\cos x)$

$\frac{dy}{dx} = \frac{1}{\sin x} \cdot \cos x - \frac{1}{1+\cos x} \cdot (-\sin x)$

$= \frac{\cos x}{\sin x} + \frac{\sin x}{1+\cos x}$

$= \cot x + \frac{\sin x}{1+\cos x}$

નિત્યસમ $\frac{\sin x}{1+\cos x} = \tan(\frac{x}{2})$ નો ઉપયોગ કરતા:

\textbf{જવાબ}: $\frac{dy}{dx} = \cot x + \tan(\frac{x}{2})$
\end{solutionbox}

\questionmarks{3(a)(2)}{3}{જો $y = \sin(x+y)$ હોય, તો $\frac{dy}{dx}$ શોધો.}

\begin{solutionbox}
બંને બાજુ વિકલન કરતા:
$\frac{dy}{dx} = \cos(x+y) \cdot \left(1 + \frac{dy}{dx}\right)$

$\frac{dy}{dx} = \cos(x+y) + \cos(x+y) \cdot \frac{dy}{dx}$

$\frac{dy}{dx} - \cos(x+y) \cdot \frac{dy}{dx} = \cos(x+y)$

$\frac{dy}{dx}[1 - \cos(x+y)] = \cos(x+y)$

\textbf{જવાબ}: $\frac{dy}{dx} = \frac{\cos(x+y)}{1-\cos(x+y)}$
\end{solutionbox}

\questionmarks{3(a)(3)}{3}{મેળવો: $\int x^2\log x dx$}

\begin{solutionbox}
ખંડશઃ સંકલનનો ઉપયોગ કરતા: $\int u dv = uv - \int v du$

ધારો કે $u = \log x$, $dv = x^2 dx$
તો $du = \frac{1}{x} dx$, $v = \frac{x^3}{3}$

$\int x^2 \log x \, dx = \log x \cdot \frac{x^3}{3} - \int \frac{x^3}{3} \cdot \frac{1}{x} dx$

$= \frac{x^3 \log x}{3} - \int \frac{x^2}{3} dx$

$= \frac{x^3 \log x}{3} - \frac{x^3}{9} + C$

\textbf{જવાબ}: $\frac{x^3}{3}(\log x - \frac{1}{3}) + C$
\end{solutionbox}

\questionmarks{3(b)}{8}{}

\questionmarks{3(b)(1)}{4}{ગતિ સમીકરણ $s = 2t^3 - 3t^2 - 12t + 7$ છે. જ્યારે પ્રવેગ શૂન્ય હોય ત્યારે s અને t શોધો.}

\begin{solutionbox}
$s = 2t^3 - 3t^2 - 12t + 7$

વેગ: $v = \frac{ds}{dt} = 6t^2 - 6t - 12$

પ્રવેગ: $a = \frac{dv}{dt} = 12t - 6$

જ્યારે પ્રવેગ = 0:
$12t - 6 = 0$
$t = \frac{1}{2}$

At $t = 1/2$:
$s = 2(\frac{1}{2})^3 - 3(\frac{1}{2})^2 - 12(\frac{1}{2}) + 7 = \frac{1}{4} - \frac{3}{4} - 6 + 7 = \frac{1}{2}$

\textbf{જવાબ}: $t = 1/2, s = 1/2$
\end{solutionbox}

\questionmarks{3(b)(2)}{4}{જો $y = 2e^{3x} + 3e^{-2x}$ હોય, તો સાબિત કરો કે $\frac{d^2y}{dx^2} - \frac{dy}{dx} - 6y = 0$}

\begin{solutionbox}
$y = 2e^{3x} + 3e^{-2x}$

$\frac{dy}{dx} = 6e^{3x} - 6e^{-2x}$

$\frac{d^2y}{dx^2} = 18e^{3x} + 12e^{-2x}$

હવે: $\frac{d^2y}{dx^2} - \frac{dy}{dx} - 6y$

$= (18e^{3x} + 12e^{-2x}) - (6e^{3x} - 6e^{-2x}) - 6(2e^{3x} + 3e^{-2x})$

$= 18e^{3x} + 12e^{-2x} - 6e^{3x} + 6e^{-2x} - 12e^{3x} - 18e^{-2x}$

$= (18-6-12)e^{3x} + (12+6-18)e^{-2x} = 0$

\textbf{Hence proved}
\end{solutionbox}

\questionmarks{3(b)(3)}{4}{$f(x) = x^3 - 3x + 11$ ની મહત્તમ અને ન્યૂનતમ કિંમત શોધો.}

\begin{solutionbox}
$f(x) = x^3 - 3x + 11$

$f'(x) = 3x^2 - 3 = 3(x^2 - 1) = 3(x-1)(x+1)$

નિર્ણાયક બિંદુઓ: $x = 1, x = -1$

$f''(x) = 6x$

$x = 1$ આગળ: $f''(1) = 6 > 0 \rightarrow$ સ્થાનીય ન્યૂનતમ
$x = -1$ આગળ: $f''(-1) = -6 < 0 \rightarrow$ સ્થાનીય મહત્તમ

$f(1) = 1 - 3 + 11 = 9$ (minimum)
$f(-1) = -1 + 3 + 11 = 13$ (maximum)

\textbf{જવાબ}: Maximum = 13 at $x = -1$, Minimum = 9 at $x = 1$
\end{solutionbox}

\questionmarks{4}{14}{કોઈપણ બે ગણો}

\questionmarks{4(a)}{6}{}

\questionmarks{4(a)(1)}{3}{મેળવો: $\int \sin 5x \sin 6x dx$}

\begin{solutionbox}
નિત્યસમનો ઉપયોગ કરતા: $\sin A \sin B = \frac{1}{2}[\cos(A-B) - \cos(A+B)]$

$\sin 5x \sin 6x = \frac{1}{2}[\cos(5x-6x) - \cos(5x+6x)]$

$= \frac{1}{2}[\cos(-x) - \cos(11x)] = \frac{1}{2}[\cos x - \cos(11x)]$

$\int \sin 5x \sin 6x \, dx = \frac{1}{2} \int [\cos x - \cos(11x)] dx$

$= \frac{1}{2}[\sin x - \frac{\sin(11x)}{11}] + C$

\textbf{જવાબ}: $\frac{1}{2}\sin x - \frac{\sin(11x)}{22} + C$
\end{solutionbox}

\questionmarks{4(a)(2)}{3}{મેળવો: $\int \frac{(1+x)e^x}{\cos^2(xe^x)} dx$}

\begin{solutionbox}
ધારો કે $u = xe^x$, તો $du = (1+x)e^x dx$

The integral becomes:
$\int \frac{du}{\cos^2 u} = \int \sec^2 u \, du = \tan u + C$

પાછું મૂકતા:
$= \tan(xe^x) + C$

\textbf{જવાબ}: $\tan(xe^x) + C$
\end{solutionbox}

\questionmarks{4(a)(3)}{3}{માહિતી: 6, 7, 10, 12, 13, 4, 8, 12 માટે પ્રમાણિત વિચલન શોધો.}

\begin{solutionbox}
Data: 6, 7, 10, 12, 13, 4, 8, 12
$n = 8$

મધ્યક = $\frac{6+7+10+12+13+4+8+12}{8} = \frac{72}{8} = 9$

\begin{answertable}{પ્રમાણિત વિચલન ગણતરી}
\begin{tabulary}{\linewidth}{|C|C|C|}
\hline
x & x-9 & (x-9)$^2$ \\ \hline
6 & -3 & 9 \\ \hline
7 & -2 & 4 \\ \hline
10 & 1 & 1 \\ \hline
12 & 3 & 9 \\ \hline
13 & 4 & 16 \\ \hline
4 & -5 & 25 \\ \hline
8 & -1 & 1 \\ \hline
12 & 3 & 9 \\ \hline
\end{tabulary}
\end{answertable}

$\sum(x-9)^2 = 74$

Standard deviation = $\sqrt{\frac{\sum(x-\bar{x})^2}{n}} = \sqrt{\frac{74}{8}} = \sqrt{9.25} = 3.04$

\textbf{જવાબ}: $\sigma = 3.04$
\end{solutionbox}

\questionmarks{4(b)}{8}{}

\questionmarks{4(b)(1)}{4}{મેળવો: $\int \frac{2x+1}{(x+1)(x-3)} dx$}

\begin{solutionbox}
આંશિક અપૂર્ણાંકનો ઉપયોગ કરતા:
$\frac{2x+1}{(x+1)(x-3)} = \frac{A}{x+1} + \frac{B}{x-3}$

$2x+1 = A(x-3) + B(x+1)$

જ્યારે $x = -1$: $2(-1)+1 = A(-4) \Rightarrow -1 = -4A \Rightarrow A = \frac{1}{4}$

જ્યારે $x = 3$: $2(3)+1 = B(4) \Rightarrow 7 = 4B \Rightarrow B = \frac{7}{4}$

$\int \frac{2x+1}{(x+1)(x-3)} dx = \frac{1}{4}\int \frac{1}{x+1} dx + \frac{7}{4}\int \frac{1}{x-3} dx$

$= \frac{1}{4}\ln|x+1| + \frac{7}{4}\ln|x-3| + C$

\textbf{જવાબ}: $\frac{1}{4}\ln|x+1| + \frac{7}{4}\ln|x-3| + C$
\end{solutionbox}

\questionmarks{4(b)(2)}{4}{મેળવો: $\int_0^{\pi/2} \frac{\sqrt{\cot x}}{\sqrt{\cot x} + \sqrt{\tan x}} dx$}

\begin{solutionbox}
ધારો કે $I = \int_0^{\pi/2} \frac{\sqrt{\cot x}}{\sqrt{\cot x} + \sqrt{\tan x}} dx$

ગુણધર્મ $\int_0^a f(x) dx = \int_0^a f(a-x) dx$ નો ઉપયોગ કરતા:

$I = \int_0^{\pi/2} \frac{\sqrt{\cot(\pi/2-x)}}{\sqrt{\cot(\pi/2-x)} + \sqrt{\tan(\pi/2-x)}} dx$

Since $\cot(\pi/2-x) = \tan x$ and $\tan(\pi/2-x) = \cot x$:

$I = \int_0^{\pi/2} \frac{\sqrt{\tan x}}{\sqrt{\tan x} + \sqrt{\cot x}} dx$

બંને પદનો સરવાળો કરતા:
$2I = \int_0^{\pi/2} \frac{\sqrt{\cot x} + \sqrt{\tan x}}{\sqrt{\cot x} + \sqrt{\tan x}} dx = \int_0^{\pi/2} 1 \, dx = \frac{\pi}{2}$

\textbf{જવાબ}: $I = \frac{\pi}{4}$
\end{solutionbox}

\questionmarks{4(b)(3)}{4}{વર્ગીકૃત માહિતી માટે સરેરાશ વિચલન શોધો}

\begin{solutionbox}
\begin{answertable}{વર્ગીકૃત માહિતી}
\begin{tabulary}{\linewidth}{|C|C|C|C|C|C|C|C|}
\hline
$x_i$ & 4 & 8 & 11 & 17 & 20 & 24 & 32 \\ \hline
$f_i$ & 3 & 5 & 9 & 5 & 4 & 3 & 1 \\ \hline
\end{tabulary}
\end{answertable}

$N = \sum f_i = 3+5+9+5+4+3+1 = 30$

મધ્યક = $\frac{\sum f_i x_i}{N} = \frac{3(4)+5(8)+9(11)+5(17)+4(20)+3(24)+1(32)}{30}$

$= \frac{12+40+99+85+80+72+32}{30} = \frac{420}{30} = 14$

\begin{answertable}{સરેરાશ વિચલન ગણતરી}
\begin{tabulary}{\linewidth}{|C|C|C|C|}
\hline
$x_i$ & $f_i$ & $|x_i-14|$ & $f_i|x_i-14|$ \\ \hline
4 & 3 & 10 & 30 \\ \hline
8 & 5 & 6 & 30 \\ \hline
11 & 9 & 3 & 27 \\ \hline
17 & 5 & 3 & 15 \\ \hline
20 & 4 & 6 & 24 \\ \hline
24 & 3 & 10 & 30 \\ \hline
32 & 1 & 18 & 18 \\ \hline
\end{tabulary}
\end{answertable}

$\sum f_i|x_i-14| = 174$

Mean deviation = $\frac{\sum f_i |x_i - \bar{x}|}{N} = \frac{174}{30} = 5.8$

\textbf{જવાબ}: Mean deviation = 5.8
\end{solutionbox}

\questionmarks{5}{14}{કોઈપણ બે ગણો}

\questionmarks{5(a)}{6}{}

\questionmarks{5(a)(1)}{3}{વર્ગીકૃત માહિતી માટે સરેરાશ વિચલન શોધો}

\begin{solutionbox}
\begin{answertable}{વર્ગીકૃત માહિતી}
\begin{tabulary}{\linewidth}{|C|C|C|C|C|C|C|C|}
\hline
Class & 30-40 & 40-50 & 50-60 & 60-70 & 70-80 & 80-90 & 90-100 \\ \hline
Freq & 3 & 7 & 12 & 15 & 8 & 3 & 2 \\ \hline
\end{tabulary}
\end{answertable}

$N = 50, \sum f_i x_i = 3100$

મધ્યક = 3100/50 = 62

\begin{answertable}{સરેરાશ વિચલન ગણતરી}
\begin{tabulary}{\linewidth}{|C|C|C|C|C|}
\hline
Class & $x_i$ & $f_i$ & $|x_i-62|$ & $f_i|x_i-62|$ \\ \hline
30-40 & 35 & 3 & 27 & 81 \\ \hline
40-50 & 45 & 7 & 17 & 119 \\ \hline
50-60 & 55 & 12 & 7 & 84 \\ \hline
60-70 & 65 & 15 & 3 & 45 \\ \hline
70-80 & 75 & 8 & 13 & 104 \\ \hline
80-90 & 85 & 3 & 23 & 69 \\ \hline
90-100 & 95 & 2 & 33 & 66 \\ \hline
\end{tabulary}
\end{answertable}

સરેરાશ વિચલન = 568/50 = 11.36

\textbf{જવાબ}: Mean deviation = 11.36
\end{solutionbox}

\questionmarks{5(a)(2)}{3}{આપેલ માહિતી માટે પ્રમાણિત વિચલન શોધો}

\begin{solutionbox}
\begin{answertable}{વર્ગીકૃત માહિતી}
\begin{tabulary}{\linewidth}{|C|C|C|C|C|C|C|C|C|C|}
\hline
Class & 60 & 61 & 62 & 63 & 64 & 65 & 66 & 67 & 68 \\ \hline
Freq & 2 & 1 & 12 & 29 & 25 & 12 & 10 & 4 & 5 \\ \hline
\end{tabulary}
\end{answertable}

$N = 100$, મધ્યક = 63.8

\begin{answertable}{પ્રમાણિત વિચલન ગણતરી}
\begin{tabulary}{\linewidth}{|C|C|C|C|C|}
\hline
$x_i$ & $f_i$ & $(x_i-63.8)$ & $(x_i-63.8)^2$ & $f_i(x_i-63.8)^2$ \\ \hline
60 & 2 & -3.8 & 14.44 & 28.88 \\ \hline
61 & 1 & -2.8 & 7.84 & 7.84 \\ \hline
62 & 12 & -1.8 & 3.24 & 38.88 \\ \hline
63 & 29 & -0.8 & 0.64 & 18.56 \\ \hline
64 & 25 & 0.2 & 0.04 & 1.00 \\ \hline
65 & 12 & 1.2 & 1.44 & 17.28 \\ \hline
66 & 10 & 2.2 & 4.84 & 48.40 \\ \hline
67 & 4 & 3.2 & 10.24 & 40.96 \\ \hline
68 & 5 & 4.2 & 17.64 & 88.20 \\ \hline
\end{tabulary}
\end{answertable}

$\sum f_i(x_i-\bar{x})^2 = 290$

પ્રમાણિત વિચલન = $\sqrt{290/100} = \sqrt{2.9} = 1.70$

\textbf{જવાબ}: $\sigma = 1.70$
\end{solutionbox}

\questionmarks{5(a)(3)}{3}{વર્ગીકૃત માહિતી માટે મધ્યક શોધો}

\begin{solutionbox}
\begin{answertable}{વર્ગીકૃત માહિતી}
\begin{tabulary}{\linewidth}{|C|C|C|C|C|C|C|}
\hline
Class & 0-20 & 20-40 & 40-60 & 60-80 & 80-100 & 100-120 \\ \hline
Freq & 26 & 31 & 35 & 42 & 82 & 71 \\ \hline
\end{tabulary}
\end{answertable}

\begin{answertable}{મધ્યક ગણતરી}
\begin{tabulary}{\linewidth}{|C|C|C|C|}
\hline
Class & Mid-value & $f_i$ & $f_i x_i$ \\ \hline
0-20 & 10 & 26 & 260 \\ \hline
20-40 & 30 & 31 & 930 \\ \hline
40-60 & 50 & 35 & 1750 \\ \hline
60-80 & 70 & 42 & 2940 \\ \hline
80-100 & 90 & 82 & 7380 \\ \hline
100-120 & 110 & 71 & 7810 \\ \hline
\end{tabulary}
\end{answertable}

$N = 287, \sum f_i x_i = 21070$

મધ્યક = $\frac{\sum f_i x_i}{N} = \frac{21070}{287} = 73.42$

\textbf{જવાબ}: Mean = 73.42
\end{solutionbox}

\questionmarks{5(b)}{8}{}

\questionmarks{5(b)(1)}{4}{વિકલ સમીકરણ $(x + y + 1)^2 \frac{dy}{dx} = 1$ ઉકેલો}

\begin{solutionbox}
ધારો કે $z = x + y + 1$, તો $\frac{dz}{dx} = 1 + \frac{dy}{dx}$
તેથી $\frac{dy}{dx} = \frac{dz}{dx} - 1$

કિંમત મૂકતા: $z^2(\frac{dz}{dx} - 1) = 1$
$z^2 \frac{dz}{dx} - z^2 = 1$
$z^2 \frac{dz}{dx} = 1 + z^2$
$\frac{z^2}{1 + z^2} dz = dx$

બંને બાજુ સંકલન કરતા:
$\int \frac{z^2}{1 + z^2} dz = \int dx$

$\int \left(1 - \frac{1}{1 + z^2}\right) dz = x + C$

$z - \tan^{-1}z = x + C$

$z = x + y + 1$ પાછું મૂકતા:
$(x + y + 1) - \tan^{-1}(x + y + 1) = x + C$

\textbf{જવાબ}: $y + 1 = \tan^{-1}(x + y + 1) + C$
\end{solutionbox}

\questionmarks{5(b)(2)}{4}{ઉકેલો: $\frac{dy}{dx} + \frac{y}{x} = e^x$, $y(0) = 2$}

\begin{solutionbox}
આ $\frac{dy}{dx} + P(x)y = Q(x)$ સ્વરૂપનું સુરેખ વિકલ સમીકરણ છે.

અહીં $P(x) = \frac{1}{x}, Q(x) = e^x$

સંકલ્યકારક અવયવ: $I.F. = e^{\int \frac{1}{x} dx} = e^{\ln|x|} = |x| = x$ ($x > 0$ માટે)

સમીકરણને $x$ વડે ગુણતા:
$x \frac{dy}{dx} + y = xe^x$

$\frac{d}{dx}(xy) = xe^x$

બંને બાજુ સંકલન કરતા:
$xy = \int xe^x dx$

$\int xe^x dx$ માટે ખંડશઃ સંકલનનો ઉપયોગ કરતા:
ધારો કે $u = x, dv = e^x dx$
તો $du = dx, v = e^x$

$\int xe^x dx = xe^x - \int e^x dx = xe^x - e^x = e^x(x-1)$

So: $xy = e^x(x-1) + C$
$y = \frac{e^x(x-1) + C}{x}$

પ્રારંભિક શરત $y(0) = 2$ નો ઉપયોગ કરતા:
જેમ $x \to 0$, આપણે L'Hôpital ના નિયમ કે શ્રેણી વિસ્તરણનો ઉપયોગ કરવો પડે.

$x = 0$ આગળ મૂળ સમીકરણ પરથી: $\frac{dy}{dx} = e^x - \frac{y}{x}$
આ સૂચવે છે કે આપણે પ્રારંભિક શરત સાથે વધુ સાવચેત રહેવું જોઈએ.

\textbf{વૈકલ્પિક અભિગમ}: સમીકરણ $x = 0$ આગળ અસામાન્ય છે, તેથી આપણે $x \neq 0$ હોય તેવા વિસ્તારમાં ઉકેલીએ છીએ.

\textbf{જવાબ}: $y = \frac{e^x(x-1) + C}{x}$ જ્યાં C સીમા શરતો દ્વારા નક્કી થાય છે.
\end{solutionbox}

\questionmarks{5(b)(3)}{4}{ઉકેલો: $y \frac{dy}{dx} = \sqrt{1 + x^2 + y^2 + x^2y^2}$}

\begin{solutionbox}
$y \frac{dy}{dx} = \sqrt{1 + x^2 + y^2 + x^2y^2}$

$y \frac{dy}{dx} = \sqrt{(1 + x^2)(1 + y^2)}$

$\frac{y dy}{\sqrt{1 + y^2}} = \sqrt{1 + x^2} dx$

બંને બાજુ સંકલન કરતા:
$\int \frac{y dy}{\sqrt{1 + y^2}} = \int \sqrt{1 + x^2} dx$

ડાબી બાજુ માટે, ધારો કે $u = 1 + y^2$, તો $du = 2y dy$:
$\int \frac{y dy}{\sqrt{1 + y^2}} = \frac{1}{2} \int \frac{du}{\sqrt{u}} = \sqrt{u} = \sqrt{1 + y^2}$

જમણી બાજુ માટે:
$\int \sqrt{1 + x^2} dx = \frac{x\sqrt{1 + x^2}}{2} + \frac{1}{2}\ln|x + \sqrt{1 + x^2}| + C$

Therefore:
\textbf{જવાબ}: $\sqrt{1 + y^2} = \frac{x\sqrt{1 + x^2}}{2} + \frac{1}{2}\ln|x + \sqrt{1 + x^2}| + C$
\end{solutionbox}

\section*{સૂત્રો}

\subsection*{શ્રેણિક પ્રક્રિયાઓ}
\begin{itemize}
\item $(A + B)^T = A^T + B^T$
\item $(AB)^T = B^T A^T$
\item $A \cdot adj(A) = |A| \cdot I$
\item 2×2 શ્રેણિક $\begin{bmatrix} a & b \\ c & d \end{bmatrix}$ માટે: $adj = \begin{bmatrix} d & -b \\ -c & a \end{bmatrix}$
\end{itemize}

\subsection*{વિકલન સૂત્રો}
\begin{itemize}
\item $\frac{d}{dx}(\sin x) = \cos x$
\item $\frac{d}{dx}(\cos x) = -\sin x$
\item $\frac{d}{dx}(\tan x) = \sec^2 x$
\item $\frac{d}{dx}(\log x) = \frac{1}{x}$
\item $\frac{d}{dx}(e^x) = e^x$
\item સાંકળ નિયમ: $\frac{d}{dx}f(g(x)) = f'(g(x)) \cdot g'(x)$
\end{itemize}

\subsection*{સંકલન સૂત્રો}
\begin{itemize}
\item $\int \sin x \, dx = -\cos x + C$
\item $\int \cos x \, dx = \sin x + C$
\item $\int \sec^2 x \, dx = \tan x + C$
\item $\int \frac{1}{x} dx = \ln|x| + C$
\item $\int e^x dx = e^x + C$
\item $\int \frac{1}{x^2 + a^2} dx = \frac{1}{a}\tan^{-1}(\frac{x}{a}) + C$
\end{itemize}

\subsection*{વિકલ સમીકરણો}
\begin{itemize}
\item \textbf{સુરેખ વિકલ સમીકરણ}: $\frac{dy}{dx} + P(x)y = Q(x)$
\item \textbf{સંકલ્યકારક અવયવ}: $I.F. = e^{\int P(x) dx}$
\item \textbf{વિયોજનીય ચલ}: $\frac{dy}{dx} = f(x)g(y) \Rightarrow \frac{dy}{g(y)} = f(x)dx$
\end{itemize}

\subsection*{આંકડાશાસ્ત્ર}
\begin{itemize}
\item \textbf{મધ્યક}: $\bar{x} = \frac{\sum x_i}{n}$ (અવર્ગીકૃત), $\bar{x} = \frac{\sum f_i x_i}{\sum f_i}$ (વર્ગીકૃત)
\item \textbf{સરેરાશ વિચલન}: $M.D. = \frac{\sum |x_i - \bar{x}|}{n}$
\item \textbf{પ્રમાણિત વિચલન}: $\sigma = \sqrt{\frac{\sum (x_i - \bar{x})^2}{n}}$
\end{itemize}

\end{document}
