\documentclass{article}

% content/resources/templates/preamble.tex
\usepackage[margin=0.6in]{geometry}
\author{Milav Dabgar}
\usepackage{amsmath,amssymb,amsthm}
\usepackage{booktabs}
\usepackage{multirow}
\usepackage{xcolor}
\usepackage{tcolorbox}
\tcbuselibrary{breakable,skins}
\usepackage[colorlinks=true,linkcolor=blue]{hyperref}
\usepackage{titlesec}
\usepackage{enumitem}
\usepackage{tikz}
\usepackage{pgfplots}
\usepackage{circuitikz}
\usepackage[version=4]{mhchem}
\usepackage{longtable}
\usepackage{array}
\usepackage{float}
\usepackage{caption}
\usepackage{listings}

\lstset{
  basicstyle=\small\ttfamily,
  breaklines=true,
  breakatwhitespace=false,
  postbreak=\mbox{\textcolor{red}{$\hookrightarrow$}\space},
  float=false,
  numbers=left,
  numberstyle=\tiny\color{gray},
  numbersep=10pt,
  xleftmargin=2em,
  keywordstyle=\color{blue},
  commentstyle=\color{green!60!black},
  stringstyle=\color{purple},
  backgroundcolor=\color{gray!5},
  showstringspaces=false,
  tabsize=2,
  captionpos=b,
  keepspaces=true,
  columns=flexible
}

\pgfplotsset{compat=1.18}
\usetikzlibrary{shapes,arrows,positioning,calc,patterns,decorations.pathmorphing,decorations.markings,arrows.meta}

% Color scheme
\definecolor{headcolor}{RGB}{0,102,204}
\definecolor{keycolor}{RGB}{220,20,60}
\definecolor{solutioncolor}{RGB}{34,139,34}
\definecolor{mnemoniccolor}{RGB}{148,0,211}
\definecolor{codecolor}{RGB}{0,0,100}

% Spacing
\setlength{\parskip}{3pt}
\setlist[itemize]{nosep}
\setlist[enumerate]{nosep}

% Title formatting
\titleformat{\section}{\Large\bfseries\color{headcolor}}{\thesection}{1em}{}
\titleformat{\subsection}{\large\bfseries\color{headcolor}}{\thesubsection}{1em}{}

% Pandoc tightlist compatibility
\providecommand{\tightlist}{%
  \setlength{\itemsep}{0pt}\setlength{\parskip}{0pt}}

% Pandoc longtable compatibility
\newcounter{none}
\def\thenone{}


% content/resources/templates/gujarati-boxes.tex
\usepackage{fontspec}
\usepackage{polyglossia}

% Set Gujarati as main language (document is primarily in Gujarati)
% Note: gloss-gujarati.ldf doesn't exist in polyglossia, but it will use hyphenation patterns
\setdefaultlanguage{gujarati}
\setotherlanguage{english}

% Configure Gujarati font properly
% Use Language=Default to prevent polyglossia from trying to add language-specific features
% that don't exist for Gujarati, which causes "empty feature" warnings
\newfontfamily\gujaratifont[Script=Gujarati,AutoFakeBold=2.5,AutoFakeSlant=0.3]{Noto Sans Gujarati}
\setmainfont[Script=Gujarati,AutoFakeBold=2.5,AutoFakeSlant=0.3]{Noto Sans Gujarati}
% Use Noto Sans Gujarati for monospace to support Gujarati in text
\setmonofont[Scale=0.9]{Noto Sans Gujarati}

% Configure English to use the same font
\newfontfamily\englishfont[Script=Gujarati,AutoFakeBold=2.5,AutoFakeSlant=0.3]{Noto Sans Gujarati}

% Translations for polyglossia
\gappto\captionsgujarati{
  \renewcommand{\tablename}{કોષ્ટક}
  \renewcommand{\figurename}{આકૃતિ}
}

% Helper for TikZ nodes to ensure Gujarati font
\newcommand{\gu}[1]{{\gujaratifont #1}}

% Custom environments
\newtcolorbox{solutionbox}{
    breakable,
    enhanced,
    colback=solutioncolor!5!white,
    colframe=solutioncolor!75!black,
    fonttitle=\bfseries,
    title=જવાબ
}

\newtcolorbox{solutionboxnobreak}{
 colback=solutioncolor!5!white,
 colframe=solutioncolor!75!black,
 fonttitle=\bfseries,
 title=જવાબ
}

\newtcolorbox{keyformula}{
 breakable,
 enhanced,
 colback=keycolor!5!white,
 colframe=keycolor!75!black,
 fonttitle=\bfseries,
 title=રાસાયણિક સમીકરણ/સૂત્ર
}

\newtcolorbox{mnemonicbox}{
 breakable,
 enhanced,
 colback=mnemoniccolor!5!white,
 colframe=mnemoniccolor!75!black,
 fonttitle=\bfseries,
 title=મેમરી ટ્રીક
}


% Custom commands for GTU solutions
% This file defines semantic commands for consistent formatting

% Question command with automatic formatting
\newcommand{\question}[2]{%
  \section*{Question #1}%
  \textbf{#2}%
}

% OR question variant
\newcommand{\questionor}[2]{%
  \section*{Question #1 OR}%
  \textbf{#2}%
}

% Proper table environment with caption
\newenvironment{answertable}[1]{%
  \begin{table}[htbp]
  \centering
  \caption{#1}
}{%
  \end{table}
}

% Proper figure environment for diagrams
\newenvironment{answerdiagram}[1]{%
  \begin{figure}[htbp]
  \centering
  \caption{#1}
}{%
  \end{figure}
}

% Semantic markup for key terms
\newcommand{\keyword}[1]{\textbf{#1}}
\newcommand{\code}[1]{\texttt{#1}}
\newcommand{\classname}[1]{\texttt{#1}}
\newcommand{\methodname}[1]{\texttt{#1}}

% Proper quotation marks
\newcommand{\mnemonic}[1]{``#1''}


\title{Applied Mathematics (4320001) - Summer 2024 Solution}
\date{June 25, 2024}

\begin{document}
\maketitle

\questionmarks{1}{14}{ખાલી જગ્યા પૂરો}

\questionmarks{1.1}{1}{શ્રેણિક $\begin{bmatrix} 1 & 2 & 3 \\ -4 & 5 & 6 \end{bmatrix}$ ની કક્ષા  = \_\_\_\_\_\_\_\_\_\_\_ છે}
\textbf{જવાબ}: (b) $2 \times 3$

\begin{solutionbox}
શ્રેણિકમાં 2 હાર અને 3 સ્તંભ છે, તેથી કક્ષા $2 \times 3$ છે.
\end{solutionbox}

\questionmarks{1.2}{1}{જો $\begin{bmatrix} x-3 & 2 \\ 4 & 0 \end{bmatrix} = \begin{bmatrix} 5 & 2 \\ 4 & 0 \end{bmatrix}$ તો $x$ = \_\_\_\_\_\_\_\_}
\textbf{જવાબ}: (d) 8

\begin{solutionbox}
સમાન શ્રેણિક માટે, અનુરૂપ ઘટકો સમાન હોવા જોઈએ:
$x - 3 = 5$
$x = 8$
\end{solutionbox}

\questionmarks{1.3}{1}{શ્રેણિક $\begin{bmatrix} -3 & 2 \\ 0 & 1 \end{bmatrix}$ નો સહઅવયવજ શ્રેણિક = \_\_\_\_\_\_\_\_\_\_\_\_\_}
\textbf{જવાબ}: (b) $\begin{bmatrix} 1 & -2 \\ 0 & -3 \end{bmatrix}$

\begin{solutionbox}
શ્રેણિક $A = \begin{bmatrix} a & b \\ c & d \end{bmatrix}$, $\text{adj}(A) = \begin{bmatrix} d & -b \\ -c & a \end{bmatrix}$ માટે
$\text{adj}\begin{bmatrix} -3 & 2 \\ 0 & 1 \end{bmatrix} = \begin{bmatrix} 1 & -2 \\ 0 & -3 \end{bmatrix}$
\end{solutionbox}

\questionmarks{1.4}{1}{કોઈપણ ચોરસ શ્રેણિક $A$ માટે, $(A^{-1})^{-1}$ = \_\_\_\_\_\_\_\_\_\_\_\_}
\textbf{જવાબ}: (b) $A$

\begin{solutionbox}
વ્યસ્ત શ્રેણિકની વ્યાખ્યા મુજબ: $(A^{-1})^{-1} = A$
\end{solutionbox}

\questionmarks{1.5}{1}{$\frac{d}{dx} \log x$ = \_\_\_\_\_\_\_\_\_}
\textbf{જવાબ}: (b) $\frac{1}{x}$

\begin{solutionbox}
લઘુગણકનું વિકલન: $\frac{d}{dx} \log x = \frac{1}{x}$
\end{solutionbox}

\questionmarks{1.6}{1}{$\frac{d}{dx}(\tan^{-1} x + \cot^{-1} x)$ = \_\_\_\_\_\_\_}
\textbf{જવાબ}: (d) 0

\begin{solutionbox}
$\tan^{-1} x + \cot^{-1} x = \frac{\pi}{2}$ (અચળ)
તેથી, $\frac{d}{dx}(\tan^{-1} x + \cot^{-1} x) = 0$
\end{solutionbox}

\questionmarks{1.7}{1}{જો $x = a \cos \theta$, $y = a \sin \theta$ તો $\frac{dy}{dx}$ = \_\_\_\_\_\_\_\_\_\_}
\textbf{જવાબ}: (a) $-\cot \theta$

\begin{solutionbox}
$\frac{dx}{d\theta} = -a \sin \theta$, $\frac{dy}{d\theta} = a \cos \theta$
$\frac{dy}{dx} = \frac{dy/d\theta}{dx/d\theta} = \frac{a \cos \theta}{-a \sin \theta} = -\cot \theta$
\end{solutionbox}

\questionmarks{1.8}{1}{$\int 5x^4 dx$ = \_\_\_\_\_\_\_\_\_\_\_\_ + $c$}
\textbf{જવાબ}: (d) $x^5$

\begin{solutionbox}
$\int 5x^4 dx = 5 \cdot \frac{x^5}{5} = x^5 + c$
\end{solutionbox}

\questionmarks{1.9}{1}{$\int_0^1 e^x dx$ = \_\_\_\_\_\_\_\_\_\_}
\textbf{જવાબ}: (a) $e - 1$

\begin{solutionbox}
$\int_0^1 e^x dx = [e^x]_0^1 = e^1 - e^0 = e - 1$
\end{solutionbox}

\questionmarks{1.10}{1}{$\int_{-1}^1 3x^2 - 2x + 1 dx$ = \_\_\_\_\_\_\_\_\_\_}
\textbf{જવાબ}: (c) 4

\begin{solutionbox}
$\int_{-1}^1 (3x^2 - 2x + 1) dx = [x^3 - x^2 + x]_{-1}^1$
$= (1 - 1 + 1) - (-1 - 1 - 1) = 1 - (-3) = 4$
\end{solutionbox}

\questionmarks{1.11}{1}{વિકલ સમીકરણ $(\frac{dy}{dx})^2 + 4y = x$ ની કક્ષા \_\_\_\_\_\_\_\_\_\_\_ છે}
\textbf{જવાબ}: (d) 1

\begin{solutionbox}
કક્ષા એ ઉચ્ચતમ વિકલિત છે. અહીં, ફક્ત પ્રથમ વિકલિત $\frac{dy}{dx}$ છે, તેથી કક્ષા = 1.
\end{solutionbox}

\questionmarks{1.12}{1}{વિકલ સમીકરણ $\frac{dy}{dx} + 3y = x$ નો સંકલ્યકારક અવયવ \_\_\_\_\_\_\_\_\_\_\_\_\_ છે}
\textbf{જવાબ}: (d) $e^{3x}$

\begin{solutionbox}
સુરેખ વિકલ સમીકરણ $\frac{dy}{dx} + Py = Q$ માટે, સંકલ્યકારક અવયવ = $e^{\int P dx}$
અહીં $P = 3$, તેથી I.F. = $e^{\int 3 dx} = e^{3x}$
\end{solutionbox}

\questionmarks{1.13}{1}{પ્રથમ દસ પ્રાકૃતિક સંખ્યાઓનો મધ્યક \_\_\_\_\_\_\_\_\_ છે}
\textbf{જવાબ}: (a) 5.5

\begin{solutionbox}
મધ્યક = $\frac{1 + 2 + 3 + ... + 10}{10} = \frac{55}{10} = 5.5$
\end{solutionbox}

\questionmarks{1.14}{1}{માહિતી 17, 15, 25, 34, 32 નો વિસ્તાર \_\_\_\_\_\_\_\_\_\_\_\_\_\_\_ છે}
\textbf{જવાબ}: (d) 19

\begin{solutionbox}
વિસ્તાર = મહત્તમ - ન્યૂનતમ = 34 - 15 = 19
\end{solutionbox}

\questionmarks{2(a)}{6}{કોઈપણ બે ગણો}

\questionmarks{2(a)(1)}{3}{જો $A = \begin{bmatrix} 1 & -1 \\ 2 & 3 \end{bmatrix}$ તો $A + A^T + I$ શોધો.}

\begin{solutionbox}
$A = \begin{bmatrix} 1 & -1 \\ 2 & 3 \end{bmatrix}$

$A^T = \begin{bmatrix} 1 & 2 \\ -1 & 3 \end{bmatrix}$

$I = \begin{bmatrix} 1 & 0 \\ 0 & 1 \end{bmatrix}$

$A + A^T + I = \begin{bmatrix} 1 & -1 \\ 2 & 3 \end{bmatrix} + \begin{bmatrix} 1 & 2 \\ -1 & 3 \end{bmatrix} + \begin{bmatrix} 1 & 0 \\ 0 & 1 \end{bmatrix}$

$= \begin{bmatrix} 3 & 1 \\ 1 & 7 \end{bmatrix}$
\end{solutionbox}

\questionmarks{2(a)(2)}{3}{જો $A = \begin{bmatrix} 2 & 3 \\ -1 & 2 \end{bmatrix}$ તો સાબિત કરો કે $A^2 - 4A + 7I_2 = 0$}
\textbf{જવાબ}: સાબિત

\begin{solutionbox}
$A = \begin{bmatrix} 2 & 3 \\ -1 & 2 \end{bmatrix}$

$A^2 = \begin{bmatrix} 2 & 3 \\ -1 & 2 \end{bmatrix} \begin{bmatrix} 2 & 3 \\ -1 & 2 \end{bmatrix} = \begin{bmatrix} 1 & 12 \\ -4 & 1 \end{bmatrix}$

$4A = 4\begin{bmatrix} 2 & 3 \\ -1 & 2 \end{bmatrix} = \begin{bmatrix} 8 & 12 \\ -4 & 8 \end{bmatrix}$

$7I_2 = \begin{bmatrix} 7 & 0 \\ 0 & 7 \end{bmatrix}$

$A^2 - 4A + 7I_2 = \begin{bmatrix} 1 & 12 \\ -4 & 1 \end{bmatrix} - \begin{bmatrix} 8 & 12 \\ -4 & 8 \end{bmatrix} + \begin{bmatrix} 7 & 0 \\ 0 & 7 \end{bmatrix}$

$= \begin{bmatrix} 0 & 0 \\ 0 & 0 \end{bmatrix} = 0$ \checkmark
\end{solutionbox}

\questionmarks{2(a)(3)}{3}{વિકલ સમીકરણ ઉકેલો: $dy - 3x^2e^{-y}dx = 0$}
\textbf{જવાબ}: $e^y = x^3 + C$

\begin{solutionbox}
$dy - 3x^2e^{-y}dx = 0$
$dy = 3x^2e^{-y}dx$
$e^y dy = 3x^2 dx$

બંને બાજુ સંકલન કરતા:
$\int e^y dy = \int 3x^2 dx$
$e^y = x^3 + C$
\end{solutionbox}

\questionmarks{2(b)}{8}{કોઈપણ બે ગણો}

\questionmarks{2(b)(1)}{4}{શ્રેણિક $\begin{bmatrix} 3 & -1 & 2 \\ 4 & 1 & -1 \\ 5 & 0 & 1 \end{bmatrix}$ નો વ્યસ્ત શ્રેણિક શોધો}
\textbf{જવાબ}: $A^{-1} = \begin{bmatrix} 1/14 & 1/14 & -1/14 \\ -9/14 & -7/14 & 11/14 \\ -5/14 & -5/14 & 1/2 \end{bmatrix}$

\begin{solutionbox}
ધારો કે $A = \begin{bmatrix} 3 & -1 & 2 \\ 4 & 1 & -1 \\ 5 & 0 & 1 \end{bmatrix}$

સૌપ્રથમ, $\det(A)$ શોધો:
$\det(A) = 3(1 \cdot 1 - (-1) \cdot 0) - (-1)(4 \cdot 1 - (-1) \cdot 5) + 2(4 \cdot 0 - 1 \cdot 5)$
$= 3(1) + 1(9) + 2(-5) = 3 + 9 - 10 = 2$

$\det(A) \neq 0$ હોવાથી, વ્યસ્ત શ્રેણિક અસ્તિત્વ ધરાવે છે.

સહઅવયવ અને સહઅવયવજ શ્રેણિક શોધતા:
$C_{11} = 1$, $C_{12} = -9$, $C_{13} = -5$
$C_{21} = 1$, $C_{22} = -7$, $C_{23} = -5$  
$C_{31} = -1$, $C_{32} = 11$, $C_{33} = 7$

$\text{adj}(A) = \begin{bmatrix} 1 & 1 & -1 \\ -9 & -7 & 11 \\ -5 & -5 & 7 \end{bmatrix}$

$A^{-1} = \frac{1}{\det(A)} \cdot \text{adj}(A) = \frac{1}{2} \begin{bmatrix} 1 & 1 & -1 \\ -9 & -7 & 11 \\ -5 & -5 & 7 \end{bmatrix}$
\end{solutionbox}

\questionmarks{2(b)(2)}{4}{જો $A + B = \begin{bmatrix} 1 & -1 \\ 3 & 0 \end{bmatrix}$ અને $A - B = \begin{bmatrix} 3 & 1 \\ 1 & 4 \end{bmatrix}$ તો $AB$ શોધો.}
\textbf{જવાબ}: $AB = \begin{bmatrix} 0 & -1 \\ 4 & -2 \end{bmatrix}$

\begin{solutionbox}
સમીકરણોનો સરવાળો કરતા:
$(A + B) + (A - B) = 2A$
$2A = \begin{bmatrix} 1 & -1 \\ 3 & 0 \end{bmatrix} + \begin{bmatrix} 3 & 1 \\ 1 & 4 \end{bmatrix} = \begin{bmatrix} 4 & 0 \\ 4 & 4 \end{bmatrix}$
$A = \begin{bmatrix} 2 & 0 \\ 2 & 2 \end{bmatrix}$

સમીકરણોની બાદબાકી કરતા:
$(A + B) - (A - B) = 2B$
$2B = \begin{bmatrix} 1 & -1 \\ 3 & 0 \end{bmatrix} - \begin{bmatrix} 3 & 1 \\ 1 & 4 \end{bmatrix} = \begin{bmatrix} -2 & -2 \\ 2 & -4 \end{bmatrix}$
$B = \begin{bmatrix} -1 & -1 \\ 1 & -2 \end{bmatrix}$

$AB = \begin{bmatrix} 2 & 0 \\ 2 & 2 \end{bmatrix} \begin{bmatrix} -1 & -1 \\ 1 & -2 \end{bmatrix} = \begin{bmatrix} -2 & -2 \\ 0 & -6 \end{bmatrix}$
\end{solutionbox}

\questionmarks{2(b)(3)}{4}{શ્રેણિકની મદદથી સુરેખ સમીકરણ સંહતિ $2x + 3y = 1$, $y - 4x = 2$ ઉકેલો.}
\textbf{જવાબ}: $x = -\frac{1}{11}$, $y = \frac{13}{11}$

\begin{solutionbox}
સમીકરણ સંહતિને $AX = B$ તરીકે લખી શકાય:
$\begin{bmatrix} 2 & 3 \\ -4 & 1 \end{bmatrix} \begin{bmatrix} x \\ y \end{bmatrix} = \begin{bmatrix} 1 \\ 2 \end{bmatrix}$

$\det(A) = 2(1) - 3(-4) = 2 + 12 = 14$

$A^{-1} = \frac{1}{14} \begin{bmatrix} 1 & -3 \\ 4 & 2 \end{bmatrix}$

$X = A^{-1}B = \frac{1}{14} \begin{bmatrix} 1 & -3 \\ 4 & 2 \end{bmatrix} \begin{bmatrix} 1 \\ 2 \end{bmatrix} = \frac{1}{14} \begin{bmatrix} -5 \\ 8 \end{bmatrix}$

Therefore: $x = -\frac{5}{14}$, $y = \frac{8}{14} = \frac{4}{7}$

Wait, MDX answer says $x = -1/11, y = 13/11$.
Let's check the calculation in the MDX text.
MDX Solution says:
$X = \frac{1}{14} \begin{bmatrix} -5 \\ 8 \end{bmatrix}$
So $x = -5/14, y = 8/14 = 4/7$.
But MDX Answer block says: $x = -1/11, y = 13/11$.
There is a contradiction in the source MDX.
Let's calculate $\det(A)$ for $2x+3y=1$ and $y-4x=2 \implies -4x+y=2$.
$A = \begin{bmatrix} 2 & 3 \\ -4 & 1 \end{bmatrix}$. $\det = 2 - (-12) = 14$.
$A^{-1} = \frac{1}{14} \begin{bmatrix} 1 & -3 \\ 4 & 2 \end{bmatrix}$.
$X = \frac{1}{14} \begin{bmatrix} 1 & -3 \\ 4 & 2 \end{bmatrix} \begin{bmatrix} 1 \\ 2 \end{bmatrix} = \frac{1}{14} \begin{bmatrix} 1-6 \\ 4+4 \end{bmatrix} = \frac{1}{14} \begin{bmatrix} -5 \\ 8 \end{bmatrix}$.
So $x = -5/14, y=4/7$.
The MDX Answer block is wrong or describes a different problem.
MDX says:
**Answer**: $x = -\frac{1}{11}$, $y = \frac{13}{11}$
**Solution**: ... Therefore: $x = -\frac{5}{14}$, $y = \frac{8}{14} = \frac{4}{7}$

I must follow the "Solution" part for the body, but usually checking what to do with the "Answer" block.
The workflow says "Strict Fidelity". I will copy EXACTLY what is in MDX, including the contradiction.
However, I will trust the computation in the Solution block for the steps.
I'll just copy the text as is.
\end{solutionbox}

\questionmarks{3(a)}{6}{કોઈપણ બે ગણો}

\questionmarks{3(a)(1)}{3}{વ્યાખ્યાની મદદથી $f(x) = e^x$ નું વિકલિત શોધો.}
\textbf{જવાબ}: $f'(x) = e^x$

\begin{solutionbox}
વ્યાખ્યાનો ઉપયોગ કરતા: $f'(x) = \lim_{h \to 0} \frac{f(x+h) - f(x)}{h}$

$f'(x) = \lim_{h \to 0} \frac{e^{x+h} - e^x}{h}$
$= \lim_{h \to 0} \frac{e^x \cdot e^h - e^x}{h}$
$= e^x \lim_{h \to 0} \frac{e^h - 1}{h}$
$= e^x \cdot 1 = e^x$
\end{solutionbox}

\questionmarks{3(a)(2)}{3}{જો $\sqrt{x} + \sqrt{y} = \sqrt{a}$ તો સાબિત કરો કે $\frac{dy}{dx} = -\sqrt{\frac{y}{x}}$}
\textbf{જવાબ}: સાબિત

\begin{solutionbox}
$\sqrt{x} + \sqrt{y} = \sqrt{a}$

$x$ ની સાપેક્ષે વિકલન કરતા:
$\frac{1}{2\sqrt{x}} + \frac{1}{2\sqrt{y}} \cdot \frac{dy}{dx} = 0$

$\frac{1}{2\sqrt{y}} \cdot \frac{dy}{dx} = -\frac{1}{2\sqrt{x}}$

$\frac{dy}{dx} = -\frac{\sqrt{y}}{\sqrt{x}} = -\sqrt{\frac{y}{x}}$ \checkmark
\end{solutionbox}

\questionmarks{3(a)(3)}{3}{કિંમત શોધો: $\int \frac{\tan x}{\sec x + \tan x} dx$}
\textbf{જવાબ}: $x - \ln|\sec x + \tan x| + C$

\begin{solutionbox}
ધારો કે $I = \int \frac{\tan x}{\sec x + \tan x} dx$

અંશ અને છેદને $(\sec x - \tan x)$ વડે ગુણતા:
$I = \int \frac{\tan x(\sec x - \tan x)}{(\sec x + \tan x)(\sec x - \tan x)} dx$
$= \int \frac{\tan x(\sec x - \tan x)}{\sec^2 x - \tan^2 x} dx$
$= \int \frac{\tan x(\sec x - \tan x)}{1} dx$
$= \int (\tan x \sec x - \tan^2 x) dx$
$= \int \tan x \sec x dx - \int (\sec^2 x - 1) dx$
$= \sec x - \tan x + x + C$
\end{solutionbox}

\questionmarks{3(b)}{8}{કોઈપણ બે ગણો}

\questionmarks{3(b)(1)}{4}{જો $e^x + e^y = e^{x+y}$ તો $\frac{dy}{dx}$ શોધો.}
\textbf{જવાબ}: $\frac{dy}{dx} = \frac{e^x(e^y - 1)}{e^y(e^x - 1)}$

\begin{solutionbox}
$e^x + e^y = e^{x+y}$

$x$ ની સાપેક્ષે વિકલન કરતા:
$e^x + e^y \frac{dy}{dx} = e^{x+y}(1 + \frac{dy}{dx})$
$e^x + e^y \frac{dy}{dx} = e^{x+y} + e^{x+y} \frac{dy}{dx}$

પુનઃગોઠવણી કરતા:
$e^x - e^{x+y} = e^{x+y} \frac{dy}{dx} - e^y \frac{dy}{dx}$
$e^x - e^{x+y} = \frac{dy}{dx}(e^{x+y} - e^y)$

$\frac{dy}{dx} = \frac{e^x - e^{x+y}}{e^{x+y} - e^y} = \frac{e^x(1 - e^y)}{e^y(e^x - 1)} = \frac{e^x(e^y - 1)}{e^y(e^x - 1)}$
\end{solutionbox}

\questionmarks{3(b)(2)}{4}{$y = 2e^{3x} + 3e^{-2x}$ માટે, સાબિત કરો કે $\frac{d^2y}{dx^2} - \frac{dy}{dx} - 6y = 0$.}
\textbf{જવાબ}: સાબિત

\begin{solutionbox}
$y = 2e^{3x} + 3e^{-2x}$

$\frac{dy}{dx} = 6e^{3x} - 6e^{-2x}$

$\frac{d^2y}{dx^2} = 18e^{3x} + 12e^{-2x}$

હવે સમીકરણ ચકાસતા:
$\frac{d^2y}{dx^2} - \frac{dy}{dx} - 6y$
$= (18e^{3x} + 12e^{-2x}) - (6e^{3x} - 6e^{-2x}) - 6(2e^{3x} + 3e^{-2x})$
$= 18e^{3x} + 12e^{-2x} - 6e^{3x} + 6e^{-2x} - 12e^{3x} - 18e^{-2x}$
$= (18 - 6 - 12)e^{3x} + (12 + 6 - 18)e^{-2x}$
$= 0 \cdot e^{3x} + 0 \cdot e^{-2x} = 0$ \checkmark
\end{solutionbox}

\questionmarks{3(b)(3)}{4}{એક ગતિમાન કણનું ગતિ સમીકરણ $s = t^3 + 3t$, $t > 0$ છે, ક્યારે વેગ અને પ્રવેગ સમાન થશે?}
\textbf{જવાબ}: At $t = 1$ second

\begin{solutionbox}
આપેલ છે: $s = t^3 + 3t$

વેગ: $v = \frac{ds}{dt} = 3t^2 + 3$
પ્રવેગ: $a = \frac{dv}{dt} = 6t$

વેગ = પ્રવેગ માટે:
$3t^2 + 3 = 6t$
$3t^2 - 6t + 3 = 0$
$t^2 - 2t + 1 = 0$
$(t - 1)^2 = 0$
$t = 1$

તેથી, $t = 1$ સેકન્ડે વેગ અને પ્રવેગ સમાન છે.
\end{solutionbox}

\questionmarks{4(a)}{6}{કોઈપણ બે ગણો}

\questionmarks{4(a)(1)}{3}{કિંમત શોધો: $\int \frac{\sin\sqrt{x}}{\sqrt{x}} dx$}
\textbf{જવાબ}: $-2\cos\sqrt{x} + C$

\begin{solutionbox}
ધારો કે $u = \sqrt{x}$, તો $du = \frac{1}{2\sqrt{x}} dx$, તેથી $dx = 2\sqrt{x} du = 2u du$

$\int \frac{\sin\sqrt{x}}{\sqrt{x}} dx = \int \frac{\sin u}{u} \cdot 2u du = 2\int \sin u du = -2\cos u + C = -2\cos\sqrt{x} + C$
\end{solutionbox}

\questionmarks{4(a)(2)}{3}{કિંમત શોધો: $\int_0^{\pi/2} \frac{\sqrt{\sin x}}{\sqrt{\cos x} + \sqrt{\sin x}} dx$}
\textbf{જવાબ}: $\frac{\pi}{4}$

\begin{solutionbox}
Let $I = \int_0^{\pi/2} \frac{\sqrt{\sin x}}{\sqrt{\cos x} + \sqrt{\sin x}} dx$

ગુણધર્મ $\int_0^a f(x) dx = \int_0^a f(a-x) dx$ નો ઉપયોગ કરતા:
$I = \int_0^{\pi/2} \frac{\sqrt{\sin(\pi/2 - x)}}{\sqrt{\cos(\pi/2 - x)} + \sqrt{\sin(\pi/2 - x)}} dx$
$= \int_0^{\pi/2} \frac{\sqrt{\cos x}}{\sqrt{\sin x} + \sqrt{\cos x}} dx$

બંને પદોનો સરવાળો કરતા:
$2I = \int_0^{\pi/2} \frac{\sqrt{\sin x} + \sqrt{\cos x}}{\sqrt{\cos x} + \sqrt{\sin x}} dx = \int_0^{\pi/2} 1 dx = \frac{\pi}{2}$

તેથી: $I = \frac{\pi}{4}$
\end{solutionbox}

\questionmarks{4(a)(3)}{3}{આવૃત્તિ વિતરણનો મધ્યક શોધો:}

\begin{solutionbox}
\begin{answertable}{આવૃત્તિ વિતરણ}
\begin{tabulary}{\linewidth}{|C|C|C|C|C|C|C|C|C|}
\hline
Age & 20-24 & 25-29 & 30-34 & 35-39 & 40-44 & 45-49 & 50-54 & 55-59 \\ \hline
Staff & 5 & 7 & 9 & 11 & 10 & 8 & 6 & 4 \\ \hline
\end{tabulary}
\end{answertable}

\textbf{જવાબ}: Mean = 37.5 years

\begin{answertable}{મધ્યક ગણતરી}
\begin{tabulary}{\linewidth}{|C|C|C|C|}
\hline
Class & Midpoint (x) & Frequency (f) & fx \\ \hline
20-24 & 22 & 5 & 110 \\ \hline
25-29 & 27 & 7 & 189 \\ \hline
30-34 & 32 & 9 & 288 \\ \hline
35-39 & 37 & 11 & 407 \\ \hline
40-44 & 42 & 10 & 420 \\ \hline
45-49 & 47 & 8 & 376 \\ \hline
50-54 & 52 & 6 & 312 \\ \hline
55-59 & 57 & 4 & 228 \\ \hline
\textbf{Total} & & \textbf{60} & \textbf{2330} \\ \hline
\end{tabulary}
\end{answertable}

Mean = $\frac{\sum fx}{\sum f} = \frac{2330}{60} = 38.83$ years
\end{solutionbox}

\questionmarks{4(b)}{8}{કોઈપણ બે ગણો}

\questionmarks{4(b)(1)}{4}{કિંમત શોધો: $\int_0^1 \frac{x^2}{1 + x^6} dx$}
\textbf{જવાબ}: $\frac{\pi}{12}$

\begin{solutionbox}
ધારો કે $u = x^3$, તો $du = 3x^2 dx$, તેથી $x^2 dx = \frac{1}{3} du$
જ્યારે $x = 0$, $u = 0$; જ્યારે $x = 1$, $u = 1$

$\int_0^1 \frac{x^2}{1 + x^6} dx = \int_0^1 \frac{1}{1 + u^2} \cdot \frac{1}{3} du = \frac{1}{3} \int_0^1 \frac{1}{1 + u^2} du$
$= \frac{1}{3} [\tan^{-1} u]_0^1 = \frac{1}{3}(\tan^{-1} 1 - \tan^{-1} 0) = \frac{1}{3} \cdot \frac{\pi}{4} = \frac{\pi}{12}$
\end{solutionbox}

\questionmarks{4(b)(2)}{4}{વક્ર $y = x^2$, $X$-અક્ષ અને $x = 2$ દ્વારા આવૃત પ્રદેશનું ક્ષેત્રફળ શોધો}
\textbf{જવાબ}: ક્ષેત્રફળ = $\frac{8}{3}$ ચોરસ એકમ

\begin{solutionbox}
આ પ્રદેશ $y = x^2$, $y = 0$ ($X$-અક્ષ), $x = 0$ અને $x = 2$ થી ઘેરાયેલો છે

ક્ષેત્રફળ = $\int_0^2 x^2 dx = \left[\frac{x^3}{3}\right]_0^2 = \frac{8}{3} - 0 = \frac{8}{3}$ ચોરસ એકમ
\end{solutionbox}

\questionmarks{4(b)(3)}{4}{નીચેના સતત આવૃત્તિ વિતરણ માટે પ્રમાણિત વિચલન શોધો:}

\begin{solutionbox}
\begin{answertable}{Grouped Data}
\begin{tabulary}{\linewidth}{|C|C|C|C|C|C|C|}
\hline
Class & 0-10 & 10-20 & 20-30 & 30-40 & 40-50 \\ \hline
Frequency & 5 & 8 & 15 & 16 & 6 \\ \hline
\end{tabulary}
\end{answertable}

\textbf{જવાબ}: Standard deviation = 10.95

\begin{answertable}{પ્રમાણિત વિચલન ગણતરી}
\begin{tabulary}{\linewidth}{|C|C|C|C|C|C|}
\hline
Class & Midpoint (x) & f & fx & $x^2$ & $fx^2$ \\ \hline
0-10 & 5 & 5 & 25 & 25 & 125 \\ \hline
10-20 & 15 & 8 & 120 & 225 & 1800 \\ \hline
20-30 & 25 & 15 & 375 & 625 & 9375 \\ \hline
30-40 & 35 & 16 & 560 & 1225 & 19600 \\ \hline
40-50 & 45 & 6 & 270 & 2025 & 12150 \\ \hline
\textbf{Total} & & \textbf{50} & \textbf{1350} & & \textbf{43050} \\ \hline
\end{tabulary}
\end{answertable}

Mean $\bar{x} = \frac{1350}{50} = 27$

Variance = $\frac{\sum fx^2}{n} - (\bar{x})^2 = \frac{43050}{50} - (27)^2 = 861 - 729 = 132$

Standard deviation = $\sqrt{132} = 11.49$
\end{solutionbox}

\questionmarks{5(a)}{6}{કોઈપણ બે ગણો}

\questionmarks{5(a)(1)}{3}{જો 25 અવલોકનોનો મધ્યક 50 હોય અને બીજા 75 અવલોકનોનો મધ્યક 60 હોય. બધા અવલોકનોને ધ્યાનમાં લેતા મિશ્ર મધ્યક શોધો.}
\textbf{જવાબ}: મિશ્ર મધ્યક = 57.5

\begin{solutionbox}
મિશ્ર મધ્યક = $\frac{n_1\bar{x_1} + n_2\bar{x_2}}{n_1 + n_2}$
$= \frac{25 \times 50 + 75 \times 60}{25 + 75} = \frac{1250 + 4500}{100} = \frac{5750}{100} = 57.5$
\end{solutionbox}

\questionmarks{5(a)(2)}{3}{નીચેના આવૃત્તિ વિતરણ માટે સરેરાશ વિચલન શોધો:}

\begin{solutionbox}
\begin{answertable}{Frequency Distribution}
\begin{tabulary}{\linewidth}{|C|C|C|C|C|C|C|}
\hline
$x_i$ & 3 & 4 & 5 & 6 & 7 & 8 \\ \hline
$f_i$ & 1 & 3 & 7 & 5 & 2 & 2 \\ \hline
\end{tabulary}
\end{answertable}

\textbf{જવાબ}: સરેરાશ વિચલન = 1.1

\begin{answertable}{સરેરાશ વિચલન ગણતરી}
\begin{tabulary}{\linewidth}{|C|C|C|C|C|}
\hline
$x_i$ & $f_i$ & $f_i x_i$ & $|x_i - \bar{x}|$ & $f_i|x_i - \bar{x}|$ \\ \hline
3 & 1 & 3 & 2 & 2 \\ \hline
4 & 3 & 12 & 1 & 3 \\ \hline
5 & 7 & 35 & 0 & 0 \\ \hline
6 & 5 & 30 & 1 & 5 \\ \hline
7 & 2 & 14 & 2 & 4 \\ \hline
8 & 2 & 16 & 3 & 6 \\ \hline
\textbf{Total} & \textbf{20} & \textbf{110} & & \textbf{20} \\ \hline
\end{tabulary}
\end{answertable}

Mean $\bar{x} = \frac{110}{20} = 5.5$

મધ્યક = 5.5 થી વિચલન લેતા:
સરેરાશ વિચલન = $\frac{\sum f_i|x_i - \bar{x}|}{\sum f_i} = \frac{22}{20} = 1.1$
\end{solutionbox}

\questionmarks{5(a)(3)}{3}{નીચેના માહિતી માટે પ્રમાણિત વિચલન શોધો:\newline 120, 132, 148, 136, 142, 140, 165, 153}
\textbf{જવાબ}: પ્રમાણિત વિચલન = 13.36

\begin{solutionbox}
\begin{answertable}{માહિતી કોષ્ટક}
\begin{tabulary}{\linewidth}{|C|C|C|}
\hline
$x$ & $x - \bar{x}$ & $(x - \bar{x})^2$ \\ \hline
120 & -19.5 & 380.25 \\ \hline
132 & -7.5 & 56.25 \\ \hline
148 & 8.5 & 72.25 \\ \hline
136 & -3.5 & 12.25 \\ \hline
142 & 2.5 & 6.25 \\ \hline
140 & 0.5 & 0.25 \\ \hline
165 & 25.5 & 650.25 \\ \hline
153 & 13.5 & 182.25 \\ \hline
\textbf{Total} & \textbf{0} & \textbf{1360} \\ \hline
\end{tabulary}
\end{answertable}

$n = 8$, $\sum x = 1116$
Mean $\bar{x} = \frac{1116}{8} = 139.5$

Variance = $\frac{\sum(x - \bar{x})^2}{n} = \frac{1360}{8} = 170$

પ્રમાણિત વિચલન = $\sqrt{170} = 13.04$
\end{solutionbox}

\questionmarks{5(b)}{8}{કોઈપણ બે ગણો}

\questionmarks{5(b)(1)}{4}{ઉકેલો: $\frac{dy}{dx} + \tan x \cdot \tan y = 0$}
\textbf{જવાબ}: $\ln|\cos y| = \ln|\cos x| + C$ or $\cos y = A\cos x$

\begin{solutionbox}
$\frac{dy}{dx} + \tan x \cdot \tan y = 0$
$\frac{dy}{dx} = -\tan x \cdot \tan y$
$\frac{dy}{\tan y} = -\tan x \, dx$
$\cot y \, dy = -\tan x \, dx$

બંને બાજુ સંકલન કરતા:
$\int \cot y \, dy = -\int \tan x \, dx$
$\ln|\sin y| = \ln|\cos x| + C_1$
$\ln|\sin y| - \ln|\cos x| = C_1$
$\ln\left|\frac{\sin y}{\cos x}\right| = C_1$

 ઘાતાંકીય લેતા:
$\frac{\sin y}{\cos x} = C$ (where $C = e^{C_1}$)
$\sin y = C \cos x$

વૈકલ્પિક સ્વરૂપ: $\cos y = A \cos x$ જ્યાં $A$ અચળ છે.
\end{solutionbox}

\questionmarks{5(b)(2)}{4}{ઉકેલો: $\frac{dy}{dx} + 2y = 3e^x$}
\textbf{જવાબ}: $y = e^x + Ce^{-2x}$

\begin{solutionbox}
આ $\frac{dy}{dx} + Py = Q$ સ્વરૂપનું પ્રથમ કક્ષાનું સુરેખ વિકલ સમીકરણ છે.
જ્યાં $P = 2$ અને $Q = 3e^x$

સંકલ્યકારક અવયવ: $I.F. = e^{\int P \, dx} = e^{\int 2 \, dx} = e^{2x}$

સમીકરણને $e^{2x}$ વડે ગુણતા:
$e^{2x}\frac{dy}{dx} + 2e^{2x}y = 3e^{3x}$

ડાબી બાજુ $ye^{2x}$ નું વિકલન છે:
$\frac{d}{dx}(ye^{2x}) = 3e^{3x}$

બંને બાજુ સંકલન કરતા:
$ye^{2x} = \int 3e^{3x} \, dx = e^{3x} + C$

તેથી: $y = e^x + Ce^{-2x}$
\end{solutionbox}

\questionmarks{5(b)(3)}{4}{ઉકેલો: $dy + 4xy^2dx = 0$; $y(0) = 1$}
\textbf{જવાબ}: $y = \frac{1}{1 + 2x^2}$

\begin{solutionbox}
$dy + 4xy^2dx = 0$
$dy = -4xy^2dx$
$\frac{dy}{y^2} = -4x \, dx$

બંને બાજુ સંકલન કરતા:
$\int y^{-2} \, dy = \int -4x \, dx$
$-\frac{1}{y} = -2x^2 + C$
$\frac{1}{y} = 2x^2 - C$

પ્રારંભિક શરત $y(0) = 1$ નો ઉપયોગ કરતા:
$\frac{1}{1} = 2(0)^2 - C$
$1 = -C$
$C = -1$

તેથી: $\frac{1}{y} = 2x^2 + 1$
$y = \frac{1}{2x^2 + 1}$
\end{solutionbox}

\section*{સૂત્રો}

\subsection*{શ્રેણિક પ્રક્રિયાઓ}
\begin{itemize}
\item \textbf{શ્રેણિક સરવાળા/બાદબાકી}: અનુરૂપ ઘટકોનો સરવાળો/બાદબાકી
\item \textbf{શ્રેણિક ગુણાકાર}: $(AB)_{ij} = \sum_{k} a_{ik}b_{kj}$
\item \textbf{પરિવર્ત શ્રેણિક}: $(A^T)_{ij} = A_{ji}$
\item \textbf{નિશ્ચાયક (2×2)}: $\det\begin{bmatrix} a & b \\ c & d \end{bmatrix} = ad - bc$
\item \textbf{વ્યસ્ત શ્રેણિક (2×2)}: $A^{-1} = \frac{1}{\det(A)}\begin{bmatrix} d & -b \\ -c & a \end{bmatrix}$
\item \textbf{સહઅવયવજ શ્રેણિક (2×2)}: $\text{adj}\begin{bmatrix} a & b \\ c & d \end{bmatrix} = \begin{bmatrix} d & -b \\ -c & a \end{bmatrix}$
\end{itemize}

\subsection*{વિકલન સૂત્રો}
\begin{itemize}
\item $\frac{d}{dx}(x^n) = nx^{n-1}$
\item $\frac{d}{dx}(e^x) = e^x$
\item $\frac{d}{dx}(\ln x) = \frac{1}{x}$
\item $\frac{d}{dx}(\sin x) = \cos x$
\item $\frac{d}{dx}(\cos x) = -\sin x$
\item $\frac{d}{dx}(\tan x) = \sec^2 x$
\item $\frac{d}{dx}(\tan^{-1} x) = \frac{1}{1+x^2}$
\item \textbf{સાંકળ નિયમ}: $\frac{d}{dx}f(g(x)) = f'(g(x)) \cdot g'(x)$
\item \textbf{ગુણાકારનો નિયમ}: $(uv)' = u'v + uv'$
\item \textbf{ભાગાકારનો નિયમ}: $(\frac{u}{v})' = \frac{u'v - uv'}{v^2}$
\end{itemize}

\subsection*{સંકલન સૂત્રો}
\begin{itemize}
\item $\int x^n \, dx = \frac{x^{n+1}}{n+1} + C$ (for $n \neq -1$)
\item $\int \frac{1}{x} \, dx = \ln|x| + C$
\item $\int e^x \, dx = e^x + C$
\item $\int \sin x \, dx = -\cos x + C$
\item $\int \cos x \, dx = \sin x + C$
\item $\int \sec^2 x \, dx = \tan x + C$
\item $\int \frac{1}{1+x^2} \, dx = \tan^{-1} x + C$
\item \textbf{ખંડશઃ સંકલન}: $\int u \, dv = uv - \int v \, du$
\end{itemize}

\subsection*{વિકલ સમીકરણો}
\begin{itemize}
\item \textbf{વિયોજનીય ચલ}: $\frac{dy}{dx} = f(x)g(y) \Rightarrow \frac{dy}{g(y)} = f(x)dx$
\item \textbf{સુરેખ વિકલ સમીકરણ}: $\frac{dy}{dx} + Py = Q$, ઉકેલ: $y \cdot I.F. = \int Q \cdot I.F. \, dx$
\item \textbf{સંકલ્યકારક અવયવ}: $I.F. = e^{\int P \, dx}$
\end{itemize}

\subsection*{આંકડાશાસ્ત્ર સૂત્રો}
\begin{itemize}
\item \textbf{મધ્યક}: $\bar{x} = \frac{\sum x_i}{n}$ (અવર્ગીકૃત), $\bar{x} = \frac{\sum f_i x_i}{\sum f_i}$ (વર્ગીકૃત)
\item \textbf{સરેરાશ વિચલન}: $M.D. = \frac{\sum |x_i - \bar{x}|}{n}$ (અવર્ગીકૃત), $M.D. = \frac{\sum f_i |x_i - \bar{x}|}{\sum f_i}$ (વર્ગીકૃત)
\item \textbf{પ્રમાણિત વિચલન}: $\sigma = \sqrt{\frac{\sum (x_i - \bar{x})^2}{n}}$ (અવર્ગીકૃત)
\item \textbf{વિચરણ}: $\sigma^2 = \frac{\sum (x_i - \bar{x})^2}{n}$
\item \textbf{વિસ્તાર}: મહત્તમ કિંમત - ન્યૂનતમ કિંમત
\item \textbf{મિશ્ર મધ્યક}: $\bar{x} = \frac{n_1\bar{x_1} + n_2\bar{x_2}}{n_1 + n_2}$
\end{itemize}

\end{document}
