\documentclass{article}

% content/resources/templates/preamble.tex
\usepackage[margin=0.6in]{geometry}
\author{Milav Dabgar}
\usepackage{amsmath,amssymb,amsthm}
\usepackage{booktabs}
\usepackage{multirow}
\usepackage{xcolor}
\usepackage{tcolorbox}
\tcbuselibrary{breakable,skins}
\usepackage[colorlinks=true,linkcolor=blue]{hyperref}
\usepackage{titlesec}
\usepackage{enumitem}
\usepackage{tikz}
\usepackage{pgfplots}
\usepackage{circuitikz}
\usepackage[version=4]{mhchem}
\usepackage{longtable}
\usepackage{array}
\usepackage{float}
\usepackage{caption}
\usepackage{listings}

\lstset{
  basicstyle=\small\ttfamily,
  breaklines=true,
  breakatwhitespace=false,
  postbreak=\mbox{\textcolor{red}{$\hookrightarrow$}\space},
  float=false,
  numbers=left,
  numberstyle=\tiny\color{gray},
  numbersep=10pt,
  xleftmargin=2em,
  keywordstyle=\color{blue},
  commentstyle=\color{green!60!black},
  stringstyle=\color{purple},
  backgroundcolor=\color{gray!5},
  showstringspaces=false,
  tabsize=2,
  captionpos=b,
  keepspaces=true,
  columns=flexible
}

\pgfplotsset{compat=1.18}
\usetikzlibrary{shapes,arrows,positioning,calc,patterns,decorations.pathmorphing,decorations.markings,arrows.meta}

% Color scheme
\definecolor{headcolor}{RGB}{0,102,204}
\definecolor{keycolor}{RGB}{220,20,60}
\definecolor{solutioncolor}{RGB}{34,139,34}
\definecolor{mnemoniccolor}{RGB}{148,0,211}
\definecolor{codecolor}{RGB}{0,0,100}

% Spacing
\setlength{\parskip}{3pt}
\setlist[itemize]{nosep}
\setlist[enumerate]{nosep}

% Title formatting
\titleformat{\section}{\Large\bfseries\color{headcolor}}{\thesection}{1em}{}
\titleformat{\subsection}{\large\bfseries\color{headcolor}}{\thesubsection}{1em}{}

% Pandoc tightlist compatibility
\providecommand{\tightlist}{%
  \setlength{\itemsep}{0pt}\setlength{\parskip}{0pt}}

% Pandoc longtable compatibility
\newcounter{none}
\def\thenone{}


% content/resources/templates/english-boxes.tex

% Custom environments
\newtcolorbox{solutionbox}{
 breakable,
 enhanced,
 colback=solutioncolor!5!white,
 colframe=solutioncolor!75!black,
 fonttitle=\bfseries,
 title=Solution
}

\newtcolorbox{solutionboxnobreak}{
 colback=solutioncolor!5!white,
 colframe=solutioncolor!75!black,
 fonttitle=\bfseries,
 title=Solution
}

\newtcolorbox{keyformula}{
 breakable,
 enhanced,
 colback=keycolor!5!white,
 colframe=keycolor!75!black,
 fonttitle=\bfseries,
 title=Key Formula
}

\newtcolorbox{mnemonicboxenv}{
 breakable,
 enhanced,
 colback=mnemoniccolor!5!white,
 colframe=mnemoniccolor!75!black,
 fonttitle=\bfseries,
 title=Mnemonic
}

\newcommand{\mnemonicbox}[1]{%
  \begin{mnemonicboxenv}
    #1
  \end{mnemonicboxenv}
}


% Custom commands for GTU solutions
% This file defines semantic commands for consistent formatting

% Question command with automatic formatting
\newcommand{\question}[2]{%
  \section*{Question #1}%
  \textbf{#2}%
}

% OR question variant
\newcommand{\questionor}[2]{%
  \section*{Question #1 OR}%
  \textbf{#2}%
}

% Proper table environment with caption
\newenvironment{answertable}[1]{%
  \begin{table}[htbp]
  \centering
  \caption{#1}
}{%
  \end{table}
}

% Proper figure environment for diagrams
\newenvironment{answerdiagram}[1]{%
  \begin{figure}[htbp]
  \centering
  \caption{#1}
}{%
  \end{figure}
}

% Semantic markup for key terms
\newcommand{\keyword}[1]{\textbf{#1}}
\newcommand{\code}[1]{\texttt{#1}}
\newcommand{\classname}[1]{\texttt{#1}}
\newcommand{\methodname}[1]{\texttt{#1}}

% Proper quotation marks
\newcommand{\mnemonic}[1]{``#1''}


\title{Applied Mathematics (4320001) - Winter 2023 Solution}
\date{January 30, 2024}

\begin{document}
\maketitle

\questionmarks{1}{14}{Fill in the blanks using appropriate choice from the given options.}

\questionmarks{1(1)}{1}{If $A = \begin{bmatrix} 1 & 2 \\ 3 & -1 \end{bmatrix}$ then $4A$ = ...}
\textbf{Answer}: (b) $\begin{bmatrix} 4 & 8 \\ 12 & -4 \end{bmatrix}$

\begin{solutionbox}
$4A = 4 \begin{bmatrix} 1 & 2 \\ 3 & -1 \end{bmatrix} = \begin{bmatrix} 4 & 8 \\ 12 & -4 \end{bmatrix}$
\end{solutionbox}

\questionmarks{1(2)}{1}{Order of the matrix $\begin{bmatrix} 1 & 1 & 2 \\ -3 & 2 & 3 \end{bmatrix}$ is ...}
\textbf{Answer}: (a) 2 × 3

\begin{solutionbox}
Matrix has 2 rows and 3 columns, so order is 2 × 3.
\end{solutionbox}

\questionmarks{1(3)}{1}{If $A = \begin{bmatrix} 1 & 1 \\ 1 & 1 \end{bmatrix}$ then $A^2$ = ...}
\textbf{Answer}: (d) $\begin{bmatrix} 2 & 2 \\ 2 & 2 \end{bmatrix}$

\begin{solutionbox}
$A^2 = \begin{bmatrix} 1 & 1 \\ 1 & 1 \end{bmatrix} \begin{bmatrix} 1 & 1 \\ 1 & 1 \end{bmatrix} = \begin{bmatrix} 2 & 2 \\ 2 & 2 \end{bmatrix}$
\end{solutionbox}

\questionmarks{1(4)}{1}{If $A = \begin{bmatrix} 2 & -1 \\ 3 & 4 \end{bmatrix}$ then adjoint of A = ...}
\textbf{Answer}: (c) $\begin{bmatrix} 4 & 1 \\ -3 & 2 \end{bmatrix}$

\begin{solutionbox}
For matrix $A = \begin{bmatrix} a & b \\ c & d \end{bmatrix}$, $adj(A) = \begin{bmatrix} d & -b \\ -c & a \end{bmatrix}$
$adj(A) = \begin{bmatrix} 4 & 1 \\ -3 & 2 \end{bmatrix}$
\end{solutionbox}

\questionmarks{1(5)}{1}{$\frac{d}{dx}(\tan x)$ = ...}
\textbf{Answer}: (d) $\sec^2x$

\begin{solutionbox}
$\frac{d}{dx}(\tan x) = \sec^2 x$
\end{solutionbox}

\questionmarks{1(6)}{1}{$\frac{d}{dx}(\sin 5x)$ = ...}
\textbf{Answer}: (b) $5\cos 5x$

\begin{solutionbox}
$\frac{d}{dx}(\sin 5x) = 5\cos 5x$ (using chain rule)
\end{solutionbox}

\questionmarks{1(7)}{1}{If function $y = f(x)$ is maximum at $x = a$ then $f'(a)$ = ...}
\textbf{Answer}: (c) 0

\begin{solutionbox}
At maximum point, first derivative equals zero: $f'(a) = 0$
\end{solutionbox}

\questionmarks{1(8)}{1}{$\int \sin x dx$ = ... + C}
\textbf{Answer}: (a) $-\cos x$

\begin{solutionbox}
$\int \sin x \, dx = -\cos x + C$
\end{solutionbox}

\questionmarks{1(9)}{1}{$\int \frac{1}{x^2+4} dx$ = ... + C}
\textbf{Answer}: (d) $\frac{1}{2}\tan^{-1}(\frac{x}{2})$

\begin{solutionbox}
$\int \frac{1}{x^2+4} dx = \frac{1}{2}\tan^{-1}\left(\frac{x}{2}\right) + C$
\end{solutionbox}

\questionmarks{1(10)}{1}{$\int_1^2 x^2 dx$ = ...}
\textbf{Answer}: (a) 7/3

\begin{solutionbox}
$\int_1^2 x^2 dx = \left[\frac{x^3}{3}\right]_1^2 = \frac{8}{3} - \frac{1}{3} = \frac{7}{3}$
\end{solutionbox}

\questionmarks{1(11)}{1}{Order of differential equation $\left(\frac{d^3y}{dx^3}\right)^4 + \frac{dy}{dx} + 5y = 0$ is ...}
\textbf{Answer}: (c) 3

\begin{solutionbox}
Order is the highest derivative present = 3
\end{solutionbox}

\questionmarks{1(12)}{1}{Integrating factor of $\frac{dy}{dx} + \frac{y}{x} = 1$ is ...}
\textbf{Answer}: (b) x

\begin{solutionbox}
I.F. = $e^{\int \frac{1}{x} dx} = e^{\ln x} = x$
\end{solutionbox}

\questionmarks{1(13)}{1}{Mean of 39,23,58,47,50,16,61 is ...}
\textbf{Answer}: (b) 42

\begin{solutionbox}
Mean = $\frac{39+23+58+47+50+16+61}{7} = \frac{294}{7} = 42$
\end{solutionbox}

\questionmarks{1(14)}{1}{Mean of first five natural numbers is ...}
\textbf{Answer}: (a) 3

\begin{solutionbox}
Mean = $\frac{1+2+3+4+5}{5} = \frac{15}{5} = 3$
\end{solutionbox}

\questionmarks{2}{14}{Attempt any two}

\questionmarks{2(a)}{6}{}

\questionmarks{2(a)(1)}{3}{If $A = \begin{bmatrix} 1 & 3 & 5 \\ -1 & 0 & 2 \\ 4 & 3 & 6 \end{bmatrix}$, $B = \begin{bmatrix} 3 & 4 & 5 \\ 5 & 4 & 3 \\ 3 & 5 & 4 \end{bmatrix}$, $C = \begin{bmatrix} 1 & 2 & 1 \\ 3 & 3 & 3 \\ 4 & 5 & 6 \end{bmatrix}$, find $3A+2B-4C$}

\begin{solutionbox}
$3A = \begin{bmatrix} 3 & 9 & 15 \\ -3 & 0 & 6 \\ 12 & 9 & 18 \end{bmatrix}$

$2B = \begin{bmatrix} 6 & 8 & 10 \\ 10 & 8 & 6 \\ 6 & 10 & 8 \end{bmatrix}$

$4C = \begin{bmatrix} 4 & 8 & 4 \\ 12 & 12 & 12 \\ 16 & 20 & 24 \end{bmatrix}$

$3A + 2B - 4C = \begin{bmatrix} 5 & 9 & 21 \\ -5 & -4 & 0 \\ 2 & -1 & 2 \end{bmatrix}$
\end{solutionbox}

\questionmarks{2(a)(2)}{3}{If $A = \begin{bmatrix} 7 & 5 \\ -1 & 2 \end{bmatrix}$, $B = \begin{bmatrix} 1 & -1 \\ 3 & 2 \end{bmatrix}$, show that $(A+B)^T = A^T + B^T$}

\begin{solutionbox}
$A + B = \begin{bmatrix} 8 & 4 \\ 2 & 4 \end{bmatrix}$

$(A + B)^T = \begin{bmatrix} 8 & 2 \\ 4 & 4 \end{bmatrix}$

$A^T = \begin{bmatrix} 7 & -1 \\ 5 & 2 \end{bmatrix}$, $B^T = \begin{bmatrix} 1 & 3 \\ -1 & 2 \end{bmatrix}$

$A^T + B^T = \begin{bmatrix} 8 & 2 \\ 4 & 4 \end{bmatrix}$

Hence proved: $(A + B)^T = A^T + B^T$
\end{solutionbox}

\questionmarks{2(a)(3)}{3}{Solve the differential equation $xy dy = (x+1)(y+1)dx$}

\begin{solutionbox}
Separating variables:
$\frac{y}{y+1} dy = \frac{x+1}{x} dx$

$\left(1 - \frac{1}{y+1}\right) dy = \left(1 + \frac{1}{x}\right) dx$

Integrating:
$y - \ln|y+1| = x + \ln|x| + C$

\textbf{Final answer}: $y - x = \ln|y+1| + \ln|x| + C$
\end{solutionbox}

\questionmarks{2(b)}{8}{}

\questionmarks{2(b)(1)}{4}{Find the inverse of matrix $\begin{bmatrix} 3 & 1 & 2 \\ 2 & -3 & -1 \\ 1 & 2 & 1 \end{bmatrix}$}

\begin{solutionbox}
Let $A = \begin{bmatrix} 3 & 1 & 2 \\ 2 & -3 & -1 \\ 1 & 2 & 1 \end{bmatrix}$

$|A| = 3(-3-(-2)) - 1(2-(-1)) + 2(4-(-3)) = 3(-1) - 1(3) + 2(7) = -3 - 3 + 14 = 8$

\textbf{Cofactors}:
\begin{itemize}
\item $C_{11} = -1, C_{12} = -3, C_{13} = 7$
\item $C_{21} = 3, C_{22} = 1, C_{23} = -5$
\item $C_{31} = 5, C_{32} = 7, C_{33} = -11$
\end{itemize}

$adj(A) = \begin{bmatrix} -1 & 3 & 5 \\ -3 & 1 & 7 \\ 7 & -5 & -11 \end{bmatrix}$

$A^{-1} = \frac{1}{8} \begin{bmatrix} -1 & 3 & 5 \\ -3 & 1 & 7 \\ 7 & -5 & -11 \end{bmatrix}$
\end{solutionbox}

\questionmarks{2(b)(2)}{4}{Solve $3x - 2y = 8, 5x + 4y = 6$ using matrix method}

\begin{solutionbox}
$\begin{bmatrix} 3 & -2 \\ 5 & 4 \end{bmatrix} \begin{bmatrix} x \\ y \end{bmatrix} = \begin{bmatrix} 8 \\ 6 \end{bmatrix}$

$|A| = 3(4) - (-2)(5) = 12 + 10 = 22$

$A^{-1} = \frac{1}{22} \begin{bmatrix} 4 & 2 \\ -5 & 3 \end{bmatrix}$

$\begin{bmatrix} x \\ y \end{bmatrix} = \frac{1}{22} \begin{bmatrix} 4 & 2 \\ -5 & 3 \end{bmatrix} \begin{bmatrix} 8 \\ 6 \end{bmatrix} = \frac{1}{22} \begin{bmatrix} 44 \\ -22 \end{bmatrix}$

\textbf{Answer}: $x = 2, y = -1$
\end{solutionbox}

\questionmarks{2(b)(3)}{4}{If $A = \begin{bmatrix} 1 & 2 & 1 \\ 2 & 3 & 1 \\ 1 & 2 & 2 \end{bmatrix}$, find $A \cdot adj(A)$}

\begin{solutionbox}
$|A| = 1(6-2) - 2(4-1) + 1(4-3) = 4 - 6 + 1 = -1$

For any matrix A: $A \cdot adj(A) = |A| \cdot I$

$A \cdot adj(A) = (-1) \begin{bmatrix} 1 & 0 & 0 \\ 0 & 1 & 0 \\ 0 & 0 & 1 \end{bmatrix} = \begin{bmatrix} -1 & 0 & 0 \\ 0 & -1 & 0 \\ 0 & 0 & -1 \end{bmatrix}$
\end{solutionbox}

\questionmarks{3}{14}{Attempt any two}

\questionmarks{3(a)}{6}{}

\questionmarks{3(a)(1)}{3}{If $y = \log(\frac{\sin x}{1+\cos x})$, find $\frac{dy}{dx}$}

\begin{solutionbox}
$y = \log(\sin x) - \log(1+\cos x)$

$\frac{dy}{dx} = \frac{1}{\sin x} \cdot \cos x - \frac{1}{1+\cos x} \cdot (-\sin x)$

$= \frac{\cos x}{\sin x} + \frac{\sin x}{1+\cos x}$

$= \cot x + \frac{\sin x}{1+\cos x}$

Using identity: $\frac{\sin x}{1+\cos x} = \tan(\frac{x}{2})$

\textbf{Answer}: $\frac{dy}{dx} = \cot x + \tan(\frac{x}{2})$
\end{solutionbox}

\questionmarks{3(a)(2)}{3}{If $y = \sin(x+y)$, find $\frac{dy}{dx}$}

\begin{solutionbox}
Differentiating both sides:
$\frac{dy}{dx} = \cos(x+y) \cdot \left(1 + \frac{dy}{dx}\right)$

$\frac{dy}{dx} = \cos(x+y) + \cos(x+y) \cdot \frac{dy}{dx}$

$\frac{dy}{dx} - \cos(x+y) \cdot \frac{dy}{dx} = \cos(x+y)$

$\frac{dy}{dx}[1 - \cos(x+y)] = \cos(x+y)$

\textbf{Answer}: $\frac{dy}{dx} = \frac{\cos(x+y)}{1-\cos(x+y)}$
\end{solutionbox}

\questionmarks{3(a)(3)}{3}{Obtain $\int x^2\log x dx$}

\begin{solutionbox}
Using integration by parts: $\int u dv = uv - \int v du$

Let $u = \log x$, $dv = x^2 dx$
Then $du = \frac{1}{x} dx$, $v = \frac{x^3}{3}$

$\int x^2 \log x \, dx = \log x \cdot \frac{x^3}{3} - \int \frac{x^3}{3} \cdot \frac{1}{x} dx$

$= \frac{x^3 \log x}{3} - \int \frac{x^2}{3} dx$

$= \frac{x^3 \log x}{3} - \frac{x^3}{9} + C$

\textbf{Answer}: $\frac{x^3}{3}(\log x - \frac{1}{3}) + C$
\end{solutionbox}

\questionmarks{3(b)}{8}{}

\questionmarks{3(b)(1)}{4}{Motion equation $s = 2t^3 - 3t^2 - 12t + 7$. Find s and t when acceleration is zero}

\begin{solutionbox}
$s = 2t^3 - 3t^2 - 12t + 7$

Velocity: $v = \frac{ds}{dt} = 6t^2 - 6t - 12$

Acceleration: $a = \frac{dv}{dt} = 12t - 6$

When acceleration = 0:
$12t - 6 = 0$
$t = \frac{1}{2}$

At $t = 1/2$:
$s = 2(\frac{1}{2})^3 - 3(\frac{1}{2})^2 - 12(\frac{1}{2}) + 7 = \frac{1}{4} - \frac{3}{4} - 6 + 7 = \frac{1}{2}$

\textbf{Answer}: $t = 1/2, s = 1/2$
\end{solutionbox}

\questionmarks{3(b)(2)}{4}{If $y = 2e^{3x} + 3e^{-2x}$, prove $\frac{d^2y}{dx^2} - \frac{dy}{dx} - 6y = 0$}

\begin{solutionbox}
$y = 2e^{3x} + 3e^{-2x}$

$\frac{dy}{dx} = 6e^{3x} - 6e^{-2x}$

$\frac{d^2y}{dx^2} = 18e^{3x} + 12e^{-2x}$

Now: $\frac{d^2y}{dx^2} - \frac{dy}{dx} - 6y$

$= (18e^{3x} + 12e^{-2x}) - (6e^{3x} - 6e^{-2x}) - 6(2e^{3x} + 3e^{-2x})$

$= 18e^{3x} + 12e^{-2x} - 6e^{3x} + 6e^{-2x} - 12e^{3x} - 18e^{-2x}$

$= (18-6-12)e^{3x} + (12+6-18)e^{-2x} = 0$

\textbf{Hence proved}
\end{solutionbox}

\questionmarks{3(b)(3)}{4}{Find maximum and minimum values of $f(x) = x^3 - 3x + 11$}

\begin{solutionbox}
$f(x) = x^3 - 3x + 11$

$f'(x) = 3x^2 - 3 = 3(x^2 - 1) = 3(x-1)(x+1)$

Critical points: $x = 1, x = -1$

$f''(x) = 6x$

At $x = 1$: $f''(1) = 6 > 0 \rightarrow$ Local minimum
At $x = -1$: $f''(-1) = -6 < 0 \rightarrow$ Local maximum

$f(1) = 1 - 3 + 11 = 9$ (minimum)
$f(-1) = -1 + 3 + 11 = 13$ (maximum)

\textbf{Answer}: Maximum = 13 at $x = -1$, Minimum = 9 at $x = 1$
\end{solutionbox}

\questionmarks{4}{14}{Attempt any two}

\questionmarks{4(a)}{6}{}

\questionmarks{4(a)(1)}{3}{Obtain $\int \sin 5x \sin 6x dx$}

\begin{solutionbox}
Using identity: $\sin A \sin B = \frac{1}{2}[\cos(A-B) - \cos(A+B)]$

$\sin 5x \sin 6x = \frac{1}{2}[\cos(5x-6x) - \cos(5x+6x)]$

$= \frac{1}{2}[\cos(-x) - \cos(11x)] = \frac{1}{2}[\cos x - \cos(11x)]$

$\int \sin 5x \sin 6x \, dx = \frac{1}{2} \int [\cos x - \cos(11x)] dx$

$= \frac{1}{2}[\sin x - \frac{\sin(11x)}{11}] + C$

\textbf{Answer}: $\frac{1}{2}\sin x - \frac{\sin(11x)}{22} + C$
\end{solutionbox}

\questionmarks{4(a)(2)}{3}{Obtain $\int \frac{(1+x)e^x}{\cos^2(xe^x)} dx$}

\begin{solutionbox}
Let $u = xe^x$, then $du = (1+x)e^x dx$

The integral becomes:
$\int \frac{du}{\cos^2 u} = \int \sec^2 u \, du = \tan u + C$

Substituting back:
$= \tan(xe^x) + C$

\textbf{Answer}: $\tan(xe^x) + C$
\end{solutionbox}

\questionmarks{4(a)(3)}{3}{Find standard deviation for data: 6,7,10,12,13,4,8,12}

\begin{solutionbox}
Data: 6, 7, 10, 12, 13, 4, 8, 12
$n = 8$

Mean = $\frac{6+7+10+12+13+4+8+12}{8} = \frac{72}{8} = 9$

\begin{answertable}{Standard Deviation Calculation}
\begin{tabulary}{\linewidth}{|C|C|C|}
\hline
x & x-9 & (x-9)$^2$ \\ \hline
6 & -3 & 9 \\ \hline
7 & -2 & 4 \\ \hline
10 & 1 & 1 \\ \hline
12 & 3 & 9 \\ \hline
13 & 4 & 16 \\ \hline
4 & -5 & 25 \\ \hline
8 & -1 & 1 \\ \hline
12 & 3 & 9 \\ \hline
\end{tabulary}
\end{answertable}

$\sum(x-9)^2 = 74$

Standard deviation = $\sqrt{\frac{\sum(x-\bar{x})^2}{n}} = \sqrt{\frac{74}{8}} = \sqrt{9.25} = 3.04$

\textbf{Answer}: $\sigma = 3.04$
\end{solutionbox}

\questionmarks{4(b)}{8}{}

\questionmarks{4(b)(1)}{4}{Obtain $\int \frac{2x+1}{(x+1)(x-3)} dx$}

\begin{solutionbox}
Using partial fractions:
$\frac{2x+1}{(x+1)(x-3)} = \frac{A}{x+1} + \frac{B}{x-3}$

$2x+1 = A(x-3) + B(x+1)$

When $x = -1$: $2(-1)+1 = A(-4) \Rightarrow -1 = -4A \Rightarrow A = \frac{1}{4}$

When $x = 3$: $2(3)+1 = B(4) \Rightarrow 7 = 4B \Rightarrow B = \frac{7}{4}$

$\int \frac{2x+1}{(x+1)(x-3)} dx = \frac{1}{4}\int \frac{1}{x+1} dx + \frac{7}{4}\int \frac{1}{x-3} dx$

$= \frac{1}{4}\ln|x+1| + \frac{7}{4}\ln|x-3| + C$

\textbf{Answer}: $\frac{1}{4}\ln|x+1| + \frac{7}{4}\ln|x-3| + C$
\end{solutionbox}

\questionmarks{4(b)(2)}{4}{Obtain $\int_0^{\pi/2} \frac{\sqrt{\cot x}}{\sqrt{\cot x} + \sqrt{\tan x}} dx$}

\begin{solutionbox}
Let $I = \int_0^{\pi/2} \frac{\sqrt{\cot x}}{\sqrt{\cot x} + \sqrt{\tan x}} dx$

Using property: $\int_0^a f(x) dx = \int_0^a f(a-x) dx$

$I = \int_0^{\pi/2} \frac{\sqrt{\cot(\pi/2-x)}}{\sqrt{\cot(\pi/2-x)} + \sqrt{\tan(\pi/2-x)}} dx$

Since $\cot(\pi/2-x) = \tan x$ and $\tan(\pi/2-x) = \cot x$:

$I = \int_0^{\pi/2} \frac{\sqrt{\tan x}}{\sqrt{\tan x} + \sqrt{\cot x}} dx$

Adding both expressions:
$2I = \int_0^{\pi/2} \frac{\sqrt{\cot x} + \sqrt{\tan x}}{\sqrt{\cot x} + \sqrt{\tan x}} dx = \int_0^{\pi/2} 1 \, dx = \frac{\pi}{2}$

\textbf{Answer}: $I = \frac{\pi}{4}$
\end{solutionbox}

\questionmarks{4(b)(3)}{4}{Find mean deviation for grouped data}

\begin{solutionbox}
\begin{answertable}{Grouped Data}
\begin{tabulary}{\linewidth}{|C|C|C|C|C|C|C|C|}
\hline
$x_i$ & 4 & 8 & 11 & 17 & 20 & 24 & 32 \\ \hline
$f_i$ & 3 & 5 & 9 & 5 & 4 & 3 & 1 \\ \hline
\end{tabulary}
\end{answertable}

$N = \sum f_i = 3+5+9+5+4+3+1 = 30$

Mean = $\frac{\sum f_i x_i}{N} = \frac{3(4)+5(8)+9(11)+5(17)+4(20)+3(24)+1(32)}{30}$

$= \frac{12+40+99+85+80+72+32}{30} = \frac{420}{30} = 14$

\begin{answertable}{Mean Deviation Calculation}
\begin{tabulary}{\linewidth}{|C|C|C|C|}
\hline
$x_i$ & $f_i$ & $|x_i-14|$ & $f_i|x_i-14|$ \\ \hline
4 & 3 & 10 & 30 \\ \hline
8 & 5 & 6 & 30 \\ \hline
11 & 9 & 3 & 27 \\ \hline
17 & 5 & 3 & 15 \\ \hline
20 & 4 & 6 & 24 \\ \hline
24 & 3 & 10 & 30 \\ \hline
32 & 1 & 18 & 18 \\ \hline
\end{tabulary}
\end{answertable}

$\sum f_i|x_i-14| = 174$

Mean deviation = $\frac{\sum f_i |x_i - \bar{x}|}{N} = \frac{174}{30} = 5.8$

\textbf{Answer}: Mean deviation = 5.8
\end{solutionbox}

\questionmarks{5}{14}{Attempt any two}

\questionmarks{5(a)}{6}{}

\questionmarks{5(a)(1)}{3}{Find mean deviation for grouped data}

\begin{solutionbox}
\begin{answertable}{Grouped Data}
\begin{tabulary}{\linewidth}{|C|C|C|C|C|C|C|C|}
\hline
Class & 30-40 & 40-50 & 50-60 & 60-70 & 70-80 & 80-90 & 90-100 \\ \hline
Freq & 3 & 7 & 12 & 15 & 8 & 3 & 2 \\ \hline
\end{tabulary}
\end{answertable}

$N = 50, \sum f_i x_i = 3100$

Mean = 3100/50 = 62

\begin{answertable}{Mean Deviation Calculation}
\begin{tabulary}{\linewidth}{|C|C|C|C|C|}
\hline
Class & $x_i$ & $f_i$ & $|x_i-62|$ & $f_i|x_i-62|$ \\ \hline
30-40 & 35 & 3 & 27 & 81 \\ \hline
40-50 & 45 & 7 & 17 & 119 \\ \hline
50-60 & 55 & 12 & 7 & 84 \\ \hline
60-70 & 65 & 15 & 3 & 45 \\ \hline
70-80 & 75 & 8 & 13 & 104 \\ \hline
80-90 & 85 & 3 & 23 & 69 \\ \hline
90-100 & 95 & 2 & 33 & 66 \\ \hline
\end{tabulary}
\end{answertable}

Mean deviation = 568/50 = 11.36

\textbf{Answer}: Mean deviation = 11.36
\end{solutionbox}

\questionmarks{5(a)(2)}{3}{Find standard deviation for given data}

\begin{solutionbox}
\begin{answertable}{Grouped Data}
\begin{tabulary}{\linewidth}{|C|C|C|C|C|C|C|C|C|C|}
\hline
Class & 60 & 61 & 62 & 63 & 64 & 65 & 66 & 67 & 68 \\ \hline
Freq & 2 & 1 & 12 & 29 & 25 & 12 & 10 & 4 & 5 \\ \hline
\end{tabulary}
\end{answertable}

$N = 100$, Mean = 63.8

\begin{answertable}{Standard Deviation Calculation}
\begin{tabulary}{\linewidth}{|C|C|C|C|C|}
\hline
$x_i$ & $f_i$ & $(x_i-63.8)$ & $(x_i-63.8)^2$ & $f_i(x_i-63.8)^2$ \\ \hline
60 & 2 & -3.8 & 14.44 & 28.88 \\ \hline
61 & 1 & -2.8 & 7.84 & 7.84 \\ \hline
62 & 12 & -1.8 & 3.24 & 38.88 \\ \hline
63 & 29 & -0.8 & 0.64 & 18.56 \\ \hline
64 & 25 & 0.2 & 0.04 & 1.00 \\ \hline
65 & 12 & 1.2 & 1.44 & 17.28 \\ \hline
66 & 10 & 2.2 & 4.84 & 48.40 \\ \hline
67 & 4 & 3.2 & 10.24 & 40.96 \\ \hline
68 & 5 & 4.2 & 17.64 & 88.20 \\ \hline
\end{tabulary}
\end{answertable}

$\sum f_i(x_i-\bar{x})^2 = 290$

Standard deviation = $\sqrt{290/100} = \sqrt{2.9} = 1.70$

\textbf{Answer}: $\sigma = 1.70$
\end{solutionbox}

\questionmarks{5(a)(3)}{3}{Find mean for grouped data}

\begin{solutionbox}
\begin{answertable}{Grouped Data}
\begin{tabulary}{\linewidth}{|C|C|C|C|C|C|C|}
\hline
Class & 0-20 & 20-40 & 40-60 & 60-80 & 80-100 & 100-120 \\ \hline
Freq & 26 & 31 & 35 & 42 & 82 & 71 \\ \hline
\end{tabulary}
\end{answertable}

\begin{answertable}{Mean Calculation}
\begin{tabulary}{\linewidth}{|C|C|C|C|}
\hline
Class & Mid-value & $f_i$ & $f_i x_i$ \\ \hline
0-20 & 10 & 26 & 260 \\ \hline
20-40 & 30 & 31 & 930 \\ \hline
40-60 & 50 & 35 & 1750 \\ \hline
60-80 & 70 & 42 & 2940 \\ \hline
80-100 & 90 & 82 & 7380 \\ \hline
100-120 & 110 & 71 & 7810 \\ \hline
\end{tabulary}
\end{answertable}

$N = 287, \sum f_i x_i = 21070$

Mean = $\frac{\sum f_i x_i}{N} = \frac{21070}{287} = 73.42$

\textbf{Answer}: Mean = 73.42
\end{solutionbox}

\questionmarks{5(b)}{8}{}

\questionmarks{5(b)(1)}{4}{Solve differential equation $(x + y + 1)^2 \frac{dy}{dx} = 1$}

\begin{solutionbox}
Let $z = x + y + 1$, then $\frac{dz}{dx} = 1 + \frac{dy}{dx}$
So $\frac{dy}{dx} = \frac{dz}{dx} - 1$

Substituting: $z^2(\frac{dz}{dx} - 1) = 1$
$z^2 \frac{dz}{dx} - z^2 = 1$
$z^2 \frac{dz}{dx} = 1 + z^2$
$\frac{z^2}{1 + z^2} dz = dx$

Integrating:
$\int \frac{z^2}{1 + z^2} dz = \int dx$

$\int \left(1 - \frac{1}{1 + z^2}\right) dz = x + C$

$z - \tan^{-1}z = x + C$

Substituting back $z = x + y + 1$:
$(x + y + 1) - \tan^{-1}(x + y + 1) = x + C$

\textbf{Answer}: $y + 1 = \tan^{-1}(x + y + 1) + C$
\end{solutionbox}

\questionmarks{5(b)(2)}{4}{Solve $\frac{dy}{dx} + \frac{y}{x} = e^x$, $y(0) = 2$}

\begin{solutionbox}
This is a linear differential equation of the form $\frac{dy}{dx} + P(x)y = Q(x)$

Here $P(x) = \frac{1}{x}, Q(x) = e^x$

Integrating factor: $I.F. = e^{\int \frac{1}{x} dx} = e^{\ln|x|} = |x| = x$ (for $x > 0$)

Multiplying the equation by $x$:
$x \frac{dy}{dx} + y = xe^x$

$\frac{d}{dx}(xy) = xe^x$

Integrating both sides:
$xy = \int xe^x dx$

Using integration by parts for $\int xe^x dx$:
Let $u = x, dv = e^x dx$
Then $du = dx, v = e^x$

$\int xe^x dx = xe^x - \int e^x dx = xe^x - e^x = e^x(x-1)$

So: $xy = e^x(x-1) + C$
$y = \frac{e^x(x-1) + C}{x}$

Using initial condition $y(0) = 2$:
As $x \to 0$, we need to use L'Hôpital's rule or series expansion.

From the original equation at $x = 0$: $\frac{dy}{dx} = e^x - \frac{y}{x}$
This suggests we need to be more careful with the initial condition.

\textbf{Alternative approach}: Since the equation has a singularity at $x = 0$, we solve in the neighborhood where $x \neq 0$.

\textbf{Answer}: $y = \frac{e^x(x-1) + C}{x}$ where C is determined by boundary conditions.
\end{solutionbox}

\questionmarks{5(b)(3)}{4}{Solve $y \frac{dy}{dx} = \sqrt{1 + x^2 + y^2 + x^2y^2}$}

\begin{solutionbox}
$y \frac{dy}{dx} = \sqrt{1 + x^2 + y^2 + x^2y^2}$

$y \frac{dy}{dx} = \sqrt{(1 + x^2)(1 + y^2)}$

$\frac{y dy}{\sqrt{1 + y^2}} = \sqrt{1 + x^2} dx$

Integrating both sides:
$\int \frac{y dy}{\sqrt{1 + y^2}} = \int \sqrt{1 + x^2} dx$

For the left side, let $u = 1 + y^2$, then $du = 2y dy$:
$\int \frac{y dy}{\sqrt{1 + y^2}} = \frac{1}{2} \int \frac{du}{\sqrt{u}} = \sqrt{u} = \sqrt{1 + y^2}$

For the right side:
$\int \sqrt{1 + x^2} dx = \frac{x\sqrt{1 + x^2}}{2} + \frac{1}{2}\ln|x + \sqrt{1 + x^2}| + C$

Therefore:
\textbf{Answer}: $\sqrt{1 + y^2} = \frac{x\sqrt{1 + x^2}}{2} + \frac{1}{2}\ln|x + \sqrt{1 + x^2}| + C$
\end{solutionbox}

\section*{Formula Cheat Sheet}

\subsection*{Matrix Operations}
\begin{itemize}
\item $(A + B)^T = A^T + B^T$
\item $(AB)^T = B^T A^T$
\item $A \cdot adj(A) = |A| \cdot I$
\item For 2×2 matrix $\begin{bmatrix} a & b \\ c & d \end{bmatrix}$: $adj = \begin{bmatrix} d & -b \\ -c & a \end{bmatrix}$
\end{itemize}

\subsection*{Differentiation Formulas}
\begin{itemize}
\item $\frac{d}{dx}(\sin x) = \cos x$
\item $\frac{d}{dx}(\cos x) = -\sin x$
\item $\frac{d}{dx}(\tan x) = \sec^2 x$
\item $\frac{d}{dx}(\log x) = \frac{1}{x}$
\item $\frac{d}{dx}(e^x) = e^x$
\item Chain rule: $\frac{d}{dx}f(g(x)) = f'(g(x)) \cdot g'(x)$
\end{itemize}

\subsection*{Integration Formulas}
\begin{itemize}
\item $\int \sin x \, dx = -\cos x + C$
\item $\int \cos x \, dx = \sin x + C$
\item $\int \sec^2 x \, dx = \tan x + C$
\item $\int \frac{1}{x} dx = \ln|x| + C$
\item $\int e^x dx = e^x + C$
\item $\int \frac{1}{x^2 + a^2} dx = \frac{1}{a}\tan^{-1}(\frac{x}{a}) + C$
\end{itemize}

\subsection*{Differential Equations}
\begin{itemize}
\item \textbf{Linear DE}: $\frac{dy}{dx} + P(x)y = Q(x)$
\item \textbf{Integrating Factor}: $I.F. = e^{\int P(x) dx}$
\item \textbf{Variable Separable}: $\frac{dy}{dx} = f(x)g(y) \Rightarrow \frac{dy}{g(y)} = f(x)dx$
\end{itemize}

\subsection*{Statistics}
\begin{itemize}
\item \textbf{Mean}: $\bar{x} = \frac{\sum x_i}{n}$ (ungrouped), $\bar{x} = \frac{\sum f_i x_i}{\sum f_i}$ (grouped)
\item \textbf{Mean Deviation}: $M.D. = \frac{\sum |x_i - \bar{x}|}{n}$
\item \textbf{Standard Deviation}: $\sigma = \sqrt{\frac{\sum (x_i - \bar{x})^2}{n}}$
\end{itemize}

\end{document}
