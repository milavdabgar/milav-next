\documentclass{article}

% content/resources/templates/preamble.tex
\usepackage[margin=0.6in]{geometry}
\author{Milav Dabgar}
\usepackage{amsmath,amssymb,amsthm}
\usepackage{booktabs}
\usepackage{multirow}
\usepackage{xcolor}
\usepackage{tcolorbox}
\tcbuselibrary{breakable,skins}
\usepackage[colorlinks=true,linkcolor=blue]{hyperref}
\usepackage{titlesec}
\usepackage{enumitem}
\usepackage{tikz}
\usepackage{pgfplots}
\usepackage{circuitikz}
\usepackage[version=4]{mhchem}
\usepackage{longtable}
\usepackage{array}
\usepackage{float}
\usepackage{caption}
\usepackage{listings}

\lstset{
  basicstyle=\small\ttfamily,
  breaklines=true,
  breakatwhitespace=false,
  postbreak=\mbox{\textcolor{red}{$\hookrightarrow$}\space},
  float=false,
  numbers=left,
  numberstyle=\tiny\color{gray},
  numbersep=10pt,
  xleftmargin=2em,
  keywordstyle=\color{blue},
  commentstyle=\color{green!60!black},
  stringstyle=\color{purple},
  backgroundcolor=\color{gray!5},
  showstringspaces=false,
  tabsize=2,
  captionpos=b,
  keepspaces=true,
  columns=flexible
}

\pgfplotsset{compat=1.18}
\usetikzlibrary{shapes,arrows,positioning,calc,patterns,decorations.pathmorphing,decorations.markings,arrows.meta}

% Color scheme
\definecolor{headcolor}{RGB}{0,102,204}
\definecolor{keycolor}{RGB}{220,20,60}
\definecolor{solutioncolor}{RGB}{34,139,34}
\definecolor{mnemoniccolor}{RGB}{148,0,211}
\definecolor{codecolor}{RGB}{0,0,100}

% Spacing
\setlength{\parskip}{3pt}
\setlist[itemize]{nosep}
\setlist[enumerate]{nosep}

% Title formatting
\titleformat{\section}{\Large\bfseries\color{headcolor}}{\thesection}{1em}{}
\titleformat{\subsection}{\large\bfseries\color{headcolor}}{\thesubsection}{1em}{}

% Pandoc tightlist compatibility
\providecommand{\tightlist}{%
  \setlength{\itemsep}{0pt}\setlength{\parskip}{0pt}}

% Pandoc longtable compatibility
\newcounter{none}
\def\thenone{}


% content/resources/templates/gujarati-boxes.tex
\usepackage{fontspec}
\usepackage{polyglossia}

% Set Gujarati as main language (document is primarily in Gujarati)
% Note: gloss-gujarati.ldf doesn't exist in polyglossia, but it will use hyphenation patterns
\setdefaultlanguage{gujarati}
\setotherlanguage{english}

% Configure Gujarati font properly
% Use Language=Default to prevent polyglossia from trying to add language-specific features
% that don't exist for Gujarati, which causes "empty feature" warnings
\newfontfamily\gujaratifont[Script=Gujarati,AutoFakeBold=2.5,AutoFakeSlant=0.3]{Noto Sans Gujarati}
\setmainfont[Script=Gujarati,AutoFakeBold=2.5,AutoFakeSlant=0.3]{Noto Sans Gujarati}
% Use Noto Sans Gujarati for monospace to support Gujarati in text
\setmonofont[Scale=0.9]{Noto Sans Gujarati}

% Configure English to use the same font
\newfontfamily\englishfont[Script=Gujarati,AutoFakeBold=2.5,AutoFakeSlant=0.3]{Noto Sans Gujarati}

% Translations for polyglossia
\gappto\captionsgujarati{
  \renewcommand{\tablename}{કોષ્ટક}
  \renewcommand{\figurename}{આકૃતિ}
}

% Helper for TikZ nodes to ensure Gujarati font
\newcommand{\gu}[1]{{\gujaratifont #1}}

% Custom environments
\newtcolorbox{solutionbox}{
    breakable,
    enhanced,
    colback=solutioncolor!5!white,
    colframe=solutioncolor!75!black,
    fonttitle=\bfseries,
    title=જવાબ
}

\newtcolorbox{solutionboxnobreak}{
 colback=solutioncolor!5!white,
 colframe=solutioncolor!75!black,
 fonttitle=\bfseries,
 title=જવાબ
}

\newtcolorbox{keyformula}{
 breakable,
 enhanced,
 colback=keycolor!5!white,
 colframe=keycolor!75!black,
 fonttitle=\bfseries,
 title=રાસાયણિક સમીકરણ/સૂત્ર
}

\newtcolorbox{mnemonicbox}{
 breakable,
 enhanced,
 colback=mnemoniccolor!5!white,
 colframe=mnemoniccolor!75!black,
 fonttitle=\bfseries,
 title=મેમરી ટ્રીક
}


% Custom commands for GTU solutions
% This file defines semantic commands for consistent formatting

% Question command with automatic formatting
\newcommand{\question}[2]{%
  \section*{Question #1}%
  \textbf{#2}%
}

% OR question variant
\newcommand{\questionor}[2]{%
  \section*{Question #1 OR}%
  \textbf{#2}%
}

% Proper table environment with caption
\newenvironment{answertable}[1]{%
  \begin{table}[htbp]
  \centering
  \caption{#1}
}{%
  \end{table}
}

% Proper figure environment for diagrams
\newenvironment{answerdiagram}[1]{%
  \begin{figure}[htbp]
  \centering
  \caption{#1}
}{%
  \end{figure}
}

% Semantic markup for key terms
\newcommand{\keyword}[1]{\textbf{#1}}
\newcommand{\code}[1]{\texttt{#1}}
\newcommand{\classname}[1]{\texttt{#1}}
\newcommand{\methodname}[1]{\texttt{#1}}

% Proper quotation marks
\newcommand{\mnemonic}[1]{``#1''}


\title{Applied Mathematics (4320001) - Winter 2022 Solution}
\date{February 23, 2022}

\begin{document}
\maketitle

\questionmarks{1}{14}{નીચેના વિકલ્પોમાંથી યોગ્ય વિકલ્પ પસંદ કરી ખાલી જગ્યા પૂરો.}

\questionmarks{1(1)}{1}{$\begin{bmatrix} 1 & 4 \\ 3 & 2 \end{bmatrix}$ શ્રેણિકની કક્ષા \underline{\hspace{2cm}} છે.}
\textbf{જવાબ}: b. 2 × 2

\begin{solutionbox}
શ્રેણિકને 2 હાર અને 2 સ્તંભ છે, તેથી કક્ષા 2 × 2 છે.
\end{solutionbox}

\questionmarks{1(2)}{1}{જો $A = \begin{bmatrix} 1 & 2 \\ -1 & 1 \end{bmatrix}$ હોય તો $2A - 3I$ = \underline{\hspace{2cm}}}
\textbf{જવાબ}: a. $\begin{bmatrix} -1 & 4 \\ -2 & -1 \end{bmatrix}$

\begin{solutionbox}
$2A = 2\begin{bmatrix} 1 & 2 \\ -1 & 1 \end{bmatrix} = \begin{bmatrix} 2 & 4 \\ -2 & 2 \end{bmatrix}$

$3I = 3\begin{bmatrix} 1 & 0 \\ 0 & 1 \end{bmatrix} = \begin{bmatrix} 3 & 0 \\ 0 & 3 \end{bmatrix}$

$2A - 3I = \begin{bmatrix} 2 & 4 \\ -2 & 2 \end{bmatrix} - \begin{bmatrix} 3 & 0 \\ 0 & 3 \end{bmatrix} = \begin{bmatrix} -1 & 4 \\ -2 & -1 \end{bmatrix}$
\end{solutionbox}

\questionmarks{1(3)}{1}{જો $A_{2×3}$ અને $B_{3×4}$ શ્રેણિક હોય તો $AB$ ની કક્ષા \underline{\hspace{2cm}} છે.}
\textbf{જવાબ}: b. 2 × 4

\begin{solutionbox}
શ્રેણિક ગુણાકાર $AB$ માટે, જો $A$ એ $m×n$ અને $B$ એ $n×p$ હોય, તો $AB$ એ $m×p$ થાય.
અહીં: $A_{2×3} \times B_{3×4} = (AB)_{2×4}$
\end{solutionbox}

\questionmarks{1(4)}{1}{જો $AB = I$ હોય તો શ્રેણિક $B$ = ...}
\textbf{જવાબ}: c. $A^{-1}$

\begin{solutionbox}
જો $AB = I$, તો $B$ એ $A$ નો વ્યસ્ત શ્રેણિક છે, એટલે કે $B = A^{-1}$
\end{solutionbox}

\questionmarks{1(5)}{1}{$\frac{d}{dx}(x^3 + 3^x + 3^3)$ = \underline{\hspace{2cm}}}
\textbf{જવાબ}: c. $3x^2 + 3^x \log 3$

\begin{solutionbox}
$\frac{d}{dx}(x^3 + 3^x + 3^3) = 3x^2 + 3^x \log 3 + 0 = 3x^2 + 3^x \log 3$
\end{solutionbox}

\questionmarks{1(6)}{1}{જો $f(x) = e^{3x}$ હોય તો $f'(0)$ = \underline{\hspace{2cm}}}
\textbf{જવાબ}: b. 3

\begin{solutionbox}
$f'(x) = 3e^{3x}$
$f'(0) = 3e^{3(0)} = 3e^0 = 3(1) = 3$
\end{solutionbox}

\questionmarks{1(7)}{1}{જો $y = e^x + 100x$ હોય તો $\frac{d^2y}{dx^2}$ = \underline{\hspace{2cm}}}
\textbf{જવાબ}: a. $e^x$

\begin{solutionbox}
$\frac{dy}{dx} = e^x + 100$
$\frac{d^2y}{dx^2} = e^x + 0 = e^x$
\end{solutionbox}

\questionmarks{1(8)}{1}{$\int \frac{1}{x^2} dx$ = \underline{\hspace{2cm}} + c}
\textbf{જવાબ}: b. $-\frac{1}{x}$

\begin{solutionbox}
$\int x^{-2} dx = \frac{x^{-2+1}}{-2+1} = \frac{x^{-1}}{-1} = -\frac{1}{x} + c$
\end{solutionbox}

\questionmarks{1(9)}{1}{$\int (\log a) dx$ = \underline{\hspace{2cm}} + c}
\textbf{જવાબ}: a. $x\log a$

\begin{solutionbox}
કેમકે $\log a$ અચળ છે:
$\int (\log a) dx = (\log a) \int dx = x\log a + c$
\end{solutionbox}

\questionmarks{1(10)}{1}{$\int_0^1 e^x dx$ = \underline{\hspace{2cm}}}
\textbf{જવાબ}: a. $e - 1$

\begin{solutionbox}
$\int_0^1 e^x dx = [e^x]_0^1 = e^1 - e^0 = e - 1$
\end{solutionbox}

\questionmarks{1(11)}{1}{વિકલ સમીકરણ $\frac{d^2y}{dx^2} - 5\frac{dy}{dx} + 6y = 0$ ની કક્ષા અને પરિમાણ અનુક્રમે \underline{\hspace{2cm}} અને \underline{\hspace{2cm}} છે.}
\textbf{જવાબ}: d. 2,1

\begin{solutionbox}
કક્ષા (Order) = મહત્તમ વિકલન = 2
પરિમાણ (Degree) = મહત્તમ વિકલનની ઘાત = 1
\end{solutionbox}

\questionmarks{1(12)}{1}{વિકલ સમીકરણ $\frac{dy}{dx} + y = 3x$ નો સંકલ્યકારક અવયવ (I.F) \underline{\hspace{2cm}} છે.}
\textbf{જવાબ}: c. $e^x$

\begin{solutionbox}
સમીકરણ $\frac{dy}{dx} + Py = Q$ માટે જ્યાં $P = 1$:
I.F. = $e^{\int P dx} = e^{\int 1 dx} = e^x$
\end{solutionbox}

\questionmarks{1(13)}{1}{પ્રથમ પાંચ પ્રાકૃતિક સંખ્યાઓનો મધ્યક \underline{\hspace{2cm}} છે.}
\textbf{જવાબ}: c. 3

\begin{solutionbox}
પ્રથમ પાંચ પ્રાકૃતિક સંખ્યાઓ: 1, 2, 3, 4, 5
મધ્યક = $\frac{1+2+3+4+5}{5} = \frac{15}{5} = 3$
\end{solutionbox}

\questionmarks{1(14)}{1}{જો અવલોકનો 11, x, 19, 21, y, 29 નો મધ્યક 20 હોય તો $x + y$ = \underline{\hspace{2cm}}}
\textbf{જવાબ}: a. 40

\begin{solutionbox}
મધ્યક = $\frac{11+x+19+21+y+29}{6} = 20$
$\frac{80+x+y}{6} = 20$
$80+x+y = 120$
$x+y = 40$
\end{solutionbox}

\questionmarks{2(a)}{6}{કોઈપણ બે ગણો}

\questionmarks{2(a)(1)}{3}{જો $A = \begin{bmatrix} 1 & 3 & 2 \\ 2 & 0 & 1 \end{bmatrix}$ અને $B = \begin{bmatrix} 2 & 1 \\ -1 & 1 \\ 1 & -1 \end{bmatrix}$ હોય તો $(AB)^T$ શોધો.}

\begin{solutionbox}
પ્રથમ $AB$ શોધો:
$AB = \begin{bmatrix} 1 & 3 & 2 \\ 2 & 0 & 1 \end{bmatrix} \begin{bmatrix} 2 & 1 \\ -1 & 1 \\ 1 & -1 \end{bmatrix}$

$AB = \begin{bmatrix} 1(2)+3(-1)+2(1) & 1(1)+3(1)+2(-1) \\ 2(2)+0(-1)+1(1) & 2(1)+0(1)+1(-1) \end{bmatrix}$

$AB = \begin{bmatrix} 2-3+2 & 1+3-2 \\ 4+0+1 & 2+0-1 \end{bmatrix} = \begin{bmatrix} 1 & 2 \\ 5 & 1 \end{bmatrix}$

$(AB)^T = \begin{bmatrix} 1 & 5 \\ 2 & 1 \end{bmatrix}$
\end{solutionbox}

\questionmarks{2(a)(2)}{3}{જો $1 + x + x^2 = 0$ અને $x^3 = 1$ હોય તો સાબિત કરો કે $\begin{bmatrix} 1 & x^2 \\ x & x \end{bmatrix} \cdot \begin{bmatrix} x & x^2 \\ 1 & x \end{bmatrix} = \begin{bmatrix} -1 & -1 \\ -1 & 2 \end{bmatrix}$}

\begin{solutionbox}
Given: $1 + x + x^2 = 0$ and $x^3 = 1$

From $1 + x + x^2 = 0$, we get $x^2 = -1 - x$

Let's compute the matrix product:
$\begin{bmatrix} 1 & x^2 \\ x & x \end{bmatrix} \cdot \begin{bmatrix} x & x^2 \\ 1 & x \end{bmatrix}$

$= \begin{bmatrix} 1(x)+x^2(1) & 1(x^2)+x^2(x) \\ x(x)+x(1) & x(x^2)+x(x) \end{bmatrix}$

$= \begin{bmatrix} x+x^2 & x^2+x^3 \\ x^2+x & x^3+x^2 \end{bmatrix}$

Since $x^3 = 1$ and $x+x^2 = -1$:
$= \begin{bmatrix} -1 & x^2+1 \\ -1 & 1+x^2 \end{bmatrix}$

Since $x^2 = -1-x$, we have $x^2+1 = -x$ and $1+x^2 = -x$

From $1+x+x^2 = 0$, if $x$ is a cube root of unity, then $x^2+1 = -x = -1$

$= \begin{bmatrix} -1 & -1 \\ -1 & 2 \end{bmatrix}$ \checkmark
\end{solutionbox}

\questionmarks{2(a)(3)}{3}{ઉકેલો: $\frac{dy}{dx} + x^2e^{-y} = 0$}

\begin{solutionbox}
$\frac{dy}{dx} = -x^2e^{-y}$

ચલને અલગ કરતા (Separating variables):
$e^y dy = -x^2 dx$
બંને બાજુ સંકલન કરતા:
$\int e^y dy = \int -x^2 dx$

$e^y = -\frac{x^3}{3} + C$

$y = \ln\left(-\frac{x^3}{3} + C\right)$
\end{solutionbox}

\questionmarks{2(b)}{8}{કોઈપણ બે ગણો}

\questionmarks{2(b)(1)}{4}{જો $A = \begin{bmatrix} 1 & 2 & 2 \\ 2 & 1 & 2 \\ 2 & 2 & 1 \end{bmatrix}$ હોય તો સાબિત કરો કે $A^2 - 4A - 5I_3 = O$}

\begin{solutionbox}
પ્રથમ $A^2$ ગણો:
$A^2 = \begin{bmatrix} 1 & 2 & 2 \\ 2 & 1 & 2 \\ 2 & 2 & 1 \end{bmatrix} \begin{bmatrix} 1 & 2 & 2 \\ 2 & 1 & 2 \\ 2 & 2 & 1 \end{bmatrix}$

$A^2 = \begin{bmatrix} 1+4+4 & 2+2+4 & 2+4+2 \\ 2+2+4 & 4+1+4 & 4+2+2 \\ 2+4+2 & 4+2+2 & 4+4+1 \end{bmatrix} = \begin{bmatrix} 9 & 8 & 8 \\ 8 & 9 & 8 \\ 8 & 8 & 9 \end{bmatrix}$

હવે $A^2 - 4A - 5I_3$ ગણો:
$4A = \begin{bmatrix} 4 & 8 & 8 \\ 8 & 4 & 8 \\ 8 & 8 & 4 \end{bmatrix}$

$5I_3 = \begin{bmatrix} 5 & 0 & 0 \\ 0 & 5 & 0 \\ 0 & 0 & 5 \end{bmatrix}$

$A^2 - 4A - 5I_3 = \begin{bmatrix} 9 & 8 & 8 \\ 8 & 9 & 8 \\ 8 & 8 & 9 \end{bmatrix} - \begin{bmatrix} 4 & 8 & 8 \\ 8 & 4 & 8 \\ 8 & 8 & 4 \end{bmatrix} - \begin{bmatrix} 5 & 0 & 0 \\ 0 & 5 & 0 \\ 0 & 0 & 5 \end{bmatrix}$

$= \begin{bmatrix} 0 & 0 & 0 \\ 0 & 0 & 0 \\ 0 & 0 & 0 \end{bmatrix} = O$
\end{solutionbox}

\questionmarks{2(b)(2)}{4}{$x$ ની કઈ કિંમત માટે શ્રેણિક $\begin{bmatrix} 3-x & 2 & 2 \\ 1 & 4-x & 1 \\ -2 & -4 & -1-x \end{bmatrix}$ અસામાન્ય શ્રેણિક બને?}

\begin{solutionbox}
જ્યારે નિશ્ચાયક શૂન્ય થાય ત્યારે શ્રેણિક અસામાન્ય બને છે.

$\det(A) = (3-x)\begin{vmatrix} 4-x & 1 \\ -4 & -1-x \end{vmatrix} - 2\begin{vmatrix} 1 & 1 \\ -2 & -1-x \end{vmatrix} + 2\begin{vmatrix} 1 & 4-x \\ -2 & -4 \end{vmatrix}$

$= (3-x)[(4-x)(-1-x) - (1)(-4)] - 2[1(-1-x) - 1(-2)] + 2[1(-4) - (4-x)(-2)]$

$= (3-x)[-(4-x)(1+x) + 4] - 2[-1-x+2] + 2[-4 + 2(4-x)]$

$= (3-x)[-4-4x+x+x^2+4] - 2[1-x] + 2[-4+8-2x]$

$= (3-x)[x^2-3x] - 2(1-x) + 2(4-2x)$

$= (3-x)x(x-3) - 2 + 2x + 8 - 4x$

$= -x(3-x)^2 + 6 - 2x$

શૂન્ય સાથે સરખાવતા:
$-x(3-x)^2 + 6 - 2x = 0$
આનાથી $x = 1, x = 2, x = 3$ મળે છે.
\end{solutionbox}

\questionmarks{2(b)(3)}{4}{શ્રેણિક પદ્ધતિથી ઉકેલો: $2y + 5x = 4$, $7x + 3y = 5$}

\begin{solutionbox}
શ્રેણિક સ્વરૂપ $AX = B$ માં લખતા:
$\begin{bmatrix} 5 & 2 \\ 7 & 3 \end{bmatrix} \begin{bmatrix} x \\ y \end{bmatrix} = \begin{bmatrix} 4 \\ 5 \end{bmatrix}$

$A^{-1}$ શોધો:
$\det(A) = 5(3) - 2(7) = 15 - 14 = 1$

$A^{-1} = \frac{1}{1}\begin{bmatrix} 3 & -2 \\ -7 & 5 \end{bmatrix} = \begin{bmatrix} 3 & -2 \\ -7 & 5 \end{bmatrix}$

$X = A^{-1}B = \begin{bmatrix} 3 & -2 \\ -7 & 5 \end{bmatrix} \begin{bmatrix} 4 \\ 5 \end{bmatrix} = \begin{bmatrix} 12-10 \\ -28+25 \end{bmatrix} = \begin{bmatrix} 2 \\ -3 \end{bmatrix}$

તેથી: $x = 2, y = -3$
\end{solutionbox}

\questionmarks{3(a)}{6}{કોઈપણ બે ગણો}

\questionmarks{3(a)(1)}{3}{વ્યાખ્યાની મદદથી $f(x) = \sqrt{x}$ નું વિકલિત શોધો.}

\begin{solutionbox}
વ્યાખ્યાનો ઉપયોગ કરતા: $f'(x) = \lim_{h \to 0} \frac{f(x+h) - f(x)}{h}$

$f'(x) = \lim_{h \to 0} \frac{\sqrt{x+h} - \sqrt{x}}{h}$

અંશનું સંમેયીકરણ કરતા:
$= \lim_{h \to 0} \frac{(\sqrt{x+h} - \sqrt{x})(\sqrt{x+h} + \sqrt{x})}{h(\sqrt{x+h} + \sqrt{x})}$

$= \lim_{h \to 0} \frac{(x+h) - x}{h(\sqrt{x+h} + \sqrt{x})}$

$= \lim_{h \to 0} \frac{h}{h(\sqrt{x+h} + \sqrt{x})}$

$= \lim_{h \to 0} \frac{1}{\sqrt{x+h} + \sqrt{x}}$

$= \frac{1}{\sqrt{x} + \sqrt{x}} = \frac{1}{2\sqrt{x}}$
\end{solutionbox}

\questionmarks{3(a)(2)}{3}{જો $x + y = \sin(xy)$ હોય તો $\frac{dy}{dx}$ શોધો.}

\begin{solutionbox}
$x$ ની સાપેક્ષે બંને બાજુ વિકલન કરતા:
$\frac{d}{dx}(x + y) = \frac{d}{dx}[\sin(xy)]$

$1 + \frac{dy}{dx} = \cos(xy) \cdot \frac{d}{dx}(xy)$

$1 + \frac{dy}{dx} = \cos(xy) \cdot \left(x\frac{dy}{dx} + y\right)$

$1 + \frac{dy}{dx} = \cos(xy) \cdot x\frac{dy}{dx} + y\cos(xy)$

$1 + \frac{dy}{dx} - x\cos(xy)\frac{dy}{dx} = y\cos(xy)$

$\frac{dy}{dx}(1 - x\cos(xy)) = y\cos(xy) - 1$

$\frac{dy}{dx} = \frac{y\cos(xy) - 1}{1 - x\cos(xy)}$
\end{solutionbox}

\questionmarks{3(a)(3)}{3}{કિંમત શોધો: $\int \frac{\sin^3x + \cos^3x}{\sin^2x \cos^2x} dx$}

\begin{solutionbox}
$\int \frac{\sin^3x + \cos^3x}{\sin^2x \cos^2x} dx = \int \frac{\sin^3x}{\sin^2x \cos^2x} dx + \int \frac{\cos^3x}{\sin^2x \cos^2x} dx$

$= \int \frac{\sin x}{\cos^2x} dx + \int \frac{\cos x}{\sin^2x} dx$

$= \int \sin x \sec^2x dx + \int \cos x \csc^2x dx$

પ્રથમ સંકલન માટે, ધારો કે $u = \cos x$, તો $du = -\sin x dx$:
$\int \sin x \sec^2x dx = -\int \frac{1}{u^2} du = \frac{1}{u} = \sec x$

બીજા સંકલન માટે, ધારો કે $v = \sin x$, તો $dv = \cos x dx$:
$\int \cos x \csc^2x dx = \int \frac{1}{v^2} dv = -\frac{1}{v} = -\csc x$

તેથી: $\int \frac{\sin^3x + \cos^3x}{\sin^2x \cos^2x} dx = \sec x - \csc x + C$
\end{solutionbox}

\questionmarks{3(b)}{8}{કોઈપણ બે ગણો}

\questionmarks{3(b)(1)}{4}{જો $y = e^x \cdot \sin x$ હોય તો સાબિત કરો કે $\frac{d^2y}{dx^2} - 2\frac{dy}{dx} + 2y = 0$}

\begin{solutionbox}
આપેલ છે: $y = e^x \sin x$
પ્રથમ વિકલિત શોધો:
$\frac{dy}{dx} = \frac{d}{dx}(e^x \sin x) = e^x \sin x + e^x \cos x = e^x(\sin x + \cos x)$

બીજું વિકલિત શોધો:
$\frac{d^2y}{dx^2} = \frac{d}{dx}[e^x(\sin x + \cos x)]$
$= e^x(\sin x + \cos x) + e^x(\cos x - \sin x)$
$= e^x[\sin x + \cos x + \cos x - \sin x]$
$= 2e^x \cos x$

હવે ચકાસતા:
$\frac{d^2y}{dx^2} - 2\frac{dy}{dx} + 2y$
$= 2e^x \cos x - 2e^x(\sin x + \cos x) + 2e^x \sin x$
$= 2e^x \cos x - 2e^x \sin x - 2e^x \cos x + 2e^x \sin x$
$= 0$

સાબિત થાય છે.
\end{solutionbox}

\questionmarks{3(b)(2)}{4}{વિધેય $f(x) = x^3 - 4x^2 + 5x + 7$ ની મહત્તમ અને ન્યૂનતમ કિંમત શોધો.}

\begin{solutionbox}
$f'(x) = 0$ લઈને નિર્ણાયક બિંદુઓ શોધો:
$f'(x) = 3x^2 - 8x + 5 = 0$

દ્વિઘાત સૂત્રનો ઉપયોગ કરતા:
$x = \frac{8 \pm \sqrt{64 - 60}}{6} = \frac{8 \pm 2}{6}$

તેથી $x = \frac{5}{3}$ અથવા $x = 1$

બીજું વિકલિત શોધો:
$f''(x) = 6x - 8$

નિર્ણાયક બિંદુઓ ચકાસતા:
- $x = 1$ માટે: $f''(1) = 6(1) - 8 = -2 < 0$ \rightarrow સ્થાનીય મહત્તમ
- $x = \frac{5}{3}$ માટે: $f''\left(\frac{5}{3}\right) = 6\left(\frac{5}{3}\right) - 8 = 10 - 8 = 2 > 0$ \rightarrow સ્થાનીય ન્યૂનતમ

વિધેયની કિંમતો ગણતા:
- $f(1) = 1 - 4 + 5 + 7 = 9$ (સ્થાનીય મહત્તમ)
- $f\left(\frac{5}{3}\right) = \left(\frac{5}{3}\right)^3 - 4\left(\frac{5}{3}\right)^2 + 5\left(\frac{5}{3}\right) + 7 = \frac{125}{27} - \frac{100}{9} + \frac{25}{3} + 7 = \frac{158}{27}$ (સ્થાનીય ન્યૂનતમ)
\end{solutionbox}

\questionmarks{3(b)(3)}{4}{એક કણનું ગતિ સમીકરણ $s = t^3 - 6t^2 + 9t$ છે તો \newline (i) $t = 3$ સેકન્ડે વેગ અને પ્રવેગ શોધો. \newline (ii) પ્રવેગ શૂન્ય થાય ત્યારે "t" શોધો.}

\begin{solutionbox}
આપેલ છે: $s = t^3 - 6t^2 + 9t$
વેગ: $v = \frac{ds}{dt} = 3t^2 - 12t + 9$
પ્રવેગ: $a = \frac{dv}{dt} = 6t - 12$

(i) $t = 3$ સેકન્ડે:
- વેગ: $v(3) = 3(9) - 12(3) + 9 = 27 - 36 + 9 = 0$ m/s
- પ્રવેગ: $a(3) = 6(3) - 12 = 18 - 12 = 6$ m/s²

(ii) જ્યારે પ્રવેગ શૂન્ય હોય:
$6t - 12 = 0$
$t = 2$ સેકન્ડ
\end{solutionbox}

\questionmarks{4(a)}{6}{કોઈપણ બે ગણો}

\questionmarks{4(a)(1)}{3}{કિંમત શોધો: $\int \frac{x}{(x+1)(x+2)} dx$}

\begin{solutionbox}
આંશિક અપૂર્ણાંકનો ઉપયોગ કરતા:
$\frac{x}{(x+1)(x+2)} = \frac{A}{x+1} + \frac{B}{x+2}$

$x = A(x+2) + B(x+1)$

$x = -1$ લેતા: $-1 = A(1) \Rightarrow A = -1$
$x = -2$ લેતા: $-2 = B(-1) \Rightarrow B = 2$

$\int \frac{x}{(x+1)(x+2)} dx = \int \left(\frac{-1}{x+1} + \frac{2}{x+2}\right) dx$

$= -\ln|x+1| + 2\ln|x+2| + C$

$= \ln\left|\frac{(x+2)^2}{x+1}\right| + C$
\end{solutionbox}

\questionmarks{4(a)(2)}{3}{કિંમત શોધો: $\int_0^{\pi/2} \frac{\sin x}{\sin x + \cos x} dx$}

\begin{solutionbox}
ધારો કે $I = \int_0^{\pi/2} \frac{\sin x}{\sin x + \cos x} dx$ ... (1)
ગુણધર્મ $\int_0^a f(x) dx = \int_0^a f(a-x) dx$ નો ઉપયોગ કરતા:

$I = \int_0^{\pi/2} \frac{\sin(\pi/2 - x)}{\sin(\pi/2 - x) + \cos(\pi/2 - x)} dx$

$= \int_0^{\pi/2} \frac{\cos x}{\cos x + \sin x} dx$ ... (2)

સમીકરણ (1) અને (2) નો સરવાળો કરતા:
$2I = \int_0^{\pi/2} \frac{\sin x + \cos x}{\sin x + \cos x} dx = \int_0^{\pi/2} 1 dx$

$2I = \left[x\right]_0^{\pi/2} = \frac{\pi}{2}$

તેથી: $I = \frac{\pi}{4}$
\end{solutionbox}

\questionmarks{4(a)(3)}{3}{જો 15, 7, 6, a, 3 નો મધ્યક 7 હોય તો "a" શોધો.}

\begin{solutionbox}
મધ્યક = $\frac{\text{અવલોકનોનો સરવાળો}}{\text{અવલોકનોની સંખ્યા}}$

$7 = \frac{15 + 7 + 6 + a + 3}{5}$

$7 = \frac{31 + a}{5}$

$35 = 31 + a$

$a = 4$
\end{solutionbox}

\questionmarks{4(b)}{8}{કોઈપણ બે ગણો}

\questionmarks{4(b)(1)}{4}{કિંમત શોધો: $\int x^2 e^x dx$}

\begin{solutionbox}
બે વખત ખંડશઃ સંકલનનો ઉપયોગ કરતા:
ધારો કે $u = x^2$, $dv = e^x dx$
તો $du = 2x dx$, $v = e^x$

$\int x^2 e^x dx = x^2 e^x - \int 2x e^x dx$

$\int 2x e^x dx$ માટે, ફરીથી ખંડશઃ સંકલન વાપરતા:
ધારો કે $u = 2x$, $dv = e^x dx$
તો $du = 2 dx$, $v = e^x$

$\int 2x e^x dx = 2x e^x - \int 2 e^x dx = 2x e^x - 2e^x$

Therefore:
$\int x^2 e^x dx = x^2 e^x - (2x e^x - 2e^x) + C$
$= x^2 e^x - 2x e^x + 2e^x + C$
$= e^x(x^2 - 2x + 2) + C$
\end{solutionbox}

\questionmarks{4(b)(2)}{4}{વક્ર $y = 2x^2$, રેખાઓ $x = 1$, $x = 3$ અને X-અક્ષ દ્વારા આવૃત પ્રદેશનું ક્ષેત્રફળ શોધો.}

\begin{solutionbox}
Area = $\int_1^3 2x^2 dx$

$= 2\int_1^3 x^2 dx$

$= 2\left[\frac{x^3}{3}\right]_1^3$

$= \frac{2}{3}[x^3]_1^3$

$= \frac{2}{3}(27 - 1)$

$= \frac{2}{3} \times 26$

$= $\frac{52}{3}$ ચોરસ એકમ
\end{solutionbox}

\questionmarks{4(b)(3)}{4}{નીચેના વર્ગીકૃત માહિતી માટે ટૂંકી રીતે મધ્યક શોધો:}

\begin{solutionbox}
\begin{answertable}{વર્ગીકૃત માહિતી}
\begin{tabulary}{\linewidth}{|C|C|C|C|C|C|C|}
\hline
Marks & 21-25 & 26-30 & 31-35 & 36-40 & 41-45 & 46-50 \\ \hline
No. of Students & 8 & 10 & 24 & 30 & 12 & 16 \\ \hline
\end{tabulary}
\end{answertable}

પદ વિચલન રીતનો ઉપયોગ કરતા:

\begin{answertable}{મધ્યક ગણતરી}
\begin{tabulary}{\linewidth}{|C|C|C|C|C|}
\hline
Class & $x_i$ & $f_i$ & $d_i = \frac{x_i - A}{h}$ & $f_i d_i$ \\ \hline
21-25 & 23 & 8 & -3 & -24 \\ \hline
26-30 & 28 & 10 & -2 & -20 \\ \hline
31-35 & 33 & 24 & -1 & -24 \\ \hline
36-40 & 38 & 30 & 0 & 0 \\ \hline
41-45 & 43 & 12 & 1 & 12 \\ \hline
46-50 & 48 & 16 & 2 & 32 \\ \hline
Total & - & 100 & - & -24 \\ \hline
\end{tabulary}
\end{answertable}

ધારેલો મધ્યક $A = 38$, વર્ગ લંબાઈ $h = 5$

Mean = $A + \frac{\sum f_i d_i}{\sum f_i} \times h$

Mean = $38 + \frac{-24}{100} \times 5 = 38 - 1.2 = 36.8$
\end{solutionbox}

\questionmarks{5(a)}{6}{કોઈપણ બે ગણો}

\questionmarks{5(a)(1)}{3}{નીચેના માહિતી માટે મધ્યક શોધો:}

\begin{solutionbox}
\begin{answertable}{વર્ગીકૃત માહિતી}
\begin{tabulary}{\linewidth}{|C|C|C|C|C|C|C|}
\hline
$x_i$ & 92 & 93 & 97 & 98 & 102 & 104 \\ \hline
$f_i$ & 3 & 2 & 3 & 2 & 6 & 4 \\ \hline
\end{tabulary}
\end{answertable}

Mean = $\frac{\sum f_i x_i}{\sum f_i}$

\begin{answertable}{પદ વિચલન ગણતરી}
\begin{tabulary}{\linewidth}{|C|C|C|}
\hline
$x_i$ & $f_i$ & $f_i x_i$ \\ \hline
92 & 3 & 276 \\ \hline
93 & 2 & 186 \\ \hline
97 & 3 & 291 \\ \hline
98 & 2 & 196 \\ \hline
102 & 6 & 612 \\ \hline
104 & 4 & 416 \\ \hline
Total & 20 & 1977 \\ \hline
\end{tabulary}
\end{answertable}

Mean = $\frac{1977}{20} = 98.85$
\end{solutionbox}

\questionmarks{5(a)(2)}{3}{4, 6, 2, 4, 5, 4, 4, 5, 3, 4 માટે સરેરાશ વિચલન શોધો.}

\begin{solutionbox}
First find the mean:
Mean = $\frac{4+6+2+4+5+4+4+5+3+4}{10} = \frac{41}{10} = 4.1$

મધ્યકથી વિચલન ગણો:

\begin{answertable}{વિચલન ગણતરી}
\begin{tabulary}{\linewidth}{|C|C|}
\hline
$x_i$ & $|x_i - \bar{x}|$ \\ \hline
4 & $|4 - 4.1| = 0.1$ \\ \hline
6 & $|6 - 4.1| = 1.9$ \\ \hline
2 & $|2 - 4.1| = 2.1$ \\ \hline
4 & $|4 - 4.1| = 0.1$ \\ \hline
5 & $|5 - 4.1| = 0.9$ \\ \hline
4 & $|4 - 4.1| = 0.1$ \\ \hline
4 & $|4 - 4.1| = 0.1$ \\ \hline
5 & $|5 - 4.1| = 0.9$ \\ \hline
3 & $|3 - 4.1| = 1.1$ \\ \hline
4 & $|4 - 4.1| = 0.1$ \\ \hline
Total & 7.4 \\ \hline
\end{tabulary}
\end{answertable}

સરેરાશ વિચલન = $\frac{\sum |x_i - \bar{x}|}{n} = \frac{7.4}{10} = 0.74$
\end{solutionbox}

\questionmarks{5(a)(3)}{3}{નીચેના અસતત આવૃત્તિ વિતરણ માટે પ્રમાણિત વિચલન શોધો:}

\begin{solutionbox}
\begin{answertable}{અસતત વર્ગીકૃત માહિતી}
\begin{tabulary}{\linewidth}{|C|C|C|C|C|C|C|C|}
\hline
$x_i$ & 4 & 8 & 11 & 17 & 20 & 24 & 32 \\ \hline
$f_i$ & 3 & 5 & 9 & 5 & 4 & 3 & 1 \\ \hline
\end{tabulary}
\end{answertable}

First find the mean:

\begin{answertable}{મધ્યક ગણતરી}
\begin{tabulary}{\linewidth}{|C|C|C|}
\hline
$x_i$ & $f_i$ & $f_i x_i$ \\ \hline
4 & 3 & 12 \\ \hline
8 & 5 & 40 \\ \hline
11 & 9 & 99 \\ \hline
17 & 5 & 85 \\ \hline
20 & 4 & 80 \\ \hline
24 & 3 & 72 \\ \hline
32 & 1 & 32 \\ \hline
Total & 30 & 420 \\ \hline
\end{tabulary}
\end{answertable}

Mean = $\frac{420}{30} = 14$

હવે પ્રમાણિત વિચલન ગણો:

\begin{answertable}{પ્રમાણિત વિચલન ગણતરી}
\begin{tabulary}{\linewidth}{|C|C|C|C|C|}
\hline
$x_i$ & $f_i$ & $x_i - \bar{x}$ & $(x_i - \bar{x})^2$ & $f_i(x_i - \bar{x})^2$ \\ \hline
4 & 3 & -10 & 100 & 300 \\ \hline
8 & 5 & -6 & 36 & 180 \\ \hline
11 & 9 & -3 & 9 & 81 \\ \hline
17 & 5 & 3 & 9 & 45 \\ \hline
20 & 4 & 6 & 36 & 144 \\ \hline
24 & 3 & 10 & 100 & 300 \\ \hline
32 & 1 & 18 & 324 & 324 \\ \hline
Total & 30 & - & - & 1374 \\ \hline
\end{tabulary}
\end{answertable}

પ્રમાણિત વિચલન = $\sqrt{\frac{\sum f_i(x_i - \bar{x})^2}{n}} = \sqrt{\frac{1374}{30}} = \sqrt{45.8} = 6.77$
\end{solutionbox}

\questionmarks{5(b)}{8}{કોઈપણ બે ગણો}

\questionmarks{5(b)(1)}{4}{ઉકેલો: $\frac{dy}{dx} + \frac{4x}{1+x^2}y = \frac{1}{(1+x^2)^2}$}

\begin{solutionbox}
આ $\frac{dy}{dx} + Py = Q$ સ્વરૂપનું સુરેખ વિકલ સમીકરણ છે.
જ્યાં $P = \frac{4x}{1+x^2}$ અને $Q = \frac{1}{(1+x^2)^2}$
સંકલ્યકારક અવયવ શોધો:
$\text{I.F.} = e^{\int P dx} = e^{\int \frac{4x}{1+x^2} dx}$

ધારો કે $u = 1+x^2$, તો $du = 2x dx$
$\int \frac{4x}{1+x^2} dx = 2\int \frac{du}{u} = 2\ln|u| = 2\ln(1+x^2)$

$\text{I.F.} = e^{2\ln(1+x^2)} = (1+x^2)^2$

The solution is:
$y \cdot (1+x^2)^2 = \int \frac{1}{(1+x^2)^2} \cdot (1+x^2)^2 dx$

$y(1+x^2)^2 = \int 1 dx = x + C$

$y = \frac{x + C}{(1+x^2)^2}$
\end{solutionbox}

\questionmarks{5(b)(2)}{4}{ઉકેલો: $(x + y + 1)^2 \frac{dy}{dx} = 1$}

\begin{solutionbox}
$(x + y + 1)^2 \frac{dy}{dx} = 1$

$\frac{dy}{dx} = \frac{1}{(x + y + 1)^2}$

ધારો કે $v = x + y + 1$, તો $\frac{dv}{dx} = 1 + \frac{dy}{dx}$
તેથી $\frac{dy}{dx} = \frac{dv}{dx} - 1$
કિંમત મૂકતા:
$\frac{dv}{dx} - 1 = \frac{1}{v^2}$

$\frac{dv}{dx} = 1 + \frac{1}{v^2} = \frac{v^2 + 1}{v^2}$

ચલને અલગ કરતા (Separating variables):
$\frac{v^2}{v^2 + 1} dv = dx$

$\left(1 - \frac{1}{v^2 + 1}\right) dv = dx$

બંને બાજુ સંકલન કરતા:
$\int \left(1 - \frac{1}{v^2 + 1}\right) dv = \int dx$

$v - \arctan(v) = x + C$

$v = x + y + 1$ પાછું મૂકતા:
$(x + y + 1) - \arctan(x + y + 1) = x + C$

$y + 1 - \arctan(x + y + 1) = C$

$y = \arctan(x + y + 1) + C - 1$
\end{solutionbox}

\questionmarks{5(b)(3)}{4}{ઉકેલો: $\frac{dy}{dx} + y = e^x$, $y(0) = 1$}

\begin{solutionbox}
આ $P = 1$ અને $Q = e^x$ સાથેનું સુરેખ વિકલ સમીકરણ છે.
સંકલ્યકારક અવયવ: $\text{I.F.} = e^{\int 1 dx} = e^x$
ઉકેલ છે:
$y \cdot e^x = \int e^x \cdot e^x dx = \int e^{2x} dx$

$ye^x = \frac{e^{2x}}{2} + C$

$y = \frac{e^x}{2} + Ce^{-x}$

પ્રારંભિક શરત $y(0) = 1$ નો ઉપયોગ કરતા:
$1 = \frac{e^0}{2} + Ce^0 = \frac{1}{2} + C$

$C = 1 - \frac{1}{2} = \frac{1}{2}$

તેથી: $y = \frac{e^x}{2} + \frac{1}{2}e^{-x} = \frac{1}{2}(e^x + e^{-x})$
\end{solutionbox}

\section*{સૂત્રો}

\subsection*{શ્રેણિક પ્રક્રિયાઓ}
\begin{itemize}
\item \textbf{શ્રેણિક ગુણાકાર}: $(AB)_{ij} = \sum_{k} A_{ik}B_{kj}$
\item \textbf{પરિવર્ત શ્રેણિક}: $(A^T)_{ij} = A_{ji}$
\item \textbf{વ્યસ્ત શ્રેણિક}: $A^{-1} = \frac{1}{|A|} \text{adj}(A)$
\item \textbf{નિશ્ચાયક 2×2}: $\begin{vmatrix} a & b \\ c & d \end{vmatrix} = ad - bc$
\end{itemize}

\subsection*{વિકલન}
\begin{itemize}
\item \textbf{મૂળભૂત નિયમો}: $\frac{d}{dx}(x^n) = nx^{n-1}$, $\frac{d}{dx}(e^x) = e^x$, $\frac{d}{dx}(\ln x) = \frac{1}{x}$
\item \textbf{સાંકળ નિયમ}: $\frac{d}{dx}[f(g(x))] = f'(g(x)) \cdot g'(x)$
\item \textbf{ગુણાકારનો નિયમ}: $\frac{d}{dx}[uv] = u'v + uv'$
\item \textbf{ગૂઢ વિધેયનું વિકલન}: બંને બાજુ વિકલન કરો, $y$ ને $x$ ના વિધેય તરીકે લો
\end{itemize}

\subsection*{સંકલન}
\begin{itemize}
\item \textbf{મૂળભૂત સંકલિતો}: $\int x^n dx = \frac{x^{n+1}}{n+1} + C$ (n $\neq$ -1)
\item \textbf{ખંડશઃ સંકલન}: $\int u dv = uv - \int v du$
\item \textbf{નિયત સંકલન}: $\int_a^b f(x) dx = F(b) - F(a)$
\end{itemize}

\subsection*{વિકલ સમીકરણો}
\begin{itemize}
\item \textbf{સુરેખ વિકલ સમીકરણ}: $\frac{dy}{dx} + Py = Q$, ઉકેલ: $y \cdot \text{I.F.} = \int Q \cdot \text{I.F.} dx$
\item \textbf{સંકલ્યકારક અવયવ}: $\text{I.F.} = e^{\int P dx}$
\item \textbf{વિયોજનીય ચલ}: $\frac{dy}{dx} = f(x)g(y) \Rightarrow \frac{dy}{g(y)} = f(x) dx$
\end{itemize}

\subsection*{આંકડાશાસ્ત્ર}
\begin{itemize}
\item \textbf{મધ્યક}: $\bar{x} = \frac{\sum f_i x_i}{\sum f_i}$
\item \textbf{સરેરાશ વિચલન}: $\text{M.D.} = \frac{\sum |x_i - \bar{x}|}{n}$
\item \textbf{પ્રમાણિત વિચલન}: $\sigma = \sqrt{\frac{\sum (x_i - \bar{x})^2}{n}}$
\end{itemize}

\end{document}
