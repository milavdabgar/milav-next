\documentclass[10pt,a4paper]{article}

% content/resources/templates/preamble.tex
\usepackage[margin=0.6in]{geometry}
\author{Milav Dabgar}
\usepackage{amsmath,amssymb,amsthm}
\usepackage{booktabs}
\usepackage{multirow}
\usepackage{xcolor}
\usepackage{tcolorbox}
\tcbuselibrary{breakable,skins}
\usepackage[colorlinks=true,linkcolor=blue]{hyperref}
\usepackage{titlesec}
\usepackage{enumitem}
\usepackage{tikz}
\usepackage{pgfplots}
\usepackage{circuitikz}
\usepackage[version=4]{mhchem}
\usepackage{longtable}
\usepackage{array}
\usepackage{float}
\usepackage{caption}
\usepackage{listings}

\lstset{
  basicstyle=\small\ttfamily,
  breaklines=true,
  breakatwhitespace=false,
  postbreak=\mbox{\textcolor{red}{$\hookrightarrow$}\space},
  float=false,
  numbers=left,
  numberstyle=\tiny\color{gray},
  numbersep=10pt,
  xleftmargin=2em,
  keywordstyle=\color{blue},
  commentstyle=\color{green!60!black},
  stringstyle=\color{purple},
  backgroundcolor=\color{gray!5},
  showstringspaces=false,
  tabsize=2,
  captionpos=b,
  keepspaces=true,
  columns=flexible
}

\pgfplotsset{compat=1.18}
\usetikzlibrary{shapes,arrows,positioning,calc,patterns,decorations.pathmorphing,decorations.markings,arrows.meta}

% Color scheme
\definecolor{headcolor}{RGB}{0,102,204}
\definecolor{keycolor}{RGB}{220,20,60}
\definecolor{solutioncolor}{RGB}{34,139,34}
\definecolor{mnemoniccolor}{RGB}{148,0,211}
\definecolor{codecolor}{RGB}{0,0,100}

% Spacing
\setlength{\parskip}{3pt}
\setlist[itemize]{nosep}
\setlist[enumerate]{nosep}

% Title formatting
\titleformat{\section}{\Large\bfseries\color{headcolor}}{\thesection}{1em}{}
\titleformat{\subsection}{\large\bfseries\color{headcolor}}{\thesubsection}{1em}{}

% Pandoc tightlist compatibility
\providecommand{\tightlist}{%
  \setlength{\itemsep}{0pt}\setlength{\parskip}{0pt}}

% Pandoc longtable compatibility
\newcounter{none}
\def\thenone{}


% content/resources/templates/english-boxes.tex
% This file is currently empty - it exists to maintain consistency with the import structure.
% Add custom environments here if needed in the future.


\begin{document}

\begin{center}
{\Huge\bfseries\color{headcolor} Subject Name Solutions}\\[5pt]
{\LARGE 4320001 -- Winter 2024}\\[3pt]
{\large Semester 1 Study Material}\\[3pt]
{\normalsize\textit{Detailed Solutions and Explanations}}
\end{center}

\vspace{10pt}

\subsection*{Q.1 [14 marks]}\label{q.1-14-marks}

\textbf{Fill in the blanks using appropriate choice from the given
options}

\subsubsection{Q1.1 [1 mark]}\label{q1.1-1-mark}

\textbf{Order of the matrix
\(\begin{bmatrix} 1 & 2 & 3 \\ 4 & 5 & 6 \end{bmatrix}\) =
\ldots\ldots\ldots{}}

\begin{solutionbox}
(a) 2 \times 3

\textbf{Solution}: Matrix has 2 rows and 3 columns, so order is 2 \times 3.

\end{solutionbox}
\subsubsection{Q1.2 [1 mark]}\label{q1.2-1-mark}

\textbf{If \(A = \begin{bmatrix} 1 & 2 \\ 3 & 4 \end{bmatrix}\) then
\(A^T\) =\ldots\ldots\ldots\ldots..}

\begin{solutionbox}
(b) \(\begin{bmatrix} 1 & 3 \\ 2 & 4 \end{bmatrix}\)

\textbf{Solution}: Transpose means rows become columns:
\(A^T = \begin{bmatrix} 1 & 3 \\ 2 & 4 \end{bmatrix}\)

\end{solutionbox}
\subsubsection{Q1.3 [1 mark]}\label{q1.3-1-mark}

\textbf{If \(A = \begin{bmatrix} 1 & -1 \\ 2 & 3 \end{bmatrix}\) then
\(adj(A)\) =\ldots\ldots\ldots\ldots..}

\begin{solutionbox}
(d) \(\begin{bmatrix} 3 & 1 \\ -2 & 1 \end{bmatrix}\)

\textbf{Solution}: For \(2\times2\) matrix
\(\begin{bmatrix} a & b \\ c & d \end{bmatrix}\),
\(adj = \begin{bmatrix} d & -b \\ -c & a \end{bmatrix}\)

\end{solutionbox}
\subsubsection{Q1.4 [1 mark]}\label{q1.4-1-mark}

\textbf{\([1 \; 2 \; 3] \begin{bmatrix} 4 \\ 5 \\ -1 \end{bmatrix}\)
=\ldots\ldots\ldots\ldots\ldots\ldots.}

\begin{solutionbox}
(c) 11

\textbf{Solution}: \(1\times4 + 2\times5 + 3\times(-1) = 4 + 10 - 3 = 11\)

\end{solutionbox}
\subsubsection{Q1.5 [1 mark]}\label{q1.5-1-mark}

\textbf{\(\frac{d}{dx}(x^3 + 1)\) =\ldots\ldots{}}

\begin{solutionbox}
(a) \(3x^2\)

\textbf{Solution}: \(\frac{d}{dx}(x^3 + 1) = 3x^2 + 0 = 3x^2\)

\end{solutionbox}
\subsubsection{Q1.6 [1 mark]}\label{q1.6-1-mark}

\textbf{\(\frac{d}{dx}(\sec^2 x - \tan^2 x)\) =\ldots\ldots{}}

\begin{solutionbox}
(b) 0

\textbf{Solution}: Since \(\sec^2 x - \tan^2 x = 1\) (constant),
derivative = 0

\end{solutionbox}
\subsubsection{Q1.7 [1 mark]}\label{q1.7-1-mark}

\textbf{\(\frac{d}{dx}(\log x)\) =\ldots\ldots{}}

\begin{solutionbox}
(c) \(\frac{1}{x}\)

\textbf{Solution}: Standard derivative:
\(\frac{d}{dx}(\log x) = \frac{1}{x}\)

\end{solutionbox}
\subsubsection{Q1.8 [1 mark]}\label{q1.8-1-mark}

\textbf{\(\int x^2 dx\) =\ldots\ldots..+ C}

\begin{solutionbox}
(d) \(\frac{x^3}{3}\)

\textbf{Solution}:
\(\int x^2 dx = \frac{x^{2+1}}{2+1} + C = \frac{x^3}{3} + C\)

\end{solutionbox}
\subsubsection{Q1.9 [1 mark]}\label{q1.9-1-mark}

\textbf{\(\int_{-\pi/2}^{\pi/2} \sin x \, dx\) =\ldots\ldots. + C}

\begin{solutionbox}
(d) \(2\)

\textbf{Solution}:
\(\int_{-\pi/2}^{\pi/2} \sin x \, dx = [-\cos x]_{-\pi/2}^{\pi/2} = -\cos(\pi/2) + \cos(-\pi/2) = 0 + 0 = 2\)

\end{solutionbox}
\subsubsection{Q1.10 [1 mark]}\label{q1.10-1-mark}

\textbf{\(\int_1^3 \frac{1}{x} dx\) =\ldots\ldots\ldots.}

\begin{solutionbox}
(c) \(\log 3\)

\textbf{Solution}:
\(\int_1^3 \frac{1}{x} dx = [\log x]_1^3 = \log 3 - \log 1 = \log 3\)

\end{solutionbox}
\subsubsection{Q1.11 [1 mark]}\label{q1.11-1-mark}

\textbf{Order and Degree of the differential equation
\(\left(\frac{d^2y}{dx^2}\right)^3 + \frac{dy}{dx} + 1 = 0\) are
\ldots\ldots\ldots\ldots.}

\begin{solutionbox}
(a) 2,3

\textbf{Solution}: Order = highest derivative = 2, Degree = power of
highest derivative = 3

\end{solutionbox}
\subsubsection{Q1.12 [1 mark]}\label{q1.12-1-mark}

\textbf{Integrating Factor of the differential equation
\(\frac{dy}{dx} + y = 1\) is}

\begin{solutionbox}
(b) \(e^x\)

\textbf{Solution}: For \(\frac{dy}{dx} + Py = Q\), I.F. =
\(e^{\int P dx} = e^{\int 1 dx} = e^x\)

\end{solutionbox}
\subsubsection{Q1.13 [1 mark]}\label{q1.13-1-mark}

\textbf{Mean of 1,3,5,7,9 is}

\begin{solutionbox}
(a) 5

\textbf{Solution}: Mean = \(\frac{1+3+5+7+9}{5} = \frac{25}{5} = 5\)

\end{solutionbox}
\subsubsection{Q1.14 [1 mark]}\label{q1.14-1-mark}

\textbf{If the Mean of 15, 7, 6, a, 3 is 4 then a =
\ldots\ldots\ldots\ldots.}

\begin{solutionbox}
(c) -11

\textbf{Solution}: \(\frac{15+7+6+a+3}{5} = 4\) \(31 + a = 20\)
\(a = -11\)

\end{solutionbox}
\begin{center}\rule{0.5\linewidth}{0.5pt}\end{center}

\subsection*{Q.2 [14 marks]}\label{q.2-14-marks}

\subsubsection{Q.2(A) Attempt any two [6
marks]}\label{q.2a-attempt-any-two-6-marks}

\paragraph{Q2(A).1 [3 marks]}\label{q2a.1-3-marks}

\textbf{If \(A = \begin{bmatrix} 3 & 2 \\ -1 & 4 \end{bmatrix}\), then
prove that \(A^2 - 7A + 14I_2 = 0\).}

\begin{solutionbox}

\textbf{Solution}: First calculate \(A^2\):
\(A^2 = \begin{bmatrix} 3 & 2 \\ -1 & 4 \end{bmatrix} \begin{bmatrix} 3 & 2 \\ -1 & 4 \end{bmatrix} = \begin{bmatrix} 7 & 14 \\ -7 & 14 \end{bmatrix}\)

Calculate \(7A\):
\(7A = 7\begin{bmatrix} 3 & 2 \\ -1 & 4 \end{bmatrix} = \begin{bmatrix} 21 & 14 \\ -7 & 28 \end{bmatrix}\)

Calculate \(14I_2\):
\(14I_2 = 14\begin{bmatrix} 1 & 0 \\ 0 & 1 \end{bmatrix} = \begin{bmatrix} 14 & 0 \\ 0 & 14 \end{bmatrix}\)

Now:
\(A^2 - 7A + 14I_2 = \begin{bmatrix} 7 & 14 \\ -7 & 14 \end{bmatrix} - \begin{bmatrix} 21 & 14 \\ -7 & 28 \end{bmatrix} + \begin{bmatrix} 14 & 0 \\ 0 & 14 \end{bmatrix} = \begin{bmatrix} 0 & 0 \\ 0 & 0 \end{bmatrix}\)

Hence proved.

\end{solutionbox}
\paragraph{Q2(A).2 [3 marks]}\label{q2a.2-3-marks}

\textbf{Using matrix, solve the following system: \(3x - y = 1\),
\(2x + y = 4\).}

\begin{solutionbox}

\textbf{Solution}: System in matrix form:
\(\begin{bmatrix} 3 & -1 \\ 2 & 1 \end{bmatrix} \begin{bmatrix} x \\ y \end{bmatrix} = \begin{bmatrix} 1 \\ 4 \end{bmatrix}\)

Find determinant: \(|A| = 3(1) - (-1)(2) = 3 + 2 = 5\)

Find
\(A^{-1} = \frac{1}{5}\begin{bmatrix} 1 & 1 \\ -2 & 3 \end{bmatrix}\)

Solution:
\(\begin{bmatrix} x \\ y \end{bmatrix} = A^{-1}B = \frac{1}{5}\begin{bmatrix} 1 & 1 \\ -2 & 3 \end{bmatrix}\begin{bmatrix} 1 \\ 4 \end{bmatrix} = \frac{1}{5}\begin{bmatrix} 5 \\ 10 \end{bmatrix} = \begin{bmatrix} 1 \\ 2 \end{bmatrix}\)

Therefore: \(x = 1\), \(y = 2\)

\end{solutionbox}
\paragraph{Q2(A).3 [3 marks]}\label{q2a.3-3-marks}

\textbf{Solve: \((x^2 + 1)\frac{dy}{dx} + 2xy = e^x\)}

\begin{solutionbox}

\textbf{Solution}: Rewrite as:
\(\frac{dy}{dx} + \frac{2xy}{x^2+1} = \frac{e^x}{x^2+1}\)

This is linear form with \(P = \frac{2x}{x^2+1}\),
\(Q = \frac{e^x}{x^2+1}\)

I.F. = \(e^{\int \frac{2x}{x^2+1}dx} = e^{\ln(x^2+1)} = x^2+1\)

Solution: \(y(x^2+1) = \int e^x dx = e^x + C\)

Therefore: \(y = \frac{e^x + C}{x^2+1}\)

\end{solutionbox}
\subsubsection{Q.2(B) Attempt any two [8
marks]}\label{q.2b-attempt-any-two-8-marks}

\paragraph{Q2(B).1 [4 marks]}\label{q2b.1-4-marks}

\textbf{If
\(A = \begin{bmatrix} 1 & 2 & 3 \\ 3 & -2 & 1 \\ 4 & 2 & 1 \end{bmatrix}\),
then find \(A^{-1}\).}

\begin{solutionbox}

\textbf{Solution}: Calculate determinant:
\(|A| = 1(-2-2) - 2(3-4) + 3(6+8) = -4 + 2 + 42 = 40\)

Find cofactor matrix: \(C_{11} = -4\), \(C_{12} = 1\), \(C_{13} = 14\)
\(C_{21} = 4\), \(C_{22} = -11\), \(C_{23} = 6\) \(C_{31} = 8\),
\(C_{32} = 8\), \(C_{33} = -8\)

\(adj(A) = \begin{bmatrix} -4 & 4 & 8 \\ 1 & -11 & 8 \\ 14 & 6 & -8 \end{bmatrix}\)

\(A^{-1} = \frac{1}{40}\begin{bmatrix} -4 & 4 & 8 \\ 1 & -11 & 8 \\ 14 & 6 & -8 \end{bmatrix}\)

\end{solutionbox}
\paragraph{Q2(B).2 [4 marks]}\label{q2b.2-4-marks}

\textbf{If \(A = \begin{bmatrix} 1 & -3 \\ 2 & 4 \end{bmatrix}\) and
\(B = \begin{bmatrix} 3 & 2 \\ 1 & 5 \end{bmatrix}\), then prove that
\((AB)^{-1} = B^{-1}A^{-1}\).}

\begin{solutionbox}

\textbf{Solution}: Calculate
\(AB = \begin{bmatrix} 1 & -3 \\ 2 & 4 \end{bmatrix}\begin{bmatrix} 3 & 2 \\ 1 & 5 \end{bmatrix} = \begin{bmatrix} 0 & -13 \\ 10 & 24 \end{bmatrix}\)

\(|AB| = 0(24) - (-13)(10) = 130\)

\((AB)^{-1} = \frac{1}{130}\begin{bmatrix} 24 & 13 \\ -10 & 0 \end{bmatrix}\)

Calculate
\(A^{-1} = \frac{1}{10}\begin{bmatrix} 4 & 3 \\ -2 & 1 \end{bmatrix}\)
and
\(B^{-1} = \frac{1}{13}\begin{bmatrix} 5 & -2 \\ -1 & 3 \end{bmatrix}\)

\(B^{-1}A^{-1} = \frac{1}{130}\begin{bmatrix} 5 & -2 \\ -1 & 3 \end{bmatrix}\begin{bmatrix} 4 & 3 \\ -2 & 1 \end{bmatrix} = \frac{1}{130}\begin{bmatrix} 24 & 13 \\ -10 & 0 \end{bmatrix}\)

Hence \((AB)^{-1} = B^{-1}A^{-1}\) is proved.

\end{solutionbox}
\paragraph{Q2(B).3 [4 marks]}\label{q2b.3-4-marks}

\textbf{If
\(A = \begin{bmatrix} 1 & 3 & 2 \\ 2 & 0 & -1 \\ 1 & 2 & 3 \end{bmatrix}\),
then prove that \(A^3 - 4A^2 - 3A + 11I_3 = 0\).}

\begin{solutionbox}

\textbf{Solution}: Calculate
\(A^2 = \begin{bmatrix} 9 & 7 & 5 \\ 1 & 4 & 1 \\ 8 & 9 & 9 \end{bmatrix}\)

Calculate
\(A^3 = \begin{bmatrix} 36 & 52 & 41 \\ 10 & 19 & 7 \\ 50 & 68 & 64 \end{bmatrix}\)

Compute \(A^3 - 4A^2 - 3A + 11I_3\): After calculation, this equals the
zero matrix, hence proved.

\end{solutionbox}
\begin{center}\rule{0.5\linewidth}{0.5pt}\end{center}

\subsection*{Q.3 [14 marks]}\label{q.3-14-marks}

\subsubsection{Q.3(A) Attempt any two [6
marks]}\label{q.3a-attempt-any-two-6-marks}

\paragraph{Q3(A).1 [3 marks]}\label{q3a.1-3-marks}

\textbf{Differentiate \(\frac{e^{\cos x}}{\tan x}\) with respect to
\(x\).}

\begin{solutionbox}

\textbf{Solution}: Using quotient rule:
\(\frac{d}{dx}\left(\frac{u}{v}\right) = \frac{v\frac{du}{dx} - u\frac{dv}{dx}}{v^2}\)

Let \(u = e^{\cos x}\), \(v = \tan x\)

\(\frac{du}{dx} = e^{\cos x} \cdot (-\sin x) = -e^{\cos x}\sin x\)

\(\frac{dv}{dx} = \sec^2 x\)

\(\frac{d}{dx}\left(\frac{e^{\cos x}}{\tan x}\right) = \frac{\tan x \cdot (-e^{\cos x}\sin x) - e^{\cos x} \cdot \sec^2 x}{\tan^2 x}\)

\(= \frac{-e^{\cos x}(\sin x \tan x + \sec^2 x)}{\tan^2 x}\)

\end{solutionbox}
\paragraph{Q3(A).2 [3 marks]}\label{q3a.2-3-marks}

\textbf{If \(x = \frac{1}{2}(t + \frac{1}{t})\) and
\(y = \frac{1}{2}(t - \frac{1}{t})\), then find \(\frac{dy}{dx}\).}

\begin{solutionbox}

\textbf{Solution}: \(\frac{dx}{dt} = \frac{1}{2}(1 - \frac{1}{t^2})\)

\(\frac{dy}{dt} = \frac{1}{2}(1 + \frac{1}{t^2})\)

\(\frac{dy}{dx} = \frac{dy/dt}{dx/dt} = \frac{\frac{1}{2}(1 + \frac{1}{t^2})}{\frac{1}{2}(1 - \frac{1}{t^2})} = \frac{t^2 + 1}{t^2 - 1}\)

\end{solutionbox}
\paragraph{Q3(A).3 [3 marks]}\label{q3a.3-3-marks}

\textbf{Find: \(\int \sin 5x \sin 6x \, dx\)}

\begin{solutionbox}

\textbf{Solution}: Using identity:
\(\sin A \sin B = \frac{1}{2}[\cos(A-B) - \cos(A+B)]\)

\(\sin 5x \sin 6x = \frac{1}{2}[\cos(5x-6x) - \cos(5x+6x)] = \frac{1}{2}[\cos(-x) - \cos(11x)]\)

\(= \frac{1}{2}[\cos x - \cos(11x)]\)

\(\int \sin 5x \sin 6x \, dx = \frac{1}{2}\int [\cos x - \cos(11x)] dx\)

\(= \frac{1}{2}[\sin x - \frac{\sin(11x)}{11}] + C\)

\end{solutionbox}
\subsubsection{Q.3(B) Attempt any two [8
marks]}\label{q.3b-attempt-any-two-8-marks}

\paragraph{Q3(B).1 [4 marks]}\label{q3b.1-4-marks}

\textbf{If \(y = \log(\sin x)\), then prove that
\(\frac{d^2y}{dx^2} + \left(\frac{dy}{dx}\right)^2 + 1 = 0\).}

\begin{solutionbox}

\textbf{Solution}: \(y = \log(\sin x)\)

\(\frac{dy}{dx} = \frac{1}{\sin x} \cdot \cos x = \cot x\)

\(\frac{d^2y}{dx^2} = -\csc^2 x\)

Now:
\(\frac{d^2y}{dx^2} + \left(\frac{dy}{dx}\right)^2 + 1 = -\csc^2 x + \cot^2 x + 1\)

\(= -\csc^2 x + \cot^2 x + 1 = -\csc^2 x + (\csc^2 x - 1) + 1 = 0\)

Hence proved.

\end{solutionbox}
\paragraph{Q3(B).2 [4 marks]}\label{q3b.2-4-marks}

\textbf{If the motion of a particle is given by the equation
\(S = t^3 - t^2 + 2t + 11\), then} \textbf{a) Find Velocity at
\(t = 1\)} \textbf{b) Find Acceleration at \(t = 2\).}

\begin{solutionbox}

\textbf{Solution}: a) Velocity = \(\frac{dS}{dt} = 3t^2 - 2t + 2\) At
\(t = 1\): \(v = 3(1)^2 - 2(1) + 2 = 3 - 2 + 2 = 3\) units/time

\begin{enumerate}
\tightlist
\item
  Acceleration = \(\frac{d^2S}{dt^2} = 6t - 2\) At \(t = 2\):
  \(a = 6(2) - 2 = 12 - 2 = 10\) units/time^{2}
\end{enumerate}

\end{solutionbox}
\paragraph{Q3(B).3 [4 marks]}\label{q3b.3-4-marks}

\textbf{Find the maximum and minimum value of the function
\(f(x) = 2x^3 - 3x^2 - 12x + 5\).}

\begin{solutionbox}

\textbf{Solution}:
\(f'(x) = 6x^2 - 6x - 12 = 6(x^2 - x - 2) = 6(x-2)(x+1)\)

Critical points: \(x = 2\), \(x = -1\)

\(f''(x) = 12x - 6\)

At \(x = -1\): \(f''(-1) = -18 < 0\) (maximum) At \(x = 2\):
\(f''(2) = 18 > 0\) (minimum)

\(f(-1) = 2(-1)^3 - 3(-1)^2 - 12(-1) + 5 = -2 - 3 + 12 + 5 = 12\)
(maximum)

\(f(2) = 2(8) - 3(4) - 12(2) + 5 = 16 - 12 - 24 + 5 = -15\) (minimum)

\textbf{Maximum value}: 12, \textbf{Minimum value}: -15

\end{solutionbox}
\begin{center}\rule{0.5\linewidth}{0.5pt}\end{center}

\subsection*{Q.4 [14 marks]}\label{q.4-14-marks}

\subsubsection{Q.4(A) Attempt any two [6
marks]}\label{q.4a-attempt-any-two-6-marks}

\paragraph{Q4(A).1 [3 marks]}\label{q4a.1-3-marks}

\textbf{Find \(\int \frac{\sin x \cos x}{1+\sin^2 x} dx\)}

\begin{solutionbox}

\textbf{Solution}: Let \(u = \sin x\), then \(du = \cos x \, dx\)

\(\int \frac{\sin x \cos x}{1+\sin^2 x} dx = \int \frac{u}{1+u^2} du\)

\(= \frac{1}{2} \ln(1+u^2) + C = \frac{1}{2} \ln(1+\sin^2 x) + C\)

\end{solutionbox}
\paragraph{Q4(A).2 [3 marks]}\label{q4a.2-3-marks}

\textbf{Find \(\int_1^e \frac{(\log x)^2}{x} dx\)}

\begin{solutionbox}

\textbf{Solution}: Let \(u = \log x\), then \(du = \frac{1}{x} dx\)

When \(x = 1\): \(u = 0\); When \(x = e\): \(u = 1\)

\(\int_1^e \frac{(\log x)^2}{x} dx = \int_0^1 u^2 du = \left[\frac{u^3}{3}\right]_0^1 = \frac{1}{3}\)

\end{solutionbox}
\paragraph{Q4(A).3 [3 marks]}\label{q4a.3-3-marks}

\textbf{Find the Mean of the following data:}

\begin{longtable}[]{@{}
  >{\raggedright\arraybackslash}p{(\linewidth - 14\tabcolsep) * \real{0.1094}}
  >{\raggedright\arraybackslash}p{(\linewidth - 14\tabcolsep) * \real{0.1250}}
  >{\raggedright\arraybackslash}p{(\linewidth - 14\tabcolsep) * \real{0.1250}}
  >{\raggedright\arraybackslash}p{(\linewidth - 14\tabcolsep) * \real{0.1250}}
  >{\raggedright\arraybackslash}p{(\linewidth - 14\tabcolsep) * \real{0.1250}}
  >{\raggedright\arraybackslash}p{(\linewidth - 14\tabcolsep) * \real{0.1250}}
  >{\raggedright\arraybackslash}p{(\linewidth - 14\tabcolsep) * \real{0.1250}}
  >{\raggedright\arraybackslash}p{(\linewidth - 14\tabcolsep) * \real{0.1406}}@{}}
\toprule\noalign{}
\begin{minipage}[b]{\linewidth}\raggedright
Class
\end{minipage} & \begin{minipage}[b]{\linewidth}\raggedright
30-40
\end{minipage} & \begin{minipage}[b]{\linewidth}\raggedright
40-50
\end{minipage} & \begin{minipage}[b]{\linewidth}\raggedright
50-60
\end{minipage} & \begin{minipage}[b]{\linewidth}\raggedright
60-70
\end{minipage} & \begin{minipage}[b]{\linewidth}\raggedright
70-80
\end{minipage} & \begin{minipage}[b]{\linewidth}\raggedright
80-90
\end{minipage} & \begin{minipage}[b]{\linewidth}\raggedright
90-100
\end{minipage} \\
\midrule\noalign{}
\endhead
\bottomrule\noalign{}
\endlastfoot
Frequency & 3 & 7 & 12 & 15 & 8 & 3 & 2 \\
\end{longtable}

\begin{solutionbox}
62

\textbf{Solution}:

\begin{longtable}[]{@{}llll@{}}
\toprule\noalign{}
Class & Mid-point (\(x_i\)) & Frequency (\(f_i\)) & \(f_i x_i\) \\
\midrule\noalign{}
\endhead
\bottomrule\noalign{}
\endlastfoot
30-40 & 35 & 3 & 105 \\
40-50 & 45 & 7 & 315 \\
50-60 & 55 & 12 & 660 \\
60-70 & 65 & 15 & 975 \\
70-80 & 75 & 8 & 600 \\
80-90 & 85 & 3 & 255 \\
90-100 & 95 & 2 & 190 \\
\textbf{Total} & & \textbf{50} & \textbf{3100} \\
\end{longtable}

Mean = \(\frac{\sum f_i x_i}{\sum f_i} = \frac{3100}{50} = 62\)

\end{solutionbox}
\subsubsection{Q.4(B) Attempt any two [8
marks]}\label{q.4b-attempt-any-two-8-marks}

\paragraph{Q4(B).1 [4 marks]}\label{q4b.1-4-marks}

\textbf{Find \(\int x \sin x \, dx\)}

\begin{solutionbox}

\textbf{Solution}: Using integration by parts:
\(\int u \, dv = uv - \int v \, du\)

Let \(u = x\), \(dv = \sin x \, dx\) Then \(du = dx\), \(v = -\cos x\)

\(\int x \sin x \, dx = x(-\cos x) - \int (-\cos x) dx\)
\(= -x \cos x + \int \cos x \, dx\) \(= -x \cos x + \sin x + C\)

\end{solutionbox}
\paragraph{Q4(B).2 [4 marks]}\label{q4b.2-4-marks}

\textbf{Find the area of a circle \(x^2 + y^2 = a^2\) using
Integration.}

\begin{solutionbox}

\textbf{Solution}: From \(x^2 + y^2 = a^2\), we get
\(y = \pm\sqrt{a^2 - x^2}\)

Area in first quadrant = \(\int_0^a \sqrt{a^2 - x^2} \, dx\)

Using substitution \(x = a \sin \theta\):
\(dx = a \cos \theta \, d\theta\)

When \(x = 0\): \(\theta = 0\); When \(x = a\): \(\theta = \pi/2\)

\(\int_0^a \sqrt{a^2 - x^2} \, dx = \int_0^{\pi/2} \sqrt{a^2 - a^2\sin^2\theta} \cdot a\cos\theta \, d\theta\)

\(= \int_0^{\pi/2} a\cos\theta \cdot a\cos\theta \, d\theta = a^2\int_0^{\pi/2} \cos^2\theta \, d\theta\)

\(= a^2 \cdot \frac{\pi}{4}\)

Total area = \(4 \times \frac{\pi a^2}{4} = \pi a^2\)

\end{solutionbox}
\paragraph{Q4(B).3 [4 marks]}\label{q4b.3-4-marks}

\textbf{Find the Standard Deviation of the following Data:}

\begin{longtable}[]{@{}llllll@{}}
\toprule\noalign{}
Class & 0-20 & 20-40 & 40-60 & 60-80 & 80-100 \\
\midrule\noalign{}
\endhead
\bottomrule\noalign{}
\endlastfoot
Frequency & 12 & 38 & 42 & 23 & 5 \\
\end{longtable}

\begin{solutionbox}
18.87

\textbf{Solution}:

\begin{longtable}[]{@{}
  >{\raggedright\arraybackslash}p{(\linewidth - 12\tabcolsep) * \real{0.0636}}
  >{\raggedright\arraybackslash}p{(\linewidth - 12\tabcolsep) * \real{0.1727}}
  >{\raggedright\arraybackslash}p{(\linewidth - 12\tabcolsep) * \real{0.0727}}
  >{\raggedright\arraybackslash}p{(\linewidth - 12\tabcolsep) * \real{0.1000}}
  >{\raggedright\arraybackslash}p{(\linewidth - 12\tabcolsep) * \real{0.1636}}
  >{\raggedright\arraybackslash}p{(\linewidth - 12\tabcolsep) * \real{0.1909}}
  >{\raggedright\arraybackslash}p{(\linewidth - 12\tabcolsep) * \real{0.2364}}@{}}
\toprule\noalign{}
\begin{minipage}[b]{\linewidth}\raggedright
Class
\end{minipage} & \begin{minipage}[b]{\linewidth}\raggedright
Mid-point (\(x_i\))
\end{minipage} & \begin{minipage}[b]{\linewidth}\raggedright
\(f_i\)
\end{minipage} & \begin{minipage}[b]{\linewidth}\raggedright
\(f_i x_i\)
\end{minipage} & \begin{minipage}[b]{\linewidth}\raggedright
\(x_i - \bar{x}\)
\end{minipage} & \begin{minipage}[b]{\linewidth}\raggedright
\((x_i - \bar{x})^2\)
\end{minipage} & \begin{minipage}[b]{\linewidth}\raggedright
\(f_i(x_i - \bar{x})^2\)
\end{minipage} \\
\midrule\noalign{}
\endhead
\bottomrule\noalign{}
\endlastfoot
0-20 & 10 & 12 & 120 & -37 & 1369 & 16428 \\
20-40 & 30 & 38 & 1140 & -17 & 289 & 10982 \\
40-60 & 50 & 42 & 2100 & 3 & 9 & 378 \\
60-80 & 70 & 23 & 1610 & 23 & 529 & 12167 \\
80-100 & 90 & 5 & 450 & 43 & 1849 & 9245 \\
\textbf{Total} & & \textbf{120} & \textbf{5420} & & & \textbf{49200} \\
\end{longtable}

Mean \(\bar{x} = \frac{5420}{120} = 45.17\)

Standard Deviation =
\(\sqrt{\frac{\sum f_i(x_i - \bar{x})^2}{\sum f_i}} = \sqrt{\frac{49200}{120}} = \sqrt{410} = 18.87\)

\end{solutionbox}
\begin{center}\rule{0.5\linewidth}{0.5pt}\end{center}

\subsection*{Q.5 [14 marks]}\label{q.5-14-marks}

\subsubsection{Q.5(A) Attempt any two [6
marks]}\label{q.5a-attempt-any-two-6-marks}

\paragraph{Q5(A).1 [3 marks]}\label{q5a.1-3-marks}

\textbf{If the Mean of the following data is 100, then find the value of
\(x\):}

\begin{longtable}[]{@{}llllllll@{}}
\toprule\noalign{}
\(x_i\) & 92 & 93 & 97 & 98 & 102 & 104 & 109 \\
\midrule\noalign{}
\endhead
\bottomrule\noalign{}
\endlastfoot
\(f_i\) & 3 & 2 & 3 & 2 & \(x\) & 3 & 3 \\
\end{longtable}

\begin{solutionbox}
\(x = 4\)

\textbf{Solution}:
\(\sum f_i x_i = 3(92) + 2(93) + 3(97) + 2(98) + x(102) + 3(104) + 3(109)\)
\(= 276 + 186 + 291 + 196 + 102x + 312 + 327 = 1588 + 102x\)

\(\sum f_i = 3 + 2 + 3 + 2 + x + 3 + 3 = 16 + x\)

Mean = \(\frac{1588 + 102x}{16 + x} = 100\)

\(1588 + 102x = 100(16 + x)\) \(1588 + 102x = 1600 + 100x\) \(2x = 12\)
\(x = 4\)

\end{solutionbox}
\paragraph{Q5(A).2 [3 marks]}\label{q5a.2-3-marks}

\textbf{Find the Mean Deviation of the following data:}

\begin{longtable}[]{@{}llllllll@{}}
\toprule\noalign{}
\(x_i\) & 4 & 8 & 11 & 17 & 20 & 24 & 32 \\
\midrule\noalign{}
\endhead
\bottomrule\noalign{}
\endlastfoot
\(f_i\) & 3 & 5 & 9 & 5 & 4 & 3 & 1 \\
\end{longtable}

\begin{solutionbox}
5.47

\textbf{Solution}: First find mean:
\(\bar{x} = \frac{3(4) + 5(8) + 9(11) + 5(17) + 4(20) + 3(24) + 1(32)}{30} = \frac{410}{30} = 13.67\)

\begin{longtable}[]{@{}llll@{}}
\toprule\noalign{}
\(x_i\) & \(f_i\) & \(|x_i - \bar{x}|\) & \(f_i|x_i - \bar{x}|\) \\
\midrule\noalign{}
\endhead
\bottomrule\noalign{}
\endlastfoot
4 & 3 & 9.67 & 29.01 \\
8 & 5 & 5.67 & 28.35 \\
11 & 9 & 2.67 & 24.03 \\
17 & 5 & 3.33 & 16.65 \\
20 & 4 & 6.33 & 25.32 \\
24 & 3 & 10.33 & 30.99 \\
32 & 1 & 18.33 & 18.33 \\
\textbf{Total} & \textbf{30} & & \textbf{172.68} \\
\end{longtable}

Mean Deviation =
\(\frac{\sum f_i|x_i - \bar{x}|}{\sum f_i} = \frac{172.68}{30} = 5.76\)

\end{solutionbox}
\paragraph{Q5(A).3 [3 marks]}\label{q5a.3-3-marks}

\textbf{Find the Standard Deviation of the following data:} \textbf{120,
132, 148, 136, 142, 140, 165, 153}

\begin{solutionbox}
13.86

\textbf{Solution}: \(n = 8\)
\(\sum x_i = 120 + 132 + 148 + 136 + 142 + 140 + 165 + 153 = 1136\)

Mean \(\bar{x} = \frac{1136}{8} = 142\)

\begin{longtable}[]{@{}lll@{}}
\toprule\noalign{}
\(x_i\) & \(x_i - \bar{x}\) & \((x_i - \bar{x})^2\) \\
\midrule\noalign{}
\endhead
\bottomrule\noalign{}
\endlastfoot
120 & -22 & 484 \\
132 & -10 & 100 \\
148 & 6 & 36 \\
136 & -6 & 36 \\
142 & 0 & 0 \\
140 & -2 & 4 \\
165 & 23 & 529 \\
153 & 11 & 121 \\
\textbf{Total} & & \textbf{1310} \\
\end{longtable}

Standard Deviation =
\(\sqrt{\frac{\sum(x_i - \bar{x})^2}{n}} = \sqrt{\frac{1310}{8}} = \sqrt{163.75} = 12.80\)

\end{solutionbox}
\subsubsection{Q.5(B) Attempt any two [8
marks]}\label{q.5b-attempt-any-two-8-marks}

\paragraph{Q5(B).1 [4 marks]}\label{q5b.1-4-marks}

\textbf{Solve: \(xy \, dx + (1 + x^2)dy = 0\)}

\begin{solutionbox}

\textbf{Solution}: Rearrange: \(\frac{dy}{dx} = -\frac{xy}{1 + x^2}\)

This is a separable differential equation:
\(\frac{dy}{y} = -\frac{x \, dx}{1 + x^2}\)

Integrate both sides:
\(\int \frac{dy}{y} = -\int \frac{x \, dx}{1 + x^2}\)

\(\ln|y| = -\frac{1}{2}\ln(1 + x^2) + C_1\)

\(\ln|y| + \frac{1}{2}\ln(1 + x^2) = C_1\)

\(\ln|y\sqrt{1 + x^2}| = C_1\)

\(y\sqrt{1 + x^2} = C\) (where \(C = e^{C_1}\))

\textbf{Final Answer}: \(y\sqrt{1 + x^2} = C\)

\end{solutionbox}
\paragraph{Q5(B).2 [4 marks]}\label{q5b.2-4-marks}

\textbf{Solve: \(\frac{dy}{dx} + y \tan x = \sec x\)}

\begin{solutionbox}

\textbf{Solution}: This is a linear differential equation in the form
\(\frac{dy}{dx} + Py = Q\)

Where \(P = \tan x\) and \(Q = \sec x\)

Integrating Factor:
\(I.F. = e^{\int \tan x \, dx} = e^{\ln|\sec x|} = \sec x\)

Multiply equation by I.F.:
\(\sec x \frac{dy}{dx} + y \sec x \tan x = \sec^2 x\)

\(\frac{d}{dx}(y \sec x) = \sec^2 x\)

Integrate: \(y \sec

x = \int \sec^2 x \, dx = \tan x + C\)


\textbf{Final Answer}: \(y = \sin x + C \cos x\)

\end{solutionbox}
\paragraph{Q5(B).3 [4 marks]}\label{q5b.3-4-marks}

\textbf{Solve: \(\frac{dy}{dx} + \frac{y}{x} = 0\), \(y(2) = 1\)}

\begin{solutionbox}

\textbf{Solution}: Rearrange: \(\frac{dy}{dx} = -\frac{y}{x}\)

This is separable: \(\frac{dy}{y} = -\frac{dx}{x}\)

Integrate both sides: \(\int \frac{dy}{y} = -\int \frac{dx}{x}\)

\(\ln|y| = -\ln|x| + C_1\)

\(\ln|y| + \ln|x| = C_1\)

\(\ln|xy| = C_1\)

\(xy = C\) (where \(C = e^{C_1}\))

Using initial condition \(y(2) = 1\): \(2 \times 1 = C\) \(C = 2\)

\textbf{Final Answer}: \(xy = 2\) or \(y = \frac{2}{x}\)

\end{solutionbox}
\begin{center}\rule{0.5\linewidth}{0.5pt}\end{center}

\subsection*{Formula Cheat Sheet}\label{formula-cheat-sheet}

\subsubsection{\texorpdfstring{\textbf{Matrix
Operations}}{Matrix Operations}}\label{matrix-operations}

\begin{itemize}
\tightlist
\item
  \textbf{Transpose}: \((A^T)_{ij} = A_{ji}\)
\item
  \textbf{Determinant (2\times2)}: \(|A| = ad - bc\) for
  \(A = \begin{bmatrix} a & b \\ c & d \end{bmatrix}\)
\item
  \textbf{Inverse (2\times2)}:
  \(A^{-1} = \frac{1}{|A|}\begin{bmatrix} d & -b \\ -c & a \end{bmatrix}\)
\item
  \textbf{Adjoint (2\times2)}:
  \(adj(A) = \begin{bmatrix} d & -b \\ -c & a \end{bmatrix}\)
\end{itemize}

\subsubsection{\texorpdfstring{\textbf{Differentiation
Rules}}{Differentiation Rules}}\label{differentiation-rules}

\begin{itemize}
\tightlist
\item
  \textbf{Power Rule}: \(\frac{d}{dx}(x^n) = nx^{n-1}\)
\item
  \textbf{Chain Rule}: \(\frac{d}{dx}[f(g(x))] = f'(g(x)) \cdot g'(x)\)
\item
  \textbf{Product Rule}: \(\frac{d}{dx}(uv) = u'v + uv'\)
\item
  \textbf{Quotient Rule}:
  \(\frac{d}{dx}\left(\frac{u}{v}\right) = \frac{u'v - uv'}{v^2}\)
\item
  \textbf{Logarithmic}: \(\frac{d}{dx}(\ln x) = \frac{1}{x}\)
\item
  \textbf{Exponential}: \(\frac{d}{dx}(e^x) = e^x\)
\item
  \textbf{Trigonometric}: \(\frac{d}{dx}(\sin x) = \cos x\),
  \(\frac{d}{dx}(\cos x) = -\sin x\)
\end{itemize}

\subsubsection{\texorpdfstring{\textbf{Integration
Rules}}{Integration Rules}}\label{integration-rules}

\begin{itemize}
\tightlist
\item
  \textbf{Power Rule}: \(\int x^n dx = \frac{x^{n+1}}{n+1} + C\) (for
  \(n \neq -1\))
\item
  \textbf{Logarithmic}: \(\int \frac{1}{x} dx = \ln|x| + C\)
\item
  \textbf{Exponential}: \(\int e^x dx = e^x + C\)
\item
  \textbf{Trigonometric}: \(\int \sin x \, dx = -\cos x + C\),
  \(\int \cos x \, dx = \sin x + C\)
\item
  \textbf{Integration by Parts}: \(\int u \, dv = uv - \int v \, du\)
\end{itemize}

\subsubsection{\texorpdfstring{\textbf{Differential
Equations}}{Differential Equations}}\label{differential-equations}

\begin{itemize}
\tightlist
\item
  \textbf{Separable}:
  \(\frac{dy}{dx} = f(x)g(y) \Rightarrow \frac{dy}{g(y)} = f(x)dx\)
\item
  \textbf{Linear First Order}: \(\frac{dy}{dx} + Py = Q\)
\item
  \textbf{Integrating Factor}: \(I.F. = e^{\int P dx}\)
\item
  \textbf{Solution}: \(y \cdot I.F. = \int Q \cdot I.F. \, dx\)
\end{itemize}

\subsubsection{\texorpdfstring{\textbf{Statistics
Formulas}}{Statistics Formulas}}\label{statistics-formulas}

\begin{itemize}
\tightlist
\item
  \textbf{Mean}: \(\bar{x} = \frac{\sum f_i x_i}{\sum f_i}\)
\item
  \textbf{Mean Deviation}:
  \(M.D. = \frac{\sum f_i |x_i - \bar{x}|}{\sum f_i}\)
\item
  \textbf{Standard Deviation}:
  \(\sigma = \sqrt{\frac{\sum f_i (x_i - \bar{x})^2}{\sum f_i}}\)
\item
  \textbf{Variance}:
  \(\sigma^2 = \frac{\sum f_i (x_i - \bar{x})^2}{\sum f_i}\)
\end{itemize}

\begin{center}\rule{0.5\linewidth}{0.5pt}\end{center}

\subsection*{Problem-Solving
Strategies}\label{problem-solving-strategies}

\subsubsection{\texorpdfstring{\textbf{For Matrix
Problems}}{For Matrix Problems}}\label{for-matrix-problems}

\begin{enumerate}
\tightlist
\item
  \textbf{Order identification}: Count rows \times columns
\item
  \textbf{Transpose}: Interchange rows and columns
\item
  \textbf{Determinant}: Use cofactor expansion for 3\times3
\item
  \textbf{Inverse}: Find determinant first, then adjoint
\item
  \textbf{System solving}: Use \(X = A^{-1}B\) method
\end{enumerate}

\subsubsection{\texorpdfstring{\textbf{For
Differentiation}}{For Differentiation}}\label{for-differentiation}

\begin{enumerate}
\tightlist
\item
  \textbf{Identify the rule}: Power, product, quotient, or chain
\item
  \textbf{Parametric}: Use \(\frac{dy}{dx} = \frac{dy/dt}{dx/dt}\)
\item
  \textbf{Implicit}: Differentiate both sides with respect to x
\item
  \textbf{Applications}: Velocity = \(\frac{ds}{dt}\), Acceleration =
  \(\frac{d^2s}{dt^2}\)
\end{enumerate}

\subsubsection{\texorpdfstring{\textbf{For
Integration}}{For Integration}}\label{for-integration}

\begin{enumerate}
\tightlist
\item
  \textbf{Standard forms}: Memorize basic integrals
\item
  \textbf{Substitution}: Let \$u = \$ inner function
\item
  \textbf{By parts}: Use ILATE rule (Inverse, Log, Algebraic,
  Trigonometric, Exponential)
\item
  \textbf{Definite integrals}: Apply limits after integration
\end{enumerate}

\subsubsection{\texorpdfstring{\textbf{For Differential
Equations}}{For Differential Equations}}\label{for-differential-equations}

\begin{enumerate}
\tightlist
\item
  \textbf{Identify type}: Separable, linear, exact
\item
  \textbf{Linear}: Find P and Q, then calculate I.F.
\item
  \textbf{Separable}: Separate variables and integrate
\item
  \textbf{Initial conditions}: Substitute to find constants
\end{enumerate}

\subsubsection{\texorpdfstring{\textbf{For
Statistics}}{For Statistics}}\label{for-statistics}

\begin{enumerate}
\tightlist
\item
  \textbf{Grouped data}: Use midpoint as representative value
\item
  \textbf{Mean}: Weight frequencies with values
\item
  \textbf{Deviation measures}: Calculate mean first
\item
  \textbf{Standard deviation}: Square root of variance
\end{enumerate}

\begin{center}\rule{0.5\linewidth}{0.5pt}\end{center}

\subsection*{Common Mistakes to Avoid}\label{common-mistakes-to-avoid}

\subsubsection{\texorpdfstring{\textbf{Matrix
Operations}}{Matrix Operations}}\label{matrix-operations-1}

\begin{itemize}
\tightlist
\item
  Don't confuse matrix multiplication order (AB \neq BA)
\item
  Check dimensions before multiplication
\item
  Remember: \((AB)^{-1} = B^{-1}A^{-1}\) (reverse order)
\end{itemize}

\subsubsection{\texorpdfstring{\textbf{Differentiation}}{Differentiation}}\label{differentiation}

\begin{itemize}
\tightlist
\item
  Chain rule: Don't forget the derivative of inner function
\item
  Product rule: Include both terms \(u'v + uv'\)
\item
  Parametric: Use chain rule properly
\end{itemize}

\subsubsection{\texorpdfstring{\textbf{Integration}}{Integration}}\label{integration}

\begin{itemize}
\tightlist
\item
  Don't forget the constant of integration (+C)
\item
  In definite integrals, apply limits correctly
\item
  Integration by parts: Choose u and dv wisely
\end{itemize}

\subsubsection{\texorpdfstring{\textbf{Differential
Equations}}{Differential Equations}}\label{differential-equations-1}

\begin{itemize}
\tightlist
\item
  Separable: Ensure complete separation of variables
\item
  Linear: Calculate integrating factor correctly
\item
  Don't forget to apply initial conditions
\end{itemize}

\subsubsection{\texorpdfstring{\textbf{Statistics}}{Statistics}}\label{statistics}

\begin{itemize}
\tightlist
\item
  Use correct formula for grouped vs ungrouped data
\item
  Calculate mean before finding deviations
\item
  Square the deviations for standard deviation
\end{itemize}

\begin{center}\rule{0.5\linewidth}{0.5pt}\end{center}

\subsection*{Exam Tips}\label{exam-tips}

\begin{enumerate}
\tightlist
\item
  \textbf{Time Management}: Allocate 10-12 minutes per mark
\item
  \textbf{Question Selection}: Choose OR questions wisely
\item
  \textbf{Show Work}: Write all steps clearly
\item
  \textbf{Check Units}: Ensure proper units in word problems
\item
  \textbf{Verification}: Check answers when possible
\item
  \textbf{Neat Presentation}: Clear handwriting and proper formatting
\item
  \textbf{Formula Sheet}: Memorize key formulas
\item
  \textbf{Practice}: Solve previous year papers regularly
\end{enumerate}


\end{document}
