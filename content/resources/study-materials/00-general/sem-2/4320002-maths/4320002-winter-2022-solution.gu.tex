\documentclass{article}

% content/resources/templates/preamble.tex
\usepackage[margin=0.6in]{geometry}
\author{Milav Dabgar}
\usepackage{amsmath,amssymb,amsthm}
\usepackage{booktabs}
\usepackage{multirow}
\usepackage{xcolor}
\usepackage{tcolorbox}
\tcbuselibrary{breakable,skins}
\usepackage[colorlinks=true,linkcolor=blue]{hyperref}
\usepackage{titlesec}
\usepackage{enumitem}
\usepackage{tikz}
\usepackage{pgfplots}
\usepackage{circuitikz}
\usepackage[version=4]{mhchem}
\usepackage{longtable}
\usepackage{array}
\usepackage{float}
\usepackage{caption}
\usepackage{listings}

\lstset{
  basicstyle=\small\ttfamily,
  breaklines=true,
  breakatwhitespace=false,
  postbreak=\mbox{\textcolor{red}{$\hookrightarrow$}\space},
  float=false,
  numbers=left,
  numberstyle=\tiny\color{gray},
  numbersep=10pt,
  xleftmargin=2em,
  keywordstyle=\color{blue},
  commentstyle=\color{green!60!black},
  stringstyle=\color{purple},
  backgroundcolor=\color{gray!5},
  showstringspaces=false,
  tabsize=2,
  captionpos=b,
  keepspaces=true,
  columns=flexible
}

\pgfplotsset{compat=1.18}
\usetikzlibrary{shapes,arrows,positioning,calc,patterns,decorations.pathmorphing,decorations.markings,arrows.meta}

% Color scheme
\definecolor{headcolor}{RGB}{0,102,204}
\definecolor{keycolor}{RGB}{220,20,60}
\definecolor{solutioncolor}{RGB}{34,139,34}
\definecolor{mnemoniccolor}{RGB}{148,0,211}
\definecolor{codecolor}{RGB}{0,0,100}

% Spacing
\setlength{\parskip}{3pt}
\setlist[itemize]{nosep}
\setlist[enumerate]{nosep}

% Title formatting
\titleformat{\section}{\Large\bfseries\color{headcolor}}{\thesection}{1em}{}
\titleformat{\subsection}{\large\bfseries\color{headcolor}}{\thesubsection}{1em}{}

% Pandoc tightlist compatibility
\providecommand{\tightlist}{%
  \setlength{\itemsep}{0pt}\setlength{\parskip}{0pt}}

% Pandoc longtable compatibility
\newcounter{none}
\def\thenone{}


% content/resources/templates/gujarati-boxes.tex
\usepackage{fontspec}
\usepackage{polyglossia}

% Set Gujarati as main language (document is primarily in Gujarati)
% Note: gloss-gujarati.ldf doesn't exist in polyglossia, but it will use hyphenation patterns
\setdefaultlanguage{gujarati}
\setotherlanguage{english}

% Configure Gujarati font properly
% Use Language=Default to prevent polyglossia from trying to add language-specific features
% that don't exist for Gujarati, which causes "empty feature" warnings
\newfontfamily\gujaratifont[Script=Gujarati,AutoFakeBold=2.5,AutoFakeSlant=0.3]{Noto Sans Gujarati}
\setmainfont[Script=Gujarati,AutoFakeBold=2.5,AutoFakeSlant=0.3]{Noto Sans Gujarati}
% Use Noto Sans Gujarati for monospace to support Gujarati in text
\setmonofont[Scale=0.9]{Noto Sans Gujarati}

% Configure English to use the same font
\newfontfamily\englishfont[Script=Gujarati,AutoFakeBold=2.5,AutoFakeSlant=0.3]{Noto Sans Gujarati}

% Translations for polyglossia
\gappto\captionsgujarati{
  \renewcommand{\tablename}{કોષ્ટક}
  \renewcommand{\figurename}{આકૃતિ}
}

% Helper for TikZ nodes to ensure Gujarati font
\newcommand{\gu}[1]{{\gujaratifont #1}}

% Custom environments
\newtcolorbox{solutionbox}{
    breakable,
    enhanced,
    colback=solutioncolor!5!white,
    colframe=solutioncolor!75!black,
    fonttitle=\bfseries,
    title=જવાબ
}

\newtcolorbox{solutionboxnobreak}{
 colback=solutioncolor!5!white,
 colframe=solutioncolor!75!black,
 fonttitle=\bfseries,
 title=જવાબ
}

\newtcolorbox{keyformula}{
 breakable,
 enhanced,
 colback=keycolor!5!white,
 colframe=keycolor!75!black,
 fonttitle=\bfseries,
 title=રાસાયણિક સમીકરણ/સૂત્ર
}

\newtcolorbox{mnemonicbox}{
 breakable,
 enhanced,
 colback=mnemoniccolor!5!white,
 colframe=mnemoniccolor!75!black,
 fonttitle=\bfseries,
 title=મેમરી ટ્રીક
}


% Custom commands for GTU solutions
% This file defines semantic commands for consistent formatting

% Question command with automatic formatting
\newcommand{\question}[2]{%
  \section*{Question #1}%
  \textbf{#2}%
}

% OR question variant
\newcommand{\questionor}[2]{%
  \section*{Question #1 OR}%
  \textbf{#2}%
}

% Proper table environment with caption
\newenvironment{answertable}[1]{%
  \begin{table}[htbp]
  \centering
  \caption{#1}
}{%
  \end{table}
}

% Proper figure environment for diagrams
\newenvironment{answerdiagram}[1]{%
  \begin{figure}[htbp]
  \centering
  \caption{#1}
}{%
  \end{figure}
}

% Semantic markup for key terms
\newcommand{\keyword}[1]{\textbf{#1}}
\newcommand{\code}[1]{\texttt{#1}}
\newcommand{\classname}[1]{\texttt{#1}}
\newcommand{\methodname}[1]{\texttt{#1}}

% Proper quotation marks
\newcommand{\mnemonic}[1]{``#1''}


\title{એન્જિનિયરિંગ મેથેમેટિક્સ (4320002) - શિયાળુ 2022 સોલ્યુશન}
\date{માર્ચ 09, 2022}

\begin{document}
\maketitle

\questionmarks{1}{14}{નીચે આપેલા વિકલ્પોમાંથી યોગ્ય વિકલ્પ પસંદ કરી ખાલી જગ્યા પૂરો.}

\questionmarks{1.1}{1}{જો A = $\begin{bmatrix} 1 & -2 \\ 2 & -1 \end{bmatrix}$ હોય તો adj.A = \_\_\_\_\_\_.}

\begin{solutionbox}
\textbf{જવાબ}: (d) $\begin{bmatrix} -1 & -2 \\ -2 & 1 \end{bmatrix}$

\textbf{ઉકેલ}:
2×2 શ્રેણિક $A = \begin{bmatrix} a & b \\ c & d \end{bmatrix}$ માટે, adj.A = $\begin{bmatrix} d & -b \\ -c & a \end{bmatrix}$

$adj.A = \begin{bmatrix} -1 & 2 \\ -2 & 1 \end{bmatrix}$
\end{solutionbox}

\questionmarks{1.2}{1}{જો A એ 2×3 અને B એ 3×4 શ્રેણિકો હોય તો AB એ \_\_\_\_\_\_ શ્રેણિક છે.}

\begin{solutionbox}
\textbf{જવાબ}: (b) 2×4

\textbf{ઉકેલ}:
શ્રેણિક ગુણાકારનો નિયમ: $(m \times n) \times (n \times p) = (m \times p)$
$(2 \times 3) \times (3 \times 4) = (2 \times 4)$
\end{solutionbox}

\questionmarks{1.3}{1}{જો $\begin{bmatrix} 0 & x \\ -2 & 4 \end{bmatrix} = \begin{bmatrix} 0 & 4 \\ -2 & 4 \end{bmatrix}$ હોય તો x = \_\_\_\_\_\_\_}

\begin{solutionbox}
\textbf{જવાબ}: (b) 4

\textbf{ઉકેલ}:
અનુરૂપ ઘટકોને સરખાવતા: $x = 4$
\end{solutionbox}

\questionmarks{1.4}{1}{જો A એ સામાન્ય શ્રેણિક (non-singular matrix) હોય તો \_\_\_\_\_\_}

\begin{solutionbox}
\textbf{જવાબ}: (d) $|A| \neq 0$

\textbf{ઉકેલ}:
જો નિશ્ચાયક શૂન્ય ન હોય તો શ્રેણિક સામાન્ય શ્રેણિક કહેવાય છે.
\end{solutionbox}

\questionmarks{1.5}{1}{$\frac{d}{dx}(e^{-\log x}) = $ \_\_\_\_\_\_\_\_\_\_\_\_\_\_}

\begin{solutionbox}
\textbf{જવાબ}: (d) x

\textbf{ઉકેલ}:
$e^{-\log x} = e^{\log x^{-1}} = x^{-1} = \frac{1}{x}$
$\frac{d}{dx}(\frac{1}{x}) = -\frac{1}{x^2}$
\end{solutionbox}

\questionmarks{1.6}{1}{જો $f(x) = \log\sqrt{x^2 + 1}$, તો $f'(0) = $ \_\_\_\_\_\_\_\_\_\_\_\_}

\begin{solutionbox}
\textbf{જવાબ}: (a) 0

\textbf{ઉકેલ}:
$f(x) = \frac{1}{2}\log(x^2 + 1)$
$f'(x) = \frac{1}{2} \cdot \frac{2x}{x^2 + 1} = \frac{x}{x^2 + 1}$
$f'(0) = \frac{0}{0 + 1} = 0$
\end{solutionbox}

\questionmarks{1.7}{1}{જો $x = \sec\theta + \tan\theta$ અને $y = \sec\theta - \tan\theta$ હોય તો $\frac{dy}{dx} = $ \_\_\_\_\_\_\_\_\_\_\_\_}

\begin{solutionbox}
\textbf{જવાબ}: (d) 1

\textbf{ઉકેલ}:
$xy = (\sec\theta + \tan\theta)(\sec\theta - \tan\theta) = \sec^2\theta - \tan^2\theta = 1$
વિકલન કરતા: $x\frac{dy}{dx} + y = 0$
$\frac{dy}{dx} = -\frac{y}{x}$
\end{solutionbox}

\questionmarks{1.8}{1}{$\int e^x(\sin x + \cos x)dx = $ \_\_\_\_\_\_\_\_\_}

\begin{solutionbox}
\textbf{જવાબ}: (b) $e^x\sin x + c$

\textbf{ઉકેલ}:
ખંડશઃ સંકલન અથવા પ્રમાણિત પરિણામનો ઉપયોગ કરતા:
$\int e^x(\sin x + \cos x)dx = e^x\sin x + c$
\end{solutionbox}

\questionmarks{1.9}{1}{$\int_{-1}^{1} x^2 + 1 dx = $ \_\_\_\_\_\_}

\begin{solutionbox}
\textbf{જવાબ}: (d) $\frac{8}{3}$

\textbf{ઉકેલ}:
$\int_{-1}^{1} (x^2 + 1)dx = [\frac{x^3}{3} + x]_{-1}^{1}$
$= (\frac{1}{3} + 1) - (\frac{-1}{3} - 1) = \frac{8}{3}$
\end{solutionbox}

\questionmarks{1.10}{1}{$\int \cot x dx = $ \_\_\_\_\_\_\_\_\_\_\_\_ + c}

\begin{solutionbox}
\textbf{જવાબ}: (a) $\log|\sin x|$

\textbf{ઉકેલ}:
$\int \cot x dx = \int \frac{\cos x}{\sin x} dx = \log|\sin x| + c$
\end{solutionbox}

\questionmarks{1.11}{1}{વિકલ સમીકરણ $\frac{d^2y}{dx^2} + x\frac{dy}{dx} + 3y = 0$ ની કક્ષા (Order) અને પરિમાણ (Degree) અનુક્રમે \_\_\_\_\_\_\_ અને \_\_\_\_\_\_\_ છે.}

\begin{solutionbox}
\textbf{જવાબ}: (a) 2, 1

\textbf{ઉકેલ}:
કક્ષા = ઉચ્ચતમ કક્ષાનું વિકલિત = 2
પરિમાણ = ઉચ્ચતમ કક્ષાના વિકલિતની ઘાત = 1
\end{solutionbox}

\questionmarks{1.12}{1}{વિકલ સમીકરણ $\frac{dy}{dx} + \frac{y}{x} = x$ નો સંકલ્પકારક અવયવ (integrating factor) \_\_\_\_ છે.}

\begin{solutionbox}
\textbf{જવાબ}: (b) $x$

\textbf{ઉકેલ}:
$\frac{dy}{dx} + P(x)y = Q(x)$ માટે, જ્યાં $P(x) = \frac{1}{x}$
I.F. = $e^{\int P(x)dx} = e^{\int \frac{1}{x}dx} = e^{\log x} = x$
\end{solutionbox}

\questionmarks{1.13}{1}{$i + i^2 + i^3 + i^4 = $ \_\_\_\_\_\_}

\begin{solutionbox}
\textbf{જવાબ}: (d) 0

\textbf{ઉકેલ}:
$i + i^2 + i^3 + i^4 = i + (-1) + (-i) + 1 = 0$
\end{solutionbox}

\questionmarks{1.14}{1}{arg(-1) = \_\_\_\_\_\_\_\_\_\_\_}

\begin{solutionbox}
\textbf{જવાબ}: (a) $\pi$

\textbf{ઉકેલ}:
$-1 = \cos\pi + i\sin\pi$, તેથી $\arg(-1) = \pi$
\end{solutionbox}

\questionmarks{2(a)}{6}{કોઈપણ બે લખો.}

\questionmarks{2(a).1}{3}{જો $A = \begin{bmatrix} 1 & 2 \\ -3 & 2 \end{bmatrix}$, $B = \begin{bmatrix} 5 & 6 \\ -2 & 3 \end{bmatrix}$ હોય તો સમીકરણ 3(X+B) + 5A = 0 પરથી શ્રેણિક X શોધો.}

\begin{solutionbox}
\textbf{ઉકેલ}:
$3(X + B) + 5A = 0$
$3X + 3B + 5A = 0$
$3X = -3B - 5A$
$X = -B - \frac{5A}{3}$

$5A = 5\begin{bmatrix} 1 & 2 \\ -3 & 2 \end{bmatrix} = \begin{bmatrix} 5 & 10 \\ -15 & 10 \end{bmatrix}$

$X = -\begin{bmatrix} 5 & 6 \\ -2 & 3 \end{bmatrix} - \frac{1}{3}\begin{bmatrix} 5 & 10 \\ -15 & 10 \end{bmatrix}$

$X = \begin{bmatrix} -5 & -6 \\ 2 & -3 \end{bmatrix} - \begin{bmatrix} \frac{5}{3} & \frac{10}{3} \\ -5 & \frac{10}{3} \end{bmatrix}$

$X = \begin{bmatrix} -\frac{20}{3} & -\frac{28}{3} \\ 7 & -\frac{19}{3} \end{bmatrix}$
\end{solutionbox}

\questionmarks{2(a).2}{3}{જો $A = \begin{bmatrix} 1 & 2 \\ 2 & 1 \end{bmatrix}$ હોય તો સાબિત કરો કે $A^2 - 4A - 5I = 0$}

\begin{solutionbox}
\textbf{ઉકેલ}:
$A^2 = \begin{bmatrix} 1 & 2 \\ 2 & 1 \end{bmatrix}\begin{bmatrix} 1 & 2 \\ 2 & 1 \end{bmatrix} = \begin{bmatrix} 5 & 4 \\ 4 & 5 \end{bmatrix}$

$4A = 4\begin{bmatrix} 1 & 2 \\ 2 & 1 \end{bmatrix} = \begin{bmatrix} 4 & 8 \\ 8 & 4 \end{bmatrix}$

$5I = 5\begin{bmatrix} 1 & 0 \\ 0 & 1 \end{bmatrix} = \begin{bmatrix} 5 & 0 \\ 0 & 5 \end{bmatrix}$

$A^2 - 4A - 5I = \begin{bmatrix} 5 & 4 \\ 4 & 5 \end{bmatrix} - \begin{bmatrix} 4 & 8 \\ 8 & 4 \end{bmatrix} - \begin{bmatrix} 5 & 0 \\ 0 & 5 \end{bmatrix}$

$= \begin{bmatrix} 0 & -4 \\ -4 & 0 \end{bmatrix} - \begin{bmatrix} 5 & 0 \\ 0 & 5 \end{bmatrix} = \begin{bmatrix} 0 & 0 \\ 0 & 0 \end{bmatrix}$

સાબિત થયું.
\end{solutionbox}

\questionmarks{2(a).3}{3}{વિકલ સમીકરણ $\frac{dy}{dx} = (x + y)^2$ ઉકેલો.}

\begin{solutionbox}
\textbf{ઉકેલ}:
ધારો કે $v = x + y$, તો $\frac{dv}{dx} = 1 + \frac{dy}{dx}$
$\frac{dy}{dx} = \frac{dv}{dx} - 1$

કિંમત મુકતા: $\frac{dv}{dx} - 1 = v^2$
$\frac{dv}{dx} = v^2 + 1$
$\frac{dv}{v^2 + 1} = dx$

સંકલન કરતા: $\int \frac{dv}{v^2 + 1} = \int dx$
$\tan^{-1}v = x + c$
$\tan^{-1}(x + y) = x + c$
$x + y = \tan(x + c)$
$y = \tan(x + c) - x$
\end{solutionbox}

\questionmarks{2(b)}{8}{કોઈપણ બે લખો.}

\questionmarks{2(b).1}{4}{જો $A = \begin{bmatrix} 3 & -1 \\ 4 & 1 \\ 5 & 0 \end{bmatrix}$ હોય તો $A^{-1}$ શોધો.}

\begin{solutionbox}
\textbf{ઉકેલ}:
આ 3×2 શ્રેણિક છે, જે ચોરસ શ્રેણિક નથી. ચોરસ શ્રેણિક ન હોય તેવા શ્રેણિકનો વ્યસ્ત અસ્તિત્વ ધરાવતો નથી.

\textbf{વૈકલ્પિક અર્થઘટન - જો તે $\begin{bmatrix} 3 & -1 & 2 \\ 4 & 1 & -1 \\ 5 & 0 & 1 \end{bmatrix}$ હોય:}

સહઅવયવજ શ્રેણિક (adjoint) પદ્ધતિનો ઉપયોગ કરીને:
$|A| = 3(1-0) + 1(4+5) + 2(0-5) = 3 + 9 - 10 = 2$

સહઅવયવો અને સહઅવયવજ શ્રેણિક ગણતરી કરો, પછી $A^{-1} = \frac{1}{|A|} \times adj(A)$
\end{solutionbox}

\questionmarks{2(b).2}{4}{શ્રેણિક પદ્ધતિનો ઉપયોગ કરીને સમીકરણ 3X-2Y=8 અને 5X+4Y=6 ઉકેલો.}

\begin{solutionbox}
\textbf{ઉકેલ}:
$\begin{bmatrix} 3 & -2 \\ 5 & 4 \end{bmatrix}\begin{bmatrix} X \\ Y \end{bmatrix} = \begin{bmatrix} 8 \\ 6 \end{bmatrix}$

$|A| = 3(4) - (-2)(5) = 12 + 10 = 22$

$A^{-1} = \frac{1}{22}\begin{bmatrix} 4 & 2 \\ -5 & 3 \end{bmatrix}$

$\begin{bmatrix} X \\ Y \end{bmatrix} = \frac{1}{22}\begin{bmatrix} 4 & 2 \\ -5 & 3 \end{bmatrix}\begin{bmatrix} 8 \\ 6 \end{bmatrix}$

$\begin{bmatrix} X \\ Y \end{bmatrix} = \frac{1}{22}\begin{bmatrix} 32 + 12 \\ -40 + 18 \end{bmatrix} = \frac{1}{22}\begin{bmatrix} 44 \\ -22 \end{bmatrix}$

$X = 2, Y = -1$
\end{solutionbox}

\questionmarks{2(b).3}{4}{જો $M = \begin{bmatrix} 2 & 3 \\ 0 & 1 \end{bmatrix}$, $N = \begin{bmatrix} 3 & 4 \\ 2 & 1 \end{bmatrix}$ હોય તો સાબિત કરો કે $(MN)^T = N^T M^T$}

\begin{solutionbox}
\textbf{ઉકેલ}:
$MN = \begin{bmatrix} 2 & 3 \\ 0 & 1 \end{bmatrix}\begin{bmatrix} 3 & 4 \\ 2 & 1 \end{bmatrix} = \begin{bmatrix} 12 & 11 \\ 2 & 1 \end{bmatrix}$

$(MN)^T = \begin{bmatrix} 12 & 2 \\ 11 & 1 \end{bmatrix}$

$M^T = \begin{bmatrix} 2 & 0 \\ 3 & 1 \end{bmatrix}$, $N^T = \begin{bmatrix} 3 & 2 \\ 4 & 1 \end{bmatrix}$

$N^T M^T = \begin{bmatrix} 3 & 2 \\ 4 & 1 \end{bmatrix}\begin{bmatrix} 2 & 0 \\ 3 & 1 \end{bmatrix} = \begin{bmatrix} 12 & 2 \\ 11 & 1 \end{bmatrix}$

તેથી $(MN)^T = N^T M^T$ સાબિત થાય છે.
\end{solutionbox}

\questionmarks{3(a)}{6}{કોઈપણ બે લખો.}

\questionmarks{3(a).1}{3}{વ્યાખ્યાનો ઉપયોગ કરીને $\sqrt{x}$ નું વિકલન કરો.}

\begin{solutionbox}
\textbf{ઉકેલ}:
$f(x) = \sqrt{x} = x^{1/2}$

વ્યાખ્યાનો ઉપયોગ કરતા: $f'(x) = \lim_{h \to 0} \frac{f(x+h) - f(x)}{h}$

$f'(x) = \lim_{h \to 0} \frac{\sqrt{x+h} - \sqrt{x}}{h}$

કરણી લેતા: $f'(x) = \lim_{h \to 0} \frac{(\sqrt{x+h} - \sqrt{x})(\sqrt{x+h} + \sqrt{x})}{h(\sqrt{x+h} + \sqrt{x})}$

$= \lim_{h \to 0} \frac{(x+h) - x}{h(\sqrt{x+h} + \sqrt{x})} = \lim_{h \to 0} \frac{h}{h(\sqrt{x+h} + \sqrt{x})}$

$= \lim_{h \to 0} \frac{1}{\sqrt{x+h} + \sqrt{x}} = \frac{1}{2\sqrt{x}}$
\end{solutionbox}

\questionmarks{3(a).2}{3}{જો $y = \log(x + \sqrt{1 + x^2})$ હોય તો $\frac{dy}{dx}$ શોધો.}

\begin{solutionbox}
\textbf{ઉકેલ}:
$y = \log(x + \sqrt{1 + x^2})$

$\frac{dy}{dx} = \frac{1}{x + \sqrt{1 + x^2}} \cdot \frac{d}{dx}(x + \sqrt{1 + x^2})$

$\frac{d}{dx}(x + \sqrt{1 + x^2}) = 1 + \frac{1}{2\sqrt{1 + x^2}} \cdot 2x = 1 + \frac{x}{\sqrt{1 + x^2}}$

$= \frac{\sqrt{1 + x^2} + x}{\sqrt{1 + x^2}}$

$\frac{dy}{dx} = \frac{1}{x + \sqrt{1 + x^2}} \cdot \frac{\sqrt{1 + x^2} + x}{\sqrt{1 + x^2}}$

$= \frac{1}{\sqrt{1 + x^2}}$
\end{solutionbox}

\questionmarks{3(a).3}{3}{$\int \frac{4 + 3\cos x}{\sin^2 x} dx$ ની કિંમત શોધો.}

\begin{solutionbox}
\textbf{ઉકેલ}:
$\int \frac{4 + 3\cos x}{\sin^2 x} dx = \int \frac{4}{\sin^2 x} dx + \int \frac{3\cos x}{\sin^2 x} dx$

$= 4\int \csc^2 x dx + 3\int \frac{\cos x}{\sin^2 x} dx$

$= -4\cot x + 3\int \sin^{-2} x \cos x dx$

બીજા સંકલન માટે, ધારો કે $u = \sin x$, $du = \cos x dx$
$3\int u^{-2} du = 3(-u^{-1}) = -\frac{3}{\sin x}$

$\int \frac{4 + 3\cos x}{\sin^2 x} dx = -4\cot x - 3\csc x + c$
\end{solutionbox}

\questionmarks{3(b)}{8}{કોઈપણ બે લખો.}

\questionmarks{3(b).1}{4}{જો $y = \log(\sin x)$ હોય તો સાબિત કરો કે $\frac{d^2y}{dx^2} + (\frac{dy}{dx})^2 + 1 = 0$}

\begin{solutionbox}
\textbf{ઉકેલ}:
$y = \log(\sin x)$

$\frac{dy}{dx} = \frac{1}{\sin x} \cdot \cos x = \cot x$

$\frac{d^2y}{dx^2} = \frac{d}{dx}(\cot x) = -\csc^2 x$

હવે, $\frac{d^2y}{dx^2} + (\frac{dy}{dx})^2 + 1 = -\csc^2 x + \cot^2 x + 1$

નિત્યસમનો ઉપયોગ કરતા: $\csc^2 x - \cot^2 x = 1$
$-\csc^2 x + \cot^2 x + 1 = -(\csc^2 x - \cot^2 x) = -1 + 1 = 0$

સાબિત થયું.
\end{solutionbox}

\questionmarks{3(b).2}{4}{જો $x + y = \sin(xy)$ હોય તો $\frac{dy}{dx}$ શોધો.}

\begin{solutionbox}
\textbf{ઉકેલ}:
$x + y = \sin(xy)$

x પ્રત્યે વિકલન કરતા:
$1 + \frac{dy}{dx} = \cos(xy) \cdot \frac{d}{dx}(xy)$

$1 + \frac{dy}{dx} = \cos(xy) \cdot (y + x\frac{dy}{dx})$

$1 + \frac{dy}{dx} = y\cos(xy) + x\cos(xy)\frac{dy}{dx}$

$1 + \frac{dy}{dx} - x\cos(xy)\frac{dy}{dx} = y\cos(xy)$

$\frac{dy}{dx}(1 - x\cos(xy)) = y\cos(xy) - 1$

$\frac{dy}{dx} = \frac{y\cos(xy) - 1}{1 - x\cos(xy)}$
\end{solutionbox}

\questionmarks{3(b).3}{4}{એક કણની ગતિ $s = t^3 - 5t^2 + 3t$ છે. જ્યારે કણ સ્થિર થાય ત્યારે પ્રવેગ શોધો.}

\begin{solutionbox}
\textbf{ઉકેલ}:
આપેલ છે: $s = t^3 - 5t^2 + 3t$

વેગ: $v = \frac{ds}{dt} = 3t^2 - 10t + 3$

પ્રવેગ: $a = \frac{dv}{dt} = 6t - 10$

સ્થિર સ્થિતિમાં, $v = 0$:
$3t^2 - 10t + 3 = 0$

દ્વિઘાત સૂત્રનો ઉપયોગ કરતા: $t = \frac{10 \pm \sqrt{100 - 36}}{6} = \frac{10 \pm 8}{6}$

$t = 3$ અથવા $t = \frac{1}{3}$

$t = 3$ સમયે: $a = 6(3) - 10 = 8$
$t = \frac{1}{3}$ સમયે: $a = 6(\frac{1}{3}) - 10 = -8$

પ્રવેગ અનુક્રમે $8$ અને $-8$ છે.
\end{solutionbox}

\questionmarks{4(a)}{6}{કોઈપણ બે લખો.}

\questionmarks{4(a).1}{3}{$\int x \sin x dx$}

\begin{solutionbox}
\textbf{ઉકેલ}:
ખંડશઃ સંકલનનો ઉપયોગ કરતા: $\int u dv = uv - \int v du$

ધારો કે $u = x$, $dv = \sin x dx$
$du = dx$, $v = -\cos x$

$\int x \sin x dx = x(-\cos x) - \int (-\cos x) dx$
$= -x\cos x + \int \cos x dx$
$= -x\cos x + \sin x + c$
\end{solutionbox}

\questionmarks{4(a).2}{3}{$\int \frac{2x + 1}{(x + 1)(x - 3)} dx$}

\begin{solutionbox}
\textbf{ઉકેલ}:
આંશિક અપૂર્ણાંકનો ઉપયોગ કરતા:
$\frac{2x + 1}{(x + 1)(x - 3)} = \frac{A}{x + 1} + \frac{B}{x - 3}$

$2x + 1 = A(x - 3) + B(x + 1)$

$x = -1$ લેતા: $-2 + 1 = A(-4) \Rightarrow A = \frac{1}{4}$
$x = 3$ લેતા: $6 + 1 = B(4) \Rightarrow B = \frac{7}{4}$

$\int \frac{2x + 1}{(x + 1)(x - 3)} dx = \frac{1}{4}\int \frac{1}{x + 1} dx + \frac{7}{4}\int \frac{1}{x - 3} dx$

$= \frac{1}{4}\log|x + 1| + \frac{7}{4}\log|x - 3| + c$
\end{solutionbox}

\questionmarks{4(a).3}{3}{સંકર સંખ્યા $z = 7 + 24i$ નું વર્ગમૂળ શોધો.}

\begin{solutionbox}
\textbf{ઉકેલ}:
ધારો કે $\sqrt{7 + 24i} = a + bi$

$(a + bi)^2 = 7 + 24i$
$a^2 - b^2 + 2abi = 7 + 24i$

સરખાવતા: $a^2 - b^2 = 7$ અને $2ab = 24$
બીજા સમીકરણ પરથી: $b = \frac{12}{a}$

કિંમત મુકતા: $a^2 - \frac{144}{a^2} = 7$
$a^4 - 7a^2 - 144 = 0$

ધારો કે $u = a^2$: $u^2 - 7u - 144 = 0$
$(u - 16)(u + 9) = 0$
$u = 16$ (ધન કિંમત લેતા)
$a^2 = 16 \Rightarrow a = 4$
$b = \frac{12}{4} = 3$

તેથી: $\sqrt{7 + 24i} = 4 + 3i$ અથવા $-(4 + 3i)$
\end{solutionbox}

\questionmarks{4(b)}{8}{કોઈપણ બે લખો.}

\questionmarks{4(b).1}{4}{$\int_0^{\pi/2} \frac{\sqrt{\sin x}}{\sqrt{\sin x} + \sqrt{\cos x}} dx$ ની કિંમત શોધો.}

\begin{solutionbox}
\textbf{ઉકેલ}:
ધારો કે $I = \int_0^{\pi/2} \frac{\sqrt{\sin x}}{\sqrt{\sin x} + \sqrt{\cos x}} dx$

ગુણધર્મનો ઉપયોગ કરતા: $\int_0^a f(x) dx = \int_0^a f(a-x) dx$

$I = \int_0^{\pi/2} \frac{\sqrt{\sin(\pi/2 - x)}}{\sqrt{\sin(\pi/2 - x)} + \sqrt{\cos(\pi/2 - x)}} dx$

$= \int_0^{\pi/2} \frac{\sqrt{\cos x}}{\sqrt{\cos x} + \sqrt{\sin x}} dx$

બંને સમીકરણોનો સરવાળો કરતા:
$2I = \int_0^{\pi/2} \frac{\sqrt{\sin x} + \sqrt{\cos x}}{\sqrt{\sin x} + \sqrt{\cos x}} dx = \int_0^{\pi/2} 1 dx = \frac{\pi}{2}$

તેથી: $I = \frac{\pi}{4}$
\end{solutionbox}

\questionmarks{4(b).2}{4}{વક્ર $y = 3x^2$, x અક્ષ અને રેખા $x = 2$ અને $x = 3$ વડે ઘેરાયેલા પ્રદેશનું ક્ષેત્રફળ શોધો.}

\begin{solutionbox}
\textbf{ઉકેલ}:
ક્ષેત્રફળ = $\int_2^3 y dx = \int_2^3 3x^2 dx$

$= 3\int_2^3 x^2 dx = 3[\frac{x^3}{3}]_2^3$

$= [x^3]_2^3 = 3^3 - 2^3 = 27 - 8 = 19$

ક્ષેત્રફળ = 19 ચોરસ એકમ
\end{solutionbox}

\questionmarks{4(b).3}{4}{સાદુરૂપ આપો: $\frac{(\cos 2\theta + i\sin 2\theta)^{-3} \cdot (\cos 3\theta - i\sin 3\theta)^2}{(\cos 2\theta - i\sin 2\theta)^{-7} \cdot (\cos 5\theta - i\sin 5\theta)^3}$}

\begin{solutionbox}
\textbf{ઉકેલ}:
યુલરના સમીકરણનો ઉપયોગ કરતા: $\cos\theta + i\sin\theta = e^{i\theta}$

$(\cos 2\theta + i\sin 2\theta)^{-3} = e^{-6i\theta}$
$(\cos 3\theta - i\sin 3\theta)^2 = e^{-6i\theta}$
$(\cos 2\theta - i\sin 2\theta)^{-7} = e^{14i\theta}$
$(\cos 5\theta - i\sin 5\theta)^3 = e^{-15i\theta}$

પદાવલિ = $\frac{e^{-6i\theta} \cdot e^{-6i\theta}}{e^{14i\theta} \cdot e^{-15i\theta}} = \frac{e^{-12i\theta}}{e^{-i\theta}} = e^{-11i\theta}$

$= \cos(-11\theta) + i\sin(-11\theta) = \cos(11\theta) - i\sin(11\theta)$
\end{solutionbox}

\questionmarks{5(a)}{6}{કોઈપણ બે લખો.}

\questionmarks{5(a).1}{3}{$\frac{4+2i}{(3+2i)(5-3i)}$ ને a+ib સ્વરૂપમાં ફેરવો.}

\begin{solutionbox}
\textbf{ઉકેલ}:
પ્રથમ, છેદનું સાદુરૂપ આપો:
$(3+2i)(5-3i) = 15 - 9i + 10i - 6i^2 = 15 + i + 6 = 21 + i$

હવે: $\frac{4+2i}{21+i}$

અનુબદ્ધ કરણી વડે ગુણતા: $\frac{4+2i}{21+i} \cdot \frac{21-i}{21-i}$

$= \frac{(4+2i)(21-i)}{(21+i)(21-i)} = \frac{84 - 4i + 42i - 2i^2}{441 - i^2}$

$= \frac{84 + 38i + 2}{441 + 1} = \frac{86 + 38i}{442} = \frac{43 + 19i}{221}$
\end{solutionbox}

\questionmarks{5(a).2}{3}{$z = 1 - \sqrt{3}i$ ને ધ્રુવીય સ્વરૂપમાં ફેરવો.}

\begin{solutionbox}
\textbf{ઉકેલ}:
$z = 1 - \sqrt{3}i$

$|z| = \sqrt{1^2 + (-\sqrt{3})^2} = \sqrt{1 + 3} = 2$

$\arg(z) = \tan^{-1}\left(\frac{-\sqrt{3}}{1}\right) = -\frac{\pi}{3}$ (કારણ કે z ચોથા ચરણમાં છે)

તેથી: $z = 2(\cos(-\frac{\pi}{3}) + i\sin(-\frac{\pi}{3})) = 2e^{-i\pi/3}$
\end{solutionbox}

\questionmarks{5(a).3}{3}{સાબિત કરો કે $(1 + \cos\theta + i\sin\theta)^n + (1 + \cos\theta - i\sin\theta)^n = 2^{n+1}\cos^n(\frac{\theta}{2})\cos(\frac{n\theta}{2})$}

\begin{solutionbox}
\textbf{ઉકેલ}:
$1 + \cos\theta + i\sin\theta = 1 + e^{i\theta} = 1 + \cos\theta + i\sin\theta$

નિત્યસમનો ઉપયોગ કરતા: $1 + \cos\theta = 2\cos^2(\frac{\theta}{2})$

$1 + \cos\theta + i\sin\theta = 2\cos^2(\frac{\theta}{2}) + 2i\sin(\frac{\theta}{2})\cos(\frac{\theta}{2})$

$= 2\cos(\frac{\theta}{2})[\cos(\frac{\theta}{2}) + i\sin(\frac{\theta}{2})] = 2\cos(\frac{\theta}{2})e^{i\theta/2}$

તે જ રીતે: $1 + \cos\theta - i\sin\theta = 2\cos(\frac{\theta}{2})e^{-i\theta/2}$

$(1 + \cos\theta + i\sin\theta)^n = 2^n\cos^n(\frac{\theta}{2})e^{in\theta/2}$

$(1 + \cos\theta - i\sin\theta)^n = 2^n\cos^n(\frac{\theta}{2})e^{-in\theta/2}$

સરવાળો = $2^n\cos^n(\frac{\theta}{2})[e^{in\theta/2} + e^{-in\theta/2}] = 2^n\cos^n(\frac{\theta}{2}) \cdot 2\cos(\frac{n\theta}{2})$

$= 2^{n+1}\cos^n(\frac{\theta}{2})\cos(\frac{n\theta}{2})$

સાબિત થયું.
\end{solutionbox}

\questionmarks{5(b)}{8}{કોઈપણ બે લખો.}

\questionmarks{5(b).1}{4}{વિકલ સમીકરણ $x\log x \frac{dy}{dx} + y = \log x^2$ ઉકેલો.}

\begin{solutionbox}
\textbf{ઉકેલ}:
$x\log x \frac{dy}{dx} + y = 2\log x$

$x\log x$ વડે ભાગતા:
$\frac{dy}{dx} + \frac{y}{x\log x} = \frac{2}{x}$

આ સુરેખ વિકલ સમીકરણ છે: $\frac{dy}{dx} + P(x)y = Q(x)$

જ્યાં $P(x) = \frac{1}{x\log x}$ અને $Q(x) = \frac{2}{x}$

\textbf{સંકલ્પકારક અવયવ}: $e^{\int P(x)dx} = e^{\int \frac{1}{x\log x}dx}$

ધારો કે $u = \log x$, તો $du = \frac{1}{x}dx$
$\int \frac{1}{x\log x}dx = \int \frac{1}{u}du = \log u = \log(\log x)$

I.F. = $e^{\log(\log x)} = \log x$

\textbf{ઉકેલ}: $y \cdot \log x = \int \frac{2}{x} \cdot \log x dx$

$= 2\int \frac{\log x}{x} dx = 2 \cdot \frac{(\log x)^2}{2} = (\log x)^2$

તેથી: $y = \frac{(\log x)^2}{\log x} = \log x$
\end{solutionbox}

\questionmarks{5(b).2}{4}{વિકલ સમીકરણ $\frac{dy}{dx} - \frac{y}{x} = e^x$ ઉકેલો.}

\begin{solutionbox}
\textbf{ઉકેલ}:
આ સુરેખ વિકલ સમીકરણ છે: $\frac{dy}{dx} + P(x)y = Q(x)$

જ્યાં $P(x) = -\frac{1}{x}$ અને $Q(x) = e^x$

\textbf{સંકલ્પકારક અવયવ}: $e^{\int P(x)dx} = e^{\int -\frac{1}{x}dx} = e^{-\log x} = \frac{1}{x}$

\textbf{ઉકેલ}: $y \cdot \frac{1}{x} = \int e^x \cdot \frac{1}{x} dx$

સંકલન $\int \frac{e^x}{x}dx$ પ્રાથમિક વિધેયોમાં દર્શાવી શકાતું નથી.

\textbf{વૈકલ્પિક અભિગમ - જો તે} $\frac{dy}{dx} + \frac{y}{x} = e^x$ \textbf{હોય:}

I.F. = $e^{\int \frac{1}{x}dx} = e^{\log x} = x$

$y \cdot x = \int e^x \cdot x dx$

ખંડશઃ સંકલનનો ઉપયોગ કરતા:
$\int xe^x dx = xe^x - \int e^x dx = xe^x - e^x = e^x(x-1)$

તેથી: $xy = e^x(x-1) + c$
$y = \frac{e^x(x-1) + c}{x}$
\end{solutionbox}

\questionmarks{5(b).3}{4}{વિકલ સમીકરણ $\sec^2x \tan y dx + \sec^2y \tan x dy = 0$ ઉકેલો, જ્યાં $y(\frac{\pi}{4}) = \frac{\pi}{4}$.}

\begin{solutionbox}
\textbf{ઉકેલ}:
$\sec^2x \tan y dx + \sec^2y \tan x dy = 0$

પદો ગોઠવતા: $\frac{\sec^2x}{\tan x}dx + \frac{\sec^2y}{\tan y}dy = 0$

$\frac{\cos x}{\sin x \cos^2 x}dx + \frac{\cos y}{\sin y \cos^2 y}dy = 0$

$\frac{1}{\sin x \cos x}dx + \frac{1}{\sin y \cos y}dy = 0$

$\frac{2}{\sin 2x}dx + \frac{2}{\sin 2y}dy = 0$

$\csc(2x)dx + \csc(2y)dy = 0$

સંકલન કરતા: $\int \csc(2x)dx + \int \csc(2y)dy = c$

$-\frac{1}{2}\log|\csc(2x) + \cot(2x)| - \frac{1}{2}\log|\csc(2y) + \cot(2y)| = c$

$\log|\csc(2x) + \cot(2x)| + \log|\csc(2y) + \cot(2y)| = -2c = k$

$|\csc(2x) + \cot(2x)| \cdot |\csc(2y) + \cot(2y)| = e^k$

શરૂઆતની શરત $y(\frac{\pi}{4}) = \frac{\pi}{4}$ નો ઉપયોગ કરતા:
$x = \frac{\pi}{4}$ મુકતા, $y = \frac{\pi}{4}$

$|\csc(\frac{\pi}{2}) + \cot(\frac{\pi}{2})| \cdot |\csc(\frac{\pi}{2}) + \cot(\frac{\pi}{2})| = |1 + 0| \cdot |1 + 0| = 1$

તેથી: $(\csc(2x) + \cot(2x))(\csc(2y) + \cot(2y)) = 1$
\end{solutionbox}

\end{document}
