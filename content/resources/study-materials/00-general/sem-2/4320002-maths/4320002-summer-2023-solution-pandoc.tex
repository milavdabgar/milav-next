\documentclass[10pt,a4paper]{article}

% content/resources/templates/preamble.tex
\usepackage[margin=0.6in]{geometry}
\author{Milav Dabgar}
\usepackage{amsmath,amssymb,amsthm}
\usepackage{booktabs}
\usepackage{multirow}
\usepackage{xcolor}
\usepackage{tcolorbox}
\tcbuselibrary{breakable,skins}
\usepackage[colorlinks=true,linkcolor=blue]{hyperref}
\usepackage{titlesec}
\usepackage{enumitem}
\usepackage{tikz}
\usepackage{pgfplots}
\usepackage{circuitikz}
\usepackage[version=4]{mhchem}
\usepackage{longtable}
\usepackage{array}
\usepackage{float}
\usepackage{caption}
\usepackage{listings}

\lstset{
  basicstyle=\small\ttfamily,
  breaklines=true,
  breakatwhitespace=false,
  postbreak=\mbox{\textcolor{red}{$\hookrightarrow$}\space},
  float=false,
  numbers=left,
  numberstyle=\tiny\color{gray},
  numbersep=10pt,
  xleftmargin=2em,
  keywordstyle=\color{blue},
  commentstyle=\color{green!60!black},
  stringstyle=\color{purple},
  backgroundcolor=\color{gray!5},
  showstringspaces=false,
  tabsize=2,
  captionpos=b,
  keepspaces=true,
  columns=flexible
}

\pgfplotsset{compat=1.18}
\usetikzlibrary{shapes,arrows,positioning,calc,patterns,decorations.pathmorphing,decorations.markings,arrows.meta}

% Color scheme
\definecolor{headcolor}{RGB}{0,102,204}
\definecolor{keycolor}{RGB}{220,20,60}
\definecolor{solutioncolor}{RGB}{34,139,34}
\definecolor{mnemoniccolor}{RGB}{148,0,211}
\definecolor{codecolor}{RGB}{0,0,100}

% Spacing
\setlength{\parskip}{3pt}
\setlist[itemize]{nosep}
\setlist[enumerate]{nosep}

% Title formatting
\titleformat{\section}{\Large\bfseries\color{headcolor}}{\thesection}{1em}{}
\titleformat{\subsection}{\large\bfseries\color{headcolor}}{\thesubsection}{1em}{}

% Pandoc tightlist compatibility
\providecommand{\tightlist}{%
  \setlength{\itemsep}{0pt}\setlength{\parskip}{0pt}}

% Pandoc longtable compatibility
\newcounter{none}
\def\thenone{}


% content/resources/templates/english-boxes.tex
% This file is currently empty - it exists to maintain consistency with the import structure.
% Add custom environments here if needed in the future.


\begin{document}

\begin{center}
{\Huge\bfseries\color{headcolor} Subject Name Solutions}\\[5pt]
{\LARGE 4320002 -- Summer 2023}\\[3pt]
{\large Semester 1 Study Material}\\[3pt]
{\normalsize\textit{Detailed Solutions and Explanations}}
\end{center}

\vspace{10pt}

\subsection*{Q.1 [14 marks]}\label{q.1-14-marks}

\textbf{Fill in the blanks using appropriate choice from the given
options.}

\subsubsection{Q.1.1 [1 mark]}\label{q.1.1-1-mark}

\textbf{Order of
\(\begin{bmatrix} 1 & 0 & 3 \\ -2 & 4 & 0 \end{bmatrix}\) is
\_\_\_\_\_\_\_\_\_\_\_.}

\begin{solutionbox}
b. \(2 \times 3\)

\textbf{Solution}: The matrix has 2 rows and 3 columns, so the order is
\(2 \times 3\).

\end{solutionbox}
\subsubsection{Q.1.2 [1 mark]}\label{q.1.2-1-mark}

\textbf{If A is of order \(2 \times 3\) and B is of order \(3 \times 2\)
then AB is of order \_\_\_\_\_\_\_\_\_.}

\begin{solutionbox}
d.~\(2 \times 2\)

\textbf{Solution}: For matrix multiplication \(AB\), if \(A\) is
\(2 \times 3\) and \(B\) is \(3 \times 2\), then \(AB\) is of order
\(2 \times 2\).

\end{solutionbox}
\subsubsection{Q.1.3 [1 mark]}\label{q.1.3-1-mark}

\textbf{If \(A = \begin{bmatrix} 1 & -1 \end{bmatrix}\) then \$A\^{}T =
\$ \_\_\_\_\_\_\_}

\begin{solutionbox}
b. \(\begin{bmatrix} 1 \\ -1 \end{bmatrix}\)

\textbf{Solution}: The transpose of a row matrix becomes a column
matrix. \(A^T = \begin{bmatrix} 1 \\ -1 \end{bmatrix}\)

\end{solutionbox}
\subsubsection{Q.1.4 [1 mark]}\label{q.1.4-1-mark}

\textbf{If \(A = \begin{bmatrix} 1 & 2 \\ 3 & 4 \end{bmatrix}\) then
\$\text{adj } A = \$ \_\_\_\_\_\_}

\begin{solutionbox}
d.~\(\begin{bmatrix} 4 & -2 \\ -3 & 1 \end{bmatrix}\)

\textbf{Solution}: For a \(2 \times 2\) matrix
\(A = \begin{bmatrix} a & b \\ c & d \end{bmatrix}\),
\(\text{adj } A = \begin{bmatrix} d & -b \\ -c & a \end{bmatrix}\)

Therefore:
\(\text{adj } A = \begin{bmatrix} 4 & -2 \\ -3 & 1 \end{bmatrix}\)

\end{solutionbox}
\subsubsection{Q.1.5 [1 mark]}\label{q.1.5-1-mark}

\textbf{\$\frac{d}{dx}(e\^{}x) = \$ \_\_\_\_\_}

\begin{solutionbox}
a. \(e^x\)

\textbf{Solution}: \(\frac{d}{dx}(e^x) = e^x\)

\end{solutionbox}
\subsubsection{Q.1.6 [1 mark]}\label{q.1.6-1-mark}

\textbf{If \(f(x) = \log x\) then \$f'(1) = \$ \_\_\_\_\_}

\begin{solutionbox}
c.~1

\textbf{Solution}: \(f'(x) = \frac{1}{x}\) \(f'(1) = \frac{1}{1} = 1\)

\end{solutionbox}
\subsubsection{Q.1.7 [1 mark]}\label{q.1.7-1-mark}

\textbf{\$\frac{d}{dx}(3\^{}\{\log\emph{3 x\}) = \$ }\_\_\_\_\_}

\begin{solutionbox}
b. \(2x\)

\textbf{Solution}: Using the property \(a^{\log_a x} = x\):
\(3^{\log_3 x} = x\) Therefore:
\(\frac{d}{dx}(3^{\log_3 x}) = \frac{d}{dx}(x) = 1\)

Wait, let me recalculate this. The expression is
\(3^{\log_3 x^2} = x^2\) \(\frac{d}{dx}(x^2) = 2x\)

\end{solutionbox}
\subsubsection{Q.1.8 [1 mark]}\label{q.1.8-1-mark}

\textbf{\$\int \sin x , dx = \$ \_\_\_\_\_}

\begin{solutionbox}
c.~\(-\cos x\)

\textbf{Solution}: \(\int \sin x \, dx = -\cos x + C\)

\end{solutionbox}
\subsubsection{Q.1.9 [1 mark]}\label{q.1.9-1-mark}

**\$\int\emph{\{-1\}\^{}\{1\} x\^{}3 , dx = \$ }\_\_\_\_**

\begin{solutionbox}
b. 0

\textbf{Solution}:
\(\int_{-1}^{1} x^3 \, dx = \left[\frac{x^4}{4}\right]_{-1}^{1} = \frac{1}{4} - \frac{1}{4} = 0\)

\end{solutionbox}
\subsubsection{Q.1.10 [1 mark]}\label{q.1.10-1-mark}

\textbf{\$\int \frac{1}{1+x^2} , dx = \$ \_\_\_\_\_}

\begin{solutionbox}
d.~\(\tan^{-1} x\)

\textbf{Solution}: \(\int \frac{1}{1+x^2} \, dx = \tan^{-1} x + C\)

\end{solutionbox}
\subsubsection{Q.1.11 [1 mark]}\label{q.1.11-1-mark}

\textbf{Order of the differential equation \(\frac{d^2y}{dx^2} - y = 0\)
is \_\_\_\_\_\_\_\_.}

\begin{solutionbox}
b. 2

\textbf{Solution}: The highest derivative is \(\frac{d^2y}{dx^2}\), so
the order is 2.

\end{solutionbox}
\subsubsection{Q.1.12 [1 mark]}\label{q.1.12-1-mark}

\textbf{The integration factor (I.F) of \(\frac{dy}{dx} + Py = Q\) is
\_\_\_\_\_\_\_\_}

\begin{solutionbox}
a. \(e^{\int P \, dx}\)

\textbf{Solution}: For a linear differential equation
\(\frac{dy}{dx} + Py = Q\), the integrating factor is
\(e^{\int P \, dx}\).

\end{solutionbox}
\subsubsection{Q.1.13 [1 mark]}\label{q.1.13-1-mark}

\textbf{If \(Z = 4 - 5i\) then \$\bar\{Z\} = \$ \_\_\_\_\_\_\_\_}

\begin{solutionbox}
c.~\(4 - 5i\)

\textbf{Solution}: Wait, this seems incorrect. If \(Z = 4 - 5i\), then
\(\bar{Z} = 4 + 5i\). The correct answer should be \(4 + 5i\).

\end{solutionbox}
\subsubsection{Q.1.14 [1 mark]}\label{q.1.14-1-mark}

\textbf{\$i\^{}\{10\} = \$ \_\_\_\_\_\_}

\begin{solutionbox}
b. -1

\textbf{Solution}:
\(i^{10} = i^{4 \cdot 2 + 2} = (i^4)^2 \cdot i^2 = 1^2 \cdot (-1) = -1\)

\end{solutionbox}
\subsection*{Q.2 (A) [6 marks]}\label{q.2-a-6-marks}

\textbf{Attempt any two.}

\subsubsection{Q.2(A).1 [3 marks]}\label{q.2a.1-3-marks}

\textbf{If \(A = \begin{bmatrix} 2 & -1 \\ 4 & 3 \end{bmatrix}\) and
\(B = \begin{bmatrix} 3 & 2 \\ 1 & 4 \end{bmatrix}\) then find the
matrix X such that \(2A + X = 3B\).}

\textbf{Solution}: \(2A + X = 3B\) \(X = 3B - 2A\)

\(2A = 2\begin{bmatrix} 2 & -1 \\ 4 & 3 \end{bmatrix} = \begin{bmatrix} 4 & -2 \\ 8 & 6 \end{bmatrix}\)

\(3B = 3\begin{bmatrix} 3 & 2 \\ 1 & 4 \end{bmatrix} = \begin{bmatrix} 9 & 6 \\ 3 & 12 \end{bmatrix}\)

\(X = \begin{bmatrix} 9 & 6 \\ 3 & 12 \end{bmatrix} - \begin{bmatrix} 4 & -2 \\ 8 & 6 \end{bmatrix} = \begin{bmatrix} 5 & 8 \\ -5 & 6 \end{bmatrix}\)

\subsubsection{Q.2(A).2 [3 marks]}\label{q.2a.2-3-marks}

\textbf{If \(A = \begin{bmatrix} 5 & 4 \\ 4 & 3 \end{bmatrix}\) and
\(B = \begin{bmatrix} 1 & 3 \\ 2 & 1 \end{bmatrix}\) then find
\((AB)^T\).}

\textbf{Solution}: First, find \(AB\):
\(AB = \begin{bmatrix} 5 & 4 \\ 4 & 3 \end{bmatrix}\begin{bmatrix} 1 & 3 \\ 2 & 1 \end{bmatrix}\)

\(AB = \begin{bmatrix} 5(1)+4(2) & 5(3)+4(1) \\ 4(1)+3(2) & 4(3)+3(1) \end{bmatrix} = \begin{bmatrix} 13 & 19 \\ 10 & 15 \end{bmatrix}\)

\((AB)^T = \begin{bmatrix} 13 & 10 \\ 19 & 15 \end{bmatrix}\)

\subsubsection{Q.2(A).3 [3 marks]}\label{q.2a.3-3-marks}

\textbf{Solve: \(\frac{dy}{dx} = x^2 \cdot e^{-y}\).}

\textbf{Solution}: \(\frac{dy}{dx} = x^2 \cdot e^{-y}\)

Separating variables: \(e^y \, dy = x^2 \, dx\)

Integrating both sides: \(\int e^y \, dy = \int x^2 \, dx\)

\(e^y = \frac{x^3}{3} + C\)

\(y = \ln\left(\frac{x^3}{3} + C\right)\)

\subsection*{Q.2 (B) [8 marks]}\label{q.2-b-8-marks}

\textbf{Attempt any two.}

\subsubsection{Q.2(B).1 [4 marks]}\label{q.2b.1-4-marks}

\textbf{If \(A = \begin{bmatrix} 2 & 3 & -1 \\ 4 & 5 & 0 \end{bmatrix}\)
and \(B = \begin{bmatrix} 1 & 2 & 4 \\ 2 & 3 & 1 \end{bmatrix}\) then
prove that \((A + B)^T = A^T + B^T\).}

\textbf{Solution}:
\(A + B = \begin{bmatrix} 2 & 3 & -1 \\ 4 & 5 & 0 \end{bmatrix} + \begin{bmatrix} 1 & 2 & 4 \\ 2 & 3 & 1 \end{bmatrix}\)

\(A + B = \begin{bmatrix} 3 & 5 & 3 \\ 6 & 8 & 1 \end{bmatrix}\)

\((A + B)^T = \begin{bmatrix} 3 & 6 \\ 5 & 8 \\ 3 & 1 \end{bmatrix}\)

\(A^T = \begin{bmatrix} 2 & 4 \\ 3 & 5 \\ -1 & 0 \end{bmatrix}\),
\(B^T = \begin{bmatrix} 1 & 2 \\ 2 & 3 \\ 4 & 1 \end{bmatrix}\)

\(A^T + B^T = \begin{bmatrix} 2 & 4 \\ 3 & 5 \\ -1 & 0 \end{bmatrix} + \begin{bmatrix} 1 & 2 \\ 2 & 3 \\ 4 & 1 \end{bmatrix} = \begin{bmatrix} 3 & 6 \\ 5 & 8 \\ 3 & 1 \end{bmatrix}\)

Therefore, \((A + B)^T = A^T + B^T\) is proved.

\subsubsection{Q.2(B).2 [4 marks]}\label{q.2b.2-4-marks}

\textbf{If
\(A = \begin{bmatrix} 2 & -1 & 0 \\ 1 & 0 & 4 \\ 1 & -1 & 1 \end{bmatrix}\)
then find \(A^{-1}\).}

\textbf{Solution}: To find \(A^{-1}\), we use the formula
\(A^{-1} = \frac{1}{|A|} \cdot \text{adj}(A)\)

First, find \(|A|\):
\(|A| = 2(0 \cdot 1 - 4 \cdot (-1)) - (-1)(1 \cdot 1 - 4 \cdot 1) + 0(1 \cdot (-1) - 0 \cdot 1)\)
\(|A| = 2(4) + 1(-3) = 8 - 3 = 5\)

Next, find cofactors:
\(C_{11} = (-1)^{1+1}\begin{vmatrix} 0 & 4 \\ -1 & 1 \end{vmatrix} = 4\)

\(C_{12} = (-1)^{1+2}\begin{vmatrix} 1 & 4 \\ 1 & 1 \end{vmatrix} = -(-3) = 3\)

\(C_{13} = (-1)^{1+3}\begin{vmatrix} 1 & 0 \\ 1 & -1 \end{vmatrix} = -1\)

\(C_{21} = (-1)^{2+1}\begin{vmatrix} -1 & 0 \\ -1 & 1 \end{vmatrix} = -(-1) = 1\)

\(C_{22} = (-1)^{2+2}\begin{vmatrix} 2 & 0 \\ 1 & 1 \end{vmatrix} = 2\)

\(C_{23} = (-1)^{2+3}\begin{vmatrix} 2 & -1 \\ 1 & -1 \end{vmatrix} = -(-1) = 1\)

\(C_{31} = (-1)^{3+1}\begin{vmatrix} -1 & 0 \\ 0 & 4 \end{vmatrix} = -4\)

\(C_{32} = (-1)^{3+2}\begin{vmatrix} 2 & 0 \\ 1 & 4 \end{vmatrix} = -(8) = -8\)

\(C_{33} = (-1)^{3+3}\begin{vmatrix} 2 & -1 \\ 1 & 0 \end{vmatrix} = 1\)

\(\text{adj}(A) = \begin{bmatrix} 4 & 1 & -4 \\ 3 & 2 & -8 \\ -1 & 1 & 1 \end{bmatrix}\)

\(A^{-1} = \frac{1}{5}\begin{bmatrix} 4 & 1 & -4 \\ 3 & 2 & -8 \\ -1 & 1 & 1 \end{bmatrix}\)

\subsubsection{Q.2(B).3 [4 marks]}\label{q.2b.3-4-marks}

\textbf{Solve the equations \(3x - y = 1, x + 2y = 5\) by matrix
method.}

\textbf{Solution}: The system can be written as \(AX = B\) where:
\(A = \begin{bmatrix} 3 & -1 \\ 1 & 2 \end{bmatrix}\),
\(X = \begin{bmatrix} x \\ y \end{bmatrix}\),
\(B = \begin{bmatrix} 1 \\ 5 \end{bmatrix}\)

\(|A| = 3(2) - (-1)(1) = 6 + 1 = 7\)

\(A^{-1} = \frac{1}{7}\begin{bmatrix} 2 & 1 \\ -1 & 3 \end{bmatrix}\)

\(X = A^{-1}B = \frac{1}{7}\begin{bmatrix} 2 & 1 \\ -1 & 3 \end{bmatrix}\begin{bmatrix} 1 \\ 5 \end{bmatrix}\)

\(X = \frac{1}{7}\begin{bmatrix} 2 + 5 \\ -1 + 15 \end{bmatrix} = \frac{1}{7}\begin{bmatrix} 7 \\ 14 \end{bmatrix} = \begin{bmatrix} 1 \\ 2 \end{bmatrix}\)

Therefore, \(x = 1\) and \(y = 2\).

\subsection*{Q.3 (A) [6 marks]}\label{q.3-a-6-marks}

\textbf{Attempt any two.}

\subsubsection{Q.3(A).1 [3 marks]}\label{q.3a.1-3-marks}

\textbf{If \(y = \frac{e^x + 1}{e^x - 1}\) then find \(\frac{dy}{dx}\).}

\textbf{Solution}: Using quotient rule:
\(\frac{d}{dx}\left(\frac{u}{v}\right) = \frac{v\frac{du}{dx} - u\frac{dv}{dx}}{v^2}\)

Let \(u = e^x + 1\) and \(v = e^x - 1\) \(\frac{du}{dx} = e^x\) and
\(\frac{dv}{dx} = e^x\)

\(\frac{dy}{dx} = \frac{(e^x - 1)(e^x) - (e^x + 1)(e^x)}{(e^x - 1)^2}\)

\(= \frac{e^{2x} - e^x - e^{2x} - e^x}{(e^x - 1)^2} = \frac{-2e^x}{(e^x - 1)^2}\)

\subsubsection{Q.3(A).2 [3 marks]}\label{q.3a.2-3-marks}

\textbf{If \(x = a\cos\theta, y = b\sin\theta\) then find
\(\frac{dy}{dx}\).}

\textbf{Solution}: \(\frac{dx}{d\theta} = -a\sin\theta\)
\(\frac{dy}{d\theta} = b\cos\theta\)

\(\frac{dy}{dx} = \frac{dy/d\theta}{dx/d\theta} = \frac{b\cos\theta}{-a\sin\theta} = -\frac{b\cos\theta}{a\sin\theta} = -\frac{b}{a}\cot\theta\)

\subsubsection{Q.3(A).3 [3 marks]}\label{q.3a.3-3-marks}

\textbf{Evaluate: \(\int \frac{\cos\sqrt{x}}{2\sqrt{x}} dx\).}

\textbf{Solution}: Let \(u = \sqrt{x}\), then
\(du = \frac{1}{2\sqrt{x}}dx\)

\(\int \frac{\cos\sqrt{x}}{2\sqrt{x}} dx = \int \cos u \, du = \sin u + C = \sin\sqrt{x} + C\)

\subsection*{Q.3 (B) [8 marks]}\label{q.3-b-8-marks}

\textbf{Attempt any two.}

\subsubsection{Q.3(B).1 [4 marks]}\label{q.3b.1-4-marks}

\textbf{Differentiate \(y = x^{\cos x}\) with respect to x.}

\textbf{Solution}: Taking natural logarithm on both sides:
\(\ln y = \cos x \ln x\)

Differentiating both sides with respect to x:
\(\frac{1}{y}\frac{dy}{dx} = \cos x \cdot \frac{1}{x} + \ln x \cdot (-\sin x)\)

\(\frac{dy}{dx} = y\left(\frac{\cos x}{x} - \sin x \ln x\right)\)

\(\frac{dy}{dx} = x^{\cos x}\left(\frac{\cos x}{x} - \sin x \ln x\right)\)

\subsubsection{Q.3(B).2 [4 marks]}\label{q.3b.2-4-marks}

\textbf{If \(y = A\cos pt + B\sin pt\), prove that
\(\frac{d^2y}{dt^2} + p^2y = 0\).}

\textbf{Solution}: \(y = A\cos pt + B\sin pt\)

\(\frac{dy}{dt} = -Ap\sin pt + Bp\cos pt\)

\(\frac{d^2y}{dt^2} = -Ap^2\cos pt - Bp^2\sin pt = -p^2(A\cos pt + B\sin pt) = -p^2y\)

Therefore: \(\frac{d^2y}{dt^2} + p^2y = -p^2y + p^2y = 0\)

\subsubsection{Q.3(B).3 [4 marks]}\label{q.3b.3-4-marks}

\textbf{The equation of motion of a particle is
\(s = t^3 + 2t^2 - 3t + 5\). Find the velocity and acceleration of the
particle at \(t = 1\) and \(t = 2\) seconds.}

\textbf{Solution}: \(s = t^3 + 2t^2 - 3t + 5\)

Velocity: \(v = \frac{ds}{dt} = 3t^2 + 4t - 3\)

Acceleration: \(a = \frac{dv}{dt} = 6t + 4\)

At \(t = 1\): \(v(1) = 3(1)^2 + 4(1) - 3 = 3 + 4 - 3 = 4\) units/sec
\(a(1) = 6(1) + 4 = 10\) units/sec^{2}

At \(t = 2\): \(v(2) = 3(2)^2 + 4(2) - 3 = 12 + 8 - 3 = 17\) units/sec
\(a(2) = 6(2) + 4 = 16\) units/sec^{2}

\subsection*{Q.4 (A) [6 marks]}\label{q.4-a-6-marks}

\textbf{Attempt any two.}

\subsubsection{Q.4(A).1 [3 marks]}\label{q.4a.1-3-marks}

\textbf{Evaluate: \(\int x \log x \, dx\).}

\textbf{Solution}: Using integration by parts:
\(\int u \, dv = uv - \int v \, du\)

Let \(u = \log x\) and \(dv = x \, dx\) Then \(du = \frac{1}{x} dx\) and
\(v = \frac{x^2}{2}\)

\(\int x \log x \, dx = \log x \cdot \frac{x^2}{2} - \int \frac{x^2}{2} \cdot \frac{1}{x} dx\)

\(= \frac{x^2 \log x}{2} - \int \frac{x}{2} dx\)

\(= \frac{x^2 \log x}{2} - \frac{x^2}{4} + C\)

\(= \frac{x^2}{2}(\log x - \frac{1}{2}) + C\)

\subsubsection{Q.4(A).2 [3 marks]}\label{q.4a.2-3-marks}

\textbf{Evaluate: \(\int_{-1}^{1} \frac{1}{1+x^2} dx\).}

\textbf{Solution}:
\(\int_{-1}^{1} \frac{1}{1+x^2} dx = [\tan^{-1} x]_{-1}^{1}\)

\(= \tan^{-1}(1) - \tan^{-1}(-1)\)

\(= \frac{\pi}{4} - \left(-\frac{\pi}{4}\right) = \frac{\pi}{2}\)

\subsubsection{Q.4(A).3 [3 marks]}\label{q.4a.3-3-marks}

\textbf{Find inverse of \(Z = 3 + 4i\).}

\textbf{Solution}: \(Z^{-1} = \frac{1}{Z} = \frac{1}{3 + 4i}\)

Multiply numerator and denominator by the conjugate:
\(Z^{-1} = \frac{1}{3 + 4i} \cdot \frac{3 - 4i}{3 - 4i} = \frac{3 - 4i}{(3)^2 + (4)^2} = \frac{3 - 4i}{9 + 16} = \frac{3 - 4i}{25}\)

\(Z^{-1} = \frac{3}{25} - \frac{4}{25}i\)

\subsection*{Q.4 (B) [8 marks]}\label{q.4-b-8-marks}

\textbf{Attempt any two.}

\subsubsection{Q.4(B).1 [4 marks]}\label{q.4b.1-4-marks}

\textbf{Evaluate:
\(\int_{0}^{\pi/2} \frac{\tan x}{\tan x + \cot x} dx\).}

\textbf{Solution}: Let
\(I = \int_{0}^{\pi/2} \frac{\tan x}{\tan x + \cot x} dx\)

Using the property \(\int_{a}^{b} f(x) dx = \int_{a}^{b} f(a+b-x) dx\):

\(I = \int_{0}^{\pi/2} \frac{\tan(\pi/2 - x)}{\tan(\pi/2 - x) + \cot(\pi/2 - x)} dx\)

\(= \int_{0}^{\pi/2} \frac{\cot x}{\cot x + \tan x} dx\)

Adding the two expressions:
\(2I = \int_{0}^{\pi/2} \frac{\tan x + \cot x}{\tan x + \cot x} dx = \int_{0}^{\pi/2} 1 \, dx = \frac{\pi}{2}\)

Therefore: \(I = \frac{\pi}{4}\)

\subsubsection{Q.4(B).2 [4 marks]}\label{q.4b.2-4-marks}

\textbf{Find the area bounded by the line \(y = x\), \(x = 5\) and the
X-axis.}

\textbf{Solution}: The region is bounded by \(y = x\), \(x = 5\), and
\(y = 0\) (X-axis).

Area =
\(\int_{0}^{5} x \, dx = \left[\frac{x^2}{2}\right]_{0}^{5} = \frac{25}{2} - 0 = \frac{25}{2}\)
square units

\subsubsection{Q.4(B).3 [4 marks]}\label{q.4b.3-4-marks}

\textbf{If \(x + iy = \left(\frac{1+i}{2-i}\right)^2\), find the value
of \(x + y\).}

\textbf{Solution}: First, simplify \(\frac{1+i}{2-i}\):
\(\frac{1+i}{2-i} \cdot \frac{2+i}{2+i} = \frac{(1+i)(2+i)}{(2-i)(2+i)} = \frac{2+i+2i+i^2}{4-i^2} = \frac{2+3i-1}{4+1} = \frac{1+3i}{5}\)

Now:
\(\left(\frac{1+3i}{5}\right)^2 = \frac{(1+3i)^2}{25} = \frac{1+6i+9i^2}{25} = \frac{1+6i-9}{25} = \frac{-8+6i}{25}\)

Therefore: \(x = -\frac{8}{25}\) and \(y = \frac{6}{25}\)

\(x + y = -\frac{8}{25} + \frac{6}{25} = -\frac{2}{25}\)

\subsection*{Q.5 (A) [6 marks]}\label{q.5-a-6-marks}

\textbf{Attempt any two.}

\subsubsection{Q.5(A).1 [3 marks]}\label{q.5a.1-3-marks}

\textbf{Find Square root of \(Z = 5 + 12i\).}

\textbf{Solution}: Let \(\sqrt{5 + 12i} = a + bi\) where
\(a, b \in \mathbb{R}\)

\((a + bi)^2 = 5 + 12i\) \(a^2 + 2abi + b^2i^2 = 5 + 12i\)
\((a^2 - b^2) + 2abi = 5 + 12i\)

Comparing real and imaginary parts: \(a^2 - b^2 = 5\) \ldots{} (1)
\(2ab = 12\) \ldots{} (2)

From (2): \(b = \frac{6}{a}\)

Substituting in (1): \(a^2 - \frac{36}{a^2} = 5\)
\(a^4 - 5a^2 - 36 = 0\)

Let \(u = a^2\): \(u^2 - 5u - 36 = 0\) \((u - 9)(u + 4) = 0\)

Since \(u = a^2 \geq 0\), we have \(u = 9\), so \(a = \pm 3\)

If \(a = 3\), then \(b = 2\) If \(a = -3\), then \(b = -2\)

Therefore: \(\sqrt{5 + 12i} = \pm(3 + 2i)\)

\subsubsection{Q.5(A).2 [3 marks]}\label{q.5a.2-3-marks}

\textbf{Find \(x, y \in \mathbb{R}\) from the equation
\((2x - y) + yi = 6 + 4i\).}

\textbf{Solution}: Comparing real and imaginary parts: Real part:
\(2x - y = 6\) \ldots{} (1) Imaginary part: \(y = 4\) \ldots{} (2)

Substituting (2) into (1): \(2x - 4 = 6\) \(2x = 10\) \(x = 5\)

Therefore: \(x = 5\) and \(y = 4\)

\subsubsection{Q.5(A).3 [3 marks]}\label{q.5a.3-3-marks}

\textbf{Find the modulus and principal argument of \(Z = 1 + i\), and
express Z into the polar form.}

\textbf{Solution}: \(Z = 1 + i\)

Modulus: \(|Z| = \sqrt{1^2 + 1^2} = \sqrt{2}\)

Principal argument:
\(\arg(Z) = \tan^{-1}\left(\frac{1}{1}\right) = \tan^{-1}(1) = \frac{\pi}{4}\)

Polar form:
\(Z = |Z|(\cos\theta + i\sin\theta) = \sqrt{2}\left(\cos\frac{\pi}{4} + i\sin\frac{\pi}{4}\right)\)

\subsection*{Q.5 (B) [8 marks]}\label{q.5-b-8-marks}

\textbf{Attempt any two.}

\subsubsection{Q.5(B).1 [4 marks]}\label{q.5b.1-4-marks}

\textbf{Solve: \(\frac{dy}{dx} = 1 + x + y + xy\).}

\textbf{Solution}:
\(\frac{dy}{dx} = 1 + x + y + xy = (1 + x) + y(1 + x) = (1 + x)(1 + y)\)

Separating variables: \(\frac{dy}{1 + y} = (1 + x) dx\)

Integrating both sides: \(\int \frac{dy}{1 + y} = \int (1 + x) dx\)

\(\ln|1 + y| = x + \frac{x^2}{2} + C\)

\(1 + y = Ae^{x + x^2/2}\) where \(A = e^C\)

\(y = Ae^{x + x^2/2} - 1\)

\subsubsection{Q.5(B).2 [4 marks]}\label{q.5b.2-4-marks}

\textbf{Solve the differential equation: \(\frac{dy}{dx} + y = e^x\).}

\textbf{Solution}: This is a first-order linear differential equation of
the form \(\frac{dy}{dx} + Py = Q\) where \(P = 1\) and \(Q = e^x\).

Integrating factor: \(I.F. = e^{\int P \, dx} = e^{\int 1 \, dx} = e^x\)

Multiplying the equation by \(e^x\):
\(e^x \frac{dy}{dx} + e^x y = e^{2x}\)

\(\frac{d}{dx}(ye^x) = e^{2x}\)

Integrating both sides: \(ye^x = \int e^{2x} dx = \frac{e^{2x}}{2} + C\)

\(y = \frac{e^x}{2} + Ce^{-x}\)

\subsubsection{Q.5(B).3 [4 marks]}\label{q.5b.3-4-marks}

\textbf{Solve the differential equation:
\(\frac{dy}{dx} - y\tan x = 1\).}

\textbf{Solution}: This is a first-order linear differential equation
where \(P = -\tan x\) and \(Q = 1\).

Integrating factor:
\(I.F. = e^{\int (-\tan x) dx} = e^{\ln|\cos x|} = \cos x\)

Multiplying the equation by \(\cos x\):
\(\cos x \frac{dy}{dx} - y\cos x \tan x = \cos x\)

\(\cos x \frac{dy}{dx} - y\sin x = \cos x\)

\(\frac{d}{dx}(y\cos x) = \cos x\)

Integrating both sides: \(y\cos

x = \int \cos x \, dx = \sin x + C\)


\(y = \tan x + \frac{C}{\cos x} = \tan x + C\sec x\)

\begin{center}\rule{0.5\linewidth}{0.5pt}\end{center}

\subsection*{Formula Cheat Sheet}\label{formula-cheat-sheet}

\subsubsection{Matrix Operations}\label{matrix-operations}

\begin{itemize}
\tightlist
\item
  \textbf{Order of Matrix}: If matrix has \(m\) rows and \(n\) columns,
  order is \(m \times n\)
\item
  \textbf{Matrix Multiplication}: \((AB)_{ij} = \sum_{k} A_{ik}B_{kj}\)
\item
  \textbf{Transpose}: \((A^T)_{ij} = A_{ji}\)
\item
  \textbf{Adjoint of 2\times2 Matrix}: If
  \(A = \begin{bmatrix} a & b \\ c & d \end{bmatrix}\), then
  \(\text{adj}(A) = \begin{bmatrix} d & -b \\ -c & a \end{bmatrix}\)
\item
  \textbf{Inverse}: \(A^{-1} = \frac{1}{|A|} \cdot \text{adj}(A)\)
\end{itemize}

\subsubsection{Differentiation}\label{differentiation}

\begin{itemize}
\tightlist
\item
  \(\frac{d}{dx}(e^x) = e^x\)
\item
  \(\frac{d}{dx}(\ln x) = \frac{1}{x}\)
\item
  \(\frac{d}{dx}(x^n) = nx^{n-1}\)
\item
  \(\frac{d}{dx}(\sin x) = \cos x\)
\item
  \(\frac{d}{dx}(\cos x) = -\sin x\)
\item
  \textbf{Chain Rule}: \(\frac{d}{dx}[f(g(x))] = f'(g(x)) \cdot g'(x)\)
\item
  \textbf{Product Rule}: \(\frac{d}{dx}(uv) = u'v + uv'\)
\item
  \textbf{Quotient Rule}:
  \(\frac{d}{dx}\left(\frac{u}{v}\right) = \frac{u'v - uv'}{v^2}\)
\item
  \textbf{Parametric}: If \(x = f(t)\) and \(y = g(t)\), then
  \(\frac{dy}{dx} = \frac{dy/dt}{dx/dt}\)
\end{itemize}

\subsubsection{Integration}\label{integration}

\begin{itemize}
\tightlist
\item
  \(\int x^n \, dx = \frac{x^{n+1}}{n+1} + C\) (for \(n \neq -1\))
\item
  \(\int e^x \, dx = e^x + C\)
\item
  \(\int \frac{1}{x} \, dx = \ln|x| + C\)
\item
  \(\int \sin x \, dx = -\cos x + C\)
\item
  \(\int \cos x \, dx = \sin x + C\)
\item
  \(\int \frac{1}{1+x^2} \, dx = \tan^{-1} x + C\)
\item
  \textbf{Integration by Parts}: \(\int u \, dv = uv - \int v \, du\)
\item
  \textbf{Definite Integration}: \(\int_a^b f(x) \, dx = F(b) - F(a)\)
  where \(F'(x) = f(x)\)
\end{itemize}

\subsubsection{Differential Equations}\label{differential-equations}

\begin{itemize}
\tightlist
\item
  \textbf{Order}: Highest derivative present
\item
  \textbf{Degree}: Power of highest derivative
\item
  \textbf{Linear DE}: \(\frac{dy}{dx} + Py = Q\)
\item
  \textbf{Integrating Factor}: \(I.F. = e^{\int P \, dx}\)
\item
  \textbf{Variable Separable}: \(\frac{dy}{dx} = f(x)g(y)\) \rightarrow
  \(\frac{dy}{g(y)} = f(x) dx\)
\end{itemize}

\subsubsection{Complex Numbers}\label{complex-numbers}

\begin{itemize}
\tightlist
\item
  \textbf{Standard Form}: \(z = a + bi\)
\item
  \textbf{Conjugate}: \(\overline{a + bi} = a - bi\)
\item
  \textbf{Modulus}: \(|a + bi| = \sqrt{a^2 + b^2}\)
\item
  \textbf{Argument}: \(\arg(z) = \tan^{-1}\left(\frac{b}{a}\right)\)
\item
  \textbf{Polar Form}: \(z = r(\cos\theta + i\sin\theta)\) where
  \(r = |z|\) and \(\theta = \arg(z)\)
\item
  \textbf{Powers of i}: \(i^1 = i\), \(i^2 = -1\), \(i^3 = -i\),
  \(i^4 = 1\)
\item
  \textbf{Inverse}: \(z^{-1} = \frac{\overline{z}}{|z|^2}\)
\end{itemize}

\subsection*{Problem-Solving
Strategies}\label{problem-solving-strategies}

\subsubsection{Matrix Problems}\label{matrix-problems}

\begin{enumerate}
\tightlist
\item
  \textbf{Check dimensions} before multiplication
\item
  \textbf{Use properties}: \((AB)^T = B^T A^T\), \((A+B)^T = A^T + B^T\)
\item
  \textbf{For inverse}: Calculate determinant first, then adjoint
\item
  \textbf{System of equations}: Write as \(AX = B\), solve
  \(X = A^{-1}B\)
\end{enumerate}

\subsubsection{Differentiation Problems}\label{differentiation-problems}

\begin{enumerate}
\tightlist
\item
  \textbf{Identify the type}: Basic, chain rule, product rule, quotient
  rule
\item
  \textbf{For implicit}: Differentiate both sides with respect to x
\item
  \textbf{For parametric}: Use \(\frac{dy}{dx} = \frac{dy/dt}{dx/dt}\)
\item
  \textbf{For logarithmic}: Take ln of both sides first
\end{enumerate}

\subsubsection{Integration Problems}\label{integration-problems}

\begin{enumerate}
\tightlist
\item
  \textbf{Check standard forms} first
\item
  \textbf{For products}: Try integration by parts (ILATE rule)
\item
  \textbf{For rational functions}: Check for substitution
\item
  \textbf{For definite integrals}: Use properties like
  \(\int_{-a}^a f(x) dx = 0\) if f(x) is odd
\end{enumerate}

\subsubsection{Differential Equations}\label{differential-equations-1}

\begin{enumerate}
\tightlist
\item
  \textbf{Identify type}: Order, degree, linear/non-linear
\item
  \textbf{For linear DE}: Find integrating factor
\item
  \textbf{For separable}: Separate variables and integrate
\item
  \textbf{Check initial conditions} if given
\end{enumerate}

\subsubsection{Complex Numbers}\label{complex-numbers-1}

\begin{enumerate}
\tightlist
\item
  \textbf{For operations}: Use standard form \(a + bi\)
\item
  \textbf{For modulus/argument}: Convert to polar form
\item
  \textbf{For powers}: Use De Moivre's theorem
\item
  \textbf{For square roots}: Let \(\sqrt{a+bi} = c+di\) and solve
\end{enumerate}

\subsection*{Common Mistakes to Avoid}\label{common-mistakes-to-avoid}

\begin{enumerate}
\tightlist
\item
  \textbf{Matrix multiplication}: Remember \(AB \neq BA\) in general
\item
  \textbf{Chain rule}: Don't forget to multiply by derivative of inner
  function
\item
  \textbf{Integration}: Remember the constant of integration
\item
  \textbf{Definite integrals}: Apply limits correctly
\item
  \textbf{Complex numbers}: \(i^2 = -1\), not \(+1\)
\item
  \textbf{Differential equations}: Don't forget integrating factor for
  linear DE
\item
  \textbf{Parametric differentiation}: Use \(\frac{dy/dt}{dx/dt}\), not
  \(\frac{dt/dy}{dt/dx}\)
\end{enumerate}

\subsection*{Exam Tips}\label{exam-tips}

\subsubsection{Time Management}\label{time-management}

\begin{itemize}
\tightlist
\item
  \textbf{Q.1 (MCQs)}: Spend 15-20 minutes maximum
\item
  \textbf{Short answers}: 3-4 minutes per question
\item
  \textbf{Long answers}: 8-10 minutes per question
\item
  \textbf{Keep 10 minutes} for final review
\end{itemize}

\subsubsection{Strategy}\label{strategy}

\begin{enumerate}
\tightlist
\item
  \textbf{Read all questions} first to identify easy ones
\item
  \textbf{Attempt easy questions} first to build confidence
\item
  \textbf{Show all steps} clearly for partial marks
\item
  \textbf{Check units} in application problems
\item
  \textbf{Verify answers} where possible (especially in matrix problems)
\end{enumerate}

\subsubsection{During Exam}\label{during-exam}

\begin{itemize}
\tightlist
\item
  \textbf{Write clearly} and organize solutions
\item
  \textbf{Draw diagrams} where helpful
\item
  \textbf{State formulas} before using them
\item
  \textbf{Don't panic} if stuck on one question - move to next
\item
  \textbf{Use remaining time} to review and check calculations
\end{itemize}

\textbf{Good Luck with your exams!}


\end{document}
