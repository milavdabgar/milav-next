\documentclass[10pt,a4paper]{article}

% content/resources/templates/preamble.tex
\usepackage[margin=0.6in]{geometry}
\author{Milav Dabgar}
\usepackage{amsmath,amssymb,amsthm}
\usepackage{booktabs}
\usepackage{multirow}
\usepackage{xcolor}
\usepackage{tcolorbox}
\tcbuselibrary{breakable,skins}
\usepackage[colorlinks=true,linkcolor=blue]{hyperref}
\usepackage{titlesec}
\usepackage{enumitem}
\usepackage{tikz}
\usepackage{pgfplots}
\usepackage{circuitikz}
\usepackage[version=4]{mhchem}
\usepackage{longtable}
\usepackage{array}
\usepackage{float}
\usepackage{caption}
\usepackage{listings}

\lstset{
  basicstyle=\small\ttfamily,
  breaklines=true,
  breakatwhitespace=false,
  postbreak=\mbox{\textcolor{red}{$\hookrightarrow$}\space},
  float=false,
  numbers=left,
  numberstyle=\tiny\color{gray},
  numbersep=10pt,
  xleftmargin=2em,
  keywordstyle=\color{blue},
  commentstyle=\color{green!60!black},
  stringstyle=\color{purple},
  backgroundcolor=\color{gray!5},
  showstringspaces=false,
  tabsize=2,
  captionpos=b,
  keepspaces=true,
  columns=flexible
}

\pgfplotsset{compat=1.18}
\usetikzlibrary{shapes,arrows,positioning,calc,patterns,decorations.pathmorphing,decorations.markings,arrows.meta}

% Color scheme
\definecolor{headcolor}{RGB}{0,102,204}
\definecolor{keycolor}{RGB}{220,20,60}
\definecolor{solutioncolor}{RGB}{34,139,34}
\definecolor{mnemoniccolor}{RGB}{148,0,211}
\definecolor{codecolor}{RGB}{0,0,100}

% Spacing
\setlength{\parskip}{3pt}
\setlist[itemize]{nosep}
\setlist[enumerate]{nosep}

% Title formatting
\titleformat{\section}{\Large\bfseries\color{headcolor}}{\thesection}{1em}{}
\titleformat{\subsection}{\large\bfseries\color{headcolor}}{\thesubsection}{1em}{}

% Pandoc tightlist compatibility
\providecommand{\tightlist}{%
  \setlength{\itemsep}{0pt}\setlength{\parskip}{0pt}}

% Pandoc longtable compatibility
\newcounter{none}
\def\thenone{}


% content/resources/templates/english-boxes.tex
% This file is currently empty - it exists to maintain consistency with the import structure.
% Add custom environments here if needed in the future.


\begin{document}

\begin{center}
{\Huge\bfseries\color{headcolor} Subject Name Solutions}\\[5pt]
{\LARGE 4320002 -- Winter 2022}\\[3pt]
{\large Semester 1 Study Material}\\[3pt]
{\normalsize\textit{Detailed Solutions and Explanations}}
\end{center}

\vspace{10pt}

\subsection*{Q.1 [14 marks]}\label{q.1-14-marks}

\textbf{Fill in the blanks using appropriate choice from the given
options.}

\subsubsection{Q1.1 [1 mark]}\label{q1.1-1-mark}

\textbf{If A = \(\begin{bmatrix} 1 & -2 \\ 2 & -1 \end{bmatrix}\) then
adj.A = \_\_\_\_\_\_.}

\begin{solutionbox}
(d) \(\begin{bmatrix} -1 & -2 \\ -2 & 1 \end{bmatrix}\)

\textbf{Solution}: For a 2\times2 matrix
\(A = \begin{bmatrix} a & b \\ c & d \end{bmatrix}\), adj.A =
\(\begin{bmatrix} d & -b \\ -c & a \end{bmatrix}\)

\(adj.A = \begin{bmatrix} -1 & 2 \\ -2 & 1 \end{bmatrix}\)

\end{solutionbox}
\subsubsection{Q1.2 [1 mark]}\label{q1.2-1-mark}

\textbf{If A is 2\times3 and B is 3\times4 matrices then AB is \_\_\_\_\_\_
matrix}

\begin{solutionbox}
(b) 2\times4

\textbf{Solution}: Matrix multiplication rule:
\((m \times n) \times (n \times p) = (m \times p)\)
\((2 \times 3) \times (3 \times 4) = (2 \times 4)\)

\end{solutionbox}
\subsubsection{Q1.3 [1 mark]}\label{q1.3-1-mark}

\textbf{If
\(\begin{bmatrix} 0 & x \\ -2 & 4 \end{bmatrix} = \begin{bmatrix} 0 & 4 \\ -2 & 4 \end{bmatrix}\)
then x = \_\_\_\_\_\_\_}

\begin{solutionbox}
(b) 4

\textbf{Solution}: Comparing corresponding elements: \(x = 4\)

\end{solutionbox}
\subsubsection{Q1.4 [1 mark]}\label{q1.4-1-mark}

\textbf{If A is non singular matrix then \_\_\_\_\_\_}

\begin{solutionbox}
(d) \(|A| \neq 0\)

\textbf{Solution}: A matrix is non-singular if its determinant is
non-zero.

\end{solutionbox}
\subsubsection{Q1.5 [1 mark]}\label{q1.5-1-mark}

\textbf{\$\frac{d}{dx}(e\^{}\{-\log x\}) = \$
\_\_\_\_\_\_\_\_\_\_\_\_\_\_}

\begin{solutionbox}
(d) x

\textbf{Solution}:
\(e^{-\log x} = e^{\log x^{-1}} = x^{-1} = \frac{1}{x}\)
\(\frac{d}{dx}(\frac{1}{x}) = -\frac{1}{x^2}\)

\end{solutionbox}
\subsubsection{Q1.6 [1 mark]}\label{q1.6-1-mark}

\textbf{If \(f(x) = \log\sqrt{x^2 + 1}\), then \$f'(0) = \$
\_\_\_\_\_\_\_\_\_\_\_\_}

\begin{solutionbox}
(a) 0

\textbf{Solution}: \(f(x) = \frac{1}{2}\log(x^2 + 1)\)
\(f'(x) = \frac{1}{2} \cdot \frac{2x}{x^2 + 1} = \frac{x}{x^2 + 1}\)
\(f'(0) = \frac{0}{0 + 1} = 0\)

\end{solutionbox}
\subsubsection{Q1.7 [1 mark]}\label{q1.7-1-mark}

\textbf{If \(x = \sec\theta + \tan\theta\) and
\(y = \sec\theta - \tan\theta\) then \$\frac{dy}{dx} = \$
\_\_\_\_\_\_\_\_\_\_\_\_}

\begin{solutionbox}
(d) 1

\textbf{Solution}:
\(xy = (\sec\theta + \tan\theta)(\sec\theta - \tan\theta) = \sec^2\theta - \tan^2\theta = 1\)
Differentiating: \(x\frac{dy}{dx} + y = 0\)
\(\frac{dy}{dx} = -\frac{y}{x}\)

\end{solutionbox}
\subsubsection{Q1.8 [1 mark]}\label{q1.8-1-mark}

\textbf{\$\int e\^{}x(\sin x + \cos x)dx = \$ \_\_\_\_\_\_\_\_\_}

\begin{solutionbox}
(b) \(e^x\sin x + c\)

\textbf{Solution}: Using integration by parts or standard result:
\(\int e^x(\sin x + \cos x)dx = e^x\sin x + c\)

\end{solutionbox}
\subsubsection{Q1.9 [1 mark]}\label{q1.9-1-mark}

\textbf{\$\int\emph{\{-1\}\^{}\{1\} x\^{}2 + 1 dx = \$ }\_\_\_\_\_}

\begin{solutionbox}
(d) \(\frac{8}{3}\)

\textbf{Solution}:
\(\int_{-1}^{1} (x^2 + 1)dx = [\frac{x^3}{3} + x]_{-1}^{1}\)
\(= (\frac{1}{3} + 1) - (\frac{-1}{3} - 1) = \frac{8}{3}\)

\end{solutionbox}
\subsubsection{Q1.10 [1 mark]}\label{q1.10-1-mark}

\textbf{\$\int \cot x dx = \$ \_\_\_\_\_\_\_\_\_\_\_\_ + c}

\begin{solutionbox}
(a) \(\log|\sin x|\)

\textbf{Solution}:
\(\int \cot x dx = \int \frac{\cos x}{\sin x} dx = \log|\sin x| + c\)

\end{solutionbox}
\subsubsection{Q1.11 [1 mark]}\label{q1.11-1-mark}

\textbf{The order \& degree of the differential equation
\(\frac{d^2y}{dx^2} + x\frac{dy}{dx} + 3y = 0\) are
respectively\_\_\_\_\_\_\_ and \_\_\_\_\_\_\_}

\begin{solutionbox}
(a) 2, 1

\textbf{Solution}: Order = highest order derivative = 2 Degree = power
of highest order derivative = 1

\end{solutionbox}
\subsubsection{Q1.12 [1 mark]}\label{q1.12-1-mark}

\textbf{The integrating factor for the differential equation
\(\frac{dy}{dx} + \frac{y}{x} = x\) is \_\_\_\_}

\begin{solutionbox}
(b) \(x\)

\textbf{Solution}: For \(\frac{dy}{dx} + P(x)y = Q(x)\), where
\(P(x) = \frac{1}{x}\) I.F. =
\(e^{\int P(x)dx} = e^{\int \frac{1}{x}dx} = e^{\log x} = x\)

\end{solutionbox}
\subsubsection{Q1.13 [1 mark]}\label{q1.13-1-mark}

\textbf{\$i + i\^{}2 + i\^{}3 + i\^{}4 = \$ \_\_\_\_\_\_}

\begin{solutionbox}
(d) 0

\textbf{Solution}: \(i + i^2 + i^3 + i^4 = i + (-1) + (-i) + 1 = 0\)

\end{solutionbox}
\subsubsection{Q1.14 [1 mark]}\label{q1.14-1-mark}

\textbf{arg(-1) = \_\_\_\_\_\_\_\_\_\_\_}

\begin{solutionbox}
(a) π

\textbf{Solution}: \(-1 = \cos\pi + i\sin\pi\), so \(\arg(-1) = \pi\)

\end{solutionbox}
\subsection*{Q.2(a) [6 marks]}\label{q.2a-6-marks}

\textbf{Attempt any two.}

\subsubsection{Q2(a).1 [3 marks]}\label{q2a.1-3-marks}

\textbf{If \(A = \begin{bmatrix} 1 & 2 \\ -3 & 2 \end{bmatrix}\),
\(B = \begin{bmatrix} 5 & 6 \\ -2 & 3 \end{bmatrix}\) then find matrix X
from equation 3(X+B) + 5A = 0}

\textbf{Solution}: \(3(X + B) + 5A = 0\) \(3X + 3B + 5A = 0\)
\(3X = -3B - 5A\) \(X = -B - \frac{5A}{3}\)

\(5A = 5\begin{bmatrix} 1 & 2 \\ -3 & 2 \end{bmatrix} = \begin{bmatrix} 5 & 10 \\ -15 & 10 \end{bmatrix}\)

\(X = -\begin{bmatrix} 5 & 6 \\ -2 & 3 \end{bmatrix} - \frac{1}{3}\begin{bmatrix} 5 & 10 \\ -15 & 10 \end{bmatrix}\)

\(X = \begin{bmatrix} -5 & -6 \\ 2 & -3 \end{bmatrix} - \begin{bmatrix} \frac{5}{3} & \frac{10}{3} \\ -5 & \frac{10}{3} \end{bmatrix}\)

\(X = \begin{bmatrix} -\frac{20}{3} & -\frac{28}{3} \\ 7 & -\frac{19}{3} \end{bmatrix}\)

\subsubsection{Q2(a).2 [3 marks]}\label{q2a.2-3-marks}

\textbf{If \(A = \begin{bmatrix} 1 & 2 \\ 2 & 1 \end{bmatrix}\) then
Prove that \(A^2 - 4A - 5I = 0\)}

\textbf{Solution}:
\(A^2 = \begin{bmatrix} 1 & 2 \\ 2 & 1 \end{bmatrix}\begin{bmatrix} 1 & 2 \\ 2 & 1 \end{bmatrix} = \begin{bmatrix} 5 & 4 \\ 4 & 5 \end{bmatrix}\)

\(4A = 4\begin{bmatrix} 1 & 2 \\ 2 & 1 \end{bmatrix} = \begin{bmatrix} 4 & 8 \\ 8 & 4 \end{bmatrix}\)

\(5I = 5\begin{bmatrix} 1 & 0 \\ 0 & 1 \end{bmatrix} = \begin{bmatrix} 5 & 0 \\ 0 & 5 \end{bmatrix}\)

\(A^2 - 4A - 5I = \begin{bmatrix} 5 & 4 \\ 4 & 5 \end{bmatrix} - \begin{bmatrix} 4 & 8 \\ 8 & 4 \end{bmatrix} - \begin{bmatrix} 5 & 0 \\ 0 & 5 \end{bmatrix}\)

\(= \begin{bmatrix} 0 & -4 \\ -4 & 0 \end{bmatrix} - \begin{bmatrix} 5 & 0 \\ 0 & 5 \end{bmatrix} = \begin{bmatrix} 0 & 0 \\ 0 & 0 \end{bmatrix}\)

Hence proved.

\subsubsection{Q2(a).3 [3 marks]}\label{q2a.3-3-marks}

\textbf{Solve differential equation \(\frac{dy}{dx} = (x + y)^2\)}

\textbf{Solution}: Let \(v = x + y\), then
\(\frac{dv}{dx} = 1 + \frac{dy}{dx}\)
\(\frac{dy}{dx} = \frac{dv}{dx} - 1\)

Substituting: \(\frac{dv}{dx} - 1 = v^2\) \(\frac{dv}{dx} = v^2 + 1\)
\(\frac{dv}{v^2 + 1} = dx\)

Integrating: \(\int \frac{dv}{v^2 + 1} = \int dx\)
\(\tan^{-1}v = x + c\) \(\tan^{-1}(x + y) = x + c\)
\(x + y = \tan(x + c)\) \(y = \tan(x + c) - x\)

\subsection*{Q.2(b) [8 marks]}\label{q.2b-8-marks}

\textbf{Attempt any two.}

\subsubsection{Q2(b).1 [4 marks]}\label{q2b.1-4-marks}

\textbf{If
\(A = \begin{bmatrix} 3 & -1 \\ 4 & 1 \\ 5 & 0 \end{bmatrix}\) then find
\(A^{-1}\)}

\textbf{Solution}: This is a 3\times2 matrix, which is non-square. Inverse
doesn't exist for non-square matrices.

\textbf{Alternative interpretation - if it's
\(\begin{bmatrix} 3 & -1 & 2 \\ 4 & 1 & -1 \\ 5 & 0 & 1 \end{bmatrix}\):}

Using adjoint method:
\(|A| = 3(1-0) + 1(4+5) + 2(0-5) = 3 + 9 - 10 = 2\)

Calculate cofactors and adjoint, then
\(A^{-1} = \frac{1}{|A|} \times adj(A)\)

\subsubsection{Q2(b).2 [4 marks]}\label{q2b.2-4-marks}

\textbf{Solve Equation 3X-2Y=8 and 5X+4Y=6 using matrices method.}

\textbf{Solution}:
\(\begin{bmatrix} 3 & -2 \\ 5 & 4 \end{bmatrix}\begin{bmatrix} X \\ Y \end{bmatrix} = \begin{bmatrix} 8 \\ 6 \end{bmatrix}\)

\(|A| = 3(4) - (-2)(5) = 12 + 10 = 22\)

\(A^{-1} = \frac{1}{22}\begin{bmatrix} 4 & 2 \\ -5 & 3 \end{bmatrix}\)

\(\begin{bmatrix} X \\ Y \end{bmatrix} = \frac{1}{22}\begin{bmatrix} 4 & 2 \\ -5 & 3 \end{bmatrix}\begin{bmatrix} 8 \\ 6 \end{bmatrix}\)

\(\begin{bmatrix} X \\ Y \end{bmatrix} = \frac{1}{22}\begin{bmatrix} 32 + 12 \\ -40 + 18 \end{bmatrix} = \frac{1}{22}\begin{bmatrix} 44 \\ -22 \end{bmatrix}\)

\(X = 2, Y = -1\)

\subsubsection{Q2(b).3 [4 marks]}\label{q2b.3-4-marks}

\textbf{If \(M = \begin{bmatrix} 2 & 3 \\ 0 & 1 \end{bmatrix}\),
\(N = \begin{bmatrix} 3 & 4 \\ 2 & 1 \end{bmatrix}\) then Prove that
\((MN)^T = N^T M^T\)}

\textbf{Solution}:
\(MN = \begin{bmatrix} 2 & 3 \\ 0 & 1 \end{bmatrix}\begin{bmatrix} 3 & 4 \\ 2 & 1 \end{bmatrix} = \begin{bmatrix} 12 & 11 \\ 2 & 1 \end{bmatrix}\)

\((MN)^T = \begin{bmatrix} 12 & 2 \\ 11 & 1 \end{bmatrix}\)

\(M^T = \begin{bmatrix} 2 & 0 \\ 3 & 1 \end{bmatrix}\),
\(N^T = \begin{bmatrix} 3 & 2 \\ 4 & 1 \end{bmatrix}\)

\(N^T M^T = \begin{bmatrix} 3 & 2 \\ 4 & 1 \end{bmatrix}\begin{bmatrix} 2 & 0 \\ 3 & 1 \end{bmatrix} = \begin{bmatrix} 12 & 2 \\ 11 & 1 \end{bmatrix}\)

Hence \((MN)^T = N^T M^T\) is proved.

\subsection*{Q.3(a) [6 marks]}\label{q.3a-6-marks}

\textbf{Attempt any two.}

\subsubsection{Q3(a).1 [3 marks]}\label{q3a.1-3-marks}

\textbf{Differentiate \(\sqrt{x}\) using the definition.}

\textbf{Solution}: \(f(x) = \sqrt{x} = x^{1/2}\)

Using definition: \(f'(x) = \lim_{h \to 0} \frac{f(x+h) - f(x)}{h}\)

\(f'(x) = \lim_{h \to 0} \frac{\sqrt{x+h} - \sqrt{x}}{h}\)

Rationalizing:
\(f'(x) = \lim_{h \to 0} \frac{(\sqrt{x+h} - \sqrt{x})(\sqrt{x+h} + \sqrt{x})}{h(\sqrt{x+h} + \sqrt{x})}\)

\(= \lim_{h \to 0} \frac{(x+h) - x}{h(\sqrt{x+h} + \sqrt{x})} = \lim_{h \to 0} \frac{h}{h(\sqrt{x+h} + \sqrt{x})}\)

\(= \lim_{h \to 0} \frac{1}{\sqrt{x+h} + \sqrt{x}} = \frac{1}{2\sqrt{x}}\)

\subsubsection{Q3(a).2 [3 marks]}\label{q3a.2-3-marks}

\textbf{If \(y = \log(x + \sqrt{1 + x^2})\) then Find \(\frac{dy}{dx}\)}

\textbf{Solution}: \(y = \log(x + \sqrt{1 + x^2})\)

\(\frac{dy}{dx} = \frac{1}{x + \sqrt{1 + x^2}} \cdot \frac{d}{dx}(x + \sqrt{1 + x^2})\)

\(\frac{d}{dx}(x + \sqrt{1 + x^2}) = 1 + \frac{1}{2\sqrt{1 + x^2}} \cdot 2x = 1 + \frac{x}{\sqrt{1 + x^2}}\)

\(= \frac{\sqrt{1 + x^2} + x}{\sqrt{1 + x^2}}\)

\(\frac{dy}{dx} = \frac{1}{x + \sqrt{1 + x^2}} \cdot \frac{\sqrt{1 + x^2} + x}{\sqrt{1 + x^2}}\)

\(= \frac{1}{\sqrt{1 + x^2}}\)

\subsubsection{Q3(a).3 [3 marks]}\label{q3a.3-3-marks}

\textbf{\(\int \frac{4 + 3\cos x}{\sin^2 x} dx\)}

\textbf{Solution}:
\(\int \frac{4 + 3\cos x}{\sin^2 x} dx = \int \frac{4}{\sin^2 x} dx + \int \frac{3\cos x}{\sin^2 x} dx\)

\(= 4\int \csc^2 x dx + 3\int \frac{\cos x}{\sin^2 x} dx\)

\(= -4\cot x + 3\int \sin^{-2} x \cos x dx\)

For the second integral, let \(u = \sin x\), \(du = \cos x dx\)
\(3\int u^{-2} du = 3(-u^{-1}) = -\frac{3}{\sin x}\)

\(\int \frac{4 + 3\cos x}{\sin^2 x} dx = -4\cot x - 3\csc x + c\)

\subsection*{Q.3(b) [8 marks]}\label{q.3b-8-marks}

\textbf{Attempt any two.}

\subsubsection{Q3(b).1 [4 marks]}\label{q3b.1-4-marks}

\textbf{If \(y = \log(\sin x)\) then prove that
\(\frac{d^2y}{dx^2} + (\frac{dy}{dx})^2 + 1 = 0\)}

\textbf{Solution}: \(y = \log(\sin x)\)

\(\frac{dy}{dx} = \frac{1}{\sin x} \cdot \cos x = \cot x\)

\(\frac{d^2y}{dx^2} = \frac{d}{dx}(\cot x) = -\csc^2 x\)

Now,
\(\frac{d^2y}{dx^2} + (\frac{dy}{dx})^2 + 1 = -\csc^2 x + \cot^2 x + 1\)

Using identity: \(\csc^2 x - \cot^2 x = 1\)
\(-\csc^2 x + \cot^2 x + 1 = -(\csc^2 x - \cot^2 x) = -1 + 1 = 0\)

Hence proved.

\subsubsection{Q3(b).2 [4 marks]}\label{q3b.2-4-marks}

\textbf{If \(x + y = \sin(xy)\) then Find \(\frac{dy}{dx}\)}

\textbf{Solution}: \(x + y = \sin(xy)\)

Differentiating both sides with respect to x:
\(1 + \frac{dy}{dx} = \cos(xy) \cdot \frac{d}{dx}(xy)\)

\(1 + \frac{dy}{dx} = \cos(xy) \cdot (y + x\frac{dy}{dx})\)

\(1 + \frac{dy}{dx} = y\cos(xy) + x\cos(xy)\frac{dy}{dx}\)

\(1 + \frac{dy}{dx} - x\cos(xy)\frac{dy}{dx} = y\cos(xy)\)

\(\frac{dy}{dx}(1 - x\cos(xy)) = y\cos(xy) - 1\)

\(\frac{dy}{dx} = \frac{y\cos(xy) - 1}{1 - x\cos(xy)}\)

\subsubsection{Q3(b).3 [4 marks]}\label{q3b.3-4-marks}

\textbf{A particle has motion of \(s = t^3 - 5t^2 + 3t\) Find the
acceleration when particle comes to rest?}

\textbf{Solution}: Given: \(s = t^3 - 5t^2 + 3t\)

Velocity: \(v = \frac{ds}{dt} = 3t^2 - 10t + 3\)

Acceleration: \(a = \frac{dv}{dt} = 6t - 10\)

At rest, \(v = 0\): \(3t^2 - 10t + 3 = 0\)

Using quadratic formula:
\(t = \frac{10 \pm \sqrt{100 - 36}}{6} = \frac{10 \pm 8}{6}\)

\(t = 3\) or \(t = \frac{1}{3}\)

At \(t = 3\): \(a = 6(3) - 10 = 8\) At \(t = \frac{1}{3}\):
\(a = 6(\frac{1}{3}) - 10 = -8\)

The accelerations are \(8\) and \(-8\) respectively.

\subsection*{Q.4(a) [6 marks]}\label{q.4a-6-marks}

\textbf{Attempt any two.}

\subsubsection{Q4(a).1 [3 marks]}\label{q4a.1-3-marks}

\textbf{\(\int x \sin x dx\)}

\textbf{Solution}: Using integration by parts:
\(\int u dv = uv - \int v du\)

Let \(u = x\), \(dv = \sin x dx\) \(du = dx\), \(v = -\cos x\)

\(\int x \sin x dx = x(-\cos x) - \int (-\cos x) dx\)
\(= -x\cos x + \int \cos x dx\) \(= -x\cos x + \sin x + c\)

\subsubsection{Q4(a).2 [3 marks]}\label{q4a.2-3-marks}

\textbf{\(\int \frac{2x + 1}{(x + 1)(x - 3)} dx\)}

\textbf{Solution}: Using partial fractions:
\(\frac{2x + 1}{(x + 1)(x - 3)} = \frac{A}{x + 1} + \frac{B}{x - 3}\)

\(2x + 1 = A(x - 3) + B(x + 1)\)

At \(x = -1\): \(-2 + 1 = A(-4) \Rightarrow

A = \frac{1}{4}\) At

\(x = 3\): \(6 + 1 = B(4) \Rightarrow B = \frac{7}{4}\)

\(\int \frac{2x + 1}{(x + 1)(x - 3)} dx = \frac{1}{4}\int \frac{1}{x + 1} dx + \frac{7}{4}\int \frac{1}{x - 3} dx\)

\(= \frac{1}{4}\log|x + 1| + \frac{7}{4}\log|x - 3| + c\)

\subsubsection{Q4(a).3 [3 marks]}\label{q4a.3-3-marks}

\textbf{Find square root of complex number \(z = 7 + 24i\)}

\textbf{Solution}: Let \(\sqrt{7 + 24i} = a + bi\)

\((a + bi)^2 = 7 + 24i\) \(a^2 - b^2 + 2abi = 7 + 24i\)

Comparing: \(a^2 - b^2 = 7\) and \(2ab = 24\) From second equation:
\(b = \frac{12}{a}\)

Substituting: \(a^2 - \frac{144}{a^2} = 7\) \(a^4 - 7a^2 - 144 = 0\)

Let \(u = a^2\): \(u^2 - 7u - 144 = 0\) \((u - 16)(u + 9) = 0\)
\(u = 16\) (taking positive value) \(a^2 = 16 \Rightarrow a = 4\)
\(b = \frac{12}{4} = 3\)

Therefore: \(\sqrt{7 + 24i} = 4 + 3i\) or \(-(4 + 3i)\)

\subsection*{Q.4(b) [8 marks]}\label{q.4b-8-marks}

\textbf{Attempt any two.}

\subsubsection{Q4(b).1 [4 marks]}\label{q4b.1-4-marks}

\textbf{\(\int_0^{\pi/2} \frac{\sqrt{\sin x}}{\sqrt{\sin x} + \sqrt{\cos x}} dx\)}

\textbf{Solution}: Let
\(I = \int_0^{\pi/2} \frac{\sqrt{\sin x}}{\sqrt{\sin x} + \sqrt{\cos x}} dx\)

Using property: \(\int_0^a f(x) dx = \int_0^a f(a-x) dx\)

\(I = \int_0^{\pi/2} \frac{\sqrt{\sin(\pi/2 - x)}}{\sqrt{\sin(\pi/2 - x)} + \sqrt{\cos(\pi/2 - x)}} dx\)

\(= \int_0^{\pi/2} \frac{\sqrt{\cos x}}{\sqrt{\cos x} + \sqrt{\sin x}} dx\)

Adding both expressions:
\(2I = \int_0^{\pi/2} \frac{\sqrt{\sin x} + \sqrt{\cos x}}{\sqrt{\sin x} + \sqrt{\cos x}} dx = \int_0^{\pi/2} 1 dx = \frac{\pi}{2}\)

Therefore: \(I = \frac{\pi}{4}\)

\subsubsection{Q4(b).2 [4 marks]}\label{q4b.2-4-marks}

\textbf{Find the area of the region bounded by the curve \(y = 3x^2\), x
axis and the line \(x = 2\) and \(x = 3\)}

\textbf{Solution}: Area = \(\int_2^3 y dx = \int_2^3 3x^2 dx\)

\(= 3\int_2^3 x^2 dx = 3[\frac{x^3}{3}]_2^3\)

\(= [x^3]_2^3 = 3^3 - 2^3 = 27 - 8 = 19\)

Area = 19 square units

\subsubsection{Q4(b).3 [4 marks]}\label{q4b.3-4-marks}

\textbf{Simplify
\(\frac{(\cos 2\theta + i\sin 2\theta)^{-3} \cdot (\cos 3\theta - i\sin 3\theta)^2}{(\cos 2\theta - i\sin 2\theta)^{-7} \cdot (\cos 5\theta - i\sin 5\theta)^3}\)}

\textbf{Solution}: Using Euler's formula:
\(\cos\theta + i\sin\theta = e^{i\theta}\)

\((\cos 2\theta + i\sin 2\theta)^{-3} = e^{-6i\theta}\)
\((\cos 3\theta - i\sin 3\theta)^2 = e^{-6i\theta}\)
\((\cos 2\theta - i\sin 2\theta)^{-7} = e^{14i\theta}\)
\((\cos 5\theta - i\sin 5\theta)^3 = e^{-15i\theta}\)

Expression =
\(\frac{e^{-6i\theta} \cdot e^{-6i\theta}}{e^{14i\theta} \cdot e^{-15i\theta}} = \frac{e^{-12i\theta}}{e^{-i\theta}} = e^{-11i\theta}\)

\(= \cos(-11\theta) + i\sin(-11\theta) = \cos(11\theta) - i\sin(11\theta)\)

\subsection*{Q.5(a) [6 marks]}\label{q.5a-6-marks}

\textbf{Attempt any two.}

\subsubsection{Q5(a).1 [3 marks]}\label{q5a.1-3-marks}

\textbf{Convert \(\frac{4+2i}{(3+2i)(5-3i)}\) in a+ib form.}

\textbf{Solution}: First, simplify the denominator:
\((3+2i)(5-3i) = 15 - 9i + 10i - 6i^2 = 15 + i + 6 = 21 + i\)

Now: \(\frac{4+2i}{21+i}\)

Multiply by conjugate: \(\frac{4+2i}{21+i} \cdot \frac{21-i}{21-i}\)

\(= \frac{(4+2i)(21-i)}{(21+i)(21-i)} = \frac{84 - 4i + 42i - 2i^2}{441 - i^2}\)

\(= \frac{84 + 38i + 2}{441 + 1} = \frac{86 + 38i}{442} = \frac{43 + 19i}{221}\)

\subsubsection{Q5(a).2 [3 marks]}\label{q5a.2-3-marks}

\textbf{Convert \(z = 1 - \sqrt{3}i\) in polar form.}

\textbf{Solution}: \(z = 1 - \sqrt{3}i\)

\(|z| = \sqrt{1^2 + (-\sqrt{3})^2} = \sqrt{1 + 3} = 2\)

\(\arg(z) = \tan^{-1}\left(\frac{-\sqrt{3}}{1}\right) = -\frac{\pi}{3}\)
(since z is in 4th quadrant)

Therefore:
\(z = 2(\cos(-\frac{\pi}{3}) + i\sin(-\frac{\pi}{3})) = 2e^{-i\pi/3}\)

\subsubsection{Q5(a).3 [3 marks]}\label{q5a.3-3-marks}

\textbf{Prove that
\((1 + \cos\theta + i\sin\theta)^n + (1 + \cos\theta - i\sin\theta)^n = 2^{n+1}\cos^n(\frac{\theta}{2})\cos(\frac{n\theta}{2})\)}

\textbf{Solution}:
\(1 + \cos\theta + i\sin\theta = 1 + e^{i\theta} = 1 + \cos\theta + i\sin\theta\)

Using identity: \(1 + \cos\theta = 2\cos^2(\frac{\theta}{2})\)

\(1 + \cos\theta + i\sin\theta = 2\cos^2(\frac{\theta}{2}) + 2i\sin(\frac{\theta}{2})\cos(\frac{\theta}{2})\)

\(= 2\cos(\frac{\theta}{2})[\cos(\frac{\theta}{2}) + i\sin(\frac{\theta}{2})] = 2\cos(\frac{\theta}{2})e^{i\theta/2}\)

Similarly:
\(1 + \cos\theta - i\sin\theta = 2\cos(\frac{\theta}{2})e^{-i\theta/2}\)

\((1 + \cos\theta + i\sin\theta)^n = 2^n\cos^n(\frac{\theta}{2})e^{in\theta/2}\)

\((1 + \cos\theta - i\sin\theta)^n = 2^n\cos^n(\frac{\theta}{2})e^{-in\theta/2}\)

Sum =
\(2^n\cos^n(\frac{\theta}{2})[e^{in\theta/2} + e^{-in\theta/2}] = 2^n\cos^n(\frac{\theta}{2}) \cdot 2\cos(\frac{n\theta}{2})\)

\(= 2^{n+1}\cos^n(\frac{\theta}{2})\cos(\frac{n\theta}{2})\)

Hence proved.

\subsection*{Q.5(b) [8 marks]}\label{q.5b-8-marks}

\textbf{Attempt any two.}

\subsubsection{Q5(b).1 [4 marks]}\label{q5b.1-4-marks}

\textbf{Solve differential equation
\(x\log x \frac{dy}{dx} + y = \log x^2\)}

\textbf{Solution}: \(x\log x \frac{dy}{dx} + y = 2\log x\)

Dividing by \(x\log x\):
\(\frac{dy}{dx} + \frac{y}{x\log x} = \frac{2}{x}\)

This is a linear differential equation: \(\frac{dy}{dx} + P(x)y = Q(x)\)

Where \(P(x) = \frac{1}{x\log x}\) and \(Q(x) = \frac{2}{x}\)

\textbf{Integrating Factor}:
\(e^{\int P(x)dx} = e^{\int \frac{1}{x\log x}dx}\)

Let \(u = \log x\), then \(du = \frac{1}{x}dx\)
\(\int \frac{1}{x\log x}dx = \int \frac{1}{u}du = \log u = \log(\log x)\)

I.F. = \(e^{\log(\log x)} = \log x\)

\textbf{Solution}: \(y \cdot \log x = \int \frac{2}{x} \cdot \log x dx\)

\(= 2\int \frac{\log x}{x} dx = 2 \cdot \frac{(\log x)^2}{2} = (\log x)^2\)

Therefore: \(y = \frac{(\log x)^2}{\log x} = \log x\)

\subsubsection{Q5(b).2 [4 marks]}\label{q5b.2-4-marks}

\textbf{Solve differential equation
\(\frac{dy}{dx} - \frac{y}{x} = e^x\)}

\textbf{Solution}: This is a linear differential equation:
\(\frac{dy}{dx} + P(x)y = Q(x)\)

Where \(P(x) = -\frac{1}{x}\) and \(Q(x) = e^x\)

\textbf{Integrating Factor}:
\(e^{\int P(x)dx} = e^{\int -\frac{1}{x}dx} = e^{-\log x} = \frac{1}{x}\)

\textbf{Solution}:
\(y \cdot \frac{1}{x} = \int e^x \cdot \frac{1}{x} dx\)

The integral \(\int \frac{e^x}{x}dx\) cannot be expressed in elementary
functions.

\textbf{Alternative approach - assuming it's}
\(\frac{dy}{dx} + \frac{y}{x} = e^x\):

I.F. = \(e^{\int \frac{1}{x}dx} = e^{\log x} = x\)

\(y \cdot x = \int e^x \cdot x dx\)

Using integration by parts:
\(\int xe^x dx = xe^x - \int e^x dx = xe^x - e^x = e^x(x-1)\)

Therefore: \(xy = e^x(x-1) + c\) \(y = \frac{e^x(x-1) + c}{x}\)

\subsubsection{Q5(b).3 [4 marks]}\label{q5b.3-4-marks}

\textbf{Solve differential equation
\(\sec^2x \tan y dx + \sec^2y \tan x dy = 0\),
\(y(\frac{\pi}{4}) = \frac{\pi}{4}\)}

\textbf{Solution}: \(\sec^2x \tan y dx + \sec^2y \tan x dy = 0\)

Rearranging: \(\frac{\sec^2x}{\tan x}dx + \frac{\sec^2y}{\tan y}dy = 0\)

\(\frac{\cos x}{\sin x \cos^2 x}dx + \frac{\cos y}{\sin y \cos^2 y}dy = 0\)

\(\frac{1}{\sin x \cos x}dx + \frac{1}{\sin y \cos y}dy = 0\)

\(\frac{2}{\sin 2x}dx + \frac{2}{\sin 2y}dy = 0\)

\(\csc(2x)dx + \csc(2y)dy = 0\)

Integrating: \(\int \csc(2x)dx + \int \csc(2y)dy = c\)

\(-\frac{1}{2}\log|\csc(2x) + \cot(2x)| - \frac{1}{2}\log|\csc(2y) + \cot(2y)| = c\)

\(\log|\csc(2x) + \cot(2x)| + \log|\csc(2y) + \cot(2y)| = -2c = k\)

\(|\csc(2x) + \cot(2x)| \cdot |\csc(2y) + \cot(2y)| = e^k\)

Using initial condition \(y(\frac{\pi}{4}) = \frac{\pi}{4}\): At
\(x = \frac{\pi}{4}\), \(y = \frac{\pi}{4}\)

\(|\csc(\frac{\pi}{2}) + \cot(\frac{\pi}{2})| \cdot |\csc(\frac{\pi}{2}) + \cot(\frac{\pi}{2})| = |1 + 0| \cdot |1 + 0| = 1\)

Therefore: \((\csc(2x) + \cot(2x))(\csc(2y) + \cot(2y)) = 1\)

\begin{center}\rule{0.5\linewidth}{0.5pt}\end{center}

\subsection*{Complete Formula Cheat
Sheet}\label{complete-formula-cheat-sheet}

\subsubsection{\texorpdfstring{\textbf{Matrix
Operations}}{Matrix Operations}}\label{matrix-operations}

\begin{longtable}[]{@{}
  >{\raggedright\arraybackslash}p{(\linewidth - 2\tabcolsep) * \real{0.5500}}
  >{\raggedright\arraybackslash}p{(\linewidth - 2\tabcolsep) * \real{0.4500}}@{}}
\toprule\noalign{}
\begin{minipage}[b]{\linewidth}\raggedright
Operation
\end{minipage} & \begin{minipage}[b]{\linewidth}\raggedright
Formula
\end{minipage} \\
\midrule\noalign{}
\endhead
\bottomrule\noalign{}
\endlastfoot
Adjoint (2\times2) & If \(A = \begin{bmatrix} a & b \\ c & d \end{bmatrix}\),
then \(adj(A) = \begin{bmatrix} d & -b \\ -c & a \end{bmatrix}\) \\
Inverse & \(A^{-1} = \frac{1}{|A|} \times adj(A)\) \\
Matrix Multiplication & \((AB)_{ij} = \sum_{k} A_{ik}B_{kj}\) \\
Transpose Property & \((AB)^T = B^T A^T\) \\
\end{longtable}

\subsubsection{\texorpdfstring{\textbf{Differentiation}}{Differentiation}}\label{differentiation}

\begin{longtable}[]{@{}ll@{}}
\toprule\noalign{}
Function & Derivative \\
\midrule\noalign{}
\endhead
\bottomrule\noalign{}
\endlastfoot
\(x^n\) & \(nx^{n-1}\) \\
\(\log x\) & \(\frac{1}{x}\) \\
\(e^x\) & \(e^x\) \\
\(\sin x\) & \(\cos x\) \\
\(\cos x\) & \(-\sin x\) \\
\(\tan x\) & \(\sec^2 x\) \\
Chain Rule & \(\frac{d}{dx}[f(g(x))] = f'(g(x)) \cdot g'(x)\) \\
Product Rule & \((uv)' = u'v + uv'\) \\
Quotient Rule & \((\frac{u}{v})' = \frac{u'v - uv'}{v^2}\) \\
\end{longtable}

\subsubsection{\texorpdfstring{\textbf{Integration}}{Integration}}\label{integration}

\begin{longtable}[]{@{}ll@{}}
\toprule\noalign{}
Function & Integral \\
\midrule\noalign{}
\endhead
\bottomrule\noalign{}
\endlastfoot
\(x^n\) & \(\frac{x^{n+1}}{n+1} + c\) \\
\(\frac{1}{x}\) & \(\log|x| + c\) \\
\(e^x\) & \(e^x + c\) \\
\(\sin x\) & \(-\cos x + c\) \\
\(\cos x\) & \(\sin x + c\) \\
\(\sec^2 x\) & \(\tan x + c\) \\
\(\csc^2 x\) & \(-\cot x + c\) \\
Integration by Parts & \(\int u dv = uv - \int v du\) \\
\end{longtable}

\subsubsection{\texorpdfstring{\textbf{Differential
Equations}}{Differential Equations}}\label{differential-equations}

\begin{longtable}[]{@{}
  >{\raggedright\arraybackslash}p{(\linewidth - 4\tabcolsep) * \real{0.2500}}
  >{\raggedright\arraybackslash}p{(\linewidth - 4\tabcolsep) * \real{0.3333}}
  >{\raggedright\arraybackslash}p{(\linewidth - 4\tabcolsep) * \real{0.4167}}@{}}
\toprule\noalign{}
\begin{minipage}[b]{\linewidth}\raggedright
Type
\end{minipage} & \begin{minipage}[b]{\linewidth}\raggedright
Method
\end{minipage} & \begin{minipage}[b]{\linewidth}\raggedright
Solution
\end{minipage} \\
\midrule\noalign{}
\endhead
\bottomrule\noalign{}
\endlastfoot
Variable Separable & \(\frac{dy}{dx} = f(x)g(y)\) &
\(\int \frac{dy}{g(y)} = \int f(x)dx\) \\
Linear DE & \(\frac{dy}{dx} + Py = Q\) &
\(y \cdot I.F. = \int Q \cdot I.F. dx\) \\
Integrating Factor & I.F. = \(e^{\int P dx}\) & - \\
\end{longtable}

\subsubsection{\texorpdfstring{\textbf{Complex
Numbers}}{Complex Numbers}}\label{complex-numbers}

\begin{longtable}[]{@{}
  >{\raggedright\arraybackslash}p{(\linewidth - 2\tabcolsep) * \real{0.5500}}
  >{\raggedright\arraybackslash}p{(\linewidth - 2\tabcolsep) * \real{0.4500}}@{}}
\toprule\noalign{}
\begin{minipage}[b]{\linewidth}\raggedright
Operation
\end{minipage} & \begin{minipage}[b]{\linewidth}\raggedright
Formula
\end{minipage} \\
\midrule\noalign{}
\endhead
\bottomrule\noalign{}
\endlastfoot
Modulus & \(|a + bi| = \sqrt{a^2 + b^2}\) \\
Argument & \(\arg(z) = \tan^{-1}(\frac{b}{a})\) \\
Polar Form & \(z = r(\cos\theta + i\sin\theta) = re^{i\theta}\) \\
Powers & \(i^1 = i, i^2 = -1, i^3 = -i, i^4 = 1\) \\
De Moivre's &
\((r(\cos\theta + i\sin\theta))^n = r^n(\cos n\theta + i\sin n\theta)\) \\
\end{longtable}

\begin{center}\rule{0.5\linewidth}{0.5pt}\end{center}

\subsection*{Problem-Solving
Strategies}\label{problem-solving-strategies}

\subsubsection{\texorpdfstring{\textbf{For Matrix
Problems:}}{For Matrix Problems:}}\label{for-matrix-problems}

\begin{enumerate}
\tightlist
\item
  \textbf{Check dimensions} first for multiplication
\item
  \textbf{Use determinant} to check if inverse exists
\item
  \textbf{Apply properties} like \((AB)^T = B^T A^T\)
\item
  \textbf{Substitute and verify} your answers
\end{enumerate}

\subsubsection{\texorpdfstring{\textbf{For
Differentiation:}}{For Differentiation:}}\label{for-differentiation}

\begin{enumerate}
\tightlist
\item
  \textbf{Identify the type} (composite, product, quotient)
\item
  \textbf{Apply appropriate rule} systematically
\item
  \textbf{Simplify step by step}
\item
  \textbf{Check using basic derivatives}
\end{enumerate}

\subsubsection{\texorpdfstring{\textbf{For
Integration:}}{For Integration:}}\label{for-integration}

\begin{enumerate}
\tightlist
\item
  \textbf{Look for standard forms} first
\item
  \textbf{Try substitution} if composite function
\item
  \textbf{Use integration by parts} for products
\item
  \textbf{Apply partial fractions} for rational functions
\end{enumerate}

\subsubsection{\texorpdfstring{\textbf{For Differential
Equations:}}{For Differential Equations:}}\label{for-differential-equations}

\begin{enumerate}
\tightlist
\item
  \textbf{Identify the type} (separable, linear, etc.)
\item
  \textbf{Find integrating factor} for linear DEs
\item
  \textbf{Separate variables} when possible
\item
  \textbf{Apply initial conditions} to find constants
\end{enumerate}

\begin{center}\rule{0.5\linewidth}{0.5pt}\end{center}

\subsection*{Common Mistakes to Avoid}\label{common-mistakes-to-avoid}

\subsubsection{\texorpdfstring{\textbf{Matrix
Operations:}}{Matrix Operations:}}\label{matrix-operations-1}

\begin{itemize}
\tightlist
\item
  \textbf{Wrong dimension calculation} in multiplication
\item
  \textbf{Forgetting to transpose} in \((AB)^T = B^T A^T\)
\item
  \textbf{Not checking if matrix is invertible} before finding inverse
\end{itemize}

\subsubsection{\texorpdfstring{\textbf{Differentiation:}}{Differentiation:}}\label{differentiation-1}

\begin{itemize}
\tightlist
\item
  \textbf{Missing chain rule} in composite functions
\item
  \textbf{Sign errors} in trigonometric derivatives
\item
  \textbf{Forgetting product rule} in multiplied functions
\end{itemize}

\subsubsection{\texorpdfstring{\textbf{Integration:}}{Integration:}}\label{integration-1}

\begin{itemize}
\tightlist
\item
  \textbf{Wrong limits} in definite integrals
\item
  \textbf{Missing constant of integration}
\item
  \textbf{Incorrect substitution} bounds
\end{itemize}

\subsubsection{\texorpdfstring{\textbf{Complex
Numbers:}}{Complex Numbers:}}\label{complex-numbers-1}

\begin{itemize}
\tightlist
\item
  \textbf{Wrong quadrant} in argument calculation
\item
  \textbf{Modulus calculation errors}
\item
  \textbf{Forgetting to rationalize} denominators
\end{itemize}

\begin{center}\rule{0.5\linewidth}{0.5pt}\end{center}

\subsection*{Exam Tips}\label{exam-tips}

\subsubsection{\texorpdfstring{\textbf{Time
Management:}}{Time Management:}}\label{time-management}

\begin{itemize}
\tightlist
\item
  \textbf{Attempt Q.1 first} (14 marks, quick fill-ups)
\item
  \textbf{Choose easier sub-questions} in each section
\item
  \textbf{Leave difficult calculations} for the end
\end{itemize}

\subsubsection{\texorpdfstring{\textbf{Answer
Presentation:}}{Answer Presentation:}}\label{answer-presentation}

\begin{itemize}
\tightlist
\item
  \textbf{Show all steps} clearly
\item
  \textbf{Box final answers}
\item
  \textbf{Use proper mathematical notation}
\item
  \textbf{Draw diagrams} where helpful
\end{itemize}

\subsubsection{\texorpdfstring{\textbf{Verification:}}{Verification:}}\label{verification}

\begin{itemize}
\tightlist
\item
  \textbf{Check dimensions} in matrix problems
\item
  \textbf{Verify differentiation} by differentiating your answer
\item
  \textbf{Substitute back} in differential equations
\item
  \textbf{Check modulus and argument} for complex numbers
\end{itemize}

\subsubsection{\texorpdfstring{\textbf{Key Formulas to
Remember:}}{Key Formulas to Remember:}}\label{key-formulas-to-remember}

\begin{itemize}
\tightlist
\item
  \textbf{Matrix inverse formula}
\item
  \textbf{Integration by parts}
\item
  \textbf{Linear DE solution method}
\item
  \textbf{Complex number polar form}
\item
  \textbf{Standard derivatives and integrals}
\end{itemize}

\begin{center}\rule{0.5\linewidth}{0.5pt}\end{center}

\textbf{Remember}: Practice is key to mastering these concepts. Work
through similar problems and focus on understanding the underlying
principles rather than just memorizing formulas.


\end{document}
