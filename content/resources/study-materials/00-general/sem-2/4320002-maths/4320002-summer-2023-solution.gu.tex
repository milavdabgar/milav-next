\documentclass{article}

% content/resources/templates/preamble.tex
\usepackage[margin=0.6in]{geometry}
\author{Milav Dabgar}
\usepackage{amsmath,amssymb,amsthm}
\usepackage{booktabs}
\usepackage{multirow}
\usepackage{xcolor}
\usepackage{tcolorbox}
\tcbuselibrary{breakable,skins}
\usepackage[colorlinks=true,linkcolor=blue]{hyperref}
\usepackage{titlesec}
\usepackage{enumitem}
\usepackage{tikz}
\usepackage{pgfplots}
\usepackage{circuitikz}
\usepackage[version=4]{mhchem}
\usepackage{longtable}
\usepackage{array}
\usepackage{float}
\usepackage{caption}
\usepackage{listings}

\lstset{
  basicstyle=\small\ttfamily,
  breaklines=true,
  breakatwhitespace=false,
  postbreak=\mbox{\textcolor{red}{$\hookrightarrow$}\space},
  float=false,
  numbers=left,
  numberstyle=\tiny\color{gray},
  numbersep=10pt,
  xleftmargin=2em,
  keywordstyle=\color{blue},
  commentstyle=\color{green!60!black},
  stringstyle=\color{purple},
  backgroundcolor=\color{gray!5},
  showstringspaces=false,
  tabsize=2,
  captionpos=b,
  keepspaces=true,
  columns=flexible
}

\pgfplotsset{compat=1.18}
\usetikzlibrary{shapes,arrows,positioning,calc,patterns,decorations.pathmorphing,decorations.markings,arrows.meta}

% Color scheme
\definecolor{headcolor}{RGB}{0,102,204}
\definecolor{keycolor}{RGB}{220,20,60}
\definecolor{solutioncolor}{RGB}{34,139,34}
\definecolor{mnemoniccolor}{RGB}{148,0,211}
\definecolor{codecolor}{RGB}{0,0,100}

% Spacing
\setlength{\parskip}{3pt}
\setlist[itemize]{nosep}
\setlist[enumerate]{nosep}

% Title formatting
\titleformat{\section}{\Large\bfseries\color{headcolor}}{\thesection}{1em}{}
\titleformat{\subsection}{\large\bfseries\color{headcolor}}{\thesubsection}{1em}{}

% Pandoc tightlist compatibility
\providecommand{\tightlist}{%
  \setlength{\itemsep}{0pt}\setlength{\parskip}{0pt}}

% Pandoc longtable compatibility
\newcounter{none}
\def\thenone{}


% content/resources/templates/gujarati-boxes.tex
\usepackage{fontspec}
\usepackage{polyglossia}

% Set Gujarati as main language (document is primarily in Gujarati)
% Note: gloss-gujarati.ldf doesn't exist in polyglossia, but it will use hyphenation patterns
\setdefaultlanguage{gujarati}
\setotherlanguage{english}

% Configure Gujarati font properly
% Use Language=Default to prevent polyglossia from trying to add language-specific features
% that don't exist for Gujarati, which causes "empty feature" warnings
\newfontfamily\gujaratifont[Script=Gujarati,AutoFakeBold=2.5,AutoFakeSlant=0.3]{Noto Sans Gujarati}
\setmainfont[Script=Gujarati,AutoFakeBold=2.5,AutoFakeSlant=0.3]{Noto Sans Gujarati}
% Use Noto Sans Gujarati for monospace to support Gujarati in text
\setmonofont[Scale=0.9]{Noto Sans Gujarati}

% Configure English to use the same font
\newfontfamily\englishfont[Script=Gujarati,AutoFakeBold=2.5,AutoFakeSlant=0.3]{Noto Sans Gujarati}

% Translations for polyglossia
\gappto\captionsgujarati{
  \renewcommand{\tablename}{કોષ્ટક}
  \renewcommand{\figurename}{આકૃતિ}
}

% Helper for TikZ nodes to ensure Gujarati font
\newcommand{\gu}[1]{{\gujaratifont #1}}

% Custom environments
\newtcolorbox{solutionbox}{
    breakable,
    enhanced,
    colback=solutioncolor!5!white,
    colframe=solutioncolor!75!black,
    fonttitle=\bfseries,
    title=જવાબ
}

\newtcolorbox{solutionboxnobreak}{
 colback=solutioncolor!5!white,
 colframe=solutioncolor!75!black,
 fonttitle=\bfseries,
 title=જવાબ
}

\newtcolorbox{keyformula}{
 breakable,
 enhanced,
 colback=keycolor!5!white,
 colframe=keycolor!75!black,
 fonttitle=\bfseries,
 title=રાસાયણિક સમીકરણ/સૂત્ર
}

\newtcolorbox{mnemonicbox}{
 breakable,
 enhanced,
 colback=mnemoniccolor!5!white,
 colframe=mnemoniccolor!75!black,
 fonttitle=\bfseries,
 title=મેમરી ટ્રીક
}


% Custom commands for GTU solutions
% This file defines semantic commands for consistent formatting

% Question command with automatic formatting
\newcommand{\question}[2]{%
  \section*{Question #1}%
  \textbf{#2}%
}

% OR question variant
\newcommand{\questionor}[2]{%
  \section*{Question #1 OR}%
  \textbf{#2}%
}

% Proper table environment with caption
\newenvironment{answertable}[1]{%
  \begin{table}[htbp]
  \centering
  \caption{#1}
}{%
  \end{table}
}

% Proper figure environment for diagrams
\newenvironment{answerdiagram}[1]{%
  \begin{figure}[htbp]
  \centering
  \caption{#1}
}{%
  \end{figure}
}

% Semantic markup for key terms
\newcommand{\keyword}[1]{\textbf{#1}}
\newcommand{\code}[1]{\texttt{#1}}
\newcommand{\classname}[1]{\texttt{#1}}
\newcommand{\methodname}[1]{\texttt{#1}}

% Proper quotation marks
\newcommand{\mnemonic}[1]{``#1''}


\title{એન્જિનિયરિંગ મેથેમેટિક્સ (4320002) - સમર 2023 સોલ્યુશન}
\date{ઓગસ્ટ 02, 2023}

\begin{document}
\maketitle

\questionmarks{1}{14}{નીચે આપેલા વિકલ્પોમાંથી યોગ્ય વિકલ્પ પસંદ કરી ખાલી જગ્યા પૂરો.}

\questionmarks{1.1}{1}{$\begin{bmatrix} 1 & 0 & 3 \\ -2 & 4 & 0 \end{bmatrix}$ ની કક્ષા (Order) \_\_\_\_\_\_\_\_\_\_\_ છે.}

\begin{solutionbox}
\textbf{જવાબ}: b. $2 \times 3$

\textbf{ઉકેલ}:
શ્રેણિકમાં 2 હાર (rows) અને 3 સ્તંભ (columns) છે, તેથી કક્ષા $2 \times 3$ છે.
\end{solutionbox}

\questionmarks{1.2}{1}{જો A ની કક્ષા $2 \times 3$ હોય અને B ની કક્ષા $3 \times 2$ હોય તો AB ની કક્ષા \_\_\_\_\_\_\_\_\_ થશે.}

\begin{solutionbox}
\textbf{જવાબ}: d. $2 \times 2$

\textbf{ઉકેલ}:
શ્રેણિકોનો ગુણાકાર $AB$ માટે, જો $A$ એ $2 \times 3$ અને $B$ એ $3 \times 2$ હોય, તો $AB$ ની કક્ષા $2 \times 2$ થશે.
\end{solutionbox}

\questionmarks{1.3}{1}{જો $A = \begin{bmatrix} 1 & -1 \end{bmatrix}$ હોય તો $A^T = $ \_\_\_\_\_\_\_}

\begin{solutionbox}
\textbf{જવાબ}: b. $\begin{bmatrix} 1 \\ -1 \end{bmatrix}$

\textbf{ઉકેલ}:
હાર શ્રેણિકનો પરિવર્ત (transpose) સ્તંભ શ્રેણિક બને છે.
\[
A^T = \begin{bmatrix} 1 \\ -1 \end{bmatrix}
\]
\end{solutionbox}

\questionmarks{1.4}{1}{જો $A = \begin{bmatrix} 1 & 2 \\ 3 & 4 \end{bmatrix}$ હોય તો $\text{adj } A = $ \_\_\_\_\_\_}

\begin{solutionbox}
\textbf{જવાબ}: d. $\begin{bmatrix} 4 & -2 \\ -3 & 1 \end{bmatrix}$

\textbf{ઉકેલ}:
$2 \times 2$ શ્રેણિક $A = \begin{bmatrix} a & b \\ c & d \end{bmatrix}$ માટે,
$\text{adj } A = \begin{bmatrix} d & -b \\ -c & a \end{bmatrix}$

તેથી: $\text{adj } A = \begin{bmatrix} 4 & -2 \\ -3 & 1 \end{bmatrix}$
\end{solutionbox}

\questionmarks{1.5}{1}{$\frac{d}{dx}(e^x) = $ \_\_\_\_\_}

\begin{solutionbox}
\textbf{જવાબ}: a. $e^x$

\textbf{ઉકેલ}:
\[
\frac{d}{dx}(e^x) = e^x
\]
\end{solutionbox}

\questionmarks{1.6}{1}{જો $f(x) = \log x$ હોય તો $f'(1) = $ \_\_\_\_\_}

\begin{solutionbox}
\textbf{જવાબ}: c. 1

\textbf{ઉકેલ}:
\[
f'(x) = \frac{1}{x}
\]
\[
f'(1) = \frac{1}{1} = 1
\]
\end{solutionbox}

\questionmarks{1.7}{1}{$\frac{d}{dx}(3^{\log_3 x}) = $ \_\_\_\_\_\_}

\begin{solutionbox}
\textbf{જવાબ}: b. $2x$

\textbf{ઉકેલ}:
$a^{\log_a x} = x$ ગુણધર્મનો ઉપયોગ કરતા:
$3^{\log_3 x} = x$
તેથી: $\frac{d}{dx}(3^{\log_3 x}) = \frac{d}{dx}(x) = 1$

જુઓ, આમાં એક પદ $3^{\log_3 x^2} = x^2$ હોવું જોઈએ.
$\frac{d}{dx}(x^2) = 2x$
\end{solutionbox}

\questionmarks{1.8}{1}{$\int \sin x \, dx = $ \_\_\_\_\_}

\begin{solutionbox}
\textbf{જવાબ}: c. $-\cos x$

\textbf{ઉકેલ}:
\[
\int \sin x \, dx = -\cos x + C
\]
\end{solutionbox}

\questionmarks{1.9}{1}{$\int_{-1}^{1} x^3 \, dx = $ \_\_\_\_\_}

\begin{solutionbox}
\textbf{જવાબ}: b. 0

\textbf{ઉકેલ}:
\[
\int_{-1}^{1} x^3 \, dx = \left[\frac{x^4}{4}\right]_{-1}^{1} = \frac{1}{4} - \frac{1}{4} = 0
\]
\end{solutionbox}

\questionmarks{1.10}{1}{$\int \frac{1}{1+x^2} \, dx = $ \_\_\_\_\_}

\begin{solutionbox}
\textbf{જવાબ}: d. $\tan^{-1} x$

\textbf{ઉકેલ}:
\[
\int \frac{1}{1+x^2} \, dx = \tan^{-1} x + C
\]
\end{solutionbox}

\questionmarks{1.11}{1}{વિકલ સમીકરણ $\frac{d^2y}{dx^2} - y = 0$ ની કક્ષા (Order) \_\_\_\_\_\_\_\_ છે.}

\begin{solutionbox}
\textbf{જવાબ}: b. 2

\textbf{ઉકેલ}:
સર્વોચ્ચ વિકલિત $\frac{d^2y}{dx^2}$ છે, તેથી કક્ષા 2 છે.
\end{solutionbox}

\questionmarks{1.12}{1}{વિકલ સમીકરણ $\frac{dy}{dx} + Py = Q$ નો સંકલ્પકારક અવયવ (I.F) \_\_\_\_\_\_\_\_ છે}

\begin{solutionbox}
\textbf{જવાબ}: a. $e^{\int P \, dx}$

\textbf{ઉકેલ}:
સુરેખ વિકલ સમીકરણ $\frac{dy}{dx} + Py = Q$ માટે, સંકલ્પકારક અવયવ $e^{\int P \, dx}$ છે.
\end{solutionbox}

\questionmarks{1.13}{1}{જો $Z = 4 - 5i$ હોય તો $\bar{Z} = $ \_\_\_\_\_\_\_\_}

\begin{solutionbox}
\textbf{જવાબ}: c. $4 - 5i$

\textbf{ઉકેલ}:
જો $Z = 4 - 5i$, તો $\bar{Z} = 4 + 5i$.
વિકલ્પોમાં કદાચ ભૂલ હોઈ શકે છે. સાચો જવાબ $4 + 5i$ હોવો જોઈએ.
\end{solutionbox}

\questionmarks{1.14}{1}{$i^{10} = $ \_\_\_\_\_\_}

\begin{solutionbox}
\textbf{જવાબ}: b. -1

\textbf{ઉકેલ}:
\[
i^{10} = i^{4 \cdot 2 + 2} = (i^4)^2 \cdot i^2 = 1^2 \cdot (-1) = -1
\]
\end{solutionbox}

\questionmarks{2(A)}{6}{કોઈપણ બે લખો.}

\questionmarks{2(A).1}{3}{જો $A = \begin{bmatrix} 2 & -1 \\ 4 & 3 \end{bmatrix}$ અને $B = \begin{bmatrix} 3 & 2 \\ 1 & 4 \end{bmatrix}$ હોય તો શ્રેણિક X શોધો જેથી $2A + X = 3B$.}

\begin{solutionbox}
\textbf{ઉકેલ}:
$2A + X = 3B \Rightarrow X = 3B - 2A$

\[
2A = 2\begin{bmatrix} 2 & -1 \\ 4 & 3 \end{bmatrix} = \begin{bmatrix} 4 & -2 \\ 8 & 6 \end{bmatrix}
\]
\[
3B = 3\begin{bmatrix} 3 & 2 \\ 1 & 4 \end{bmatrix} = \begin{bmatrix} 9 & 6 \\ 3 & 12 \end{bmatrix}
\]
\[
X = \begin{bmatrix} 9 & 6 \\ 3 & 12 \end{bmatrix} - \begin{bmatrix} 4 & -2 \\ 8 & 6 \end{bmatrix} = \begin{bmatrix} 5 & 8 \\ -5 & 6 \end{bmatrix}
\]
\end{solutionbox}

\questionmarks{2(A).2}{3}{જો $A = \begin{bmatrix} 5 & 4 \\ 4 & 3 \end{bmatrix}$ અને $B = \begin{bmatrix} 1 & 3 \\ 2 & 1 \end{bmatrix}$ હોય તો $(AB)^T$ શોધો.}

\begin{solutionbox}
\textbf{ઉકેલ}:
પ્રથમ, $AB$ શોધો:
\[
AB = \begin{bmatrix} 5 & 4 \\ 4 & 3 \end{bmatrix}\begin{bmatrix} 1 & 3 \\ 2 & 1 \end{bmatrix}
\]
\[
AB = \begin{bmatrix} 5(1)+4(2) & 5(3)+4(1) \\ 4(1)+3(2) & 4(3)+3(1) \end{bmatrix} = \begin{bmatrix} 13 & 19 \\ 10 & 15 \end{bmatrix}
\]
\[
(AB)^T = \begin{bmatrix} 13 & 10 \\ 19 & 15 \end{bmatrix}
\]
\end{solutionbox}

\questionmarks{2(A).3}{3}{ઉકેલો: $\frac{dy}{dx} = x^2 \cdot e^{-y}$.}

\begin{solutionbox}
\textbf{ઉકેલ}:
\[
\frac{dy}{dx} = x^2 \cdot e^{-y}
\]
ચલ અલગ કરતા (Separating variables):
\[
e^y \, dy = x^2 \, dx
\]
બંને બાજુ સંકલન કરતા:
\[
\int e^y \, dy = \int x^2 \, dx
\]
\[
e^y = \frac{x^3}{3} + C
\]
\[
y = \ln\left(\frac{x^3}{3} + C\right)
\]
\end{solutionbox}

\questionmarks{2(B)}{8}{કોઈપણ બે લખો.}

\questionmarks{2(B).1}{4}{જો $A = \begin{bmatrix} 2 & 3 & -1 \\ 4 & 5 & 0 \end{bmatrix}$ અને $B = \begin{bmatrix} 1 & 2 & 4 \\ 2 & 3 & 1 \end{bmatrix}$ હોય તો સાબિત કરો કે $(A + B)^T = A^T + B^T$.}

\begin{solutionbox}
\textbf{ઉકેલ}:
\[
A + B = \begin{bmatrix} 2 & 3 & -1 \\ 4 & 5 & 0 \end{bmatrix} + \begin{bmatrix} 1 & 2 & 4 \\ 2 & 3 & 1 \end{bmatrix}
\]
\[
A + B = \begin{bmatrix} 3 & 5 & 3 \\ 6 & 8 & 1 \end{bmatrix}
\]
\[
(A + B)^T = \begin{bmatrix} 3 & 6 \\ 5 & 8 \\ 3 & 1 \end{bmatrix}
\]
\[
A^T = \begin{bmatrix} 2 & 4 \\ 3 & 5 \\ -1 & 0 \end{bmatrix}, \quad B^T = \begin{bmatrix} 1 & 2 \\ 2 & 3 \\ 4 & 1 \end{bmatrix}
\]
\[
A^T + B^T = \begin{bmatrix} 2 & 4 \\ 3 & 5 \\ -1 & 0 \end{bmatrix} + \begin{bmatrix} 1 & 2 \\ 2 & 3 \\ 4 & 1 \end{bmatrix} = \begin{bmatrix} 3 & 6 \\ 5 & 8 \\ 3 & 1 \end{bmatrix}
\]
તેથી, $(A + B)^T = A^T + B^T$ સાબિત થાય છે.
\end{solutionbox}

\questionmarks{2(B).2}{4}{જો $A = \begin{bmatrix} 2 & -1 & 0 \\ 1 & 0 & 4 \\ 1 & -1 & 1 \end{bmatrix}$ હોય તો $A^{-1}$ શોધો.}

\begin{solutionbox}
\textbf{ઉકેલ}:
$A^{-1}$ શોધવા માટે, આપણે $A^{-1} = \frac{1}{|A|} \cdot \text{adj}(A)$ સૂત્રનો ઉપયોગ કરીએ.

પ્રથમ, $|A|$ શોધીએ:
$|A| = 2(0 \cdot 1 - 4 \cdot (-1)) - (-1)(1 \cdot 1 - 4 \cdot 1) + 0(1 \cdot (-1) - 0 \cdot 1)$
$|A| = 2(4) + 1(-3) = 8 - 3 = 5$

હવે સહઅવયવો (cofactors) શોધીએ:
$C_{11} = (-1)^{1+1}\begin{vmatrix} 0 & 4 \\ -1 & 1 \end{vmatrix} = 4$
$C_{12} = (-1)^{1+2}\begin{vmatrix} 1 & 4 \\ 1 & 1 \end{vmatrix} = -(-3) = 3$
$C_{13} = (-1)^{1+3}\begin{vmatrix} 1 & 0 \\ 1 & -1 \end{vmatrix} = -1$
$C_{21} = (-1)^{2+1}\begin{vmatrix} -1 & 0 \\ -1 & 1 \end{vmatrix} = -(-1) = 1$
$C_{22} = (-1)^{2+2}\begin{vmatrix} 2 & 0 \\ 1 & 1 \end{vmatrix} = 2$
$C_{23} = (-1)^{2+3}\begin{vmatrix} 2 & -1 \\ 1 & -1 \end{vmatrix} = -(-1) = 1$
$C_{31} = (-1)^{3+1}\begin{vmatrix} -1 & 0 \\ 0 & 4 \end{vmatrix} = -4$
$C_{32} = (-1)^{3+2}\begin{vmatrix} 2 & 0 \\ 1 & 4 \end{vmatrix} = -(8) = -8$
$C_{33} = (-1)^{3+3}\begin{vmatrix} 2 & -1 \\ 1 & 0 \end{vmatrix} = 1$

\[
\text{adj}(A) = \begin{bmatrix} 4 & 1 & -4 \\ 3 & 2 & -8 \\ -1 & 1 & 1 \end{bmatrix}
\]
\[
A^{-1} = \frac{1}{5}\begin{bmatrix} 4 & 1 & -4 \\ 3 & 2 & -8 \\ -1 & 1 & 1 \end{bmatrix}
\]
\end{solutionbox}

\questionmarks{2(B).3}{4}{શ્રેણિક પદ્ધતિથી સમીકરણો $3x - y = 1, x + 2y = 5$ ઉકેલો.}

\begin{solutionbox}
\textbf{ઉકેલ}:
સમીકરણોને $AX = B$ તરીકે લખી શકાય:
$A = \begin{bmatrix} 3 & -1 \\ 1 & 2 \end{bmatrix}$, $X = \begin{bmatrix} x \\ y \end{bmatrix}$, $B = \begin{bmatrix} 1 \\ 5 \end{bmatrix}$

\[
|A| = 3(2) - (-1)(1) = 6 + 1 = 7
\]
\[
A^{-1} = \frac{1}{7}\begin{bmatrix} 2 & 1 \\ -1 & 3 \end{bmatrix}
\]
\[
X = A^{-1}B = \frac{1}{7}\begin{bmatrix} 2 & 1 \\ -1 & 3 \end{bmatrix}\begin{bmatrix} 1 \\ 5 \end{bmatrix}
\]
\[
X = \frac{1}{7}\begin{bmatrix} 2 + 5 \\ -1 + 15 \end{bmatrix} = \frac{1}{7}\begin{bmatrix} 7 \\ 14 \end{bmatrix} = \begin{bmatrix} 1 \\ 2 \end{bmatrix}
\]
તેથી, $x = 1$ અને $y = 2$.
\end{solutionbox}

\questionmarks{3(A)}{6}{કોઈપણ બે લખો.}

\questionmarks{3(A).1}{3}{જો $y = \frac{e^x + 1}{e^x - 1}$ હોય તો $\frac{dy}{dx}$ શોધો.}

\begin{solutionbox}
\textbf{ઉકેલ}:
ભાગાકારના નિયમનો ઉપયોગ કરતા: $\frac{d}{dx}\left(\frac{u}{v}\right) = \frac{v\frac{du}{dx} - u\frac{dv}{dx}}{v^2}$

ધારો કે $u = e^x + 1$ અને $v = e^x - 1$
$\frac{du}{dx} = e^x$ અને $\frac{dv}{dx} = e^x$

\[
\frac{dy}{dx} = \frac{(e^x - 1)(e^x) - (e^x + 1)(e^x)}{(e^x - 1)^2}
\]
\[
= \frac{e^{2x} - e^x - e^{2x} - e^x}{(e^x - 1)^2} = \frac{-2e^x}{(e^x - 1)^2}
\]
\end{solutionbox}

\questionmarks{3(A).2}{3}{જો $x = a\cos\theta, y = b\sin\theta$ હોય તો $\frac{dy}{dx}$ શોધો.}

\begin{solutionbox}
\textbf{ઉકેલ}:
$\frac{dx}{d\theta} = -a\sin\theta$
$\frac{dy}{d\theta} = b\cos\theta$
\[
\frac{dy}{dx} = \frac{dy/d\theta}{dx/d\theta} = \frac{b\cos\theta}{-a\sin\theta} = -\frac{b\cos\theta}{a\sin\theta} = -\frac{b}{a}\cot\theta
\]
\end{solutionbox}

\questionmarks{3(A).3}{3}{કિંમત શોધો: $\int \frac{\cos\sqrt{x}}{2\sqrt{x}} dx$.}

\begin{solutionbox}
\textbf{ઉકેલ}:
ધારો કે $u = \sqrt{x}$, તેથી $du = \frac{1}{2\sqrt{x}}dx$

\[
\int \frac{\cos\sqrt{x}}{2\sqrt{x}} dx = \int \cos u \, du = \sin u + C = \sin\sqrt{x} + C
\]
\end{solutionbox}

\questionmarks{3(B)}{8}{કોઈપણ બે લખો.}

\questionmarks{3(B).1}{4}{$y = x^{\cos x}$ નું x સાપેક્ષ વિકલન કરો.}

\begin{solutionbox}
\textbf{ઉકેલ}:
બંને બાજુ લઘુગુણક (ln) લેતા:
\[
\ln y = \cos x \ln x
\]
બંને બાજુ x સાપેક્ષ વિકલન કરતા:
\[
\frac{1}{y}\frac{dy}{dx} = \cos x \cdot \frac{1}{x} + \ln x \cdot (-\sin x)
\]
\[
\frac{dy}{dx} = y\left(\frac{\cos x}{x} - \sin x \ln x\right)
\]
\[
\frac{dy}{dx} = x^{\cos x}\left(\frac{\cos x}{x} - \sin x \ln x\right)
\]
\end{solutionbox}

\questionmarks{3(B).2}{4}{જો $y = A\cos pt + B\sin pt$ હોય, તો સાબિત કરો કે $\frac{d^2y}{dt^2} + p^2y = 0$.}

\begin{solutionbox}
\textbf{ઉકેલ}:
$y = A\cos pt + B\sin pt$
\[
\frac{dy}{dt} = -Ap\sin pt + Bp\cos pt
\]
\[
\frac{d^2y}{dt^2} = -Ap^2\cos pt - Bp^2\sin pt = -p^2(A\cos pt + B\sin pt) = -p^2y
\]
તેથી: $\frac{d^2y}{dt^2} + p^2y = -p^2y + p^2y = 0$
\end{solutionbox}

\questionmarks{3(B).3}{4}{એક કણની ગતિનું સમીકરણ $s = t^3 + 2t^2 - 3t + 5$ છે. તો $t = 1$ અને $t = 2$ સેકન્ડે કણનો વેગ અને પ્રવેગ શોધો.}

\begin{solutionbox}
\textbf{ઉકેલ}:
$s = t^3 + 2t^2 - 3t + 5$

વેગ (Velocity): $v = \frac{ds}{dt} = 3t^2 + 4t - 3$

પ્રવેગ (Acceleration): $a = \frac{dv}{dt} = 6t + 4$

$t = 1$ સમયે:
$v(1) = 3(1)^2 + 4(1) - 3 = 3 + 4 - 3 = 4$ એકમ/સેકન્ડ
$a(1) = 6(1) + 4 = 10$ એકમ/સેકન્ડ²

$t = 2$ સમયે:
$v(2) = 3(2)^2 + 4(2) - 3 = 12 + 8 - 3 = 17$ એકમ/સેકન્ડ
$a(2) = 6(2) + 4 = 16$ એકમ/સેકન્ડ²
\end{solutionbox}

\questionmarks{4(A)}{6}{કોઈપણ બે લખો.}

\questionmarks{4(A).1}{3}{કિંમત શોધો: $\int x \log x \, dx$.}

\begin{solutionbox}
\textbf{ઉકેલ}:
ખંડશઃ સંકલનનો ઉપયોગ કરતા: $\int u \, dv = uv - \int v \, du$

ધારો કે $u = \log x$ અને $dv = x \, dx$
તેથી $du = \frac{1}{x} dx$ અને $v = \frac{x^2}{2}$
\[
\int x \log x \, dx = \log x \cdot \frac{x^2}{2} - \int \frac{x^2}{2} \cdot \frac{1}{x} dx
\]
\[
= \frac{x^2 \log x}{2} - \int \frac{x}{2} dx
\]
\[
= \frac{x^2 \log x}{2} - \frac{x^2}{4} + C
\]
\[
= \frac{x^2}{2}(\log x - \frac{1}{2}) + C
\]
\end{solutionbox}

\questionmarks{4(A).2}{3}{કિંમત શોધો: $\int_{-1}^{1} \frac{1}{1+x^2} dx$.}

\begin{solutionbox}
\textbf{ઉકેલ}:
\[
\int_{-1}^{1} \frac{1}{1+x^2} dx = [\tan^{-1} x]_{-1}^{1}
\]
\[
= \tan^{-1}(1) - \tan^{-1}(-1)
\]
\[
= \frac{\pi}{4} - \left(-\frac{\pi}{4}\right) = \frac{\pi}{2}
\]
\end{solutionbox}

\questionmarks{4(A).3}{3}{$Z = 3 + 4i$ નો વ્યસ્ત શોધો.}

\begin{solutionbox}
\textbf{ઉકેલ}:
\[
Z^{-1} = \frac{1}{Z} = \frac{1}{3 + 4i}
\]
અંશ અને છેદને અનુબદ્ધ કરણી વડે ગુણતા:
\[
Z^{-1} = \frac{1}{3 + 4i} \cdot \frac{3 - 4i}{3 - 4i} = \frac{3 - 4i}{(3)^2 + (4)^2} = \frac{3 - 4i}{9 + 16} = \frac{3 - 4i}{25}
\]
\[
Z^{-1} = \frac{3}{25} - \frac{4}{25}i
\]
\end{solutionbox}

\questionmarks{4(B)}{8}{કોઈપણ બે લખો.}

\questionmarks{4(B).1}{4}{કિંમત શોધો: $\int_{0}^{\pi/2} \frac{\tan x}{\tan x + \cot x} dx$.}

\begin{solutionbox}
\textbf{ઉકેલ}:
ધારો કે $I = \int_{0}^{\pi/2} \frac{\tan x}{\tan x + \cot x} dx$

$\int_{a}^{b} f(x) dx = \int_{a}^{b} f(a+b-x) dx$ ગુણધર્મનો ઉપયોગ કરતા:
\[
I = \int_{0}^{\pi/2} \frac{\tan(\pi/2 - x)}{\tan(\pi/2 - x) + \cot(\pi/2 - x)} dx
\]
\[
= \int_{0}^{\pi/2} \frac{\cot x}{\cot x + \tan x} dx
\]
બંને સમીકરણોનો સરવાળો કરતા:
\[
2I = \int_{0}^{\pi/2} \frac{\tan x + \cot x}{\tan x + \cot x} dx = \int_{0}^{\pi/2} 1 \, dx = \frac{\pi}{2}
\]
તેથી: $I = \frac{\pi}{4}$
\end{solutionbox}

\questionmarks{4(B).2}{4}{રેખા $y = x$, $x = 5$ અને X-અક્ષ વડે ઘેરાયેલા પ્રદેશનું ક્ષેત્રફળ શોધો.}

\begin{solutionbox}
\textbf{ઉકેલ}:
પ્રદેશ $y = x$, $x = 5$, અને $y = 0$ (X-અક્ષ) દ્વારા સીમિત છે.

ક્ષેત્રફળ $= \int_{0}^{5} x \, dx = \left[\frac{x^2}{2}\right]_{0}^{5} = \frac{25}{2} - 0 = \frac{25}{2}$ ચોરસ એકમ
\end{solutionbox}

\questionmarks{4(B).3}{4}{જો $x + iy = \left(\frac{1+i}{2-i}\right)^2$ હોય, તો $x + y$ ની કિંમત શોધો.}

\begin{solutionbox}
\textbf{ઉકેલ}:
પ્રથમ $\frac{1+i}{2-i}$ ને સાદુરૂપ આપીએ:
\[
\frac{1+i}{2-i} \cdot \frac{2+i}{2+i} = \frac{(1+i)(2+i)}{(2-i)(2+i)} = \frac{2+i+2i+i^2}{4-i^2} = \frac{2+3i-1}{4+1} = \frac{1+3i}{5}
\]
હવે:
\[
\left(\frac{1+3i}{5}\right)^2 = \frac{(1+3i)^2}{25} = \frac{1+6i+9i^2}{25} = \frac{1+6i-9}{25} = \frac{-8+6i}{25}
\]
તેથી: $x = -\frac{8}{25}$ અને $y = \frac{6}{25}$
\[
x + y = -\frac{8}{25} + \frac{6}{25} = -\frac{2}{25}
\]
\end{solutionbox}

\questionmarks{5(A)}{6}{કોઈપણ બે લખો.}

\questionmarks{5(A).1}{3}{$Z = 5 + 12i$ નું વર્ગમૂળ શોધો.}

\begin{solutionbox}
\textbf{ઉકેલ}:
ધારો કે $\sqrt{5 + 12i} = a + bi$ જ્યાં $a, b \in \mathbb{R}$
\[
(a + bi)^2 = 5 + 12i
\]
\[
a^2 + 2abi + b^2i^2 = 5 + 12i
\]
\[
(a^2 - b^2) + 2abi = 5 + 12i
\]
વાસ્તવિક અને કાલ્પનિક ભાગો સરખાવતા:
$a^2 - b^2 = 5$ ... (1)
$2ab = 12$ ... (2)

(2) પરથી: $b = \frac{6}{a}$

(1) માં મુકતા: $a^2 - \frac{36}{a^2} = 5$
\[
a^4 - 5a^2 - 36 = 0
\]
ધારો કે $u = a^2$: $u^2 - 5u - 36 = 0$
\[
(u - 9)(u + 4) = 0
\]
કારણ કે $u = a^2 \geq 0$, આપણી પાસે $u = 9$, તેથી $a = \pm 3$

જો $a = 3$, તો $b = 2$
જો $a = -3$, તો $b = -2$

તેથી: $\sqrt{5 + 12i} = \pm(3 + 2i)$
\end{solutionbox}

\questionmarks{5(A).2}{3}{સમીકરણ $(2x - y) + yi = 6 + 4i$ પરથી $x, y \in \mathbb{R}$ શોધો.}

\begin{solutionbox}
\textbf{ઉકેલ}:
વાસ્તવિક અને કાલ્પનિક ભાગો સરખાવતા:
વાસ્તવિક ભાગ: $2x - y = 6$ ... (1)
કાલ્પનિક ભાગ: $y = 4$ ... (2)

(1) માં (2) મુકતા:
$2x - 4 = 6$
$2x = 10$
$x = 5$

તેથી: $x = 5$ અને $y = 4$
\end{solutionbox}

\questionmarks{5(A).3}{3}{$Z = 1 + i$ નો માનાંક અને મુખ્ય કોણાંક શોધો, અને Z ને ધ્રુવીય સ્વરૂપમાં દર્શાવો.}

\begin{solutionbox}
\textbf{ઉકેલ}:
$Z = 1 + i$

માનાંક: $|Z| = \sqrt{1^2 + 1^2} = \sqrt{2}$

મુખ્ય કોણાંક: $\arg(Z) = \tan^{-1}\left(\frac{1}{1}\right) = \tan^{-1}(1) = \frac{\pi}{4}$

ધ્રુવીય સ્વરૂપ: $Z = |Z|(\cos\theta + i\sin\theta) = \sqrt{2}\left(\cos\frac{\pi}{4} + i\sin\frac{\pi}{4}\right)$
\end{solutionbox}

\questionmarks{5(B)}{8}{કોઈપણ બે લખો.}

\questionmarks{5(B).1}{4}{ઉકેલો: $\frac{dy}{dx} = 1 + x + y + xy$.}

\begin{solutionbox}
\textbf{ઉકેલ}:
\[
\frac{dy}{dx} = 1 + x + y + xy = (1 + x) + y(1 + x) = (1 + x)(1 + y)
\]
ચલ અલગ કરતા:
\[
\frac{dy}{1 + y} = (1 + x) dx
\]
બંને બાજુ સંકલન કરતા:
\[
\int \frac{dy}{1 + y} = \int (1 + x) dx
\]
\[
\ln|1 + y| = x + \frac{x^2}{2} + C
\]
\[
1 + y = Ae^{x + x^2/2} \quad \text{જ્યાં } A = e^C
\]
\[
y = Ae^{x + x^2/2} - 1
\]
\end{solutionbox}

\questionmarks{5(B).2}{4}{વિકલ સમીકરણ $\frac{dy}{dx} + y = e^x$ ઉકેલો.}

\begin{solutionbox}
\textbf{ઉકેલ}:
આ $\frac{dy}{dx} + Py = Q$ સ્વરૂપનું પ્રથમ કક્ષાનું સુરેખ વિકલ સમીકરણ છે, જ્યાં $P = 1$ અને $Q = e^x$.

સંકલ્પકારક અવયવ: $I.F. = e^{\int P \, dx} = e^{\int 1 \, dx} = e^x$

સમીકરણને $e^x$ વડે ગુણતા:
\[
e^x \frac{dy}{dx} + e^x y = e^{2x}
\]
\[
\frac{d}{dx}(ye^x) = e^{2x}
\]
બંને બાજુ સંકલન કરતા:
\[
ye^x = \int e^{2x} dx = \frac{e^{2x}}{2} + C
\]
\[
y = \frac{e^x}{2} + Ce^{-x}
\]
\end{solutionbox}

\questionmarks{5(B).3}{4}{વિકલ સમીકરણ $\frac{dy}{dx} - y\tan x = 1$ ઉકેલો.}

\begin{solutionbox}
\textbf{ઉકેલ}:
આ પ્રથમ કક્ષાનું સુરેખ વિકલ સમીકરણ છે, જ્યાં $P = -\tan x$ અને $Q = 1$.

સંકલ્પકારક અવયવ: $I.F. = e^{\int (-\tan x) dx} = e^{\ln|\cos x|} = \cos x$

સમીકરણને $\cos x$ વડે ગુણતા:
\[
\cos x \frac{dy}{dx} - y\cos x \tan x = \cos x
\]
\[
\cos x \frac{dy}{dx} - y\sin x = \cos x
\]
\[
\frac{d}{dx}(y\cos x) = \cos x
\]
બંને બાજુ સંકલન કરતા:
\[
y\cos x = \int \cos x \, dx = \sin x + C
\]
\[
y = \tan x + \frac{C}{\cos x} = \tan x + C\sec x
\]
\end{solutionbox}

\end{document}
