\documentclass{article}

% content/resources/templates/preamble.tex
\usepackage[margin=0.6in]{geometry}
\author{Milav Dabgar}
\usepackage{amsmath,amssymb,amsthm}
\usepackage{booktabs}
\usepackage{multirow}
\usepackage{xcolor}
\usepackage{tcolorbox}
\tcbuselibrary{breakable,skins}
\usepackage[colorlinks=true,linkcolor=blue]{hyperref}
\usepackage{titlesec}
\usepackage{enumitem}
\usepackage{tikz}
\usepackage{pgfplots}
\usepackage{circuitikz}
\usepackage[version=4]{mhchem}
\usepackage{longtable}
\usepackage{array}
\usepackage{float}
\usepackage{caption}
\usepackage{listings}

\lstset{
  basicstyle=\small\ttfamily,
  breaklines=true,
  breakatwhitespace=false,
  postbreak=\mbox{\textcolor{red}{$\hookrightarrow$}\space},
  float=false,
  numbers=left,
  numberstyle=\tiny\color{gray},
  numbersep=10pt,
  xleftmargin=2em,
  keywordstyle=\color{blue},
  commentstyle=\color{green!60!black},
  stringstyle=\color{purple},
  backgroundcolor=\color{gray!5},
  showstringspaces=false,
  tabsize=2,
  captionpos=b,
  keepspaces=true,
  columns=flexible
}

\pgfplotsset{compat=1.18}
\usetikzlibrary{shapes,arrows,positioning,calc,patterns,decorations.pathmorphing,decorations.markings,arrows.meta}

% Color scheme
\definecolor{headcolor}{RGB}{0,102,204}
\definecolor{keycolor}{RGB}{220,20,60}
\definecolor{solutioncolor}{RGB}{34,139,34}
\definecolor{mnemoniccolor}{RGB}{148,0,211}
\definecolor{codecolor}{RGB}{0,0,100}

% Spacing
\setlength{\parskip}{3pt}
\setlist[itemize]{nosep}
\setlist[enumerate]{nosep}

% Title formatting
\titleformat{\section}{\Large\bfseries\color{headcolor}}{\thesection}{1em}{}
\titleformat{\subsection}{\large\bfseries\color{headcolor}}{\thesubsection}{1em}{}

% Pandoc tightlist compatibility
\providecommand{\tightlist}{%
  \setlength{\itemsep}{0pt}\setlength{\parskip}{0pt}}

% Pandoc longtable compatibility
\newcounter{none}
\def\thenone{}


% content/resources/templates/gujarati-boxes.tex
\usepackage{fontspec}
\usepackage{polyglossia}

% Set Gujarati as main language (document is primarily in Gujarati)
% Note: gloss-gujarati.ldf doesn't exist in polyglossia, but it will use hyphenation patterns
\setdefaultlanguage{gujarati}
\setotherlanguage{english}

% Configure Gujarati font properly
% Use Language=Default to prevent polyglossia from trying to add language-specific features
% that don't exist for Gujarati, which causes "empty feature" warnings
\newfontfamily\gujaratifont[Script=Gujarati,AutoFakeBold=2.5,AutoFakeSlant=0.3]{Noto Sans Gujarati}
\setmainfont[Script=Gujarati,AutoFakeBold=2.5,AutoFakeSlant=0.3]{Noto Sans Gujarati}
% Use Noto Sans Gujarati for monospace to support Gujarati in text
\setmonofont[Scale=0.9]{Noto Sans Gujarati}

% Configure English to use the same font
\newfontfamily\englishfont[Script=Gujarati,AutoFakeBold=2.5,AutoFakeSlant=0.3]{Noto Sans Gujarati}

% Translations for polyglossia
\gappto\captionsgujarati{
  \renewcommand{\tablename}{કોષ્ટક}
  \renewcommand{\figurename}{આકૃતિ}
}

% Helper for TikZ nodes to ensure Gujarati font
\newcommand{\gu}[1]{{\gujaratifont #1}}

% Custom environments
\newtcolorbox{solutionbox}{
    breakable,
    enhanced,
    colback=solutioncolor!5!white,
    colframe=solutioncolor!75!black,
    fonttitle=\bfseries,
    title=જવાબ
}

\newtcolorbox{solutionboxnobreak}{
 colback=solutioncolor!5!white,
 colframe=solutioncolor!75!black,
 fonttitle=\bfseries,
 title=જવાબ
}

\newtcolorbox{keyformula}{
 breakable,
 enhanced,
 colback=keycolor!5!white,
 colframe=keycolor!75!black,
 fonttitle=\bfseries,
 title=રાસાયણિક સમીકરણ/સૂત્ર
}

\newtcolorbox{mnemonicbox}{
 breakable,
 enhanced,
 colback=mnemoniccolor!5!white,
 colframe=mnemoniccolor!75!black,
 fonttitle=\bfseries,
 title=મેમરી ટ્રીક
}


% Custom commands for GTU solutions
% This file defines semantic commands for consistent formatting

% Question command with automatic formatting
\newcommand{\question}[2]{%
  \section*{Question #1}%
  \textbf{#2}%
}

% OR question variant
\newcommand{\questionor}[2]{%
  \section*{Question #1 OR}%
  \textbf{#2}%
}

% Proper table environment with caption
\newenvironment{answertable}[1]{%
  \begin{table}[htbp]
  \centering
  \caption{#1}
}{%
  \end{table}
}

% Proper figure environment for diagrams
\newenvironment{answerdiagram}[1]{%
  \begin{figure}[htbp]
  \centering
  \caption{#1}
}{%
  \end{figure}
}

% Semantic markup for key terms
\newcommand{\keyword}[1]{\textbf{#1}}
\newcommand{\code}[1]{\texttt{#1}}
\newcommand{\classname}[1]{\texttt{#1}}
\newcommand{\methodname}[1]{\texttt{#1}}

% Proper quotation marks
\newcommand{\mnemonic}[1]{``#1''}


\title{એન્જિનિયરિંગ મેથેમેટિક્સ (4320002) - સમર 2024 સોલ્યુશન}
\date{જૂન 26, 2024}

\begin{document}
\maketitle

\questionmarks{1}{14}{નીચે આપેલા વિકલ્પોમાંથી યોગ્ય વિકલ્પ પસંદ કરી ખાલી જગ્યા પૂરો.}

\questionmarks{1.1}{1}{શ્રેણિક $A = \begin{bmatrix} 1 & 2 \\ 0 & -1 \\ 3 & 4 \end{bmatrix}$ ની કક્ષા (Order) \_\_\_\_\_\_ છે.}

\begin{solutionbox}
\textbf{જવાબ}: (b) 3 × 2

\textbf{ઉકેલ}:
શ્રેણિકની કક્ષા (હાર) × (સ્તંભ) દ્વારા આપવામાં આવે છે
શ્રેણિક A માં 3 હાર અને 2 સ્તંભ છે
તેથી, કક્ષા = 3 × 2
\end{solutionbox}

\questionmarks{1.2}{1}{જો $A = \begin{bmatrix} \sin \theta & -\cos \theta \\ \cos \theta & \sin \theta \end{bmatrix}$ હોય તો $A^{-1} = $ \_\_\_\_\_\_}

\begin{solutionbox}
\textbf{જવાબ}: (d) $A^T$

\textbf{ઉકેલ}:
લાંબિક શ્રેણિક (orthogonal matrices) માટે, $A^{-1} = A^T$
કેમ કે $AA^T = I$, આપણને $A^{-1} = A^T$ મળે છે.
\end{solutionbox}

\questionmarks{1.3}{1}{$\begin{bmatrix} 1 & 2 \\ 5 & 0 \end{bmatrix} \times \begin{bmatrix} -1 & 6 \\ 2 & 1 \end{bmatrix} = $ \_\_\_\_\_\_}

\begin{solutionbox}
\textbf{જવાબ}: (a) $\begin{bmatrix} 3 & 8 \\ -5 & 30 \end{bmatrix}$

\textbf{ઉકેલ}:
\[
\begin{bmatrix} 1 & 2 \\ 5 & 0 \end{bmatrix} \times \begin{bmatrix} -1 & 6 \\ 2 & 1 \end{bmatrix}
\]
\[
= \begin{bmatrix} 1(-1) + 2(2) & 1(6) + 2(1) \\ 5(-1) + 0(2) & 5(6) + 0(1) \end{bmatrix}
\]
\[
= \begin{bmatrix} -1 + 4 & 6 + 2 \\ -5 + 0 & 30 + 0 \end{bmatrix} = \begin{bmatrix} 3 & 8 \\ -5 & 30 \end{bmatrix}
\]
\end{solutionbox}

\questionmarks{1.4}{1}{જો $A = \begin{bmatrix} a & c \\ b & d \end{bmatrix}$ હોય તો $A^T = $ \_\_\_\_\_\_}

\begin{solutionbox}
\textbf{જવાબ}: (b) $\begin{bmatrix} a & b \\ c & d \end{bmatrix}$

\textbf{ઉકેલ}:
શ્રેણિકનો પરિવર્ત (transpose) હાર અને સ્તંભની અદલાબદલી કરીને મેળવવામાં આવે છે.
\[
A^T = \begin{bmatrix} a & b \\ c & d \end{bmatrix}
\]
\end{solutionbox}

\questionmarks{1.5}{1}{$\frac{d}{dx}(4^x) = $ \_\_\_\_\_\_}

\begin{solutionbox}
\textbf{જવાબ}: (a) $4^x \log_e 4$

\textbf{ઉકેલ}:
$\frac{d}{dx}(a^x) = a^x \ln a$
તેથી, $\frac{d}{dx}(4^x) = 4^x \ln 4 = 4^x \log_e 4$
\end{solutionbox}

\questionmarks{1.6}{1}{$\frac{d}{dx}(\sin^2 x + \cos^2 x) = $ \_\_\_\_\_\_}

\begin{solutionbox}
\textbf{જવાબ}: (b) 0

\textbf{ઉકેલ}:
$\sin^2 x + \cos^2 x = 1$ (ત્રિકોણમિતીય નિત્યસમ)
$\frac{d}{dx}(1) = 0$
\end{solutionbox}

\questionmarks{1.7}{1}{જો $x = \sin \theta, y = \cos \theta$ હોય તો $\frac{dy}{dx} = $ \_\_\_\_\_\_}

\begin{solutionbox}
\textbf{જવાબ}: (d) $-\cot \theta$

\textbf{ઉકેલ}:
$\frac{dx}{d\theta} = \cos \theta$, $\frac{dy}{d\theta} = -\sin \theta$
\[
\frac{dy}{dx} = \frac{dy/d\theta}{dx/d\theta} = \frac{-\sin \theta}{\cos \theta} = -\tan \theta = -\cot \theta
\]
\end{solutionbox}

\questionmarks{1.8}{1}{$\int x^7 dx = $ \_\_\_\_\_\_}

\begin{solutionbox}
\textbf{જવાબ}: (c) $\frac{x^8}{8}$

\textbf{ઉકેલ}:
$\int x^n dx = \frac{x^{n+1}}{n+1} + c$
$\int x^7 dx = \frac{x^8}{8} + c$
\end{solutionbox}

\questionmarks{1.9}{1}{$\int_{-2}^{2} x^5 dx = $ \_\_\_\_\_\_}

\begin{solutionbox}
\textbf{જવાબ}: (b) 0

\textbf{ઉકેલ}:
$x^5$ એ અયુગ્મ (odd) વિધેય છે.
અયુગ્મ વિધેયો માટે, $\int_{-a}^{a} f(x) dx = 0$
તેથી, $\int_{-2}^{2} x^5 dx = 0$
\end{solutionbox}

\questionmarks{1.10}{1}{$\int \frac{\cos x}{\sin x} dx = $ \_\_\_\_\_\_}

\begin{solutionbox}
\textbf{જવાબ}: (d) $\log|\sin x|$

\textbf{ઉકેલ}:
ધારો કે $u = \sin x$, તેથી $du = \cos x dx$
\[
\int \frac{\cos x}{\sin x} dx = \int \frac{du}{u} = \log|u| + c = \log|\sin x| + c
\]
\end{solutionbox}

\questionmarks{1.11}{1}{વિકલ સમીકરણ $\left(\frac{d^3y}{dx^3}\right)^2 + \left(\frac{d^2y}{dx^2}\right)^4 + y = 0$ ની કક્ષા (order) \_\_\_\_\_\_ છે}

\begin{solutionbox}
\textbf{જવાબ}: (a) 3

\textbf{ઉકેલ}:
વિકલ સમીકરણની કક્ષા એટલે તેમાં રહેલા ઉચ્ચતમ કક્ષાના વિકલિતની કક્ષા.
ઉચ્ચતમ વિકલિત $\frac{d^3y}{dx^3}$ છે, તેથી કક્ષા = 3
\end{solutionbox}

\questionmarks{1.12}{1}{વિકલ સમીકરણ $\frac{dy}{dx} + y = 3x$ નો સંકલ્પકારક અવયવ (integrating factor) \_\_\_\_\_\_ છે}

\begin{solutionbox}
\textbf{જવાબ}: (c) $e^x$

\textbf{ઉકેલ}:
સુરેખ વિકલ સમીકરણ $\frac{dy}{dx} + Py = Q$ માટે
સંકલ્પકારક અવયવ = $e^{\int P dx} = e^{\int 1 dx} = e^x$
\end{solutionbox}

\questionmarks{1.13}{1}{$i^7 = $ \_\_\_\_\_\_}

\begin{solutionbox}
\textbf{જવાબ}: (b) $-i$

\textbf{ઉકેલ}:
$i^1 = i, i^2 = -1, i^3 = -i, i^4 = 1$
$i^7 = i^4 \cdot i^3 = 1 \cdot (-i) = -i$
\end{solutionbox}

\questionmarks{1.14}{1}{$\arg(1+i) = $ \_\_\_\_\_\_}

\begin{solutionbox}
\textbf{જવાબ}: (c) $\frac{\pi}{4}$

\textbf{ઉકેલ}:
$\arg(a + bi) = \tan^{-1}\left(\frac{b}{a}\right)$
$\arg(1 + i) = \tan^{-1}\left(\frac{1}{1}\right) = \tan^{-1}(1) = \frac{\pi}{4}$
\end{solutionbox}

\questionmarks{2(A)}{6}{કોઈપણ બે લખો.}

\questionmarks{2(A).1}{3}{જો $A = \begin{bmatrix} 2 & 1 \\ 3 & 0 \end{bmatrix}$ અને $B = \begin{bmatrix} 4 & -1 \\ 2 & 3 \end{bmatrix}$ હોય તો સાબિત કરો કે $(A + B)^T = A^T + B^T$}

\begin{solutionbox}
\textbf{ઉકેલ}:
\[
A + B = \begin{bmatrix} 2 & 1 \\ 3 & 0 \end{bmatrix} + \begin{bmatrix} 4 & -1 \\ 2 & 3 \end{bmatrix} = \begin{bmatrix} 6 & 0 \\ 5 & 3 \end{bmatrix}
\]
\[
(A + B)^T = \begin{bmatrix} 6 & 5 \\ 0 & 3 \end{bmatrix}
\]
\[
A^T = \begin{bmatrix} 2 & 3 \\ 1 & 0 \end{bmatrix}, \quad B^T = \begin{bmatrix} 4 & 2 \\ -1 & 3 \end{bmatrix}
\]
\[
A^T + B^T = \begin{bmatrix} 2 & 3 \\ 1 & 0 \end{bmatrix} + \begin{bmatrix} 4 & 2 \\ -1 & 3 \end{bmatrix} = \begin{bmatrix} 6 & 5 \\ 0 & 3 \end{bmatrix}
\]
તેથી, $(A + B)^T = A^T + B^T$ \checkmark \textbf{સાબિત થયું}
\end{solutionbox}

\questionmarks{2(A).2}{3}{જો $A = \begin{bmatrix} 1 & 1 \\ 2 & 3 \end{bmatrix}$ હોય તો દર્શાવો કે $A \cdot A^{-1} = I$}

\begin{solutionbox}
\textbf{ઉકેલ}:
પ્રથમ, $A^{-1}$ શોધો:
$|A| = 1(3) - 1(2) = 3 - 2 = 1$

\[
A^{-1} = \frac{1}{|A|} \text{adj}(A) = \frac{1}{1} \begin{bmatrix} 3 & -1 \\ -2 & 1 \end{bmatrix} = \begin{bmatrix} 3 & -1 \\ -2 & 1 \end{bmatrix}
\]
હવે $A \cdot A^{-1} = I$ ચકાસીએ:
\[
A \cdot A^{-1} = \begin{bmatrix} 1 & 1 \\ 2 & 3 \end{bmatrix} \begin{bmatrix} 3 & -1 \\ -2 & 1 \end{bmatrix}
\]
\[
= \begin{bmatrix} 1(3) + 1(-2) & 1(-1) + 1(1) \\ 2(3) + 3(-2) & 2(-1) + 3(1) \end{bmatrix}
\]
\[
= \begin{bmatrix} 3 - 2 & -1 + 1 \\ 6 - 6 & -2 + 3 \end{bmatrix} = \begin{bmatrix} 1 & 0 \\ 0 & 1 \end{bmatrix} = I \quad \text{\checkmark \textbf{સાબિત થયું}}
\]
\end{solutionbox}

\questionmarks{2(A).3}{3}{વિકલ સમીકરણ $x dy + y dx = 0$ ઉકેલો.}

\begin{solutionbox}
\textbf{ઉકેલ}:
$x dy + y dx = 0$
$x dy = -y dx$
$\frac{dy}{y} = -\frac{dx}{x}$

બંને બાજુ સંકલન કરતા:
\[
\int \frac{dy}{y} = -\int \frac{dx}{x}
\]
\[
\ln|y| = -\ln|x| + c_1
\]
\[
\ln|y| + \ln|x| = c_1
\]
\[
\ln|xy| = c_1
\]
\[
|xy| = e^{c_1} = c \quad \text{(જ્યાં } c = e^{c_1} \text{ અચળાંક છે)}
\]
તેથી, $xy = \pm c$ અથવા \textbf{$xy = k$} જ્યાં $k$ એક સ્વૈર અચળાંક છે.
\end{solutionbox}

\questionmarks{2(B)}{8}{કોઈપણ બે લખો.}

\questionmarks{2(B).1}{4}{જો $A = \begin{bmatrix} 3 & 1 \\ -1 & 2 \end{bmatrix}$ હોય તો દર્શાવો કે $A^2 - 5A + 7I = 0$}

\begin{solutionbox}
\textbf{ઉકેલ}:
પ્રથમ, $A^2$ ગણો:
\[
A^2 = \begin{bmatrix} 3 & 1 \\ -1 & 2 \end{bmatrix} \begin{bmatrix} 3 & 1 \\ -1 & 2 \end{bmatrix}
\]
\[
= \begin{bmatrix} 3(3) + 1(-1) & 3(1) + 1(2) \\ -1(3) + 2(-1) & -1(1) + 2(2) \end{bmatrix}
\]
\[
= \begin{bmatrix} 9 - 1 & 3 + 2 \\ -3 - 2 & -1 + 4 \end{bmatrix} = \begin{bmatrix} 8 & 5 \\ -5 & 3 \end{bmatrix}
\]
હવે $5A$ ગણો:
\[
5A = 5\begin{bmatrix} 3 & 1 \\ -1 & 2 \end{bmatrix} = \begin{bmatrix} 15 & 5 \\ -5 & 10 \end{bmatrix}
\]
તથા $7I$:
\[
7I = 7\begin{bmatrix} 1 & 0 \\ 0 & 1 \end{bmatrix} = \begin{bmatrix} 7 & 0 \\ 0 & 7 \end{bmatrix}
\]
હવે $A^2 - 5A + 7I = 0$ ચકાસીએ:
\[
A^2 - 5A + 7I = \begin{bmatrix} 8 & 5 \\ -5 & 3 \end{bmatrix} - \begin{bmatrix} 15 & 5 \\ -5 & 10 \end{bmatrix} + \begin{bmatrix} 7 & 0 \\ 0 & 7 \end{bmatrix}
\]
\[
= \begin{bmatrix} 8 - 15 + 7 & 5 - 5 + 0 \\ -5 + 5 + 0 & 3 - 10 + 7 \end{bmatrix} = \begin{bmatrix} 0 & 0 \\ 0 & 0 \end{bmatrix} = 0 \quad \text{\checkmark \textbf{સાબિત થયું}}
\]
\end{solutionbox}

\questionmarks{2(B).2}{4}{જો $A = \begin{bmatrix} -4 & -3 & -3 \\ 1 & 0 & 1 \\ 4 & 4 & 3 \end{bmatrix}$ હોય તો સાબિત કરો કે $\text{adj } A = A$}

\begin{solutionbox}
\textbf{ઉકેલ}:
$\text{adj } A$ શોધવા માટે, આપણે સહઅવયવ શ્રેણિક શોધવો પડશે અને પછી તેનો પરિવર્ત કરવો પડશે.

સહઅવયવો (Cofactors):
$C_{11} = (-1)^{1+1} \begin{vmatrix} 0 & 1 \\ 4 & 3 \end{vmatrix} = 0(3) - 1(4) = -4$

$C_{12} = (-1)^{1+2} \begin{vmatrix} 1 & 1 \\ 4 & 3 \end{vmatrix} = -(1(3) - 1(4)) = -(3-4) = 1$

$C_{13} = (-1)^{1+3} \begin{vmatrix} 1 & 0 \\ 4 & 4 \end{vmatrix} = 1(4) - 0(4) = 4$

$C_{21} = (-1)^{2+1} \begin{vmatrix} -3 & -3 \\ 4 & 3 \end{vmatrix} = -((-3)(3) - (-3)(4)) = -(-9+12) = -3$

$C_{22} = (-1)^{2+2} \begin{vmatrix} -4 & -3 \\ 4 & 3 \end{vmatrix} = (-4)(3) - (-3)(4) = -12+12 = 0$

$C_{23} = (-1)^{2+3} \begin{vmatrix} -4 & -3 \\ 4 & 4 \end{vmatrix} = -((-4)(4) - (-3)(4)) = -(-16+12) = -(-4) = 4$

$C_{31} = (-1)^{3+1} \begin{vmatrix} -3 & -3 \\ 0 & 1 \end{vmatrix} = (-3)(1) - (-3)(0) = -3$

$C_{32} = (-1)^{3+2} \begin{vmatrix} -4 & -3 \\ 1 & 1 \end{vmatrix} = -((-4)(1) - (-3)(1)) = -(-4+3) = -(-1) = 1$

$C_{33} = (-1)^{3+3} \begin{vmatrix} -4 & -3 \\ 1 & 0 \end{vmatrix} = (-4)(0) - (-3)(1) = 0+3 = 3$

સહઅવયવ શ્રેણિક = $\begin{bmatrix} -4 & 1 & 4 \\ -3 & 0 & 4 \\ -3 & 1 & 3 \end{bmatrix}$

\[
\text{adj } A = \text{(સહઅવયવ શ્રેણિક)}^T = \begin{bmatrix} -4 & -3 & -3 \\ 1 & 0 & 1 \\ 4 & 4 & 3 \end{bmatrix} = A \quad \text{\checkmark \textbf{સાબિત થયું}}
\]
\end{solutionbox}

\questionmarks{2(B).3}{4}{શ્રેણિક પદ્ધતિનો ઉપયોગ કરીને નીચેના સુરેખ સમીકરણો ઉકેલો: $3x + 2y = 5$, $2x - y = 1$}

\begin{solutionbox}
\textbf{ઉકેલ}:
સમીકરણોને $AX = B$ તરીકે લખી શકાય:
$A = \begin{bmatrix} 3 & 2 \\ 2 & -1 \end{bmatrix}$, $X = \begin{bmatrix} x \\ y \end{bmatrix}$, $B = \begin{bmatrix} 5 \\ 1 \end{bmatrix}$

$|A| = 3(-1) - 2(2) = -3 - 4 = -7$ શોધો.

\[
A^{-1} = \frac{1}{-7} \begin{bmatrix} -1 & -2 \\ -2 & 3 \end{bmatrix} = \begin{bmatrix} \frac{1}{7} & \frac{2}{7} \\ \frac{2}{7} & -\frac{3}{7} \end{bmatrix}
\]
\[
X = A^{-1}B = \begin{bmatrix} \frac{1}{7} & \frac{2}{7} \\ \frac{2}{7} & -\frac{3}{7} \end{bmatrix} \begin{bmatrix} 5 \\ 1 \end{bmatrix}
\]
\[
= \begin{bmatrix} \frac{1}{7}(5) + \frac{2}{7}(1) \\ \frac{2}{7}(5) - \frac{3}{7}(1) \end{bmatrix} = \begin{bmatrix} \frac{5+2}{7} \\ \frac{10-3}{7} \end{bmatrix} = \begin{bmatrix} 1 \\ 1 \end{bmatrix}
\]
તેથી, \textbf{$x = 1, y = 1$}
\end{solutionbox}

\questionmarks{3(A)}{6}{કોઈપણ બે લખો.}

\questionmarks{3(A).1}{3}{વિકલનની વ્યાખ્યાનો ઉપયોગ કરીને $x^5$ નું $x$ પ્રત્યે વિકલન શોધો.}

\begin{solutionbox}
\textbf{ઉકેલ}:
વ્યાખ્યા અનુસાર: $\frac{dy}{dx} = \lim_{h \to 0} \frac{f(x+h) - f(x)}{h}$

$f(x) = x^5$ માટે:
\[
\frac{d}{dx}(x^5) = \lim_{h \to 0} \frac{(x+h)^5 - x^5}{h}
\]
દ્વિપદી પ્રમેયનો ઉપયોગ કરતા: $(x+h)^5 = x^5 + 5x^4h + 10x^3h^2 + 10x^2h^3 + 5xh^4 + h^5$

\[
\frac{d}{dx}(x^5) = \lim_{h \to 0} \frac{x^5 + 5x^4h + 10x^3h^2 + 10x^2h^3 + 5xh^4 + h^5 - x^5}{h}
\]
\[
= \lim_{h \to 0} \frac{5x^4h + 10x^3h^2 + 10x^2h^3 + 5xh^4 + h^5}{h}
\]
\[
= \lim_{h \to 0} (5x^4 + 10x^3h + 10x^2h^2 + 5xh^3 + h^4)
\]
\[
= 5x^4 + 0 + 0 + 0 + 0 = 5x^4
\]
તેથી, \textbf{$\frac{d}{dx}(x^5) = 5x^4$}
\end{solutionbox}

\questionmarks{3(A).2}{3}{જો $y = \frac{x^2-1}{x^2+1}$ હોય તો $\frac{dy}{dx}$ શોધો.}

\begin{solutionbox}
\textbf{ઉકેલ}:
ભાગાકારના નિયમનો ઉપયોગ કરતા: $\frac{d}{dx}\left(\frac{u}{v}\right) = \frac{v\frac{du}{dx} - u\frac{dv}{dx}}{v^2}$

અહીં, $u = x^2 - 1$, $v = x^2 + 1$
$\frac{du}{dx} = 2x$, $\frac{dv}{dx} = 2x$

\[
\frac{dy}{dx} = \frac{(x^2+1)(2x) - (x^2-1)(2x)}{(x^2+1)^2}
\]
\[
= \frac{2x(x^2+1) - 2x(x^2-1)}{(x^2+1)^2}
\]
\[
= \frac{2x[(x^2+1) - (x^2-1)]}{(x^2+1)^2}
\]
\[
= \frac{2x[x^2+1-x^2+1]}{(x^2+1)^2}
\]
\[
= \frac{2x \cdot 2}{(x^2+1)^2} = \frac{4x}{(x^2+1)^2}
\]
તેથી, \textbf{$\frac{dy}{dx} = \frac{4x}{(x^2+1)^2}$}
\end{solutionbox}

\questionmarks{3(A).3}{3}{સંકલન શોધો: $\int \frac{x^2+5x+6}{x^2+2x} dx$}

\begin{solutionbox}
\textbf{ઉકેલ}:
પ્રથમ, બહુપદી ભાગાકાર કરતા:
\[
\frac{x^2+5x+6}{x^2+2x} = 1 + \frac{3x+6}{x^2+2x}
\]
\[
\int \frac{x^2+5x+6}{x^2+2x} dx = \int \left(1 + \frac{3x+6}{x^2+2x}\right) dx
\]
\[
= \int 1 dx + \int \frac{3x+6}{x^2+2x} dx
\]
\[
= x + \int \frac{3x+6}{x(x+2)} dx
\]
બીજા સંકલન માટે:
$\frac{3x+6}{x(x+2)} = \frac{3(x+2)}{x(x+2)} = \frac{3}{x}$

\[
\int \frac{3x+6}{x(x+2)} dx = \int \frac{3}{x} dx = 3\ln|x| + c
\]
તેથી: $\int \frac{x^2+5x+6}{x^2+2x} dx = x + 3\ln|x| + c$
\end{solutionbox}

\questionmarks{3(B)}{8}{કોઈપણ બે લખો.}

\questionmarks{3(B).1}{4}{જો $y = \log(\sec x + \tan x)$ હોય તો $\frac{dy}{dx}$ શોધો.}

\begin{solutionbox}
\textbf{ઉકેલ}:
$y = \log(\sec x + \tan x)$

\[
\frac{dy}{dx} = \frac{1}{\sec x + \tan x} \cdot \frac{d}{dx}(\sec x + \tan x)
\]
$\frac{d}{dx}(\sec x) = \sec x \tan x$
$\frac{d}{dx}(\tan x) = \sec^2 x$

\[
\frac{dy}{dx} = \frac{1}{\sec x + \tan x} \cdot (\sec x \tan x + \sec^2 x)
\]
\[
= \frac{\sec x(\tan x + \sec x)}{\sec x + \tan x}
\]
\[
= \frac{\sec x(\sec x + \tan x)}{\sec x + \tan x} = \sec x
\]
તેથી, \textbf{$\frac{dy}{dx} = \sec x$}
\end{solutionbox}

\questionmarks{3(B).2}{4}{જો $y = 2e^{3x} + 3e^{-2x}$ હોય તો સાબિત કરો કે $\frac{d^2y}{dx^2} - \frac{dy}{dx} - 6y = 0$}

\begin{solutionbox}
\textbf{ઉકેલ}:
$y = 2e^{3x} + 3e^{-2x}$

પ્રથમ વિકલિત:
$\frac{dy}{dx} = 2(3e^{3x}) + 3(-2e^{-2x}) = 6e^{3x} - 6e^{-2x}$

દ્વિતીય વિકલિત:
$\frac{d^2y}{dx^2} = 6(3e^{3x}) - 6(-2e^{-2x}) = 18e^{3x} + 12e^{-2x}$

હવે સમીકરણ ચકાસીએ:
$\frac{d^2y}{dx^2} - \frac{dy}{dx} - 6y$
\[
= (18e^{3x} + 12e^{-2x}) - (6e^{3x} - 6e^{-2x}) - 6(2e^{3x} + 3e^{-2x})
\]
\[
= 18e^{3x} + 12e^{-2x} - 6e^{3x} + 6e^{-2x} - 12e^{3x} - 18e^{-2x}
\]
\[
= e^{3x}(18 - 6 - 12) + e^{-2x}(12 + 6 - 18)
\]
\[
= e^{3x}(0) + e^{-2x}(0) = 0 \quad \text{\checkmark \textbf{સાબિત થયું}}
\]
\end{solutionbox}

\questionmarks{3(B).3}{4}{વિધેય $f(x) = x^3 - 3x + 11$ ની મહત્તમ અને ન્યૂનતમ કિંમતો શોધો.}

\begin{solutionbox}
\textbf{ઉકેલ}:
$f(x) = x^3 - 3x + 11$

પ્રથમ વિકલિત: $f'(x) = 3x^2 - 3 = 3(x^2 - 1) = 3(x-1)(x+1)$

નિર્ણાયક બિંદુઓ માટે, $f'(x) = 0$ લેતા:
$3(x-1)(x+1) = 0$
$x = 1$ અથવા $x = -1$

દ્વિતીય વિકલિત: $f''(x) = 6x$

$x = 1$ આગળ: $f''(1) = 6 > 0$ $\rightarrow$ સ્થાનીય ન્યૂનતમ
$x = -1$ આગળ: $f''(-1) = -6 < 0$ $\rightarrow$ સ્થાનીય મહત્તમ

વિધેયની કિંમતો:
$x = 1$ આગળ: $f(1) = 1^3 - 3(1) + 11 = 1 - 3 + 11 = 9$
$x = -1$ આગળ: $f(-1) = (-1)^3 - 3(-1) + 11 = -1 + 3 + 11 = 13$

તેથી:
\begin{itemize}
    \item \textbf{સ્થાનીય મહત્તમ કિંમત = 13 (જ્યાં $x = -1$)}
    \item \textbf{સ્થાનીય ન્યૂનતમ કિંમત = 9 (જ્યાં $x = 1$)}
\end{itemize}
\end{solutionbox}

\questionmarks{4(A)}{6}{કોઈપણ બે લખો.}

\questionmarks{4(A).1}{3}{કિંમત શોધો: $\int \frac{\cos(\log x)}{x} dx$}

\begin{solutionbox}
\textbf{ઉકેલ}:
ધારો કે $u = \log x$, તેથી $du = \frac{1}{x} dx$

\[
\int \frac{\cos(\log x)}{x} dx = \int \cos u \, du = \sin u + c
\]
પાછું મુકતા: $u = \log x$

તેથી, \textbf{$\int \frac{\cos(\log x)}{x} dx = \sin(\log x) + c$}
\end{solutionbox}

\questionmarks{4(A).2}{3}{કિંમત શોધો: $\int x \sin x \, dx$}

\begin{solutionbox}
\textbf{ઉકેલ}:
ખંડશઃ સંકલનનો ઉપયોગ કરતા: $\int u \, dv = uv - \int v \, du$

ધારો કે $u = x$ અને $dv = \sin x \, dx$
તેથી $du = dx$ અને $v = -\cos x$

\[
\int x \sin x \, dx = x(-\cos x) - \int (-\cos x) dx
\]
\[
= -x \cos x + \int \cos x \, dx
\]
\[
= -x \cos x + \sin x + c
\]
તેથી, \textbf{$\int x \sin x \, dx = \sin x - x \cos x + c$}
\end{solutionbox}

\questionmarks{4(A).3}{3}{જો $(2x - y) + 2y i = 6 + 4i$ હોય તો $x$ અને $y$ શોધો.}

\begin{solutionbox}
\textbf{ઉકેલ}:
$(2x - y) + 2y i = 6 + 4i$

વાસ્તવિક અને કાલ્પનિક ભાગો સરખાવતા:
વાસ્તવિક ભાગ: $2x - y = 6$ ... (1)
કાલ્પનિક ભાગ: $2y = 4$ ... (2)

સમીકરણ (2) પરથી: $y = 2$

સમીકરણ (1) માં મુકતા:
$2x - 2 = 6$
$2x = 8$
$x = 4$

તેથી, \textbf{$x = 4$ અને $y = 2$}
\end{solutionbox}

\questionmarks{4(B)}{8}{કોઈપણ બે લખો.}

\questionmarks{4(B).1}{4}{વક્ર $y = x^2$, રેખાઓ $x = 1$, $x = 2$ અને X-અક્ષ વડે ઘેરાયેલા પ્રદેશનું ક્ષેત્રફળ શોધો.}

\begin{solutionbox}
\textbf{ઉકેલ}:
માગેલ ક્ષેત્રફળ નીચે મુજબ છે:
\[
A = \int_1^2 x^2 \, dx
\]
\[
A = \left[\frac{x^3}{3}\right]_1^2
\]
\[
= \frac{2^3}{3} - \frac{1^3}{3}
\]
\[
= \frac{8}{3} - \frac{1}{3}
\]
\[
= \frac{7}{3} \text{ ચોરસ એકમ}
\]
તેથી, \textbf{ક્ષેત્રફળ = $\frac{7}{3}$ ચોરસ એકમ}
\end{solutionbox}

\questionmarks{4(B).2}{4}{નિયત સંકલન $\int_0^{\pi/2} \frac{\sec x}{\sec x + \csc x} dx$ ની કિંમત શોધો.}

\begin{solutionbox}
\textbf{ઉકેલ}:
ધારો કે $I = \int_0^{\pi/2} \frac{\sec x}{\sec x + \csc x} dx$

$\int_0^a f(x) dx = \int_0^a f(a-x) dx$ ગુણધર્મનો ઉપયોગ કરતા:
\[
I = \int_0^{\pi/2} \frac{\sec(\pi/2 - x)}{\sec(\pi/2 - x) + \csc(\pi/2 - x)} dx
\]
કારણ કે $\sec(\pi/2 - x) = \csc x$ અને $\csc(\pi/2 - x) = \sec x$:
\[
I = \int_0^{\pi/2} \frac{\csc x}{\csc x + \sec x} dx
\]
બંને સમીકરણોનો સરવાળો કરતા:
\[
2I = \int_0^{\pi/2} \frac{\sec x}{\sec x + \csc x} dx + \int_0^{\pi/2} \frac{\csc x}{\sec x + \csc x} dx
\]
\[
2I = \int_0^{\pi/2} \frac{\sec x + \csc x}{\sec x + \csc x} dx = \int_0^{\pi/2} 1 \, dx = \frac{\pi}{2}
\]
તેથી, $I = \frac{\pi}{4}$

\textbf{જવાબ: $\int_0^{\pi/2} \frac{\sec x}{\sec x + \csc x} dx = \frac{\pi}{4}$}
\end{solutionbox}

\questionmarks{4(B).3}{4}{જો $\alpha + i\beta = \frac{1}{a + ib}$ હોય તો સાબિત કરો કે $(\alpha^2 + \beta^2)(a^2 + b^2) = 1$}

\begin{solutionbox}
\textbf{ઉકેલ}:
આપેલ છે: $\alpha + i\beta = \frac{1}{a + ib}$

જમણી બાજુ કરણી લેતા (Rationalizing):
\[
\alpha + i\beta = \frac{1}{a + ib} \cdot \frac{a - ib}{a - ib} = \frac{a - ib}{a^2 + b^2}
\]
\[
\alpha + i\beta = \frac{a}{a^2 + b^2} - i\frac{b}{a^2 + b^2}
\]
વાસ્તવિક અને કાલ્પનિક ભાગો સરખાવતા:
$\alpha = \frac{a}{a^2 + b^2}$ અને $\beta = -\frac{b}{a^2 + b^2}$

હવે $\alpha^2 + \beta^2$ ગણતા:
\[
\alpha^2 + \beta^2 = \left(\frac{a}{a^2 + b^2}\right)^2 + \left(-\frac{b}{a^2 + b^2}\right)^2
\]
\[
= \frac{a^2}{(a^2 + b^2)^2} + \frac{b^2}{(a^2 + b^2)^2}
\]
\[
= \frac{a^2 + b^2}{(a^2 + b^2)^2} = \frac{1}{a^2 + b^2}
\]
તેથી:
$(\alpha^2 + \beta^2)(a^2 + b^2) = \frac{1}{a^2 + b^2} \cdot (a^2 + b^2) = 1$ \checkmark \textbf{સાબિત થયું}
\end{solutionbox}

\questionmarks{5(A)}{6}{કોઈપણ બે લખો.}

\questionmarks{5(A).1}{3}{સંકર સંખ્યા $\frac{2+3i}{3+2i}$ ની અનુબદ્ધ સંકર સંખ્યા અને માનાંક શોધો.}

\begin{solutionbox}
\textbf{ઉકેલ}:
પ્રથમ, સંકર સંખ્યાને સાદુરૂપ આપીએ:
\[
\frac{2+3i}{3+2i} = \frac{2+3i}{3+2i} \cdot \frac{3-2i}{3-2i}
\]
\[
= \frac{(2+3i)(3-2i)}{(3+2i)(3-2i)}
\]
\[
= \frac{6 - 4i + 9i - 6i^2}{9 - 4i^2}
\]
\[
= \frac{6 + 5i - 6(-1)}{9 - 4(-1)}
\]
\[
= \frac{6 + 5i + 6}{9 + 4} = \frac{12 + 5i}{13}
\]
તેથી $\frac{2+3i}{3+2i} = \frac{12}{13} + \frac{5}{13}i$

\textbf{અનુબદ્ધ કરણી (Conjugate)}: $\overline{\frac{2+3i}{3+2i}} = \frac{12}{13} - \frac{5}{13}i$

\textbf{માનાંક (Modulus)}: $\left|\frac{2+3i}{3+2i}\right| = \sqrt{\left(\frac{12}{13}\right)^2 + \left(\frac{5}{13}\right)^2}$
\[
= \sqrt{\frac{144}{169} + \frac{25}{169}} = \sqrt{\frac{169}{169}} = \sqrt{1} = 1
\]
\end{solutionbox}

\questionmarks{5(A).2}{3}{સાદુરૂપ આપો: $\frac{(\cos 3\theta + i \sin 3\theta)^{-4} (\cos \theta - i \sin \theta)^{-5}}{(\cos 2\theta - i \sin 2\theta)^7}$}

\begin{solutionbox}
\textbf{ઉકેલ}:
દ-મદ-પ્રેમય (De Moivre's theorem) નો ઉપયોગ કરતા: $(\cos \theta + i \sin \theta)^n = \cos n\theta + i \sin n\theta$

વળી, $\cos \theta - i \sin \theta = \cos(-\theta) + i \sin(-\theta)$

$(\cos 3\theta + i \sin 3\theta)^{-4} = \cos(-12\theta) + i \sin(-12\theta)$

$(\cos \theta - i \sin \theta)^{-5} = (\cos(-\theta) + i \sin(-\theta))^{-5} = \cos(5\theta) + i \sin(5\theta)$

$(\cos 2\theta - i \sin 2\theta)^7 = (\cos(-2\theta) + i \sin(-2\theta))^7 = \cos(-14\theta) + i \sin(-14\theta)$

તેથી:
\[
\frac{(\cos 3\theta + i \sin 3\theta)^{-4} (\cos \theta - i \sin \theta)^{-5}}{(\cos 2\theta - i \sin 2\theta)^7}
\]
\[
= \frac{[\cos(-12\theta) + i \sin(-12\theta)][\cos(5\theta) + i \sin(5\theta)]}{\cos(-14\theta) + i \sin(-14\theta)}
\]
\[
= \frac{\cos(-12\theta + 5\theta) + i \sin(-12\theta + 5\theta)}{\cos(-14\theta) + i \sin(-14\theta)}
\]
\[
= \frac{\cos(-7\theta) + i \sin(-7\theta)}{\cos(-14\theta) + i \sin(-14\theta)}
\]
\[
= \cos(-7\theta + 14\theta) + i \sin(-7\theta + 14\theta)
\]
\[
= \cos(7\theta) + i \sin(7\theta)
\]
\end{solutionbox}

\questionmarks{5(A).3}{3}{સંકર સંખ્યા $1 + \sqrt{3}i$ ને ધ્રુવીય સ્વરૂપમાં દર્શાવો.}

\begin{solutionbox}
\textbf{ઉકેલ}:
સંકર સંખ્યા $z = a + bi$ માટે, ધ્રુવીય સ્વરૂપ $z = r(\cos \theta + i \sin \theta)$ છે.

અહીં, $a = 1$, $b = \sqrt{3}$

\textbf{માનાંક (Modulus)}: $r = |z| = \sqrt{a^2 + b^2} = \sqrt{1^2 + (\sqrt{3})^2} = \sqrt{1 + 3} = \sqrt{4} = 2$

\textbf{કોણાંક (Argument)}: $\theta = \tan^{-1}\left(\frac{b}{a}\right) = \tan^{-1}\left(\frac{\sqrt{3}}{1}\right) = \tan^{-1}(\sqrt{3}) = \frac{\pi}{3}$

તેથી, ધ્રુવીય સ્વરૂપ છે:
\textbf{$1 + \sqrt{3}i = 2\left(\cos \frac{\pi}{3} + i \sin \frac{\pi}{3}\right)$}
\end{solutionbox}

\questionmarks{5(B)}{8}{કોઈપણ બે લખો.}

\questionmarks{5(B).1}{4}{ઉકેલો: $\tan y \, dx + \tan x \sec^2 y \, dy = 0$}

\begin{solutionbox}
\textbf{ઉકેલ}:
$\tan y \, dx + \tan x \sec^2 y \, dy = 0$

પદો અલગ કરતા: $\tan y \, dx = -\tan x \sec^2 y \, dy$

$\frac{dx}{\tan x} = -\frac{\sec^2 y \, dy}{\tan y}$

$\frac{\cos x}{\sin x} dx = -\frac{dy}{\sin y \cos y}$

$\cot x \, dx = -\frac{dy}{\sin y \cos y}$

કારણ કે $\frac{1}{\sin y \cos y} = \frac{2}{2\sin y \cos y} = \frac{2}{\sin 2y}$:

$\cot x \, dx = -\frac{2 dy}{\sin 2y}$

બંને બાજુ સંકલન કરતા:
$\int \cot x \, dx = -2 \int \csc(2y) \, dy$

$\ln|\sin x| = -2 \cdot \left(-\frac{1}{2}\ln|\csc(2y) + \cot(2y)|\right) + c$

$\ln|\sin x| = \ln|\csc(2y) + \cot(2y)| + c$

તેથી: \textbf{$\sin x \cdot [\csc(2y) + \cot(2y)] = k$} જ્યાં $k$ એક અચળાંક છે.
\end{solutionbox}

\questionmarks{5(B).2}{4}{ઉકેલો: $x \frac{dy}{dx} - y = x^2$}

\begin{solutionbox}
\textbf{ઉકેલ}:
$x \frac{dy}{dx} - y = x^2$

$x$ વડે ભાગતા: $\frac{dy}{dx} - \frac{y}{x} = x$

આ $\frac{dy}{dx} + Py = Q$ સ્વરૂપનું સુરેખ વિકલ સમીકરણ છે.

અહીં, $P = -\frac{1}{x}$ અને $Q = x$

સંકલ્પકારક અવયવ (I.F.): $I.F. = e^{\int P dx} = e^{\int -\frac{1}{x} dx} = e^{-\ln|x|} = \frac{1}{x}$

સમીકરણને I.F. વડે ગુણતા:
$\frac{1}{x} \frac{dy}{dx} - \frac{y}{x^2} = 1$

આને આમ લખી શકાય: $\frac{d}{dx}\left(\frac{y}{x}\right) = 1$

સંકલન કરતા: $\frac{y}{x} = x + c$

તેથી: \textbf{$y = x^2 + cx$}
\end{solutionbox}

\questionmarks{5(B).3}{4}{ઉકેલો: $\frac{dy}{dx} + \frac{y}{x} = e^x$, $y(0) = 3$}

\begin{solutionbox}
\textbf{ઉકેલ}:
આ સુરેખ વિકલ સમીકરણ છે: $\frac{dy}{dx} + \frac{y}{x} = e^x$

અહીં, $P = \frac{1}{x}$, $Q = e^x$

સંકલ્પકારક અવયવ: $I.F. = e^{\int \frac{1}{x} dx} = e^{\ln|x|} = x$ (ધારો કે $x > 0$)

સમીકરણને I.F. વડે ગુણતા:
$x \frac{dy}{dx} + y = xe^x$

આને આમ લખી શકાય: $\frac{d}{dx}(xy) = xe^x$

બંને બાજુ સંકલન કરતા:
$xy = \int xe^x dx$

ખંડશઃ સંકલનનો ઉપયોગ કરતા $\int xe^x dx$:
ધારો કે $u = x$, $dv = e^x dx$
તેથી $du = dx$, $v = e^x$

$\int xe^x dx = xe^x - \int e^x dx = xe^x - e^x = e^x(x-1)$

તેથી: $xy = e^x(x-1) + c$

તેથી: $y = \frac{e^x(x-1) + c}{x}$

\textbf{સામાન્ય ઉકેલ: $y = \frac{e^x(x-1) + c}{x}$ જ્યાં $x \neq 0$}
\end{solutionbox}

\end{document}
