\documentclass{article}

% content/resources/templates/preamble.tex
\usepackage[margin=0.6in]{geometry}
\author{Milav Dabgar}
\usepackage{amsmath,amssymb,amsthm}
\usepackage{booktabs}
\usepackage{multirow}
\usepackage{xcolor}
\usepackage{tcolorbox}
\tcbuselibrary{breakable,skins}
\usepackage[colorlinks=true,linkcolor=blue]{hyperref}
\usepackage{titlesec}
\usepackage{enumitem}
\usepackage{tikz}
\usepackage{pgfplots}
\usepackage{circuitikz}
\usepackage[version=4]{mhchem}
\usepackage{longtable}
\usepackage{array}
\usepackage{float}
\usepackage{caption}
\usepackage{listings}

\lstset{
  basicstyle=\small\ttfamily,
  breaklines=true,
  breakatwhitespace=false,
  postbreak=\mbox{\textcolor{red}{$\hookrightarrow$}\space},
  float=false,
  numbers=left,
  numberstyle=\tiny\color{gray},
  numbersep=10pt,
  xleftmargin=2em,
  keywordstyle=\color{blue},
  commentstyle=\color{green!60!black},
  stringstyle=\color{purple},
  backgroundcolor=\color{gray!5},
  showstringspaces=false,
  tabsize=2,
  captionpos=b,
  keepspaces=true,
  columns=flexible
}

\pgfplotsset{compat=1.18}
\usetikzlibrary{shapes,arrows,positioning,calc,patterns,decorations.pathmorphing,decorations.markings,arrows.meta}

% Color scheme
\definecolor{headcolor}{RGB}{0,102,204}
\definecolor{keycolor}{RGB}{220,20,60}
\definecolor{solutioncolor}{RGB}{34,139,34}
\definecolor{mnemoniccolor}{RGB}{148,0,211}
\definecolor{codecolor}{RGB}{0,0,100}

% Spacing
\setlength{\parskip}{3pt}
\setlist[itemize]{nosep}
\setlist[enumerate]{nosep}

% Title formatting
\titleformat{\section}{\Large\bfseries\color{headcolor}}{\thesection}{1em}{}
\titleformat{\subsection}{\large\bfseries\color{headcolor}}{\thesubsection}{1em}{}

% Pandoc tightlist compatibility
\providecommand{\tightlist}{%
  \setlength{\itemsep}{0pt}\setlength{\parskip}{0pt}}

% Pandoc longtable compatibility
\newcounter{none}
\def\thenone{}


% content/resources/templates/gujarati-boxes.tex
\usepackage{fontspec}
\usepackage{polyglossia}

% Set Gujarati as main language (document is primarily in Gujarati)
% Note: gloss-gujarati.ldf doesn't exist in polyglossia, but it will use hyphenation patterns
\setdefaultlanguage{gujarati}
\setotherlanguage{english}

% Configure Gujarati font properly
% Use Language=Default to prevent polyglossia from trying to add language-specific features
% that don't exist for Gujarati, which causes "empty feature" warnings
\newfontfamily\gujaratifont[Script=Gujarati,AutoFakeBold=2.5,AutoFakeSlant=0.3]{Noto Sans Gujarati}
\setmainfont[Script=Gujarati,AutoFakeBold=2.5,AutoFakeSlant=0.3]{Noto Sans Gujarati}
% Use Noto Sans Gujarati for monospace to support Gujarati in text
\setmonofont[Scale=0.9]{Noto Sans Gujarati}

% Configure English to use the same font
\newfontfamily\englishfont[Script=Gujarati,AutoFakeBold=2.5,AutoFakeSlant=0.3]{Noto Sans Gujarati}

% Translations for polyglossia
\gappto\captionsgujarati{
  \renewcommand{\tablename}{કોષ્ટક}
  \renewcommand{\figurename}{આકૃતિ}
}

% Helper for TikZ nodes to ensure Gujarati font
\newcommand{\gu}[1]{{\gujaratifont #1}}

% Custom environments
\newtcolorbox{solutionbox}{
    breakable,
    enhanced,
    colback=solutioncolor!5!white,
    colframe=solutioncolor!75!black,
    fonttitle=\bfseries,
    title=જવાબ
}

\newtcolorbox{solutionboxnobreak}{
 colback=solutioncolor!5!white,
 colframe=solutioncolor!75!black,
 fonttitle=\bfseries,
 title=જવાબ
}

\newtcolorbox{keyformula}{
 breakable,
 enhanced,
 colback=keycolor!5!white,
 colframe=keycolor!75!black,
 fonttitle=\bfseries,
 title=રાસાયણિક સમીકરણ/સૂત્ર
}

\newtcolorbox{mnemonicbox}{
 breakable,
 enhanced,
 colback=mnemoniccolor!5!white,
 colframe=mnemoniccolor!75!black,
 fonttitle=\bfseries,
 title=મેમરી ટ્રીક
}


% Custom commands for GTU solutions
% This file defines semantic commands for consistent formatting

% Question command with automatic formatting
\newcommand{\question}[2]{%
  \section*{Question #1}%
  \textbf{#2}%
}

% OR question variant
\newcommand{\questionor}[2]{%
  \section*{Question #1 OR}%
  \textbf{#2}%
}

% Proper table environment with caption
\newenvironment{answertable}[1]{%
  \begin{table}[htbp]
  \centering
  \caption{#1}
}{%
  \end{table}
}

% Proper figure environment for diagrams
\newenvironment{answerdiagram}[1]{%
  \begin{figure}[htbp]
  \centering
  \caption{#1}
}{%
  \end{figure}
}

% Semantic markup for key terms
\newcommand{\keyword}[1]{\textbf{#1}}
\newcommand{\code}[1]{\texttt{#1}}
\newcommand{\classname}[1]{\texttt{#1}}
\newcommand{\methodname}[1]{\texttt{#1}}

% Proper quotation marks
\newcommand{\mnemonic}[1]{``#1''}


\title{એન્જિનિયરિંગ મેથેમેટિક્સ (4320002) - સમર 2022 સોલ્યુશન}
\date{સપ્ટેમ્બર 06, 2022}

\begin{document}
\maketitle

\questionmarks{1}{14}{નીચે આપેલા વિકલ્પોમાંથી યોગ્ય વિકલ્પ પસંદ કરી ખાલી જગ્યા પૂરો}

\questionmarks{1.1}{1}{જો $A_{2\times3}$ અને $B_{3\times4}$ બે શ્રેણિકો હોય તો AB ની કક્ષા (order) શોધો =\_\_\_\_\_\_}

\begin{solutionbox}
\textbf{જવાબ}: b. $2\times4$

\textbf{ઉકેલ}:
જ્યારે શ્રેણિકોનો ગુણાકાર કરવામાં આવે છે, જો $A$ ની કક્ષા $m\times n$ હોય અને $B$ ની કક્ષા $n\times p$ હોય, તો $AB$ ની કક્ષા $m\times p$ થશે.
અહીં આપેલ છે: $A_{2\times3}$ અને $B_{3\times4}$
તેથી, $AB$ ની કક્ષા $2\times4$ થશે.
\end{solutionbox}

\questionmarks{1.2}{1}{જો $A = [1\ 3\ 2]$ અને $B = \begin{bmatrix} 1 \\ 2 \\ 1 \end{bmatrix}$ હોય તો AB શોધો =\_\_\_\_\_\_}

\begin{solutionbox}
\textbf{જવાબ}: b. 9

\textbf{ઉકેલ}:
\[
AB = [1\ 3\ 2] \begin{bmatrix} 1 \\ 2 \\ 1 \end{bmatrix} = 1(1) + 3(2) + 2(1) = 1 + 6 + 2 = 9
\]
\end{solutionbox}

\questionmarks{1.3}{1}{$A.I_2 = A$ હોય તો $I_2$ =\_\_\_\_\_\_}

\begin{solutionbox}
\textbf{જવાબ}: c. $\begin{bmatrix} 1 & 0 \\ 0 & 1 \end{bmatrix}$

\textbf{ઉકેલ}:
$I_2$ એ $2\times2$ કક્ષાનો એકમ શ્રેણિક (identity matrix) છે, જેમાં મુખ્ય વિકર્ણ પર 1 અને અન્ય જગ્યાએ 0 હોય છે.
\end{solutionbox}

\questionmarks{1.4}{1}{જો $\frac{d}{dx}(\sin^2 x + \cos^2 x) =$ \_\_\_\_\_\_}

\begin{solutionbox}
\textbf{જવાબ}: b. 0

\textbf{ઉકેલ}:
જેમ કે $\sin^2 x + \cos^2 x = 1$ (મૂળભૂત ત્રિકોણમિતીય નિત્યસમ)
\[
\frac{d}{dx}(\sin^2 x + \cos^2 x) = \frac{d}{dx}(1) = 0
\]
\end{solutionbox}

\questionmarks{1.5}{1}{$\frac{d}{dx}(\cot x) =$ \_\_\_\_\_\_}

\begin{solutionbox}
\textbf{જવાબ}: d. $-\csc^2 x$

\textbf{ઉકેલ}:
\[
\frac{d}{dx}(\cot x) = -\csc^2 x
\]
\end{solutionbox}

\questionmarks{1.6}{1}{$\frac{d}{dx}\log(\sin x)$ તો $\frac{d^2y}{dx^2} =$ \_\_\_\_\_\_ શોધો}

\begin{solutionbox}
\textbf{જવાબ}: d. $-\cot^2 x$

\textbf{ઉકેલ}:
ધારો કે $y = \log(\sin x)$
\[
\frac{dy}{dx} = \frac{1}{\sin x} \cdot \cos x = \cot x
\]
\[
\frac{d^2y}{dx^2} = \frac{d}{dx}(\cot x) = -\csc^2 x
\]
જોકે, $\csc^2 x = 1 + \cot^2 x$ હોવાથી, જવાબ $-\csc^2 x$ છે.
\end{solutionbox}

\questionmarks{1.7}{1}{$\frac{d}{dx}(\frac{1}{x}) =$ \_\_\_\_\_\_}

\begin{solutionbox}
\textbf{જવાબ}: c. $-\frac{1}{x^2}$

\textbf{ઉકેલ}:
\[
\frac{d}{dx}\left(\frac{1}{x}\right) = \frac{d}{dx}(x^{-1}) = -1 \cdot x^{-2} = -\frac{1}{x^2}
\]
\end{solutionbox}

\questionmarks{1.8}{1}{જો $\int x^5 dx =$ \_\_\_\_\_\_+ c}

\begin{solutionbox}
\textbf{જવાબ}: a. $\frac{x^6}{6}$

\textbf{ઉકેલ}:
\[
\int x^5 dx = \frac{x^{5+1}}{5+1} + c = \frac{x^6}{6} + c
\]
\end{solutionbox}

\questionmarks{1.9}{1}{$\int_0^{2\pi} (\sin^2 \theta + \cos^2 \theta)d\theta =$ \_\_\_\_\_\_+ c}

\begin{solutionbox}
\textbf{જવાબ}: a. $2\pi$

\textbf{ઉકેલ}:
\[
\int_0^{2\pi} (\sin^2 \theta + \cos^2 \theta)d\theta = \int_0^{2\pi} 1 \, d\theta = [\theta]_0^{2\pi} = 2\pi - 0 = 2\pi
\]
\end{solutionbox}

\questionmarks{1.10}{1}{$\int_{-1}^{1} x^3 dx =$ \_\_\_\_\_\_+ c}

\begin{solutionbox}
\textbf{જવાબ}: c. 0

\textbf{ઉકેલ}:
\[
\int_{-1}^{1} x^3 dx = \left[\frac{x^4}{4}\right]_{-1}^{1} = \frac{1^4}{4} - \frac{(-1)^4}{4} = \frac{1}{4} - \frac{1}{4} = 0
\]
\end{solutionbox}

\questionmarks{1.11}{1}{વિકલ સમીકરણ $x^2 \frac{d^2y}{dx^2} + 3y^2 = 0$ ની કક્ષા (order) અને પરિમાણ (degree) =\_\_\_\_\_\_ છે}

\begin{solutionbox}
\textbf{જવાબ}: c. 2 અને 1

\textbf{ઉકેલ}:
કક્ષા (Order) એ સર્વોચ્ચ વિકલિત છે = 2 ($\frac{d^2y}{dx^2}$ પરથી)
પરિમાણ (Degree) એ સર્વોચ્ચ વિકલિતની ઘાત છે = 1
\end{solutionbox}

\questionmarks{1.12}{1}{વિકલ સમીકરણ $\frac{dy}{dx} + py = Q$ નો સંકલ્પકારક અવયવ (integrating factor) \_\_\_\_\_\_ છે}

\begin{solutionbox}
\textbf{જવાબ}: c. $e^{\int p dx}$

\textbf{ઉકેલ}:
પ્રથમ કક્ષાના સુરેખ વિકલ સમીકરણ $\frac{dy}{dx} + py = Q$ માટે, સંકલ્પકારક અવયવ $e^{\int p dx}$ છે.
\end{solutionbox}

\questionmarks{1.13}{1}{$i^4 =$ \_\_\_\_\_\_}

\begin{solutionbox}
\textbf{જવાબ}: a. 1

\textbf{ઉકેલ}:
\[
i^4 = (i^2)^2 = (-1)^2 = 1
\]
\end{solutionbox}

\questionmarks{1.14}{1}{(3+4i)(4-5i) =\_\_\_\_\_\_}

\begin{solutionbox}
\textbf{જવાબ}: d. -32+ i

\textbf{ઉકેલ}:
\begin{align*}
(3+4i)(4-5i) &= 3(4) + 3(-5i) + 4i(4) + 4i(-5i) \\
&= 12 - 15i + 16i - 20i^2 \\
&= 12 + i - 20(-1) \\
&= 12 + i + 20 = 32 + i
\end{align*}

ફરીથી ગણતરી કરતા:
$(3+4i)(4-5i) = 12 - 15i + 16i - 20i^2 = 12 + i + 20 = 32 + i$

સાચો જવાબ b. 32+ i હોવો જોઈએ, પરંતુ વિકલ્પ d માં -32+ i આપેલ છે. વિકલ્પોમાં ભૂલ હોઈ શકે છે.
\end{solutionbox}

\questionmarks{2(a)}{6}{કોઈપણ બે લખો}

\questionmarks{2.1}{3}{જો $A = \begin{bmatrix} 1 & -1 & 1 \\ 3 & 2 & 1 \end{bmatrix}$ અને $B = \begin{bmatrix} 1 & 2 \\ 4 & 2 \\ 1 & 7 \end{bmatrix}$ હોય તો AB અને BA શોધો.}

\begin{solutionbox}
\textbf{ઉકેલ}:

\textbf{AB ની ગણતરી:}
\[
AB = \begin{bmatrix} 1 & -1 & 1 \\ 3 & 2 & 1 \end{bmatrix} \begin{bmatrix} 1 & 2 \\ 4 & 2 \\ 1 & 7 \end{bmatrix}
\]
\[
AB = \begin{bmatrix} 1(1) + (-1)(4) + 1(1) & 1(2) + (-1)(2) + 1(7) \\ 3(1) + 2(4) + 1(1) & 3(2) + 2(2) + 1(7) \end{bmatrix}
\]
\[
AB = \begin{bmatrix} 1 - 4 + 1 & 2 - 2 + 7 \\ 3 + 8 + 1 & 6 + 4 + 7 \end{bmatrix} = \begin{bmatrix} -2 & 7 \\ 12 & 17 \end{bmatrix}
\]

\textbf{BA ની ગણતરી:}
\[
BA = \begin{bmatrix} 1 & 2 \\ 4 & 2 \\ 1 & 7 \end{bmatrix} \begin{bmatrix} 1 & -1 & 1 \\ 3 & 2 & 1 \end{bmatrix}
\]
\[
BA = \begin{bmatrix} 1(1) + 2(3) & 1(-1) + 2(2) & 1(1) + 2(1) \\ 4(1) + 2(3) & 4(-1) + 2(2) & 4(1) + 2(1) \\ 1(1) + 7(3) & 1(-1) + 7(2) & 1(1) + 7(1) \end{bmatrix}
\]
\[
BA = \begin{bmatrix} 7 & 3 & 3 \\ 10 & 0 & 6 \\ 22 & 13 & 8 \end{bmatrix}
\]
\end{solutionbox}

\questionmarks{2.2}{3}{જો $A = \begin{bmatrix} -1 & 2 \\ 3 & 1 \end{bmatrix}$ હોય તો સાબિત કરો કે $A^2 - 7I_2 = 0$}

\begin{solutionbox}
\textbf{ઉકેલ}:
\[
A^2 = \begin{bmatrix} -1 & 2 \\ 3 & 1 \end{bmatrix} \begin{bmatrix} -1 & 2 \\ 3 & 1 \end{bmatrix}
\]
\[
A^2 = \begin{bmatrix} (-1)(-1) + (2)(3) & (-1)(2) + (2)(1) \\ (3)(-1) + (1)(3) & (3)(2) + (1)(1) \end{bmatrix}
\]
\[
A^2 = \begin{bmatrix} 1 + 6 & -2 + 2 \\ -3 + 3 & 6 + 1 \end{bmatrix} = \begin{bmatrix} 7 & 0 \\ 0 & 7 \end{bmatrix}
\]
\[
7I_2 = 7\begin{bmatrix} 1 & 0 \\ 0 & 1 \end{bmatrix} = \begin{bmatrix} 7 & 0 \\ 0 & 7 \end{bmatrix}
\]
તેથી,
\[
A^2 - 7I_2 = \begin{bmatrix} 7 & 0 \\ 0 & 7 \end{bmatrix} - \begin{bmatrix} 7 & 0 \\ 0 & 7 \end{bmatrix} = \begin{bmatrix} 0 & 0 \\ 0 & 0 \end{bmatrix} = 0
\]
સાબિત થયું.
\end{solutionbox}

\questionmarks{2.3}{3}{$\frac{2+3i}{4-3i}$ નો વ્યસ્ત સંકર સંખ્યા શોધો}

\begin{solutionbox}
\textbf{ઉકેલ}:
પ્રથમ, $\frac{2+3i}{4-3i}$ શોધીએ:
\[
\frac{2+3i}{4-3i} = \frac{(2+3i)(4+3i)}{(4-3i)(4+3i)} = \frac{8 + 6i + 12i + 9i^2}{16 - 9i^2}
\]
\[
= \frac{8 + 18i - 9}{16 + 9} = \frac{-1 + 18i}{25} = -\frac{1}{25} + \frac{18}{25}i
\]
સંકર સંખ્યા $z = a + bi$ નો વ્યસ્ત $\frac{1}{z} = \frac{\bar{z}}{|z|^2}$ છે

ધારો કે $z = -\frac{1}{25} + \frac{18}{25}i$
\[
|z|^2 = \left(-\frac{1}{25}\right)^2 + \left(\frac{18}{25}\right)^2 = \frac{1}{625} + \frac{324}{625} = \frac{325}{625} = \frac{13}{25}
\]
\[
\bar{z} = -\frac{1}{25} - \frac{18}{25}i
\]
\[
\frac{1}{z} = \frac{-\frac{1}{25} - \frac{18}{25}i}{\frac{13}{25}} = \frac{-1 - 18i}{13}
\]
\end{solutionbox}

\questionmarks{2(b)}{8}{કોઈપણ બે લખો}

\questionmarks{2.1}{4}{2y+5x-4 =0 અને 7x +3y = 5 સમીકરણો શ્રેણિક પદ્ધતિનો ઉપયોગ કરીને ઉકેલો.}

\begin{solutionbox}
\textbf{ઉકેલ}:
સમીકરણો આ રીતે લખી શકાય:
\begin{align*}
5x + 2y &= 4 \\
7x + 3y &= 5
\end{align*}
શ્રેણિક સ્વરૂપમાં: $\begin{bmatrix} 5 & 2 \\ 7 & 3 \end{bmatrix} \begin{bmatrix} x \\ y \end{bmatrix} = \begin{bmatrix} 4 \\ 5 \end{bmatrix}$

ધારો કે $A = \begin{bmatrix} 5 & 2 \\ 7 & 3 \end{bmatrix}$
\[
|A| = 5(3) - 2(7) = 15 - 14 = 1
\]
\[
A^{-1} = \frac{1}{|A|} \begin{bmatrix} 3 & -2 \\ -7 & 5 \end{bmatrix} = \begin{bmatrix} 3 & -2 \\ -7 & 5 \end{bmatrix}
\]
\[
\begin{bmatrix} x \\ y \end{bmatrix} = A^{-1} \begin{bmatrix} 4 \\ 5 \end{bmatrix} = \begin{bmatrix} 3 & -2 \\ -7 & 5 \end{bmatrix} \begin{bmatrix} 4 \\ 5 \end{bmatrix}
\]
\[
\begin{bmatrix} x \\ y \end{bmatrix} = \begin{bmatrix} 3(4) + (-2)(5) \\ -7(4) + 5(5) \end{bmatrix} = \begin{bmatrix} 12 - 10 \\ -28 + 25 \end{bmatrix} = \begin{bmatrix} 2 \\ -3 \end{bmatrix}
\]
તેથી, $x = 2$ અને $y = -3$.
\end{solutionbox}

\questionmarks{2.2}{4}{જો $A = \begin{bmatrix} 2 & -2 \\ 3 & 1 \end{bmatrix}$ અને $B = \begin{bmatrix} -1 & 5 \\ 4 & -3 \end{bmatrix}$ હોય તો સાબિત કરો કે $(AB)^T = B^T.A^T$}

\begin{solutionbox}
\textbf{ઉકેલ}:
પ્રથમ, $AB$ શોધીએ:
\[
AB = \begin{bmatrix} 2 & -2 \\ 3 & 1 \end{bmatrix} \begin{bmatrix} -1 & 5 \\ 4 & -3 \end{bmatrix}
\]
\[
AB = \begin{bmatrix} 2(-1) + (-2)(4) & 2(5) + (-2)(-3) \\ 3(-1) + 1(4) & 3(5) + 1(-3) \end{bmatrix}
\]
\[
AB = \begin{bmatrix} -2 - 8 & 10 + 6 \\ -3 + 4 & 15 - 3 \end{bmatrix} = \begin{bmatrix} -10 & 16 \\ 1 & 12 \end{bmatrix}
\]
\[
(AB)^T = \begin{bmatrix} -10 & 1 \\ 16 & 12 \end{bmatrix}
\]
હવે, $B^T$ અને $A^T$ શોધીએ:
$A^T = \begin{bmatrix} 2 & 3 \\ -2 & 1 \end{bmatrix}$, $B^T = \begin{bmatrix} -1 & 4 \\ 5 & -3 \end{bmatrix}$
\[
B^T \cdot A^T = \begin{bmatrix} -1 & 4 \\ 5 & -3 \end{bmatrix} \begin{bmatrix} 2 & 3 \\ -2 & 1 \end{bmatrix}
\]
\[
B^T \cdot A^T = \begin{bmatrix} -1(2) + 4(-2) & -1(3) + 4(1) \\ 5(2) + (-3)(-2) & 5(3) + (-3)(1) \end{bmatrix}
\]
\[
B^T \cdot A^T = \begin{bmatrix} -2 - 8 & -3 + 4 \\ 10 + 6 & 15 - 3 \end{bmatrix} = \begin{bmatrix} -10 & 1 \\ 16 & 12 \end{bmatrix}
\]
કારણ કે $(AB)^T = B^T \cdot A^T$, ગુણધર્મ સાબિત થાય છે.
\end{solutionbox}

\questionmarks{2.3}{4}{સાદુરૂપ આપો: $\frac{(\cos2\theta+i\sin2\theta)^{-3}.(\cos3\theta-i\sin3\theta)^2}{(\cos2\theta+i\sin2\theta)^{-7}.(\cos5\theta-i\sin5\theta)^3}$}

\begin{solutionbox}
\textbf{ઉકેલ}:
દ-મોઈવ (De Moivre) પ્રમેયનો ઉપયોગ કરતા: $(\cos\theta + i\sin\theta)^n = \cos(n\theta) + i\sin(n\theta)$

$(\cos2\theta+i\sin2\theta)^{-3} = \cos(-6\theta) + i\sin(-6\theta) = \cos(6\theta) - i\sin(6\theta)$

$(\cos3\theta-i\sin3\theta)^2 = (\cos(-3\theta) + i\sin(-3\theta))^2 = \cos(-6\theta) + i\sin(-6\theta) = \cos(6\theta) - i\sin(6\theta)$

$(\cos2\theta+i\sin2\theta)^{-7} = \cos(-14\theta) + i\sin(-14\theta) = \cos(14\theta) - i\sin(14\theta)$

$(\cos5\theta-i\sin5\theta)^3 = (\cos(-5\theta) + i\sin(-5\theta))^3 = \cos(-15\theta) + i\sin(-15\theta) = \cos(15\theta) - i\sin(15\theta)$

તેથી:
\[
\frac{[\cos(6\theta) - i\sin(6\theta)][\cos(6\theta) - i\sin(6\theta)]}{[\cos(14\theta) - i\sin(14\theta)][\cos(15\theta) - i\sin(15\theta)]}
\]
\[
= \frac{[\cos(6\theta) - i\sin(6\theta)]^2}{[\cos(14\theta) - i\sin(14\theta)][\cos(15\theta) - i\sin(15\theta)]}
\]
\[
= \frac{\cos(12\theta) - i\sin(12\theta)}{\cos(29\theta) - i\sin(29\theta)}
\]
\[
= \cos(12\theta - 29\theta) + i\sin(12\theta - 29\theta) = \cos(-17\theta) + i\sin(-17\theta) = \cos(17\theta) - i\sin(17\theta)
\]
\end{solutionbox}

\questionmarks{3(a)}{6}{કોઈપણ બે લખો}

\questionmarks{3.1}{3}{જો $y = \frac{1+\tan x}{1-\tan x}$ હોય તો $\frac{dy}{dx}$ શોધો}

\begin{solutionbox}
\textbf{ઉકેલ}:
ભાગાકારના નિયમનો ઉપયોગ કરતા: $\frac{d}{dx}\left[\frac{u}{v}\right] = \frac{v\frac{du}{dx} - u\frac{dv}{dx}}{v^2}$

ધારો કે $u = 1+\tan x$ અને $v = 1-\tan x$
\[
\frac{du}{dx} = \sec^2 x \quad \text{અને} \quad \frac{dv}{dx} = -\sec^2 x
\]
\[
\frac{dy}{dx} = \frac{(1-\tan x)(\sec^2 x) - (1+\tan x)(-\sec^2 x)}{(1-\tan x)^2}
\]
\[
= \frac{(1-\tan x)\sec^2 x + (1+\tan x)\sec^2 x}{(1-\tan x)^2}
\]
\[
= \frac{\sec^2 x[(1-\tan x) + (1+\tan x)]}{(1-\tan x)^2}
\]
\[
= \frac{2\sec^2 x}{(1-\tan x)^2}
\]
\end{solutionbox}

\questionmarks{3.2}{3}{વિકલનની વ્યાખ્યાનો ઉપયોગ કરી $x^3$ નું $x$ સાપેક્ષ વિકલન કરો.}

\begin{solutionbox}
\textbf{ઉકેલ}:
વ્યાખ્યાનો ઉપયોગ કરતા: $\frac{dy}{dx} = \lim_{h \to 0} \frac{f(x+h) - f(x)}{h}$

$f(x) = x^3$ માટે:
\[
\frac{d}{dx}(x^3) = \lim_{h \to 0} \frac{(x+h)^3 - x^3}{h}
\]
\[
= \lim_{h \to 0} \frac{x^3 + 3x^2h + 3xh^2 + h^3 - x^3}{h}
\]
\[
= \lim_{h \to 0} \frac{3x^2h + 3xh^2 + h^3}{h}
\]
\[
= \lim_{h \to 0} \frac{h(3x^2 + 3xh + h^2)}{h}
\]
\[
= \lim_{h \to 0} (3x^2 + 3xh + h^2)
\]
\[
= 3x^2 + 0 + 0 = 3x^2
\]
\end{solutionbox}

\questionmarks{3.3}{3}{સાદુરૂપ આપો: $\int \frac{4+3\cos x}{\sin^2 x} dx$}

\begin{solutionbox}
\textbf{ઉકેલ}:
\[
\int \frac{4+3\cos x}{\sin^2 x} dx = \int \frac{4}{\sin^2 x} dx + \int \frac{3\cos x}{\sin^2 x} dx
\]
\[
= 4\int \csc^2 x \, dx + 3\int \frac{\cos x}{\sin^2 x} dx
\]
પ્રથમ સંકલન માટે: $\int \csc^2 x \, dx = -\cot x$

બીજા સંકલન માટે, ધારો કે $u = \sin x$, તેથી $du = \cos x \, dx$:
\[
\int \frac{\cos x}{\sin^2 x} dx = \int \frac{1}{u^2} du = -\frac{1}{u} = -\frac{1}{\sin x} = -\csc x
\]
તેથી:
\[
\int \frac{4+3\cos x}{\sin^2 x} dx = 4(-\cot x) + 3(-\csc x) + C = -4\cot x - 3\csc x + C
\]
\end{solutionbox}

\questionmarks{3(b)}{8}{કોઈપણ બે લખો}

\questionmarks{3.1}{4}{જો $y = \log\left(\frac{\cos x}{1+\sin x}\right)$ હોય તો $\frac{dy}{dx}$ શોધો}

\begin{solutionbox}
\textbf{ઉકેલ}:
\[
y = \log\left(\frac{\cos x}{1+\sin x}\right) = \log(\cos x) - \log(1+\sin x)
\]
\[
\frac{dy}{dx} = \frac{d}{dx}[\log(\cos x)] - \frac{d}{dx}[\log(1+\sin x)]
\]
\[
= \frac{1}{\cos x} \cdot (-\sin x) - \frac{1}{1+\sin x} \cdot \cos x
\]
\[
= -\frac{\sin x}{\cos x} - \frac{\cos x}{1+\sin x}
\]
\[
= -\tan x - \frac{\cos x}{1+\sin x}
\]
આગળ સાદુરૂપ આપતા:
\[
= -\frac{\sin x(1+\sin x) + \cos^2 x}{\cos x(1+\sin x)}
\]
\[
= -\frac{\sin x + \sin^2 x + \cos^2 x}{\cos x(1+\sin x)}
\]
\[
= -\frac{\sin x + 1}{\cos x(1+\sin x)} = -\frac{1}{\cos x} = -\sec x
\]
\end{solutionbox}

\questionmarks{3.2}{4}{વિધેય $f(x) = 2x^3 - 15x^2 + 36x + 10$ ની મહત્તમ અને ન્યૂનતમ કિંમત શોધો.}

\begin{solutionbox}
\textbf{ઉકેલ}:
મહત્તમ અને ન્યૂનતમ કિંમત શોધવા માટે, આપણે $f'(x) = 0$ લઈએ:
\[
f'(x) = 6x^2 - 30x + 36 = 6(x^2 - 5x + 6) = 6(x-2)(x-3)
\]
$f'(x) = 0$ લેતા: $x = 2$ અથવા $x = 3$

બીજા વિકલિતની કસોટીનો ઉપયોગ કરીએ:
$f''(x) = 12x - 30$

$x = 2$ માટે: $f''(2) = 24 - 30 = -6 < 0 \to$ સ્થાનીય મહત્તમ (Local maximum)

$x = 3$ માટે: $f''(3) = 36 - 30 = 6 > 0 \to$ સ્થાનીય ન્યૂનતમ (Local minimum)

\textbf{કિંમતો:}
\[
f(2) = 2(8) - 15(4) + 36(2) + 10 = 16 - 60 + 72 + 10 = 38
\]
\[
f(3) = 2(27) - 15(9) + 36(3) + 10 = 54 - 135 + 108 + 10 = 37
\]

તેથી:
\begin{itemize}
    \item સ્થાનીય મહત્તમ કિંમત: 38 (at $x = 2$)
    \item સ્થાનીય ન્યૂનતમ કિંમત: 37 (at $x = 3$)
\end{itemize}
\end{solutionbox}

\questionmarks{3.3}{4}{જો $y = 2e^{-3x} + 3e^{2x}$ હોય તો સાબિત કરો કે $y_2 + y_1 - 6y = 0$.}

\begin{solutionbox}
\textbf{ઉકેલ}:
આપેલ: $y = 2e^{-3x} + 3e^{2x}$
\[
y_1 = \frac{dy}{dx} = 2(-3)e^{-3x} + 3(2)e^{2x} = -6e^{-3x} + 6e^{2x}
\]
\[
y_2 = \frac{d^2y}{dx^2} = -6(-3)e^{-3x} + 6(2)e^{2x} = 18e^{-3x} + 12e^{2x}
\]
હવે $y_2 + y_1 - 6y = 0$ ચકાસીએ:
\[
y_2 + y_1 - 6y = (18e^{-3x} + 12e^{2x}) + (-6e^{-3x} + 6e^{2x}) - 6(2e^{-3x} + 3e^{2x})
\]
\[
= 18e^{-3x} + 12e^{2x} - 6e^{-3x} + 6e^{2x} - 12e^{-3x} - 18e^{2x}
\]
\[
= (18 - 6 - 12)e^{-3x} + (12 + 6 - 18)e^{2x}
\]
\[
= 0 \cdot e^{-3x} + 0 \cdot e^{2x} = 0
\]
સાબિત થયું.
\end{solutionbox}

\questionmarks{4(a)}{6}{કોઈપણ બે લખો}

\questionmarks{4.1}{3}{કિંમત શોધો: $\int \frac{x^2}{1+x^6} dx$}

\begin{solutionbox}
\textbf{ઉકેલ}:
ધારો કે $u = x^3$, તેથી $du = 3x^2 dx$, એટલે કે $x^2 dx = \frac{1}{3} du$

\[
\int \frac{x^2}{1+x^6} dx = \int \frac{1}{1+(x^3)^2} \cdot x^2 dx = \int \frac{1}{1+u^2} \cdot \frac{1}{3} du
\]
\[
= \frac{1}{3} \int \frac{1}{1+u^2} du = \frac{1}{3} \tan^{-1}(u) + C
\]
\[
= \frac{1}{3} \tan^{-1}(x^3) + C
\]
\end{solutionbox}

\questionmarks{4.2}{3}{કિંમત શોધો: $\int x \log x \, dx$}

\begin{solutionbox}
\textbf{ઉકેલ}:
ખંડશઃ સંકલન (integration by parts) નો ઉપયોગ કરતા: $\int u \, dv = uv - \int v \, du$

ધારો કે $u = \log x$ અને $dv = x \, dx$
તેથી $du = \frac{1}{x} dx$ અને $v = \frac{x^2}{2}$
\[
\int x \log x \, dx = \log x \cdot \frac{x^2}{2} - \int \frac{x^2}{2} \cdot \frac{1}{x} dx
\]
\[
= \frac{x^2 \log x}{2} - \int \frac{x}{2} dx
\]
\[
= \frac{x^2 \log x}{2} - \frac{x^2}{4} + C
\]
\[
= \frac{x^2}{2}(\log x - \frac{1}{2}) + C
\]
\end{solutionbox}

\questionmarks{4.3}{3}{વિકલ સમીકરણ $x dy + y dx = 0$ ઉકેલો.}

\begin{solutionbox}
\textbf{ઉકેલ}:
આપેલ સમીકરણ: $x dy + y dx = 0$

આને આ રીતે લખી શકાય: $x dy = -y dx$

ચલ અલગ કરતા (Separating variables): $\frac{dy}{y} = -\frac{dx}{x}$

બંને બાજુ સંકલન કરતા:
\[
\int \frac{dy}{y} = \int -\frac{dx}{x}
\]
\[
\log|y| = -\log|x| + C_1
\]
\[
\log|y| + \log|x| = C_1
\]
\[
\log|xy| = C_1
\]
\[
|xy| = e^{C_1} = C \quad (\text{જ્યાં } C = e^{C_1})
\]
તેથી: $xy = \pm C$

વ્યાપક ઉકેલ છે: $xy = k$ (જ્યાં $k$ સ્વૈર અચળાંક છે)
\end{solutionbox}

\questionmarks{4(b)}{8}{કોઈપણ બે લખો}

\questionmarks{4.1}{4}{કિંમત શોધો: $\int_1^e \frac{(\log x)^2}{x} dx$}

\begin{solutionbox}
\textbf{ઉકેલ}:
ધારો કે $u = \log x$, તેથી $du = \frac{1}{x} dx$

જ્યારે $x = 1$: $u = \log 1 = 0$
જ્યારે $x = e$: $u = \log e = 1$
\[
\int_1^e \frac{(\log x)^2}{x} dx = \int_0^1 u^2 du
\]
\[
= \left[\frac{u^3}{3}\right]_0^1 = \frac{1^3}{3} - \frac{0^3}{3} = \frac{1}{3}
\]
\end{solutionbox}

\questionmarks{4.2}{4}{કિંમત શોધો: $\int_0^{\pi/2} \frac{\sec x}{\sec x + \cos x} dx$}

\begin{solutionbox}
\textbf{ઉકેલ}:
ધારો કે $I = \int_0^{\pi/2} \frac{\sec x}{\sec x + \cos x} dx$

પ્રથમ, વિધેયને સાદુરૂપ આપીએ:
\[
\frac{\sec x}{\sec x + \cos x} = \frac{\frac{1}{\cos x}}{\frac{1}{\cos x} + \cos x} = \frac{\frac{1}{\cos x}}{\frac{1 + \cos^2 x}{\cos x}} = \frac{1}{1 + \cos^2 x}
\]
તેથી $I = \int_0^{\pi/2} \frac{1}{1 + \cos^2 x} dx$

આદેશ $\tan(x/2) = t$ લેતા:
$\cos x = \frac{1-t^2}{1+t^2}$, $dx = \frac{2dt}{1+t^2}$

જ્યારે $x = 0$: $t = 0$
જ્યારે $x = \pi/2$: $t = 1$
\[
I = \int_0^1 \frac{1}{1 + \left(\frac{1-t^2}{1+t^2}\right)^2} \cdot \frac{2dt}{1+t^2}
\]
સાદુરૂપ આપ્યા બાદ (જેમાં બીજગણિતનો ઉપયોગ થાય છે), આનો જવાબ આવે છે:
\[
I = \frac{\pi}{2\sqrt{2}}
\]
\end{solutionbox}

\questionmarks{4.3}{4}{વિકલ સમીકરણ $\frac{dy}{dx} + \frac{y}{x} = e^x$, $y(0) = 2$ ઉકેલો.}

\begin{solutionbox}
\textbf{ઉકેલ}:
આ $\frac{dy}{dx} + P(x)y = Q(x)$ સ્વરૂપનું પ્રથમ કક્ષાનું સુરેખ વિકલ સમીકરણ છે.

અહીં, $P(x) = \frac{1}{x}$ અને $Q(x) = e^x$

સંકલ્પકારક અવયવ છે: $\mu(x) = e^{\int P(x) dx} = e^{\int \frac{1}{x} dx} = e^{\log|x|} = |x| = x$ ($x > 0$ માટે)

સમીકરણને સંકલ્પકારક અવયવ વડે ગુણતા:
\[
x\frac{dy}{dx} + y = xe^x
\]
ડાબી બાજુ $\frac{d}{dx}(xy)$ છે, તેથી:
\[
\frac{d}{dx}(xy) = xe^x
\]
બંને બાજુ સંકલન કરતા:
\[
xy = \int xe^x dx
\]
ખંડશઃ સંકલનનો ઉપયોગ કરતા $\int xe^x dx$:
ધારો કે $u = x$, $dv = e^x dx$
તેથી $du = dx$, $v = e^x$

\[
\int xe^x dx = xe^x - \int e^x dx = xe^x - e^x + C = e^x(x-1) + C
\]
તેથી: $xy = e^x(x-1) + C$
\[
y = \frac{e^x(x-1) + C}{x}
\]
પ્રારંભિક શરત $y(0) = 2$ નો ઉપયોગ કરતા:
આમાં એક સમસ્યા છે કારણ કે $x = 0$ પર ઉકેલ અવ્યાખ્યાયિત છે.

જો આપણે પ્રારંભિક શરત $x = 1$ માનીએ, અને $y(1) = 2$:
\[
2 = \frac{e^1(1-1) + C}{1} = \frac{0 + C}{1} = C
\]
તેથી $C = 2$, અને ઉકેલ છે:
\[
y = \frac{e^x(x-1) + 2}{x}
\]
\end{solutionbox}

\questionmarks{5(a)}{6}{કોઈપણ બે લખો}

\questionmarks{5.1}{3}{$\frac{3+7i}{1-i}$ ની અનુબદ્ધ સંકર સંખ્યા અને માનાંક શોધો.}

\begin{solutionbox}
\textbf{ઉકેલ}:
પ્રથમ, $\frac{3+7i}{1-i}$ ને સાદુરૂપ આપીએ:
\[
\frac{3+7i}{1-i} = \frac{(3+7i)(1+i)}{(1-i)(1+i)} = \frac{3 + 3i + 7i + 7i^2}{1 - i^2}
\]
\[
= \frac{3 + 10i - 7}{1 + 1} = \frac{-4 + 10i}{2} = -2 + 5i
\]
\textbf{અનુબદ્ધ:} $-2 + 5i$ ની અનુબદ્ધ સંકર સંખ્યા $-2 - 5i$ છે

\textbf{માનાંક:} $|{-2 + 5i}| = \sqrt{(-2)^2 + (5)^2} = \sqrt{4 + 25} = \sqrt{29}$
\end{solutionbox}

\questionmarks{5.2}{3}{સંકર સંખ્યા $3-4i$ નું વર્ગમૂળ શોધો.}

\begin{solutionbox}
\textbf{ઉકેલ}:
ધારો કે $\sqrt{3-4i} = a + bi$ જ્યાં $a, b \in \mathbb{R}$

તેથી $(a + bi)^2 = 3 - 4i$
\[
a^2 + 2abi + (bi)^2 = 3 - 4i
\]
\[
a^2 - b^2 + 2abi = 3 - 4i
\]
વાસ્તવિક અને કાલ્પનિક ભાગોને સરખાવતા:
$a^2 - b^2 = 3$ ... (1)
$2ab = -4$ ... (2)

સમીકરણ (2) પરથી: $b = -\frac{2}{a}$

સમીકરણ (1) માં મુકતા:
\[
a^2 - \left(-\frac{2}{a}\right)^2 = 3
\]
\[
a^2 - \frac{4}{a^2} = 3
\]
\[
a^4 - 3a^2 - 4 = 0
\]
ધારો કે $u = a^2$: $u^2 - 3u - 4 = 0$
\[
(u-4)(u+1) = 0
\]
તેથી $u = 4$ અથવા $u = -1$

કારણ કે $u = a^2 \geq 0$, આપણી પાસે $u = 4$, તેથી $a^2 = 4$

તેથી $a = \pm 2$

જો $a = 2$: $b = -\frac{2}{2} = -1$
જો $a = -2$: $b = -\frac{2}{-2} = 1$

બે વર્ગમૂળ છે: $2 - i$ અને $-2 + i$
\end{solutionbox}

\questionmarks{5.3}{3}{$y = (\sin x)^{\tan x}$ હોય તો $\frac{dy}{dx}$ શોધો}

\begin{solutionbox}
\textbf{ઉકેલ}:
બંને બાજુ લઘુગુણક (logarithm) લેતા:
\[
\log y = \tan x \log(\sin x)
\]
બંને બાજુ $x$ સાપેક્ષે વિકલન કરતા:
\[
\frac{1}{y} \frac{dy}{dx} = \frac{d}{dx}[\tan x \log(\sin x)]
\]
જમણી બાજુ ગુણાકારના નિયમનો ઉપયોગ કરતા:
\[
\frac{1}{y} \frac{dy}{dx} = \sec^2 x \log(\sin x) + \tan x \cdot \frac{\cos x}{\sin x}
\]
\[
\frac{1}{y} \frac{dy}{dx} = \sec^2 x \log(\sin x) + \tan x \cdot \cot x
\]
\[
\frac{1}{y} \frac{dy}{dx} = \sec^2 x \log(\sin x) + 1
\]
તેથી:
\[
\frac{dy}{dx} = y[\sec^2 x \log(\sin x) + 1]
\]
\[
\frac{dy}{dx} = (\sin x)^{\tan x}[\sec^2 x \log(\sin x) + 1]
\]
\end{solutionbox}

\questionmarks{5(b)}{8}{કોઈપણ બે લખો}

\questionmarks{5.1}{4}{વિકલ સમીકરણ $\tan y \, dx + \tan x \sec^2 y \, dy = 0$ નો ઉકેલ શોધો.}

\begin{solutionbox}
\textbf{ઉકેલ}:
આપેલ સમીકરણ: $\tan y \, dx + \tan x \sec^2 y \, dy = 0$

ગોઠવતા: $\tan y \, dx = -\tan x \sec^2 y \, dy$
\[
\frac{\tan y}{\sec^2 y} dy = -\tan x \, dx
\]
\[
\frac{\sin y / \cos y}{1/\cos^2 y} dy = -\tan x \, dx
\]
\[
\frac{\sin y}{\cos y} \cdot \cos^2 y \, dy = -\tan x \, dx
\]
\[
\sin y \cos y \, dy = -\tan x \, dx
\]
બંને બાજુ સંકલન કરતા:
\[
\int \sin y \cos y \, dy = -\int \tan x \, dx
\]
ડાબી બાજુ માટે, ધારો કે $u = \sin y$, તેથી $du = \cos y \, dy$:
$\int \sin y \cos y \, dy = \int u \, du = \frac{u^2}{2} = \frac{\sin^2 y}{2}$

જમણી બાજુ માટે:
$-\int \tan x \, dx = -\int \frac{\sin x}{\cos x} dx = \log|\cos x| + C_1$

તેથી:
\[
\frac{\sin^2 y}{2} = \log|\cos x| + C
\]
\[
\sin^2 y = 2\log|\cos x| + K \quad (\text{જ્યાં } K = 2C)
\]
\end{solutionbox}

\questionmarks{5.2}{4}{જો $A = \begin{bmatrix} 3 & -1 & 2 \\ 4 & 1 & -1 \\ 5 & 0 & 1 \end{bmatrix}$ હોય તો $A^{-1}$ શોધો.}

\begin{solutionbox}
\textbf{ઉકેલ}:
$A^{-1}$ શોધવા માટે, આપણે $A^{-1} = \frac{1}{|A|} \text{adj}(A)$ સૂત્રનો ઉપયોગ કરીએ.

પ્રથમ, $|A|$ શોધીએ:
\[
|A| = 3\begin{vmatrix} 1 & -1 \\ 0 & 1 \end{vmatrix} - (-1)\begin{vmatrix} 4 & -1 \\ 5 & 1 \end{vmatrix} + 2\begin{vmatrix} 4 & 1 \\ 5 & 0 \end{vmatrix}
\]
\[
= 3(1 \cdot 1 - (-1) \cdot 0) + 1(4 \cdot 1 - (-1) \cdot 5) + 2(4 \cdot 0 - 1 \cdot 5)
\]
\[
= 3(1) + 1(4 + 5) + 2(0 - 5) = 3 + 9 - 10 = 2
\]
હવે સહઅવયવ શ્રેણિક (cofactor matrix) શોધીએ:
$C_{11} = 1$, $C_{12} = -9$, $C_{13} = -5$
$C_{21} = 1$, $C_{22} = -7$, $C_{23} = -5$
$C_{31} = -1$, $C_{32} = 11$, $C_{33} = 7$

સહઅવયવ શ્રેણિક છે: $C = \begin{bmatrix} 1 & -9 & -5 \\ 1 & -7 & -5 \\ -1 & 11 & 7 \end{bmatrix}$

સહઅવયવ શ્રેણિકનો પરિવર્ત (transpose) એ સહલગ્ન શ્રેણિક (adjugate matrix) છે:
\[
\text{adj}(A) = \begin{bmatrix} 1 & 1 & -1 \\ -9 & -7 & 11 \\ -5 & -5 & 7 \end{bmatrix}
\]
તેથી:
\[
A^{-1} = \frac{1}{2}\begin{bmatrix} 1 & 1 & -1 \\ -9 & -7 & 11 \\ -5 & -5 & 7 \end{bmatrix} = \begin{bmatrix} 1/2 & 1/2 & -1/2 \\ -9/2 & -7/2 & 11/2 \\ -5/2 & -5/2 & 7/2 \end{bmatrix}
\]
\end{solutionbox}

\questionmarks{5.3}{4}{જો $x = a(\theta - \sin\theta)$, $y = a(1 - \cos\theta)$ તો $\frac{dy}{dx}$ શોધો.}

\begin{solutionbox}
\textbf{ઉકેલ}:
આ પ્રચલ સમીકરણો છે. $\frac{dy}{dx}$ શોધવા માટે:
\[
\frac{dy}{dx} = \frac{dy/d\theta}{dx/d\theta}
\]
પ્રથમ, $\frac{dx}{d\theta}$ શોધીએ:
$x = a(\theta - \sin\theta)$
$\frac{dx}{d\theta} = a(1 - \cos\theta)$

પછી, $\frac{dy}{d\theta}$ શોધીએ:
$y = a(1 - \cos\theta)$
$\frac{dy}{d\theta} = a\sin\theta$

તેથી:
\[
\frac{dy}{dx} = \frac{a\sin\theta}{a(1 - \cos\theta)} = \frac{\sin\theta}{1 - \cos\theta}
\]
$1 - \cos\theta = 2\sin^2(\theta/2)$ અને $\sin\theta = 2\sin(\theta/2)\cos(\theta/2)$ સૂત્રોનો ઉપયોગ કરતા:
\[
\frac{dy}{dx} = \frac{2\sin(\theta/2)\cos(\theta/2)}{2\sin^2(\theta/2)} = \frac{\cos(\theta/2)}{\sin(\theta/2)} = \cot(\theta/2)
\]
\end{solutionbox}

\end{document}
