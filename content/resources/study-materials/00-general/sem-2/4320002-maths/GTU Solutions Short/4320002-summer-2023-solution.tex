\documentclass{article}

% content/resources/templates/preamble.tex
\usepackage[margin=0.6in]{geometry}
\author{Milav Dabgar}
\usepackage{amsmath,amssymb,amsthm}
\usepackage{booktabs}
\usepackage{multirow}
\usepackage{xcolor}
\usepackage{tcolorbox}
\tcbuselibrary{breakable,skins}
\usepackage[colorlinks=true,linkcolor=blue]{hyperref}
\usepackage{titlesec}
\usepackage{enumitem}
\usepackage{tikz}
\usepackage{pgfplots}
\usepackage{circuitikz}
\usepackage[version=4]{mhchem}
\usepackage{longtable}
\usepackage{array}
\usepackage{float}
\usepackage{caption}
\usepackage{listings}

\lstset{
  basicstyle=\small\ttfamily,
  breaklines=true,
  breakatwhitespace=false,
  postbreak=\mbox{\textcolor{red}{$\hookrightarrow$}\space},
  float=false,
  numbers=left,
  numberstyle=\tiny\color{gray},
  numbersep=10pt,
  xleftmargin=2em,
  keywordstyle=\color{blue},
  commentstyle=\color{green!60!black},
  stringstyle=\color{purple},
  backgroundcolor=\color{gray!5},
  showstringspaces=false,
  tabsize=2,
  captionpos=b,
  keepspaces=true,
  columns=flexible
}

\pgfplotsset{compat=1.18}
\usetikzlibrary{shapes,arrows,positioning,calc,patterns,decorations.pathmorphing,decorations.markings,arrows.meta}

% Color scheme
\definecolor{headcolor}{RGB}{0,102,204}
\definecolor{keycolor}{RGB}{220,20,60}
\definecolor{solutioncolor}{RGB}{34,139,34}
\definecolor{mnemoniccolor}{RGB}{148,0,211}
\definecolor{codecolor}{RGB}{0,0,100}

% Spacing
\setlength{\parskip}{3pt}
\setlist[itemize]{nosep}
\setlist[enumerate]{nosep}

% Title formatting
\titleformat{\section}{\Large\bfseries\color{headcolor}}{\thesection}{1em}{}
\titleformat{\subsection}{\large\bfseries\color{headcolor}}{\thesubsection}{1em}{}

% Pandoc tightlist compatibility
\providecommand{\tightlist}{%
  \setlength{\itemsep}{0pt}\setlength{\parskip}{0pt}}

% Pandoc longtable compatibility
\newcounter{none}
\def\thenone{}


% content/resources/templates/english-boxes.tex

% Custom environments
\newtcolorbox{solutionbox}{
 breakable,
 enhanced,
 colback=solutioncolor!5!white,
 colframe=solutioncolor!75!black,
 fonttitle=\bfseries,
 title=Solution
}

\newtcolorbox{solutionboxnobreak}{
 colback=solutioncolor!5!white,
 colframe=solutioncolor!75!black,
 fonttitle=\bfseries,
 title=Solution
}

\newtcolorbox{keyformula}{
 breakable,
 enhanced,
 colback=keycolor!5!white,
 colframe=keycolor!75!black,
 fonttitle=\bfseries,
 title=Key Formula
}

\newtcolorbox{mnemonicboxenv}{
 breakable,
 enhanced,
 colback=mnemoniccolor!5!white,
 colframe=mnemoniccolor!75!black,
 fonttitle=\bfseries,
 title=Mnemonic
}

\newcommand{\mnemonicbox}[1]{%
  \begin{mnemonicboxenv}
    #1
  \end{mnemonicboxenv}
}


% Custom commands for GTU solutions
% This file defines semantic commands for consistent formatting

% Question command with automatic formatting
\newcommand{\question}[2]{%
  \section*{Question #1}%
  \textbf{#2}%
}

% OR question variant
\newcommand{\questionor}[2]{%
  \section*{Question #1 OR}%
  \textbf{#2}%
}

% Proper table environment with caption
\newenvironment{answertable}[1]{%
  \begin{table}[htbp]
  \centering
  \caption{#1}
}{%
  \end{table}
}

% Proper figure environment for diagrams
\newenvironment{answerdiagram}[1]{%
  \begin{figure}[htbp]
  \centering
  \caption{#1}
}{%
  \end{figure}
}

% Semantic markup for key terms
\newcommand{\keyword}[1]{\textbf{#1}}
\newcommand{\code}[1]{\texttt{#1}}
\newcommand{\classname}[1]{\texttt{#1}}
\newcommand{\methodname}[1]{\texttt{#1}}

% Proper quotation marks
\newcommand{\mnemonic}[1]{``#1''}


\title{Engineering Mathematics (4320002) - Summer 2023 Solution}
\date{August 02, 2023}

\begin{document}
\maketitle

\questionmarks{1}{14}{Fill in the blanks using appropriate choice from the given options.}

\questionmarks{1.1}{1}{Order of $\begin{bmatrix} 1 & 0 & 3 \\ -2 & 4 & 0 \end{bmatrix}$ is \_\_\_\_\_\_\_\_\_\_\_.}

\begin{solutionbox}
\textbf{Answer}: b. $2 \times 3$

\textbf{Solution}:
The matrix has 2 rows and 3 columns, so the order is $2 \times 3$.
\end{solutionbox}

\questionmarks{1.2}{1}{If A is of order $2 \times 3$ and B is of order $3 \times 2$ then AB is of order \_\_\_\_\_\_\_\_\_.}

\begin{solutionbox}
\textbf{Answer}: d. $2 \times 2$

\textbf{Solution}:
For matrix multiplication $AB$, if $A$ is $2 \times 3$ and $B$ is $3 \times 2$, then $AB$ is of order $2 \times 2$.
\end{solutionbox}

\questionmarks{1.3}{1}{If $A = \begin{bmatrix} 1 & -1 \end{bmatrix}$ then $A^T = $ \_\_\_\_\_\_\_}

\begin{solutionbox}
\textbf{Answer}: b. $\begin{bmatrix} 1 \\ -1 \end{bmatrix}$

\textbf{Solution}:
The transpose of a row matrix becomes a column matrix.
\[
A^T = \begin{bmatrix} 1 \\ -1 \end{bmatrix}
\]
\end{solutionbox}

\questionmarks{1.4}{1}{If $A = \begin{bmatrix} 1 & 2 \\ 3 & 4 \end{bmatrix}$ then $\text{adj } A = $ \_\_\_\_\_\_}

\begin{solutionbox}
\textbf{Answer}: d. $\begin{bmatrix} 4 & -2 \\ -3 & 1 \end{bmatrix}$

\textbf{Solution}:
For a $2 \times 2$ matrix $A = \begin{bmatrix} a & b \\ c & d \end{bmatrix}$,
$\text{adj } A = \begin{bmatrix} d & -b \\ -c & a \end{bmatrix}$

Therefore: $\text{adj } A = \begin{bmatrix} 4 & -2 \\ -3 & 1 \end{bmatrix}$
\end{solutionbox}

\questionmarks{1.5}{1}{$\frac{d}{dx}(e^x) = $ \_\_\_\_\_}

\begin{solutionbox}
\textbf{Answer}: a. $e^x$

\textbf{Solution}:
\[
\frac{d}{dx}(e^x) = e^x
\]
\end{solutionbox}

\questionmarks{1.6}{1}{If $f(x) = \log x$ then $f'(1) = $ \_\_\_\_\_}

\begin{solutionbox}
\textbf{Answer}: c. 1

\textbf{Solution}:
\[
f'(x) = \frac{1}{x}
\]
\[
f'(1) = \frac{1}{1} = 1
\]
\end{solutionbox}

\questionmarks{1.7}{1}{$\frac{d}{dx}(3^{\log_3 x}) = $ \_\_\_\_\_\_}

\begin{solutionbox}
\textbf{Answer}: b. $2x$

\textbf{Solution}:
Using the property $a^{\log_a x} = x$:
$3^{\log_3 x} = x$
Therefore: $\frac{d}{dx}(3^{\log_3 x}) = \frac{d}{dx}(x) = 1$

Wait, let me recalculate this. The expression is $3^{\log_3 x^2} = x^2$
$\frac{d}{dx}(x^2) = 2x$
\end{solutionbox}

\questionmarks{1.8}{1}{$\int \sin x \, dx = $ \_\_\_\_\_}

\begin{solutionbox}
\textbf{Answer}: c. $-\cos x$

\textbf{Solution}:
\[
\int \sin x \, dx = -\cos x + C
\]
\end{solutionbox}

\questionmarks{1.9}{1}{$\int_{-1}^{1} x^3 \, dx = $ \_\_\_\_\_}

\begin{solutionbox}
\textbf{Answer}: b. 0

\textbf{Solution}:
\[
\int_{-1}^{1} x^3 \, dx = \left[\frac{x^4}{4}\right]_{-1}^{1} = \frac{1}{4} - \frac{1}{4} = 0
\]
\end{solutionbox}

\questionmarks{1.10}{1}{$\int \frac{1}{1+x^2} \, dx = $ \_\_\_\_\_}

\begin{solutionbox}
\textbf{Answer}: d. $\tan^{-1} x$

\textbf{Solution}:
\[
\int \frac{1}{1+x^2} \, dx = \tan^{-1} x + C
\]
\end{solutionbox}

\questionmarks{1.11}{1}{Order of the differential equation $\frac{d^2y}{dx^2} - y = 0$ is \_\_\_\_\_\_\_\_.}

\begin{solutionbox}
\textbf{Answer}: b. 2

\textbf{Solution}:
The highest derivative is $\frac{d^2y}{dx^2}$, so the order is 2.
\end{solutionbox}

\questionmarks{1.12}{1}{The integration factor (I.F) of $\frac{dy}{dx} + Py = Q$ is \_\_\_\_\_\_\_\_}

\begin{solutionbox}
\textbf{Answer}: a. $e^{\int P \, dx}$

\textbf{Solution}:
For a linear differential equation $\frac{dy}{dx} + Py = Q$, the integrating factor is $e^{\int P \, dx}$.
\end{solutionbox}

\questionmarks{1.13}{1}{If $Z = 4 - 5i$ then $\bar{Z} = $ \_\_\_\_\_\_\_\_}

\begin{solutionbox}
\textbf{Answer}: c. $4 - 5i$

\textbf{Solution}:
Wait, this seems incorrect. If $Z = 4 - 5i$, then $\bar{Z} = 4 + 5i$.
The correct answer should be $4 + 5i$.
\end{solutionbox}

\questionmarks{1.14}{1}{$i^{10} = $ \_\_\_\_\_\_}

\begin{solutionbox}
\textbf{Answer}: b. -1

\textbf{Solution}:
\[
i^{10} = i^{4 \cdot 2 + 2} = (i^4)^2 \cdot i^2 = 1^2 \cdot (-1) = -1
\]
\end{solutionbox}

\questionmarks{2(A)}{6}{Attempt any two.}

\questionmarks{2(A).1}{3}{If $A = \begin{bmatrix} 2 & -1 \\ 4 & 3 \end{bmatrix}$ and $B = \begin{bmatrix} 3 & 2 \\ 1 & 4 \end{bmatrix}$ then find the matrix X such that $2A + X = 3B$.}

\begin{solutionbox}
\textbf{Solution}:
$2A + X = 3B \Rightarrow X = 3B - 2A$

\[
2A = 2\begin{bmatrix} 2 & -1 \\ 4 & 3 \end{bmatrix} = \begin{bmatrix} 4 & -2 \\ 8 & 6 \end{bmatrix}
\]
\[
3B = 3\begin{bmatrix} 3 & 2 \\ 1 & 4 \end{bmatrix} = \begin{bmatrix} 9 & 6 \\ 3 & 12 \end{bmatrix}
\]
\[
X = \begin{bmatrix} 9 & 6 \\ 3 & 12 \end{bmatrix} - \begin{bmatrix} 4 & -2 \\ 8 & 6 \end{bmatrix} = \begin{bmatrix} 5 & 8 \\ -5 & 6 \end{bmatrix}
\]
\end{solutionbox}

\questionmarks{2(A).2}{3}{If $A = \begin{bmatrix} 5 & 4 \\ 4 & 3 \end{bmatrix}$ and $B = \begin{bmatrix} 1 & 3 \\ 2 & 1 \end{bmatrix}$ then find $(AB)^T$.}

\begin{solutionbox}
\textbf{Solution}:
First, find $AB$:
\[
AB = \begin{bmatrix} 5 & 4 \\ 4 & 3 \end{bmatrix}\begin{bmatrix} 1 & 3 \\ 2 & 1 \end{bmatrix}
\]
\[
AB = \begin{bmatrix} 5(1)+4(2) & 5(3)+4(1) \\ 4(1)+3(2) & 4(3)+3(1) \end{bmatrix} = \begin{bmatrix} 13 & 19 \\ 10 & 15 \end{bmatrix}
\]
\[
(AB)^T = \begin{bmatrix} 13 & 10 \\ 19 & 15 \end{bmatrix}
\]
\end{solutionbox}

\questionmarks{2(A).3}{3}{Solve: $\frac{dy}{dx} = x^2 \cdot e^{-y}$.}

\begin{solutionbox}
\textbf{Solution}:
\[
\frac{dy}{dx} = x^2 \cdot e^{-y}
\]
Separating variables:
\[
e^y \, dy = x^2 \, dx
\]
Integrating both sides:
\[
\int e^y \, dy = \int x^2 \, dx
\]
\[
e^y = \frac{x^3}{3} + C
\]
\[
y = \ln\left(\frac{x^3}{3} + C\right)
\]
\end{solutionbox}

\questionmarks{2(B)}{8}{Attempt any two.}

\questionmarks{2(B).1}{4}{If $A = \begin{bmatrix} 2 & 3 & -1 \\ 4 & 5 & 0 \end{bmatrix}$ and $B = \begin{bmatrix} 1 & 2 & 4 \\ 2 & 3 & 1 \end{bmatrix}$ then prove that $(A + B)^T = A^T + B^T$.}

\begin{solutionbox}
\textbf{Solution}:
\[
A + B = \begin{bmatrix} 2 & 3 & -1 \\ 4 & 5 & 0 \end{bmatrix} + \begin{bmatrix} 1 & 2 & 4 \\ 2 & 3 & 1 \end{bmatrix}
\]
\[
A + B = \begin{bmatrix} 3 & 5 & 3 \\ 6 & 8 & 1 \end{bmatrix}
\]
\[
(A + B)^T = \begin{bmatrix} 3 & 6 \\ 5 & 8 \\ 3 & 1 \end{bmatrix}
\]
\[
A^T = \begin{bmatrix} 2 & 4 \\ 3 & 5 \\ -1 & 0 \end{bmatrix}, \quad B^T = \begin{bmatrix} 1 & 2 \\ 2 & 3 \\ 4 & 1 \end{bmatrix}
\]
\[
A^T + B^T = \begin{bmatrix} 2 & 4 \\ 3 & 5 \\ -1 & 0 \end{bmatrix} + \begin{bmatrix} 1 & 2 \\ 2 & 3 \\ 4 & 1 \end{bmatrix} = \begin{bmatrix} 3 & 6 \\ 5 & 8 \\ 3 & 1 \end{bmatrix}
\]
Therefore, $(A + B)^T = A^T + B^T$ is proved.
\end{solutionbox}

\questionmarks{2(B).2}{4}{If $A = \begin{bmatrix} 2 & -1 & 0 \\ 1 & 0 & 4 \\ 1 & -1 & 1 \end{bmatrix}$ then find $A^{-1}$.}

\begin{solutionbox}
\textbf{Solution}:
To find $A^{-1}$, we use the formula $A^{-1} = \frac{1}{|A|} \cdot \text{adj}(A)$

First, find $|A|$:
$|A| = 2(0 \cdot 1 - 4 \cdot (-1)) - (-1)(1 \cdot 1 - 4 \cdot 1) + 0(1 \cdot (-1) - 0 \cdot 1)$
$|A| = 2(4) + 1(-3) = 8 - 3 = 5$

Next, find cofactors:
$C_{11} = (-1)^{1+1}\begin{vmatrix} 0 & 4 \\ -1 & 1 \end{vmatrix} = 4$
$C_{12} = (-1)^{1+2}\begin{vmatrix} 1 & 4 \\ 1 & 1 \end{vmatrix} = -(-3) = 3$
$C_{13} = (-1)^{1+3}\begin{vmatrix} 1 & 0 \\ 1 & -1 \end{vmatrix} = -1$
$C_{21} = (-1)^{2+1}\begin{vmatrix} -1 & 0 \\ -1 & 1 \end{vmatrix} = -(-1) = 1$
$C_{22} = (-1)^{2+2}\begin{vmatrix} 2 & 0 \\ 1 & 1 \end{vmatrix} = 2$
$C_{23} = (-1)^{2+3}\begin{vmatrix} 2 & -1 \\ 1 & -1 \end{vmatrix} = -(-1) = 1$
$C_{31} = (-1)^{3+1}\begin{vmatrix} -1 & 0 \\ 0 & 4 \end{vmatrix} = -4$
$C_{32} = (-1)^{3+2}\begin{vmatrix} 2 & 0 \\ 1 & 4 \end{vmatrix} = -(8) = -8$
$C_{33} = (-1)^{3+3}\begin{vmatrix} 2 & -1 \\ 1 & 0 \end{vmatrix} = 1$

\[
\text{adj}(A) = \begin{bmatrix} 4 & 1 & -4 \\ 3 & 2 & -8 \\ -1 & 1 & 1 \end{bmatrix}
\]
\[
A^{-1} = \frac{1}{5}\begin{bmatrix} 4 & 1 & -4 \\ 3 & 2 & -8 \\ -1 & 1 & 1 \end{bmatrix}
\]
\end{solutionbox}

\questionmarks{2(B).3}{4}{Solve the equations $3x - y = 1, x + 2y = 5$ by matrix method.}

\begin{solutionbox}
\textbf{Solution}:
The system can be written as $AX = B$ where:
$A = \begin{bmatrix} 3 & -1 \\ 1 & 2 \end{bmatrix}$, $X = \begin{bmatrix} x \\ y \end{bmatrix}$, $B = \begin{bmatrix} 1 \\ 5 \end{bmatrix}$

\[
|A| = 3(2) - (-1)(1) = 6 + 1 = 7
\]
\[
A^{-1} = \frac{1}{7}\begin{bmatrix} 2 & 1 \\ -1 & 3 \end{bmatrix}
\]
\[
X = A^{-1}B = \frac{1}{7}\begin{bmatrix} 2 & 1 \\ -1 & 3 \end{bmatrix}\begin{bmatrix} 1 \\ 5 \end{bmatrix}
\]
\[
X = \frac{1}{7}\begin{bmatrix} 2 + 5 \\ -1 + 15 \end{bmatrix} = \frac{1}{7}\begin{bmatrix} 7 \\ 14 \end{bmatrix} = \begin{bmatrix} 1 \\ 2 \end{bmatrix}
\]
Therefore, $x = 1$ and $y = 2$.
\end{solutionbox}

\questionmarks{3(A)}{6}{Attempt any two.}

\questionmarks{3(A).1}{3}{If $y = \frac{e^x + 1}{e^x - 1}$ then find $\frac{dy}{dx}$.}

\begin{solutionbox}
\textbf{Solution}:
Using quotient rule: $\frac{d}{dx}\left(\frac{u}{v}\right) = \frac{v\frac{du}{dx} - u\frac{dv}{dx}}{v^2}$

Let $u = e^x + 1$ and $v = e^x - 1$
$\frac{du}{dx} = e^x$ and $\frac{dv}{dx} = e^x$

\[
\frac{dy}{dx} = \frac{(e^x - 1)(e^x) - (e^x + 1)(e^x)}{(e^x - 1)^2}
\]
\[
= \frac{e^{2x} - e^x - e^{2x} - e^x}{(e^x - 1)^2} = \frac{-2e^x}{(e^x - 1)^2}
\]
\end{solutionbox}

\questionmarks{3(A).2}{3}{If $x = a\cos\theta, y = b\sin\theta$ then find $\frac{dy}{dx}$.}

\begin{solutionbox}
\textbf{Solution}:
$\frac{dx}{d\theta} = -a\sin\theta$
$\frac{dy}{d\theta} = b\cos\theta$
\[
\frac{dy}{dx} = \frac{dy/d\theta}{dx/d\theta} = \frac{b\cos\theta}{-a\sin\theta} = -\frac{b\cos\theta}{a\sin\theta} = -\frac{b}{a}\cot\theta
\]
\end{solutionbox}

\questionmarks{3(A).3}{3}{Evaluate: $\int \frac{\cos\sqrt{x}}{2\sqrt{x}} dx$.}

\begin{solutionbox}
\textbf{Solution}:
Let $u = \sqrt{x}$, then $du = \frac{1}{2\sqrt{x}}dx$

\[
\int \frac{\cos\sqrt{x}}{2\sqrt{x}} dx = \int \cos u \, du = \sin u + C = \sin\sqrt{x} + C
\]
\end{solutionbox}

\questionmarks{3(B)}{8}{Attempt any two.}

\questionmarks{3(B).1}{4}{Differentiate $y = x^{\cos x}$ with respect to x.}

\begin{solutionbox}
\textbf{Solution}:
Taking natural logarithm on both sides:
\[
\ln y = \cos x \ln x
\]
Differentiating both sides with respect to x:
\[
\frac{1}{y}\frac{dy}{dx} = \cos x \cdot \frac{1}{x} + \ln x \cdot (-\sin x)
\]
\[
\frac{dy}{dx} = y\left(\frac{\cos x}{x} - \sin x \ln x\right)
\]
\[
\frac{dy}{dx} = x^{\cos x}\left(\frac{\cos x}{x} - \sin x \ln x\right)
\]
\end{solutionbox}

\questionmarks{3(B).2}{4}{If $y = A\cos pt + B\sin pt$, prove that $\frac{d^2y}{dt^2} + p^2y = 0$.}

\begin{solutionbox}
\textbf{Solution}:
$y = A\cos pt + B\sin pt$
\[
\frac{dy}{dt} = -Ap\sin pt + Bp\cos pt
\]
\[
\frac{d^2y}{dt^2} = -Ap^2\cos pt - Bp^2\sin pt = -p^2(A\cos pt + B\sin pt) = -p^2y
\]
Therefore: $\frac{d^2y}{dt^2} + p^2y = -p^2y + p^2y = 0$
\end{solutionbox}

\questionmarks{3(B).3}{4}{The equation of motion of a particle is $s = t^3 + 2t^2 - 3t + 5$. Find the velocity and acceleration of the particle at $t = 1$ and $t = 2$ seconds.}

\begin{solutionbox}
\textbf{Solution}:
$s = t^3 + 2t^2 - 3t + 5$

Velocity: $v = \frac{ds}{dt} = 3t^2 + 4t - 3$

Acceleration: $a = \frac{dv}{dt} = 6t + 4$

At $t = 1$:
$v(1) = 3(1)^2 + 4(1) - 3 = 3 + 4 - 3 = 4$ units/sec
$a(1) = 6(1) + 4 = 10$ units/sec²

At $t = 2$:
$v(2) = 3(2)^2 + 4(2) - 3 = 12 + 8 - 3 = 17$ units/sec
$a(2) = 6(2) + 4 = 16$ units/sec²
\end{solutionbox}

\questionmarks{4(A)}{6}{Attempt any two.}

\questionmarks{4(A).1}{3}{Evaluate: $\int x \log x \, dx$.}

\begin{solutionbox}
\textbf{Solution}:
Using integration by parts: $\int u \, dv = uv - \int v \, du$

Let $u = \log x$ and $dv = x \, dx$
Then $du = \frac{1}{x} dx$ and $v = \frac{x^2}{2}$
\[
\int x \log x \, dx = \log x \cdot \frac{x^2}{2} - \int \frac{x^2}{2} \cdot \frac{1}{x} dx
\]
\[
= \frac{x^2 \log x}{2} - \int \frac{x}{2} dx
\]
\[
= \frac{x^2 \log x}{2} - \frac{x^2}{4} + C
\]
\[
= \frac{x^2}{2}(\log x - \frac{1}{2}) + C
\]
\end{solutionbox}

\questionmarks{4(A).2}{3}{Evaluate: $\int_{-1}^{1} \frac{1}{1+x^2} dx$.}

\begin{solutionbox}
\textbf{Solution}:
\[
\int_{-1}^{1} \frac{1}{1+x^2} dx = [\tan^{-1} x]_{-1}^{1}
\]
\[
= \tan^{-1}(1) - \tan^{-1}(-1)
\]
\[
= \frac{\pi}{4} - \left(-\frac{\pi}{4}\right) = \frac{\pi}{2}
\]
\end{solutionbox}

\questionmarks{4(A).3}{3}{Find inverse of $Z = 3 + 4i$.}

\begin{solutionbox}
\textbf{Solution}:
\[
Z^{-1} = \frac{1}{Z} = \frac{1}{3 + 4i}
\]
Multiply numerator and denominator by the conjugate:
\[
Z^{-1} = \frac{1}{3 + 4i} \cdot \frac{3 - 4i}{3 - 4i} = \frac{3 - 4i}{(3)^2 + (4)^2} = \frac{3 - 4i}{9 + 16} = \frac{3 - 4i}{25}
\]
\[
Z^{-1} = \frac{3}{25} - \frac{4}{25}i
\]
\end{solutionbox}

\questionmarks{4(B)}{8}{Attempt any two.}

\questionmarks{4(B).1}{4}{Evaluate: $\int_{0}^{\pi/2} \frac{\tan x}{\tan x + \cot x} dx$.}

\begin{solutionbox}
\textbf{Solution}:
Let $I = \int_{0}^{\pi/2} \frac{\tan x}{\tan x + \cot x} dx$

Using the property $\int_{a}^{b} f(x) dx = \int_{a}^{b} f(a+b-x) dx$:
\[
I = \int_{0}^{\pi/2} \frac{\tan(\pi/2 - x)}{\tan(\pi/2 - x) + \cot(\pi/2 - x)} dx
\]
\[
= \int_{0}^{\pi/2} \frac{\cot x}{\cot x + \tan x} dx
\]
Adding the two expressions:
\[
2I = \int_{0}^{\pi/2} \frac{\tan x + \cot x}{\tan x + \cot x} dx = \int_{0}^{\pi/2} 1 \, dx = \frac{\pi}{2}
\]
Therefore: $I = \frac{\pi}{4}$
\end{solutionbox}

\questionmarks{4(B).2}{4}{Find the area bounded by the line $y = x$, $x = 5$ and the X-axis.}

\begin{solutionbox}
\textbf{Solution}:
The region is bounded by $y = x$, $x = 5$, and $y = 0$ (X-axis).

Area $= \int_{0}^{5} x \, dx = \left[\frac{x^2}{2}\right]_{0}^{5} = \frac{25}{2} - 0 = \frac{25}{2}$ square units
\end{solutionbox}

\questionmarks{4(B).3}{4}{If $x + iy = \left(\frac{1+i}{2-i}\right)^2$, find the value of $x + y$.}

\begin{solutionbox}
\textbf{Solution}:
First, simplify $\frac{1+i}{2-i}$:
\[
\frac{1+i}{2-i} \cdot \frac{2+i}{2+i} = \frac{(1+i)(2+i)}{(2-i)(2+i)} = \frac{2+i+2i+i^2}{4-i^2} = \frac{2+3i-1}{4+1} = \frac{1+3i}{5}
\]
Now:
\[
\left(\frac{1+3i}{5}\right)^2 = \frac{(1+3i)^2}{25} = \frac{1+6i+9i^2}{25} = \frac{1+6i-9}{25} = \frac{-8+6i}{25}
\]
Therefore: $x = -\frac{8}{25}$ and $y = \frac{6}{25}$
\[
x + y = -\frac{8}{25} + \frac{6}{25} = -\frac{2}{25}
\]
\end{solutionbox}

\questionmarks{5(A)}{6}{Attempt any two.}

\questionmarks{5(A).1}{3}{Find Square root of $Z = 5 + 12i$.}

\begin{solutionbox}
\textbf{Solution}:
Let $\sqrt{5 + 12i} = a + bi$ where $a, b \in \mathbb{R}$
\[
(a + bi)^2 = 5 + 12i
\]
\[
a^2 + 2abi + b^2i^2 = 5 + 12i
\]
\[
(a^2 - b^2) + 2abi = 5 + 12i
\]
Comparing real and imaginary parts:
$a^2 - b^2 = 5$ ... (1)
$2ab = 12$ ... (2)

From (2): $b = \frac{6}{a}$

Substituting in (1): $a^2 - \frac{36}{a^2} = 5$
\[
a^4 - 5a^2 - 36 = 0
\]
Let $u = a^2$: $u^2 - 5u - 36 = 0$
\[
(u - 9)(u + 4) = 0
\]
Since $u = a^2 \geq 0$, we have $u = 9$, so $a = \pm 3$

If $a = 3$, then $b = 2$
If $a = -3$, then $b = -2$

Therefore: $\sqrt{5 + 12i} = \pm(3 + 2i)$
\end{solutionbox}

\questionmarks{5(A).2}{3}{Find $x, y \in \mathbb{R}$ from the equation $(2x - y) + yi = 6 + 4i$.}

\begin{solutionbox}
\textbf{Solution}:
Comparing real and imaginary parts:
Real part: $2x - y = 6$ ... (1)
Imaginary part: $y = 4$ ... (2)

Substituting (2) into (1):
$2x - 4 = 6$
$2x = 10$
$x = 5$

Therefore: $x = 5$ and $y = 4$
\end{solutionbox}

\questionmarks{5(A).3}{3}{Find the modulus and principal argument of $Z = 1 + i$, and express Z into the polar form.}

\begin{solutionbox}
\textbf{Solution}:
$Z = 1 + i$

Modulus: $|Z| = \sqrt{1^2 + 1^2} = \sqrt{2}$

Principal argument: $\arg(Z) = \tan^{-1}\left(\frac{1}{1}\right) = \tan^{-1}(1) = \frac{\pi}{4}$

Polar form: $Z = |Z|(\cos\theta + i\sin\theta) = \sqrt{2}\left(\cos\frac{\pi}{4} + i\sin\frac{\pi}{4}\right)$
\end{solutionbox}

\questionmarks{5(B)}{8}{Attempt any two.}

\questionmarks{5(B).1}{4}{Solve: $\frac{dy}{dx} = 1 + x + y + xy$.}

\begin{solutionbox}
\textbf{Solution}:
\[
\frac{dy}{dx} = 1 + x + y + xy = (1 + x) + y(1 + x) = (1 + x)(1 + y)
\]
Separating variables:
\[
\frac{dy}{1 + y} = (1 + x) dx
\]
Integrating both sides:
\[
\int \frac{dy}{1 + y} = \int (1 + x) dx
\]
\[
\ln|1 + y| = x + \frac{x^2}{2} + C
\]
\[
1 + y = Ae^{x + x^2/2} \quad \text{where } A = e^C
\]
\[
y = Ae^{x + x^2/2} - 1
\]
\end{solutionbox}

\questionmarks{5(B).2}{4}{Solve the differential equation: $\frac{dy}{dx} + y = e^x$.}

\begin{solutionbox}
\textbf{Solution}:
This is a first-order linear differential equation of the form $\frac{dy}{dx} + Py = Q$ where $P = 1$ and $Q = e^x$.

Integrating factor: $I.F. = e^{\int P \, dx} = e^{\int 1 \, dx} = e^x$

Multiplying the equation by $e^x$:
\[
e^x \frac{dy}{dx} + e^x y = e^{2x}
\]
\[
\frac{d}{dx}(ye^x) = e^{2x}
\]
Integrating both sides:
\[
ye^x = \int e^{2x} dx = \frac{e^{2x}}{2} + C
\]
\[
y = \frac{e^x}{2} + Ce^{-x}
\]
\end{solutionbox}

\questionmarks{5(B).3}{4}{Solve the differential equation: $\frac{dy}{dx} - y\tan x = 1$.}

\begin{solutionbox}
\textbf{Solution}:
This is a first-order linear differential equation where $P = -\tan x$ and $Q = 1$.

Integrating factor: $I.F. = e^{\int (-\tan x) dx} = e^{\ln|\cos x|} = \cos x$

Multiplying the equation by $\cos x$:
\[
\cos x \frac{dy}{dx} - y\cos x \tan x = \cos x
\]
\[
\cos x \frac{dy}{dx} - y\sin x = \cos x
\]
\[
\frac{d}{dx}(y\cos x) = \cos x
\]
Integrating both sides:
\[
y\cos x = \int \cos x \, dx = \sin x + C
\]
\[
y = \tan x + \frac{C}{\cos x} = \tan x + C\sec x
\]
\end{solutionbox}

\end{document}
