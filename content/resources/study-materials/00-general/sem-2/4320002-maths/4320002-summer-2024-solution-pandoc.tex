\documentclass[10pt,a4paper]{article}

% content/resources/templates/preamble.tex
\usepackage[margin=0.6in]{geometry}
\author{Milav Dabgar}
\usepackage{amsmath,amssymb,amsthm}
\usepackage{booktabs}
\usepackage{multirow}
\usepackage{xcolor}
\usepackage{tcolorbox}
\tcbuselibrary{breakable,skins}
\usepackage[colorlinks=true,linkcolor=blue]{hyperref}
\usepackage{titlesec}
\usepackage{enumitem}
\usepackage{tikz}
\usepackage{pgfplots}
\usepackage{circuitikz}
\usepackage[version=4]{mhchem}
\usepackage{longtable}
\usepackage{array}
\usepackage{float}
\usepackage{caption}
\usepackage{listings}

\lstset{
  basicstyle=\small\ttfamily,
  breaklines=true,
  breakatwhitespace=false,
  postbreak=\mbox{\textcolor{red}{$\hookrightarrow$}\space},
  float=false,
  numbers=left,
  numberstyle=\tiny\color{gray},
  numbersep=10pt,
  xleftmargin=2em,
  keywordstyle=\color{blue},
  commentstyle=\color{green!60!black},
  stringstyle=\color{purple},
  backgroundcolor=\color{gray!5},
  showstringspaces=false,
  tabsize=2,
  captionpos=b,
  keepspaces=true,
  columns=flexible
}

\pgfplotsset{compat=1.18}
\usetikzlibrary{shapes,arrows,positioning,calc,patterns,decorations.pathmorphing,decorations.markings,arrows.meta}

% Color scheme
\definecolor{headcolor}{RGB}{0,102,204}
\definecolor{keycolor}{RGB}{220,20,60}
\definecolor{solutioncolor}{RGB}{34,139,34}
\definecolor{mnemoniccolor}{RGB}{148,0,211}
\definecolor{codecolor}{RGB}{0,0,100}

% Spacing
\setlength{\parskip}{3pt}
\setlist[itemize]{nosep}
\setlist[enumerate]{nosep}

% Title formatting
\titleformat{\section}{\Large\bfseries\color{headcolor}}{\thesection}{1em}{}
\titleformat{\subsection}{\large\bfseries\color{headcolor}}{\thesubsection}{1em}{}

% Pandoc tightlist compatibility
\providecommand{\tightlist}{%
  \setlength{\itemsep}{0pt}\setlength{\parskip}{0pt}}

% Pandoc longtable compatibility
\newcounter{none}
\def\thenone{}


% content/resources/templates/english-boxes.tex
% This file is currently empty - it exists to maintain consistency with the import structure.
% Add custom environments here if needed in the future.


\begin{document}

\begin{center}
{\Huge\bfseries\color{headcolor} Subject Name Solutions}\\[5pt]
{\LARGE 4320002 -- Summer 2024}\\[3pt]
{\large Semester 1 Study Material}\\[3pt]
{\normalsize\textit{Detailed Solutions and Explanations}}
\end{center}

\vspace{10pt}

\section*{Engineering Mathematics (4320002) - Summer 2024
Solutions}\label{engineering-mathematics-4320002---summer-2024-solutions}

\subsection*{Q.1 [14 marks]}\label{q.1-14-marks}

\textbf{Fill in the blanks using appropriate choice from the given
options.}

\subsubsection{Q.1.1 [1 mark]}\label{q.1.1-1-mark}

\textbf{Order of the matrix
\(A = \begin{bmatrix} 1 & 2 \\ 0 & -1 \\ 3 & 4 \end{bmatrix}\) is
\_\_\_\_\_\_.}

\begin{solutionbox}
(b) 3 \times 2

\textbf{Solution}: Order of a matrix is given by (number of rows) \times
(number of columns) Matrix A has 3 rows and 2 columns Therefore, order =
3 \times 2

\end{solutionbox}
\subsubsection{Q.1.2 [1 mark]}\label{q.1.2-1-mark}

\textbf{If
\(A = \begin{bmatrix} \sin \theta & -\cos \theta \\ \cos \theta & \sin \theta \end{bmatrix}\)
then \$A\^{}\{-1\} = \$ \_\_\_\_\_\_}

\begin{solutionbox}
(d) \(A^T\)

\textbf{Solution}: For orthogonal matrices, \(A^{-1} = A^T\) Since
\(AA^T = I\), we have \(A^{-1} = A^T\)

\end{solutionbox}
\subsubsection{Q.1.3 [1 mark]}\label{q.1.3-1-mark}

**\$

\begin{bmatrix} 1 & 2 \\ 5 & 0 \end{bmatrix}

\times 

\begin{bmatrix} -1 & 6 \\ 2 & 1 \end{bmatrix}

= \$ \_\_\_\_\_\_**

\begin{solutionbox}
(a) \(\begin{bmatrix} 3 & 8 \\ -5 & 30 \end{bmatrix}\)

\textbf{Solution}:
\(\begin{bmatrix} 1 & 2 \\ 5 & 0 \end{bmatrix} \times \begin{bmatrix} -1 & 6 \\ 2 & 1 \end{bmatrix}\)

\(= \begin{bmatrix} 1(-1) + 2(2) & 1(6) + 2(1) \\ 5(-1) + 0(2) & 5(6) + 0(1) \end{bmatrix}\)

\(= \begin{bmatrix} -1 + 4 & 6 + 2 \\ -5 + 0 & 30 + 0 \end{bmatrix} = \begin{bmatrix} 3 & 8 \\ -5 & 30 \end{bmatrix}\)

\end{solutionbox}
\subsubsection{Q.1.4 [1 mark]}\label{q.1.4-1-mark}

\textbf{If \(A = \begin{bmatrix} a & c \\ b & d \end{bmatrix}\) then
\$A\^{}T = \$ \_\_\_\_\_\_}

\begin{solutionbox}
(b) \(\begin{bmatrix} a & b \\ c & d \end{bmatrix}\)

\textbf{Solution}: Transpose of a matrix is obtained by interchanging
rows and columns \(A^T = \begin{bmatrix} a & b \\ c & d \end{bmatrix}\)

\end{solutionbox}
\subsubsection{Q.1.5 [1 mark]}\label{q.1.5-1-mark}

\textbf{\$\frac{d}{dx}(4\^{}x) = \$ \_\_\_\_\_\_}

\begin{solutionbox}
(a) \(4^x \log_e 4\)

\textbf{Solution}: \(\frac{d}{dx}(a^x) = a^x \ln a\) Therefore,
\(\frac{d}{dx}(4^x) = 4^x \ln 4 = 4^x \log_e 4\)

\end{solutionbox}
\subsubsection{Q.1.6 [1 mark]}\label{q.1.6-1-mark}

\textbf{\$\frac{d}{dx}(\sin\^{}2 x + \cos\^{}2 x) = \$ \_\_\_\_\_\_}

\begin{solutionbox}
(b) 0

\textbf{Solution}: \(\sin^2 x + \cos^2 x = 1\) (trigonometric identity)
\(\frac{d}{dx}(1) = 0\)

\end{solutionbox}
\subsubsection{Q.1.7 [1 mark]}\label{q.1.7-1-mark}

\textbf{If \(x = \sin \theta, y = \cos \theta\) then \$\frac{dy}{dx} =
\$ \_\_\_\_\_\_}

\begin{solutionbox}
(d) \(-\cot \theta\)

\textbf{Solution}: \(\frac{dx}{d\theta} = \cos \theta\),
\(\frac{dy}{d\theta} = -\sin \theta\)
\(\frac{dy}{dx} = \frac{dy/d\theta}{dx/d\theta} = \frac{-\sin \theta}{\cos \theta} = -\tan \theta = -\cot \theta\)

\end{solutionbox}
\subsubsection{Q.1.8 [1 mark]}\label{q.1.8-1-mark}

\textbf{\$\int x\^{}7 dx = \$ \_\_\_\_\_\_}

\begin{solutionbox}
(c) \(\frac{x^8}{8}\)

\textbf{Solution}: \(\int x^n dx = \frac{x^{n+1}}{n+1} + c\)
\(\int x^7 dx = \frac{x^8}{8} + c\)

\end{solutionbox}
\subsubsection{Q.1.9 [1 mark]}\label{q.1.9-1-mark}

\textbf{\$\int\emph{\{-2\}\^{}\{2\} x\^{}5 dx = \$ }\_\_\_\_\_}

\begin{solutionbox}
(b) 0

\textbf{Solution}: \(x^5\) is an odd function For odd functions,
\(\int_{-a}^{a} f(x) dx = 0\) Therefore, \(\int_{-2}^{2} x^5 dx = 0\)

\end{solutionbox}
\subsubsection{Q.1.10 [1 mark]}\label{q.1.10-1-mark}

\textbf{\$\int \frac{\cos x}{\sin x} dx = \$ \_\_\_\_\_\_}

\begin{solutionbox}
(d) \(\log|\sin x|\)

\textbf{Solution}: Let \(u = \sin x\), then \(du = \cos x dx\)
\(\int \frac{\cos x}{\sin x} dx = \int \frac{du}{u} = \log|u| + c = \log|\sin x| + c\)

\end{solutionbox}
\subsubsection{Q.1.11 [1 mark]}\label{q.1.11-1-mark}

\textbf{The order of the differential equation
\(\left(\frac{d^3y}{dx^3}\right)^2 + \left(\frac{d^2y}{dx^2}\right)^4 + y = 0\)
is \_\_\_\_\_\_}

\begin{solutionbox}
(a) 3

\textbf{Solution}: Order of a differential equation is the highest order
derivative present Highest derivative is \(\frac{d^3y}{dx^3}\), so order
= 3

\end{solutionbox}
\subsubsection{Q.1.12 [1 mark]}\label{q.1.12-1-mark}

\textbf{An integrating factor of the differential equation
\(\frac{dy}{dx} + y = 3x\) is \_\_\_\_\_\_}

\begin{solutionbox}
(c) \(e^x\)

\textbf{Solution}: For linear differential equation
\(\frac{dy}{dx} + Py = Q\) Integrating factor =
\(e^{\int P dx} = e^{\int 1 dx} = e^x\)

\end{solutionbox}
\subsubsection{Q.1.13 [1 mark]}\label{q.1.13-1-mark}

\textbf{\$i\^{}7 = \$ \_\_\_\_\_\_}

\begin{solutionbox}
(b) \(-i\)

\textbf{Solution}: \(i^1 = i, i^2 = -1, i^3 = -i, i^4 = 1\)
\(i^7 = i^4 \cdot i^3 = 1 \cdot (-i) = -i\)

\end{solutionbox}
\subsubsection{Q.1.14 [1 mark]}\label{q.1.14-1-mark}

\textbf{\$\arg(1+i) = \$ \_\_\_\_\_\_}

\begin{solutionbox}
(c) \(\frac{\pi}{4}\)

\textbf{Solution}: \(\arg(a + bi) = \tan^{-1}\left(\frac{b}{a}\right)\)
\(\arg(1 + i) = \tan^{-1}\left(\frac{1}{1}\right) = \tan^{-1}(1) = \frac{\pi}{4}\)

\end{solutionbox}
\subsection*{Q.2 (A) [6 marks]}\label{q.2-a-6-marks}

\textbf{Attempt any two}

\subsubsection{Q.2 (A).1 [3 marks]}\label{q.2-a.1-3-marks}

\textbf{If \(A = \begin{bmatrix} 2 & 1 \\ 3 & 0 \end{bmatrix}\) and
\(B = \begin{bmatrix} 4 & -1 \\ 2 & 3 \end{bmatrix}\) then prove that
\((A + B)^T = A^T + B^T\)}

\textbf{Solution}:
\(A + B = \begin{bmatrix} 2 & 1 \\ 3 & 0 \end{bmatrix} + \begin{bmatrix} 4 & -1 \\ 2 & 3 \end{bmatrix} = \begin{bmatrix} 6 & 0 \\ 5 & 3 \end{bmatrix}\)

\((A + B)^T = \begin{bmatrix} 6 & 5 \\ 0 & 3 \end{bmatrix}\)

\(A^T = \begin{bmatrix} 2 & 3 \\ 1 & 0 \end{bmatrix}\),
\(B^T = \begin{bmatrix} 4 & 2 \\ -1 & 3 \end{bmatrix}\)

\(A^T + B^T = \begin{bmatrix} 2 & 3 \\ 1 & 0 \end{bmatrix} + \begin{bmatrix} 4 & 2 \\ -1 & 3 \end{bmatrix} = \begin{bmatrix} 6 & 5 \\ 0 & 3 \end{bmatrix}\)

Therefore, \((A + B)^T = A^T + B^T\) ✓ \textbf{Proved}

\subsubsection{Q.2 (A).2 [3 marks]}\label{q.2-a.2-3-marks}

\textbf{If \(A = \begin{bmatrix} 1 & 1 \\ 2 & 3 \end{bmatrix}\) then
show that \(A \cdot A^{-1} = I\)}

\textbf{Solution}: First, find \(A^{-1}\):
\(|A| = 1(3) - 1(2) = 3 - 2 = 1\)

\(A^{-1} = \frac{1}{|A|} \text{adj}(A) = \frac{1}{1} \begin{bmatrix} 3 & -1 \\ -2 & 1 \end{bmatrix} = \begin{bmatrix} 3 & -1 \\ -2 & 1 \end{bmatrix}\)

Now verify \(A \cdot A^{-1} = I\):
\(A \cdot A^{-1} = \begin{bmatrix} 1 & 1 \\ 2 & 3 \end{bmatrix} \begin{bmatrix} 3 & -1 \\ -2 & 1 \end{bmatrix}\)

\(= \begin{bmatrix} 1(3) + 1(-2) & 1(-1) + 1(1) \\ 2(3) + 3(-2) & 2(-1) + 3(1) \end{bmatrix}\)

\(= \begin{bmatrix} 3 - 2 & -1 + 1 \\ 6 - 6 & -2 + 3 \end{bmatrix} = \begin{bmatrix} 1 & 0 \\ 0 & 1 \end{bmatrix} = I\)
✓ \textbf{Proved}

\subsubsection{Q.2 (A).3 [3 marks]}\label{q.2-a.3-3-marks}

\textbf{Solve the differential equation \(x dy + y dx = 0\)}

\textbf{Solution}: \(x dy + y dx = 0\) \(x dy = -y dx\)
\(\frac{dy}{y} = -\frac{dx}{x}\)

Integrating both sides: \(\int \frac{dy}{y} = -\int \frac{dx}{x}\)
\(\ln|y| = -\ln|x| + c_1\) \(\ln|y| + \ln|x| = c_1\) \(\ln|xy| = c_1\)
\(|xy| = e^{c_1} = c\) (where \(c = e^{c_1}\) is a constant)

Therefore, \(xy = \pm c\) or \textbf{\(xy = k\)} where \(k\) is an
arbitrary constant.

\subsection*{Q.2 (B) [8 marks]}\label{q.2-b-8-marks}

\textbf{Attempt any two}

\subsubsection{Q.2 (B).1 [4 marks]}\label{q.2-b.1-4-marks}

\textbf{If \(A = \begin{bmatrix} 3 & 1 \\ -1 & 2 \end{bmatrix}\) then
show that \(A^2 - 5A + 7I = 0\)}

\textbf{Solution}: First, calculate \(A^2\):
\(A^2 = \begin{bmatrix} 3 & 1 \\ -1 & 2 \end{bmatrix} \begin{bmatrix} 3 & 1 \\ -1 & 2 \end{bmatrix}\)

\(= \begin{bmatrix} 3(3) + 1(-1) & 3(1) + 1(2) \\ -1(3) + 2(-1) & -1(1) + 2(2) \end{bmatrix}\)

\(= \begin{bmatrix} 9 - 1 & 3 + 2 \\ -3 - 2 & -1 + 4 \end{bmatrix} = \begin{bmatrix} 8 & 5 \\ -5 & 3 \end{bmatrix}\)

Now calculate \(5A\):
\(5A = 5\begin{bmatrix} 3 & 1 \\ -1 & 2 \end{bmatrix} = \begin{bmatrix} 15 & 5 \\ -5 & 10 \end{bmatrix}\)

And \(7I\):
\(7I = 7\begin{bmatrix} 1 & 0 \\ 0 & 1 \end{bmatrix} = \begin{bmatrix} 7 & 0 \\ 0 & 7 \end{bmatrix}\)

Now verify \(A^2 - 5A + 7I = 0\):
\(A^2 - 5A + 7I = \begin{bmatrix} 8 & 5 \\ -5 & 3 \end{bmatrix} - \begin{bmatrix} 15 & 5 \\ -5 & 10 \end{bmatrix} + \begin{bmatrix} 7 & 0 \\ 0 & 7 \end{bmatrix}\)

\(= \begin{bmatrix} 8 - 15 + 7 & 5 - 5 + 0 \\ -5 + 5 + 0 & 3 - 10 + 7 \end{bmatrix} = \begin{bmatrix} 0 & 0 \\ 0 & 0 \end{bmatrix} = 0\)
✓ \textbf{Proved}

\subsubsection{Q.2 (B).2 [4 marks]}\label{q.2-b.2-4-marks}

\textbf{If
\(A = \begin{bmatrix} -4 & -3 & -3 \\ 1 & 0 & 1 \\ 4 & 4 & 3 \end{bmatrix}\)
then prove that \(\text{adj } A = A\)}

\textbf{Solution}: To find adj A, we need to find the cofactor matrix
and then transpose it.

Cofactors:
\(C_{11} = (-1)^{1+1} \begin{vmatrix} 0 & 1 \\ 4 & 3 \end{vmatrix} = 0(3) - 1(4) = -4\)

\(C_{12} = (-1)^{1+2} \begin{vmatrix} 1 & 1 \\ 4 & 3 \end{vmatrix} = -(1(3) - 1(4)) = -(3-4) = 1\)

\(C_{13} = (-1)^{1+3} \begin{vmatrix} 1 & 0 \\ 4 & 4 \end{vmatrix} = 1(4) - 0(4) = 4\)

\(C_{21} = (-1)^{2+1} \begin{vmatrix} -3 & -3 \\ 4 & 3 \end{vmatrix} = -((-3)(3) - (-3)(4)) = -(-9+12) = -3\)

\(C_{22} = (-1)^{2+2} \begin{vmatrix} -4 & -3 \\ 4 & 3 \end{vmatrix} = (-4)(3) - (-3)(4) = -12+12 = 0\)

\(C_{23} = (-1)^{2+3} \begin{vmatrix} -4 & -3 \\ 4 & 4 \end{vmatrix} = -((-4)(4) - (-3)(4)) = -(-16+12) = -(-4) = 4\)

\(C_{31} = (-1)^{3+1} \begin{vmatrix} -3 & -3 \\ 0 & 1 \end{vmatrix} = (-3)(1) - (-3)(0) = -3\)

\(C_{32} = (-1)^{3+2} \begin{vmatrix} -4 & -3 \\ 1 & 1 \end{vmatrix} = -((-4)(1) - (-3)(1)) = -(-4+3) = -(-1) = 1\)

\(C_{33} = (-1)^{3+3} \begin{vmatrix} -4 & -3 \\ 1 & 0 \end{vmatrix} = (-4)(0) - (-3)(1) = 0+3 = 3\)

Cofactor matrix =
\(\begin{bmatrix} -4 & 1 & 4 \\ -3 & 0 & 4 \\ -3 & 1 & 3 \end{bmatrix}\)

\(\text{adj } A = \text{(Cofactor matrix)}^T = \begin{bmatrix} -4 & -3 & -3 \\ 1 & 0 & 1 \\ 4 & 4 & 3 \end{bmatrix} = A\)
✓ \textbf{Proved}

\subsubsection{Q.2 (B).3 [4 marks]}\label{q.2-b.3-4-marks}

\textbf{Solve the following system of linear equations using matrix:
\(3x + 2y = 5\), \(2x - y = 1\)}

\textbf{Solution}: The system can be written as \(AX = B\) where:
\(A = \begin{bmatrix} 3 & 2 \\ 2 & -1 \end{bmatrix}\),
\(X = \begin{bmatrix} x \\ y \end{bmatrix}\),
\(B = \begin{bmatrix} 5 \\ 1 \end{bmatrix}\)

Find \(|A| = 3(-1) - 2(2) = -3 - 4 = -7\)

\(A^{-1} = \frac{1}{-7} \begin{bmatrix} -1 & -2 \\ -2 & 3 \end{bmatrix} = \begin{bmatrix} \frac{1}{7} & \frac{2}{7} \\ \frac{2}{7} & -\frac{3}{7} \end{bmatrix}\)

\(X = A^{-1}B = \begin{bmatrix} \frac{1}{7} & \frac{2}{7} \\ \frac{2}{7} & -\frac{3}{7} \end{bmatrix} \begin{bmatrix} 5 \\ 1 \end{bmatrix}\)

\(= \begin{bmatrix} \frac{1}{7}(5) + \frac{2}{7}(1) \\ \frac{2}{7}(5) - \frac{3}{7}(1) \end{bmatrix} = \begin{bmatrix} \frac{5+2}{7} \\ \frac{10-3}{7} \end{bmatrix} = \begin{bmatrix} 1 \\ 1 \end{bmatrix}\)

Therefore, \textbf{\(x = 1,

y = 1\)}


\subsection*{Q.3 (A) [6 marks]}\label{q.3-a-6-marks}

\textbf{Attempt any two}

\subsubsection{Q.3 (A).1 [3 marks]}\label{q.3-a.1-3-marks}

\textbf{Using definition of differentiation find the derivative of
\(x^5\) with respect to \(x\)}

\textbf{Solution}: By definition:
\(\frac{dy}{dx} = \lim_{h \to 0} \frac{f(x+h) - f(x)}{h}\)

For \(f(x) = x^5\):
\(\frac{d}{dx}(x^5) = \lim_{h \to 0} \frac{(x+h)^5 - x^5}{h}\)

Using binomial theorem:
\((x+h)^5 = x^5 + 5x^4h + 10x^3h^2 + 10x^2h^3 + 5xh^4 + h^5\)

\(\frac{d}{dx}(x^5) = \lim_{h \to 0} \frac{x^5 + 5x^4h + 10x^3h^2 + 10x^2h^3 + 5xh^4 + h^5 - x^5}{h}\)

\(= \lim_{h \to 0} \frac{5x^4h + 10x^3h^2 + 10x^2h^3 + 5xh^4 + h^5}{h}\)

\(= \lim_{h \to 0} (5x^4 + 10x^3h + 10x^2h^2 + 5xh^3 + h^4)\)

\(= 5x^4 + 0 + 0 + 0 + 0 = 5x^4\)

Therefore, \textbf{\(\frac{d}{dx}(x^5) = 5x^4\)}

\subsubsection{Q.3 (A).2 [3 marks]}\label{q.3-a.2-3-marks}

\textbf{Find \(\frac{dy}{dx}\) if \(y = \frac{x^2-1}{x^2+1}\)}

\textbf{Solution}: Using quotient rule:
\(\frac{d}{dx}\left(\frac{u}{v}\right) = \frac{v\frac{du}{dx} - u\frac{dv}{dx}}{v^2}\)

Here, \(u = x^2 - 1\), \(v = x^2 + 1\) \(\frac{du}{dx} = 2x\),
\(\frac{dv}{dx} = 2x\)

\(\frac{dy}{dx} = \frac{(x^2+1)(2x) - (x^2-1)(2x)}{(x^2+1)^2}\)

\(= \frac{2x(x^2+1) - 2x(x^2-1)}{(x^2+1)^2}\)

\(= \frac{2x[(x^2+1) - (x^2-1)]}{(x^2+1)^2}\)

\(= \frac{2x[x^2+1-x^2+1]}{(x^2+1)^2}\)

\(= \frac{2x \cdot 2}{(x^2+1)^2} = \frac{4x}{(x^2+1)^2}\)

Therefore, \textbf{\(\frac{dy}{dx} = \frac{4x}{(x^2+1)^2}\)}

\subsubsection{Q.3 (A).3 [3 marks]}\label{q.3-a.3-3-marks}

\textbf{Evaluate the integral \(\int \frac{x^2+5x+6}{x^2+2x} dx\)}

\textbf{Solution}: First, perform polynomial long division:
\(\frac{x^2+5x+6}{x^2+2x} = 1 + \frac{3x+6}{x^2+2x}\)

\(\int \frac{x^2+5x+6}{x^2+2x} dx = \int \left(1 + \frac{3x+6}{x^2+2x}\right) dx\)

\(= \int 1 dx + \int \frac{3x+6}{x^2+2x} dx\)

\(= x + \int \frac{3x+6}{x(x+2)} dx\)

For the second integral, use partial fractions:
\(\frac{3x+6}{x(x+2)} = \frac{A}{x} + \frac{B}{x+2}\)

\(3x + 6 = A(x+2) + Bx\)

When \(x = 0\): \(6 = 2A\), so \(A = 3\) When \(x = -2\):
\(-6 + 6 = -2B\), so \(B = 0\)

Wait, let me recalculate: When \(x = -2\):
\(3(-2) + 6 = -6 + 6 = 0 = B(-2)\) When \(x = 0\): \(6 = 2A\), so
\(A = 3\)

Actually: \(3x + 6 = 3(x + 2)\) So
\(\frac{3x+6}{x(x+2)} = \frac{3(x+2)}{x(x+2)} = \frac{3}{x}\)

\(\int \frac{3x+6}{x(x+2)} dx = \int \frac{3}{x} dx = 3\ln|x| + c_1\)

Therefore: \(\int \frac{x^2+5x+6}{x^2+2x} dx = x + 3\ln|x| + c\)

\subsection*{Q.3 (B) [8 marks]}\label{q.3-b-8-marks}

\textbf{Attempt any two}

\subsubsection{Q.3 (B).1 [4 marks]}\label{q.3-b.1-4-marks}

\textbf{If \(y = \log(\sec x + \tan x)\) then find \(\frac{dy}{dx}\)}

\textbf{Solution}: \(y = \log(\sec x + \tan x)\)

\(\frac{dy}{dx} = \frac{1}{\sec x + \tan x} \cdot \frac{d}{dx}(\sec x + \tan x)\)

\(\frac{d}{dx}(\sec x) = \sec x \tan x\)
\(\frac{d}{dx}(\tan x) = \sec^2 x\)

\(\frac{dy}{dx} = \frac{1}{\sec x + \tan x} \cdot (\sec x \tan x + \sec^2 x)\)

\(= \frac{\sec x(\tan x + \sec x)}{\sec x + \tan x}\)

\(= \frac{\sec x(\sec x + \tan x)}{\sec x + \tan x} = \sec x\)

Therefore, \textbf{\(\frac{dy}{dx} = \sec x\)}

\subsubsection{Q.3 (B).2 [4 marks]}\label{q.3-b.2-4-marks}

\textbf{If \(y = 2e^{3x} + 3e^{-2x}\) then prove that
\(\frac{d^2y}{dx^2} - \frac{dy}{dx} - 6y = 0\)}

\textbf{Solution}: \(y = 2e^{3x} + 3e^{-2x}\)

First derivative:
\(\frac{dy}{dx} = 2(3e^{3x}) + 3(-2e^{-2x}) = 6e^{3x} - 6e^{-2x}\)

Second derivative:
\(\frac{d^2y}{dx^2} = 6(3e^{3x}) - 6(-2e^{-2x}) = 18e^{3x} + 12e^{-2x}\)

Now verify the equation: \(\frac{d^2y}{dx^2} - \frac{dy}{dx} - 6y\)

\(= (18e^{3x} + 12e^{-2x}) - (6e^{3x} - 6e^{-2x}) - 6(2e^{3x} + 3e^{-2x})\)

\(= 18e^{3x} + 12e^{-2x} - 6e^{3x} + 6e^{-2x} - 12e^{3x} - 18e^{-2x}\)

\(= e^{3x}(18 - 6 - 12) + e^{-2x}(12 + 6 - 18)\)

\(= e^{3x}(0) + e^{-2x}(0) = 0\) ✓ \textbf{Proved}

\subsubsection{Q.3 (B).3 [4 marks]}\label{q.3-b.3-4-marks}

\textbf{Find the maximum and minimum value of function
\(f(x) = x^3 - 3x + 11\)}

\textbf{Solution}: \(f(x) = x^3 - 3x + 11\)

First derivative: \(f'(x) = 3x^2 - 3 = 3(x^2 - 1) = 3(x-1)(x+1)\)

For critical points, set \(f'(x) = 0\): \(3(x-1)(x+1) = 0\) \(x = 1\) or
\(x = -1\)

Second derivative: \(f''(x) = 6x\)

At \(x = 1\): \(f''(1) = 6 > 0\) \rightarrow Local minimum At \(x = -1\):
\(f''(-1) = -6 < 0\) \rightarrow Local maximum

Function values: At \(x = 1\):
\(f(1) = 1^3 - 3(1) + 11 = 1 - 3 + 11 = 9\) At \(x = -1\):
\(f(-1) = (-1)^3 - 3(-1) + 11 = -1 + 3 + 11 = 13\)

Therefore:

\begin{itemize}
\tightlist
\item
  \textbf{Local maximum value = 13 at \(x = -1\)}
\item
  \textbf{Local minimum value = 9 at \(x = 1\)}
\end{itemize}

\subsection*{Q.4 (A) [6 marks]}\label{q.4-a-6-marks}

\textbf{Attempt any two}

\subsubsection{Q.4 (A).1 [3 marks]}\label{q.4-a.1-3-marks}

\textbf{Evaluate the integral \(\int \frac{\cos(\log x)}{x} dx\)}

\textbf{Solution}: Let \(u = \log x\), then \(du = \frac{1}{x} dx\)

\(\int \frac{\cos(\log x)}{x} dx = \int \cos u \, du = \sin u + c\)

Substituting back: \(u = \log x\)

Therefore,
\textbf{\(\int \frac{\cos(\log x)}{x} dx = \sin(\log x) + c\)}

\subsubsection{Q.4 (A).2 [3 marks]}\label{q.4-a.2-3-marks}

\textbf{Evaluate the integral \(\int x \sin x \, dx\)}

\textbf{Solution}: Using integration by parts:
\(\int u \, dv = uv - \int v \, du\)

Let \(u = x\) and \(dv = \sin x \, dx\) Then \(du = dx\) and
\(v = -\cos x\)

\(\int x \sin x \, dx = x(-\cos x) - \int (-\cos x) dx\)

\(= -x \cos x + \int \cos x \, dx\)

\(= -x \cos x + \sin x + c\)

Therefore, \textbf{\(\int x \sin x \, dx = \sin x - x \cos x + c\)}

\subsubsection{Q.4 (A).3 [3 marks]}\label{q.4-a.3-3-marks}

\textbf{If \((2x - y) + 2y i = 6 + 4i\) then find \(x\) and \(y\)}

\textbf{Solution}: \((2x - y) + 2y i = 6 + 4i\)

Comparing real and imaginary parts: Real part: \(2x - y = 6\) \ldots{}
(1) Imaginary part: \(2y = 4\) \ldots{} (2)

From equation (2): \(y = 2\)

Substituting in equation (1): \(2x - 2 = 6\) \(2x = 8\) \(x = 4\)

Therefore, \textbf{\(x = 4\) and \(y = 2\)}

\subsection*{Q.4 (B) [8 marks]}\label{q.4-b-8-marks}

\textbf{Attempt any two}

\subsubsection{Q.4 (B).1 [4 marks]}\label{q.4-b.1-4-marks}

\textbf{Find the area of the region bounded by the curve \(y = x^2\),
lines \(x = 1\), \(x = 2\) and X-axis}

\textbf{Solution}: The required area is given by:
\(A = \int_1^2 x^2 \, dx\)

\(A = \left[\frac{x^3}{3}\right]_1^2\)

\(= \frac{2^3}{3} - \frac{1^3}{3}\)

\(= \frac{8}{3} - \frac{1}{3}\)

\(= \frac{7}{3}\) square units

Therefore, \textbf{Area = \(\frac{7}{3}\) square units}

\subsubsection{Q.4 (B).2 [4 marks]}\label{q.4-b.2-4-marks}

\textbf{Evaluate the definite integral
\(\int_0^{\pi/2} \frac{\sec x}{\sec x + \csc x} dx\)}

\textbf{Solution}: Let
\(I = \int_0^{\pi/2} \frac{\sec x}{\sec x + \csc x} dx\)

Using the property \(\int_0^a f(x) dx = \int_0^a f(a-x) dx\):

\(I = \int_0^{\pi/2} \frac{\sec(\pi/2 - x)}{\sec(\pi/2 - x) + \csc(\pi/2 - x)} dx\)

Since \(\sec(\pi/2 - x) = \csc x\) and \(\csc(\pi/2 - x) = \sec x\):

\(I = \int_0^{\pi/2} \frac{\csc x}{\csc x + \sec x} dx\)

Adding both expressions:
\(2I = \int_0^{\pi/2} \frac{\sec x}{\sec x + \csc x} dx + \int_0^{\pi/2} \frac{\csc x}{\sec x + \csc x} dx\)

\(2I = \int_0^{\pi/2} \frac{\sec x + \csc x}{\sec x + \csc x} dx = \int_0^{\pi/2} 1 \, dx = \frac{\pi}{2}\)

Therefore, \(I = \frac{\pi}{4}\)

\begin{solutionbox}
\textbf{Answer:
\(\int_0^{\pi/2} \frac{\sec x}{\sec x + \csc x} dx = \frac{\pi}{4}\)}

\end{solutionbox}
\subsubsection{Q.4 (B).3 [4 marks]}\label{q.4-b.3-4-marks}

\textbf{If \(\alpha + i\beta = \frac{1}{a + ib}\) then prove that
\((\alpha^2 + \beta^2)(a^2 + b^2) = 1\)}

\textbf{Solution}: Given: \(\alpha + i\beta = \frac{1}{a + ib}\)

Rationalizing the right side:
\(\alpha + i\beta = \frac{1}{a + ib} \cdot \frac{a - ib}{a - ib} = \frac{a - ib}{a^2 + b^2}\)

\(\alpha + i\beta = \frac{a}{a^2 + b^2} - i\frac{b}{a^2 + b^2}\)

Comparing real and imaginary parts: \(\alpha = \frac{a}{a^2 + b^2}\) and
\(\beta = -\frac{b}{a^2 + b^2}\)

Now calculating \(\alpha^2 + \beta^2\):
\(\alpha^2 + \beta^2 = \left(\frac{a}{a^2 + b^2}\right)^2 + \left(-\frac{b}{a^2 + b^2}\right)^2\)

\(= \frac{a^2}{(a^2 + b^2)^2} + \frac{b^2}{(a^2 + b^2)^2}\)

\(= \frac{a^2 + b^2}{(a^2 + b^2)^2} = \frac{1}{a^2 + b^2}\)

Therefore:
\((\alpha^2 + \beta^2)(a^2 + b^2) = \frac{1}{a^2 + b^2} \cdot (a^2 + b^2) = 1\)
✓ \textbf{Proved}

\subsection*{Q.5 (A) [6 marks]}\label{q.5-a-6-marks}

\textbf{Attempt any two}

\subsubsection{Q.5 (A).1 [3 marks]}\label{q.5-a.1-3-marks}

\textbf{Find conjugate and modulus of complex number
\(\frac{2+3i}{3+2i}\)}

\textbf{Solution}: First, simplify the complex number by rationalizing:
\(\frac{2+3i}{3+2i} = \frac{2+3i}{3+2i} \cdot \frac{3-2i}{3-2i}\)

\(= \frac{(2+3i)(3-2i)}{(3+2i)(3-2i)}\)

\(= \frac{6 - 4i + 9i - 6i^2}{9 - 4i^2}\)

\(= \frac{6 + 5i - 6(-1)}{9 - 4(-1)}\)

\(= \frac{6 + 5i + 6}{9 + 4} = \frac{12 + 5i}{13}\)

So \(\frac{2+3i}{3+2i} = \frac{12}{13} + \frac{5}{13}i\)

\textbf{Conjugate}:
\(\overline{\frac{2+3i}{3+2i}} = \frac{12}{13} - \frac{5}{13}i\)

\textbf{Modulus}:
\(\left|\frac{2+3i}{3+2i}\right| = \sqrt{\left(\frac{12}{13}\right)^2 + \left(\frac{5}{13}\right)^2}\)

\(= \sqrt{\frac{144}{169} + \frac{25}{169}} = \sqrt{\frac{169}{169}} = \sqrt{1} = 1\)

\subsubsection{Q.5 (A).2 [3 marks]}\label{q.5-a.2-3-marks}

\textbf{Simplify:
\(\frac{(\cos 3\theta + i \sin 3\theta)^{-4} (\cos \theta - i \sin \theta)^{-5}}{(\cos 2\theta - i \sin 2\theta)^7}\)}

\textbf{Solution}: Using De Moivre's theorem:
\((\cos \theta + i \sin \theta)^n = \cos n\theta + i \sin n\theta\)

Also, \(\cos \theta - i \sin \theta = \cos(-\theta) + i \sin(-\theta)\)

\((\cos 3\theta + i \sin 3\theta)^{-4} = \cos(-12\theta) + i \sin(-12\theta)\)

\((\cos \theta - i \sin \theta)^{-5} = (\cos(-\theta) + i \sin(-\theta))^{-5} = \cos(5\theta) + i \sin(5\theta)\)

\((\cos 2\theta - i \sin 2\theta)^7 = (\cos(-2\theta) + i \sin(-2\theta))^7 = \cos(-14\theta) + i \sin(-14\theta)\)

Therefore:
\(\frac{(\cos 3\theta + i \sin 3\theta)^{-4} (\cos \theta - i \sin \theta)^{-5}}{(\cos 2\theta - i \sin 2\theta)^7}\)

\(= \frac{[\cos(-12\theta) + i \sin(-12\theta)][\cos(5\theta) + i \sin(5\theta)]}{\cos(-14\theta) + i \sin(-14\theta)}\)

\(= \frac{\cos(-12\theta + 5\theta) + i \sin(-12\theta + 5\theta)}{\cos(-14\theta) + i \sin(-14\theta)}\)

\(= \frac{\cos(-7\theta) + i \sin(-7\theta)}{\cos(-14\theta) + i \sin(-14\theta)}\)

\(= \cos(-7\theta + 14\theta) + i \sin(-7\theta + 14\theta)\)

\(= \cos(7\theta) + i \sin(7\theta)\)

\subsubsection{Q.5 (A).3 [3 marks]}\label{q.5-a.3-3-marks}

\textbf{Express Complex number \(1 + \sqrt{3}i\) into polar form}

\textbf{Solution}: For complex number \(z = a + bi\), polar form is
\(z = r(\cos \theta + i \sin \theta)\)

Here, \(a = 1\), \(b = \sqrt{3}\)

\textbf{Modulus}:
\(r = |z| = \sqrt{a^2 + b^2} = \sqrt{1^2 + (\sqrt{3})^2} = \sqrt{1 + 3} = \sqrt{4} = 2\)

\textbf{Argument}:
\(\theta = \tan^{-1}\left(\frac{b}{a}\right) = \tan^{-1}\left(\frac{\sqrt{3}}{1}\right) = \tan^{-1}(\sqrt{3}) = \frac{\pi}{3}\)

Therefore, the polar form is:
\textbf{\(1 + \sqrt{3}i = 2\left(\cos \frac{\pi}{3} + i \sin \frac{\pi}{3}\right)\)}

\subsection*{Q.5 (B) [8 marks]}\label{q.5-b-8-marks}

\textbf{Attempt any two}

\subsubsection{Q.5 (B).1 [4 marks]}\label{q.5-b.1-4-marks}

\textbf{Solve: \(\tan y \, dx + \tan x \sec^2 y \, dy = 0\)}

\textbf{Solution}: \(\tan y \, dx + \tan x \sec^2 y \, dy = 0\)

Rearranging: \(\tan y \, dx = -\tan x \sec^2 y \, dy\)

\(\frac{dx}{\tan x} = -\frac{\sec^2 y \, dy}{\tan y}\)

\(\frac{\cos x}{\sin x} dx = -\frac{dy}{\sin y \cos y}\)

\(\cot x \, dx = -\frac{dy}{\sin y \cos y}\)

Since
\(\frac{1}{\sin y \cos y} = \frac{2}{2\sin y \cos y} = \frac{2}{\sin 2y}\):

\(\cot x \, dx = -\frac{2 dy}{\sin 2y}\)

Integrating both sides: \(\int \cot x \, dx = -2 \int \csc(2y) \, dy\)

\(\ln|\sin x| = -2 \cdot \left(-\frac{1}{2}\ln|\csc(2y) + \cot(2y)|\right) + c\)

\(\ln|\sin x| = \ln|\csc(2y) + \cot(2y)| + c\)

Therefore: \textbf{\(\sin x \cdot [\csc(2y) + \cot(2y)] = k\)} where
\(k\) is a constant.

\subsubsection{Q.5 (B).2 [4 marks]}\label{q.5-b.2-4-marks}

\textbf{Solve: \(x \frac{dy}{dx} - y = x^2\)}

\textbf{Solution}: \(x \frac{dy}{dx} - y = x^2\)

Dividing by \(x\): \(\frac{dy}{dx} - \frac{y}{x} = x\)

This is a linear differential equation of the form
\(\frac{dy}{dx} + Py = Q\)

Here, \(P = -\frac{1}{x}\) and \(Q = x\)

Integrating factor:
\(I.F. = e^{\int P dx} = e^{\int -\frac{1}{x} dx} = e^{-\ln|x|} = \frac{1}{x}\)

Multiplying the equation by I.F.:
\(\frac{1}{x} \frac{dy}{dx} - \frac{y}{x^2} = 1\)

This can be written as: \(\frac{d}{dx}\left(\frac{y}{x}\right) = 1\)

Integrating: \(\frac{y}{x} = x + c\)

Therefore: \textbf{\(y = x^2 + cx\)}

\subsubsection{Q.5 (B).3 [4 marks]}\label{q.5-b.3-4-marks}

\textbf{Solve: \(\frac{dy}{dx} + \frac{y}{x} = e^x\), \(y(0) = 3\)}

\textbf{Solution}: This is a linear differential equation:
\(\frac{dy}{dx} + \frac{y}{x} = e^x\)

Here, \(P = \frac{1}{x}\) and \(Q = e^x\)

Integrating factor:
\(I.F. = e^{\int \frac{1}{x} dx} = e^{\ln|x|} = |x| = x\) (assuming
\(x > 0\))

Multiplying the equation by I.F.: \(x \frac{dy}{dx} + y = xe^x\)

This can be written as: \(\frac{d}{dx}(xy) = xe^x\)

Integrating both sides: \(xy = \int xe^x dx\)

Using integration by parts for \(\int xe^x dx\): Let \(u = x\),
\(dv = e^x dx\) Then \(du = dx\), \(v = e^x\)

\(\int xe^x dx = xe^x - \int e^x dx = xe^x - e^x = e^x(x-1)\)

So: \(xy = e^x(x-1) + c\)

Therefore: \(y = \frac{e^x(x-1) + c}{x}\)

Using initial condition \(y(0) = 3\): This presents a problem as we have
division by zero. Let me reconsider the approach.

Actually, let's solve this more carefully. The equation
\(\frac{dy}{dx} + \frac{y}{x} = e^x\) with \(y(0) = 3\) has an issue
because at \(x = 0\), we have division by zero.

For the general solution away from \(x = 0\):
\(y = \frac{e^x(x-1) + c}{x}\)

The initial condition suggests we need to examine the behavior near
\(x = 0\).

\textbf{General solution: \(y = \frac{e^x(x-1) + c}{x}\) for
\(x \neq 0\)}

\begin{center}\rule{0.5\linewidth}{0.5pt}\end{center}

\subsection*{Formula Cheat Sheet}\label{formula-cheat-sheet}

\subsubsection{\texorpdfstring{\textbf{Matrix
Operations}}{Matrix Operations}}\label{matrix-operations}

\begin{itemize}
\tightlist
\item
  Matrix multiplication: \((AB)_{ij} = \sum_{k} A_{ik}B_{kj}\)
\item
  Inverse of 2\times2 matrix:
  \(A^{-1} = \frac{1}{|A|}\begin{bmatrix} d & -b \\ -c & a \end{bmatrix}\)
  for \(A = \begin{bmatrix} a & b \\ c & d \end{bmatrix}\)
\item
  Determinant: \(|A| = ad - bc\)
\end{itemize}

\subsubsection{\texorpdfstring{\textbf{Differentiation
Rules}}{Differentiation Rules}}\label{differentiation-rules}

\begin{itemize}
\tightlist
\item
  Power rule: \(\frac{d}{dx}(x^n) = nx^{n-1}\)
\item
  Product rule: \(\frac{d}{dx}(uv) = u\frac{dv}{dx} + v\frac{du}{dx}\)
\item
  Quotient rule:
  \(\frac{d}{dx}\left(\frac{u}{v}\right) = \frac{v\frac{du}{dx} - u\frac{dv}{dx}}{v^2}\)
\item
  Chain rule: \(\frac{d}{dx}[f(g(x))] = f'(g(x)) \cdot g'(x)\)
\end{itemize}

\subsubsection{\texorpdfstring{\textbf{Integration
Rules}}{Integration Rules}}\label{integration-rules}

\begin{itemize}
\tightlist
\item
  Power rule: \(\int x^n dx = \frac{x^{n+1}}{n+1} + c\) (for
  \(n \neq -1\))
\item
  Integration by parts: \(\int u \, dv = uv - \int v \, du\)
\item
  Fundamental theorem: \(\int_a^b f(x) dx = F(b) - F(a)\)
\end{itemize}

\subsubsection{\texorpdfstring{\textbf{Differential
Equations}}{Differential Equations}}\label{differential-equations}

\begin{itemize}
\tightlist
\item
  Linear first order: \(\frac{dy}{dx} + Py = Q\), Solution:
  \(y \cdot I.F. = \int Q \cdot I.F. \, dx\)
\item
  Integrating factor: \(I.F. = e^{\int P dx}\)
\item
  Variable separable: \(\frac{dy}{dx} = f(x)g(y)\) \rightarrow
  \(\frac{dy}{g(y)} = f(x)dx\)
\end{itemize}

\subsubsection{\texorpdfstring{\textbf{Complex
Numbers}}{Complex Numbers}}\label{complex-numbers}

\begin{itemize}
\tightlist
\item
  Polar form: \(z = r(\cos \theta + i \sin \theta)\)
\item
  Modulus: \(|a + bi| = \sqrt{a^2 + b^2}\)
\item
  Argument: \(\arg(a + bi) = \tan^{-1}(b/a)\)
\item
  De Moivre's theorem:
  \((\cos \theta + i \sin \theta)^n = \cos(n\theta) + i \sin(n\theta)\)
\end{itemize}

\subsection*{Problem-Solving
Strategies}\label{problem-solving-strategies}

\begin{enumerate}
\tightlist
\item
  \textbf{Matrix Problems}: Always check dimensions before
  multiplication
\item
  \textbf{Differentiation}: Identify which rule applies (product,
  quotient, chain)
\item
  \textbf{Integration}: Look for substitution opportunities first
\item
  \textbf{Differential Equations}: Identify type (separable vs linear)
  before solving
\item
  \textbf{Complex Numbers}: Convert to standard form before operations
\end{enumerate}

\subsection*{Common Mistakes to Avoid}\label{common-mistakes-to-avoid}

\begin{enumerate}
\tightlist
\item
  \textbf{Matrix multiplication}: Order matters - \(AB \neq BA\) in
  general
\item
  \textbf{Differentiation}: Don't forget the chain rule for composite
  functions
\item
  \textbf{Integration}: Always add the constant of integration
\item
  \textbf{Complex numbers}: Be careful with signs when rationalizing
\end{enumerate}

\subsection*{Exam Tips}\label{exam-tips}

\begin{enumerate}
\tightlist
\item
  \textbf{Time management}: Allocate time based on marks (1 mark = 2-3
  minutes)
\item
  \textbf{Show work}: Partial marks are awarded for correct steps
\item
  \textbf{Check units}: Ensure final answers have appropriate units
\item
  \textbf{Verify}: When possible, substitute back to check answers
\end{enumerate}


\end{document}
