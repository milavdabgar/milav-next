\documentclass[10pt,a4paper]{article}

% content/resources/templates/preamble.tex
\usepackage[margin=0.6in]{geometry}
\author{Milav Dabgar}
\usepackage{amsmath,amssymb,amsthm}
\usepackage{booktabs}
\usepackage{multirow}
\usepackage{xcolor}
\usepackage{tcolorbox}
\tcbuselibrary{breakable,skins}
\usepackage[colorlinks=true,linkcolor=blue]{hyperref}
\usepackage{titlesec}
\usepackage{enumitem}
\usepackage{tikz}
\usepackage{pgfplots}
\usepackage{circuitikz}
\usepackage[version=4]{mhchem}
\usepackage{longtable}
\usepackage{array}
\usepackage{float}
\usepackage{caption}
\usepackage{listings}

\lstset{
  basicstyle=\small\ttfamily,
  breaklines=true,
  breakatwhitespace=false,
  postbreak=\mbox{\textcolor{red}{$\hookrightarrow$}\space},
  float=false,
  numbers=left,
  numberstyle=\tiny\color{gray},
  numbersep=10pt,
  xleftmargin=2em,
  keywordstyle=\color{blue},
  commentstyle=\color{green!60!black},
  stringstyle=\color{purple},
  backgroundcolor=\color{gray!5},
  showstringspaces=false,
  tabsize=2,
  captionpos=b,
  keepspaces=true,
  columns=flexible
}

\pgfplotsset{compat=1.18}
\usetikzlibrary{shapes,arrows,positioning,calc,patterns,decorations.pathmorphing,decorations.markings,arrows.meta}

% Color scheme
\definecolor{headcolor}{RGB}{0,102,204}
\definecolor{keycolor}{RGB}{220,20,60}
\definecolor{solutioncolor}{RGB}{34,139,34}
\definecolor{mnemoniccolor}{RGB}{148,0,211}
\definecolor{codecolor}{RGB}{0,0,100}

% Spacing
\setlength{\parskip}{3pt}
\setlist[itemize]{nosep}
\setlist[enumerate]{nosep}

% Title formatting
\titleformat{\section}{\Large\bfseries\color{headcolor}}{\thesection}{1em}{}
\titleformat{\subsection}{\large\bfseries\color{headcolor}}{\thesubsection}{1em}{}

% Pandoc tightlist compatibility
\providecommand{\tightlist}{%
  \setlength{\itemsep}{0pt}\setlength{\parskip}{0pt}}

% Pandoc longtable compatibility
\newcounter{none}
\def\thenone{}


% content/resources/templates/english-boxes.tex
% This file is currently empty - it exists to maintain consistency with the import structure.
% Add custom environments here if needed in the future.


\begin{document}

\begin{center}
{\Huge\bfseries\color{headcolor} Subject Name Solutions}\\[5pt]
{\LARGE 4320002 -- Winter 2023}\\[3pt]
{\large Semester 1 Study Material}\\[3pt]
{\normalsize\textit{Detailed Solutions and Explanations}}
\end{center}

\vspace{10pt}

\subsection*{Q.1 Fill in the blanks [14
marks]}\label{q.1-fill-in-the-blanks-14-marks}

\subsubsection{Q1.1 [1 mark]}\label{q1.1-1-mark}

\textbf{Order of the matrix
\(\begin{bmatrix} 2 & 5 \\ 7 & 8 \end{bmatrix}\) is \_\_\_\_\_\_\_\_\_}

\begin{solutionbox}
(d) \(2 \times 2\)

\textbf{Solution}: The matrix has 2 rows and 2 columns, so its order is
\(2 \times 2\).

\end{solutionbox}
\subsubsection{Q1.2 [1 mark]}\label{q1.2-1-mark}

**\$

\begin{bmatrix} 4 & 3 \\ 6 & 2 \end{bmatrix}

\begin{itemize}
\item
  \begin{bmatrix} 1 & 5 \\ 5 & 8 \end{bmatrix}

  = \$ \_\_\_\_\_\_\_\_\_**
\end{itemize}

\begin{solutionbox}
(a) \(\begin{bmatrix} 5 & 8 \\ 11 & 10 \end{bmatrix}\)

\textbf{Solution}:
\(\begin{bmatrix} 4 & 3 \\ 6 & 2 \end{bmatrix} + \begin{bmatrix} 1 & 5 \\ 5 & 8 \end{bmatrix} = \begin{bmatrix} 4+1 & 3+5 \\ 6+5 & 2+8 \end{bmatrix} = \begin{bmatrix} 5 & 8 \\ 11 & 10 \end{bmatrix}\)

\end{solutionbox}
\subsubsection{Q1.3 [1 mark]}\label{q1.3-1-mark}

\textbf{Which of the following is a square matrix?}

\begin{solutionbox}
(c) \(\begin{bmatrix} 1 & 3 \\ 5 & 4 \end{bmatrix}\)

\textbf{Solution}: A square matrix has equal number of rows and columns.
Only option (c) has \(2 \times 2\) dimensions.

\end{solutionbox}
\subsubsection{Q1.4 [1 mark]}\label{q1.4-1-mark}

\textbf{If \(A = [3]\) and \(B = [4]\) then \$A \cdot B = \$
\_\_\_\_\_\_\_\_\_}

\begin{solutionbox}
(b) 12

\textbf{Solution}:
\(A \cdot B = [3] \times [4] = [3 \times 4] = [12] = 12\)

\end{solutionbox}
\subsubsection{Q1.5 [1 mark]}\label{q1.5-1-mark}

\textbf{\$\frac{d}{dx}\sin x = \$ \_\_\_\_\_\_\_\_\_}

\begin{solutionbox}
(d) \(\cos x\)

\textbf{Solution}: The derivative of \(\sin x\) is \(\cos x\).

\end{solutionbox}
\subsubsection{Q1.6 [1 mark]}\label{q1.6-1-mark}

\textbf{If \(f(x) = xe^x\) then \$f'(0) = \$ \_\_\_\_\_\_\_\_\_}

\begin{solutionbox}
(b) 1

\textbf{Solution}: Using product rule:
\(f'(x) = \frac{d}{dx}(xe^x) = e^x + xe^x = e^x(1 + x)\)
\(f'(0) = e^0(1 + 0) = 1 \times 1 = 1\)

\end{solutionbox}
\subsubsection{Q1.7 [1 mark]}\label{q1.7-1-mark}

\textbf{If \(y = x^2\) then \$\frac{d^2y}{dx^2} = \$ \_\_\_\_\_\_\_\_\_}

\begin{solutionbox}
(b) 2

\textbf{Solution}: \(y = x^2\) \(\frac{dy}{dx} = 2x\)
\(\frac{d^2y}{dx^2} = 2\)

\end{solutionbox}
\subsubsection{Q1.8 [1 mark]}\label{q1.8-1-mark}

\textbf{\$\int \cos x dx = \$ \_\_\_\_\_\_\_\_\_ \(+ c\)}

\begin{solutionbox}
(a) \(\sin x\)

\textbf{Solution}: \(\int \cos x dx = \sin x + c\)

\end{solutionbox}
\subsubsection{Q1.9 [1 mark]}\label{q1.9-1-mark}

\textbf{\$\int\emph{0\^{}1 x dx = \$ }\_\_\_\_\_\_\_\_}

\begin{solutionbox}
(c) \(\frac{1}{2}\)

\textbf{Solution}:
\(\int_0^1 x dx = \left[\frac{x^2}{2}\right]_0^1 = \frac{1^2}{2} - \frac{0^2}{2} = \frac{1}{2}\)

\end{solutionbox}
\subsubsection{Q1.10 [1 mark]}\label{q1.10-1-mark}

\textbf{\$\int \frac{1}{1+x^2} dx = \$ \_\_\_\_\_\_\_\_\_ \(+ c\)}

\begin{solutionbox}
(a) \(\tan^{-1} x\)

\textbf{Solution}: \(\int \frac{1}{1+x^2} dx = \tan^{-1} x + c\)

\end{solutionbox}
\subsubsection{Q1.11 [1 mark]}\label{q1.11-1-mark}

\textbf{Order of differential equation \(x\sin y + xy = x\) is
\_\_\_\_\_\_\_\_\_}

\begin{solutionbox}
(b) 1

\textbf{Solution}: The equation can be written as
\(\frac{dy}{dx} = \frac{1-xy}{\sin y}\). The highest order derivative is
first order.

\end{solutionbox}
\subsubsection{Q1.12 [1 mark]}\label{q1.12-1-mark}

\textbf{Integration factor of \(\frac{dy}{dx} + y = x\) is
\_\_\_\_\_\_\_\_\_}

\begin{solutionbox}
(d) \(e^x\)

\textbf{Solution}: For \(\frac{dy}{dx} + Py = Q\), integration factor
\(= e^{\int P dx} = e^{\int 1 dx} = e^x\)

\end{solutionbox}
\subsubsection{Q1.13 [1 mark]}\label{q1.13-1-mark}

\textbf{\$i\^{}2 = \$ \_\_\_\_\_\_\_\_\_}

\begin{solutionbox}
(b) -1

\textbf{Solution}: By definition, \(i^2 = -1\)

\end{solutionbox}
\subsubsection{Q1.14 [1 mark]}\label{q1.14-1-mark}

\textbf{\$(2+3i)(2-3i) = \$ \_\_\_\_\_\_\_\_\_}

\begin{solutionbox}
(c) 13

\textbf{Solution}:
\((2+3i)(2-3i) = 2^2 - (3i)^2 = 4 - 9i^2 = 4 - 9(-1) = 4 + 9 = 13\)

\end{solutionbox}
\subsection*{Q.2(A) Attempt any two [6
marks]}\label{q.2a-attempt-any-two-6-marks}

\subsubsection{Q2.1(A)(1) [3 marks]}\label{q2.1a1-3-marks}

\textbf{If \(A = \begin{bmatrix} 2 & 5 \\ -1 & 3 \end{bmatrix}\),
\(B = \begin{bmatrix} 5 & 8 \\ 4 & 6 \end{bmatrix}\) and
\(C = \begin{bmatrix} 4 & 2 \\ 1 & 5 \end{bmatrix}\) then find
\(2A + 3B - C\)}

\textbf{Solution}:
\(2A = 2\begin{bmatrix} 2 & 5 \\ -1 & 3 \end{bmatrix} = \begin{bmatrix} 4 & 10 \\ -2 & 6 \end{bmatrix}\)

\(3B = 3\begin{bmatrix} 5 & 8 \\ 4 & 6 \end{bmatrix} = \begin{bmatrix} 15 & 24 \\ 12 & 18 \end{bmatrix}\)

\(2A + 3B = \begin{bmatrix} 4 & 10 \\ -2 & 6 \end{bmatrix} + \begin{bmatrix} 15 & 24 \\ 12 & 18 \end{bmatrix} = \begin{bmatrix} 19 & 34 \\ 10 & 24 \end{bmatrix}\)

\(2A + 3B - C = \begin{bmatrix} 19 & 34 \\ 10 & 24 \end{bmatrix} - \begin{bmatrix} 4 & 2 \\ 1 & 5 \end{bmatrix} = \begin{bmatrix} 15 & 32 \\ 9 & 19 \end{bmatrix}\)

\subsubsection{Q2.1(A)(2) [3 marks]}\label{q2.1a2-3-marks}

\textbf{If \(M = \begin{bmatrix} 1 & 4 \\ 3 & 7 \end{bmatrix}\) and
\(N = \begin{bmatrix} 6 & 9 \\ 0 & 5 \end{bmatrix}\) then prove that
\((M+N)^T = M^T + N^T\)}

\textbf{Solution}:
\(M + N = \begin{bmatrix} 1 & 4 \\ 3 & 7 \end{bmatrix} + \begin{bmatrix} 6 & 9 \\ 0 & 5 \end{bmatrix} = \begin{bmatrix} 7 & 13 \\ 3 & 12 \end{bmatrix}\)

\((M+N)^T = \begin{bmatrix} 7 & 3 \\ 13 & 12 \end{bmatrix}\)

\(M^T = \begin{bmatrix} 1 & 3 \\ 4 & 7 \end{bmatrix}\),
\(N^T = \begin{bmatrix} 6 & 0 \\ 9 & 5 \end{bmatrix}\)

\(M^T + N^T = \begin{bmatrix} 1 & 3 \\ 4 & 7 \end{bmatrix} + \begin{bmatrix} 6 & 0 \\ 9 & 5 \end{bmatrix} = \begin{bmatrix} 7 & 3 \\ 13 & 12 \end{bmatrix}\)

Hence, \((M+N)^T = M^T + N^T\) is proved.

\subsubsection{Q2.1(A)(3) [3 marks]}\label{q2.1a3-3-marks}

\textbf{Solve differential equation: \(x\frac{dy}{dx} + y = xy\)}

\textbf{Solution}: \(x\frac{dy}{dx} + y = xy\)
\(\frac{dy}{dx} + \frac{y}{x} = y\)
\(\frac{dy}{dx} = y - \frac{y}{x} = y\left(1 - \frac{1}{x}\right) = y\left(\frac{x-1}{x}\right)\)

Separating variables: \(\frac{dy}{y} = \frac{x-1}{x}dx\)

Integrating:
\(\ln|y| = \int\frac{x-1}{x}dx = \int\left(1 - \frac{1}{x}\right)dx = x - \ln|x| + C\)

\(y = Ae^{x-\ln|x|} = A\frac{e^x}{x}\)

\subsection*{Q.2(B) Attempt any two [8
marks]}\label{q.2b-attempt-any-two-8-marks}

\subsubsection{Q2.1(B)(1) [4 marks]}\label{q2.1b1-4-marks}

\textbf{Solve equations \(2x + 3y = 8\), \(3x + 4y = 11\) using matrix
method}

\textbf{Solution}: Writing in matrix form: \(AX = B\)
\(\begin{bmatrix} 2 & 3 \\ 3 & 4 \end{bmatrix}\begin{bmatrix} x \\ y \end{bmatrix} = \begin{bmatrix} 8 \\ 11 \end{bmatrix}\)

Finding \(A^{-1}\): \(|A| = 2(4) - 3(3) = 8 - 9 = -1\)

\(A^{-1} = \frac{1}{|A|}\begin{bmatrix} 4 & -3 \\ -3 & 2 \end{bmatrix} = \begin{bmatrix} -4 & 3 \\ 3 & -2 \end{bmatrix}\)

\(X = A^{-1}B = \begin{bmatrix} -4 & 3 \\ 3 & -2 \end{bmatrix}\begin{bmatrix} 8 \\ 11 \end{bmatrix} = \begin{bmatrix} -32+33 \\ 24-22 \end{bmatrix} = \begin{bmatrix} 1 \\ 2 \end{bmatrix}\)

Therefore: \(x = 1,

y = 2\)


\subsubsection{Q2.1(B)(2) [4 marks]}\label{q2.1b2-4-marks}

\textbf{If \(A = \begin{bmatrix} 3 & 2 \\ 1 & 4 \end{bmatrix}\) and
\(B = \begin{bmatrix} 1 & 2 \\ 0 & 1 \end{bmatrix}\) then prove that
\((AB)^T = B^T A^T\)}

\textbf{Solution}:
\(AB = \begin{bmatrix} 3 & 2 \\ 1 & 4 \end{bmatrix}\begin{bmatrix} 1 & 2 \\ 0 & 1 \end{bmatrix} = \begin{bmatrix} 3 & 8 \\ 1 & 6 \end{bmatrix}\)

\((AB)^T = \begin{bmatrix} 3 & 1 \\ 8 & 6 \end{bmatrix}\)

\(A^T = \begin{bmatrix} 3 & 1 \\ 2 & 4 \end{bmatrix}\),
\(B^T = \begin{bmatrix} 1 & 0 \\ 2 & 1 \end{bmatrix}\)

\(B^T A^T = \begin{bmatrix} 1 & 0 \\ 2 & 1 \end{bmatrix}\begin{bmatrix} 3 & 1 \\ 2 & 4 \end{bmatrix} = \begin{bmatrix} 3 & 1 \\ 8 & 6 \end{bmatrix}\)

Hence, \((AB)^T = B^T A^T\) is proved.

\subsubsection{Q2.1(B)(3) [4 marks]}\label{q2.1b3-4-marks}

\textbf{If \(A = \begin{bmatrix} 2 & 3 \\ -1 & 2 \end{bmatrix}\) then
prove that \(A^2 - 4A + 7I = O\)}

\textbf{Solution}:
\(A^2 = \begin{bmatrix} 2 & 3 \\ -1 & 2 \end{bmatrix}\begin{bmatrix} 2 & 3 \\ -1 & 2 \end{bmatrix} = \begin{bmatrix} 1 & 12 \\ -4 & 1 \end{bmatrix}\)

\(4A = 4\begin{bmatrix} 2 & 3 \\ -1 & 2 \end{bmatrix} = \begin{bmatrix} 8 & 12 \\ -4 & 8 \end{bmatrix}\)

\(7I = 7\begin{bmatrix} 1 & 0 \\ 0 & 1 \end{bmatrix} = \begin{bmatrix} 7 & 0 \\ 0 & 7 \end{bmatrix}\)

\(A^2 - 4A + 7I = \begin{bmatrix} 1 & 12 \\ -4 & 1 \end{bmatrix} - \begin{bmatrix} 8 & 12 \\ -4 & 8 \end{bmatrix} + \begin{bmatrix} 7 & 0 \\ 0 & 7 \end{bmatrix} = \begin{bmatrix} 0 & 0 \\ 0 & 0 \end{bmatrix} = O\)

Hence proved.

\subsection*{Q.3(A) Attempt any two [6
marks]}\label{q.3a-attempt-any-two-6-marks}

\subsubsection{Q3.1(A)(1) [3 marks]}\label{q3.1a1-3-marks}

\textbf{Find derivative of \(f(x) = e^x\) using definition of
differentiation}

\textbf{Solution}: Using definition:
\(f'(x) = \lim_{h \to 0} \frac{f(x+h) - f(x)}{h}\)

\(f'(x) = \lim_{h \to 0} \frac{e^{x+h} - e^x}{h} = \lim_{h \to 0} \frac{e^x \cdot e^h - e^x}{h}\)

\(= \lim_{h \to 0} \frac{e^x(e^h - 1)}{h} = e^x \lim_{h \to 0} \frac{e^h - 1}{h}\)

Since \(\lim_{h \to 0} \frac{e^h - 1}{h} = 1\)

Therefore: \(f'(x) = e^x\)

\subsubsection{Q3.1(A)(2) [3 marks]}\label{q3.1a2-3-marks}

\textbf{If \(y = \log(\sin x)\) then find \(\frac{dy}{dx}\)}

\textbf{Solution}: \(y = \log(\sin x)\)

Using chain rule:
\(\frac{dy}{dx} = \frac{1}{\sin x} \cdot \frac{d}{dx}(\sin x) = \frac{1}{\sin x} \cdot \cos x = \frac{\cos x}{\sin x} = \cot x\)

\subsubsection{Q3.1(A)(3) [3 marks]}\label{q3.1a3-3-marks}

\textbf{Evaluate: \(\int\left(4x^3 + 3x^2 + \frac{2}{x}\right)dx\)}

\textbf{Solution}: \(\int\left(4x^3 + 3x^2 + \frac{2}{x}\right)dx\)

\(= \int 4x^3 dx + \int 3x^2 dx + \int \frac{2}{x} dx\)

\(= 4 \cdot \frac{x^4}{4} + 3 \cdot \frac{x^3}{3} + 2\ln|x| + C\)

\(= x^4 + x^3 + 2\ln|x| + C\)

\subsection*{Q.3(B) Attempt any two [8
marks]}\label{q.3b-attempt-any-two-8-marks}

\subsubsection{Q3.1(B)(1) [4 marks]}\label{q3.1b1-4-marks}

\textbf{If \(y = e^{\tan x} + \log(\sin x)\) then find
\(\frac{dy}{dx}\)}

\textbf{Solution}: \(y = e^{\tan x} + \log(\sin x)\)

\(\frac{dy}{dx} = \frac{d}{dx}[e^{\tan x}] + \frac{d}{dx}[\log(\sin x)]\)

For first term: \(\frac{d}{dx}[e^{\tan x}] = e^{\tan x} \cdot \sec^2 x\)

For second term:
\(\frac{d}{dx}[\log(\sin x)] = \frac{1}{\sin x} \cdot \cos x = \cot x\)

Therefore: \(\frac{dy}{dx} = e^{\tan x} \sec^2 x + \cot x\)

\subsubsection{Q3.1(B)(2) [4 marks]}\label{q3.1b2-4-marks}

\textbf{The equation of motion of a particle is \(s = t^4 + 3t\). Find
its velocity and acceleration at \(t = 2\) sec}

\textbf{Solution}: Given: \(s = t^4 + 3t\)

Velocity: \(v = \frac{ds}{dt} = 4t^3 + 3\)

At \(t = 2\): \(v = 4(2)^3 + 3 = 4(8) + 3 = 32 + 3 = 35\) units/sec

Acceleration: \(a = \frac{dv}{dt} = \frac{d^2s}{dt^2} = 12t^2\)

At \(t = 2\): \(a = 12(2)^2 = 12(4) = 48\) units/sec^{2}

\subsubsection{Q3.1(B)(3) [4 marks]}\label{q3.1b3-4-marks}

\textbf{Find the maximum and minimum value of the function
\(f(x) = 2x^3 - 3x^2 - 12x + 5\)}

\textbf{Solution}: \(f(x) = 2x^3 - 3x^2 - 12x + 5\)

\(f'(x) = 6x^2 - 6x - 12 = 6(x^2 - x - 2) = 6(x-2)(x+1)\)

For critical points: \(f'(x) = 0\) \(6(x-2)(x+1) = 0\) \(x = 2\) or
\(x = -1\)

\(f''(x) = 12x - 6\)

At \(x = -1\): \(f''(-1) = 12(-1) - 6 = -18 < 0\) (Maximum) At
\(x = 2\): \(f''(2) = 12(2) - 6 = 18 > 0\) (Minimum)

\(f(-1) = 2(-1)^3 - 3(-1)^2 - 12(-1) + 5 = -2 - 3 + 12 + 5 = 12\)
(Maximum) \(f(2) = 2(8) - 3(4) - 12(2) + 5 = 16 - 12 - 24 + 5 = -15\)
(Minimum)

\textbf{Maximum value}: 12 at \(x = -1\) \textbf{Minimum value}: -15 at
\(x = 2\)

\subsection*{Q.4(A) Attempt any two [6
marks]}\label{q.4a-attempt-any-two-6-marks}

\subsubsection{Q4.1(A)(1) [3 marks]}\label{q4.1a1-3-marks}

\textbf{Evaluate: \(\int xe^x dx\)}

\textbf{Solution}: Using integration by parts:
\(\int u dv = uv - \int v du\)

Let \(u = x\), \(dv = e^x dx\) Then \(du = dx\), \(v = e^x\)

\(\int xe^x dx = x \cdot e^x - \int e^x dx = xe^x - e^x + C = e^x(x-1) + C\)

\subsubsection{Q4.1(A)(2) [3 marks]}\label{q4.1a2-3-marks}

\textbf{Evaluate: \(\int \frac{dx}{\sqrt{9-4x^2}}\)}

\textbf{Solution}:
\(\int \frac{dx}{\sqrt{9-4x^2}} = \int \frac{dx}{\sqrt{9(1-\frac{4x^2}{9})}} = \int \frac{dx}{3\sqrt{1-\left(\frac{2x}{3}\right)^2}}\)

Let \(\frac{2x}{3} = \sin \theta\), then \(x = \frac{3\sin \theta}{2}\),
\(dx = \frac{3\cos \theta}{2} d\theta\)

\(= \int \frac{\frac{3\cos \theta}{2} d\theta}{3\sqrt{1-\sin^2 \theta}} = \int \frac{\frac{3\cos \theta}{2} d\theta}{3\cos \theta} = \int \frac{1}{2} d\theta = \frac{\theta}{2} + C\)

\(= \frac{1}{2}\sin^{-1}\left(\frac{2x}{3}\right) + C\)

\subsubsection{Q4.1(A)(3) [3 marks]}\label{q4.1a3-3-marks}

\textbf{Find complex conjugate of \(\frac{1-i}{1+i}\)}

\textbf{Solution}:
\(\frac{1-i}{1+i} = \frac{(1-i)(1-i)}{(1+i)(1-i)} = \frac{(1-i)^2}{1-i^2} = \frac{1-2i+i^2}{1-(-1)} = \frac{1-2i-1}{2} = \frac{-2i}{2} = -i\)

Complex conjugate of \(-i\) is \(\overline{-i} = i\)

\subsection*{Q.4(B) Attempt any two [8
marks]}\label{q.4b-attempt-any-two-8-marks}

\subsubsection{Q4.1(B)(1) [4 marks]}\label{q4.1b1-4-marks}

\textbf{Evaluate:
\(\int_0^{\pi/2} \frac{\sqrt{\cos x}}{\sqrt{\cos x} + \sqrt{\sin x}} dx\)}

\textbf{Solution}: Let
\(I = \int_0^{\pi/2} \frac{\sqrt{\cos x}}{\sqrt{\cos x} + \sqrt{\sin x}} dx\)

Using property: \(\int_0^a f(x)dx = \int_0^a f(a-x)dx\)

\(I = \int_0^{\pi/2} \frac{\sqrt{\cos(\pi/2-x)}}{\sqrt{\cos(\pi/2-x)} + \sqrt{\sin(\pi/2-x)}} dx = \int_0^{\pi/2} \frac{\sqrt{\sin x}}{\sqrt{\sin x} + \sqrt{\cos x}} dx\)

Adding both expressions:
\(2I = \int_0^{\pi/2} \frac{\sqrt{\cos x} + \sqrt{\sin x}}{\sqrt{\cos x} + \sqrt{\sin x}} dx = \int_0^{\pi/2} 1 dx = \frac{\pi}{2}\)

Therefore: \(I = \frac{\pi}{4}\)

\subsubsection{Q4.1(B)(2) [4 marks]}\label{q4.1b2-4-marks}

\textbf{Find the area of circle \(x^2 + y^2 = a^2\) using integration}

\textbf{Solution}: For circle \(x^2 + y^2 = a^2\), we have
\(y = \pm\sqrt{a^2-x^2}\)

Area of circle = \(4 \times\) Area in first quadrant
\(= 4\int_0^a \sqrt{a^2-x^2} dx\)

Let \(x = a\sin \theta\), \(dx = a\cos \theta d\theta\) When \(x = 0\),
\(\theta = 0\); when \(x = a\), \(\theta = \pi/2\)

\(= 4\int_0^{\pi/2} \sqrt{a^2-a^2\sin^2 \theta} \cdot a\cos \theta d\theta\)
\(= 4\int_0^{\pi/2} a\cos \theta \cdot a\cos \theta d\theta\)
\(= 4a^2\int_0^{\pi/2} \cos^2 \theta d\theta\)
\(= 4a^2 \cdot \frac{\pi}{4} = \pi a^2\)

\subsubsection{Q4.1(B)(3) [4 marks]}\label{q4.1b3-4-marks}

\textbf{Simplify:
\(\frac{(\cos 3\theta + i\sin 3\theta)^4 \cdot (\cos \theta - i\sin \theta)^5}{(\cos 2\theta - i\sin 2\theta)^3 \cdot (\cos 12\theta + i\sin 12\theta)}\)}

\textbf{Solution}: Using De Moivre's theorem:
\((\cos \theta + i\sin \theta)^n = \cos n\theta + i\sin n\theta\)

Numerator:
\((\cos 3\theta + i\sin 3\theta)^4 \cdot (\cos \theta - i\sin \theta)^5\)
\(= (\cos 12\theta + i\sin 12\theta) \cdot (\cos(-5\theta) + i\sin(-5\theta))\)
\(= \cos(12\theta - 5\theta) + i\sin(12\theta - 5\theta)\)
\(= \cos 7\theta + i\sin 7\theta\)

Denominator:
\((\cos 2\theta - i\sin 2\theta)^3 \cdot (\cos 12\theta + i\sin 12\theta)\)
\(= (\cos(-6\theta) + i\sin(-6\theta)) \cdot (\cos 12\theta + i\sin 12\theta)\)
\(= \cos(-6\theta + 12\theta) + i\sin(-6\theta + 12\theta)\)
\(= \cos 6\theta + i\sin 6\theta\)

Result:
\(\frac{\cos 7\theta + i\sin 7\theta}{\cos 6\theta + i\sin 6\theta} = \cos(7\theta - 6\theta) + i\sin(7\theta - 6\theta) = \cos \theta + i\sin \theta\)

\subsection*{Q.5(A) Attempt any two [6
marks]}\label{q.5a-attempt-any-two-6-marks}

\subsubsection{Q5.1(A)(1) [3 marks]}\label{q5.1a1-3-marks}

\textbf{If \((3x - 7) + 2iy = 5y + (5 + x)i\) then find value of x and
y}

\textbf{Solution}: \((3x - 7) + 2iy = 5y + (5 + x)i\)

Comparing real and imaginary parts: Real parts: \(3x - 7 = 5y\) \ldots{}
(1) Imaginary parts: \(2y = 5 + x\) \ldots{} (2)

From equation (2): \(x = 2y - 5\) \ldots{} (3)

Substituting (3) in (1): \(3(2y - 5) - 7 = 5y\) \(6y - 15 - 7 = 5y\)
\(6y - 22 = 5y\) \(y = 22\)

From (3): \(x = 2(22) - 5 = 44 - 5 = 39\)

Therefore: \(x = 39,

y = 22\)


\subsubsection{Q5.1(A)(2) [3 marks]}\label{q5.1a2-3-marks}

\textbf{Convert \(z = 1 + \sqrt{3}i\) into polar form}

\textbf{Solution}: \(z = 1 + \sqrt{3}i\)

Modulus:
\(|z| = \sqrt{1^2 + (\sqrt{3})^2} = \sqrt{1 + 3} = \sqrt{4} = 2\)

Argument:
\(\arg(z) = \tan^{-1}\left(\frac{\sqrt{3}}{1}\right) = \tan^{-1}(\sqrt{3}) = \frac{\pi}{3}\)

Polar form:
\(z = |z|(\cos \theta + i\sin \theta) = 2\left(\cos \frac{\pi}{3} + i\sin \frac{\pi}{3}\right)\)

\subsubsection{Q5.1(A)(3) [3 marks]}\label{q5.1a3-3-marks}

\textbf{Express \(\frac{4 + 2i}{(3 + 2i)(5 - 3i)}\) in \(a + ib\) form}

\textbf{Solution}: First, simplify denominator:
\((3 + 2i)(5 - 3i) = 15 - 9i + 10i - 6i^2 = 15 + i - 6(-1) = 15 + i + 6 = 21 + i\)

\(\frac{4 + 2i}{21 + i} = \frac{(4 + 2i)(21 - i)}{(21 + i)(21 - i)} = \frac{84 - 4i + 42i - 2i^2}{21^2 - i^2} = \frac{84 + 38i + 2}{441 + 1} = \frac{86 + 38i}{442}\)

\(= \frac{86}{442} + \frac{38}{442}i = \frac{43}{221} + \frac{19}{221}i\)

\subsection*{Q.5(B) Attempt any two [8
marks]}\label{q.5b-attempt-any-two-8-marks}

\subsubsection{Q5.1(B)(1) [4 marks]}\label{q5.1b1-4-marks}

\textbf{Solve differential equation: \(\frac{dy}{dx} + 2y = 3e^x\)}

\textbf{Solution}: This is a first-order linear differential equation of
the form \(\frac{dy}{dx} + Py = Q\)

Here: \(P = 2\), \(Q = 3e^x\)

Integration factor: \(\mu = e^{\int P dx} = e^{\int 2 dx} = e^{2x}\)

Multiplying equation by \(\mu\):
\(e^{2x}\frac{dy}{dx} + 2e^{2x}y = 3e^{2x} \cdot e^x = 3e^{3x}\)

This gives: \(\frac{d}{dx}(ye^{2x}) = 3e^{3x}\)

Integrating both sides:
\(ye^{2x} = \int 3e^{3x} dx = 3 \cdot \frac{e^{3x}}{3} + C = e^{3x} + C\)

Therefore: \(y = \frac{e^{3x} + C}{e^{2x}} = e^x + Ce^{-2x}\)

\subsubsection{Q5.1(B)(2) [4 marks]}\label{q5.1b2-4-marks}

\textbf{Solve differential equation: \(\frac{dy}{dx} = (x + y)^2\)}

\textbf{Solution}: Let \(v = x + y\), then
\(\frac{dv}{dx} = 1 + \frac{dy}{dx}\)

So \(\frac{dy}{dx} = \frac{dv}{dx} - 1\)

Substituting in the original equation: \(\frac{dv}{dx} - 1 = v^2\)
\(\frac{dv}{dx} = v^2 + 1\)

Separating variables: \(\frac{dv}{v^2 + 1} = dx\)

Integrating both sides: \(\int \frac{dv}{v^2 + 1} = \int dx\)
\(\tan^{-1}(v) = x + C\) \(v = \tan(x + C)\)

Substituting back: \(x + y = \tan(x + C)\) Therefore:
\(y = \tan(x + C) - x\)

\subsubsection{Q5.1(B)(3) [4 marks]}\label{q5.1b3-4-marks}

\textbf{Solve differential equation:
\(\frac{dy}{dx} + \frac{y}{x} = e^x\), \(y(0) = 2\)}

\textbf{Solution}: This is a first-order linear differential equation:
\(\frac{dy}{dx} + \frac{y}{x} = e^x\)

Here: \(P = \frac{1}{x}\), \(Q = e^x\)

Integration factor:
\(\mu = e^{\int \frac{1}{x} dx} = e^{\ln|x|} = |x| = x\) (for \(x > 0\))

Multiplying equation by \(\mu = x\): \(x\frac{dy}{dx} +

y = xe^x\)


This gives: \(\frac{d}{dx}(xy) = xe^x\)

Integrating both sides using integration by parts: \(xy = \int xe^x dx\)

For \(\int xe^x dx\): Let \(u = x\), \(dv = e^x dx\) Then \(du = dx\),
\(v = e^x\)
\(\int xe^x dx = xe^x - \int e^x dx = xe^x - e^x = e^x(x-1)\)

So: \(xy = e^x(x-1) + C\) \(y = \frac{e^x(x-1) + C}{x}\)

Using initial condition \(y(0) = 2\): This presents a problem as we have
division by zero. The equation needs to be solved more carefully near
\(x = 0\).

For the general solution:
\(y = e^x\left(1 - \frac{1}{x}\right) + \frac{C}{x}\)

\begin{center}\rule{0.5\linewidth}{0.5pt}\end{center}

\subsection*{Formula Cheat Sheet}\label{formula-cheat-sheet}

\subsubsection{\texorpdfstring{\textbf{Matrix
Operations}}{Matrix Operations}}\label{matrix-operations}

\begin{itemize}
\tightlist
\item
  Matrix addition: \((A + B)_{ij} = A_{ij} + B_{ij}\)
\item
  Matrix multiplication: \((AB)_{ij} = \sum_{k} A_{ik}B_{kj}\)
\item
  Transpose: \((A^T)_{ij} = A_{ji}\)
\item
  Inverse of 2\times2 matrix:
  \(A^{-1} = \frac{1}{|A|}\begin{bmatrix} d & -b \\ -c & a \end{bmatrix}\)
  where \(A = \begin{bmatrix} a & b \\ c & d \end{bmatrix}\)
\end{itemize}

\subsubsection{\texorpdfstring{\textbf{Differentiation
Formulas}}{Differentiation Formulas}}\label{differentiation-formulas}

\begin{itemize}
\tightlist
\item
  \(\frac{d}{dx}(x^n) = nx^{n-1}\)
\item
  \(\frac{d}{dx}(e^x) = e^x\)
\item
  \(\frac{d}{dx}(\sin x) = \cos x\)
\item
  \(\frac{d}{dx}(\cos x) = -\sin x\)
\item
  \(\frac{d}{dx}(\tan x) = \sec^2 x\)
\item
  \(\frac{d}{dx}(\ln x) = \frac{1}{x}\)
\item
  Product rule: \((uv)' = u'v + uv'\)
\item
  Chain rule: \(\frac{d}{dx}f(g(x)) = f'(g(x)) \cdot g'(x)\)
\end{itemize}

\subsubsection{\texorpdfstring{\textbf{Integration
Formulas}}{Integration Formulas}}\label{integration-formulas}

\begin{itemize}
\tightlist
\item
  \(\int x^n dx = \frac{x^{n+1}}{n+1} + C\) (for \(n \neq -1\))
\item
  \(\int \frac{1}{x} dx = \ln|x| + C\)
\item
  \(\int e^x dx = e^x + C\)
\item
  \(\int \sin x dx = -\cos x + C\)
\item
  \(\int \cos x dx = \sin x + C\)
\item
  \(\int \sec^2 x dx = \tan x + C\)
\item
  \(\int \frac{1}{1+x^2} dx = \tan^{-1} x + C\)
\item
  \(\int \frac{1}{\sqrt{1-x^2}} dx = \sin^{-1} x + C\)
\end{itemize}

\subsubsection{\texorpdfstring{\textbf{Differential
Equations}}{Differential Equations}}\label{differential-equations}

\begin{itemize}
\tightlist
\item
  First-order linear: \(\frac{dy}{dx} + Py = Q\)
\item
  Integration factor: \(\mu = e^{\int P dx}\)
\item
  Solution: \(y = \frac{1}{\mu}\left[\int \mu Q dx + C\right]\)
\item
  Variable separable: \(\frac{dy}{dx} = f(x)g(y)\) \rightarrow
  \(\frac{dy}{g(y)} = f(x)dx\)
\end{itemize}

\subsubsection{\texorpdfstring{\textbf{Complex
Numbers}}{Complex Numbers}}\label{complex-numbers}

\begin{itemize}
\tightlist
\item
  \(i^2 = -1\), \(i^3 = -i\), \(i^4 = 1\)
\item
  Modulus: \(|a + bi| = \sqrt{a^2 + b^2}\)
\item
  Argument: \(\arg(a + bi) = \tan^{-1}\left(\frac{b}{a}\right)\)
\item
  Polar form: \(z = r(\cos \theta + i\sin \theta)\)
\item
  De Moivre's theorem:
  \((\cos \theta + i\sin \theta)^n = \cos n\theta + i\sin n\theta\)
\end{itemize}

\begin{center}\rule{0.5\linewidth}{0.5pt}\end{center}

\subsection*{Problem-Solving
Strategies}\label{problem-solving-strategies}

\subsubsection{\texorpdfstring{\textbf{Matrix
Problems}}{Matrix Problems}}\label{matrix-problems}

\begin{enumerate}
\tightlist
\item
  \textbf{Always check dimensions} before performing operations
\item
  \textbf{For matrix equations}: Use inverse method \(X = A^{-1}B\)
\item
  \textbf{For transpose properties}: Use \((AB)^T = B^T A^T\)
\item
  \textbf{For matrix powers}: Calculate step by step, look for patterns
\end{enumerate}

\subsubsection{\texorpdfstring{\textbf{Differentiation
Problems}}{Differentiation Problems}}\label{differentiation-problems}

\begin{enumerate}
\tightlist
\item
  \textbf{Identify the type}: Product, quotient, chain rule, or implicit
\item
  \textbf{For complex functions}: Break down using appropriate rules
\item
  \textbf{For applications}: Remember \(v = \frac{ds}{dt}\),
  \(a = \frac{dv}{dt}\)
\item
  \textbf{For maxima/minima}: Find critical points where \(f'(x) = 0\)
\end{enumerate}

\subsubsection{\texorpdfstring{\textbf{Integration
Problems}}{Integration Problems}}\label{integration-problems}

\begin{enumerate}
\tightlist
\item
  \textbf{Recognize standard forms} first
\item
  \textbf{For substitution}: Look for \(f'(x)\) when \(f(x)\) appears
\item
  \textbf{For integration by parts}: Choose \(u\) as LIATE (Log, Inverse
  trig, Algebraic, Trig, Exponential)
\item
  \textbf{For definite integrals}: Use fundamental theorem or properties
\end{enumerate}

\subsubsection{\texorpdfstring{\textbf{Differential
Equations}}{Differential Equations}}\label{differential-equations-1}

\begin{enumerate}
\tightlist
\item
  \textbf{Identify the type}: Linear, separable, or exact
\item
  \textbf{For linear equations}: Find integration factor systematically
\item
  \textbf{For separable equations}: Separate variables completely before
  integrating
\item
  \textbf{Always check initial conditions} if given
\end{enumerate}

\subsubsection{\texorpdfstring{\textbf{Complex
Numbers}}{Complex Numbers}}\label{complex-numbers-1}

\begin{enumerate}
\tightlist
\item
  \textbf{For operations}: Convert to \(a + bi\) form first
\item
  \textbf{For polar form}: Calculate modulus and argument carefully
\item
  \textbf{For powers}: Use De Moivre's theorem
\item
  \textbf{For division}: Multiply by conjugate of denominator
\end{enumerate}

\begin{center}\rule{0.5\linewidth}{0.5pt}\end{center}

\subsection*{Common Mistakes to Avoid}\label{common-mistakes-to-avoid}

\subsubsection{\texorpdfstring{\textbf{Matrix
Operations}}{Matrix Operations}}\label{matrix-operations-1}

\begin{itemize}
\tightlist
\item
  ❌ \textbf{Don't assume} \(AB = BA\) (matrix multiplication is not
  commutative)
\item
  ❌ \textbf{Don't forget} to check if matrices can be multiplied (inner
  dimensions must match)
\item
  ❌ \textbf{Don't confuse} transpose with inverse
\end{itemize}

\subsubsection{\texorpdfstring{\textbf{Differentiation}}{Differentiation}}\label{differentiation}

\begin{itemize}
\tightlist
\item
  ❌ \textbf{Don't forget} the chain rule for composite functions
\item
  ❌ \textbf{Don't mix up} \(\frac{d}{dx}(\sin x) = \cos x\) and
  \(\frac{d}{dx}(\cos x) = -\sin x\)
\item
  ❌ \textbf{Don't forget} to use product rule when multiplying
  functions
\end{itemize}

\subsubsection{\texorpdfstring{\textbf{Integration}}{Integration}}\label{integration}

\begin{itemize}
\tightlist
\item
  ❌ \textbf{Don't forget} the constant of integration \(+C\)
\item
  ❌ \textbf{Don't confuse} indefinite and definite integrals
\item
  ❌ \textbf{Don't forget} to substitute limits properly in definite
  integrals
\end{itemize}

\subsubsection{\texorpdfstring{\textbf{Complex
Numbers}}{Complex Numbers}}\label{complex-numbers-2}

\begin{itemize}
\tightlist
\item
  ❌ \textbf{Don't forget} \(i^2 = -1\) when expanding
\item
  ❌ \textbf{Don't confuse} modulus with real part
\item
  ❌ \textbf{Don't forget} to rationalize denominators with complex
  numbers
\end{itemize}

\begin{center}\rule{0.5\linewidth}{0.5pt}\end{center}

\subsection*{Exam Tips}\label{exam-tips}

\subsubsection{\texorpdfstring{\textbf{Time
Management}}{Time Management}}\label{time-management}

\begin{itemize}
\tightlist
\item
  \textbf{Spend 2-3 minutes} reading the entire paper first
\item
  \textbf{Attempt easier questions first} to build confidence
\item
  \textbf{Reserve 15 minutes} at the end for review
\end{itemize}

\subsubsection{\texorpdfstring{\textbf{Writing
Strategy}}{Writing Strategy}}\label{writing-strategy}

\begin{itemize}
\tightlist
\item
  \textbf{Show all steps clearly} - partial marks are often awarded
\item
  \textbf{Draw diagrams where helpful} - especially for geometry
  problems
\item
  \textbf{Write final answers clearly} and box them if possible
\end{itemize}

\subsubsection{\texorpdfstring{\textbf{Calculation
Tips}}{Calculation Tips}}\label{calculation-tips}

\begin{itemize}
\tightlist
\item
  \textbf{Double-check arithmetic} - many marks are lost due to
  calculation errors
\item
  \textbf{Use calculator efficiently} but don't become dependent on it
\item
  \textbf{Cross-verify answers} using different methods when possible
\end{itemize}

\subsubsection{\texorpdfstring{\textbf{Question
Selection}}{Question Selection}}\label{question-selection}

\begin{itemize}
\tightlist
\item
  \textbf{In OR questions}, choose the one you're most confident about
\item
  \textbf{Don't spend too much time} on any single question
\item
  \textbf{If stuck}, move on and return later with fresh perspective
\end{itemize}

\textbf{Good luck with your exam preparation!}


\end{document}
