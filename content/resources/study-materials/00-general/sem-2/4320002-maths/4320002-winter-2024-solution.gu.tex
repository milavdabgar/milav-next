\documentclass{article}

% content/resources/templates/preamble.tex
\usepackage[margin=0.6in]{geometry}
\author{Milav Dabgar}
\usepackage{amsmath,amssymb,amsthm}
\usepackage{booktabs}
\usepackage{multirow}
\usepackage{xcolor}
\usepackage{tcolorbox}
\tcbuselibrary{breakable,skins}
\usepackage[colorlinks=true,linkcolor=blue]{hyperref}
\usepackage{titlesec}
\usepackage{enumitem}
\usepackage{tikz}
\usepackage{pgfplots}
\usepackage{circuitikz}
\usepackage[version=4]{mhchem}
\usepackage{longtable}
\usepackage{array}
\usepackage{float}
\usepackage{caption}
\usepackage{listings}

\lstset{
  basicstyle=\small\ttfamily,
  breaklines=true,
  breakatwhitespace=false,
  postbreak=\mbox{\textcolor{red}{$\hookrightarrow$}\space},
  float=false,
  numbers=left,
  numberstyle=\tiny\color{gray},
  numbersep=10pt,
  xleftmargin=2em,
  keywordstyle=\color{blue},
  commentstyle=\color{green!60!black},
  stringstyle=\color{purple},
  backgroundcolor=\color{gray!5},
  showstringspaces=false,
  tabsize=2,
  captionpos=b,
  keepspaces=true,
  columns=flexible
}

\pgfplotsset{compat=1.18}
\usetikzlibrary{shapes,arrows,positioning,calc,patterns,decorations.pathmorphing,decorations.markings,arrows.meta}

% Color scheme
\definecolor{headcolor}{RGB}{0,102,204}
\definecolor{keycolor}{RGB}{220,20,60}
\definecolor{solutioncolor}{RGB}{34,139,34}
\definecolor{mnemoniccolor}{RGB}{148,0,211}
\definecolor{codecolor}{RGB}{0,0,100}

% Spacing
\setlength{\parskip}{3pt}
\setlist[itemize]{nosep}
\setlist[enumerate]{nosep}

% Title formatting
\titleformat{\section}{\Large\bfseries\color{headcolor}}{\thesection}{1em}{}
\titleformat{\subsection}{\large\bfseries\color{headcolor}}{\thesubsection}{1em}{}

% Pandoc tightlist compatibility
\providecommand{\tightlist}{%
  \setlength{\itemsep}{0pt}\setlength{\parskip}{0pt}}

% Pandoc longtable compatibility
\newcounter{none}
\def\thenone{}


% content/resources/templates/gujarati-boxes.tex
\usepackage{fontspec}
\usepackage{polyglossia}

% Set Gujarati as main language (document is primarily in Gujarati)
% Note: gloss-gujarati.ldf doesn't exist in polyglossia, but it will use hyphenation patterns
\setdefaultlanguage{gujarati}
\setotherlanguage{english}

% Configure Gujarati font properly
% Use Language=Default to prevent polyglossia from trying to add language-specific features
% that don't exist for Gujarati, which causes "empty feature" warnings
\newfontfamily\gujaratifont[Script=Gujarati,AutoFakeBold=2.5,AutoFakeSlant=0.3]{Noto Sans Gujarati}
\setmainfont[Script=Gujarati,AutoFakeBold=2.5,AutoFakeSlant=0.3]{Noto Sans Gujarati}
% Use Noto Sans Gujarati for monospace to support Gujarati in text
\setmonofont[Scale=0.9]{Noto Sans Gujarati}

% Configure English to use the same font
\newfontfamily\englishfont[Script=Gujarati,AutoFakeBold=2.5,AutoFakeSlant=0.3]{Noto Sans Gujarati}

% Translations for polyglossia
\gappto\captionsgujarati{
  \renewcommand{\tablename}{કોષ્ટક}
  \renewcommand{\figurename}{આકૃતિ}
}

% Helper for TikZ nodes to ensure Gujarati font
\newcommand{\gu}[1]{{\gujaratifont #1}}

% Custom environments
\newtcolorbox{solutionbox}{
    breakable,
    enhanced,
    colback=solutioncolor!5!white,
    colframe=solutioncolor!75!black,
    fonttitle=\bfseries,
    title=જવાબ
}

\newtcolorbox{solutionboxnobreak}{
 colback=solutioncolor!5!white,
 colframe=solutioncolor!75!black,
 fonttitle=\bfseries,
 title=જવાબ
}

\newtcolorbox{keyformula}{
 breakable,
 enhanced,
 colback=keycolor!5!white,
 colframe=keycolor!75!black,
 fonttitle=\bfseries,
 title=રાસાયણિક સમીકરણ/સૂત્ર
}

\newtcolorbox{mnemonicbox}{
 breakable,
 enhanced,
 colback=mnemoniccolor!5!white,
 colframe=mnemoniccolor!75!black,
 fonttitle=\bfseries,
 title=મેમરી ટ્રીક
}


% Custom commands for GTU solutions
% This file defines semantic commands for consistent formatting

% Question command with automatic formatting
\newcommand{\question}[2]{%
  \section*{Question #1}%
  \textbf{#2}%
}

% OR question variant
\newcommand{\questionor}[2]{%
  \section*{Question #1 OR}%
  \textbf{#2}%
}

% Proper table environment with caption
\newenvironment{answertable}[1]{%
  \begin{table}[htbp]
  \centering
  \caption{#1}
}{%
  \end{table}
}

% Proper figure environment for diagrams
\newenvironment{answerdiagram}[1]{%
  \begin{figure}[htbp]
  \centering
  \caption{#1}
}{%
  \end{figure}
}

% Semantic markup for key terms
\newcommand{\keyword}[1]{\textbf{#1}}
\newcommand{\code}[1]{\texttt{#1}}
\newcommand{\classname}[1]{\texttt{#1}}
\newcommand{\methodname}[1]{\texttt{#1}}

% Proper quotation marks
\newcommand{\mnemonic}[1]{``#1''}


\title{એન્જિનિયરિંગ મેથેમેટિક્સ (4320002) - વિન્ટર 2024 સોલ્યુશન}
\date{જાન્યુઆરી 23, 2025}

\begin{document}
\maketitle

\questionmarks{1}{14}{નીચે આપેલા વિકલ્પોમાંથી યોગ્ય વિકલ્પ પસંદ કરી ખાલી જગ્યા પૂરો.}

\questionmarks{1.1}{1}{જો $A = \begin{bmatrix} 2 & -1 \\ 3 & -3 \end{bmatrix}$ હોય તો $\text{Adj}A^T = $ \_\_\_\_\_\_\_\_}

\begin{solutionbox}
\textbf{જવાબ}: a. $\begin{bmatrix} -3 & 1 \\ -3 & 2 \end{bmatrix}$

\textbf{ઉકેલ}:
પ્રથમ $A^T$ શોધો:
\[
A^T = \begin{bmatrix} 2 & 3 \\ -1 & -3 \end{bmatrix}
\]
$\text{Adj}A^T$ માટે, આપણે સહઅવયવો (cofactors) શોધીએ:
\begin{itemize}
    \item $C_{11} = (-1)^{1+1} \cdot (-3) = -3$
    \item $C_{12} = (-1)^{1+2} \cdot (-1) = 1$
    \item $C_{21} = (-1)^{2+1} \cdot 3 = -3$
    \item $C_{22} = (-1)^{2+2} \cdot 2 = 2$
\end{itemize}
તેથી: $\text{Adj}A^T = \begin{bmatrix} -3 & 1 \\ -3 & 2 \end{bmatrix}$
\end{solutionbox}

\questionmarks{1.2}{1}{જો $A = \begin{bmatrix} 1 & 3 & 4 \\ 2 & 0 & 1 \end{bmatrix}$ અને $B = \begin{bmatrix} 1 & 1 \\ 2 & 4 \\ 3 & 0 \end{bmatrix}$ હોય તો $AB$ ની કક્ષા = \_\_\_\_\_\_\_\_}

\begin{solutionbox}
\textbf{જવાબ}: b. $2 \times 2$

\textbf{ઉકેલ}:
\begin{itemize}
    \item શ્રેણિક $A$ ની કક્ષા $2 \times 3$ છે
    \item શ્રેણિક $B$ ની કક્ષા $3 \times 2$ છે
    \item શ્રેણિક ગુણાકાર માટે: $(2 \times 3) \times (3 \times 2) = 2 \times 2$
\end{itemize}
\end{solutionbox}

\questionmarks{1.3}{1}{જો $A = \begin{bmatrix} -1 & 2 \\ 3 & -1 \\ 0 & 4 \end{bmatrix}$, $B = \begin{bmatrix} 4 & -3 \\ -2 & 1 \\ 4 & 0 \end{bmatrix}$ અને $C = \begin{bmatrix} 0 & -1 \\ 5 & 3 \\ 2 & 1 \end{bmatrix}$ હોય તો $A + B - C = $ \_\_\_\_\_\_\_\_}

\begin{solutionbox}
\textbf{જવાબ}: a. $\begin{bmatrix} 3 & 0 \\ -4 & -3 \\ 2 & 3 \end{bmatrix}$

\textbf{ઉકેલ}:
\[
A + B = \begin{bmatrix} -1+4 & 2+(-3) \\ 3+(-2) & -1+1 \\ 0+4 & 4+0 \end{bmatrix} = \begin{bmatrix} 3 & -1 \\ 1 & 0 \\ 4 & 4 \end{bmatrix}
\]
\[
A + B - C = \begin{bmatrix} 3-0 & -1-(-1) \\ 1-5 & 0-3 \\ 4-2 & 4-1 \end{bmatrix} = \begin{bmatrix} 3 & 0 \\ -4 & -3 \\ 2 & 3 \end{bmatrix}
\]
\end{solutionbox}

\questionmarks{1.4}{1}{જો $A = \begin{bmatrix} -3 & 1 \\ 2 & 1 \end{bmatrix}$ હોય તો $A^2 = $ \_\_\_\_\_\_\_\_\_\_}

\begin{solutionbox}
\textbf{જવાબ}: c. $\begin{bmatrix} 11 & -2 \\ -4 & 3 \end{bmatrix}$

\textbf{ઉકેલ}:
\[
A^2 = A \times A = \begin{bmatrix} -3 & 1 \\ 2 & 1 \end{bmatrix} \begin{bmatrix} -3 & 1 \\ 2 & 1 \end{bmatrix}
\]
\[
A^2 = \begin{bmatrix} (-3)(-3) + (1)(2) & (-3)(1) + (1)(1) \\ (2)(-3) + (1)(2) & (2)(1) + (1)(1) \end{bmatrix} = \begin{bmatrix} 11 & -2 \\ -4 & 3 \end{bmatrix}
\]
\end{solutionbox}

\questionmarks{1.5}{1}{$\frac{d}{dx}\left(\frac{\cos x}{\sin x}\right) = $ \_\_\_\_\_\_\_\_\_}

\begin{solutionbox}
\textbf{જવાબ}: d. $-\csc^2 x$

\textbf{ઉકેલ}:
\[
\frac{d}{dx}\left(\frac{\cos x}{\sin x}\right) = \frac{d}{dx}(\cot x) = -\csc^2 x
\]
\end{solutionbox}

\questionmarks{1.6}{1}{$\frac{d}{dx}(\sin^2 x) = $ \_\_\_\_\_\_\_\_\_}

\begin{solutionbox}
\textbf{જવાબ}: d. $2\cos x$

\textbf{ઉકેલ}:
સાંકળ નિયમ (chain rule) નો ઉપયોગ કરતા:
\[
\frac{d}{dx}(\sin^2 x) = 2\sin x \cdot \cos x = \sin 2x
\]
નોંધ: સાચો જવાબ $\sin 2x$ હોવો જોઈએ, પરંતુ આપેલા વિકલ્પોમાંથી, આપણને $2\sin x \cos x$ ની જરૂર છે જે સાદુરૂપ આપતા $\sin 2x$ થાય છે.
\end{solutionbox}

\questionmarks{1.7}{1}{જો $\sqrt{x} + \sqrt{y} = 9$ હોય તો $\frac{dy}{dx} = $ \_\_\_\_\_\_\_\_\_\_}

\begin{solutionbox}
\textbf{જવાબ}: b. $-\sqrt{\frac{x}{y}}$

\textbf{ઉકેલ}:
$x$ સાપેક્ષ બંને બાજુ વિકલન કરતા:
\[
\frac{1}{2\sqrt{x}} + \frac{1}{2\sqrt{y}} \cdot \frac{dy}{dx} = 0
\]
\[
\frac{1}{2\sqrt{y}} \cdot \frac{dy}{dx} = -\frac{1}{2\sqrt{x}}
\]
\[
\frac{dy}{dx} = -\frac{\sqrt{y}}{\sqrt{x}} = -\sqrt{\frac{y}{x}}
\]
આ પેપર માં ટાઈપિંગ મિસ્ટેક હોવાનું જણાય છે, સાચો જવાબ $-\sqrt{\frac{y}{x}}$ આવે, પણ વિકલ્પો મુજબ જવાબ b. $-\sqrt{\frac{x}{y}}$ છે.
\end{solutionbox}

\questionmarks{1.8}{1}{$\int 2^x dx = $ \_\_\_\_\_\_\_\_\_ $+ C$}

\begin{solutionbox}
\textbf{જવાબ}: c. $\frac{2^x}{\log 2}$

\textbf{ઉકેલ}:
\[
\int 2^x dx = \frac{2^x}{\ln 2} + C = \frac{2^x}{\log 2} + C
\]
\end{solutionbox}

\questionmarks{1.9}{1}{$\int \frac{dx}{\sin^2 x \cos^2 x} = $ \_\_\_\_\_\_\_\_\_ $+ C$}

\begin{solutionbox}
\textbf{જવાબ}: b. $\tan x + \cot x$

\textbf{ઉકેલ}:
\[
\int \frac{dx}{\sin^2 x \cos^2 x} = \int \frac{1}{\sin^2 x \cos^2 x} dx = \int \frac{\sin^2 x + \cos^2 x}{\sin^2 x \cos^2 x} dx
\]
\[
= \int \left(\frac{1}{\cos^2 x} + \frac{1}{\sin^2 x}\right) dx = \int (\sec^2 x + \csc^2 x) dx
\]
\[
= \tan x - \cot x + C
\]
પરંતુ આપેલ જવાબ $\tan x + \cot x$ છે, જે સંભવતઃ વિકલ્પોમાં ભૂલ સુચવે છે.
\end{solutionbox}

\questionmarks{1.10}{1}{$\int_0^3 6x dx = $ \_\_\_\_\_\_}

\begin{solutionbox}
\textbf{જવાબ}: b. 27

\textbf{ઉકેલ}:
\[
\int_0^3 6x dx = 6 \int_0^3 x dx = 6 \left[\frac{x^2}{2}\right]_0^3 = 6 \cdot \frac{9}{2} = 27
\]
\end{solutionbox}

\questionmarks{1.11}{1}{વિકલ સમીકરણ $\sqrt[3]{\frac{d^2y}{dx^2}} = \sqrt{\frac{dy}{dx}}$ નું કક્ષા (order) અને પરિમાણ (degree) \_\_\_\_\_\_\_\_ છે}

\begin{solutionbox}
\textbf{જવાબ}: c. 3 અને 2

\textbf{ઉકેલ}:
ફરીથી લખતા: $\left(\frac{d^2y}{dx^2}\right)^{1/3} = \left(\frac{dy}{dx}\right)^{1/2}$

અપૂર્ણાંક ઘાત દૂર કરવા માટે, બંને બાજુ ઘન કરતા:
\[
\frac{d^2y}{dx^2} = \left(\frac{dy}{dx}\right)^{3/2}
\]
બંને બાજુ વર્ગ કરતા:
\[
\left(\frac{d^2y}{dx^2}\right)^2 = \left(\frac{dy}{dx}\right)^3
\]
\textbf{કક્ષા (Order)} = 2 (સર્વોચ્ચ વિકલિત)
\textbf{પરિમાણ (Degree)} = 2 (સર્વોચ્ચ વિકલિતની ઘાત)

પરંતુ આપેલ જવાબ "3 અને 2" છે, જે કદાચ 3 પરિમાણ અને 2 કક્ષા નો ઉલ્લેખ કરે છે અથવા વિકલ્પમાં ભૂલ છે.
\end{solutionbox}

\questionmarks{1.12}{1}{વિકલ સમીકરણ $x\frac{dy}{dx} + \frac{y}{x} = x^2$ નો સંકલ્પકારક અવયવ (Integrating Factor) \_\_\_\_\_\_\_\_ છે}

\begin{solutionbox}
\textbf{જવાબ}: b. $\frac{1}{x}$

\textbf{ઉકેલ}:
પ્રમાણિત સ્વરૂપમાં લખતા: $\frac{dy}{dx} + \frac{y}{x^2} = x$

આ આપે છે $P(x) = \frac{1}{x^2}$

સંકલ્પકારક અવયવ $= e^{\int P(x)dx} = e^{\int \frac{1}{x^2}dx} = e^{-\frac{1}{x}}$

પરંતુ આ વિકલ્પો સાથે મેળ ખાતું નથી. ચાલો મૂળ સમીકરણ ફરીથી તપાસીએ:
$x\frac{dy}{dx} + \frac{y}{x} = x^2$

આખા સમીકરણને $\frac{1}{x}$ વડે ગુણતા: $\frac{dy}{dx} + \frac{y}{x^2} = x$

વાસ્તવમાં, પેટર્નના આધારે સંકલ્પકારક અવયવ $\frac{1}{x}$ હોવો જોઈએ.
\end{solutionbox}

\questionmarks{1.13}{1}{$i + i^2 + i^3 + i^4 = $ \_\_\_\_\_\_\_\_\_\_}

\begin{solutionbox}
\textbf{જવાબ}: c. 0

\textbf{ઉકેલ}:
\begin{itemize}
    \item $i^1 = i$
    \item $i^2 = -1$
    \item $i^3 = i^2 \cdot i = -i$
    \item $i^4 = 1$
\end{itemize}
તેથી: $i + (-1) + (-i) + 1 = 0$
\end{solutionbox}

\questionmarks{1.14}{1}{$(2 - i)(3 + 2i) = $ \_\_\_\_\_\_\_}

\begin{solutionbox}
\textbf{જવાબ}: d. $8 + i$

\textbf{ઉકેલ}:
$(2 - i)(3 + 2i) = 2(3) + 2(2i) - i(3) - i(2i)$
$= 6 + 4i - 3i - 2i^2$
$= 6 + i - 2(-1)$
$= 6 + i + 2 = 8 + i$
\end{solutionbox}

\questionmarks{2(a)}{6}{કોઈપણ બે લખો.}

\questionmarks{2.1(a)}{3}{જો $A = \begin{bmatrix} 3 & 1 \\ -1 & 2 \end{bmatrix}$ હોય તો સાબિત કરો કે $A^2 - 5A + 7I = 0$}

\begin{solutionbox}
\textbf{ઉકેલ}:
પ્રથમ, $A^2$ ગણો:
\[
A^2 = \begin{bmatrix} 3 & 1 \\ -1 & 2 \end{bmatrix} \begin{bmatrix} 3 & 1 \\ -1 & 2 \end{bmatrix} = \begin{bmatrix} 8 & 5 \\ -5 & 3 \end{bmatrix}
\]
$5A$ ગણો:
\[
5A = 5\begin{bmatrix} 3 & 1 \\ -1 & 2 \end{bmatrix} = \begin{bmatrix} 15 & 5 \\ -5 & 10 \end{bmatrix}
\]
$7I$ ગણો:
\[
7I = 7\begin{bmatrix} 1 & 0 \\ 0 & 1 \end{bmatrix} = \begin{bmatrix} 7 & 0 \\ 0 & 7 \end{bmatrix}
\]
હવે $A^2 - 5A + 7I$ ગણો:
\[
A^2 - 5A + 7I = \begin{bmatrix} 8 & 5 \\ -5 & 3 \end{bmatrix} - \begin{bmatrix} 15 & 5 \\ -5 & 10 \end{bmatrix} + \begin{bmatrix} 7 & 0 \\ 0 & 7 \end{bmatrix}
\]
\[
= \begin{bmatrix} 8-15+7 & 5-5+0 \\ -5+5+0 & 3-10+7 \end{bmatrix} = \begin{bmatrix} 0 & 0 \\ 0 & 0 \end{bmatrix}
\]
તેથી સાબિત થાય છે કે: $A^2 - 5A + 7I = 0$
\end{solutionbox}

\questionmarks{2.2(a)}{3}{જો $A = \begin{bmatrix} -4 & -3 & -3 \\ 1 & 0 & 1 \\ 4 & 4 & 3 \end{bmatrix}$ હોય તો Adj.A શોધો.}

\begin{solutionbox}
\textbf{ઉકેલ}:
Adj.A શોધવા માટે, આપણે સહઅવયવ શ્રેણિક (cofactor matrix) શોધવો પડશે.

\textbf{સહઅવયવો (Cofactors)}:
\begin{itemize}
    \item $C_{11} = (-1)^{1+1} \begin{vmatrix} 0 & 1 \\ 4 & 3 \end{vmatrix} = -4$
    \item $C_{12} = (-1)^{1+2} \begin{vmatrix} 1 & 1 \\ 4 & 3 \end{vmatrix} = -(3-4) = 1$
    \item $C_{13} = (-1)^{1+3} \begin{vmatrix} 1 & 0 \\ 4 & 4 \end{vmatrix} = 4$
    \item $C_{21} = (-1)^{2+1} \begin{vmatrix} -3 & -3 \\ 4 & 3 \end{vmatrix} = -(-9+12) = -3$
    \item $C_{22} = (-1)^{2+2} \begin{vmatrix} -4 & -3 \\ 4 & 3 \end{vmatrix} = -12+12 = 0$
    \item $C_{23} = (-1)^{2+3} \begin{vmatrix} -4 & -3 \\ 4 & 4 \end{vmatrix} = -(-16+12) = 4$
    \item $C_{31} = (-1)^{3+1} \begin{vmatrix} -3 & -3 \\ 0 & 1 \end{vmatrix} = -3$
    \item $C_{32} = (-1)^{3+2} \begin{vmatrix} -4 & -3 \\ 1 & 1 \end{vmatrix} = -(-4+3) = 1$
    \item $C_{33} = (-1)^{3+3} \begin{vmatrix} -4 & -3 \\ 1 & 0 \end{vmatrix} = 3$
\end{itemize}
\textbf{સહઅવયવ શ્રેણિક} = $\begin{bmatrix} -4 & 1 & 4 \\ -3 & 0 & 4 \\ -3 & 1 & 3 \end{bmatrix}$

\textbf{Adj.A} = $\begin{bmatrix} -4 & -3 & -3 \\ 1 & 0 & 1 \\ 4 & 4 & 3 \end{bmatrix}$
\end{solutionbox}

\questionmarks{2.3(a)}{3}{વિકલ સમીકરણ ઉકેલો: $y(1 + x)dx + x(1 + y)dy = 0$}

\begin{solutionbox}
\textbf{ઉકેલ}:
ફરીથી ગોઠવતા: $y(1 + x)dx = -x(1 + y)dy$
\[
\frac{y(1 + x)}{x(1 + y)} = -\frac{dy}{dx}
\]
\[
\frac{y}{x} \cdot \frac{1 + x}{1 + y} = -\frac{dy}{dx}
\]
ચલ અલગ કરતા:
\[
\frac{1 + y}{y} dy = -\frac{1 + x}{x} dx
\]
\[
\left(1 + \frac{1}{y}\right) dy = -\left(1 + \frac{1}{x}\right) dx
\]
બંને બાજુ સંકલન કરતા:
\[
\int \left(1 + \frac{1}{y}\right) dy = -\int \left(1 + \frac{1}{x}\right) dx
\]
\[
y + \ln|y| = -(x + \ln|x|) + C
\]
\[
y + \ln|y| + x + \ln|x| = C
\]
\[
x + y + \ln|xy| = C
\]
\end{solutionbox}

\questionmarks{2(b)}{8}{કોઈપણ બે લખો.}

\questionmarks{2.1(b)}{4}{જો $A = \begin{bmatrix} 1 & 2 \\ -2 & 0 \end{bmatrix}$ અને $B = \begin{bmatrix} 3 & -2 \\ 2 & -4 \end{bmatrix}$ હોય તો દર્શાવો કે $(AB)^T = B^T A^T$}

\begin{solutionbox}
\textbf{ઉકેલ}:
\textbf{પગલું 1}: $AB$ ગણો
\[
AB = \begin{bmatrix} 1 & 2 \\ -2 & 0 \end{bmatrix} \begin{bmatrix} 3 & -2 \\ 2 & -4 \end{bmatrix} = \begin{bmatrix} 7 & -10 \\ -6 & 4 \end{bmatrix}
\]
\textbf{પગલું 2}: $(AB)^T$ શોધો
\[
(AB)^T = \begin{bmatrix} 7 & -6 \\ -10 & 4 \end{bmatrix}
\]
\textbf{પગલું 3}: $A^T$ અને $B^T$ ગણો
\[
A^T = \begin{bmatrix} 1 & -2 \\ 2 & 0 \end{bmatrix}, \quad B^T = \begin{bmatrix} 3 & 2 \\ -2 & -4 \end{bmatrix}
\]
\textbf{પગલું 4}: $B^T A^T$ ગણો
\[
B^T A^T = \begin{bmatrix} 3 & 2 \\ -2 & -4 \end{bmatrix} \begin{bmatrix} 1 & -2 \\ 2 & 0 \end{bmatrix} = \begin{bmatrix} 7 & -6 \\ -10 & 4 \end{bmatrix}
\]
તેથી $(AB)^T = B^T A^T$ સાબિત થાય છે.
\end{solutionbox}

\questionmarks{2.2(b)}{4}{જો $A = \begin{bmatrix} -4 & -3 \\ 4 & 2 \end{bmatrix}$ હોય તો સાબિત કરો કે $A \cdot A^{-1} = I$}

\begin{solutionbox}
\textbf{ઉકેલ}:
\textbf{પગલું 1}: $|A|$ શોધો
\[
|A| = (-4)(2) - (-3)(4) = -8 + 12 = 4
\]
\textbf{પગલું 2}: $A^{-1}$ શોધો
\[
A^{-1} = \frac{1}{|A|} \text{adj}(A) = \frac{1}{4} \begin{bmatrix} 2 & 3 \\ -4 & -4 \end{bmatrix} = \begin{bmatrix} 1/2 & 3/4 \\ -1 & -1 \end{bmatrix}
\]
\textbf{પગલું 3}: $A \cdot A^{-1}$ ગણો
\[
A \cdot A^{-1} = \begin{bmatrix} -4 & -3 \\ 4 & 2 \end{bmatrix} \begin{bmatrix} 1/2 & 3/4 \\ -1 & -1 \end{bmatrix}
\]
\[
= \begin{bmatrix} -2+3 & -3+3 \\ 2-2 & 3-2 \end{bmatrix} = \begin{bmatrix} 1 & 0 \\ 0 & 1 \end{bmatrix} = I
\]
તેથી સાબિત થાય છે કે: $A \cdot A^{-1} = I$
\end{solutionbox}

\questionmarks{2.3(b)}{4}{શ્રેણિક પદ્ધતિનો ઉપયોગ કરીને આપેલા સમીકરણો ઉકેલો: $5x + 3y = 11$ અને $3x - 2y = -1$}

\begin{solutionbox}
\textbf{ઉકેલ}:
સમીકરણોને $AX = B$ તરીકે લખી શકાય:
\[
A = \begin{bmatrix} 5 & 3 \\ 3 & -2 \end{bmatrix}, \quad X = \begin{bmatrix} x \\ y \end{bmatrix}, \quad B = \begin{bmatrix} 11 \\ -1 \end{bmatrix}
\]
\textbf{પગલું 1}: $|A|$ શોધો
\[
|A| = 5(-2) - 3(3) = -10 - 9 = -19
\]
\textbf{પગલું 2}: $A^{-1}$ શોધો
\[
A^{-1} = \frac{1}{-19} \begin{bmatrix} -2 & -3 \\ -3 & 5 \end{bmatrix} = \begin{bmatrix} 2/19 & 3/19 \\ 3/19 & -5/19 \end{bmatrix}
\]
\textbf{પગલું 3}: $X = A^{-1}B$ ઉકેલો
\[
X = \begin{bmatrix} 2/19 & 3/19 \\ 3/19 & -5/19 \end{bmatrix} \begin{bmatrix} 11 \\ -1 \end{bmatrix} = \begin{bmatrix} 22/19 - 3/19 \\ 33/19 + 5/19 \end{bmatrix} = \begin{bmatrix} 1 \\ 2 \end{bmatrix}
\]
તેથી: $x = 1, y = 2$
\end{solutionbox}

\questionmarks{3(a)}{6}{કોઈપણ બે લખો.}

\questionmarks{3.1(a)}{3}{જો $y = \log\sqrt{\frac{a+x}{a-x}}$ હોય તો $\frac{dy}{dx}$ શોધો.}

\begin{solutionbox}
\textbf{ઉકેલ}:
\[
y = \log\sqrt{\frac{a+x}{a-x}} = \frac{1}{2}\log\left(\frac{a+x}{a-x}\right)
\]
\[
y = \frac{1}{2}[\log(a+x) - \log(a-x)]
\]
$x$ સાપેક્ષ વિકલન કરતા:
\[
\frac{dy}{dx} = \frac{1}{2}\left[\frac{1}{a+x} - \frac{1}{a-x} \cdot (-1)\right]
\]
\[
= \frac{1}{2}\left[\frac{1}{a+x} + \frac{1}{a-x}\right]
\]
\[
= \frac{1}{2} \cdot \frac{(a-x) + (a+x)}{(a+x)(a-x)}
\]
\[
= \frac{1}{2} \cdot \frac{2a}{a^2-x^2} = \frac{a}{a^2-x^2}
\]
\end{solutionbox}

\questionmarks{3.2(a)}{3}{જો $y = (\sin x)^x$ હોય તો $\frac{dy}{dx}$ શોધો.}

\begin{solutionbox}
\textbf{ઉકેલ}:
લઘુગુણક લેતા:
\[
\ln y = x \ln(\sin x)
\]
$x$ સાપેક્ષ બંને બાજુ વિકલન કરતા:
\[
\frac{1}{y} \cdot \frac{dy}{dx} = \ln(\sin x) + x \cdot \frac{\cos x}{\sin x}
\]
\[
\frac{1}{y} \cdot \frac{dy}{dx} = \ln(\sin x) + x \cot x
\]
\[
\frac{dy}{dx} = y[\ln(\sin x) + x \cot x]
\]
\[
= (\sin x)^x [\ln(\sin x) + x \cot x]
\]
\end{solutionbox}

\questionmarks{3.3(a)}{3}{સાદુરૂપ આપો: $\int \frac{x^2+5x+6}{x^2+2x} dx$}

\begin{solutionbox}
\textbf{ઉકેલ}:
પ્રથમ, બહુપદી ભાગાકાર કરો:
\[
\frac{x^2+5x+6}{x^2+2x} = \frac{x^2+2x+3x+6}{x^2+2x} = 1 + \frac{3x+6}{x^2+2x}
\]
\[
= 1 + \frac{3x+6}{x(x+2)} = 1 + \frac{3(x+2)}{x(x+2)} = 1 + \frac{3}{x}
\]
તેથી:
\[
\int \frac{x^2+5x+6}{x^2+2x} dx = \int \left(1 + \frac{3}{x}\right) dx = x + 3\ln|x| + C
\]
\end{solutionbox}

\questionmarks{3(b)}{8}{કોઈપણ બે લખો.}

\questionmarks{3.1(b)}{4}{જો $x = e^\theta(\cos\theta + \sin\theta)$ અને $y = e^\theta(\cos\theta - \sin\theta)$ હોય તો $\frac{dy}{dx}$ શોધો.}

\begin{solutionbox}
\textbf{ઉકેલ}:
\textbf{રીત}: પ્રચલ વિકલન (parametric differentiation) $\frac{dy}{dx} = \frac{dy/d\theta}{dx/d\theta}$ નો ઉપયોગ કરો

$\frac{dx}{d\theta}$ શોધો:
\[
\frac{dx}{d\theta} = \frac{d}{d\theta}[e^\theta(\cos\theta + \sin\theta)]
\]
\[
= e^\theta(\cos\theta + \sin\theta) + e^\theta(-\sin\theta + \cos\theta)
\]
\[
= e^\theta[(\cos\theta + \sin\theta) + (\cos\theta - \sin\theta)]
\]
\[
= e^\theta \cdot 2\cos\theta = 2e^\theta\cos\theta
\]
$\frac{dy}{d\theta}$ શોધો:
\[
\frac{dy}{d\theta} = \frac{d}{d\theta}[e^\theta(\cos\theta - \sin\theta)]
\]
\[
= e^\theta(\cos\theta - \sin\theta) + e^\theta(-\sin\theta - \cos\theta)
\]
\[
= e^\theta[(\cos\theta - \sin\theta) - (\sin\theta + \cos\theta)]
\]
\[
= e^\theta(-2\sin\theta) = -2e^\theta\sin\theta
\]
તેથી:
\[
\frac{dy}{dx} = \frac{dy/d\theta}{dx/d\theta} = \frac{-2e^\theta\sin\theta}{2e^\theta\cos\theta} = -\tan\theta
\]
\end{solutionbox}

\questionmarks{3.2(b)}{4}{જો $y = \log(\sin x)$ હોય તો દર્શાવો કે: $\frac{d^2y}{dx^2} + \left(\frac{dy}{dx}\right)^2 + 1 = 0$}

\begin{solutionbox}
\textbf{ઉકેલ}:
પ્રથમ વિકલિત શોધો:
\[
\frac{dy}{dx} = \frac{1}{\sin x} \cdot \cos x = \cot x
\]
દ્વિતીય વિકલિત શોધો:
\[
\frac{d^2y}{dx^2} = \frac{d}{dx}(\cot x) = -\csc^2 x
\]
હવે આપેલ પદાવલિમાં મુકો:
\[
\frac{d^2y}{dx^2} + \left(\frac{dy}{dx}\right)^2 + 1
\]
\[
= -\csc^2 x + \cot^2 x + 1
\]
\[
= -\csc^2 x + \cot^2 x + 1
\]
નિત્યસમ $\csc^2 x = 1 + \cot^2 x$ નો ઉપયોગ કરતા:
\[
= -(1 + \cot^2 x) + \cot^2 x + 1
\]
\[
= -1 - \cot^2 x + \cot^2 x + 1 = 0
\]
જે સાબિત થાય છે.
\end{solutionbox}

\questionmarks{3.3(b)}{4}{જ્યારે ગતિ કરતા કણોનું સમીકરણ $S = t^3 - 6t^2 + 9t + 4$ હોય, તો આપેલા પ્રશ્નો ઉકેલો: (1) જ્યારે $a = 0$ હોય, ત્યારે 'v' અને 's' શોધો (2) જ્યારે $v = 0$ હોય ત્યારે 'a' અને 's' શોધો}

\begin{solutionbox}
\textbf{ઉકેલ}:
આપેલ: $S = t^3 - 6t^2 + 9t + 4$

વેગ (Velocity): $v = \frac{dS}{dt} = 3t^2 - 12t + 9$

પ્રવેગ (Acceleration): $a = \frac{dv}{dt} = 6t - 12$

\textbf{(1) જ્યારે $a = 0$ હોય:}
\[
6t - 12 = 0 \Rightarrow t = 2
\]
$t = 2$ પર:
\begin{itemize}
    \item $v = 3(4) - 12(2) + 9 = 12 - 24 + 9 = -3$
    \item $s = (2)^3 - 6(2)^2 + 9(2) + 4 = 8 - 24 + 18 + 4 = 6$
\end{itemize}

\textbf{(2) જ્યારે $v = 0$ હોય:}
\[
3t^2 - 12t + 9 = 0
\]
\[
t^2 - 4t + 3 = 0
\]
\[
(t - 1)(t - 3) = 0
\]
\[
t = 1 \text{ અથવા } t = 3
\]
$t = 1$ પર:
\begin{itemize}
    \item $a = 6(1) - 12 = -6$
    \item $s = 1 - 6 + 9 + 4 = 8$
\end{itemize}
$t = 3$ પર:
\begin{itemize}
    \item $a = 6(3) - 12 = 6$
    \item $s = 27 - 54 + 27 + 4 = 4$
\end{itemize}
\end{solutionbox}

\questionmarks{4(a)}{6}{કોઈપણ બે લખો.}

\questionmarks{4.1(a)}{3}{$\int \frac{(1-3x)^2}{x^3} dx$ : કિંમત શોધો}

\begin{solutionbox}
\textbf{ઉકેલ}:
અંશનું વિસ્તરણ કરો:
\[
(1-3x)^2 = 1 - 6x + 9x^2
\]
\[
\int \frac{(1-3x)^2}{x^3} dx = \int \frac{1 - 6x + 9x^2}{x^3} dx
\]
\[
= \int \left(\frac{1}{x^3} - \frac{6x}{x^3} + \frac{9x^2}{x^3}\right) dx
\]
\[
= \int \left(x^{-3} - 6x^{-2} + 9x^{-1}\right) dx
\]
\[
= \frac{x^{-2}}{-2} - 6 \cdot \frac{x^{-1}}{-1} + 9\ln|x| + C
\]
\[
= -\frac{1}{2x^2} + \frac{6}{x} + 9\ln|x| + C
\]
\end{solutionbox}

\questionmarks{4.2(a)}{3}{$\int x \cdot e^{3x} dx$ : કિંમત શોધો}

\begin{solutionbox}
\textbf{ઉકેલ}:
ખંડશઃ સંકલન (integration by parts) નો ઉપયોગ કરતા: $\int u \, dv = uv - \int v \, du$

ધારો કે $u = x$ અને $dv = e^{3x} dx$

તો $du = dx$ અને $v = \frac{e^{3x}}{3}$
\[
\int x \cdot e^{3x} dx = x \cdot \frac{e^{3x}}{3} - \int \frac{e^{3x}}{3} dx
\]
\[
= \frac{xe^{3x}}{3} - \frac{1}{3} \cdot \frac{e^{3x}}{3} + C
\]
\[
= \frac{xe^{3x}}{3} - \frac{e^{3x}}{9} + C
\]
\[
= \frac{e^{3x}}{9}(3x - 1) + C
\]
\end{solutionbox}

\questionmarks{4.3(a)}{3}{સંકર સંખ્યા $\sqrt{3} - i$ નું વર્ગમૂળ શોધો}

\begin{solutionbox}
\textbf{ઉકેલ}:
ધારો કે $z = \sqrt{3} - i$

પ્રથમ, ધ્રુવીય સ્વરૂપમાં ફેરવો:
\begin{itemize}
    \item $|z| = \sqrt{(\sqrt{3})^2 + (-1)^2} = \sqrt{3 + 1} = 2$
    \item $\arg(z) = \arctan\left(\frac{-1}{\sqrt{3}}\right) = -\frac{\pi}{6}$ (4th ચરણ)
\end{itemize}
તેથી $z = 2e^{-i\pi/6} = 2(\cos(-\pi/6) + i\sin(-\pi/6))$

વર્ગમૂળ માટે, આપણે ઉપયોગ કરીએ છીએ:
\[
\sqrt{z} = \sqrt{|z|} \cdot e^{i\arg(z)/2}
\]
\[
\sqrt{z} = \sqrt{2} \cdot e^{-i\pi/12}
\]
\[
= \sqrt{2}\left(\cos\left(-\frac{\pi}{12}\right) + i\sin\left(-\frac{\pi}{12}\right)\right)
\]
બીજું વર્ગમૂળ નીચે મુજબ છે:
\[
\sqrt{z} = \sqrt{2} \cdot e^{i(\pi - \pi/12)} = \sqrt{2} \cdot e^{i11\pi/12}
\]
બે વર્ગમૂળો છે:
\[
\sqrt{2}e^{-i\pi/12} \text{ અને } \sqrt{2}e^{i11\pi/12}
\]
\end{solutionbox}

\questionmarks{4(b)}{8}{કોઈપણ બે લખો.}

\questionmarks{4.1(b)}{4}{કિંમત શોધો: $\int_0^{\pi/2} \frac{\sin x}{\cos x + \sin x} dx$}

\begin{solutionbox}
\textbf{ઉકેલ}:
ધારો કે $I = \int_0^{\pi/2} \frac{\sin x}{\cos x + \sin x} dx$

ગુણધર્મનો ઉપયોગ કરતા: $\int_0^a f(x) dx = \int_0^a f(a-x) dx$
\[
I = \int_0^{\pi/2} \frac{\sin(\pi/2 - x)}{\cos(\pi/2 - x) + \sin(\pi/2 - x)} dx
\]
\[
= \int_0^{\pi/2} \frac{\cos x}{\sin x + \cos x} dx
\]
બંને સમીકરણો ઉમેરતા:
\[
I + I = \int_0^{\pi/2} \frac{\sin x}{\cos x + \sin x} dx + \int_0^{\pi/2} \frac{\cos x}{\sin x + \cos x} dx
\]
\[
2I = \int_0^{\pi/2} \frac{\sin x + \cos x}{\cos x + \sin x} dx = \int_0^{\pi/2} 1 \, dx = \frac{\pi}{2}
\]
તેથી: $I = \frac{\pi}{4}$
\end{solutionbox}

\questionmarks{4.2(b)}{4}{વર્તુળ $x^2 + y^2 = a^2$ ના ક્ષેત્રફળનું સમીકરણ શોધો.}

\begin{solutionbox}
\textbf{ઉકેલ}:
$a$ ત્રિજ્યા વાળા વર્તુળનું ક્ષેત્રફળ સંકલન નો ઉપયોગ કરીને શોધી શકાય છે.

$x^2 + y^2 = a^2$ પરથી, $y = \pm\sqrt{a^2 - x^2}$

ક્ષેત્રફળ છે:
\[
A = \int_{-a}^{a} 2\sqrt{a^2 - x^2} \, dx
\]
આદેશ $x = a\sin\theta$, $dx = a\cos\theta \, d\theta$ લેતા

જ્યારે $x = -a$, $\theta = -\pi/2$; જ્યારે $x = a$, $\theta = \pi/2$
\[
A = \int_{-\pi/2}^{\pi/2} 2\sqrt{a^2 - a^2\sin^2\theta} \cdot a\cos\theta \, d\theta
\]
\[
= \int_{-\pi/2}^{\pi/2} 2a\cos\theta \cdot a\cos\theta \, d\theta
\]
\[
= 2a^2 \int_{-\pi/2}^{\pi/2} \cos^2\theta \, d\theta
\]
$\cos^2\theta = \frac{1 + \cos(2\theta)}{2}$ નો ઉપયોગ કરતા:
\[
A = 2a^2 \int_{-\pi/2}^{\pi/2} \frac{1 + \cos(2\theta)}{2} d\theta
\]
\[
= a^2 \int_{-\pi/2}^{\pi/2} (1 + \cos(2\theta)) d\theta
\]
\[
= a^2 \left[\theta + \frac{\sin(2\theta)}{2}\right]_{-\pi/2}^{\pi/2}
\]
\[
= a^2 \left[\frac{\pi}{2} + 0 - \left(-\frac{\pi}{2} + 0\right)\right] = a^2 \cdot \pi
\]
તેથી, વર્તુળનું ક્ષેત્રફળ $A = \pi a^2$ છે.
\end{solutionbox}

\questionmarks{4.3(b)}{4}{જો $z_1 = 3 + 4i$ અને $z_2 = 2 - i$ હોય તો $z_1 + z_2$, $z_1 - z_2$, $z_1 \times z_2$ અને $z_1 \div z_2$ શોધો.}

\begin{solutionbox}
\textbf{ઉકેલ}:
આપેલ: $z_1 = 3 + 4i$ અને $z_2 = 2 - i$

\textbf{(1) સરવાળો:}
\[
z_1 + z_2 = (3 + 4i) + (2 - i) = 5 + 3i
\]
\textbf{(2) બાદબાકી:}
\[
z_1 - z_2 = (3 + 4i) - (2 - i) = 1 + 5i
\]
\textbf{(3) ગુણાકાર:}
\[
z_1 \times z_2 = (3 + 4i)(2 - i)
\]
\[
= 3(2) + 3(-i) + 4i(2) + 4i(-i)
\]
\[
= 6 - 3i + 8i - 4i^2
\]
\[
= 6 + 5i - 4(-1) = 6 + 5i + 4 = 10 + 5i
\]
\textbf{(4) ભાગાકાર:}
\[
z_1 \div z_2 = \frac{3 + 4i}{2 - i}
\]
અંશ અને છેદને છેદના અનુબદ્ધ વડે ગુણતા:
\[
= \frac{(3 + 4i)(2 + i)}{(2 - i)(2 + i)}
\]
\[
= \frac{6 + 3i + 8i + 4i^2}{4 - i^2}
\]
\[
= \frac{6 + 11i - 4}{4 + 1} = \frac{2 + 11i}{5} = \frac{2}{5} + \frac{11}{5}i
\]
\end{solutionbox}

\questionmarks{5(a)}{6}{કોઈપણ બે લખો.}

\questionmarks{5.1(a)}{3}{સંકર સંખ્યા $(2 - 3i)(-2 + i)$ નો માનાંક અને અનુબદ્ધ સ્વરૂપ શોધો.}

\begin{solutionbox}
\textbf{ઉકેલ}:
પ્રથમ, સંકર સંખ્યાઓનો ગુણાકાર કરો:
\[
(2 - 3i)(-2 + i) = 2(-2) + 2(i) - 3i(-2) - 3i(i)
\]
\[
= -4 + 2i + 6i - 3i^2
\]
\[
= -4 + 8i - 3(-1) = -4 + 8i + 3 = -1 + 8i
\]
ધારો કે $z = -1 + 8i$

\textbf{માનાંક (Modulus):}
\[
|z| = \sqrt{(-1)^2 + 8^2} = \sqrt{1 + 64} = \sqrt{65}
\]
\textbf{અનુબદ્ધ (Conjugate):}
\[
\overline{z} = -1 - 8i
\]
\end{solutionbox}

\questionmarks{5.2(a)}{3}{સંકર સંખ્યા $\frac{1+i}{1-i}$ નો મુખ્ય કોણાંક (Principal Argument) શોધો.}

\begin{solutionbox}
\textbf{ઉકેલ}:
પ્રથમ, સંકર સંખ્યાનું સાદુરૂપ આપો:
\[
\frac{1+i}{1-i} = \frac{(1+i)(1+i)}{(1-i)(1+i)} = \frac{(1+i)^2}{1-i^2}
\]
\[
= \frac{1 + 2i + i^2}{1 - (-1)} = \frac{1 + 2i - 1}{2} = \frac{2i}{2} = i
\]
$z = i = 0 + 1i$ માટે:
\begin{itemize}
    \item વાસ્તવિક ભાગ = 0
    \item કાલ્પનિક ભાગ = 1 > 0
\end{itemize}
સંકર સંખ્યા $i$ ધન કાલ્પનિક અક્ષ પર છે.

\textbf{મુખ્ય કોણાંક} = $\frac{\pi}{2}$
\end{solutionbox}

\questionmarks{5.3(a)}{3}{દર્શાવો કે: $\frac{(\cos 2\theta + i\sin 2\theta)^3 (\cos 3\theta - i\sin 3\theta)^2}{(\cos 4\theta + i\sin 4\theta)^5 (\cos 5\theta - i\sin 4\theta)^5} = 1$}

\begin{solutionbox}
\textbf{ઉકેલ}:
દ-મદ મેવરેના પ્રમેય (De Moivre's theorem) નો ઉપયોગ કરતા: $(\cos\theta + i\sin\theta)^n = \cos(n\theta) + i\sin(n\theta)$

\textbf{અંશ:}
\[
(\cos 2\theta + i\sin 2\theta)^3 = \cos(6\theta) + i\sin(6\theta)
\]
\[
(\cos 3\theta - i\sin 3\theta)^2 = (\cos(-3\theta) + i\sin(-3\theta))^2 = \cos(-6\theta) + i\sin(-6\theta)
\]
અંશ = $[\cos(6\theta) + i\sin(6\theta)][\cos(-6\theta) + i\sin(-6\theta)]$

$(a + bi)(c + di) = (ac - bd) + (ad + bc)i$ અને $\cos(-\theta) = \cos\theta$, $\sin(-\theta) = -\sin\theta$ નો ઉપયોગ કરતા:
\[
= \cos(6\theta)\cos(6\theta) - \sin(6\theta)(-\sin(6\theta)) + i[\cos(6\theta)(-\sin(6\theta)) + \sin(6\theta)\cos(6\theta)]
\]
\[
= \cos^2(6\theta) + \sin^2(6\theta) + i[0] = 1
\]
\textbf{છેદ:}
\[
(\cos 4\theta + i\sin 4\theta)^5 = \cos(20\theta) + i\sin(20\theta)
\]
નોંધ: પ્રશ્નમાં ભૂલ જણાય છે. ધારો કે તે $(\cos 5\theta - i\sin 5\theta)^5$ હોવું જોઈએ:
\[
(\cos 5\theta - i\sin 5\theta)^5 = \cos(-25\theta) + i\sin(-25\theta)
\]
પદાવલિ 1 થવા માટે, અંશ અને છેદ સમાન હોવા જરૂરી છે.
\end{solutionbox}

\questionmarks{5(b)}{8}{કોઈપણ બે લખો.}

\questionmarks{5.1(b)}{4}{વિકલ સમીકરણ ઉકેલો: $\frac{dy}{dx} = \frac{y}{x} + x\sin\left(\frac{y}{x}\right)$}

\begin{solutionbox}
\textbf{ઉકેલ}:
આ સમપરિમાણ વિકલ સમીકરણ છે. ધારો કે $v = \frac{y}{x}$, તેથી $y = vx$ અને $\frac{dy}{dx} = v + x\frac{dv}{dx}$

કિંમત મુકતા:
\[
v + x\frac{dv}{dx} = v + x\sin v
\]
\[
x\frac{dv}{dx} = x\sin v
\]
\[
\frac{dv}{dx} = \sin v
\]
ચલ અલગ કરતા:
\[
\frac{dv}{\sin v} = \frac{dx}{x}
\]
\[
\csc v \, dv = \frac{dx}{x}
\]
બંને બાજુ સંકલન કરતા:
\[
\int \csc v \, dv = \int \frac{dx}{x}
\]
\[
-\ln|\csc v + \cot v| = \ln|x| + C
\]
\[
\ln|\csc v + \cot v| = -\ln|x| + C_1
\]
\[
\csc v + \cot v = \frac{A}{x} \quad (\text{જ્યાં } A = e^{C_1})
\]
$v = \frac{y}{x}$ પાછું મુકતા:
\[
\csc\left(\frac{y}{x}\right) + \cot\left(\frac{y}{x}\right) = \frac{A}{x}
\]
\end{solutionbox}

\questionmarks{5.2(b)}{4}{વિકલ સમીકરણ ઉકેલો: $\frac{dy}{dx} = \frac{y}{x} + x^2$}

\begin{solutionbox}
\textbf{ઉકેલ}:
આ સુરેખ વિકલ સમીકરણ છે. પ્રમાણિત સ્વરૂપમાં લખતા:
\[
\frac{dy}{dx} - \frac{y}{x} = x^2
\]
અહીં, $P(x) = -\frac{1}{x}$ અને $Q(x) = x^2$

\textbf{સંકલ્પકારક અવયવ (Integrating factor):}
\[
\mu(x) = e^{\int P(x)dx} = e^{\int -\frac{1}{x}dx} = e^{-\ln|x|} = \frac{1}{x}
\]
સમીકરણને સંકલ્પકારક અવયવ વડે ગુણતા:
\[
\frac{1}{x}\frac{dy}{dx} - \frac{1}{x} \cdot \frac{y}{x} = \frac{1}{x} \cdot x^2
\]
\[
\frac{1}{x}\frac{dy}{dx} - \frac{y}{x^2} = x
\]
ડાબી બાજુ $\frac{y}{x}$ નું વિકલિત છે:
\[
\frac{d}{dx}\left(\frac{y}{x}\right) = x
\]
બંને બાજુ સંકલન કરતા:
\[
\frac{y}{x} = \int x \, dx = \frac{x^2}{2} + C
\]
તેથી:
\[
y = x\left(\frac{x^2}{2} + C\right) = \frac{x^3}{2} + Cx
\]
\end{solutionbox}

\questionmarks{5.3(b)}{4}{વિકલ સમીકરણ ઉકેલો: $(e^y + 1)\cos x \, dx + e^y \sin x \, dy = 0$}

\begin{solutionbox}
\textbf{ઉકેલ}:
ફરીથી ગોઠવતા:
\[
(e^y + 1)\cos x \, dx = -e^y \sin x \, dy
\]
ચલ અલગ કરતા:
\[
\frac{\cos x}{\sin x} dx = -\frac{e^y}{e^y + 1} dy
\]
\[
\cot x \, dx = -\frac{e^y}{e^y + 1} dy
\]
બંને બાજુ સંકલન કરતા:
\[
\int \cot x \, dx = -\int \frac{e^y}{e^y + 1} dy
\]
ડાબી બાજુ માટે:
\[
\int \cot x \, dx = \int \frac{\cos x}{\sin x} dx = \ln|\sin x| + C_1
\]
જમણી બાજુ માટે, ધારો કે $u = e^y + 1$, તો $du = e^y dy$:
\[
-\int \frac{e^y}{e^y + 1} dy = -\int \frac{1}{u} du = -\ln|u| + C_2 = -\ln|e^y + 1| + C_2
\]
ભેગું કરતા:
\[
\ln|\sin x| = -\ln|e^y + 1| + C
\]
\[
\ln|\sin x| + \ln|e^y + 1| = C
\]
\[
\ln|\sin x(e^y + 1)| = C
\]
\[
\sin x(e^y + 1) = A \quad (\text{જ્યાં } A = e^C)
\]
આ વિકલ સમીકરણનો સામાન્ય ઉકેલ છે.
\end{solutionbox}

\end{document}

