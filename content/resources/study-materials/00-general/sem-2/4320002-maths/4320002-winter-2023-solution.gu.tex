\documentclass{article}

% content/resources/templates/preamble.tex
\usepackage[margin=0.6in]{geometry}
\author{Milav Dabgar}
\usepackage{amsmath,amssymb,amsthm}
\usepackage{booktabs}
\usepackage{multirow}
\usepackage{xcolor}
\usepackage{tcolorbox}
\tcbuselibrary{breakable,skins}
\usepackage[colorlinks=true,linkcolor=blue]{hyperref}
\usepackage{titlesec}
\usepackage{enumitem}
\usepackage{tikz}
\usepackage{pgfplots}
\usepackage{circuitikz}
\usepackage[version=4]{mhchem}
\usepackage{longtable}
\usepackage{array}
\usepackage{float}
\usepackage{caption}
\usepackage{listings}

\lstset{
  basicstyle=\small\ttfamily,
  breaklines=true,
  breakatwhitespace=false,
  postbreak=\mbox{\textcolor{red}{$\hookrightarrow$}\space},
  float=false,
  numbers=left,
  numberstyle=\tiny\color{gray},
  numbersep=10pt,
  xleftmargin=2em,
  keywordstyle=\color{blue},
  commentstyle=\color{green!60!black},
  stringstyle=\color{purple},
  backgroundcolor=\color{gray!5},
  showstringspaces=false,
  tabsize=2,
  captionpos=b,
  keepspaces=true,
  columns=flexible
}

\pgfplotsset{compat=1.18}
\usetikzlibrary{shapes,arrows,positioning,calc,patterns,decorations.pathmorphing,decorations.markings,arrows.meta}

% Color scheme
\definecolor{headcolor}{RGB}{0,102,204}
\definecolor{keycolor}{RGB}{220,20,60}
\definecolor{solutioncolor}{RGB}{34,139,34}
\definecolor{mnemoniccolor}{RGB}{148,0,211}
\definecolor{codecolor}{RGB}{0,0,100}

% Spacing
\setlength{\parskip}{3pt}
\setlist[itemize]{nosep}
\setlist[enumerate]{nosep}

% Title formatting
\titleformat{\section}{\Large\bfseries\color{headcolor}}{\thesection}{1em}{}
\titleformat{\subsection}{\large\bfseries\color{headcolor}}{\thesubsection}{1em}{}

% Pandoc tightlist compatibility
\providecommand{\tightlist}{%
  \setlength{\itemsep}{0pt}\setlength{\parskip}{0pt}}

% Pandoc longtable compatibility
\newcounter{none}
\def\thenone{}


% content/resources/templates/gujarati-boxes.tex
\usepackage{fontspec}
\usepackage{polyglossia}

% Set Gujarati as main language (document is primarily in Gujarati)
% Note: gloss-gujarati.ldf doesn't exist in polyglossia, but it will use hyphenation patterns
\setdefaultlanguage{gujarati}
\setotherlanguage{english}

% Configure Gujarati font properly
% Use Language=Default to prevent polyglossia from trying to add language-specific features
% that don't exist for Gujarati, which causes "empty feature" warnings
\newfontfamily\gujaratifont[Script=Gujarati,AutoFakeBold=2.5,AutoFakeSlant=0.3]{Noto Sans Gujarati}
\setmainfont[Script=Gujarati,AutoFakeBold=2.5,AutoFakeSlant=0.3]{Noto Sans Gujarati}
% Use Noto Sans Gujarati for monospace to support Gujarati in text
\setmonofont[Scale=0.9]{Noto Sans Gujarati}

% Configure English to use the same font
\newfontfamily\englishfont[Script=Gujarati,AutoFakeBold=2.5,AutoFakeSlant=0.3]{Noto Sans Gujarati}

% Translations for polyglossia
\gappto\captionsgujarati{
  \renewcommand{\tablename}{કોષ્ટક}
  \renewcommand{\figurename}{આકૃતિ}
}

% Helper for TikZ nodes to ensure Gujarati font
\newcommand{\gu}[1]{{\gujaratifont #1}}

% Custom environments
\newtcolorbox{solutionbox}{
    breakable,
    enhanced,
    colback=solutioncolor!5!white,
    colframe=solutioncolor!75!black,
    fonttitle=\bfseries,
    title=જવાબ
}

\newtcolorbox{solutionboxnobreak}{
 colback=solutioncolor!5!white,
 colframe=solutioncolor!75!black,
 fonttitle=\bfseries,
 title=જવાબ
}

\newtcolorbox{keyformula}{
 breakable,
 enhanced,
 colback=keycolor!5!white,
 colframe=keycolor!75!black,
 fonttitle=\bfseries,
 title=રાસાયણિક સમીકરણ/સૂત્ર
}

\newtcolorbox{mnemonicbox}{
 breakable,
 enhanced,
 colback=mnemoniccolor!5!white,
 colframe=mnemoniccolor!75!black,
 fonttitle=\bfseries,
 title=મેમરી ટ્રીક
}


% Custom commands for GTU solutions
% This file defines semantic commands for consistent formatting

% Question command with automatic formatting
\newcommand{\question}[2]{%
  \section*{Question #1}%
  \textbf{#2}%
}

% OR question variant
\newcommand{\questionor}[2]{%
  \section*{Question #1 OR}%
  \textbf{#2}%
}

% Proper table environment with caption
\newenvironment{answertable}[1]{%
  \begin{table}[htbp]
  \centering
  \caption{#1}
}{%
  \end{table}
}

% Proper figure environment for diagrams
\newenvironment{answerdiagram}[1]{%
  \begin{figure}[htbp]
  \centering
  \caption{#1}
}{%
  \end{figure}
}

% Semantic markup for key terms
\newcommand{\keyword}[1]{\textbf{#1}}
\newcommand{\code}[1]{\texttt{#1}}
\newcommand{\classname}[1]{\texttt{#1}}
\newcommand{\methodname}[1]{\texttt{#1}}

% Proper quotation marks
\newcommand{\mnemonic}[1]{``#1''}


\title{એન્જિનિયરિંગ મેથેમેટિક્સ (4320002) - વિન્ટર 2023 સોલ્યુશન}
\date{જાન્યુઆરી 31, 2024}

\begin{document}
\maketitle

\questionmarks{1}{14}{નીચે આપેલા વિકલ્પોમાંથી યોગ્ય વિકલ્પ પસંદ કરી ખાલી જગ્યા પૂરો.}

\questionmarks{1.1}{1}{શ્રેણિક $\begin{bmatrix} 2 & 5 \\ 7 & 8 \end{bmatrix}$ ની કક્ષા (Order) \_\_\_\_\_\_\_\_\_ છે.}

\begin{solutionbox}
\textbf{જવાબ}: (d) $2 \times 2$

\textbf{ઉકેલ}:
શ્રેણિકમાં 2 હાર અને 2 સ્તંભ છે, તેથી તેની કક્ષા $2 \times 2$ છે.
\end{solutionbox}

\questionmarks{1.2}{1}{$\begin{bmatrix} 4 & 3 \\ 6 & 2 \end{bmatrix} + \begin{bmatrix} 1 & 5 \\ 5 & 8 \end{bmatrix} = $ \_\_\_\_\_\_\_\_\_}

\begin{solutionbox}
\textbf{જવાબ}: (a) $\begin{bmatrix} 5 & 8 \\ 11 & 10 \end{bmatrix}$

\textbf{ઉકેલ}:
\[
\begin{bmatrix} 4 & 3 \\ 6 & 2 \end{bmatrix} + \begin{bmatrix} 1 & 5 \\ 5 & 8 \end{bmatrix} = \begin{bmatrix} 4+1 & 3+5 \\ 6+5 & 2+8 \end{bmatrix} = \begin{bmatrix} 5 & 8 \\ 11 & 10 \end{bmatrix}
\]
\end{solutionbox}

\questionmarks{1.3}{1}{નીચેનામાંથી કયો ચોરસ શ્રેણિક છે?}

\begin{solutionbox}
\textbf{જવાબ}: (c) $\begin{bmatrix} 1 & 3 \\ 5 & 4 \end{bmatrix}$

\textbf{ઉકેલ}:
ચોરસ શ્રેણિકમાં હાર અને સ્તંભની સંખ્યા સમાન હોય છે. માત્ર વિકલ્પ (c) માં $2 \times 2$ પરિમાણ છે.
\end{solutionbox}

\questionmarks{1.4}{1}{જો $A = [3]$ અને $B = [4]$ હોય તો $A \cdot B = $ \_\_\_\_\_\_\_\_\_}

\begin{solutionbox}
\textbf{જવાબ}: (b) 12

\textbf{ઉકેલ}:
\[
A \cdot B = [3] \times [4] = [3 \times 4] = [12] = 12
\]
\end{solutionbox}

\questionmarks{1.5}{1}{$\frac{d}{dx}\sin x = $ \_\_\_\_\_\_\_\_\_}

\begin{solutionbox}
\textbf{જવાબ}: (d) $\cos x$

\textbf{ઉકેલ}:
$\sin x$ નું વિકલન $\cos x$ છે.
\end{solutionbox}

\questionmarks{1.6}{1}{જો $f(x) = xe^x$ હોય તો $f'(0) = $ \_\_\_\_\_\_\_\_\_}

\begin{solutionbox}
\textbf{જવાબ}: (b) 1

\textbf{ઉકેલ}:
ગુણાકારના નિયમનો ઉપયોગ કરતા: $f'(x) = \frac{d}{dx}(xe^x) = e^x + xe^x = e^x(1 + x)$
\[
f'(0) = e^0(1 + 0) = 1 \times 1 = 1
\]
\end{solutionbox}

\questionmarks{1.7}{1}{જો $y = x^2$ હોય તો $\frac{d^2y}{dx^2} = $ \_\_\_\_\_\_\_\_\_}

\begin{solutionbox}
\textbf{જવાબ}: (b) 2

\textbf{ઉકેલ}:
\[
y = x^2
\]
\[
\frac{dy}{dx} = 2x
\]
\[
\frac{d^2y}{dx^2} = 2
\]
\end{solutionbox}

\questionmarks{1.8}{1}{$\int \cos x dx = $ \_\_\_\_\_\_\_\_\_ $+ c$}

\begin{solutionbox}
\textbf{જવાબ}: (a) $\sin x$

\textbf{ઉકેલ}:
\[
\int \cos x dx = \sin x + c
\]
\end{solutionbox}

\questionmarks{1.9}{1}{$\int_0^1 x dx = $ \_\_\_\_\_\_\_\_\_}

\begin{solutionbox}
\textbf{જવાબ}: (c) $\frac{1}{2}$

\textbf{ઉકેલ}:
\[
\int_0^1 x dx = \left[\frac{x^2}{2}\right]_0^1 = \frac{1^2}{2} - \frac{0^2}{2} = \frac{1}{2}
\]
\end{solutionbox}

\questionmarks{1.10}{1}{$\int \frac{1}{1+x^2} dx = $ \_\_\_\_\_\_\_\_\_ $+ c$}

\begin{solutionbox}
\textbf{જવાબ}: (a) $\tan^{-1} x$

\textbf{ઉકેલ}:
\[
\int \frac{1}{1+x^2} dx = \tan^{-1} x + c
\]
\end{solutionbox}

\questionmarks{1.11}{1}{વિકલ સમીકરણ $x\sin y + xy = x$ ની કક્ષા (Order) \_\_\_\_\_\_\_\_\_ છે}

\begin{solutionbox}
\textbf{જવાબ}: (b) 1

\textbf{ઉકેલ}:
સમીકરણને $\frac{dy}{dx} = \frac{1-xy}{\sin y}$ તરીકે લખી શકાય. સૌથી મોટું વિકલિત પ્રથમ કક્ષાનું છે.
\end{solutionbox}

\questionmarks{1.12}{1}{$\frac{dy}{dx} + y = x$ નો સંકલ્પકારક અવયવ (I.F.) \_\_\_\_\_\_\_\_\_ છે}

\begin{solutionbox}
\textbf{જવાબ}: (d) $e^x$

\textbf{ઉકેલ}:
$\frac{dy}{dx} + Py = Q$ માટે, સંકલ્પકારક અવયવ $= e^{\int P dx} = e^{\int 1 dx} = e^x$
\end{solutionbox}

\questionmarks{1.13}{1}{$i^2 = $ \_\_\_\_\_\_\_\_\_}

\begin{solutionbox}
\textbf{જવાબ}: (b) -1

\textbf{ઉકેલ}:
વ્યાખ્યા મુજબ, $i^2 = -1$
\end{solutionbox}

\questionmarks{1.14}{1}{$(2+3i)(2-3i) = $ \_\_\_\_\_\_\_\_\_}

\begin{solutionbox}
\textbf{જવાબ}: (c) 13

\textbf{ઉકેલ}:
\[
(2+3i)(2-3i) = 2^2 - (3i)^2 = 4 - 9i^2 = 4 - 9(-1) = 4 + 9 = 13
\]
\end{solutionbox}

\questionmarks{2(A)}{6}{કોઈપણ બે લખો.}

\questionmarks{2(A).1}{3}{જો $A = \begin{bmatrix} 2 & 5 \\ -1 & 3 \end{bmatrix}$, $B = \begin{bmatrix} 5 & 8 \\ 4 & 6 \end{bmatrix}$ અને $C = \begin{bmatrix} 4 & 2 \\ 1 & 5 \end{bmatrix}$ હોય તો $2A + 3B - C$ શોધો.}

\begin{solutionbox}
\textbf{ઉકેલ}:
\[
2A = 2\begin{bmatrix} 2 & 5 \\ -1 & 3 \end{bmatrix} = \begin{bmatrix} 4 & 10 \\ -2 & 6 \end{bmatrix}
\]
\[
3B = 3\begin{bmatrix} 5 & 8 \\ 4 & 6 \end{bmatrix} = \begin{bmatrix} 15 & 24 \\ 12 & 18 \end{bmatrix}
\]
\[
2A + 3B = \begin{bmatrix} 4 & 10 \\ -2 & 6 \end{bmatrix} + \begin{bmatrix} 15 & 24 \\ 12 & 18 \end{bmatrix} = \begin{bmatrix} 19 & 34 \\ 10 & 24 \end{bmatrix}
\]
\[
2A + 3B - C = \begin{bmatrix} 19 & 34 \\ 10 & 24 \end{bmatrix} - \begin{bmatrix} 4 & 2 \\ 1 & 5 \end{bmatrix} = \begin{bmatrix} 15 & 32 \\ 9 & 19 \end{bmatrix}
\]
\end{solutionbox}

\questionmarks{2(A).2}{3}{જો $M = \begin{bmatrix} 1 & 4 \\ 3 & 7 \end{bmatrix}$ અને $N = \begin{bmatrix} 6 & 9 \\ 0 & 5 \end{bmatrix}$ હોય તો સાબિત કરો કે $(M+N)^T = M^T + N^T$}

\begin{solutionbox}
\textbf{ઉકેલ}:
\[
M + N = \begin{bmatrix} 1 & 4 \\ 3 & 7 \end{bmatrix} + \begin{bmatrix} 6 & 9 \\ 0 & 5 \end{bmatrix} = \begin{bmatrix} 7 & 13 \\ 3 & 12 \end{bmatrix}
\]
\[
(M+N)^T = \begin{bmatrix} 7 & 3 \\ 13 & 12 \end{bmatrix}
\]
\[
M^T = \begin{bmatrix} 1 & 3 \\ 4 & 7 \end{bmatrix}, \quad N^T = \begin{bmatrix} 6 & 0 \\ 9 & 5 \end{bmatrix}
\]
\[
M^T + N^T = \begin{bmatrix} 1 & 3 \\ 4 & 7 \end{bmatrix} + \begin{bmatrix} 6 & 0 \\ 9 & 5 \end{bmatrix} = \begin{bmatrix} 7 & 3 \\ 13 & 12 \end{bmatrix}
\]
તેથી, $(M+N)^T = M^T + N^T$ સાબિત થાય છે.
\end{solutionbox}

\questionmarks{2(A).3}{3}{વિકલ સમીકરણ ઉકેલો: $x\frac{dy}{dx} + y = xy$}

\begin{solutionbox}
\textbf{ઉકેલ}:
\[
x\frac{dy}{dx} + y = xy
\]
\[
\frac{dy}{dx} + \frac{y}{x} = y
\]
\[
\frac{dy}{dx} = y - \frac{y}{x} = y\left(1 - \frac{1}{x}\right) = y\left(\frac{x-1}{x}\right)
\]
ચલ અલગ કરતા (Separating variables):
\[
\frac{dy}{y} = \frac{x-1}{x}dx
\]
સંકલન કરતા:
\[
\ln|y| = \int\frac{x-1}{x}dx = \int\left(1 - \frac{1}{x}\right)dx = x - \ln|x| + C
\]
\[
y = Ae^{x-\ln|x|} = A\frac{e^x}{x}
\]
\end{solutionbox}

\questionmarks{2(B)}{8}{કોઈપણ બે લખો.}

\questionmarks{2(B).1}{4}{શ્રેણિક પદ્ધતિથી સમીકરણો $2x + 3y = 8$, $3x + 4y = 11$ ઉકેલો.}

\begin{solutionbox}
\textbf{ઉકેલ}:
શ્રેણિક સ્વરૂપમાં લખતા: $AX = B$
\[
\begin{bmatrix} 2 & 3 \\ 3 & 4 \end{bmatrix}\begin{bmatrix} x \\ y \end{bmatrix} = \begin{bmatrix} 8 \\ 11 \end{bmatrix}
\]
$A^{-1}$ શોધતા:
\[
|A| = 2(4) - 3(3) = 8 - 9 = -1
\]
\[
A^{-1} = \frac{1}{|A|}\begin{bmatrix} 4 & -3 \\ -3 & 2 \end{bmatrix} = \begin{bmatrix} -4 & 3 \\ 3 & -2 \end{bmatrix}
\]
\[
X = A^{-1}B = \begin{bmatrix} -4 & 3 \\ 3 & -2 \end{bmatrix}\begin{bmatrix} 8 \\ 11 \end{bmatrix} = \begin{bmatrix} -32+33 \\ 24-22 \end{bmatrix} = \begin{bmatrix} 1 \\ 2 \end{bmatrix}
\]
તેથી: $x = 1, y = 2$
\end{solutionbox}

\questionmarks{2(B).2}{4}{જો $A = \begin{bmatrix} 3 & 2 \\ 1 & 4 \end{bmatrix}$ અને $B = \begin{bmatrix} 1 & 2 \\ 0 & 1 \end{bmatrix}$ હોય તો સાબિત કરો કે $(AB)^T = B^T A^T$}

\begin{solutionbox}
\textbf{ઉકેલ}:
\[
AB = \begin{bmatrix} 3 & 2 \\ 1 & 4 \end{bmatrix}\begin{bmatrix} 1 & 2 \\ 0 & 1 \end{bmatrix} = \begin{bmatrix} 3 & 8 \\ 1 & 6 \end{bmatrix}
\]
\[
(AB)^T = \begin{bmatrix} 3 & 1 \\ 8 & 6 \end{bmatrix}
\]
\[
A^T = \begin{bmatrix} 3 & 1 \\ 2 & 4 \end{bmatrix}, \quad B^T = \begin{bmatrix} 1 & 0 \\ 2 & 1 \end{bmatrix}
\]
\[
B^T A^T = \begin{bmatrix} 1 & 0 \\ 2 & 1 \end{bmatrix}\begin{bmatrix} 3 & 1 \\ 2 & 4 \end{bmatrix} = \begin{bmatrix} 3 & 1 \\ 8 & 6 \end{bmatrix}
\]
તેથી, $(AB)^T = B^T A^T$ સાબિત થાય છે.
\end{solutionbox}

\questionmarks{2(B).3}{4}{જો $A = \begin{bmatrix} 2 & 3 \\ -1 & 2 \end{bmatrix}$ હોય તો સાબિત કરો કે $A^2 - 4A + 7I = O$}

\begin{solutionbox}
\textbf{ઉકેલ}:
\[
A^2 = \begin{bmatrix} 2 & 3 \\ -1 & 2 \end{bmatrix}\begin{bmatrix} 2 & 3 \\ -1 & 2 \end{bmatrix} = \begin{bmatrix} 1 & 12 \\ -4 & 1 \end{bmatrix}
\]
\[
4A = 4\begin{bmatrix} 2 & 3 \\ -1 & 2 \end{bmatrix} = \begin{bmatrix} 8 & 12 \\ -4 & 8 \end{bmatrix}
\]
\[
7I = 7\begin{bmatrix} 1 & 0 \\ 0 & 1 \end{bmatrix} = \begin{bmatrix} 7 & 0 \\ 0 & 7 \end{bmatrix}
\]
\[
A^2 - 4A + 7I = \begin{bmatrix} 1 & 12 \\ -4 & 1 \end{bmatrix} - \begin{bmatrix} 8 & 12 \\ -4 & 8 \end{bmatrix} + \begin{bmatrix} 7 & 0 \\ 0 & 7 \end{bmatrix} = \begin{bmatrix} 0 & 0 \\ 0 & 0 \end{bmatrix} = O
\]
સાબિત થાય છે.
\end{solutionbox}

\questionmarks{3(A)}{6}{કોઈપણ બે લખો.}

\questionmarks{3(A).1}{3}{વિકલનની વ્યાખ્યાનો ઉપયોગ કરી $f(x) = e^x$ નું વિકલિત શોધો.}

\begin{solutionbox}
\textbf{ઉકેલ}:
વ્યાખ્યાનો ઉપયોગ કરતા: $f'(x) = \lim_{h \to 0} \frac{f(x+h) - f(x)}{h}$
\[
f'(x) = \lim_{h \to 0} \frac{e^{x+h} - e^x}{h} = \lim_{h \to 0} \frac{e^x \cdot e^h - e^x}{h}
\]
\[
= \lim_{h \to 0} \frac{e^x(e^h - 1)}{h} = e^x \lim_{h \to 0} \frac{e^h - 1}{h}
\]
કારણ કે $\lim_{h \to 0} \frac{e^h - 1}{h} = 1$
તેથી: $f'(x) = e^x$
\end{solutionbox}

\questionmarks{3(A).2}{3}{જો $y = \log(\sin x)$ હોય તો $\frac{dy}{dx}$ શોધો.}

\begin{solutionbox}
\textbf{ઉકેલ}:
\[
y = \log(\sin x)
\]
સાંકળ નિયમ (Chain rule) નો ઉપયોગ કરતા:
\[
\frac{dy}{dx} = \frac{1}{\sin x} \cdot \frac{d}{dx}(\sin x) = \frac{1}{\sin x} \cdot \cos x = \frac{\cos x}{\sin x} = \cot x
\]
\end{solutionbox}

\questionmarks{3(A).3}{3}{કિંમત શોધો: $\int\left(4x^3 + 3x^2 + \frac{2}{x}\right)dx$}

\begin{solutionbox}
\textbf{ઉકેલ}:
\[
\int\left(4x^3 + 3x^2 + \frac{2}{x}\right)dx
\]
\[
= \int 4x^3 dx + \int 3x^2 dx + \int \frac{2}{x} dx
\]
\[
= 4 \cdot \frac{x^4}{4} + 3 \cdot \frac{x^3}{3} + 2\ln|x| + C
\]
\[
= x^4 + x^3 + 2\ln|x| + C
\]
\end{solutionbox}

\questionmarks{3(B)}{8}{કોઈપણ બે લખો.}

\questionmarks{3(B).1}{4}{જો $y = e^{\tan x} + \log(\sin x)$ હોય તો $\frac{dy}{dx}$ શોધો.}

\begin{solutionbox}
\textbf{ઉકેલ}:
\[
y = e^{\tan x} + \log(\sin x)
\]
\[
\frac{dy}{dx} = \frac{d}{dx}[e^{\tan x}] + \frac{d}{dx}[\log(\sin x)]
\]
પ્રથમ પદ માટે: $\frac{d}{dx}[e^{\tan x}] = e^{\tan x} \cdot \sec^2 x$
બીજા પદ માટે: $\frac{d}{dx}[\log(\sin x)] = \frac{1}{\sin x} \cdot \cos x = \cot x$

તેથી: $\frac{dy}{dx} = e^{\tan x} \sec^2 x + \cot x$
\end{solutionbox}

\questionmarks{3(B).2}{4}{એક કણની ગતિનું સમીકરણ $s = t^4 + 3t$ છે. તો $t = 2$ સેકન્ડે તેનો વેગ અને પ્રવેગ શોધો.}

\begin{solutionbox}
\textbf{ઉકેલ}:
આપેલ છે: $s = t^4 + 3t$

વેગ: $v = \frac{ds}{dt} = 4t^3 + 3$
$t = 2$ સમયે: $v = 4(2)^3 + 3 = 4(8) + 3 = 32 + 3 = 35$ એકમ/સેકન્ડ

પ્રવેગ: $a = \frac{dv}{dt} = \frac{d^2s}{dt^2} = 12t^2$
$t = 2$ સમયે: $a = 12(2)^2 = 12(4) = 48$ એકમ/સેકન્ડ²
\end{solutionbox}

\questionmarks{3(B).3}{4}{વિધેય $f(x) = 2x^3 - 3x^2 - 12x + 5$ ની મહત્તમ અને ન્યૂનતમ કિંમત શોધો.}

\begin{solutionbox}
\textbf{ઉકેલ}:
\[
f(x) = 2x^3 - 3x^2 - 12x + 5
\]
\[
f'(x) = 6x^2 - 6x - 12 = 6(x^2 - x - 2) = 6(x-2)(x+1)
\]
નિર્ણાયક બિંદુઓ (Critical points) માટે: $f'(x) = 0 \Rightarrow 6(x-2)(x+1) = 0 \Rightarrow x = 2$ અથવા $x = -1$

\[
f''(x) = 12x - 6
\]
$x = -1$ પર: $f''(-1) = 12(-1) - 6 = -18 < 0$ (મહત્તમ)
$x = 2$ પર: $f''(2) = 12(2) - 6 = 18 > 0$ (ન્યૂનતમ)

$f(-1) = 2(-1)^3 - 3(-1)^2 - 12(-1) + 5 = -2 - 3 + 12 + 5 = 12$ (મહત્તમ)
$f(2) = 2(8) - 3(4) - 12(2) + 5 = 16 - 12 - 24 + 5 = -15$ (ન્યૂનતમ)

\textbf{મહત્તમ કિંમત}: 12 ($x = -1$ પર)
\textbf{ન્યૂનતમ કિંમત}: -15 ($x = 2$ પર)
\end{solutionbox}

\questionmarks{4(A)}{6}{કોઈપણ બે લખો.}

\questionmarks{4(A).1}{3}{કિંમત શોધો: $\int xe^x dx$}

\begin{solutionbox}
\textbf{ઉકેલ}:
ખંડશઃ સંકલનનો ઉપયોગ કરતા: $\int u dv = uv - \int v du$

ધારો કે $u = x$, $dv = e^x dx$
તેથી $du = dx$, $v = e^x$

\[
\int xe^x dx = x \cdot e^x - \int e^x dx = xe^x - e^x + C = e^x(x-1) + C
\]
\end{solutionbox}

\questionmarks{4(A).2}{3}{કિંમત શોધો: $\int \frac{dx}{\sqrt{9-4x^2}}$}

\begin{solutionbox}
\textbf{ઉકેલ}:
\[
\int \frac{dx}{\sqrt{9-4x^2}} = \int \frac{dx}{\sqrt{9(1-\frac{4x^2}{9})}} = \int \frac{dx}{3\sqrt{1-\left(\frac{2x}{3}\right)^2}}
\]
ધારો કે $\frac{2x}{3} = \sin \theta$, તેથી $x = \frac{3\sin \theta}{2}$, $dx = \frac{3\cos \theta}{2} d\theta$

\[
= \int \frac{\frac{3\cos \theta}{2} d\theta}{3\sqrt{1-\sin^2 \theta}} = \int \frac{\frac{3\cos \theta}{2} d\theta}{3\cos \theta} = \int \frac{1}{2} d\theta = \frac{\theta}{2} + C
\]
\[
= \frac{1}{2}\sin^{-1}\left(\frac{2x}{3}\right) + C
\]
\end{solutionbox}

\questionmarks{4(A).3}{3}{$\frac{1-i}{1+i}$ નો અનુબદ્ધ સંકર સંખ્યા શોધો.}

\begin{solutionbox}
\textbf{ઉકેલ}:
\[
\frac{1-i}{1+i} = \frac{(1-i)(1-i)}{(1+i)(1-i)} = \frac{(1-i)^2}{1-i^2} = \frac{1-2i+i^2}{1-(-1)} = \frac{1-2i-1}{2} = \frac{-2i}{2} = -i
\]
$-i$ ની અનુબદ્ધ સંકર સંખ્યા $\overline{-i} = i$ છે.
\end{solutionbox}

\questionmarks{4(B)}{8}{કોઈપણ બે લખો.}

\questionmarks{4(B).1}{4}{કિંમત શોધો: $\int_0^{\pi/2} \frac{\sqrt{\cos x}}{\sqrt{\cos x} + \sqrt{\sin x}} dx$}

\begin{solutionbox}
\textbf{ઉકેલ}:
ધારો કે $I = \int_0^{\pi/2} \frac{\sqrt{\cos x}}{\sqrt{\cos x} + \sqrt{\sin x}} dx$

ગુણધર્મ: $\int_0^a f(x)dx = \int_0^a f(a-x)dx$ નો ઉપયોગ કરતા,

\[
I = \int_0^{\pi/2} \frac{\sqrt{\cos(\pi/2-x)}}{\sqrt{\cos(\pi/2-x)} + \sqrt{\sin(\pi/2-x)}} dx = \int_0^{\pi/2} \frac{\sqrt{\sin x}}{\sqrt{\sin x} + \sqrt{\cos x}} dx
\]
બંને સમીકરણોનો સરવાળો કરતા:
\[
2I = \int_0^{\pi/2} \frac{\sqrt{\cos x} + \sqrt{\sin x}}{\sqrt{\cos x} + \sqrt{\sin x}} dx = \int_0^{\pi/2} 1 dx = \frac{\pi}{2}
\]
તેથી: $I = \frac{\pi}{4}$
\end{solutionbox}

\questionmarks{4(B).2}{4}{વર્તુળ $x^2 + y^2 = a^2$ નું ક્ષેત્રફળ સંકલનની મદદથી શોધો.}

\begin{solutionbox}
\textbf{ઉકેલ}:
વર્તુળ $x^2 + y^2 = a^2$ માટે, $y = \pm\sqrt{a^2-x^2}$

વર્તુળનું ક્ષેત્રફળ = $4 \times$ પ્રથમ ચરણમાં ક્ષેત્રફળ
\[
= 4\int_0^a \sqrt{a^2-x^2} dx
\]
ધારો કે $x = a\sin \theta$, $dx = a\cos \theta d\theta$
જ્યારે $x = 0$, $\theta = 0$; જ્યારે $x = a$, $\theta = \pi/2$

\[
= 4\int_0^{\pi/2} \sqrt{a^2-a^2\sin^2 \theta} \cdot a\cos \theta d\theta
\]
\[
= 4\int_0^{\pi/2} a\cos \theta \cdot a\cos \theta d\theta
\]
\[
= 4a^2\int_0^{\pi/2} \cos^2 \theta d\theta
\]
\[
= 4a^2 \cdot \frac{\pi}{4} = \pi a^2
\]
\end{solutionbox}

\questionmarks{4(B).3}{4}{સાદુરૂપ આપો: $\frac{(\cos 3\theta + i\sin 3\theta)^4 \cdot (\cos \theta - i\sin \theta)^5}{(\cos 2\theta - i\sin 2\theta)^3 \cdot (\cos 12\theta + i\sin 12\theta)}$}

\begin{solutionbox}
\textbf{ઉકેલ}:
દ-દમ્વારેના પ્રમેય (De Moivre's theorem) નો ઉપયોગ કરતા: $(\cos \theta + i\sin \theta)^n = \cos n\theta + i\sin n\theta$

અંશ:
\[
(\cos 3\theta + i\sin 3\theta)^4 \cdot (\cos \theta - i\sin \theta)^5
\]
\[
= (\cos 12\theta + i\sin 12\theta) \cdot (\cos(-5\theta) + i\sin(-5\theta))
\]
\[
= \cos(12\theta - 5\theta) + i\sin(12\theta - 5\theta)
\]
\[
= \cos 7\theta + i\sin 7\theta
\]
છેદ:
\[
(\cos 2\theta - i\sin 2\theta)^3 \cdot (\cos 12\theta + i\sin 12\theta)
\]
\[
= (\cos(-6\theta) + i\sin(-6\theta)) \cdot (\cos 12\theta + i\sin 12\theta)
\]
\[
= \cos(-6\theta + 12\theta) + i\sin(-6\theta + 12\theta)
\]
\[
= \cos 6\theta + i\sin 6\theta
\]
પરિણામ:
\[
\frac{\cos 7\theta + i\sin 7\theta}{\cos 6\theta + i\sin 6\theta} = \cos(7\theta - 6\theta) + i\sin(7\theta - 6\theta) = \cos \theta + i\sin \theta
\]
\end{solutionbox}

\questionmarks{5(A)}{6}{કોઈપણ બે લખો.}

\questionmarks{5(A).1}{3}{જો $(3x - 7) + 2iy = 5y + (5 + x)i$ હોય તો x અને y ની કિંમત શોધો.}

\begin{solutionbox}
\textbf{ઉકેલ}:
\[
(3x - 7) + 2iy = 5y + (5 + x)i
\]
વાસ્તવિક અને કાલ્પનિક ભાગો સરખાવતા:
વાસ્તવિક ભાગ: $3x - 7 = 5y$ ... (1)
કાલ્પનિક ભાગ: $2y = 5 + x$ ... (2)

સમીકરણ (2) પરથી: $x = 2y - 5$ ... (3)

(1) માં (3) મુકતા:
\[
3(2y - 5) - 7 = 5y
\]
\[
6y - 15 - 7 = 5y
\]
\[
6y - 22 = 5y \Rightarrow y = 22
\]
(3) પરથી: $x = 2(22) - 5 = 44 - 5 = 39$

તેથી: $x = 39, y = 22$
\end{solutionbox}

\questionmarks{5(A).2}{3}{$z = 1 + \sqrt{3}i$ ને ધ્રુવીય સ્વરૂપમાં ફેરવો.}

\begin{solutionbox}
\textbf{ઉકેલ}:
\[
z = 1 + \sqrt{3}i
\]
માનાંક: $|z| = \sqrt{1^2 + (\sqrt{3})^2} = \sqrt{1 + 3} = \sqrt{4} = 2$

કોણાંક: $\arg(z) = \tan^{-1}\left(\frac{\sqrt{3}}{1}\right) = \tan^{-1}(\sqrt{3}) = \frac{\pi}{3}$

ધ્રુવીય સ્વરૂપ: $z = |z|(\cos \theta + i\sin \theta) = 2\left(\cos \frac{\pi}{3} + i\sin \frac{\pi}{3}\right)$
\end{solutionbox}

\questionmarks{5(A).3}{3}{$\frac{4 + 2i}{(3 + 2i)(5 - 3i)}$ ને $a + ib$ સ્વરૂપમાં દર્શાવો.}

\begin{solutionbox}
\textbf{ઉકેલ}:
પ્રથમ છેદનું સાદુરૂપ આપીએ:
\[
(3 + 2i)(5 - 3i) = 15 - 9i + 10i - 6i^2 = 15 + i - 6(-1) = 15 + i + 6 = 21 + i
\]
\[
\frac{4 + 2i}{21 + i} = \frac{(4 + 2i)(21 - i)}{(21 + i)(21 - i)} = \frac{84 - 4i + 42i - 2i^2}{21^2 - i^2} = \frac{84 + 38i + 2}{441 + 1} = \frac{86 + 38i}{442}
\]
\[
= \frac{86}{442} + \frac{38}{442}i = \frac{43}{221} + \frac{19}{221}i
\]
\end{solutionbox}

\questionmarks{5(B)}{8}{કોઈપણ બે લખો.}

\questionmarks{5(B).1}{4}{વિકલ સમીકરણ ઉકેલો: $\frac{dy}{dx} + 2y = 3e^x$}

\begin{solutionbox}
\textbf{ઉકેલ}:
આ $\frac{dy}{dx} + Py = Q$ સ્વરૂપનું પ્રથમ કક્ષાનું સુરેખ વિકલ સમીકરણ છે.

અહીં: $P = 2$, $Q = 3e^x$

સંકલ્પકારક અવયવ: $\mu = e^{\int P dx} = e^{\int 2 dx} = e^{2x}$

સમીકરણને $\mu$ વડે ગુણતા:
\[
e^{2x}\frac{dy}{dx} + 2e^{2x}y = 3e^{2x} \cdot e^x = 3e^{3x}
\]
આ આપે છે: $\frac{d}{dx}(ye^{2x}) = 3e^{3x}$

બંને બાજુ સંકલન કરતા:
\[
ye^{2x} = \int 3e^{3x} dx = 3 \cdot \frac{e^{3x}}{3} + C = e^{3x} + C
\]
તેથી: $y = \frac{e^{3x} + C}{e^{2x}} = e^x + Ce^{-2x}$
\end{solutionbox}

\questionmarks{5(B).2}{4}{વિકલ સમીકરણ ઉકેલો: $\frac{dy}{dx} = (x + y)^2$}

\begin{solutionbox}
\textbf{ઉકેલ}:
ધારો કે $v = x + y$, તેથી $\frac{dv}{dx} = 1 + \frac{dy}{dx}$

તેથી $\frac{dy}{dx} = \frac{dv}{dx} - 1$

મૂળ સમીકરણમાં મુકતા:
\[
\frac{dv}{dx} - 1 = v^2
\]
\[
\frac{dv}{dx} = v^2 + 1
\]
ચલ અલગ કરતા:
\[
\frac{dv}{v^2 + 1} = dx
\]
બંને બાજુ સંકલન કરતા:
\[
\int \frac{dv}{v^2 + 1} = \int dx
\]
\[
\tan^{-1}(v) = x + C
\]
\[
v = \tan(x + C)
\]
પાછું મુકતા: $x + y = \tan(x + C)$
તેથી: $y = \tan(x + C) - x$
\end{solutionbox}

\questionmarks{5(B).3}{4}{વિકલ સમીકરણ ઉકેલો: $\frac{dy}{dx} + \frac{y}{x} = e^x$, $y(0) = 2$}

\begin{solutionbox}
\textbf{ઉકેલ}:
આ પ્રથમ કક્ષાનું સુરેખ વિકલ સમીકરણ છે: $\frac{dy}{dx} + \frac{y}{x} = e^x$

અહીં: $P = \frac{1}{x}$, $Q = e^x$

સંકલ્પકારક અવયવ: $\mu = e^{\int \frac{1}{x} dx} = e^{\ln|x|} = |x| = x$ (માટે $x > 0$)

સમીકરણને $\mu = x$ વડે ગુણતા:
\[
x\frac{dy}{dx} + y = xe^x
\]
આ આપે છે: $\frac{d}{dx}(xy) = xe^x$

બંને બાજુ સંકલન કરતા (ખંડશઃ સંકલનનો ઉપયોગ કરીને):
\[
xy = \int xe^x dx
\]
$\int xe^x dx$ માટે: ધારો કે $u = x$, $dv = e^x dx$
તેથી $du = dx$, $v = e^x$
\[
\int xe^x dx = xe^x - \int e^x dx = xe^x - e^x = e^x(x-1)
\]
તેથી: $xy = e^x(x-1) + C$
\[
y = \frac{e^x(x-1) + C}{x}
\]
પ્રારંભિક શરત $y(0) = 2$ નો ઉપયોગ કરતા:
અહીં શૂન્ય વડે ભાગાકારની સમસ્યા ઉદ્ભવે છે. સમીકરણનો ઉકેલ $x = 0$ ની નજીક વધુ કાળજીપૂર્વક શોધવો પડે.

સામાન્ય ઉકેલ માટે: $y = e^x\left(1 - \frac{1}{x}\right) + \frac{C}{x}$
\end{solutionbox}

\end{document}
