\documentclass[10pt,a4paper]{article}

% content/resources/templates/preamble.tex
\usepackage[margin=0.6in]{geometry}
\author{Milav Dabgar}
\usepackage{amsmath,amssymb,amsthm}
\usepackage{booktabs}
\usepackage{multirow}
\usepackage{xcolor}
\usepackage{tcolorbox}
\tcbuselibrary{breakable,skins}
\usepackage[colorlinks=true,linkcolor=blue]{hyperref}
\usepackage{titlesec}
\usepackage{enumitem}
\usepackage{tikz}
\usepackage{pgfplots}
\usepackage{circuitikz}
\usepackage[version=4]{mhchem}
\usepackage{longtable}
\usepackage{array}
\usepackage{float}
\usepackage{caption}
\usepackage{listings}

\lstset{
  basicstyle=\small\ttfamily,
  breaklines=true,
  breakatwhitespace=false,
  postbreak=\mbox{\textcolor{red}{$\hookrightarrow$}\space},
  float=false,
  numbers=left,
  numberstyle=\tiny\color{gray},
  numbersep=10pt,
  xleftmargin=2em,
  keywordstyle=\color{blue},
  commentstyle=\color{green!60!black},
  stringstyle=\color{purple},
  backgroundcolor=\color{gray!5},
  showstringspaces=false,
  tabsize=2,
  captionpos=b,
  keepspaces=true,
  columns=flexible
}

\pgfplotsset{compat=1.18}
\usetikzlibrary{shapes,arrows,positioning,calc,patterns,decorations.pathmorphing,decorations.markings,arrows.meta}

% Color scheme
\definecolor{headcolor}{RGB}{0,102,204}
\definecolor{keycolor}{RGB}{220,20,60}
\definecolor{solutioncolor}{RGB}{34,139,34}
\definecolor{mnemoniccolor}{RGB}{148,0,211}
\definecolor{codecolor}{RGB}{0,0,100}

% Spacing
\setlength{\parskip}{3pt}
\setlist[itemize]{nosep}
\setlist[enumerate]{nosep}

% Title formatting
\titleformat{\section}{\Large\bfseries\color{headcolor}}{\thesection}{1em}{}
\titleformat{\subsection}{\large\bfseries\color{headcolor}}{\thesubsection}{1em}{}

% Pandoc tightlist compatibility
\providecommand{\tightlist}{%
  \setlength{\itemsep}{0pt}\setlength{\parskip}{0pt}}

% Pandoc longtable compatibility
\newcounter{none}
\def\thenone{}


% content/resources/templates/english-boxes.tex
% This file is currently empty - it exists to maintain consistency with the import structure.
% Add custom environments here if needed in the future.


\begin{document}

\begin{center}
{\Huge\bfseries\color{headcolor} Mathematics-I Solutions}\\[5pt]
{\LARGE DI01000021 -- Summer 2025}\\[3pt]
{\large Semester 1 Study Material}\\[3pt]
{\normalsize\textit{Detailed Solutions and Explanations}}
\end{center}

\vspace{10pt}

\subsection*{Q.1 [14 marks]}\label{q.1-14-marks}

\textbf{Fill in the blanks/MCQs using appropriate choice from the given
options}

\subsubsection{Q1.1 [1 mark]}\label{q1.1-1-mark}

**\$\log\emph{3 1 = \$ }\_\_\_**

\begin{solutionbox}
d.~0

\textbf{Solution}: For any base \(a > 0, a \neq 1\): \(\log_a 1 = 0\)
Therefore: \(\log_3 1 = 0\)

\end{solutionbox}
\subsubsection{Q1.2 [1 mark]}\label{q1.2-1-mark}

\textbf{If \(f(x) = e^{x-1}\) then \$f(1) = \$ \_\_\_\_}

\begin{solutionbox}
c.~1

\textbf{Solution}: \(f(x) = e^{x-1}\) \(f(1) = e^{1-1} = e^0 = 1\)

\end{solutionbox}
\subsubsection{Q1.3 [1 mark]}\label{q1.3-1-mark}

**\$\log\emph{5 125 = \$ }\_\_\_**

\begin{solutionbox}
b. 3

\textbf{Solution}: \(\log_5 125 = \log_5 5^3 = 3\) Since \(5^3 = 125\)

\end{solutionbox}
\subsubsection{Q1.4 [1 mark]}\label{q1.4-1-mark}

\textbf{If \(f(x) = x^3 - 7\) then \$f(-2) = \$ \_\_\_\_}

\begin{solutionbox}
c.~-15

\textbf{Solution}: \(f(x) = x^3 - 7\)
\(f(-2) = (-2)^3 - 7 = -8 - 7 = -15\)

\end{solutionbox}
\subsubsection{Q1.5 [1 mark]}\label{q1.5-1-mark}

\textbf{Principal period of \(\cos x\) is \_\_\_\_}

\begin{solutionbox}
c.~\(2\pi\)

\textbf{Solution}: The cosine function repeats every \(2\pi\) radians,
so its principal period is \(2\pi\).

\end{solutionbox}
\subsubsection{Q1.6 [1 mark]}\label{q1.6-1-mark}

\textbf{\$150^\circ = \$ \_\_\_\_}

\begin{solutionbox}
a. \(\frac{5\pi}{6}\)

\textbf{Solution}: Converting degrees to radians:
\(150^\circ = 150 \times \frac{\pi}{180} = \frac{5\pi}{6}\)

\end{solutionbox}
\subsubsection{Q1.7 [1 mark]}\label{q1.7-1-mark}

\textbf{\$\sin\^{}\{-1\}x + \cos\^{}\{-1\}x = \$ \_\_\_\_}

\begin{solutionbox}
a. \(\frac{\pi}{2}\)

\textbf{Solution}: This is a standard identity:
\(\sin^{-1}x + \cos^{-1}x = \frac{\pi}{2}\) for \(x \in [-1, 1]\)

\end{solutionbox}
\subsubsection{Q1.8 [1 mark]}\label{q1.8-1-mark}

\textbf{(1,0,0) \times (1,0,0) = \_\_\_\_}

\begin{solutionbox}
d.~(0,0,0)

\textbf{Solution}: Cross product of any vector with itself is zero
vector: \((1,0,0) \times (1,0,0) = (0,0,0)\)

\end{solutionbox}
\subsubsection{Q1.9 [1 mark]}\label{q1.9-1-mark}

\textbf{If \(\vec{a} = 4\hat{i} - 3\hat{j}\) then
\$\textbar{}\vec{a}\textbar{} = \$ \_\_\_\_}

\begin{solutionbox}
b. 5

\textbf{Solution}:
\(|\vec{a}| = \sqrt{4^2 + (-3)^2} = \sqrt{16 + 9} = \sqrt{25} = 5\)

\end{solutionbox}
\subsubsection{Q1.10 [1 mark]}\label{q1.10-1-mark}

\textbf{If a line makes an angle \(45^\circ\) with positive x-axis then slope
of the line is \_\_\_\_}

\begin{solutionbox}
c.~1

\textbf{Solution}: Slope \(m = \tan(45^\circ) = 1\)

\end{solutionbox}
\subsubsection{Q1.11 [1 mark]}\label{q1.11-1-mark}

\textbf{Radius of the circle \(x^2 + y^2 = 4\) is \_\_\_\_}

\begin{solutionbox}
d.~2

\textbf{Solution}: Standard form: \(x^2 + y^2 = r^2\) Comparing:
\(r^2 = 4\), so \(r = 2\)

\end{solutionbox}
\subsubsection{Q1.12 [1 mark]}\label{q1.12-1-mark}

**\$\lim\emph{\{x \to 0\} \frac{e^x - 1}{x} = \$ }\_\_\_**

\begin{solutionbox}
a. 1

\textbf{Solution}: This is a standard limit:
\(\lim_{x \to 0} \frac{e^x - 1}{x} = 1\)

\end{solutionbox}
\subsubsection{Q1.13 [1 mark]}\label{q1.13-1-mark}

**\$\lim\emph{\{x \to 0\} \frac{\sin 3x}{x} = \$ }\_\_\_**

\begin{solutionbox}
d.~3

\textbf{Solution}:
\(\lim_{x \to 0} \frac{\sin 3x}{x} = \lim_{x \to 0} \frac{\sin 3x}{3x} \times 3 = 1 \times 3 = 3\)

\end{solutionbox}
\subsubsection{Q1.14 [1 mark]}\label{q1.14-1-mark}

**\$\lim\emph{\{n \to \infty\} \frac{5n + 4}{4n + 5} = \$ }\_\_\_**

\begin{solutionbox}
c.~5/4

\textbf{Solution}:
\(\lim_{n \to \infty} \frac{5n + 4}{4n + 5} = \lim_{n \to \infty} \frac{5 + \frac{4}{n}}{4 + \frac{5}{n}} = \frac{5}{4}\)

\end{solutionbox}
\begin{center}\rule{0.5\linewidth}{0.5pt}\end{center}

\subsection*{Q.2 (A) [6 marks]}\label{q.2-a-6-marks}

\textbf{Attempt any two}

\subsubsection{Q2(A).1 [3 marks]}\label{q2a.1-3-marks}

\textbf{Find value:
\(\begin{vmatrix} 1 & 2 & 3 \\ 4 & 5 & 6 \\ 7 & 8 & 9 \end{vmatrix}\)}

\begin{solutionbox}
0

\textbf{Solution}:
\(\begin{vmatrix} 1 & 2 & 3 \\ 4 & 5 & 6 \\ 7 & 8 & 9 \end{vmatrix} = 1(5 \times 9 - 6 \times 8) - 2(4 \times 9 - 6 \times 7) + 3(4 \times 8 - 5 \times 7)\)

\(= 1(45 - 48) - 2(36 - 42) + 3(32 - 35)\) \(= 1(-3) - 2(-6) + 3(-3)\)
\(= -3 + 12 - 9 = 0\)

\end{solutionbox}
\subsubsection{Q2(A).2 [3 marks]}\label{q2a.2-3-marks}

\textbf{Prove that:
\(\log\left(\frac{x^p}{x^q}\right) + \log\left(\frac{x^q}{x^r}\right) + \log\left(\frac{x^r}{x^p}\right) = 0\)}

\textbf{Solution}: LHS =
\(\log\left(\frac{x^p}{x^q}\right) + \log\left(\frac{x^q}{x^r}\right) + \log\left(\frac{x^r}{x^p}\right)\)

Using logarithm properties:
\(= \log(x^p) - \log(x^q) + \log(x^q) - \log(x^r) + \log(x^r) - \log(x^p)\)
\(= p\log x - q\log x + q\log x - r\log x + r\log x - p\log x\) \(= 0\)
= RHS

\subsubsection{Q2(A).3 [3 marks]}\label{q2a.3-3-marks}

\textbf{Find value: \(\tan(75^\circ)\)}

\begin{solutionbox}
\(2 + \sqrt{3}\)

\textbf{Solution}: \(\tan(75^\circ) = \tan(45^\circ + 30^\circ)\)

Using \(\tan(A + B) = \frac{\tan A + \tan B}{1 - \tan A \tan B}\):

\(\tan(75^\circ) = \frac{\tan 45^\circ + \tan 30^\circ}{1 - \tan 45^\circ \tan 30^\circ} = \frac{1 + \frac{1}{\sqrt{3}}}{1 - 1 \times \frac{1}{\sqrt{3}}} = \frac{1 + \frac{1}{\sqrt{3}}}{1 - \frac{1}{\sqrt{3}}}\)

\(= \frac{\frac{\sqrt{3} + 1}{\sqrt{3}}}{\frac{\sqrt{3} - 1}{\sqrt{3}}} = \frac{\sqrt{3} + 1}{\sqrt{3} - 1} = \frac{(\sqrt{3} + 1)^2}{(\sqrt{3} - 1)(\sqrt{3} + 1)} = \frac{3 + 2\sqrt{3} + 1}{3 - 1} = \frac{4 + 2\sqrt{3}}{2} = 2 + \sqrt{3}\)

\end{solutionbox}
\begin{center}\rule{0.5\linewidth}{0.5pt}\end{center}

\subsection*{Q.2 (B) [8 marks]}\label{q.2-b-8-marks}

\textbf{Attempt any two}

\subsubsection{Q2(B).1 [4 marks]}\label{q2b.1-4-marks}

\textbf{Prove that:
\(\frac{1}{\log_{12} 120} + \frac{1}{\log_2 120} + \frac{1}{\log_5 120} = 1\)}

\textbf{Solution}: Using change of base formula:
\(\frac{1}{\log_a b} = \log_b a\)

LHS = \(\log_{120} 12 + \log_{120} 2 + \log_{120} 5\)

Using logarithm properties:
\(= \log_{120}(12 \times 2 \times 5) = \log_{120} 120 = 1\) = RHS

\subsubsection{Q2(B).2 [4 marks]}\label{q2b.2-4-marks}

\textbf{Solve:
\(\begin{vmatrix} x & 1 & 1 \\ 1 & 2 & 1 \\ 0 & 0 & 3 \end{vmatrix} = 3\)}

\textbf{Solution}: Expanding along third row:
\(\begin{vmatrix} x & 1 & 1 \\ 1 & 2 & 1 \\ 0 & 0 & 3 \end{vmatrix} = 3 \begin{vmatrix} x & 1 \\ 1 & 2 \end{vmatrix}\)

\(= 3(2x - 1) = 6x - 3\)

Given: \(6x - 3 = 3\) \(6x = 6\) \(x = 1\)

\subsubsection{Q2(B).3 [4 marks]}\label{q2b.3-4-marks}

\textbf{If \(f(x) = \frac{1-x}{1+x}\) prove that: (i)
\(f(x) + f\left(\frac{1}{x}\right) = 0\) (ii) \(f(x) \times f(-x) = 1\)}

\textbf{Solution}: Given: \(f(x) = \frac{1-x}{1+x}\)

\textbf{(i)}
\(f\left(\frac{1}{x}\right) = \frac{1-\frac{1}{x}}{1+\frac{1}{x}} = \frac{\frac{x-1}{x}}{\frac{x+1}{x}} = \frac{x-1}{x+1} = -\frac{1-x}{1+x} = -f(x)\)

Therefore: \(f(x) + f\left(\frac{1}{x}\right) = f(x) + (-f(x)) = 0\)

\textbf{(ii)} \(f(-x) = \frac{1-(-x)}{1+(-x)} = \frac{1+x}{1-x}\)

\(f(x) \times f(-x) = \frac{1-x}{1+x} \times \frac{1+x}{1-x} = 1\)

\begin{center}\rule{0.5\linewidth}{0.5pt}\end{center}

\subsection*{Q.3 (A) [6 marks]}\label{q.3-a-6-marks}

\textbf{Attempt any two}

\subsubsection{Q3(A).1 [3 marks]}\label{q3a.1-3-marks}

\textbf{Prove that:
\(\frac{\sin(180^\circ - x) + \cosec(180^\circ - x) + \tan(180^\circ + x)}{\cos(90^\circ + x) + \sec(90^\circ + x) + \cot(90^\circ + x)} = -3\)}

\textbf{Solution}: Using trigonometric identities:

\begin{itemize}
\tightlist
\item
  \(\sin(180^\circ - x) = \sin x\)
\item
  \(\cosec(180^\circ - x) = \cosec x\)
\item
  \(\tan(180^\circ + x) = \tan x\)
\item
  \(\cos(90^\circ + x) = -\sin x\)
\item
  \(\sec(90^\circ + x) = -\cosec x\)
\item
  \(\cot(90^\circ + x) = -\tan x\)
\end{itemize}

Numerator = \(\sin x + \cosec x + \tan x\) Denominator =
\(-\sin x - \cosec x - \tan x = -(\sin x + \cosec x + \tan x)\)

Therefore:
\(\frac{\sin x + \cosec x + \tan x}{-(\sin x + \cosec x + \tan x)} = -1 \neq -3\)

\textbf{Note}: There appears to be an error in the problem statement or
expected answer.

\subsubsection{Q3(A).2 [3 marks]}\label{q3a.2-3-marks}

\textbf{Prove that:
\(\tan^{-1}\left(\frac{1}{3}\right) + \tan^{-1}\left(\frac{1}{2}\right) = 45^\circ\)}

\textbf{Solution}: Using
\(\tan^{-1}A + \tan^{-1}B = \tan^{-1}\left(\frac{A+B}{1-AB}\right)\):

\(\tan^{-1}\left(\frac{1}{3}\right) + \tan^{-1}\left(\frac{1}{2}\right) = \tan^{-1}\left(\frac{\frac{1}{3} + \frac{1}{2}}{1 - \frac{1}{3} \times \frac{1}{2}}\right)\)

\(= \tan^{-1}\left(\frac{\frac{5}{6}}{1 - \frac{1}{6}}\right) = \tan^{-1}\left(\frac{\frac{5}{6}}{\frac{5}{6}}\right) = \tan^{-1}(1) = 45^\circ\)

\subsubsection{Q3(A).3 [3 marks]}\label{q3a.3-3-marks}

\textbf{Find out equation of the line whose X-intercept is 3 and
Y-intercept is 2.}

\textbf{Solution}: Using intercept form:
\(\frac{x}{a} + \frac{y}{b} = 1\)

Where \(a = 3\) (x-intercept) and \(b = 2\) (y-intercept)

\(\frac{x}{3} + \frac{y}{2} = 1\)

Multiplying by 6: \(2x + 3y = 6\)

\begin{center}\rule{0.5\linewidth}{0.5pt}\end{center}

\subsection*{Q.3 (B) [8 marks]}\label{q.3-b-8-marks}

\textbf{Attempt any two}

\subsubsection{Q3(B).1 [4 marks]}\label{q3b.1-4-marks}

\textbf{Prove that:
\(\tan(70^\circ) = \frac{\cos(25^\circ) + \sin(25^\circ)}{\cos(25^\circ) - \sin(25^\circ)}\)}

\textbf{Solution}: RHS =
\(\frac{\cos(25^\circ) + \sin(25^\circ)}{\cos(25^\circ) - \sin(25^\circ)}\)

Dividing numerator and denominator by \(\cos(25^\circ)\):

\(= \frac{1 + \tan(25^\circ)}{1 - \tan(25^\circ)}\)

Using \(\tan(45^\circ + θ) = \frac{1 + \tan θ}{1 - \tan θ}\):

\(= \tan(45^\circ + 25^\circ) = \tan(70^\circ)\) = LHS

\subsubsection{Q3(B).2 [4 marks]}\label{q3b.2-4-marks}

\textbf{Prove that:
\(\frac{\sin θ + \sin 2θ + \sin 3θ}{\cos θ + \cos 2θ + \cos 3θ} = \tan 2θ\)}

\textbf{Solution}: Using sum-to-product formulas:

Numerator:
\(\sin θ + \sin 3θ + \sin 2θ = 2\sin 2θ \cos θ + \sin 2θ = \sin 2θ(2\cos θ + 1)\)

Denominator:
\(\cos θ + \cos 3θ + \cos 2θ = 2\cos 2θ \cos θ + \cos 2θ = \cos 2θ(2\cos θ + 1)\)

Therefore:
\(\frac{\sin 2θ(2\cos θ + 1)}{\cos 2θ(2\cos θ + 1)} = \frac{\sin 2θ}{\cos 2θ} = \tan 2θ\)

\subsubsection{Q3(B).3 [4 marks]}\label{q3b.3-4-marks}

\textbf{If \(\vec{a} = (1,2,3)\), \(\vec{b} = (4,0,0)\) and
\(\vec{c} = (2,0,1)\) find \(2\vec{a} + 3\vec{b} - 5\vec{c}\)}

\textbf{Solution}: \(2\vec{a} = 2(1,2,3) = (2,4,6)\)
\(3\vec{b} = 3(4,0,0) = (12,0,0)\) \(5\vec{c} = 5(2,0,1) = (10,0,5)\)

\(2\vec{a} + 3\vec{b} - 5\vec{c} = (2,4,6) + (12,0,0) - (10,0,5)\)
\(= (2+12-10, 4+0-0, 6+0-5)\) \(= (4,4,1)\)

\begin{center}\rule{0.5\linewidth}{0.5pt}\end{center}

\subsection*{Q.4 (A) [6 marks]}\label{q.4-a-6-marks}

\textbf{Attempt any two}

\subsubsection{Q4(A).1 [3 marks]}\label{q4a.1-3-marks}

\textbf{If the vectors \(\vec{a} = \hat{i} - 2\hat{j} + 3\hat{k}\) and
\(\vec{b} = 2\hat{i} + m\hat{j} - 4\hat{k}\) are perpendicular, find m.}

\textbf{Solution}: For perpendicular vectors:
\(\vec{a} \cdot \vec{b} = 0\)

\(\vec{a} \cdot \vec{b} = (1)(2) + (-2)(m) + (3)(-4) = 2 - 2m - 12 = -10 - 2m\)

Setting equal to zero: \(-10 - 2m = 0\) \(2m = -10\) \(m = -5\)

\subsubsection{Q4(A).2 [3 marks]}\label{q4a.2-3-marks}

\textbf{Find the direction cosines and direction angles of the vector
\(\vec{a} = 5\hat{i} - 12\hat{k}\)}

\textbf{Solution}: \(\vec{a} = 5\hat{i} + 0\hat{j} - 12\hat{k}\)

Magnitude:
\(|\vec{a}| = \sqrt{5^2 + 0^2 + (-12)^2} = \sqrt{25 + 144} = \sqrt{169} = 13\)

Direction cosines:

\begin{itemize}
\tightlist
\item
  \(l = \frac{5}{13}\)
\item
  \(m = \frac{0}{13} = 0\)\\
\item
  \(n = \frac{-12}{13}\)
\end{itemize}

Direction angles:

\begin{itemize}
\tightlist
\item
  \(α = \cos^{-1}\left(\frac{5}{13}\right)\)
\item
  \(β = \cos^{-1}(0) = 90^\circ\)
\item
  \(γ = \cos^{-1}\left(\frac{-12}{13}\right)\)
\end{itemize}

\subsubsection{Q4(A).3 [3 marks]}\label{q4a.3-3-marks}

\textbf{Find out equation of the circle having center at \((2, -3)\) and
radius 3.}

\textbf{Solution}: Standard form: \((x - h)^2 + (y - k)^2 = r^2\)

Where \((h, k) = (2, -3)\) and \(r = 3\)

\((x - 2)^2 + (y + 3)^2 = 9\)

Expanding: \(x^2 - 4x + 4 + y^2 + 6y + 9 = 9\)
\(x^2 + y^2 - 4x + 6y + 4 = 0\)

\begin{center}\rule{0.5\linewidth}{0.5pt}\end{center}

\subsection*{Q.4 (B) [8 marks]}\label{q.4-b-8-marks}

\textbf{Attempt any two}

\subsubsection{Q4(B).1 [4 marks]}\label{q4b.1-4-marks}

\textbf{Show that the angle between vectors
\(\vec{a} = \hat{i} + 2\hat{j}\) and
\(\vec{b} = \hat{i} + \hat{j} + 3\hat{k}\) is
\(\sin^{-1}\sqrt{\frac{46}{55}}\)}

\textbf{Solution}:
\(\vec{a} \cdot \vec{b} = (1)(1) + (2)(1) + (0)(3) = 1 + 2 = 3\)

\(|\vec{a}| = \sqrt{1^2 + 2^2} = \sqrt{5}\)
\(|\vec{b}| = \sqrt{1^2 + 1^2 + 3^2} = \sqrt{11}\)

\(\cos θ = \frac{\vec{a} \cdot \vec{b}}{|\vec{a}||\vec{b}|} = \frac{3}{\sqrt{5}\sqrt{11}} = \frac{3}{\sqrt{55}}\)

\(\sin^2 θ = 1 - \cos^2 θ = 1 - \frac{9}{55} = \frac{46}{55}\)

Therefore: \(θ = \sin^{-1}\sqrt{\frac{46}{55}}\)

\subsubsection{Q4(B).2 [4 marks]}\label{q4b.2-4-marks}

\textbf{Under effect of the forces \(2\hat{i} + \hat{j} + \hat{k}\) and
\(\hat{i} + 3\hat{j} - \hat{k}\) a particle moves from the point
\((1,2,-3)\) to the point \((5,3,7)\). Find out work done.}

\textbf{Solution}: Net force:
\(\vec{F} = (2\hat{i} + \hat{j} + \hat{k}) + (\hat{i} + 3\hat{j} - \hat{k}) = 3\hat{i} + 4\hat{j}\)

Displacement: \(\vec{s} = (5,3,7) - (1,2,-3) = (4,1,10)\)

Work done:
\(W = \vec{F} \cdot \vec{s} = (3)(4) + (4)(1) + (0)(10) = 12 + 4 = 16\)
units

\subsubsection{Q4(B).3 [4 marks]}\label{q4b.3-4-marks}

\textbf{Evaluate: \(\lim_{x \to 0} \frac{2^x - 5^x}{x}\)}

\textbf{Solution}: Using L'Hôpital's rule or the derivative definition:

\(\lim_{x \to 0} \frac{2^x - 5^x}{x} = \lim_{x \to 0} \frac{2^x \ln 2 - 5^x \ln 5}{1}\)

\(= 2^0 \ln 2 - 5^0 \ln 5 = \ln 2 - \ln 5 = \ln\left(\frac{2}{5}\right)\)

\begin{center}\rule{0.5\linewidth}{0.5pt}\end{center}

\subsection*{Q.5 (A) [6 marks]}\label{q.5-a-6-marks}

\textbf{Attempt any two}

\subsubsection{Q5(A).1 [3 marks]}\label{q5a.1-3-marks}

\textbf{Evaluate:
\(\lim_{x \to 0} \left(1 + \frac{3x}{7}\right)^{\frac{1}{x}}\)}

\textbf{Solution}: Let
\(y = \left(1 + \frac{3x}{7}\right)^{\frac{1}{x}}\)

Taking natural log:
\(\ln y = \frac{1}{x} \ln\left(1 + \frac{3x}{7}\right)\)

\(\lim_{x \to 0} \ln y = \lim_{x \to 0} \frac{\ln\left(1 + \frac{3x}{7}\right)}{x}\)

Using L'Hôpital's rule:
\(= \lim_{x \to 0} \frac{\frac{3/7}{1 + \frac{3x}{7}}}{1} = \frac{3}{7}\)

Therefore: \(\lim_{x \to 0} y = e^{3/7}\)

\subsubsection{Q5(A).2 [3 marks]}\label{q5a.2-3-marks}

\textbf{Evaluate: \(\lim_{x \to 3} \frac{x^2 - 5x + 6}{x^2 - 9}\)}

\textbf{Solution}: Factoring numerator: \(x^2 - 5x + 6 = (x-2)(x-3)\)
Factoring denominator: \(x^2 - 9 = (x-3)(x+3)\)

\(\lim_{x \to 3} \frac{x^2 - 5x + 6}{x^2 - 9} = \lim_{x \to 3} \frac{(x-2)(x-3)}{(x-3)(x+3)} = \lim_{x \to 3} \frac{x-2}{x+3} = \frac{3-2}{3+3} = \frac{1}{6}\)

\subsubsection{Q5(A).3 [3 marks]}\label{q5a.3-3-marks}

\textbf{Evaluate: \(\lim_{x \to 0} \frac{\sqrt{4+x} - 2}{x}\)}

\textbf{Solution}: Rationalizing the numerator:

\(\lim_{x \to 0} \frac{\sqrt{4+x} - 2}{x} \times \frac{\sqrt{4+x} + 2}{\sqrt{4+x} + 2}\)

\(= \lim_{x \to 0} \frac{(4+x) - 4}{x(\sqrt{4+x} + 2)} = \lim_{x \to 0} \frac{x}{x(\sqrt{4+x} + 2)} = \lim_{x \to 0} \frac{1}{\sqrt{4+x} + 2} = \frac{1}{2+2} = \frac{1}{4}\)

\begin{center}\rule{0.5\linewidth}{0.5pt}\end{center}

\subsection*{Q.5 (B) [8 marks]}\label{q.5-b-8-marks}

\textbf{Attempt any two}

\subsubsection{Q5(B).1 [4 marks]}\label{q5b.1-4-marks}

\textbf{Find out equation of the line passing through points \((1,2)\)
and \((2,1)\).}

\textbf{Solution}: Using two-point form:
\(\frac{y - y_1}{y_2 - y_1} = \frac{x - x_1}{x_2 - x_1}\)

\(\frac{y - 2}{1 - 2} = \frac{x - 1}{2 - 1}\)

\(\frac{y - 2}{-1} = \frac{x - 1}{1}\)

\(y - 2 = -(x - 1) = -x + 1\)

\(x + y = 3\)

\subsubsection{Q5(B).2 [4 marks]}\label{q5b.2-4-marks}

\textbf{Find equation of the line that passes through \((-3, 2)\) and
parallel to the line \(x - 2y + 1 = 0\)}

\textbf{Solution}: The given line \(x - 2y + 1 = 0\) has slope
\(m = \frac{1}{2}\)

Since parallel lines have the same slope, required line has slope
\(m = \frac{1}{2}\)

Using point-slope form: \(y - y_1 = m(x - x_1)\)

\(y - 2 = \frac{1}{2}(x - (-3))\)

\(y - 2 = \frac{1}{2}(x + 3)\)

\(2y - 4 = x + 3\)

\(x - 2y + 7 = 0\)

\subsubsection{Q5(B).3 [4 marks]}\label{q5b.3-4-marks}

\textbf{Find out center and radius of the circle:
\(x^2 + y^2 + 6x - 4y - 3 = 0\)}

\textbf{Solution}: Completing the square:

\(x^2 + 6x + y^2 - 4y = 3\)

\((x^2 + 6x + 9) + (y^2 - 4y + 4) = 3 + 9 + 4\)

\((x + 3)^2 + (y - 2)^2 = 16\)

\textbf{Center}: \((-3, 2)\) \textbf{Radius}: \(r = \sqrt{16} = 4\)

\begin{center}\rule{0.5\linewidth}{0.5pt}\end{center}

\subsection*{Formula Cheat Sheet}\label{formula-cheat-sheet}

\subsubsection{Logarithms}\label{logarithms}

\begin{itemize}
\tightlist
\item
  \(\log_a 1 = 0\)
\item
  \(\log_a a = 1\)\\
\item
  \(\log_a(xy) = \log_a x + \log_a y\)
\item
  \(\log_a\left(\frac{x}{y}\right) = \log_a x - \log_a y\)
\end{itemize}

\subsubsection{Trigonometry}\label{trigonometry}

\begin{itemize}
\tightlist
\item
  \(\sin^{-1}x + \cos^{-1}x = \frac{\pi}{2}\)
\item
  \(\tan(A \pm B) = \frac{\tan A \pm \tan B}{1 \mp \tan A \tan B}\)
\item
  \(\sin(180^\circ - x) = \sin x\), \(\cos(90^\circ + x) = -\sin x\)
\end{itemize}

\subsubsection{Vectors}\label{vectors}

\begin{itemize}
\tightlist
\item
  \(|\vec{a}| = \sqrt{a_1^2 + a_2^2 + a_3^2}\)
\item
  \(\vec{a} \cdot \vec{b} = |\vec{a}||\vec{b}|\cos θ\)
\item
  For perpendicular vectors: \(\vec{a} \cdot \vec{b} = 0\)
\end{itemize}

\subsubsection{Coordinate Geometry}\label{coordinate-geometry}

\begin{itemize}
\tightlist
\item
  Two-point form:
  \(\frac{y - y_1}{y_2 - y_1} = \frac{x - x_1}{x_2 - x_1}\)
\item
  Circle: \((x - h)^2 + (y - k)^2 = r^2\)
\item
  Parallel lines have equal slopes
\end{itemize}

\subsubsection{Limits}\label{limits}

\begin{itemize}
\tightlist
\item
  \(\lim_{x \to 0} \frac{\sin x}{x} = 1\)
\item
  \(\lim_{x \to 0} \frac{e^x - 1}{x} = 1\)
\item
  \(\lim_{x \to \infty} \frac{ax + b}{cx + d} = \frac{a}{c}\)
\end{itemize}

\subsection*{Problem-Solving
Strategies}\label{problem-solving-strategies}

\begin{enumerate}
\tightlist
\item
  \textbf{Logarithms}: Use properties to simplify expressions
\item
  \textbf{Trigonometry}: Apply compound angle formulas and identities
\item
  \textbf{Vectors}: Remember dot and cross product properties
\end{enumerate}

\subsection*{Common Mistakes to Avoid}\label{common-mistakes-to-avoid}

\subsubsection{Logarithms}\label{logarithms-1}

\begin{itemize}
\tightlist
\item
  \textbf{Mistake}: Confusing \(\log_a b\) with \(\log_b a\)
\item
  \textbf{Solution}: Remember change of base:
  \(\frac{1}{\log_a b} = \log_b a\)
\end{itemize}

\subsubsection{Trigonometry}\label{trigonometry-1}

\begin{itemize}
\tightlist
\item
  \textbf{Mistake}: Wrong angle conversions between degrees and radians
\item
  \textbf{Solution}: Always use \(180^\circ = \pi\) radians for conversion
\end{itemize}

\subsubsection{Vectors}\label{vectors-1}

\begin{itemize}
\tightlist
\item
  \textbf{Mistake}: Confusing dot product with cross product
\item
  \textbf{Solution}: Dot product gives scalar, cross product gives
  vector
\end{itemize}

\subsubsection{Limits}\label{limits-1}

\begin{itemize}
\tightlist
\item
  \textbf{Mistake}: Direct substitution in indeterminate forms
\item
  \textbf{Solution}: Use algebraic manipulation, L'Hôpital's rule, or
  standard limits
\end{itemize}

\subsubsection{Determinants}\label{determinants}

\begin{itemize}
\tightlist
\item
  \textbf{Mistake}: Sign errors in expansion
\item
  \textbf{Solution}: Follow the checkerboard pattern carefully
\end{itemize}

\subsection*{Exam Tips}\label{exam-tips}

\subsubsection{Time Management}\label{time-management}

\begin{itemize}
\tightlist
\item
  \textbf{Q1 (14 marks)}: 20-25 minutes - Quick calculations
\item
  \textbf{Q2-Q5}: 35-40 minutes each - Show all steps clearly
\end{itemize}

\subsubsection{Strategy}\label{strategy}

\begin{enumerate}
\tightlist
\item
  \textbf{Read all questions first} - Choose easier OR options
\item
  \textbf{Start with Q1} - Build confidence with MCQs
\item
  \textbf{Show work clearly} - Partial credit is available
\item
  \textbf{Use standard formulas} - Don't derive unless asked
\end{enumerate}

\subsubsection{Key Points to Remember}\label{key-points-to-remember}

\begin{itemize}
\tightlist
\item
  Always write the final answer clearly
\item
  Use proper mathematical notation
\item
  Draw diagrams where helpful
\item
  Check units in physics-related problems (work, force)
\end{itemize}

\subsubsection{Calculator Usage}\label{calculator-usage}

\begin{itemize}
\tightlist
\item
  Scientific calculator allowed
\item
  Use for complex arithmetic only
\item
  Show the setup before calculating
\item
  Round final answers appropriately
\end{itemize}

\subsubsection{Common Formula
Applications}\label{common-formula-applications}

\paragraph{Standard Limits (Memory
aids)}\label{standard-limits-memory-aids}

\begin{verbatim}
lim(x\rightarrow0) sin(x)/x = 1         "Sine over x is one"
lim(x\rightarrow0) (e^x - 1)/x = 1      "e minus one over x is one"  
lim(x\rightarrow0) (a^x - 1)/x = ln(a)  "General exponential form"
\end{verbatim}

\paragraph{Trigonometric Identities (Quick
Reference)}\label{trigonometric-identities-quick-reference}

\begin{center}
\textbf{Mermaid Diagram (Code)}
\begin{verbatim}
{Shaded}
{Highlighting}[]
graph LR
    A[sin^{2θ + cos^{2}θ = 1] {-}{-}{} B[1 + tan^{2}θ = sec^{2}θ]}
    A {-{-}{} C[1 + cot^{2}θ = cosec^{2}θ]}
    D["sin(A)"] {-{-}{} E[sin A cos B  cos A sin B]}
    F["cos(A)"] {-{-}{} G[cos A cos B  sin A sin B]}
{Highlighting}
{Shaded}
\end{verbatim}
\end{center}

\paragraph{Vector Operations
(Step-by-step)}\label{vector-operations-step-by-step}

\begin{enumerate}
\tightlist
\item
  \textbf{Magnitude}: \(|\vec{a}| = \sqrt{sum \, of \, squares}\)
\item
  \textbf{Dot Product}:
  \(\vec{a} \cdot \vec{b} = a_1b_1 + a_2b_2 + a_3b_3\)
\item
  \textbf{Angle}:
  \(\cos θ = \frac{\vec{a} \cdot \vec{b}}{|\vec{a}||\vec{b}|}\)
\end{enumerate}

\paragraph{Circle Equations (Forms)}\label{circle-equations-forms}

\begin{longtable}[]{@{}
  >{\raggedright\arraybackslash}p{(\linewidth - 4\tabcolsep) * \real{0.2069}}
  >{\raggedright\arraybackslash}p{(\linewidth - 4\tabcolsep) * \real{0.3448}}
  >{\raggedright\arraybackslash}p{(\linewidth - 4\tabcolsep) * \real{0.4483}}@{}}
\toprule\noalign{}
\begin{minipage}[b]{\linewidth}\raggedright
Form
\end{minipage} & \begin{minipage}[b]{\linewidth}\raggedright
Equation
\end{minipage} & \begin{minipage}[b]{\linewidth}\raggedright
When to Use
\end{minipage} \\
\midrule\noalign{}
\endhead
\bottomrule\noalign{}
\endlastfoot
Standard & \((x-h)^{2} + (y-k)^{2} = r^{2}\) & Given center and radius \\
General & \(x^{2} + y^{2} + Dx + Ey + F = 0\) & Need to find center/radius \\
Complete Square & \((x+D/2)^{2} + (y+E/2)^{2} = (D^{2}+E^{2}-4F)/4\) & Converting
general to standard \\
\end{longtable}

\subsubsection{Problem-Specific
Strategies}\label{problem-specific-strategies}

\paragraph{For Determinant Problems}\label{for-determinant-problems}

\begin{enumerate}
\tightlist
\item
  Look for zeros to simplify expansion
\item
  Use row/column operations if allowed
\item
  Remember: if two rows/columns are proportional, determinant = 0
\end{enumerate}

\paragraph{For Limit Problems}\label{for-limit-problems}

\begin{verbatim}
    Start with limit
         |
    Direct substitution?
      /        {}
    Yes         No (0/0, /, etc.)
     |           |
   Answer    Try factoring/
             rationalization
                 |
             Still indeterminate?
                 |
            L{Hôpitals Rule}
                 |
             Find answer
\end{verbatim}

\paragraph{For Vector Problems}\label{for-vector-problems}

\begin{itemize}
\tightlist
\item
  \textbf{Step 1}: Write vectors in component form
\item
  \textbf{Step 2}: Apply required operation (dot/cross product)
\item
  \textbf{Step 3}: Simplify and find magnitude if needed
\item
  \textbf{Step 4}: Check perpendicularity condition
  (\(\vec{a} \cdot \vec{b} = 0\))
\end{itemize}

\paragraph{For Coordinate Geometry}\label{for-coordinate-geometry}

\begin{itemize}
\tightlist
\item
  \textbf{Line problems}: Identify what's given (points, slope,
  parallel/perpendicular)
\item
  \textbf{Circle problems}: Identify center and radius from given
  information
\item
  \textbf{Always} check your equation by substituting known points
\end{itemize}

\subsubsection{Memory Techniques}\label{memory-techniques}

\paragraph{Logarithm Properties (MNEMONIC:
``PLUS'')}\label{logarithm-properties-mnemonic-plus}

\begin{itemize}
\tightlist
\item
  \textbf{P}roduct: \(\log(ab) = \log a + \log b\)
\item
  \textbf{L}imit: \(\log_a 1 = 0\)
\item
  \textbf{U}nity: \(\log_a a = 1\)\\
\item
  \textbf{S}ubtraction: \(\log(a/b) = \log a - \log b\)
\end{itemize}

\paragraph{Trigonometric Values (30^\circ, 45^\circ,
60^\circ)}\label{trigonometric-values-30-45-60}

\begin{longtable}[]{@{}llll@{}}
\toprule\noalign{}
Angle & sin & cos & tan \\
\midrule\noalign{}
\endhead
\bottomrule\noalign{}
\endlastfoot
30^\circ & 1/2 & \sqrt3/2 & 1/\sqrt3 \\
45^\circ & 1/\sqrt2 & 1/\sqrt2 & 1 \\
60^\circ & \sqrt3/2 & 1/2 & \sqrt3 \\
\end{longtable}

\textbf{Memory aid}: ``1, 2, 3'' under square roots for sin values (30^\circ
to 60^\circ)

\subsubsection{Final Review Checklist}\label{final-review-checklist}

Before submitting your paper:

\begin{itemize}
\tightlist
\item[$\square$]
  All questions attempted as required
\item[$\square$]
  Final answers clearly marked
\item[$\square$]
  Units included where applicable\\
\item[$\square$]
  No arithmetic errors in simple calculations
\item[$\square$]
  Proper mathematical notation used
\item[$\square$]
  Diagrams labeled clearly (if drawn)
\end{itemize}

\subsubsection{Quick Problem Solving
Guide}\label{quick-problem-solving-guide}

\paragraph{If you're stuck on a
problem:}\label{if-youre-stuck-on-a-problem}

\begin{enumerate}
\tightlist
\item
  \textbf{Read the problem again} - Often missed details become clear
\item
  \textbf{Try a different approach} - Multiple methods usually exist
\item
  \textbf{Work backwards} - Start from what you want to prove/find
\item
  \textbf{Use elimination} - In MCQs, eliminate obviously wrong options
\item
  \textbf{Move on and return} - Don't spend too much time on one problem
\end{enumerate}

\paragraph{Last 15 minutes strategy:}\label{last-15-minutes-strategy}

\begin{itemize}
\tightlist
\item
  Focus on completing MCQs in Q1
\item
  Check arithmetic in longer problems
\item
  Ensure all final answers are clearly marked
\item
  Review any skipped parts of questions
\end{itemize}

Remember: This exam tests fundamental concepts. Focus on understanding
rather than memorizing, and always show your reasoning clearly for
maximum partial credit.


\end{document}
