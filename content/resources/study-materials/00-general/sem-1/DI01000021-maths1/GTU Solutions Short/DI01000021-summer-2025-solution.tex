\documentclass{article}

% content/resources/templates/preamble.tex
\usepackage[margin=0.6in]{geometry}
\author{Milav Dabgar}
\usepackage{amsmath,amssymb,amsthm}
\usepackage{booktabs}
\usepackage{multirow}
\usepackage{xcolor}
\usepackage{tcolorbox}
\tcbuselibrary{breakable,skins}
\usepackage[colorlinks=true,linkcolor=blue]{hyperref}
\usepackage{titlesec}
\usepackage{enumitem}
\usepackage{tikz}
\usepackage{pgfplots}
\usepackage{circuitikz}
\usepackage[version=4]{mhchem}
\usepackage{longtable}
\usepackage{array}
\usepackage{float}
\usepackage{caption}
\usepackage{listings}

\lstset{
  basicstyle=\small\ttfamily,
  breaklines=true,
  breakatwhitespace=false,
  postbreak=\mbox{\textcolor{red}{$\hookrightarrow$}\space},
  float=false,
  numbers=left,
  numberstyle=\tiny\color{gray},
  numbersep=10pt,
  xleftmargin=2em,
  keywordstyle=\color{blue},
  commentstyle=\color{green!60!black},
  stringstyle=\color{purple},
  backgroundcolor=\color{gray!5},
  showstringspaces=false,
  tabsize=2,
  captionpos=b,
  keepspaces=true,
  columns=flexible
}

\pgfplotsset{compat=1.18}
\usetikzlibrary{shapes,arrows,positioning,calc,patterns,decorations.pathmorphing,decorations.markings,arrows.meta}

% Color scheme
\definecolor{headcolor}{RGB}{0,102,204}
\definecolor{keycolor}{RGB}{220,20,60}
\definecolor{solutioncolor}{RGB}{34,139,34}
\definecolor{mnemoniccolor}{RGB}{148,0,211}
\definecolor{codecolor}{RGB}{0,0,100}

% Spacing
\setlength{\parskip}{3pt}
\setlist[itemize]{nosep}
\setlist[enumerate]{nosep}

% Title formatting
\titleformat{\section}{\Large\bfseries\color{headcolor}}{\thesection}{1em}{}
\titleformat{\subsection}{\large\bfseries\color{headcolor}}{\thesubsection}{1em}{}

% Pandoc tightlist compatibility
\providecommand{\tightlist}{%
  \setlength{\itemsep}{0pt}\setlength{\parskip}{0pt}}

% Pandoc longtable compatibility
\newcounter{none}
\def\thenone{}


% content/resources/templates/english-boxes.tex

% Custom environments
\newtcolorbox{solutionbox}{
 breakable,
 enhanced,
 colback=solutioncolor!5!white,
 colframe=solutioncolor!75!black,
 fonttitle=\bfseries,
 title=Solution
}

\newtcolorbox{solutionboxnobreak}{
 colback=solutioncolor!5!white,
 colframe=solutioncolor!75!black,
 fonttitle=\bfseries,
 title=Solution
}

\newtcolorbox{keyformula}{
 breakable,
 enhanced,
 colback=keycolor!5!white,
 colframe=keycolor!75!black,
 fonttitle=\bfseries,
 title=Key Formula
}

\newtcolorbox{mnemonicboxenv}{
 breakable,
 enhanced,
 colback=mnemoniccolor!5!white,
 colframe=mnemoniccolor!75!black,
 fonttitle=\bfseries,
 title=Mnemonic
}

\newcommand{\mnemonicbox}[1]{%
  \begin{mnemonicboxenv}
    #1
  \end{mnemonicboxenv}
}


% Custom commands for GTU solutions
% This file defines semantic commands for consistent formatting

% Question command with automatic formatting
\newcommand{\question}[2]{%
  \section*{Question #1}%
  \textbf{#2}%
}

% OR question variant
\newcommand{\questionor}[2]{%
  \section*{Question #1 OR}%
  \textbf{#2}%
}

% Proper table environment with caption
\newenvironment{answertable}[1]{%
  \begin{table}[htbp]
  \centering
  \caption{#1}
}{%
  \end{table}
}

% Proper figure environment for diagrams
\newenvironment{answerdiagram}[1]{%
  \begin{figure}[htbp]
  \centering
  \caption{#1}
}{%
  \end{figure}
}

% Semantic markup for key terms
\newcommand{\keyword}[1]{\textbf{#1}}
\newcommand{\code}[1]{\texttt{#1}}
\newcommand{\classname}[1]{\texttt{#1}}
\newcommand{\methodname}[1]{\texttt{#1}}

% Proper quotation marks
\newcommand{\mnemonic}[1]{``#1''}


\title{Mathematics-I (DI01000021) - Summer 2025 Solution}
\date{May 30, 2025}

\begin{document}
\maketitle

\questionmarks{Q.1}{14}{Fill in the blanks/MCQs using appropriate choice from the given options}

\questionmarks{Q1.1}{1}{$\log_3 1 = $ \_\_\_\_}
\begin{solutionbox}
\textbf{Answer}: d. 0

\textbf{Solution}:
For any base $a > 0, a \neq 1$: $\log_a 1 = 0$
Therefore: $\log_3 1 = 0$
\end{solutionbox}

\questionmarks{Q1.2}{1}{If $f(x) = e^{x-1}$ then $f(1) = $ \_\_\_\_}
\begin{solutionbox}
\textbf{Answer}: c. 1

\textbf{Solution}:
$f(x) = e^{x-1}$
$f(1) = e^{1-1} = e^0 = 1$
\end{solutionbox}

\questionmarks{Q1.3}{1}{$\log_5 125 = $ \_\_\_\_}
\begin{solutionbox}
\textbf{Answer}: b. 3

\textbf{Solution}:
$\log_5 125 = \log_5 5^3 = 3$
Since $5^3 = 125$
\end{solutionbox}

\questionmarks{Q1.4}{1}{If $f(x) = x^3 - 7$ then $f(-2) = $ \_\_\_\_}
\begin{solutionbox}
\textbf{Answer}: c. -15

\textbf{Solution}:
$f(x) = x^3 - 7$
$f(-2) = (-2)^3 - 7 = -8 - 7 = -15$
\end{solutionbox}

\questionmarks{Q1.5}{1}{Principal period of $\cos x$ is \_\_\_\_}
\begin{solutionbox}
\textbf{Answer}: c. $2\pi$

\textbf{Solution}:
The cosine function repeats every $2\pi$ radians, so its principal period is $2\pi$.
\end{solutionbox}

\questionmarks{Q1.6}{1}{$150\textdegree = $ \_\_\_\_}
\begin{solutionbox}
\textbf{Answer}: a. $\frac{5\pi}{6}$

\textbf{Solution}:
Converting degrees to radians: $150\textdegree = 150 \times \frac{\pi}{180} = \frac{5\pi}{6}$
\end{solutionbox}

\questionmarks{Q1.7}{1}{$\sin^{-1}x + \cos^{-1}x = $ \_\_\_\_}
\begin{solutionbox}
\textbf{Answer}: a. $\frac{\pi}{2}$

\textbf{Solution}:
This is a standard identity: $\sin^{-1}x + \cos^{-1}x = \frac{\pi}{2}$ for $x \in [-1, 1]$
\end{solutionbox}

\questionmarks{Q1.8}{1}{$(1,0,0) \times (1,0,0) = $ \_\_\_\_}
\begin{solutionbox}
\textbf{Answer}: d. (0,0,0)

\textbf{Solution}:
Cross product of any vector with itself is zero vector:
$(1,0,0) \times (1,0,0) = (0,0,0)$
\end{solutionbox}

\questionmarks{Q1.9}{1}{If $\vec{a} = 4\hat{i} - 3\hat{j}$ then $|\vec{a}| = $ \_\_\_\_}
\begin{solutionbox}
\textbf{Answer}: b. 5

\textbf{Solution}:
$|\vec{a}| = \sqrt{4^2 + (-3)^2} = \sqrt{16 + 9} = \sqrt{25} = 5$
\end{solutionbox}

\questionmarks{Q1.10}{1}{If a line makes an angle $45\textdegree$ with positive x-axis then slope of the line is \_\_\_\_}
\begin{solutionbox}
\textbf{Answer}: c. 1

\textbf{Solution}:
Slope $m = \tan(45\textdegree) = 1$
\end{solutionbox}

\questionmarks{Q1.11}{1}{Radius of the circle $x^2 + y^2 = 4$ is \_\_\_\_}
\begin{solutionbox}
\textbf{Answer}: d. 2

\textbf{Solution}:
Standard form: $x^2 + y^2 = r^2$
Comparing: $r^2 = 4$, so $r = 2$
\end{solutionbox}

\questionmarks{Q1.12}{1}{$\lim\limits_{x \to 0} \frac{e^x - 1}{x} = $ \_\_\_\_}
\begin{solutionbox}
\textbf{Answer}: a. 1

\textbf{Solution}:
This is a standard limit: $\lim\limits_{x \to 0} \frac{e^x - 1}{x} = 1$
\end{solutionbox}

\questionmarks{Q1.13}{1}{$\lim\limits_{x \to 0} \frac{\sin 3x}{x} = $ \_\_\_\_}
\begin{solutionbox}
\textbf{Answer}: d. 3

\textbf{Solution}:
$\lim\limits_{x \to 0} \frac{\sin 3x}{x} = \lim\limits_{x \to 0} \frac{\sin 3x}{3x} \times 3 = 1 \times 3 = 3$
\end{solutionbox}

\questionmarks{Q1.14}{1}{$\lim\limits_{n \to \infty} \frac{5n + 4}{4n + 5} = $ \_\_\_\_}
\begin{solutionbox}
\textbf{Answer}: c. 5/4

\textbf{Solution}:
$\lim\limits_{n \to \infty} \frac{5n + 4}{4n + 5} = \lim\limits_{n \to \infty} \frac{5 + \frac{4}{n}}{4 + \frac{5}{n}} = \frac{5}{4}$
\end{solutionbox}

\questionmarks{Q.2 (A)}{6}{Attempt any two}

\questionmarks{Q2(A).1}{3}{Find value: $\begin{vmatrix} 1 & 2 & 3 \\ 4 & 5 & 6 \\ 7 & 8 & 9 \end{vmatrix}$}
\begin{solutionbox}
\textbf{Answer}: 0

\textbf{Solution}:
$\begin{vmatrix} 1 & 2 & 3 \\ 4 & 5 & 6 \\ 7 & 8 & 9 \end{vmatrix} = 1(5 \times 9 - 6 \times 8) - 2(4 \times 9 - 6 \times 7) + 3(4 \times 8 - 5 \times 7)$
$= 1(45 - 48) - 2(36 - 42) + 3(32 - 35)$
$= 1(-3) - 2(-6) + 3(-3)$
$= -3 + 12 - 9 = 0$
\end{solutionbox}

\questionmarks{Q2(A).2}{3}{Prove that: $\log\left(\frac{x^p}{x^q}\right) + \log\left(\frac{x^q}{x^r}\right) + \log\left(\frac{x^r}{x^p}\right) = 0$}
\begin{solutionbox}
\textbf{Solution}:
LHS = $\log\left(\frac{x^p}{x^q}\right) + \log\left(\frac{x^q}{x^r}\right) + \log\left(\frac{x^r}{x^p}\right)$

Using logarithm properties:
$= \log(x^p) - \log(x^q) + \log(x^q) - \log(x^r) + \log(x^r) - \log(x^p)$
$= p\log x - q\log x + q\log x - r\log x + r\log x - p\log x$
$= 0$ = RHS
\end{solutionbox}

\questionmarks{Q2(A).3}{3}{Find value: $\tan(75\textdegree)$}
\begin{solutionbox}
\textbf{Answer}: $2 + \sqrt{3}$

\textbf{Solution}:
$\tan(75\textdegree) = \tan(45\textdegree + 30\textdegree)$

Using $\tan(A + B) = \frac{\tan A + \tan B}{1 - \tan A \tan B}$:

$\tan(75\textdegree) = \frac{\tan 45\textdegree + \tan 30\textdegree}{1 - \tan 45\textdegree \tan 30\textdegree} = \frac{1 + \frac{1}{\sqrt{3}}}{1 - 1 \times \frac{1}{\sqrt{3}}} = \frac{1 + \frac{1}{\sqrt{3}}}{1 - \frac{1}{\sqrt{3}}}$

$= \frac{\frac{\sqrt{3} + 1}{\sqrt{3}}}{\frac{\sqrt{3} - 1}{\sqrt{3}}} = \frac{\sqrt{3} + 1}{\sqrt{3} - 1} = \frac{(\sqrt{3} + 1)^2}{(\sqrt{3} - 1)(\sqrt{3} + 1)} = \frac{3 + 2\sqrt{3} + 1}{3 - 1} = \frac{4 + 2\sqrt{3}}{2} = 2 + \sqrt{3}$
\end{solutionbox}

\questionmarks{Q.2 (B)}{8}{Attempt any two}

\questionmarks{Q2(B).1}{4}{Prove that: $\frac{1}{\log_{12} 120} + \frac{1}{\log_2 120} + \frac{1}{\log_5 120} = 1$}
\begin{solutionbox}
\textbf{Solution}:
Using change of base formula: $\frac{1}{\log_a b} = \log_b a$

LHS = $\log_{120} 12 + \log_{120} 2 + \log_{120} 5$

Using logarithm properties:
$= \log_{120}(12 \times 2 \times 5) = \log_{120} 120 = 1$ = RHS
\end{solutionbox}

\questionmarks{Q2(B).2}{4}{Solve: $\begin{vmatrix} x & 1 & 1 \\ 1 & 2 & 1 \\ 0 & 0 & 3 \end{vmatrix} = 3$}
\begin{solutionbox}
\textbf{Solution}:
Expanding along third row:
$\begin{vmatrix} x & 1 & 1 \\ 1 & 2 & 1 \\ 0 & 0 & 3 \end{vmatrix} = 3 \begin{vmatrix} x & 1 \\ 1 & 2 \end{vmatrix}$

$= 3(2x - 1) = 6x - 3$

Given: $6x - 3 = 3$
$6x = 6$
$x = 1$
\end{solutionbox}

\questionmarks{Q2(B).3}{4}{If $f(x) = \frac{1-x}{1+x}$ prove that: (i) $f(x) + f\left(\frac{1}{x}\right) = 0$ (ii) $f(x) \times f(-x) = 1$}
\begin{solutionbox}
\textbf{Solution}:
Given: $f(x) = \frac{1-x}{1+x}$

\textbf{(i)} $f\left(\frac{1}{x}\right) = \frac{1-\frac{1}{x}}{1+\frac{1}{x}} = \frac{\frac{x-1}{x}}{\frac{x+1}{x}} = \frac{x-1}{x+1} = -\frac{1-x}{1+x} = -f(x)$

Therefore: $f(x) + f\left(\frac{1}{x}\right) = f(x) + (-f(x)) = 0$

\textbf{(ii)} $f(-x) = \frac{1-(-x)}{1+(-x)} = \frac{1+x}{1-x}$

$f(x) \times f(-x) = \frac{1-x}{1+x} \times \frac{1+x}{1-x} = 1$
\end{solutionbox}

\questionmarks{Q.3 (A)}{6}{Attempt any two}

\questionmarks{Q3(A).1}{3}{Prove that: $\frac{\sin(180\textdegree - x) + \cosec(180\textdegree - x) + \tan(180\textdegree + x)}{\cos(90\textdegree + x) + \sec(90\textdegree + x) + \cot(90\textdegree + x)} = -3$}
\begin{solutionbox}
\textbf{Solution}:
Using trigonometric identities:

\begin{itemize}
\item $\sin(180\textdegree - x) = \sin x$
\item $\cosec(180\textdegree - x) = \cosec x$
\item $\tan(180\textdegree + x) = \tan x$
\item $\cos(90\textdegree + x) = -\sin x$
\item $\sec(90\textdegree + x) = -\cosec x$
\item $\cot(90\textdegree + x) = -\tan x$
\end{itemize}

Numerator = $\sin x + \cosec x + \tan x$
Denominator = $-\sin x - \cosec x - \tan x = -(\sin x + \cosec x + \tan x)$

Therefore: $\frac{\sin x + \cosec x + \tan x}{-(\sin x + \cosec x + \tan x)} = -1 \neq -3$

\textbf{Note}: There appears to be an error in the problem statement or expected answer.
\end{solutionbox}

\questionmarks{Q3(A).2}{3}{Prove that: $\tan^{-1}\left(\frac{1}{3}\right) + \tan^{-1}\left(\frac{1}{2}\right) = 45\textdegree$}
\begin{solutionbox}
\textbf{Solution}:
Using $\tan^{-1}A + \tan^{-1}B = \tan^{-1}\left(\frac{A+B}{1-AB}\right)$:

$\tan^{-1}\left(\frac{1}{3}\right) + \tan^{-1}\left(\frac{1}{2}\right) = \tan^{-1}\left(\frac{\frac{1}{3} + \frac{1}{2}}{1 - \frac{1}{3} \times \frac{1}{2}}\right)$

$= \tan^{-1}\left(\frac{\frac{5}{6}}{1 - \frac{1}{6}}\right) = \tan^{-1}\left(\frac{\frac{5}{6}}{\frac{5}{6}}\right) = \tan^{-1}(1) = 45\textdegree$
\end{solutionbox}

\questionmarks{Q3(A).3}{3}{Find out equation of the line whose X-intercept is 3 and Y-intercept is 2.}
\begin{solutionbox}
\textbf{Solution}:
Using intercept form: $\frac{x}{a} + \frac{y}{b} = 1$

Where $a = 3$ (x-intercept) and $b = 2$ (y-intercept)

$\frac{x}{3} + \frac{y}{2} = 1$

Multiplying by 6: $2x + 3y = 6$
\end{solutionbox}

\questionmarks{Q.3 (B)}{8}{Attempt any two}

\questionmarks{Q3(B).1}{4}{Prove that: $\tan(70\textdegree) = \frac{\cos(25\textdegree) + \sin(25\textdegree)}{\cos(25\textdegree) - \sin(25\textdegree)}$}
\begin{solutionbox}
\textbf{Solution}:
RHS = $\frac{\cos(25\textdegree) + \sin(25\textdegree)}{\cos(25\textdegree) - \sin(25\textdegree)}$

Dividing numerator and denominator by $\cos(25\textdegree)$:

$= \frac{1 + \tan(25\textdegree)}{1 - \tan(25\textdegree)}$

Using $\tan(45\textdegree + \theta) = \frac{1 + \tan \theta}{1 - \tan \theta}$:

$= \tan(45\textdegree + 25\textdegree) = \tan(70\textdegree)$ = LHS
\end{solutionbox}

\questionmarks{Q3(B).2}{4}{Prove that: $\frac{\sin \theta + \sin 2\theta + \sin 3\theta}{\cos \theta + \cos 2\theta + \cos 3\theta} = \tan 2\theta$}
\begin{solutionbox}
\textbf{Solution}:
Using sum-to-product formulas:

Numerator: $\sin \theta + \sin 3\theta + \sin 2\theta = 2\sin 2\theta \cos \theta + \sin 2\theta = \sin 2\theta(2\cos \theta + 1)$

Denominator: $\cos \theta + \cos 3\theta + \cos 2\theta = 2\cos 2\theta \cos \theta + \cos 2\theta = \cos 2\theta(2\cos \theta + 1)$

Therefore: $\frac{\sin 2\theta(2\cos \theta + 1)}{\cos 2\theta(2\cos \theta + 1)} = \frac{\sin 2\theta}{\cos 2\theta} = \tan 2\theta$
\end{solutionbox}

\questionmarks{Q3(B).3}{4}{If $\vec{a} = (1,2,3)$, $\vec{b} = (4,0,0)$ and $\vec{c} = (2,0,1)$ find $2\vec{a} + 3\vec{b} - 5\vec{c}$}
\begin{solutionbox}
\textbf{Solution}:
$2\vec{a} = 2(1,2,3) = (2,4,6)$
$3\vec{b} = 3(4,0,0) = (12,0,0)$
$5\vec{c} = 5(2,0,1) = (10,0,5)$

$2\vec{a} + 3\vec{b} - 5\vec{c} = (2,4,6) + (12,0,0) - (10,0,5)$
$= (2+12-10, 4+0-0, 6+0-5)$
$= (4,4,1)$
\end{solutionbox}

\questionmarks{Q.4 (A)}{6}{Attempt any two}

\questionmarks{Q4(A).1}{3}{If the vectors $\vec{a} = \hat{i} - 2\hat{j} + 3\hat{k}$ and $\vec{b} = 2\hat{i} + m\hat{j} - 4\hat{k}$ are perpendicular, find m.}
\begin{solutionbox}
\textbf{Solution}:
For perpendicular vectors: $\vec{a} \cdot \vec{b} = 0$

$\vec{a} \cdot \vec{b} = (1)(2) + (-2)(m) + (3)(-4) = 2 - 2m - 12 = -10 - 2m$

Setting equal to zero: $-10 - 2m = 0$
$2m = -10$
$m = -5$
\end{solutionbox}

\questionmarks{Q4(A).2}{3}{Find the direction cosines and direction angles of the vector $\vec{a} = 5\hat{i} - 12\hat{k}$}
\begin{solutionbox}
\textbf{Solution}:
$\vec{a} = 5\hat{i} + 0\hat{j} - 12\hat{k}$

Magnitude: $|\vec{a}| = \sqrt{5^2 + 0^2 + (-12)^2} = \sqrt{25 + 144} = \sqrt{169} = 13$

Direction cosines:
\begin{itemize}
\item $l = \frac{5}{13}$
\item $m = \frac{0}{13} = 0$
\item $n = \frac{-12}{13}$
\end{itemize}

Direction angles:
\begin{itemize}
\item $\alpha = \cos^{-1}\left(\frac{5}{13}\right)$
\item $\beta = \cos^{-1}(0) = 90\textdegree$
\item $\gamma = \cos^{-1}\left(\frac{-12}{13}\right)$
\end{itemize}
\end{solutionbox}

\questionmarks{Q4(A).3}{3}{Find out equation of the circle having center at $(2, -3)$ and radius 3.}
\begin{solutionbox}
\textbf{Solution}:
Standard form: $(x - h)^2 + (y - k)^2 = r^2$

Where $(h, k) = (2, -3)$ and $r = 3$

$(x - 2)^2 + (y + 3)^2 = 9$

Expanding: $x^2 - 4x + 4 + y^2 + 6y + 9 = 9$
$x^2 + y^2 - 4x + 6y + 4 = 0$
\end{solutionbox}

\questionmarks{Q.4 (B)}{8}{Attempt any two}

\questionmarks{Q4(B).1}{4}{Show that the angle between vectors $\vec{a} = \hat{i} + 2\hat{j}$ and $\vec{b} = \hat{i} + \hat{j} + 3\hat{k}$ is $\sin^{-1}\sqrt{\frac{46}{55}}$}
\begin{solutionbox}
\textbf{Solution}:
$\vec{a} \cdot \vec{b} = (1)(1) + (2)(1) + (0)(3) = 1 + 2 = 3$

$|\vec{a}| = \sqrt{1^2 + 2^2} = \sqrt{5}$
$|\vec{b}| = \sqrt{1^2 + 1^2 + 3^2} = \sqrt{11}$

$\cos \theta = \frac{\vec{a} \cdot \vec{b}}{|\vec{a}||\vec{b}|} = \frac{3}{\sqrt{5}\sqrt{11}} = \frac{3}{\sqrt{55}}$

$\sin^2 \theta = 1 - \cos^2 \theta = 1 - \frac{9}{55} = \frac{46}{55}$

Therefore: $\theta = \sin^{-1}\sqrt{\frac{46}{55}}$
\end{solutionbox}

\questionmarks{Q4(B).2}{4}{Under effect of the forces $2\hat{i} + \hat{j} + \hat{k}$ and $\hat{i} + 3\hat{j} - \hat{k}$ a particle moves from the point $(1,2,-3)$ to the point $(5,3,7)$. Find out work done.}
\begin{solutionbox}
\textbf{Solution}:
Net force: $\vec{F} = (2\hat{i} + \hat{j} + \hat{k}) + (\hat{i} + 3\hat{j} - \hat{k}) = 3\hat{i} + 4\hat{j}$

Displacement: $\vec{s} = (5,3,7) - (1,2,-3) = (4,1,10)$

Work done: $W = \vec{F} \cdot \vec{s} = (3)(4) + (4)(1) + (0)(10) = 12 + 4 = 16$ units
\end{solutionbox}

\questionmarks{Q4(B).3}{4}{Evaluate: $\lim\limits_{x \to 0} \frac{2^x - 5^x}{x}$}
\begin{solutionbox}
\textbf{Solution}:
Using L'Hôpital's rule or the derivative definition:

$\lim\limits_{x \to 0} \frac{2^x - 5^x}{x} = \lim\limits_{x \to 0} \frac{2^x \ln 2 - 5^x \ln 5}{1}$

$= 2^0 \ln 2 - 5^0 \ln 5 = \ln 2 - \ln 5 = \ln\left(\frac{2}{5}\right)$
\end{solutionbox}

\questionmarks{Q.5 (A)}{6}{Attempt any two}

\questionmarks{Q5(A).1}{3}{Evaluate: $\lim\limits_{x \to 0} \left(1 + \frac{3x}{7}\right)^{\frac{1}{x}}$}
\begin{solutionbox}
\textbf{Solution}:
Let $y = \left(1 + \frac{3x}{7}\right)^{\frac{1}{x}}$

Taking natural log: $\ln y = \frac{1}{x} \ln\left(1 + \frac{3x}{7}\right)$

$\lim\limits_{x \to 0} \ln y = \lim\limits_{x \to 0} \frac{\ln\left(1 + \frac{3x}{7}\right)}{x}$

Using L'Hôpital's rule: $= \lim\limits_{x \to 0} \frac{\frac{3/7}{1 + \frac{3x}{7}}}{1} = \frac{3}{7}$

Therefore: $\lim\limits_{x \to 0} y = e^{3/7}$
\end{solutionbox}

\questionmarks{Q5(A).2}{3}{Evaluate: $\lim\limits_{x \to 3} \frac{x^2 - 5x + 6}{x^2 - 9}$}
\begin{solutionbox}
\textbf{Solution}:
Factoring numerator: $x^2 - 5x + 6 = (x-2)(x-3)$
Factoring denominator: $x^2 - 9 = (x-3)(x+3)$

$\lim\limits_{x \to 3} \frac{x^2 - 5x + 6}{x^2 - 9} = \lim\limits_{x \to 3} \frac{(x-2)(x-3)}{(x-3)(x+3)} = \lim\limits_{x \to 3} \frac{x-2}{x+3} = \frac{3-2}{3+3} = \frac{1}{6}$
\end{solutionbox}

\questionmarks{Q5(A).3}{3}{Evaluate: $\lim\limits_{x \to 0} \frac{\sqrt{4+x} - 2}{x}$}
\begin{solutionbox}
\textbf{Solution}:
Rationalizing the numerator:

$\lim\limits_{x \to 0} \frac{\sqrt{4+x} - 2}{x} \times \frac{\sqrt{4+x} + 2}{\sqrt{4+x} + 2}$

$= \lim\limits_{x \to 0} \frac{(4+x) - 4}{x(\sqrt{4+x} + 2)} = \lim\limits_{x \to 0} \frac{x}{x(\sqrt{4+x} + 2)} = \lim\limits_{x \to 0} \frac{1}{\sqrt{4+x} + 2} = \frac{1}{2+2} = \frac{1}{4}$
\end{solutionbox}

\questionmarks{Q.5 (B)}{8}{Attempt any two}

\questionmarks{Q5(B).1}{4}{Find out equation of the line passing through points $(1,2)$ and $(2,1)$.}
\begin{solutionbox}
\textbf{Solution}:
Using two-point form: $\frac{y - y_1}{y_2 - y_1} = \frac{x - x_1}{x_2 - x_1}$

$\frac{y - 2}{1 - 2} = \frac{x - 1}{2 - 1}$

$\frac{y - 2}{-1} = \frac{x - 1}{1}$

$y - 2 = -(x - 1) = -x + 1$

$x + y = 3$
\end{solutionbox}

\questionmarks{Q5(B).2}{4}{Find equation of the line that passes through $(-3, 2)$ and parallel to the line $x - 2y + 1 = 0$}
\begin{solutionbox}
\textbf{Solution}:
The given line $x - 2y + 1 = 0$ has slope $m = \frac{1}{2}$

Since parallel lines have the same slope, required line has slope $m = \frac{1}{2}$

Using point-slope form: $y - y_1 = m(x - x_1)$

$y - 2 = \frac{1}{2}(x - (-3))$

$y - 2 = \frac{1}{2}(x + 3)$

$2y - 4 = x + 3$

$x - 2y + 7 = 0$
\end{solutionbox}

\questionmarks{Q5(B).3}{4}{Find out center and radius of the circle: $x^2 + y^2 + 6x - 4y - 3 = 0$}
\begin{solutionbox}
\textbf{Solution}:
Completing the square:

$x^2 + 6x + y^2 - 4y = 3$

$(x^2 + 6x + 9) + (y^2 - 4y + 4) = 3 + 9 + 4$

$(x + 3)^2 + (y - 2)^2 = 16$

\textbf{Center}: $(-3, 2)$
\textbf{Radius}: $r = \sqrt{16} = 4$
\end{solutionbox}

\newpage
\section*{Formula Cheat Sheet}

\subsection*{Logarithms}
\begin{itemize}
\item $\log_a 1 = 0$
\item $\log_a a = 1$
\item $\log_a(xy) = \log_a x + \log_a y$
\item $\log_a\left(\frac{x}{y}\right) = \log_a x - \log_a y$
\end{itemize}

\subsection*{Trigonometry}
\begin{itemize}
\item $\sin^{-1}x + \cos^{-1}x = \frac{\pi}{2}$
\item $\tan(A \pm B) = \frac{\tan A \pm \tan B}{1 \mp \tan A \tan B}$
\item $\sin(180\textdegree - x) = \sin x$, $\cos(90\textdegree + x) = -\sin x$
\end{itemize}

\textbf{Trigonometric Identities Map}:

\begin{figure}[h]
\centering
\begin{tikzpicture}[
    node distance=2cm,
    level distance=3.5cm,
    every node/.style={gtu block, align=center, text width=4cm, font=\small}
]
    \node (id1) {$\sin^2\theta + \cos^2\theta = 1$};
    \node (id2) [right of=id1, xshift=3cm] {$1 + \tan^2\theta = \sec^2\theta$};
    \node (id3) [below of=id2] {$1 + \cot^2\theta = \cosec^2\theta$};
    
    \node (sumsin) [below of=id1] {$\sin(A\pm B)$};
    \node (expsin) [right of=sumsin, xshift=3cm] {$\sin A \cos B \pm \cos A \sin B$};
    
    \node (sumcos) [below of=sumsin] {$\cos(A\pm B)$};
    \node (expcos) [right of=sumcos, xshift=3cm] {$\cos A \cos B \mp \sin A \sin B$};

    \draw [gtu arrow] (id1) -- (id2);
    \draw [gtu arrow] (id1) |- (id3);
    \draw [gtu arrow] (sumsin) -- (expsin);
    \draw [gtu arrow] (sumcos) -- (expcos);

\end{tikzpicture}
\caption{Trigonometric Identities}
\end{figure}

\subsection*{Vectors}
\begin{itemize}
\item $|\vec{a}| = \sqrt{a_1^2 + a_2^2 + a_3^2}$
\item $\vec{a} \cdot \vec{b} = |\vec{a}||\vec{b}|\cos \theta$
\item For perpendicular vectors: $\vec{a} \cdot \vec{b} = 0$
\end{itemize}

\subsection*{Limits Strategy}

\begin{figure}[h]
\centering
\begin{tikzpicture}[node distance=2.5cm, auto]
    \node [gtu block] (start) {Start with limit};
    \node [gtu decision, aspect=2, below of=start] (sub) {Direct substitution?};
    \node [gtu block, below left of=sub, xshift=-2cm] (yes) {Answer};
    \node [gtu block, below right of=sub, xshift=2cm] (no) {Indeterminate Form (0/0, $\infty/\infty$)};
    \node [gtu block, below of=no] (factor) {Try factoring/rationalization};
    \node [gtu decision, aspect=2, below of=factor] (check) {Still indeterminate?};
    \node [gtu block, below of=check] (lhop) {L'Hôpital's Rule};
    \node [gtu block, below of=lhop] (ans) {Find answer};

    \path [gtu arrow] (start) -- (sub);
    \path [gtu arrow] (sub) -- node {Yes} (yes);
    \path [gtu arrow] (sub) -- node {No} (no);
    \path [gtu arrow] (no) -- (factor);
    \path [gtu arrow] (factor) -- (check);
    \path [gtu arrow] (check.east) -- ++(1,0) |- node [near start] {No} (ans.east);
    \path [gtu arrow] (check) -- node {Yes} (lhop);
    \path [gtu arrow] (lhop) -- (ans);
    
\end{tikzpicture}
\caption{Limits Solving Strategy}
\end{figure}

\subsection*{Coordinate Geometry}
\begin{itemize}
\item Two-point form: $\frac{y - y_1}{y_2 - y_1} = \frac{x - x_1}{x_2 - x_1}$
\item Circle: $(x - h)^2 + (y - k)^2 = r^2$
\item Parallel lines have equal slopes
\end{itemize}

\subsection*{Memory Techniques (Logarithms)}
\begin{mnemonicbox}
\mnemonic{PLUS: Product, Limit, Unity, Subtraction}
\begin{itemize}
\item \textbf{P}roduct: $\log(ab) = \log a + \log b$
\item \textbf{L}imit: $\log_a 1 = 0$
\item \textbf{U}nity: $\log_a a = 1$
\item \textbf{S}ubtraction: $\log(a/b) = \log a - \log b$
\end{itemize}
\end{mnemonicbox}

\end{document}

