\documentclass[a4paper,12pt]{article}
\usepackage[utf8]{inputenc}
\usepackage[T1]{fontenc}
\usepackage[margin=1in]{geometry}
\usepackage{amsmath, amssymb, mathtools}
\usepackage{tcolorbox}
\usepackage{enumitem}
\usepackage{fancyhdr}
\usepackage{titlesec}
\usepackage{tikz}
\usepackage{pgfplots}
\usepackage{hyperref}
\usepackage{longtable}
\usepackage{array}

\pgfplotsset{compat=1.18}

% Header and Footer
\pagestyle{fancy}
\fancyhf{}
\fancyhead[L]{Mathematics-I (DI01000021)}
\fancyhead[R]{Summer 2025 Solution}
\fancyfoot[C]{\thepage}

% Title Formatting
\titleformat{\section}{\large\bfseries}{\thesection}{1em}{}
\titleformat{\subsection}{\bfseries}{\thesubsection}{1em}{}

% Custom Environments
\newtcolorbox{solutionbox}{
    colback=gray!5,
    colframe=gray!40,
    boxrule=0.5pt,
    arc=2pt,
    left=5pt, right=5pt, top=5pt, bottom=5pt
}

\begin{document}

\begin{center}
    {\Large\textbf{Mathematics-I (DI01000021) - Summer 2025 Solution}}\\[0.5cm]
    \textit{Date: 2025-05-30}
\end{center}

\section*{Q.1 [14 marks]}
\textbf{Fill in the blanks/MCQs using appropriate choice from the given options}

\subsection*{Q1.1 [1 mark]}
$\log_3 1 = $ \underline{\hspace{2cm}}

\textbf{Answer}: d. 0

\begin{solutionbox}
\textbf{Solution}:\\
For any base $a > 0, a \neq 1$: $\log_a 1 = 0$\\
Therefore: $\log_3 1 = 0$
\end{solutionbox}

\subsection*{Q1.2 [1 mark]}
If $f(x) = e^{x-1}$ then $f(1) = $ \underline{\hspace{2cm}}

\textbf{Answer}: c. 1

\begin{solutionbox}
\textbf{Solution}:\\
$f(x) = e^{x-1}$\\
$f(1) = e^{1-1} = e^0 = 1$
\end{solutionbox}

\subsection*{Q1.3 [1 mark]}
$\log_5 125 = $ \underline{\hspace{2cm}}

\textbf{Answer}: b. 3

\begin{solutionbox}
\textbf{Solution}:\\
$\log_5 125 = \log_5 5^3 = 3$\\
Since $5^3 = 125$
\end{solutionbox}

\subsection*{Q1.4 [1 mark]}
If $f(x) = x^3 - 7$ then $f(-2) = $ \underline{\hspace{2cm}}

\textbf{Answer}: c. -15

\begin{solutionbox}
\textbf{Solution}:\\
$f(x) = x^3 - 7$\\
$f(-2) = (-2)^3 - 7 = -8 - 7 = -15$
\end{solutionbox}

\subsection*{Q1.5 [1 mark]}
Principal period of $\cos x$ is \underline{\hspace{2cm}}

\textbf{Answer}: c. $2\pi$

\begin{solutionbox}
\textbf{Solution}:\\
The cosine function repeats every $2\pi$ radians, so its principal period is $2\pi$.
\end{solutionbox}

\subsection*{Q1.6 [1 mark]}
$150^\circ = $ \underline{\hspace{2cm}}

\textbf{Answer}: a. $\frac{5\pi}{6}$

\begin{solutionbox}
\textbf{Solution}:\\
Converting degrees to radians: $150^\circ = 150 \times \frac{\pi}{180} = \frac{5\pi}{6}$
\end{solutionbox}

\subsection*{Q1.7 [1 mark]}
$\sin^{-1}x + \cos^{-1}x = $ \underline{\hspace{2cm}}

\textbf{Answer}: a. $\frac{\pi}{2}$

\begin{solutionbox}
\textbf{Solution}:\\
This is a standard identity: $\sin^{-1}x + \cos^{-1}x = \frac{\pi}{2}$ for $x \in [-1, 1]$
\end{solutionbox}

\subsection*{Q1.8 [1 mark]}
$(1,0,0) \times (1,0,0) = $ \underline{\hspace{2cm}}

\textbf{Answer}: d. (0,0,0)

\begin{solutionbox}
\textbf{Solution}:\\
Cross product of any vector with itself is zero vector:\\
$(1,0,0) \times (1,0,0) = (0,0,0)$
\end{solutionbox}

\subsection*{Q1.9 [1 mark]}
If $\vec{a} = 4\hat{i} - 3\hat{j}$ then $|\vec{a}| = $ \underline{\hspace{2cm}}

\textbf{Answer}: b. 5

\begin{solutionbox}
\textbf{Solution}:\\
$|\vec{a}| = \sqrt{4^2 + (-3)^2} = \sqrt{16 + 9} = \sqrt{25} = 5$
\end{solutionbox}

\subsection*{Q1.10 [1 mark]}
If a line makes an angle $45^\circ$ with positive x-axis then slope of the line is \underline{\hspace{2cm}}

\textbf{Answer}: c. 1

\begin{solutionbox}
\textbf{Solution}:\\
Slope $m = \tan(45^\circ) = 1$
\end{solutionbox}

\subsection*{Q1.11 [1 mark]}
Radius of the circle $x^2 + y^2 = 4$ is \underline{\hspace{2cm}}

\textbf{Answer}: d. 2

\begin{solutionbox}
\textbf{Solution}:\\
Standard form: $x^2 + y^2 = r^2$\\
Comparing: $r^2 = 4$, so $r = 2$
\end{solutionbox}

\subsection*{Q1.12 [1 mark]}
$\lim_{x \to 0} \frac{e^x - 1}{x} = $ \underline{\hspace{2cm}}

\textbf{Answer}: a. 1

\begin{solutionbox}
\textbf{Solution}:\\
This is a standard limit: $\lim_{x \to 0} \frac{e^x - 1}{x} = 1$
\end{solutionbox}

\subsection*{Q1.13 [1 mark]}
$\lim_{x \to 0} \frac{\sin 3x}{x} = $ \underline{\hspace{2cm}}

\textbf{Answer}: d. 3

\begin{solutionbox}
\textbf{Solution}:\\
$\lim_{x \to 0} \frac{\sin 3x}{x} = \lim_{x \to 0} \frac{\sin 3x}{3x} \times 3 = 1 \times 3 = 3$
\end{solutionbox}

\subsection*{Q1.14 [1 mark]}
$\lim_{n \to \infty} \frac{5n + 4}{4n + 5} = $ \underline{\hspace{2cm}}

\textbf{Answer}: c. 5/4

\begin{solutionbox}
\textbf{Solution}:\\
$\lim_{n \to \infty} \frac{5n + 4}{4n + 5} = \lim_{n \to \infty} \frac{5 + \frac{4}{n}}{4 + \frac{5}{n}} = \frac{5}{4}$
\end{solutionbox}

\section*{Q.2 (A) [6 marks]}
\textbf{Attempt any two}

\subsection*{Q2(A).1 [3 marks]}
\textbf{Find value: $\begin{vmatrix} 1 & 2 & 3 \\ 4 & 5 & 6 \\ 7 & 8 & 9 \end{vmatrix}$}

\textbf{Answer}: 0

\begin{solutionbox}
\textbf{Solution}:\\
$\begin{vmatrix} 1 & 2 & 3 \\ 4 & 5 & 6 \\ 7 & 8 & 9 \end{vmatrix} = 1(5 \times 9 - 6 \times 8) - 2(4 \times 9 - 6 \times 7) + 3(4 \times 8 - 5 \times 7)$\\
$= 1(45 - 48) - 2(36 - 42) + 3(32 - 35)$\\
$= 1(-3) - 2(-6) + 3(-3)$\\
$= -3 + 12 - 9 = 0$
\end{solutionbox}

\subsection*{Q2(A).2 [3 marks]}
\textbf{Prove that: $\log\left(\frac{x^p}{x^q}\right) + \log\left(\frac{x^q}{x^r}\right) + \log\left(\frac{x^r}{x^p}\right) = 0$}

\begin{solutionbox}
\textbf{Solution}:\\
LHS = $\log\left(\frac{x^p}{x^q}\right) + \log\left(\frac{x^q}{x^r}\right) + \log\left(\frac{x^r}{x^p}\right)$\\
Using logarithm properties:\\
$= \log(x^p) - \log(x^q) + \log(x^q) - \log(x^r) + \log(x^r) - \log(x^p)$\\
$= p\log x - q\log x + q\log x - r\log x + r\log x - p\log x$\\
$= 0$ = RHS
\end{solutionbox}

\subsection*{Q2(A).3 [3 marks]}
\textbf{Find value: $\tan(75^\circ)$}

\textbf{Answer}: $2 + \sqrt{3}$

\begin{solutionbox}
\textbf{Solution}:\\
$\tan(75^\circ) = \tan(45^\circ + 30^\circ)$\\
Using $\tan(A + B) = \frac{\tan A + \tan B}{1 - \tan A \tan B}$:\\
$\tan(75^\circ) = \frac{\tan 45^\circ + \tan 30^\circ}{1 - \tan 45^\circ \tan 30^\circ} = \frac{1 + \frac{1}{\sqrt{3}}}{1 - 1 \times \frac{1}{\sqrt{3}}} = \frac{1 + \frac{1}{\sqrt{3}}}{1 - \frac{1}{\sqrt{3}}}$\\
$= \frac{\frac{\sqrt{3} + 1}{\sqrt{3}}}{\frac{\sqrt{3} - 1}{\sqrt{3}}} = \frac{\sqrt{3} + 1}{\sqrt{3} - 1} = \frac{(\sqrt{3} + 1)^2}{(\sqrt{3} - 1)(\sqrt{3} + 1)} = \frac{3 + 2\sqrt{3} + 1}{3 - 1} = \frac{4 + 2\sqrt{3}}{2} = 2 + \sqrt{3}$
\end{solutionbox}

\section*{Q.2 (B) [8 marks]}
\textbf{Attempt any two}

\subsection*{Q2(B).1 [4 marks]}
\textbf{Prove that: $\frac{1}{\log_{12} 120} + \frac{1}{\log_2 120} + \frac{1}{\log_5 120} = 1$}

\begin{solutionbox}
\textbf{Solution}:\\
Using change of base formula: $\frac{1}{\log_a b} = \log_b a$\\
LHS = $\log_{120} 12 + \log_{120} 2 + \log_{120} 5$\\
Using logarithm properties:\\
$= \log_{120}(12 \times 2 \times 5) = \log_{120} 120 = 1$ = RHS
\end{solutionbox}

\subsection*{Q2(B).2 [4 marks]}
\textbf{Solve: $\begin{vmatrix} x & 1 & 1 \\ 1 & 2 & 1 \\ 0 & 0 & 3 \end{vmatrix} = 3$}

\begin{solutionbox}
\textbf{Solution}:\\
Expanding along third row:\\
$\begin{vmatrix} x & 1 & 1 \\ 1 & 2 & 1 \\ 0 & 0 & 3 \end{vmatrix} = 3 \begin{vmatrix} x & 1 \\ 1 & 2 \end{vmatrix}$\\
$= 3(2x - 1) = 6x - 3$\\
Given: $6x - 3 = 3$\\
$6x = 6$\\
$x = 1$
\end{solutionbox}

\subsection*{Q2(B).3 [4 marks]}
\textbf{If $f(x) = \frac{1-x}{1+x}$ prove that: (i) $f(x) + f\left(\frac{1}{x}\right) = 0$ (ii) $f(x) \times f(-x) = 1$}

\begin{solutionbox}
\textbf{Solution}:\\
Given: $f(x) = \frac{1-x}{1+x}$\\
\textbf{(i)} $f\left(\frac{1}{x}\right) = \frac{1-\frac{1}{x}}{1+\frac{1}{x}} = \frac{\frac{x-1}{x}}{\frac{x+1}{x}} = \frac{x-1}{x+1} = -\frac{1-x}{1+x} = -f(x)$\\
Therefore: $f(x) + f\left(\frac{1}{x}\right) = f(x) + (-f(x)) = 0$\\
\textbf{(ii)} $f(-x) = \frac{1-(-x)}{1+(-x)} = \frac{1+x}{1-x}$\\
$f(x) \times f(-x) = \frac{1-x}{1+x} \times \frac{1+x}{1-x} = 1$
\end{solutionbox}

\section*{Q.3 (A) [6 marks]}
\textbf{Attempt any two}

\subsection*{Q3(A).1 [3 marks]}
\textbf{Prove that: $\frac{\sin(180^\circ - x) + \text{cosec}(180^\circ - x) + \tan(180^\circ + x)}{\cos(90^\circ + x) + \sec(90^\circ + x) + \cot(90^\circ + x)} = -3$}

\begin{solutionbox}
\textbf{Solution}:\\
Using trigonometric identities:\\
- $\sin(180^\circ - x) = \sin x$\\
- $\text{cosec}(180^\circ - x) = \text{cosec } x$\\
- $\tan(180^\circ + x) = \tan x$\\
- $\cos(90^\circ + x) = -\sin x$\\
- $\sec(90^\circ + x) = -\text{cosec } x$\\
- $\cot(90^\circ + x) = -\tan x$\\
Numerator = $\sin x + \text{cosec } x + \tan x$\\
Denominator = $-\sin x - \text{cosec } x - \tan x = -(\sin x + \text{cosec } x + \tan x)$\\
Therefore: $\frac{\sin x + \text{cosec } x + \tan x}{-(\sin x + \text{cosec } x + \tan x)} = -1 \neq -3$\\
\textbf{Note}: There appears to be an error in the problem statement or expected answer.
\end{solutionbox}

\subsection*{Q3(A).2 [3 marks]}
\textbf{Prove that: $\tan^{-1}\left(\frac{1}{3}\right) + \tan^{-1}\left(\frac{1}{2}\right) = 45^\circ$}

\begin{solutionbox}
\textbf{Solution}:\\
Using $\tan^{-1}A + \tan^{-1}B = \tan^{-1}\left(\frac{A+B}{1-AB}\right)$:\\
$\tan^{-1}\left(\frac{1}{3}\right) + \tan^{-1}\left(\frac{1}{2}\right) = \tan^{-1}\left(\frac{\frac{1}{3} + \frac{1}{2}}{1 - \frac{1}{3} \times \frac{1}{2}}\right)$\\
$= \tan^{-1}\left(\frac{\frac{5}{6}}{1 - \frac{1}{6}}\right) = \tan^{-1}\left(\frac{\frac{5}{6}}{\frac{5}{6}}\right) = \tan^{-1}(1) = 45^\circ$
\end{solutionbox}

\subsection*{Q3(A).3 [3 marks]}
\textbf{Find out equation of the line whose X-intercept is 3 and Y-intercept is 2.}

\begin{solutionbox}
\textbf{Solution}:\\
Using intercept form: $\frac{x}{a} + \frac{y}{b} = 1$\\
Where $a = 3$ (x-intercept) and $b = 2$ (y-intercept)\\
$\frac{x}{3} + \frac{y}{2} = 1$\\
Multiplying by 6: $2x + 3y = 6$
\end{solutionbox}

\section*{Q.3 (B) [8 marks]}
\textbf{Attempt any two}

\subsection*{Q3(B).1 [4 marks]}
\textbf{Prove that: $\tan(70^\circ) = \frac{\cos(25^\circ) + \sin(25^\circ)}{\cos(25^\circ) - \sin(25^\circ)}$}

\begin{solutionbox}
\textbf{Solution}:\\
RHS = $\frac{\cos(25^\circ) + \sin(25^\circ)}{\cos(25^\circ) - \sin(25^\circ)}$\\
Dividing numerator and denominator by $\cos(25^\circ)$:\\
$= \frac{1 + \tan(25^\circ)}{1 - \tan(25^\circ)}$\\
Using $\tan(45^\circ + \theta) = \frac{1 + \tan \theta}{1 - \tan \theta}$:\\
$= \tan(45^\circ + 25^\circ) = \tan(70^\circ)$ = LHS
\end{solutionbox}

\subsection*{Q3(B).2 [4 marks]}
\textbf{Prove that: $\frac{\sin \theta + \sin 2\theta + \sin 3\theta}{\cos \theta + \cos 2\theta + \cos 3\theta} = \tan 2\theta$}

\begin{solutionbox}
\textbf{Solution}:\\
Using sum-to-product formulas:\\
Numerator: $\sin \theta + \sin 3\theta + \sin 2\theta = 2\sin 2\theta \cos \theta + \sin 2\theta = \sin 2\theta(2\cos \theta + 1)$\\
Denominator: $\cos \theta + \cos 3\theta + \cos 2\theta = 2\cos 2\theta \cos \theta + \cos 2\theta = \cos 2\theta(2\cos \theta + 1)$\\
Therefore: $\frac{\sin 2\theta(2\cos \theta + 1)}{\cos 2\theta(2\cos \theta + 1)} = \frac{\sin 2\theta}{\cos 2\theta} = \tan 2\theta$
\end{solutionbox}

\subsection*{Q3(B).3 [4 marks]}
\textbf{If $\vec{a} = (1,2,3)$, $\vec{b} = (4,0,0)$ and $\vec{c} = (2,0,1)$ find $2\vec{a} + 3\vec{b} - 5\vec{c}$}

\begin{solutionbox}
\textbf{Solution}:\\
$2\vec{a} = 2(1,2,3) = (2,4,6)$\\
$3\vec{b} = 3(4,0,0) = (12,0,0)$\\
$5\vec{c} = 5(2,0,1) = (10,0,5)$\\
$2\vec{a} + 3\vec{b} - 5\vec{c} = (2,4,6) + (12,0,0) - (10,0,5)$\\
$= (2+12-10, 4+0-0, 6+0-5)$\\
$= (4,4,1)$
\end{solutionbox}

\section*{Q.4 (A) [6 marks]}
\textbf{Attempt any two}

\subsection*{Q4(A).1 [3 marks]}
\textbf{If the vectors $\vec{a} = \hat{i} - 2\hat{j} + 3\hat{k}$ and $\vec{b} = 2\hat{i} + m\hat{j} - 4\hat{k}$ are perpendicular, find m.}

\begin{solutionbox}
\textbf{Solution}:\\
For perpendicular vectors: $\vec{a} \cdot \vec{b} = 0$\\
$\vec{a} \cdot \vec{b} = (1)(2) + (-2)(m) + (3)(-4) = 2 - 2m - 12 = -10 - 2m$\\
Setting equal to zero: $-10 - 2m = 0$\\
$2m = -10$\\
$m = -5$
\end{solutionbox}

\subsection*{Q4(A).2 [3 marks]}
\textbf{Find the direction cosines and direction angles of the vector $\vec{a} = 5\hat{i} - 12\hat{k}$}

\begin{solutionbox}
\textbf{Solution}:\\
$\vec{a} = 5\hat{i} + 0\hat{j} - 12\hat{k}$\\
Magnitude: $|\vec{a}| = \sqrt{5^2 + 0^2 + (-12)^2} = \sqrt{25 + 144} = \sqrt{169} = 13$\\
Direction cosines:\\
- $l = \frac{5}{13}$\\
- $m = \frac{0}{13} = 0$\\
- $n = \frac{-12}{13}$\\
Direction angles:\\
- $\alpha = \cos^{-1}\left(\frac{5}{13}\right)$\\
- $\beta = \cos^{-1}(0) = 90^\circ$\\
- $\gamma = \cos^{-1}\left(\frac{-12}{13}\right)$
\end{solutionbox}

\subsection*{Q4(A).3 [3 marks]}
\textbf{Find out equation of the circle having center at $(2, -3)$ and radius 3.}

\begin{solutionbox}
\textbf{Solution}:\\
Standard form: $(x - h)^2 + (y - k)^2 = r^2$\\
Where $(h, k) = (2, -3)$ and $r = 3$\\
$(x - 2)^2 + (y + 3)^2 = 9$\\
Expanding: $x^2 - 4x + 4 + y^2 + 6y + 9 = 9$\\
$x^2 + y^2 - 4x + 6y + 4 = 0$
\end{solutionbox}

\section*{Q.4 (B) [8 marks]}
\textbf{Attempt any two}

\subsection*{Q4(B).1 [4 marks]}
\textbf{Show that the angle between vectors $\vec{a} = \hat{i} + 2\hat{j}$ and $\vec{b} = \hat{i} + \hat{j} + 3\hat{k}$ is $\sin^{-1}\sqrt{\frac{46}{55}}$}

\begin{solutionbox}
\textbf{Solution}:\\
$\vec{a} \cdot \vec{b} = (1)(1) + (2)(1) + (0)(3) = 1 + 2 = 3$\\
$|\vec{a}| = \sqrt{1^2 + 2^2} = \sqrt{5}$\\
$|\vec{b}| = \sqrt{1^2 + 1^2 + 3^2} = \sqrt{11}$\\
$\cos \theta = \frac{\vec{a} \cdot \vec{b}}{|\vec{a}||\vec{b}|} = \frac{3}{\sqrt{5}\sqrt{11}} = \frac{3}{\sqrt{55}}$\\
$\sin^2 \theta = 1 - \cos^2 \theta = 1 - \frac{9}{55} = \frac{46}{55}$\\
Therefore: $\theta = \sin^{-1}\sqrt{\frac{46}{55}}$
\end{solutionbox}

\subsection*{Q4(B).2 [4 marks]}
\textbf{Under effect of the forces $2\hat{i} + \hat{j} + \hat{k}$ and $\hat{i} + 3\hat{j} - \hat{k}$ a particle moves from the point $(1,2,-3)$ to the point $(5,3,7)$. Find out work done.}

\begin{solutionbox}
\textbf{Solution}:\\
Net force: $\vec{F} = (2\hat{i} + \hat{j} + \hat{k}) + (\hat{i} + 3\hat{j} - \hat{k}) = 3\hat{i} + 4\hat{j}$\\
Displacement: $\vec{s} = (5,3,7) - (1,2,-3) = (4,1,10)$\\
Work done: $W = \vec{F} \cdot \vec{s} = (3)(4) + (4)(1) + (0)(10) = 12 + 4 = 16$ units
\end{solutionbox}

\subsection*{Q4(B).3 [4 marks]}
\textbf{Evaluate: $\lim_{x \to 0} \frac{2^x - 5^x}{x}$}

\begin{solutionbox}
\textbf{Solution}:\\
Using L'Hôpital's rule or the derivative definition:\\
$\lim_{x \to 0} \frac{2^x - 5^x}{x} = \lim_{x \to 0} \frac{2^x \ln 2 - 5^x \ln 5}{1}$\\
$= 2^0 \ln 2 - 5^0 \ln 5 = \ln 2 - \ln 5 = \ln\left(\frac{2}{5}\right)$
\end{solutionbox}

\section*{Q.5 (A) [6 marks]}
\textbf{Attempt any two}

\subsection*{Q5(A).1 [3 marks]}
\textbf{Evaluate: $\lim_{x \to 0} \left(1 + \frac{3x}{7}\right)^{\frac{1}{x}}$}

\begin{solutionbox}
\textbf{Solution}:\\
Let $y = \left(1 + \frac{3x}{7}\right)^{\frac{1}{x}}$\\
Taking natural log: $\ln y = \frac{1}{x} \ln\left(1 + \frac{3x}{7}\right)$\\
$\lim_{x \to 0} \ln y = \lim_{x \to 0} \frac{\ln\left(1 + \frac{3x}{7}\right)}{x}$\\
Using L'Hôpital's rule: $= \lim_{x \to 0} \frac{\frac{3/7}{1 + \frac{3x}{7}}}{1} = \frac{3}{7}$\\
Therefore: $\lim_{x \to 0} y = e^{3/7}$
\end{solutionbox}

\subsection*{Q5(A).2 [3 marks]}
\textbf{Evaluate: $\lim_{x \to 3} \frac{x^2 - 5x + 6}{x^2 - 9}$}

\begin{solutionbox}
\textbf{Solution}:\\
Factoring numerator: $x^2 - 5x + 6 = (x-2)(x-3)$\\
Factoring denominator: $x^2 - 9 = (x-3)(x+3)$\\
$\lim_{x \to 3} \frac{x^2 - 5x + 6}{x^2 - 9} = \lim_{x \to 3} \frac{(x-2)(x-3)}{(x-3)(x+3)} = \lim_{x \to 3} \frac{x-2}{x+3} = \frac{3-2}{3+3} = \frac{1}{6}$
\end{solutionbox}

\subsection*{Q5(A).3 [3 marks]}
\textbf{Evaluate: $\lim_{x \to 0} \frac{\sqrt{4+x} - 2}{x}$}

\begin{solutionbox}
\textbf{Solution}:\\
Rationalizing the numerator:\\
$\lim_{x \to 0} \frac{\sqrt{4+x} - 2}{x} \times \frac{\sqrt{4+x} + 2}{\sqrt{4+x} + 2}$\\
$= \lim_{x \to 0} \frac{(4+x) - 4}{x(\sqrt{4+x} + 2)} = \lim_{x \to 0} \frac{x}{x(\sqrt{4+x} + 2)} = \lim_{x \to 0} \frac{1}{\sqrt{4+x} + 2} = \frac{1}{2+2} = \frac{1}{4}$
\end{solutionbox}

\section*{Q.5 (B) [8 marks]}
\textbf{Attempt any two}

\subsection*{Q5(B).1 [4 marks]}
\textbf{Find out equation of the line passing through points $(1,2)$ and $(2,1)$.}

\begin{solutionbox}
\textbf{Solution}:\\
Using two-point form: $\frac{y - y_1}{y_2 - y_1} = \frac{x - x_1}{x_2 - x_1}$\\
$\frac{y - 2}{1 - 2} = \frac{x - 1}{2 - 1}$\\
$\frac{y - 2}{-1} = \frac{x - 1}{1}$\\
$y - 2 = -(x - 1) = -x + 1$\\
$x + y = 3$
\end{solutionbox}

\subsection*{Q5(B).2 [4 marks]}
\textbf{Find equation of the line that passes through $(-3, 2)$ and parallel to the line $x - 2y + 1 = 0$}

\begin{solutionbox}
\textbf{Solution}:\\
The given line $x - 2y + 1 = 0$ has slope $m = \frac{1}{2}$\\
Since parallel lines have the same slope, required line has slope $m = \frac{1}{2}$\\
Using point-slope form: $y - y_1 = m(x - x_1)$\\
$y - 2 = \frac{1}{2}(x - (-3))$\\
$y - 2 = \frac{1}{2}(x + 3)$\\
$2y - 4 = x + 3$\\
$x - 2y + 7 = 0$
\end{solutionbox}

\subsection*{Q5(B).3 [4 marks]}
\textbf{Find out center and radius of the circle: $x^2 + y^2 + 6x - 4y - 3 = 0$}

\begin{solutionbox}
\textbf{Solution}:\\
Completing the square:\\
$x^2 + 6x + y^2 - 4y = 3$\\
$(x^2 + 6x + 9) + (y^2 - 4y + 4) = 3 + 9 + 4$\\
$(x + 3)^2 + (y - 2)^2 = 16$\\
\textbf{Center}: $(-3, 2)$\\
\textbf{Radius}: $r = \sqrt{16} = 4$
\end{solutionbox}

\newpage
\section*{Formula Cheat Sheet}

\subsection*{Logarithms}
\begin{itemize}
    \item $\log_a 1 = 0$
    \item $\log_a a = 1$
    \item $\log_a(xy) = \log_a x + \log_a y$
    \item $\log_a\left(\frac{x}{y}\right) = \log_a x - \log_a y$
\end{itemize}

\subsection*{Trigonometry}
\begin{itemize}
    \item $\sin^{-1}x + \cos^{-1}x = \frac{\pi}{2}$
    \item $\tan(A \pm B) = \frac{\tan A \pm \tan B}{1 \mp \tan A \tan B}$
    \item $\sin(180^\circ - x) = \sin x$, $\cos(90^\circ + x) = -\sin x$
\end{itemize}

\subsection*{Vectors}
\begin{itemize}
    \item $|\vec{a}| = \sqrt{a_1^2 + a_2^2 + a_3^2}$
    \item $\vec{a} \cdot \vec{b} = |\vec{a}||\vec{b}|\cos \theta$
    \item For perpendicular vectors: $\vec{a} \cdot \vec{b} = 0$
\end{itemize}

\subsection*{Coordinate Geometry}
\begin{itemize}
    \item Two-point form: $\frac{y - y_1}{y_2 - y_1} = \frac{x - x_1}{x_2 - x_1}$
    \item Circle: $(x - h)^2 + (y - k)^2 = r^2$
    \item Parallel lines have equal slopes
\end{itemize}

\subsection*{Limits}
\begin{itemize}
    \item $\lim_{x \to 0} \frac{\sin x}{x} = 1$
    \item $\lim_{x \to 0} \frac{e^x - 1}{x} = 1$
    \item $\lim_{x \to \infty} \frac{ax + b}{cx + d} = \frac{a}{c}$
\end{itemize}

\end{document}
