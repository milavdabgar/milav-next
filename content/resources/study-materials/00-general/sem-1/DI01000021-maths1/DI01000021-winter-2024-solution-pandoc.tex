\documentclass[10pt,a4paper]{article}

% content/resources/templates/preamble.tex
\usepackage[margin=0.6in]{geometry}
\author{Milav Dabgar}
\usepackage{amsmath,amssymb,amsthm}
\usepackage{booktabs}
\usepackage{multirow}
\usepackage{xcolor}
\usepackage{tcolorbox}
\tcbuselibrary{breakable,skins}
\usepackage[colorlinks=true,linkcolor=blue]{hyperref}
\usepackage{titlesec}
\usepackage{enumitem}
\usepackage{tikz}
\usepackage{pgfplots}
\usepackage{circuitikz}
\usepackage[version=4]{mhchem}
\usepackage{longtable}
\usepackage{array}
\usepackage{float}
\usepackage{caption}
\usepackage{listings}

\lstset{
  basicstyle=\small\ttfamily,
  breaklines=true,
  breakatwhitespace=false,
  postbreak=\mbox{\textcolor{red}{$\hookrightarrow$}\space},
  float=false,
  numbers=left,
  numberstyle=\tiny\color{gray},
  numbersep=10pt,
  xleftmargin=2em,
  keywordstyle=\color{blue},
  commentstyle=\color{green!60!black},
  stringstyle=\color{purple},
  backgroundcolor=\color{gray!5},
  showstringspaces=false,
  tabsize=2,
  captionpos=b,
  keepspaces=true,
  columns=flexible
}

\pgfplotsset{compat=1.18}
\usetikzlibrary{shapes,arrows,positioning,calc,patterns,decorations.pathmorphing,decorations.markings,arrows.meta}

% Color scheme
\definecolor{headcolor}{RGB}{0,102,204}
\definecolor{keycolor}{RGB}{220,20,60}
\definecolor{solutioncolor}{RGB}{34,139,34}
\definecolor{mnemoniccolor}{RGB}{148,0,211}
\definecolor{codecolor}{RGB}{0,0,100}

% Spacing
\setlength{\parskip}{3pt}
\setlist[itemize]{nosep}
\setlist[enumerate]{nosep}

% Title formatting
\titleformat{\section}{\Large\bfseries\color{headcolor}}{\thesection}{1em}{}
\titleformat{\subsection}{\large\bfseries\color{headcolor}}{\thesubsection}{1em}{}

% Pandoc tightlist compatibility
\providecommand{\tightlist}{%
  \setlength{\itemsep}{0pt}\setlength{\parskip}{0pt}}

% Pandoc longtable compatibility
\newcounter{none}
\def\thenone{}


% content/resources/templates/english-boxes.tex
% This file is currently empty - it exists to maintain consistency with the import structure.
% Add custom environments here if needed in the future.


\begin{document}

\begin{center}
{\Huge\bfseries\color{headcolor} Mathematics-I Solutions}\\[5pt]
{\LARGE DI01000021 -- Winter 2024}\\[3pt]
{\large Semester 1 Study Material}\\[3pt]
{\normalsize\textit{Detailed Solutions and Explanations}}
\end{center}

\vspace{10pt}

\subsection*{Q.1 [14 marks]}\label{q.1-14-marks}

\textbf{Fill in the blanks/MCQs using appropriate choice from the given
options.}

\subsubsection{Q1.1 [1 mark]}\label{q1.1-1-mark}

**\$

\begin{vmatrix} 5 & 1 \\ 2 & 3 \end{vmatrix}

= \$ \_\_\_\_\_\_\_**

\begin{solutionbox}
b. 13

\textbf{Solution}: For 2\times2 determinant
\(\begin{vmatrix} a & b \\ c & d \end{vmatrix} = ad - bc\)

\(\begin{vmatrix} 5 & 1 \\ 2 & 3 \end{vmatrix} = (5 \times 3) - (1 \times 2) = 15 - 2 = 13\)

\end{solutionbox}
\subsubsection{Q1.2 [1 mark]}\label{q1.2-1-mark}

\textbf{If \(\begin{vmatrix} x & 1 \\ 2 & 1 \end{vmatrix} = 0\) then \$x
= \$ \_\_\_\_\_\_\_}

\begin{solutionbox}
b. 2

\textbf{Solution}:
\(\begin{vmatrix} x & 1 \\ 2 & 1 \end{vmatrix} = x \times 1 - 1 \times 2 = x - 2 = 0\)

Therefore, \(x = 2\)

\end{solutionbox}
\subsubsection{Q1.3 [1 mark]}\label{q1.3-1-mark}

\textbf{If \(f(x) = x^2\) then \$f(-1) = \$ \_\_\_\_\_\_\_}

\begin{solutionbox}
a. 1

\textbf{Solution}: \(f(x) = x^2\) \(f(-1) = (-1)^2 = 1\)

\end{solutionbox}
\subsubsection{Q1.4 [1 mark]}\label{q1.4-1-mark}

\textbf{\$\log\emph{\{10\} 1 = \$ }\_\_\_\_\_\_}

\begin{solutionbox}
b. 0

\textbf{Solution}: By logarithm property: \(\log_a 1 = 0\) for any base
\(a > 0\) Therefore, \(\log_{10} 1 = 0\)

\end{solutionbox}
\subsubsection{Q1.5 [1 mark]}\label{q1.5-1-mark}

\textbf{\$\sin \frac{\pi}{2} + \cos \frac{\pi}{2} = \$ \_\_\_\_\_\_\_}

\begin{solutionbox}
c.~1

\textbf{Solution}: \(\sin \frac{\pi}{2} = 1\) and
\(\cos \frac{\pi}{2} = 0\) Therefore,
\(\sin \frac{\pi}{2} + \cos \frac{\pi}{2} = 1 + 0 = 1\)

\end{solutionbox}
\subsubsection{Q1.6 [1 mark]}\label{q1.6-1-mark}

\textbf{\$\tan\^{}\{-1\}(1) = \$ \_\_\_\_\_\_\_}

\begin{solutionbox}
a. \(\frac{\pi}{4}\)

\textbf{Solution}: \(\tan \frac{\pi}{4} = 1\) Therefore,
\(\tan^{-1}(1) = \frac{\pi}{4}\)

\end{solutionbox}
\subsubsection{Q1.7 [1 mark]}\label{q1.7-1-mark}

\textbf{\(\frac{2\pi}{3}\) radian = \_\_\_\_\_\_\_ degree}

\begin{solutionbox}
d.~120

\textbf{Solution}: To convert radians to degrees:
\(\text{degrees} = \text{radians} \times \frac{180}{\pi}\)
\(\frac{2\pi}{3} \times \frac{180}{\pi} = \frac{2 \times 180}{3} = \frac{360}{3} = 120^\circ\)

\end{solutionbox}
\subsubsection{Q1.8 [1 mark]}\label{q1.8-1-mark}

\textbf{\$\hat{i} \times \hat{j} = \$ \_\_\_\_\_\_\_}

\begin{solutionbox}
c.~\(\hat{k}\)

\textbf{Solution}: By right-hand rule for cross product:
\(\hat{i} \times \hat{j} = \hat{k}\)

\end{solutionbox}
\subsubsection{Q1.9 [1 mark]}\label{q1.9-1-mark}

\textbf{\$\textbar{}\hat{i} + \hat{j} + \hat{k}\textbar{} = \$
\_\_\_\_\_\_\_}

\begin{solutionbox}
d.~\(\sqrt{3}\)

\textbf{Solution}:
\(|\hat{i} + \hat{j} + \hat{k}| = \sqrt{1^2 + 1^2 + 1^2} = \sqrt{3}\)

\end{solutionbox}
\subsubsection{Q1.10 [1 mark]}\label{q1.10-1-mark}

\textbf{Slope of line \(2x + y - 3 = 0\) is \_\_\_\_\_\_\_}

\begin{solutionbox}
a. -2

\textbf{Solution}: Convert to slope-intercept form: \(y = -2x + 3\)
Slope = coefficient of \(x = -2\)

\end{solutionbox}
\subsubsection{Q1.11 [1 mark]}\label{q1.11-1-mark}

\textbf{Radius of circle \(x^2 + y^2 = 81\) is \_\_\_\_\_\_\_}

\begin{solutionbox}
b. 9

\textbf{Solution}: Standard form: \(x^2 + y^2 = r^2\) Here,
\(r^2 = 81\), so \(r = 9\)

\end{solutionbox}
\subsubsection{Q1.12 [1 mark]}\label{q1.12-1-mark}

\textbf{\$\lim\emph{\{n \to \infty\} \frac{1}{n} = \$ }\_\_\_\_\_\_}

\begin{solutionbox}
c.~0

\textbf{Solution}: As \(n\) approaches infinity, \(\frac{1}{n}\)
approaches 0

\end{solutionbox}
\subsubsection{Q1.13 [1 mark]}\label{q1.13-1-mark}

\textbf{\$\lim\emph{\{x \to 1\} (x\^{}2 + x + 1) = \$ }\_\_\_\_\_\_}

\begin{solutionbox}
a. 3

\textbf{Solution}: Direct substitution:
\((1)^2 + (1) + 1 = 1 + 1 + 1 = 3\)

\end{solutionbox}
\subsubsection{Q1.14 [1 mark]}\label{q1.14-1-mark}

\textbf{\$\lim\emph{\{\theta \to 0\} \frac{\tan \theta}{\theta} = \$
}\_\_\_\_\_\_}

\begin{solutionbox}
b. 1

\textbf{Solution}: This is a standard limit:
\(\lim_{\theta \to 0} \frac{\tan \theta}{\theta} = 1\)

\end{solutionbox}
\subsection*{Q.2 (A) [6 marks]}\label{q.2-a-6-marks}

\textbf{Attempt any two}

\subsubsection{Q2.1 [3 marks]}\label{q2.1-3-marks}

\textbf{Find the value of
\(\begin{vmatrix} 1 & 3 & 1 \\ 2 & -1 & 0 \\ 4 & -2 & 5 \end{vmatrix}\)}

\begin{solutionbox}

\textbf{Solution}: Using expansion along second row (has zero):
\(= -2\begin{vmatrix} 3 & 1 \\ -2 & 5 \end{vmatrix} + (-1)\begin{vmatrix} 1 & 1 \\ 4 & 5 \end{vmatrix} + 0\)

\(= -2(15 + 2) - 1(5 - 4)\) \(= -2(17) - 1(1)\) \(= -34 - 1 = -35\)


\begin{longtable}[]{@{}lll@{}}
\toprule\noalign{}
Step & Calculation & Result \\
\midrule\noalign{}
\endhead
\bottomrule\noalign{}
\endlastfoot
Minor 1 & \((3 \times 5) - (1 \times -2)\) & 17 \\
Minor 2 & \((1 \times 5) - (1 \times 4)\) & 1 \\
Final & \(-2(17) - 1(1)\) & -35 \\
\end{longtable}

\end{solutionbox}
\subsubsection{Q2.2 [3 marks]}\label{q2.2-3-marks}

\textbf{If \(f(x) = x^3 + 5\) then find \(f(0)\), \(f(1)\) and
\(f(-1)\)}

\begin{solutionbox}

\textbf{Solution}: Given: \(f(x) = x^3 + 5\)

\(f(0) = (0)^3 + 5 = 0 + 5 = 5\) \(f(1) = (1)^3 + 5 = 1 + 5 = 6\)
\(f(-1) = (-1)^3 + 5 = -1 + 5 = 4\)


\begin{longtable}[]{@{}lll@{}}
\toprule\noalign{}
Input & Calculation & Output \\
\midrule\noalign{}
\endhead
\bottomrule\noalign{}
\endlastfoot
\(f(0)\) & \(0^3 + 5\) & 5 \\
\(f(1)\) & \(1^3 + 5\) & 6 \\
\(f(-1)\) & \((-1)^3 + 5\) & 4 \\
\end{longtable}

\end{solutionbox}
\subsubsection{Q2.3 [3 marks]}\label{q2.3-3-marks}

\textbf{Prove that
\(\tan^{-1}\left(\frac{1}{2}\right) + \tan^{-1}\left(\frac{1}{3}\right) = \frac{\pi}{4}\)}

\begin{solutionbox}

\textbf{Solution}: Using formula:
\(\tan^{-1}a + \tan^{-1}b = \tan^{-1}\left(\frac{a+b}{1-ab}\right)\)

Let \(a = \frac{1}{2}\), \(b = \frac{1}{3}\)

\(\tan^{-1}\left(\frac{1}{2}\right) + \tan^{-1}\left(\frac{1}{3}\right) = \tan^{-1}\left(\frac{\frac{1}{2} + \frac{1}{3}}{1 - \frac{1}{2} \times \frac{1}{3}}\right)\)

\(= \tan^{-1}\left(\frac{\frac{5}{6}}{1 - \frac{1}{6}}\right) = \tan^{-1}\left(\frac{\frac{5}{6}}{\frac{5}{6}}\right) = \tan^{-1}(1) = \frac{\pi}{4}\)

Hence proved.

\end{solutionbox}
\subsection*{Q.2 (B) [8 marks]}\label{q.2-b-8-marks}

\textbf{Attempt any two}

\subsubsection{Q2.1 [4 marks]}\label{q2.1-4-marks}

\textbf{If \(f(x) = \frac{x-1}{x+1}\) then prove that
\(f(x) \cdot f(-x) = 1\)}

\begin{solutionbox}

\textbf{Solution}: Given: \(f(x) = \frac{x-1}{x+1}\)

First find \(f(-x)\):
\(f(-x) = \frac{(-x)-1}{(-x)+1} = \frac{-x-1}{-x+1} = \frac{-(x+1)}{-(x-1)} = \frac{x+1}{x-1}\)

Now calculate \(f(x) \cdot f(-x)\):
\(f(x) \cdot f(-x) = \frac{x-1}{x+1} \cdot \frac{x+1}{x-1} = \frac{(x-1)(x+1)}{(x+1)(x-1)} = 1\)

Hence proved.

\end{solutionbox}
\subsubsection{Q2.2 [4 marks]}\label{q2.2-4-marks}

\textbf{If
\(\log\left(\frac{x+y}{2}\right) = \frac{1}{2}(\log x + \log y)\) then
prove that \(x = y\)}

\begin{solutionbox}

\textbf{Solution}: Given:
\(\log\left(\frac{x+y}{2}\right) = \frac{1}{2}(\log x + \log y)\)

Using logarithm properties:
\(\frac{1}{2}(\log x + \log y) = \frac{1}{2}\log(xy) = \log\sqrt{xy}\)

So: \(\log\left(\frac{x+y}{2}\right) = \log\sqrt{xy}\)

Taking antilog: \(\frac{x+y}{2} = \sqrt{xy}\)

Squaring both sides: \(\left(\frac{x+y}{2}\right)^2 = xy\)

\(\frac{(x+y)^2}{4} = xy\)

\((x+y)^2 = 4xy\)

\(x^2 + 2xy + y^2 = 4xy\)

\(x^2 - 2xy + y^2 = 0\)

\((x-y)^2 = 0\)

Therefore, \(x = y\). Hence proved.

\end{solutionbox}
\subsubsection{Q2.3 [4 marks]}\label{q2.3-4-marks}

\textbf{Solve \(\log(x+3) + \log(x-3) = \log 27\)}

\begin{solutionbox}

\textbf{Solution}: Given: \(\log(x+3) + \log(x-3) = \log 27\)

Using logarithm property: \(\log a + \log b = \log(ab)\)
\(\log[(x+3)(x-3)] = \log 27\)

Taking antilog: \((x+3)(x-3) = 27\)

\(x^2 - 9 = 27\)

\(x^2 = 36\)

\(x = \pm 6\)

\textbf{Check validity:}

\begin{itemize}
\tightlist
\item
  For \(x = 6\): \(x+3 = 9 > 0\) and \(x-3 = 3 > 0\) ✓
\item
  For \(x = -6\): \(x+3 = -3 < 0\) (invalid for logarithm)
\end{itemize}

Therefore, \(x = 6\)

\end{solutionbox}
\subsection*{Q.3 (A) [6 marks]}\label{q.3-a-6-marks}

\textbf{Attempt any two}

\subsubsection{Q3.1 [3 marks]}\label{q3.1-3-marks}

\textbf{Prove that
\(\frac{\sin\left(\frac{\pi}{2}+\theta\right)}{\cos(\pi-\theta)} + \frac{\tan(\pi-\theta)}{\cot\left(\frac{3\pi}{2}-\theta\right)} + \frac{\text{cosec}\left(\frac{\pi}{2}-\theta\right)}{\sec(\pi+\theta)} = -3\)}

\begin{solutionbox}

\textbf{Solution}: Using trigonometric identities:

\(\sin\left(\frac{\pi}{2}+\theta\right) = \cos\theta\)
\(\cos(\pi-\theta) = -\cos\theta\) \(\tan(\pi-\theta) = -\tan\theta\)
\(\cot\left(\frac{3\pi}{2}-\theta\right) = \tan\theta\)
\(\text{cosec}\left(\frac{\pi}{2}-\theta\right) = \sec\theta\)
\(\sec(\pi+\theta) = -\sec\theta\)

Substituting:
\(\frac{\cos\theta}{-\cos\theta} + \frac{-\tan\theta}{\tan\theta} + \frac{\sec\theta}{-\sec\theta}\)

\(= -1 + (-1) + (-1) = -3\)

Hence proved.

\end{solutionbox}
\subsubsection{Q3.2 [3 marks]}\label{q3.2-3-marks}

\textbf{Prove that
\(\tan 55^\circ = \frac{\cos 10^\circ + \sin 10^\circ}{\cos 10^\circ - \sin 10^\circ}\)}

\begin{solutionbox}

\textbf{Solution}: We know that \(\tan 55^\circ = \tan(45^\circ + 10^\circ)\)

Using formula:
\(\tan(A + B) = \frac{\tan A + \tan B}{1 - \tan A \tan B}\)

\(\tan 55^\circ = \frac{\tan 45^\circ + \tan 10^\circ}{1 - \tan 45^\circ \tan 10^\circ} = \frac{1 + \tan 10^\circ}{1 - \tan 10^\circ}\)

Now, \(\tan 10^\circ = \frac{\sin 10^\circ}{\cos 10^\circ}\)

\(\tan 55^\circ = \frac{1 + \frac{\sin 10^\circ}{\cos 10^\circ}}{1 - \frac{\sin 10^\circ}{\cos 10^\circ}} = \frac{\cos 10^\circ + \sin 10^\circ}{\cos 10^\circ - \sin 10^\circ}\)

Hence proved.

\end{solutionbox}
\subsubsection{Q3.3 [3 marks]}\label{q3.3-3-marks}

\textbf{If \(\vec{a} = 2\hat{i} + 3\hat{j} + \hat{k}\),
\(\vec{b} = \hat{i} + \hat{j} + \hat{k}\) and
\(\vec{c} = 3\hat{i} + \hat{j} + \hat{k}\) then find
\(2\vec{a} + \vec{b} - \vec{c}\)}

\begin{solutionbox}

\textbf{Solution}: Given: \(\vec{a} = 2\hat{i} + 3\hat{j} + \hat{k}\)
\(\vec{b} = \hat{i} + \hat{j} + \hat{k}\)
\(\vec{c} = 3\hat{i} + \hat{j} + \hat{k}\)

\(2\vec{a} = 2(2\hat{i} + 3\hat{j} + \hat{k}) = 4\hat{i} + 6\hat{j} + 2\hat{k}\)

\(2\vec{a} + \vec{b} - \vec{c} = (4\hat{i} + 6\hat{j} + 2\hat{k}) + (\hat{i} + \hat{j} + \hat{k}) - (3\hat{i} + \hat{j} + \hat{k})\)

\(= (4 + 1 - 3)\hat{i} + (6 + 1 - 1)\hat{j} + (2 + 1 - 1)\hat{k}\)

\(= 2\hat{i} + 6\hat{j} + 2\hat{k}\)

\end{solutionbox}
\subsection*{Q.3 (B) [8 marks]}\label{q.3-b-8-marks}

\textbf{Attempt any two}

\subsubsection{Q3.1 [4 marks]}\label{q3.1-4-marks}

\textbf{Prove that
\(\frac{\sin(x-y)}{\cos x \cos y} + \frac{\sin(y-z)}{\cos y \cos z} + \frac{\sin(z-x)}{\cos z \cos x} = 0\)}

\begin{solutionbox}

\textbf{Solution}: Using identity:
\(\sin(A-B) = \sin A \cos B - \cos A \sin B\)

\(\frac{\sin(x-y)}{\cos x \cos y} = \frac{\sin x \cos y - \cos x \sin y}{\cos x \cos y} = \tan x - \tan y\)

Similarly: \(\frac{\sin(y-z)}{\cos y \cos z} = \tan y - \tan z\)
\(\frac{\sin(z-x)}{\cos z \cos x} = \tan z - \tan x\)

Adding all three:
\((\tan x - \tan y) + (\tan y - \tan z) + (\tan z - \tan x) = 0\)

Hence proved.

\end{solutionbox}
\subsubsection{Q3.2 [4 marks]}\label{q3.2-4-marks}

\textbf{Draw graph of \(y = \cos x\) for \(0 \leq x \leq \pi\)}

\begin{solutionbox}

\textbf{Solution}:

\begin{verbatim}
      y
      \^{}
    1 |     •
      |    / {}
      |   /   {}
      |  /     {}
    0 |•{-{-}{-}{-}{-}{-}{-}{-}{-}•{-}{-}{-}{-}{-}{-}{-}{-}{-} x}
      |          {           π}
      |           {         /}
      |            {       /}
   {-1 |             •{-}{-}{-}{-}{-}•}
      |           π/2     
      0          π/2      π
\end{verbatim}

\textbf{Table of values:}

\begin{longtable}[]{@{}llllll@{}}
\toprule\noalign{}
x & 0 & π/4 & π/2 & 3π/4 & π \\
\midrule\noalign{}
\endhead
\bottomrule\noalign{}
\endlastfoot
y & 1 & \sqrt2/2 & 0 & -\sqrt2/2 & -1 \\
\end{longtable}

\end{solutionbox}
\subsubsection{Q3.3 [4 marks]}\label{q3.3-4-marks}

\textbf{Find equation of line passing through (1, 2) and (-3, 1)}

\begin{solutionbox}

\textbf{Solution}: Given points: \((x_1, y_1) = (1, 2)\) and
\((x_2, y_2) = (-3, 1)\)

Slope:
\(m = \frac{y_2 - y_1}{x_2 - x_1} = \frac{1 - 2}{-3 - 1} = \frac{-1}{-4} = \frac{1}{4}\)

Using point-slope form: \(y - y_1 = m(x - x_1)\)
\(y - 2 = \frac{1}{4}(x - 1)\) \(4(y - 2) = x - 1\) \(4y - 8 = x - 1\)
\(x - 4y + 7 = 0\)

\textbf{Equation:} \(x - 4y + 7 = 0\)

\end{solutionbox}
\subsection*{Q.4 (A) [6 marks]}\label{q.4-a-6-marks}

\textbf{Attempt any two}

\subsubsection{Q4.1 [3 marks]}\label{q4.1-3-marks}

\textbf{Find unit vector perpendicular to
\(\vec{a} = \hat{i} - 3\hat{j} + \hat{k}\) and
\(\vec{b} = 2\hat{i} + \hat{j} + 2\hat{k}\)}

\begin{solutionbox}

\textbf{Solution}: Cross product:
\(\vec{a} \times \vec{b} = \begin{vmatrix} \hat{i} & \hat{j} & \hat{k} \\ 1 & -3 & 1 \\ 2 & 1 & 2 \end{vmatrix}\)

\(= \hat{i}[(-3)(2) - (1)(1)] - \hat{j}[(1)(2) - (1)(2)] + \hat{k}[(1)(1) - (-3)(2)]\)
\(= \hat{i}(-6-1) - \hat{j}(2-2) + \hat{k}(1+6)\)
\(= -7\hat{i} + 0\hat{j} + 7\hat{k}\)

Magnitude:
\(|\vec{a} \times \vec{b}| = \sqrt{(-7)^2 + 0^2 + 7^2} = \sqrt{49 + 49} = 7\sqrt{2}\)

Unit vector:
\(\hat{n} = \frac{-7\hat{i} + 7\hat{k}}{7\sqrt{2}} = \frac{-\hat{i} + \hat{k}}{\sqrt{2}}\)

\end{solutionbox}
\subsubsection{Q4.2 [3 marks]}\label{q4.2-3-marks}

\textbf{Forces (1, 2, 1) and (2, -1, 3) act on a particle and the
particle moves from point (2, 3, 1) to (4, 6, 2). Find the work done.}

\begin{solutionbox}

\textbf{Solution}: Resultant force:
\(\vec{F} = (1, 2, 1) + (2, -1, 3) = (3, 1, 4)\)

Displacement: \(\vec{s} = (4, 6, 2) - (2, 3, 1) = (2, 3, 1)\)

Work done:
\(W = \vec{F} \cdot \vec{s} = (3)(2) + (1)(3) + (4)(1) = 6 + 3 + 4 = 13\)
units

\end{solutionbox}
\subsubsection{Q4.3 [3 marks]}\label{q4.3-3-marks}

\textbf{Show that lines \(2x - 3y + 5 = 0\) and \(8x - 12y - 3 = 0\) are
parallel lines.}

\begin{solutionbox}

\textbf{Solution}: For line \(2x - 3y + 5 = 0\): slope
\(m_1 = \frac{2}{3}\) For line \(8x - 12y - 3 = 0\): slope
\(m_2 = \frac{8}{12} = \frac{2}{3}\)

Since \(m_1 = m_2 = \frac{2}{3}\), the lines are parallel.


\begin{longtable}[]{@{}lll@{}}
\toprule\noalign{}
Line & Standard Form & Slope \\
\midrule\noalign{}
\endhead
\bottomrule\noalign{}
\endlastfoot
Line 1 & \(2x - 3y + 5 = 0\) & \(\frac{2}{3}\) \\
Line 2 & \(8x - 12y - 3 = 0\) & \(\frac{2}{3}\) \\
\end{longtable}

\end{solutionbox}
\subsection*{Q.4 (B) [8 marks]}\label{q.4-b-8-marks}

\textbf{Attempt any two}

\subsubsection{Q4.1 [4 marks]}\label{q4.1-4-marks}

\textbf{Show that angle between
\(\vec{a} = \hat{i} + \hat{j} - \hat{k}\) and
\(\vec{b} = 2\hat{i} - 2\hat{j} + \hat{k}\) is
\(\sin^{-1}\left(\frac{\sqrt{26}}{27}\right)\)}

\begin{solutionbox}

\textbf{Solution}:
\(\vec{a} \cdot \vec{b} = (1)(2) + (1)(-2) + (-1)(1) = 2 - 2 - 1 = -1\)

\(|\vec{a}| = \sqrt{1^2 + 1^2 + (-1)^2} = \sqrt{3}\)
\(|\vec{b}| = \sqrt{2^2 + (-2)^2 + 1^2} = \sqrt{9} = 3\)

\(\cos\theta = \frac{\vec{a} \cdot \vec{b}}{|\vec{a}||\vec{b}|} = \frac{-1}{\sqrt{3} \times 3} = \frac{-1}{3\sqrt{3}}\)

\(\sin^2\theta = 1 - \cos^2\theta = 1 - \frac{1}{27} = \frac{26}{27}\)

Therefore,
\(\sin\theta = \frac{\sqrt{26}}{3\sqrt{3}} = \frac{\sqrt{26}}{\sqrt{27}}\)

Hence, \(\theta = \sin^{-1}\left(\frac{\sqrt{26}}{\sqrt{27}}\right)\)

\end{solutionbox}
\subsubsection{Q4.2 [4 marks]}\label{q4.2-4-marks}

\textbf{If \(\vec{a} = (1, 1, 1)\), \(\vec{b} = (2, 0, 1)\) and
\(\vec{c} = (-2, 1, 0)\) then find
\(\vec{a} \cdot (\vec{b} \times \vec{c})\)}

\begin{solutionbox}

\textbf{Solution}: First find \(\vec{b} \times \vec{c}\):
\(\vec{b} \times \vec{c} = \begin{vmatrix} \hat{i} & \hat{j} & \hat{k} \\ 2 & 0 & 1 \\ -2 & 1 & 0 \end{vmatrix}\)

\(= \hat{i}(0 \times 0 - 1 \times 1) - \hat{j}(2 \times 0 - 1 \times (-2)) + \hat{k}(2 \times 1 - 0 \times (-2))\)
\(= \hat{i}(-1) - \hat{j}(2) + \hat{k}(2)\)
\(= -\hat{i} - 2\hat{j} + 2\hat{k}\)

Now find \(\vec{a} \cdot (\vec{b} \times \vec{c})\):
\(\vec{a} \cdot (\vec{b} \times \vec{c}) = (1, 1, 1) \cdot (-1, -2, 2)\)
\(= (1)(-1) + (1)(-2) + (1)(2) = -1 - 2 + 2 = -1\)

\end{solutionbox}
\subsubsection{Q4.3 [4 marks]}\label{q4.3-4-marks}

\textbf{Evaluate \(\lim_{\theta \to 0} \frac{\sin 4\theta}{\theta}\)}

\begin{solutionbox}

\textbf{Solution}:
\(\lim_{\theta \to 0} \frac{\sin 4\theta}{\theta} = \lim_{\theta \to 0} \frac{\sin 4\theta}{4\theta} \times 4\)

Using standard limit \(\lim_{x \to 0} \frac{\sin x}{x} = 1\):

Let \(u = 4\theta\), then as \(\theta \to 0\), \(u \to 0\)

\(\lim_{\theta \to 0} \frac{\sin 4\theta}{4\theta} = \lim_{u \to 0} \frac{\sin u}{u} = 1\)

Therefore,
\(\lim_{\theta \to 0} \frac{\sin 4\theta}{\theta} = 4 \times 1 = 4\)

\end{solutionbox}
\subsection*{Q.5 (A) [6 marks]}\label{q.5-a-6-marks}

\textbf{Attempt any two}

\subsubsection{Q5.1 [3 marks]}\label{q5.1-3-marks}

\textbf{Evaluate \(\lim_{x \to 9} \frac{x^2 - 81}{x - 9}\)}

\begin{solutionbox}

\textbf{Solution}: Direct substitution gives \(\frac{0}{0}\) form.

Factor the numerator: \(x^2 - 81 = (x-9)(x+9)\)

\(\lim_{x \to 9} \frac{x^2 - 81}{x - 9} = \lim_{x \to 9} \frac{(x-9)(x+9)}{x-9}\)

\(= \lim_{x \to 9} (x+9) = 9 + 9 = 18\)

\end{solutionbox}
\subsubsection{Q5.2 [3 marks]}\label{q5.2-3-marks}

\textbf{Evaluate
\(\lim_{x \to \infty} \left(1 + \frac{3}{x}\right)^{2x}\)}

\begin{solutionbox}

\textbf{Solution}: Let \(y = \left(1 + \frac{3}{x}\right)^{2x}\)

Taking natural logarithm: \(\ln y = 2x \ln\left(1 + \frac{3}{x}\right)\)

As \(x \to \infty\), \(\frac{3}{x} \to 0\)

Using \(\ln(1+u) \approx u\) for small \(u\):
\(\ln y = 2x \times \frac{3}{x} = 6\)

Therefore, \(y = e^6\)

\end{solutionbox}
\subsubsection{Q5.3 [3 marks]}\label{q5.3-3-marks}

\textbf{Evaluate \(\lim_{x \to 1} \frac{x - 1}{x^2 + x - 2}\)}

\begin{solutionbox}

\textbf{Solution}: Factor the denominator: \(x^2 + x - 2 = (x+2)(x-1)\)

\(\lim_{x \to 1} \frac{x - 1}{x^2 + x - 2} = \lim_{x \to 1} \frac{x-1}{(x+2)(x-1)}\)

\(= \lim_{x \to 1} \frac{1}{x+2} = \frac{1}{1+2} = \frac{1}{3}\)

\end{solutionbox}
\subsection*{Q.5 (B) [8 marks]}\label{q.5-b-8-marks}

\textbf{Attempt any two}

\subsubsection{Q5.1 [4 marks]}\label{q5.1-4-marks}

\textbf{Find the equation of line passing through the point (2, -3) and
having slope 4.}

\begin{solutionbox}

\textbf{Solution}: Using point-slope form: \(y - y_1 = m(x - x_1)\)

Given: \((x_1, y_1) = (2, -3)\) and slope \(m = 4\)

\(y - (-3) = 4(x - 2)\) \(y + 3 = 4x - 8\) \(y = 4x - 11\)

\textbf{Equation:} \(y = 4x - 11\) or \$4x - y - 11 = 0

\end{solutionbox}
\subsubsection{Q5.2 [4 marks]}\label{q5.2-4-marks}

\textbf{For what value of m, lines \(7x + y - 1 = 0\) and
\(3x - my + 2 = 0\) are perpendicular to each other.}

\begin{solutionbox}

\textbf{Solution}: For perpendicular lines, product of slopes = -1

For line \(7x + y - 1 = 0\): slope \(m_1 = -7\) For line
\(3x - my + 2 = 0\): slope \(m_2 = \frac{3}{m}\)

Condition: \(m_1 \times m_2 = -1\) \((-7) \times \frac{3}{m} = -1\)
\(\frac{-21}{m} = -1\) \(21 = m\)

Therefore, \(m = 21\)


\begin{longtable}[]{@{}lll@{}}
\toprule\noalign{}
Line & Standard Form & Slope \\
\midrule\noalign{}
\endhead
\bottomrule\noalign{}
\endlastfoot
Line 1 & \(7x + y - 1 = 0\) & \(-7\) \\
Line 2 & \(3x - my + 2 = 0\) & \(\frac{3}{m}\) \\
\end{longtable}

\textbf{Verification:} When \(m = 21\), slopes are \(-7\) and
\(\frac{3}{21} = \frac{1}{7}\) Product: \((-7) \times \frac{1}{7} = -1\)
✓

\end{solutionbox}
\subsubsection{Q5.3 [4 marks]}\label{q5.3-4-marks}

\textbf{Find the centre and radius of the circle
\(4x^2 + 4y^2 + 8x - 12y - 3 = 0\)}

\begin{solutionbox}

\textbf{Solution}: First, divide by 4 to get standard form:
\(x^2 + y^2 + 2x - 3y - \frac{3}{4} = 0\)

Complete the square for x and y terms: \(x^2 + 2x = (x+1)^2 - 1\)
\(y^2 - 3y = \left(y - \frac{3}{2}\right)^2 - \frac{9}{4}\)

Substituting:
\((x+1)^2 - 1 + \left(y - \frac{3}{2}\right)^2 - \frac{9}{4} - \frac{3}{4} = 0\)

\((x+1)^2 + \left(y - \frac{3}{2}\right)^2 = 1 + \frac{9}{4} + \frac{3}{4} = 1 + 3 = 4\)

\textbf{Centre:} \((-1, \frac{3}{2})\) \textbf{Radius:}
\(r = \sqrt{4} = 2\)


\begin{longtable}[]{@{}ll@{}}
\toprule\noalign{}
Component & Value \\
\midrule\noalign{}
\endhead
\bottomrule\noalign{}
\endlastfoot
Centre (h,k) & \((-1, \frac{3}{2})\) \\
Radius & 2 \\
Standard Form & \((x+1)^2 + (y-\frac{3}{2})^2 = 4\) \\
\end{longtable}

\end{solutionbox}
\begin{center}\rule{0.5\linewidth}{0.5pt}\end{center}

\subsection*{Formula Cheat Sheet}\label{formula-cheat-sheet}

\subsubsection{\texorpdfstring{\textbf{Determinants}}{Determinants}}\label{determinants}

\begin{itemize}
\tightlist
\item
  \textbf{2\times2 Determinant:}
  \(\begin{vmatrix} a & b \\ c & d \end{vmatrix} = ad - bc\)
\item
  \textbf{3\times3 Determinant:} Expand along any row/column
\end{itemize}

\subsubsection{\texorpdfstring{\textbf{Functions \&
Logarithms}}{Functions \& Logarithms}}\label{functions-logarithms}

\begin{itemize}
\tightlist
\item
  \textbf{Basic:} \(\log_a 1 = 0\), \(\log_a a = 1\)
\item
  \textbf{Properties:} \(\log(ab) = \log a + \log b\),
  \(\log\left(\frac{a}{b}\right) = \log a - \log b\)
\end{itemize}

\subsubsection{\texorpdfstring{\textbf{Trigonometry}}{Trigonometry}}\label{trigonometry}

\begin{itemize}
\tightlist
\item
  \textbf{Basic Values:} \(\sin 0^\circ = 0\), \(\sin 30^\circ = \frac{1}{2}\),
  \(\sin 45^\circ = \frac{\sqrt{2}}{2}\), \(\sin 60^\circ = \frac{\sqrt{3}}{2}\),
  \(\sin 90^\circ = 1\)
\item
  \textbf{Conversion:} Radians to degrees: \(\times \frac{180}{\pi}\)
\item
  \textbf{Identities:} \(\sin^2\theta + \cos^2\theta = 1\)
\item
  \textbf{Inverse:} \(\tan^{-1}(1) = \frac{\pi}{4}\)
\end{itemize}

\subsubsection{\texorpdfstring{\textbf{Vectors}}{Vectors}}\label{vectors}

\begin{itemize}
\tightlist
\item
  \textbf{Magnitude:} \(|\vec{a}| = \sqrt{a_x^2 + a_y^2 + a_z^2}\)
\item
  \textbf{Dot Product:}
  \(\vec{a} \cdot \vec{b} = a_x b_x + a_y b_y + a_z b_z\)
\item
  \textbf{Cross Product:} \(\hat{i} \times \hat{j} = \hat{k}\),
  \(\hat{j} \times \hat{k} = \hat{i}\),
  \(\hat{k} \times \hat{i} = \hat{j}\)
\item
  \textbf{Work Done:} \(W = \vec{F} \cdot \vec{s}\)
\end{itemize}

\subsubsection{\texorpdfstring{\textbf{Coordinate
Geometry}}{Coordinate Geometry}}\label{coordinate-geometry}

\begin{itemize}
\tightlist
\item
  \textbf{Slope:} \(m = \frac{y_2 - y_1}{x_2 - x_1}\)
\item
  \textbf{Point-Slope Form:} \(y - y_1 = m(x - x_1)\)
\item
  \textbf{Parallel Lines:} Same slope
\item
  \textbf{Perpendicular Lines:} Product of slopes = -1
\item
  \textbf{Circle:} \((x-h)^2 + (y-k)^2 = r^2\)
\end{itemize}

\subsubsection{\texorpdfstring{\textbf{Limits}}{Limits}}\label{limits}

\begin{itemize}
\tightlist
\item
  \textbf{Standard Limits:} \(\lim_{x \to 0} \frac{\sin x}{x} = 1\),
  \(\lim_{x \to 0} \frac{\tan x}{x} = 1\)
\item
  \textbf{Factorization:} Use for \(\frac{0}{0}\) forms
\item
  \textbf{L'Hôpital's Rule:} For indeterminate forms
\end{itemize}

\begin{center}\rule{0.5\linewidth}{0.5pt}\end{center}

\subsection*{Problem-Solving
Strategies}\label{problem-solving-strategies}

\subsubsection{\texorpdfstring{\textbf{For
Determinants:}}{For Determinants:}}\label{for-determinants}

\begin{enumerate}
\tightlist
\item
  \textbf{Choose the row/column with most zeros for expansion}
\item
  \textbf{Use cofactor expansion systematically}
\item
  \textbf{Check calculations by expanding along different rows}
\end{enumerate}

\subsubsection{\texorpdfstring{\textbf{For
Functions:}}{For Functions:}}\label{for-functions}

\begin{enumerate}
\tightlist
\item
  \textbf{Direct substitution first}
\item
  \textbf{Use function properties and definitions}
\item
  \textbf{Check domain restrictions}
\end{enumerate}

\subsubsection{\texorpdfstring{\textbf{For
Trigonometry:}}{For Trigonometry:}}\label{for-trigonometry}

\begin{enumerate}
\tightlist
\item
  \textbf{Convert all angles to same unit (degrees or radians)}
\item
  \textbf{Use standard angle values}
\item
  \textbf{Apply appropriate identities}
\item
  \textbf{Simplify step by step}
\end{enumerate}

\subsubsection{\texorpdfstring{\textbf{For
Vectors:}}{For Vectors:}}\label{for-vectors}

\begin{enumerate}
\tightlist
\item
  \textbf{Write components clearly}
\item
  \textbf{Use right-hand rule for cross products}
\item
  \textbf{Check units and directions}
\item
  \textbf{Verify with geometric interpretation}
\end{enumerate}

\subsubsection{\texorpdfstring{\textbf{For Coordinate
Geometry:}}{For Coordinate Geometry:}}\label{for-coordinate-geometry}

\begin{enumerate}
\tightlist
\item
  \textbf{Plot points when possible}
\item
  \textbf{Use appropriate formulas based on given information}
\item
  \textbf{Check parallel/perpendicular conditions}
\item
  \textbf{Complete the square for circles}
\end{enumerate}

\subsubsection{\texorpdfstring{\textbf{For
Limits:}}{For Limits:}}\label{for-limits}

\begin{enumerate}
\tightlist
\item
  \textbf{Try direct substitution first}
\item
  \textbf{Factor polynomials for \(\frac{0}{0}\) forms}
\item
  \textbf{Use standard limit formulas}
\item
  \textbf{Apply L'Hôpital's rule for indeterminate forms}
\end{enumerate}

\begin{center}\rule{0.5\linewidth}{0.5pt}\end{center}

\subsection*{Common Mistakes to Avoid}\label{common-mistakes-to-avoid}

\subsubsection{\texorpdfstring{\textbf{Determinants:}}{Determinants:}}\label{determinants-1}

\begin{itemize}
\tightlist
\item
  \textbf{❌ Wrong sign in calculations}
\item
  \textbf{✅ Follow cofactor signs carefully: \((-1)^{i+j}\)}
\end{itemize}

\subsubsection{\texorpdfstring{\textbf{Logarithms:}}{Logarithms:}}\label{logarithms}

\begin{itemize}
\tightlist
\item
  \textbf{❌ \(\log(a+b) = \log a + \log b\) (WRONG)}
\item
  \textbf{✅ \(\log(ab) = \log a + \log b\) (CORRECT)}
\end{itemize}

\subsubsection{\texorpdfstring{\textbf{Trigonometry:}}{Trigonometry:}}\label{trigonometry-1}

\begin{itemize}
\tightlist
\item
  \textbf{❌ Mixing degrees and radians}
\item
  \textbf{✅ Convert to same unit first}
\end{itemize}

\subsubsection{\texorpdfstring{\textbf{Vectors:}}{Vectors:}}\label{vectors-1}

\begin{itemize}
\tightlist
\item
  \textbf{❌ \(\vec{a} \times \vec{b} = \vec{b} \times \vec{a}\)
  (WRONG)}
\item
  \textbf{✅ \(\vec{a} \times \vec{b} = -(\vec{b} \times \vec{a})\)
  (CORRECT)}
\end{itemize}

\subsubsection{\texorpdfstring{\textbf{Slopes:}}{Slopes:}}\label{slopes}

\begin{itemize}
\tightlist
\item
  \textbf{❌ Confusing parallel and perpendicular conditions}
\item
  \textbf{✅ Parallel: same slope, Perpendicular: product = -1}
\end{itemize}

\subsubsection{\texorpdfstring{\textbf{Limits:}}{Limits:}}\label{limits-1}

\begin{itemize}
\tightlist
\item
  \textbf{❌ Direct substitution without checking indeterminate forms}
\item
  \textbf{✅ Check for \(\frac{0}{0}\) or \(\frac{\infty}{\infty}\)
  first}
\end{itemize}

\begin{center}\rule{0.5\linewidth}{0.5pt}\end{center}

\subsection*{Exam Tips}\label{exam-tips}

\subsubsection{\texorpdfstring{\textbf{Time
Management:}}{Time Management:}}\label{time-management}

\begin{itemize}
\tightlist
\item
  \textbf{Spend 2 minutes per mark} (14 marks = 28 minutes for Q1)
\item
  \textbf{Start with familiar questions}
\item
  \textbf{Leave difficult problems for the end}
\end{itemize}

\subsubsection{\texorpdfstring{\textbf{Calculation
Tips:}}{Calculation Tips:}}\label{calculation-tips}

\begin{itemize}
\tightlist
\item
  \textbf{Show all steps clearly}
\item
  \textbf{Use tables for organized presentation}
\item
  \textbf{Double-check arithmetic}
\item
  \textbf{Write final answers clearly}
\end{itemize}

\subsubsection{\texorpdfstring{\textbf{Writing
Strategy:}}{Writing Strategy:}}\label{writing-strategy}

\begin{itemize}
\tightlist
\item
  \textbf{Write given information first}
\item
  \textbf{State formulas before using them}
\item
  \textbf{Include units where applicable}
\item
  \textbf{Box or underline final answers}
\end{itemize}

\subsubsection{\texorpdfstring{\textbf{Last-Minute
Checks:}}{Last-Minute Checks:}}\label{last-minute-checks}

\begin{itemize}
\tightlist
\item
  \textbf{Verify all calculations}
\item
  \textbf{Check if answers are reasonable}
\item
  \textbf{Ensure all parts are attempted}
\item
  \textbf{Review question requirements}
\end{itemize}

\begin{mnemonicbox}
\textbf{``Some People Have Curly
Brown Hair Through Proper Brushing''}

\begin{itemize}
\tightlist
\item
  \textbf{S}in 0^\circ = 0, \textbf{P}i/6 = 1/2, \textbf{H}alf = \sqrt2/2,
  \textbf{C}os complement, etc.
\end{itemize}

\textbf{Remember:} Mathematics is about \textbf{understanding patterns},
not memorizing formulas. Practice regularly and \textbf{think step by
step}!

\end{mnemonicbox}
\begin{center}\rule{0.5\linewidth}{0.5pt}\end{center}

\subsection*{Quick Reference Table}\label{quick-reference-table}

\begin{longtable}[]{@{}
  >{\raggedright\arraybackslash}p{(\linewidth - 4\tabcolsep) * \real{0.2414}}
  >{\raggedright\arraybackslash}p{(\linewidth - 4\tabcolsep) * \real{0.4483}}
  >{\raggedright\arraybackslash}p{(\linewidth - 4\tabcolsep) * \real{0.3103}}@{}}
\toprule\noalign{}
\begin{minipage}[b]{\linewidth}\raggedright
Topic
\end{minipage} & \begin{minipage}[b]{\linewidth}\raggedright
Key Formula
\end{minipage} & \begin{minipage}[b]{\linewidth}\raggedright
Example
\end{minipage} \\
\midrule\noalign{}
\endhead
\bottomrule\noalign{}
\endlastfoot
Determinant 2\times2 & \(ad - bc\) &
\(\begin{vmatrix} 2 & 3 \\ 1 & 4 \end{vmatrix} = 8-3 = 5\) \\
Slope & \(\frac{y_2-y_1}{x_2-x_1}\) & Points (1,2), (3,8):
\(m = \frac{8-2}{3-1} = 3\) \\
Distance & \(\sqrt{(x_2-x_1)^2+(y_2-y_1)^2}\) & Between (0,0), (3,4):
\(d = 5\) \\
Circle & \((x-h)^2+(y-k)^2=r^2\) & Center (1,2), radius 3 \\
Limit & \(\lim_{x \to a} f(x)\) & Direct substitution or factoring \\
\end{longtable}

\textbf{Final Tip:} Keep practicing and stay confident! 🎯


\end{document}
