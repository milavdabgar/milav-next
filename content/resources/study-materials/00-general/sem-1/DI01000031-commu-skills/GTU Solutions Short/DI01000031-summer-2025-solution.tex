%% METADATA
%% subject-code: DI01000031
%% subject-name: Communication Skills in English
%% semester: 1
%% examination: Summer-2025
%% date: 2025-06-05
%% description: Solution guide for Communication Skills in English (DI01000031) Summer 2025 exam
%% summary: Detailed solutions and explanations for the Summer 2025 exam of Communication Skills in English (DI01000031)
%% tags: study-material, solutions, communication-skills, DI01000031, 2025, summer
%% END METADATA

\documentclass{article}

% content/resources/templates/preamble.tex
\usepackage[margin=0.6in]{geometry}
\author{Milav Dabgar}
\usepackage{amsmath,amssymb,amsthm}
\usepackage{booktabs}
\usepackage{multirow}
\usepackage{xcolor}
\usepackage{tcolorbox}
\tcbuselibrary{breakable,skins}
\usepackage[colorlinks=true,linkcolor=blue]{hyperref}
\usepackage{titlesec}
\usepackage{enumitem}
\usepackage{tikz}
\usepackage{pgfplots}
\usepackage{circuitikz}
\usepackage[version=4]{mhchem}
\usepackage{longtable}
\usepackage{array}
\usepackage{float}
\usepackage{caption}
\usepackage{listings}

\lstset{
  basicstyle=\small\ttfamily,
  breaklines=true,
  breakatwhitespace=false,
  postbreak=\mbox{\textcolor{red}{$\hookrightarrow$}\space},
  float=false,
  numbers=left,
  numberstyle=\tiny\color{gray},
  numbersep=10pt,
  xleftmargin=2em,
  keywordstyle=\color{blue},
  commentstyle=\color{green!60!black},
  stringstyle=\color{purple},
  backgroundcolor=\color{gray!5},
  showstringspaces=false,
  tabsize=2,
  captionpos=b,
  keepspaces=true,
  columns=flexible
}

\pgfplotsset{compat=1.18}
\usetikzlibrary{shapes,arrows,positioning,calc,patterns,decorations.pathmorphing,decorations.markings,arrows.meta}

% Color scheme
\definecolor{headcolor}{RGB}{0,102,204}
\definecolor{keycolor}{RGB}{220,20,60}
\definecolor{solutioncolor}{RGB}{34,139,34}
\definecolor{mnemoniccolor}{RGB}{148,0,211}
\definecolor{codecolor}{RGB}{0,0,100}

% Spacing
\setlength{\parskip}{3pt}
\setlist[itemize]{nosep}
\setlist[enumerate]{nosep}

% Title formatting
\titleformat{\section}{\Large\bfseries\color{headcolor}}{\thesection}{1em}{}
\titleformat{\subsection}{\large\bfseries\color{headcolor}}{\thesubsection}{1em}{}

% Pandoc tightlist compatibility
\providecommand{\tightlist}{%
  \setlength{\itemsep}{0pt}\setlength{\parskip}{0pt}}

% Pandoc longtable compatibility
\newcounter{none}
\def\thenone{}


\title{Communication Skills in English (DI01000031) - Summer 2025 Solution}
\date{June 05, 2025}

% PDF Metadata
\hypersetup{
  pdftitle={Communication Skills in English (DI01000031) - Summer 2025 Solution},
  pdfsubject={GTU Exam Solution - Summer 2025},
  pdfauthor={Milav Dabgar},
  pdfkeywords={study-material, solutions, communication-skills, DI01000031, 2025, summer},
  pdfcreator={XeLaTeX}
}

\begin{document}
\maketitle

\setcounter{tocdepth}{5}
\tableofcontents
\newpage

% ========================================
% QUESTION 1: Multiple Choice Questions (14 marks)
% ========================================

\section{Question 1 [14 marks]}

\textbf{Choose the correct answer for the following questions.}

\begin{enumerate}
    \item \textbf{Factors that disturb the process of communication is called \_\_\_\_\_\_\_\_\_\_}
    
    \textbf{Answer}: (C) Barriers.

    \item \textbf{The response to a sender's message is called}
    
    \textbf{Answer}: (A) Feedback.

    \item \textbf{who visited the stream regularly?}
    
    \textbf{Answer}: (B) the leopard

    \item \textbf{Which is a part of the process of communication from the following?}
    
    \textbf{Answer}: (C) Encoding

    \item \textbf{Physical appearance is an example of \_\_\_\_\_\_\_\_\_\_\_\_\_\_\_\_ communication}
    
    \textbf{Answer}: (D) non-verbal

    \item \textbf{What made Bob realise that the cop was not Jimmy Wells?}
    
    \textbf{Answer}: (B) His Nose

    \item \textbf{linguistic barrier means barrier related to}
    
    \textbf{Answer}: (A) Language

    \item \textbf{The communication process starts from}
    
    \textbf{Answer}: (A) Sender

    \item \textbf{"He gives his harness bells a shake". Who is he?}
    
    \textbf{Answer}: (D) The horse

    \item \textbf{"Stopping by woods on a snowy evening" is written by}
    
    \textbf{Answer}: (A) Robert Frost

    \item \textbf{during our verbal communication, we should use}
    
    \textbf{Answer}: (B) Simple words.

    \item \textbf{The hunters used to sell leopard's skin in \_\_\_\_\_\_\_\_\_\_\_}
    
    \textbf{Answer}: (D) Delhi

    \item \textbf{Which of the following is not a form of nonverbal communication?}
    
    \textbf{Answer}: (C) Skype.

    \item \textbf{how long had Bob bean under arrest for?}
    
    \textbf{Answer}: (B) 10 mins.
\end{enumerate}

% ========================================
% QUESTION 2(A): Short Notes (6 marks)
% ========================================

\section{Question 2}

\subsection{Question 2(A) [6 marks]}
\textbf{Attempt any two.}

\subsubsection{1. Define the term communication and mention the types of communication.}

\textbf{Answer}:
Communication is the process of exchanging information, ideas, feelings, or emotions between two or more people through a common medium or channel. It involves a sender, message, channel, receiver, and feedback.

\textbf{Types of Communication:}
\begin{itemize}
    \item \textbf{Verbal Communication}: Uses words (Oral and Written).
    \item \textbf{Non-Verbal Communication}: Body language, gestures, facial expressions, tone of voice.
    \item \textbf{Formal Communication}: Official and structured (e.g., reports, emails).
    \item \textbf{Informal Communication}: Casual and unstructured (e.g., grapevine, gossip).
\end{itemize}

\subsubsection{2. Where did the poet stop in the poem and how did the horse react to it?}

\textbf{Answer}:
In "Stopping by Woods on a Snowy Evening," the poet stopped by the woods between the woods and a frozen lake on the "darkest evening of the year." The horse, accustomed to stopping only at farmhouses, found it strange to stop in the middle of nowhere without a farmhouse nearby. The horse reacted by shaking his harness bells, as if asking if there was some mistake in stopping there.

\subsubsection{3. "The birds and animals knew that that trust had been violated." What does the author want to convey by the statement?}

\textbf{Answer}:
In "The Leopard," the author conveys that the harmony between humans and nature was broken when the hunters arrived to kill the leopard. The wildlife sensed the malicious intent of the humans, contrasting with the author's peaceful presence earlier. This violation of trust signifies the disruption of the natural order and safety that the animals previously felt, highlighting the destructive nature of human greed compared to the innocent coexistence of the wild.

% ========================================
% QUESTION 2(B): Short Notes (8 marks)
% ========================================

\subsection{Question 2(B) [8 marks]}
\textbf{Attempt any two.}

\subsubsection{1. Write a short note on barriers of communication}

\textbf{Answer}:
Communication barriers are obstacles that distort or prevent the message from reaching the receiver effectively.

\begin{table}[H]
\centering
\caption{Types of Barriers}
\begin{tabularx}{\textwidth}{lXX}
\toprule
\textbf{Barrier} & \textbf{Description} & \textbf{Example} \\
\midrule
Physical & Environmental factors & Noise, distance, poor signal \\
Linguistic & Language/Semantic difficulties & Jargon, unknown language \\
Psychological & Mental/Emotional state & Prejudice, anger, lack of attention \\
Cultural & Differences in customs/beliefs & Gestures with different meanings \\
Organizational & Structure/Hierarchy issues & Complex hierarchy, information overload \\
\bottomrule
\end{tabularx}
\end{table}

Overcoming these barriers requires active listening, clear language, improper feedback, and choosing the right channel.

\paragraph{Mnemonic:}
\emph{"PPLCO: Physical, Psychological, Linguistic, Cultural, Organizational"}

\subsubsection{2. Talk about the central idea of the poem 'Stopping by Woods On a Snowy Evening'.}

\textbf{Answer}:
The central idea of Robert Frost's poem is the conflict between the attraction of nature's beauty and the demands of worldly responsibilities. The woods, "lovely, dark and deep," represent a peaceful escape and the temptation to rest or perhaps even death/oblivion. However, the "promises to keep" and "miles to go" symbolise the duties and obligations the speaker must fulfill before he can rest. It highlights the human condition of balancing desire for solitude with social and personal commitments.

\paragraph{Mnemonic:}
\emph{"Beauty vs Duty, Rest vs Responsibility"}

\subsubsection{3. Describe the ravine.}

\textbf{Answer}:
In "The Leopard," the ravine is described as a deep, shadowed place with steep sides. It was a "deep, dark, silent gorge" where the sun hardly reached the bottom. A stream flowed through it, and it was covered with dense vegetation, ferns, and rocks. It was a secluded and almost mystical place where the leopard lived, representing the raw, untouched beauty and mystery of nature. The ravine provided a perfect hideout and habitat for the leopard until the hunters invaded its sanctity.

% ========================================
% QUESTION 3(A): Grammar (6 marks)
% ========================================

\section{Question 3}

\subsection{Question 3(A) [6 marks]}
\textbf{Attempt any two.}

\subsubsection{1. Identify sentence pattern of the following sentences:}

\textbf{1.1 The thief was punished.}

\textbf{Answer}: Subject + Verb (Passive) / SV

\textbf{1.2 The guests came suddenly.}

\textbf{Answer}: Subject + Verb + Adverb (SVA)

\textbf{1.3 His mother is a doctor.}

\textbf{Answer}: Subject + Verb + Complement (SVC)

\subsubsection{2. Use proper preposition in the following sentence:}

\textbf{\_\_\_\_\_\_\_\_\_ Monday, he started his work \_\_\_\_\_\_\_\_ 9 o'clock \_\_\_\_\_\_\_\_\_\_\_ the morning.}

\textbf{Answer}: \textbf{On} Monday, he started his work \textbf{at} 9 o'clock \textbf{in} the morning.

\subsubsection{3. Select the right verb for the following sentences:}

\textbf{3.1 Gulliver's Travels \_\_\_\_\_\_\_\_\_ written by Jonathan Swift. (was,were)}

\textbf{Answer}: was

\textbf{3.2 Diamond and Platinum \_\_\_\_\_\_\_\_\_\_\_ precious metals. (is,are)}

\textbf{Answer}: are

\textbf{3.3 Time and tide \_\_\_\_\_\_\_\_\_\_ for no man. (wait,waits)}

\textbf{Answer}: wait (or waits - traditionally 'wait', modern usage sometimes 'waits', but 'wait' is standard proverb)

% ========================================
% QUESTION 3(B): Grammar (8 marks)
% ========================================

\subsection{Question 3(B) [8 marks]}
\textbf{Attempt any two}

\subsubsection{1. Fill in the blanks using proper verb}

\textbf{1.1 My friend \_\_\_\_\_\_ (work)in the garden and suddenly he\_\_\_\_\_\_\_\_\_ (hear) noise from outside.}

\textbf{Answer}: was working, heard

\textbf{1.2 She\_\_\_\_\_\_\_\_\_ (leave) restaurant before her friends \_\_\_\_\_\_\_\_\_ (arrive).}

\textbf{Answer}: had left, arrived

\subsubsection{2. Fill in the blanks using proper verb}

\textbf{2.1 I \_\_\_\_\_\_\_ speak French fluently when I was a child but now I \_\_\_\_\_\_\_\_\_\_ not.}

\textbf{Answer}: could, can

\textbf{2.2 They \_\_\_\_\_\_\_\_\_\_\_ be out. All lights are off. We \_\_\_\_\_\_\_\_\_\_ come back tomorrow.}

\textbf{Answer}: must, will / should

\subsubsection{3. Fill in the blanks using proper verb}

\textbf{3.1 My father \_\_\_\_\_\_\_\_\_\_\_ (wash) his car every Sunday. He \_\_\_\_\_\_\_\_\_\_\_\_ (do) it now because of yesterday's heavy rain.}

\textbf{Answer}: washes, is doing

\textbf{3.2 You \_\_\_\_\_\_\_\_\_ not go to the market late at night. (Could, should)}

\textbf{Answer}: should

\textbf{3.3 Tek an umbrella. It \_\_\_\_\_\_\_\_\_\_ rain later. (Might, can)}

\textbf{Answer}: Might

% ========================================
% QUESTION 4(A): Grammar (6 marks)
% ========================================

\section{Question 4}

\subsection{Question 4(A) [6 marks]}
\textbf{Attempt any two.}

\subsubsection{1. Form three sentences of the sentence pattern: S+V+O}

\textbf{Answer}:
\begin{itemize}
    \item I (S) play (V) cricket (O).
    \item She (S) eats (V) an apple (O).
    \item They (S) completed (V) the work (O).
\end{itemize}

\subsubsection{2. Fill in the blanks in the following lines:}

\textbf{She \_\_\_\_\_\_\_ (be) a fashion designer. She \_\_\_\_\_\_\_\_ (make) new clothes ever since she   \_\_\_\_\_\_\_\_ (be) a young girl.}

\textbf{Answer}: is, has been making / makes, was

\subsubsection{3. Use proper verb given in the bracket for the following lines :}

\textbf{Plastic bags \_\_\_\_\_\_\_\_ not be used because they cause pollution. These bags \_\_\_\_\_\_\_ to be   melted and reprocessed, which \_\_\_\_\_\_\_ make them even more harmful. (Need, might, should)}

\textbf{Answer}: should, need, might

% ========================================
% QUESTION 4(B): Sentence Patterns (8 marks)
% ========================================

\subsection{Question 4(B) [8 marks]}
\textbf{Attempt any two.}

\subsubsection{1. Identify adjective: "It was a sunny day and my little puppy jumped onto our red couch and   played with his new toy."}

\textbf{Answer}: sunny, little, red, new

\subsubsection{2. Join the sentence using the appropriate connector.}

\textbf{2.1 The police officer came. He was a friend of my father's. (Use of Who)}

\textbf{Answer}: The police officer who came was a friend of my father's.

\textbf{2.2 The children were playing. It started raining. ( Use of When)}

\textbf{Answer}: The children were playing when it started raining.

\textbf{2.3 She left the job. She did not mention the reason. ( Use Why)}

\textbf{Answer}: She did not mention why she left the job.

\textbf{2.4 He did not go to college. He was not keeping well. (Use because)}

\textbf{Answer}: He did not go to college because he was not keeping well.

\subsubsection{3. Select the right verb for the following sentences:}

\textbf{3.1 The rise and fall of the empire \_\_\_\_\_\_\_ dependent on its ruler. (is,are)}

\textbf{Answer}: is

\textbf{3.2 Two hours \_\_\_\_\_\_\_\_\_\_ enough for them to finish the assignment. (is, are)}

\textbf{Answer}: is

\textbf{3.3 The committee \_\_\_\_\_\_\_\_\_\_\_\_\_ taken the decision. (has, have)}

\textbf{Answer}: has

\textbf{3.4 His knowledge of various languages \_\_\_\_\_\_\_\_\_\_ appalling. ( is,are)}

\textbf{Answer}: is

% ========================================
% QUESTION 5(A): Paragraph Writing (6 marks)
% ========================================

\section{Question 5}

\subsection{Question 5(A) [6 marks]}
\textbf{Attempt any one}

\subsubsection{1. Write a paragraph on your favourite festival.}

\textbf{Answer}:
My favourite festival is Diwali, the festival of lights. It symbolizes the victory of light over darkness and good over evil. Days before Diwali, we clean and decorate our houses with rangolis and beautiful lights. On the day of Diwali, everyone wears new clothes and worships Goddess Lakshmi for prosperity. We light earthen lamps (diyas) and burst crackers to celebrate. It is a time for family gatherings, exchanging sweets, and spreading joy. I love Diwali because it brings excitement, happiness, and togetherness, filling the atmosphere with positivity and light.

\subsubsection{2. Write a paragraph on 'Climate Change: Human Impact on the Environment'}

\textbf{Answer}:
Climate change is one of the most pressing issues facing our planet today, largely driven by human activities. The burning of fossil fuels, deforestation, and industrial pollution have significantly increased the levels of greenhouse gases in the atmosphere, leading to global warming. This has resulted in melting glaciers, rising sea levels, and unpredictable weather patterns. Human impact on the environment extends to the destruction of biodiversity and pollution of oceans. It is crucial for us to adopt sustainable practices, reduce our carbon footprint, and protect nature to ensure a habitable planet for future generations.

% ========================================
% QUESTION 5(B): Email Drafts (8 marks)
% ========================================

\subsection{Question 5(B) [8 marks]}
\textbf{Attempt any one}

\subsubsection{1. Draft an email to make an inquiry of Air conditioners for your newly constructed branch office.}

\textbf{Answer}:

\begin{lstlisting}[basicstyle=\ttfamily\small, frame=single, breaklines=true]
From: branchmanager@company.com
To: sales@coolworld.com
Subject: Inquiry for Bulk Purchase of Air Conditioners

Dear Sales Team,

I am writing to inquire about purchasing air conditioners for our newly constructed branch office in Ahmedabad. We are looking for energy-efficient split AC units suitable for an office environment.

We require approximately 15 units of 1.5 Ton capacity and 5 units of 2 Ton capacity. Please provide us with a quotation that includes:
1. Product specifications and features (preferably 5-star rated).
2. Unit price and total cost.
3. Installation charges and warranty details.
4. Delivery timeline.
5. Any discount available for bulk purchase.

We are interested in brands like Voltas, Daikin, or Hitachi. We would appreciate a prompt response with your best offer.

Sincerely,

Rohan Patel
Branch Manager
XYZ Corp., Ahmedabad
Mo: 98XXXXXXXX
\end{lstlisting}

\subsubsection{2. You have placed an order for raincoats and umbrellas, but the order has not been delivered on time. Draft an email of complaint to the supplier about the delay.}

\textbf{Answer}:

\begin{lstlisting}[basicstyle=\ttfamily\small, frame=single, breaklines=true]
From: purchase@retailsstore.com
To: orders@rainwearsuppliers.com
Subject: Complaint Regarding Delay in Order Delivery - Order #4521

Dear Supplier,

I am writing to express my dissatisfaction regarding the delay in the delivery of our order No. 4521, placed on May 20, 2025. The order consisted of 100 raincoats and 200 umbrellas, with a promised delivery date of May 30, 2025.

It has now been a week past the delivery date, and we have not yet received the consignment. This delay is affecting our sales as the monsoon season has already begun.

We request you to kindly check the status of our order and expedite the delivery immediately. If the order cannot be delivered within the next 2 days, we will be forced to cancel it and source the products elsewhere.

Please provide an update on this matter urgently.

Regards,

Anjali Mehta
Purchase Manager
City Retail Store
Mo: 98XXXXXXXX
\end{lstlisting}

\end{document}
