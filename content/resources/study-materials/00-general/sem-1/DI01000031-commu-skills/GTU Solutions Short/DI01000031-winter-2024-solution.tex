%% METADATA
%% subject-code: DI01000031
%% subject-name: Communication Skills in English
%% semester: 1
%% examination: Winter-2024
%% date: 2025-01-04
%% description: Solution guide for Communication Skills in English (DI01000031) Winter 2024 exam
%% summary: Detailed solutions and explanations for the Winter 2024 exam of Communication Skills in English (DI01000031)
%% tags: study-material, solutions, communication-skills, DI01000031, 2024, winter
%% END METADATA

\documentclass{article}

% content/resources/templates/preamble.tex
\usepackage[margin=0.6in]{geometry}
\author{Milav Dabgar}
\usepackage{amsmath,amssymb,amsthm}
\usepackage{booktabs}
\usepackage{multirow}
\usepackage{xcolor}
\usepackage{tcolorbox}
\tcbuselibrary{breakable,skins}
\usepackage[colorlinks=true,linkcolor=blue]{hyperref}
\usepackage{titlesec}
\usepackage{enumitem}
\usepackage{tikz}
\usepackage{pgfplots}
\usepackage{circuitikz}
\usepackage[version=4]{mhchem}
\usepackage{longtable}
\usepackage{array}
\usepackage{float}
\usepackage{caption}
\usepackage{listings}

\lstset{
  basicstyle=\small\ttfamily,
  breaklines=true,
  breakatwhitespace=false,
  postbreak=\mbox{\textcolor{red}{$\hookrightarrow$}\space},
  float=false,
  numbers=left,
  numberstyle=\tiny\color{gray},
  numbersep=10pt,
  xleftmargin=2em,
  keywordstyle=\color{blue},
  commentstyle=\color{green!60!black},
  stringstyle=\color{purple},
  backgroundcolor=\color{gray!5},
  showstringspaces=false,
  tabsize=2,
  captionpos=b,
  keepspaces=true,
  columns=flexible
}

\pgfplotsset{compat=1.18}
\usetikzlibrary{shapes,arrows,positioning,calc,patterns,decorations.pathmorphing,decorations.markings,arrows.meta}

% Color scheme
\definecolor{headcolor}{RGB}{0,102,204}
\definecolor{keycolor}{RGB}{220,20,60}
\definecolor{solutioncolor}{RGB}{34,139,34}
\definecolor{mnemoniccolor}{RGB}{148,0,211}
\definecolor{codecolor}{RGB}{0,0,100}

% Spacing
\setlength{\parskip}{3pt}
\setlist[itemize]{nosep}
\setlist[enumerate]{nosep}

% Title formatting
\titleformat{\section}{\Large\bfseries\color{headcolor}}{\thesection}{1em}{}
\titleformat{\subsection}{\large\bfseries\color{headcolor}}{\thesubsection}{1em}{}

% Pandoc tightlist compatibility
\providecommand{\tightlist}{%
  \setlength{\itemsep}{0pt}\setlength{\parskip}{0pt}}

% Pandoc longtable compatibility
\newcounter{none}
\def\thenone{}


\title{Communication Skills in English (DI01000031) - Winter 2024 Solution}
\date{January 04, 2025}

% PDF Metadata
\hypersetup{
  pdftitle={Communication Skills in English (DI01000031) - Winter 2024 Solution},
  pdfsubject={GTU Exam Solution - Winter 2024},
  pdfauthor={Milav Dabgar},
  pdfkeywords={study-material, solutions, communication-skills, DI01000031, 2024, winter},
  pdfcreator={XeLaTeX}
}

\begin{document}
\maketitle

\setcounter{tocdepth}{5}
\tableofcontents
\newpage

% ========================================
% QUESTION 1: Multiple Choice Questions (14 marks)
% ========================================

\section{Question 1 [14 marks]}

\textbf{Choose the correct option:}

\begin{enumerate}
    \item \textbf{What is non-verbal communication?}
    
    \textbf{Answer}: (b) Non-verbal communication is about exchanging information without speaking words

    \item \textbf{Which of the following is an example of body language?}
    
    \textbf{Answer}: (e) All of the above

    \item \textbf{What does effective communication require?}
    
    \textbf{Answer}: (e) All of the above

    \item \textbf{How will you know if communication was successful?}
    
    \textbf{Answer}: (c) It has the desired outcome

    \item \textbf{What is paralanguage?}
    
    \textbf{Answer}: (b) How something is said, rather than what is said

    \item \textbf{What is efficient communication?}
    
    \textbf{Answer}: (b) Spending the minimum amount of time and effort to get the communication message across successfully

    \item \textbf{All communication is verbal}
    
    \textbf{Answer}: (b) False

    \item \textbf{The author first saw the leopard when …}
    
    \textbf{Answer}: (c) he was crossing the stream

    \item \textbf{….. is called Hill of Fairies}
    
    \textbf{Answer}: (d) Pari Tibba

    \item \textbf{The location of the story "After Twenty Years" is ….}
    
    \textbf{Answer}: (d) New York

    \item \textbf{The man was waiting for his \_\_\_\_\_\_\_\_\_\_\_\_\_\_\_\_\_\_.}
    
    \textbf{Answer}: (b) friend

    \item \textbf{The woods were filled up with…….}
    
    \textbf{Answer}: (c) snow

    \item \textbf{The horse shakes its harness bell to ask if there is a …….}
    
    \textbf{Answer}: (b) mistake

    \item \textbf{Who wrote the note?}
    
    \textbf{Answer}: (b) Jimmy
\end{enumerate}

% ========================================
% QUESTION 2(A): Short Notes (6 marks)
% ========================================

\section{Question 2}

\subsection{Question 2(A) [6 marks]}
\textbf{Write a short note. (Attempt any two)}

\subsubsection{1. Barriers to Communication}

\textbf{Answer}:
Communication barriers are obstacles that prevent effective exchange of information between sender and receiver.

\begin{table}[H]
\centering
\caption{Types of Communication Barriers}
\begin{tabularx}{\textwidth}{lXX}
\toprule
\textbf{Type} & \textbf{Examples} & \textbf{Impact} \\
\midrule
Physical & Noise, distance, poor network & Message distortion \\
Psychological & Bias, emotions, attitudes & Misinterpretation \\
Semantic & Jargon, language differences & Confusion \\
Organizational & Hierarchy, information overload & Delays, filtering \\
\bottomrule
\end{tabularx}
\end{table}

\begin{itemize}
    \item \textbf{Physical Barriers}: Environmental factors like noise, distance, or technical problems that interfere with message transmission
    \item \textbf{Psychological Barriers}: Mental factors such as prejudices, emotions, or stress that affect understanding
    \item \textbf{Semantic Barriers}: Problems with meaning due to different languages, jargon, or unclear expressions
    \item \textbf{Organizational Barriers}: Structural issues in workplace communication like complex hierarchies or information overload
\end{itemize}

\paragraph{Mnemonic:}
\emph{"PESO: Physical, Emotional, Semantic, Organizational"}

\subsubsection{2. The friendship of Jimmy and Bob}

\textbf{Answer}:
In "After Twenty Years," O. Henry portrays a deep friendship tested by divergent life paths and moral obligations.

\begin{table}[H]
\centering
\caption{Friendship Dynamics}
\begin{tabularx}{\textwidth}{lXX}
\toprule
\textbf{Aspect} & \textbf{Bob's Perspective} & \textbf{Jimmy's Perspective} \\
\midrule
Loyalty & Travels miles to keep promise & Arranges meeting but can't arrest friend \\
Memory & Remembers Jimmy fondly & Recognizes Bob despite changes \\
Conflict & Unaware of moral dilemma & Torn between duty and friendship \\
Resolution & Receives explanatory note & Chooses compromise solution \\
\bottomrule
\end{tabularx}
\end{table}

\begin{itemize}
    \item \textbf{Twenty-Year Promise}: Both friends committed to meeting at the same spot after twenty years
    \item \textbf{Divergent Paths}: Bob became a criminal while Jimmy became a police officer
    \item \textbf{Moral Dilemma}: Jimmy's professional duty conflicts with personal loyalty
    \item \textbf{Compromise Solution}: Jimmy arranges for another officer to make the arrest while explaining through a note
\end{itemize}

\paragraph{Mnemonic:}
\emph{"Promise, Path, Problem, Peace"}

\subsubsection{3. Central idea of the poem "Stopping by Woods on a Snowy Evening"}

\textbf{Answer}:
Robert Frost's poem explores the conflict between appreciating natural beauty and fulfilling life's responsibilities.

\begin{table}[H]
\centering
\caption{Central Themes}
\begin{tabularx}{\textwidth}{lXX}
\toprule
\textbf{Element} & \textbf{Symbolism} & \textbf{Meaning} \\
\midrule
Woods & Peaceful escape, nature's beauty & Temptation to pause from life's journey \\
Snow & Purity, tranquility & Natural beauty that attracts the speaker \\
Horse & Practical reality & Reminder of obligations and responsibilities \\
Journey & Life's path & Continuing duties despite momentary attractions \\
\bottomrule
\end{tabularx}
\end{table}

\begin{itemize}
    \item \textbf{Nature's Appeal}: The speaker is attracted to the quiet beauty of snow-filled woods
    \item \textbf{Responsibility vs. Desire}: Torn between staying to enjoy beauty and continuing journey
    \item \textbf{Life's Obligations}: "Promises to keep" and "miles to go" represent ongoing duties
    \item \textbf{Final Choice}: Chooses responsibility over momentary pleasure, continuing the journey
\end{itemize}

\paragraph{Mnemonic:}
\emph{"Beauty Beckons, Duty Demands, Journey Continues"}

% ========================================
% QUESTION 2(B): Answer in Brief (8 marks)
% ========================================

\subsection{Question 2(B) [8 marks]}
\textbf{Attempt any two.}

\subsubsection{1. Answer in brief}

\textbf{(i) Why did the author return to mountains?}

\textbf{Answer}:
In "The Leopard," the author returned to the mountains to escape the chaos of city life and find peace in nature. He was drawn to the solitude and beauty of the mountain environment, particularly to observe wildlife in their natural habitat. The mountains offered him a sanctuary where he could connect with nature and experience the harmony of the wilderness away from human interference.

\textbf{(ii) What did the stranger say to the policeman?}

\textbf{Answer}:
In "After Twenty Years," the stranger (Bob) told the policeman that he was waiting for his friend Jimmy Wells, with whom he had made an appointment twenty years ago to meet at that exact spot. He explained their friendship from childhood, described how successful he had become out West, and expressed confidence that Jimmy would honor their commitment despite the passage of time.

\subsubsection{2. Answer in brief}

\textbf{(i) What does the poet say about the owner of the woods?}

\textbf{Answer}:
In "Stopping by Woods on a Snowy Evening," the poet mentions that the owner of the woods lives in the village and will not see him stopping there. This suggests the owner is absent, allowing the speaker to pause and admire the woods privately. The reference emphasizes the speaker's temporary intrusion into someone else's property while enjoying a moment of natural beauty.

\textbf{(ii) What are the communication skills?}

\textbf{Answer}:
Communication skills are abilities that enable effective exchange of information, ideas, and emotions. These include verbal skills (speaking and writing clearly), non-verbal skills (body language, facial expressions), listening skills (active attention and understanding), and interpersonal skills (empathy, feedback, and adaptation to different audiences). They are essential for personal and professional success.

\subsubsection{3. Answer in brief}

\textbf{(i) What kind of man was his friend Jimmy?}

\textbf{Answer}:
Jimmy Wells was portrayed as an honest, principled, and loyal man who became a police officer. He was someone who kept his promises (meeting Bob after twenty years) but also upheld his professional duties and moral obligations. Jimmy showed both compassion and integrity by arranging for another officer to arrest Bob while explaining his actions through a personal note, demonstrating his internal conflict between friendship and duty.

\textbf{(ii) What is the basic model of communication?}

\textbf{Answer}:
The basic model of communication includes five essential elements: Sender (who creates the message), Message (the information being communicated), Channel (the medium of transmission), Receiver (who interprets the message), and Feedback (response confirming understanding). This linear model shows how information flows from one person to another, with potential for noise or barriers to interfere with effective transmission.

% ========================================
% QUESTION 3(A): Fill in the blanks (6 marks)
% ========================================

\section{Question 3}

\subsection{Question 3(A) [6 marks]}
\textbf{Attempt any two.}

\subsubsection{1. Fill in the blanks using correct form of verbs given in bracket}

(i) Whenever we meet, we \textbf{plan} a trip.

(ii) Vijay \textbf{was waiting} for me when I arrived.

(iii) Shikhar Dhawan \textbf{scored} a century in the last match.

\subsubsection{2. Fill in the blanks using correct form of verbs given in bracket}

(i) It \textbf{is raining} outside now.

(ii) Can you \textbf{help} me move this heavy table?

(iii) Hello Samay, I \textbf{haven't seen} you for ages. How are you?

\subsubsection{3. Fill in the blanks using correct form of verbs given in bracket}

(i) \textbf{Had} you ever \textbf{visited} China before your trip in 2006?

(ii) Who \textbf{invented} the computer?

(iii) Yesterday evening the phone \textbf{rang} three times while we \textbf{were having} dinner.

% ========================================
% QUESTION 3(B): Grammar (8 marks)
% ========================================

\subsection{Question 3(B) [8 marks]}
\textbf{Attempt any two}

\subsubsection{1. Identify the underlined part of speech}

(i) They \textbf{have been living} in Switzerland for seven years.

\textbf{Answer}: (D) verb

(ii) \textbf{They} themselves admitted the misconduct.

\textbf{Answer}: (a) pronoun

(iii) We are meeting \textbf{at} the cafe.

\textbf{Answer}: (c) preposition

(iv) Usha runs \textbf{fast}.

\textbf{Answer}: (c) adverb

\subsubsection{2. Choose the correct answer}

(i) Wait here \textbf{until} I get back.

\textbf{Answer}: (b) until

(ii) It was raining, \textbf{so} we turned back

\textbf{Answer}: (b) so

(iii) He is good \textbf{at} math and science.

\textbf{Answer}: (c) at

(iv) \textbf{Because} he was late, he missed the bus.

\textbf{Answer}: (a) because

\subsubsection{3. Join the sentences using appropriate conjunction}

(i) Take your umbrella \textbf{otherwise} you will get wet.

(ii) Mr. Patil \textbf{who} is known to me is a professor.

(iii) \textbf{Although} I was tired, I managed to finish the work.

(iv) The book \textbf{which} you bought yesterday is very useful.

% ========================================
% QUESTION 4(A): Modal Auxiliaries (6 marks)
% ========================================

\section{Question 4}

\subsection{Question 4(A) [6 marks]}
\textbf{Attempt any two.}

\subsubsection{1. Choose the appropriate option}

(i) We \textbf{should} keep promises.

(ii) \textbf{Will} you lend me a pen, please?

(iii) You \textbf{must} not speak loudly in the hospital.

\subsubsection{2. Fill in the blanks with appropriate modal auxiliary}

(i) She \textbf{may} come tomorrow. (Possibility)

(ii) We \textbf{should} honour our parents. (Moral obligation)

(iii) Tomorrow \textbf{will} be a holiday. (Future)

\subsubsection{3. Replace the modal auxiliaries with appropriate ones}

(i) \textbf{May} God bless you!

(ii) \textbf{Could} you lend me your scooter, please?

(iii) A patient \textbf{should} follow the doctor's advice.

% ========================================
% QUESTION 4(B): Sentence Patterns (8 marks)
% ========================================

\subsection{Question 4(B) [8 marks]}
\textbf{Attempt any two.}

\subsubsection{1. Identify the sentence pattern of given sentences}

(i) She / sings / a song

\textbf{Answer}: Subject + Verb + Object (SVO)

(ii) People/ cried.

\textbf{Answer}: Subject + Verb (SV)

(iii) You/ are/ intelligent.

\textbf{Answer}: Subject + Verb + Complement (SVC)

(iv) Lata/ sang /sweetly.

\textbf{Answer}: Subject + Verb + Adverb (SVA)

\subsubsection{2. Pick up the right verb and rewrite the sentence}

(i) The deputy along with thirty miners \textbf{was} killed.

(ii) None of them \textbf{attend} to their work these days.

(iii) The secretary and the member \textbf{have} come to visit the institute today.

(iv) My uncle and guide \textbf{is} my best friend.

\subsubsection{3. Correct the sentences}

(i) Apple pie and custard \textbf{is} my favourite dish.

(ii) Each of the boxes \textbf{weighs} 10 kgs.

(iii) Bread and butter \textbf{is} the primary need of human being.

(iv) The trouble with these guys \textbf{is} their rustic approach.

% ========================================
% QUESTION 5(A): Paragraph Writing (6 marks)
% ========================================

\section{Question 5}

\subsection{Question 5(A) [6 marks]}
\textbf{Attempt any two.}

\subsubsection{1. Write a paragraph on "Importance of Trees"}

\textbf{Answer}:
Trees play a vital role in maintaining ecological balance and supporting life on Earth. They act as natural air purifiers by absorbing carbon dioxide and releasing oxygen through photosynthesis, helping combat air pollution and climate change. Trees prevent soil erosion, maintain groundwater levels, and provide habitat for countless species of birds and animals. Economically, they supply timber, fruits, medicines, and various forest products that sustain human livelihood. Trees also offer psychological benefits by creating peaceful environments and reducing stress. Their shade provides natural cooling, reducing the need for air conditioning and conserving energy. During monsoons, trees help prevent floods by absorbing excess water. In urban areas, they improve air quality and enhance the beauty of landscapes. The presence of trees increases property values and creates pleasant living spaces. Therefore, protecting existing forests and planting more trees is essential for environmental sustainability and human survival.

\subsubsection{2. Write a paragraph on "Diwali"}

\textbf{Answer}:
Diwali, also known as the Festival of Lights, is one of the most significant and widely celebrated festivals in India. This five-day festival symbolizes the victory of light over darkness, good over evil, and knowledge over ignorance. The celebration commemorates Lord Rama's return to Ayodhya after defeating Ravana and completing fourteen years of exile. People clean and decorate their homes with colorful rangoli patterns, diyas (oil lamps), and electric lights to welcome prosperity and happiness. Families gather to perform Lakshmi Puja, seeking blessings for wealth and prosperity from Goddess Lakshmi. The festival involves exchanging sweets and gifts with relatives and friends, strengthening social bonds and relationships. Fireworks and crackers light up the night sky, creating a spectacular display of colors and sounds. Traditional sweets like laddu, barfi, and gulab jamun are prepared and shared with everyone. Markets buzz with activity as people shop for new clothes, jewelry, and decorative items. Diwali transcends religious boundaries and brings communities together in celebration of hope, renewal, and the triumph of positive forces.

\subsubsection{3. Answer the following questions}

(i) \textbf{What is the full form of email?}

\textbf{Answer}: Electronic mail

(ii) \textbf{What is the use of subject line?}

\textbf{Answer}: The subject line summarizes the email's purpose and helps the recipient understand the message content at a glance, making it easier to prioritize and organize emails.

(iii) \textbf{What does BCC stand for?}

\textbf{Answer}: Blind Carbon Copy

(iv) \textbf{The tone of your email should be\_\_\_\_\_\_\_(aggressive/polite)}

\textbf{Answer}: polite

% ========================================
% QUESTION 5(B): Email Drafts (8 marks)
% ========================================

\subsection{Question 5(B) [8 marks]}
\textbf{Attempt any two}

\subsubsection{1. Draft an email asking for the illustrated catalogue and quotation of certain surgical tools required by your hospital}

\textbf{Answer}:

\begin{lstlisting}[basicstyle=\ttfamily\small, frame=single, breaklines=true]
From: procurement@cityhospital.com
To: sales@medicalinstruments.com
Subject: Request for Catalog and Quotation - Surgical Instruments

Dear Sales Manager,

I am writing on behalf of City Hospital, a 300-bed multi-specialty hospital located in Ahmedabad, Gujarat. We are looking to procure high-quality surgical instruments for our operation theaters and require your assistance.

We would appreciate if you could provide us with:

1. Your latest illustrated catalog featuring surgical instruments
2. Detailed quotations for the following items:
   - General Surgery Kit (complete set) - 5 units
   - Laparoscopic Instruments Set - 3 units
   - Orthopedic Surgery Tools - 2 complete sets
   - Cardiovascular Surgery Instruments - 1 complete set
   - Sterilization Equipment - As per catalog

Please include the following information in your quotation:
- Unit prices and total cost
- Technical specifications
- Warranty details and service support
- Delivery timeframe
- Payment terms and conditions
- Available discounts for bulk orders

As we are expanding our surgical facilities, this represents a significant procurement opportunity. We would prefer to schedule a product demonstration at our facility if possible.

Please send the catalog and quotation to this email address. For any clarifications, you may contact me at 079-12345678.

We look forward to your prompt response and establishing a long-term business relationship.

Best regards,

Dr. Priya Sharma
Procurement Officer
City Hospital, Ahmedabad
Email: procurement@cityhospital.com
Tel: 079-12345678
\end{lstlisting}

\subsubsection{2. One of your customers has complained that the chairs supplied by you are of inferior quality and not in accordance with the samples shown to him. Draft an email expressing your regrets and showing willingness to replace the goods}

\textbf{Answer}:

\begin{lstlisting}[basicstyle=\ttfamily\small, frame=single, breaklines=true]
From: customercare@premiumfurniture.com
To: customer@email.com
Subject: Response to Your Complaint - Quality Issue with Chair Order

Dear Mr. [Customer Name],

Reference: Your complaint dated [Date] regarding Order No. PF/2024/567

I sincerely apologize for the inconvenience caused by the inferior quality of chairs delivered to you. After receiving your complaint, I immediately investigated the matter with our quality control and production teams.

We acknowledge that the chairs supplied do not match the quality standards of the samples shown to you during the ordering process. This is completely unacceptable and goes against our commitment to delivering premium furniture products.

Upon investigation, we found that there was an error in our production line which resulted in the dispatch of substandard products. We take full responsibility for this oversight.

To rectify this situation immediately, we are pleased to offer the following resolution:

1. We will collect the defective chairs from your premises at no cost to you
2. We will replace them with new chairs matching exactly the sample quality within 7 working days
3. Our quality manager will personally inspect the replacement items before dispatch
4. As a goodwill gesture, we are offering a 15% discount on your next order

Our collection team will contact you within 24 hours to schedule a convenient time for pickup and replacement delivery. Additionally, we have implemented stricter quality control measures to ensure such incidents do not recur.

We value your business and apologize once again for this unacceptable experience. Please feel free to contact me directly at 98XXXXXXXX for any immediate concerns.

Thank you for your patience and understanding.

Sincerely,

Rajesh Kumar
Customer Relations Manager
Premium Furniture Ltd.
Email: customercare@premiumfurniture.com
Direct: 98XXXXXXXX
\end{lstlisting}

\subsubsection{3. True/False}

(i) The email fonts should colourful and fancy.

\textbf{Answer}: False

(ii) Proof reading emails before hitting send is preferred.

\textbf{Answer}: True

(iii) The subject line should be long and descriptive.

\textbf{Answer}: False

(iv) Email is one of the easiest modes of communication.

\textbf{Answer}: True

\end{document}

