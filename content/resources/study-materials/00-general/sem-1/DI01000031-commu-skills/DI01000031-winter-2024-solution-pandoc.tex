\documentclass[10pt,a4paper]{article}

% content/resources/templates/preamble.tex
\usepackage[margin=0.6in]{geometry}
\author{Milav Dabgar}
\usepackage{amsmath,amssymb,amsthm}
\usepackage{booktabs}
\usepackage{multirow}
\usepackage{xcolor}
\usepackage{tcolorbox}
\tcbuselibrary{breakable,skins}
\usepackage[colorlinks=true,linkcolor=blue]{hyperref}
\usepackage{titlesec}
\usepackage{enumitem}
\usepackage{tikz}
\usepackage{pgfplots}
\usepackage{circuitikz}
\usepackage[version=4]{mhchem}
\usepackage{longtable}
\usepackage{array}
\usepackage{float}
\usepackage{caption}
\usepackage{listings}

\lstset{
  basicstyle=\small\ttfamily,
  breaklines=true,
  breakatwhitespace=false,
  postbreak=\mbox{\textcolor{red}{$\hookrightarrow$}\space},
  float=false,
  numbers=left,
  numberstyle=\tiny\color{gray},
  numbersep=10pt,
  xleftmargin=2em,
  keywordstyle=\color{blue},
  commentstyle=\color{green!60!black},
  stringstyle=\color{purple},
  backgroundcolor=\color{gray!5},
  showstringspaces=false,
  tabsize=2,
  captionpos=b,
  keepspaces=true,
  columns=flexible
}

\pgfplotsset{compat=1.18}
\usetikzlibrary{shapes,arrows,positioning,calc,patterns,decorations.pathmorphing,decorations.markings,arrows.meta}

% Color scheme
\definecolor{headcolor}{RGB}{0,102,204}
\definecolor{keycolor}{RGB}{220,20,60}
\definecolor{solutioncolor}{RGB}{34,139,34}
\definecolor{mnemoniccolor}{RGB}{148,0,211}
\definecolor{codecolor}{RGB}{0,0,100}

% Spacing
\setlength{\parskip}{3pt}
\setlist[itemize]{nosep}
\setlist[enumerate]{nosep}

% Title formatting
\titleformat{\section}{\Large\bfseries\color{headcolor}}{\thesection}{1em}{}
\titleformat{\subsection}{\large\bfseries\color{headcolor}}{\thesubsection}{1em}{}

% Pandoc tightlist compatibility
\providecommand{\tightlist}{%
  \setlength{\itemsep}{0pt}\setlength{\parskip}{0pt}}

% Pandoc longtable compatibility
\newcounter{none}
\def\thenone{}


% content/resources/templates/english-boxes.tex
% This file is currently empty - it exists to maintain consistency with the import structure.
% Add custom environments here if needed in the future.


\begin{document}

\begin{center}
{\Huge\bfseries\color{headcolor} Subject Name Solutions}\\[5pt]
{\LARGE DI01000031 -- Winter 2024}\\[3pt]
{\large Semester 1 Study Material}\\[3pt]
{\normalsize\textit{Detailed Solutions and Explanations}}
\end{center}

\vspace{10pt}

\subsection*{Question 1 [14 marks]}\label{question-1-14-marks}

\textbf{Choose the correct option:}

\begin{enumerate}
\item
\begin{solutionbox}
\textbf{What is non-verbal communication?}  (b)
  Non-verbal communication is about exchanging information without
  speaking words
\item
  \textbf{Which of the following is an example of body language?}
\end{solutionbox}
\begin{solutionbox}
(e) All of the above
\item
\end{solutionbox}
\begin{solutionbox}
\textbf{What does effective communication require?}
  (e) All of the above
\item
  \textbf{How will you know if communication was successful?}
\end{solutionbox}
\begin{solutionbox}
(c) It has the desired outcome
\item
\end{solutionbox}
\begin{solutionbox}
\textbf{What is paralanguage?}  (b) How something is
  said, rather than what is said
\item
\end{solutionbox}
\begin{solutionbox}
\textbf{What is efficient communication?}  (b)
  Spending the minimum amount of time and effort to get the
  communication message across successfully
\item
\end{solutionbox}
\begin{solutionbox}
\textbf{All communication is verbal}  (b) False
\item
  \textbf{The author first saw the leopard when \ldots{}}
\end{solutionbox}
\begin{solutionbox}
(c) he was crossing the stream
\item
\end{solutionbox}
\begin{solutionbox}
\textbf{\ldots.. is called Hill of Fairies}  (d) Pari
  Tibba
\item
  \textbf{The location of the story ``After Twenty Years'' is \ldots.}
\end{solutionbox}
\begin{solutionbox}
(d) New York
\item
  \textbf{The man was waiting for his
\end{solutionbox}
\begin{solutionbox}
\_\_\_\_\_\_\_\_\_\_\_\_\_\_\_\_\_\_.}  (b) friend
\item
\end{solutionbox}
\begin{solutionbox}
\textbf{The woods were filled up with\ldots\ldots.}
  (c) snow
\item
  \textbf{The horse shakes its harness bell to ask if there is a
\end{solutionbox}
\begin{solutionbox}
\ldots\ldots.}  (b) mistake
\item
\end{solutionbox}
\begin{solutionbox}
\textbf{Who wrote the note?}  (b) Jimmy
\end{enumerate}

\end{solutionbox}
\subsection*{Question 2(A) [6 marks]}\label{q2a}

\textbf{Write a short note. (Attempt any two)}

\subsubsection{1. Barriers to
Communication}\label{barriers-to-communication}

\begin{solutionbox}
Communication barriers are obstacles that prevent
effective exchange of information between sender and receiver.


\vspace{-5pt}
\captionof{table}{Types of Communication Barriers}
\vspace{-10pt}
\begin{longtable}[]{@{}lll@{}}
\toprule\noalign{}
Type & Examples & Impact \\
\midrule\noalign{}
\endhead
\bottomrule\noalign{}
\endlastfoot
Physical & Noise, distance, poor network & Message distortion \\
Psychological & Bias, emotions, attitudes & Misinterpretation \\
Semantic & Jargon, language differences & Confusion \\
Organizational & Hierarchy, information overload & Delays, filtering \\
\end{longtable}

\begin{itemize}
\tightlist
\item
  \textbf{Physical Barriers}: Environmental factors like noise,
  distance, or technical problems that interfere with message
  transmission
\item
  \textbf{Psychological Barriers}: Mental factors such as prejudices,
  emotions, or stress that affect understanding
\item
  \textbf{Semantic Barriers}: Problems with meaning due to different
  languages, jargon, or unclear expressions
\item
  \textbf{Organizational Barriers}: Structural issues in workplace
  communication like complex hierarchies or information overload
\end{itemize}

\end{solutionbox}
\begin{mnemonicbox}
``PESO: Physical, Emotional, Semantic,
Organizational''

\end{mnemonicbox}
\subsubsection{2. The friendship of Jimmy and
Bob}\label{the-friendship-of-jimmy-and-bob}

\begin{solutionbox}
In ``After Twenty Years,'' O. Henry portrays a deep
friendship tested by divergent life paths and moral obligations.


\vspace{-5pt}
\captionof{table}{Friendship Dynamics}
\vspace{-10pt}
\begin{longtable}[]{@{}
  >{\raggedright\arraybackslash}p{(\linewidth - 4\tabcolsep) * \real{0.1778}}
  >{\raggedright\arraybackslash}p{(\linewidth - 4\tabcolsep) * \real{0.4000}}
  >{\raggedright\arraybackslash}p{(\linewidth - 4\tabcolsep) * \real{0.4222}}@{}}
\toprule\noalign{}
\begin{minipage}[b]{\linewidth}\raggedright
Aspect
\end{minipage} & \begin{minipage}[b]{\linewidth}\raggedright
Bob's Perspective
\end{minipage} & \begin{minipage}[b]{\linewidth}\raggedright
Jimmy's Perspective
\end{minipage} \\
\midrule\noalign{}
\endhead
\bottomrule\noalign{}
\endlastfoot
Loyalty & Travels miles to keep promise & Arranges meeting but can't
arrest friend \\
Memory & Remembers Jimmy fondly & Recognizes Bob despite changes \\
Conflict & Unaware of moral dilemma & Torn between duty and
friendship \\
Resolution & Receives explanatory note & Chooses compromise solution \\
\end{longtable}

\begin{itemize}
\tightlist
\item
  \textbf{Twenty-Year Promise}: Both friends committed to meeting at the
  same spot after twenty years
\item
  \textbf{Divergent Paths}: Bob became a criminal while Jimmy became a
  police officer
\item
  \textbf{Moral Dilemma}: Jimmy's professional duty conflicts with
  personal loyalty
\item
  \textbf{Compromise Solution}: Jimmy arranges for another officer to
  make the arrest while explaining through a note
\end{itemize}

\end{solutionbox}
\begin{mnemonicbox}
``Promise, Path, Problem, Peace''

\end{mnemonicbox}
\subsubsection{3. Central idea of the poem ``Stopping by Woods on a
Snowy
Evening''}\label{central-idea-of-the-poem-stopping-by-woods-on-a-snowy-evening}

\begin{solutionbox}
Robert Frost's poem explores the conflict between
appreciating natural beauty and fulfilling life's responsibilities.


\vspace{-5pt}
\captionof{table}{Central Themes}
\vspace{-10pt}
\begin{longtable}[]{@{}
  >{\raggedright\arraybackslash}p{(\linewidth - 4\tabcolsep) * \real{0.3103}}
  >{\raggedright\arraybackslash}p{(\linewidth - 4\tabcolsep) * \real{0.3793}}
  >{\raggedright\arraybackslash}p{(\linewidth - 4\tabcolsep) * \real{0.3103}}@{}}
\toprule\noalign{}
\begin{minipage}[b]{\linewidth}\raggedright
Element
\end{minipage} & \begin{minipage}[b]{\linewidth}\raggedright
Symbolism
\end{minipage} & \begin{minipage}[b]{\linewidth}\raggedright
Meaning
\end{minipage} \\
\midrule\noalign{}
\endhead
\bottomrule\noalign{}
\endlastfoot
Woods & Peaceful escape, nature's beauty & Temptation to pause from
life's journey \\
Snow & Purity, tranquility & Natural beauty that attracts the speaker \\
Horse & Practical reality & Reminder of obligations and
responsibilities \\
Journey & Life's path & Continuing duties despite momentary
attractions \\
\end{longtable}

\begin{itemize}
\tightlist
\item
  \textbf{Nature's Appeal}: The speaker is attracted to the quiet beauty
  of snow-filled woods
\item
  \textbf{Responsibility vs.~Desire}: Torn between staying to enjoy
  beauty and continuing journey
\item
  \textbf{Life's Obligations}: ``Promises to keep'' and ``miles to go''
  represent ongoing duties
\item
  \textbf{Final Choice}: Chooses responsibility over momentary pleasure,
  continuing the journey
\end{itemize}

\end{solutionbox}
\begin{mnemonicbox}
``Beauty Beckons, Duty Demands, Journey Continues''

\end{mnemonicbox}
\subsection*{Question 2(B) [8 marks]}\label{q2b}

\textbf{Attempt any two.}

\subsubsection{1. Answer in brief}\label{answer-in-brief}

\textbf{(i) Why did the author return to mountains?}

\begin{solutionbox}
In ``The Leopard,'' the author returned to the
mountains to escape the chaos of city life and find peace in nature. He
was drawn to the solitude and beauty of the mountain environment,
particularly to observe wildlife in their natural habitat. The mountains
offered him a sanctuary where he could connect with nature and
experience the harmony of the wilderness away from human interference.

\textbf{(ii) What did the stranger say to the policeman?}

\end{solutionbox}
\begin{solutionbox}
In ``After Twenty Years,'' the stranger (Bob) told the
policeman that he was waiting for his friend Jimmy Wells, with whom he
had made an appointment twenty years ago to meet at that exact spot. He
explained their friendship from childhood, described how successful he
had become out West, and expressed confidence that Jimmy would honor
their commitment despite the passage of time.

\end{solutionbox}
\subsubsection{2. Answer in brief}\label{answer-in-brief-1}

\textbf{(i) What does the poet say about the owner of the woods?}

\begin{solutionbox}
In ``Stopping by Woods on a Snowy Evening,'' the poet
mentions that the owner of the woods lives in the village and will not
see him stopping there. This suggests the owner is absent, allowing the
speaker to pause and admire the woods privately. The reference
emphasizes the speaker's temporary intrusion into someone else's
property while enjoying a moment of natural beauty.

\textbf{(ii) What are the communication skills?}

\end{solutionbox}
\begin{solutionbox}
Communication skills are abilities that enable
effective exchange of information, ideas, and emotions. These include
verbal skills (speaking and writing clearly), non-verbal skills (body
language, facial expressions), listening skills (active attention and
understanding), and interpersonal skills (empathy, feedback, and
adaptation to different audiences). They are essential for personal and
professional success.

\end{solutionbox}
\subsubsection{3. Answer in brief}\label{answer-in-brief-2}

\textbf{(i) What kind of man was his friend Jimmy?}

\begin{solutionbox}
Jimmy Wells was portrayed as an honest, principled, and
loyal man who became a police officer. He was someone who kept his
promises (meeting Bob after twenty years) but also upheld his
professional duties and moral obligations. Jimmy showed both compassion
and integrity by arranging for another officer to arrest Bob while
explaining his actions through a personal note, demonstrating his
internal conflict between friendship and duty.

\textbf{(ii) What is the basic model of communication?}

\end{solutionbox}
\begin{solutionbox}
The basic model of communication includes five
essential elements: Sender (who creates the message), Message (the
information being communicated), Channel (the medium of transmission),
Receiver (who interprets the message), and Feedback (response confirming
understanding). This linear model shows how information flows from one
person to another, with potential for noise or barriers to interfere
with effective transmission.

\end{solutionbox}
\subsection*{Question 3(A) [6 marks]}\label{q3a}

\textbf{Attempt any two.}

\subsubsection{1. Fill in the blanks using correct form of verbs given
in
bracket}\label{fill-in-the-blanks-using-correct-form-of-verbs-given-in-bracket}

\begin{enumerate}
\tightlist
\item
  Whenever we meet, we \textbf{plan} a trip.
\item
  Vijay \textbf{was waiting} for me when I arrived.
\item
  Shikhar Dhawan \textbf{scored} a century in the last match.
\end{enumerate}

\subsubsection{2. Fill in the blanks using correct form of verbs given
in
bracket}\label{fill-in-the-blanks-using-correct-form-of-verbs-given-in-bracket-1}

\begin{enumerate}
\tightlist
\item
  It \textbf{is raining} outside now.
\item
  Can you \textbf{help} me move this heavy table?
\item
  Hello Samay, I \textbf{haven't seen} you for ages. How are you?
\end{enumerate}

\subsubsection{3. Fill in the blanks using correct form of verbs given
in
bracket}\label{fill-in-the-blanks-using-correct-form-of-verbs-given-in-bracket-2}

\begin{enumerate}
\tightlist
\item
  \textbf{Had} you ever \textbf{visited} China before your trip in 2006?
\item
  Who \textbf{invented} the computer?
\item
  Yesterday evening the phone \textbf{rang} three times while we
  \textbf{were having} dinner.
\end{enumerate}

\subsection*{Question 3(B) [8 marks]}\label{q3b}

\textbf{Attempt any two}

\subsubsection{1. Identify the underlined part of
speech}\label{identify-the-underlined-part-of-speech}

\begin{enumerate}
\item
  They \textbf{have been living} in Switzerland for seven years.
\begin{solutionbox}
(D) verb
\item
\end{solutionbox}
\begin{solutionbox}
\textbf{They} themselves admitted the misconduct.  (a)
  pronoun
\item
\end{solutionbox}
\begin{solutionbox}
We are meeting \textbf{at} the cafe.  (c) preposition
\item
\end{solutionbox}
\begin{solutionbox}
Usha runs \textbf{fast}.  (c) adverb
\end{enumerate}

\end{solutionbox}
\subsubsection{2. Choose the correct
answer}\label{choose-the-correct-answer}

\begin{enumerate}
\item
\begin{solutionbox}
Wait here \textbf{until} I get back.  (b) until
\item
\end{solutionbox}
\begin{solutionbox}
It was raining, \textbf{so} we turned back  (b) so
\item
\end{solutionbox}
\begin{solutionbox}
He is good \textbf{at} math and science.  (c) at
\item
\end{solutionbox}
\begin{solutionbox}
\textbf{Because} he was late, he missed the bus.  (a)
  because
\end{enumerate}

\end{solutionbox}
\subsubsection{3. Join the sentences using appropriate
conjunction}\label{join-the-sentences-using-appropriate-conjunction}

\begin{enumerate}
\item
  Take your umbrella \textbf{otherwise} you will get wet.
\item
  Mr.~Patil \textbf{who} is known to me is a professor.
\item
  \textbf{Although} I was tired, I managed to finish the work.
\item
  The book \textbf{which} you bought yesterday is very useful.
\end{enumerate}

\subsection*{Question 4(A) [6 marks]}\label{q4a}

\textbf{Attempt any two.}

\subsubsection{1. Choose the appropriate
option}\label{choose-the-appropriate-option}

\begin{enumerate}
\tightlist
\item
  We \textbf{should} keep promises.
\item
  \textbf{Will} you lend me a pen, please?
\item
  You \textbf{must} not speak loudly in the hospital.
\end{enumerate}

\subsubsection{2. Fill in the blanks with appropriate modal
auxiliary}\label{fill-in-the-blanks-with-appropriate-modal-auxiliary}

\begin{enumerate}
\tightlist
\item
  She \textbf{may} come tomorrow. (Possibility)
\item
  We \textbf{should} honour our parents. (Moral obligation)
\item
  Tomorrow \textbf{will} be a holiday. (Future)
\end{enumerate}

\subsubsection{3. Replace the modal auxiliaries with appropriate
ones}\label{replace-the-modal-auxiliaries-with-appropriate-ones}

\begin{enumerate}
\tightlist
\item
  \textbf{May} God bless you!
\item
  \textbf{Could} you lend me your scooter, please?
\item
  A patient \textbf{should} follow the doctor's advice.
\end{enumerate}

\subsection*{Question 4(B) [8 marks]}\label{q4b}

\textbf{Attempt any two.}

\subsubsection{1. Identify the sentence pattern of given
sentences}\label{identify-the-sentence-pattern-of-given-sentences}

\begin{enumerate}
\item
\begin{solutionbox}
She / sings / a song  Subject + Verb + Object (SVO)
\item
\end{solutionbox}
\begin{solutionbox}
People/ cried.  Subject + Verb (SV)
\item
\end{solutionbox}
\begin{solutionbox}
You/ are/ intelligent.  Subject + Verb + Complement
  (SVC)
\item
\end{solutionbox}
\begin{solutionbox}
Lata/ sang /sweetly.  Subject + Verb + Adverb (SVA)
\end{enumerate}

\end{solutionbox}
\subsubsection{2. Pick up the right verb and rewrite the
sentence}\label{pick-up-the-right-verb-and-rewrite-the-sentence}

\begin{enumerate}
\item
  The deputy along with thirty miners \textbf{was} killed.
\item
  None of them \textbf{attend} to their work these days.
\item
  The secretary and the member \textbf{have} come to visit the institute
  today.
\item
  My uncle and guide \textbf{is} my best friend.
\end{enumerate}

\subsubsection{3. Correct the sentences}\label{correct-the-sentences}

\begin{enumerate}
\item
  Apple pie and custard \textbf{is} my favourite dish.
\item
  Each of the boxes \textbf{weighs} 10 kgs.
\item
  Bread and butter \textbf{is} the primary need of human being.
\item
  The trouble with these guys \textbf{is} their rustic approach.
\end{enumerate}

\subsection*{Question 5(A) [6 marks]}\label{q5a}

\textbf{Attempt any two.}

\subsubsection{1. Write a paragraph on ``Importance of
Trees''}\label{write-a-paragraph-on-importance-of-trees}

\begin{solutionbox}
Trees play a vital role in maintaining ecological
balance and supporting life on Earth. They act as natural air purifiers
by absorbing carbon dioxide and releasing oxygen through photosynthesis,
helping combat air pollution and climate change. Trees prevent soil
erosion, maintain groundwater levels, and provide habitat for countless
species of birds and animals. Economically, they supply timber, fruits,
medicines, and various forest products that sustain human livelihood.
Trees also offer psychological benefits by creating peaceful
environments and reducing stress. Their shade provides natural cooling,
reducing the need for air conditioning and conserving energy. During
monsoons, trees help prevent floods by absorbing excess water. In urban
areas, they improve air quality and enhance the beauty of landscapes.
The presence of trees increases property values and creates pleasant
living spaces. Therefore, protecting existing forests and planting more
trees is essential for environmental sustainability and human survival.

\end{solutionbox}
\subsubsection{2. Write a paragraph on
``Diwali''}\label{write-a-paragraph-on-diwali}

\begin{solutionbox}
Diwali, also known as the Festival of Lights, is one of
the most significant and widely celebrated festivals in India. This
five-day festival symbolizes the victory of light over darkness, good
over evil, and knowledge over ignorance. The celebration commemorates
Lord Rama's return to Ayodhya after defeating Ravana and completing
fourteen years of exile. People clean and decorate their homes with
colorful rangoli patterns, diyas (oil lamps), and electric lights to
welcome prosperity and happiness. Families gather to perform Lakshmi
Puja, seeking blessings for wealth and prosperity from Goddess Lakshmi.
The festival involves exchanging sweets and gifts with relatives and
friends, strengthening social bonds and relationships. Fireworks and
crackers light up the night sky, creating a spectacular display of
colors and sounds. Traditional sweets like laddu, barfi, and gulab jamun
are prepared and shared with everyone. Markets buzz with activity as
people shop for new clothes, jewelry, and decorative items. Diwali
transcends religious boundaries and brings communities together in
celebration of hope, renewal, and the triumph of positive forces.

\end{solutionbox}
\subsubsection{3. Answer the following
questions}\label{answer-the-following-questions}

\begin{enumerate}
\item
\begin{solutionbox}
\textbf{What is the full form of email?}  Electronic
  mail
\item
\end{solutionbox}
\begin{solutionbox}
\textbf{What is the use of subject line?}  The subject
  line summarizes the email's purpose and helps the recipient understand
  the message content at a glance, making it easier to prioritize and
  organize emails.
\item
\end{solutionbox}
\begin{solutionbox}
\textbf{What does BCC stand for?}  Blind Carbon Copy
\item
  \textbf{The tone of your email should
\end{solutionbox}
\begin{solutionbox}
be\_\_\_\_\_\_\_(aggressive/polite)}  polite
\end{enumerate}

\end{solutionbox}
\subsection*{Question 5(B) [8 marks]}\label{q5b}

\textbf{Attempt any two}

\subsubsection{1. Draft an email asking for the illustrated catalogue
and quotation of certain surgical tools required by your
hospital}\label{draft-an-email-asking-for-the-illustrated-catalogue-and-quotation-of-certain-surgical-tools-required-by-your-hospital}

\begin{solutionbox}

\begin{verbatim}
From: procurement@cityhospital.com
To: sales@medicalinstruments.com
Subject: Request for Catalog and Quotation - Surgical Instruments

Dear Sales Manager,

I am writing on behalf of City Hospital, a 300-bed multi-specialty hospital located in Ahmedabad, Gujarat. We are looking to procure high-quality surgical instruments for our operation theaters and require your assistance.

We would appreciate if you could provide us with:

1. Your latest illustrated catalog featuring surgical instruments
2. Detailed quotations for the following items:
   - General Surgery Kit (complete set) - 5 units
   - Laparoscopic Instruments Set - 3 units
   - Orthopedic Surgery Tools - 2 complete sets
   - Cardiovascular Surgery Instruments - 1 complete set
   - Sterilization Equipment - As per catalog

Please include the following information in your quotation:
- Unit prices and total cost
- Technical specifications
- Warranty details and service support
- Delivery timeframe
- Payment terms and conditions
- Available discounts for bulk orders

As we are expanding our surgical facilities, this represents a significant procurement opportunity. We would prefer to schedule a product demonstration at our facility if possible.

Please send the catalog and quotation to this email address. For any clarifications, you may contact me at 079-12345678.

We look forward to your prompt response and establishing a long-term business relationship.

Best regards,

Dr. Priya Sharma
Procurement Officer
City Hospital, Ahmedabad
Email: procurement@cityhospital.com
Tel: 079-12345678
\end{verbatim}

\end{solutionbox}
\subsubsection{2. One of your customers has complained that the chairs
supplied by you are of inferior quality and not in accordance with the
samples shown to him. Draft an email expressing your regrets and showing
willingness to replace the
goods}\label{one-of-your-customers-has-complained-that-the-chairs-supplied-by-you-are-of-inferior-quality-and-not-in-accordance-with-the-samples-shown-to-him.-draft-an-email-expressing-your-regrets-and-showing-willingness-to-replace-the-goods}

\begin{solutionbox}

\begin{verbatim}
From: customercare@premiumfurniture.com
To: customer@email.com
Subject: Response to Your Complaint - Quality Issue with Chair Order

Dear Mr. [Customer Name],

Reference: Your complaint dated [Date] regarding Order No. PF/2024/567

I sincerely apologize for the inconvenience caused by the inferior quality of chairs delivered to you. After receiving your complaint, I immediately investigated the matter with our quality control and production teams.

We acknowledge that the chairs supplied do not match the quality standards of the samples shown to you during the ordering process. This is completely unacceptable and goes against our commitment to delivering premium furniture products.

Upon investigation, we found that there was an error in our production line which resulted in the dispatch of substandard products. We take full responsibility for this oversight.

To rectify this situation immediately, we are pleased to offer the following resolution:

1. We will collect the defective chairs from your premises at no cost to you
2. We will replace them with new chairs matching exactly the sample quality within 7 working days
3. Our quality manager will personally inspect the replacement items before dispatch
4. As a goodwill gesture, we are offering a 15% discount on your next order

Our collection team will contact you within 24 hours to schedule a convenient time for pickup and replacement delivery. Additionally, we have implemented stricter quality control measures to ensure such incidents do not recur.

We value your business and apologize once again for this unacceptable experience. Please feel free to contact me directly at 98XXXXXXXX for any immediate concerns.

Thank you for your patience and understanding.

Sincerely,

Rajesh Kumar
Customer Relations Manager
Premium Furniture Ltd.
Email: customercare@premiumfurniture.com
Direct: 98XXXXXXXX
\end{verbatim}

\end{solutionbox}
\subsubsection{3. True/False}\label{truefalse}

\begin{enumerate}
\item
\begin{solutionbox}
The email fonts should colourful and fancy.  False
\item
  Proof reading emails before hitting send is preferred.
\end{solutionbox}
\begin{solutionbox}
True
\item
\end{solutionbox}
\begin{solutionbox}
The subject line should be long and descriptive.
  False
\item
\end{solutionbox}
\begin{solutionbox}
Email is one of the easiest modes of communication.
  True
\end{enumerate}

\end{solutionbox}

\end{document}
