\documentclass[10pt,a4paper]{article}

% content/resources/templates/preamble.tex
\usepackage[margin=0.6in]{geometry}
\author{Milav Dabgar}
\usepackage{amsmath,amssymb,amsthm}
\usepackage{booktabs}
\usepackage{multirow}
\usepackage{xcolor}
\usepackage{tcolorbox}
\tcbuselibrary{breakable,skins}
\usepackage[colorlinks=true,linkcolor=blue]{hyperref}
\usepackage{titlesec}
\usepackage{enumitem}
\usepackage{tikz}
\usepackage{pgfplots}
\usepackage{circuitikz}
\usepackage[version=4]{mhchem}
\usepackage{longtable}
\usepackage{array}
\usepackage{float}
\usepackage{caption}
\usepackage{listings}

\lstset{
  basicstyle=\small\ttfamily,
  breaklines=true,
  breakatwhitespace=false,
  postbreak=\mbox{\textcolor{red}{$\hookrightarrow$}\space},
  float=false,
  numbers=left,
  numberstyle=\tiny\color{gray},
  numbersep=10pt,
  xleftmargin=2em,
  keywordstyle=\color{blue},
  commentstyle=\color{green!60!black},
  stringstyle=\color{purple},
  backgroundcolor=\color{gray!5},
  showstringspaces=false,
  tabsize=2,
  captionpos=b,
  keepspaces=true,
  columns=flexible
}

\pgfplotsset{compat=1.18}
\usetikzlibrary{shapes,arrows,positioning,calc,patterns,decorations.pathmorphing,decorations.markings,arrows.meta}

% Color scheme
\definecolor{headcolor}{RGB}{0,102,204}
\definecolor{keycolor}{RGB}{220,20,60}
\definecolor{solutioncolor}{RGB}{34,139,34}
\definecolor{mnemoniccolor}{RGB}{148,0,211}
\definecolor{codecolor}{RGB}{0,0,100}

% Spacing
\setlength{\parskip}{3pt}
\setlist[itemize]{nosep}
\setlist[enumerate]{nosep}

% Title formatting
\titleformat{\section}{\Large\bfseries\color{headcolor}}{\thesection}{1em}{}
\titleformat{\subsection}{\large\bfseries\color{headcolor}}{\thesubsection}{1em}{}

% Pandoc tightlist compatibility
\providecommand{\tightlist}{%
  \setlength{\itemsep}{0pt}\setlength{\parskip}{0pt}}

% Pandoc longtable compatibility
\newcounter{none}
\def\thenone{}


% content/resources/templates/gujarati-boxes.tex
\usepackage{fontspec}
\usepackage{polyglossia}

% Set Gujarati as main language (document is primarily in Gujarati)
% Note: gloss-gujarati.ldf doesn't exist in polyglossia, but it will use hyphenation patterns
\setdefaultlanguage{gujarati}
\setotherlanguage{english}

% Configure Gujarati font properly
% Use Language=Default to prevent polyglossia from trying to add language-specific features
% that don't exist for Gujarati, which causes "empty feature" warnings
\newfontfamily\gujaratifont[Script=Gujarati,AutoFakeBold=2.5,AutoFakeSlant=0.3]{Noto Sans Gujarati}
\setmainfont[Script=Gujarati,AutoFakeBold=2.5,AutoFakeSlant=0.3]{Noto Sans Gujarati}
% Use Noto Sans Gujarati for monospace to support Gujarati in text
\setmonofont[Scale=0.9]{Noto Sans Gujarati}

% Configure English to use the same font
\newfontfamily\englishfont[Script=Gujarati,AutoFakeBold=2.5,AutoFakeSlant=0.3]{Noto Sans Gujarati}

% Translations for polyglossia
\gappto\captionsgujarati{
  \renewcommand{\tablename}{કોષ્ટક}
  \renewcommand{\figurename}{આકૃતિ}
}

% Helper for TikZ nodes to ensure Gujarati font
\newcommand{\gu}[1]{{\gujaratifont #1}}

% Custom environments
\newtcolorbox{solutionbox}{
    breakable,
    enhanced,
    colback=solutioncolor!5!white,
    colframe=solutioncolor!75!black,
    fonttitle=\bfseries,
    title=જવાબ
}

\newtcolorbox{solutionboxnobreak}{
 colback=solutioncolor!5!white,
 colframe=solutioncolor!75!black,
 fonttitle=\bfseries,
 title=જવાબ
}

\newtcolorbox{keyformula}{
 breakable,
 enhanced,
 colback=keycolor!5!white,
 colframe=keycolor!75!black,
 fonttitle=\bfseries,
 title=રાસાયણિક સમીકરણ/સૂત્ર
}

\newtcolorbox{mnemonicbox}{
 breakable,
 enhanced,
 colback=mnemoniccolor!5!white,
 colframe=mnemoniccolor!75!black,
 fonttitle=\bfseries,
 title=મેમરી ટ્રીક
}


\begin{document}

\begin{center}
{\Huge\bfseries\color{headcolor} Modern Physics (Gujarati)}\\[5pt]
{\LARGE DI01000061 -- Winter 2024}\\[3pt]
{\large Semester 1 Study Material}\\[3pt]
{\normalsize\textit{Detailed Solutions and Explanations}}
\end{center}

\vspace{10pt}

\subsection*{પ્રશ્ન 1 - ખાલી જગ્યા પૂરો/બહુવિકલ્પ પ્રશ્નો [14
ગુણ]}\label{uxaaauxab0uxab6uxaa8-1---uxa96uxab2-uxa9cuxa97uxaaf-uxaaauxab0uxaacuxab9uxab5uxa95uxab2uxaaa-uxaaauxab0uxab6uxaa8-14-uxa97uxaa3}

\begin{solutionbox}

\begin{longtable}[]{@{}llll@{}}
\toprule\noalign{}
પ્રશ્ન & જવાબ & પ્રશ્ન & જવાબ \\
\midrule\noalign{}
\endhead
\bottomrule\noalign{}
\endlastfoot
(1) & (a) Si & (8) & (b) 0.5 Hz \\
(2) & (a) 1.50 & (9) & (a) 300000 km/s \\
(3) & (b) વધારે & (10) & (b) ઘન \\
(4) & (c) 4 & (11) & (a) શૃંગ અને ગર્ત \\
(5) & (d) પૂર્ણ આંતરિક પરાવર્તન & (12) & (b) એકરંગી \\
(6) & (d) આવૃત્તિ & (13) & (a) સિંગલ મોડ \\
(7) & (a) કુલંબ & (14) & (b) 45^\circ \\
\end{longtable}

\end{solutionbox}
\begin{mnemonicbox}
``સિલિકોન ગ્લાસ બ્રિજ ઓપ્ટિક આવૃત્તિ કુલંબ Hz ઘન શૃંગ મોનો
સિંગલ 45''

\end{mnemonicbox}
\subsection*{પ્રશ્ન 2(A) - કોઈપણ બેના જવાબ આપો [6
ગુણ]}\label{q2a}

\subsubsection{પ્રશ્ન 2(A)(1) [3
ગુણ]}\label{uxaaauxab0uxab6uxaa8-2a1-3-uxa97uxaa3}

\textbf{ચોકસાઈ અને સચોટતા વચ્ચેનો તફાવત આપો.}

\begin{solutionbox}

\begin{longtable}[]{@{}lll@{}}
\toprule\noalign{}
પરિમાણ & ચોકસાઈ (Accuracy) & સચોટતા (Precision) \\
\midrule\noalign{}
\endhead
\bottomrule\noalign{}
\endlastfoot
વ્યાખ્યા & સાચા મૂલ્યની નજીક & પુનરાવર્તિત માપનોની સુસંગતતા \\
કેન્દ્ર & સાચું હોવું & પુનઃઉત્પાદન \\
ભૂલનો પ્રકાર & વ્યવસ્થિત ભૂલ & અવ્યવસ્થિત ભૂલ \\
ઉદાહરણ & લક્ષ્યમાં મારવું & સમાન જગ્યાએ વારંવાર મારવું \\
\end{longtable}

\begin{itemize}
\tightlist
\item
  \textbf{ચોકસાઈ}: માપ વાસ્તવિક મૂલ્યની કેટલી નજીક છે
\item
  \textbf{સચોટતા}: પુનરાવર્તિત માપન એકબીજાની કેટલી નજીક છે
\end{itemize}

\end{solutionbox}
\begin{mnemonicbox}
``ચોકસાઈ વાસ્તવિક લક્ષ્ય, સચોટતા સુસંગત પુનરાવર્તન''

\end{mnemonicbox}
\subsubsection{પ્રશ્ન 2(A)(2) [3
ગુણ]}\label{uxaaauxab0uxab6uxaa8-2a2-3-uxa97uxaa3}

\textbf{માઇક્રોમીટર સ્ક્રૂ દ્વારા માપવામાં આવતા ગોળાનો વ્યાસ નક્કી કરો, મુખ્ય
માપપટ્ટીનું માપ 5 mm અને વર્તુળાકાર માપપટ્ટીનો 50મો વિભાગ બેઝ લાઇન સાથે મેચ થાય છે.
આ સાધનની લ.મા.શ 0.01 mm છે.}

\begin{solutionbox}

\begin{verbatim}
આપેલ:
મુખ્ય માપપટ્ટી વાંચન (MSR) = 5 mm
વર્તુળાકાર માપપટ્ટી વાંચન (CSR) = 50 વિભાગ
લઘુતમ માપશક્તિ (LC) = 0.01 mm

સૂત્ર: કુલ વાંચન = MSR + (CSR \times LC)
કુલ વાંચન = 5 + (50 \times 0.01)
કુલ વાંચન = 5 + 0.5 = 5.5 mm
\end{verbatim}

\textbf{ગોળાનો વ્યાસ = 5.5 mm}

\end{solutionbox}
\begin{mnemonicbox}
``મુખ્ય વાંચન + વર્તુળાકાર \times લઘુતમ માપશક્તિ''

\end{mnemonicbox}
\subsubsection{પ્રશ્ન 2(A)(3) [3
ગુણ]}\label{uxaaauxab0uxab6uxaa8-2a3-3-uxa97uxaa3}

\textbf{જ્યારે 4 \muF કેપેસિટન્સ ધરાવતા કેપેસિટરને 12 volt બેટરી સાથે જોડતા કેપેસિટરની
બંને પ્લેટ પર સંગ્રહિત થતાં વિદ્યુતભારના જથ્થાની ગણતરી કરો.}

\begin{solutionbox}

\begin{verbatim}
આપેલ:
કેપેસિટન્સ (C) = 4 µF = 4 \times 10^{-}^{6} F
વોલ્ટેજ (V) = 12 V

સૂત્ર: Q = CV
Q = 4 \times 10^{-}^{6} \times 12
Q = 48 \times 10^{-}^{6} C
Q = 48 µC
\end{verbatim}

\textbf{સંગ્રહિત વિદ્યુતભાર = 48 µC}

\end{solutionbox}
\begin{mnemonicbox}
``ચાર્જ બરાબર કેપેસિટન્સ ગુણ્યે વોલ્ટેજ''

\end{mnemonicbox}
\subsection*{પ્રશ્ન 2(B) - કોઈપણ બેના જવાબ આપો [8
ગુણ]}\label{q2b}

\subsubsection{પ્રશ્ન 2(B)(1) [4
ગુણ]}\label{uxaaauxab0uxab6uxaa8-2b1-4-uxa97uxaa3}

\textbf{યોગ્ય નામકરણ સાથે માઇક્રોમીટર સ્ક્રૂ ગેજની આકૃતિ દોરો.}

\begin{solutionbox}

\begin{verbatim}
                     રેચેટ
                        |
                        v
    +{-{-}{-}{-}{-}{-}{-}{-}{-}{-}{-}{-}{-}{-}{-}{-}{-}{-}+}
    |                  |
    |   થિમ્બલ સ્કેલ       |
    |        ||        |
    +{-{-}{-}{-}{-}{-}{-}{-}||{-}{-}{-}{-}{-}{-}{-}{-}+}
             ||
             ||  સ્પિંડલ
             ||
    +{-{-}{-}{-}{-}{-}{-}{-}||{-}{-}{-}{-}{-}{-}{-}{-}+}
    |    મુખ્ય સ્કેલ       |
    |  0   5   10  15  |
    +{-{-}||{-}{-}{-}{-}{-}{-}{-}{-}{-}{-}||{-}{-}+}
       ||          ||
    એન્વિલ         ફ્રેમ
\end{verbatim}

\textbf{મુખ્ય ઘટકો}:

\begin{itemize}
\tightlist
\item
  \textbf{ફ્રેમ}: U-આકારનું માળખું જે આધાર પૂરો પાડે
\item
  \textbf{એન્વિલ}: વસ્તુ મૂકવા માટે સ્થિર જડબો
\item
  \textbf{સ્પિંડલ}: ગતિશીલ સ્ક્રૂ મેકેનિઝમ
\item
  \textbf{થિમ્બલ સ્કેલ}: 50 વિભાગ સાથે વર્તુળાકાર સ્કેલ
\item
  \textbf{મુખ્ય સ્કેલ}: mm માં રેખીય સ્કેલ
\item
  \textbf{રેચેટ}: સુસંગત દબાણ લાગુ કરવા માટે
\end{itemize}

\end{solutionbox}
\begin{mnemonicbox}
``ફ્રેમ એન્વિલ સ્પિંડલ થિમ્બલ મુખ્ય રેચેટ''

\end{mnemonicbox}
\subsubsection{પ્રશ્ન 2(B)(2) [4
ગુણ]}\label{uxaaauxab0uxab6uxaa8-2b2-4-uxa97uxaa3}

\textbf{વર્નિયર કેલિપર્સ માટે યોગ્ય આકૃતિ સાથે શૂન્ય, ધન અને ઋણ ત્રુટીઓ સમજાવો અને આ
પ્રકારની ત્રુટીઓ દૂર કરવા માટેના જરૂરી પગલાંની યાદી બનાવો.}

\begin{solutionbox}

\textbf{ત્રુટીના પ્રકારો}:

\begin{longtable}[]{@{}
  >{\raggedright\arraybackslash}p{(\linewidth - 4\tabcolsep) * \real{0.5161}}
  >{\raggedright\arraybackslash}p{(\linewidth - 4\tabcolsep) * \real{0.2258}}
  >{\raggedright\arraybackslash}p{(\linewidth - 4\tabcolsep) * \real{0.2581}}@{}}
\toprule\noalign{}
\begin{minipage}[b]{\linewidth}\raggedright
ત્રુટીનો પ્રકાર
\end{minipage} & \begin{minipage}[b]{\linewidth}\raggedright
સ્થિતિ
\end{minipage} & \begin{minipage}[b]{\linewidth}\raggedright
વાંચન
\end{minipage} \\
\midrule\noalign{}
\endhead
\bottomrule\noalign{}
\endlastfoot
શૂન્ય ત્રુટિ & વર્નિયરની શૂન્ય રેખા મુખ્ય સ્કેલની શૂન્ય સાથે મેળ ખાતી નથી & જડબા બંધ હોય
ત્યારે શૂન્ય અલાવાનું વાંચન \\
ધન ત્રુટિ & વર્નિયર શૂન્ય મુખ્ય સ્કેલ શૂન્યની જમણી બાજુએ & સુધારો ઉમેરો \\
ઋણ ત્રુટિ & વર્નિયર શૂન્ય મુખ્ય સ્કેલ શૂન્યની ડાબી બાજુએ & સુધારો બાદ કરો \\
\end{longtable}

\textbf{આકૃતિ}:

\begin{verbatim}
શૂન્ય ત્રુટિ:
મુખ્ય સ્કેલ:  |0|1|2|3|4|5|
વર્નિયર:      |0|1|2|3|4|

ધન ત્રુટિ:
મુખ્ય સ્કેલ:  |0|1|2|3|4|5|
વર્નિયર:       |0|1|2|3|4|

ઋણ ત્રુટિ:
મુખ્ય સ્કેલ:  |0|1|2|3|4|5|
વર્નિયર:     |0|1|2|3|4|
\end{verbatim}

\textbf{ત્રુટીઓ દૂર કરવાના પગલાં}:

\begin{itemize}
\tightlist
\item
  \textbf{શૂન્ય ત્રુટિ તપાસો} માપન પહેલાં
\item
  \textbf{અંતિમ વાંચનમાં સુધારો લાગુ કરો}
\item
  \textbf{જડબાઓ સાફ કરો} કચરો અટકાવવા માટે
\item
  \textbf{સાવચેતીથી હાથ વણો} યાંત્રિક નુકસાન ટાળવા માટે
\end{itemize}

\end{solutionbox}
\begin{mnemonicbox}
``તપાસો સાફ કરો સુધારો સાવચેતી''

\end{mnemonicbox}
\subsubsection{પ્રશ્ન 2(B)(3) [4
ગુણ]}\label{uxaaauxab0uxab6uxaa8-2b3-4-uxa97uxaa3}

\textbf{સાદા લોલકનો આવર્તકાળ શોધવાના પ્રયોગમાં અવલોકનો 1.96 s, 1.98 s, 2.00
s, 2.02 s, 2.04 s છે. નિરપેક્ષ ત્રુટિ, સરેરાશ નિરપેક્ષ ત્રુટિ, સાપેક્ષ ત્રુટિ અને
પ્રતિશત ત્રુટિની ગણતરી કરો.}

\begin{solutionbox}

\begin{verbatim}
અવલોકનો: 1.96, 1.98, 2.00, 2.02, 2.04 s

સરેરાશ મૂલ્ય = (1.96 + 1.98 + 2.00 + 2.02 + 2.04) \div 5 = 2.00 s

નિરપેક્ષ ત્રુટીઓ: |xi - સરેરાશ|
|1.96 - 2.00| = 0.04 s
|1.98 - 2.00| = 0.02 s
|2.00 - 2.00| = 0.00 s
|2.02 - 2.00| = 0.02 s
|2.04 - 2.00| = 0.04 s

સરેરાશ નિરપેક્ષ ત્રુટિ = (0.04 + 0.02 + 0.00 + 0.02 + 0.04) \div 5 = 0.024 s

સાપેક્ષ ત્રુટિ = સરેરાશ નિરપેક્ષ ત્રુટિ \div સરેરાશ મૂલ્ય = 0.024 \div 2.00 = 0.012

પ્રતિશત ત્રુટિ = સાપેક્ષ ત્રુટિ \times 100 = 0.012 \times 100 = 1.2%
\end{verbatim}

\textbf{પરિણામો}: સરેરાશ નિરપેક્ષ ત્રુટિ = 0.024 s, સાપેક્ષ ત્રુટિ = 0.012,
પ્રતિશત ત્રુટિ = 1.2\%

\end{solutionbox}
\begin{mnemonicbox}
``સરેરાશ નિરપેક્ષ સાપેક્ષ પ્રતિશત''

\end{mnemonicbox}
\subsection*{પ્રશ્ન 3(A) - કોઈપણ બેના જવાબ આપો [6
ગુણ]}\label{q3a}

\subsubsection{પ્રશ્ન 3(A)(1) [3
ગુણ]}\label{uxaaauxab0uxab6uxaa8-3a1-3-uxa97uxaa3}

\textbf{વ્યાખ્યાઓ કરો: વિદ્યુત ફ્લક્સ, વિદ્યુતક્ષેત્ર, વીજસ્થિતિમાનનો તફાવત}

\begin{solutionbox}

\begin{longtable}[]{@{}
  >{\raggedright\arraybackslash}p{(\linewidth - 6\tabcolsep) * \real{0.1852}}
  >{\raggedright\arraybackslash}p{(\linewidth - 6\tabcolsep) * \real{0.3333}}
  >{\raggedright\arraybackslash}p{(\linewidth - 6\tabcolsep) * \real{0.2222}}
  >{\raggedright\arraybackslash}p{(\linewidth - 6\tabcolsep) * \real{0.2593}}@{}}
\toprule\noalign{}
\begin{minipage}[b]{\linewidth}\raggedright
શબ્દ
\end{minipage} & \begin{minipage}[b]{\linewidth}\raggedright
વ્યાખ્યા
\end{minipage} & \begin{minipage}[b]{\linewidth}\raggedright
એકમ
\end{minipage} & \begin{minipage}[b]{\linewidth}\raggedright
સૂત્ર
\end{minipage} \\
\midrule\noalign{}
\endhead
\bottomrule\noalign{}
\endlastfoot
વિદ્યુત ફ્લક્સ & સપાટીમાંથી પસાર થતી વિદ્યુત ક્ષેત્ર રેખાઓની સંખ્યા & Nm^{2}/C & \Phi =
E·A \\
વિદ્યુતક્ષેત્ર & એકમ ધન આવેશ પર લાગતું બળ & N/C & E = F/q \\
વીજસ્થિતિમાનનો તફાવત & બે બિંદુઓ વચ્ચે એકમ આવેશ દીઠ કામ & વોલ્ટ & V = W/q \\
\end{longtable}

\begin{itemize}
\tightlist
\item
  \textbf{વિદ્યુત ફ્લક્સ}: સપાટીમાં પ્રવેશતી ક્ષેત્ર રેખાઓનું માપ
\item
  \textbf{વિદ્યુતક્ષેત્ર}: વિદ્યુત બળ ક્રિયા કરતો વિસ્તાર
\item
  \textbf{વીજસ્થિતિમાનનો તફાવત}: એકમ આવેશ દીઠ ઊર્જાનો તફાવત
\end{itemize}

\end{solutionbox}
\begin{mnemonicbox}
``ફ્લક્સ ક્ષેત્ર બળ, કામ વોટ્સ વોલ્ટ્સ''

\end{mnemonicbox}
\subsubsection{પ્રશ્ન 3(A)(2) [3
ગુણ]}\label{uxaaauxab0uxab6uxaa8-3a2-3-uxa97uxaa3}

\textbf{જ્યારે ત્રણ જુદા જુદા કેપેસિટરોને શ્રેણીમાં જોડવામાં આવે ત્યારે જરૂરી સર્કિટ
ડાયાગ્રામ સાથે સમકક્ષ કેપેસિટન્સ માટેનું સૂત્ર મેળવો.}

\begin{solutionbox}

\textbf{સર્કિટ ડાયાગ્રામ}:

\begin{verbatim}
    +{-{-}{-}{-}||{-}{-}{-}{-}||{-}{-}{-}{-}||{-}{-}{-}{-}+}
    |    C1    C2    C3    |
    |                      |
    +{-{-}{-}{-}{-}{-}{-}{-}{-}{-}V{-}{-}{-}{-}{-}{-}{-}{-}{-}{-}{-}+}
\end{verbatim}

\textbf{વ્યુત્પત્તિ}:

\begin{itemize}
\tightlist
\item
  \textbf{સમાન આવેશ} Q દરેક કેપેસિટર દ્વારા વહે છે
\item
  \textbf{વોલ્ટેજ વિભાજન}: V = V_{1} + V_{2} + V_{3}
\item
  \textbf{દરેક કેપેસિટર માટે}: V_{1} = Q/C_{1}, V_{2} = Q/C_{2}, V_{3} = Q/C_{3}
\item
  \textbf{કુલ વોલ્ટેજ}: V = Q/C_{1} + Q/C_{2} + Q/C_{3} = Q(1/C_{1} + 1/C_{2} + 1/C_{3})
\item
  \textbf{સમકક્ષ માટે}: V = Q/Cs
\item
  \textbf{તેથી}: 1/Cs = 1/C_{1} + 1/C_{2} + 1/C_{3}
\end{itemize}

\textbf{સૂત્ર}: \textbf{1/Cs = 1/C_{1} + 1/C_{2} + 1/C_{3}}

\end{solutionbox}
\begin{mnemonicbox}
``શ્રેણી વિપરીત સરવાળો, સમાન આવેશ વિભાજિત વોલ્ટેજ''

\end{mnemonicbox}
\subsubsection{પ્રશ્ન 3(A)(3) [3
ગુણ]}\label{uxaaauxab0uxab6uxaa8-3a3-3-uxa97uxaa3}

\textbf{વ્યાખ્યાઓ કરો: ઇન્ફ્રાસોનિક ધ્વનિ, શ્રાવ્ય ધ્વનિ, અલ્ટ્રાસોનિક ધ્વનિ}

\begin{solutionbox}

\begin{longtable}[]{@{}llll@{}}
\toprule\noalign{}
ધ્વનિનો પ્રકાર & આવૃત્તિ શ્રેણી & લાક્ષણિકતાઓ & ઉપયોગો \\
\midrule\noalign{}
\endhead
\bottomrule\noalign{}
\endlastfoot
ઇન્ફ્રાસોનિક & 20 Hz થી નીચે & મનુષ્યને સંભળાતું નથી & ભૂકંપ શોધ \\
શ્રાવ્ય & 20 Hz થી 20 kHz & મનુષ્યને સંભળાય છે & વાતચીત, સંગીત \\
અલ્ટ્રાસોનિક & 20 kHz થી ઉપર & મનુષ્યને સંભળાતું નથી & તબીબી ઇમેજિંગ, SONAR \\
\end{longtable}

\begin{itemize}
\tightlist
\item
  \textbf{ઇન્ફ્રાસોનિક}: માનવ શ્રવણથી નીચેની ઓછી આવૃત્તિ
\item
  \textbf{શ્રાવ્ય}: માનવો માટે સામાન્ય શ્રવણ શ્રેણી
\item
  \textbf{અલ્ટ્રાસોનિક}: માનવ શ્રવણથી ઉપરની ઊંચી આવૃત્તિ
\end{itemize}

\end{solutionbox}
\begin{mnemonicbox}
``ઇન્ફ્રા-નીચે, શ્રાવ્ય-વચ્ચે, અલ્ટ્રા-ઉપર''

\end{mnemonicbox}
\subsection*{પ્રશ્ન 3(B) - કોઈપણ બેના જવાબ આપો [8
ગુણ]}\label{q3b}

\subsubsection{પ્રશ્ન 3(B)(1) [4
ગુણ]}\label{uxaaauxab0uxab6uxaa8-3b1-4-uxa97uxaa3}

\textbf{સમાંતર પ્લેટ કેપેસિટર માટે C = \epsilon_{0}A/d સાબિત કરો.}

\begin{solutionbox}

\textbf{આકૃતિ}:

\begin{verbatim}
    +{-{-}{-}{-}{-}{-}{-}{-}+  +{-}{-}{-}{-}{-}{-}{-}{-}+}
    |   +Q   |  |   {-Q   |}
    |        |  |        |
    | પ્લેટ1   |  | પ્લેટ2    |
    |   A    |  |   A    |
    +{-{-}{-}{-}{-}{-}{-}{-}+  +{-}{-}{-}{-}{-}{-}{-}{-}+}
         {{-}{-}{-}d{-}{-}{-}}
\end{verbatim}

\textbf{વ્યુત્પત્તિ}:

\begin{itemize}
\tightlist
\item
  \textbf{પ્લેટો વચ્ચે વિદ્યુત ક્ષેત્ર}: E = \sigma/\epsilon_{0} = Q/(\epsilon_{0}A)
\item
  \textbf{વીજસ્થિતિમાનનો તફાવત}: V = E \times d = Qd/(\epsilon_{0}A)
\item
  \textbf{કેપેસિટન્સની વ્યાખ્યા}: C = Q/V
\item
  \textbf{બદલીને}: C = Q \div [Qd/(\epsilon_{0}A)] = \epsilon_{0}A/d
\end{itemize}

\textbf{અંતિમ સૂત્ર}: \textbf{C = \epsilon_{0}A/d}

જ્યાં:

\begin{itemize}
\tightlist
\item
  \textbf{\epsilon_{0}}: મુક્ત અવકાશની વિદ્યુત પ્રવેશ્યતા
\item
  \textbf{A}: પ્લેટોનું ક્ષેત્રફળ
\item
  \textbf{d}: પ્લેટો વચ્ચેનું અંતર
\end{itemize}

\end{solutionbox}
\begin{mnemonicbox}
``કેપેસિટન્સ બરાબર એપ્સિલોન-શૂન્ય ક્ષેત્રફળ ભાગુ અંતર''

\end{mnemonicbox}
\subsubsection{પ્રશ્ન 3(B)(2) [4
ગુણ]}\label{uxaaauxab0uxab6uxaa8-3b2-4-uxa97uxaa3}

\textbf{વિદ્યુતક્ષેત્ર રેખાઓની લાક્ષણિકતાઓ સૂચિબદ્ધ કરો.}

\begin{solutionbox}

\textbf{મુખ્ય લાક્ષણિકતાઓ}:

\begin{itemize}
\tightlist
\item
  \textbf{દિશા}: ધન આવેશથી ઋણ આવેશ તરફ
\item
  \textbf{ઘનતા}: ક્ષેત્રની મજબૂતાઈ દર્શાવે છે
\item
  \textbf{નિરંતર}: મુક્ત અવકાશમાં ક્યારેય તૂટતી નથી
\item
  \textbf{બિન-છેદન}: બે રેખાઓ ક્યારેય પાર કરતી નથી
\item
  \textbf{લંબ}: વાહક સપાટી પર લંબ હોય છે
\item
  \textbf{બંધ લૂપ}: ફક્ત બદલાતા ચુંબકીય ક્ષેત્રની આસપાસ
\item
  \textbf{સ્પર્શક}: કોઈપણ બિંદુએ ક્ષેત્રની દિશા આપે છે
\item
  \textbf{સમાન અંતર}: સમાન ક્ષેત્રના વિસ્તારોમાં
\end{itemize}

\textbf{ગુણધર્મો}:

\begin{itemize}
\tightlist
\item
  \textbf{ધન આવેશ}થી શરુ થાય છે
\item
  \textbf{ઋણ આવેશ}પર સમાપ્ત થાય છે
\item
  \textbf{વધુ ઘનતા} મજબૂત ક્ષેત્ર દર્શાવે છે
\item
  \textbf{ક્યારેય છેદન નથી} કરતી
\end{itemize}

\end{solutionbox}
\begin{mnemonicbox}
``ધન થી ઋણ, ઘન મજબૂત, ક્યારેય છેદે નહીં, હંમેશા લંબ''

\end{mnemonicbox}
\subsubsection{પ્રશ્ન 3(B)(3) [4
ગુણ]}\label{uxaaauxab0uxab6uxaa8-3b3-4-uxa97uxaa3}

\textbf{અલ્ટ્રાસોનિક તરંગોના ઉત્પાદન માટે ઉપયોગમાં લેવામાં આવતી મેગ્નેટોસ્ટ્રિક્શન
પદ્ધતિની રચના અને કાર્યપદ્ધતિનું વર્ણન કરો.}

\begin{solutionbox}

\textbf{રચના}:

\begin{verbatim}
    Oscillator {- Coil {-} Nickel Rod {-} Horn}
                    |       |          |
                   AC    Vibrates   Amplifies
\end{verbatim}

\textbf{ઘટકો}:

\begin{itemize}
\tightlist
\item
  \textbf{નિકલ રોડ}: મેગ્નેટોસ્ટ્રિક્ટિવ પદાર્થ
\item
  \textbf{કોઇલ}: રોડની આસપાસ ઇલેક્ટ્રોમેગ્નેટ
\item
  \textbf{AC ઓસિલેટર}: ઊંચી આવૃત્તિનો પ્રવાહ સ્ત્રોત
\item
  \textbf{હોર્ન}: ધ્વનિ વર્ધક અને ટ્રાન્સમિટર
\end{itemize}

\textbf{કાર્યપદ્ધતિ}:

\begin{itemize}
\tightlist
\item
  \textbf{AC પ્રવાહ} કોઇલમાંથી વહે છે
\item
  \textbf{ચુંબકીય ક્ષેત્ર} ઝડપથી બદલાય છે
\item
  \textbf{નિકલ રોડ} વિસ્તૃત અને સંકુચિત થાય છે
\item
  \textbf{યાંત્રિક કંપનો} ઉત્પન્ન થાય છે
\item
  \textbf{અલ્ટ્રાસોનિક તરંગો} ઉત્પન્ન થાય છે
\end{itemize}

\textbf{ઉપયોગો}: તબીબી ઇમેજિંગ, સફાઈ, વેલ્ડિંગ

\end{solutionbox}
\begin{mnemonicbox}
``AC કોઇલ નિકલને કંપાવે છે, અલ્ટ્રાસોનિક બનાવે છે''

\end{mnemonicbox}
\subsection*{પ્રશ્ન 4(A) - કોઈપણ બેના જવાબ આપો [6
ગુણ]}\label{q4a}

\subsubsection{પ્રશ્ન 4(A)(1) [3
ગુણ]}\label{uxaaauxab0uxab6uxaa8-4a1-3-uxa97uxaa3}

\textbf{એક રેડિયો સ્ટેશન 9.26 \times 10^{7} Hz આવૃત્તિવાળા તરંગોનું ઉત્સર્જન કરે છે. જો આ
તરંગોની ઝડપ 3.00 \times 10^{8} m/s હોય તો તેની તરંગલંબાઈ શોધો.}

\begin{solutionbox}

\begin{verbatim}
આપેલ:
આવૃત્તિ (f) = 9.26 \times 10^{7} Hz
ઝડપ (c) = 3.00 \times 10^{8} m/s

સૂત્ર: c = f\lambda
તેથી: \lambda = c/f

\lambda = (3.00 \times 10^{8}) \div (9.26 \times 10^{7})
\lambda = 3.24 m
\end{verbatim}

\textbf{તરંગલંબાઈ = 3.24 m}

\end{solutionbox}
\begin{mnemonicbox}
``ઝડપ બરાબર આવૃત્તિ ગુણ્યે તરંગલંબાઈ''

\end{mnemonicbox}
\subsubsection{પ્રશ્ન 4(A)(2) [3
ગુણ]}\label{uxaaauxab0uxab6uxaa8-4a2-3-uxa97uxaa3}

\textbf{સ્નેલનો નિયમ જણાવો અને માધ્યમનો વક્રીભવનાંક સમજાવો.}

\begin{solutionbox}

\textbf{સ્નેલનો નિયમ}: n_{1} sin \theta_{1} = n_{2} sin \theta_{2}

જ્યાં:

\begin{itemize}
\tightlist
\item
  \textbf{n_{1}, n_{2}}: માધ્યમ 1 અને 2 ના વક્રીભવનાંક
\item
  \textbf{\theta_{1}, \theta_{2}}: આપાત અને વક્રીભવન કોણ
\end{itemize}

\textbf{વક્રીભવનાંક}:

\begin{longtable}[]{@{}lll@{}}
\toprule\noalign{}
પ્રકાર & વ્યાખ્યા & સૂત્ર \\
\midrule\noalign{}
\endhead
\bottomrule\noalign{}
\endlastfoot
નિરપેક્ષ & શૂન્યાવકાશમાં પ્રકાશની ઝડપ અને માધ્યમમાં ઝડપનો ગુણોત્તર & n = c/v \\
સાપેક્ષ & બે માધ્યમોમાં ઝડપનો ગુણોત્તર & n_{2}_{1} = v_{1}/v_{2} \\
\end{longtable}

\begin{itemize}
\tightlist
\item
  \textbf{ઊંચો વક્રીભવનાંક}: ઘન માધ્યમ, ધીમો પ્રકાશ
\item
  \textbf{નીચો વક્રીભવનાંક}: વિરળ માધ્યમ, ઝડપી પ્રકાશ
\end{itemize}

\end{solutionbox}
\begin{mnemonicbox}
``સ્નેલ સાઇન ગુણોત્તર સ્થિર, ઘન પ્રકાશ ધીમો કરે''

\end{mnemonicbox}
\subsubsection{પ્રશ્ન 4(A)(3) [3
ગુણ]}\label{uxaaauxab0uxab6uxaa8-4a3-3-uxa97uxaa3}

\textbf{સરખામણી કરો: સામાન્ય પ્રકાશ અને LASER}

\begin{solutionbox}

\begin{longtable}[]{@{}lll@{}}
\toprule\noalign{}
ગુણધર્મ & સામાન્ય પ્રકાશ & LASER \\
\midrule\noalign{}
\endhead
\bottomrule\noalign{}
\endlastfoot
સુસંગતતા & અસુસંગત & સુસંગત \\
રંગ & બહુરંગી & એકરંગી \\
દિશા & વિકીર્ણ & સમાંતર કિરણ \\
તીવ્રતા & ઓછી & ખૂબ વધારે \\
કલા & અવ્યવસ્થિત & સ્થિર કલા સંબંધ \\
તરંગલંબાઈ & બહુવિધ તરંગલંબાઈ & એકલ તરંગલંબાઈ \\
\end{longtable}

\textbf{મુખ્ય તફાવતો}:

\begin{itemize}
\tightlist
\item
  \textbf{LASER}: સુસંગત, એકરંગી, સમાંતર, તીવ્ર
\item
  \textbf{સામાન્ય}: અસુસંગત, બહુરંગી, વિકીર્ણ, ઓછી તીવ્ર
\end{itemize}

\end{solutionbox}
\begin{mnemonicbox}
``LASER: સુસંગત એકરંગી સમાંતર તીવ્ર''

\end{mnemonicbox}
\subsection*{પ્રશ્ન 4(B) - કોઈપણ બેના જવાબ આપો [8
ગુણ]}\label{q4b}

\subsubsection{પ્રશ્ન 4(B)(1) [4
ગુણ]}\label{uxaaauxab0uxab6uxaa8-4b1-4-uxa97uxaa3}

\textbf{જરૂરી આકૃતિ સાથે ઓપ્ટિકલ ફાઇબરની રચના દર્શાવો.}

\begin{solutionbox}

\textbf{ઓપ્ટિકલ ફાઇબર રચના}:

\begin{verbatim}
    |{{-}{-}{-}{-}{-}{-} કોર {-}{-}{-}{-}{-}{-}|}
    |                    |
    +{-{-}{-}{-}{-}{-}{-}{-}{-}{-}{-}{-}{-}{-}{-}{-}{-}{-}{-}{-}+  {-} ક્લેડિંગ}
    |   વધારે n_{1           |}
    |                    |  {{-} ઓછો n_{2}}
    +{-{-}{-}{-}{-}{-}{-}{-}{-}{-}{-}{-}{-}{-}{-}{-}{-}{-}{-}{-}+}
    |                    |
    |   સુરક્ષાત્મક           |  {{-} જેકેટ}
    |   કોટિંગ             |
    +{-{-}{-}{-}{-}{-}{-}{-}{-}{-}{-}{-}{-}{-}{-}{-}{-}{-}{-}{-}+}
\end{verbatim}

\textbf{ઘટકો}:

\begin{longtable}[]{@{}llll@{}}
\toprule\noalign{}
ઘટક & સામગ્રી & કાર્ય & વક્રીભવનાંક \\
\midrule\noalign{}
\endhead
\bottomrule\noalign{}
\endlastfoot
કોર & કાચ/પ્લાસ્ટિક & પ્રકાશ સંચાર & વધારે (n_{1}) \\
ક્લેડિંગ & કાચ & પૂર્ણ આંતરિક પરાવર્તન & ઓછો (n_{2}) \\
જેકેટ & પ્લાસ્ટિક & સુરક્ષા & - \\
\end{longtable}

\textbf{કાર્યપદ્ધતિ}:

\begin{itemize}
\tightlist
\item
  પ્રકાશ \textbf{કોર}માં સ્વીકૃતિ કોણ પર પ્રવેશે છે
\item
  કોર-ક્લેડિંગ સીમા પર \textbf{પૂર્ણ આંતરિક પરાવર્તન}
\item
  પ્રકાશ કોરમાં \textbf{ઝિગઝેગ માર્ગ}માં મુસાફરી કરે છે
\item
  \textbf{n_{1} \textgreater{} n_{2}} પ્રકાશ કેદ સુનિશ્ચિત કરે છે
\end{itemize}

\end{solutionbox}
\begin{mnemonicbox}
``કોર ક્લેડિંગ જેકેટ, વધારે ઓછો સુરક્ષા''

\end{mnemonicbox}
\subsubsection{પ્રશ્ન 4(B)(2) [4
ગુણ]}\label{uxaaauxab0uxab6uxaa8-4b2-4-uxa97uxaa3}

\textbf{ઇજનેરી અને મેડિકલ ક્ષેત્રે LASER ના ઉપયોગોની યાદી આપો.}

\begin{solutionbox}

\textbf{ઇજનેરિંગ ઉપયોગો}:

\begin{itemize}
\tightlist
\item
  \textbf{કટિંગ અને વેલ્ડિંગ}: ચોક્કસ ધાતુ કાપવા
\item
  \textbf{3D પ્રિંટિંગ}: લેઝર સિન્ટરિંગ
\item
  \textbf{માપન}: અંતર અને સર્વેક્ષણ
\item
  \textbf{સંચાર}: ઓપ્ટિકલ ફાઇબર સિસ્ટમ
\item
  \textbf{સામગ્રી પ્રક્રિયા}: સપાટી કઠિનીકરણ
\item
  \textbf{બારકોડ સ્કેનિંગ}: રિટેઇલ અને ઇન્વેન્ટરી
\end{itemize}

\textbf{તબીબી ઉપયોગો}:

\begin{itemize}
\tightlist
\item
  \textbf{શસ્ત્રક્રિયા}: ચોક્કસ પેશી કાપવા
\item
  \textbf{આંખની સારવાર}: સુધારાત્મક શસ્ત્રક્રિયા
\item
  \textbf{કેન્સર સારવાર}: ગાંઠનો નાશ
\item
  \textbf{નિદાન}: સ્પેક્ટ્રોસ્કોપી
\item
  \textbf{દંત ચિકિત્સા}: કેવિટી સારવાર
\item
  \textbf{ચામડીની સારવાર}: કોસ્મેટિક પ્રક્રિયાઓ
\end{itemize}

\textbf{ફાયદા}: \textbf{ચોકસાઈ, બિન-સંપર્ક, જંતુરહિત, ન્યૂનતમ નુકસાન}

\end{solutionbox}
\begin{mnemonicbox}
``ઇજનેરિંગ: કાપ વેલ્ડ માપ સંચાર, મેડિકલ: શસ્ત્રક્રિયા આંખ કેન્સર
નિદાન''

\end{mnemonicbox}
\subsubsection{પ્રશ્ન 4(B)(3) [4
ગુણ]}\label{uxaaauxab0uxab6uxaa8-4b3-4-uxa97uxaa3}

\textbf{P-type અને N-type અર્ધવાહકો સમજાવો.}

\begin{solutionbox}

\textbf{N-type અર્ધવાહક}:

\begin{longtable}[]{@{}ll@{}}
\toprule\noalign{}
ગુણધર્મ & N-type \\
\midrule\noalign{}
\endhead
\bottomrule\noalign{}
\endlastfoot
ડોપન્ટ & ફોસ્ફોરસ, આર્સેનિક (5 વેલેન્સ ઇલેક્ટ્રોન) \\
મુખ્ય વાહકો & ઇલેક્ટ્રોન \\
ગૌણ વાહકો & હોલ્સ \\
આવેશ & નકારાત્મક \\
\end{longtable}

\textbf{P-type અર્ધવાહક}:

\begin{longtable}[]{@{}ll@{}}
\toprule\noalign{}
ગુણધર્મ & P-type \\
\midrule\noalign{}
\endhead
\bottomrule\noalign{}
\endlastfoot
ડોપન્ટ & બોરોન, એલ્યુમિનિયમ (3 વેલેન્સ ઇલેક્ટ્રોન) \\
મુખ્ય વાહકો & હોલ્સ \\
ગૌણ વાહકો & ઇલેક્ટ્રોન \\
આવેશ & સકારાત્મક \\
\end{longtable}

\textbf{રચના પ્રક્રિયા}:

\begin{itemize}
\tightlist
\item
  \textbf{N-type}: પંચસંયોજક અણુઓ ઇલેક્ટ્રોન દાન કરે છે
\item
  \textbf{P-type}: ત્રિસંયોજક અણુઓ ઇલેક્ટ્રોન સ્વીકારે છે, હોલ્સ બનાવે છે
\item
  \textbf{ડોપિંગ}: અશુદ્ધતાઓનો નિયંત્રિત ઉમેરો
\item
  \textbf{વાહકતા}: મુક્ત વાહકોને કારણે વધે છે
\end{itemize}

\end{solutionbox}
\begin{mnemonicbox}
``N-type નકારાત્મક ઇલેક્ટ્રોન, P-type સકારાત્મક હોલ્સ''

\end{mnemonicbox}
\subsection*{પ્રશ્ન 5(A) - કોઈપણ બેના જવાબ આપો [6
ગુણ]}\label{q5a}

\subsubsection{પ્રશ્ન 5(A)(1) [3
ગુણ]}\label{uxaaauxab0uxab6uxaa8-5a1-3-uxa97uxaa3}

\textbf{ઊર્જા બેન્ડ ગેપના આધારે વાહકો, અર્ધવાહકો અને અવાહકોનું વર્ગીકરણ કરો.}

\begin{solutionbox}

\begin{longtable}[]{@{}llll@{}}
\toprule\noalign{}
સામગ્રી & ઊર્જા બેન્ડ ગેપ & લાક્ષણિકતાઓ & ઉદાહરણો \\
\midrule\noalign{}
\endhead
\bottomrule\noalign{}
\endlastfoot
વાહક & કોઈ ગેપ નથી (0 eV) & વેલેન્સ અને વહન બેન્ડ ઓવરલેપ & તાંબુ, ચાંદી \\
અર્ધવાહક & નાનો ગેપ (1-3 eV) & મધ્યમ બેન્ડ ગેપ & સિલિકોન, જર્મેનિયમ \\
અવાહક & મોટો ગેપ (\textgreater3 eV) & પહોળો બેન્ડ ગેપ & કાચ, રબર \\
\end{longtable}

\textbf{ઊર્જા બેન્ડ આકૃતિ}:

\begin{verbatim}
વાહક        અર્ધવાહક      અવાહક
   
   CB            CB              CB
   {-{-}            {-}{-}              {-}{-}}
   VB            VB              VB
   
કોઈ ગેપ        નાનો ગેપ       મોટો ગેપ
\end{verbatim}

\begin{itemize}
\tightlist
\item
  \textbf{CB}: વહન બેન્ડ
\item
  \textbf{VB}: વેલેન્સ બેન્ડ
\item
  \textbf{ગેપ વિદ્યુત વાહકતા} નક્કી કરે છે
\end{itemize}

\end{solutionbox}
\begin{mnemonicbox}
``કોઈ ગેપ વાહે, નાનો ગેપ અર્ધ, મોટો ગેપ અવાહ''

\end{mnemonicbox}
\subsubsection{પ્રશ્ન 5(A)(2) [3
ગુણ]}\label{uxaaauxab0uxab6uxaa8-5a2-3-uxa97uxaa3}

\textbf{જરૂરી ટ્રુથ ટેબલ સાથે OR અને AND લોજિક ગેટ સમજાવો.}

\begin{solutionbox}

\textbf{OR ગેટ}:

\begin{longtable}[]{@{}lll@{}}
\toprule\noalign{}
A & B & Y = A + B \\
\midrule\noalign{}
\endhead
\bottomrule\noalign{}
\endlastfoot
0 & 0 & 0 \\
0 & 1 & 1 \\
1 & 0 & 1 \\
1 & 1 & 1 \\
\end{longtable}

\textbf{AND ગેટ}:

\begin{longtable}[]{@{}lll@{}}
\toprule\noalign{}
A & B & Y = A · B \\
\midrule\noalign{}
\endhead
\bottomrule\noalign{}
\endlastfoot
0 & 0 & 0 \\
0 & 1 & 0 \\
1 & 0 & 0 \\
1 & 1 & 1 \\
\end{longtable}

\textbf{પ્રતીકો}:

\begin{verbatim}
OR ગેટ:       A {-{-}{-}{-}}
                   {{-}{-}{-}{-} Y}
             B {-{-}{-}{-}/}

AND ગેટ:      A {-{-}{-}{-}}
                   \&{-{-}{-}{-} Y}
             B {-{-}{-}{-}/}
\end{verbatim}

\begin{itemize}
\tightlist
\item
  \textbf{OR}: કોઈપણ ઇનપુટ HIGH હોય ત્યારે આઉટપુટ HIGH
\item
  \textbf{AND}: બધા ઇનપુટ HIGH હોય ત્યારે આઉટપુટ HIGH
\end{itemize}

\end{solutionbox}
\begin{mnemonicbox}
``OR: કોઈ પણ હાઈ બનાવે હાઈ, AND: બધા હાઈ બનાવે હાઈ''

\end{mnemonicbox}
\subsubsection{પ્રશ્ન 5(A)(3) [3
ગુણ]}\label{uxaaauxab0uxab6uxaa8-5a3-3-uxa97uxaa3}

\textbf{વોલ્ટેજ રેગ્યુલેટર તરીકે ઝેનર ડાયોડના ઉપયોગનું વર્ણન કરો.}

\begin{solutionbox}

\textbf{સર્કિટ આકૃતિ}:

\begin{verbatim}
    Vin {-{-}{-}{-}[Rs]{-}{-}{-}{-}+{-}{-}{-}{-}Vout}
                     |
                   [Zener]
                     |
                    GND
\end{verbatim}

\textbf{કાર્યપદ્ધતિ}:

\begin{itemize}
\tightlist
\item
  \textbf{ફોરવર્ડ બાયાસ}: સામાન્ય ડાયોડની જેમ કાર્ય કરે છે
\item
  \textbf{રિવર્સ બાયાસ}: ઝેનર વોલ્ટેજ પર બ્રેકડાઉન
\item
  \textbf{વોલ્ટેજ રેગ્યુલેશન}: સ્થિર Vout = Vz જાળવે છે
\item
  \textbf{શ્રેણી રેઝિસ્ટર}: ઝેનર દ્વારા કરંટ મર્યાદિત કરે છે
\end{itemize}

\textbf{લાક્ષણિકતાઓ}:

\begin{itemize}
\tightlist
\item
  \textbf{ઝેનર વોલ્ટેજ}: સ્થિર બ્રેકડાઉન વોલ્ટેજ
\item
  \textbf{કરંટ શ્રેણી}: વિશાળ ઓપરેટિંગ રેન્જ
\item
  \textbf{તાપમાન સ્થિરતા}: સારી વોલ્ટેજ સ્થિરતા
\item
  \textbf{પાવર રેટિંગ}: મહત્તમ પાવર વટાવવું નહીં
\end{itemize}

\textbf{ઉપયોગો}: પાવર સપ્લાય, વોલ્ટેજ રેફરન્સ, સંરક્ષણ સર્કિટ

\end{solutionbox}
\begin{mnemonicbox}
``ઝેનર ઉત્સાહથી વોલ્ટેજ વિવિધતા છતાં જાળવે છે''

\end{mnemonicbox}
\subsection*{પ્રશ્ન 5(B) - કોઈપણ બેના જવાબ આપો [8
ગુણ]}\label{q5b}

\subsubsection{પ્રશ્ન 5(B)(1) [4
ગુણ]}\label{uxaaauxab0uxab6uxaa8-5b1-4-uxa97uxaa3}

\textbf{જરૂરી સર્કિટ સાથે પૂર્ણ તરંગ રેક્ટિફાયર સમજાવો તથા ઇનપુટ અને આઉટપુટ તરંગો
દોરો.}

\begin{solutionbox}

\textbf{સેન્ટર-ટેપ પૂર્ણ તરંગ રેક્ટિફાયર}:

\begin{verbatim}
    AC Input {-{-}{-}{-}+{-}{-}{-}{-}[D1]{-}{-}{-}{-}+{-}{-}{-}{-} સકારાત્મક આઉટપુટ}
                 |             |
            ટ્રાન્સફોર્મર    લોડ (RL)
                 |             |
                 +{-{-}{-}{-}[D2]{-}{-}{-}{-}+{-}{-}{-}{-} કોમન}
\end{verbatim}

\textbf{કાર્યપદ્ધતિ}:

\begin{itemize}
\tightlist
\item
  \textbf{સકારાત્મક અર્ધ ચક્ર}: D1 વાહે છે, D2 બંધ
\item
  \textbf{નકારાત્મક અર્ધ ચક્ર}: D2 વાહે છે, D1 બંધ
\item
  \textbf{બંને અર્ધ}: લોડમાંથી સમાન દિશામાં કરંટ વહે છે
\end{itemize}

\textbf{તરંગરૂપો}:

\begin{verbatim}
ઇનપુટ:      /{  /  /  /}
          /  {/  /  /  }
         /                {}

આઉટપુટ:   /{    /    /}
         /  {  /    /  }
        /    {/    /    }
\end{verbatim}

\textbf{ફાયદા}: બહેતર કાર્યક્ષમતા, ઓછો રિપલ, બહેતર ટ્રાન્સફોર્મર ઉપયોગ

\end{solutionbox}
\begin{mnemonicbox}
``પૂર્ણ તરંગ પૂર્ણ ચક્ર વાપરે, બહેતર કાર્યક્ષમતા બહેતર આઉટપુટ''

\end{mnemonicbox}
\subsubsection{પ્રશ્ન 5(B)(2) [4
ગુણ]}\label{uxaaauxab0uxab6uxaa8-5b2-4-uxa97uxaa3}

\textbf{P-N જંકશન ડાયોડની ફોરવર્ડ અને રિવર્સ લાક્ષણિકતાઓ દર્શાવો.}

\begin{solutionbox}

\textbf{ફોરવર્ડ બાયાસ લાક્ષણિકતાઓ}:

\begin{longtable}[]{@{}lll@{}}
\toprule\noalign{}
વોલ્ટેજ શ્રેણી & કરંટ & વર્તન \\
\midrule\noalign{}
\endhead
\bottomrule\noalign{}
\endlastfoot
0 થી 0.3V (Si) & ખૂબ નાનો & કટ-ઇન વોલ્ટેજ \\
0.7V થી ઉપર & ઘાતાંકીય વધારો & વાહક \\
\end{longtable}

\textbf{રિવર્સ બાયાસ લાક્ષણિકતાઓ}:

\begin{longtable}[]{@{}lll@{}}
\toprule\noalign{}
વોલ્ટેજ શ્રેણી & કરંટ & વર્તન \\
\midrule\noalign{}
\endhead
\bottomrule\noalign{}
\endlastfoot
0 થી બ્રેકડાઉન & રિવર્સ સેચ્યુરેશન & લીકેજ કરંટ \\
બ્રેકડાઉન વોલ્ટેજ & તીવ્ર વધારો & એવેલાન્ચ બ્રેકડાઉન \\
\end{longtable}

\textbf{I-V લાક્ષણિક વક્ર}:

\begin{verbatim}
       I(mA)
        |
        |    ફોરવર્ડ
       /|     બાયાસ
      / |
     /  |
{-{-}{-}{-}+{-}{-}{-}+{-}{-}{-}{-}V(V)}
  {-20   0.7  }
    |
    |રિવર્સ
    |બાયાસ
\end{verbatim}

\textbf{મુખ્ય બિંદુઓ}:

\begin{itemize}
\tightlist
\item
  \textbf{ફોરવર્ડ}: ઓછો પ્રતિકાર, વધારે કરંટ
\item
  \textbf{રિવર્સ}: વધારે પ્રતિકાર, ઓછો કરંટ
\item
  \textbf{કટ-ઇન વોલ્ટેજ}: સિલિકોન માટે 0.7V, જર્મેનિયમ માટે 0.3V
\end{itemize}

\end{solutionbox}
\begin{mnemonicbox}
``ફોરવર્ડ વહેવું, રિવર્સ પ્રતિકાર''

\end{mnemonicbox}
\subsubsection{પ્રશ્ન 5(B)(3) [4
ગુણ]}\label{uxaaauxab0uxab6uxaa8-5b3-4-uxa97uxaa3}

\textbf{LED નો સિદ્ધાંત લખો અને તેની રચના અને કાર્યપદ્ધતિ સમજાવો.}

\begin{solutionbox}

\textbf{સિદ્ધાંત}: \textbf{ઇલેક્ટ્રોલ્યુમિનેસન્સ} - વિદ્યુત ઊર્જાનું પ્રકાશ ઊર્જામાં સીધું
રૂપાંતર

\textbf{રચના}:

\begin{verbatim}
    પ્રકાશ આઉટપુટ
        ↑
    +{-{-}{-}{-}{-}{-}{-}+}
    | P{-type|  {-} એનોડ}
    +{-{-}{-}{-}{-}{-}{-}+}
    |જંકશન   |
    +{-{-}{-}{-}{-}{-}{-}+}
    | N{-type|  {-} કેથોડ}
    +{-{-}{-}{-}{-}{-}{-}+}
\end{verbatim}

\textbf{ઉપયોગમાં લેવાતી સામગ્રી}:

\begin{longtable}[]{@{}lll@{}}
\toprule\noalign{}
રંગ & સામગ્રી & તરંગલંબાઈ \\
\midrule\noalign{}
\endhead
\bottomrule\noalign{}
\endlastfoot
લાલ & GaAs & 700 nm \\
લીલો & GaP & 550 nm \\
વાદળી & GaN & 470 nm \\
\end{longtable}

\textbf{કાર્યપદ્ધતિ}:

\begin{itemize}
\tightlist
\item
  \textbf{ફોરવર્ડ બાયાસ}: ઇલેક્ટ્રોન અને હોલ્સ જંકશન પર પુનઃસંયોજન
\item
  \textbf{ઊર્જા મુક્તિ}: પુનઃસંયોજન દરમિયાન ફોટોન ઉત્સર્જન
\item
  \textbf{પ્રકાશનો રંગ}: બેન્ડ ગેપ ઊર્જા પર આધાર
\item
  \textbf{કાર્યક્ષમતા}: ઊંચું વિદ્યુત થી ઓપ્ટિકલ રૂપાંતર
\end{itemize}

\textbf{ઉપયોગો}: ડિસ્પ્લે, ઇન્ડિકેટર, લાઇટિંગ, ઓપ્ટિકલ કમ્યુનિકેશન

\end{solutionbox}
\begin{mnemonicbox}
``LED: પ્રકાશ ઉત્સર્જક ડાયોડ, ઇલેક્ટ્રોન અને હોલ્સ નૃત્ય કરી
પ્રકાશ બનાવે છે''

\end{mnemonicbox}

\end{document}
