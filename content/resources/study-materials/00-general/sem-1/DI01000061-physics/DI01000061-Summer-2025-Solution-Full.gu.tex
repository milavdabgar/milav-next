%% METADATA
%% subject-code: DI01000061
%% subject-name: Modern Physics
%% semester: 1
%% examination: Summer-2025
%% date: 11-06-2025
%% description: Solution guide for Modern Physics in Gujarati
%% tags: study-material, solutions, gtu, DI01000061, gujarati
%% END METADATA

\documentclass{article}
% GTU Solutions - Gujarati Preamble
% Includes common preamble + Gujarati font setup

% Basic setup
\usepackage[margin=1in]{geometry}
\author{Milav Dabgar}

% Math and tables
\usepackage{amsmath,amssymb,amsthm}
\usepackage{booktabs}
\usepackage{tabularx}
\usepackage{graphicx}
\usepackage{float}  % Required for [H] float placement

% Code listings with syntax highlighting
\usepackage{xcolor}
\usepackage{listings}
\lstset{
  basicstyle=\small\ttfamily,
  breaklines=true,
  numbers=left,
  numberstyle=\tiny\color{gray},
  xleftmargin=2em,
  frame=single,
  showstringspaces=false,
  tabsize=2,
  keywordstyle=\color{blue},
  commentstyle=\color{green!60!black},
  stringstyle=\color{purple}
}

% Optional: TikZ for diagrams (remove if not needed)
\usepackage{tikz}
\usepackage{circuitikz}
\usetikzlibrary{shapes,arrows,positioning,calc}

% Header/footer with author and website
\usepackage{fancyhdr}
\usepackage{lastpage}

\pagestyle{fancy}
\fancyhf{}
\fancyhead[L]{\small\itshape\leftmark}
\fancyhead[R]{\small Milav Dabgar}
\fancyfoot[L]{\small\href{https://www.milav.in}{www.milav.in}}
\fancyfoot[R]{\small Page \thepage\ of \pageref{LastPage}}
\renewcommand{\headrulewidth}{0.4pt}
\renewcommand{\footrulewidth}{0.4pt}

% Hyperref (load before fontspec for Gujarati)
\usepackage[
  colorlinks=true,
  linkcolor=blue,
  urlcolor=blue,
  citecolor=blue,
  pdfauthor={Milav Dabgar},
  pdfsubject={GTU Exam Solutions},
  pdfkeywords={GTU, Java, Programming, Solutions, Gujarati},
  bookmarks=true
]{hyperref}

% Gujarati font setup
\usepackage{fontspec}
\usepackage{polyglossia}
\setdefaultlanguage{gujarati}
\setotherlanguage{english}
\newfontfamily\gujaratifont[Script=Gujarati,AutoFakeBold=2.5,AutoFakeSlant=0.3]{Noto Sans Gujarati}
\setmainfont[Script=Gujarati,AutoFakeBold=2.5,AutoFakeSlant=0.3]{Noto Sans Gujarati}
\setmonofont[Scale=0.9]{Noto Sans Gujarati}
\newfontfamily\englishfont[Script=Gujarati,AutoFakeBold=2.5,AutoFakeSlant=0.3]{Noto Sans Gujarati}
\gappto\captionsgujarati{
  \renewcommand{\tablename}{કોષ્ટક}
  \renewcommand{\figurename}{આકૃતિ}
}
\newcommand{\gu}[1]{{\gujaratifont #1}}


\title{Modern Physics (DI01000061) - Summer 2025 Solution}
\date{11 જૂન, 2025}

\hypersetup{
  pdftitle={Modern Physics (DI01000061) - Summer 2025 Solution (Gujarati)},
  pdfsubject={GTU Exam Solution - Summer-2025},
  pdfauthor={Milav Dabgar},
  pdfkeywords={study-material, solutions, gtu, DI01000061, gujarati},
  pdfcreator={xelatex}
}

\begin{document}
\maketitle

\setcounter{tocdepth}{5}
\tableofcontents
\newpage

% ========================================
% QUESTION 1: MCQs (14 marks)
% Demonstrates: 14 single mark questions
% ========================================

\section{પ્રશ્ન 1}

\subsection{પ્રશ્ન 1(a) [14 marks]}
\textbf{આપેલ વિકલ્પોમાંથી યોગ્ય પસંદગીનો ઉપયોગ કરીને ખાલી જગ્યાઓ પૂરો / બહુવિકલ્પી પ્રશ્નોના જવાબ આપો.}

\subsubsection{ઉકેલ}

\begin{enumerate}
    \item \textbf{તાપમાનનો SI એકમ \_\_\_\_\_\_\_\_\_\_ છે.}
    \begin{enumerate}
        \item કેલ્વિન
        \item ફેરનહીટ
        \item સેલ્સિયસ
        \item આમાંથી એકપણ નહીં
    \end{enumerate}
    \textbf{જવાબ:} (1) કેલ્વિન
    \paragraph{સમજૂતી:} તાપમાનનો SI (ઇન્ટરનેશનલ સિસ્ટમ ઓફ યુનિટ્સ) એકમ કેલ્વિન (K) છે. સેલ્સિયસ અને ફેરનહીટ અન્ય સ્કેલ છે પરંતુ SI એકમ નથી.
    \subparagraph{નોંધ:} નિરપેક્ષ શૂન્ય 0 K છે.
    \paragraph{મેમરી ટ્રીક:} \emph{Kelvin is King of SI units.}

    \item \textbf{Coulomb (કુલંબ) એ \_\_\_\_\_\_\_\_\_\_ નો SI એકમ છે.}
    \begin{enumerate}
        \item વિદ્યુત પ્રવાહ
        \item વિદ્યુત સ્થિતિમાન
        \item વિદ્યુતભાર
        \item વિદ્યુત ક્ષેત્ર
    \end{enumerate}
    \textbf{જવાબ:} (3) વિદ્યુતભાર
    \paragraph{સમજૂતી:} કુલંબ (C) એ વિદ્યુતભાર (Electric Charge) ના એકમ તરીકે વ્યાખ્યાયિત છે.
    \paragraph{મેમરી ટ્રીક:} \emph{C for Charge, C for Coulomb.}

    \item \textbf{0.0031 માં \_\_\_\_\_\_\_\_\_\_ સાર્થક અંકો (significant digits) છે.}
    \begin{enumerate}
        \item 5
        \item 4
        \item 2
        \item 3
    \end{enumerate}
    \textbf{જવાબ:} (3) 2
    \paragraph{સમજૂતી:} આગળના શૂન્યો ક્યારેય સાર્થક હોતા નથી. તેથી, 0.0031 માં, ફક્ત 3 અને 1 અંકો જ સાર્થક છે.
    \paragraph{મેમરી ટ્રીક:} \emph{Leading zeros lose value.}

    \item \textbf{P-type સેમિકન્ડક્ટર બનાવવા માટે કયા પ્રકારની અશુદ્ધિ ઉમેરવામાં આવે છે?}
    \begin{enumerate}
        \item Trivalent (ત્રિસંયોજક)
        \item Tetravalent (ચતુર્સંયોજક)
        \item Pentavalent (પંચસંયોજક)
        \item આમાંથી એકપણ નહીં
    \end{enumerate}
    \textbf{જવાબ:} (1) Trivalent (ત્રિસંયોજક)
    \paragraph{સમજૂતી:} Trivalent અશુદ્ધિઓ (જેમ કે બોરોન, એલ્યુમિનિયમ) માં 3 વેલેન્સ ઇલેક્ટ્રોન હોય છે, જે સેમિકન્ડક્ટર લેટીસમાં ''હોલ્સ'' (પોઝિટિવ ચાર્જ કેરિયર્સ) બનાવે છે, તેથી P-type.
    \paragraph{મેમરી ટ્રીક:} \emph{P for P-type, P comes after T? No. Trivalent starts with T, Three valence electrons.}

    \item \textbf{નીચેનામાંથી કયું બિન-યાંત્રિક (non-mechanical) તરંગ છે?}
    \begin{enumerate}
        \item ધ્વનિ તરંગ (sound wave)
        \item પ્રકાશ તરંગ (light wave)
        \item પાણીના તરંગો (water wave)
        \item અલ્ટાસોનિક તરંગ (Ultrasonic wave)
    \end{enumerate}
    \textbf{જવાબ:} (2) પ્રકાશ તરંગ (light wave)
    \paragraph{સમજૂતી:} પ્રકાશ તરંગો ઇલેક્ટ્રોમેગ્નેટિક તરંગો છે અને પ્રસરણ માટે ભૌતિક માધ્યમની જરૂર નથી, તેથી તે બિન-યાંત્રિક છે. ધ્વનિ, પાણી અને અલ્ટાસોનિક તરંગોને માધ્યમની જરૂર છે.
    \paragraph{મેમરી ટ્રીક:} \emph{Light travels through vacuum, so it's non-mechanical.}

    \item \textbf{વિદ્યુતભારનું ક્વોન્ટમીકરણ (quantization) \_\_\_\_\_\_\_\_\_\_ દ્વારા દર્શાવવામાં આવે છે.}
    \begin{enumerate}
        \item \(Q=ne\)
        \item \(Q=n/e\)
        \item \(Q=e\)
        \item આમાંથી એકપણ નહીં
    \end{enumerate}
    \textbf{જવાબ:} (1) \(Q=ne\)
    \paragraph{સમજૂતી:} ચાર્જ ક્વોન્ટમીકરણનો અર્થ એ છે કે ચાર્જ ડિસ્ક્રિટ પેકેટોમાં અસ્તિત્વ ધરાવે છે. કુલ ચાર્જ \(Q\) એ મૂળભૂત ચાર્જ \(e\) (\(1.6 \times 10^{-19} C\)) નો પૂર્ણાંક ગુણાંક \(n\) છે.
    \paragraph{મેમરી ટ્રીક:} \emph{Q = ne (Quarter Needs Energy? No, Quantity = Number * electron).}

    \item \textbf{ઈન્ફ્રાસોનિક તરંગોની ફ્રીક્વન્સી રેન્જ \_\_\_\_\_\_\_\_\_\_ છે.}
    \begin{enumerate}
        \item 20KHz થી વધુ
        \item 10KHz થી વધુ
        \item 20Hz થી 20KHz વચ્ચે
        \item 20Hz થી ઓછું
    \end{enumerate}
    \textbf{જવાબ:} (4) 20Hz થી ઓછું
    \paragraph{સમજૂતી:} ઈન્ફ્રાસોનિક તરંગોની ફ્રીક્વન્સી સાંભળી શકાય તેવા રેન્જ (20 Hz થી ઓછી) કરતા નીચે હોય છે. Audible 20 Hz થી 20 kHz છે. Ultrasonic 20 kHz થી ઉપર છે.
    \paragraph{મેમરી ટ્રીક:} \emph{Infra means Below (20 Hz).}

    \item \textbf{સ્નેલનો નિયમ \_\_\_\_\_\_\_\_\_\_ સાથે સંબંધિત છે.}
    \begin{enumerate}
        \item પ્રકાશનું પ્રસારણ (transmission)
        \item પ્રકાશનું વિવર્તન (diffraction)
        \item પ્રકાશનું પરાવર્તન (reflection)
        \item પ્રકાશનું વક્રીભવન (refraction)
    \end{enumerate}
    \textbf{જવાબ:} (4) પ્રકાશનું વક્રીભવન (refraction)
    \paragraph{સમજૂતી:} સ્નેલનો નિયમ આપાતકોણ અને વક્રીભૂતકોણ વચ્ચેના સંબંધનું વર્ણન કરે છે જ્યારે પ્રકાશ બે અલગ અલગ આઇસોટ્રોપિક માધ્યમો વચ્ચે પસાર થાય છે.
    \paragraph{મેમરી ટ્રીક:} \emph{Snell spells Refraction cells.}

    \item \textbf{ઓપ્ટિકલ ફાઈબર પ્રકાશના \_\_\_\_\_\_\_\_\_\_ સિદ્ધાંત પર કાર્ય કરે છે.}
    \begin{enumerate}
        \item ધ્રુવીભવન (Polarisation)
        \item વક્રીભવન (Refraction)
        \item પરાવર્તન (Reflection)
        \item પૂર્ણ આંતરિક પરાવર્તન (Total internal reflection)
    \end{enumerate}
    \textbf{જવાબ:} (4) પૂર્ણ આંતરિક પરાવર્તન (Total internal reflection)
    \paragraph{સમજૂતી:} ઓપ્ટિકલ ફાઇબર્સ પૂર્ણ આંતરિક પરાવર્તન (TIR) નો ઉપયોગ કરીને કોરની અંદર પ્રકાશને સીમિત કરીને કાર્ય કરે છે, જે તેને ન્યૂનતમ નુકસાન સાથે લાંબા અંતરની મુસાફરી કરવાની મંજૂરી આપે છે.
    \paragraph{મેમરી ટ્રીક:} \emph{Fiber traps light totally inside.}

    \item \textbf{લેસર રેડિયેશન \_\_\_\_\_\_\_\_\_\_ છે.}
    \begin{enumerate}
        \item મોનોક્રોમેટિક (એકરંગી)
        \item યુનિડાયરેક્શનલ (એકદિશીય)
        \item અત્યંત સુસંગત (Highly coherent)
        \item આપેલ તમામ
    \end{enumerate}
    \textbf{જવાબ:} (4) આપેલ તમામ
    \paragraph{સમજૂતી:} LASER (Light Amplification by Stimulated Emission of Radiation) મોનોક્રોમેટિક (એક રંગ), યુનિડાયરેક્શનલ (ઓછું ડાયવર્જન્સ) અને કોહેરન્ટ (in phase) હોવા દ્વારા વર્ગીકૃત થયેલ છે.
    \paragraph{મેમરી ટ્રીક:} \emph{Laser is Perfect Light: one color, one direction, one phase.}

    \item \textbf{ઓપ્ટિકલ ફાઈબર કેબલમાં, અંદરના ભાગને \_\_\_\_\_\_\_\_\_\_ કહેવામાં આવે છે.}
    \begin{enumerate}
        \item શીથ (Sheath)
        \item ક્લેડીંગ (Cladding)
        \item કોર (Core)
        \item આમાંથી એકપણ નહીં
    \end{enumerate}
    \textbf{જવાબ:} (3) કોર (Core)
    \paragraph{સમજૂતી:} કોર એ ઓપ્ટિકલ ફાઈબરનો મધ્ય ભાગ છે જ્યાં પ્રકાશનું પ્રસારણ થાય છે. તે ક્લેડીંગથી ઘેરાયેલું છે જેનો રિફ્રેક્ટિવ ઇન્ડેક્સ ઓછો હોય છે.
    \paragraph{મેમરી ટ્રીક:} \emph{Core is the Center.}

    \item \textbf{નીચેનામાંથી ક્યુ સેમિકન્ડક્ટર મટીરીયલ છે?}
    \begin{enumerate}
        \item એલ્યુમિનિયમ
        \item સિલિકોન
        \item ગેલિયમ
        \item આર્સેનિક
    \end{enumerate}
    \textbf{જવાબ:} (2) સિલિકોન
    \paragraph{સમજૂતી:} સિલિકોન (Si) અને જર્મેનિયમ (Ge) સૌથી સામાન્ય નેજ (intrinsic) સેમિકન્ડક્ટર મટીરીયલ્સ છે. એલ્યુમિનિયમ વાહક છે. ગેલિયમ અને આર્સેનિકનો ઉપયોગ કમ્પાઉન્ડ સેમિકન્ડક્ટર્સ (GaAs) માં થાય છે.
    \paragraph{મેમરી ટ્રીક:} \emph{Silicon Valley is Semiconductor Valley.}

    \item \textbf{PN જંકશન પર ફોરવર્ડ બાયસ વોલ્ટેજ વધારતા, ડેપ્લેશન લેયરની પહોળાઈ \_\_\_\_\_\_\_\_\_\_.}
    \begin{enumerate}
        \item કોઈ ફેરફાર થતો નથી
        \item વધે છે
        \item ઘટે છે
        \item આમાંથી એકપણ નહીં
    \end{enumerate}
    \textbf{જવાબ:} (3) ઘટે છે
    \paragraph{સમજૂતી:} ફોરવર્ડ બાયસમાં, બાહ્ય વિદ્યુત ક્ષેત્ર પોટેન્શિયલ બેરિયરનો વિરોધ કરે છે, મેજોરિટી કેરિયર્સને જંકશન તરફ ધકેલે છે, જે ડેપ્લેશન લેયરની પહોળાઈ ઘટાડે છે.
    \paragraph{મેમરી ટ્રીક:} \emph{Forward pushes narrower.}

    \item \textbf{LEDs \_\_\_\_\_\_\_\_\_\_ ઘટનાને કારણે પ્રકાશ ઉત્સર્જન કરે છે.}
    \begin{enumerate}
        \item ઇલેક્ટ્રોમેગ્નેટિક ઇન્ડક્શન
        \item ઇલેક્ટ્રોસ્ટેટિક ડિસ્ચાર્જ
        \item ઇલેક્ટ્રોલ્યુમિનેસન્સ (Electroluminescence)
        \item થર્મલ એમિશન
    \end{enumerate}
    \textbf{જવાબ:} (3) ઇલેક્ટ્રોલ્યુમિનેસન્સ (Electroluminescence)
    \paragraph{સમજૂતી:} ઇલેક્ટ્રોલ્યુમિનેસન્સ એ એક ઓપ્ટિકલ અને ઇલેક્ટ્રિકલ ઘટના છે જેમાં પદાર્થ ઇલેક્ટ્રિક કરંટ પસાર થવાના પ્રતિભાવમાં પ્રકાશનું ઉત્સર્જન કરે છે.
    \paragraph{મેમરી ટ્રીક:} \emph{Electron to Luminesce (Light).}
\end{enumerate}

% ========================================
% QUESTION 2: Short Questions (14 marks)
% Demonstrates: 3 marks and 4 marks questions
% ========================================

\section{પ્રશ્ન 2}

\noindent \textbf{પ્રશ્ન 2(a) [6 marks]} \\
\textbf{નીચેના પ્રશ્નોના જવાબ આપો. (3 માંથી કોઈપણ 2)}

\subsection{પ્રશ્ન 2(a)(1) [3 marks]}
\textbf{માઇક્રોમીટર સ્ક્રૂ ગેજની સ્વચ્છ આકૃતિ દોરો અને તેના વિવિધ ભાગોના નામ લખો.}

\subsubsection{ઉકેલ}
માઇક્રોમીટર સ્ક્રૂ ગેજ એક ચોકસાઇ સાધન છે જેનો ઉપયોગ ઉચ્ચ ચોકસાઈ સાથે નાના અંતરને માપવા માટે થાય છે.

\begin{figure}[H]
    \centering
    \includegraphics[width=0.6\linewidth]{micrometer_sketch.png}
    \caption{માઇક્રોમીટર સ્ક્રૂ ગેજ}\label{fig:micrometer}
\end{figure}

\paragraph{માઇક્રોમીટર સ્ક્રૂ ગેજના ભાગો:}
\begin{itemize}
    \item \textbf{ફ્રેમ (Frame):} તે U-આકારની ધાતુની ફ્રેમ છે જે અન્ય ભાગોને એકસાથે રાખે છે. તે સાધનને મજબૂતાઈ પૂરી પાડે છે.
    \item \textbf{એરણ (Anvil):} તે ફ્રેમના એક છેડે આવેલ નાનો નિશ્ચિત સ્ટડ છે જેની સામે વસ્તુ રાખવામાં આવે છે.
    \item \textbf{સ્પિન્ડલ (Spindle):} તે એક જંગમ નળાકાર ભાગ છે જે થિમ્બલને ફેરવવામાં આવે ત્યારે એરણ તરફ ખસે છે.
    \item \textbf{સ્લીવ (Main Scale):} તે મિલિમીટર (mm) માં અંકિત રેખીય સ્કેલ (મુખ્ય માપપટ્ટી) સાથેનો સ્થિર નળાકાર છે.
    \item \textbf{થિમ્બલ (Circular Scale):} તે સ્લીવ ઉપર ફરતો ચોકસાઈ માટેનો ભાગ છે. તે વર્તુળાકાર સ્કેલ વિભાગો (સામાન્ય રીતે 50 અથવા 100) ધરાવે છે.
    \item \textbf{રેચેટ (Ratchet):} તેનો ઉપયોગ વસ્તુ પર એકસમાન દબાણ લાગુ કરવા માટે થાય છે. જ્યારે સ્પિન્ડલ વસ્તુને સ્પર્શે છે ત્યારે તે 'ક્લિક' અવાજ કરે છે, જે વધુ પડતા દબાણને અટકાવે છે.
\end{itemize}

\paragraph{મેમરી ટ્રીક:} \emph{FAST Ratchet (Frame, Anvil, Spindle, Thimble, Ratchet).}

\subsection{પ્રશ્ન 2(a)(2) [3 marks]}
\textbf{ગાણિતિક સૂત્ર સાથે કુલંબનો નિયમ સમજાવો.}

\subsubsection{ઉકેલ}
\textbf{કુલંબનો નિયમ} જણાવે છે કે બે બિંદુવત વિદ્યુતભારો વચ્ચેના આકર્ષણ અથવા અપાકર્ષણના ઇલેક્ટ્રોસ્ટેટિક બળનું મૂલ્ય ચાર્જના મૂલ્યોના ગુણાકારના સમપ્રમાણમાં અને તેમની વચ્ચેના અંતરના વર્ગના વ્યસ્ત પ્રમાણમાં હોય છે. બળ બે ચાર્જને જોડતી રેખા સાથે કાર્ય કરે છે.

\paragraph{ગાણિતિક સૂત્ર:}
\[ F = k \frac{|q_1 q_2|}{r^2} \]
જ્યાં:
\begin{itemize}
    \item \(F\) = ઇલેક્ટ્રોસ્ટેટિક બળ (ન્યૂટન)
    \item \(q_1, q_2\) = ચાર્જના મૂલ્યો (કુલંબ)
    \item \(r\) = ચાર્જ વચ્ચેનું અંતર (મીટર)
    \item \(k\) = કુલંબનો અચળાંક (\(\approx 9 \times 10^9 \, N \cdot m^2 / C^2\))
\end{itemize}
\[ k = \frac{1}{4\pi\epsilon_0} \]
જ્યાં \(\epsilon_0\) મુક્ત અવકાશની પરમિટિવિટી છે.

\paragraph{મેમરી ટ્રીક:} \emph{Force depends on Product of charges and Inverse Square of distance.}

\paragraph{મહત્વ:}
કુલંબનો નિયમ એ વિદ્યુતશાસ્ત્રનો પાયાનો નિયમ છે. તે બે સ્થિર બિંદુવત વિદ્યુતભારો વચ્ચે લાગતા બળનું મૂલ્ય શોધવા માટે વપરાય છે. આ નિયમ ન્યૂટનના ગુરુત્વાકર્ષણના નિયમ જેવો જ વ્યસ્ત વર્ગનો નિયમ છે, પરંતુ અહીં બળ આકર્ષી અથવા અપાકર્ષી હોઈ શકે છે, જ્યારે ગુરુત્વાકર્ષણ બળ માત્ર આકર્ષી હોય છે. આ બળ પરમાણુની સ્થિરતા માટે જવાબદાર છે.

\subsection{પ્રશ્ન 2(a)(3) [3 marks]}
\textbf{એક વર્નિયર કેલિપર્સની મુખ્ય માપપટ્ટી મિલિમીટરમાં અંકિત કરેલ છે. વર્નિયર સ્કેલના 20 વિભાગો તેના મુખ્ય સ્કેલના 19 વિભાગો બરાબર છે, તો તેની લઘુત્તમ માપ શક્તિ (Least Count) શોધો.}

\subsubsection{ઉકેલ}
વર્નિયર કેલિપર્સ જેવા ચોકસાઇવાળા સાધનો 1 mm કરતા નાની લંબાઈ માપવા માટે બનાવવામાં આવ્યા છે, જે સામાન્ય મીટર પટ્ટી ચોકસાઈથી માપી શકતી નથી. લઘુત્તમ માપ શક્તિ સાધનની ન્યૂનતમ ચોકસાઈ દર્શાવે છે. લઘુત્તમ માપનાનું ઓછું મૂલ્ય ઉચ્ચ સચોટતા સૂચવે છે. અહીં, એક મુખ્ય સ્કેલ વિભાગ અને એક વર્નિયર સ્કેલ વિભાગ વચ્ચેનો તફાવત આપણને આ ચોકસાઈ આપે છે.

\paragraph{આપેલ માહિતી:}
\begin{itemize}
    \item 1 મુખ્ય સ્કેલ વિભાગ (MSD) = 1 mm (કારણ કે mm માં કેલિબ્રેટ થયેલ છે)
    \item 20 વર્નિયર સ્કેલ વિભાગ (VSD) = 19 મુખ્ય સ્કેલ વિભાગ (MSD)
\end{itemize}

\paragraph{ગણતરી:}
1 VSD નું મૂલ્ય:
\[ 20 \text{ VSD} = 19 \text{ MSD} \]
\[ 1 \text{ VSD} = \frac{19}{20} \text{ MSD} = \frac{19}{20} \times 1 \text{ mm} = 0.95 \text{ mm} \]

લઘુત્તમ માપ શક્તિ માટે સૂત્ર:
\[ \text{LC} = 1 \text{ MSD} - 1 \text{ VSD} \]
\[ \text{LC} = 1 \text{ mm} - 0.95 \text{ mm} \]
\[ \text{LC} = 0.05 \text{ mm} \]

વૈકલ્પિક રીતે:
\[ \text{LC} = \frac{1 \text{ MSD}}{\text{Total VSD}} = \frac{1 \text{ mm}}{20} = 0.05 \text{ mm} \]
આનો અર્થ એ છે કે સાધન 0.05 મીમી સુધીનો નરી આંખે જોઈ શકાય તેવો ફેરફાર ચોકસાઈથી માપી શકે છે.

\paragraph{જવાબ:} વર્નિયર કેલિપર્સની લઘુત્તમ માપ શક્તિ \textbf{0.05 mm} છે.

\paragraph{મેમરી ટ્રીક:} \emph{LC = 1MSD - 1VSD.}

\noindent \textbf{પ્રશ્ન 2(b) [8 marks]} \\
\textbf{નીચેના પ્રશ્નોના જવાબ આપો. (3 માંથી કોઈપણ 2)}

\subsection{પ્રશ્ન 2(b)(1) [4 marks]}
\textbf{આકૃતિ સાથે વર્નિયર કેલિપર્સની ધન (positive) અને ઋણ (negative) ત્રુટિ સમજાવો.}

\subsubsection{ઉકેલ}
\textbf{શૂન્ય ત્રુટિ} ત્યારે થાય છે જ્યારે જડબાં સંપૂર્ણપણે બંધ હોય ત્યારે વર્નિયર સ્કેલનું શૂન્ય ચિહ્ન મુખ્ય સ્કેલના શૂન્ય ચિહ્ન સાથે એકરુપ થતું નથી.

\paragraph{1. ધન શૂન્ય ત્રુટિ (Positive Zero Error):}
જો વર્નિયર સ્કેલનું શૂન્ય મુખ્ય સ્કેલ શૂન્યની \textbf{જમણી} બાજુએ હોય, તો ત્રુટિ ધન છે.
\begin{itemize}
    \item \textbf{સુધારો:} અવલોકન કરેલ રીડિંગમાંથી ત્રુટિ મૂલ્ય \textbf{બાદ} કરવામાં આવે છે.
    \item \emph{ઉદાહરણ:} જો 3જો વર્નિયર વિભાગ એકરુપ થાય, તો ત્રુટિ = \(+ (3 \times LC)\).
\end{itemize}

\paragraph{2. ઋણ શૂન્ય ત્રુટિ (Negative Zero Error):}
જો વર્નિયર સ્કેલનું શૂન્ય મુખ્ય સ્કેલ શૂન્યની \textbf{ડાબી} બાજુએ હોય, તો ત્રુટિ ઋણ છે.
\begin{itemize}
    \item \textbf{સુધારો:} અવલોકન કરેલ રીડિંગમાં ત્રુટિ મૂલ્ય \textbf{ઉમેરવામાં} આવે છે (કેમ કે ઋણની બાદબાકી એટલે સરવાળો). આ સુધારો કરવો ખૂબ જ જરૂરી છે અન્યથા માપન હંમેશા ઓછું આવશે.
    \item \emph{ઉદાહરણ:} જો 7મો વર્નિયર વિભાગ એકરુપ થાય (કુલ 10), તો ત્રુટિ = \(-(10-7) \times LC\).
\end{itemize}

\begin{figure}[H]
    \centering
    \includegraphics[width=0.8\linewidth]{vernier_calipers_error.png}
    \caption{વર્નિયર કેલિપર્સ શૂન્ય ત્રુટિ}\label{fig:vernier-error}
\end{figure}

\paragraph{ચકાસણી:}
ત્રુટિ ચકાસવા માટે, વર્નિયર કેલિપર્સના બંને જડબાં (Jaws) ને એકબીજા સાથે સંપૂર્ણપણે ભીડાવી દેવા જોઈએ. જો શૂન્ય સામે શૂન્ય સેટ ન થાય, તો જ સાધનમાં શૂન્ય ત્રુટિ છે તેમ કહેવાય.

\paragraph{મેમરી ટ્રીક:} \emph{Right is Positive (Subtracted), Left is Negative (Added).}

\subsection{પ્રશ્ન 2(b)(2) [4 marks]}
\textbf{વિદ્યુત ક્ષેત્ર રેખાઓના કોઈપણ ચાર ગુણધર્મો વર્ણવો.}

\subsubsection{ઉકેલ}
\textbf{વિદ્યુત ક્ષેત્ર રેખાઓ} એ કાલ્પનિક રેખાઓ છે જે વિદ્યુત ક્ષેત્રની દિશા અને તીવ્રતા દર્શાવે છે.

\paragraph{ગુણધર્મો:}
\begin{enumerate}
    \item \textbf{શરૂઆત અને અંત:} ક્ષેત્ર રેખાઓ \textbf{ધન} વિદ્યુતભારોમાંથી ઉદ્ભવે છે અને \textbf{ઋણ} વિદ્યુતભારો પર સમાપ્ત થાય છે. તે બંધ લૂપ્સ બનાવતા નથી (ચુંબકીય ક્ષેત્ર રેખાઓથી વિપરીત).
    \item \textbf{દિશા:} કોઈપણ બિંદુએ ક્ષેત્ર રેખાનો સ્પર્શક તે બિંદુએ વિદ્યુત ક્ષેત્રની દિશા આપે છે.
    \item \textbf{છેદન નથી:} બે વિદ્યુત ક્ષેત્ર રેખાઓ એકબીજાને \textbf{ક્યારેય છેદતી નથી}. જો તે છેદે, તો એક બિંદુ પર વિદ્યુત ક્ષેત્રની બે દિશાઓ હશે, જે અશક્ય છે.
    \item \textbf{ઘનતા:} ક્ષેત્ર રેખાઓની નિકટતા (ઘનતા) વિદ્યુત ક્ષેત્રની \textbf{તીવ્રતા} સૂચવે છે. નજીકની રેખાઓનો અર્થ મજબૂત ક્ષેત્ર થાય છે; છૂટી રેખાઓનો અર્થ નબળું ક્ષેત્ર થાય છે.
    \item \textbf{લંબ:} ક્ષેત્ર રેખાઓ હંમેશા વિદ્યુતભારિત વાહક ની સપાટીને લંબ હોય છે.
\end{enumerate}

\paragraph{મેમરી ટ્રીક:} \emph{Positive to Negative, Tangent direction, Never Cross, Density is Strength.}

\subsection{પ્રશ્ન 2(b)(3) [4 marks]}
\textbf{એક સાદા લોલકના આવર્તકાળના અવલોકનો 2.42s, 2.56s, 2.63s, 2.71s અને 2.80s છે. તો આવર્તકાળના અવલોકનમાં પ્રતિશત ત્રુટિ શોધો.}

\subsubsection{ઉકેલ}
પ્રતિશત ત્રુટિ શોધવા માટે, આપણે એક પદ્ધતિસરની પ્રક્રિયા અનુસરીએ છીએ: સરેરાશ મૂલ્ય (સાચું મૂલ્ય) શોધો, પછી વ્યક્તિગત નિરપેક્ષ ત્રુટિઓ શોધો, આ ત્રુટિઓની સરેરાશ કાઢો, અને છેલ્લે તેને સરેરાશના ટકા તરીકે દર્શાવો.

\paragraph{પગલું 1: સરેરાશ આવર્તકાળ (\(\bar{T}\))}
અવલોકનમાં રહેલી રેન્ડમ (યાદચ્છિક) ત્રુટિઓને ઘટાડવા માટે આપણે અંકગણિતીય સરેરાશ લઈએ છીએ.
\[ \bar{T} = \frac{2.42 + 2.56 + 2.63 + 2.71 + 2.80}{5} \]
\[ \bar{T} = \frac{13.12}{5} = \textbf{2.624 s} \]

\paragraph{પગલું 2: નિરપેક્ષ ત્રુટિઓ (\(|\Delta T_i|\))}
વ્યક્તિગત માપન અને સરેરાશ મૂલ્ય વચ્ચેનો તફાવત.
\begin{itemize}
    \item \(|\Delta T_1| = |2.42 - 2.624| = 0.204\)
    \item \(|\Delta T_2| = |2.56 - 2.624| = 0.064\)
    \item \(|\Delta T_3| = |2.63 - 2.624| = 0.006\)
    \item \(|\Delta T_4| = |2.71 - 2.624| = 0.086\)
    \item \(|\Delta T_5| = |2.80 - 2.624| = 0.176\)
\end{itemize}

\paragraph{પગલું 3: સરેરાશ નિરપેક્ષ ત્રુટિ (\(\Delta \bar{T}\))}
બધી નિરપેક્ષ ત્રુટિઓની સરેરાશ.
\[ \Delta \bar{T} = \frac{0.204 + 0.064 + 0.006 + 0.086 + 0.176}{5} \]
\[ \Delta \bar{T} = \frac{0.536}{5} = \textbf{0.1072 s} \]

\paragraph{પગલું 4: સાપેક્ષ ત્રુટિ (\(\delta T\))}
સરેરાશ નિરપેક્ષ ત્રુટિ અને સરેરાશ મૂલ્યનો ગુણોત્તર.
\[ \delta T = \frac{\Delta \bar{T}}{\bar{T}} = \frac{0.1072}{2.624} \approx 0.04085 \]

\paragraph{પગલું 5: પ્રતિશત ત્રુટિ}
\[ \% \text{ Error} = \delta T \times 100\% \]
\[ \% \text{ Error} = 0.04085 \times 100\% \approx \textbf{4.1\%} \]

\paragraph{જવાબ:} આવર્તકાળના અવલોકનમાં પ્રતિશત ત્રુટિ આશરે \textbf{4.1\%} છે.

\paragraph{મેમરી ટ્રીક:} \emph{Percentage Error = (Mean Absolute Error / Mean Value) * 100.}

% ========================================
% QUESTION 3: Short Questions (14 marks)
% Demonstrates: 3 marks and 4 marks questions
% ========================================

\section{પ્રશ્ન 3}

\noindent \textbf{પ્રશ્ન 3(a)} \\
\textbf{નીચેના પ્રશ્નોના જવાબ આપો. (3 માંથી કોઈપણ 2)}

\subsection{પ્રશ્ન 3(a)(1) [3 marks]}
\textbf{સાંગત તરાંગ અને લાંબગત તરાંગ િચ્ચેના કોઈપણ ત્રણ તફાિત લખો.}

\subsubsection{ઉકેલ}
\paragraph{તફાવત:}
\begin{table}[H]
\caption{સંગત તરંગો અને લંબગત તરંગો વચ્ચેનો તફાવત}
\centering
\begin{tabularx}{\textwidth}{|X|X|}
\hline
\textbf{સંગત તરંગો (Longitudinal Waves)} & \textbf{લંબગત તરંગો (Transverse Waves)} \\
\hline
1. માધ્યમના કણો તરંગ પ્રસરણની દિશાને \textbf{સમાંતર} કંપન કરે છે. & 1. માધ્યમના કણો તરંગ પ્રસરણની દિશાને \textbf{લંબ} કંપન કરે છે. \\
\hline
2. તે \textbf{સંઘનન (compressions)} અને \textbf{વિઘનન (rarefactions)} સ્વરૂપે આગળ વધે છે. & 2. તે \textbf{શૃંગ (crests)} અને \textbf{ગર્ત (troughs)} સ્વરૂપે આગળ વધે છે. \\
\hline
3. તે ઘન, પ્રવાહી અને વાયુઓમાં પ્રસરી શકે છે. & 3. તે ઘન અને પ્રવાહીની સપાટી પર પ્રસરી શકે છે, પરંતુ વાયુઓ અથવા પ્રવાહીની અંદર નહીં. \\
\hline
4. આ તરંગોનું \textbf{ધ્રુવીભવન (polarization) કરી શકાતું નથી}. & 4. આ તરંગોનું \textbf{ધ્રુવીભવન કરી શકાય છે}. \\
\hline
5. ઉદાહરણ: હવામાં ધ્વનિ તરંગો, P-સિસ્મિક તરંગો. & 5. ઉદાહરણ: પ્રકાશના તરંગો, રેડિયો તરંગો, S-સિસ્મિક તરંગો. \\
\hline
\end{tabularx}
\end{table}

\paragraph{મેમરી ટ્રીક:} \emph{Longitudinal = Parallel (Sound), Transverse = Perpendicular (Light).}

\subsection{પ્રશ્ન 3(a)(2) [3 marks]}
\textbf{અલ્ટાસોનિક તરંગોની કોઈપણ બે ઉપયોગીતા સવિસ્તાર સમજાવો.}

\subsubsection{ઉકેલ}
\paragraph{વ્યાખ્યા:} \textbf{અલ્ટાસોનિક તરંગો (Ultrasonic waves)} એ ધ્વનિ તરંગો છે જેની આવૃત્તિ માનવ શ્રવણ શક્તિની ઉચ્ચ સીમા (20 kHz થી વધુ) કરતા હોય છે.

\begin{enumerate}
    \item \textbf{SONAR (Sound Navigation And Ranging):}
    \begin{itemize}
        \item પાણીની અંદરની વસ્તુઓ (સબમરીન, ડૂબેલા જહાજ) શોધવા અને સમુદ્રની ઊંડાઈ માપવા માટે વપરાય છે.
        \item તે પરાવર્તન (echo) ના સિદ્ધાંત પર કામ કરે છે. અલ્ટાસોનિક તરંગો નીચે મોકલવામાં આવે છે, વસ્તુ પરથી પરાવર્તિત થાય છે અને તેને શોધી કાઢવામાં આવે છે. સમયનો તફાવત અંતર આપે છે ($d = v \times t / 2$).
    \end{itemize}
    \item \textbf{તબીબી ઇમેજિંગ (Ultrasound Scanning):}
    \begin{itemize}
        \item શરીરના આંતરિક અવયવો (દા.ત. ગર્ભાવસ્થા દરમિયાન ગર્ભ, કિડની સ્ટોન, હૃદય) ની છબીઓ બનાવવા માટે વપરાય છે.
        \item જુદી જુદી પેશીઓ અલ્ટ્રાસાઉન્ડને અલગ રીતે પરાવર્તિત કરે છે, હાનિકારક રેડિયેશન વગર મોનિટર પર છબી બનાવે છે (X-rays થી વિપરીત).
    \end{itemize}
\end{enumerate}

\paragraph{મેમરી ટ્રીક:} \emph{Sonar for Sea, Ultrasound for See (inside body).}

\subsection{પ્રશ્ન 3(a)(3) [3 marks]}
\textbf{બે વિદ્યુતભારો \(20 \mu C\) અને \(10 \mu C\) હવામાં 0.02 m ના અંતરે મુકેલા છે. તો તેમની વચ્ચે લાગતું વિદ્યુતબળ અથવા કુલંબ બળ શોધો. K નું મુલ્ય \(9 \times 10^9 \, N m^2 / C^2\) છે.}

\subsubsection{ઉકેલ}
\textbf{આપેલ માહિતી:}
\begin{itemize}
    \item વિદ્યુતભાર \(q_1 = 20 \, \mu C = 20 \times 10^{-6} \, C\)
    \item વિદ્યુતભાર \(q_2 = 10 \, \mu C = 10 \times 10^{-6} \, C\)
    \item અંતર \(r = 0.02 \, m = 2 \times 10^{-2} \, m\)
    \item અચળાંક \(k = 9 \times 10^9 \, N \cdot m^2 / C^2\)
\end{itemize}

\textbf{સૂત્ર:}
કુલંબનો નિયમ:
\[ F = k \frac{|q_1 q_2|}{r^2} \]

\textbf{ગણતરી:}
\[ F = 9 \times 10^9 \times \frac{(20 \times 10^{-6}) \times (10 \times 10^{-6})}{(2 \times 10^{-2})^2} \]
\[ F = 9 \times 10^9 \times \frac{200 \times 10^{-12}}{4 \times 10^{-4}} \]
\[ F = \frac{9 \times 200}{4} \times 10^{9 - 12 - (-4)} \]
\[ F = 450 \times 10^1 \]
\[ F = 4500 \, N \]

\paragraph{નિષ્કર્ષ:}
બંને વિદ્યુતભાર ધન હોવાથી (\(+20 \mu C\) અને \(+10 \mu C\)), બળ \textbf{અપાકર્ષી} (repulsive) પ્રકારનું હશે. આ પ્રચંડ બળ બે વિદ્યુતભારોને જોડતી રેખા પર લાગે છે.

\paragraph{જવાબ:} વિદ્યુતભારો વચ્ચે લાગતુ વિદ્યુતબળ \textbf{4500 N} (અપાકર્ષી) છે.

\paragraph{મેમરી ટ્રીક:} \emph{F = k q1 q2 / r squared.}


\noindent \textbf{પ્રશ્ન 3(b)} \\
\textbf{નીચેના પ્રશ્નોના જવાબ આપો. (3 માંથી કોઈપણ 2)}

\subsection{પ્રશ્ન 3(b)(1) [4 marks]}
\textbf{વ્યાખ્યા આપો: (1) ચોક્કસાઈ (Accuracy) (2) સચોટતા (Precision) (3) વિદ્યુત ફ્લક્સ (Electric Flux) (4) વિદ્યુત સ્થિતિમાન (Electric Potential)}

\subsubsection{ઉકેલ}
\paragraph{વ્યાખ્યાઓ:}
\begin{enumerate}
    \item \textbf{ચોકસાઈ (Accuracy):} માપેલું મૂલ્ય \textbf{સાચા (standard) મૂલ્ય}ની કેટલી નજીક છે તેને ચોકસાઈ કહે છે. તે માપન કેટલું સાચું છે તે દર્શાવે છે અને પ્રયોગમાં રહેલી વ્યવસ્થિત ત્રુટિઓ પર આધાર રાખે છે. ઉચ્ચ ચોકસાઈ એટલે ઓછી ત્રુટિ.
    \item \textbf{સચોટતા (Precision):} બે કે તેથી વધુ માપેલા મુલ્યો \textbf{એકબીજાની કેટલી નજીક} છે તેને સચોટતા કહે છે. તે માપનનું વિભેદન (resolution) દર્શાવે છે. તે તકનીકી રીતે માપન સાધનની લઘુત્તમ માપ શક્તિ પર આધાર રાખે છે. ઉચ્ચ સચોટતા એટલે મૂલ્યો એકબીજાની ખૂબ નજીક હોય છે.
    \item \textbf{વિદ્યુત ફ્લક્સ (\(\Phi_E\)):} આપેલ ક્ષેત્રફળમાંથી લંબરૂપે પસાર થતી વિદ્યુતક્ષેત્ર રેખાઓની કુલ સંખ્યા. તે અદિશ રાશિ છે. જો ક્ષેત્રરેખાઓ બંધ પૃષ્ઠમાં પ્રવેશે તો ફ્લક્સ ઋણ અને બહાર નીકળે તો ધન ગણાય છે. સૂત્ર: \(\Phi_E = \vec{E} \cdot \vec{A}\). એકમ: \(N \cdot m^2/C\).
    \item \textbf{વિદ્યુત સ્થિતિમાન (\(V\)):} અનંત અંતરેથી એકમ ધન વિદ્યુતભારને આપેલ બિંદુએ લાવવા માટે વિદ્યુતક્ષેત્રની વિરુદ્ધ કરવા પડતા કાર્યને તે બિંદુ આગળનું વિદ્યુત સ્થિતિમાન કહે છે. તે સ્થાનનો અદિશ ગુણધર્મ છે. સૂત્ર: \(V = W/q\). એકમ: Volt (\(V\)).
\end{enumerate}

\paragraph{મેમરી ટ્રીક:} \emph{Accuracy is Truth, Precision is Repetition, Flux is Flow lines, Potential is Work/Charge.}

\subsection{પ્રશ્ન 3(b)(2) [4 marks]}
\textbf{એક ધ્વનિતરંગની આવૃત્તિ 500Hz અને વેગ 1500m/s છે તો તેની તરંગલંબાઈ શોધો.}

\subsubsection{ઉકેલ}
\textbf{આપેલ માહિતી:}
\begin{itemize}
    \item આવૃત્તિ ($f$) = 500 Hz
    \item વેગ ($v$) = 1500 m/s
\end{itemize}

\textbf{સૂત્ર:}
તરંગ વેગ, આવૃત્તિ અને તરંગલંબાઈ વચ્ચેનો મૂળભૂત સંબંધ છે:
\[ v = f \times \lambda \]
જ્યાં:
\begin{itemize}
    \item \(v\) એ તરંગ વેગ (m/s) છે
    \item \(f\) એ આવૃત્તિ (Hz) છે
    \item \(\lambda\) એ તરંગલંબાઈ (m) છે
\end{itemize}

\textbf{ગણતરી:}
તરંગલંબાઈ શોધવા માટે સૂત્રને ફરીથી ગોઠવતા:
\[ \lambda = \frac{v}{f} \]
આપેલ કિંમતો મૂકતા:
\[ \lambda = \frac{1500}{500} \]
\[ \lambda = 3 \, m \]
આનો અર્થ એ છે કે તરંગના દરેક ચક્ર માટે, તે 3 મીટરનું ક્ષૈતિજ અંતર કાપે છે.

\paragraph{અર્થઘટન:}
આ ધ્વનિ તરંગ એક પૂર્ણ દોલન દરમિયાન 3 મીટરનું અંતર કાપે છે એટલે કે તેની તરંગલંબાઈ 3 મીટર છે. 500 Hz આવૃત્તિ મનુષ્યની સાંભળવાની ક્ષમતા (20 Hz થી 20 kHz) ની વચ્ચે આવે છે, તેથી આ અવાજ આપણે સ્પષ્ટ રીતે સાંભળી શકીએ છીએ. વધુમાં, હવામાં ધ્વનિનો વેગ સામાન્ય તાપમાને આશરે 343 m/s હોય છે, પરંતુ અહીં આપેલ વેગ 1500 m/s છે જે દર્શાવે છે કે આ તરંગ પાણી અથવા કોઈ અન્ય પ્રવાહી માધ્યમમાં પ્રસરી રહ્યું છે.

\paragraph{જવાબ:} ધ્વનિતરંગની તરંગલંબાઈ \textbf{3 મીટર} છે.

\paragraph{મેમરી ટ્રીક:} \emph{Velocity equals Frequency times Wavelength.}

\subsection{પ્રશ્ન 3(b)(3) [4 marks]}
\textbf{બે પ્લેટ વચ્ચેનું અંતર 1mm છે, જો આપણે કેપેસીટન્સ 0.1 F મેળવવું હોય તો પ્લેટનું ક્ષેત્રફળ કેટલું જોઈએ? \(\epsilon_0 = 8.85 \times 10^{-12} F/m\)}

\subsubsection{ઉકેલ}
\textbf{આપેલ માહિતી:}
\begin{itemize}
    \item અંતર (\(d\)) = 1 mm = \(1 \times 10^{-3} \, m\)
    \item કેપેસીટન્સ (\(C\)) = 0.1 F
    \item મુક્ત અવકાશની પરમિટિવિટી (\(\epsilon_0\)) = \(8.85 \times 10^{-12} \, F/m\)
\end{itemize}

\textbf{સૂત્ર:}
સમાંતર પ્લેટ કેપેસીટર:
\[ C = \frac{\epsilon_0 A}{d} \]
જ્યાં \(A\) એ પ્લેટનું ક્ષેત્રફળ છે.

\textbf{ગણતરી:}
ક્ષેત્રફળ \(A\) શોધવા માટે સૂત્રને ફરીથી ગોઠવતા:
\[ A = \frac{C \times d}{\epsilon_0} \]

કિંમતો મૂકતા:
\[ A = \frac{0.1 \times 10^{-3}}{8.85 \times 10^{-12}} \]
\[ A = \frac{10^{-4}}{8.85} \times 10^{12} \]
\[ A = \frac{1}{8.85} \times 10^{8} \]
\[ A \approx 0.11299 \times 10^8 \, m^2 \]
\[ A \approx 1.13 \times 10^7 \, m^2 \]

\paragraph{મહત્વ:}
ગણતરી કરેલ ક્ષેત્રફળ (\(\approx 11.3 \, km^2\)) અત્યંત વિશાળ છે. આ દર્શાવે છે કે સાદા સમાંતર પ્લેટ કેપેસીટરથી 0.1 F જેટલું ઊંચું કેપેસીટન્સ મેળવવા માટે અવ્યવહારુ ક્ષેત્રફળની જરૂર પડે છે. વ્યવહારમાં, આપણે આ હાંસલ કરવા માટે ખાસ ઉત્પાદન તકનીકો (જેમ કે રોલિંગ) અને ડાઈઇલેક્ટ્રિક્સનો ઉપયોગ કરીએ છીએ.

\paragraph{જવાબ:} પ્લેટનું જરૂરી ક્ષેત્રફળ આશરે \(\mathbf{1.13 \times 10^7 \, m^2}\) છે.

\paragraph{મેમરી ટ્રીક:} \emph{C = Epsilon A over d.}

% ========================================
% QUESTION 4: Short Questions (14 marks)
% Demonstrates: 3 marks and 4 marks questions with Diagrams
% ========================================

\section{પ્રશ્ન 4}

\noindent \textbf{પ્રશ્ન 4(a)} \\
\textbf{નીચેના પ્રશ્નોના જવાબ આપો. (3 માંથી કોઈપણ 2)}

\subsection{પ્રશ્ન 4(a)(1) [3 marks]}
\textbf{અલ્ટાસોનિક તરંગોના કોઈપણ ત્રણ ગુણધર્મો લખો.}

\subsubsection{ઉકેલ}
\paragraph{અલ્ટાસોનિક તરંગોના ગુણધર્મો:}
\begin{enumerate}
    \item \textbf{ઉચ્ચ ઊર્જા અને આવૃત્તિ:} તેમની આવૃત્તિ 20kHz કરતાં વધુ હોય છે. ઉચ્ચ આવૃત્તિને કારણે, તેઓ ખૂબ ઊર્જા ધરાવે છે, જે તેમને ડ્રિલિંગ અને સફાઈ હેતુઓ માટે ઉપયોગી બનાવે છે.
    \item \textbf{દિશાકીયતા (તીક્ષ્ણ બીમ):} તેમની તરંગલંબાઈ ખૂબ નાની હોવાથી, તેઓ વધારે ફેલાયા વગર (અવગણ્ય વિવર્તન) તીક્ષ્ણ બીમ તરીકે ચોક્કસ દિશામાં મુસાફરી કરી શકે છે.
    \item \textbf{પરાવર્તન અને પડઘા:} તેઓ પ્રકાશ તરંગોની જેમ પરાવર્તનના નિયમોનું પાલન કરે છે. જ્યારે તેઓ કોઈ અવરોધ સાથે અથડાય છે, ત્યારે તેઓ પાછા પરાવર્તિત થાય છે, અને પડઘા ઉત્પન્ન કરે છે. આ ગુણધર્મ SONAR ટેકનોલોજીનો આધાર છે.
    \item \textbf{ભેદન શક્તિ:} તેઓ ઘણા પદાર્થોમાં (જેમ કે ધાતુના બ્લોક્સ) પ્રવેશી શકે છે પરંતુ તિરાડો અથવા ખામીઓ દ્વારા પરાવર્તિત થાય છે, જે નોન-ડિસ્ટ્રક્ટિવ ટેસ્ટિંગ (NDT) માટે ઉપયોગી છે.
\end{enumerate}

\paragraph{મેમરી ટ્રીક:} \emph{High Energy, Sharp Beam, Bounces like Light.}

\subsection{પ્રશ્ન 4(a)(2) [3 marks]}
\textbf{લેસર પ્રકાશ અને સામાન્ય પ્રકાશ વચ્ચેના કોઈપણ ત્રણ તફાવત લખો.}

\subsubsection{ઉકેલ}
\paragraph{તફાવત:}
\begin{table}[H]
\caption{સામાન્ય પ્રકાશ અને લેસર પ્રકાશ વચ્ચેનો તફાવત}
\centering
\begin{tabularx}{\textwidth}{|X|X|}
\hline
\textbf{સામાન્ય પ્રકાશ (Common Light)} & \textbf{લેસર પ્રકાશ (Laser Light)} \\
\hline
1. \textbf{પોલીક્રોમેટિક (Polychromatic):} ઘણી તરંગલંબાઇઓ (મિશ્રિત વિવિધ રંગો) ના તરંગો ધરાવે છે. & 1. \textbf{મોનોક્રોમેટિક (Monochromatic):} એક જ ચોક્કસ તરંગલંબાઇ (એક વિશિષ્ટ રંગ) ના તરંગો ધરાવે છે. \\
\hline
2. \textbf{અસંગત (Incoherent):} તરંગો એકબીજા સાથે કળામાં હોતા નથી; શૃંગ અને ગર્ત મેળ ખાતા નથી. & 2. \textbf{સુસંગત (Coherent):} તરંગો એકબીજા સાથે કળામાં હોય છે; બધા શૃંગ અને ગર્ત સંપૂર્ણપણે સંરેખિત (align) થાય છે. \\
\hline
3. \textbf{વિકેન્દ્રિત (Divergent):} બધી દિશાઓમાં ઝડપથી ફેલાય છે (દા.ત., બલ્બનો પ્રકાશ રૂમ ભરે છે). & 3. \textbf{ઉચ્ચ દિશાકીય (Highly Directional):} ઓછામાં ઓછા ફેલાવા સાથે લાંબા અંતર સુધી સાંકડા, સમાંતર બીમ તરીકે મુસાફરી કરે છે. \\
\hline
4. \textbf{ઓછી તીવ્રતા:} અંતર સાથે તીવ્રતા ઝડપથી ઘટે છે. & 4. \textbf{ઉચ્ચ તીવ્રતા:} ઊર્જા નાના વિસ્તારમાં કેન્દ્રિત હોય છે, જે તેને ખૂબ જ તેજસ્વી અને શક્તિશાળી બનાવે છે. \\
\hline
\end{tabularx}
\end{table}

\paragraph{મેમરી ટ્રીક:} \emph{Laser is MCD (Monochromatic, Coherent, Directional).}

\subsection{પ્રશ્ન 4(a)(3) [3 marks]}
\textbf{પ્રકાશનું પૂર્ણ આંતરિક પરાવર્તન આકૃતિ સાથે સમજાવો.}

\subsubsection{ઉકેલ}
\textbf{પૂર્ણ આંતરિક પરાવર્તન (Total Internal Reflection - TIR):}
જ્યારે પ્રકાશનું કિરણ \textbf{પ્રકાશીય ઘટ્ટ માધ્યમ} (દા.ત., કાચ, પાણી) માંથી \textbf{પ્રકાશીય પાતળા માધ્યમ} (દા.ત., હવા) માં પ્રવેશે છે અને આપાતકોણ ($i$) તે માધ્યમોની જોડ માટેના \textbf{ક્રાંતિકોણ} ($C$) કરતા \textbf{મોટો} હોય, ત્યારે કિરણ પાતળા માધ્યમમાં વક્રીભવન પામતું નથી પરંતુ ઘટ્ટ માધ્યમમાં પાછું પરાવર્તિત થાય છે. આ ઘટનાને પૂર્ણ આંતરિક પરાવર્તન કહેવાય છે.

\begin{figure}[H]
\centering
\begin{tikzpicture}[scale=0.8]
    % Mediums
    \fill[cyan!10] (-5,-3) rectangle (5,0);
    \draw[thick] (-5,0) -- (5,0);
    \node at (3.5,-0.5) {Denser ($n_1$)};
    \node at (3.5,0.5) {Rarer ($n_2$)};
    
    % Source
    \coordinate (S) at (0,-2.5);
    \fill (S) circle (2pt) node[below] {Source};
    
    % Ray 1: Refraction (Small angle)
    \draw[red, thick, ->] (S) -- (-2,0);
    \draw[red, thick, ->] (-2,0) -- (-3,1.5);
    \draw[dashed] (-2,-0.5) -- (-2,1.5); % Normal
    
    % Ray 2: Critical Angle (90 degree refraction)
    \draw[orange, thick, ->] (S) -- (1.5,0);
    \draw[orange, thick, ->] (1.5,0) -- (3.5,0);
    \draw[dashed] (1.5,-0.5) -- (1.5,1.5); % Normal
    \node at (1.1, -0.4) {\(i=C\)};
    
    % Ray 3: TIR (i > C)
    \draw[blue, thick, ->] (S) -- (3.5,0);
    \draw[blue, thick, ->] (3.5,0) -- (2,-2.5) node[midway, right] {TIR};
    \draw[dashed] (3.5,-0.5) -- (3.5,1.5); % Normal
    \node at (3.2, -0.4) {\(i>C\)};
    
\end{tikzpicture}
\caption{પૂર્ણ આંતરિક પરાવર્તન (TIR)}
\end{figure}

\paragraph{શરતો:}
\begin{enumerate}
    \item પ્રકાશ ઘટ્ટ માધ્યમમાંથી પાતળા માધ્યમમાં જવો જોઈએ.
    \item આપાતકોણ ($i$) એ ક્રાંતિકોણ ($C$) કરતા મોટો હોવો જોઈએ (\(i > C\)).
\end{enumerate}

\paragraph{મેમરી ટ્રીક:} \emph{Denser to Rarer, Angle > Critical = Reflection.}

\noindent \textbf{પ્રશ્ન 4(b)} \\
\textbf{નીચેના પ્રશ્નોના જવાબ આપો. (3 માંથી કોઈપણ 2)}

\subsection{પ્રશ્ન 4(b)(1) [4 marks]}
\textbf{સંજ્ઞા દોરો: (1) p-n જંકશન ડાયોડ (2) ઝેનર ડાયોડ અને વ્યાખ્યા આપો: (3) વેલેન્સ ઇલેક્ટ્રોન (4) ડોપિંગ પ્રક્રિયા.}

\subsubsection{ઉકેલ}

\paragraph{1. સંજ્ઞાઓ:}
\begin{figure}[H]
    \centering
    \begin{tikzpicture}
        % PN Junction Diode
        \draw (0,0) to[D, l=p-n Junction Diode] (3,0);
        
        % Zener Diode
        \draw (5,0) to[zD, l=Zener Diode] (8,0);
    \end{tikzpicture}
    \caption{p-n જંકશન ડાયોડ અને ઝેનર ડાયોડની સંજ્ઞાઓ}
\end{figure}

\paragraph{2. વ્યાખ્યાઓ:}

\begin{enumerate}
    \item[(3)] \textbf{વેલેન્સ ઇલેક્ટ્રોન:} પરમાણુની \textbf{સૌથી બહારની કક્ષા} (શેલ) માં રહેલા ઇલેક્ટ્રોનને વેલેન્સ ઇલેક્ટ્રોન કહે છે. તેઓ અંદરના ઇલેક્ટ્રોનની તુલનામાં ન્યુક્લિયસ સાથે શિથિલ રીતે બંધાયેલા હોય છે. આ ઇલેક્ટ્રોન તત્વની સંયોજકતા નક્કી કરવા અને રાસાયણિક બંધન તેમજ પદાર્થમાં વિદ્યુત વહનમાં ભાગ લેવા માટે જવાબદાર છે. રાસાયણિક પ્રક્રિયાઓમાં ભાગ લેતા હોવાથી તે ખૂબ મહત્વના છે.
    \item[(4)] \textbf{ડોપિંગ પ્રક્રિયા:} શુદ્ધ અર્ધવાહક (જેમ કે Si અથવા Ge) ની વિદ્યુત વાહકતા વધારવા માટે તેમાં ઇરાદાપૂર્વક નિયંત્રિત માત્રામાં યોગ્ય \textbf{અશુદ્ધિ} (જેને ડોપન્ટ કહેવાય છે) ઉમેરવાની પ્રક્રિયાને ડોપિંગ કહે છે. ડોપિંગને કારણે સેમિકન્ડક્ટરની વાહકતા લાખો ગણી વધી જાય છે, જે ઇલેક્ટ્રોનિક ઉપકરણો બનાવવા માટે જરૂરી છે. ઉદાહરણ તરીકે, આર્સેનિક ઉમેરવાથી n-પ્રકાર અને ગેલિયમ ઉમેરવાથી p-પ્રકારના અર્ધવાહક બને છે.
\end{enumerate}

\paragraph{મેમરી ટ્રીક:} \emph{Valence = Outer, Doping = Adding Impurity.}

\subsection{પ્રશ્ન 4(b)(2) [4 marks]}
\textbf{ઓપ્ટીકલ ફાઈબરની રચના આકૃતિ સાથે સવિસ્તાર સમજાવો.}

\subsubsection{ઉકેલ}
\paragraph{રચના:} \textbf{ઓપ્ટીકલ ફાઈબરની રચના:}
ઓપ્ટીકલ ફાઈબર ત્રણ સમકેન્દ્રિય નળાકાર સ્તરો ધરાવે છે:

\begin{enumerate}
    \item \textbf{કોર (Core):} ઉચ્ચ ગુણવત્તાવાળા કાચ અથવા પ્લાસ્ટિકથી બનેલો સૌથી અંદરનો મધ્ય ભાગ. તે પ્રકાશ સિગ્નલનું વહન કરે છે. તેનો વક્રીભવનાંક ($n_1$) ઊંચો હોય છે.
    \item \textbf{ક્લેડીંગ (Cladding):} કોરની આસપાસનું મધ્યમ સ્તર. તેનો વક્રીભવનાંક કોર કરતા \textbf{સહેજ ઓછો} ($n_2 < n_1$) હોય છે. આ શરતને કારણે પૂર્ણ આંતરિક પરાવર્તન (TIR) થાય છે અને પ્રકાશ કોરની અંદર જળવાઈ રહે છે.
    \item \textbf{બફર કોટિંગ/જેકેટ (Jacket):} સૌથી બહારનું પ્લાસ્ટિકનું સ્તર જે ફાઈબરને ભેજ, ભૌતિક નુકસાન અને તૂટવાથી બચાવે છે.
\end{enumerate}

\begin{figure}[H]
\centering
\begin{tikzpicture}
    % Core
    \fill[cyan!30] (0,1) rectangle (6,2);
    \draw[thick] (0,1) -- (6,1);
    \draw[thick] (0,2) -- (6,2);
    \node at (3,1.5) {Core ($n_1$)};
    
    % Cladding
    \fill[gray!20] (0,0) rectangle (6,1);
    \fill[gray!20] (0,2) rectangle (6,3);
    \draw[thick] (0,0) -- (6,0);
    \draw[thick] (0,3) -- (6,3);
    \node at (3,0.5) {Cladding ($n_2$)};
    \node at (3,2.5) {Cladding ($n_2$)};
    
    % Propagating light ray
    \draw[red, thick, ->] (0,1.5) -- (1,2) -- (2,1) -- (3,2) -- (4,1) -- (5,2) -- (6,1.5);
    \node[red, right] at (6,1.5) {Light Signal};

    % Labelling n1 > n2
    \node[right] at (6.2, 1.5) {\(n_1 > n_2\) (TIR Condition)};
\end{tikzpicture}
\caption{ઓપ્ટીકલ ફાઈબરની રચના}
\end{figure}

\paragraph{મેમરી ટ્રીક:} \emph{Core keeps Light (High n), Cladding reflects (Low n), Jacket protects.}

\subsection{પ્રશ્ન 4(b)(3) [4 marks]}
\textbf{બ્રીજ રેકટીફાયર તેની સર્કિટ, ઈનપુટ અને આઉટપુટ વેવફોર્મ સાથે સમજાવો.}

\subsubsection{ઉકેલ}
\textbf{બ્રીજ રેકટીફાયર:}
બ્રીજ રેકટીફાયર AC વોલ્ટેજને પૂર્ણ-તરંગ (Full-wave) DC વોલ્ટેજમાં રૂપાંતરિત કરવા માટે બ્રીજ કન્ફિગરેશનમાં ગોઠવાયેલા \textbf{ચાર ડાયોડ} ($D_1, D_2, D_3, D_4$) નો ઉપયોગ કરે છે.

\paragraph{કાર્યપદ્ધતિ:}
\begin{itemize}
    \item \textbf{ધન ચક્ર (Positive Cycle):} ડાયોડ \(D_1\) અને \(D_3\) ફોરવર્ડ બાયસ (વાહક) બને છે, જ્યારે \(D_2\) અને \(D_4\) રિવર્સ બાયસ રહે છે. લોડમાંથી પ્રવાહ વહે છે.
    \item \textbf{ઋણ ચક્ર (Negative Cycle):} ડાયોડ \(D_2\) અને \(D_4\) વાહક બને છે, જ્યારે \(D_1\) અને \(D_3\) રિવર્સ બાયસ રહે છે. લોડમાંથી પ્રવાહ \textbf{તે જ દિશામાં} વહે છે.
\end{itemize}

\begin{figure}[H]
\centering
\begin{tikzpicture}[scale=0.9]
    \draw (0,0) node[left] {A} -- (1.5,1.5) coordinate (top);
    \draw (0,0) -- (1.5,-1.5) coordinate (bottom);
    \draw (3,0) node[right] {B} -- (1.5,1.5);
    \draw (3,0) -- (1.5,-1.5);
    
    % Diodes
    \draw (0.8,0.8) node[rotate=-45] {\(\blacktriangleright\)}; % D1ish
    \draw (2.2,0.8) node[rotate=-135] {\(\blacktriangleright\)}; % D2ish
    \draw (0.8,-0.8) node[rotate=45] {\(\blacktriangleright\)}; % D4ish
    \draw (2.2,-0.8) node[rotate=135] {\(\blacktriangleright\)}; % D3ish
    
    % Labels (Simulation)
    \node at (0.6,1) {\(D_1\)};
    \node at (2.4,1) {\(D_2\)};
    \node at (0.6,-1) {\(D_4\)};
    \node at (2.4,-1) {\(D_3\)};
    
    % AC Input
    \draw (-2, 0) to[sinusoidal voltage source] (0,0);
    
    % Load - Taken from Top and Bottom
    \draw (1.5,1.5) -- (1.5, 2.5) -- (4, 2.5) to[R, l=\(R_L\)] (4, -2.5) -- (1.5, -2.5) -- (1.5, -1.5);
    
\end{tikzpicture}
\caption{બ્રીજ રેકટીફાયર સર્કિટ}
\end{figure}

\begin{figure}[H]
    \centering
    \begin{tikzpicture}[scale=0.6]
        % Input
        \draw[->] (0,0) -- (7,0) node[right] {t};
        \draw[->] (0,-1.5) -- (0,1.5) node[above] {Input AC};
        \draw[blue, thick] plot[domain=0:6.5, samples=100] (\x, {sin(\x*100)});
        
        % Output
        \begin{scope}[yshift=-4cm]
            \draw[->] (0,0) -- (7,0) node[right] {t};
            \draw[->] (0,0) -- (0,1.5) node[above] {Output DC};
            \draw[red, thick] plot[domain=0:6.5, samples=100] (\x, {abs(sin(\x*100))});
        \end{scope}
    \end{tikzpicture}
    \caption{ઇનપુટ અને આઉટપુટ વેવફોર્મ}
\end{figure}

\paragraph{મેમરી ટ્રીક:} \emph{4 Diodes Bridge, Full Wave Output (All humps positive).}

% ========================================
% QUESTION 5: Short Questions (14 marks)
% Demonstrates: 3 marks and 4 marks questions with Circuit Diagrams and Logic Gates
% ========================================

\section{પ્રશ્ન 5}

\noindent \textbf{પ્રશ્ન 5(a)} \\
\textbf{નીચેના પ્રશ્નોના જવાબ આપો. (3 માંથી કોઈપણ 2)}

\subsection{પ્રશ્ન 5(a)(1) [3 marks]}
\textbf{OR, AND અને NOT ગેટ સમજાવો.}

\subsubsection{ઉકેલ}
\paragraph{વ્યાખ્યાઓ:} \textbf{લોજિક ગેટ્સ (Logic Gates):}
\begin{enumerate}
    \item \textbf{OR ગેટ:} તેમાં બે અથવા વધુ ઇનપુટ અને એક આઉટપુટ હોય છે. જો \textbf{કોઈપણ} ઇનપુટ HIGH (1) હોય તો આઉટપુટ HIGH (1) મળે છે. બુલિયન સમીકરણ: \(Y = A + B\).
    \item \textbf{AND ગેટ:} તેમાં બે અથવા વધુ ઇનપુટ અને એક આઉટપુટ હોય છે. જો \textbf{બધા} ઇનપુટ HIGH (1) હોય તો જ આઉટપુટ HIGH (1) મળે છે. બુલિયન સમીકરણ: \(Y = A \cdot B\).
    \item \textbf{NOT ગેટ (Inverter):} તેમાં માત્ર એક ઇનપુટ અને એક આઉટપુટ હોય છે. આઉટપુટ ઇનપુટ કરતા ઉલટું હોય છે. બુલિયન સમીકરણ: \(Y = \bar{A}\).
\end{enumerate}

\begin{figure}[H]
    \centering
    \begin{tikzpicture}
        % OR Gate
        \node[or port] (or) at (0,0) {};
        \node[left] at (or.in 1) {A};
        \node[left] at (or.in 2) {B};
        \node[right] at (or.out) {Y=A+B};
        \node[below] at (0,-0.8) {OR Gate};

        % AND Gate
        \node[and port] (and) at (4,0) {};
        \node[left] at (and.in 1) {A};
        \node[left] at (and.in 2) {B};
        \node[right] at (and.out) {\(Y=A \cdot B\)};
        \node[below] at (4,-0.8) {AND Gate};

        % NOT Gate
        \node[not port] (not) at (8,0) {};
        \node[left] at (not.in) {A};
        \node[right] at (not.out) {\(Y=\bar{A}\)};
        \node[below] at (8,-0.8) {NOT Gate};
    \end{tikzpicture}
    \caption{લોજિક ગેટ્સની સંજ્ઞાઓ}
\end{figure}

\paragraph{મેમરી ટ્રીક:} \emph{OR=Any, AND=All, NOT=Inverse.}

\subsection{પ્રશ્ન 5(a)(2) [3 marks]}
\textbf{n-પ્રકારનું અર્ધવાહક સમજાવો.}

\subsubsection{ઉકેલ}
\paragraph{સમજૂતી:} \textbf{n-પ્રકાર અર્ધવાહક (n-type Semiconductor):}
\begin{itemize}
    \item શુદ્ધ અર્ધવાહક (જેમ કે સિલિકોન, જર્મેનિયમ) માં પેન્ટાવેલેન્ટ અશુદ્ધિ (Group 15, જેમ કે ફોસ્ફરસ, આર્સેનિક, એન્ટિમોની) ઉમેરીને આ અશુદ્ધ અર્ધવાહક બનાવવામાં આવે છે.
    \item અશુદ્ધિ પરમાણુમાં 5 વેલેન્સ ઇલેક્ટ્રોન હોય છે. 4 પાડોશી Si અણુઓ સાથે સહસંયોજક બંધ બનાવે છે, અને \textbf{5મો ઇલેક્ટ્રોન મુક્ત રહે છે}.
    \item અહીં, ઇલેક્ટ્રોન \textbf{બહુમતી વિદ્યુતભાર વાહકો (majority carriers)} છે અને હોલ લઘુમતી વિદ્યુતભાર વાહકો છે.
    \item તે વિદ્યુતની દ્રષ્ટિએ તટસ્થ છે.
\end{itemize}

\begin{figure}[H]
    \centering
    \begin{tikzpicture}[scale=0.8]
        % Atom Grid
        \foreach \x in {0,2,4}
            \foreach \y in {0,2,4} {
                \draw (\x,\y) circle (0.4);
            }
        
        % Labels
        \node at (0,0) {Si}; \node at (2,0) {Si}; \node at (4,0) {Si};
        \node at (0,2) {Si}; \node at (2,2) {As}; \node at (4,2) {Si}; % Center is Impurity
        \node at (0,4) {Si}; \node at (2,4) {Si}; \node at (4,4) {Si};
        
        % Bonds
        \draw (0.4,0) -- (1.6,0); \draw (2.4,0) -- (3.6,0);
        \draw (0.4,2) -- (1.6,2); \draw (2.4,2) -- (3.6,2);
        \draw (0.4,4) -- (1.6,4); \draw (2.4,4) -- (3.6,4);
        
        \draw (0,0.4) -- (0,1.6); \draw (2,0.4) -- (2,1.6); \draw (4,0.4) -- (4,1.6);
        \draw (0,2.4) -- (0,3.6); \draw (2,2.4) -- (2,3.6); \draw (4,2.4) -- (4,3.6);
        
        % Electrons (Dots on bonds)
        \foreach \x in {0.8, 1.2, 2.8, 3.2}
            \foreach \y in {0,2,4} \fill (\x,\y) circle (2pt);
            
        \foreach \x in {0,2,4}
            \foreach \y in {0.8, 1.2, 2.8, 3.2} \fill (\x,\y) circle (2pt);
            
        % Free Electron
        \fill[red] (2.8, 2.8) circle (3pt) node[right, above] {Free Electron ($e^-$)};
        \node[below] at (2, -0.5) {n-type Silicon Lattice};
    \end{tikzpicture}
    \caption{n-પ્રકાર અર્ધવાહક (પેન્ટાવેલેન્ટ ડોપિંગ)}
\end{figure}

\paragraph{મેમરી ટ્રીક:} \emph{n-type = Negative (electron) majority, Pentavalent impurity.}

\subsection{પ્રશ્ન 5(a)(3) [3 marks]}
\textbf{પ્રકાશ હવાના માધ્યમમાંથી કાચમાં પ્રવેશે છે. કાચનો વક્રીભવનાંક 1.56 છે. તો પ્રકાશનો કાચમાં વેગ શોધો. \(C=3 \times 10^8 m/s\).}

\subsubsection{ઉકેલ}
\textbf{આપેલ માહિતી:}
\begin{itemize}
    \item કાચનો વક્રીભવનાંક (\(n\)) = 1.56
    \item શૂન્યાવકાશમાં પ્રકાશનો વેગ (\(c\)) = \(3 \times 10^8 \, m/s\)
\end{itemize}

\textbf{સૂત્ર:}
વક્રીભવનાંક (\(n\)) એ શૂન્યાવકાશમાં પ્રકાશના વેગ (\(c\)) અને માધ્યમમાં પ્રકાશના વેગ (\(v\)) નો ગુણોત્તર છે.
\[ n = \frac{c}{v} \]

\textbf{ગણતરી:}
કાચમાં વેગ શોધવા માટે સૂત્રને ફરીથી ગોઠવતા:
\[ v = \frac{c}{n} \]
કિંમતો મૂકતા:
\[ v = \frac{3 \times 10^8}{1.56} \]

\[ v \approx 1.923 \times 10^8 \, m/s \]
આ સૂચવે છે કે પ્રકાશ શૂન્યાવકાશ કરતાં કાચમાં આશરે 1.56 ગણો ધીમો પ્રવાસ કરે છે.

\paragraph{જવાબ:} કાચમાં પ્રકાશનો વેગ આશરે \(\mathbf{1.92 \times 10^8 \, m/s}\) છે.

\paragraph{નિષ્કર્ષ:}
વક્રીભવનાંક હંમેશા 1 કરતા મોટો હોય છે, જે દર્શાવે છે કે પ્રકાશ શૂન્યાવકાશ કરતા માધ્યમમાં ધીમો ગતિ કરે છે. હીરા જેવા ઘટ્ટ માધ્યમો માટે આ કિંમત વધુ મોટી હોય છે. આપણે જોઈએ છીએ કે વેગમાં ઘટાડો નોંધપાત્ર છે.




\paragraph{મેમરી ટ્રીક:} \emph{v = c / n (Velocity decreases in denser medium).}

\noindent \textbf{પ્રશ્ન 5(b)} \\
\textbf{નીચેના પ્રશ્નોના જવાબ આપો. (3 માંથી કોઈપણ 2)}

\subsection{પ્રશ્ન 5(b)(1) [4 marks]}
\textbf{p-n જંકશન ડાયોડની ફોરવર્ડ બાયસ લાક્ષણિકતા સમજાવો.}

\subsubsection{ઉકેલ}
\textbf{ફોરવર્ડ બાયસ લાક્ષણિકતાઓ:}
\begin{itemize}
    \item જ્યારે બેટરીના \textbf{ધન છેડા}ને ડાયોડના \textbf{p-વિભાગ} સાથે અને \textbf{ઋણ છેડા}ને \textbf{n-વિભાગ} સાથે જોડવામાં આવે ત્યારે તે ફોરવર્ડ બાયસમાં છે તેમ કહેવાય.
    \paragraph{સમજૂતી:}
    \item \textbf{Si અને Ge એ સેમિકન્ડક્ટર છે:} સિલિકોન (Si) અને જર્મેનિયમ (Ge) એ સેમિકન્ડક્ટર સામગ્રી છે.
    \item \textbf{પોટેન્શિયલ બેરિયર:} p-n જંકશનમાં ડેપ્લેશન લેયરને પાર કરવા માટે જરૂરી લઘુત્તમ વોલ્ટેજ. Si માટે \(0.7V\), Ge માટે \(0.3V\). આ ઊર્જા બેન્ડ ગેપ પર આધારિત છે.
    \item ડેપ્લેશન સ્તરની પહોળાઈ \textbf{ઘટે} છે અને પોટેન્શિયલ બેરિયર ઘટે છે, જેનાથી મેજોરિટી કેરિયર્સ સરળતાથી જંકશન પાર કરી શકે છે.
    \item મિલી-એમ્પિયર (\(mA\)) ક્રમનો મોટો પ્રવાહ વહે છે.
    \item થ્રેસોલ્ડ વોલ્ટેજ (Si માટે \(0.7V\), Ge માટે \(0.3V\)) વટાવ્યા પછી વોલ્ટેજ સાથે પ્રવાહ ઘાતાંકીય રીતે વધે છે. આ વોલ્ટેજને \textbf{ની-વોલ્ટેજ (Knee Voltage)} કહે છે.
\end{itemize}

\begin{figure}[H]
    \centering
    \begin{tikzpicture}[scale=0.8]
        % Circuit
        \draw (0,0) to[D, l=Diode] (3,0);
        \draw (3,0) -- (3,-2) -- (0,-2) to[battery1, l=Battery] (0,0);
        \node at (1.5, -2.5) {Forward Bias Connection};
        
        % Graph
        \begin{scope}[xshift=5cm, yshift=-1.5cm]
            \draw[->] (0,0) -- (3,0) node[right] {\(V_F\) (Volts)};
            \draw[->] (0,0) -- (0,3) node[above] {\(I_F\) (mA)};
            \draw[blue, thick] plot[domain=0:2.5, samples=100] (\x, {0.1*exp(2*\x)-0.1});
            \node at (2,2) {Exponential Rise};
            \node at (1.5,-0.5) {VI Characteristic};
        \end{scope}
    \end{tikzpicture}
    \caption{ફોરવર્ડ બાયસ સર્કિટ અને V-I લાક્ષણિકતા}
\end{figure}

\paragraph{મેમરી ટ્રીક:} \emph{Forward = P to Plus, Current Flows, Barrier Lowers.}

\subsection{પ્રશ્ન 5(b)(2) [4 marks]}
\textbf{ઝેનર ડાયોડનું વોલ્ટેજ રેગ્યુલેટર તરીકેનું કાર્ય સમજાવો.}

\subsubsection{ઉકેલ}
\paragraph{કાર્ય:} \textbf{ઝેનર ડાયોડ વોલ્ટેજ રેગ્યુલેટર તરીકે:}

\begin{itemize}
    \item ઝેનર ડાયોડ \textbf{રિવર્સ બ્રેકડાઉન વિસ્તાર}માં કાર્ય કરવા માટે બનાવેલ છે.
    \item આ વિસ્તારમાં, ઝેનર ડાયોડમાંથી વહેતો પ્રવાહ ($I_Z$) બદલાય તો પણ તેની બે છેડા વચ્ચેનો વોલ્ટેજ ($V_Z$) \textbf{અચળ} રહે છે.
    \item તેને લોડ અવરોધ ($R_L$) સાથે \textbf{સમાંતર}માં અને રિવર્સ બાયસમાં જોડવામાં આવે છે.
    \item ઇનપુટ વોલ્ટેજ ($V_{in}$) માં થતી વધઘટને શોષવા માટે શ્રેણી અવરોધ ($R_S$) જોડવામાં આવે છે.
    \item જો \(V_{in}\) વધે, તો ઝેનરમાંથી પ્રવાહ વધે છે, \(R_S\) પર વોલ્ટેજ ડ્રોપ વધે છે, પરંતુ લોડ પરનો વોલ્ટેજ (\(V_Z\)) અચળ રહે છે.
\end{itemize}

\begin{figure}[H]
    \centering
    \begin{tikzpicture}
        \draw (0,2) to[V, l=\(V_{in}\) (Unregulated)] (0,0);
        \draw (0,2) to[R, l=\(R_S\)] (3,2);
        \draw (3,2) to[zD, l=\(V_Z\)] (3,0);
        \draw (3,2) -- (6,2) to[R, l=\(R_L\) (Load)] (6,0) -- (0,0);
        
        \node at (6,2.3) {\(V_{out} = V_Z\) (Constant)};
        \draw[->] (1.5, 2.2) -- (2.5, 2.2) node[midway, above] {\(I\)};
        \draw[->] (3.2, 1.5) -- (3.2, 0.5) node[midway, right] {\(I_Z\)};
        \draw[->] (5.8, 1.5) -- (5.8, 0.5) node[midway, left] {\(I_L\)};
        \node at (4.5, 2.5) {\(I = I_Z + I_L\)};
    \end{tikzpicture}
    \caption{ઝેનર ડાયોડ વોલ્ટેજ રેગ્યુલેટર સર્કિટ}
\end{figure}

\paragraph{મેમરી ટ્રીક:} \emph{Zener in Reverse Parallel, Constant Voltage.}

\subsection{પ્રશ્ન 5(b)(3) [4 marks]}
\textbf{એક ઓપ્ટીકલ ફાઈબરના કોર અને ક્લેડીંગ માટેના વક્રીભવનાંકના મુલ્યો અનુક્રમે 1.48 અને 1.45 છે. ઓપ્ટીકલ ફાઈબર માટે ન્યુમેરિકલ એપર્ચર અને એકસેપ્ટન્સ કોણની ગણતરી કરો.}

\subsubsection{ઉકેલ}
\textbf{આપેલ માહિતી:}
\begin{itemize}
    \item કોરનો વક્રીભવનાંક ($n_1$) = 1.48
    \item ક્લેડીંગનો વક્રીભવનાંક ($n_2$) = 1.45
\end{itemize}

\textbf{સૂત્રો:}
\begin{enumerate}
    \item \textbf{ન્યુમેરિકલ એપર્ચર (NA):}
    \[ NA = \sqrt{n_1^2 - n_2^2} \]
    \item \textbf{એકસેપ્ટન્સ કોણ (\(\theta_a\)):}
    \[ \theta_a = \sin^{-1}(NA) \]
\end{enumerate}

\textbf{ગણતરી:}
\textbf{1. ન્યુમેરિકલ એપર્ચર (NA):}
આ ફાઈબરની પ્રકાશ એકત્ર કરવાની ક્ષમતા દર્શાવે છે.
\[ NA = \sqrt{(1.48)^2 - (1.45)^2} \]
\[ NA = \sqrt{2.1904 - 2.1025} \]
\[ NA = \sqrt{0.0879} \]
\[ NA \approx 0.2965 \]

\textbf{2. એકસેપ્ટન્સ કોણ (\(\theta_a\)):}
સૌથી મોટો આપાતકોણ કે જેના પર પ્રકાશ કોરમાં પ્રવેશી શકે છે.
\[ \theta_a = \sin^{-1}(NA) \]
\[ \theta_a = \sin^{-1}(0.2965) \]
\[ \theta_a \approx 17.25^\circ \]

\paragraph{જવાબ:}
\begin{itemize}
    \item ન્યુમેરિકલ એપર્ચર (NA) = \textbf{0.2965}
    \item એકસેપ્ટન્સ કોણ (\(\theta_a\)) = \(\mathbf{17.25^\circ}\)
\end{itemize}

\paragraph{મહત્વ:}
0.2965 નું ન્યુમેરિકલ એપર્ચર ફાઈબરની પ્રકાશ ગ્રહણ કરવાની ક્ષમતા દર્શાવે છે. ઊંચું NA એટલે ફાઈબર વધારે ખૂણેથી આવતા પ્રકાશને અંદર દાખલ કરી શકે છે. લગભગ \(17^\circ\) નો એકસેપ્ટન્સ કોણ એ શંકુ દર્શાવે છે જેની અંદર પ્રકાશ કિરણો દાખલ થવાથી પૂર્ણ આંતરિક પરાવર્તનની શરતનું પાલન થાય છે. આ ફાઈબર ઓછી શક્તિવાળા ટૂંકા અંતરના સંચાર માટે યોગ્ય છે. લાંબા અંતરના સંચાર માટે, સામાન્ય રીતે ઓછા NA અને નાના એકસેપ્ટન્સ કોણવાળા સિંગલ-મોડ ફાઈબરનો ઉપયોગ થાય છે જેથી ડિસ્પરઝન (ફેલાવો) ઘટાડી શકાય.

\paragraph{મેમરી ટ્રીક:} \emph{NA = Root(n1 sq - n2 sq).}

\end{document}




