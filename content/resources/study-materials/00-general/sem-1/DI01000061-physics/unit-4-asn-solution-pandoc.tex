\documentclass[10pt,a4paper]{article}

% content/resources/templates/preamble.tex
\usepackage[margin=0.6in]{geometry}
\author{Milav Dabgar}
\usepackage{amsmath,amssymb,amsthm}
\usepackage{booktabs}
\usepackage{multirow}
\usepackage{xcolor}
\usepackage{tcolorbox}
\tcbuselibrary{breakable,skins}
\usepackage[colorlinks=true,linkcolor=blue]{hyperref}
\usepackage{titlesec}
\usepackage{enumitem}
\usepackage{tikz}
\usepackage{pgfplots}
\usepackage{circuitikz}
\usepackage[version=4]{mhchem}
\usepackage{longtable}
\usepackage{array}
\usepackage{float}
\usepackage{caption}
\usepackage{listings}

\lstset{
  basicstyle=\small\ttfamily,
  breaklines=true,
  breakatwhitespace=false,
  postbreak=\mbox{\textcolor{red}{$\hookrightarrow$}\space},
  float=false,
  numbers=left,
  numberstyle=\tiny\color{gray},
  numbersep=10pt,
  xleftmargin=2em,
  keywordstyle=\color{blue},
  commentstyle=\color{green!60!black},
  stringstyle=\color{purple},
  backgroundcolor=\color{gray!5},
  showstringspaces=false,
  tabsize=2,
  captionpos=b,
  keepspaces=true,
  columns=flexible
}

\pgfplotsset{compat=1.18}
\usetikzlibrary{shapes,arrows,positioning,calc,patterns,decorations.pathmorphing,decorations.markings,arrows.meta}

% Color scheme
\definecolor{headcolor}{RGB}{0,102,204}
\definecolor{keycolor}{RGB}{220,20,60}
\definecolor{solutioncolor}{RGB}{34,139,34}
\definecolor{mnemoniccolor}{RGB}{148,0,211}
\definecolor{codecolor}{RGB}{0,0,100}

% Spacing
\setlength{\parskip}{3pt}
\setlist[itemize]{nosep}
\setlist[enumerate]{nosep}

% Title formatting
\titleformat{\section}{\Large\bfseries\color{headcolor}}{\thesection}{1em}{}
\titleformat{\subsection}{\large\bfseries\color{headcolor}}{\thesubsection}{1em}{}

% Pandoc tightlist compatibility
\providecommand{\tightlist}{%
  \setlength{\itemsep}{0pt}\setlength{\parskip}{0pt}}

% Pandoc longtable compatibility
\newcounter{none}
\def\thenone{}


% content/resources/templates/english-boxes.tex
% This file is currently empty - it exists to maintain consistency with the import structure.
% Add custom environments here if needed in the future.


\begin{document}

\begin{center}
{\Huge\bfseries\color{headcolor} Subject Name Solutions}\\[5pt]
{\LARGE unit -- Study Material}\\[3pt]
{\large Semester 1 Study Material}\\[3pt]
{\normalsize\textit{Detailed Solutions and Explanations}}
\end{center}

\vspace{10pt}

\section*{Unit-4. LASER and Fiber Optics -
Solutions}\label{unit-4.-laser-and-fiber-optics---solutions}

\subsection*{Part A: Short Answers (1 or 2
marks)}\label{part-a-short-answers-1-or-2-marks}

\subsubsection{(1) Write Snell's law}\label{write-snells-law}

\begin{solutionbox}
Snell's law states that when light passes from one
medium to another, the ratio of sine of angle of incidence to the sine
of angle of refraction is constant and equal to the ratio of refractive
indices of the two media.

\textbf{Mathematical form:}

\begin{verbatim}
n_{1} sin \theta_{1} = n_{2} sin \theta_{2}
\end{verbatim}

or

\begin{verbatim}
sin i / sin

r = n_{2} / n_{1} = constant

\end{verbatim}

Where:

\begin{itemize}
\tightlist
\item
  n_{1} = refractive index of first medium
\item
  n_{2} = refractive index of second medium
\item
  \theta_{1} (or i) = angle of incidence
\item
  \theta_{2} (or r) = angle of refraction
\end{itemize}

\end{solutionbox}
\begin{center}\rule{0.5\linewidth}{0.5pt}\end{center}

\subsubsection{(2) Write full form of
LASER}\label{write-full-form-of-laser}

\begin{solutionbox}
LASER stands for \textbf{Light Amplification by
Stimulated Emission of Radiation}.

\begin{itemize}
\tightlist
\item
  \textbf{L} - Light
\item
  \textbf{A} - Amplification
\item
  \textbf{S} - Stimulated
\item
  \textbf{E} - Emission
\item
  \textbf{R} - Radiation
\end{itemize}

\end{solutionbox}
\begin{center}\rule{0.5\linewidth}{0.5pt}\end{center}

\subsubsection{(3) What is monochromatic light and polychromatic
light?}\label{what-is-monochromatic-light-and-polychromatic-light}

\begin{solutionbox}

\textbf{Monochromatic Light:} Light consisting of only one wavelength or
single color. It has a single frequency and energy. Example: Laser
light, sodium vapor lamp light.

\textbf{Polychromatic Light:} Light consisting of multiple wavelengths
or many colors. It contains waves of different frequencies. Example:
White light, sunlight.

\end{solutionbox}
\begin{center}\rule{0.5\linewidth}{0.5pt}\end{center}

\subsubsection{(4) Write properties of laser
light}\label{write-properties-of-laser-light}

\begin{solutionbox}
Properties of laser light are:

\begin{enumerate}
\tightlist
\item
  \textbf{Monochromaticity} - Single wavelength/color (single frequency
  light)
\item
  \textbf{Coherence} - All waves are in phase (temporal and spatial
  coherence)
\item
  \textbf{Directionality} - Highly directional beam, propagates in one
  specific direction with minimal divergence
\item
  \textbf{High Intensity} - Concentrated energy in a narrow beam, very
  powerful and bright
\item
  \textbf{Polarization} - Waves vibrate in a single plane
\item
  \textbf{Low Divergence} - Beam remains parallel even after traveling
  long distances
\item
  \textbf{Focused Beam} - Can be focused to a very small area,
  depositing energy in a tiny spot
\item
  \textbf{High Speed} - Travels at tremendous velocity (speed of light)
\end{enumerate}

\end{solutionbox}
\begin{center}\rule{0.5\linewidth}{0.5pt}\end{center}

\subsubsection{(5) Define: Absolute refractive index, Critical
angle}\label{define-absolute-refractive-index-critical-angle}

\begin{solutionbox}

\textbf{Absolute Refractive Index:} The ratio of speed of light in
vacuum (or air) to the speed of light in that medium.

\begin{verbatim}
n = c / v
\end{verbatim}

Where:

\begin{itemize}
\tightlist
\item
  n = absolute refractive index
\item
  c = speed of light in vacuum (3 \times 10^{8} m/s)
\item
  v = speed of light in the medium
\end{itemize}

\textbf{Critical Angle:} The angle of incidence in the denser medium for
which the angle of refraction in the rarer medium becomes 90^\circ. Beyond
this angle, total internal reflection occurs.

\begin{verbatim}
sin \thetac = n_{2} / n_{1}  (where n_{1} > n_{2})
\end{verbatim}

\end{solutionbox}
\begin{center}\rule{0.5\linewidth}{0.5pt}\end{center}

\subsection*{Part B: Detailed Answers (2 or 3
marks)}\label{part-b-detailed-answers-2-or-3-marks}

\subsubsection{(1) Explain refraction of light with figure and
examples}\label{explain-refraction-of-light-with-figure-and-examples}

\begin{solutionbox}

\textbf{Refraction of Light:} The phenomenon of change in direction
(bending) of light when it passes from one transparent medium to another
is called refraction. This occurs because the speed of light varies in
different media.

\textbf{Diagram:}

\begin{figure}
\centering
\pandocbounded{\includegraphics[keepaspectratio,alt={Refraction through Glass Slab}]{unit-4-book-scan_artifacts/image_000000_14970c3e348b2ecca5b78372097cf29812965c6414b9609dbff8a1e32ddccda5.png}}
\caption{Refraction through Glass Slab}
\end{figure}

\textbf{ASCII Representation:}

\begin{verbatim}
    Air (Rarer Medium - n_{1})
         |
      PQ |  \leftarrow Normal N_{1}
    i ↗  |  
   P/___Q|______________ AB (Glass Surface)
     ↘  |
   r  ↘ |  Glass (Denser Medium)
       ↘|  
        R  \leftarrow Refracted Ray QR
        |
        |  \leftarrow Normal N_{2}
     e ↗|  
      / |______________ CD (Glass Surface)
     /  |
    S   |  Air (Rarer Medium)
        ↓  Emergent Ray RS
\end{verbatim}

\textbf{Explanation:}

\begin{itemize}
\tightlist
\item
  When light travels from rarer to denser medium (air to glass), it
  bends towards the normal (\theta_{2} \textless{} \theta_{1})
\item
  When light travels from denser to rarer medium (glass to air), it
  bends away from the normal (\theta_{2} \textgreater{} \theta_{1})
\item
  The bending occurs because light travels at different speeds in
  different media
\end{itemize}

\textbf{Examples:}

\begin{enumerate}
\tightlist
\item
  A pencil partially immersed in water appears bent
\item
  A coin at the bottom of a water-filled vessel appears raised
\item
  Twinkling of stars due to atmospheric refraction
\item
  Formation of mirages in deserts
\item
  Lenses in spectacles correct vision through refraction
\end{enumerate}

\end{solutionbox}
\begin{center}\rule{0.5\linewidth}{0.5pt}\end{center}

\subsubsection{(2) Explain refractive index. What is absolute refractive
index?}\label{explain-refractive-index.-what-is-absolute-refractive-index}

\begin{solutionbox}

\textbf{Refractive Index:} It is a measure of how much light slows down
when it passes through a medium. It determines the bending of light rays
at the interface between two media.

\textbf{Types:}

\begin{enumerate}
\item
  \textbf{Relative Refractive Index:} The ratio of speed of light in
  first medium to speed of light in second medium.

\begin{verbatim}
n_{2}_{1} = v_{1} / v_{2} = n_{2} / n_{1}
\end{verbatim}
\item
  \textbf{Absolute Refractive Index:} The ratio of speed of light in
  vacuum (or air) to the speed of light in that medium.

\begin{verbatim}
n = c / v
\end{verbatim}

  Where:

  \begin{itemize}
  \tightlist
  \item
    c = 3 \times 10^{8} m/s (speed of light in vacuum)
  \item
    v = speed of light in the medium
  \end{itemize}
\end{enumerate}

\textbf{Properties:}

\begin{itemize}
\tightlist
\item
  Always greater than 1 for any medium (except vacuum where n = 1)
\item
  Higher refractive index means light travels slower in that medium
\item
  Depends on wavelength of light (causes dispersion)
\end{itemize}

\textbf{Refractive Indices of Common Materials (at 20^\circC):}

\begin{longtable}[]{@{}ll@{}}
\toprule\noalign{}
Medium & Refractive Index \\
\midrule\noalign{}
\endhead
\bottomrule\noalign{}
\endlastfoot
Vacuum/Air & 1.000 \\
Ice (0^\circC) & 1.31 \\
Water & 1.333 \\
Ethyl Alcohol & 1.362 \\
Glycerine & 1.473 \\
Glass/Benzene & 1.501 \\
Quartz & 1.554 \\
Polystyrene & 1.595 \\
Diamond & 2.417 \\
\end{longtable}

\textbf{Key Points:}

\begin{itemize}
\tightlist
\item
  Optically denser medium has higher refractive index
\item
  Optically rarer medium has lower refractive index
\item
  Higher n means light travels slower in that medium
\end{itemize}

\end{solutionbox}
\begin{center}\rule{0.5\linewidth}{0.5pt}\end{center}

\subsubsection{(3) Explain total internal reflection in detail with
figure}\label{explain-total-internal-reflection-in-detail-with-figure}

\begin{solutionbox}

\textbf{Total Internal Reflection (TIR):} When light traveling from a
denser medium to a rarer medium is completely reflected back into the
denser medium at the interface, the phenomenon is called total internal
reflection.

\textbf{Conditions for TIR:}

\begin{enumerate}
\tightlist
\item
  Light must travel from denser to rarer medium (n_{1} \textgreater{} n_{2})
\item
  Angle of incidence must be greater than critical angle (i
  \textgreater{} \thetac)
\end{enumerate}

\textbf{Diagram from Textbook:}

\begin{figure}
\centering
\pandocbounded{\includegraphics[keepaspectratio,alt={Total Internal Reflection}]{unit-4-book-scan_artifacts/image_000001_76b4f6c445184617690172eaf4b98d3d4a0759f39efc5cbf3f63e2f62f8a388c.png}}
\caption{Total Internal Reflection}
\end{figure}

\emph{Figure: Total internal reflection in water showing light rays at
different angles - some refracted, one at critical angle, and some
totally reflected}

\begin{figure}
\centering
\pandocbounded{\includegraphics[keepaspectratio,alt={Critical Angle Illustration}]{unit-4-book-scan_artifacts/image_000002_0d572298feb37d0d4db17560d6aa4455b66b0c76c994bc779ed5e306b9f3be95.png}}
\caption{Critical Angle Illustration}
\end{figure}

\emph{Figure: Derivation of critical angle formula using Snell's law}

\textbf{ASCII Representation:}

\begin{verbatim}
        Denser Medium (Water - n_{1})  |    Rarer Medium (Air - n_{2})
                                    |
    A        B         C            |
     \       |        /             |
      \      |       /              |
    \theta_{1} \     |\thetac    / \theta_{3}            |
        \    |     /                |
         \   |    /                 |    
          \  |   /                  |
___________\◄|__/___________________| 90^\circ (Critical Ray)
            \|_/                    |
             X                      |
            /|\                     |
           / | \                    |
          /  |  \                   |
         /   |   \                  |
       r_{1}   TIR  r_{3}                 |
     (Reflected)  (Totally Reflected)|
                                    |
\end{verbatim}

\textbf{Explanation:}

\begin{itemize}
\tightlist
\item
  Ray A: i \textless{} \thetac \rightarrow Refraction occurs (bends away from normal)
\item
Ray B:

i = \thetac \rightarrow Refracted ray grazes along the surface (r = 90^\circ)

\item
  Ray C: i \textgreater{} \thetac \rightarrow Total internal reflection occurs
\end{itemize}

\textbf{Critical Angle Formula:}

\begin{verbatim}
sin \thetac = n_{2} / n_{1}  (where n_{1} > n_{2})

or

\thetac = sin^{-}^{1}(n_{2} / n_{1})
\end{verbatim}

\textbf{Examples of Critical Angles:}

\begin{enumerate}
\tightlist
\item
  \textbf{Glass-Air Interface:}

  \begin{itemize}
  \tightlist
  \item
    n_{1} (glass) = 1.50, n_{2} (air) = 1.00
  \item
    \thetac = sin^{-}^{1}(1.00/1.50) = sin^{-}^{1}(0.667) = \textbf{41.1^\circ}
  \end{itemize}
\item
  \textbf{Water-Air Interface:}

  \begin{itemize}
  \tightlist
  \item
    n_{1} (water) = 1.33, n_{2} (air) = 1.00
  \item
    \thetac = sin^{-}^{1}(1.00/1.33) = sin^{-}^{1}(0.752) = \textbf{48.75^\circ}
  \end{itemize}
\item
  \textbf{Diamond-Air Interface:}

  \begin{itemize}
  \tightlist
  \item
    n_{1} (diamond) = 2.42, n_{2} (air) = 1.00
  \item
    \thetac = sin^{-}^{1}(1.00/2.42) = sin^{-}^{1}(0.413) = \textbf{24.4^\circ}
  \item
    This small critical angle explains diamond's brilliant sparkle
  \end{itemize}
\end{enumerate}

\textbf{Applications:}

\begin{enumerate}
\tightlist
\item
  Optical fibers for data transmission
\item
  Prisms in periscopes and binoculars
\item
  Brilliance of diamonds (due to very small critical angle)
\item
  Mirage formation in deserts
\item
  Endoscopy in medical field
\item
  Sparkling of crystal glass
\end{enumerate}

\end{solutionbox}
\begin{center}\rule{0.5\linewidth}{0.5pt}\end{center}

\subsubsection{(4) Give difference between common light and laser
light}\label{give-difference-between-common-light-and-laser-light}

\begin{solutionbox}

\begin{longtable}[]{@{}
  >{\raggedright\arraybackslash}p{(\linewidth - 4\tabcolsep) * \real{0.2778}}
  >{\raggedright\arraybackslash}p{(\linewidth - 4\tabcolsep) * \real{0.3611}}
  >{\raggedright\arraybackslash}p{(\linewidth - 4\tabcolsep) * \real{0.3611}}@{}}
\toprule\noalign{}
\begin{minipage}[b]{\linewidth}\raggedright
Property
\end{minipage} & \begin{minipage}[b]{\linewidth}\raggedright
Common Light
\end{minipage} & \begin{minipage}[b]{\linewidth}\raggedright
Laser Light
\end{minipage} \\
\midrule\noalign{}
\endhead
\bottomrule\noalign{}
\endlastfoot
\textbf{Monochromaticity} & Polychromatic (multiple wavelengths) &
Monochromatic (single wavelength) \\
\textbf{Coherence} & Non-coherent (waves out of phase) & Highly coherent
(waves in phase) \\
\textbf{Directionality} & Spreads in all directions & Highly
directional, narrow beam \\
\textbf{Divergence} & High divergence & Very low divergence (parallel
beam) \\
\textbf{Intensity} & Low intensity & Very high intensity \\
\textbf{Source} & Spontaneous emission & Stimulated emission \\
\textbf{Polarization} & Unpolarized or randomly polarized & Polarized \\
\textbf{Energy} & Low energy density & High energy density \\
\textbf{Phase relationship} & Random phase & Constant phase
relationship \\
\textbf{Example} & Bulb, tube light, sunlight & He-Ne laser, CO_{2} laser,
ruby laser \\
\end{longtable}

\textbf{Visual Comparison from Textbook:}

\begin{figure}
\centering
\pandocbounded{\includegraphics[keepaspectratio,alt={Light Comparison - Normal Bulb}]{unit-4-book-scan_artifacts/image_000004_7e3c7d9559d59b85da8324d7117e3a809a4c9517ec7f5432d5135b50d85d4869.png}}
\caption{Light Comparison - Normal Bulb}
\end{figure}

\begin{figure}
\centering
\pandocbounded{\includegraphics[keepaspectratio,alt={Light Comparison - Sodium Lamp vs Laser}]{unit-4-book-scan_artifacts/image_000005_aed4eadc15c5221067d197c2bfca305d7f9dfe9ccf59cf00ff2ecb8b1be046ed.png}}
\caption{Light Comparison - Sodium Lamp vs Laser}
\end{figure}

\textbf{ASCII Representation:}

\begin{verbatim}
Figure 4.4 - Normal Bulb Light:
    ╱─────╲  (Multiple wavelengths)
   ╱   \lambda_{1}  ╲ (spreads in all directions)
  │  \lambda_{2} \lambda_{3} │ 
   ╲  \lambda_{4}   ╱
    ╲─────╱

Figure 4.5 - Sodium Lamp (Monochromatic):
    ╱─────╲  (Single wavelength \lambda)
   ╱   \lambda   ╲ (spreads in all directions)
  │    \lambda   │ 
   ╲   \lambda   ╱
    ╲─────╱

Figure 4.6 - Laser Light:
       ║║║  (Single wavelength \lambda)
       ║║║  (Parallel, unidirectional)
    ═══╬╬╬═══
       ║║║
       ║║║
\end{verbatim}

\end{solutionbox}
\begin{center}\rule{0.5\linewidth}{0.5pt}\end{center}

\subsubsection{(5) Write at least 6 applications of laser light in
various fields in
detail}\label{write-at-least-6-applications-of-laser-light-in-various-fields-in-detail}

\begin{solutionbox}

\textbf{Applications of LASER:}

\begin{enumerate}
\tightlist
\item
  \textbf{Medical Field:}

  \begin{itemize}
  \tightlist
  \item
    \textbf{Laser Surgery:} Precision cutting in eye surgery (LASIK),
    tumor removal
  \item
    \textbf{Laser Therapy:} Treatment of kidney stones, dental
    procedures
  \item
    \textbf{Endoscopy:} Internal examination using fiber optic cables
    with laser
  \item
    \textbf{Bloodless Surgery:} Cauterization of blood vessels during
    cutting
  \end{itemize}
\item
  \textbf{Communication:}

  \begin{itemize}
  \tightlist
  \item
    \textbf{Optical Fiber Communication:} High-speed internet, telephone
    networks
  \item
    \textbf{Satellite Communication:} Long-distance signal transmission
  \item
    \textbf{Data transfer:} Laser beams carry more information than
    radio waves
  \end{itemize}
\item
  \textbf{Industrial Applications:}

  \begin{itemize}
  \tightlist
  \item
    \textbf{Cutting and Welding:} Precision cutting of metals, fabrics,
    plastics
  \item
    \textbf{Drilling:} Making micro-holes in hard materials like
    diamonds
  \item
    \textbf{Surface Treatment:} Heat treatment, hardening of metals
  \item
    \textbf{3D Printing:} Laser sintering and stereolithography
  \end{itemize}
\item
  \textbf{Military and Defense:}

  \begin{itemize}
  \tightlist
  \item
    \textbf{Range Finders:} Measuring distance to targets accurately
  \item
    \textbf{Guided Missiles:} Laser-guided weapons systems
  \item
    \textbf{LIDAR:} Detection and ranging for surveillance
  \item
    \textbf{Anti-missile systems:} Destroying incoming projectiles
  \end{itemize}
\item
  \textbf{Scientific Research:}

  \begin{itemize}
  \tightlist
  \item
    \textbf{Spectroscopy:} Analyzing material composition
  \item
    \textbf{Holography:} Creating 3D images
  \item
    \textbf{Laser Cooling:} Cooling atoms to near absolute zero
  \item
    \textbf{Plasma Physics:} Creating high-temperature plasmas
  \end{itemize}
\item
  \textbf{Entertainment and Commercial:}

  \begin{itemize}
  \tightlist
  \item
    \textbf{Laser Light Shows:} Concerts, events, displays
  \item
    \textbf{Barcode Scanners:} Retail checkout systems
  \item
    \textbf{CD/DVD/Blu-ray Players:} Reading optical discs
  \item
    \textbf{Laser Printers:} High-quality printing
  \end{itemize}
\item
  \textbf{Measurement and Instrumentation:}

  \begin{itemize}
  \tightlist
  \item
    \textbf{Precision Measurement:} Interferometry for extremely
    accurate measurements
  \item
    \textbf{Surveying:} Land surveying and construction alignment
  \item
    \textbf{Speed Detection:} Traffic speed radars
  \end{itemize}
\end{enumerate}

\end{solutionbox}
\begin{center}\rule{0.5\linewidth}{0.5pt}\end{center}

\subsubsection{(6) Describe types of optical fiber in
detail}\label{describe-types-of-optical-fiber-in-detail}

\begin{solutionbox}

Optical fibers are classified based on two criteria:

\textbf{A. Based on Mode of Propagation:}

\textbf{1. Single Mode Fiber (SMF):}

\begin{itemize}
\tightlist
\item
  \textbf{Core diameter:} Very small (8-10 \mum)
\item
  \textbf{Cladding diameter:} 125 \mum
\item
  \textbf{Mode:} Only one mode (fundamental mode) propagates
\item
  \textbf{Transmission:} Light travels straight along the axis
\item
  \textbf{Dispersion:} Very low intermodal dispersion
\item
  \textbf{Bandwidth:} Very high (\textgreater100 GHz)
\item
  \textbf{Distance:} Long distance (\textgreater50 km)
\item
  \textbf{Cost:} Expensive
\item
  \textbf{Applications:} Long-haul telecommunications, cable TV,
  internet backbone
\end{itemize}

\begin{verbatim}
Single Mode Fiber:
    ═══════════════════════►
    ═══════════════════════► (Straight path)
    ═══════════════════════►
    |     Core (8-10\mum)    |
\end{verbatim}

\textbf{2. Multi Mode Fiber (MMF):}

\begin{itemize}
\tightlist
\item
  \textbf{Core diameter:} Large (50-62.5 \mum)
\item
  \textbf{Cladding diameter:} 125 \mum
\item
  \textbf{Mode:} Multiple modes propagate simultaneously
\item
  \textbf{Transmission:} Light takes multiple paths (zigzag)
\item
  \textbf{Dispersion:} High intermodal dispersion
\item
  \textbf{Bandwidth:} Lower than SMF (up to 1 GHz)
\item
  \textbf{Distance:} Short distance (\textless2 km)
\item
  \textbf{Cost:} Less expensive
\item
  \textbf{Applications:} LANs, short-distance communication, building
  networks
\end{itemize}

\begin{verbatim}
Multi Mode Fiber:
    ═══╱╲╱╲╱╲╱╲╱╲══► (Multiple paths)
    ═══════════════►
    ══╲╱╲╱╲╱╲╱═════►
    | Core (50-62.5\mum) |
\end{verbatim}

\textbf{B. Based on Refractive Index Profile:}

\textbf{1. Step Index Fiber:}

\begin{itemize}
\tightlist
\item
  \textbf{Profile:} Uniform refractive index in core, sharp boundary
  with cladding
\item
  \textbf{Core RI:} Constant (n_{1})
\item
  \textbf{Cladding RI:} Constant (n_{2}), where n_{1} \textgreater{} n_{2}
\item
  \textbf{Types:} Can be single mode or multimode
\item
  \textbf{Dispersion:} High in multimode version
\item
  \textbf{Applications:} Short-distance, low-cost applications
\end{itemize}

\begin{verbatim}
Refractive Index Profile:
    n_{1} |████████|
       |████████|
    n_{2} |        |________
       └────────┴────────► radius
          core  cladding
\end{verbatim}

\textbf{2. Graded Index Fiber:}

\begin{itemize}
\tightlist
\item
  \textbf{Profile:} Refractive index gradually decreases from center to
  edge of core
\item
  \textbf{Core RI:} Maximum at center, decreases parabolically toward
  edges
\item
  \textbf{Cladding RI:} Constant (n_{2})
\item
  \textbf{Types:} Always multimode
\item
  \textbf{Dispersion:} Lower than step index (paths equalized)
\item
  \textbf{Applications:} Medium-distance, higher bandwidth applications
\end{itemize}

\begin{figure}
\centering
\pandocbounded{\includegraphics[keepaspectratio,alt={Step Index and Graded Index Fibers}]{unit-4-book-scan_artifacts/image_000012_9b632dd9ddefd1f582541ac938fc7e815cebffd1ac9e9b1a2a928547b45068fc.png}}
\caption{Step Index and Graded Index Fibers}
\end{figure}

\emph{Figure: Comparison of refractive index profiles and light
propagation in step-index and graded-index optical fibers}

\begin{verbatim}
Step Index - Refractive Index Profile:
    n_{1} |████████|
       |████████|
    n_{2} |        |________
       └────────┴────────► radius
          core  cladding
    
    Light Path: ══╲╱╲╱╲╱══► (zigzag)

Graded Index - Refractive Index Profile:
    n_{1} |   ╱══╲   |
       |  ╱    ╲  |
    n_{2} | ╱      ╲ |________
       └─────────┴────────► radius
          core   cladding

    Light Path: ══╱══╲══╱══╲══► (sinusoidal)
\end{verbatim}

\textbf{Comparison Table:}

\begin{longtable}[]{@{}llll@{}}
\toprule\noalign{}
Parameter & Step Index (MM) & Graded Index (MM) & Single Mode \\
\midrule\noalign{}
\endhead
\bottomrule\noalign{}
\endlastfoot
Core Diameter & 50-200 \mum & 50-62.5 \mum & 8-10 \mum \\
Bandwidth & Low & Medium & Very High \\
Dispersion & High & Medium & Very Low \\
Distance & Short & Medium & Long \\
Cost & Low & Medium & High \\
\end{longtable}

\end{solutionbox}
\begin{center}\rule{0.5\linewidth}{0.5pt}\end{center}

\subsubsection{(7) Write applications of optical fiber in various fields
in
detail}\label{write-applications-of-optical-fiber-in-various-fields-in-detail}

\begin{solutionbox}

\begin{figure}
\centering
\pandocbounded{\includegraphics[keepaspectratio,alt={Optical Fiber Applications}]{unit-4-book-scan_artifacts/image_000003_2055878789c865cb3690ccad3b720baa96b9de11b64962866256a170986c3da0.png}}
\caption{Optical Fiber Applications}
\end{figure}

\emph{Figure: Optical fiber being used for data transmission based on
total internal reflection principle}

\textbf{Applications of Optical Fiber:}

\textbf{1. Telecommunications:}

\begin{itemize}
\tightlist
\item
  \textbf{Long-Distance Communication:} Intercontinental submarine
  cables
\item
  \textbf{Telephone Networks:} Voice transmission over fiber optic
  cables
\item
  \textbf{Internet Backbone:} High-speed internet infrastructure (FTTH -
  Fiber to the Home)
\item
  \textbf{Mobile Networks:} Backhaul connections between cell towers
\item
  \textbf{Advantages:} High bandwidth, low signal loss, immune to
  electromagnetic interference
\end{itemize}

\textbf{2. Medical Field:}

\begin{itemize}
\tightlist
\item
  \textbf{Endoscopy:} Internal examination of body organs (gastroscopy,
  colonoscopy)
\item
  \textbf{Laser Surgery:} Delivering laser beams precisely to target
  tissues
\item
  \textbf{Photodynamic Therapy:} Cancer treatment using light-activated
  drugs
\item
  \textbf{Medical Imaging:} Fiber optic microscopy and imaging systems
\item
  \textbf{Advantages:} Flexibility, minimal invasion, precise light
  delivery
\end{itemize}

\textbf{3. Industrial Applications:}

\begin{itemize}
\tightlist
\item
  \textbf{Sensors:} Temperature sensors, pressure sensors in harsh
  environments
\item
  \textbf{Lighting:} Fiber optic lighting for displays, decorations, and
  hazardous areas
\item
  \textbf{Inspection:} Inspecting hard-to-reach areas in machinery and
  structures
\item
  \textbf{Process Control:} Monitoring industrial processes in real-time
\item
  \textbf{Advantages:} Immunity to electrical noise, safe in explosive
  environments
\end{itemize}

\textbf{4. Military and Defense:}

\begin{itemize}
\tightlist
\item
  \textbf{Secure Communication:} Difficult to tap, no electromagnetic
  emission
\item
  \textbf{Battlefield Networks:} Lightweight, reliable communication
  systems
\item
  \textbf{Navigation Systems:} Fiber optic gyroscopes for aircraft and
  missiles
\item
  \textbf{Sonar Systems:} Underwater detection and communication
\item
  \textbf{Advantages:} Secure, lightweight, resistant to jamming
\end{itemize}

\textbf{5. Cable Television (CATV):}

\begin{itemize}
\tightlist
\item
  \textbf{Broadcast:} Distributing TV signals to multiple subscribers
\item
  \textbf{Video on Demand:} High-quality video streaming services
\item
  \textbf{Interactive TV:} Two-way communication for interactive
  services
\item
  \textbf{Advantages:} High bandwidth, multiple channels, better picture
  quality
\end{itemize}

\textbf{6. Computer Networking:}

\begin{itemize}
\tightlist
\item
  \textbf{Local Area Networks (LANs):} High-speed data transfer in
  offices and campuses
\item
  \textbf{Data Centers:} Interconnecting servers and storage systems
\item
  \textbf{Storage Networks:} SANs (Storage Area Networks)
\item
  \textbf{Advantages:} High data rates, long distances without repeaters
\end{itemize}

\textbf{7. Automotive Industry:}

\begin{itemize}
\tightlist
\item
  \textbf{Vehicle Networks:} Connecting various electronic systems in
  cars
\item
  \textbf{Safety Systems:} ABS, airbag deployment sensors
\item
  \textbf{Entertainment:} In-vehicle multimedia distribution
\item
  \textbf{Advantages:} Lightweight, immune to electromagnetic
  interference
\end{itemize}

\textbf{8. Aerospace:}

\begin{itemize}
\tightlist
\item
  \textbf{Aircraft Systems:} Flight control, avionics communication
\item
  \textbf{Spacecraft:} Data transmission in satellites
\item
  \textbf{Flight Data Recorders:} Black boxes using fiber optic
  technology
\item
  \textbf{Advantages:} Lightweight, reliable in extreme conditions
\end{itemize}

\textbf{9. Scientific Research:}

\begin{itemize}
\tightlist
\item
  \textbf{Spectroscopy:} Delivering light to and from spectrometers
\item
  \textbf{Laser Systems:} Beam delivery in laboratories
\item
  \textbf{Astronomy:} Connecting telescopes to analysis instruments
\item
  \textbf{Particle Physics:} Timing and triggering systems in detectors
\end{itemize}

\textbf{10. Oil and Gas Industry:}

\begin{itemize}
\tightlist
\item
  \textbf{Downhole Monitoring:} Temperature and pressure sensors in oil
  wells
\item
  \textbf{Pipeline Monitoring:} Detecting leaks and intrusions
\item
  \textbf{Seismic Sensing:} Distributed acoustic sensing for exploration
\item
  \textbf{Advantages:} Operates in high-temperature, high-pressure
  environments
\end{itemize}

\end{solutionbox}
\begin{center}\rule{0.5\linewidth}{0.5pt}\end{center}

\subsubsection{(8) Explain construction of optical fiber with
figure}\label{explain-construction-of-optical-fiber-with-figure}

\begin{solutionbox}

\textbf{Construction of Optical Fiber:}

An optical fiber consists of three main parts:

\begin{figure}
\centering
\pandocbounded{\includegraphics[keepaspectratio,alt={Optical Fiber Structure}]{unit-4-book-scan_artifacts/image_000007_76d19da86696fd4fc6257a5556f40e8b6317fdeeebaa10f7d2810a4b9c4964f2.png}}
\caption{Optical Fiber Structure}
\end{figure}

\textbf{Cross-Sectional View:}

\begin{verbatim}
    ╔═══════════════════════════════════╗
    ║  Buffer Coating/Sheath/Jacket     ║
    ║    (Protective plastic layer)     ║
    ║  ╔═════════════════════════════╗  ║
    ║  ║      Cladding               ║  ║
    ║  ║   (Glass/Plastic, n_{2})       ║  ║
    ║  ║   ┌─────────────────────┐   ║  ║
    ║  ║   │      Core           │   ║  ║
    ║  ║   │  (Glass/Plastic)    │   ║  ║
    ║  ║   │   (n_{1} > n_{2})         │   ║  ║
    ║  ║   │  ~50-200 \mum (MM)    │   ║  ║
    ║  ║   │  ~8-10 \mum (SM)      │   ║  ║
    ║  ║   └─────────────────────┘   ║  ║
    ║  ║      ~125 \mum diameter       ║  ║
    ║  ╚═════════════════════════════╝  ║
    ║        ~250-900 \mum total          ║
    ╚═══════════════════════════════════╝

Longitudinal View:

    ═══════════════════════════════════════►
    Light propagates through core via TIR
    ═══════════════════════════════════════►
\end{verbatim}

\begin{figure}
\centering
\pandocbounded{\includegraphics[keepaspectratio,alt={Fiber Optic Cable Cross-Section}]{unit-4-book-scan_artifacts/image_000008_53457beac663b3666b8a27c3a0940b10efaa314aa02373fcfc36c1639a5a7ee1.png}}
\caption{Fiber Optic Cable Cross-Section}
\end{figure}

\emph{Figure: Cross-section of a typical telecommunication optical fiber
cable showing 6 optical fibers, insulated copper wires, central steel
wire for strength, and aluminum sheath}

\textbf{Detailed Structure:}

\textbf{1. Core (Inner Part):}

\begin{itemize}
\tightlist
\item
  \textbf{Material:} Ultra-pure silica glass (SiO_{2}) or plastic
\item
  \textbf{Diameter:}

  \begin{itemize}
  \tightlist
  \item
    Single mode: 8-10 \mum
  \item
    Multimode: 50-62.5 \mum
  \end{itemize}
\item
  \textbf{Refractive Index:} High (n_{1} \approx 1.48-1.50)
\item
  \textbf{Doping:} May be doped with germania (GeO_{2}) to increase
  refractive index
\item
  \textbf{Function:} Carries light signals through total internal
  reflection
\end{itemize}

\textbf{2. Cladding (Middle Part):}

\begin{itemize}
\tightlist
\item
  \textbf{Material:} Silica glass with lower refractive index
\item
  \textbf{Diameter:} 125 \mum (standard)
\item
  \textbf{Refractive Index:} Lower than core (n_{2} \approx 1.46)
\item
  \textbf{Function:}

  \begin{itemize}
  \tightlist
  \item
    Provides optical confinement by reflecting light back into core
  \item
    Reduces light loss
  \item
    Adds mechanical strength
  \end{itemize}
\end{itemize}

\textbf{3. Protective Jacket/Buffer Coating (Outer Part):}

\begin{itemize}
\tightlist
\item
  \textbf{Material:} Plastic polymer (polyimide, acrylate, or PVC)
\item
  \textbf{Thickness:} 250-900 \mum
\item
  \textbf{Function:}

  \begin{itemize}
  \tightlist
  \item
    Protects fiber from physical damage
  \item
    Provides mechanical strength
  \item
    Prevents moisture ingress
  \item
    Color coding for identification
  \end{itemize}
\end{itemize}

\textbf{Additional Layers (in cables):}

\begin{itemize}
\tightlist
\item
  \textbf{Strength Members:} Kevlar or steel wires for tensile strength
\item
  \textbf{Outer Sheath:} Additional plastic covering for environmental
  protection
\end{itemize}

\textbf{Diagram with Light Propagation:}

\begin{figure}
\centering
\pandocbounded{\includegraphics[keepaspectratio,alt={Light Propagation in Optical Fiber}]{unit-4-book-scan_artifacts/image_000009_e0dec5824494a470fa998dd811ce81c419d7e98b2611dcc7b0b5aefad75b01ab.png}}
\caption{Light Propagation in Optical Fiber}
\end{figure}

\textbf{ASCII Representation:}

\begin{verbatim}
    Incident Light
         ↓
    ╔═══│═══════════════════════╗
    ║   ↓                       ║
    ║  ╔↓═══╲               ╗   ║  
    ║  ║  ═══╲═════╱       ║   ║  Cladding (n_{2})
    ║  ║   ═══╲═══╱═══     ║   ║  
    ║  ║    ═══X═══╱       ║   ║  Core (n_{1} > n_{2})
    ║  ║     ══╱═══════    ║   ║  
    ║  ║      ╱═════════   ║   ║  Cladding (n_{2})
    ║  ╚═════════════════════╝  ║
    ║    TIR at core-cladding   ║  Buffer Coating
    ║      interface (\theta > \thetac)   ║
    ╚═══════════════════════════╝
                              \rightarrow  Output Light
\end{verbatim}

\textbf{Acceptance Angle and Numerical Aperture:}

\begin{figure}
\centering
\pandocbounded{\includegraphics[keepaspectratio,alt={Acceptance Angle Diagram}]{unit-4-book-scan_artifacts/image_000010_ae771083844832b8c76a1ef581fdc3f8a14064c63a34bb5658717eff2c86c235.png}}
\caption{Acceptance Angle Diagram}
\end{figure}

\emph{Figure: Light ray entering optical fiber at acceptance angle \theta_{a},
showing refraction at fiber end and total internal reflection at
core-cladding interface}

\begin{figure}
\centering
\pandocbounded{\includegraphics[keepaspectratio,alt={Acceptance Cone}]{unit-4-book-scan_artifacts/image_000011_d2163fa7fa2b00ab527d92325cf2ed7ab4e665fa8fa12a9cd10cf4a20c06ce6f.png}}
\caption{Acceptance Cone}
\end{figure}

\emph{Figure: Acceptance cone showing the solid angle (2\theta_{a}) within which
light can enter and propagate through the fiber}

\textbf{Key Points:}

\begin{itemize}
\tightlist
\item
  Core has higher refractive index than cladding (n_{1} \textgreater{} n_{2})
\item
  Light is confined to core by total internal reflection
\item
  Typical core/cladding ratio determines fiber characteristics
\item
  Multiple fibers bundled together form fiber optic cables
\item
  The difference in refractive indices (\Deltan = n_{1} - n_{2}) is small
  (\textasciitilde0.01-0.02)
\item
  \textbf{Acceptance Angle (\theta_{a}):} Maximum angle at which light can enter
  the fiber and still propagate via TIR
\item
  \textbf{Numerical Aperture (NA):} sin \theta_{a} = \sqrt(n_{1}^{2} - n_{2}^{2}) - measures
  light-gathering capacity
\item
  Larger NA means better light-gathering ability
\item
  Typical NA values range from 0.13 to 0.50
\end{itemize}

\end{solutionbox}
\begin{center}\rule{0.5\linewidth}{0.5pt}\end{center}

\subsubsection{(9) Describe advantages of optical fiber over coaxial
cable}\label{describe-advantages-of-optical-fiber-over-coaxial-cable}

\begin{solutionbox}

\textbf{Advantages of Optical Fiber over Coaxial Cable:}

\textbf{1. Higher Bandwidth:}

\begin{itemize}
\tightlist
\item
  \textbf{Optical Fiber:} Can carry data at rates of Terabits per second
  (Tbps)
\item
  \textbf{Coaxial Cable:} Limited to Megabits to Gigabits per second
\item
  \textbf{Reason:} Light has much higher frequency than electrical
  signals
\item
  \textbf{Impact:} Supports more simultaneous communications
\end{itemize}

\textbf{2. Lower Signal Attenuation (Loss):}

\begin{itemize}
\tightlist
\item
  \textbf{Optical Fiber:} 0.2-0.5 dB/km loss
\item
  \textbf{Coaxial Cable:} 10-30 dB/km loss
\item
  \textbf{Benefit:} Signals travel longer distances without
  amplification/repeaters
\item
  \textbf{Result:} Fewer repeaters needed, lower installation and
  maintenance costs
\end{itemize}

\textbf{3. Immunity to Electromagnetic Interference (EMI):}

\begin{itemize}
\tightlist
\item
  \textbf{Optical Fiber:} Not affected by electromagnetic fields, radio
  frequency interference
\item
  \textbf{Coaxial Cable:} Susceptible to EMI, crosstalk from nearby
  cables
\item
  \textbf{Advantage:} Cleaner signals, fewer errors
\item
  \textbf{Applications:} Suitable for electrically noisy environments
  (factories, power plants)
\end{itemize}

\textbf{4. Higher Security:}

\begin{itemize}
\tightlist
\item
  \textbf{Optical Fiber:} Difficult to tap without detection; no
  electromagnetic radiation
\item
  \textbf{Coaxial Cable:} Can be tapped easily; radiates electromagnetic
  signals
\item
  \textbf{Benefit:} More secure for sensitive data transmission
\item
  \textbf{Use Cases:} Military, banking, government communications
\end{itemize}

\textbf{5. Lighter and Smaller:}

\begin{itemize}
\tightlist
\item
  \textbf{Optical Fiber:} Weighs \textasciitilde10-20 g/km; diameter
  \textasciitilde125 \mum
\item
  \textbf{Coaxial Cable:} Weighs \textasciitilde100-300 g/km; diameter
  \textasciitilde5-10 mm
\item
  \textbf{Advantages:}

  \begin{itemize}
  \tightlist
  \item
    Easier to install in tight spaces
  \item
    Lower transportation costs
  \item
    More cables in same conduit
  \end{itemize}
\end{itemize}

\textbf{6. Non-Conductive:}

\begin{itemize}
\tightlist
\item
  \textbf{Optical Fiber:} Made of glass/plastic; doesn't conduct
  electricity
\item
  \textbf{Coaxial Cable:} Metal conductor; can conduct electricity
\item
  \textbf{Benefits:}

  \begin{itemize}
  \tightlist
  \item
    No spark hazard in flammable environments
  \item
    No ground loop problems
  \item
    Safe during lightning strikes
  \item
    Can be used in high-voltage environments
  \end{itemize}
\end{itemize}

\textbf{7. Better Signal Quality:}

\begin{itemize}
\tightlist
\item
  \textbf{Optical Fiber:} Digital signals maintain integrity over long
  distances
\item
  \textbf{Coaxial Cable:} Analog signals degrade with distance
\item
  \textbf{Result:} Less signal distortion, better data integrity
\end{itemize}

\textbf{8. Longer Distance Transmission:}

\begin{itemize}
\tightlist
\item
  \textbf{Optical Fiber:} Up to 100+ km without repeaters
\item
  \textbf{Coaxial Cable:} Maximum \textasciitilde1-2 km without
  amplification
\item
  \textbf{Benefit:} Reduced infrastructure costs
\end{itemize}

\textbf{9. Corrosion Resistance:}

\begin{itemize}
\tightlist
\item
  \textbf{Optical Fiber:} Glass/plastic not affected by moisture,
  chemicals
\item
  \textbf{Coaxial Cable:} Metal conductors can corrode
\item
  \textbf{Longevity:} Longer lifespan in harsh environments
\end{itemize}

\textbf{10. Future-Proof:}

\begin{itemize}
\tightlist
\item
  \textbf{Optical Fiber:} Bandwidth can be upgraded without changing
  cables (upgrade equipment only)
\item
  \textbf{Coaxial Cable:} Limited upgrade potential
\item
  \textbf{Economic:} Better long-term investment
\end{itemize}

\textbf{Comparison Table:}

\begin{longtable}[]{@{}lll@{}}
\toprule\noalign{}
Feature & Optical Fiber & Coaxial Cable \\
\midrule\noalign{}
\endhead
\bottomrule\noalign{}
\endlastfoot
Bandwidth & Very High (Tbps) & Moderate (Mbps-Gbps) \\
Attenuation & 0.2-0.5 dB/km & 10-30 dB/km \\
Distance & 100+ km & 1-2 km \\
EMI Immunity & Excellent & Poor \\
Weight & Light (\textasciitilde10-20 g/km) & Heavy
(\textasciitilde100-300 g/km) \\
Size & Small (125 \mum) & Large (5-10 mm) \\
Security & Very High & Low \\
Cost & Higher initial & Lower initial \\
Installation & Requires expertise & Easier \\
Durability & High & Moderate \\
\end{longtable}

\textbf{Disadvantages of Optical Fiber:}

\begin{enumerate}
\tightlist
\item
  Higher initial installation cost
\item
  Requires specialized equipment and training
\item
  More fragile (glass fibers can break)
\item
  Difficult to splice and terminate
\item
  Cannot carry electrical power (unlike copper)
\end{enumerate}

\end{solutionbox}
\begin{center}\rule{0.5\linewidth}{0.5pt}\end{center}

\subsection*{Part C: Numerical Solutions (3
marks)}\label{part-c-numerical-solutions-3-marks}

\subsubsection{(1) Velocity of light in air is 3 \times 10^{8} m/s and in liquid
1.8 \times 10^{8} m/s, so find out refractive index of
liquid}\label{velocity-of-light-in-air-is-3-10ux2078-ms-and-in-liquid-1.8-10ux2078-ms-so-find-out-refractive-index-of-liquid}

\textbf{Given:}

\begin{itemize}
\tightlist
\item
  Velocity of light in air, c = 3 \times 10^{8} m/s
\item
  Velocity of light in liquid, v = 1.8 \times 10^{8} m/s
\end{itemize}

\textbf{To Find:}

\begin{itemize}
\tightlist
\item
  Refractive index of liquid, n = ?
\end{itemize}

\textbf{Formula:}

\begin{verbatim}
n = c / v
\end{verbatim}

\textbf{Solution:}

\begin{verbatim}
n = c / v
n = (3 \times 10^{8}) / (1.8 \times 10^{8})
n = 3 / 1.8
n = 1.67
\end{verbatim}

\begin{solutionbox}
The refractive index of the liquid is \textbf{1.67}

\end{solutionbox}
\begin{center}\rule{0.5\linewidth}{0.5pt}\end{center}

\subsubsection{(2) Velocity of light in air is 3 \times 10^{8} m/s and in glass
2 \times 10^{8} m/s, so find out refractive index of
glass}\label{velocity-of-light-in-air-is-3-10ux2078-ms-and-in-glass-2-10ux2078-ms-so-find-out-refractive-index-of-glass}

\textbf{Given:}

\begin{itemize}
\tightlist
\item
  Velocity of light in air, c = 3 \times 10^{8} m/s
\item
  Velocity of light in glass, v = 2 \times 10^{8} m/s
\end{itemize}

\textbf{To Find:}

\begin{itemize}
\tightlist
\item
  Refractive index of glass, n = ?
\end{itemize}

\textbf{Formula:}

\begin{verbatim}
n = c / v
\end{verbatim}

\textbf{Solution:}

\begin{verbatim}
n = c / v
n = (3 \times 10^{8}) / (2 \times 10^{8})
n = 3 / 2
n = 1.5
\end{verbatim}

\begin{solutionbox}
The refractive index of the glass is \textbf{1.5}

\end{solutionbox}
\begin{center}\rule{0.5\linewidth}{0.5pt}\end{center}

\subsubsection{(3) Light enters in glass medium from air. Glass has
refractive index is 1.56. So find velocity of light in
glass}\label{light-enters-in-glass-medium-from-air.-glass-has-refractive-index-is-1.56.-so-find-velocity-of-light-in-glass}

\textbf{Given:}

\begin{itemize}
\tightlist
\item
  Velocity of light in air, c = 3 \times 10^{8} m/s
\item
  Refractive index of glass, n = 1.56
\end{itemize}

\textbf{To Find:}

\begin{itemize}
\tightlist
\item
  Velocity of light in glass, v = ?
\end{itemize}

\textbf{Formula:}

\begin{verbatim}
n = c / v
Therefore, v = c / n
\end{verbatim}

\textbf{Solution:}

\begin{verbatim}
v = c / n
v = (3 \times 10^{8}) / 1.56
v = 1.923 \times 10^{8} m/s
\end{verbatim}

\begin{solutionbox}
The velocity of light in glass is \textbf{1.923 \times 10^{8}
m/s} or \textbf{1.92 \times 10^{8} m/s}

\end{solutionbox}
\begin{center}\rule{0.5\linewidth}{0.5pt}\end{center}

\subsubsection{(4) One optical fiber has value of refractive indices are
1.563 and 1.498, respectively. Calculate acceptance angle of optical
fiber}\label{one-optical-fiber-has-value-of-refractive-indices-are-1.563-and-1.498-respectively.-calculate-acceptance-angle-of-optical-fiber}

\textbf{Given:}

\begin{itemize}
\tightlist
\item
  Refractive index of core, n_{1} = 1.563
\item
  Refractive index of cladding, n_{2} = 1.498
\item
  Refractive index of air, n_{0} = 1
\end{itemize}

\textbf{To Find:}

\begin{itemize}
\tightlist
\item
  Acceptance angle, \theta_{a} = ?
\end{itemize}

\textbf{Formulas:}

\begin{verbatim}
Numerical Aperture (NA) = \sqrt(n_{1}^{2} - n_{2}^{2})
Acceptance Angle: sin \theta_{a} = NA / n_{0}
\end{verbatim}

\textbf{Solution:}

\textbf{Step 1:} Calculate Numerical Aperture (NA)

\begin{verbatim}
NA = \sqrt(n_{1}^{2} - n_{2}^{2})
NA = \sqrt(1.563^{2} - 1.498^{2})
NA = \sqrt(2.4430 - 2.2440)
NA = \sqrt0.1990
NA = 0.446
\end{verbatim}

\textbf{Step 2:} Calculate Acceptance Angle

\begin{verbatim}
sin \theta_{a} = NA / n_{0}
sin \theta_{a} = 0.446 / 1
sin \theta_{a} = 0.446
\theta_{a} = sin^{-}^{1}(0.446)
\theta_{a} = 26.5^\circ
\end{verbatim}

\begin{solutionbox}

\begin{itemize}
\tightlist
\item
  \textbf{Numerical Aperture (NA) = 0.446}
\item
  \textbf{Acceptance Angle (\theta_{a}) = 26.5^\circ}
\end{itemize}

\end{solutionbox}
\begin{center}\rule{0.5\linewidth}{0.5pt}\end{center}

\subsubsection{(5) An optical fiber has value of refractive indices are
1.48 and 1.45, respectively. Calculate acceptance angle and numerical
aperture of optical
fiber}\label{an-optical-fiber-has-value-of-refractive-indices-are-1.48-and-1.45-respectively.-calculate-acceptance-angle-and-numerical-aperture-of-optical-fiber}

\textbf{Given:}

\begin{itemize}
\tightlist
\item
  Refractive index of core, n_{1} = 1.48
\item
  Refractive index of cladding, n_{2} = 1.45
\item
  Refractive index of air, n_{0} = 1
\end{itemize}

\textbf{To Find:}

\begin{itemize}
\tightlist
\item
  Numerical Aperture (NA) = ?
\item
  Acceptance angle, \theta_{a} = ?
\end{itemize}

\textbf{Formulas:}

\begin{verbatim}
Numerical Aperture (NA) = \sqrt(n_{1}^{2} - n_{2}^{2})
Acceptance Angle: sin \theta_{a} = NA / n_{0}
\end{verbatim}

\textbf{Solution:}

\textbf{Step 1:} Calculate Numerical Aperture (NA)

\begin{verbatim}
NA = \sqrt(n_{1}^{2} - n_{2}^{2})
NA = \sqrt(1.48^{2} - 1.45^{2})
NA = \sqrt(2.1904 - 2.1025)
NA = \sqrt0.0879
NA = 0.2965
NA \approx 0.297
\end{verbatim}

\textbf{Step 2:} Calculate Acceptance Angle

\begin{verbatim}
sin \theta_{a} = NA / n_{0}
sin \theta_{a} = 0.297 / 1
sin \theta_{a} = 0.297
\theta_{a} = sin^{-}^{1}(0.297)
\theta_{a} = 17.27^\circ
\theta_{a} \approx 17.3^\circ
\end{verbatim}

\begin{solutionbox}

\begin{itemize}
\tightlist
\item
  \textbf{Numerical Aperture (NA) = 0.297 or 0.30}
\item
  \textbf{Acceptance Angle (\theta_{a}) = 17.3^\circ}
\end{itemize}

\textbf{Note:} Numerical Aperture (NA) is an important parameter that
indicates:

\begin{itemize}
\tightlist
\item
  The light-gathering capacity of the optical fiber
\item
  The larger the NA, the more light can enter the fiber
\item
  Maximum half-angle at which light can enter the fiber core
\end{itemize}

\end{solutionbox}
\begin{center}\rule{0.5\linewidth}{0.5pt}\end{center}

\subsection*{Additional Important
Formulas}\label{additional-important-formulas}

\subsubsection{Refractive Index}\label{refractive-index}

\begin{verbatim}
n = c /

v = sin i / sin

r = \lambda_{0} / \lambda_{m}

\end{verbatim}

\subsubsection{Critical Angle}\label{critical-angle}

\begin{verbatim}
sin \thetac = n_{2} / n_{1}  (where n_{1} > n_{2})
\end{verbatim}

\subsubsection{Numerical Aperture}\label{numerical-aperture}

\begin{verbatim}
NA = \sqrt(n_{1}^{2} - n_{2}^{2}) = n_{0} sin \theta_{a}
\end{verbatim}

Where:

\begin{itemize}
\tightlist
\item
  n_{1} = core refractive index
\item
  n_{2} = cladding refractive index
\item
  n_{0} = refractive index of medium from which light enters (usually air,
  n_{0} = 1)
\item
  \theta_{a} = acceptance angle
\end{itemize}

\subsubsection{Fractional Refractive Index
Change}\label{fractional-refractive-index-change}

\begin{verbatim}
\Delta = (n_{1} - n_{2}) / n_{1}
\end{verbatim}

\subsubsection{For small \Delta}\label{for-small-ux3b4}

\begin{verbatim}
NA \approx n_{1}\sqrt(2\Delta)
\end{verbatim}

\begin{center}\rule{0.5\linewidth}{0.5pt}\end{center}

\emph{End of Solutions}


\end{document}
