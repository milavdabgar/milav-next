\documentclass[10pt,a4paper]{article}
\usepackage[margin=0.6in]{geometry}
\usepackage{amsmath,amssymb,amsthm}
\usepackage{booktabs}
\usepackage{multirow}
\usepackage{xcolor}
\usepackage{tcolorbox}
\tcbuselibrary{breakable}
\usepackage[colorlinks=true,linkcolor=blue]{hyperref}
\usepackage{titlesec}
\usepackage{enumitem}
\usepackage{tikz}
\usepackage{circuitikz}
\usetikzlibrary{shapes,arrows,positioning,calc}

% XeLaTeX for Gujarati support
\usepackage{fontspec}
\usepackage{polyglossia}
\setmainlanguage{gujarati}
\setotherlanguage{english}

% Gujarati font with proper script features
\setmainfont{Noto Sans Gujarati}[
  Script=Gujarati,
  Renderer=Harfbuzz,
  Language=Gujarati
]
\newfontfamily\englishfont{Latin Modern Roman}

% Color scheme
\definecolor{headcolor}{RGB}{0,102,204}
\definecolor{keycolor}{RGB}{220,20,60}
\definecolor{solutioncolor}{RGB}{34,139,34}
\definecolor{mnemoniccolor}{RGB}{148,0,211}

% Custom environments
\newtcolorbox{solutionbox}{
 breakable,
 colback=solutioncolor!5!white,
 colframe=solutioncolor!75!black,
 fonttitle=\bfseries,
 title=ઉકેલ
}

\newtcolorbox{solutionboxnobreak}{
 colback=solutioncolor!5!white,
 colframe=solutioncolor!75!black,
 fonttitle=\bfseries,
 title=ઉકેલ
}

\newtcolorbox{keyformula}{
 colback=keycolor!5!white,
 colframe=keycolor!75!black,
 fonttitle=\bfseries,
 title=મુખ્ય સૂત્ર
}

\newtcolorbox{mnemonicbox}{
 colback=mnemoniccolor!5!white,
 colframe=mnemoniccolor!75!black,
 fonttitle=\bfseries,
 title=મેમરી ટ્રીક
}

% Spacing
\setlength{\parskip}{3pt}
\setlist[itemize]{nosep}
\setlist[enumerate]{nosep}

% Title formatting
\titleformat{\section}{\Large\bfseries\color{headcolor}}{\thesection}{1em}{}
\titleformat{\subsection}{\large\bfseries\color{headcolor}}{\thesubsection}{1em}{}

\begin{document}

\begin{center}
{\Huge\bfseries\color{headcolor} આધુનિક ભૌતિકશાસ્ત્ર ઉકેલો}\\[5pt]
{\LARGE DI01000061 -- શિયાળો 2024}\\[3pt]
{\large સેમેસ્ટર 1 અભ્યાસ સામગ્રી}\\[3pt]
{\normalsize\textit{વિગતવાર ઉકેલો અને સમજૂતીઓ}}
\end{center}

\vspace{10pt}

%----------------------------------------
\section*{પ્રશ્ન 1 -- ખાલી જગ્યા પૂરો/બહુવિકલ્પ પ્રશ્નો [14 ગુણ]}

\begin{solutionbox}
\textbf{જવાબ કોષ્ટક:}

\begin{center}
\begin{tabular}{|c|c|c|c|}
\hline
\textbf{પ્રશ્ન} & \textbf{જવાબ} & \textbf{પ્રશ્ન} & \textbf{જવાબ} \\
\hline
(1) & (a) Si & (8) & (b) 0.5 Hz \\
(2) & (a) 1.50 & (9) & (a) 300000 km/s \\
(3) & (b) વધારે & (10) & (b) ઘન \\
(4) & (c) 4 & (11) & (a) શૃંગ અને ગર્ત \\
(5) & (d) પૂર્ણ આંતરિક પરાવર્તન & (12) & (b) એકરંગી \\
(6) & (d) આવૃત્તિ & (13) & (a) સિંગલ મોડ \\
(7) & (a) કુલંબ & (14) & (b) 45° \\
\hline
\end{tabular}
\end{center}
\end{solutionbox}

\begin{mnemonicbox}
``સિલિકોન ગ્લાસ બ્રિજ ઓપ્ટિક આવૃત્તિ કુલંબ Hz ઘન શૃંગ મોનો સિંગલ 45''
\end{mnemonicbox}

%----------------------------------------
\section*{પ્રશ્ન 2(A) -- કોઈપણ બેના જવાબ આપો [6 ગુણ]}

\subsection*{પ્રશ્ન 2(A)(1) [3 ગુણ]}
\textbf{ચોકસાઈ અને સચોટતા વચ્ચેનો તફાવત આપો.}

\begin{solutionbox}
\begin{center}
\begin{tabular}{|l|p{5.5cm}|p{5.5cm}|}
\hline
\textbf{પરિમાણ} & \textbf{ચોકસાઈ (Accuracy)} & \textbf{સચોટતા (Precision)} \\
\hline
વ્યાખ્યા & સાચા મૂલ્યની નજીક & પુનરાવર્તિત માપનોની સુસંગતતા \\
\hline
કેન્દ્ર & સાચું હોવું & પુનઃઉત્પાદન \\
\hline
ભૂલનો પ્રકાર & વ્યવસ્થિત ભૂલ & અવ્યવસ્થિત ભૂલ \\
\hline
ઉદાહરણ & લક્ષ્યમાં મારવું & સમાન જગ્યાએ વારંવાર મારવું \\
\hline
\end{tabular}
\end{center}

\textbf{મુખ્ય મુદ્દાઓ:}
\begin{itemize}
\item \textbf{ચોકસાઈ}: માપ વાસ્તવિક મૂલ્યની કેટલી નજીક છે
\item \textbf{સચોટતા}: પુનરાવર્તિત માપન એકબીજાની કેટલી નજીક છે
\end{itemize}
\end{solutionbox}

\begin{mnemonicbox}
``ચોકસાઈ વાસ્તવિક લક્ષ્ય, સચોટતા સુસંગત પુનરાવર્તન''
\end{mnemonicbox}

\subsection*{પ્રશ્ન 2(A)(2) [3 ગુણ]}
\textbf{માઇક્રોમીટર સ્ક્રૂ દ્વારા માપવામાં આવતા ગોળાનો વ્યાસ નક્કી કરો, મુખ્ય માપપટ્ટીનું માપ 5 mm અને વર્તુળાકાર માપપટ્ટીનો 50મો વિભાગ બેઝ લાઇન સાથે મેચ થાય છે. આ સાધનની લ.મા.શ 0.01 mm છે.}

\begin{solutionbox}
\textbf{આપેલ:}
\begin{align*}
\text{મુખ્ય માપપટ્ટી વાંચન (MSR)} &= 5\text{ mm}\\
\text{વર્તુળાકાર માપપટ્ટી વાંચન (CSR)} &= 50\text{ વિભાગ}\\
\text{લઘુતમ માપશક્તિ (LC)} &= 0.01\text{ mm}
\end{align*}

\textbf{સૂત્ર:}
\[\text{કુલ વાંચન} = \text{MSR} + (\text{CSR} \times \text{LC})\]

\textbf{ગણતરી:}
\begin{align*}
\text{કુલ વાંચન} &= 5 + (50 \times 0.01)\\
&= 5 + 0.5\\
&= 5.5\text{ mm}
\end{align*}

\textbf{જવાબ: ગોળાનો વ્યાસ = 5.5 mm}
\end{solutionbox}

\begin{mnemonicbox}
``મુખ્ય વાંચન + વર્તુળાકાર $\times$ લઘુતમ માપશક્તિ''
\end{mnemonicbox}

\subsection*{પ્રશ્ન 2(A)(3) [3 ગુણ]}
\textbf{જ્યારે 4 $\mu$F કેપેસિટન્સ ધરાવતા કેપેસિટરને 12 volt બેટરી સાથે જોડતા કેપેસિટરની બંને પ્લેટ પર સંગ્રહિત થતાં વિદ્યુતભારના જથ્થાની ગણતરી કરો.}

\begin{solutionbox}
\textbf{આપેલ:}
\begin{align*}
\text{કેપેસિટન્સ } (C) &= 4~\mu\text{F} = 4 \times 10^{-6}\text{ F}\\
\text{વોલ્ટેજ } (V) &= 12\text{ V}
\end{align*}

\begin{keyformula}
\[Q = CV\]
\end{keyformula}

\textbf{ગણતરી:}
\begin{align*}
Q &= 4 \times 10^{-6} \times 12\\
&= 48 \times 10^{-6}\text{ C}\\
&= 48~\mu\text{C}
\end{align*}

\textbf{જવાબ: સંગ્રહિત વિદ્યુતભાર = 48 $\mu$C}
\end{solutionbox}

\begin{mnemonicbox}
``ચાર્જ બરાબર કેપેસિટન્સ ગુણ્યે વોલ્ટેજ''
\end{mnemonicbox}

%----------------------------------------
\section*{પ્રશ્ન 2(B) -- કોઈપણ બેના જવાબ આપો [8 ગુણ]}

\subsection*{પ્રશ્ન 2(B)(1) [4 ગુણ]}
\textbf{યોગ્ય નામકરણ સાથે માઇક્રોમીટર સ્ક્રૂ ગેજની આકૃતિ દોરો.}

\begin{solutionbox}
\textbf{માઇક્રોમીટર સ્ક્રૂ ગેજ આકૃતિ:}

\begin{center}
\begin{tikzpicture}[scale=1.2]
% Frame
\draw[thick] (0,0) -- (0,2) -- (0.5,2.5) -- (5,2.5) -- (5,0) -- cycle;
% Anvil (fixed)
\draw[fill=gray!30,thick] (0.5,0.8) rectangle (1,1.6);
% Spindle (movable)
\draw[fill=gray!30,thick] (3,0.8) rectangle (3.5,1.6);
% Main sleeve/barrel
\draw[thick] (2.5,0.7) rectangle (4.5,1.7);
% Thimble
\draw[thick] (3.5,0.5) -- (4.5,0.5) -- (4.5,1.9) -- (3.5,1.9);
% Ratchet
\draw[thick] (4.5,1.2) circle (0.3);
\draw[thick] (4.5,1.2) -- (4.8,1.2);
\draw[thick] (4.5,1.2) -- (4.5,1.5);
\draw[thick] (4.5,1.2) -- (4.5,0.9);

% Labels
\node[below] at (0.75,0.8) {\small એન્વિલ};
\node[below] at (3.25,0.4) {\small સ્પિંડલ};
\node[above] at (3.5,2.5) {\small મુખ્ય સ્કેલ};
\node[right] at (5,1.4) {\small થિમ્બલ સ્કેલ};
\node[above] at (4.8,1.5) {\small રેચેટ};
\node[below] at (2.5,0) {\small ફ્રેમ};

% Main scale markings
\foreach \x in {2.6,2.8,3.0,3.2,3.4}
    \draw (\x,1.65) -- (\x,1.55);
\node[above,font=\tiny] at (2.6,1.65) {0};
\node[above,font=\tiny] at (3.0,1.65) {5};
\node[above,font=\tiny] at (3.4,1.65) {10};

% Thimble scale markings
\foreach \y in {0.6,0.8,1.0,1.2,1.4,1.6,1.8}
    \draw (4.4,\y) -- (4.5,\y);
\end{tikzpicture}
\end{center}

\textbf{મુખ્ય ઘટકો:}
\begin{itemize}
\item \textbf{ફ્રેમ}: U-આકારનું માળખું જે આધાર પૂરો પાડે
\item \textbf{એન્વિલ}: વસ્તુ મૂકવા માટે સ્થિર જડબો
\item \textbf{સ્પિંડલ}: ગતિશીલ સ્ક્રૂ મેકેનિઝમ
\item \textbf{થિમ્બલ સ્કેલ}: 50 વિભાગ સાથે વર્તુળાકાર સ્કેલ
\item \textbf{મુખ્ય સ્કેલ}: mm માં રેખીય સ્કેલ
\item \textbf{રેચેટ}: સુસંગત દબાણ લાગુ કરવા માટે
\end{itemize}
\end{solutionbox}

\begin{mnemonicbox}
``ફ્રેમ એન્વિલ સ્પિંડલ થિમ્બલ મુખ્ય રેચેટ''
\end{mnemonicbox}

\subsection*{પ્રશ્ન 2(B)(2) [4 ગુણ]}
\textbf{વર્નિયર કેલિપર્સ માટે યોગ્ય આકૃતિ સાથે શૂન્ય, ધન અને ઋણ ત્રુટીઓ સમજાવો અને આ પ્રકારની ત્રુટીઓ દૂર કરવા માટેના જરૂરી પગલાંની યાદી બનાવો.}

\begin{solutionbox}
\textbf{ત્રુટીના પ્રકારો:}

\begin{center}
\begin{tabular}{|l|p{5cm}|p{4cm}|}
\hline
\textbf{ત્રુટીનો પ્રકાર} & \textbf{સ્થિતિ} & \textbf{વાંચન} \\
\hline
શૂન્ય ત્રુટિ & વર્નિયરની શૂન્ય રેખા મુખ્ય સ્કેલની શૂન્ય સાથે મેળ ખાતી નથી & જડબા બંધ હોય ત્યારે શૂન્ય અલાવાનું વાંચન \\
\hline
ધન ત્રુટિ & વર્નિયર શૂન્ય મુખ્ય સ્કેલ શૂન્યની જમણી બાજુએ & સુધારો ઉમેરો \\
\hline
ઋણ ત્રુટિ & વર્નિયર શૂન્ય મુખ્ય સ્કેલ શૂન્યની ડાબી બાજુએ & સુધારો બાદ કરો \\
\hline
\end{tabular}
\end{center}

\vspace{5pt}
\textbf{આકૃતિઓ:}

\begin{center}
\begin{tikzpicture}[scale=0.8]
% Zero Error
\node at (-1,2) {\textbf{શૂન્ય ત્રુટિ:}};
\draw[thick] (0,1.5) -- (6,1.5);
\foreach \x in {0,1,2,3,4,5}
    \draw (\x,1.5) -- (\x,1.3) node[below,font=\scriptsize] {\x};
\draw[thick,blue] (0,0.8) -- (5,0.8);
\foreach \x in {0,1,2,3,4}
    \draw[blue] (\x,0.8) -- (\x,1.0);
\node[blue] at (0,0.5) {\scriptsize 0};

% Positive Error
\node at (-1,0) {\textbf{ધન ત્રુટિ:}};
\draw[thick] (0,-0.5) -- (6,-0.5);
\foreach \x in {0,1,2,3,4,5}
    \draw (\x,-0.5) -- (\x,-0.7) node[below,font=\scriptsize] {\x};
\draw[thick,red] (0.3,-1.2) -- (5.3,-1.2);
\foreach \x in {0.3,1.3,2.3,3.3,4.3}
    \draw[red] (\x,-1.2) -- (\x,-1.0);
\node[red] at (0.3,-1.5) {\scriptsize 0};

% Negative Error  
\node at (-1,-2) {\textbf{ઋણ ત્રુટિ:}};
\draw[thick] (0,-2.5) -- (6,-2.5);
\foreach \x in {0,1,2,3,4,5}
    \draw (\x,-2.5) -- (\x,-2.7) node[below,font=\scriptsize] {\x};
\draw[thick,green!50!black] (-0.2,-3.2) -- (4.8,-3.2);
\foreach \x in {-0.2,0.8,1.8,2.8,3.8}
    \draw[green!50!black] (\x,-3.2) -- (\x,-3.0);
\node[green!50!black] at (-0.2,-3.5) {\scriptsize 0};
\end{tikzpicture}
\end{center}

\textbf{ત્રુટીઓ દૂર કરવાના પગલાં:}
\begin{enumerate}
\item \textbf{શૂન્ય ત્રુટિ તપાસો} માપન પહેલાં
\item \textbf{અંતિમ વાંચનમાં સુધારો લાગુ કરો}
\item \textbf{જડબાઓ સાફ કરો} કચરો અટકાવવા માટે
\item \textbf{સાવચેતીથી હાથ વણો} યાંત્રિક નુકસાન ટાળવા માટે
\end{enumerate}
\end{solutionbox}

\begin{mnemonicbox}
``તપાસો સાફ કરો સુધારો સાવચેતી''
\end{mnemonicbox}

\subsection*{પ્રશ્ન 2(B)(3) [4 ગુણ]}
\textbf{સાદા લોલકનો આવર્તકાળ શોધવાના પ્રયોગમાં અવલોકનો 1.96 s, 1.98 s, 2.00 s, 2.02 s, 2.04 s છે. નિરપેક્ષ ત્રુટિ, સરેરાશ નિરપેક્ષ ત્રુટિ, સાપેક્ષ ત્રુટિ અને પ્રતિશત ત્રુટિની ગણતરી કરો.}

\begin{solutionbox}
\textbf{અવલોકનો:} 1.96, 1.98, 2.00, 2.02, 2.04 s

\textbf{સરેરાશ મૂલ્ય:}
\[\bar{x} = \frac{1.96 + 1.98 + 2.00 + 2.02 + 2.04}{5} = \frac{10.00}{5} = 2.00\text{ s}\]

\textbf{નિરપેક્ષ ત્રુટીઓ:} $|x_i - \bar{x}|$
\begin{center}
\begin{tabular}{|c|c|c|}
\hline
\textbf{અવલોકન} & \textbf{મૂલ્ય (s)} & \textbf{નિરપેક્ષ ત્રુટિ (s)} \\
\hline
1 & 1.96 & $|1.96 - 2.00| = 0.04$ \\
2 & 1.98 & $|1.98 - 2.00| = 0.02$ \\
3 & 2.00 & $|2.00 - 2.00| = 0.00$ \\
4 & 2.02 & $|2.02 - 2.00| = 0.02$ \\
5 & 2.04 & $|2.04 - 2.00| = 0.04$ \\
\hline
\end{tabular}
\end{center}

\textbf{સરેરાશ નિરપેક્ષ ત્રુટિ:}
\[\Delta x_{\text{mean}} = \frac{0.04 + 0.02 + 0.00 + 0.02 + 0.04}{5} = \frac{0.12}{5} = 0.024\text{ s}\]

\textbf{સાપેક્ષ ત્રુટિ:}
\[\text{સાપેક્ષ ત્રુટિ} = \frac{\Delta x_{\text{mean}}}{\bar{x}} = \frac{0.024}{2.00} = 0.012\]

\textbf{પ્રતિશત ત્રુટિ:}
\[\text{પ્રતિશત ત્રુટિ} = \text{સાપેક્ષ ત્રુટિ} \times 100 = 0.012 \times 100 = 1.2\%\]

\textbf{પરિણામો:}
\begin{itemize}
\item સરેરાશ નિરપેક્ષ ત્રુટિ = 0.024 s
\item સાપેક્ષ ત્રુટિ = 0.012
\item પ્રતિશત ત્રુટિ = 1.2\%
\end{itemize}
\end{solutionbox}

\begin{mnemonicbox}
``સરેરાશ નિરપેક્ષ સાપેક્ષ પ્રતિશત''
\end{mnemonicbox}

%----------------------------------------
\section*{પ્રશ્ન 3(A) -- કોઈપણ બેના જવાબ આપો [6 ગુણ]}

\subsection*{પ્રશ્ન 3(A)(1) [3 ગુણ]}
\textbf{વ્યાખ્યાઓ કરો: વિદ્યુત ફ્લક્સ, વિદ્યુતક્ષેત્ર, વીજસ્થિતિમાનનો તફાવત}

\begin{solutionbox}
\begin{center}
\begin{tabular}{|l|p{4cm}|c|c|}
\hline
\textbf{શબ્દ} & \textbf{વ્યાખ્યા} & \textbf{એકમ} & \textbf{સૂત્ર} \\
\hline
વિદ્યુત ફ્લક્સ & સપાટીમાંથી પસાર થતી વિદ્યુત ક્ષેત્ર રેખાઓની સંખ્યા & Nm²/C & $\Phi = E \cdot A$ \\
\hline
વિદ્યુતક્ષેત્ર & એકમ ધન આવેશ પર લાગતું બળ & N/C & $E = F/q$ \\
\hline
વીજસ્થિતિમાનનો તફાવત & બે બિંદુઓ વચ્ચે એકમ આવેશ દીઠ કામ & વોલ્ટ & $V = W/q$ \\
\hline
\end{tabular}
\end{center}

\textbf{મુખ્ય બિંદુઓ:}
\begin{itemize}
\item \textbf{વિદ્યુત ફ્લક્સ}: સપાટીમાં પ્રવેશતી ક્ષેત્ર રેખાઓનું માપ
\item \textbf{વિદ્યુતક્ષેત્ર}: વિદ્યુત બળ ક્રિયા કરતો વિસ્તાર
\item \textbf{વીજસ્થિતિમાનનો તફાવત}: એકમ આવેશ દીઠ ઊર્જાનો તફાવત
\end{itemize}
\end{solutionbox}

\begin{mnemonicbox}
``ફ્લક્સ ક્ષેત્ર બળ, કામ વોટ્સ વોલ્ટ્સ''
\end{mnemonicbox}

\subsection*{પ્રશ્ન 3(A)(2) [3 ગુણ]}
\textbf{જ્યારે ત્રણ જુદા જુદા કેપેસિટરોને શ્રેણીમાં જોડવામાં આવે ત્યારે જરૂરી સર્કિટ ડાયાગ્રામ સાથે સમકક્ષ કેપેસિટન્સ માટેનું સૂત્ર મેળવો.}

\begin{solutionbox}
\textbf{સર્કિટ ડાયાગ્રામ:}

\begin{center}
\begin{circuitikz}[scale=1.2]
\draw (0,0) to[battery1, l=$V$] (0,2)
      to[short] (1,2)
      to[C, l=$C_1$] (2.5,2)
      to[C, l=$C_2$] (4,2)
      to[C, l=$C_3$] (5.5,2)
      to[short] (6,2)
      to[short] (6,0)
      to[short] (0,0);
\end{circuitikz}
\end{center}

\textbf{વ્યુત્પત્તિ:}

\begin{itemize}
\item \textbf{સમાન આવેશ} $Q$ દરેક કેપેસિટર દ્વારા વહે છે
\item \textbf{વોલ્ટેજ વિભાજન}: $V = V_1 + V_2 + V_3$
\item \textbf{દરેક કેપેસિટર માટે}: $V_1 = \frac{Q}{C_1}$, $V_2 = \frac{Q}{C_2}$, $V_3 = \frac{Q}{C_3}$
\item \textbf{કુલ વોલ્ટેજ}: 
\[V = \frac{Q}{C_1} + \frac{Q}{C_2} + \frac{Q}{C_3} = Q\left(\frac{1}{C_1} + \frac{1}{C_2} + \frac{1}{C_3}\right)\]
\item \textbf{સમકક્ષ માટે}: $V = \frac{Q}{C_s}$
\end{itemize}

\begin{keyformula}
\[\frac{1}{C_s} = \frac{1}{C_1} + \frac{1}{C_2} + \frac{1}{C_3}\]
\end{keyformula}
\end{solutionbox}

\begin{mnemonicbox}
``શ્રેણી વિપરીત સરવાળો, સમાન આવેશ વિભાજિત વોલ્ટેજ''
\end{mnemonicbox}

\subsection*{પ્રશ્ન 3(A)(3) [3 ગુણ]}
\textbf{વ્યાખ્યાઓ કરો: ઇન્ફ્રાસોનિક ધ્વનિ, શ્રાવ્ય ધ્વનિ, અલ્ટ્રાસોનિક ધ્વનિ}

\begin{solutionbox}
\begin{center}
\begin{tabular}{|l|c|p{3.5cm}|p{3cm}|}
\hline
\textbf{ધ્વનિનો પ્રકાર} & \textbf{આવૃત્તિ શ્રેણી} & \textbf{લાક્ષણિકતાઓ} & \textbf{ઉપયોગો} \\
\hline
ઇન્ફ્રાસોનિક & 20 Hz થી નીચે & મનુષ્યને સંભળાતું નથી & ભૂકંપ શોધ \\
\hline
શ્રાવ્ય & 20 Hz થી 20 kHz & મનુષ્યને સંભળાય છે & વાતચીત, સંગીત \\
\hline
અલ્ટ્રાસોનિક & 20 kHz થી ઉપર & મનુષ્યને સંભળાતું નથી & તબીબી ઇમેજિંગ, SONAR \\
\hline
\end{tabular}
\end{center}

\textbf{વિગતો:}
\begin{itemize}
\item \textbf{ઇન્ફ્રાસોનિક}: માનવ શ્રવણથી નીચેની ઓછી આવૃત્તિ
\item \textbf{શ્રાવ્ય}: માનવો માટે સામાન્ય શ્રવણ શ્રેણી
\item \textbf{અલ્ટ્રાસોનિક}: માનવ શ્રવણથી ઉપરની ઊંચી આવૃત્તિ
\end{itemize}
\end{solutionbox}

\begin{mnemonicbox}
``ઇન્ફ્રા-નીચે, શ્રાવ્ય-વચ્ચે, અલ્ટ્રા-ઉપર''
\end{mnemonicbox}

%----------------------------------------
\section*{પ્રશ્ન 3(B) -- કોઈપણ બેના જવાબ આપો [8 ગુણ]}

\subsection*{પ્રશ્ન 3(B)(1) [4 ગુણ]}
\textbf{સમાંતર પ્લેટ કેપેસિટર માટે $C = \varepsilon_0 A/d$ સાબિત કરો.}

\begin{solutionbox}
\textbf{આકૃતિ:}

\begin{center}
\begin{tikzpicture}[scale=1.0]
% Plate 1 (positive)
\draw[thick,fill=red!20] (0,0) rectangle (0.3,3);
\foreach \y in {0.3,0.9,1.5,2.1,2.7}
    \node at (0.15,\y) {+};
\node[left] at (0,1.5) {$+Q$};

% Plate 2 (negative)
\draw[thick,fill=blue!20] (3,0) rectangle (3.3,3);
\foreach \y in {0.3,0.9,1.5,2.1,2.7}
    \node at (3.15,\y) {--};
\node[right] at (3.3,1.5) {$-Q$};

% Electric field lines
\foreach \y in {0.5,1.0,1.5,2.0,2.5}
    \draw[->,thick,blue] (0.4,\y) -- (2.9,\y);

% Distance label
\draw[<->,thick] (0.3,-0.5) -- (3,-0.5) node[midway,below] {$d$};

% Area label
\draw[<->,thick] (-0.5,0) -- (-0.5,3) node[midway,left] {$A$};

\node[above] at (1.65,3.2) {વિદ્યુતક્ષેત્ર $E$};
\end{tikzpicture}
\end{center}

\textbf{વ્યુત્પત્તિ:}

\begin{enumerate}
\item \textbf{પ્લેટો વચ્ચે વિદ્યુત ક્ષેત્ર}:
\[E = \frac{\sigma}{\varepsilon_0} = \frac{Q}{\varepsilon_0 A}\]

\item \textbf{વીજસ્થિતિમાનનો તફાવત}:
\[V = E \times d = \frac{Qd}{\varepsilon_0 A}\]

\item \textbf{કેપેસિટન્સની વ્યાખ્યા}:
\[C = \frac{Q}{V}\]

\item \textbf{બદલીને}:
\[C = \frac{Q}{\frac{Qd}{\varepsilon_0 A}} = \frac{\varepsilon_0 A}{d}\]
\end{enumerate}

\begin{keyformula}
\[C = \frac{\varepsilon_0 A}{d}\]

જ્યાં:
\begin{itemize}
\item $\varepsilon_0$: મુક્ત અવકાશની વિદ્યુત પ્રવેશ્યતા
\item $A$: પ્લેટોનું ક્ષેત્રફળ
\item $d$: પ્લેટો વચ્ચેનું અંતર
\end{itemize}
\end{keyformula}
\end{solutionbox}

\begin{mnemonicbox}
``કેપેસિટન્સ બરાબર એપ્સિલોન-શૂન્ય ક્ષેત્રફળ ભાગુ અંતર''
\end{mnemonicbox}

\subsection*{પ્રશ્ન 3(B)(2) [4 ગુણ]}
\textbf{વિદ્યુતક્ષેત્ર રેખાઓની લાક્ષણિકતાઓ સૂચિબદ્ધ કરો.}

\begin{solutionbox}
\textbf{મુખ્ય લાક્ષણિકતાઓ:}

\begin{enumerate}
\item \textbf{દિશા}: ધન આવેશથી ઋણ આવેશ તરફ
\item \textbf{ઘનતા}: ક્ષેત્રની મજબૂતાઈ દર્શાવે છે
\item \textbf{નિરંતર}: મુક્ત અવકાશમાં ક્યારેય તૂટતી નથી
\item \textbf{બિન-છેદન}: બે રેખાઓ ક્યારેય પાર કરતી નથી
\item \textbf{લંબ}: વાહક સપાટી પર લંબ હોય છે
\item \textbf{બંધ લૂપ}: ફક્ત બદલાતા ચુંબકીય ક્ષેત્રની આસપાસ
\item \textbf{સ્પર્શક}: કોઈપણ બિંદુએ ક્ષેત્રની દિશા આપે છે
\item \textbf{સમાન અંતર}: સમાન ક્ષેત્રના વિસ્તારોમાં
\end{enumerate}

\textbf{ગુણધર્મો:}
\begin{itemize}
\item \textbf{ધન આવેશ}થી શરુ થાય છે
\item \textbf{ઋણ આવેશ}પર સમાપ્ત થાય છે
\item \textbf{વધુ ઘનતા} મજબૂત ક્ષેત્ર દર્શાવે છે
\item \textbf{ક્યારેય છેદન નથી} કરતી
\end{itemize}
\end{solutionbox}

\begin{mnemonicbox}
``ધન થી ઋણ, ઘન મજબૂત, ક્યારેય છેદે નહીં, હંમેશા લંબ''
\end{mnemonicbox}

\subsection*{પ્રશ્ન 3(B)(3) [4 ગુણ]}
\textbf{અલ્ટ્રાસોનિક તરંગોના ઉત્પાદન માટે ઉપયોગમાં લેવામાં આવતી મેગ્નેટોસ્ટ્રિક્શન પદ્ધતિની રચના અને કાર્યપદ્ધતિનું વર્ણન કરો.}

\begin{solutionbox}
\textbf{રચના:}

\begin{center}
\begin{tikzpicture}[scale=1.0,>=stealth]
\node[draw,rectangle,minimum width=2cm,minimum height=1cm] (osc) at (0,0) {ઓસિલેટર};
\node[draw,rectangle,minimum width=1.5cm,minimum height=1cm] (coil) at (3,0) {કોઇલ};
\node[draw,rectangle,minimum width=2cm,minimum height=1cm,fill=gray!30] (rod) at (6,0) {નિકલ રોડ};
\node[draw,rectangle,minimum width=1.5cm,minimum height=1cm] (horn) at (9,0) {હોર્ન};

\draw[->,thick] (osc) -- (coil) node[midway,above] {AC};
\draw[->,thick] (coil) -- (rod) node[midway,above] {કંપન};
\draw[->,thick] (rod) -- (horn) node[midway,above] {વર્ધન};
\draw[->,thick,red] (horn) -- (11,0) node[right] {અલ્ટ્રાસોનિક};
\end{tikzpicture}
\end{center}

\textbf{ઘટકો:}
\begin{itemize}
\item \textbf{નિકલ રોડ}: મેગ્નેટોસ્ટ્રિક્ટિવ પદાર્થ
\item \textbf{કોઇલ}: રોડની આસપાસ ઇલેક્ટ્રોમેગ્નેટ
\item \textbf{AC ઓસિલેટર}: ઊંચી આવૃત્તિનો પ્રવાહ સ્ત્રોત
\item \textbf{હોર્ન}: ધ્વનિ વર્ધક અને ટ્રાન્સમિટર
\end{itemize}

\textbf{કાર્યપદ્ધતિ:}
\begin{enumerate}
\item \textbf{AC પ્રવાહ} કોઇલમાંથી વહે છે
\item \textbf{ચુંબકીય ક્ષેત્ર} ઝડપથી બદલાય છે
\item \textbf{નિકલ રોડ} વિસ્તૃત અને સંકુચિત થાય છે
\item \textbf{યાંત્રિક કંપનો} ઉત્પન્ન થાય છે
\item \textbf{અલ્ટ્રાસોનિક તરંગો} ઉત્પન્ન થાય છે
\end{enumerate}

\textbf{ઉપયોગો}: તબીબી ઇમેજિંગ, સફાઈ, વેલ્ડિંગ
\end{solutionbox}

\begin{mnemonicbox}
``AC કોઇલ નિકલને કંપાવે છે, અલ્ટ્રાસોનિક બનાવે છે''
\end{mnemonicbox}

%----------------------------------------
\section*{પ્રશ્ન 4(A) -- કોઈપણ બેના જવાબ આપો [6 ગુણ]}

\subsection*{પ્રશ્ન 4(A)(1) [3 ગુણ]}
\textbf{એક રેડિયો સ્ટેશન $9.26 \times 10^7$ Hz આવૃત્તિવાળા તરંગોનું ઉત્સર્જન કરે છે. જો આ તરંગોની ઝડપ $3.00 \times 10^8$ m/s હોય તો તેની તરંગલંબાઈ શોધો.}

\begin{solutionbox}
\textbf{આપેલ:}
\begin{align*}
\text{આવૃત્તિ } (f) &= 9.26 \times 10^7\text{ Hz}\\
\text{ઝડપ } (c) &= 3.00 \times 10^8\text{ m/s}
\end{align*}

\begin{keyformula}
\[c = f\lambda\]
\[\text{તેથી: } \lambda = \frac{c}{f}\]
\end{keyformula}

\textbf{ગણતરી:}
\begin{align*}
\lambda &= \frac{3.00 \times 10^8}{9.26 \times 10^7}\\
&= 3.24\text{ m}
\end{align*}

\textbf{જવાબ: તરંગલંબાઈ = 3.24 m}
\end{solutionbox}

\begin{mnemonicbox}
``ઝડપ બરાબર આવૃત્તિ ગુણ્યે તરંગલંબાઈ''
\end{mnemonicbox}

\subsection*{પ્રશ્ન 4(A)(2) [3 ગુણ]}
\textbf{સ્નેલનો નિયમ જણાવો અને માધ્યમનો વક્રીભવનાંક સમજાવો.}

\begin{solutionbox}
\begin{keyformula}
\textbf{સ્નેલનો નિયમ:}
\[n_1 \sin\theta_1 = n_2 \sin\theta_2\]

જ્યાં:
\begin{itemize}
\item $n_1, n_2$: માધ્યમ 1 અને 2 ના વક્રીભવનાંક
\item $\theta_1, \theta_2$: આપાત અને વક્રીભવન કોણ
\end{itemize}
\end{keyformula}

\textbf{વક્રીભવનાંક:}

\begin{center}
\begin{tabular}{|l|p{5cm}|c|}
\hline
\textbf{પ્રકાર} & \textbf{વ્યાખ્યા} & \textbf{સૂત્ર} \\
\hline
નિરપેક્ષ & શૂન્યાવકાશમાં પ્રકાશની ઝડપ અને માધ્યમમાં ઝડપનો ગુણોત્તર & $n = c/v$ \\
\hline
સાપેક્ષ & બે માધ્યમોમાં ઝડપનો ગુણોત્તર & $n_{21} = v_1/v_2$ \\
\hline
\end{tabular}
\end{center}

\textbf{મુખ્ય બિંદુઓ:}
\begin{itemize}
\item \textbf{ઊંચો વક્રીભવનાંક}: ઘન માધ્યમ, ધીમો પ્રકાશ
\item \textbf{નીચો વક્રીભવનાંક}: વિરળ માધ્યમ, ઝડપી પ્રકાશ
\end{itemize}
\end{solutionbox}

\begin{mnemonicbox}
``સ્નેલ સાઇન ગુણોત્તર સ્થિર, ઘન પ્રકાશ ધીમો કરે''
\end{mnemonicbox}

\subsection*{પ્રશ્ન 4(A)(3) [3 ગુણ]}
\textbf{સરખામણી કરો: સામાન્ય પ્રકાશ અને LASER}

\begin{solutionbox}
\begin{center}
\begin{tabular}{|l|p{4.5cm}|p{4.5cm}|}
\hline
\textbf{ગુણધર્મ} & \textbf{સામાન્ય પ્રકાશ} & \textbf{LASER} \\
\hline
સુસંગતતા & અસુસંગત & સુસંગત \\
\hline
રંગ & બહુરંગી & એકરંગી \\
\hline
દિશા & વિકીર્ણ & સમાંતર કિરણ \\
\hline
તીવ્રતા & ઓછી & ખૂબ વધારે \\
\hline
કલા & અવ્યવસ્થિત & સ્થિર કલા સંબંધ \\
\hline
તરંગલંબાઈ & બહુવિધ તરંગલંબાઈ & એકલ તરંગલંબાઈ \\
\hline
\end{tabular}
\end{center}

\textbf{મુખ્ય તફાવતો:}
\begin{itemize}
\item \textbf{LASER}: સુસંગત, એકરંગી, સમાંતર, તીવ્ર
\item \textbf{સામાન્ય}: અસુસંગત, બહુરંગી, વિકીર્ણ, ઓછી તીવ્ર
\end{itemize}
\end{solutionbox}

\begin{mnemonicbox}
``LASER: સુસંગત એકરંગી સમાંતર તીવ્ર''
\end{mnemonicbox}

%----------------------------------------
\section*{પ્રશ્ન 4(B) -- કોઈપણ બેના જવાબ આપો [8 ગુણ]}

\subsection*{પ્રશ્ન 4(B)(1) [4 ગુણ]}
\textbf{જરૂરી આકૃતિ સાથે ઓપ્ટિકલ ફાઇબરની રચના દર્શાવો.}

\begin{solutionbox}
\textbf{ઓપ્ટિકલ ફાઇબર રચના:}

\begin{center}
\begin{tikzpicture}[scale=1.2]
% Core
\draw[thick,fill=yellow!30] (0,0.5) rectangle (6,1.5);
\node at (3,1) {કોર ($n_1$ વધારે)};

% Cladding top
\draw[thick,fill=blue!20] (0,1.5) rectangle (6,2);
% Cladding bottom
\draw[thick,fill=blue!20] (0,0) rectangle (6,0.5);
\node at (3,1.75) {\small ક્લેડિંગ ($n_2$ ઓછો)};

% Jacket top
\draw[thick,fill=gray!40] (0,2) rectangle (6,2.3);
% Jacket bottom
\draw[thick,fill=gray!40] (0,-0.3) rectangle (6,0);
\node at (3,-0.6) {\small સુરક્ષાત્મક જેકેટ};

% Light ray path
\draw[->,red,thick] (0,1) -- (1.5,1.4) -- (3,1) -- (4.5,1.4) -- (6,1);
\node[red] at (6.5,1) {પ્રકાશ};

% Labels
\draw[<->,thick] (6.5,0.5) -- (6.5,1.5) node[midway,right] {કોર};
\draw[<->,thick] (6.5,0) -- (6.5,2) node[midway,right,xshift=0.8cm] {ક્લેડિંગ};
\end{tikzpicture}
\end{center}

\textbf{ઘટકો:}

\begin{center}
\begin{tabular}{|l|c|p{3cm}|c|}
\hline
\textbf{ઘટક} & \textbf{સામગ્રી} & \textbf{કાર્ય} & \textbf{વક્રીભવનાંક} \\
\hline
કોર & કાચ/પ્લાસ્ટિક & પ્રકાશ સંચાર & વધારે ($n_1$) \\
\hline
ક્લેડિંગ & કાચ & પૂર્ણ આંતરિક પરાવર્તન & ઓછો ($n_2$) \\
\hline
જેકેટ & પ્લાસ્ટિક & સુરક્ષા & -- \\
\hline
\end{tabular}
\end{center}

\textbf{કાર્યપદ્ધતિ:}
\begin{itemize}
\item પ્રકાશ \textbf{કોર}માં સ્વીકૃતિ કોણ પર પ્રવેશે છે
\item કોર-ક્લેડિંગ સીમા પર \textbf{પૂર્ણ આંતરિક પરાવર્તન}
\item પ્રકાશ કોરમાં \textbf{ઝિગઝેગ માર્ગ}માં મુસાફરી કરે છે
\item \textbf{$n_1 > n_2$} પ્રકાશ કેદ સુનિશ્ચિત કરે છે
\end{itemize}
\end{solutionbox}

\begin{mnemonicbox}
``કોર ક્લેડિંગ જેકેટ, વધારે ઓછો સુરક્ષા''
\end{mnemonicbox}

\subsection*{પ્રશ્ન 4(B)(2) [4 ગુણ]}
\textbf{ઇજનેરી અને મેડિકલ ક્ષેત્રે LASER ના ઉપયોગોની યાદી આપો.}

\begin{solutionbox}
\textbf{ઇજનેરિંગ ઉપયોગો:}
\begin{enumerate}
\item \textbf{કટિંગ અને વેલ્ડિંગ}: ચોક્કસ ધાતુ કાપવા
\item \textbf{3D પ્રિંટિંગ}: લેઝર સિન્ટરિંગ
\item \textbf{માપન}: અંતર અને સર્વેક્ષણ
\item \textbf{સંચાર}: ઓપ્ટિકલ ફાઇબર સિસ્ટમ
\item \textbf{સામગ્રી પ્રક્રિયા}: સપાટી કઠિનીકરણ
\item \textbf{બારકોડ સ્કેનિંગ}: રિટેઇલ અને ઇન્વેન્ટરી
\end{enumerate}

\textbf{તબીબી ઉપયોગો:}
\begin{enumerate}
\item \textbf{શસ્ત્રક્રિયા}: ચોક્કસ પેશી કાપવા
\item \textbf{આંખની સારવાર}: સુધારાત્મક શસ્ત્રક્રિયા
\item \textbf{કેન્સર સારવાર}: ગાંઠનો નાશ
\item \textbf{નિદાન}: સ્પેક્ટ્રોસ્કોપી
\item \textbf{દંત ચિકિત્સા}: કેવિટી સારવાર
\item \textbf{ચામડીની સારવાર}: કોસ્મેટિક પ્રક્રિયાઓ
\end{enumerate}

\textbf{ફાયદા}: ચોકસાઈ, બિન-સંપર્ક, જંતુરહિત, ન્યૂનતમ નુકસાન
\end{solutionbox}

\begin{mnemonicbox}
``ઇજનેરિંગ: કાપ વેલ્ડ માપ સંચાર, મેડિકલ: શસ્ત્રક્રિયા આંખ કેન્સર નિદાન''
\end{mnemonicbox}

\subsection*{પ્રશ્ન 4(B)(3) [4 ગુણ]}
\textbf{P-type અને N-type અર્ધવાહકો સમજાવો.}

\begin{solutionbox}
\textbf{N-type અર્ધવાહક:}

\begin{center}
\begin{tabular}{|l|p{8cm}|}
\hline
\textbf{ગુણધર્મ} & \textbf{N-type} \\
\hline
ડોપન્ટ & ફોસ્ફોરસ, આર્સેનિક (5 વેલેન્સ ઇલેક્ટ્રોન) \\
\hline
મુખ્ય વાહકો & ઇલેક્ટ્રોન \\
\hline
ગૌણ વાહકો & હોલ્સ \\
\hline
આવેશ & નકારાત્મક \\
\hline
\end{tabular}
\end{center}

\vspace{5pt}
\textbf{P-type અર્ધવાહક:}

\begin{center}
\begin{tabular}{|l|p{8cm}|}
\hline
\textbf{ગુણધર્મ} & \textbf{P-type} \\
\hline
ડોપન્ટ & બોરોન, એલ્યુમિનિયમ (3 વેલેન્સ ઇલેક્ટ્રોન) \\
\hline
મુખ્ય વાહકો & હોલ્સ \\
\hline
ગૌણ વાહકો & ઇલેક્ટ્રોન \\
\hline
આવેશ & સકારાત્મક \\
\hline
\end{tabular}
\end{center}

\textbf{રચના પ્રક્રિયા:}
\begin{itemize}
\item \textbf{N-type}: પંચસંયોજક અણુઓ ઇલેક્ટ્રોન દાન કરે છે
\item \textbf{P-type}: ત્રિસંયોજક અણુઓ ઇલેક્ટ્રોન સ્વીકારે છે, હોલ્સ બનાવે છે
\item \textbf{ડોપિંગ}: અશુદ્ધતાઓનો નિયંત્રિત ઉમેરો
\item \textbf{વાહકતા}: મુક્ત વાહકોને કારણે વધે છે
\end{itemize}
\end{solutionbox}

\begin{mnemonicbox}
``N-type નકારાત્મક ઇલેક્ટ્રોન, P-type સકારાત્મક હોલ્સ''
\end{mnemonicbox}

%----------------------------------------
\section*{પ્રશ્ન 5(A) -- કોઈપણ બેના જવાબ આપો [6 ગુણ]}

\subsection*{પ્રશ્ન 5(A)(1) [3 ગુણ]}
\textbf{ઊર્જા બેન્ડ ગેપના આધારે વાહકો, અર્ધવાહકો અને અવાહકોનું વર્ગીકરણ કરો.}

\begin{solutionbox}
\begin{center}
\begin{tabular}{|l|c|p{3.5cm}|p{2.5cm}|}
\hline
\textbf{સામગ્રી} & \textbf{ઊર્જા બેન્ડ ગેપ} & \textbf{લાક્ષણિકતાઓ} & \textbf{ઉદાહરણો} \\
\hline
વાહક & કોઈ ગેપ નથી (0 eV) & વેલેન્સ અને વહન બેન્ડ ઓવરલેપ & તાંબુ, ચાંદી \\
\hline
અર્ધવાહક & નાનો ગેપ (1-3 eV) & મધ્યમ બેન્ડ ગેપ & સિલિકોન, જર્મેનિયમ \\
\hline
અવાહક & મોટો ગેપ ($>$3 eV) & પહોળો બેન્ડ ગેપ & કાચ, રબર \\
\hline
\end{tabular}
\end{center}

\textbf{ઊર્જા બેન્ડ આકૃતિ:}

\begin{center}
\begin{tikzpicture}[scale=0.9]
% Conductor
\node at (1.5,3) {\textbf{વાહક}};
\draw[thick,fill=blue!20] (0,1.5) rectangle (3,2.5);
\node at (1.5,2) {CB/VB};
\node at (1.5,0.5) {કોઈ ગેપ};

% Semiconductor
\node at (5.5,3) {\textbf{અર્ધવાહક}};
\draw[thick,fill=blue!20] (4,2.3) rectangle (7,2.8);
\node[right] at (7,2.55) {\scriptsize CB};
\draw[thick,fill=green!20] (4,1.5) rectangle (7,2.0);
\node[right] at (7,1.75) {\scriptsize VB};
\draw[<->,thick] (7.3,2.0) -- (7.3,2.3) node[midway,right] {\tiny 1-3 eV};
\node at (5.5,0.5) {નાનો ગેપ};

% Insulator
\node at (9.5,3) {\textbf{અવાહક}};
\draw[thick,fill=blue!20] (8,2.5) rectangle (11,2.8);
\node[right] at (11,2.65) {\scriptsize CB};
\draw[thick,fill=green!20] (8,1.5) rectangle (11,1.8);
\node[right] at (11,1.65) {\scriptsize VB};
\draw[<->,thick] (11.3,1.8) -- (11.3,2.5) node[midway,right] {\tiny $>$3 eV};
\node at (9.5,0.5) {મોટો ગેપ};
\end{tikzpicture}
\end{center}

\textbf{નોંધ:} CB = વહન બેન્ડ, VB = વેલેન્સ બેન્ડ
\end{solutionbox}

\begin{mnemonicbox}
``કોઈ ગેપ વાહે, નાનો ગેપ અર્ધ, મોટો ગેપ અવાહ''
\end{mnemonicbox}

\subsection*{પ્રશ્ન 5(A)(2) [3 ગુણ]}
\textbf{જરૂરી ટ્રુથ ટેબલ સાથે OR અને AND લોજિક ગેટ સમજાવો.}

\begin{solutionbox}
\textbf{OR ગેટ:}

\begin{minipage}{0.45\textwidth}
\begin{center}
\begin{tabular}{|c|c|c|}
\hline
\textbf{A} & \textbf{B} & \textbf{Y = A + B} \\
\hline
0 & 0 & 0 \\
0 & 1 & 1 \\
1 & 0 & 1 \\
1 & 1 & 1 \\
\hline
\end{tabular}
\end{center}
\end{minipage}
\hfill
\begin{minipage}{0.45\textwidth}
\begin{center}
\begin{tikzpicture}[scale=0.8]
\draw[thick] (0,0.5) -- (0.5,0.5);
\draw[thick] (0,1.5) -- (0.5,1.5);
\draw[thick] (0.5,0.3) arc (-90:90:0.7);
\draw[thick] (2.5,1) -- (3,1);
\node[left] at (0,1.5) {A};
\node[left] at (0,0.5) {B};
\node[right] at (3,1) {Y};
\end{tikzpicture}
\end{center}
\end{minipage}

\vspace{5pt}
\textbf{AND ગેટ:}

\begin{minipage}{0.45\textwidth}
\begin{center}
\begin{tabular}{|c|c|c|}
\hline
\textbf{A} & \textbf{B} & \textbf{Y = A · B} \\
\hline
0 & 0 & 0 \\
0 & 1 & 0 \\
1 & 0 & 0 \\
1 & 1 & 1 \\
\hline
\end{tabular}
\end{center}
\end{minipage}
\hfill
\begin{minipage}{0.45\textwidth}
\begin{center}
\begin{tikzpicture}[scale=0.8]
\draw[thick] (0,0.5) -- (1,0.5) -- (1,0);
\draw[thick] (0,1.5) -- (1,1.5) -- (1,2);
\draw[thick] (1,0) -- (1,2);
\draw[thick] (1,0) arc (-90:90:1);
\draw[thick] (2.5,1) -- (3,1);
\node[left] at (0,1.5) {A};
\node[left] at (0,0.5) {B};
\node[right] at (3,1) {Y};
\end{tikzpicture}
\end{center}
\end{minipage}

\vspace{5pt}
\textbf{મુખ્ય બિંદુઓ:}
\begin{itemize}
\item \textbf{OR}: કોઈપણ ઇનપુટ HIGH હોય ત્યારે આઉટપુટ HIGH
\item \textbf{AND}: બધા ઇનપુટ HIGH હોય ત્યારે આઉટપુટ HIGH
\end{itemize}
\end{solutionbox}

\begin{mnemonicbox}
``OR: કોઈ પણ હાઈ બનાવે હાઈ, AND: બધા હાઈ બનાવે હાઈ''
\end{mnemonicbox}

\subsection*{પ્રશ્ન 5(A)(3) [3 ગુણ]}
\textbf{વોલ્ટેજ રેગ્યુલેટર તરીકે ઝેનર ડાયોડના ઉપયોગનું વર્ણન કરો.}

\begin{solutionbox}
\textbf{સર્કિટ આકૃતિ:}

\begin{center}
\begin{circuitikz}[scale=1.2]
\draw (0,0) to[V, l=$V_{in}$] (0,2)
      to[R, l=$R_s$] (3,2)
      to[short] (4,2)
      to[short, -o] (5,2) node[right] {$V_{out}$};
\draw (3,2) to[zzD, l=$Z$] (3,0);
\draw (4,2) to[R, l=$R_L$] (4,0);
\draw (0,0) to[short] (5,0) node[right] {GND};
\draw (5,0) to[short, -o] (5,0);
\end{circuitikz}
\end{center}

\textbf{કાર્યપદ્ધતિ:}
\begin{itemize}
\item \textbf{ફોરવર્ડ બાયાસ}: સામાન્ય ડાયોડની જેમ કાર્ય કરે છે
\item \textbf{રિવર્સ બાયાસ}: ઝેનર વોલ્ટેજ પર બ્રેકડાઉન
\item \textbf{વોલ્ટેજ રેગ્યુલેશન}: સ્થિર $V_{out} = V_z$ જાળવે છે
\item \textbf{શ્રેણી રેઝિસ્ટર}: ઝેનર દ્વારા કરંટ મર્યાદિત કરે છે
\end{itemize}

\textbf{લાક્ષણિકતાઓ:}
\begin{itemize}
\item \textbf{ઝેનર વોલ્ટેજ}: સ્થિર બ્રેકડાઉન વોલ્ટેજ
\item \textbf{કરંટ શ્રેણી}: વિશાળ ઓપરેટિંગ રેન્જ
\item \textbf{તાપમાન સ્થિરતા}: સારી વોલ્ટેજ સ્થિરતા
\item \textbf{પાવર રેટિંગ}: મહત્તમ પાવર વટાવવું નહીં
\end{itemize}

\textbf{ઉપયોગો}: પાવર સપ્લાય, વોલ્ટેજ રેફરન્સ, સંરક્ષણ સર્કિટ
\end{solutionbox}

\begin{mnemonicbox}
``ઝેનર ઉત્સાહથી વોલ્ટેજ વિવિધતા છતાં જાળવે છે''
\end{mnemonicbox}

%----------------------------------------
\section*{પ્રશ્ન 5(B) -- કોઈપણ બેના જવાબ આપો [8 ગુણ]}

\subsection*{પ્રશ્ન 5(B)(1) [4 ગુણ]}
\textbf{જરૂરી સર્કિટ સાથે પૂર્ણ તરંગ રેક્ટિફાયર સમજાવો તથા ઇનપુટ અને આઉટપુટ તરંગો દોરો.}

\begin{solutionbox}
\textbf{સેન્ટર-ટેપ પૂર્ણ તરંગ રેક્ટિફાયર:}

\begin{center}
\begin{circuitikz}[scale=1.0]
% Transformer
\draw (0,0) to[sinusoidal voltage source, l=AC] (0,2);
\draw (0,2) -- (1,2);
\draw (0,0) -- (1,0);
\draw (1,0) to[inductor] (2,0);
\draw (1,2) to[inductor] (2,2);
\draw (2,1) node[circ] {} -- (2.5,1);

% Diodes
\draw (2.5,2) to[D, l=$D_1$] (4,2);
\draw (2.5,0) to[D, l=$D_2$] (4,0);
\draw (2.5,2) -- (2.5,1) -- (2.5,0);

% Load
\draw (4,2) to[R, l=$R_L$] (4,0);
\draw (4,2) -- (5,2) node[right] {$+$};
\draw (4,0) -- (5,0) node[right] {$-$};

\node[above] at (1.5,2.3) {ટ્રાન્સફોર્મર};
\end{circuitikz}
\end{center}

\textbf{કાર્યપદ્ધતિ:}
\begin{itemize}
\item \textbf{સકારાત્મક અર્ધ ચક્ર}: $D_1$ વાહે છે, $D_2$ બંધ
\item \textbf{નકારાત્મક અર્ધ ચક્ર}: $D_2$ વાહે છે, $D_1$ બંધ
\item \textbf{બંને અર્ધ}: લોડમાંથી સમાન દિશામાં કરંટ વહે છે
\end{itemize}

\textbf{ફાયદા}: બહેતર કાર્યક્ષમતા, ઓછો રિપલ, બહેતર ટ્રાન્સફોર્મર ઉપયોગ
\end{solutionbox}

\begin{mnemonicbox}
``પૂર્ણ તરંગ પૂર્ણ ચક્ર વાપરે, બહેતર કાર્યક્ષમતા બહેતર આઉટપુટ''
\end{mnemonicbox}

\subsection*{પ્રશ્ન 5(B)(2) [4 ગુણ]}
\textbf{P-N જંકશન ડાયોડની ફોરવર્ડ અને રિવર્સ લાક્ષણિકતાઓ દર્શાવો.}

\begin{solutionbox}
\textbf{P-N જંકશન રચના:}

\begin{center}
\begin{tikzpicture}[scale=1.0]
% P-type region
\draw[thick,fill=red!20] (0,0) rectangle (2,1.5);
\node at (1,0.75) {\textbf{P-type}};
\node[below] at (1,0) {\scriptsize હોલ્સ (+)};

% N-type region
\draw[thick,fill=blue!20] (2,0) rectangle (4,1.5);
\node at (3,0.75) {\textbf{N-type}};
\node[below] at (3,0) {\scriptsize ઇલેક્ટ્રોન (-)};

% Junction line
\draw[very thick,red] (2,0) -- (2,1.5);
\node[right,rotate=90] at (2,0.75) {\scriptsize જંકશન};

% Depletion region indicator
\draw[dashed] (1.7,0) -- (1.7,1.5);
\draw[dashed] (2.3,0) -- (2.3,1.5);
\node[above,font=\tiny] at (2,1.6) {ડિપ્લીશન રીજન};

% Terminals
\draw[thick] (0,0.75) -- (-0.5,0.75) node[left] {એનોડ};
\draw[thick] (4,0.75) -- (4.5,0.75) node[right] {કેથોડ};
\end{tikzpicture}
\end{center}

\textbf{ફોરવર્ડ બાયાસ લાક્ષણિકતાઓ:}

\begin{center}
\begin{tabular}{|l|c|l|}
\hline
\textbf{વોલ્ટેજ શ્રેણી} & \textbf{કરંટ} & \textbf{વર્તન} \\
\hline
0 થી 0.3V (Si) & ખૂબ નાનો & કટ-ઇન વોલ્ટેજ \\
\hline
0.7V થી ઉપર & ઘાતાંકીય વધારો & વાહક \\
\hline
\end{tabular}
\end{center}

\textbf{રિવર્સ બાયાસ લાક્ષણિકતાઓ:}

\begin{center}
\begin{tabular}{|l|c|l|}
\hline
\textbf{વોલ્ટેજ શ્રેણી} & \textbf{કરંટ} & \textbf{વર્તન} \\
\hline
0 થી બ્રેકડાઉન & રિવર્સ સેચ્યુરેશન & લીકેજ કરંટ \\
\hline
બ્રેકડાઉન વોલ્ટેજ & તીવ્ર વધારો & એવેલાન્ચ બ્રેકડાઉન \\
\hline
\end{tabular}
\end{center}
\end{solutionbox}

\begin{solutionboxnobreak}
\textbf{I-V લાક્ષણિક વક્ર:}

\begin{center}
\begin{tikzpicture}[scale=0.9]
% Axes - shorter to fit better
\draw[->,thick] (-2.5,0) -- (1.5,0) node[right] {$V$};
\draw[->,thick] (0,-1.5) -- (0,2.5) node[above] {$I$ (mA)};

% Forward characteristic - exponential curve (much reduced height)
\draw[thick,blue,domain=0:1.2,samples=40] plot (\x,{0.03*exp(4*\x)-0.03});

% Reverse characteristic - flat then breakdown
\draw[thick,blue] (-2,-0.08) -- (-0.05,-0.08);
\draw[thick,blue] (-2,-0.08) -- (-2,-1.3);

% Voltage markers
\node[below] at (0.8,0) {\scriptsize 0.7V};
\node[below] at (-2,0) {\scriptsize $-V_B$};

% Current direction labels
\node[blue] at (0.8,0.6) {\scriptsize ફોરવર્ડ};
\node[blue] at (-1.2,-0.4) {\scriptsize રિવર્સ};
\node at (-2,-0.6) {\scriptsize બ્રેકડાઉન};
\end{tikzpicture}
\end{center}

\textbf{મુખ્ય બિંદુઓ:}
\begin{itemize}
\item \textbf{ફોરવર્ડ}: ઓછો પ્રતિકાર, વધારે કરંટ
\item \textbf{રિવર્સ}: વધારે પ્રતિકાર, ઓછો કરંટ  
\item \textbf{કટ-ઇન વોલ્ટેજ}: સિલિકોન માટે 0.7V, જર્મેનિયમ માટે 0.3V
\end{itemize}
\end{solutionboxnobreak}

\begin{mnemonicbox}
``ફોરવર્ડ વહેવું, રિવર્સ પ્રતિકાર''
\end{mnemonicbox}

\subsection*{પ્રશ્ન 5(B)(3) [4 ગુણ]}
\textbf{LED નો સિદ્ધાંત લખો અને તેની રચના અને કાર્યપદ્ધતિ સમજાવો.}

\begin{solutionbox}
\textbf{સિદ્ધાંત}: \textbf{ઇલેક્ટ્રોલ્યુમિનેસન્સ} -- વિદ્યુત ઊર્જાનું પ્રકાશ ઊર્જામાં સીધું રૂપાંતર

\textbf{રચના:}

\begin{center}
\begin{tikzpicture}[scale=1.1]
% LED structure
\draw[thick,fill=red!20] (1,2) -- (1,2.5) -- (2,2.5) -- (2,2) -- cycle;
\draw[thick,fill=blue!20] (1,1.5) -- (1,2) -- (2,2) -- (2,1.5) -- cycle;
\draw[thick,fill=gray!30] (0.8,2.5) rectangle (2.2,2.8);
\draw[thick,fill=gray!30] (0.8,1.2) rectangle (2.2,1.5);
% Labels
\node at (2.8,2.25) {P-type (એનોડ)};
\node at (2.8,1.75) {જંકશન};
\node at (2.8,1.35) {N-type (કેથોડ)};
% Light arrows
\draw[->,yellow!80!orange,thick] (1.5,2.8) -- (1.5,3.4) node[above] {પ્રકાશ};
\draw[->,yellow!80!orange,thick] (1.2,2.7) -- (0.8,3.2);
\draw[->,yellow!80!orange,thick] (1.8,2.7) -- (2.2,3.2);
% Terminals
\draw[thick] (1,1.2) -- (1,0.8) node[below] {$-$};
\draw[thick] (2,2.8) -- (2,3.2) node[above] {$+$};
\end{tikzpicture}
\end{center}

\textbf{ઉપયોગમાં લેવાતી સામગ્રી:}
\begin{center}
\begin{tabular}{|l|c|c|}
\hline
\textbf{રંગ} & \textbf{સામગ્રી} & \textbf{તરંગલંબાઈ} \\
\hline
લાલ & GaAs & 700 nm \\
\hline
લીલો & GaP & 550 nm \\
\hline
વાદળી & GaN & 470 nm \\
\hline
\end{tabular}
\end{center}
\end{solutionbox}

\begin{solutionbox}
\textbf{કાર્યપદ્ધતિ:}
\begin{enumerate}
\item \textbf{ફોરવર્ડ બાયાસ}: ઇલેક્ટ્રોન અને હોલ્સ જંકશન પર પુનઃસંયોજન
\item \textbf{ઊર્જા મુક્તિ}: પુનઃસંયોજન દરમિયાન ફોટોન ઉત્સર્જન
\item \textbf{પ્રકાશનો રંગ}: બેન્ડ ગેપ ઊર્જા પર આધાર
\item \textbf{કાર્યક્ષમતા}: ઊંચું વિદ્યુત થી ઓપ્ટિકલ રૂપાંતર
\end{enumerate}

\textbf{ઉપયોગો}: ડિસ્પ્લે, ઇન્ડિકેટર, લાઇટિંગ, ઓપ્ટિકલ કમ્યુનિકેશન
\end{solutionbox}

\begin{mnemonicbox}
``LED: પ્રકાશ ઉત્સર્જક ડાયોડ, ઇલેક્ટ્રોન અને હોલ્સ નૃત્ય કરી પ્રકાશ બનાવે છે''
\end{mnemonicbox}

\vspace{15pt}
\begin{center}
\rule{0.8\textwidth}{0.4pt}\\[5pt]
{\large\textbf{--- ઉકેલોનો અંત ---}}\\[3pt]
{\small આધુનિક ભૌતિકશાસ્ત્ર (DI01000061) -- શિયાળો 2024}
\end{center}

\end{document}
