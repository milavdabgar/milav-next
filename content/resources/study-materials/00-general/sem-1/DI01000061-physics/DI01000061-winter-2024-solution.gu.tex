\documentclass{article}

% content/resources/templates/preamble.tex
\usepackage[margin=0.6in]{geometry}
\author{Milav Dabgar}
\usepackage{amsmath,amssymb,amsthm}
\usepackage{booktabs}
\usepackage{multirow}
\usepackage{xcolor}
\usepackage{tcolorbox}
\tcbuselibrary{breakable,skins}
\usepackage[colorlinks=true,linkcolor=blue]{hyperref}
\usepackage{titlesec}
\usepackage{enumitem}
\usepackage{tikz}
\usepackage{pgfplots}
\usepackage{circuitikz}
\usepackage[version=4]{mhchem}
\usepackage{longtable}
\usepackage{array}
\usepackage{float}
\usepackage{caption}
\usepackage{listings}

\lstset{
  basicstyle=\small\ttfamily,
  breaklines=true,
  breakatwhitespace=false,
  postbreak=\mbox{\textcolor{red}{$\hookrightarrow$}\space},
  float=false,
  numbers=left,
  numberstyle=\tiny\color{gray},
  numbersep=10pt,
  xleftmargin=2em,
  keywordstyle=\color{blue},
  commentstyle=\color{green!60!black},
  stringstyle=\color{purple},
  backgroundcolor=\color{gray!5},
  showstringspaces=false,
  tabsize=2,
  captionpos=b,
  keepspaces=true,
  columns=flexible
}

\pgfplotsset{compat=1.18}
\usetikzlibrary{shapes,arrows,positioning,calc,patterns,decorations.pathmorphing,decorations.markings,arrows.meta}

% Color scheme
\definecolor{headcolor}{RGB}{0,102,204}
\definecolor{keycolor}{RGB}{220,20,60}
\definecolor{solutioncolor}{RGB}{34,139,34}
\definecolor{mnemoniccolor}{RGB}{148,0,211}
\definecolor{codecolor}{RGB}{0,0,100}

% Spacing
\setlength{\parskip}{3pt}
\setlist[itemize]{nosep}
\setlist[enumerate]{nosep}

% Title formatting
\titleformat{\section}{\Large\bfseries\color{headcolor}}{\thesection}{1em}{}
\titleformat{\subsection}{\large\bfseries\color{headcolor}}{\thesubsection}{1em}{}

% Pandoc tightlist compatibility
\providecommand{\tightlist}{%
  \setlength{\itemsep}{0pt}\setlength{\parskip}{0pt}}

% Pandoc longtable compatibility
\newcounter{none}
\def\thenone{}


% content/resources/templates/gujarati-boxes.tex
\usepackage{fontspec}
\usepackage{polyglossia}

% Set Gujarati as main language (document is primarily in Gujarati)
% Note: gloss-gujarati.ldf doesn't exist in polyglossia, but it will use hyphenation patterns
\setdefaultlanguage{gujarati}
\setotherlanguage{english}

% Configure Gujarati font properly
% Use Language=Default to prevent polyglossia from trying to add language-specific features
% that don't exist for Gujarati, which causes "empty feature" warnings
\newfontfamily\gujaratifont[Script=Gujarati,AutoFakeBold=2.5,AutoFakeSlant=0.3]{Noto Sans Gujarati}
\setmainfont[Script=Gujarati,AutoFakeBold=2.5,AutoFakeSlant=0.3]{Noto Sans Gujarati}
% Use Noto Sans Gujarati for monospace to support Gujarati in text
\setmonofont[Scale=0.9]{Noto Sans Gujarati}

% Configure English to use the same font
\newfontfamily\englishfont[Script=Gujarati,AutoFakeBold=2.5,AutoFakeSlant=0.3]{Noto Sans Gujarati}

% Translations for polyglossia
\gappto\captionsgujarati{
  \renewcommand{\tablename}{કોષ્ટક}
  \renewcommand{\figurename}{આકૃતિ}
}

% Helper for TikZ nodes to ensure Gujarati font
\newcommand{\gu}[1]{{\gujaratifont #1}}

% Custom environments
\newtcolorbox{solutionbox}{
    breakable,
    enhanced,
    colback=solutioncolor!5!white,
    colframe=solutioncolor!75!black,
    fonttitle=\bfseries,
    title=જવાબ
}

\newtcolorbox{solutionboxnobreak}{
 colback=solutioncolor!5!white,
 colframe=solutioncolor!75!black,
 fonttitle=\bfseries,
 title=જવાબ
}

\newtcolorbox{keyformula}{
 breakable,
 enhanced,
 colback=keycolor!5!white,
 colframe=keycolor!75!black,
 fonttitle=\bfseries,
 title=રાસાયણિક સમીકરણ/સૂત્ર
}

\newtcolorbox{mnemonicbox}{
 breakable,
 enhanced,
 colback=mnemoniccolor!5!white,
 colframe=mnemoniccolor!75!black,
 fonttitle=\bfseries,
 title=મેમરી ટ્રીક
}


% Custom commands for GTU solutions
% This file defines semantic commands for consistent formatting

% Question command with automatic formatting
\newcommand{\question}[2]{%
  \section*{Question #1}%
  \textbf{#2}%
}

% OR question variant
\newcommand{\questionor}[2]{%
  \section*{Question #1 OR}%
  \textbf{#2}%
}

% Proper table environment with caption
\newenvironment{answertable}[1]{%
  \begin{table}[htbp]
  \centering
  \caption{#1}
}{%
  \end{table}
}

% Proper figure environment for diagrams
\newenvironment{answerdiagram}[1]{%
  \begin{figure}[htbp]
  \centering
  \caption{#1}
}{%
  \end{figure}
}

% Semantic markup for key terms
\newcommand{\keyword}[1]{\textbf{#1}}
\newcommand{\code}[1]{\texttt{#1}}
\newcommand{\classname}[1]{\texttt{#1}}
\newcommand{\methodname}[1]{\texttt{#1}}

% Proper quotation marks
\newcommand{\mnemonic}[1]{``#1''}


\title{આધુનિક ભૌતિકશાસ્ત્ર (DI01000061) - શિયાળો 2024 ઉકેલ}
\date{જાન્યુઆરી 9, 2025}

\begin{document}
\maketitle

\questionmarks{1}{14}{ખાલી જગ્યા પૂરો/બહુવિકલ્પ પ્રશ્નો}

\begin{solutionbox}
\begin{center}
\captionof{table}{MCQ જવાબો}
\begin{tabulary}{\linewidth}{|C|L|C|L|}
\hline
\textbf{ક્રમ} & \textbf{જવાબ} & \textbf{ક્રમ} & \textbf{જવાબ} \\ \hline
(1) & (a) Si & (8) & (b) 0.5 Hz \\ \hline
(2) & (a) 1.50 & (9) & (a) 300000 km/s \\ \hline
(3) & (b) વધારે & (10) & (b) ઘન \\ \hline
(4) & (c) 4 & (11) & (a) શૃંગ અને ગર્ત \\ \hline
(5) & (d) પૂર્ણ આંતરિક પરાવર્તન & (12) & (b) એકરંગી \\ \hline
(6) & (d) આવૃત્તિ & (13) & (a) સિંગલ મોડ \\ \hline
(7) & (a) કુલંબ & (14) & (b) 45\textdegree \\ \hline
\end{tabulary}
\end{center}
\end{solutionbox}

\begin{mnemonicbox}
\mnemonic{સિલિકોન ગ્લાસ બ્રિજ ઓપ્ટિક આવૃત્તિ કુલંબ Hz ઘન શૃંગ મોનો સિંગલ 45}
\end{mnemonicbox}

\questionmarks{2(A)}{6}{કોઈપણ બેના જવાબ આપો}

\questionmarks{2(A)(1)}{3}{ચોકસાઈ અને સચોટતા વચ્ચેનો તફાવત આપો.}

\begin{solutionbox}
\begin{center}
\captionof{table}{ચોકસાઈ અને સચોટતા તફાવત}
\begin{tabulary}{\linewidth}{|L|L|L|}
\hline
\textbf{પરિમાણ} & \textbf{ચોકસાઈ (Accuracy)} & \textbf{સચોટતા (Precision)} \\ \hline
વ્યાખ્યા & સાચા મૂલ્યની નજીક & પુનરાવર્તિત માપનોની સુસંગતતા \\ \hline
કેન્દ્ર & સાચું હોવું & પુનઃઉત્પાદન \\ \hline
ભૂલનો પ્રકાર & વ્યવસ્થિત ભૂલ & અવ્યવસ્થિત ભૂલ \\ \hline
ઉદાહરણ & લક્ષ્યમાં મારવું & સમાન જગ્યાએ વારંવાર મારવું \\ \hline
\end{tabulary}
\end{center}

\begin{itemize}
    \item \keyword{ચોકસાઈ}: માપ વાસ્તવિક મૂલ્યની કેટલી નજીક છે
    \item \keyword{સચોટતા}: પુનરાવર્તિત માપન એકબીજાની કેટલી નજીક છે
\end{itemize}
\end{solutionbox}

\begin{mnemonicbox}
\mnemonic{ચોકસાઈ વાસ્તવિક લક્ષ્ય, સચોટતા સુસંગત પુનરાવર્તન}
\end{mnemonicbox}

\questionmarks{2(A)(2)}{3}{માઇક્રોમીટર સ્ક્રૂ દ્વારા માપવામાં આવતા ગોળાનો વ્યાસ નક્કી કરો, મુખ્ય માપપટ્ટીનું માપ 5 mm અને વર્તુળાકાર માપપટ્ટીનો 50મો વિભાગ બેઝ લાઇન સાથે મેચ થાય છે. આ સાધનની લ.મા.શ 0.01 mm છે.}

\begin{solutionbox}
\textbf{આપેલ:}
\begin{itemize}
    \item મુખ્ય માપપટ્ટી વાંચન (MSR) = 5 mm
    \item વર્તુળાકાર માપપટ્ટી વાંચન (CSR) = 50 વિભાગ
    \item લઘુતમ માપશક્તિ (LC) = 0.01 mm
\end{itemize}

\textbf{સૂત્ર:}
\[ \text{કુલ વાંચન} = \text{MSR} + (\text{CSR} \times \text{LC}) \]

\textbf{ગણતરી:}
\begin{align*}
\text{કુલ વાંચન} &= 5 + (50 \times 0.01) \\
&= 5 + 0.5 \\
&= 5.5 \text{ mm}
\end{align*}

\textbf{ગોળાનો વ્યાસ = 5.5 mm}
\end{solutionbox}

\begin{mnemonicbox}
\mnemonic{મુખ્ય વાંચન + વર્તુળાકાર * લઘુતમ માપશક્તિ}
\end{mnemonicbox}

\questionmarks{2(A)(3)}{3}{જ્યારે 4 $\mu$F કેપેસિટન્સ ધરાવતા કેપેસિટરને 12 volt બેટરી સાથે જોડતા કેપેસિટરની બંને પ્લેટ પર સંગ્રહિત થતાં વિદ્યુતભારના જથ્થાની ગણતરી કરો.}

\begin{solutionbox}
\textbf{આપેલ:}
\begin{itemize}
    \item કેપેસિટન્સ ($C$) = 4 $\mu$F = $4 \times 10^{-6}$ F
    \item વોલ્ટેજ ($V$) = 12 V
\end{itemize}

\textbf{સૂત્ર:}
\[ Q = CV \]

\textbf{ગણતરી:}
\begin{align*}
Q &= 4 \times 10^{-6} \times 12 \\
Q &= 48 \times 10^{-6} \text{ C} \\
Q &= 48 \ \mu\text{C}
\end{align*}

\textbf{સંગ્રહિત વિદ્યુતભાર = 48 $\mu$C}
\end{solutionbox}

\begin{mnemonicbox}
\mnemonic{ચાર્જ બરાબર કેપેસિટન્સ ગુણ્યે વોલ્ટેજ}
\end{mnemonicbox}

\questionmarks{2(B)}{8}{કોઈપણ બેના જવાબ આપો}

\questionmarks{2(B)(1)}{4}{યોગ્ય નામકરણ સાથે માઇક્રોમીટર સ્ક્રૂ ગેજની આકૃતિ દોરો.}

\begin{solutionbox}
\begin{center}
\begin{tikzpicture}[x=1cm, y=1cm, scale=0.8, transform shape]
    % Frame
    \draw[thick, fill=gray!20] (0,0) arc (180:360:2.5) -- (5,0) -- (5,0.5) -- (0,0.5) -- cycle;
    \draw[thick, fill=gray!20] (0,0.5) -- (0,2);
    \draw[thick, fill=gray!20] (5,0.5) -- (5,2);
    
    % Anvil
    \draw[thick, fill=black!70] (0.2,1.5) rectangle (1,2);
    \node[below, font=\small] at (0.6,1.5) {એન્વિલ};

    % Spindle
    \draw[thick, fill=gray!40] (3.5,1.6) rectangle (1,1.9);
    \node[below, font=\small] at (2.25,1.6) {સ્પિંડલ};

    % Sleeve/Main Scale
    \draw[thick, fill=white] (3.5,1.4) rectangle (6.5,2.1);
    \draw (3.5,1.75) -- (6.5,1.75); % Baseline
    \foreach \x in {3.5, 4.0, ..., 6.0}
        \draw (\x, 1.75) -- (\x, 1.6);
    \foreach \x in {3.75, 4.25, ..., 6.25}
        \draw (\x, 1.75) -- (\x, 1.9);
    \node[below, font=\scriptsize] at (3.5,1.4) {0};
    \node[below, font=\scriptsize] at (4.5,1.4) {5};
    \node[below, font=\scriptsize] at (5.5,1.4) {10};
    \node[below, font=\small] at (5,1.2) {મુખ્ય સ્કેલ};

    % Thimble
    \draw[thick, fill=white] (6.5,1.3) rectangle (8.5,2.2);
    \foreach \y in {1.4, 1.5, ..., 2.1}
        \draw (6.5,\y) -- (6.8,\y);
    \node[right, font=\scriptsize] at (6.8,1.75) {0};
    \node[right, font=\scriptsize] at (6.8,2.0) {5};
    \node[below, font=\small] at (7.5,1.3) {થિમ્બલ};

    % Ratchet
    \draw[thick, fill=black!80] (8.5,1.5) rectangle (9,2);
    \draw[thick, fill=black!80] (9,1.6) rectangle (9.3,1.9);
    \node[right, font=\small] at (9.3,1.75) {રેચેટ};

    % Labels
    \node[font=\bfseries] at (2.5,-0.5) {ફ્રેમ};
\end{tikzpicture}
\captionof{figure}{માઇક્રોમીટર સ્ક્રૂ ગેજ}
\end{center}

\textbf{મુખ્ય ઘટકો:}
\begin{itemize}
    \item \keyword{ફ્રેમ}: U-આકારનું માળખું જે આધાર પૂરો પાડે
    \item \keyword{એન્વિલ}: વસ્તુ મૂકવા માટે સ્થિર જડબો
    \item \keyword{સ્પિંડલ}: ગતિશીલ સ્ક્રૂ મેકેનિઝમ
    \item \keyword{થિમ્બલ સ્કેલ}: 50 વિભાગ સાથે વર્તુળાકાર સ્કેલ
    \item \keyword{મુખ્ય સ્કેલ}: mm માં રેખીય સ્કેલ
    \item \keyword{રેચેટ}: સુસંગત દબાણ લાગુ કરવા માટે
\end{itemize}
\end{solutionbox}

\begin{mnemonicbox}
\mnemonic{ફ્રેમ એન્વિલ સ્પિંડલ થિમ્બલ મુખ્ય રેચેટ}
\end{mnemonicbox}

\questionmarks{2(B)(2)}{4}{વર્નિયર કેલિપર્સ માટે યોગ્ય આકૃતિ સાથે શૂન્ય, ધન અને ઋણ ત્રુટીઓ સમજાવો અને આ પ્રકારની ત્રુટીઓ દૂર કરવા માટેના જરૂરી પગલાંની યાદી બનાવો.}

\begin{solutionbox}
\textbf{ત્રુટીના પ્રકારો:}
\begin{center}
\captionof{table}{વર્નિયર કેલિપર્સ ત્રુટીઓ}
\begin{tabulary}{\linewidth}{|L|L|L|}
\hline
\textbf{ત્રુટીનો પ્રકાર} & \textbf{સ્થિતિ} & \textbf{વાંચન} \\ \hline
શૂન્ય ત્રુટિ & વર્નિયરની શૂન્ય રેખા મુખ્ય સ્કેલની શૂન્ય સાથે મેળ ખાતી નથી & જડબા બંધ હોય ત્યારે શૂન્ય અલાવાનું વાંચન \\ \hline
ધન ત્રુટિ & વર્નિયર શૂન્ય મુખ્ય સ્કેલ શૂન્યની જમણી બાજુએ & સુધારો ઉમેરો \\ \hline
ઋણ ત્રુટિ & વર્નિયર શૂન્ય મુખ્ય સ્કેલ શૂન્યની ડાબી બાજુએ & સુધારો બાદ કરો \\ \hline
\end{tabulary}
\end{center}

\textbf{આકૃતિ:}
\begin{center}
\begin{tikzpicture}[scale=0.8]
    % Zero Error
    \node[font=\bfseries] at (2, 2.5) {શૂન્ય ત્રુટિ નથી};
    \draw (0,1.5) rectangle (4,2.2); % Main
    \foreach \x in {0,0.5,...,4} \draw (\x,1.5) -- (\x,1.7);
    \node[above] at (1,2.2) {0};
    \draw (0,0.8) rectangle (4,1.5); % Vernier
    \foreach \x in {0,0.5,...,4} \draw (\x,1.5) -- (\x,1.3);
    \node[below] at (1,0.8) {0};
    \node at (2, 1.5) [draw, circle, red, inner sep=2pt] {};

    % Positive Error
    \node[font=\bfseries] at (7, 2.5) {ધન ત્રુટિ};
    \draw (5,1.5) rectangle (9,2.2); % Main
    \foreach \x in {5,5.5,...,9} \draw (\x,1.5) -- (\x,1.7);
    \node[above] at (6,2.2) {0};
    \draw (5,0.8) rectangle (9,1.5); % Vernier
    \foreach \x in {5,5.5,...,9} \draw (\x+0.2,1.5) -- (\x+0.2,1.3); % Shifted right
    \node[below] at (6.2,0.8) {0};
    \draw[->, blue] (6,1.1) -- (6.2,1.1);

    % Negative Error
    \node[font=\bfseries] at (12, 2.5) {ઋણ ત્રુટિ};
    \draw (10,1.5) rectangle (14,2.2); % Main
    \foreach \x in {10,10.5,...,14} \draw (\x,1.5) -- (\x,1.7);
    \node[above] at (11,2.2) {0};
    \draw (10,0.8) rectangle (14,1.5); % Vernier
    \foreach \x in {10,10.5,...,14} \draw (\x-0.2,1.5) -- (\x-0.2,1.3); % Shifted left
    \node[below] at (10.8,0.8) {0};
    \draw[->, blue] (11,1.1) -- (10.8,1.1);
\end{tikzpicture}
\captionof{figure}{શૂન્ય ત્રુટીઓ}
\end{center}

\textbf{ત્રુટીઓ દૂર કરવાના પગલાં:}
\begin{itemize}
    \item \keyword{શૂન્ય ત્રુટિ તપાસો} માપન પહેલાં
    \item \keyword{સુધારો લાગુ કરો} અંતિમ વાંચનમાં
    \item \keyword{જડબાઓ સાફ કરો} કચરો અટકાવવા માટે
    \item \keyword{સાવચેતીથી હાથ વણો} યાંત્રિક નુકસાન ટાળવા માટે
\end{itemize}
\end{solutionbox}

\begin{mnemonicbox}
\mnemonic{તપાસો સાફ કરો સુધારો સાવચેતી}
\end{mnemonicbox}

\questionmarks{2(B)(3)}{4}{સાદા લોલકનો આવર્તકાળ શોધવાના પ્રયોગમાં અવલોકનો 1.96 s, 1.98 s, 2.00 s, 2.02 s, 2.04 s છે. નિરપેક્ષ ત્રુટિ, સરેરાશ નિરપેક્ષ ત્રુટિ, સાપેક્ષ ત્રુટિ અને પ્રતિશત ત્રુટિની ગણતરી કરો.}

\begin{solutionbox}
\textbf{અવલોકનો:} 1.96, 1.98, 2.00, 2.02, 2.04 s

\textbf{ગણતરી:}

1. \textbf{સરેરાશ મૂલ્ય}:
\[ \bar{x} = \frac{1.96 + 1.98 + 2.00 + 2.02 + 2.04}{5} = \frac{10.00}{5} = 2.00 \text{ s} \]

2. \textbf{નિરપેક્ષ ત્રુટીઓ} ($|\Delta x_i| = |x_i - \bar{x}|$):
\begin{itemize}
    \item $|1.96 - 2.00| = 0.04$ s
    \item $|1.98 - 2.00| = 0.02$ s
    \item $|2.00 - 2.00| = 0.00$ s
    \item $|2.02 - 2.00| = 0.02$ s
    \item $|2.04 - 2.00| = 0.04$ s
\end{itemize}

3. \textbf{સરેરાશ નિરપેક્ષ ત્રુટિ}:
\[ \overline{\Delta x} = \frac{0.04 + 0.02 + 0.00 + 0.02 + 0.04}{5} = \frac{0.12}{5} = 0.024 \text{ s} \]

4. \textbf{સાપેક્ષ ત્રુટિ}:
\[ \delta x = \frac{\overline{\Delta x}}{\bar{x}} = \frac{0.024}{2.00} = 0.012 \]

5. \textbf{પ્રતિશત ત્રુટિ}:
\[ \% \text{ ત્રુટિ} = 0.012 \times 100 = 1.2\% \]

\textbf{પરિણામો:} સરેરાશ નિરપેક્ષ ત્રુટિ = 0.024 s, સાપેક્ષ ત્રુટિ = 0.012, પ્રતિશત ત્રુટિ = 1.2\%
\end{solutionbox}

\begin{mnemonicbox}
\mnemonic{સરેરાશ નિરપેક્ષ સાપેક્ષ પ્રતિશત}
\end{mnemonicbox}

\questionmarks{3(A)}{6}{કોઈપણ બેના જવાબ આપો}

\questionmarks{3(A)(1)}{3}{વ્યાખ્યાઓ કરો: વિદ્યુત ફ્લક્સ, વિદ્યુતક્ષેત્ર, વીજસ્થિતિમાનનો તફાવત}

\begin{solutionbox}
\begin{center}
\captionof{table}{વ્યાખ્યાઓ}
\begin{tabulary}{\linewidth}{|L|L|C|L|}
\hline
\textbf{શબ્દ} & \textbf{વ્યાખ્યા} & \textbf{એકમ} & \textbf{સૂત્ર} \\ \hline
વિદ્યુત ફ્લક્સ & સપાટીમાંથી પસાર થતી વિદ્યુત ક્ષેત્ર રેખાઓની સંખ્યા & Nm$^2$/C & $\Phi = E \cdot A$ \\ \hline
વિદ્યુતક્ષેત્ર & એકમ ધન આવેશ પર લાગતું બળ & N/C & $E = F/q$ \\ \hline
વીજસ્થિતિમાનનો તફાવત & બે બિંદુઓ વચ્ચે એકમ આવેશ દીઠ કામ & વોલ્ટ & $V = W/q$ \\ \hline
\end{tabulary}
\end{center}

\begin{itemize}
    \item \keyword{વિદ્યુત ફ્લક્સ}: સપાટીમાં પ્રવેશતી ક્ષેત્ર રેખાઓનું માપ
    \item \keyword{વિદ્યુતક્ષેત્ર}: વિદ્યુત બળ ક્રિયા કરતો વિસ્તાર
    \item \keyword{વીજસ્થિતિમાનનો તફાવત}: એકમ આવેશ દીઠ ઊર્જાનો તફાવત
\end{itemize}
\end{solutionbox}

\begin{mnemonicbox}
\mnemonic{ફ્લક્સ ક્ષેત્ર બળ, કામ વોટ્સ વોલ્ટ્સ}
\end{mnemonicbox}

\questionmarks{3(A)(2)}{3}{જ્યારે ત્રણ જુદા જુદા કેપેસિટરોને શ્રેણીમાં જોડવામાં આવે ત્યારે જરૂરી સર્કિટ ડાયાગ્રામ સાથે સમકક્ષ કેપેસિટન્સ માટેનું સૂત્ર મેળવો.}

\begin{solutionbox}
\textbf{સર્કિટ ડાયાગ્રામ:}
\begin{center}
\begin{tikzpicture}
    \draw (0,0) to[battery1, l=$V$] (0,2) -- (1,2)
        to[C, l=$C_1$] (2.5,2)
        to[C, l=$C_2$] (4,2)
        to[C, l=$C_3$] (5.5,2) -- (6.5,2) -- (6.5,0) -- (0,0);
\end{tikzpicture}
\captionof{figure}{કેપેસિટરો શ્રેણીમાં}
\end{center}

\textbf{વ્યુત્પત્તિ:}
\begin{itemize}
    \item \keyword{સમાન આવેશ} $Q$ દરેક કેપેસિટર દ્વારા વહે છે.
    \item \keyword{વોલ્ટેજ વિભાજન}: $V = V_1 + V_2 + V_3$
    \item દરેક કેપેસિટર માટે: $V_1 = Q/C_1, V_2 = Q/C_2, V_3 = Q/C_3$
    \item કુલ વોલ્ટેજ:
    \[ V = \frac{Q}{C_1} + \frac{Q}{C_2} + \frac{Q}{C_3} = Q \left( \frac{1}{C_1} + \frac{1}{C_2} + \frac{1}{C_3} \right) \]
    \item સમકક્ષ કેપેસિટર $C_s$ માટે: $V = Q/C_s$
    \item સમીકરણોની સરખામણી:
    \[ \frac{1}{C_s} = \frac{1}{C_1} + \frac{1}{C_2} + \frac{1}{C_3} \]
\end{itemize}

\textbf{સૂત્ર:}
\[ \frac{1}{C_s} = \frac{1}{C_1} + \frac{1}{C_2} + \frac{1}{C_3} \]
\end{solutionbox}

\begin{mnemonicbox}
\mnemonic{શ્રેણી વિપરીત સરવાળો, સમાન આવેશ વિભાજિત વોલ્ટેજ}
\end{mnemonicbox}

\questionmarks{3(A)(3)}{3}{વ્યાખ્યાઓ કરો: ઇન્ફ્રાસોનિક ધ્વનિ, શ્રાવ્ય ધ્વનિ, અલ્ટ્રાસોનિક ધ્વનિ}

\begin{solutionbox}
\begin{center}
\captionof{table}{ધ્વનિના પ્રકારો}
\begin{tabulary}{\linewidth}{|L|L|L|L|}
\hline
\textbf{ધ્વનિનો પ્રકાર} & \textbf{આવૃત્તિ શ્રેણી} & \textbf{લાક્ષણિકતાઓ} & \textbf{ઉપયોગો} \\ \hline
ઇન્ફ્રાસોનિક & 20 Hz થી નીચે & મનુષ્યને સંભળાતું નથી & ભૂકંપ શોધ \\ \hline
શ્રાવ્ય & 20 Hz થી 20 kHz & મનુષ્યને સંભળાય છે & વાતચીત, સંગીત \\ \hline
અલ્ટ્રાસોનિક & 20 kHz થી ઉપર & મનુષ્યને સંભળાતું નથી & તબીબી ઇમેજિંગ, SONAR \\ \hline
\end{tabulary}
\end{center}

\begin{itemize}
    \item \keyword{ઇન્ફ્રાસોનિક}: માનવ શ્રવણથી નીચેની ઓછી આવૃત્તિ
    \item \keyword{શ્રાવ્ય}: માનવો માટે સામાન્ય શ્રવણ શ્રેણી
    \item \keyword{અલ્ટ્રાસોનિક}: માનવ શ્રવણથી ઉપરની ઊંચી આવૃત્તિ
\end{itemize}
\end{solutionbox}

\begin{mnemonicbox}
\mnemonic{ઇન્ફ્રા-નીચે, શ્રાવ્ય-વચ્ચે, અલ્ટ્રા-ઉપર}
\end{mnemonicbox}

\questionmarks{3(B)}{8}{કોઈપણ બેના જવાબ આપો}

\questionmarks{3(B)(1)}{4}{સમાંતર પ્લેટ કેપેસિટર માટે $C = \epsilon_0 A/d$ સાબિત કરો.}

\begin{solutionbox}
\textbf{આકૃતિ:}
\begin{center}
\begin{tikzpicture}
    % Plate 1
    \draw[thick, fill=gray!20] (0, 2) rectangle (4, 2.2);
    \node at (2, 2.1) {$+Q$};
    \node[left] at (0, 2.1) {પ્લેટ 1 (ક્ષેત્રફળ $A$)};
    
    % Plate 2
    \draw[thick, fill=gray!20] (0, 0) rectangle (4, 0.2);
    \node at (2, 0.1) {$-Q$};
    \node[left] at (0, 0.1) {પ્લેટ 2 (ક્ષેત્રફળ $A$)};
    
    % Field Lines
    \foreach \x in {0.5, 1.0, ..., 3.5}
        \draw[->] (\x, 2) -- (\x, 0.2);
    \node at (4.5, 1.1) {$\vec{E}$};
    
    % Distance
    \draw[<->] (4.2, 0.2) -- (4.2, 2);
    \node[right] at (4.2, 1.1) {$d$};
\end{tikzpicture}
\captionof{figure}{સમાંતર પ્લેટ કેપેસિટર}
\end{center}

\textbf{વ્યુત્પત્તિ:}
\begin{itemize}
    \item પ્લેટો વચ્ચે \keyword{વિદ્યુતક્ષેત્ર}: $E = \sigma/\epsilon_0 = Q/(\epsilon_0 A)$
    \item \keyword{વીજસ્થિતિમાનનો તફાવત}: $V = E \times d = Qd/(\epsilon_0 A)$
    \item \keyword{કેપેસિટન્સ વ્યાખ્યા}: $C = Q/V$
    \item $V$ ની કિંમત મૂકતા:
    \[ C = \frac{Q}{Qd/(\epsilon_0 A)} = \frac{\epsilon_0 A}{d} \]
\end{itemize}

\textbf{અંતિમ સૂત્ર:}
\[ C = \frac{\epsilon_0 A}{d} \]

જ્યાં:
\begin{itemize}
    \item $\epsilon_0$: મુક્ત અવકાશની પરમિટિવિટી
    \item $A$: પ્લેટોનું ક્ષેત્રફળ
    \item $d$: પ્લેટો વચ્ચેનું અંતર
\end{itemize}
\end{solutionbox}

\begin{mnemonicbox}
\mnemonic{કેપેસિટન્સ બરાબર એપસીલોન-ઝીરો એરિયા ભાગ્યા અંતર}
\end{mnemonicbox}

\questionmarks{3(B)(2)}{4}{વિદ્યુત ક્ષેત્ર રેખાઓની લાક્ષણિકતાઓ જણાવો.}

\begin{solutionbox}
\textbf{મુખ્ય લાક્ષણિકતાઓ:}
\begin{enumerate}
    \item \keyword{દિશા}: ધન વીજભારથી શરૂ થાય છે અને ઋણ વીજભાર પર સમાપ્ત થાય છે.
    \item \keyword{ગીચતા}: રેખાઓની નિકટતા ક્ષેત્રની પ્રબળતા સૂચવે છે (ગીચ = પ્રબળ).
    \item \keyword{સતત}: તે વીજભાર મુક્ત વિસ્તારમાં તૂટ્યા વિના સતત વણાંકો છે.
    \item \keyword{છેદતી નથી}: બે ક્ષેત્ર રેખાઓ ક્યારેય એકબીજાને છેદતી નથી (નહિતર એક બિંદુ પર બે દિશાઓ હોય).
    \item \keyword{લંબ}: તે હંમેશા વિદ્યુતભારિત સુવાહકની સપાટીને લંબ હોય છે.
    \item \keyword{બંધ લૂપ}: તે બંધ લૂપ રચતી નથી (સ્થિત વિદ્યુત ક્ષેત્ર સંરક્ષી છે).
    \item \keyword{સ્પર્શક}: કોઈપણ બિંદુએ રેખાનો સ્પર્શક વિદ્યુત ક્ષેત્રની દિશા આપે છે.
\end{enumerate}
\end{solutionbox}

\begin{mnemonicbox}
\mnemonic{ધન થી ઋણ, ગીચ એટલે પ્રબળ, ક્યારેય છેદે નહીં, હંમેશા લંબ}
\end{mnemonicbox}

\questionmarks{3(B)(3)}{4}{અલ્ટ્રાસોનિક તરંગોના ઉત્પાદન માટે વપરાતી મેગ્નેટોસ્ટ્રિક્શન પદ્ધતિની કાર્યપદ્ધતિ અને રચના વર્ણવો.}

\begin{solutionbox}
\textbf{રચના:}
\begin{center}
\begin{tikzpicture}[node distance=1.5cm, auto]
    \node [gtu block] (osc) {AC ઓસિલેટર};
    \node [gtu block, right=1cm of osc] (coil) {કોઇલ};
    \node [gtu block, right=1cm of coil] (rod) {નિકલ સળિયો};
    \node [gtu block, right=1cm of rod] (horn) {હોર્ન};
    
    \path [gtu arrow] (osc) -- (coil);
    \path [gtu arrow] (coil) -- node[above] {ચુંબકીય ક્ષેત્ર} (rod);
    \path [gtu arrow] (rod) -- node[above] {કંપનો} (horn);
    \path [gtu arrow] (horn) -- node[above] {અલ્ટ્રાસોનિક તરંગો} +(3,2);
\end{tikzpicture}
\captionof{figure}{મેગ્નેટોસ્ટ્રિક્શન ઓસિલેટર બ્લોક ડાયાગ્રામ}
\end{center}

\textbf{ઘટકો:}
\begin{itemize}
    \item \keyword{નિકલ સળિયો}: ફેરોમેગ્નેટિક સામગ્રી જે મેગ્નેટોસ્ટ્રિક્શન અસર દર્શાવે છે.
    \item \keyword{કોઇલ}: ચુંબકીય ક્ષેત્ર ઉત્પન્ન કરવા માટે સળિયાની આસપાસ વીંટાળેલ સોલેનોઇડ.
    \item \keyword{AC ઓસિલેટર}: ઉચ્ચ-આવૃત્તિના અલ્ટરનેટિંગ કરંટનો સ્ત્રોત.
    \item \keyword{હોર્ન}: ધ્વનિ ઊર્જાને કાર્યક્ષમ રીતે પ્રસારિત કરવા માટે.
\end{itemize}

\textbf{કાર્ય સિદ્ધાંત:}
\begin{itemize}
    \item જ્યારે \keyword{AC પ્રવાહ} કોઇલમાંથી વહે છે, ત્યારે ઝડપથી બદલાતું \keyword{ચુંબકીય ક્ષેત્ર} ઉત્પન્ન થાય છે.
    \item ફેરોમેગ્નેટિક સળિયો લાગુ AC ની બમણી આવૃત્તિએ \keyword{મેગ્નેટોસ્ટ્રિક્શન} (લંબાઈમાં ફેરફાર) અનુભવે છે.
    \item આ સળિયામાં \keyword{યાંત્રિક કંપનો} ઉત્પન્ન કરે છે.
    \item જો આવૃત્તિ સળિયાની પ્રાકૃતિક આવૃત્તિ સાથે મેળ ખાય, તો અનુનાદ થાય છે, જે ઉચ્ચ-તીવ્રતાના \keyword{અલ્ટ્રાસોનિક તરંગો} ઉત્પન્ન કરે છે.
\end{itemize}
\end{solutionbox}

\begin{mnemonicbox}
\mnemonic{AC કોઇલ નિકલ કંપાવે, અલ્ટ્રાસોનિક બનાવે}
\end{mnemonicbox}

\questionmarks{4(A)}{6}{કોઈપણ બેના જવાબ આપો}

\questionmarks{4(A)(1)}{3}{એક રેડિયો સ્ટેશન તેના રેડિયો સિગ્નલો $9.26 \times 10^7$ Hz પર પ્રસારિત કરે છે. જો તરંગો $3.00 \times 10^8$ m/s ની ઝડપે મુસાફરી કરતા હોય તો તરંગલંબાઇ શોધો.}

\begin{solutionbox}
\textbf{આપેલ:}
\begin{itemize}
    \item આવૃત્તિ ($f$) = $9.26 \times 10^7$ Hz
    \item ઝડપ ($c$) = $3.00 \times 10^8$ m/s
\end{itemize}

\textbf{સૂત્ર:}
\[ c = f\lambda \implies \lambda = \frac{c}{f} \]

\textbf{ગણતરી:}
\begin{align*}
\lambda &= \frac{3.00 \times 10^8}{9.26 \times 10^7} \\
\lambda &= \frac{300}{92.6} \times 10^0 \\
\lambda &\approx 3.24 \text{ m}
\end{align*}

\textbf{તરંગલંબાઇ = 3.24 m}
\end{solutionbox}

\begin{mnemonicbox}
\mnemonic{ઝડપ બરાબર આવૃત્તિ ગુણ્યે તરંગલંબાઇ}
\end{mnemonicbox}

\questionmarks{4(A)(2)}{3}{સ્નેલનો નિયમ લખો અને માધ્યમનો વક્રીભવનાંક સમજાવો.}

\begin{solutionbox}
\textbf{સ્નેલનો નિયમ:}
આપેલ માધ્યમોની જોડ માટે આપાતકોણના સાઈન અને વક્રીભૂતકોણના સાઈનનો ગુણોત્તર અચળ હોય છે.
\[ n_1 \sin \theta_1 = n_2 \sin \theta_2 \]
જ્યાં:
\begin{itemize}
    \item $n_1, n_2$: માધ્યમ 1 અને 2 ના વક્રીભવનાંક
    \item $\theta_1, \theta_2$: આપાતકોણ અને વક્રીભૂતકોણ
\end{itemize}

\textbf{વક્રીભવનાંક:}
\begin{center}
\captionof{table}{વક્રીભવનાંક પ્રકારો}
\begin{tabulary}{\linewidth}{|L|L|L|}
\hline
\textbf{પ્રકાર} & \textbf{વ્યાખ્યા} & \textbf{સૂત્ર} \\ \hline
નિરપેક્ષ & શૂન્યાવકાશમાં પ્રકાશની ઝડપ અને માધ્યમમાં ઝડપનો ગુણોત્તર & $n = c/v$ \\ \hline
સાપેક્ષ & બે માધ્યમોમાં પ્રકાશની ઝડપનો ગુણોત્તર & $n_{21} = v_1/v_2$ \\ \hline
\end{tabulary}
\end{center}
વધારે વક્રીભવનાંક એટલે પ્રકાશીય રીતે ઘટ્ટ માધ્યમ જ્યાં પ્રકાશ ધીમો ચાલે છે.
\end{solutionbox}

\begin{mnemonicbox}
\mnemonic{સ્નેલ કહે સાઈન ગુણોત્તર અચળ, ઘટ્ટ ધીમો પાડે પ્રકાશ}
\end{mnemonicbox}

\questionmarks{4(A)(3)}{3}{સરખાવો: સામાન્ય પ્રકાશ અને LASER}

\begin{solutionbox}
\begin{center}
\captionof{table}{સામાન્ય પ્રકાશ અને LASER}
\begin{tabulary}{\linewidth}{|L|L|L|}
\hline
\textbf{ગુણધર્મ} & \textbf{સામાન્ય પ્રકાશ} & \textbf{LASER} \\ \hline
સુસંગતતા & અસંગત & સુસંગત (Coherent) \\ \hline
રંગ & બહુરંગી (Polychromatic) & એકરંગી (Monochromatic) \\ \hline
દિશા & ફેલાયેલ (Divergent) & સમાંતર બીમ \\ \hline
તીવ્રતા & ઓછી & ખૂબ વધારે \\ \hline
કળા & યદચ્છ & નિશ્ચિત કળા સંબંધ \\ \hline
તરંગલંબાઇ & અનેક તરંગલંબાઇ & એક જ તરંગલંબાઇ \\ \hline
\end{tabulary}
\end{center}
\end{solutionbox}

\begin{mnemonicbox}
\mnemonic{LASER: સુસંગત એકરંગી સમાંતર તીવ્ર}
\end{mnemonicbox}

\questionmarks{4(B)}{8}{કોઈપણ બેના જવાબ આપો}

\questionmarks{4(B)(1)}{4}{જરૂરી આકૃતિ સાથે ઓપ્ટિકલ ફાઇબરની રચના સમજાવો.}

\begin{solutionbox}
\textbf{ઓપ્ટિકલ ફાઇબર રચના:}
\begin{center}
\begin{tikzpicture}
    % Core
    \fill[cyan!20] (0,1) rectangle (8,2);
    \draw[thick] (0,1) -- (8,1);
    \draw[thick] (0,2) -- (8,2);
    \node at (4,1.5) {કોર (Core) ($n_1$)};
    
    % Cladding
    \fill[gray!20] (0,0.5) rectangle (8,1);
    \fill[gray!20] (0,2) rectangle (8,2.5);
    \draw[thick] (0,0.5) -- (8,0.5);
    \draw[thick] (0,2.5) -- (8,2.5);
    \node at (4,2.25) {ક્લેડીંગ (Cladding) ($n_2, n_1 > n_2$)};
    \node at (4,0.75) {ક્લેડીંગ};
    
    % Jacket
    \draw[thick, pattern=north east lines] (0,0) rectangle (8,0.5);
    \draw[thick, pattern=north east lines] (0,2.5) rectangle (8,3);
    \node[right] at (8,2.75) {રક્ષણાત્મક જેકેટ};
    \node[right] at (8,0.25) {રક્ષણાત્મક જેકેટ};

    % Light ray
    \draw[red, thick, ->] (0,1.5) -- (1,2) -- (2,1) -- (3,2) -- (4,1);
    \node[red, right] at (4,1) {પ્રકાશ સંકેત (TIR)};
\end{tikzpicture}
\captionof{figure}{ઓપ્ટિકલ ફાઇબરની રચના}
\end{center}

\textbf{ઘટકો:}
\begin{center}
\captionof{table}{ફાઇબરના ઘટકો}
\begin{tabulary}{\linewidth}{|L|L|L|L|}
\hline
\textbf{ઘટક} & \textbf{સામગ્રી} & \textbf{કાર્ય} & \textbf{વક્રીભવનાંક} \\ \hline
કોર & ગ્લાસ/પ્લાસ્ટિક & પ્રકાશ પ્રસારણ & વધારે ($n_1$) \\ \hline
ક્લેડીંગ & ગ્લાસ & પૂર્ણ આંતરિક પરાવર્તન & ઓછો ($n_2$) \\ \hline
જેકેટ & પ્લાસ્ટિક & રક્ષણ & - \\ \hline
\end{tabulary}
\end{center}

\textbf{કાર્ય સિદ્ધાંત:} પ્રકાશ કોરમાં \keyword{પૂર્ણ આંતરિક પરાવર્તન (TIR)} દ્વારા મુસાફરી કરે છે કારણ કે $n_1 > n_2$.
\end{solutionbox}

\begin{mnemonicbox}
\mnemonic{કોર ક્લેડીંગ જેકેટ, વધારે ઓછો રક્ષણ}
\end{mnemonicbox}

\questionmarks{4(B)(2)}{4}{એન્જિનિયરિંગ અને મેડિકલ ક્ષેત્રે LASER ના ઉપયોગો લખો.}

\begin{solutionbox}
\textbf{એન્જિનિયરિંગ ઉપયોગો:}
\begin{itemize}
    \item \keyword{કટિંગ અને વેલ્ડિંગ}: ધાતુનું સચોટ કટિંગ.
    \item \keyword{3D પ્રિન્ટીંગ}: લેસર સિન્ટરિંગ.
    \item \keyword{માપન}: અંતર માપન અને સર્વેક્ષણ (LIDAR).
    \item \keyword{કોમ્યુનિકેશન}: ઓપ્ટિકલ ફાઇબર સિસ્ટમ્સ.
    \item \keyword{સામગ્રી પ્રક્રિયા}: સપાટી સખત કરવી.
    \item \keyword{બારકોડ સ્કેનિંગ}: રિટેલ અને ઇન્વેન્ટરી.
\end{itemize}

\textbf{મેડિકલ ઉપયોગો:}
\begin{itemize}
    \item \keyword{સર્જરી}: સચોટ પેશી કાપવા (રક્તહીન શસ્ત્રક્રિયા).
    \item \keyword{આંખની સારવાર}: LASIK સુધારાત્મક શસ્ત્રક્રિયા.
    \item \keyword{કેન્સર સારવાર}: ગાંઠનો નાશ.
    \item \keyword{નિદાન}: સ્પેક્ટ્રોસ્કોપી.
    \item \keyword{દંત ચિકિત્સા}: કેવીટી સારવાર.
    \item \keyword{ત્વચા સારવાર}: કોસ્મેટિક પ્રક્રિયાઓ (વાળ દૂર કરવા).
\end{itemize}
\end{solutionbox}

\begin{mnemonicbox}
\mnemonic{એન્જિનિયરિંગ: કાપવું વેલ્ડ માપવું, મેડિકલ: સર્જરી આંખ કેન્સર સારવાર}
\end{mnemonicbox}

\questionmarks{4(B)(3)}{4}{P-type અને N-type અર્ધવાહકો સમજાવો.}

\begin{solutionbox}
\begin{center}
\captionof{table}{N-type અને P-type અર્ધવાહકો}
\begin{tabulary}{\linewidth}{|L|L|L|}
\hline
\textbf{ગુણધર્મ} & \textbf{N-type} & \textbf{P-type} \\ \hline
ડોપન્ટ & ફોસ્ફરસ, આર્સેનિક (પેન્ટાવેલેન્ટ) & બોરોન, એલ્યુમિનિયમ (ટ્રાઈવેલેન્ટ) \\ \hline
મેજોરિટી કેરિયર્સ & ઇલેક્ટ્રોન & હોલ્સ \\ \hline
માઈનોરિટી કેરિયર્સ & હોલ્સ & ઇલેક્ટ્રોન \\ \hline
વીજભાર & વિદ્યુત તટસ્થ & વિદ્યુત તટસ્થ \\ \hline
રચના & ડોનર અશુદ્ધિ ઇલેક્ટ્રોન ઉમેરે છે & એક્સેપ્ટર અશુદ્ધિ હોલ્સ બનાવે છે \\ \hline
\end{tabulary}
\end{center}
બંને પ્રકારો શુદ્ધ અર્ધવાહકો (જેમ કે Si અથવા Ge) માં ચોક્કસ અશુદ્ધિઓ ઉમેરીને (\keyword{ડોપિંગ}) વાહકતા વધારવા માટે રચાય છે.
\end{solutionbox}

\begin{mnemonicbox}
\mnemonic{N-type નેગેટિવ ઇલેક્ટ્રોન, P-type પોઝિટિવ હોલ્સ}
\end{mnemonicbox}

\questionmarks{5(A)}{6}{કોઈપણ બેના જવાબ આપો}

\questionmarks{5(A)(1)}{3}{એનર્જી બેન્ડ ગેપના આધારે સુવાહક, અર્ધવાહક અને અવાહકનું વર્ગીકરણ કરો.}

\begin{solutionbox}
\begin{center}
\captionof{table}{પદાર્થોનું વર્ગીકરણ}
\begin{tabulary}{\linewidth}{|L|L|L|}
\hline
\textbf{પદાર્થ} & \textbf{એનર્જી બેન્ડ ગેપ} & \textbf{લાક્ષણિકતાઓ} \\ \hline
સુવાહક & ગેપ નથી (0 eV) & વેલેન્સ અને કન્ડક્શન બેન્ડ ઓવરલેપ થાય છે \\ \hline
અર્ધવાહક & નાનો ગેપ (1-3 eV) & મધ્યમ બેન્ડ ગેપ \\ \hline
અવાહક & મોટો ગેપ (>3 eV) & પહોળો બેન્ડ ગેપ \\ \hline
\end{tabulary}
\end{center}

\textbf{એનર્જી બેન્ડ ડાયાગ્રામ:}
\begin{center}
\begin{tikzpicture}[scale=0.7]
    % Conductor
    \node at (2,4) {સુવાહક};
    \fill[gray!30] (0,0) rectangle (4,2.2); % VB
    \fill[gray!50] (0,1.8) rectangle (4,3.5); % CB overlap
    \draw (0,0) rectangle (4,2.2);
    \draw (0,1.8) rectangle (4,3.5);
    \node at (2,1) {VB};
    \node at (2,3) {CB};
    \node[right] at (4,2) {ઓવરલેપ};

    % Semiconductor
    \node at (7,4) {અર્ધવાહક};
    \fill[gray!30] (5,0) rectangle (9,1.5); % VB
    \fill[gray!50] (5,2.5) rectangle (9,3.5); % CB
    \draw (5,0) rectangle (9,1.5);
    \draw (5,2.5) rectangle (9,3.5);
    \node at (7,0.75) {VB};
    \node at (7,3) {CB};
    \node at (7,2) {ગેપ $\approx$ 1 eV};

    % Insulator
    \node at (12,4) {અવાહક};
    \fill[gray!30] (10,0) rectangle (14,1.5); % VB
    \fill[gray!50] (10,3.5) rectangle (14,4.5); % CB
    \draw (10,0) rectangle (14,1.5);
    \draw (10,3.5) rectangle (14,4.5);
    \node at (12,0.75) {VB};
    \node at (12,4) {CB};
    \node at (12,2.5) {ગેપ > 3 eV};
\end{tikzpicture}
\captionof{figure}{એનર્જી બેન્ડ ડાયાગ્રામ}
\end{center}
\end{solutionbox}

\begin{mnemonicbox}
\mnemonic{ગેપ નથી સુવાહક, નાનો ગેપ અર્ધ, મોટો ગેપ અવાહક}
\end{mnemonicbox}

\questionmarks{5(A)(2)}{3}{જરૂરી ટ્રુથ ટેબલ સાથે OR અને AND લોજિક ગેટ સમજાવો.}

\begin{solutionbox}
\begin{center}
\begin{tabular}{cc}
\begin{minipage}{0.45\linewidth}
\begin{center}
\textbf{OR ગેટ}
\[ Y = A + B \]
\begin{tikzpicture}
    \node[or port] (or) at (0,0) {};
    \draw (or.in 1) -- ++(-0.5,0) node[left] {A};
    \draw (or.in 2) -- ++(-0.5,0) node[left] {B};
    \draw (or.out) -- ++(0.5,0) node[right] {Y};
\end{tikzpicture}

\begin{tabular}{|c|c|c|}
\hline
A & B & Y \\ \hline
0 & 0 & 0 \\ \hline
0 & 1 & 1 \\ \hline
1 & 0 & 1 \\ \hline
1 & 1 & 1 \\ \hline
\end{tabular}
\end{center}
\end{minipage}
&
\begin{minipage}{0.45\linewidth}
\begin{center}
\textbf{AND ગેટ}
\[ Y = A \cdot B \]
\begin{tikzpicture}
    \node[and port] (and) at (0,0) {};
    \draw (and.in 1) -- ++(-0.5,0) node[left] {A};
    \draw (and.in 2) -- ++(-0.5,0) node[left] {B};
    \draw (and.out) -- ++(0.5,0) node[right] {Y};
\end{tikzpicture}

\begin{tabular}{|c|c|c|}
\hline
A & B & Y \\ \hline
0 & 0 & 0 \\ \hline
0 & 1 & 0 \\ \hline
1 & 0 & 0 \\ \hline
1 & 1 & 1 \\ \hline
\end{tabular}
\end{center}
\end{minipage}
\end{tabular}
\end{center}
\end{solutionbox}

\begin{mnemonicbox}
\mnemonic{OR: કોઈ પણ એક, AND: બધા જ}
\end{mnemonicbox}

\questionmarks{5(A)(3)}{3}{વોલ્ટેજ રેગ્યુલેટર તરીકે ઝેનર ડાયોડનો ઉપયોગ વર્ણવો.}

\begin{solutionbox}
\textbf{સર્કિટ ડાયાગ્રામ:}
\begin{center}
\begin{tikzpicture}
    \draw (0,2) to[short, o-] (1,2)
        to[R, l=$R_s$] (3,2) -- (4,2)
        to[short, -o] (5,2) node[right] {$V_{out}$};
    \draw (4,2) to[zD, l=$V_z$, invert] (4,0);
    \draw (0,0) to[short, o-o] (5,0);
    \node[left] at (0,1) {$V_{in}$};
    \node[left] at (0,2) {+};
    \node[left] at (0,0) {-};
\end{tikzpicture}
\captionof{figure}{ઝેનર વોલ્ટેજ રેગ્યુલેટર}
\end{center}

\textbf{કાર્ય સિદ્ધાંત:}
\begin{itemize}
    \item \keyword{રિવર્સ બાયસ}: ઝેનર ડાયોડ રિવર્સ બાયસમાં જોડાયેલ છે.
    \item \keyword{બ્રેકડાઉન}: જ્યારે $V_{in}$ ઝેનર વોલ્ટેજ $V_z$ કરતાં વધી જાય છે, ત્યારે ડાયોડ બ્રેકડાઉન ક્ષેત્રમાં વાહન કરે છે.
    \item \keyword{રેગ્યુલેશન}: ઇનપુટ વોલ્ટેજ અથવા લોડ કરંટમાં ફેરફાર હોવા છતાં ઝેનર પર વોલ્ટેજ અચળ ($V_z$) રહે છે.
    \item \keyword{શ્રેણી અવરોધ} ($R_s$): ઝેનર ડાયોડને સુરક્ષિત કરવા માટે કરંટ મર્યાદિત કરે છે.
\end{itemize}
\end{solutionbox}

\begin{mnemonicbox}
\mnemonic{ઝેનર વોલ્ટેજ જાળવી રાખે છે, ફેરફારો છતાં}
\end{mnemonicbox}

\questionmarks{5(B)}{8}{કોઈપણ બેના જવાબ આપો}

\questionmarks{5(B)(1)}{4}{જરૂરી સર્કિટ સાથે પૂર્ણ તરંગ રેક્ટિફાયર સમજાવો અને ઇનપુટ અને આઉટપુટ વેવફોર્મ્સ દોરો.}

\begin{solutionbox}
\textbf{સર્કિટ ડાયાગ્રામ (સેન્ટર-ટેપ્ડ):}
\begin{center}
\begin{tikzpicture}
    % Center Tap Rectifier
    \draw (0,2) to[sinusoidal voltage source, l=$V_{in}$] (0,0);
    \draw (0,2) -- (1,2); \draw (0,0) -- (1,0); 
    % Transformer
    \draw (1.5,2.5) -- (1.5,-0.5); \draw (1.6,2.5) -- (1.6,-0.5); % core
    \draw (1.2,2) to[L] (1.2,0); % Primary
    \draw (1.9,2.5) to[L] (1.9,-0.5); % Secondary
    % Diodes
    \draw (1.9,2.5) -- (2.5,2.5) to[D, l=$D_1$] (4,2.5) -- (5,2.5) -- (5,1);
    \draw (1.9,-0.5) -- (2.5,-0.5) to[D, l=$D_2$] (4,-0.5) -- (5,-0.5) -- (5,1);
    % Load
    \draw (1.9,1) -- (2.5,1) -- (2.5,1) to[R, l=$R_L$] (5,1);
    % Output
    \node at (5.5,1) {$+$}; \node at (2.2,1) {$-$};
\end{tikzpicture}
\captionof{figure}{પૂર્ણ તરંગ રેક્ટિફાયર}
\end{center}

\textbf{વેવફોર્મ્સ:}
\begin{center}
\begin{tikzpicture}
    % Input
    \draw[->] (0,0) -- (4,0) node[right] {$t$};
    \draw[->] (0,-1) -- (0,1) node[above] {$V_{in}$};
    \draw[blue, thick] plot[domain=0:3.8, samples=100] (\x, {0.8*sin(3.14*\x r)});
    
    % Output
    \draw[->] (5,0) -- (9,0) node[right] {$t$};
    \draw[->] (5,-1) -- (5,1) node[above] {$V_{out}$};
    \draw[red, thick] plot[domain=5:8.8, samples=100] (\x, {0.8*abs(sin(3.14*(\x-5) r))});
\end{tikzpicture}
\captionof{figure}{ઇનપુટ અને આઉટપુટ વેવફોર્મ્સ}
\end{center}
\end{solutionbox}

\begin{mnemonicbox}
\mnemonic{પૂર્ણ તરંગ પૂર્ણ ચક્ર વાપરે, સારી કાર્યક્ષમતા}
\end{mnemonicbox}

\questionmarks{5(B)(2)}{4}{PN જંકશન ડાયોડની ફોરવર્ડ અને રિવર્સ લાક્ષણિકતાઓ દર્શાવો.}

\begin{solutionbox}
\textbf{I-V લાક્ષણિકતા વક્ર:}
\begin{center}
\begin{tikzpicture}
    \draw[->] (-3,0) -- (3,0) node[right] {$V$};
    \draw[->] (0,-3) -- (0,3) node[above] {$I$};
    % Forward
    \draw[blue, thick] (0,0) -- (0.5,0) to[out=0, in=260] (0.7,0.1) -- (1,2.5);
    \node[right] at (1,2) {ફોરવર્ડ બાયસ};
    \node[below] at (0.7,0) {$0.7V$};
    
    % Reverse
    \draw[red, thick] (0,0) -- (-2, -0.1) -- (-2.2, -2.5);
    \node[left] at (-1.5,-1) {રિવર્સ બાયસ};
    \node[above] at (-2.2,0) {$V_{BR}$};
\end{tikzpicture}
\captionof{figure}{PN જંકશન ડાયોડ લાક્ષણિકતાઓ}
\end{center}

\textbf{લાક્ષણિકતાઓ:}
\begin{itemize}
    \item \keyword{ફોરવર્ડ બાયસ}: કટ-ઇન વોલ્ટેજ (Si માટે 0.7V) પછી ડાયોડ નોંધપાત્ર રીતે વહન કરે છે.
    \item \keyword{રિવર્સ બાયસ}: બ્રેકડાઉન વોલ્ટેજ સુધી નહિવત્ લીકેજ કરંટ.
    \item \keyword{કટ-ઇન વોલ્ટેજ}: વોલ્ટેજ જે પછી કરંટ ઝડપથી વધવાનું શરૂ કરે છે.
    \item \keyword{બ્રેકડાઉન}: ઉચ્ચ રિવર્સ વોલ્ટેજ પર રિવર્સ કરંટમાં તીવ્ર વધારો.
\end{itemize}
\end{solutionbox}

\begin{mnemonicbox}
\mnemonic{ફોરવર્ડ આપે, રિવર્સ રોકે}
\end{mnemonicbox}

\questionmarks{5(B)(3)}{4}{LED નો સિદ્ધાંત લખો અને તેની રચના અને કાર્ય સમજાવો.}

\begin{solutionbox}
\textbf{સિદ્ધાંત:} \keyword{ઇલેક્ટ્રો્લ્યુમિનિસેન્સ} - કેરિયર રિકોમ્બિનેશન દરમિયાન વિદ્યુત ઊર્જાનું પ્રકાશ ઊર્જામાં રૂપાંતર.

\textbf{રચના:}
\begin{center}
\begin{tikzpicture}
    % Layers
    \fill[red!20] (0,2) rectangle (4,3); \node at (2,2.5) {P-પ્રકાર પ્રદેશ};
    \fill[blue!20] (0,1) rectangle (4,2); \node at (2,1.5) {N-પ્રકાર પ્રદેશ};
    \fill[gray!20] (0,0) rectangle (4,0.5); \node at (2,0.25) {સબસ્ટ્રેટ/સંપર્ક};
    
    % Junction
    \draw[dashed] (0,2) -- (4,2); 
    \node[right] at (4,2) {જંકશન};
    
    % Contacts
    \draw[thick] (1,3) -- (1,3.5); \draw[thick] (3,3) -- (3,3.5);
    \node[above] at (2,3.5) {એનોડ (+)};
    
    \draw[thick] (2,0) -- (2,-0.5);
    \node[below] at (2,-0.5) {કેથોડ (-)};
    
    % Light
    \draw[yellow!80!orange, thick, ->] (0.5,2) -- (-0.5,2.5);
    \draw[yellow!80!orange, thick, ->] (3.5,2) -- (4.5,2.5);
    \node at (5,2.5) {પ્રકાશ $\lambda$};
\end{tikzpicture}
\captionof{figure}{LED રચના}
\end{center}

\textbf{કાર્ય:}
\begin{itemize}
    \item તે \keyword{ફોરવર્ડ બાયસ} માં કાર્ય કરે છે.
    \item N-પ્રદેશમાંથી ઇલેક્ટ્રોન P-પ્રદેશમાં જાય છે અને હોલ્સ સાથે પુનઃસંયોજન કરે છે.
    \item ઊર્જા \keyword{ફોટોન} (પ્રકાશ) ના સ્વરૂપમાં મુક્ત થાય છે.
    \item રંગ અર્ધવાહક સામગ્રીના \keyword{બેન્ડ ગેપ} પર આધારિત છે (દા.ત., લાલ માટે GaAs).
\end{itemize}
\end{solutionbox}

\begin{mnemonicbox}
\mnemonic{LED: લાઈટ એમિટિંગ ડાયોડ, પ્રકાશ આપે}
\end{mnemonicbox}

\end{document}
