\documentclass[10pt,a4paper]{article}

% content/resources/templates/preamble.tex
\usepackage[margin=0.6in]{geometry}
\author{Milav Dabgar}
\usepackage{amsmath,amssymb,amsthm}
\usepackage{booktabs}
\usepackage{multirow}
\usepackage{xcolor}
\usepackage{tcolorbox}
\tcbuselibrary{breakable,skins}
\usepackage[colorlinks=true,linkcolor=blue]{hyperref}
\usepackage{titlesec}
\usepackage{enumitem}
\usepackage{tikz}
\usepackage{pgfplots}
\usepackage{circuitikz}
\usepackage[version=4]{mhchem}
\usepackage{longtable}
\usepackage{array}
\usepackage{float}
\usepackage{caption}
\usepackage{listings}

\lstset{
  basicstyle=\small\ttfamily,
  breaklines=true,
  breakatwhitespace=false,
  postbreak=\mbox{\textcolor{red}{$\hookrightarrow$}\space},
  float=false,
  numbers=left,
  numberstyle=\tiny\color{gray},
  numbersep=10pt,
  xleftmargin=2em,
  keywordstyle=\color{blue},
  commentstyle=\color{green!60!black},
  stringstyle=\color{purple},
  backgroundcolor=\color{gray!5},
  showstringspaces=false,
  tabsize=2,
  captionpos=b,
  keepspaces=true,
  columns=flexible
}

\pgfplotsset{compat=1.18}
\usetikzlibrary{shapes,arrows,positioning,calc,patterns,decorations.pathmorphing,decorations.markings,arrows.meta}

% Color scheme
\definecolor{headcolor}{RGB}{0,102,204}
\definecolor{keycolor}{RGB}{220,20,60}
\definecolor{solutioncolor}{RGB}{34,139,34}
\definecolor{mnemoniccolor}{RGB}{148,0,211}
\definecolor{codecolor}{RGB}{0,0,100}

% Spacing
\setlength{\parskip}{3pt}
\setlist[itemize]{nosep}
\setlist[enumerate]{nosep}

% Title formatting
\titleformat{\section}{\Large\bfseries\color{headcolor}}{\thesection}{1em}{}
\titleformat{\subsection}{\large\bfseries\color{headcolor}}{\thesubsection}{1em}{}

% Pandoc tightlist compatibility
\providecommand{\tightlist}{%
  \setlength{\itemsep}{0pt}\setlength{\parskip}{0pt}}

% Pandoc longtable compatibility
\newcounter{none}
\def\thenone{}


% content/resources/templates/english-boxes.tex
% This file is currently empty - it exists to maintain consistency with the import structure.
% Add custom environments here if needed in the future.


\begin{document}

\begin{center}
{\Huge\bfseries\color{headcolor} Subject Name Solutions}\\[5pt]
{\LARGE unit -- Study Material}\\[3pt]
{\large Semester 1 Study Material}\\[3pt]
{\normalsize\textit{Detailed Solutions and Explanations}}
\end{center}

\vspace{10pt}

\section*{Unit-2. Electrostatics -
Solutions}\label{unit-2.-electrostatics---solutions}

\subsection*{Part A: Definitions with Standard Units (1 or 2
marks)}\label{part-a-definitions-with-standard-units-1-or-2-marks}

\subsubsection{(1) Give definitions with its standard
unit:}\label{give-definitions-with-its-standard-unit}

\paragraph{\texorpdfstring{\textbf{Electric Field
(E):}}{Electric Field (E):}}\label{electric-field-e}

\textbf{Definition:} The electric field at a point is defined as the
force experienced by a unit positive charge placed at that point. It
represents the effect of an electric charge in the surrounding space.

\textbf{Mathematical Formula:}

\begin{verbatim}
E = F / q_{0}
\end{verbatim}

Where:

\begin{itemize}
\tightlist
\item
  E = Electric field intensity
\item
  F = Electric force experienced
\item
  q_{0} = Test charge (very small positive charge)
\end{itemize}

For a point charge Q at distance r:

\begin{verbatim}
E = kQ / r^{2}  or

E = Q / (4πε_{0}r^{2})

\end{verbatim}

\textbf{SI Unit:} Newton per Coulomb (N/C) or Volt per meter (V/m)

\textbf{Direction:} Electric field is a vector quantity. It points away
from positive charges and towards negative charges.

\begin{center}\rule{0.5\linewidth}{0.5pt}\end{center}

\paragraph{\texorpdfstring{\textbf{Electric Potential
(V):}}{Electric Potential (V):}}\label{electric-potential-v}

\textbf{Definition:} The electric potential at a point in an electric
field is defined as the amount of work done in bringing a unit positive
charge from infinity to that point without acceleration.

\textbf{Mathematical Formula:}

\begin{verbatim}
V = W / q
\end{verbatim}

Where:

\begin{itemize}
\tightlist
\item
  V = Electric potential
\item
  W = Work done
\item
  q = Charge
\end{itemize}

For a point charge Q at distance r:

\begin{verbatim}
V = kQ / r  or

V = Q / (4πε_{0}r)

\end{verbatim}

\textbf{SI Unit:} Volt (V) or Joule per Coulomb (J/C)

\textbf{Note:} Electric potential is a scalar quantity.

\begin{center}\rule{0.5\linewidth}{0.5pt}\end{center}

\paragraph{\texorpdfstring{\textbf{Electric Potential Difference (ΔV or
V):}}{Electric Potential Difference (ΔV or V):}}\label{electric-potential-difference-ux3b4v-or-v}

\textbf{Definition:} The potential difference between two points in an
electric field is defined as the work done in moving a unit positive
charge from one point to another against the electric field.

\textbf{Mathematical Formula:}

\begin{verbatim}
V_{2} - V_{1} = W / q

or

ΔV = W / q
\end{verbatim}

Where:

\begin{itemize}
\tightlist
\item
  V_{2} - V_{1} = Potential difference between points 2 and 1
\item
  W = Work done
\item
  q = Charge moved
\end{itemize}

\textbf{Relation with Electric Field:}

\begin{verbatim}
V = -E \times d  (for uniform field)
\end{verbatim}

\textbf{SI Unit:} Volt (V)

\textbf{Note:} Potential difference is also called voltage.

\begin{center}\rule{0.5\linewidth}{0.5pt}\end{center}

\paragraph{\texorpdfstring{\textbf{Electric Flux
(Φ):}}{Electric Flux (Φ):}}\label{electric-flux-ux3c6}

\textbf{Definition:} Electric flux through a surface is defined as the
total number of electric field lines passing perpendicularly through
that surface. It measures the quantity of electric field passing through
a given area.

\begin{figure}
\centering
\pandocbounded{\includegraphics[keepaspectratio,alt={Electric Flux Diagram}]{unit-2-book-scan_artifacts/image_000015_ea8cf122d11e80250b910248f3f29e6bed211818bf5f4d9ef8da1dddc744d680.png}}
\caption{Electric Flux Diagram}
\end{figure}

\emph{Figure: Electric flux through different orientations of surface
area in electric field}

\textbf{Mathematical Formula:}

\begin{verbatim}
Φ = E ·

A = EA cos θ

\end{verbatim}

Where:

\begin{itemize}
\tightlist
\item
  Φ = Electric flux
\item
  E = Electric field intensity
\item
  A = Area of the surface
\item
  θ = Angle between electric field and normal to the surface
\end{itemize}

\textbf{Special Cases:}

\begin{itemize}
\tightlist
\item
When

θ = 0^\circ (perpendicular):

Φ = EA (maximum)

\item
When

θ = 90^\circ (parallel):

Φ = 0 (minimum)

\end{itemize}

\textbf{SI Unit:} Newton meter^{2} per Coulomb (N·m^{2}/C) or Volt meter (V·m)

\begin{center}\rule{0.5\linewidth}{0.5pt}\end{center}

\paragraph{\texorpdfstring{\textbf{Capacitor:}}{Capacitor:}}\label{capacitor}

\textbf{Definition:} A capacitor is an electrical device that stores
electrical energy in the form of electric charge. It consists of two
conducting plates separated by an insulating material (dielectric).

\textbf{Construction:} Two parallel conducting plates separated by a
small distance with air or dielectric material between them.

\textbf{Function:}

\begin{itemize}
\tightlist
\item
  Stores electric charge and energy
\item
  Blocks DC and allows AC
\item
  Used in filtering, timing circuits, energy storage
\end{itemize}

\textbf{Types:}

\begin{enumerate}
\tightlist
\item
  Fixed capacitors (paper, mica, ceramic, electrolytic)
\item
  Variable capacitors
\item
  Parallel plate capacitors
\item
  Spherical capacitors
\item
  Cylindrical capacitors
\end{enumerate}

\begin{center}\rule{0.5\linewidth}{0.5pt}\end{center}

\paragraph{\texorpdfstring{\textbf{Capacitance
(C):}}{Capacitance (C):}}\label{capacitance-c}

\textbf{Definition:} Capacitance is the ability of a capacitor to store
electric charge. It is defined as the ratio of charge stored on one
plate to the potential difference between the plates.

\textbf{Mathematical Formula:}

\begin{verbatim}
C = Q / V
\end{verbatim}

Where:

\begin{itemize}
\tightlist
\item
  C = Capacitance
\item
  Q = Charge stored on one plate
\item
  V = Potential difference between plates
\end{itemize}

\textbf{For Parallel Plate Capacitor:}

\begin{verbatim}
C = ε_{0}εᵣA /

d = ε_{0}KA / d

\end{verbatim}

Where:

\begin{itemize}
\tightlist
\item
  ε_{0} = Permittivity of free space = 8.85 \times 10^{-}^{1}^{2} F/m
\item
  εᵣ or K = Relative permittivity (dielectric constant)
\item
  A = Area of each plate
\item
  d = Distance between plates
\end{itemize}

\textbf{SI Unit:} Farad (F)

\textbf{Practical Units:}

\begin{itemize}
\tightlist
\item
  Microfarad (μF) = 10^{-}^{6} F
\item
  Nanofarad (nF) = 10^{-}^{9} F
\item
  Picofarad (pF) = 10^{-}^{1}^{2} F
\end{itemize}

\textbf{Note:} Capacitance depends on:

\begin{enumerate}
\tightlist
\item
  Area of plates (C ∝ A)
\item
  Distance between plates (C ∝ 1/d)
\item
  Dielectric medium (C ∝ K)
\end{enumerate}

\begin{center}\rule{0.5\linewidth}{0.5pt}\end{center}

\subsection*{Part B: Detailed Answers (2 or 3
marks)}\label{part-b-detailed-answers-2-or-3-marks}

\subsubsection{(1) Explain Coulomb's law with mathematical
formula.}\label{explain-coulombs-law-with-mathematical-formula.}

\begin{solutionbox}

\textbf{Coulomb's Law:} French scientist Charles Augustin de Coulomb
(1736-1806) conducted experiments to find the force between two electric
charges and formulated Coulomb's law.

\begin{figure}
\centering
\pandocbounded{\includegraphics[keepaspectratio,alt={Coulomb's Law Illustration}]{unit-2-book-scan_artifacts/image_000002_93bff9bbc0efb7c065572eca4cde6eeac24915da10f24946bda8ea40f0e5f055.png}}
\caption{Coulomb's Law Illustration}
\end{figure}

\emph{Figure: Electric force between two point charges q_{1} and q_{2}}

\textbf{Statement:} ``The electric force (Coulombian force) between two
stationary point charges is directly proportional to the product of
their charges and inversely proportional to the square of the distance
between them. This force acts along the line joining the two charges.''

\textbf{Mathematical Formula:}

\begin{verbatim}
F ∝ q_{1}q_{2}  (Force is directly proportional to product of charges)

F ∝ 1/r^{2}  (Force is inversely proportional to square of distance)

Combining both:

F = k(q_{1}q_{2})/r^{2}
\end{verbatim}

\textbf{Where:}

\begin{itemize}
\tightlist
\item
  F = Electric force between charges (N)
\item
  q_{1}, q_{2} = Magnitudes of two point charges (C)
\item
  r = Distance between the charges (m)
\item
  k = Coulomb's constant = 9 \times 10^{9} N·m^{2}/C^{2}
\end{itemize}

\textbf{Alternative Form:}

\begin{verbatim}
F = (1/4πε_{0}) \times (q_{1}q_{2})/r^{2}
\end{verbatim}

Where:

\begin{itemize}
\tightlist
\item
  ε_{0} = Permittivity of free space = 8.85 \times 10^{-}^{1}^{2} C^{2}/N·m^{2}
\item
  k = 1/(4πε_{0}) = 8.9875 \times 10^{9} \approx 9 \times 10^{9} N·m^{2}/C^{2}
\end{itemize}

\textbf{In a Medium:}

\begin{verbatim}
F = (1/4πε_{0}εᵣ) \times (q_{1}q_{2})/r^{2} = k(q_{1}q_{2})/(εᵣr^{2})
\end{verbatim}

Where:

\begin{itemize}
\tightlist
\item
  εᵣ = Relative permittivity or dielectric constant (K)
\end{itemize}

\textbf{Nature of Force:}

\begin{enumerate}
\item
  \textbf{Like charges:} If both charges are of the same sign (both
  positive or both negative), the force is \textbf{repulsive} (pushes
  them apart)
\item
  \textbf{Unlike charges:} If charges are of opposite signs (one
  positive, one negative), the force is \textbf{attractive} (pulls them
  together)
\end{enumerate}

\textbf{Vector Form:}

\begin{verbatim}
F⃗_{1}_{2} = k(q_{1}q_{2})/r^{2} \times r̂_{1}_{2}
\end{verbatim}

Where r̂_{1}_{2} is the unit vector from q_{1} to q_{2}.

\textbf{Key Points:}

\begin{enumerate}
\tightlist
\item
  Coulomb's law is valid only for \textbf{stationary point charges}
\item
  It is a \textbf{fundamental law} of nature
\item
  Similar to Newton's law of gravitation in form
\item
  Electric force is \textbf{much stronger} than gravitational force
  (\textasciitilde10^{3}^{9} times)
\item
  The law can be applied to large charged objects if the distance
  between them is much larger than their size
\item
  Permittivity (ε) represents the resistance of the medium that impedes
  the electric field
\end{enumerate}

\textbf{Comparison with Gravitational Force:}

\begin{longtable}[]{@{}lll@{}}
\toprule\noalign{}
Property & Gravitational Force & Electric Force \\
\midrule\noalign{}
\endhead
\bottomrule\noalign{}
\endlastfoot
Formula &

F = Gm_{1}m_{2}/r^{2} &

F = kq_{1}q_{2}/r^{2} \\

Nature & Always attractive & Attractive or repulsive \\
Strength & Very weak & Very strong \\
Depends on & Mass & Charge \\
Constant &

G = 6.67\times10^{-}^{1}^{1} &

k = 9\times10^{9} \\

\end{longtable}

\end{solutionbox}
\begin{center}\rule{0.5\linewidth}{0.5pt}\end{center}

\subsubsection{(2) Explain characteristics of Electric field lines with
figures.}\label{explain-characteristics-of-electric-field-lines-with-figures.}

\begin{solutionbox}

\textbf{Electric Field Lines:} Michael Faraday introduced the concept of
electric field lines (also called ``electric lines of force''). The
geometric representation of an electric field is called electric field
lines.

\textbf{Definition:} An electric field line is a curve drawn in an
electric field such that the tangent to the curve at any point gives the
direction of the net electric field at that point.

\begin{figure}
\centering
\pandocbounded{\includegraphics[keepaspectratio,alt={Electric Field Lines - Single Charges}]{unit-2-book-scan_artifacts/image_000010_2c0f1ad08ec6b2c11509a5a9553987e436dec3440498983c03a28655fd5979b0.png}}
\caption{Electric Field Lines - Single Charges}
\end{figure}

\emph{Figure: Electric field lines for positive and negative charges}

\begin{figure}
\centering
\pandocbounded{\includegraphics[keepaspectratio,alt={Electric Field Lines - Multiple Charges}]{unit-2-book-scan_artifacts/image_000011_c74b95eb27cda2adebf9a9ef3e6b8830746f1269312c1d49a4334e230c5a06a1.png}}
\caption{Electric Field Lines - Multiple Charges}
\end{figure}

\emph{Figure: Electric field lines between positive and negative charges
showing force direction}

\textbf{Characteristics of Electric Field Lines:}

\textbf{1. Origin and Termination:}

\begin{itemize}
\tightlist
\item
  Electric field lines \textbf{start from positive charges} and
  \textbf{end at negative charges}
\item
  For isolated positive charge: lines radiate outward to infinity
\item
  For isolated negative charge: lines come from infinity and converge
  inward
\end{itemize}

\begin{verbatim}
Positive Charge (+Q):          Negative Charge (-Q):
        ↗  ↑  ↖                      ↙  ↓  ↘
      ↗    |    ↖                  ↙    |    ↘
    \rightarrow  (+Q)  \leftarrow                \rightarrow  (-Q)  \leftarrow
      ↘    |    ↗                  ↖    |    ↗
        ↘  ↓  ↗                      ↖  ↑  ↙
\end{verbatim}

\textbf{2. Direction of Electric Field:}

\begin{itemize}
\tightlist
\item
  The \textbf{tangent} at any point on an electric field line indicates
  the \textbf{direction of the electric field} at that point
\item
  It shows the direction in which a positive test charge would move if
  placed at that point
\end{itemize}

\begin{figure}
\centering
\pandocbounded{\includegraphics[keepaspectratio,alt={Tangent to Field Lines}]{unit-2-book-scan_artifacts/image_000012_21b0825f5fdc61f7262a761818af0398996cba4b3aa92f48bbdb97f08713056b.png}}
\caption{Tangent to Field Lines}
\end{figure}

\emph{Figure: Tangent at points P_{1} and P_{2} showing electric field
direction E_{1} and E_{2}}

\textbf{3. Non-Intersection:}

\begin{itemize}
\tightlist
\item
  \textbf{Two electric field lines never intersect or cross each other}
\item
  If they intersect, there would be two directions of electric field at
  that point, which is impossible
\item
  A charge at the intersection would experience force in two directions
  simultaneously, which contradicts the definition
\end{itemize}

\begin{figure}
\centering
\pandocbounded{\includegraphics[keepaspectratio,alt={Field Lines Never Intersect}]{unit-2-book-scan_artifacts/image_000013_a70c1cbec720d8ccd994f48d6bb9cac55a42406a06d4b59982fb2ce14515ec5d.png}}
\caption{Field Lines Never Intersect}
\end{figure}

\emph{Figure: Two field lines cannot intersect as it would give two
field directions at point P}

\textbf{4. Field Intensity and Line Density:}

\begin{itemize}
\tightlist
\item
  The \textbf{density} (closeness) of electric field lines indicates the
  \textbf{strength of the electric field}
\item
  \textbf{Closely spaced lines} \rightarrow Strong electric field (high intensity)
\item
  \textbf{Widely spaced lines} \rightarrow Weak electric field (low intensity)
\item
  Number of lines passing through a unit area is proportional to field
  strength
\end{itemize}

\begin{figure}
\centering
\pandocbounded{\includegraphics[keepaspectratio,alt={Field Line Density}]{unit-2-book-scan_artifacts/image_000014_d5b17b9b69152cfe38f61a2a47062cc4e5f4404694f727d39d519c1157ad2244.png}}
\caption{Field Line Density}
\end{figure}

\emph{Figure: More field lines through A_{1} indicates stronger field than
through A_{2}}

\textbf{5. Uniform Electric Field:}

\begin{itemize}
\tightlist
\item
  Electric field lines of a \textbf{uniform electric field} are:

  \begin{itemize}
  \tightlist
  \item
    Mutually \textbf{parallel}
  \item
    \textbf{Equidistant} from each other
  \item
    Example: Field between two parallel charged plates
  \end{itemize}
\end{itemize}

\begin{verbatim}
Uniform Electric Field:
    ++++++++++++++++
    ║ ║ ║ ║ ║ ║ ║ ║
    ║ ║ ║ ║ ║ ║ ║ ║
    ║ ║ ║ ║ ║ ║ ║ ║
    ───────────────
\end{verbatim}

\textbf{6. Imaginary Nature:}

\begin{itemize}
\tightlist
\item
  Electric field lines are \textbf{imaginary}, but \textbf{electric
  field is real}
\item
  They are a visual tool to represent the field
\item
  Actual field exists continuously in space
\end{itemize}

\textbf{7. Perpendicular to Conducting Surface:}

\begin{itemize}
\tightlist
\item
  Electric field lines are always \textbf{perpendicular} to the
  conducting surface
\item
  This applies both when leaving and entering the charge
\item
  \textbf{Reason:} Electric field parallel to conducting surface is zero
\item
  No electric force exists parallel to the conducting surface
\end{itemize}

\textbf{8. Open Curves:}

\begin{itemize}
\tightlist
\item
  Electric field lines \textbf{do not form closed loops}
\item
  They always have a beginning (positive charge) and end (negative
  charge)
\item
  Unlike magnetic field lines which form closed loops
\end{itemize}

\textbf{Examples of Field Line Patterns:}

\textbf{a) Two Positive Charges:}

\begin{verbatim}
     ↗  ↑  ↖            ↗  ↑  ↖
   ↗    |    ↖        ↗    |    ↖
 \rightarrow  (+Q)      \leftrightarrow    \leftrightarrow      (+Q)  \leftarrow
   ↘    |    ↗        ↘    |    ↗
     ↘  ↓  ↗            ↘  ↓  ↗
\end{verbatim}

\emph{Field lines repel each other, never connect}

\textbf{b) Positive and Negative Charges (Dipole):}

\begin{verbatim}
     ↗  \rightarrow  \rightarrow  \rightarrow  \rightarrow  ↘
   ↗              ↘
 (+Q) \rightarrow \rightarrow \rightarrow \rightarrow \rightarrow (-Q)
   ↘              ↗
     ↘  \rightarrow  \rightarrow  \rightarrow  \rightarrow  ↗
\end{verbatim}

\emph{Field lines flow from + to -}

\textbf{Summary Table:}

\begin{longtable}[]{@{}ll@{}}
\toprule\noalign{}
Characteristic & Description \\
\midrule\noalign{}
\endhead
\bottomrule\noalign{}
\endlastfoot
Start Point & Positive charge \\
End Point & Negative charge \\
Direction & Tangent at any point \\
Intersection & Never cross \\
Density & Indicates field strength \\
Conductor & Perpendicular to surface \\
Nature & Imaginary lines, real field \\
Closed Loop & No, always open curves \\
\end{longtable}

\end{solutionbox}
\begin{center}\rule{0.5\linewidth}{0.5pt}\end{center}

\subsubsection{(3) Write short note on parallel plate
capacitor.}\label{write-short-note-on-parallel-plate-capacitor.}

\begin{solutionbox}

\textbf{Parallel Plate Capacitor:} A parallel plate capacitor consists
of two large conducting plates of equal area placed parallel to each
other and separated by a small distance with air or a dielectric medium
between them.

\textbf{Construction:}

\begin{verbatim}
    ++++++++++++++++  \leftarrow Plate 1 (+Q charge)
         ↓ ↓ ↓
    ║ ║ ║ ║ ║ ║ ║   \leftarrow Electric field (uniform)
         ↓ ↓ ↓
    ────────────────  \leftarrow Plate 2 (-Q charge)
         ↑
      Distance d
    
    \leftarrow───── A ─────\rightarrow  (Area of each plate)
\end{verbatim}

\textbf{Components:}

\begin{enumerate}
\tightlist
\item
  \textbf{Two Conducting Plates:}

  \begin{itemize}
  \tightlist
  \item
    Made of metal (copper, aluminum)
  \item
    Same area A
  \item
    Placed parallel to each other
  \end{itemize}
\item
  \textbf{Dielectric Medium:}

  \begin{itemize}
  \tightlist
  \item
    Insulating material between plates
  \item
    Air, paper, mica, ceramic, plastic, etc.
  \item
    Prevents direct contact and discharge
  \end{itemize}
\item
  \textbf{Separation (d):}

  \begin{itemize}
  \tightlist
  \item
    Small distance between plates
  \item
    Typically much smaller than plate dimensions
  \item
    d \textless\textless{} \sqrtA
  \end{itemize}
\end{enumerate}

\textbf{Working Principle:}

When connected to a battery:

\begin{enumerate}
\tightlist
\item
  One plate gets \textbf{positive charge (+Q)}
\item
  Other plate gets \textbf{equal negative charge (-Q)}
\item
  Electric field is established between plates
\item
  Energy is stored in the electric field
\end{enumerate}

\textbf{Electric Field:}

Inside the capacitor (between plates):

\begin{verbatim}
E = σ/ε_{0} = Q/(ε_{0}A)
\end{verbatim}

Where:

\begin{itemize}
\tightlist
\item
  σ = Surface charge density = Q/A
\item
  ε_{0} = Permittivity of free space
\end{itemize}

\textbf{Potential Difference:}

\begin{verbatim}
V = Ed = Qd/(ε_{0}A)
\end{verbatim}

\textbf{Capacitance:}

The capacitance of a parallel plate capacitor is given by:

\begin{verbatim}
C = Q/V = ε_{0}A/d
\end{verbatim}

\textbf{With Dielectric:}

\begin{verbatim}
C = ε_{0}εᵣA/d = ε_{0}KA/d = εA/d
\end{verbatim}

Where:

\begin{itemize}
\tightlist
\item
  C = Capacitance (F)
\item
  ε_{0} = 8.85 \times 10^{-}^{1}^{2} F/m
\item
  εᵣ or K = Dielectric constant
\item
  A = Area of each plate (m^{2})
\item
  d = Distance between plates (m)
\item
  ε = ε_{0}εᵣ = Permittivity of medium
\end{itemize}

\textbf{Factors Affecting Capacitance:}

\begin{enumerate}
\tightlist
\item
  \textbf{Area of Plates (A):}

  \begin{itemize}
  \tightlist
  \item
    C ∝ A
  \item
    Larger area \rightarrow More charge storage \rightarrow Higher capacitance
  \end{itemize}
\item
  \textbf{Distance Between Plates (d):}

  \begin{itemize}
  \tightlist
  \item
    C ∝ 1/d
  \item
    Smaller distance \rightarrow Stronger field \rightarrow Higher capacitance
  \end{itemize}
\item
  \textbf{Dielectric Medium:}

  \begin{itemize}
  \tightlist
  \item
    C ∝ K (dielectric constant)
  \item
    Higher K \rightarrow Higher capacitance
  \item
    Air: K = 1, Paper: K \approx 3.7, Mica: K \approx 5.5
  \end{itemize}
\end{enumerate}

\textbf{Energy Stored:}

The energy stored in a parallel plate capacitor:

\begin{verbatim}
U = (1/2)QV = (1/2)CV^{2} = Q^{2}/(2C)
\end{verbatim}

\textbf{Characteristics:}

\begin{enumerate}
\tightlist
\item
  \textbf{Uniform Electric Field:}

  \begin{itemize}
  \tightlist
  \item
    Field between plates is uniform (except at edges)
  \item
    Parallel and equally spaced field lines
  \end{itemize}
\item
  \textbf{High Capacitance:}

  \begin{itemize}
  \tightlist
  \item
    Relatively high capacitance for given size
  \item
    Depends on area, separation, and dielectric
  \end{itemize}
\item
  \textbf{Linear Device:}

  \begin{itemize}
  \tightlist
  \item
    Q ∝ V (charge proportional to voltage)
  \item
    Constant capacitance
  \end{itemize}
\end{enumerate}

\textbf{Applications:}

\begin{enumerate}
\tightlist
\item
  \textbf{Energy Storage:}

  \begin{itemize}
  \tightlist
  \item
    Camera flash circuits
  \item
    Power supplies
  \end{itemize}
\item
  \textbf{Filtering:}

  \begin{itemize}
  \tightlist
  \item
    Smoothing voltage in power supplies
  \item
    Signal processing
  \end{itemize}
\item
  \textbf{Timing Circuits:}

  \begin{itemize}
  \tightlist
  \item
    Oscillators
  \item
    Timers
  \end{itemize}
\item
  \textbf{Coupling/Decoupling:}

  \begin{itemize}
  \tightlist
  \item
    Blocking DC, allowing AC
  \item
    Separating circuit stages
  \end{itemize}
\item
  \textbf{Tuning:}

  \begin{itemize}
  \tightlist
  \item
    Radio and TV circuits
  \item
    Resonant circuits
  \end{itemize}
\end{enumerate}

\textbf{Advantages:}

\begin{itemize}
\tightlist
\item
  Simple construction
\item
  Predictable capacitance
\item
  Can handle high voltages
\item
  Low cost
\item
  Reliable
\end{itemize}

\textbf{Limitations:}

\begin{itemize}
\tightlist
\item
  Fixed capacitance (unless variable design)
\item
  Limited voltage rating
\item
  Can discharge rapidly (short circuit hazard)
\item
  Electrolytic types have polarity
\end{itemize}

\textbf{Practical Considerations:}

\begin{enumerate}
\tightlist
\item
  \textbf{Fringing Effect:}

  \begin{itemize}
  \tightlist
  \item
    Field lines curve at edges
  \item
    Formula assumes negligible edge effects
  \item
    Valid when d \textless\textless{} \sqrtA
  \end{itemize}
\item
  \textbf{Breakdown Voltage:}

  \begin{itemize}
  \tightlist
  \item
    Maximum voltage before dielectric breaks down
  \item
    Depends on dielectric strength and thickness
  \end{itemize}
\item
  \textbf{Leakage Current:}

  \begin{itemize}
  \tightlist
  \item
    Small current through dielectric
  \item
    Causes gradual discharge
  \end{itemize}
\end{enumerate}

\end{solutionbox}
\begin{center}\rule{0.5\linewidth}{0.5pt}\end{center}

\subsubsection{(4) Explain series connection of capacitors in
detail.}\label{explain-series-connection-of-capacitors-in-detail.}

\begin{solutionbox}

\textbf{Series Connection of Capacitors:} When capacitors are connected
end-to-end such that the negative plate of one capacitor is connected to
the positive plate of the next capacitor, they are said to be connected
in series.

\textbf{Circuit Diagram:}

\begin{verbatim}
    ───┤├─────┤├─────┤├───
       C_{1}     C_{2}     C_{3}
       
    +  -  +  -  +  -
    Q  Q  Q  Q  Q  Q
\end{verbatim}

\textbf{Alternative Representation:}

\begin{verbatim}
         C_{1}        C_{2}        C_{3}
    +───┤├───+───┤├───+───┤├───+
    │        -   +       -   +   │
   ━┷━                          ━┷━
    V                            -
\end{verbatim}

\textbf{Characteristics of Series Connection:}

\textbf{1. Same Charge on All Capacitors:}

\begin{itemize}
\item
  When connected to a battery, the same charge Q flows through each
  capacitor
\item
  Charge on each capacitor: \textbf{Q_{1} = Q_{2} = Q_{3} = Q}

  \textbf{Reason:} When a charge +Q is stored on the positive plate of
  C_{1}, it induces -Q on its negative plate. This -Q repels equal amount
  from the positive plate of C_{2}, leaving +Q on it, and so on.
\end{itemize}

\textbf{2. Different Potential Differences:}

\begin{itemize}
\item
  Total voltage divides among capacitors
\item
  V = V_{1} + V_{2} + V_{3}

  Where:
\item
  V_{1} = Q/C_{1} (voltage across C_{1})
\item
  V_{2} = Q/C_{2} (voltage across C_{2})
\item
  V_{3} = Q/C_{3} (voltage across C_{3})
\end{itemize}

\textbf{3. Equivalent Capacitance:}

The total or equivalent capacitance (C_{s}) of capacitors in series is
given by:

\begin{verbatim}
1/C_{s} = 1/C_{1} + 1/C_{2} + 1/C_{3} + ... + 1/C_{n}
\end{verbatim}

\textbf{Derivation:}

Starting with the definition of capacitance:

\begin{verbatim}
C = Q/V  \rightarrow

V = Q/C

\end{verbatim}

For series connection:

\begin{verbatim}
V = V_{1} + V_{2} + V_{3}

V = Q/C_{1} + Q/C_{2} + Q/C_{3}

V = Q(1/C_{1} + 1/C_{2} + 1/C_{3})
\end{verbatim}

If C_{s} is the equivalent capacitance:

\begin{verbatim}
V = Q/C_{s}
\end{verbatim}

Comparing:

\begin{verbatim}
Q/C_{s} = Q(1/C_{1} + 1/C_{2} + 1/C_{3})

1/C_{s} = 1/C_{1} + 1/C_{2} + 1/C_{3}
\end{verbatim}

\textbf{For Two Capacitors in Series:}

\begin{verbatim}
1/C_{s} = 1/C_{1} + 1/C_{2} = (C_{1} + C_{2})/(C_{1}C_{2})

C_{s} = (C_{1}C_{2})/(C_{1} + C_{2})

C_{s} = Product / Sum
\end{verbatim}

\textbf{For n Equal Capacitors (C each) in Series:}

\begin{verbatim}
1/C_{s} = n/C

C_{s} = C/n
\end{verbatim}

\textbf{Important Points:}

\begin{enumerate}
\tightlist
\item
  \textbf{Equivalent Capacitance Decreases:}

  \begin{itemize}
  \tightlist
  \item
    C_{s} \textless{} smallest individual capacitance
  \item
    Series connection reduces total capacitance
  \end{itemize}
\item
  \textbf{Effective Distance Increases:}

  \begin{itemize}
  \tightlist
  \item
    Series connection is like increasing plate separation
  \item
    Since C ∝ 1/d, capacitance decreases
  \end{itemize}
\item
  \textbf{Voltage Distribution:}

  \begin{itemize}
  \tightlist
  \item
    Smaller capacitor gets larger voltage
  \item
    V ∝ 1/C
  \item
    V_{1}/V_{2} = C_{2}/C_{1}
  \end{itemize}
\item
  \textbf{Same Energy Distribution:}

  \begin{itemize}
  \tightlist
  \item
    Energy in each: U_{1} = Q^{2}/(2C_{1})
  \item
    Total energy: U = Q^{2}/(2C_{s})
  \end{itemize}
\end{enumerate}

\textbf{Example Calculation:}

For C_{1} = 10 μF, C_{2} = 20 μF, C_{3} = 30 μF in series:

\begin{verbatim}
1/C_{s} = 1/10 + 1/20 + 1/30
1/C_{s} = (6 + 3 + 2)/60 = 11/60
C_{s} = 60/11 = 5.45 μF
\end{verbatim}

\textbf{Applications:}

\begin{enumerate}
\tightlist
\item
  \textbf{Voltage Division:}

  \begin{itemize}
  \tightlist
  \item
    Obtaining different voltages from single source
  \item
    High voltage rating circuits
  \end{itemize}
\item
  \textbf{Increased Voltage Rating:}

  \begin{itemize}
  \tightlist
  \item
    Total voltage rating = sum of individual ratings
  \item
    Used when higher voltage handling needed
  \end{itemize}
\item
  \textbf{Fine Tuning:}

  \begin{itemize}
  \tightlist
  \item
    Achieving specific capacitance values
  \item
    Precision applications
  \end{itemize}
\end{enumerate}

\textbf{Advantages:}

\begin{itemize}
\tightlist
\item
  Same charge on all capacitors
\item
  Higher voltage handling capability
\item
  Simple voltage division
\end{itemize}

\textbf{Disadvantages:}

\begin{itemize}
\tightlist
\item
  Reduced total capacitance
\item
  If one capacitor fails (open circuit), entire circuit fails
\item
  Unequal voltage distribution
\end{itemize}

\textbf{Comparison with Parallel:}

\begin{longtable}[]{@{}lll@{}}
\toprule\noalign{}
Property & Series & Parallel \\
\midrule\noalign{}
\endhead
\bottomrule\noalign{}
\endlastfoot
Charge & Same (Q) & Different (Q = Q_{1}+Q_{2}+Q_{3}) \\
Voltage & Different (V = V_{1}+V_{2}+V_{3}) & Same (V) \\
Capacitance & 1/C = 1/C_{1}+1/C_{2}+1/C_{3} &

C = C_{1}+C_{2}+C_{3} \\

Effect & Decreases & Increases \\
\end{longtable}

\end{solutionbox}
\begin{center}\rule{0.5\linewidth}{0.5pt}\end{center}

\subsubsection{(5) Explain parallel connection of capacitors in
detail.}\label{explain-parallel-connection-of-capacitors-in-detail.}

\begin{solutionbox}

\textbf{Parallel Connection of Capacitors:} When all the positive plates
of capacitors are connected to one common point and all negative plates
are connected to another common point, the capacitors are said to be
connected in parallel.

\textbf{Circuit Diagram:}

\begin{verbatim}
         C_{1}    C_{2}    C_{3}
    +────┤├────┤├────┤├────+
    │                       │
   ━┷━                     ━┷━
    V                       -
\end{verbatim}

\textbf{Alternative Representation:}

\begin{verbatim}
         ┌───┤├─── C_{1}
    +────┤
    │    ├───┤├─── C_{2}
   ━┷━   │
    V    └───┤├─── C_{3}
         │
    ─────┴───────── -
\end{verbatim}

\textbf{Characteristics of Parallel Connection:}

\textbf{1. Same Potential Difference:}

\begin{itemize}
\item
  All capacitors have the same voltage across them
\item
  \textbf{V_{1} = V_{2} = V_{3} = V}

  \textbf{Reason:} All positive plates connected to same point (positive
  terminal) and all negative plates to another point (negative terminal)
\end{itemize}

\textbf{2. Different Charges:}

\begin{itemize}
\item
  Total charge divides among capacitors
\item
  Each stores charge according to its capacitance
\item
  \textbf{Q = Q_{1} + Q_{2} + Q_{3}}

  Where:
\item
  Q_{1} = C_{1}V (charge on C_{1})
\item
  Q_{2} = C_{2}V (charge on C_{2})
\item
  Q_{3} = C_{3}V (charge on C_{3})
\end{itemize}

\textbf{3. Equivalent Capacitance:}

The total or equivalent capacitance (C_{p}) of capacitors in parallel is
given by:

\begin{verbatim}
C_{p} = C_{1} + C_{2} + C_{3} + ... + C_{n}
\end{verbatim}

\textbf{Derivation:}

Starting with the definition of capacitance:

\begin{verbatim}
Q = CV
\end{verbatim}

For parallel connection:

\begin{verbatim}
Q = Q_{1} + Q_{2} + Q_{3}

Q = C_{1}V + C_{2}V + C_{3}V

Q = V(C_{1} + C_{2} + C_{3})
\end{verbatim}

If C_{p} is the equivalent capacitance:

\begin{verbatim}
Q = C_{p}V
\end{verbatim}

Comparing:

\begin{verbatim}
C_{p}V = V(C_{1} + C_{2} + C_{3})

C_{p} = C_{1} + C_{2} + C_{3}
\end{verbatim}

\textbf{For n Equal Capacitors (C each) in Parallel:}

\begin{verbatim}
C_{p} = nC
\end{verbatim}

\textbf{Important Points:}

\begin{enumerate}
\tightlist
\item
  \textbf{Equivalent Capacitance Increases:}

  \begin{itemize}
  \tightlist
  \item
    C_{p} \textgreater{} largest individual capacitance
  \item
    Parallel connection increases total capacitance
  \end{itemize}
\item
  \textbf{Effective Area Increases:}

  \begin{itemize}
  \tightlist
  \item
    Parallel connection is like increasing plate area
  \item
    Since C ∝ A, capacitance increases
  \end{itemize}
\item
  \textbf{Charge Distribution:}

  \begin{itemize}
  \tightlist
  \item
    Larger capacitor stores more charge
  \item
    Q ∝ C
  \item
    Q_{1}/Q_{2} = C_{1}/C_{2}
  \end{itemize}
\item
  \textbf{Current Distribution:}

  \begin{itemize}
  \tightlist
  \item
    Total current: I = I_{1} + I_{2} + I_{3}
  \item
    Each branch carries different current
  \end{itemize}
\item
  \textbf{Energy Storage:}

  \begin{itemize}
  \tightlist
  \item
    Energy in each: U_{1} = ½C_{1}V^{2}
  \item
    Total energy: U = ½C_{p}V^{2}
  \item
    U = U_{1} + U_{2} + U_{3}
  \end{itemize}
\end{enumerate}

\textbf{Example Calculation:}

For C_{1} = 10 μF, C_{2} = 20 μF, C_{3} = 30 μF in parallel:

\begin{verbatim}
C_{p} = C_{1} + C_{2} + C_{3}
C_{p} = 10 + 20 + 30
C_{p} = 60 μF
\end{verbatim}

If V = 12V, charges are:

\begin{verbatim}
Q_{1} = C_{1}V = 10 \times 12 = 120 μC
Q_{2} = C_{2}V = 20 \times 12 = 240 μC
Q_{3} = C_{3}V = 30 \times 12 = 360 μC
Total

Q = 120 + 240 + 360 = 720 μC

\end{verbatim}

Verification:

\begin{verbatim}
Q = C_{p}V = 60 \times 12 = 720 μC ✓
\end{verbatim}

\textbf{Applications:}

\begin{enumerate}
\tightlist
\item
  \textbf{Increased Capacitance:}

  \begin{itemize}
  \tightlist
  \item
    Getting higher capacitance from smaller units
  \item
    When large capacitance needed
  \end{itemize}
\item
  \textbf{Increased Current Capacity:}

  \begin{itemize}
  \tightlist
  \item
    Current divided among capacitors
  \item
    Reduces stress on individual capacitors
  \end{itemize}
\item
  \textbf{Power Factor Correction:}

  \begin{itemize}
  \tightlist
  \item
    Banks of capacitors in parallel
  \item
    Industrial power systems
  \end{itemize}
\item
  \textbf{Energy Storage:}

  \begin{itemize}
  \tightlist
  \item
    More energy stored than individual capacitors
  \item
    UPS systems, electric vehicles
  \end{itemize}
\item
  \textbf{Filtering:}

  \begin{itemize}
  \tightlist
  \item
    Multiple frequency filtering
  \item
    Different capacitors for different frequencies
  \end{itemize}
\end{enumerate}

\textbf{Advantages:}

\begin{itemize}
\tightlist
\item
  Increased total capacitance
\item
  Same voltage across all
\item
  If one fails (short circuit), others continue working
\item
  Redundancy and reliability
\item
  Easy to add or remove capacitors
\end{itemize}

\textbf{Disadvantages:}

\begin{itemize}
\tightlist
\item
  Takes more space
\item
  Higher cost (more capacitors)
\item
  If one shorts, entire circuit affected
\end{itemize}

\textbf{Comparison Table:}

\begin{longtable}[]{@{}lll@{}}
\toprule\noalign{}
Property & Series & Parallel \\
\midrule\noalign{}
\endhead
\bottomrule\noalign{}
\endlastfoot
\textbf{Connection} & End-to-end & All +ve together, all -ve together \\
\textbf{Charge} & Q_{1} = Q_{2} = Q_{3} = Q & Q = Q_{1} + Q_{2} + Q_{3} \\
\textbf{Voltage} & V = V_{1} + V_{2} + V_{3} & V_{1} = V_{2} = V_{3} = V \\
\textbf{Capacitance} & 1/C = 1/C_{1} + 1/C_{2} + 1/C_{3} & C = C_{1} + C_{2} + C_{3} \\
\textbf{Effect on C} & Decreases (C \textless{} smallest) & Increases (C
\textgreater{} largest) \\
\textbf{Analogy} & Like increasing d & Like increasing A \\
\textbf{Use} & High voltage circuits & High capacitance needs \\
\end{longtable}

\textbf{Practical Example - Three Identical Capacitors:}

For C_{1} = C_{2} = C_{3} = C = 10 μF:

\textbf{Series:}

\begin{verbatim}
C_{s} = C/n = 10/3 = 3.33 μF
\end{verbatim}

\textbf{Parallel:}

\begin{verbatim}
C_{p} = nC = 3 \times 10 = 30 μF
\end{verbatim}

This shows parallel gives 9 times more capacitance than series!

\textbf{Mixed Connections:}

Real circuits often use both series and parallel combinations:

\begin{verbatim}
Example: (C_{1} ∥ C_{2}) in series with C_{3}
\end{verbatim}

Step 1: Find parallel combination

\begin{verbatim}
C_{1}_{2} = C_{1} + C_{2}
\end{verbatim}

Step 2: Combine in series with C_{3}

\begin{verbatim}
1/C_{t}_{o}_{t}_{a}_{l} = 1/C_{1}_{2} + 1/C_{3}
\end{verbatim}

\end{solutionbox}
\begin{center}\rule{0.5\linewidth}{0.5pt}\end{center}

\subsubsection{(6) Explain effect of dielectric material on the
capacitance of parallel
plate.}\label{explain-effect-of-dielectric-material-on-the-capacitance-of-parallel-plate.}

\begin{solutionbox}

\textbf{Dielectric Material:} A dielectric is an insulating
(non-conducting) material that can be polarized by an electric field.
When placed between the plates of a capacitor, it significantly affects
the capacitance.

\textbf{Common Dielectric Materials:}

\begin{itemize}
\tightlist
\item
  Air (K = 1.0)
\item
  Paper (K \approx 3.7)
\item
  Mica (K \approx 5.5)
\item
  Glass (K \approx 4.5-10)
\item
  Ceramic (K \approx 6-1000)
\item
  Polyester (K \approx 3.3)
\item
  Teflon (K \approx 2.1)
\end{itemize}

\textbf{Effect of Dielectric on Capacitance:}

\textbf{1. Without Dielectric (Air/Vacuum):}

Capacitance of parallel plate capacitor:

\begin{verbatim}
C_{0} = ε_{0}A/d
\end{verbatim}

Where:

\begin{itemize}
\tightlist
\item
  C_{0} = Capacitance with air/vacuum
\item
  ε_{0} = 8.85 \times 10^{-}^{1}^{2} F/m (permittivity of free space)
\item
  A = Area of plates
\item
  d = Distance between plates
\end{itemize}

\textbf{2. With Dielectric Material:}

When dielectric is inserted:

\begin{verbatim}
C = ε_{0}εᵣA/d = ε_{0}KA/d

C = KC_{0}
\end{verbatim}

Where:

\begin{itemize}
\tightlist
\item
  C = New capacitance with dielectric
\item
  εᵣ or K = Relative permittivity (dielectric constant)
\item
  K \textgreater{} 1 for all materials except vacuum
\end{itemize}

\textbf{Dielectric Constant (K):}

\begin{verbatim}
K = C/C_{0} = Capacitance with dielectric / Capacitance without dielectric
\end{verbatim}

\textbf{Key Observation:}

\begin{itemize}
\tightlist
\item
  \textbf{Capacitance increases by factor K}
\item
  \textbf{K is always \geq 1} (K = 1 for vacuum)
\item
  Higher K \rightarrow Higher capacitance
\end{itemize}

\textbf{How Dielectric Increases Capacitance:}

\textbf{Physical Mechanism:}

\begin{enumerate}
\tightlist
\item
  \textbf{Polarization:}

  \begin{itemize}
  \tightlist
  \item
    Dielectric molecules align in electric field
  \item
    Positive charges slightly toward negative plate
  \item
    Negative charges slightly toward positive plate
  \end{itemize}
\end{enumerate}

\begin{verbatim}
Without Dielectric:         With Dielectric:
+++++++++++                 +++++++++++
║ ║ ║ ║ ║                  \oplus ⊖ \oplus ⊖ \oplus ⊖
║ ║ ║ ║ ║     \rightarrow            ⊖ \oplus ⊖ \oplus ⊖ \oplus
───────────                 ⊖ \oplus ⊖ \oplus ⊖ \oplus
                            ───────────
    E_{0}                           E
\end{verbatim}

\begin{enumerate}
\tightlist
\item
  \textbf{Induced Surface Charges:}

  \begin{itemize}
  \tightlist
  \item
    Negative charges appear on dielectric surface near positive plate
  \item
    Positive charges appear on surface near negative plate
  \item
    These are called bound charges (cannot move freely)
  \end{itemize}
\item
  \textbf{Reduced Electric Field:}

  \begin{itemize}
  \tightlist
  \item
    Internal electric field opposes applied field
  \item
    Net field: E = E_{0}/K
  \item
    External field partially cancelled
  \end{itemize}
\item
  \textbf{Same Charge, Lower Voltage:}

  \begin{itemize}
  \tightlist
  \item
    Voltage: V = Ed
  \item
    Since E decreases, V decreases
  \item
    V = V_{0}/K
  \item
    Capacitance: C = Q/V increases
  \end{itemize}
\end{enumerate}

\textbf{Mathematical Analysis:}

\textbf{At Constant Charge (Q constant):}

Without dielectric:

\begin{verbatim}
V_{0} = Q/C_{0}
E_{0} = V_{0}/d
\end{verbatim}

With dielectric:

\begin{verbatim}
E = E_{0}/K
V = Ed = (E_{0}/K)d = V_{0}/K
C = Q/V = Q/(V_{0}/K) = K(Q/V_{0}) = KC_{0}
\end{verbatim}

\textbf{At Constant Voltage (Battery connected):}

Without dielectric:

\begin{verbatim}
Q_{0} = C_{0}V
\end{verbatim}

With dielectric:

\begin{verbatim}
C = KC_{0}
Q = CV = KC_{0}V = KQ_{0}
\end{verbatim}

More charge flows from battery!

\textbf{Effects of Dielectric:}

\begin{longtable}[]{@{}llll@{}}
\toprule\noalign{}
Property & Without Dielectric & With Dielectric (K) & Change \\
\midrule\noalign{}
\endhead
\bottomrule\noalign{}
\endlastfoot
\textbf{Capacitance} & C_{0} & KC_{0} & Increases K times \\
\textbf{Electric Field} & E_{0} & E_{0}/K & Decreases K times \\
\textbf{Voltage} (Q constant) & V_{0} & V_{0}/K & Decreases K times \\
\textbf{Charge} (V constant) & Q_{0} & KQ_{0} & Increases K times \\
\textbf{Energy} (Q constant) & U_{0} & U_{0}/K & Decreases K times \\
\textbf{Energy} (V constant) & U_{0} & KU_{0} & Increases K times \\
\end{longtable}

\textbf{Energy Considerations:}

\textbf{Case 1: Battery Disconnected (Q constant):}

\begin{verbatim}
U_{0} = Q^{2}/(2C_{0})
U = Q^{2}/(2C) = Q^{2}/(2KC_{0}) = U_{0}/K
\end{verbatim}

Energy decreases! Where does it go?

\begin{itemize}
\tightlist
\item
  Converted to mechanical work (dielectric pulled in)
\item
  Heat due to molecular alignment
\end{itemize}

\textbf{Case 2: Battery Connected (V constant):}

\begin{verbatim}
U_{0} = ½C_{0}V^{2}
U = ½CV^{2} = ½(KC_{0})V^{2} = KU_{0}
\end{verbatim}

Energy increases! From where?

\begin{itemize}
\tightlist
\item
  Battery supplies additional energy
\item
  Work done in polarizing dielectric
\end{itemize}

\textbf{Advantages of Using Dielectric:}

\begin{enumerate}
\tightlist
\item
  \textbf{Increased Capacitance:}

  \begin{itemize}
  \tightlist
  \item
    Get more capacitance without increasing size
  \item
    C = KC_{0} where K can be 2-1000
  \end{itemize}
\item
  \textbf{Higher Breakdown Voltage:}

  \begin{itemize}
  \tightlist
  \item
    Dielectric can withstand higher fields than air
  \item
    Prevents spark/discharge between plates
  \item
    Typical: Air \textasciitilde3 kV/mm, Mica \textasciitilde200 kV/mm
  \end{itemize}
\item
  \textbf{Reduced Physical Size:}

  \begin{itemize}
  \tightlist
  \item
    For given capacitance, can reduce area or increase distance
  \item
    More compact capacitors
  \end{itemize}
\item
  \textbf{Mechanical Support:}

  \begin{itemize}
  \tightlist
  \item
    Keeps plates at fixed distance
  \item
    Prevents short circuit
  \item
    Structural stability
  \end{itemize}
\item
  \textbf{Protection:}

  \begin{itemize}
  \tightlist
  \item
    Prevents moisture, dust entry
  \item
    Longer life
  \end{itemize}
\end{enumerate}

\textbf{Practical Applications:}

\begin{enumerate}
\tightlist
\item
  \textbf{Commercial Capacitors:}

  \begin{itemize}
  \tightlist
  \item
    Paper: K \approx 3.7 (cheap, low voltage)
  \item
    Mica: K \approx 5.5 (precision, stable)
  \item
    Ceramic: K up to 1000 (compact, high C)
  \end{itemize}
\item
  \textbf{Electrolytic Capacitors:}

  \begin{itemize}
  \tightlist
  \item
    Aluminum oxide: K \approx 8
  \item
    Tantalum oxide: K \approx 25
  \item
    Very high capacitance in small size
  \end{itemize}
\item
  \textbf{Variable Capacitors:}

  \begin{itemize}
  \tightlist
  \item
    Changing dielectric position changes C
  \item
    Used in radio tuning
  \end{itemize}
\end{enumerate}

\textbf{Breakdown and Dielectric Strength:}

\textbf{Dielectric Strength:} Maximum electric field a dielectric can
withstand before breaking down (conducting).

\begin{verbatim}
E_max = V_max/d
\end{verbatim}

\begin{longtable}[]{@{}ll@{}}
\toprule\noalign{}
Material & Dielectric Strength (kV/mm) \\
\midrule\noalign{}
\endhead
\bottomrule\noalign{}
\endlastfoot
Air & 3 \\
Paper & 16 \\
Mica & 200 \\
Teflon & 60 \\
Glass & 30 \\
\end{longtable}

\textbf{Summary Formula:}

\begin{verbatim}
C = ε_{0}εᵣA/d = ε_{0}KA/d = εA/d

Where:
-

ε = ε_{0}εᵣ = absolute permittivity of medium

- ε_{0} = 8.85 \times 10^{-}^{1}^{2} F/m
- εᵣ = K = dielectric constant (\geq 1)
\end{verbatim}

\textbf{Conclusion:}

Inserting a dielectric between capacitor plates: ✓ Increases capacitance
by factor K ✓ Decreases electric field by factor K ✓ Increases breakdown
voltage ✓ Makes capacitors more compact ✓ Provides mechanical support
and protection

This makes dielectric materials essential for practical capacitor design
and applications.

\end{solutionbox}
\begin{center}\rule{0.5\linewidth}{0.5pt}\end{center}

\subsection*{Part C: Numerical Solutions (3
marks)}\label{part-c-numerical-solutions-3-marks}

\subsubsection{(1) Two charges with value of 20 µC and 10 µC are
separated 0.02 m distance in air. Find electric force or coulomb force
between these charges. K value is 9 \times 10^{9} N
m^{2}/C^{2}.}\label{two-charges-with-value-of-20-uxb5c-and-10-uxb5c-are-separated-0.02-m-distance-in-air.-find-electric-force-or-coulomb-force-between-these-charges.-k-value-is-9-10ux2079-n-muxb2cuxb2.}

\textbf{Given:}

\begin{itemize}
\tightlist
\item
  Charge 1: q_{1} = 20 μC = 20 \times 10^{-}^{6} C
\item
  Charge 2: q_{2} = 10 μC = 10 \times 10^{-}^{6} C
\item
Distance:

r = 0.02

m = 2 \times 10^{-}^{2} m

\item
  Coulomb's constant: k = 9 \times 10^{9} N·m^{2}/C^{2}
\item
  Medium: Air (εᵣ = 1)
\end{itemize}

\textbf{To Find:}

\begin{itemize}
\tightlist
\item
  Electric force: F = ?
\end{itemize}

\textbf{Formula:}

\begin{verbatim}
F = k(q_{1}q_{2})/r^{2}
\end{verbatim}

\textbf{Solution:}

\begin{verbatim}
F = k(q_{1}q_{2})/r^{2}

F = (9 \times 10^{9}) \times (20 \times 10^{-}^{6}) \times (10 \times 10^{-}^{6}) / (0.02)^{2}

F = (9 \times 10^{9}) \times (200 \times 10^{-}^{1}^{2}) / (4 \times 10^{-}^{4})

F = (1800 \times 10^{-}^{3}) / (4 \times 10^{-}^{4})

F = (1800 \times 10^{-}^{3}) \times (10^{4}/4)

F = 1800 \times 10^{1} / 4

F = 18000 / 4

F = 4500 N

F = 4.5 \times 10^{3} N
\end{verbatim}

\textbf{Alternative Method:}

\begin{verbatim}
F = (9 \times 10^{9}) \times (20 \times 10^{-}^{6}) \times (10 \times 10^{-}^{6}) / (2 \times 10^{-}^{2})^{2}

F = (9 \times 20 \times 10) \times 10^{9}^{-}^{6}^{-}^{6} / (4 \times 10^{-}^{4})

F = 1800 \times 10^{-}^{3} / (4 \times 10^{-}^{4})

F = 1800/4 \times 10^{-}^{3}^{+}^{4}

F = 450 \times 10^{1}

F = 4500 N
\end{verbatim}

\begin{solutionbox}
The electric force (Coulomb force) between the two
charges is \textbf{4500 N or 4.5 \times 10^{3} N}.

\textbf{Nature of Force:} Since both charges are positive, the force is
\textbf{repulsive}.

\end{solutionbox}
\begin{center}\rule{0.5\linewidth}{0.5pt}\end{center}

\subsubsection{(2) 1600 Joule of work is done in moving a charge 25
coulomb from one point to the other. Calculate the potential difference
between the
points.}\label{joule-of-work-is-done-in-moving-a-charge-25-coulomb-from-one-point-to-the-other.-calculate-the-potential-difference-between-the-points.}

\textbf{Given:}

\begin{itemize}
\tightlist
\item
  Work done: W = 1600 J
\item
  Charge moved: q = 25 C
\end{itemize}

\textbf{To Find:}

\begin{itemize}
\tightlist
\item
  Potential difference: V = ?
\end{itemize}

\textbf{Formula:}

\begin{verbatim}
V = W/q
\end{verbatim}

Where:

\begin{itemize}
\tightlist
\item
  V = Potential difference (Volt)
\item
  W = Work done (Joule)
\item
  q = Charge (Coulomb)
\end{itemize}

\textbf{Solution:}

\begin{verbatim}
V = W/q

V = 1600/25

V = 64 V
\end{verbatim}

\begin{solutionbox}
The potential difference between the two points is
\textbf{64 Volts}.

\textbf{Physical Meaning:}

\begin{itemize}
\tightlist
\item
  64 Joules of work is needed to move 1 Coulomb of charge between these
  points
\item
  This also means the electric potential at the first point is 64 V
  higher than at the second point
\end{itemize}

\end{solutionbox}
\begin{center}\rule{0.5\linewidth}{0.5pt}\end{center}

\subsubsection{(3) A capacitor gets a charge 60 µC when it is connected
to a battery of e.m.f. 12 V. Calculate the capacitance of the
capacitor.}\label{a-capacitor-gets-a-charge-60-uxb5c-when-it-is-connected-to-a-battery-of-e.m.f.-12-v.-calculate-the-capacitance-of-the-capacitor.}

\textbf{Given:}

\begin{itemize}
\tightlist
\item
Charge stored:

Q = 60 μC = 60 \times 10^{-}^{6} C

\item
  Voltage applied: V = 12 V
\end{itemize}

\textbf{To Find:}

\begin{itemize}
\tightlist
\item
  Capacitance: C = ?
\end{itemize}

\textbf{Formula:}

\begin{verbatim}
C = Q/V
\end{verbatim}

Where:

\begin{itemize}
\tightlist
\item
  C = Capacitance (Farad)
\item
  Q = Charge stored (Coulomb)
\item
  V = Voltage (Volt)
\end{itemize}

\textbf{Solution:}

\begin{verbatim}
C = Q/V

C = (60 \times 10^{-}^{6})/12

C = 5 \times 10^{-}^{6} F

C = 5 μF
\end{verbatim}

\begin{solutionbox}
The capacitance of the capacitor is \textbf{5 μF
(microfarad)} or \textbf{5 \times 10^{-}^{6} F}.

\textbf{Verification:}

\begin{verbatim}
Q = CV = 5 \times 10^{-}^{6} \times 12 = 60 \times 10^{-}^{6}

C = 60 μC ✓

\end{verbatim}

\end{solutionbox}
\begin{center}\rule{0.5\linewidth}{0.5pt}\end{center}

\subsubsection{(4) Three capacitors of 10µF are connected in series and
parallel connections in circuit. Find out total capacitance in both
cases.}\label{three-capacitors-of-10uxb5f-are-connected-in-series-and-parallel-connections-in-circuit.-find-out-total-capacitance-in-both-cases.}

\textbf{Given:}

\begin{itemize}
\tightlist
\item
  Three identical capacitors
\item
  C_{1} = C_{2} = C_{3} = C = 10 μF
\item
  n = 3 (number of capacitors)
\end{itemize}

\textbf{To Find:}

\begin{itemize}
\tightlist
\item
  \begin{enumerate}
  \tightlist
  \item
    Total capacitance in series: C_{s} = ?
  \end{enumerate}
\item
  \begin{enumerate}
  \tightlist
  \item
    Total capacitance in parallel: C_{p} = ?
  \end{enumerate}
\end{itemize}

\textbf{Case (a): Series Connection}

\textbf{Formula:}

\begin{verbatim}
For n identical capacitors in series:
C_{s} = C/n
\end{verbatim}

Or using general formula:

\begin{verbatim}
1/C_{s} = 1/C_{1} + 1/C_{2} + 1/C_{3}
\end{verbatim}

\textbf{Solution:}

\textbf{Method 1 (Direct):}

\begin{verbatim}
C_{s} = C/n

C_{s} = 10/3

C_{s} = 3.33 μF
\end{verbatim}

\textbf{Method 2 (General formula):}

\begin{verbatim}
1/C_{s} = 1/C_{1} + 1/C_{2} + 1/C_{3}

1/C_{s} = 1/10 + 1/10 + 1/10

1/C_{s} = 3/10

C_{s} = 10/3

C_{s} = 3.33 μF
\end{verbatim}

\begin{solutionbox}
Total capacitance in series = \textbf{3.33 μF}
or \textbf{10/3 μF}

\end{solutionbox}
\begin{center}\rule{0.5\linewidth}{0.5pt}\end{center}

\textbf{Case (b): Parallel Connection}

\textbf{Formula:}

\begin{verbatim}
For n identical capacitors in parallel:
C_{p} = nC
\end{verbatim}

Or using general formula:

\begin{verbatim}
C_{p} = C_{1} + C_{2} + C_{3}
\end{verbatim}

\textbf{Solution:}

\textbf{Method 1 (Direct):}

\begin{verbatim}
C_{p} = nC

C_{p} = 3 \times 10

C_{p} = 30 μF
\end{verbatim}

\textbf{Method 2 (General formula):}

\begin{verbatim}
C_{p} = C_{1} + C_{2} + C_{3}

C_{p} = 10 + 10 + 10

C_{p} = 30 μF
\end{verbatim}

\begin{solutionbox}
Total capacitance in parallel = \textbf{30
μF}

\end{solutionbox}
\begin{center}\rule{0.5\linewidth}{0.5pt}\end{center}

\textbf{Summary:}

\begin{longtable}[]{@{}lll@{}}
\toprule\noalign{}
Connection & Formula & Total Capacitance \\
\midrule\noalign{}
\endhead
\bottomrule\noalign{}
\endlastfoot
\textbf{Series} & C_{s} = C/n & 3.33 μF \\
\textbf{Parallel} & C_{p} = nC & 30 μF \\
\end{longtable}

\textbf{Observation:}

\begin{itemize}
\tightlist
\item
  Parallel gives 9 times more capacitance than series (30/3.33 \approx 9)
\item
  Series: C_{s} \textless{} individual capacitance
\item
  Parallel: C_{p} \textgreater{} individual capacitance
\end{itemize}

\begin{center}\rule{0.5\linewidth}{0.5pt}\end{center}

\subsubsection{(5) Plate area of one parallel plate capacitor is 10 mm^{2},
which are separated with 1mm distance in air. Calculate capacitance of
capacitor.}\label{plate-area-of-one-parallel-plate-capacitor-is-10-mmuxb2-which-are-separated-with-1mm-distance-in-air.-calculate-capacitance-of-capacitor.}

\textbf{Given:}

\begin{itemize}
\tightlist
\item
Area of plates:

A = 10 mm^{2} = 10 \times 10^{-}^{6} m^{2} = 10^{-}^{5} m^{2}

\item
Distance between plates:

d = 1 mm = 1 \times 10^{-}^{3}

m = 10^{-}^{3} m

\item
Medium: Air (εᵣ = 1,

K = 1)

\item
  Permittivity of free space: ε_{0} = 8.85 \times 10^{-}^{1}^{2} F/m
\end{itemize}

\textbf{To Find:}

\begin{itemize}
\tightlist
\item
  Capacitance: C = ?
\end{itemize}

\textbf{Formula:}

\begin{verbatim}
C = ε_{0}εᵣA/d = ε_{0}A/d  (for air, εᵣ = 1)
\end{verbatim}

\textbf{Solution:}

\begin{verbatim}
C = ε_{0}A/d

C = (8.85 \times 10^{-}^{1}^{2}) \times (10^{-}^{5}) / (10^{-}^{3})

C = 8.85 \times 10^{-}^{1}^{2}^{-}^{5}^{+}^{3}

C = 8.85 \times 10^{-}^{1}^{4} F
\end{verbatim}

\textbf{Converting to picofarads:}

\begin{verbatim}
C = 8.85 \times 10^{-}^{1}^{4} F
C = 0.0885 \times 10^{-}^{1}^{2} F
C = 0.0885 pF
\end{verbatim}

\begin{solutionbox}
The capacitance of the capacitor is \textbf{8.85 \times
10^{-}^{1}^{4} F} or \textbf{0.0885 pF (picofarad)}.

\textbf{Note:} This is a very small capacitance due to the small plate
area. Practical capacitors have much larger plate areas.

\end{solutionbox}
\begin{center}\rule{0.5\linewidth}{0.5pt}\end{center}

\subsubsection{(6) The distance between the plates is 1mm, if we want to
get capacitance of 1F, how much area of plate should
be?}\label{the-distance-between-the-plates-is-1mm-if-we-want-to-get-capacitance-of-1f-how-much-area-of-plate-should-be}

\textbf{Given:}

\begin{itemize}
\tightlist
\item
Distance between plates:

d = 1 mm = 10^{-}^{3} m

\item
  Required capacitance: C = 1 F
\item
  Medium: Air (assuming air, εᵣ = 1)
\item
  Permittivity of free space: ε_{0} = 8.85 \times 10^{-}^{1}^{2} F/m
\end{itemize}

\textbf{To Find:}

\begin{itemize}
\tightlist
\item
  Area of plates: A = ?
\end{itemize}

\textbf{Formula:}

\begin{verbatim}
C = ε_{0}A/d
\end{verbatim}

Rearranging for A:

\begin{verbatim}
A = Cd/ε_{0}
\end{verbatim}

\textbf{Solution:}

\begin{verbatim}
A = Cd/ε_{0}

A = (1) \times (10^{-}^{3}) / (8.85 \times 10^{-}^{1}^{2})

A = 10^{-}^{3} / (8.85 \times 10^{-}^{1}^{2})

A = (1/8.85) \times 10^{-}^{3}^{+}^{1}^{2}

A = 0.113 \times 10^{9} m^{2}

A = 1.13 \times 10^{8} m^{2}

A = 113 \times 10^{6} m^{2}

A = 113,000,000 m^{2}
\end{verbatim}

\textbf{Converting to square kilometers:}

\begin{verbatim}
A = 113 \times 10^{6} m^{2}
A = 113 km^{2}
\end{verbatim}

\begin{solutionbox}
The area of the plate should be \textbf{1.13 \times 10^{8} m^{2}}
or approximately \textbf{113 km^{2}}.

\textbf{Physical Interpretation:} This enormous area (about 10.6 km \times
10.6 km) demonstrates why:

\begin{enumerate}
\tightlist
\item
  \textbf{1 Farad is a huge capacitance} - rarely used in practice
\item
  \textbf{Practical capacitors} use microfarads (μF), nanofarads (nF),
  or picofarads (pF)
\item
  \textbf{Dielectric materials} with high K values are essential to
  achieve reasonable capacitance in practical sizes
\item
  \textbf{Electrolytic capacitors} can achieve high capacitance (up to
  mF) by using very thin dielectric layers and special construction
\end{enumerate}

\textbf{With Dielectric:} If we use a dielectric with K = 1000:

\begin{verbatim}
A = Cd/(ε_{0}K) = 113 \times 10^{6}/1000 = 113,000 m^{2} \approx 0.113 km^{2}
\end{verbatim}

Still very large, but more practical!

\end{solutionbox}
\begin{center}\rule{0.5\linewidth}{0.5pt}\end{center}

\subsubsection{(7) As per the below circuit, calculate total capacitance
value.}\label{as-per-the-below-circuit-calculate-total-capacitance-value.}

\textbf{Note:} Since the circuit diagram is referenced but not visible
in the text, I'll provide solutions for common circuit configurations:

\begin{center}\rule{0.5\linewidth}{0.5pt}\end{center}

\textbf{Case A: Series-Parallel Combination}

Assuming circuit: (C_{1} ∥ C_{2}) in series with C_{3}

Example: C_{1} = 10 μF, C_{2} = 20 μF, C_{3} = 30 μF

\begin{verbatim}
        ┌───┤├─── C_{1} (10μF)
    +───┤           
        └───┤├─── C_{2} (20μF)
        │
    ────┤├──────── C_{3} (30μF)
\end{verbatim}

\textbf{Solution:}

\textbf{Step 1:} Find parallel combination of C_{1} and C_{2}

\begin{verbatim}
C_{1}_{2} = C_{1} + C_{2}
C_{1}_{2} = 10 + 20 = 30 μF
\end{verbatim}

\textbf{Step 2:} Combine C_{1}_{2} in series with C_{3}

\begin{verbatim}
1/C_{t}_{o}_{t}_{a}_{l} = 1/C_{1}_{2} + 1/C_{3}
1/C_{t}_{o}_{t}_{a}_{l} = 1/30 + 1/30
1/C_{t}_{o}_{t}_{a}_{l} = 2/30
C_{t}_{o}_{t}_{a}_{l} = 30/2 = 15 μF
\end{verbatim}

\begin{solutionbox}
Total capacitance = \textbf{15 μF}

\end{solutionbox}
\begin{center}\rule{0.5\linewidth}{0.5pt}\end{center}

\textbf{Case B: Series-Parallel Combination (Alternative)}

Assuming circuit: (C_{1} in series with C_{2}) parallel with C_{3}

Example: C_{1} = 6 μF, C_{2} = 6 μF, C_{3} = 4 μF

\begin{verbatim}
    ┌───┤├───┤├───┐
    │   C_{1}    C_{2}  │
+───┤            ├───
    │    C_{3}      │
    └─────┤├─────┘
\end{verbatim}

\textbf{Solution:}

\textbf{Step 1:} Find series combination of C_{1} and C_{2}

\begin{verbatim}
1/C_{1}_{2} = 1/C_{1} + 1/C_{2}
1/C_{1}_{2} = 1/6 + 1/6 = 2/6
C_{1}_{2} = 6/2 = 3 μF
\end{verbatim}

\textbf{Step 2:} Combine C_{1}_{2} in parallel with C_{3}

\begin{verbatim}
C_{t}_{o}_{t}_{a}_{l} = C_{1}_{2} + C_{3}
C_{t}_{o}_{t}_{a}_{l} = 3 + 4 = 7 μF
\end{verbatim}

\begin{solutionbox}
Total capacitance = \textbf{7 μF}

\end{solutionbox}
\begin{center}\rule{0.5\linewidth}{0.5pt}\end{center}

\textbf{Case C: Complex Network}

Assuming bridge network or Wheatstone arrangement

\textbf{General Approach:}

\begin{enumerate}
\tightlist
\item
  \textbf{Identify series and parallel sections}
\item
  \textbf{Simplify step by step} from outermost to innermost
\item
  \textbf{Use equivalent capacitance formulas} at each step
\item
  \textbf{Continue until single equivalent capacitance} remains
\end{enumerate}

\textbf{Steps:}

\begin{itemize}
\tightlist
\item
  Mark parallel combinations and add: C = C_{1} + C_{2} + \ldots{}
\item
  Mark series combinations and use: 1/C = 1/C_{1} + 1/C_{2} + \ldots{}
\item
  Replace simplified sections with equivalent values
\item
  Repeat until circuit reduces to single capacitor
\end{itemize}

\begin{center}\rule{0.5\linewidth}{0.5pt}\end{center}

\textbf{Common Circuit Configurations:}

\textbf{Configuration 1: Three in Series}

\begin{verbatim}
───┤├───┤├───┤├───
   C_{1}   C_{2}   C_{3}

Result: 1/C = 1/C_{1} + 1/C_{2} + 1/C_{3}
\end{verbatim}

\textbf{Configuration 2: Three in Parallel}

\begin{verbatim}
   ┌──┤├──┐
   │  C_{1}  │
───┤──┤├──┤───
   │  C_{2}  │
   └──┤├──┘
      C_{3}

Result: C = C_{1} + C_{2} + C_{3}
\end{verbatim}

\textbf{Configuration 3: Mixed (Two in Series, then Parallel with
Third)}

\begin{verbatim}
    ┌──┤├──┤├──┐
    │  C_{1}  C_{2} │
+───┤        ├───
    │   C_{3}   │
    └───┤├───┘

Step 1: C_{1}_{2} = (C_{1}C_{2})/(C_{1}+C_{2})
Step 2: C_{t}_{o}_{t}_{a}_{l} = C_{1}_{2} + C_{3}
\end{verbatim}

\textbf{Please provide the specific circuit diagram for an exact
solution!}

\begin{center}\rule{0.5\linewidth}{0.5pt}\end{center}

\subsection*{Additional Important
Formulas}\label{additional-important-formulas}

\subsubsection{Electric Charge:}\label{electric-charge}

\begin{verbatim}
Q = ne  (n = 1, 2, 3, ...)
e = 1.6 \times 10^{-}^{1}^{9} C (elementary charge)
\end{verbatim}

\subsubsection{Coulomb's Law:}\label{coulombs-law}

\begin{verbatim}
F = kq_{1}q_{2}/r^{2}
k = 9 \times 10^{9} N·m^{2}/C^{2}
ε_{0} = 8.85 \times 10^{-}^{1}^{2} F/m
k = 1/(4πε_{0})
\end{verbatim}

\subsubsection{Electric Field:}\label{electric-field}

\begin{verbatim}
E = F/q_{0} = kQ/r^{2}
E (uniform) = V/d
\end{verbatim}

\subsubsection{Electric Potential:}\label{electric-potential}

\begin{verbatim}
V = kQ/r
V = W/q
ΔV = V_{2} - V_{1}
\end{verbatim}

\subsubsection{Electric Flux:}\label{electric-flux}

\begin{verbatim}
Φ = E·A = EA cos θ
\end{verbatim}

\subsubsection{Capacitance:}\label{capacitance}

\begin{verbatim}
C = Q/V
C (parallel plate) = ε_{0}εᵣA/d
C (with dielectric) = KC_{0}
\end{verbatim}

\subsubsection{Series Capacitors:}\label{series-capacitors}

\begin{verbatim}
1/C_{s} = 1/C_{1} + 1/C_{2} + 1/C_{3} + ...
For 2: C_{s} = C_{1}C_{2}/(C_{1}+C_{2})
For n equal: C_{s} = C/n
\end{verbatim}

\subsubsection{Parallel Capacitors:}\label{parallel-capacitors}

\begin{verbatim}
C_{p} = C_{1} + C_{2} + C_{3} + ...
For n equal: C_{p} = nC
\end{verbatim}

\subsubsection{Energy in Capacitor:}\label{energy-in-capacitor}

\begin{verbatim}
U = ½QV = ½CV^{2} = Q^{2}/(2C)
\end{verbatim}

\subsubsection{Constants:}\label{constants}

\begin{verbatim}
e = 1.6 \times 10^{-}^{1}^{9} C
k = 9 \times 10^{9} N·m^{2}/C^{2}
ε_{0} = 8.85 \times 10^{-}^{1}^{2} F/m
1 μF = 10^{-}^{6} F
1 nF = 10^{-}^{9} F
1 pF = 10^{-}^{1}^{2} F
\end{verbatim}

\begin{center}\rule{0.5\linewidth}{0.5pt}\end{center}

\emph{End of Solutions}


\end{document}
