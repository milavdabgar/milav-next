%% METADATA
%% subject-code: DI01000061
%% subject-name: આધુનિક ભૌતિકશાસ્ત્ર
%% semester: 1
%% examination: Winter-2024
%% date: 09-01-2025
%% description: આધુનિક ભૌતિકશાસ્ત્ર (DI01000061) માટે ઉકેલ માર્ગદર્શિકા
%% tags: study-material, solutions, gtu, DI01000061, physics
%% END METADATA

\documentclass{article}
% GTU Solutions - Gujarati Preamble
% Includes common preamble + Gujarati font setup

% Basic setup
\usepackage[margin=1in]{geometry}
\author{Milav Dabgar}

% Math and tables
\usepackage{amsmath,amssymb,amsthm}
\usepackage{booktabs}
\usepackage{tabularx}
\usepackage{graphicx}
\usepackage{float}  % Required for [H] float placement

% Code listings with syntax highlighting
\usepackage{xcolor}
\usepackage{listings}
\lstset{
  basicstyle=\small\ttfamily,
  breaklines=true,
  numbers=left,
  numberstyle=\tiny\color{gray},
  xleftmargin=2em,
  frame=single,
  showstringspaces=false,
  tabsize=2,
  keywordstyle=\color{blue},
  commentstyle=\color{green!60!black},
  stringstyle=\color{purple}
}

% Optional: TikZ for diagrams (remove if not needed)
\usepackage{tikz}
\usepackage{circuitikz}
\usetikzlibrary{shapes,arrows,positioning,calc}

% Header/footer with author and website
\usepackage{fancyhdr}
\usepackage{lastpage}

\pagestyle{fancy}
\fancyhf{}
\fancyhead[L]{\small\itshape\leftmark}
\fancyhead[R]{\small Milav Dabgar}
\fancyfoot[L]{\small\href{https://www.milav.in}{www.milav.in}}
\fancyfoot[R]{\small Page \thepage\ of \pageref{LastPage}}
\renewcommand{\headrulewidth}{0.4pt}
\renewcommand{\footrulewidth}{0.4pt}

% Hyperref (load before fontspec for Gujarati)
\usepackage[
  colorlinks=true,
  linkcolor=blue,
  urlcolor=blue,
  citecolor=blue,
  pdfauthor={Milav Dabgar},
  pdfsubject={GTU Exam Solutions},
  pdfkeywords={GTU, Java, Programming, Solutions, Gujarati},
  bookmarks=true
]{hyperref}

% Gujarati font setup
\usepackage{fontspec}
\usepackage{polyglossia}
\setdefaultlanguage{gujarati}
\setotherlanguage{english}
\newfontfamily\gujaratifont[Script=Gujarati,AutoFakeBold=2.5,AutoFakeSlant=0.3]{Noto Sans Gujarati}
\setmainfont[Script=Gujarati,AutoFakeBold=2.5,AutoFakeSlant=0.3]{Noto Sans Gujarati}
\setmonofont[Scale=0.9]{Noto Sans Gujarati}
\newfontfamily\englishfont[Script=Gujarati,AutoFakeBold=2.5,AutoFakeSlant=0.3]{Noto Sans Gujarati}
\gappto\captionsgujarati{
  \renewcommand{\tablename}{કોષ્ટક}
  \renewcommand{\figurename}{આકૃતિ}
}
\newcommand{\gu}[1]{{\gujaratifont #1}}


\title{આધુનિક ભૌતિકશાસ્ત્ર (DI01000061) - Winter 2024 ઉકેલ}
\date{જાન્યુઆરી 9, 2025}

\hypersetup{
  pdftitle={આધુનિક ભૌતિકશાસ્ત્ર (DI01000061) - Winter 2024 ઉકેલ},
  pdfsubject={GTU Exam Solution - Winter-2024},
  pdfauthor={Milav Dabgar},
  pdfkeywords={study-material, solutions, gtu, DI01000061, physics},
  pdfcreator={xelatex}
}

\begin{document}
\maketitle

\setcounter{tocdepth}{5}
\tableofcontents
\newpage

% ========================================
% QUESTION 1: MCQs and Fill in the Blanks (14 marks)
% Demonstrates: Short answer format, multiple topics
% ========================================

\section{પ્રશ્ન 1}
\textbf{યોગ્ય વિકલ્પ પસંદ કરી ખાલી જગ્યા પૂરો / બહુવિકલ્પ પ્રશ્નોના જવાબ આપો.}

\subsection{પ્રશ્ન 1(1) [1 marks]}
\textbf{નીચેનામાંથી કયું અર્ધવાહક છે?}

\textbf{વિકલ્પો:} (a) Si (b) Cu (c) Fe (d) Ni

\subsubsection{ઉકેલ}
\paragraph{જવાબ:} (a) Si

\paragraph{સમજૂતી:}
સિલિકોન (Si) એ ગ્રુપ 14 નો તત્વ છે જેમાં 4 વેલેન્સ ઇલેક્ટ્રોન હોય છે, જે તેને આંતરિક અર્ધવાહક બનાવે છે. કોપર (Cu), આયર્ન (Fe) અને નિકલ (Ni) એ ધાતુઓ છે જે મુક્ત ઇલેક્ટ્રોનને કારણે ઉચ્ચ વિદ્યુત વાહકતા ધરાવે છે.

\subparagraph{નોંધ:} જર્મેનિયમ (Ge) પણ ગ્રુપ 14 નો અર્ધવાહક છે.

\paragraph{મેમરી ટ્રીક:} \emph{Si = Semiconductor, Cu/Fe/Ni = Metals.}

\subsection{પ્રશ્ન 1(2) [1 marks]}
\textbf{કાચનો વક્રીભવનાંક \_\_\_\_\_ છે.}

\textbf{વિકલ્પો:} (a) 1.50 (b) 1.33 (c) 1.00 (d) 2.43

\subsubsection{ઉકેલ}
\paragraph{જવાબ:} (a) 1.50

\paragraph{સમજૂતી:}
સામાન્ય કાચનો વક્રીભવનાંક આશરે 1.50 હોય છે. પાણીનો \(n = 1.33\), હવા/શૂન્યાવકાશનો \(n = 1.00\) અને હીરાનો \(n = 2.43\) હોય છે.

\subparagraph{નોંધ:} ઉચ્ચ વક્રીભવનાંક એટલે તે માધ્યમમાં પ્રકાશ ધીમો મુસાફરી કરે છે.

\paragraph{મેમરી ટ્રીક:} \emph{Glass = 1.5, Water = 1.33, Diamond = 2.43.}

\subsection{પ્રશ્ન 1(3) [1 marks]}
\textbf{જ્યારે આપાતકોણ ક્રાંતિકોણ કરતાં \_\_\_\_\_ થાય ત્યારે પૂર્ણ આંતરિક પરાવર્તન થાય છે.}

\textbf{વિકલ્પો:} (a) સમાન (b) વધારે (c) ઓછો (d) આ પૈકી કોઈ નહીં

\subsubsection{ઉકેલ}
\paragraph{જવાબ:} (b) વધારે

\paragraph{સમજૂતી:}
પૂર્ણ આંતરિક પરાવર્તન (TIR) ત્યારે થાય છે જ્યારે પ્રકાશ ઘટ્ટ માધ્યમમાંથી પાતળા માધ્યમમાં જાય છે અને આપાત કોણ \(i\) ક્રાંતિકોણ \(C\) કરતા મોટો હોય, એટલે કે \(i > C\).

\paragraph{મેમરી ટ્રીક:} \emph{TIR when i > C (Greater than Critical).}

\subsection{પ્રશ્ન 1(4) [1 marks]}
\textbf{બ્રિજ રેકટીફાયરમાં કેટલા P-N જંકશન ડાયોડનો ઉપયોગ થાય છે?}

\textbf{વિકલ્પો:} (a) 2 (b) 3 (c) 4 (d) 5

\subsubsection{ઉકેલ}
\paragraph{જવાબ:} (c) 4

\paragraph{સમજૂતી:}
બ્રિજ રેકટીફાયર AC ને પૂર્ણ-તરંગ DC માં રૂપાંતરિત કરવા માટે બ્રિજ કન્ફિગરેશનમાં ગોઠવાયેલા 4 ડાયોડનો ઉપયોગ કરે છે. આ બંને અર્ધ ચક્ર દરમિયાન લોડમાંથી પ્રવાહને એક જ દિશામાં વહેવાની મંજૂરી આપે છે.

\paragraph{મેમરી ટ્રીક:} \emph{Bridge = 4 Diodes.}

\subsection{પ્રશ્ન 1(5) [1 marks]}
\textbf{ઓપ્ટિકલ ફાઈબર \_\_\_\_\_ ના સિદ્ધાંત પર કાર્ય કરે છે.}

\textbf{વિકલ્પો:} (a) વ્યતિકરણ (b) વક્રીભવન (c) ધ્રુવીભવન (d) પૂર્ણ આંતરિક પરાવર્તન

\subsubsection{ઉકેલ}
\paragraph{જવાબ:} (d) પૂર્ણ આંતરિક પરાવર્તન

\paragraph{સમજૂતી:}
ઓપ્ટિકલ ફાઈબર કોર-ક્લેડિંગ ઇન્ટરફેસ પર પુનરાવર્તિત પૂર્ણ આંતરિક પરાવર્તન દ્વારા પ્રકાશ સંકેતો પ્રસારિત કરે છે, જ્યાં કોર ક્લેડિંગ કરતા ઊંચો વક્રીભવનાંક ધરાવે છે.

\paragraph{મેમરી ટ્રીક:} \emph{Fiber = TIR (Total Internal Reflection).}

\subsection{પ્રશ્ન 1(6) [1 marks]}
\textbf{એકમ સમયમાં થતાં દોલનોની સંખ્યાને \_\_\_\_\_ કહે છે.}

\textbf{વિકલ્પો:} (a) આવર્તકાળ (b) તરંગલંબાઈ (c) કંપવિસ્તાર (d) આવૃત્તિ

\subsubsection{ઉકેલ}
\paragraph{જવાબ:} (d) આવૃત્તિ

\paragraph{સમજૂતી:}
આવૃત્તિ (\(f\)) એ એકમ સમય દીઠ સંપૂર્ણ દોલનોની સંખ્યા તરીકે વ્યાખ્યાયિત થાય છે. તેનો એકમ Hertz (Hz) અથવા \(s^{-1}\) છે. સંબંધ છે \(f = 1/T\), જ્યાં \(T\) આવર્તકાળ છે.

\paragraph{મેમરી ટ્રીક:} \emph{Frequency = Oscillations per second.}

\subsection{પ્રશ્ન 1(7) [1 marks]}
\textbf{વિદ્યુતભારનો એસ.આઈ. એકમ \_\_\_\_\_ છે.}

\textbf{વિકલ્પો:} (a) કુલંબ (b) એમ્પિયર (c) વોલ્ટ (d) ફેરાડે

\subsubsection{ઉકેલ}
\paragraph{જવાબ:} (a) કુલંબ

\paragraph{સમજૂતી:}
વિદ્યુતભારનો SI એકમ કુલંબ (C) છે. એક કુલંબ એ એક સેકન્ડમાં એક એમ્પિયરના પ્રવાહ દ્વારા પરિવહન થયેલો ભાર છે (\(1\,C = 1\,A \times 1\,s\)).

\paragraph{મેમરી ટ્રીક:} \emph{Charge = Coulomb (C).}

\subsection{પ્રશ્ન 1(8) [1 marks]}
\textbf{જો સાદા લોલકનો આવર્તકાળ 2 સેકન્ડ હોય તો તેની આવૃત્તિ \_\_\_\_\_ હશે.}

\textbf{વિકલ્પો:} (a) 2 Hz (b) 0.5 Hz (c) 0.2 Hz (d) 5 Hz

\subsubsection{ઉકેલ}
\paragraph{જવાબ:} (b) 0.5 Hz

\paragraph{સમજૂતી:}
આવૃત્તિ (\(f\)) અને આવર્તકાળ (\(T\)) વચ્ચેનો સંબંધ વિલોમ છે. આવૃત્તિ આપણને જણાવે છે કે પ્રતિ સેકન્ડ કેટલા ચક્ર પૂર્ણ થાય છે, જ્યારે આવર્તકાળ જણાવે છે કે એક ચક્રને કેટલો સમય લાગે છે.

\paragraph{ગણતરી:}
\[ f = \frac{1}{T} = \frac{1}{2} = 0.5 \, \text{Hz} \]

\paragraph{મેમરી ટ્રીક:} \emph{f = 1/T.}

\subsection{પ્રશ્ન 1(9) [1 marks]}
\textbf{પ્રકાશનો શૂન્યાવકાશમાં વેગ \_\_\_\_\_ હોય છે.}

\textbf{વિકલ્પો:} (a) 300000 km/s (b) 300000 m/s (c) 341 km/s (d) 341 m/s

\subsubsection{ઉકેલ}
\paragraph{જવાબ:} (a) 300000 km/s

\paragraph{સમજૂતી:}
શૂન્યાવકાશમાં પ્રકાશની ઝડપ \(c = 3 \times 10^8\,m/s = 300000\,km/s\) છે. આ ભૌતિકશાસ્ત્રમાં એક મૂળભૂત અચળાંક છે. નોંધ: 341 m/s એ હવામાં ધ્વનિની ઝડપ છે.

\paragraph{મેમરી ટ્રીક:} \emph{c = 3\texttimes10\textsuperscript{8} m/s = 300000 km/s.}

\subsection{પ્રશ્ન 1(10) [1 marks]}
\textbf{ધ્વનિ તરંગોનો વેગ \_\_\_\_\_ માં મહત્તમ હોય છે.}

\textbf{વિકલ્પો:} (a) પ્રવાહી (b) ઘન (c) વાયુ (d) શૂન્યાવકાશ

\subsubsection{ઉકેલ}
\paragraph{જવાબ:} (b) ઘન

\paragraph{સમજૂતી:}
ધ્વનિ ઘનમાં સૌથી ઝડપથી મુસાફરી કરે છે કારણ કે તેમાં અણુઓ એકબીજાની ખૂબ નજીક હોય છે અને મજબૂત આંતર-આણ્વિક બળો હોય છે. ક્રમ: \(v_{solid} > v_{liquid} > v_{gas}\). ધ્વનિ શૂન્યાવકાશમાં મુસાફરી કરી શકતું નથી.

\paragraph{મેમરી ટ્રીક:} \emph{Solids = Fastest sound.}

\subsection{પ્રશ્ન 1(11) [1 marks]}
\textbf{પ્રકાશના તરંગનું પ્રસરણ \_\_\_\_\_ ને આભારી છે.}

\textbf{વિકલ્પો:} (a) શૃંગ અને ગર્ત (b) સંઘનન અને વિઘનન (c) ફક્ત સંઘનન (d) ફક્ત વિઘનન

\subsubsection{ઉકેલ}
\paragraph{જવાબ:} (a) શૃંગ અને ગર્ત

\paragraph{સમજૂતી:}
પ્રકાશ તરંગો એ ટ્રાન્સવર્સ ઇલેક્ટ્રોમેગ્નેટિક તરંગો છે જે વૈકલ્પિક શૃંગ અને ગર્ત દ્વારા પ્રસરે છે. સંઘનન અને વિઘનન એ ધ્વનિ જેવા લોન્ગીટ્યુડિનલ તરંગો સાથે સંકળાયેલા છે.

\paragraph{મેમરી ટ્રીક:} \emph{Light = Transverse = Crest/Trough.}

\subsection{પ્રશ્ન 1(12) [1 marks]}
\textbf{LASER વિકિરણ \_\_\_\_\_ છે.}

\textbf{વિકલ્પો:} (a) બહુરંગી (b) એકરંગી (c) ઓછું તીવ્ર (d) આ પૈકી કોઈ નહીં

\subsubsection{ઉકેલ}
\paragraph{જવાબ:} (b) એકરંગી

\paragraph{સમજૂતી:}
LASER (Light Amplification by Stimulated Emission of Radiation) એકરંગી પ્રકાશ ઉત્પન્ન કરે છે, એટલે કે તેની એક જ ચોક્કસ તરંગલંબાઈ હોય છે. તે સુસંગત અને અત્યંત તીવ્ર પણ હોય છે.

\paragraph{મેમરી ટ્રીક:} \emph{LASER = Monochromatic, Coherent, Directional.}

\subsection{પ્રશ્ન 1(13) [1 marks]}
\textbf{કયો ફાઈબર લાંબી બેન્ડવિથ આપે છે?}

\textbf{વિકલ્પો:} (a) સિંગલ મોડ (b) મલ્ટી મોડ સ્ટેપ ઇન્ડેક્સ (c) સ્ટેપ ઇન્ડેક્સ (d) આ પૈકી કોઈ નહીં

\subsubsection{ઉકેલ}
\paragraph{જવાબ:} (a) સિંગલ મોડ

\paragraph{સમજૂતી:}
સિંગલ મોડ ફાઈબરમાં ખૂબ જ નાનો કોર વ્યાસ હોય છે અને કેવળ એક જ મોડના પ્રકાશ પ્રસરણને મંજૂરી આપે છે, જે લઘુત્તમ વિક્ષેપણ અને મહત્તમ બેન્ડવિથમાં પરિણમે છે. તેનો ઉપયોગ લાંબા અંતરના સંદેશાવ્યવહાર માટે થાય છે.

\paragraph{મેમરી ટ્રીક:} \emph{Single mode = Long distance, High bandwidth.}

\subsection{પ્રશ્ન 1(14) [1 marks]}
\textbf{0.5 ન્યૂમેરિકલ એપર્ચર ધરાવતા ઓપ્ટિકલ ફાઈબરનો એક્સેપ્ટન્સ એંગલ કેટલો હશે?}

\textbf{વિકલ્પો:} (a) \(30^\circ\) (b) \(45^\circ\) (c) \(60^\circ\) (d) \(15^\circ\)

\subsubsection{ઉકેલ}
\paragraph{જવાબ:} (a) \(30^\circ\)

\paragraph{સમજૂતી:}
એક્સેપ્ટન્સ એંગલ એ મહત્તમ કોણ છે જેના પર પ્રકાશ ઓપ્ટિકલ ફાઈબરમાં પ્રવેશ કરી શકે છે અને હજુ પણ પૂર્ણ આંતરિક પરાવર્તન દ્વારા પ્રસારિત થઈ શકે છે. તે ન્યૂમેરિકલ એપર્ચર પર આધાર રાખે છે, જે ફાઈબરની પ્રકાશ-એકત્રીકરણ ક્ષમતાનું માપ છે.

\paragraph{ગણતરી:}
ન્યૂમેરિકલ એપર્ચર (NA) એ એક્સેપ્ટન્સ એંગલ (\(\theta_a\)) સાથે નીચેના સંબંધ દ્વારા જોડાયેલ છે:
\[ NA = \sin(\theta_a) \]
\[ 0.5 = \sin(\theta_a) \]
\[ \theta_a = \sin^{-1}(0.5) = 30^\circ \]

\paragraph{મેમરી ટ્રીક:} \emph{NA = sin(acceptance angle).}

% ========================================
% QUESTION 2: Theory Questions (14 marks)
% Q2(A): 3 marks each, Q2(B): 4 marks each
% ========================================

\section{પ્રશ્ન 2}

\subsection{પ્રશ્ન 2(a)(1) [3 marks]}
\textbf{ચોકસાઈ અને સચોટતા વચ્ચેનો તફાવત આપો.}

\subsubsection{ઉકેલ}
\paragraph{વ્યાખ્યાઓ:}
\begin{itemize}
    \item \textbf{ચોકસાઈ (Accuracy):} માપેલું મૂલ્ય \textbf{સાચા (standard) મૂલ્ય}ની કેટલી નજીક છે તેને ચોકસાઈ કહે છે. તે માપન કેટલું સાચું છે તે દર્શાવે છે અને વ્યવસ્થિત ત્રુટિઓ પર આધાર રાખે છે. ઉચ્ચ ચોકસાઈ એટલે ઓછી ત્રુટિ.
    \item \textbf{સચોટતા (Precision):} બે કે તેથી વધુ માપેલા મૂલ્યો \textbf{એકબીજાની કેટલી નજીક} છે તેને સચોટતા કહે છે. તે માપનનું વિભેદન અથવા પુનરુત્પાદનક્ષમતા દર્શાવે છે. તે સાધનની લઘુત્તમ માપ શક્તિ પર આધાર રાખે છે. ઉચ્ચ સચોટતા એટલે મૂલ્યો એકબીજાની ખૂબ નજીક હોય છે.
\end{itemize}

\paragraph{મુખ્ય તફાવત:}
માપન ચોક્કસ હોય વગર સચોટ હોઈ શકે છે (સતત ખોટું), અથવા સચોટ હોય વગર ચોક્કસ હોઈ શકે છે (સરેરાશ સાચું પરંતુ વિખરાયેલું).

\paragraph{ઉદાહરણ:}
જો સાચું મૂલ્ય 10.0 cm હોય:
\begin{itemize}
    \item ઉચ્ચ ચોકસાઈ, ઉચ્ચ સચોટતા: 10.0, 10.1, 9.9 cm
    \item નીચી ચોકસાઈ, ઉચ્ચ સચોટતા: 8.0, 8.1, 7.9 cm (સતત ઓછું)
    \item ઉચ્ચ ચોકસાઈ, નીચી સચોટતા: 10.0, 12.0, 8.0 cm (સાચા મૂલ્યની આસપાસ વિખરાયેલું)
\end{itemize}

\paragraph{મેમરી ટ્રીક:} \emph{Accuracy = Correctness, Precision = Consistency.}

\subsection{પ્રશ્ન 2(a)(2) [3 marks]}
\textbf{માઇક્રોમીટર સ્ક્રૂ દ્વારા માપવામાં આવતા ગોળાનો વ્યાસ નક્કી કરવા, મુખ્ય માપપટ્ટીનું માપ 5 mm અને વર્તુળાકાર માપપટ્ટીનો 50 મો વિભાગ બેઝ લાઇન સાથે મેળ ખાય છે. આ સાધનની લ.મા.શ 0.01 mm છે.}

\subsubsection{ઉકેલ}
\paragraph{આપેલ માહિતી:}
\begin{itemize}
    \item મુખ્ય માપપટ્ટી વાંચન (MSR) = 5 mm
    \item વર્તુળાકાર માપપટ્ટી વિભાગ (CSD) = 50
    \item લઘુત્તમ માપ શક્તિ (LC) = 0.01 mm
\end{itemize}

\paragraph{સૂત્ર:}
માઇક્રોમીટર સ્ક્રૂ ગેજ માટે:
\[ \text{Reading} = \text{MSR} + (\text{CSD} \times \text{LC}) \]

\paragraph{ગણતરી:}
\[ \text{Diameter} = 5 + (50 \times 0.01) \]
\[ \text{Diameter} = 5 + 0.50 = 5.50 \, \text{mm} \]

\paragraph{જવાબ:}
ગોળાનો વ્યાસ \textbf{5.50 mm} છે.

\paragraph{મેમરી ટ્રીક:} \emph{MSR + (CSD × LC) = Total Reading.}

\subsection{પ્રશ્ન 2(a)(3) [3 marks]}
\textbf{જ્યારે 4 \(\mu\)F કેપેસિટન્સ ધરાવતા કેપેસિટરને 12 volt બેટરી સાથે જોડતા કેપેસિટરની બંને પ્લેટ પર સંગ્રહિત થતાં વિદ્યુતભારના જથ્થાની ગણતરી કરો.}

\subsubsection{ઉકેલ}
\paragraph{આપેલ માહિતી:}
\begin{itemize}
    \item કેપેસિટન્સ, \(C = 4\,\mu F = 4 \times 10^{-6}\,F\)
    \item વોલ્ટેજ, \(V = 12\,V\)
\end{itemize}

\paragraph{સૂત્ર:}
કેપેસિટર પર સંગ્રહિત થતો ભાર આપવામાં આવે છે:
\[ Q = CV \]
જ્યાં \(Q\) એ કુલંબમાં ભાર છે, \(C\) એ ફેરાડમાં કેપેસિટન્સ છે, અને \(V\) એ વોલ્ટમાં સંભવિત તફાવત છે. આ મૂળભૂત સંબંધ દર્શાવે છે કે સંગ્રહિત ભાર કેપેસિટન્સ અને વોલ્ટેજ બંનેના સીધા પ્રમાણમાં છે. કેપેસિટર તેની પ્લેટો વચ્ચે વિદ્યુત ક્ષેત્રમાં વિદ્યુત ઊર્જા સંગ્રહિત કરે છે.

\paragraph{ગણતરી:}
\[ Q = 4 \times 10^{-6} \times 12 \]
\[ Q = 48 \times 10^{-6} \, C \]
\[ Q = 48 \, \mu C \]

\paragraph{જવાબ:}
બંને પ્લેટ પર સંગ્રહિત થયેલા વિદ્યુતભારનું પ્રમાણ \textbf{48 \(\mu\)C} છે. નોંધ કરો કે એક પ્લેટ પર +48 \(\mu\)C અને બીજી પર -48 \(\mu\)C છે, જે ચોખ્ખો ભાર શૂન્ય બનાવે છે, પરંતુ દરેક પ્લેટ સમાન માત્રાનો ભાર સંગ્રહિત કરે છે.

\paragraph{મેમરી ટ્રીક:} \emph{Q = CV (Charge = Capacitance × Voltage).}

\subsection{પ્રશ્ન 2(b)(1) [4 marks]}
\textbf{યોગ્ય નામકરણ સાથે માઇક્રોમીટર સ્ક્રૂ ગેજની આકૃતિ દોરો.}

\subsubsection{ઉકેલ}
\paragraph{આકૃતિ:}
\begin{figure}[H]
\centering
\begin{tikzpicture}[scale=1.2]
    % Frame
    \draw[thick] (0,0) -- (0,1.5) -- (6,1.5) -- (6,0) -- cycle;
    % Anvil
    \fill[gray!30] (0.5,0.5) rectangle (1,1);
    \draw[thick] (0.5,0.5) rectangle (1,1);
    \node[below] at (0.75,0.5) {\small Anvil};
    % Spindle
    \fill[gray!30] (4,0.5) rectangle (4.5,1);
    \draw[thick] (4,0.5) rectangle (4.5,1);
    \node[below] at (4.25,0.5) {\small Spindle};
    % Sleeve (Main Scale)
    \draw[thick] (2,0.6) rectangle (4,0.9);
    \draw (2.2,0.6) -- (2.2,0.9);
    \draw (2.4,0.6) -- (2.4,0.9);
    \draw (2.6,0.6) -- (2.6,0.9);
    \node[above] at (3,0.9) {\small Main Scale};
    % Thimble (Circular Scale)
    \draw[thick,fill=gray!20] (3.5,0.4) rectangle (4.5,1.1);
    \draw (3.7,0.4) -- (3.7,1.1);
    \draw (3.9,0.4) -- (3.9,1.1);
    \draw (4.1,0.4) -- (4.1,1.1);
    \draw (4.3,0.4) -- (4.3,1.1);
    \node[above] at (4,1.1) {\small Thimble};
    \node[below] at (4,0.4) {\small Circular Scale};
    % Ratchet
    \draw[thick,fill=gray!40] (4.8,0.6) circle (0.3);
    \node[right] at (5.1,0.6) {\small Ratchet};
    % Frame label
    \node[left] at (0,0.75) {\small Frame};
\end{tikzpicture}
\caption{માઇક્રોમીટર સ્ક્રૂ ગેજ}
\end{figure}

\paragraph{મુખ્ય ભાગો:}
\begin{enumerate}
    \item \textbf{ફ્રેમ (Frame):} C આકારનું કઠણ શરીર જે બધા ભાગોને પકડી રાખે છે
    \item \textbf{એન્વિલ (Anvil):} સ્થિર છેડો જેની સામે વસ્તુ મૂકવામાં આવે છે
    \item \textbf{સ્પિન્ડલ (Spindle):} ગતિશીલ ભાગ જે એન્વિલ તરફ આગળ વધે છે
    \item \textbf{સ્લીવ (Sleeve) - મુખ્ય માપપટ્ટી:} મિલિમીટર વિભાગો બતાવે છે
    \item \textbf{થિમ્બલ (Thimble) - વર્તુળાકાર માપપટ્ટી:} અંશ વિભાગો બતાવે છે (0-50 અથવા 0-100)
    \item \textbf{રેચેટ (Ratchet):} માપન દરમિયાન એકરૂપ દબાણ સુનિશ્ચિત કરે છે
\end{enumerate}

\paragraph{મેમરી ટ્રીક:} \emph{Frame-Anvil-Spindle-Sleeve-Thimble-Ratchet (FASSTR).}

\subsection{પ્રશ્ન 2(b)(2) [4 marks]}
\textbf{વર્નિયર કેલિપર્સ માટે યોગ્ય આકૃતિ સાથે શૂન્ય, ધન અને ઋણ ત્રુટીઓ સમજાવો અને આ પ્રકારની ત્રુટીઓ દૂર કરવા માટેના જરૂરી પગલાંની યાદી બનાવો.}

\subsubsection{ઉકેલ}
\paragraph{ત્રુટીના પ્રકારો:}

\begin{enumerate}
    \item \textbf{શૂન્ય ત્રુટિ:} જ્યારે જડબા બંધ હોય, જો વર્નિયર માપપટ્ટીનો શૂન્ય મુખ્ય માપપટ્ટીના શૂન્ય સાથે સંપાત થતો ન હોય, તો સાધનમાં શૂન્ય ત્રુટિ હોય છે.
    
    \item \textbf{ધન શૂન્ય ત્રુટિ:} જ્યારે વર્નિયર માપપટ્ટીનો શૂન્ય મુખ્ય માપપટ્ટીના શૂન્યની \textit{જમણી બાજુએ} હોય. વાંચન વાસ્તવિક કરતાં \textit{વધુ} હોય છે, તેથી આપણે ત્રુટિ \textit{બાદ} કરીએ છીએ.
    
    \item \textbf{ઋણ શૂન્ય ત્રુટિ:} જ્યારે વર્નિયર માપપટ્ટીનો શૂન્ય મુખ્ય માપપટ્ટીના શૂન્યની \textit{ડાબી બાજુએ} હોય. વાંચન વાસ્તવિક કરતાં \textit{ઓછું} હોય છે, તેથી આપણે ત્રુટિ \textit{ઉમેરીએ} છીએ.
\end{enumerate}

\paragraph{ત્રુટિઓ દૂર કરવાના પગલાં:}
\begin{enumerate}
    \item જબડા બંધ હોય ત્યારે શૂન્ય ત્રુટિ નોંધો
    \item ધન ત્રુટિ માટે: માપેલા વાંચનમાંથી ત્રુટિ બાદ કરો
    \item ઋણ ત્રુટિ માટે: માપેલા વાંચનમાં ત્રુટિ ઉમેરો
    \item સૂત્ર: સુધારેલું વાંચન = માપેલું વાંચન - શૂન્ય ત્રુટિ
    \item જો શૂન્ય ત્રુટિ ચાલુ રહે, તો સાધનને ટેકનિશિયન દ્વારા કેલિબ્રેશનની જરૂર છે
\end{enumerate}

\paragraph{મેમરી ટ્રીક:} \emph{Positive error = Subtract, Negative error = Add.}

\subsection{પ્રશ્ન 2(b)(3) [4 marks]}
\textbf{સાદા લોલકનો આવર્તકાળ શોધવાના પ્રયોગમાં અવલોકનો 1.96 s, 1.98 s, 2.00 s, 2.02 s, 2.04 s છે. નિરપેક્ષ ત્રુટિ, સરેરાશ નિરપેક્ષ ત્રુટિ, સાપેક્ષ ત્રુટિ અને પ્રતિશત ત્રુટિની ગણતરી કરો.}

\subsubsection{ઉકેલ}
\paragraph{આપેલ માહિતી:}
અવલોકનો: \(T_1 = 1.96\,s\), \(T_2 = 1.98\,s\), \(T_3 = 2.00\,s\), \(T_4 = 2.02\,s\), \(T_5 = 2.04\,s\)

\paragraph{ગણતરીઓ:}

\subparagraph{સરેરાશ મૂલ્ય:}
સરેરાશ મૂલ્ય એ બધા અવલોકનોનો અંકગણિત સરેરાશ છે:
\[ T_{mean} = \frac{T_1 + T_2 + T_3 + T_4 + T_5}{5} = \frac{1.96 + 1.98 + 2.00 + 2.02 + 2.04}{5} = \frac{10.00}{5} = 2.00\,s \]
આ આવર્તકાળનું સૌથી સંભવિત મૂલ્ય રજૂ કરે છે.

\subparagraph{નિરપેક્ષ ત્રુટિઓ:}
પ્રત્યેક અવલોકન માટે નિરપેક્ષ ત્રુટિ એ સરેરાશમાંથી નિરપેક્ષ તફાવત છે:
\[ \Delta T_1 = |T_{mean} - T_1| = |2.00 - 1.96| = 0.04\,s \]
\[ \Delta T_2 = |2.00 - 1.98| = 0.02\,s \]
\[ \Delta T_3 = |2.00 - 2.00| = 0.00\,s \]
\[ \Delta T_4 = |2.00 - 2.02| = 0.02\,s \]
\[ \Delta T_5 = |2.00 - 2.04| = 0.04\,s \]

\subparagraph{સરેરાશ નિરપેક્ષ ત્રુટિ:}
સરેરાશ નિરપેક્ષ ત્રુટિ એ બધી વ્યક્તિગત નિરપેક્ષ ત્રુટિઓનો સરેરાશ છે:
\[ \Delta T_{mean} = \frac{\Delta T_1 + \Delta T_2 + \Delta T_3 + \Delta T_4 + \Delta T_5}{5} = \frac{0.04 + 0.02 + 0.00 + 0.02 + 0.04}{5} = \frac{0.12}{5} = 0.024\,s \]
આ આપણા માપનમાં અનિશ્ચિતતા દર્શાવે છે.

\subparagraph{સાપેક્ષ ત્રુટિ:}
સાપેક્ષ ત્રુટિ એ સરેરાશ નિરપેક્ષ ત્રુટિ અને સરેરાશ મૂલ્યનો ગુણોત્તર છે:
\[ \text{Relative Error} = \frac{\Delta T_{mean}}{T_{mean}} = \frac{0.024}{2.00} = 0.012 \]
આ એક પરિમાણવિહીન રાશિ છે જે ભાગલક્ષી અનિશ્ચિતતા દર્શાવે છે.

\subparagraph{પ્રતિશત ત્રુટિ:}
પ્રતિશત ત્રુટિ સાપેક્ષ ત્રુટિને ટકાવારી તરીકે વ્યક્ત કરે છે:
\[ \text{Percentage Error} = \text{Relative Error} \times 100\% = 0.012 \times 100\% = 1.2\% \]

\paragraph{જવાબો:}
\begin{itemize}
    \item સરેરાશ નિરપેક્ષ ત્રુટિ = 0.024 s
    \item સાપેક્ષ ત્રુટિ = 0.012
    \item પ્રતિશત ત્રુટિ = 1.2\%
\end{itemize}

\paragraph{મહત્વ:}
1.2\% ની પ્રતિશત ત્રુટિ આપણા માપનોમાં ઉચ્ચ સચોટતા દર્શાવે છે. પ્રયોગ ઓછામાં ઓછી રેન્ડમ ત્રુટિઓ સાથે કાળજીપૂર્વક હાથ ધરવામાં આવ્યો હતો.

\paragraph{મેમરી ટ્રીક:} \emph{Mean → Absolute → Relative → Percentage.}

\end{document}
