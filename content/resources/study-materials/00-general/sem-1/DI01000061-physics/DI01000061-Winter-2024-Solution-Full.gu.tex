%% METADATA
%% subject-code: DI01000061
%% subject-name: આધુનિક ભૌતિકશાસ્ત્ર
%% semester: 1
%% examination: Winter-2024
%% date: 09-01-2025
%% description: આધુનિક ભૌતિકશાસ્ત્ર (DI01000061) માટે ઉકેલ માર્ગદર્શિકા
%% tags: study-material, solutions, gtu, DI01000061, physics
%% END METADATA

\documentclass{article}
% GTU Solutions - Gujarati Preamble
% Includes common preamble + Gujarati font setup

% Basic setup
\usepackage[margin=1in]{geometry}
\author{Milav Dabgar}

% Math and tables
\usepackage{amsmath,amssymb,amsthm}
\usepackage{booktabs}
\usepackage{tabularx}
\usepackage{graphicx}
\usepackage{float}  % Required for [H] float placement

% Code listings with syntax highlighting
\usepackage{xcolor}
\usepackage{listings}
\lstset{
  basicstyle=\small\ttfamily,
  breaklines=true,
  numbers=left,
  numberstyle=\tiny\color{gray},
  xleftmargin=2em,
  frame=single,
  showstringspaces=false,
  tabsize=2,
  keywordstyle=\color{blue},
  commentstyle=\color{green!60!black},
  stringstyle=\color{purple}
}

% Optional: TikZ for diagrams (remove if not needed)
\usepackage{tikz}
\usepackage{circuitikz}
\usetikzlibrary{shapes,arrows,positioning,calc}

% Header/footer with author and website
\usepackage{fancyhdr}
\usepackage{lastpage}

\pagestyle{fancy}
\fancyhf{}
\fancyhead[L]{\small\itshape\leftmark}
\fancyhead[R]{\small Milav Dabgar}
\fancyfoot[L]{\small\href{https://www.milav.in}{www.milav.in}}
\fancyfoot[R]{\small Page \thepage\ of \pageref{LastPage}}
\renewcommand{\headrulewidth}{0.4pt}
\renewcommand{\footrulewidth}{0.4pt}

% Hyperref (load before fontspec for Gujarati)
\usepackage[
  colorlinks=true,
  linkcolor=blue,
  urlcolor=blue,
  citecolor=blue,
  pdfauthor={Milav Dabgar},
  pdfsubject={GTU Exam Solutions},
  pdfkeywords={GTU, Java, Programming, Solutions, Gujarati},
  bookmarks=true
]{hyperref}

% Gujarati font setup
\usepackage{fontspec}
\usepackage{polyglossia}
\setdefaultlanguage{gujarati}
\setotherlanguage{english}
\newfontfamily\gujaratifont[Script=Gujarati,AutoFakeBold=2.5,AutoFakeSlant=0.3]{Noto Sans Gujarati}
\setmainfont[Script=Gujarati,AutoFakeBold=2.5,AutoFakeSlant=0.3]{Noto Sans Gujarati}
\setmonofont[Scale=0.9]{Noto Sans Gujarati}
\newfontfamily\englishfont[Script=Gujarati,AutoFakeBold=2.5,AutoFakeSlant=0.3]{Noto Sans Gujarati}
\gappto\captionsgujarati{
  \renewcommand{\tablename}{કોષ્ટક}
  \renewcommand{\figurename}{આકૃતિ}
}
\newcommand{\gu}[1]{{\gujaratifont #1}}


\title{આધુનિક ભૌતિકશાસ્ત્ર (DI01000061) - Winter 2024 ઉકેલ}
\date{જાન્યુઆરી 9, 2025}

\hypersetup{
  pdftitle={આધુનિક ભૌતિકશાસ્ત્ર (DI01000061) - Winter 2024 ઉકેલ},
  pdfsubject={GTU Exam Solution - Winter-2024},
  pdfauthor={Milav Dabgar},
  pdfkeywords={study-material, solutions, gtu, DI01000061, physics},
  pdfcreator={xelatex}
}

\begin{document}
\maketitle

\setcounter{tocdepth}{5}
\tableofcontents
\newpage

% ========================================
% QUESTION 1: MCQs and Fill in the Blanks (14 marks)
% Demonstrates: Short answer format, multiple topics
% ========================================

\section{પ્રશ્ન 1}
\textbf{યોગ્ય વિકલ્પ પસંદ કરી ખાલી જગ્યા પૂરો / બહુવિકલ્પ પ્રશ્નોના જવાબ આપો.}

\subsection{પ્રશ્ન 1(1) [1 marks]}
\textbf{નીચેનામાંથી કયું અર્ધવાહક છે?}

\textbf{વિકલ્પો:} (a) Si (b) Cu (c) Fe (d) Ni

\subsubsection{ઉકેલ}
\paragraph{જવાબ:} (a) Si

\paragraph{સમજૂતી:}
સિલિકોન (Si) એ ગ્રુપ 14 નો તત્વ છે જેમાં 4 વેલેન્સ ઇલેક્ટ્રોન હોય છે, જે તેને આંતરિક અર્ધવાહક બનાવે છે. કોપર (Cu), આયર્ન (Fe) અને નિકલ (Ni) એ ધાતુઓ છે જે મુક્ત ઇલેક્ટ્રોનને કારણે ઉચ્ચ વિદ્યુત વાહકતા ધરાવે છે.

\subparagraph{નોંધ:} જર્મેનિયમ (Ge) પણ ગ્રુપ 14 નો અર્ધવાહક છે.

\paragraph{મેમરી ટ્રીક:} \emph{Si = Semiconductor, Cu/Fe/Ni = Metals.}

\subsection{પ્રશ્ન 1(2) [1 marks]}
\textbf{કાચનો વક્રીભવનાંક \_\_\_\_\_ છે.}

\textbf{વિકલ્પો:} (a) 1.50 (b) 1.33 (c) 1.00 (d) 2.43

\subsubsection{ઉકેલ}
\paragraph{જવાબ:} (a) 1.50

\paragraph{સમજૂતી:}
સામાન્ય કાચનો વક્રીભવનાંક આશરે 1.50 હોય છે. પાણીનો \(n = 1.33\), હવા/શૂન્યાવકાશનો \(n = 1.00\) અને હીરાનો \(n = 2.43\) હોય છે.

\subparagraph{નોંધ:} ઉચ્ચ વક્રીભવનાંક એટલે તે માધ્યમમાં પ્રકાશ ધીમો મુસાફરી કરે છે.

\paragraph{મેમરી ટ્રીક:} \emph{Glass = 1.5, Water = 1.33, Diamond = 2.43.}

\subsection{પ્રશ્ન 1(3) [1 marks]}
\textbf{જ્યારે આપાતકોણ ક્રાંતિકોણ કરતાં \_\_\_\_\_ થાય ત્યારે પૂર્ણ આંતરિક પરાવર્તન થાય છે.}

\textbf{વિકલ્પો:} (a) સમાન (b) વધારે (c) ઓછો (d) આ પૈકી કોઈ નહીં

\subsubsection{ઉકેલ}
\paragraph{જવાબ:} (b) વધારે

\paragraph{સમજૂતી:}
પૂર્ણ આંતરિક પરાવર્તન (TIR) ત્યારે થાય છે જ્યારે પ્રકાશ ઘટ્ટ માધ્યમમાંથી પાતળા માધ્યમમાં જાય છે અને આપાત કોણ \(i\) ક્રાંતિકોણ \(C\) કરતા મોટો હોય, એટલે કે \(i > C\).

\paragraph{મેમરી ટ્રીક:} \emph{TIR when i > C (Greater than Critical).}

\subsection{પ્રશ્ન 1(4) [1 marks]}
\textbf{બ્રિજ રેકટીફાયરમાં કેટલા P-N જંકશન ડાયોડનો ઉપયોગ થાય છે?}

\textbf{વિકલ્પો:} (a) 2 (b) 3 (c) 4 (d) 5

\subsubsection{ઉકેલ}
\paragraph{જવાબ:} (c) 4

\paragraph{સમજૂતી:}
બ્રિજ રેકટીફાયર AC ને પૂર્ણ-તરંગ DC માં રૂપાંતરિત કરવા માટે બ્રિજ કન્ફિગરેશનમાં ગોઠવાયેલા 4 ડાયોડનો ઉપયોગ કરે છે. આ બંને અર્ધ ચક્ર દરમિયાન લોડમાંથી પ્રવાહને એક જ દિશામાં વહેવાની મંજૂરી આપે છે.

\paragraph{મેમરી ટ્રીક:} \emph{Bridge = 4 Diodes.}

\subsection{પ્રશ્ન 1(5) [1 marks]}
\textbf{ઓપ્ટિકલ ફાઈબર \_\_\_\_\_ ના સિદ્ધાંત પર કાર્ય કરે છે.}

\textbf{વિકલ્પો:} (a) વ્યતિકરણ (b) વક્રીભવન (c) ધ્રુવીભવન (d) પૂર્ણ આંતરિક પરાવર્તન

\subsubsection{ઉકેલ}
\paragraph{જવાબ:} (d) પૂર્ણ આંતરિક પરાવર્તન

\paragraph{સમજૂતી:}
ઓપ્ટિકલ ફાઈબર કોર-ક્લેડિંગ ઇન્ટરફેસ પર પુનરાવર્તિત પૂર્ણ આંતરિક પરાવર્તન દ્વારા પ્રકાશ સંકેતો પ્રસારિત કરે છે, જ્યાં કોર ક્લેડિંગ કરતા ઊંચો વક્રીભવનાંક ધરાવે છે.

\paragraph{મેમરી ટ્રીક:} \emph{Fiber = TIR (Total Internal Reflection).}

\subsection{પ્રશ્ન 1(6) [1 marks]}
\textbf{એકમ સમયમાં થતાં દોલનોની સંખ્યાને \_\_\_\_\_ કહે છે.}

\textbf{વિકલ્પો:} (a) આવર્તકાળ (b) તરંગલંબાઈ (c) કંપવિસ્તાર (d) આવૃત્તિ

\subsubsection{ઉકેલ}
\paragraph{જવાબ:} (d) આવૃત્તિ

\paragraph{સમજૂતી:}
આવૃત્તિ (\(f\)) એ એકમ સમય દીઠ સંપૂર્ણ દોલનોની સંખ્યા તરીકે વ્યાખ્યાયિત થાય છે. તેનો એકમ Hertz (Hz) અથવા \(s^{-1}\) છે. સંબંધ છે \(f = 1/T\), જ્યાં \(T\) આવર્તકાળ છે.

\paragraph{મેમરી ટ્રીક:} \emph{Frequency = Oscillations per second.}

\subsection{પ્રશ્ન 1(7) [1 marks]}
\textbf{વિદ્યુતભારનો એસ.આઈ. એકમ \_\_\_\_\_ છે.}

\textbf{વિકલ્પો:} (a) કુલંબ (b) એમ્પિયર (c) વોલ્ટ (d) ફેરાડે

\subsubsection{ઉકેલ}
\paragraph{જવાબ:} (a) કુલંબ

\paragraph{સમજૂતી:}
વિદ્યુતભારનો SI એકમ કુલંબ (C) છે. એક કુલંબ એ એક સેકન્ડમાં એક એમ્પિયરના પ્રવાહ દ્વારા પરિવહન થયેલો ભાર છે (\(1\,C = 1\,A \times 1\,s\)).

\paragraph{મેમરી ટ્રીક:} \emph{Charge = Coulomb (C).}

\subsection{પ્રશ્ન 1(8) [1 marks]}
\textbf{જો સાદા લોલકનો આવર્તકાળ 2 સેકન્ડ હોય તો તેની આવૃત્તિ \_\_\_\_\_ હશે.}

\textbf{વિકલ્પો:} (a) 2 Hz (b) 0.5 Hz (c) 0.2 Hz (d) 5 Hz

\subsubsection{ઉકેલ}
\paragraph{જવાબ:} (b) 0.5 Hz

\paragraph{સમજૂતી:}
આવૃત્તિ (\(f\)) અને આવર્તકાળ (\(T\)) વચ્ચેનો સંબંધ વિલોમ છે. આવૃત્તિ આપણને જણાવે છે કે પ્રતિ સેકન્ડ કેટલા ચક્ર પૂર્ણ થાય છે, જ્યારે આવર્તકાળ જણાવે છે કે એક ચક્રને કેટલો સમય લાગે છે.

\paragraph{ગણતરી:}
\[ f = \frac{1}{T} = \frac{1}{2} = 0.5 \, \text{Hz} \]

\paragraph{મેમરી ટ્રીક:} \emph{f = 1/T.}

\subsection{પ્રશ્ન 1(9) [1 marks]}
\textbf{પ્રકાશનો શૂન્યાવકાશમાં વેગ \_\_\_\_\_ હોય છે.}

\textbf{વિકલ્પો:} (a) 300000 km/s (b) 300000 m/s (c) 341 km/s (d) 341 m/s

\subsubsection{ઉકેલ}
\paragraph{જવાબ:} (a) 300000 km/s

\paragraph{સમજૂતી:}
શૂન્યાવકાશમાં પ્રકાશની ઝડપ \(c = 3 \times 10^8\,m/s = 300000\,km/s\) છે. આ ભૌતિકશાસ્ત્રમાં એક મૂળભૂત અચળાંક છે. નોંધ: 341 m/s એ હવામાં ધ્વનિની ઝડપ છે.

\paragraph{મેમરી ટ્રીક:} \emph{c = 3\texttimes10\textsuperscript{8} m/s = 300000 km/s.}

\subsection{પ્રશ્ન 1(10) [1 marks]}
\textbf{ધ્વનિ તરંગોનો વેગ \_\_\_\_\_ માં મહત્તમ હોય છે.}

\textbf{વિકલ્પો:} (a) પ્રવાહી (b) ઘન (c) વાયુ (d) શૂન્યાવકાશ

\subsubsection{ઉકેલ}
\paragraph{જવાબ:} (b) ઘન

\paragraph{સમજૂતી:}
ધ્વનિ ઘનમાં સૌથી ઝડપથી મુસાફરી કરે છે કારણ કે તેમાં અણુઓ એકબીજાની ખૂબ નજીક હોય છે અને મજબૂત આંતર-આણ્વિક બળો હોય છે. ક્રમ: \(v_{solid} > v_{liquid} > v_{gas}\). ધ્વનિ શૂન્યાવકાશમાં મુસાફરી કરી શકતું નથી.

\paragraph{મેમરી ટ્રીક:} \emph{Solids = Fastest sound.}

\subsection{પ્રશ્ન 1(11) [1 marks]}
\textbf{પ્રકાશના તરંગનું પ્રસરણ \_\_\_\_\_ ને આભારી છે.}

\textbf{વિકલ્પો:} (a) શૃંગ અને ગર્ત (b) સંઘનન અને વિઘનન (c) ફક્ત સંઘનન (d) ફક્ત વિઘનન

\subsubsection{ઉકેલ}
\paragraph{જવાબ:} (a) શૃંગ અને ગર્ત

\paragraph{સમજૂતી:}
પ્રકાશ તરંગો એ ટ્રાન્સવર્સ ઇલેક્ટ્રોમેગ્નેટિક તરંગો છે જે વૈકલ્પિક શૃંગ અને ગર્ત દ્વારા પ્રસરે છે. સંઘનન અને વિઘનન એ ધ્વનિ જેવા લોન્ગીટ્યુડિનલ તરંગો સાથે સંકળાયેલા છે.

\paragraph{મેમરી ટ્રીક:} \emph{Light = Transverse = Crest/Trough.}

\subsection{પ્રશ્ન 1(12) [1 marks]}
\textbf{LASER વિકિરણ \_\_\_\_\_ છે.}

\textbf{વિકલ્પો:} (a) બહુરંગી (b) એકરંગી (c) ઓછું તીવ્ર (d) આ પૈકી કોઈ નહીં

\subsubsection{ઉકેલ}
\paragraph{જવાબ:} (b) એકરંગી

\paragraph{સમજૂતી:}
LASER (Light Amplification by Stimulated Emission of Radiation) એકરંગી પ્રકાશ ઉત્પન્ન કરે છે, એટલે કે તેની એક જ ચોક્કસ તરંગલંબાઈ હોય છે. તે સુસંગત અને અત્યંત તીવ્ર પણ હોય છે.

\paragraph{મેમરી ટ્રીક:} \emph{LASER = Monochromatic, Coherent, Directional.}

\subsection{પ્રશ્ન 1(13) [1 marks]}
\textbf{કયો ફાઈબર લાંબી બેન્ડવિથ આપે છે?}

\textbf{વિકલ્પો:} (a) સિંગલ મોડ (b) મલ્ટી મોડ સ્ટેપ ઇન્ડેક્સ (c) સ્ટેપ ઇન્ડેક્સ (d) આ પૈકી કોઈ નહીં

\subsubsection{ઉકેલ}
\paragraph{જવાબ:} (a) સિંગલ મોડ

\paragraph{સમજૂતી:}
સિંગલ મોડ ફાઈબરમાં ખૂબ જ નાનો કોર વ્યાસ હોય છે અને કેવળ એક જ મોડના પ્રકાશ પ્રસરણને મંજૂરી આપે છે, જે લઘુત્તમ વિક્ષેપણ અને મહત્તમ બેન્ડવિથમાં પરિણમે છે. તેનો ઉપયોગ લાંબા અંતરના સંદેશાવ્યવહાર માટે થાય છે.

\paragraph{મેમરી ટ્રીક:} \emph{Single mode = Long distance, High bandwidth.}

\subsection{પ્રશ્ન 1(14) [1 marks]}
\textbf{0.5 ન્યૂમેરિકલ એપર્ચર ધરાવતા ઓપ્ટિકલ ફાઈબરનો એક્સેપ્ટન્સ એંગલ કેટલો હશે?}

\textbf{વિકલ્પો:} (a) \(30^\circ\) (b) \(45^\circ\) (c) \(60^\circ\) (d) \(15^\circ\)

\subsubsection{ઉકેલ}
\paragraph{જવાબ:} (a) \(30^\circ\)

\paragraph{સમજૂતી:}
એક્સેપ્ટન્સ એંગલ એ મહત્તમ કોણ છે જેના પર પ્રકાશ ઓપ્ટિકલ ફાઈબરમાં પ્રવેશ કરી શકે છે અને હજુ પણ પૂર્ણ આંતરિક પરાવર્તન દ્વારા પ્રસારિત થઈ શકે છે. તે ન્યૂમેરિકલ એપર્ચર પર આધાર રાખે છે, જે ફાઈબરની પ્રકાશ-એકત્રીકરણ ક્ષમતાનું માપ છે.

\paragraph{ગણતરી:}
ન્યૂમેરિકલ એપર્ચર (NA) એ એક્સેપ્ટન્સ એંગલ (\(\theta_a\)) સાથે નીચેના સંબંધ દ્વારા જોડાયેલ છે:
\[ NA = \sin(\theta_a) \]
\[ 0.5 = \sin(\theta_a) \]
\[ \theta_a = \sin^{-1}(0.5) = 30^\circ \]

\paragraph{મેમરી ટ્રીક:} \emph{NA = sin(acceptance angle).}

% ========================================
% QUESTION 2: Theory Questions (14 marks)
% Q2(A): 3 marks each, Q2(B): 4 marks each
% ========================================

\section{પ્રશ્ન 2}

\subsection{પ્રશ્ન 2(a)(1) [3 marks]}
\textbf{ચોકસાઈ અને સચોટતા વચ્ચેનો તફાવત આપો.}

\subsubsection{ઉકેલ}
\paragraph{સરખામણી ટેબલ:}
\begin{table}[H]
\caption{ચોકસાઈ વિરુદ્ધ સચોટતા}
\centering
\begin{tabularx}{\textwidth}{|X|X|}
\hline
\textbf{ચોકસાઈ (Accuracy)} & \textbf{સચોટતા (Precision)} \\ \hline
સાચા/standard મૂલ્યની નજીકતા & માપનો એકબીજાની નજીકતા \\ \hline
માપનની યોગ્યતા દર્શાવે છે & પુનરુત્પાદનક્ષમતા/સંગતતા દર્શાવે છે \\ \hline
વ્યવસ્થિત ત્રુટિઓ પર આધાર રાખે છે & સાધનની લઘુત્તમ માપ શક્તિ પર આધાર રાખે છે \\ \hline
ઉચ્ચ ચોકસાઈ = ઓછી ત્રુટિ & ઉચ્ચ સચોટતા = મૂલ્યો એક જગ્યાએ એકત્ર \\ \hline
ઉદાહરણ: લક્ષ્ય કેન્દ્ર પર હણવું & ઉદાહરણ: એક જગ્યાએ એકત્રિત તીરો \\ \hline
\end{tabularx}
\end{table}

\paragraph{મુખ્ય મુદ્દાઓ:}
\begin{itemize}
    \item માપન ચોક્કસ હોય વગર સચોટ હોઈ શકે છે (સતત ખોટા મૂલ્યો)
    \item માપન સચોટ હોય વગર ચોક્કસ હોઈ શકે છે (સાચા મૂલ્યની આસપાસ વિખરાયેલા)
    \item આદર્શ માપન ચોક્કસ અને સચોટ બંને હોય છે
\end{itemize}

\paragraph{ઉદાહરણ:}
જો સાચું મૂલ્ય = 10.0 cm હોય:
\begin{itemize}
    \item ઉચ્ચ ચોકસાઈ, ઉચ્ચ સચોટતા: 10.0, 10.1, 9.9 cm (સાચા નજીક, એકત્રિત)
    \item નીચી ચોકસાઈ, ઉચ્ચ સચોટતા: 8.0, 8.1, 7.9 cm (ખોટું પણ સંગત)
    \item ઉચ્ચ ચોકસાઈ, નીચી સચોટતા: 10.0, 12.0, 8.0 cm (સરેરાશ સાચું, વિખરાયેલા)
\end{itemize}

\paragraph{મેમરી ટ્રીક:} \emph{Accuracy = Correctness, Precision = Consistency.}

\subsection{પ્રશ્ન 2(a)(2) [3 marks]}
\textbf{માઇક્રોમીટર સ્ક્રૂ દ્વારા માપવામાં આવતા ગોળાનો વ્યાસ નક્કી કરવા, મુખ્ય માપપટ્ટીનું માપ 5 mm અને વર્તુળાકાર માપપટ્ટીનો 50 મો વિભાગ બેઝ લાઇન સાથે મેળ ખાય છે. આ સાધનની લ.મા.શ 0.01 mm છે.}

\subsubsection{ઉકેલ}
\paragraph{આપેલ માહિતી:}
\begin{itemize}
    \item મુખ્ય માપપટ્ટી વાંચન (MSR) = 5 mm
    \item વર્તુળાકાર માપપટ્ટી વિભાગ (CSD) = 50
    \item લઘુત્તમ માપ શક્તિ (LC) = 0.01 mm
\end{itemize}

\paragraph{સૂત્ર:}
માઇક્રોમીટર સ્ક્રૂ ગેજ માટે, કુલ વાંચન એ મુખ્ય માપપટ્ટી વાંચન અને વર્તુળાકાર માપપટ્ટી વિભાગ અને લઘુત્તમ માપ શક્તિના ગુણાકારનો સરવાળો છે:
\[ \text{Reading} = \text{MSR} + (\text{CSD} \times \text{LC}) \]
મુખ્ય માપપટ્ટી વાંચન પૂર્ણાંક મિલિમીટર આપે છે, જ્યારે વર્તુળાકાર માપપટ્ટી અપૂર્ણાંક ભાગ આપે છે. લઘુત્તમ માપ શક્તિ (0.01 mm) એટલે કે દરેક વર્તુળાકાર માપ વિભાગ 0.01 mm દર્શાવે છે. જ્યારે 50મો વિભાગ બેઝલાઇન સાથે મેળ ખાય છે, તે મુખ્ય માપપટ્ટી વાંચનથી વધારાના 0.50 mm સૂચવે છે.

\paragraph{ગણતરી:}
\[ \text{Diameter} = 5 + (50 \times 0.01) \]
\[ \text{Diameter} = 5 + 0.50 = 5.50 \, \text{mm} \]

\paragraph{જવાબ:}
ગોળાનો વ્યાસ \textbf{5.50 mm} છે. સાધનની લઘુત્તમ માપ શક્તિને કારણે આ માપન ±0.01 mm ની ચોકસાઇ ધરાવે છે.

\paragraph{મેમરી ટ્રીક:} \emph{MSR + (CSD × LC) = Total Reading.}

\subsection{પ્રશ્ન 2(a)(3) [3 marks]}
\textbf{જ્યારે 4 \(\mu\)F કેપેસિટન્સ ધરાવતા કેપેસિટરને 12 volt બેટરી સાથે જોડતા કેપેસિટરની બંને પ્લેટ પર સંગ્રહિત થતાં વિદ્યુતભારના જથ્થાની ગણતરી કરો.}

\subsubsection{ઉકેલ}
\paragraph{આપેલ માહિતી:}
\begin{itemize}
    \item કેપેસિટન્સ, \(C = 4\,\mu F = 4 \times 10^{-6}\,F\)
    \item વોલ્ટેજ, \(V = 12\,V\)
\end{itemize}

\paragraph{સૂત્ર:}
કેપેસિટર પર સંગ્રહિત થતો ભાર આપવામાં આવે છે:
\[ Q = CV \]
જ્યાં \(Q\) એ કુલંબમાં ભાર છે, \(C\) એ ફેરાડમાં કેપેસિટન્સ છે, અને \(V\) એ વોલ્ટમાં સંભવિત તફાવત છે. આ મૂળભૂત સંબંધ દર્શાવે છે કે સંગ્રહિત ભાર કેપેસિટન્સ અને વોલ્ટેજ બંનેના સીધા પ્રમાણમાં છે. કેપેસિટર તેની પ્લેટો વચ્ચે વિદ્યુત ક્ષેત્રમાં વિદ્યુત ઊર્જા સંગ્રહિત કરે છે.

\paragraph{ગણતરી:}
\[ Q = 4 \times 10^{-6} \times 12 \]
\[ Q = 48 \times 10^{-6} \, C \]
\[ Q = 48 \, \mu C \]

\paragraph{જવાબ:}
બંને પ્લેટ પર સંગ્રહિત થયેલા વિદ્યુતભારનું પ્રમાણ \textbf{48 \(\mu\)C} છે. નોંધ કરો કે એક પ્લેટ પર +48 \(\mu\)C અને બીજી પર -48 \(\mu\)C છે, જે ચોખ્ખો ભાર શૂન્ય બનાવે છે, પરંતુ દરેક પ્લેટ સમાન માત્રાનો ભાર સંગ્રહિત કરે છે.

\paragraph{મેમરી ટ્રીક:} \emph{Q = CV (Charge = Capacitance × Voltage).}

\subsection{પ્રશ્ન 2(b)(1) [4 marks]}
\textbf{યોગ્ય નામકરણ સાથે માઇક્રોમીટર સ્ક્રૂ ગેજની આકૃતિ દોરો.}

\subsubsection{ઉકેલ}
\paragraph{આકૃતિ:}
\begin{figure}[H]
\centering
\begin{tikzpicture}[scale=1.2]
    % Frame
    \draw[thick] (0,0) -- (0,1.5) -- (6,1.5) -- (6,0) -- cycle;
    % Anvil
    \fill[gray!30] (0.5,0.5) rectangle (1,1);
    \draw[thick] (0.5,0.5) rectangle (1,1);
    \node[below] at (0.75,0.5) {\small Anvil};
    % Spindle
    \fill[gray!30] (4,0.5) rectangle (4.5,1);
    \draw[thick] (4,0.5) rectangle (4.5,1);
    \node[below] at (4.25,0.5) {\small Spindle};
    % Sleeve (Main Scale)
    \draw[thick] (2,0.6) rectangle (4,0.9);
    \draw (2.2,0.6) -- (2.2,0.9);
    \draw (2.4,0.6) -- (2.4,0.9);
    \draw (2.6,0.6) -- (2.6,0.9);
    \node[above] at (3,0.9) {\small Main Scale};
    % Thimble (Circular Scale)
    \draw[thick,fill=gray!20] (3.5,0.4) rectangle (4.5,1.1);
    \draw (3.7,0.4) -- (3.7,1.1);
    \draw (3.9,0.4) -- (3.9,1.1);
    \draw (4.1,0.4) -- (4.1,1.1);
    \draw (4.3,0.4) -- (4.3,1.1);
    \node[above] at (4,1.1) {\small Thimble};
    \node[below] at (4,0.4) {\small Circular Scale};
    % Ratchet
    \draw[thick,fill=gray!40] (4.8,0.6) circle (0.3);
    \node[right] at (5.1,0.6) {\small Ratchet};
    % Frame label
    \node[left] at (0,0.75) {\small Frame};
\end{tikzpicture}
\caption{માઇક્રોમીટર સ્ક્રૂ ગેજ}
\end{figure}

\paragraph{મુખ્ય ભાગો:}
\begin{enumerate}
    \item \textbf{ફ્રેમ (Frame):} C આકારનું કઠણ શરીર જે બધા ભાગોને પકડી રાખે છે
    \item \textbf{એન્વિલ (Anvil):} સ્થિર છેડો જેની સામે વસ્તુ મૂકવામાં આવે છે
    \item \textbf{સ્પિન્ડલ (Spindle):} ગતિશીલ ભાગ જે એન્વિલ તરફ આગળ વધે છે
    \item \textbf{સ્લીવ (Sleeve) - મુખ્ય માપપટ્ટી:} મિલિમીટર વિભાગો બતાવે છે
    \item \textbf{થિમ્બલ (Thimble) - વર્તુળાકાર માપપટ્ટી:} અંશ વિભાગો બતાવે છે (0-50 અથવા 0-100)
    \item \textbf{રેચેટ (Ratchet):} માપન દરમિયાન એકરૂપ દબાણ સુનિશ્ચિત કરે છે
\end{enumerate}

\paragraph{મેમરી ટ્રીક:} \emph{Frame-Anvil-Spindle-Sleeve-Thimble-Ratchet (FASSTR).}

\subsection{પ્રશ્ન 2(b)(2) [4 marks]}
\textbf{વર્નિયર કેલિપર્સ માટે યોગ્ય આકૃતિ સાથે શૂન્ય, ધન અને ઋણ ત્રુટીઓ સમજાવો અને આ પ્રકારની ત્રુટીઓ દૂર કરવા માટેના જરૂરી પગલાંની યાદી બનાવો.}

\subsubsection{ઉકેલ}
\paragraph{ત્રુટીના પ્રકારો:}

\begin{enumerate}
    \item \textbf{શૂન્ય ત્રુટિ:} જ્યારે જડબા બંધ હોય, જો વર્નિયર માપપટ્ટીનો શૂન્ય મુખ્ય માપપટ્ટીના શૂન્ય સાથે સંપાત થતો ન હોય, તો સાધનમાં શૂન્ય ત્રુટિ હોય છે.
    
    \item \textbf{ધન શૂન્ય ત્રુટિ:} જ્યારે વર્નિયર માપપટ્ટીનો શૂન્ય મુખ્ય માપપટ્ટીના શૂન્યની \textit{જમણી બાજુએ} હોય. વાંચન વાસ્તવિક કરતાં \textit{વધુ} હોય છે, તેથી આપણે ત્રુટિ \textit{બાદ} કરીએ છીએ.
    
    \item \textbf{ઋણ શૂન્ય ત્રુટિ:} જ્યારે વર્નિયર માપપટ્ટીનો શૂન્ય મુખ્ય માપપટ્ટીના શૂન્યની \textit{ડાબી બાજુએ} હોય. વાંચન વાસ્તવિક કરતાં \textit{ઓછું} હોય છે, તેથી આપણે ત્રુટિ \textit{ઉમેરીએ} છીએ.
\end{enumerate}

\paragraph{ત્રુટિઓ દૂર કરવાના પગલાં:}
\begin{enumerate}
    \item જબડા બંધ હોય ત્યારે શૂન્ય ત્રુટિ નોંધો
    \item ધન ત્રુટિ માટે: માપેલા વાંચનમાંથી ત્રુટિ બાદ કરો
    \item ઋણ ત્રુટિ માટે: માપેલા વાંચનમાં ત્રુટિ ઉમેરો
    \item સૂત્ર: સુધારેલું વાંચન = માપેલું વાંચન - શૂન્ય ત્રુટિ
    \item જો શૂન્ય ત્રુટિ ચાલુ રહે, તો સાધનને ટેકનિશિયન દ્વારા કેલિબ્રેશનની જરૂર છે
\end{enumerate}

\paragraph{મેમરી ટ્રીક:} \emph{Positive error = Subtract, Negative error = Add.}

\subsection{પ્રશ્ન 2(b)(3) [4 marks]}
\textbf{સાદા લોલકનો આવર્તકાળ શોધવાના પ્રયોગમાં અવલોકનો 1.96 s, 1.98 s, 2.00 s, 2.02 s, 2.04 s છે. નિરપેક્ષ ત્રુટિ, સરેરાશ નિરપેક્ષ ત્રુટિ, સાપેક્ષ ત્રુટિ અને પ્રતિશત ત્રુટિની ગણતરી કરો.}

\subsubsection{ઉકેલ}
\paragraph{આપેલ માહિતી:}
અવલોકનો: \(T_1 = 1.96\,s\), \(T_2 = 1.98\,s\), \(T_3 = 2.00\,s\), \(T_4 = 2.02\,s\), \(T_5 = 2.04\,s\)

\paragraph{ગણતરીઓ:}

\subparagraph{સરેરાશ મૂલ્ય:}
સરેરાશ મૂલ્ય એ બધા અવલોકનોનો અંકગણિત સરેરાશ છે:
\[ T_{mean} = \frac{T_1 + T_2 + T_3 + T_4 + T_5}{5} = \frac{1.96 + 1.98 + 2.00 + 2.02 + 2.04}{5} = \frac{10.00}{5} = 2.00\,s \]
આ આવર્તકાળનું સૌથી સંભવિત મૂલ્ય રજૂ કરે છે.

\subparagraph{નિરપેક્ષ ત્રુટિઓ:}
પ્રત્યેક અવલોકન માટે નિરપેક્ષ ત્રુટિ એ સરેરાશમાંથી નિરપેક્ષ તફાવત છે:
\[ \Delta T_1 = |T_{mean} - T_1| = |2.00 - 1.96| = 0.04\,s \]
\[ \Delta T_2 = |2.00 - 1.98| = 0.02\,s \]
\[ \Delta T_3 = |2.00 - 2.00| = 0.00\,s \]
\[ \Delta T_4 = |2.00 - 2.02| = 0.02\,s \]
\[ \Delta T_5 = |2.00 - 2.04| = 0.04\,s \]

\subparagraph{સરેરાશ નિરપેક્ષ ત્રુટિ:}
સરેરાશ નિરપેક્ષ ત્રુટિ એ બધી વ્યક્તિગત નિરપેક્ષ ત્રુટિઓનો સરેરાશ છે:
\[ \Delta T_{mean} = \frac{\Delta T_1 + \Delta T_2 + \Delta T_3 + \Delta T_4 + \Delta T_5}{5} = \frac{0.04 + 0.02 + 0.00 + 0.02 + 0.04}{5} = \frac{0.12}{5} = 0.024\,s \]
આ આપણા માપનમાં અનિશ્ચિતતા દર્શાવે છે.

\subparagraph{સાપેક્ષ ત્રુટિ:}
સાપેક્ષ ત્રુટિ એ સરેરાશ નિરપેક્ષ ત્રુટિ અને સરેરાશ મૂલ્યનો ગુણોત્તર છે:
\[ \text{Relative Error} = \frac{\Delta T_{mean}}{T_{mean}} = \frac{0.024}{2.00} = 0.012 \]
આ એક પરિમાણવિહીન રાશિ છે જે ભાગલક્ષી અનિશ્ચિતતા દર્શાવે છે.

\subparagraph{પ્રતિશત ત્રુટિ:}
પ્રતિશત ત્રુટિ સાપેક્ષ ત્રુટિને ટકાવારી તરીકે વ્યક્ત કરે છે:
\[ \text{Percentage Error} = \text{Relative Error} \times 100\% = 0.012 \times 100\% = 1.2\% \]

\paragraph{જવાબો:}
\begin{itemize}
    \item સરેરાશ નિરપેક્ષ ત્રુટિ = 0.024 s
    \item સાપેક્ષ ત્રુટિ = 0.012
    \item પ્રતિશત ત્રુટિ = 1.2\%
\end{itemize}

\paragraph{મહત્વ:}
1.2\% ની પ્રતિશત ત્રુટિ આપણા માપનોમાં ઉચ્ચ સચોટતા દર્શાવે છે. પ્રયોગ ઓછામાં ઓછી રેન્ડમ ત્રુટિઓ સાથે કાળજીપૂર્વક હાથ ધરવામાં આવ્યો હતો.

\paragraph{મેમરી ટ્રીક:} \emph{Mean → Absolute → Relative → Percentage.}

% ========================================
% QUESTION 3: Electromagnetics & Waves (14 marks)
% Q3(A): 3 marks each, Q3(B): 4 marks each
% ========================================

\section{પ્રશ્ન 3}

\subsection{પ્રશ્ન 3(a)(1) [3 marks]}
\textbf{વ્યાખ્યાયિત કરો: વિદ્યુત ફ્લક્સ, વિદ્યુતક્ષેત્ર, વીજસ્થિતિમાનનો તફાવત}

\subsubsection{ઉકેલ}
\paragraph{વ્યાખ્યાઓ:}

\subparagraph{વિદ્યુત ફ્લક્સ (\(\Phi_E\)):}
સપાટી દ્વારા પસાર થતી વિદ્યુત ક્ષેત્ર રેખાઓની કુલ સંખ્યાને વિદ્યુત ફ્લક્સ કહે છે. ગણિતીય રીતે, \(\Phi_E = \vec{E} \cdot \vec{A} = EA\cos\theta\), જ્યાં \(\theta\) એ ક્ષેત્ર અને ક્ષેત્રફળ સદિશ વચ્ચેનો કોણ છે. SI એકમ: \(N \cdot m^2/C\) અથવા \(V \cdot m\).

\subparagraph{વિદ્યુતક્ષેત્ર (\(\vec{E}\)):}
કોઈ બિંદુ પર વિદ્યુતક્ષેત્ર એ તે બિંદુએ મૂકેલા એકમ ધન ભાર દ્વારા અનુભવાતું બળ છે. તેને \(\vec{E} = \vec{F}/q_0\) તરીકે વ્યાખ્યાયિત કરવામાં આવે છે, જ્યાં \(q_0\) નાનો પરીક્ષણ ભાર છે. SI એકમ: \(N/C\) અથવા \(V/m\). દિશા: ધન ભારથી દૂર, ઋણ ભાર તરફ.

\subparagraph{સ્થિતિમાન તફાવત (V):}
બે બિંદુઓ વચ્ચેનો સ્થિતિમાન તફાવત એ નાના ધન ભારને એક બિંદુથી બીજા બિંદુ સુધી ખસેડવામાં એકમ ભાર દીઠ કરવામાં આવતું કાર્ય છે. ગણિતીય રીતે, \(V = W/q\). SI એકમ: Volt (V) અથવા Joule/Coulomb (J/C). તે એકમ ભાર દીઠ ઊર્જા રજૂ કરે છે.

\paragraph{મેમરી ટ્રીક:} \emph{Flux = Field lines, Field = Force/charge, Potential = Work/charge.}

\subsection{પ્રશ્ન 3(a)(2) [3 marks]}
\textbf{જ્યારે ત્રણ જુદા જુદા કેપેસિટરોને શ્રેણીમાં જોડવામાં આવે ત્યારે જરૂરી સર્કિટ ડાયાગ્રામ સાથે સમકક્ષ કેપેસિટેન્સ માટેનું સૂત્ર મેળવો.}

\subsubsection{ઉકેલ}
\paragraph{સર્કિટ આકૃતિ:}
\begin{figure}[H]
\centering
\begin{tikzpicture}[scale=1]
    \draw (0,0) to[battery, l=V] (0,3);
    \draw (0,3) -- (1,3);
    \draw (1,2.5) -- (1,3.5);
    \draw (1.3,2.7) -- (1.3,3.3);
    \node[above] at (1.15,3.5) {\(C_1\)};
    \draw (1.3,3) -- (3,3);
    \draw (3,2.5) -- (3,3.5);
    \draw (3.3,2.7) -- (3.3,3.3);
    \node[above] at (3.15,3.5) {\(C_2\)};
    \draw (3.3,3) -- (5,3);
    \draw (5,2.5) -- (5,3.5);
    \draw (5.3,2.7) -- (5.3,3.3);
    \node[above] at (5.15,3.5) {\(C_3\)};
    \draw (5.3,3) -- (6,3) -- (6,0) -- (0,0);
\end{tikzpicture}
\caption{શ્રેણીમાં ત્રણ કેપેસિટર}
\end{figure}

\paragraph{વ્યુત્પત્તિ:}
શ્રેણીમાં કેપેસિટર માટે:
\begin{itemize}
    \item બધા કેપેસિટર પર સમાન ભાર \(Q\)
    \item કુલ વોલ્ટેજ \(V = V_1 + V_2 + V_3\)
    \item દરેક માટે: \(V_1 = Q/C_1\), \(V_2 = Q/C_2\), \(V_3 = Q/C_3\)
    \item સમકક્ષ માટે: \(V = Q/C_{eq}\)
\end{itemize}

બદલી કરતાં:
\[ \frac{Q}{C_{eq}} = \frac{Q}{C_1} + \frac{Q}{C_2} + \frac{Q}{C_3} \]
\[ \frac{1}{C_{eq}} = \frac{1}{C_1} + \frac{1}{C_2} + \frac{1}{C_3} \]

\paragraph{પરિણામ:}
\(n\) કેપેસિટર શ્રેણીમાં માટે: \(\frac{1}{C_{eq}} = \sum_{i=1}^{n} \frac{1}{C_i}\)

\paragraph{મેમરી ટ્રીક:} \emph{Series: Add reciprocals (like resistors in parallel).}

\subsection{પ્રશ્ન 3(a)(3) [3 marks]}
\textbf{વ્યાખ્યાયિત કરો: ઇન્ફ્રાસોનિક ધ્વનિ, શ્રાવ્ય ધ્વનિ, અલ્ટ્રાસોનિક ધ્વનિ}

\subsubsection{ઉકેલ}
\paragraph{વ્યાખ્યાઓ:}

\subparagraph{ઇન્ફ્રાસોનિક ધ્વનિ:}
20 Hz થી નીચી આવૃત્તિ ધરાવતા ધ્વનિ તરંગોને (માનવ શ્રવણ વિસ્તારથી નીચે) ઇન્ફ્રાસોનિક અથવા સબસોનિક ધ્વનિ કહે છે. ઉદાહરણો: ભૂકંપ તરંગો, વ્હેલનો અવાજ, હાથીનો સંદેશાવ્યવહાર. માનવી આને સાંભળી શકતા નથી પરંતુ કંપનો અનુભવી શકે છે. ઉપયોગ: ભૂસ્તરશાસ્ત્રીય અભ્યાસ, પ્રાણી વર્તન સંશોધન.

\subparagraph{શ્રાવ્ય ધ્વનિ:}
20 Hz થી 20,000 Hz (20 kHz) ની આવૃત્તિ વિસ્તાર ધરાવતા ધ્વનિ તરંગો જે સામાન્ય માનવ કાન દ્વારા શોધી શકાય છે તેને શ્રાવ્ય ધ્વનિ કહે છે. આ માનવ શ્રવણની વિસ્તાર છે. મોટાભાગની વાણી 250 Hz થી 6000 Hz વચ્ચે થાય છે. સંગીત વાદ્યો આ વિસ્તારમાં ધ્વનિ ઉત્પન્ન કરે છે.

\subparagraph{અલ્ટ્રાસોનિક ધ્વનિ:}
20 kHz થી ઉપરની આવૃત્તિ ધરાવતા ધ્વનિ તરંગોને (માનવ શ્રવણ વિસ્તારથી ઉપર) અલ્ટ્રાસોનિક ધ્વનિ કહે છે. ઉદાહરણો: શ્વાનની સીટી (25 kHz), ચામાચીડિયાની નેવિગેશન (100 kHz સુધી), તબીબી અલ્ટ્રાસાઉન્ડ (1-18 MHz). ઉપયોગો: SONAR, તબીબી ઇમેજિંગ, સફાઈ, વેલ્ડિંગ, અંતર માપન.

\paragraph{મેમરી ટ્રીક:} \emph{Infra < 20 Hz, Audible = 20 Hz - 20 kHz, Ultra > 20 kHz.}

\subsection{પ્રશ્ન 3(b)(1) [4 marks]}
\textbf{સમાંતર પ્લેટ કેપેસિટર માટે \(C = \frac{\epsilon_0 A}{d}\) સાબિત કરો.}

\subsubsection{ઉકેલ}
\paragraph{રચના:}
સમાંતર પ્લેટ કેપેસિટર જેમાં:
\begin{itemize}
    \item પ્લેટ ક્ષેત્રફળ = \(A\)
    \item પ્લેટો વચ્ચેનું અંતર = \(d\)
    \item પ્લેટો પરનો ભાર = \(+Q\) અને \(-Q\)
    \item મુક્ત અવકાશની પરવાનગીતા = \(\epsilon_0\)
\end{itemize}

\paragraph{વ્યુત્પત્તિ:}

\subparagraph{પગલું 1 - વિદ્યુતક્ષેત્ર:}
સમાંતર પ્લેટ કેપેસિટર માટે, પ્લેટો વચ્ચે વિદ્યુતક્ષેત્ર સમાન છે. સપાટી ભાર ઘનતા \(\sigma\) ધરાવતા વાહક માટે ગૌસના નિયમનો ઉપયોગ કરીને, વિદ્યુતક્ષેત્ર છે: \(E = \frac{\sigma}{\epsilon_0} = \frac{Q}{\epsilon_0 A}\)
જ્યાં \(\sigma = Q/A\) સપાટી ભાર ઘનતા છે. આ ક્ષેત્ર ધન થી ઋણ પ્લેટ તરફ નિર્દેશ કરે છે અને પ્લેટો વચ્ચેની આખી જગ્યામાં સતત રહે છે.

\subparagraph{પગલું 2 - સ્થિતિમાન તફાવત:}
પ્લેટો વચ્ચેનો સ્થિતિમાન તફાવત એ એકમ ભાર એક પ્લેટથી બીજી પ્લેટ તરફ ખસેડવામાં કરવામાં આવતું કાર્ય છે. કારણ કે ક્ષેત્ર સમાન છે:
\(V = Ed = \frac{Qd}{\epsilon_0 A}\)
આ સંબંધ દર્શાવે છે ક પ્લેટ અંતર અને ભાર સાથે વોલ્ટેજ રેખીય રીતે વધે છે.

\subparagraph{પગલું 3 - કેપેસિટન્સ:}
વ્યાખ્યા પ્રમાણે, કેપેસિટન્સ \(C = \frac{Q}{V}\). વોલ્ટેજ માટેના સમીકરણને બદલી કરતાં:
\[ C = \frac{Q}{\frac{Qd}{\epsilon_0 A}} = \frac{Q \epsilon_0 A}{Qd} = \frac{\epsilon_0 A}{d} \]

\paragraph{પરિણામ:}
\[ \boxed{C = \frac{\epsilon_0 A}{d}} \]

આ દર્શાવે છે કે કેપેસિટન્સ:
\begin{itemize}
    \item પ્લેટ ક્ષેત્રફળ \(A\) ના સીધા પ્રમાણમાં છે: મોટું ક્ષેત્રફળ વધુ ભાર સંગ્રહિત કરે છે
    \item અંતર \(d\) ના વિપરીત પ્રમાણમાં છે: નજીકની પ્લેટો મજબૂત ક્ષેત્ર બનાવે છે
    \item ભાર \(Q\) અથવા વોલ્ટેજ \(V\) થી સ્વતંત્ર છે: કેપેસિટન્સ ભૌમિતિક ગુણધર્મ છે
\end{itemize}

\paragraph{મેમરી ટ્રીક:} \emph{C = \(\epsilon\)A/d (Epsilon-Area/distance).}

\subsection{પ્રશ્ન 3(b)(2) [4 marks]}
\textbf{વિદ્યુતક્ષેત્ર રેખાઓની લાક્ષણિકતાઓ સૂચિબદ્ધ કરો.}

\subsubsection{ઉકેલ}
\paragraph{વિદ્યુતક્ષેત્ર રેખાઓની લાક્ષણિકતાઓ:}

\begin{enumerate}
    \item \textbf{ઉદ્ભવ અને સમાપ્તિ:} વિદ્યુતક્ષેત્ર રેખાઓ ધન ભારથી શરૂ થાય છે અને ઋણ ભાર પર સમાપ્ત થાય છે. ઋણ ભારની ગેરહાજરીમાં, તેઓ અનંત સુધી વિસ્તરે છે.
    
    \item \textbf{દિશા:} કોઈપણ બિંદુ પર ક્ષેત્ર રેખાની સ્પર્શરેખા તે બિંદુ પર વિદ્યુતક્ષેત્રની દિશા આપે છે. ક્ષેત્ર +ve થી દૂર અને -ve તરફ નિર્દેશ કરે છે.
    
    \item \textbf{છેદન નથી:} બે વિદ્યુતક્ષેત્ર રેખાઓ ક્યારેય એકબીજાને છેદતી નથી. જો તેઓ છેદે, તો છેદબિંદુ પર વિદ્યુતક્ષેત્રની બે દિશાઓ હશે, જે અશક્ય છે.
    
    \item \textbf{ઘનતા અને તાકાત:} એકમ ક્ષેત્રફળ દીઠ ક્ષેત્ર રેખાઓની સંખ્યા (ઘનતા) વિદ્યુતક્ષેત્રના પરિમાણના પ્રમાણમાં છે. નજીકની રેખાઓ મજબૂત ક્ષેત્ર દર્શાવે છે.
    
    \item \textbf{લંબતા:} સંતુલનમાં વાહકની સપાટી સાથે વિદ્યુતક્ષેત્ર રેખાઓ હંમેશા લંબ હોય છે. વાહકની અંદર, વિદ્યુતક્ષેત્ર શૂન્ય છે.
    
    \item \textbf{સાતત્ય:} ક્ષેત્ર રેખાઓ કોઈ તૂટક વગરની સતત વક્ર છે. તેઓ ખાલી જગ્યામાં શરૂ અથવા સમાપ્ત થતી નથી (અનંત સિવાય).
    
    \item \textbf{સમપ્રમાણતા:} સમપ્રમાણ ભાર વિતરણ માટે, ક્ષેત્ર રેખાઓ અનુરૂપ સમપ્રમાણતા દર્શાવે છે. ઉદાહરણ: બિંદુ ભાર માટે રેડિયલ રેખાઓ, અનંત સમતલ માટે સમાંતર રેખાઓ.
    
    \item \textbf{ભૌતિક નથી:} ક્ષેત્ર રેખાઓ કલ્પનાશીલ રેખાઓ છે જેનો ઉપયોગ દૃશ્યાવલોકન માટે થાય છે. તેઓ ભારોના વાસ્તવિક ભૌતિક માર્ગનું પ્રતિનિધિત્વ કરતી નથી.
\end{enumerate}

\paragraph{મેમરી ટ્રીક:} \emph{+ve to -ve, Never cross, Density = Strength, Perpendicular to surfaces.}

\subsection{પ્રશ્ન 3(b)(3) [4 marks]}
\textbf{અલ્ટ્રાસોનિક તરંગોના ઉત્પાદન માટે ઉપયોગમાં લેવામાં આવતી મેગ્નેટોસ્ટ્રિક્શન પદ્ધતિની રચના અને કાર્યપદ્ધતિનું વર્ણન કરો.}

\subsubsection{ઉકેલ}
\paragraph{સિદ્ધાંત:}
મેગ્નેટોસ્ટ્રિક્શન એ ફેરોમેગ્નેટિક પદાર્થો (લોખંડ, નિકલ, કોબાલ્ટ)નો ગુણધર્મ છે જ્યારે ચુંબકીય ક્ષેત્રમાં મૂકવામાં આવે ત્યારે તેમના પરિમાણો બદલાય છે. જ્યારે ક્ષેત્ર બદલાય છે, ત્યારે પદાર્થ વિસ્તરે છે અથવા સંકોચાય છે, યાંત્રિક કંપનો ઉત્પન્ન કરે છે.

\paragraph{રચના આકૃતિ:}
\begin{figure}[H]
\centering
\begin{tikzpicture}[scale=1.2]
    % Rod
    \draw[thick,fill=gray!30] (0,0.5) rectangle (6,1.5);
    \node at (3,1) {Magnetostrictive Rod (Nickel)};
    % Coil windings
    \foreach \x in {1.5,1.8,...,4.5} {
        \draw[thick,blue] (\x,0.4) -- (\x,0.5) -- (\x+0.2,0.5) -- (\x+0.2,1.5) -- (\x+0.3,1.5) -- (\x+0.3,0.5);
    }
    \node[blue] at (3,0) {Copper Coil};
    % AC Source
    \draw[thick] (0,1) -- (-0.5,1);
    \draw[thick] (-0.5,0.8) rectangle (-1.5,1.2);
    \node at (-1,1) {\small AC};
    \node[below] at (-1,0.8) {\small Source};
    % Reflector plates
    \draw[thick,fill=gray!50] (-0.3,0.3) rectangle (0,1.7);
    \draw[thick,fill=gray!50] (6,0.3) rectangle (6.3,1.7);
    \node[left,rotate=90] at (-0.15,1) {\tiny Reflector};
    \node[right,rotate=90] at (6.15,1) {\tiny Reflector};
    % Waves
    \draw[->,thick,red] (6.5,1) -- (7.5,1);
    \node[red,right] at (7.5,1) {Ultrasonic waves};
\end{tikzpicture}
\caption{મેગ્નેટોસ્ટ્રિક્શન અલ્ટ્રાસોનિક જનરેટર}
\end{figure}

\paragraph{રચના ભાગો:}
\begin{itemize}
    \item \textbf{મેગ્નેટોસ્ટ્રિક્ટિવ સળિયો:} ફેરોમેગ્નેટિક પદાર્થની સળિયો (નિકલ અથવા લોખંડ-નિકલ મિશ્રધાતુ) જે કંપન કરે છે
    \item \textbf{કોઇલ:} સળિયાની આસપાસ લપેટેલી ઇન્સ્યુલેટેડ તાંબાના તારની કોઇલ જે ચુંબકીય ક્ષેત્ર બનાવે છે
    \item \textbf{AC સ્રોત:} ઉચ્ચ-આવૃત્તિ વૈકલ્પિક પ્રવાહ (20-100 kHz) પાવર સપ્લાઇ
    \item \textbf{પરાવર્તક પ્લેટ્સ:} તરંગોને પરાવર્તિત અને કેન્દ્રિત કરવા છેડે ધાતુની પ્લેટ્સ
\end{itemize}

\paragraph{કાર્યપદ્ધતિ:}
\begin{enumerate}
    \item કોઇલમાંથી AC પ્રવાહ વહે છે, વૈકલ્પિક ચુંબકીય ક્ષેત્ર બનાવે છે
    \item મેગ્નેટોસ્ટ્રિક્ટિવ સળિયો AC ની આવૃત્તિ પર સામયિક વિસ્તરણ અને સંકોચન પસાર કરે છે
    \item જ્યારે આવૃત્તિ સળિયાની કુદરતી આવૃત્તિ (અનુનાદ) સાથે મેળ ખાય છે, મહત્તમ કંપનો થાય છે
    \item આ યાંત્રિક કંપનો આસપાસના માધ્યમમાં અલ્ટ્રાસોનિક તરંગો ઉત્પન્ન કરે છે
    \item સળિયાની લંબાઈ \(L\) એવી પસંદ કરવામાં આવે છે કે અનુનાદ માટે \(L = n\lambda/2\)
\end{enumerate}

\paragraph{ફાયદા:}
ઉચ્ચ શક્તિ અલ્ટ્રાસોનિક તરંગો, સરળ રચના, વિશ્વસનીય કામગીરી.

\paragraph{ગેરફાયદા:}
નીચી અલ્ટ્રાસોનિક આવૃત્તિઓ (< 100 kHz) સુધી મર્યાદિત, ઉચ્ચ આવૃત્તિઓ પર કાર્યક્ષમતા ઘટે છે, સળિયાનું ગરમ થવું.

\paragraph{ઉપયોગો:}
અલ્ટ્રાસોનિક સફાઈ, ડ્રિલિંગ, વેલ્ડિંગ, SONAR સિસ્ટમ્સ.

\paragraph{મેમરી ટ્રીક:} \emph{Magnetic field → Dimension change → Vibrations → Ultrasound.}

% ================================================================
% QUESTION 4 \& 5: Gujarati equivalents with same structure/diagrams
% Following exact same content as English for parity
% ================================================================

\section{પ્રશ્ન 4}

\subsection{પ્રશ્ન 4(a)(1) [3 marks]}
\textbf{એક રેડિઓસ્ટેશન 100 MHz આવૃત્તિવાળા તરંગોનું ઉત્સર્જન કરે છે. જો આ તરંગોની ઝડપ \(3 \times 10^8\,m/s\) હોય તો તેની તરંગલંબાઈ શોધો.}

\subsubsection{ઉકેલ}
\paragraph{આપેલ માહિતી:}
\begin{itemize}
    \item આવૃત્તિ, \(f = 100\,MHz = 100 \times 10^6\,Hz = 10^8\,Hz\)
    \item તરંગની ઝડપ, \(v = 3 \times 10^8\,m/s\)
\end{itemize}

\paragraph{સૂત્ર:}
બધા વિદ્યુતચુંબકીય તરંગો માટે તરંગની ઝડપ, આવૃત્તિ અને તરંગલંબાઈ વચ્ચેનો મૂળભૂત સંબંધ:
\[ v = f\lambda \]
જ્યાં \(v\) તરંગની ઝડપ m/s માં, \(f\) આવૃત્તિ Hz માં, અને \(\lambda\) (lambda) તરંગલંબાઈ મીટરમાં છે. આ સમીકરણ દર્શાવે છે કે આપેલ તરંગ ઝડપ માટે તરંગલંબાઈ અને આવૃત્તિ વિપરીત પ્રમાણમાં છે. વધુ આવૃત્તિનો અર્થ નાની તરંગલંબાઈ અને પ્રત્yeકનું ઉલ્ટું.

\paragraph{ગણતરી:}
તરંગલંબાઈ મેળવવા સૂત્રને પુનઃરચના કરતાં:
\[ \lambda = \frac{v}{f} = \frac{3 \times 10^8\,m/s}{10^8\,Hz} = 3\,m \]

\paragraph{જવાબ:}
રેડિઓ સિગ્નલ્સની તરંગલંબાઈ \textbf{3 મીટર} છે. આ વિદ્યુતચુંબકીય સ્પેક્ટ્રમના રેડિઓ તરંગ પ્રદેશમાં છે, પ્રસારણ માટે આદર્શ કારણ કે: (1) આ તરંગો વાયુમંડળ દ્વારા લાંબા અંતર સુધી મુસાફરી કરી શકે છે, (2) ઇમારતો અને અવરોધોમાં પ્રવેશી શકે છે, (3) યોગ્ય આકારની એન્ટેનાની જરૂર (quarter-wavelength = 75 cm), અને (4) અવરોધોની આસપાસ સારા વિવર્તન ગુણધર્મો.

\paragraph{મેમરી ટ્રીક:} \emph{\(\lambda\) = v/f (wavelength = speed/frequency).}

\subsection{પ્રશ્ન 4(a)(2) [3 marks]}
\textbf{સ્નેલનો નિયમ જણાવો અને માધ્યમનો વક્રીભવનાંક સમજાવો.}

\subsubsection{ઉકેલ}
\paragraph{સ્નેલનો નિયમ:}
જ્યારે પ્રકાશ કિરણ એક માધ્યમમાંથી બીજા માધ્યમમાં પસાર થાય છે, ત્યારે આપાતન કોણની સાઇન અને વક્રીભવનના કોણની સાઇનનો ગુણોત્તર અચળ રહે છે. ગણિતીય રીતે:
\[ \frac{\sin i}{\sin r} = \text{constant} = \frac{n_2}{n_1} \]
અથવા: \( n_1 \sin i = n_2 \sin r \)

જ્યાં \(i\) આપાતન કોણ, \(r\) વક્રીભવનનો કોણ, \(n_1\) અને \(n_2\) માધ્યમ 1 અને 2 ના વક્રીભવનાંક છે.

\paragraph{વક્રીભવનાંક:}
માધ્યમનો વક્રીભવનાંક (\(n\)) એ શૂન્યમાં પ્રકાશની ઝડપ અને તે માધ્યમમાં પ્રકાશની ઝડપનો ગુણોત્તર છે:
\[ n = \frac{c}{v} \]

જ્યાં \(c = 3 \times 10^8\,m/s\) શૂન્યમાં પ્રકાશની ઝડપ, અને \(v\) માધ્યમમાં પ્રકાશની ઝડપ છે. વક્રીભવનાંક પરિમાણવિહીન છે અને હંમેશા \(\geq 1\). વધુ વક્રીભવનાંક એટલે કે પ્રકાશ તે માધ્યમમાં ધીમો ચાલે છે અને વિરલ માધ્યમમાંથી પ્રવેશતા સમયે લંબ તરફ વધુ વળે છે.

\paragraph{ઉદાહરણો:}
\begin{itemize}
    \item હવા: \(n \approx 1.0003 \approx 1\)
    \item પાણી: \(n \approx 1.33\)
    \item કાચ: \(n \approx 1.5\)
    \item હીરો: \(n \approx 2.42\)
\end{itemize}

\paragraph{મેમરી ટ્રીક:} \emph{n\(_1\)sin(i) = n\(_2\)sin(r), n = c/v.}

\subsection{પ્રશ્ન 4(a)(3) [3 marks]}
\textbf{સરખામણી કરો: સામાન્ય પ્રકાશ અને LASER}

\subsubsection{ઉકેલ}
\paragraph{સરખામણી ટેબલ:}
\begin{table}[H]
\caption{સામાન્ય પ્રકાશ વિ. LASER}
\centering
\begin{tabularx}{\textwidth}{|X|X|}
\hline
\textbf{સામાન્ય પ્રકાશ} & \textbf{LASER} \\ \hline
બહુવર્ણી (અનેક તરંગલંબાઈ) & એકવર્ણી (એક તરંગલંબાઈ) \\ \hline
અસંગત (રેન્ડમ ફેઝ) & સંગત (અચળ ફેઝ સંબંધ) \\ \hline
અપસારી કિરણપુંજ (ફેલાય છે) & અત્યંત દિશાત્મક (સમાંતર કિરણ) \\ \hline
ઓછી તીવ્રતા & ખૂબ ઉચ્ચ તીવ્રતા \\ \hline
ઉદાહરણ: બલ્બ, સૂર્યપ્રકાશ & ઉદાહરણ: He-Ne laser, CO\(_2\) laser \\ \hline
ઉપયોગો: સામાન્ય લાઇટિંગ & ઉપયોગો: શસ્ત્રક્રિયા, કટિંગ, સંચાર \\ \hline
\end{tabularx}
\end{table}

\paragraph{મુખ્ય તફાવતો:}
\begin{enumerate}
    \item \textbf{એકવર્ણીતા:} LASER એક રંગ/તરંગલંબાઈ ઉત્સર્જિત કરે છે, સામાન્ય પ્રકાશમાં રંગોનું મિશ્રણ હોય છે
    \item \textbf{સંગતતા:} LASER તરંગો ફેઝમાં હોય છે, સામાન્ય પ્રકાશ તરંગોનો રેન્ડમ ફેઝ હોય છે
    \item \textbf{દિશાત્મકતા:} LASER અત્યંત દિશાત્મક છે, સામાન્ય પ્રકાશ બધી દિશામાં ફેલાય છે
    \item \textbf{તીવ્રતા:} LASER કેન્દ્રિત ઉચ્ચ તીવ્રતા ધરાવે છે, સામાન્ય પ્રકાશ ઓછી સુતીવ્રતા ધરાવે છે
\end{enumerate}

\paragraph{મેમરી ટ્રીક:} \emph{LASER = Monochromatic + Coherent + Directional + Intense.}

\subsection{પ્રશ્ન 4(b)(1) [4 marks]}
\textbf{જરૂરી આકૃતિ સાથે ઓપ્ટિકલ ફાઇબરની રચના દર્શાવો.}

\subsubsection{ઉકેલ}
\paragraph{રચના આકૃતિ:}
\begin{figure}[H]
\centering
\begin{tikzpicture}[scale=1.2]
    % Core
    \fill[blue!30] (0,0.3) rectangle (8,0.7);
    \draw[thick] (0,0.3) -- (8,0.3);
    \draw[thick] (0,0.7) -- (8,0.7);
    \node at (4,0.5) {Core (n\(_1\))};
    % Cladding
    \fill[blue!10] (0,0) rectangle (8,0.3);
    \fill[blue!10] (0,0.7) rectangle (8,1);
    \draw[thick] (0,0) -- (8,0);
    \draw[thick] (0,1) -- (8,1);
    \node at (4,0.15) {\small Cladding (n\(_2\))};
    \node at (4,0.85) {\small Cladding (n\(_2\))};
    % Jacket
    \draw[very thick] (-0.2,-0.2) rectangle (8.2,1.2);
   \node[right] at (8.3,0.5) {Protective Jacket};
    % Light ray
    \draw[->,red,thick] (-0.5,0.5) -- (0,0.6);
    \draw[->,red,thick] (0,0.6) -- (2,0.4) -- (4,0.6) -- (6,0.4) -- (8,0.5);
    \node[red,left] at (-0.5,0.5) {Light};
\end{tikzpicture}
\caption{ઓપ્ટિકલ ફાઇબર રચના}
\end{figure}

\paragraph{ભાગો:}
\begin{enumerate}
    \item \textbf{કોર:} ઉચ્ચ વક્રીભવનાંક (\(n_1\)) સાથે મધ્ય સિલિન્ડર. પ્રકાશ કોર દ્વારા સંપૂર્ણ આંતરિક પરાવર્તન દ્વારા મુસાફરી કરે છે. વ્યાસ: 8-10 \(\mu\)m (single mode) અથવા 50-100 \(\mu\)m (multimode).
    
    \item \textbf{ક્લેડિંગ:} નીચા વક્રીભવનાંક (\(n_2 < n_1\)) સાથે કોરની આસપાસ. સંપૂર્ણ આંતરિક પરાવર્તન સુનિશ્ચિત કરે છે અને પ્રકાશ લીકેજ રોકે છે. જાડાઈ: સામાન્ય રીતે 125 \(\mu\)m.
    
    \item \textbf{સંરક્ષક જેકેટ:} બાહ્ય પોલિમર કોટિંગ ફાઇબરને ભૌતિક નુકસાન, ભેજ અને યાંત્રિક તાણથી સુરક્ષિત કરે છે.
\end{enumerate}

\paragraph{કાર્ય સિદ્ધાંત:}
પ્રકાશ કોર-ક્લેડિંગ સીમા પર સંપૂર્ણ આંતરિક પરાવર્તન દ્વારા કોર દ્વારા મુસાફરી કરે છે અને આવશ્યક શરતો: \(n_1 > n_2\) અને આપાતન કોણ \(>\) ક્રાંતિક કોણ.

\paragraph{મેમરી ટ્રીક:} \emph{Core (high n) + Cladding (low n) + Jacket = Fiber.}

\subsection{પ્રશ્ન 4(b)(2) [4 marks]}
\textbf{ઇજનેરી અને મેડિકલ ક્ષેત્રે LASER ના ઉપયોગોની યાદી આપો.}

\subsubsection{ઉકેલ}
\paragraph{ઇજનેરી ઉપયોગો:}
\begin{enumerate}
    \item \textbf{સામગ્રી પ્રક્રિયા:} ધાતુ અને પ્લાસ્ટિકનું ઉચ્ચ ચોકસાઈ સાથે laser કટિંગ, વેલ્ડિંગ, ડ્રિલિંગ
    \item \textbf{સંચાર:} લાંબા અંતર પર હાઇ-સ્પીડ ડેટા પ્રસારણ માટે ફાઇબર ઓપ્ટિક સંચાર
    \item \textbf{માપન:} અંતર માપન (LIDAR), બાંધકામમાં સંરેખણ, સર્વેક્ષણ
    \item \textbf{ઉત્પાદન:} 3D printing, laser engraving, સપાટી સારવાર અને સખ્તીકરણ
    \item \textbf{હોલોગ્રાફી:} સુરક્ષા અને પ્રદર્શન માટે 3D હોલોગ્રામ બનાવવા
    \item \textbf{બારકોડ સ્કેનિંગ:} રિટેલ ચેકઆઉટ સિસ્ટમ્સ અને ઇન્વેન્ટરી મેનેજમેન્ટ
    \item \textbf{સંરક્ષણ:} Laser guided missiles, range finders, target designation
\end{enumerate}

\paragraph{તબીબી ઉપયોગો:}
\begin{enumerate}
    \item \textbf{શસ્ત્રક્રિયા:} રક્તહીન કટિંગ, ચોક્કસ પેશી દૂર કરવા માટે laser scalpel
    \item \textbf{નેત્રવિજ્ઞાન:} દ્રષ્ટિ સુધારણા માટે LASIK આંખની શસ્ત્રક્રિયા, રેટિના સમારકામ
    \item \textbf{ચર્મરોગવિજ્ઞાન:} ટેટૂ દૂર કરવું, ત્વચા પુનરુત્થાન, વાળ દૂર કરવા
    \item \textbf{દંતચિકિત્સા:} પોલાણ સારવાર, પેઢાની શસ્ત્રક્રિયા, દાંત સફેદ કરવા
    \item \textbf{કેન્સર સારવાર:} ગાંઠ દૂર કરવી, photodynamic therapy
    \item \textbf{નિદાન:} Laser microscopy, રક્ત વિશ્લેષણ, પેશી imaging
    \item \textbf{સૌંદર્ય:} સરસ ઘટાડો, ડાઘ સારવાર
\end{enumerate}

\paragraph{ફાયદા:}
ઉચ્ચ ચોકસાઈ, ન્યૂનતમ રક્તસ્રાવ, ઝડપી હીલિંગ, ઘટાડેલું ચેપનું જોખમ, બિન-સંપર્ક કામગીરી.

\paragraph{મેમરી ટ્રીક:} \emph{Engineering: Cut-Weld-Communicate-Measure. Medical: Surgery-Eye-Skin-Cancer.}

\subsection{પ્રશ્ન 4(b)(3) [4 marks]}
\textbf{P-type અને N-type અર્ધવાહકો સમજાવો.}

\subsubsection{ઉકેલ}
\paragraph{N-Type અર્ધવાહક:}
\begin{itemize}
    \item \textbf{ડોપિંગ:} શુદ્ધ અર્ધવાહક (Si અથવા Ge) ને પંચસંયોજી અશુદ્ધિ (P, As, Sb) સાથે ડોપ કરવામાં આવે છે
    \item \textbf{બહુમતી વાહકો:} ઇલેક્ટ્રોન (દાતા અણુઓમાંથી મુક્ત ઇલેક્ટ્રોન)
    \item \textbf{લઘુમતી વાહકો:} હોલ
    \item \textbf{દાતા અણુઓ:} પંચસંયોજી અણુઓ વધારાનું ઇલેક્ટ્રોન દાન કરે છે
    \item \textbf{ભાર:} એકંદરે વિદ્યુત તટસ્થ પરંતુ ઇલેક્ટ્રોન-સમૃદ્ધ
\end{itemize}

\paragraph{P-Type અર્ધવાહક:}
\begin{itemize}
    \item \textbf{ડોપિંગ:} શુદ્ધ અર્ધવાહક ત્રિસંયોજી અશુદ્ધિ (B, Al, Ga, In) સાથે ડોપ કરવામાં આવે છે
    \item \textbf{બહુમતી વાહકો:} હોલ (ઇલેક્ટ્રોનની ગેરહાજરી)
    \item \textbf{લઘુમતી વાહકો:} ઇલેક્ટ્રોન
    \item \textbf{સ્વીકારક અણુઓ:} ત્રિસંયોજી અણુઓ ઇલેક્ટ્રોન સ્વીકારે છે, હોલ બનાવે છે
    \item \textbf{ભાર:} એકંદરે વિદ્યુત તટસ્થ પરંતુ હોલ-સમૃદ્ધ
\end{itemize}

\paragraph{સરખામણી:}
\begin{itemize}
    \item N-type પાસે વધારાના ઇલેક્ટ્રોન, P-type પાસે વધારાના હોલ
    \item N-type Group V તત્વો વાપરે છે, P-type Group III તત્વો વાપરે છે
    \item બંને વિદ્યુત રીતે તટસ્થ છે
    \item બંનેની વાહકતા શુદ્ધ અર્ધવાહક કરતાં વધુ છે
\end{itemize}

\paragraph{ઉપયોગો:}
ડાયોડ, ટ્રાન્ઝિસ્ટર, સૌર કોષો, LED અને અન્ય અર્ધવાહક ઉપકરણોમાં PN જંકશન બનાવવા માટે વપરાય છે.

\paragraph{મેમરી ટ્રીક:} \emph{N = Negative carriers (electrons), P = Positive carriers (holes).}

% ========================================
% QUESTION 5: Semiconductors & Logic (14 marks)
% Q5(A): 3 marks each, Q5(B): 4 marks each
% ========================================

\section{પ્રશ્ન 5}

\subsection{પ્રશ્ન 5(a)(1) [3 marks]}
\textbf{એનર્જી બેન્ડગેપના આધારે વાહકો, અર્ધવાહકો અને અવાહકોનું વર્ગીકરણ કરો.}

\subsubsection{ઉકેલ}
\paragraph{વર્ગીકરણ ટેબલ:}
\begin{table}[H]
\caption{એનર્જી બેન્ડ ગેપ પર આધારિત વર્ગીકરણ}
\centering
\begin{tabularx}{\textwidth}{|X|X|X|X|}
\hline
\textbf{પ્રકાર} & \textbf{બેન્ડ ગેપ (E\(_g\))} & \textbf{ઉદાહરણ} & \textbf{વાહકતા} \\ \hline
વાહકો & કોઈ બેન્ડ ગેપ નથી (E\(_g\) = 0) & Cu, Ag, Al, Au & ખૂબ ઉચ્ચ (\(10^7\,S/m\)) \\ \hline
અર્ધવાહકો & નાનો ગેપ (0.1 થી 3 eV) & Si (1.1 eV), Ge (0.7 eV) & મધ્યમ (\(10^{-4}\) થી \(10^4\,S/m\)) \\ \hline
અવાહકો & મોટો ગેપ (\(>3\) eV) & Diamond (5.5 eV), Glass & ખૂબ ઓછી (\(10^{-15}\,S/m\)) \\ \hline
\end{tabularx}
\end{table}

\paragraph{વિગતવાર વર્ણન:}
\begin{enumerate}
    \item \textbf{વાહકો:} સંયોજકતા અને વહન બેન્ડ overlap થાય છે. ઇલેક્ટ્રોન વધારાની ઊર્જાની જરૂર વગર મુક્તપણે ખસી શકે છે. બધા તાપમાને ઉચ્ચ વાહકતા.
    
    \item \textbf{અર્ધવાહકો:} નાનો પ્રતિબંધિત ઊર્જા અંતર. ઓરડાના તાપમાને, થર્મલ ઊર્જા કેટલાક ઇલેક્ટ્રોન માટે વહન બેન્ડમાં કૂદવા માટે પૂરતું ઉત્તેજન પૂરું પાડે છે. તાપમાન સાથે વાહકતા વધે છે.
    
    \item \textbf{અવાહકો:} ખૂબ મોટો ઊર્જા અંતર. ઇલેક્ટ્રોન ઉચ્ચ તાપમાને પણ સરળતાથી વહન બેન્ડમાં કૂદી શકતા નથી. નબળા વાહક રહે છે.
\end{enumerate}

\paragraph{મેમરી ટ્રીક:} \emph{Conductors: Zero gap. Semiconductors: Small gap. Insulators: Large gap.}

\subsection{પ્રશ્ન 5(a)(2) [3 marks]}
\textbf{જરૂરી ટ્રુથ ટેબલ સાથે OR અને AND લોજીક ગેટ સમજાવો.}

\subsubsection{ઉકેલ}
\paragraph{OR ગેટ:}
જો કોઈપણ ઇનપુટ HIGH (1) હોય તો આઉટપુટ HIGH (1) છે.

\textbf{ટ્રુથ ટેબલ:}
\begin{table}[H]
\caption{OR ગેટ ટ્રુથ ટેબલ}
\centering
\begin{tabularx}{0.7\textwidth}{|X|X|X|}
\hline
\textbf{A} & \textbf{B} & \textbf{Y = A + B} \\ \hline
0 & 0 & 0 \\ \hline
0 & 1 & 1 \\ \hline
1 & 0 & 1 \\ \hline
1 & 1 & 1 \\ \hline
\end{tabularx}
\end{table}

\textbf{બુલિયન અભિવ્યક્તિ:} \( Y = A + B \)

\paragraph{AND ગેટ:}
જો બધા ઇનપુટ HIGH (1) હોય તો જ આઉટપુટ HIGH (1) છે.

\textbf{ટ્રુથ ટેબલ:}
\begin{table}[H]
\caption{AND ગેટ ટ્રુથ ટેબલ}
\centering
\begin{tabularx}{0.7\textwidth}{|X|X|X|}
\hline
\textbf{A} & \textbf{B} & \textbf{Y = A · B} \\ \hline
0 & 0 & 0 \\ \hline
0 & 1 & 0 \\ \hline
1 & 0 & 0 \\ \hline
1 & 1 & 1 \\ \hline
\end{tabularx}
\end{table}

\textbf{બુલિયન અભિવ્યક્તિ:} \( Y = A \cdot B \) અથવા \( Y = AB \)

\paragraph{સારાંશ:}
OR ગેટ: સરવાળા કામગીરી, AND ગેટ: ગુણાકાર કામગીરી. બંને ડિજિટલ સર્કિટ ડિઝાઇનમાં વપરાતા મૂળભૂત ગેટ્સ છે.

\paragraph{મેમરી ટ્રીક:} \emph{OR = Any HIGH gives HIGH. AND = All HIGH gives HIGH.}

\subsection{પ્રશ્ન 5(a)(3) [3 marks]}
\textbf{વોલ્ટેજ રેગ્યુલેટર તરીકે ઝેનર ડાયોડના ઉપયોગનું વર્ણન કરો.}

\subsubsection{ઉકેલ}
\paragraph{સિદ્ધાંત:}
ઝેનર ડાયોડ reverse bias breakdown પ્રદેશમાં કામ કરે છે. તે પ્રવાહમાં ફેરફાર છતાં તેના ટર્મિનલ્સ પર અચળ વોલ્ટેજ જાળવે છે. આ ગુણધર્મ તેને વોલ્ટેજ નિયમન માટે આદર્શ બનાવે છે.

\paragraph{કાર્યપદ્ધતિ:}
\begin{enumerate}
    \item ઝેનર ડાયોડ reverse bias માં લોડની સમાંતર જોડાયેલ છે
    \item શ્રેણી પ્રતિકાર \(R_s\) પ્રવાહ મર્યાદિત કરવા માટે જોડાયેલ છે
    \item જ્યારે ઇનપુટ વોલ્ટેજ વધે છે, ઝેનર પ્રવાહ વધે છે પરંતુ વોલ્ટેજ \(V_Z\) પર અચળ રહે છે
    \item જ્યારે ઇનપુટ ઘટે છે (પરંતુ \(V_Z\) ઉપર રહે છે), ઝેનર અચળ આઉટપુટ જાળવવા માટે પ્રવાહ સમાયોજિત કરે છે
    \item લોડ ઝેનર breakdown voltage સમાન સ્થિર વોલ્ટેજ મેળવે છે
\end{enumerate}

\paragraph{ઉપયોગો:}
\begin{itemize}
    \item પાવર સપ્લાઇ વોલ્ટેજ સ્થિરીકરણ
    \item સંવેદનશીલ ઇલેક્ટ્રોનિક ઘટકોનું રક્ષણ
    \item સંદર્ભ વોલ્ટેજ નિર્માણ
    \item વોલ્ટેજ મર્યાદા સર્કિટ્સ
\end{itemize}

\paragraph{ફાયદા:}
સરળ, ઓછી કિંમત, વિશ્વસનીય, ઝડપી પ્રતિભાવ.

\paragraph{મર્યાદાઓ:}
મર્યાદિત પાવર હેન્ડલિંગ, વેરિએબલ આઉટપુટ વોલ્ટેજ માટે યોગ્ય નથી.

\paragraph{મેમરી ટ્રીક:} \emph{Zener in reverse bias = Constant voltage regulator.}

\subsection{પ્રશ્ન 5(b)(1) [4 marks]}
\textbf{જરૂરી સર્કિટ સાથે પૂર્ણ તરંગ રેક્ટિફાયર સમજાવો અને ઇનપુટ અને આઉટપુટ તરંગો દોરો.}

\subsubsection{ઉકેલ}
\paragraph{સર્કિટ રચના:}
પૂર્ણ તરંગ રેક્ટિફાયર center-tapped ટ્રાન્સફોર્મર સાથે બે ડાયોડ (D\(_1\) અને D\(_2\)) જોડીને alternating current (AC) ને pulsating direct current (DC) માં રૂપાંતરિત કરે છે. સેકંડરી વિન્ડીંગનું center tap grounded છે, વિપરીત ધ્રુવતાના બે સમાન વોલ્ટેજ સ્રોતો બનાવે છે. Load resistor center tap અને બંને ડાયોડના cathodes વચ્ચે જોડાયેલ છે.

\paragraph{કાર્ય સિદ્ધાંત:}
\begin{itemize}
    \item \textbf{ધન અર્ધ  ચક્ર:} ઉપરનું ટર્મિનલ positive છે. ડાયોડ D\(_1\) forward biased થાય છે અને વહન કરે છે જ્યારે D\(_2\) reverse biased છે અને બ્લોક કરે છે. પ્રવાહ ground થી ઉપરના ટર્મિનલ તરફ load દ્વારા વહે છે.
    \item \textbf{ઋણ અર્ધ ચક્ર:} નીચેનું ટર્મિનલ positive છે. ડાયોડ D\(_2\) forward biased થાય છે અને વહન કરે છે જ્યારે D\(_1\) reverse biased છે. પ્રવાહ ground થી નીચેના ટર્મિનલ તરફ load દ્વારા સમાન દિશામાં વહે છે.
    \item બંને અર્ધ ચક્રો load દ્વારા સમાન દિશામાં આઉટપુટ પ્રવાહ ઉત્પન્ન કરે છે, તેથી ``full wave'' rectification. આ અર્ધ તરંગ rectifier કરતાં મુખ્ય ફાયદો છે જે ફક્ત એક અર્ધ ચક્રનો ઉપયોગ કરે છે.
\end{itemize}

\paragraph{લાક્ષણિકતાઓ:}
\begin{itemize}
    \item કાર્યક્ષમતા: 81.2\% (અર્ધ તરંગના 40.6\% કરતાં ઘણું સારું)
    \item Ripple આવૃત્તિ: \(2f\) (input frequency કરતાં બમણી, ફિલ્ટર કરવું સરળ)
    \item આઉટપુટ વોલ્ટેજ: \(V_{dc} = \frac{2V_m}{\pi} \approx 0.636V_m\)
    \item Peak Inverse Voltage (PIV): \(2V_m\) (દરેક ડાયોડે ટકી શકવું જોઈએ)
    \item અર્ધ તરંગ કરતાં સારો transformer utilization factor
\end{itemize}

\paragraph{ફાયદા:}
ઉચ્ચ કાર્યક્ષમતા (81.2\%), સારો ripple factor (ઓછો ripple), ઉચ્ચ DC આઉટપુટ વોલ્ટેજ, સારું transformer ઉપયોગ, ઉચ્ચ ripple frequency ને કારણે સરળ ફિલ્ટરિંગ.

\paragraph{ગેરફાયદા:}
Center-tapped transformer જરૂરી (વધુ ખર્ચાળ), transformer secondary નો ફક્ત અડધો ભાગ એક સમયે ઉપયોગ, ડાયોડ માટે ઉચ્ચ PIV રેટિંગ જરૂરી.

\paragraph{ઉપયોગો:}
ઇલેક્ટ્રોનિક સર્કિટ માટે DC પાવર સપ્લાય, બેટરી ચાર્જર, AM radio receivers માં signal demodulation, voltage regulators.

\paragraph{મેમરી ટ્રીક:} \emph{Full wave = Both halves utilized, 2 diodes, 81\% efficiency.}

\subsection{પ્રશ્ન 5(b)(2) [4 marks]}
\textbf{P-N જંકશન ડાયોડની ફોરવર્ડ અને રિવર્સ લાક્ષણિકતાઓ દર્શાવો.}

\subsubsection{ઉકેલ}
\paragraph{ફોરવર્ડ બાયસ લાક્ષણિકતાઓ:}
\begin{itemize}
    \item P-બાજુ positive સાથે, N-બાજુ negative સાથે જોડાયેલ
    \item Depletion પ્રદેશ સંકુચિત થાય છે, અવરોધ સંભાવના ઘટે છે
    \item \textbf{Cut-in/Threshold Voltage:} Si: 0.7V, Ge: 0.3V
    \item Cut-in નીચે: નહીવત્ પ્રવાહ (થોડા \(\mu\)A)
    \item Cut-in ઉપર: પ્રવાહ ઘાતાંકીય રીતે વધે છે
    \item Forward પ્રતિકાર: ખૂબ ઓછો (થોડા ohms)
\end{itemize}

\paragraph{રિવર્સ બાયસ લાક્ષણિકતાઓ:}
\begin{itemize}
    \item P-બાજુ negative સાથે, N-બાજુ positive સાથે જોડાયેલ
    \item Depletion પ્રદેશ વિસ્તૃત થાય છે, અવરોધ સંભાવના વધે છે
    \item \textbf{Reverse Saturation Current:} ખૂબ નાનો (થોડા \(\mu\)A થી nA)
    \item વિશાળ વોલ્ટેજ શ્રેણી માટે લગભગ અચળ રહે છે
    \item \textbf{Breakdown Voltage:} ચોક્કસ reverse voltage પછી, પ્રવાહ ઝડપથી વધે છે
    \item Reverse પ્રતિકાર: ખૂબ ઉચ્ચ (ઘણા M\(\Omega\))
\end{itemize}

\paragraph{મુખ્ય મુદ્દા:}
\begin{itemize}
    \item ડાયોડ forward bias માં સરળતાથી વહન કરે છે, reverse bias માં અવરોધે છે
    \item Forward થી reverse પ્રતિકારનો ગુણોત્તર: \(10^6\) થી \(10^8\)
    \item પ્રવાહ માટે one-way valve તરીકે કાર્ય કરે છે
\end{itemize}

\paragraph{ઉપયોગો:}
રેક્ટિફાયર, clipping સર્કિટ્સ, clamping સર્કિટ્સ, voltage regulators.

\paragraph{મેમરી ટ્રીક:} \emph{Forward = Low R, High I. Reverse = High R, Low I.}

\subsection{પ્રશ્ન 5(b)(3) [4 marks]}
\textbf{LED નો સિદ્ધાંત લખો અને તેની રચના અને કાર્યપદ્ધતિ સમજાવો.}

\subsubsection{ઉકેલ}
\paragraph{સિદ્ધાંત:}
Light Emitting Diode (LED) electroluminescence ના સિદ્ધાંત પર કામ કરે છે. જ્યારે P-N જંકશન forward biased હોય છે, ઇલેક્ટ્રોન અને હોલ પુનઃસંયોજન કરે છે, ફોટોન (પ્રકાશ) તરીકે ઊર્જા છોડે છે. ફોટોનની ઊર્જા રંગ નક્કી કરે છે: \(E = hf = \frac{hc}{\lambda}\).

\paragraph{રચના:}
\begin{itemize}
    \item સંયોજન અર્ધવાહકો (GaAs, GaP, GaAsP, GaN) નું P-N જંકશન બનાવેલું
    \item પારદર્શક epoxy resin dome લેન્સ તરીકે કાર્ય કરે છે
    \item પ્રકાશને ઉપર તરફ દિશા આપવા માટે પરાવર્તક કપ
    \item એનોડ (+) અને કેથોડ (-) માટે ધાતુ સંપર્કો
    \item લાંબી lead એનોડ છે, ટૂંકી કેથોડ છે
\end{itemize}

\paragraph{કાર્યપદ્ધતિ:}
\begin{enumerate}
    \item LED forward biased છે (anode થી +ve, cathode થી -ve)
    \item N-પ્રદેશમાંથી ઇલેક્ટ્રોન અને P-પ્રદેશમાંથી હોલ જંકશન તરફ ખસે છે
    \item જંકશન પર, પુનઃસંયોજન થાય છે
    \item ફોટોન (પ્રકાશ) તરીકે ઊર્જા છૂટે છે
    \item રંગ અર્ધવાહક સામગ્રીના બેન્ડ ગેપ પર આધાર રાખે છે
\end{enumerate}

\paragraph{LED રંગો:}
\begin{itemize}
    \item લાલ: GaAsP (1.8 eV)
    \item લીલો: GaP (2.2 eV)
    \item વાદળી: GaN (2.9 eV)
    \item સફેદ: Blue LED + Yellow phosphor coating
\end{itemize}

\paragraph{ફાયદા:}
લાંબુ જીવન (50,000+ કલાક), ઓછો પાવર વપરાશ, ઝડપી સ્વિચિંગ, કોમ્પેક્ટ કદ, ટકાઊ, mercury નથી.

\paragraph{ઉપયોગો:}
ડિસ્પ્લે, indicators, traffic lights, automotive lighting, street lights, backlighting.

\paragraph{મેમરી ટ્રીક:} \emph{LED = Forward bias → Recombination → Photon emission → Light!}

\end{document}

