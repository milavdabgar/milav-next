\documentclass[10pt,a4paper]{article}

% content/resources/templates/preamble.tex
\usepackage[margin=0.6in]{geometry}
\author{Milav Dabgar}
\usepackage{amsmath,amssymb,amsthm}
\usepackage{booktabs}
\usepackage{multirow}
\usepackage{xcolor}
\usepackage{tcolorbox}
\tcbuselibrary{breakable,skins}
\usepackage[colorlinks=true,linkcolor=blue]{hyperref}
\usepackage{titlesec}
\usepackage{enumitem}
\usepackage{tikz}
\usepackage{pgfplots}
\usepackage{circuitikz}
\usepackage[version=4]{mhchem}
\usepackage{longtable}
\usepackage{array}
\usepackage{float}
\usepackage{caption}
\usepackage{listings}

\lstset{
  basicstyle=\small\ttfamily,
  breaklines=true,
  breakatwhitespace=false,
  postbreak=\mbox{\textcolor{red}{$\hookrightarrow$}\space},
  float=false,
  numbers=left,
  numberstyle=\tiny\color{gray},
  numbersep=10pt,
  xleftmargin=2em,
  keywordstyle=\color{blue},
  commentstyle=\color{green!60!black},
  stringstyle=\color{purple},
  backgroundcolor=\color{gray!5},
  showstringspaces=false,
  tabsize=2,
  captionpos=b,
  keepspaces=true,
  columns=flexible
}

\pgfplotsset{compat=1.18}
\usetikzlibrary{shapes,arrows,positioning,calc,patterns,decorations.pathmorphing,decorations.markings,arrows.meta}

% Color scheme
\definecolor{headcolor}{RGB}{0,102,204}
\definecolor{keycolor}{RGB}{220,20,60}
\definecolor{solutioncolor}{RGB}{34,139,34}
\definecolor{mnemoniccolor}{RGB}{148,0,211}
\definecolor{codecolor}{RGB}{0,0,100}

% Spacing
\setlength{\parskip}{3pt}
\setlist[itemize]{nosep}
\setlist[enumerate]{nosep}

% Title formatting
\titleformat{\section}{\Large\bfseries\color{headcolor}}{\thesection}{1em}{}
\titleformat{\subsection}{\large\bfseries\color{headcolor}}{\thesubsection}{1em}{}

% Pandoc tightlist compatibility
\providecommand{\tightlist}{%
  \setlength{\itemsep}{0pt}\setlength{\parskip}{0pt}}

% Pandoc longtable compatibility
\newcounter{none}
\def\thenone{}


% content/resources/templates/english-boxes.tex
% This file is currently empty - it exists to maintain consistency with the import structure.
% Add custom environments here if needed in the future.


\begin{document}

\begin{center}
{\Huge\bfseries\color{headcolor} Modern Physics Solutions}\\[5pt]
{\LARGE DI01000061 -- Winter 2024}\\[3pt]
{\large Semester 1 Study Material}\\[3pt]
{\normalsize\textit{Detailed Solutions and Explanations}}
\end{center}

\vspace{10pt}

\subsection*{Question 1 - Fill in the blanks/MCQs [14
marks]}\label{question-1---fill-in-the-blanksmcqs-14-marks}

\begin{solutionbox}

\begin{longtable}[]{@{}llll@{}}
\toprule\noalign{}
Question & Answer & Question & Answer \\
\midrule\noalign{}
\endhead
\bottomrule\noalign{}
\endlastfoot
(1) & (a) Si & (8) & (b) 0.5 Hz \\
(2) & (a) 1.50 & (9) & (a) 300000 km/s \\
(3) & (b) greater than & (10) & (b) solid \\
(4) & (c) 4 & (11) & (a) crest and trough \\
(5) & (d) Total internal reflection & (12) & (b) monochromatic \\
(6) & (d) frequency & (13) & (a) Single mode \\
(7) & (a) Coulomb & (14) & (b) 45^\circ \\
\end{longtable}

\end{solutionbox}
\begin{mnemonicbox}
``Silicon Glass Bridge Optic Frequency Coulomb Hz
Solid Crest Mono Single 45''

\end{mnemonicbox}
\subsection*{Question 2(A) - Attempt any two [6
marks]}\label{q2a}

\subsubsection{Question 2(A)(1) [3
marks]}\label{question-2a1-3-marks}

\textbf{Differentiate between accuracy and precision.}

\begin{solutionbox}

\begin{longtable}[]{@{}
  >{\raggedright\arraybackslash}p{(\linewidth - 4\tabcolsep) * \real{0.3438}}
  >{\raggedright\arraybackslash}p{(\linewidth - 4\tabcolsep) * \real{0.3125}}
  >{\raggedright\arraybackslash}p{(\linewidth - 4\tabcolsep) * \real{0.3438}}@{}}
\toprule\noalign{}
\begin{minipage}[b]{\linewidth}\raggedright
Parameter
\end{minipage} & \begin{minipage}[b]{\linewidth}\raggedright
Accuracy
\end{minipage} & \begin{minipage}[b]{\linewidth}\raggedright
Precision
\end{minipage} \\
\midrule\noalign{}
\endhead
\bottomrule\noalign{}
\endlastfoot
Definition & Closeness to true value & Consistency of repeated
measurements \\
Focus & Correctness & Reproducibility \\
Error Type & Systematic error & Random error \\
Example & Hitting bullseye & Hitting same spot repeatedly \\
\end{longtable}

\begin{itemize}
\tightlist
\item
  \textbf{Accuracy}: How close measurement is to actual value
\item
  \textbf{Precision}: How close repeated measurements are to each other
\end{itemize}

\end{solutionbox}
\begin{mnemonicbox}
``Accurate Aims Actual, Precise Repeats Reliably''

\end{mnemonicbox}
\subsubsection{Question 2(A)(2) [3
marks]}\label{question-2a2-3-marks}

\textbf{Determine the diameter of a sphere measured by micrometer screw,
main scale reading is 5 mm and 50th division of circular scale is
coinciding with base line. The least count of this instrument is 0.01
mm.}

\begin{solutionbox}

\begin{verbatim}
Given:
Main Scale Reading (MSR) = 5 mm
Circular Scale Reading (CSR) = 50 divisions
Least Count (LC) = 0.01 mm

Formula: Total Reading = MSR + (CSR \times LC)
Total Reading = 5 + (50 \times 0.01)
Total Reading = 5 + 0.5 = 5.5 mm
\end{verbatim}

\textbf{Diameter of sphere = 5.5 mm}

\end{solutionbox}
\begin{mnemonicbox}
``Main Scale Reading + Circular \times Least Count''

\end{mnemonicbox}
\subsubsection{Question 2(A)(3) [3
marks]}\label{question-2a3-3-marks}

\textbf{Calculate the amount of electric charge stored on either plate
of a capacitor of capacitance 4 µF when connected across 12 volt
battery.}

\begin{solutionbox}

\begin{verbatim}
Given:
Capacitance (C) = 4 µF = 4 \times 10^{-}^{6} F
Voltage (V) = 12 V

Formula: Q = CV
Q = 4 \times 10^{-}^{6} \times 12
Q = 48 \times 10^{-}^{6} C
Q = 48 µC
\end{verbatim}

\textbf{Electric charge stored = 48 µC}

\end{solutionbox}
\begin{mnemonicbox}
``Charge equals Capacitance times Voltage''

\end{mnemonicbox}
\subsection*{Question 2(B) - Attempt any two [8
marks]}\label{q2b}

\subsubsection{Question 2(B)(1) [4
marks]}\label{question-2b1-4-marks}

\textbf{Draw a sketch of micrometer screw gauge with proper
nomenclature.}

\begin{solutionbox}

\begin{verbatim}
                     Ratchet
                        |
                        v
    +{-{-}{-}{-}{-}{-}{-}{-}{-}{-}{-}{-}{-}{-}{-}{-}{-}{-}+}
    |                  |
    |   Thimble Scale  |
    |        ||        |
    +{-{-}{-}{-}{-}{-}{-}{-}||{-}{-}{-}{-}{-}{-}{-}{-}+}
             ||
             ||  Spindle
             ||
    +{-{-}{-}{-}{-}{-}{-}{-}||{-}{-}{-}{-}{-}{-}{-}{-}+}
    |    Main Scale    |
    |  0   5   10  15  |
    +{-{-}||{-}{-}{-}{-}{-}{-}{-}{-}{-}{-}||{-}{-}+}
       ||          ||
    Anvil         Frame
\end{verbatim}

\textbf{Main Components}:

\begin{itemize}
\tightlist
\item
  \textbf{Frame}: U-shaped structure providing support
\item
  \textbf{Anvil}: Fixed jaw for placing object
\item
  \textbf{Spindle}: Movable screw mechanism
\item
  \textbf{Thimble Scale}: Circular scale with 50 divisions
\item
  \textbf{Main Scale}: Linear scale in mm
\item
  \textbf{Ratchet}: For consistent pressure application
\end{itemize}

\end{solutionbox}
\begin{mnemonicbox}
``Frame Anvil Spindle Thimble Main Ratchet''

\end{mnemonicbox}
\subsubsection{Question 2(B)(2) [4
marks]}\label{question-2b2-4-marks}

\textbf{Explain the zero, positive and negative errors for vernier
calipers with proper diagram and list necessary steps to remove these
types of errors.}

\begin{solutionbox}

\textbf{Types of Errors}:

\begin{longtable}[]{@{}
  >{\raggedright\arraybackslash}p{(\linewidth - 4\tabcolsep) * \real{0.3750}}
  >{\raggedright\arraybackslash}p{(\linewidth - 4\tabcolsep) * \real{0.3438}}
  >{\raggedright\arraybackslash}p{(\linewidth - 4\tabcolsep) * \real{0.2812}}@{}}
\toprule\noalign{}
\begin{minipage}[b]{\linewidth}\raggedright
Error Type
\end{minipage} & \begin{minipage}[b]{\linewidth}\raggedright
Condition
\end{minipage} & \begin{minipage}[b]{\linewidth}\raggedright
Reading
\end{minipage} \\
\midrule\noalign{}
\endhead
\bottomrule\noalign{}
\endlastfoot
Zero Error & Zero line of vernier doesn't coincide with main scale zero
& Non-zero reading when jaws closed \\
Positive Error & Vernier zero is right of main scale zero & Add
correction \\
Negative Error & Vernier zero is left of main scale zero & Subtract
correction \\
\end{longtable}

\textbf{Diagram}:

\begin{verbatim}
Zero Error:
Main Scale:  |0|1|2|3|4|5|
Vernier:      |0|1|2|3|4|

Positive Error:
Main Scale:  |0|1|2|3|4|5|
Vernier:       |0|1|2|3|4|

Negative Error:
Main Scale:  |0|1|2|3|4|5|
Vernier:     |0|1|2|3|4|
\end{verbatim}

\textbf{Steps to Remove Errors}:

\begin{itemize}
\tightlist
\item
  \textbf{Check zero error} before measurement
\item
  \textbf{Apply correction} to final reading
\item
  \textbf{Clean jaws} regularly to prevent debris
\item
  \textbf{Handle carefully} to avoid mechanical damage
\end{itemize}

\end{solutionbox}
\begin{mnemonicbox}
``Check Clean Correct Carefully''

\end{mnemonicbox}
\subsubsection{Question 2(B)(3) [4
marks]}\label{question-2b3-4-marks}

\textbf{In an experiment of finding the periodic time of a simple
pendulum, the observations are 1.96 s, 1.98 s, 2.00 s, 2.02 s, 2.04 s.
Calculate absolute error, mean absolute error, relative error and
percentage error.}

\begin{solutionbox}

\begin{verbatim}
Observations: 1.96, 1.98, 2.00, 2.02, 2.04 s

Mean value = (1.96 + 1.98 + 2.00 + 2.02 + 2.04) \div 5 = 2.00 s

Absolute errors: |xi - mean|
|1.96 - 2.00| = 0.04 s
|1.98 - 2.00| = 0.02 s
|2.00 - 2.00| = 0.00 s
|2.02 - 2.00| = 0.02 s
|2.04 - 2.00| = 0.04 s

Mean absolute error = (0.04 + 0.02 + 0.00 + 0.02 + 0.04) \div 5 = 0.024 s

Relative error = Mean absolute error \div Mean value = 0.024 \div 2.00 = 0.012

Percentage error = Relative error \times 100 = 0.012 \times 100 = 1.2%
\end{verbatim}

\textbf{Results}: Mean absolute error = 0.024 s, Relative error = 0.012,
Percentage error = 1.2\%

\end{solutionbox}
\begin{mnemonicbox}
``Mean Absolute Relative Percentage''

\end{mnemonicbox}
\subsection*{Question 3(A) - Attempt any two [6
marks]}\label{q3a}

\subsubsection{Question 3(A)(1) [3
marks]}\label{question-3a1-3-marks}

\textbf{Define: Electric flux, Electric field, Potential Difference}

\begin{solutionbox}

\begin{longtable}[]{@{}
  >{\raggedright\arraybackslash}p{(\linewidth - 6\tabcolsep) * \real{0.1818}}
  >{\raggedright\arraybackslash}p{(\linewidth - 6\tabcolsep) * \real{0.3636}}
  >{\raggedright\arraybackslash}p{(\linewidth - 6\tabcolsep) * \real{0.1818}}
  >{\raggedright\arraybackslash}p{(\linewidth - 6\tabcolsep) * \real{0.2727}}@{}}
\toprule\noalign{}
\begin{minipage}[b]{\linewidth}\raggedright
Term
\end{minipage} & \begin{minipage}[b]{\linewidth}\raggedright
Definition
\end{minipage} & \begin{minipage}[b]{\linewidth}\raggedright
Unit
\end{minipage} & \begin{minipage}[b]{\linewidth}\raggedright
Formula
\end{minipage} \\
\midrule\noalign{}
\endhead
\bottomrule\noalign{}
\endlastfoot
Electric Flux & Number of electric field lines passing through a surface
& Nm^{2}/C & Φ = E·A \\
Electric Field & Force per unit positive charge & N/C & E = F/q \\
Potential Difference & Work done per unit charge between two points &
Volt & V = W/q \\
\end{longtable}

\begin{itemize}
\tightlist
\item
  \textbf{Electric flux}: Measure of field lines penetrating surface
\item
  \textbf{Electric field}: Region where electric force acts on charges
\item
  \textbf{Potential difference}: Energy difference per unit charge
\end{itemize}

\end{solutionbox}
\begin{mnemonicbox}
``Flux Field Force, Work Watts Volts''

\end{mnemonicbox}
\subsubsection{Question 3(A)(2) [3
marks]}\label{question-3a2-3-marks}

\textbf{Derive the formula for equivalent capacitance when three
different capacitors are connected in series with necessary circuit
diagram.}

\begin{solutionbox}

\textbf{Circuit Diagram}:

\begin{verbatim}
    +{-{-}{-}{-}||{-}{-}{-}{-}||{-}{-}{-}{-}||{-}{-}{-}{-}+}
    |    C1    C2    C3    |
    |                      |
    +{-{-}{-}{-}{-}{-}{-}{-}{-}{-}V{-}{-}{-}{-}{-}{-}{-}{-}{-}{-}{-}+}
\end{verbatim}

\textbf{Derivation}:

\begin{itemize}
\tightlist
\item
  \textbf{Same charge} Q flows through each capacitor
\item
  \textbf{Voltage divides}: V = V_{1} + V_{2} + V_{3}
\item
  \textbf{For each capacitor}: V_{1} = Q/C_{1}, V_{2} = Q/C_{2}, V_{3} = Q/C_{3}
\item
  \textbf{Total voltage}: V = Q/C_{1} + Q/C_{2} + Q/C_{3} = Q(1/C_{1} + 1/C_{2} + 1/C_{3})
\item
  \textbf{For equivalent}: V = Q/Cs
\item
  \textbf{Therefore}: 1/Cs = 1/C_{1} + 1/C_{2} + 1/C_{3}
\end{itemize}

\textbf{Formula}: \textbf{1/Cs = 1/C_{1} + 1/C_{2} + 1/C_{3}}

\end{solutionbox}
\begin{mnemonicbox}
``Series Sums reciprocals, Same charge Splits
voltage''

\end{mnemonicbox}
\subsubsection{Question 3(A)(3) [3
marks]}\label{question-3a3-3-marks}

\textbf{Define: Infrasonic sound, Audible Sound, Ultrasonic sound}

\begin{solutionbox}

\begin{longtable}[]{@{}
  >{\raggedright\arraybackslash}p{(\linewidth - 6\tabcolsep) * \real{0.2034}}
  >{\raggedright\arraybackslash}p{(\linewidth - 6\tabcolsep) * \real{0.2712}}
  >{\raggedright\arraybackslash}p{(\linewidth - 6\tabcolsep) * \real{0.2881}}
  >{\raggedright\arraybackslash}p{(\linewidth - 6\tabcolsep) * \real{0.2373}}@{}}
\toprule\noalign{}
\begin{minipage}[b]{\linewidth}\raggedright
Sound Type
\end{minipage} & \begin{minipage}[b]{\linewidth}\raggedright
Frequency Range
\end{minipage} & \begin{minipage}[b]{\linewidth}\raggedright
Characteristics
\end{minipage} & \begin{minipage}[b]{\linewidth}\raggedright
Applications
\end{minipage} \\
\midrule\noalign{}
\endhead
\bottomrule\noalign{}
\endlastfoot
Infrasonic & Below 20 Hz & Inaudible to humans & Earthquake detection \\
Audible & 20 Hz to 20 kHz & Audible to humans & Communication, music \\
Ultrasonic & Above 20 kHz & Inaudible to humans & Medical imaging,
SONAR \\
\end{longtable}

\begin{itemize}
\tightlist
\item
  \textbf{Infrasonic}: Low frequency sounds below human hearing
\item
  \textbf{Audible}: Normal hearing range for humans
\item
  \textbf{Ultrasonic}: High frequency sounds above human hearing
\end{itemize}

\end{solutionbox}
\begin{mnemonicbox}
``Infra-Below, Audible-Between, Ultra-Above''

\end{mnemonicbox}
\subsection*{Question 3(B) - Attempt any two [8
marks]}\label{q3b}

\subsubsection{Question 3(B)(1) [4
marks]}\label{question-3b1-4-marks}

\textbf{Prove C = ε_{0}A/d for parallel plate capacitor.}

\begin{solutionbox}

\textbf{Diagram}:

\begin{verbatim}
    +{-{-}{-}{-}{-}{-}{-}{-}+  +{-}{-}{-}{-}{-}{-}{-}{-}+}
    |   +Q   |  |   {-Q   |}
    |        |  |        |
    | Plate1 |  | Plate2 |
    |   A    |  |   A    |
    +{-{-}{-}{-}{-}{-}{-}{-}+  +{-}{-}{-}{-}{-}{-}{-}{-}+}
         {{-}{-}{-}d{-}{-}{-}}
\end{verbatim}

\textbf{Derivation}:

\begin{itemize}
\tightlist
\item
  \textbf{Electric field} between plates: E = σ/ε_{0} = Q/(ε_{0}A)
\item
  \textbf{Potential difference}: V = E \times d = Qd/(ε_{0}A)
\item
  \textbf{Capacitance definition}: C = Q/V
\item
  \textbf{Substituting}: C = Q \div [Qd/(ε_{0}A)] = ε_{0}A/d
\end{itemize}

\textbf{Final Formula}: \textbf{C = ε_{0}A/d}

Where:

\begin{itemize}
\tightlist
\item
  \textbf{ε_{0}}: Permittivity of free space
\item
  \textbf{A}: Area of plates
\item
  \textbf{d}: Distance between plates
\end{itemize}

\end{solutionbox}
\begin{mnemonicbox}
``Capacitance equals epsilon-zero Area over
distance''

\end{mnemonicbox}
\subsubsection{Question 3(B)(2) [4
marks]}\label{question-3b2-4-marks}

\textbf{List the characteristics of electric field lines.}

\begin{solutionbox}

\textbf{Key Characteristics}:

\begin{itemize}
\tightlist
\item
  \textbf{Direction}: From positive to negative charge
\item
  \textbf{Density}: Indicates field strength
\item
  \textbf{Continuous}: Never break in free space
\item
  \textbf{Non-intersecting}: No two lines cross
\item
  \textbf{Perpendicular}: To conductor surface
\item
  \textbf{Closed loops}: Only around changing magnetic fields
\item
  \textbf{Tangent}: Gives field direction at any point
\item
  \textbf{Uniform spacing}: In uniform field regions
\end{itemize}

\textbf{Properties}:

\begin{itemize}
\tightlist
\item
  Start from \textbf{positive charges}
\item
  End at \textbf{negative charges}
\item
  \textbf{Higher density} means stronger field
\item
  \textbf{Never intersect} each other
\end{itemize}

\end{solutionbox}
\begin{mnemonicbox}
``Positive to Negative, Dense means Strong, Never
cross, Always perpendicular''

\end{mnemonicbox}
\subsubsection{Question 3(B)(3) [4
marks]}\label{question-3b3-4-marks}

\textbf{Describe working and construction of magnetostriction method
used for production of ultrasonic waves.}

\begin{solutionbox}

\textbf{Construction}:

\begin{verbatim}
    Oscillator {- Coil {-} Nickel Rod {-} Horn}
                    |       |          |
                   AC    Vibrates   Amplifies
\end{verbatim}

\textbf{Components}:

\begin{itemize}
\tightlist
\item
  \textbf{Nickel rod}: Magnetostrictive material
\item
  \textbf{Coil}: Electromagnet around rod
\item
  \textbf{AC oscillator}: High frequency current source
\item
  \textbf{Horn}: Sound amplifier and transmitter
\end{itemize}

\textbf{Working Principle}:

\begin{itemize}
\tightlist
\item
  \textbf{AC current} flows through coil
\item
  \textbf{Magnetic field} changes rapidly
\item
  \textbf{Nickel rod} expands and contracts
\item
  \textbf{Mechanical vibrations} produced
\item
  \textbf{Ultrasonic waves} generated
\end{itemize}

\textbf{Applications}: Medical imaging, cleaning, welding

\end{solutionbox}
\begin{mnemonicbox}
``AC Coil Makes Nickel vibrate, Creates Ultrasonic''

\end{mnemonicbox}
\subsection*{Question 4(A) - Attempt any two [6
marks]}\label{q4a}

\subsubsection{Question 4(A)(1) [3
marks]}\label{question-4a1-3-marks}

\textbf{A radio station broadcasts its radio signals at 9.26 \times 10^{7} Hz.
Find the wavelength if the waves travel at a speed of 3.00 \times 10^{8} m/s.}

\begin{solutionbox}

\begin{verbatim}
Given:
Frequency (f) = 9.26 \times 10^{7} Hz
Speed (c) = 3.00 \times 10^{8} m/s

Formula: c = fλ
Therefore: λ = c/f

λ = (3.00 \times 10^{8}) \div (9.26 \times 10^{7})
λ = 3.24 m
\end{verbatim}

\textbf{Wavelength = 3.24 m}

\end{solutionbox}
\begin{mnemonicbox}
``Speed equals frequency times wavelength''

\end{mnemonicbox}
\subsubsection{Question 4(A)(2) [3
marks]}\label{question-4a2-3-marks}

\textbf{State the Snell's law and explain refractive index of media.}

\begin{solutionbox}

\textbf{Snell's Law}: n_{1} sin θ_{1} = n_{2} sin θ_{2}

Where:

\begin{itemize}
\tightlist
\item
  \textbf{n_{1}, n_{2}}: Refractive indices of media 1 and 2
\item
  \textbf{θ_{1}, θ_{2}}: Angles of incidence and refraction
\end{itemize}

\textbf{Refractive Index}:

\begin{longtable}[]{@{}lll@{}}
\toprule\noalign{}
Type & Definition & Formula \\
\midrule\noalign{}
\endhead
\bottomrule\noalign{}
\endlastfoot
Absolute & Speed of light in vacuum to medium & n = c/v \\
Relative & Ratio of speeds in two media & n_{2}_{1} = v_{1}/v_{2} \\
\end{longtable}

\begin{itemize}
\tightlist
\item
  \textbf{Higher refractive index}: Denser medium, slower light
\item
  \textbf{Lower refractive index}: Rarer medium, faster light
\end{itemize}

\end{solutionbox}
\begin{mnemonicbox}
``Snell Says Sine ratio constant, Dense slows Down
light''

\end{mnemonicbox}
\subsubsection{Question 4(A)(3) [3
marks]}\label{question-4a3-3-marks}

\textbf{Compare: Ordinary light and LASER}

\begin{solutionbox}

\begin{longtable}[]{@{}lll@{}}
\toprule\noalign{}
Property & Ordinary Light & LASER \\
\midrule\noalign{}
\endhead
\bottomrule\noalign{}
\endlastfoot
Coherence & Incoherent & Coherent \\
Color & Polychromatic & Monochromatic \\
Direction & Divergent & Parallel beam \\
Intensity & Low & Very high \\
Phase & Random & Fixed phase relationship \\
Wavelength & Multiple wavelengths & Single wavelength \\
\end{longtable}

\textbf{Key Differences}:

\begin{itemize}
\tightlist
\item
  \textbf{LASER}: Coherent, monochromatic, parallel, intense
\item
  \textbf{Ordinary}: Incoherent, polychromatic, divergent, less intense
\end{itemize}

\end{solutionbox}
\begin{mnemonicbox}
``LASER: Coherent Monochromatic Parallel Intense''

\end{mnemonicbox}
\subsection*{Question 4(B) - Attempt any two [8
marks]}\label{q4b}

\subsubsection{Question 4(B)(1) [4
marks]}\label{question-4b1-4-marks}

\textbf{Demonstrate the structure of an optical fiber with necessary
diagram.}

\begin{solutionbox}

\textbf{Optical Fiber Structure}:

\begin{verbatim}
    |{{-}{-}{-}{-}{-}{-} Core {-}{-}{-}{-}{-}{-}|}
    |                    |
    +{-{-}{-}{-}{-}{-}{-}{-}{-}{-}{-}{-}{-}{-}{-}{-}{-}{-}{-}{-}+  {-} Cladding}
    |   Higher n_{1        |}
    |                    |  {{-} Lower n_{2}}
    +{-{-}{-}{-}{-}{-}{-}{-}{-}{-}{-}{-}{-}{-}{-}{-}{-}{-}{-}{-}+}
    |                    |
    |   Protective       |  {{-} Jacket}
    |   Coating          |
    +{-{-}{-}{-}{-}{-}{-}{-}{-}{-}{-}{-}{-}{-}{-}{-}{-}{-}{-}{-}+}
\end{verbatim}

\textbf{Components}:

\begin{longtable}[]{@{}llll@{}}
\toprule\noalign{}
Component & Material & Function & Refractive Index \\
\midrule\noalign{}
\endhead
\bottomrule\noalign{}
\endlastfoot
Core & Glass/Plastic & Light transmission & Higher (n_{1}) \\
Cladding & Glass & Total internal reflection & Lower (n_{2}) \\
Jacket & Plastic & Protection & - \\
\end{longtable}

\textbf{Working Principle}:

\begin{itemize}
\tightlist
\item
  Light enters \textbf{core} at acceptance angle
\item
  \textbf{Total internal reflection} at core-cladding boundary
\item
  Light travels in \textbf{zigzag path} through core
\item
  \textbf{n_{1} \textgreater{} n_{2}} ensures light confinement
\end{itemize}

\end{solutionbox}
\begin{mnemonicbox}
``Core Cladding Jacket, Higher Lower Protection''

\end{mnemonicbox}
\subsubsection{Question 4(B)(2) [4
marks]}\label{question-4b2-4-marks}

\textbf{List applications of LASER in engineering and medical field.}

\begin{solutionbox}

\textbf{Engineering Applications}:

\begin{itemize}
\tightlist
\item
  \textbf{Cutting and welding}: Precision metal cutting
\item
  \textbf{3D printing}: Laser sintering
\item
  \textbf{Measurement}: Distance and surveying
\item
  \textbf{Communication}: Optical fiber systems
\item
  \textbf{Material processing}: Surface hardening
\item
  \textbf{Barcode scanning}: Retail and inventory
\end{itemize}

\textbf{Medical Applications}:

\begin{itemize}
\tightlist
\item
  \textbf{Surgery}: Precise tissue cutting
\item
  \textbf{Eye treatment}: Corrective surgery
\item
  \textbf{Cancer treatment}: Tumor destruction
\item
  \textbf{Diagnostics}: Spectroscopy
\item
  \textbf{Dentistry}: Cavity treatment
\item
  \textbf{Skin treatment}: Cosmetic procedures
\end{itemize}

\textbf{Advantages}: \textbf{Precision, non-contact, sterile, minimal
damage}

\end{solutionbox}
\begin{mnemonicbox}
``Engineering: Cut Weld Measure Communicate, Medical:
Surgery Eye Cancer Diagnose''

\end{mnemonicbox}
\subsubsection{Question 4(B)(3) [4
marks]}\label{question-4b3-4-marks}

\textbf{Explain P-type and N-type semiconductors.}

\begin{solutionbox}

\textbf{N-type Semiconductor}:

\begin{longtable}[]{@{}ll@{}}
\toprule\noalign{}
Property & N-type \\
\midrule\noalign{}
\endhead
\bottomrule\noalign{}
\endlastfoot
Dopant & Phosphorus, Arsenic (5 valence electrons) \\
Majority carriers & Electrons \\
Minority carriers & Holes \\
Charge & Negative \\
\end{longtable}

\textbf{P-type Semiconductor}:

\begin{longtable}[]{@{}ll@{}}
\toprule\noalign{}
Property & P-type \\
\midrule\noalign{}
\endhead
\bottomrule\noalign{}
\endlastfoot
Dopant & Boron, Aluminum (3 valence electrons) \\
Majority carriers & Holes \\
Minority carriers & Electrons \\
Charge & Positive \\
\end{longtable}

\textbf{Formation Process}:

\begin{itemize}
\tightlist
\item
  \textbf{N-type}: Pentavalent atoms donate electrons
\item
  \textbf{P-type}: Trivalent atoms accept electrons, create holes
\item
  \textbf{Doping}: Controlled addition of impurities
\item
  \textbf{Conductivity}: Increases due to free carriers
\end{itemize}

\end{solutionbox}
\begin{mnemonicbox}
``N-type Negative electrons, P-type Positive holes''

\end{mnemonicbox}
\subsection*{Question 5(A) - Attempt any two [6
marks]}\label{q5a}

\subsubsection{Question 5(A)(1) [3
marks]}\label{question-5a1-3-marks}

\textbf{Classify conductors, semiconductors and insulators based on
energy band gap.}

\begin{solutionbox}

\begin{longtable}[]{@{}
  >{\raggedright\arraybackslash}p{(\linewidth - 6\tabcolsep) * \real{0.1887}}
  >{\raggedright\arraybackslash}p{(\linewidth - 6\tabcolsep) * \real{0.3019}}
  >{\raggedright\arraybackslash}p{(\linewidth - 6\tabcolsep) * \real{0.3208}}
  >{\raggedright\arraybackslash}p{(\linewidth - 6\tabcolsep) * \real{0.1887}}@{}}
\toprule\noalign{}
\begin{minipage}[b]{\linewidth}\raggedright
Material
\end{minipage} & \begin{minipage}[b]{\linewidth}\raggedright
Energy Band Gap
\end{minipage} & \begin{minipage}[b]{\linewidth}\raggedright
Characteristics
\end{minipage} & \begin{minipage}[b]{\linewidth}\raggedright
Examples
\end{minipage} \\
\midrule\noalign{}
\endhead
\bottomrule\noalign{}
\endlastfoot
Conductor & No gap (0 eV) & Valence and conduction bands overlap &
Copper, Silver \\
Semiconductor & Small gap (1-3 eV) & Moderate band gap & Silicon,
Germanium \\
Insulator & Large gap (\textgreater3 eV) & Wide band gap & Glass,
Rubber \\
\end{longtable}

\textbf{Energy Band Diagram}:

\begin{verbatim}
Conductor    Semiconductor    Insulator
   
   CB            CB              CB
   {-{-}            {-}{-}              {-}{-}}
   VB            VB              VB
   
No Gap        Small Gap       Large Gap
\end{verbatim}

\begin{itemize}
\tightlist
\item
  \textbf{CB}: Conduction Band
\item
  \textbf{VB}: Valence Band
\item
  \textbf{Gap determines} electrical conductivity
\end{itemize}

\end{solutionbox}
\begin{mnemonicbox}
``No gap Conducts, Small gap Semi, Large gap
Insulates''

\end{mnemonicbox}
\subsubsection{Question 5(A)(2) [3
marks]}\label{question-5a2-3-marks}

\textbf{Explain OR and AND logic gates with necessary truth table.}

\begin{solutionbox}

\textbf{OR Gate}:

\begin{longtable}[]{@{}lll@{}}
\toprule\noalign{}
A & B & Y = A + B \\
\midrule\noalign{}
\endhead
\bottomrule\noalign{}
\endlastfoot
0 & 0 & 0 \\
0 & 1 & 1 \\
1 & 0 & 1 \\
1 & 1 & 1 \\
\end{longtable}

\textbf{AND Gate}:

\begin{longtable}[]{@{}lll@{}}
\toprule\noalign{}
A & B & Y = A · B \\
\midrule\noalign{}
\endhead
\bottomrule\noalign{}
\endlastfoot
0 & 0 & 0 \\
0 & 1 & 0 \\
1 & 0 & 0 \\
1 & 1 & 1 \\
\end{longtable}

\textbf{Symbols}:

\begin{verbatim}
OR Gate:     A {-{-}{-}{-}}
                   {{-}{-}{-}{-} Y}
             B {-{-}{-}{-}/}

AND Gate:    A {-{-}{-}{-}}
                   \&{-{-}{-}{-} Y}
             B {-{-}{-}{-}/}
\end{verbatim}

\begin{itemize}
\tightlist
\item
  \textbf{OR}: Output HIGH when any input is HIGH
\item
  \textbf{AND}: Output HIGH when all inputs are HIGH
\end{itemize}

\end{solutionbox}
\begin{mnemonicbox}
``OR: Any high makes high, AND: All high makes high''

\end{mnemonicbox}
\subsubsection{Question 5(A)(3) [3
marks]}\label{question-5a3-3-marks}

\textbf{Describe the use of Zener diode as a voltage regulator.}

\begin{solutionbox}

\textbf{Circuit Diagram}:

\begin{verbatim}
    Vin {-{-}{-}{-}[Rs]{-}{-}{-}{-}+{-}{-}{-}{-}Vout}
                     |
                   [Zener]
                     |
                    GND
\end{verbatim}

\textbf{Working Principle}:

\begin{itemize}
\tightlist
\item
  \textbf{Forward bias}: Acts like normal diode
\item
  \textbf{Reverse bias}: Breaks down at Zener voltage
\item
  \textbf{Voltage regulation}: Maintains constant Vout = Vz
\item
  \textbf{Series resistor}: Limits current through Zener
\end{itemize}

\textbf{Characteristics}:

\begin{itemize}
\tightlist
\item
  \textbf{Zener voltage}: Constant breakdown voltage
\item
  \textbf{Current range}: Wide operating range
\item
  \textbf{Temperature stability}: Good voltage stability
\item
  \textbf{Power rating}: Must not exceed maximum power
\end{itemize}

\textbf{Applications}: Power supplies, voltage references, protection
circuits

\end{solutionbox}
\begin{mnemonicbox}
``Zener Zealously maintains Voltage despite
Variations''

\end{mnemonicbox}
\subsection*{Question 5(B) - Attempt any two [8
marks]}\label{q5b}

\subsubsection{Question 5(B)(1) [4
marks]}\label{question-5b1-4-marks}

\textbf{Explain full wave rectifier with necessary circuit and draw
input and output waveforms.}

\begin{solutionbox}

\textbf{Center-tap Full Wave Rectifier}:

\begin{verbatim}
    AC Input {-{-}{-}{-}+{-}{-}{-}{-}[D1]{-}{-}{-}{-}+{-}{-}{-}{-} Positive Output}
                 |             |
            Transformer     Load (RL)
                 |             |
                 +{-{-}{-}{-}[D2]{-}{-}{-}{-}+{-}{-}{-}{-} Common}
\end{verbatim}

\textbf{Working}:

\begin{itemize}
\tightlist
\item
  \textbf{Positive half cycle}: D1 conducts, D2 off
\item
  \textbf{Negative half cycle}: D2 conducts, D1 off
\item
  \textbf{Both halves}: Current flows through load in same direction
\end{itemize}

\textbf{Waveforms}:

\begin{verbatim}
Input:     /{  /  /  /}
          /  {/  /  /  }
         /                {}

Output:   /{    /    /}
         /  {  /    /  }
        /    {/    /    }
\end{verbatim}

\textbf{Advantages}: Better efficiency, lower ripple, better transformer
utilization

\end{solutionbox}
\begin{mnemonicbox}
``Full wave uses Full cycle, Better efficiency Better
output''

\end{mnemonicbox}
\subsubsection{Question 5(B)(2) [4
marks]}\label{question-5b2-4-marks}

\textbf{Demonstrate forward and reverse characteristics of P-N junction
diode.}

\begin{solutionbox}

\textbf{Forward Bias Characteristics}:

\begin{longtable}[]{@{}lll@{}}
\toprule\noalign{}
Voltage Range & Current & Behavior \\
\midrule\noalign{}
\endhead
\bottomrule\noalign{}
\endlastfoot
0 to 0.3V (Si) & Very small & Cut-in voltage \\
Above 0.7V & Exponential increase & Conducting \\
\end{longtable}

\textbf{Reverse Bias Characteristics}:

\begin{longtable}[]{@{}lll@{}}
\toprule\noalign{}
Voltage Range & Current & Behavior \\
\midrule\noalign{}
\endhead
\bottomrule\noalign{}
\endlastfoot
0 to breakdown & Reverse saturation & Leakage current \\
Breakdown voltage & Sharp increase & Avalanche breakdown \\
\end{longtable}

\textbf{I-V Characteristic Curve}:

\begin{verbatim}
       I(mA)
        |
        |    Forward
       /|     bias
      / |
     /  |
{-{-}{-}{-}+{-}{-}{-}+{-}{-}{-}{-}V(V)}
  {-20   0.7  }
    |
    |Reverse
    |bias
\end{verbatim}

\textbf{Key Points}:

\begin{itemize}
\tightlist
\item
  \textbf{Forward}: Low resistance, high current
\item
  \textbf{Reverse}: High resistance, low current
\item
  \textbf{Cut-in voltage}: 0.7V for Silicon, 0.3V for Germanium
\end{itemize}

\end{solutionbox}
\begin{mnemonicbox}
``Forward Flow, Reverse Resist''

\end{mnemonicbox}
\subsubsection{Question 5(B)(3) [4
marks]}\label{question-5b3-4-marks}

\textbf{Write the principle of LED and explain its construction and
working.}

\begin{solutionbox}

\textbf{Principle}: \textbf{Electroluminescence} - Direct conversion of
electrical energy to light energy

\textbf{Construction}:

\begin{verbatim}
    Light Output
        ↑
    +{-{-}{-}{-}{-}{-}{-}+}
    | P{-type|  {-} Anode}
    +{-{-}{-}{-}{-}{-}{-}+}
    |Junction|
    +{-{-}{-}{-}{-}{-}{-}+}
    | N{-type|  {-} Cathode}
    +{-{-}{-}{-}{-}{-}{-}+}
\end{verbatim}

\textbf{Materials Used}:

\begin{longtable}[]{@{}lll@{}}
\toprule\noalign{}
Color & Material & Wavelength \\
\midrule\noalign{}
\endhead
\bottomrule\noalign{}
\endlastfoot
Red & GaAs & 700 nm \\
Green & GaP & 550 nm \\
Blue & GaN & 470 nm \\
\end{longtable}

\textbf{Working}:

\begin{itemize}
\tightlist
\item
  \textbf{Forward bias}: Electrons and holes recombine at junction
\item
  \textbf{Energy release}: Photons emitted during recombination
\item
  \textbf{Light color}: Depends on band gap energy
\item
  \textbf{Efficiency}: High electrical to optical conversion
\end{itemize}

\textbf{Applications}: Displays, indicators, lighting, optical
communication

\end{solutionbox}
\begin{mnemonicbox}
``LED: Light Emitting Diode, Electrons and holes
Dance to make Light''

\end{mnemonicbox}

\end{document}
