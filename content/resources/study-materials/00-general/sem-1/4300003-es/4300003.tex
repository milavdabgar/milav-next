\documentclass[10pt,a4paper]{article}

% content/resources/templates/preamble.tex
\usepackage[margin=0.6in]{geometry}
\author{Milav Dabgar}
\usepackage{amsmath,amssymb,amsthm}
\usepackage{booktabs}
\usepackage{multirow}
\usepackage{xcolor}
\usepackage{tcolorbox}
\tcbuselibrary{breakable,skins}
\usepackage[colorlinks=true,linkcolor=blue]{hyperref}
\usepackage{titlesec}
\usepackage{enumitem}
\usepackage{tikz}
\usepackage{pgfplots}
\usepackage{circuitikz}
\usepackage[version=4]{mhchem}
\usepackage{longtable}
\usepackage{array}
\usepackage{float}
\usepackage{caption}
\usepackage{listings}

\lstset{
  basicstyle=\small\ttfamily,
  breaklines=true,
  breakatwhitespace=false,
  postbreak=\mbox{\textcolor{red}{$\hookrightarrow$}\space},
  float=false,
  numbers=left,
  numberstyle=\tiny\color{gray},
  numbersep=10pt,
  xleftmargin=2em,
  keywordstyle=\color{blue},
  commentstyle=\color{green!60!black},
  stringstyle=\color{purple},
  backgroundcolor=\color{gray!5},
  showstringspaces=false,
  tabsize=2,
  captionpos=b,
  keepspaces=true,
  columns=flexible
}

\pgfplotsset{compat=1.18}
\usetikzlibrary{shapes,arrows,positioning,calc,patterns,decorations.pathmorphing,decorations.markings,arrows.meta}

% Color scheme
\definecolor{headcolor}{RGB}{0,102,204}
\definecolor{keycolor}{RGB}{220,20,60}
\definecolor{solutioncolor}{RGB}{34,139,34}
\definecolor{mnemoniccolor}{RGB}{148,0,211}
\definecolor{codecolor}{RGB}{0,0,100}

% Spacing
\setlength{\parskip}{3pt}
\setlist[itemize]{nosep}
\setlist[enumerate]{nosep}

% Title formatting
\titleformat{\section}{\Large\bfseries\color{headcolor}}{\thesection}{1em}{}
\titleformat{\subsection}{\large\bfseries\color{headcolor}}{\thesubsection}{1em}{}

% Pandoc tightlist compatibility
\providecommand{\tightlist}{%
  \setlength{\itemsep}{0pt}\setlength{\parskip}{0pt}}

% Pandoc longtable compatibility
\newcounter{none}
\def\thenone{}


% content/resources/templates/english-boxes.tex
% This file is currently empty - it exists to maintain consistency with the import structure.
% Add custom environments here if needed in the future.


\begin{document}

\begin{center}
{\Huge\bfseries\color{headcolor} Environment and Sustainability Solutions}\\[5pt]
{\LARGE 4300003 -- Study Material}\\[3pt]
{\large Semester 1 Study Material}\\[3pt]
{\normalsize\textit{Detailed Solutions and Explanations}}
\end{center}

\vspace{10pt}

\subsection*{GUJARAT TECHNOLOGICAL UNIVERSITY
(GTU)}\label{gujarat-technological-university-gtu}

\subsection*{Competency-focused Outcome-based Green Curriculum-2021
(COGC-2021)}\label{competency-focused-outcome-based-green-curriculum-2021-cogc-2021}

\textbf{Semester - I}

\subsection*{Course Title: Environment and Sustainability (Course Code:
4300003)}\label{course-title-environment-and-sustainability-course-code-4300003}

{\def\LTcaptype{none} % do not increment counter
\begin{longtable}[]{@{}
  >{\raggedright\arraybackslash}p{(\linewidth - 2\tabcolsep) * \real{0.7059}}
  >{\raggedright\arraybackslash}p{(\linewidth - 2\tabcolsep) * \real{0.2941}}@{}}
\toprule\noalign{}
\begin{minipage}[b]{\linewidth}\raggedright
Diploma programme in which this course is offered
\end{minipage} & \begin{minipage}[b]{\linewidth}\raggedright
Semester in which offered
\end{minipage} \\
\midrule\noalign{}
\endhead
\bottomrule\noalign{}
\endlastfoot
Chemical, Mechatronics, Computer & First \\
Civil, Environment, Mining, Architectural Assistantship, Mechanical,
Automobile, Marine, Metallurgy, Fabrication, Electrical, Electronics and
Communication, Instrumentation and Control, Bio Medical, Power
Electronics, IT, Textile Manufacturing, Textile Processing, Textile
Design, Printing, Plastics, Ceramics, CACDDM , Computer Science and
Engineering. & Second \\
\end{longtable}
}

\subsection*{1. RATIONALE}\label{rationale}

For a country to progress, sustainable development is one of the key
factors. Environment conservation and hazard management is of much
importance to every citizen of India. Considerable amount of energy is
being wasted. Energy saved is energy produced. Environmental pollution
is on the rise due to rampant industrial mismanagement and indiscipline.
Renewable energy is one of the answers to the energy crisis and also to
reduce environmental pollution. Therefore this course has been designed
to develop a general awareness of these and related issues so that the
every student will start acting as a responsible citizen to make the
country and the world a better place to live in.

\subsection*{2. COMPETENCY}\label{competency}

The purpose of this course is to help the student to attain the
following industry identified competency through various teaching
learning experiences:

\begin{itemize}
\tightlist
\item
  Adopt the sustainable practices to resolve the environment related
  issues.
\end{itemize}

\subsection*{3. COURSE OUTCOMES (Cos)}\label{course-outcomes-cos}

The practical exercises, the underpinning knowledge and the relevant
soft skills associated with this competency are to be developed in the
student to display the following COs:

\begin{itemize}
\tightlist
\item
  \begin{enumerate}
  \def\labelenumi{\alph{enumi})}
  \tightlist
  \item
    Adopt relevant ecofriendly product in the given situation to protect
    ecosystem
  \end{enumerate}
\item
  \begin{enumerate}
  \def\labelenumi{\alph{enumi})}
  \setcounter{enumi}{1}
  \tightlist
  \item
    use relevant method of pollution reduction in the given situation
  \end{enumerate}
\item
  \begin{enumerate}
  \def\labelenumi{\alph{enumi})}
  \setcounter{enumi}{2}
  \tightlist
  \item
    Use of renewable resources of energy for sustainable development
  \end{enumerate}
\item
  \begin{enumerate}
  \def\labelenumi{\alph{enumi})}
  \setcounter{enumi}{3}
  \tightlist
  \item
    Use the relevant techniques in given context to reduce impact due to
    climate change Use relevant laws and policies for developing the
    sustainable environmental development
  \end{enumerate}
\end{itemize}

\subsection*{4. TEACHING AND EXAMINATION
SCHEME}\label{teaching-and-examination-scheme}

{\def\LTcaptype{none} % do not increment counter
\begin{longtable}[]{@{}
  >{\raggedright\arraybackslash}p{(\linewidth - 16\tabcolsep) * \real{0.1014}}
  >{\raggedright\arraybackslash}p{(\linewidth - 16\tabcolsep) * \real{0.1014}}
  >{\raggedright\arraybackslash}p{(\linewidth - 16\tabcolsep) * \real{0.1014}}
  >{\raggedright\arraybackslash}p{(\linewidth - 16\tabcolsep) * \real{0.0878}}
  >{\raggedright\arraybackslash}p{(\linewidth - 16\tabcolsep) * \real{0.1216}}
  >{\raggedright\arraybackslash}p{(\linewidth - 16\tabcolsep) * \real{0.1216}}
  >{\raggedright\arraybackslash}p{(\linewidth - 16\tabcolsep) * \real{0.1216}}
  >{\raggedright\arraybackslash}p{(\linewidth - 16\tabcolsep) * \real{0.1216}}
  >{\raggedright\arraybackslash}p{(\linewidth - 16\tabcolsep) * \real{0.1216}}@{}}
\toprule\noalign{}
\begin{minipage}[b]{\linewidth}\raggedright
Teaching Scheme
\end{minipage} & \begin{minipage}[b]{\linewidth}\raggedright
Teaching Scheme
\end{minipage} & \begin{minipage}[b]{\linewidth}\raggedright
Teaching Scheme
\end{minipage} & \begin{minipage}[b]{\linewidth}\raggedright
Total Credits
\end{minipage} & \begin{minipage}[b]{\linewidth}\raggedright
Examination Scheme
\end{minipage} & \begin{minipage}[b]{\linewidth}\raggedright
Examination Scheme
\end{minipage} & \begin{minipage}[b]{\linewidth}\raggedright
Examination Scheme
\end{minipage} & \begin{minipage}[b]{\linewidth}\raggedright
Examination Scheme
\end{minipage} & \begin{minipage}[b]{\linewidth}\raggedright
Examination Scheme
\end{minipage} \\
\midrule\noalign{}
\endhead
\bottomrule\noalign{}
\endlastfoot
(In Hours) & (In Hours) & (In Hours) & (L+T/2+P/2) & Theory Marks &
Theory Marks & Practical Marks & Practical Marks & Total \\
L & T & P & C & CA & ESE & CA & ESE & Marks \\
3 & 0 & 0 & 3 & 30* & 70 & 0 & 0 & 100 \\
\end{longtable}
}

\begin{itemize}
\tightlist
\item
  (*): Out of 30 marks under the theory CA, 10 marks are for assessment
  of the microproject to facilitate integration of COs and the remaining
  20 marks is the average of 2 tests to be taken during the semester for
  the assessing the attainment of the cognitive domain UOs required for
  the attainment of the COs .
\end{itemize}

Legends: L -Lecture; T - Tutorial/Teacher Guided Theory Practice; P
-Practical; C -Credit, CA - Continuous Assessment; ESE -End Semester
Examination.

\subsection*{5. SUGGESTED PRACTICAL EXERCISES - Not
Applicable}\label{suggested-practical-exercises---not-applicable}

The following practical outcomes (PrOs) that are the sub-components of
the COs. Some of the PrOs marked '*' are compulsory, as they are crucial
for that particular CO at the `Precision Level' of Dave's Taxonomy
related to `Psychomotor Domain' .

\subsection*{Note}\label{note}

\begin{itemize}
\tightlist
\item
  \begin{enumerate}
  \def\labelenumi{\roman{enumi}.}
  \tightlist
  \item
    More Practical Exercises can be designed and offered by the
    respective course teacher to develop the industry relevant
    skills/outcomes to match the COs. The above table is only a
    suggestive list .
  \end{enumerate}
\item
  \begin{enumerate}
  \def\labelenumi{\roman{enumi}.}
  \setcounter{enumi}{1}
  \tightlist
  \item
    The following are some sample `Process' and `Product' related skills
    (more may be added/deleted depending on the course) that occur in
    the above listed Practical Exercises of this course required which
    are embedded in the COs and ultimately the competency..
  \end{enumerate}
\end{itemize}

{\def\LTcaptype{none} % do not increment counter
\begin{longtable}[]{@{}
  >{\raggedright\arraybackslash}p{(\linewidth - 4\tabcolsep) * \real{0.0952}}
  >{\raggedright\arraybackslash}p{(\linewidth - 4\tabcolsep) * \real{0.6825}}
  >{\raggedright\arraybackslash}p{(\linewidth - 4\tabcolsep) * \real{0.2222}}@{}}
\toprule\noalign{}
\begin{minipage}[b]{\linewidth}\raggedright
S. No.
\end{minipage} & \begin{minipage}[b]{\linewidth}\raggedright
Sample Performance Indicators for the PrOs
\end{minipage} & \begin{minipage}[b]{\linewidth}\raggedright
Weightage in \%
\end{minipage} \\
\midrule\noalign{}
\endhead
\bottomrule\noalign{}
\endlastfoot
1 & Prepare of experimental setup & 20 \\
2 & Operate the equipment setup or circuit & 20 \\
3 & Follow safe practices measures & 10 \\
4 & Record observations correctly & 20 \\
5 & Interpret the result and conclude & 30 \\
Total & Total & 100 \\
\end{longtable}
}

\subsection*{6. MAJOR EQUIPMENT/ INSTRUMENTS REQUIRED - (Not
Applicable)}\label{major-equipment-instruments-required---not-applicable}

These major equipment with broad specifications for the PrOs is a guide
to procure them by the administrators to usher in uniformity of
practicals in all institutions across the state.

\subsection*{7. AFFECTIVE DOMAIN
OUTCOMES}\label{affective-domain-outcomes}

The following sample Affective Domain Outcomes (ADOs) are embedded in
many of the above mentioned COs and PrOs. More could be added to fulfil
the development of this competency.

\begin{itemize}
\tightlist
\item
  \begin{enumerate}
  \def\labelenumi{\alph{enumi})}
  \tightlist
  \item
    Work as a leader/a team member.
  \end{enumerate}
\item
  \begin{enumerate}
  \def\labelenumi{\alph{enumi})}
  \setcounter{enumi}{1}
  \tightlist
  \item
    Follow ethical practices.
  \end{enumerate}
\item
  \begin{enumerate}
  \def\labelenumi{\alph{enumi})}
  \setcounter{enumi}{2}
  \tightlist
  \item
    Practice environmental friendly methods and processes. (Environment
    related)
  \end{enumerate}
\end{itemize}

The ADOs are best developed through the laboratory/field based
exercises. Moreover, the level of achievement of the ADOs according to
Krathwohl's `Affective Domain Taxonomy' should gradually increase as
planned below:

\begin{itemize}
\tightlist
\item
  \begin{enumerate}
  \def\labelenumi{\roman{enumi}.}
  \tightlist
  \item
    `Valuing Level' in 1 st year
  \end{enumerate}
\item
  \begin{enumerate}
  \def\labelenumi{\roman{enumi}.}
  \setcounter{enumi}{1}
  \tightlist
  \item
    `Organization Level' in 2 nd year.
  \end{enumerate}
\item
  \begin{enumerate}
  \def\labelenumi{\roman{enumi}.}
  \setcounter{enumi}{2}
  \tightlist
  \item
    `Characterization Level' in 3 rd year.
  \end{enumerate}
\end{itemize}

\subsection*{8. UNDERPINNING THEORY}\label{underpinning-theory}

Only the major Underpinning Theory is formulated as higher level UOs of
Revised Bloom's taxonomy in order development of the COs and competency
is not missed out by the students and teachers. If required, more such
higher level UOs could be included by the course teacher to focus on
attainment of COs and competency.

{\def\LTcaptype{none} % do not increment counter
\begin{longtable}[]{@{}
  >{\raggedright\arraybackslash}p{(\linewidth - 4\tabcolsep) * \real{0.3333}}
  >{\raggedright\arraybackslash}p{(\linewidth - 4\tabcolsep) * \real{0.3333}}
  >{\raggedright\arraybackslash}p{(\linewidth - 4\tabcolsep) * \real{0.3333}}@{}}
\toprule\noalign{}
\begin{minipage}[b]{\linewidth}\raggedright
Unit
\end{minipage} & \begin{minipage}[b]{\linewidth}\raggedright
Unit Outcomes (UOs) (4 to 6 UOs at Application and above level)
\end{minipage} & \begin{minipage}[b]{\linewidth}\raggedright
Topics and Sub-topics
\end{minipage} \\
\midrule\noalign{}
\endhead
\bottomrule\noalign{}
\endlastfoot
Unit - IEcosystem & 1a. Explain the Structure with components of the
given Ecosystem 1b. Explain Carbon, Nitrogen, Sulphur and phosphorus
cycle for the given ecosystem. 1c. Justify the need to conserve the
given Ecosystem on the w.r.t. following points: carrying capacity of
earth Biomes, Ecologically sensitive area 1d. Explain the term
biodiversity with its importance. 1e. Illustrate the importance of IUCN
red list in environmental engineering. 1f. Calculate global ecological
overshoot and virtual water requirement of given natural and man-made
materials. & 1.1 Structure and components of ecosystem1.2 Types of
Ecosystem, changes in ecosystem1.3 Various natural cycles like carbon,
Nitrogen, Sulphur, Phosphorus 1.4 Ecosystem conservation, carrying
capacity of earth, Biomes in India, (ESA) Ecologically sensitive areas
1.5 Bio diversity, its need and importance, International Union for
Conservation of Nature (IUCN) red list 1.6 Concept of Ecological foot
print, virtual water, global ecological overshoot \\
Unit - IIPollution and its types & 2a. Explain the term, `pollution and
pollutant' in the given situation. 2b. Classify the air pollution on the
basis of its source 2c. Use relevant equipment to control given type of
air pollution. 2d. Explain relevant techniques of treatment to deal with
given type of water pollution. 2e. Apply relevant techniques of Solid
waste management based on its characteristics. 2f. Explain drawbacks of
noise pollution in given situation. 2g. Describe the environmental
degradation due to Plastic waste and E- waste & 2.1. Definition of
pollution and pollutant 2.2. Air pollution, classification and its
sources 2.3. Air pollution control Equipments 2.4. Water pollution,
pollution parameters like BOD,COD, pH, Total suspended solids,
Turbidity, Total Solids 2.5. Waste water treatment like primary,
secondary and tertiary 2.6. Solid waste generation, sources and
characteristics of Muncipal solid waste 2.7. Collection and disposal of
Muncipal waste and Hazardous waste 2.8. Noise pollution- its effects,
sources and measurement 2.9. Plastic waste and its hazard 2.10. E waste
and its hazard \\
Unit- IIIRenewable sources of energy & 3a. Justify the need of renewable
energy adopting relevant energy policy in given situation. 3b. Explain
the working of the solar thermal and PV systems with sketch in given
situation. 3c. Justify the need of Advanced collector, Solar Pond, Solar
water heater, Solar dryer in the given system. 3d. Emphasize the
importance of wind power in India 3e. Select the relevant type of wind
turbines in the given situation. 3f. Identify the relevant types of
Sources of biomass energy. 3g. Draw the neat labelled diagram of simple
biogas plant to explain its working.3h. Identify the sources of the
energy generation for the given situation. & 3.1 Need of Renewable
energy and energy policy 3.2 Solar energy: National solar mission 3.3
Features of solar thermal and PV systems Advanced collector, Solar Pond,
Solar water heater, Solar dryer, polycrystalline, monocrystalline and
thin film PV systems 3.4 Wind Energy: Growth of wind power in India 3.5
Types of wind turbines - Vertical axis wind turbines (VAWT) and
horizontal axis wind turbines (HAWT) 3.6 Types of HAWTs - drag and lift
types 3.7 Biomass: Overview of biomass as energy source. Thermal
characteristics of biomass as fuel 3.8 Anaerobic digestion, Biogas
production mechanism, utilization and storage. 3.9 New energy sources:
Geothermal energy, Ocean energy sources, Tidal energy conversion,
Hydrogen energy \\
Unit- IV Climate Change & 4a. Explain the term, `climate change' in
context of environment. 4b. Describe the ill effects of Global warming
due to various causes arising in the given situation. 4c. Explain the
term, `greenhouse effect' with its causes. 4d. Relate the impact of
Ozone depletion in climate change due to its causes. 4e. Justify the
need of relevant Climate change management system to reduce the impact
of climate change in the given context. & 4.1 Identify Factors affecting
climate change in given locality. 4.2 Definition of climate change 4.3
Global warming-causes, effect, process 4.4 Greenhouse effect 4.5 Ozone
depletion 4.6 Factors affecting climate change 4.7 Impact and mitigation
4.8 Climate change management \\
Unit- VEnvironme ntal legislation and sustainable practices & 5.a Use
relevant policy or law in relation with environment in given situation
5.b Relate the relevant provision of given act in given situation. 5.c
Explain the necessity of the Environmental management system in given
situation. 5.d Use the principle of Rain water harvesting in the given
situation. 5.e Justify the necessity of Green building in India. 5.f.
Adopt the relevant rating system for energy calculation for the given
building. 5.f Explain the terms, `Cradle to cradle concept' and `Life
cycle analysis' 5.g Emphasize the importance of Carbon credit system in
India. 5.h Explain the importance of 5R concept. & 5.1 Environmental
policies in India 5.2 Air act, water act, Environment protection act,
wild life protection act, Forest conservation act, Biodiversity act 5.3
Environmental management system: ISO 14000, definition and benefits 5.4
Rain water harvesting 5.5 Green building and rating system in India 5.6
Cradle to cradle concept and Life cycle analysis 5.7 Green label 5.8
Carbon credit system its advantages and disadvantages 5.9 Concept of
5R(Refuse, Reduce, Reuse, Repurpose, Recycle) 5.10 Eco tourism:
advantages and disadvantages \\
\end{longtable}
}

Note : The UOs need to be formulated at the `Application Level' and
above of Revised Bloom's Taxonomy' to accelerate the attainment of the
COs and the competency.

\subsection*{9. SUGGESTED SPECIFICATION TABLE FOR QUESTION PAPER
DESIGN}\label{suggested-specification-table-for-question-paper-design}

{\def\LTcaptype{none} % do not increment counter
\begin{longtable}[]{@{}
  >{\raggedright\arraybackslash}p{(\linewidth - 12\tabcolsep) * \real{0.0469}}
  >{\raggedright\arraybackslash}p{(\linewidth - 12\tabcolsep) * \real{0.2708}}
  >{\raggedright\arraybackslash}p{(\linewidth - 12\tabcolsep) * \real{0.0781}}
  >{\raggedright\arraybackslash}p{(\linewidth - 12\tabcolsep) * \real{0.1510}}
  >{\raggedright\arraybackslash}p{(\linewidth - 12\tabcolsep) * \real{0.1510}}
  >{\raggedright\arraybackslash}p{(\linewidth - 12\tabcolsep) * \real{0.1510}}
  >{\raggedright\arraybackslash}p{(\linewidth - 12\tabcolsep) * \real{0.1510}}@{}}
\toprule\noalign{}
\begin{minipage}[b]{\linewidth}\raggedright
Unit No.
\end{minipage} & \begin{minipage}[b]{\linewidth}\raggedright
Unit Title
\end{minipage} & \begin{minipage}[b]{\linewidth}\raggedright
Teaching Hours
\end{minipage} & \begin{minipage}[b]{\linewidth}\raggedright
Distribution of Theory Marks
\end{minipage} & \begin{minipage}[b]{\linewidth}\raggedright
Distribution of Theory Marks
\end{minipage} & \begin{minipage}[b]{\linewidth}\raggedright
Distribution of Theory Marks
\end{minipage} & \begin{minipage}[b]{\linewidth}\raggedright
Distribution of Theory Marks
\end{minipage} \\
\midrule\noalign{}
\endhead
\bottomrule\noalign{}
\endlastfoot
Unit No. & Unit Title & Teaching Hours & R Level & U Level & A & Total
Marks \\
I & Ecosystem & 08 & 6 & 6 & 2 & 14 \\
II & Pollution and its types & 10 & 4 & 6 & 6 & 16 \\
III & Renewable sources of energy & 10 & 4 & 6 & 6 & 16 \\
IV & Climate Change & 08 & 4 & 6 & 4 & 14 \\
V & Environmental legislation and sustainable practices & 06 & 5 & 3 & 2
& 10 \\
Total & Total & 42 & 12 & 28 & 30 & 70 \\
\end{longtable}
}

Legends: R=Remember, U=Understand, A=Apply and above (Revised Bloom's
taxonomy) Note : This specification table provides general guidelines to
assist student for their learning and to teachers to teach and question
paper designers/setters to formulate test items/questions assess the
attainment of the UOs. The actual distribution of marks at different
taxonomy levels (of R, U and A) in the question paper may vary slightly
from above table.

\subsection*{10. SUGGESTED STUDENT
ACTIVITIES}\label{suggested-student-activities}

Other than the classroom and laboratory learning, following are the
suggested student-related co-curricular activities which can be
undertaken to accelerate the attainment of the various outcomes in this
course: Students should conduct following activities in group and
prepare reports of about 5 pages for each activity, also collect/record
physical evidences for their (student's) portfolio which will be useful
for their placement interviews:

\begin{itemize}
\tightlist
\item
  \begin{enumerate}
  \def\labelenumi{\alph{enumi})}
  \tightlist
  \item
    Prepare specification of some renewable sources of energy.
  \end{enumerate}
\item
  \begin{enumerate}
  \def\labelenumi{\alph{enumi})}
  \setcounter{enumi}{1}
  \tightlist
  \item
    Undertake micro-projects in teams
  \end{enumerate}
\item
  \begin{enumerate}
  \def\labelenumi{\alph{enumi})}
  \setcounter{enumi}{2}
  \tightlist
  \item
    Give seminar on any relevant topic.
  \end{enumerate}
\item
  \begin{enumerate}
  \def\labelenumi{\alph{enumi})}
  \setcounter{enumi}{3}
  \tightlist
  \item
    Undertake a market survey of different green materials.
  \end{enumerate}
\item
  \begin{enumerate}
  \def\labelenumi{\alph{enumi})}
  \setcounter{enumi}{4}
  \tightlist
  \item
    Prepare showcase portfolios.
  \end{enumerate}
\item
  \begin{enumerate}
  \def\labelenumi{\alph{enumi})}
  \setcounter{enumi}{5}
  \tightlist
  \item
    Prepare report on various issues related to environment and
    sustainable development
  \end{enumerate}
\item
  \begin{enumerate}
  \def\labelenumi{\alph{enumi})}
  \setcounter{enumi}{6}
  \tightlist
  \item
    Publish a research paper on themes related to environment and
    sustainable development.
  \end{enumerate}
\item
  \begin{enumerate}
  \def\labelenumi{\alph{enumi})}
  \setcounter{enumi}{7}
  \tightlist
  \item
    Compare the pollution (water, air and noise) data of various cities
    with standard values as laid by pollution control board.
  \end{enumerate}
\item
  \begin{enumerate}
  \def\labelenumi{\roman{enumi})}
  \tightlist
  \item
    Undertake some small mini projects on various issues related to
    environment and sustainable development.
  \end{enumerate}
\item
  \begin{enumerate}
  \def\labelenumi{\alph{enumi})}
  \setcounter{enumi}{9}
  \tightlist
  \item
    Submit a report on visit to an energy park
  \end{enumerate}
\item
  \begin{enumerate}
  \def\labelenumi{\alph{enumi})}
  \setcounter{enumi}{10}
  \tightlist
  \item
    Prepare power point on clean and green technologies
  \end{enumerate}
\item
  \begin{enumerate}
  \def\labelenumi{\alph{enumi})}
  \setcounter{enumi}{11}
  \tightlist
  \item
    Submit a report on visit to garbage disposal system in your
    city/town.
  \end{enumerate}
\item
  \begin{enumerate}
  \def\labelenumi{\alph{enumi})}
  \setcounter{enumi}{12}
  \tightlist
  \item
    Submit a report on analysis of the life cycle of any one or two
    eco-friendly product/s.
  \end{enumerate}
\item
  \begin{enumerate}
  \def\labelenumi{\alph{enumi})}
  \setcounter{enumi}{13}
  \tightlist
  \item
    Calculate ecological footprint using various calculator available on
    web with a report recommending ways and means to reduce ecological
    footprint.
  \end{enumerate}
\item
  \begin{enumerate}
  \def\labelenumi{\alph{enumi})}
  \setcounter{enumi}{14}
  \tightlist
  \item
    Give seminar on relevant topic.
  \end{enumerate}
\item
  \begin{enumerate}
  \def\labelenumi{\alph{enumi})}
  \setcounter{enumi}{15}
  \tightlist
  \item
    Undertake micro-projects.
  \end{enumerate}
\end{itemize}

\subsection*{11. SUGGESTED SPECIAL INSTRUCTIONAL STRATEGIES (if
any)}\label{suggested-special-instructional-strategies-if-any}

These are sample strategies, which the teacher can use to accelerate the
attainment of the various outcomes in this course:

\begin{itemize}
\tightlist
\item
  \begin{enumerate}
  \def\labelenumi{\alph{enumi})}
  \tightlist
  \item
    Massive open online courses ( MOOCs ) may be used to teach various
    topics/sub topics.
  \end{enumerate}
\item
  \begin{enumerate}
  \def\labelenumi{\alph{enumi})}
  \setcounter{enumi}{1}
  \tightlist
  \item
    Guide student(s) in undertaking micro-projects.
  \end{enumerate}
\item
  \begin{enumerate}
  \def\labelenumi{\alph{enumi})}
  \setcounter{enumi}{2}
  \tightlist
  \item
    `L' in section No.~4 means different types of teaching methods that
    are to be employed by teachers to develop the outcomes.
  \end{enumerate}
\item
  \begin{enumerate}
  \def\labelenumi{\alph{enumi})}
  \setcounter{enumi}{3}
  \tightlist
  \item
    About 20\% of the topics/sub-topics which are relatively simpler or
    descriptive in nature is to be given to the students for
    self-learning , but to be assessed using different assessment
    methods.
  \end{enumerate}
\item
  \begin{enumerate}
  \def\labelenumi{\alph{enumi})}
  \setcounter{enumi}{4}
  \tightlist
  \item
    With respect to section No.10 , teachers need to ensure to create
    opportunities and provisions for co-curricular activities .
  \end{enumerate}
\end{itemize}

\subsection*{f) Guide students on how to address issues on environment
and
sustainability}\label{f-guide-students-on-how-to-address-issues-on-environment-and-sustainability}

\begin{itemize}
\tightlist
\item
  \begin{enumerate}
  \def\labelenumi{\alph{enumi})}
  \setcounter{enumi}{6}
  \tightlist
  \item
    Guide students for using data manuals.
  \end{enumerate}
\item
  \begin{enumerate}
  \def\labelenumi{\alph{enumi})}
  \setcounter{enumi}{7}
  \tightlist
  \item
    Guide students for using data manuals.
  \end{enumerate}
\item
  \begin{enumerate}
  \def\labelenumi{\roman{enumi})}
  \tightlist
  \item
    Arrange visit to nearby industries and workshops for understanding
    various sources of pollution.
  \end{enumerate}
\item
  \begin{enumerate}
  \def\labelenumi{\alph{enumi})}
  \setcounter{enumi}{9}
  \tightlist
  \item
    Use video/animation films to explain various processes related to
    environment and sustainable development
  \end{enumerate}
\item
  \begin{enumerate}
  \def\labelenumi{\alph{enumi})}
  \setcounter{enumi}{10}
  \tightlist
  \item
    Use different instructional strategies in classroom teaching.
  \end{enumerate}
\item
  \begin{enumerate}
  \def\labelenumi{\alph{enumi})}
  \setcounter{enumi}{11}
  \tightlist
  \item
    Write the report on properties of various eco-friendly construction
    materials like Stone, aggregate of different sizes, timber, lime,
    bitumen, Bricks, tiles, precast concrete products, Water proofing
    material, Termite proofing material, Thermal insulating material,
    plaster of Paris, paints, distemper, and varnishes.
  \end{enumerate}
\item
  \begin{enumerate}
  \def\labelenumi{\alph{enumi})}
  \setcounter{enumi}{12}
  \tightlist
  \item
    Display various technical brochures of recent projects/themes
    related to environment and sustainable development
  \end{enumerate}
\item
  \begin{enumerate}
  \def\labelenumi{\alph{enumi})}
  \setcounter{enumi}{13}
  \tightlist
  \item
    Visit the Pollution control board office and its various projects to
    demonstrate the various practices adopted for control of Pollution
  \end{enumerate}
\end{itemize}

\subsection*{12. SUGGESTED
MICRO-PROJECTS}\label{suggested-micro-projects}

Only one micro-project is planned to be undertaken by a student that
needs to be assigned to him/her in the beginning of the semester. In the
first four semesters, the micro-project are group-based. However, in the
fifth and sixth semesters, it should be preferably be individually
undertaken to build up the skill and confidence in every student to
become problem solver so that s/he contributes to the projects of the
industry. In special situations where groups have to be formed for
micro-projects, the number of students in the group should not exceed
three.

The micro-project could be industry application based, internet-based,
workshop based, laboratory-based or field-based. Each micro-project
should encompass two or more COs which are in fact, an integration of
PrOs, UOs and ADOs. Each student will have to maintain dated work diary
consisting of individual contribution in the project work and give a
seminar presentation of it before submission. The total duration of the
micro-project should not be less than 16 (sixteen) student engagement
hours during the course. The student ought to submit micro-project by
the end of the semester to develop the industry-oriented COs.

A suggestive list of micro-projects is given here. This has to match the
competency and the COs. Similar micro-projects could be added by the
concerned course teacher:

\begin{itemize}
\tightlist
\item
  \begin{enumerate}
  \def\labelenumi{\alph{enumi})}
  \tightlist
  \item
    Natural cycles : Build a Chart showing different natural cycles like
    Carbon, Nitrogen,Sulphur and phosphorus cycle.)
  \end{enumerate}
\item
  \begin{enumerate}
  \def\labelenumi{\alph{enumi})}
  \setcounter{enumi}{1}
  \tightlist
  \item
    Solar Energy: Build a model of Solar water heater/Solar cooker
  \end{enumerate}
\item
  \begin{enumerate}
  \def\labelenumi{\alph{enumi})}
  \setcounter{enumi}{2}
  \tightlist
  \item
    Wind energy: Build a model of wind mill
  \end{enumerate}
\item
  \begin{enumerate}
  \def\labelenumi{\alph{enumi})}
  \setcounter{enumi}{3}
  \tightlist
  \item
    Best out of waste : Build useful items from waste materials like
    used plastic bottles, discarded pens etc.
  \end{enumerate}
\item
  \begin{enumerate}
  \def\labelenumi{\alph{enumi})}
  \setcounter{enumi}{4}
  \tightlist
  \item
    Compare the pollution (water, air and noise) data of various cities
    with standard values as laid by pollution control board.
  \end{enumerate}
\item
  \begin{enumerate}
  \def\labelenumi{\alph{enumi})}
  \setcounter{enumi}{5}
  \tightlist
  \item
    Surf different websites related environment and sustainable
    development, Pollution control.
  \end{enumerate}
\item
  \begin{enumerate}
  \def\labelenumi{\alph{enumi})}
  \setcounter{enumi}{6}
  \tightlist
  \item
    Prepare energy audit report of any residential building.
  \end{enumerate}
\item
  \begin{enumerate}
  \def\labelenumi{\alph{enumi})}
  \setcounter{enumi}{7}
  \tightlist
  \item
    Collect relevant information about the software used in pollution
    control.
  \end{enumerate}
\item
  \begin{enumerate}
  \def\labelenumi{\alph{enumi})}
  \setcounter{enumi}{14}
  \tightlist
  \item
    Visit to ongoing project and study various aspects related to
    environment and sustainable development
  \end{enumerate}
\end{itemize}

\subsection*{13. SUGGESTED LEARNING
RESOURCES}\label{suggested-learning-resources}

{\def\LTcaptype{none} % do not increment counter
\begin{longtable}[]{@{}
  >{\raggedright\arraybackslash}p{(\linewidth - 6\tabcolsep) * \real{0.0374}}
  >{\raggedright\arraybackslash}p{(\linewidth - 6\tabcolsep) * \real{0.3209}}
  >{\raggedright\arraybackslash}p{(\linewidth - 6\tabcolsep) * \real{0.3209}}
  >{\raggedright\arraybackslash}p{(\linewidth - 6\tabcolsep) * \real{0.3209}}@{}}
\toprule\noalign{}
\begin{minipage}[b]{\linewidth}\raggedright
S. No.
\end{minipage} & \begin{minipage}[b]{\linewidth}\raggedright
Title of Book
\end{minipage} & \begin{minipage}[b]{\linewidth}\raggedright
Author
\end{minipage} & \begin{minipage}[b]{\linewidth}\raggedright
Publication with place, year and ISBN
\end{minipage} \\
\midrule\noalign{}
\endhead
\bottomrule\noalign{}
\endlastfoot
1 & Renewable Energy Technologies: A Practical Guide for Beginners &
Solanki, Chetan Singh & PHI Learning, New Delhi, 2010 Print Book ISBN:
9788120334342 eBook ISBN: 9789354437151 \\
2 & Ecology and Control of the Natural Environment & Izrael,Y.A. &
Kluwer Academic Publisher eBook ISBN: 978-94-011-3390-6 Softcover ISBN:
978-94-010-5499-7 \\
3 & Green Technologies and Environmental Sustainability & Singh, Ritu,
Kumar, Sanjeev & Springer International Publishing, 2017 eBook ISBN
978-3-319-50654-8 \\
4 & Environmental Noise Pollution and Its Control & G.R. Chhatwal, M.
Satake, M.C. Mehra, Mohan Katyal, T. Katyal, T. Nagahiro & Anmol
Publications, New Delhi ISBN: 8170411378 ISBN: 8170411378 \\
5 & Wind Power Plants and Project Development & Earnest, Joshua \&
Wizelius, Tore & PHI Learning, New Delhi, 2011 ISBN-10: 8120351274
ISBN-13: 978-8120351271 \\
6 & Renewable Energy Sources and Emerging Technologies & Kothari, D.P.
Singal, K.C., Ranjan, Rakesh & PHI Learning, New Delhi, 2009 ISBN-13 -
978-8120344709 \\
7 & Environmental Studies & Anandita Basak & Pearson Publications ISBN
8131785688, 9788131785683 ISBN: 9788131721186, 8131721183 \\
8 & Environmental Science and Engineering & Aloka Debi & University
Press ISBN: 9788173718113 ISBN-10: 8173716080 ISBN-13: 978-8173716089 \\
9 & Coping With Natural Hazards: Indian Context & K. S. Valadia & Orient
Longman ISBN-10: 8125027351 ISBN-13: 978-8125027355 \\
10 & Introduction to Engineering and Environment & Edward S. Rubin & Mc
Graw Hill Publications ISBN-10 : 0071181857 ISBN-13 : 978-0071181853 \\
\end{longtable}
}

\subsection*{14. SOFTWARE/LEARNING
WEBSITES}\label{softwarelearning-websites}

\begin{itemize}
\tightlist
\item
  \begin{enumerate}
  \def\labelenumi{\alph{enumi})}
  \tightlist
  \item
    www.nptel.iitm.ac.in
  \end{enumerate}
\item
  \begin{enumerate}
  \def\labelenumi{\alph{enumi})}
  \setcounter{enumi}{1}
  \tightlist
  \item
    www.khanacademy
  \end{enumerate}
\item
  \begin{enumerate}
  \def\labelenumi{\alph{enumi})}
  \setcounter{enumi}{2}
  \tightlist
  \item
    http://www1.eere.energy.gov/wind/wind\_animation.html
  \end{enumerate}
\item
  \begin{enumerate}
  \def\labelenumi{\alph{enumi})}
  \setcounter{enumi}{3}
  \tightlist
  \item
    http://www.nrel.gov/learning/re\_solar.html
  \end{enumerate}
\item
  \begin{enumerate}
  \def\labelenumi{\alph{enumi})}
  \setcounter{enumi}{4}
  \tightlist
  \item
    http://www.nrel.gov/learning/re\_biomass.html
  \end{enumerate}
\item
  \begin{enumerate}
  \def\labelenumi{\alph{enumi})}
  \setcounter{enumi}{5}
  \tightlist
  \item
    http://www.mnre.gov.in/schemes/grid-connected/biomass-powercogen/
  \end{enumerate}
\item
  \begin{enumerate}
  \def\labelenumi{\alph{enumi})}
  \setcounter{enumi}{6}
  \tightlist
  \item
    http://www.epa.gov/climatestudents/
  \end{enumerate}
\item
  \begin{enumerate}
  \def\labelenumi{\alph{enumi})}
  \setcounter{enumi}{7}
  \tightlist
  \item
    http://www.climatecentral.org
  \end{enumerate}
\item
  \begin{enumerate}
  \def\labelenumi{\roman{enumi})}
  \tightlist
  \item
    http://www.envis.nic.in/
  \end{enumerate}
\item
  \begin{enumerate}
  \def\labelenumi{\alph{enumi})}
  \setcounter{enumi}{9}
  \tightlist
  \item
    https://www.overshootday.org/
  \end{enumerate}
\item
  \begin{enumerate}
  \def\labelenumi{\alph{enumi})}
  \setcounter{enumi}{10}
  \tightlist
  \item
    http://www.footprintcalculator.org/
  \end{enumerate}
\item
  \begin{enumerate}
  \def\labelenumi{\alph{enumi})}
  \setcounter{enumi}{11}
  \tightlist
  \item
    https://www.carbonfootprint.com/calculator.aspx
  \end{enumerate}
\end{itemize}

\subsection*{15. PO-COMPETENCY-CO
MAPPING}\label{po-competency-co-mapping}

Legend: ' 3' for high, ' 2 ' for medium, `1' for low or `-' for the
relevant correlation of each competency, CO, with PO/ PSO

{\def\LTcaptype{none} % do not increment counter
\begin{longtable}[]{@{}
  >{\raggedright\arraybackslash}p{(\linewidth - 18\tabcolsep) * \real{0.1000}}
  >{\raggedright\arraybackslash}p{(\linewidth - 18\tabcolsep) * \real{0.1000}}
  >{\raggedright\arraybackslash}p{(\linewidth - 18\tabcolsep) * \real{0.1000}}
  >{\raggedright\arraybackslash}p{(\linewidth - 18\tabcolsep) * \real{0.1000}}
  >{\raggedright\arraybackslash}p{(\linewidth - 18\tabcolsep) * \real{0.1000}}
  >{\raggedright\arraybackslash}p{(\linewidth - 18\tabcolsep) * \real{0.1000}}
  >{\raggedright\arraybackslash}p{(\linewidth - 18\tabcolsep) * \real{0.1000}}
  >{\raggedright\arraybackslash}p{(\linewidth - 18\tabcolsep) * \real{0.1000}}
  >{\raggedright\arraybackslash}p{(\linewidth - 18\tabcolsep) * \real{0.1000}}
  >{\raggedright\arraybackslash}p{(\linewidth - 18\tabcolsep) * \real{0.1000}}@{}}
\toprule\noalign{}
\begin{minipage}[b]{\linewidth}\raggedright
Semester II
\end{minipage} & \begin{minipage}[b]{\linewidth}\raggedright
Environment and Sustainability (Course Code:
\ldots\ldots\ldots\ldots\ldots\ldots\ldots{} )
\end{minipage} & \begin{minipage}[b]{\linewidth}\raggedright
Environment and Sustainability (Course Code:
\ldots\ldots\ldots\ldots\ldots\ldots\ldots{} )
\end{minipage} & \begin{minipage}[b]{\linewidth}\raggedright
Environment and Sustainability (Course Code:
\ldots\ldots\ldots\ldots\ldots\ldots\ldots{} )
\end{minipage} & \begin{minipage}[b]{\linewidth}\raggedright
Environment and Sustainability (Course Code:
\ldots\ldots\ldots\ldots\ldots\ldots\ldots{} )
\end{minipage} & \begin{minipage}[b]{\linewidth}\raggedright
Environment and Sustainability (Course Code:
\ldots\ldots\ldots\ldots\ldots\ldots\ldots{} )
\end{minipage} & \begin{minipage}[b]{\linewidth}\raggedright
Environment and Sustainability (Course Code:
\ldots\ldots\ldots\ldots\ldots\ldots\ldots{} )
\end{minipage} & \begin{minipage}[b]{\linewidth}\raggedright
Environment and Sustainability (Course Code:
\ldots\ldots\ldots\ldots\ldots\ldots\ldots{} )
\end{minipage} & \begin{minipage}[b]{\linewidth}\raggedright
Environment and Sustainability (Course Code:
\ldots\ldots\ldots\ldots\ldots\ldots\ldots{} )
\end{minipage} & \begin{minipage}[b]{\linewidth}\raggedright
Environment and Sustainability (Course Code:
\ldots\ldots\ldots\ldots\ldots\ldots\ldots{} )
\end{minipage} \\
\midrule\noalign{}
\endhead
\bottomrule\noalign{}
\endlastfoot
Competency \& Course Outcomes & PO 1 Basic \& specific knowledg e &
Discipline PO 2 Proble m Analysi s & PO 3 Design/ develop ment of
solutio ns & PO 4 Engineering Tools, Experiment ation \&Testing & PO 5
Engineering practices for society, sustainability \& environment & PO 6
Project Manageme nt & PO 7 Life-long learning & PSO 1 Environm ental
planning \& deisgn & PSO 2 Execution \& Maintenan ce \\
Competency - Adopt the sustainable practices to resolve the environment
related issues & Competency - Adopt the sustainable practices to resolve
the environment related issues & Competency - Adopt the sustainable
practices to resolve the environment related issues & Competency - Adopt
the sustainable practices to resolve the environment related issues &
Competency - Adopt the sustainable practices to resolve the environment
related issues & Competency - Adopt the sustainable practices to resolve
the environment related issues & Competency - Adopt the sustainable
practices to resolve the environment related issues & Competency - Adopt
the sustainable practices to resolve the environment related issues &
Competency - Adopt the sustainable practices to resolve the environment
related issues & Competency - Adopt the sustainable practices to resolve
the environment related issues \\
a. Adopt relevant ecofriendly product in the given situation to protect
ecosystem & 2 & 1 & 1 & - & 2 & 1 & 1 & 2 & 2 \\
b. use relevant method of pollution reduction in the given situation & 2
& 2 & 1 & 1 & 2 & - & 2 & 2 & 2 \\
c.~Use of renewable resources of energy for sustainable development & 2
& 2 & 2 & 1 & 2 & 2 & 1 & 2 & 2 \\
d.~Use the relevant techniques in given context to reduce impact due to
climate change & 2 & 2 & 2 & 1 & 2 & 1 & 2 & 2 & 2 \\
e. Use relevant laws and policies for developing the sustainable
environmental development & 2 & 2 & 2 & 1 & 1 & 1 & 1 & 2 & 2 \\
\end{longtable}
}

\subsection*{16. COURSE CURRICULUM DEVELOPMENT
COMMITTEE}\label{course-curriculum-development-committee}

\subsection*{GTU Resource Persons}\label{gtu-resource-persons}

{\def\LTcaptype{none} % do not increment counter
\begin{longtable}[]{@{}
  >{\raggedright\arraybackslash}p{(\linewidth - 8\tabcolsep) * \real{0.0569}}
  >{\raggedright\arraybackslash}p{(\linewidth - 8\tabcolsep) * \real{0.1626}}
  >{\raggedright\arraybackslash}p{(\linewidth - 8\tabcolsep) * \real{0.4309}}
  >{\raggedright\arraybackslash}p{(\linewidth - 8\tabcolsep) * \real{0.0976}}
  >{\raggedright\arraybackslash}p{(\linewidth - 8\tabcolsep) * \real{0.2520}}@{}}
\toprule\noalign{}
\begin{minipage}[b]{\linewidth}\raggedright
S. No.
\end{minipage} & \begin{minipage}[b]{\linewidth}\raggedright
Name and Designation
\end{minipage} & \begin{minipage}[b]{\linewidth}\raggedright
Institute
\end{minipage} & \begin{minipage}[b]{\linewidth}\raggedright
Contact No.
\end{minipage} & \begin{minipage}[b]{\linewidth}\raggedright
Email
\end{minipage} \\
\midrule\noalign{}
\endhead
\bottomrule\noalign{}
\endlastfoot
1 & Dr.~Jayesh Shah & Ass. Dean GTU, Pacific School of Engineering,
Surat & 9825436342 & jayesh.shah.23021971 @gmail.com \\
2 & Mrs.~Jini Sunil & Shri K.J. Polytechnic, Bharuch & 9601880636 &
jinivt@rediffmail.com \\
\end{longtable}
}

\subsection*{NITTTR Resource Persons}\label{nitttr-resource-persons}

{\def\LTcaptype{none} % do not increment counter
\begin{longtable}[]{@{}
  >{\raggedright\arraybackslash}p{(\linewidth - 8\tabcolsep) * \real{0.0638}}
  >{\raggedright\arraybackslash}p{(\linewidth - 8\tabcolsep) * \real{0.5000}}
  >{\raggedright\arraybackslash}p{(\linewidth - 8\tabcolsep) * \real{0.0532}}
  >{\raggedright\arraybackslash}p{(\linewidth - 8\tabcolsep) * \real{0.1170}}
  >{\raggedright\arraybackslash}p{(\linewidth - 8\tabcolsep) * \real{0.2660}}@{}}
\toprule\noalign{}
\begin{minipage}[b]{\linewidth}\raggedright
S. No
\end{minipage} & \begin{minipage}[b]{\linewidth}\raggedright
Name and Designation
\end{minipage} & \begin{minipage}[b]{\linewidth}\raggedright
Dept.
\end{minipage} & \begin{minipage}[b]{\linewidth}\raggedright
Contact No.
\end{minipage} & \begin{minipage}[b]{\linewidth}\raggedright
Email
\end{minipage} \\
\midrule\noalign{}
\endhead
\bottomrule\noalign{}
\endlastfoot
1 & Dr.~V.D.Patil, Associate Professor, DCEEE & DCEEE & 9422346736 &
vdpatil@nitttrbpl.ac.in \\
2 & Prof.~M.C.Paliwal, Associate Professor, DCEEE & DCEEE & 9407271980 &
mcpaliwal@nitttrbpl.ac.in \\
\end{longtable}
}

\end{document}