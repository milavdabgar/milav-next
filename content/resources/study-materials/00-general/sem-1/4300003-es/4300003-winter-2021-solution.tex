\documentclass[10pt,a4paper]{article}

% content/resources/templates/preamble.tex
\usepackage[margin=0.6in]{geometry}
\author{Milav Dabgar}
\usepackage{amsmath,amssymb,amsthm}
\usepackage{booktabs}
\usepackage{multirow}
\usepackage{xcolor}
\usepackage{tcolorbox}
\tcbuselibrary{breakable,skins}
\usepackage[colorlinks=true,linkcolor=blue]{hyperref}
\usepackage{titlesec}
\usepackage{enumitem}
\usepackage{tikz}
\usepackage{pgfplots}
\usepackage{circuitikz}
\usepackage[version=4]{mhchem}
\usepackage{longtable}
\usepackage{array}
\usepackage{float}
\usepackage{caption}
\usepackage{listings}

\lstset{
  basicstyle=\small\ttfamily,
  breaklines=true,
  breakatwhitespace=false,
  postbreak=\mbox{\textcolor{red}{$\hookrightarrow$}\space},
  float=false,
  numbers=left,
  numberstyle=\tiny\color{gray},
  numbersep=10pt,
  xleftmargin=2em,
  keywordstyle=\color{blue},
  commentstyle=\color{green!60!black},
  stringstyle=\color{purple},
  backgroundcolor=\color{gray!5},
  showstringspaces=false,
  tabsize=2,
  captionpos=b,
  keepspaces=true,
  columns=flexible
}

\pgfplotsset{compat=1.18}
\usetikzlibrary{shapes,arrows,positioning,calc,patterns,decorations.pathmorphing,decorations.markings,arrows.meta}

% Color scheme
\definecolor{headcolor}{RGB}{0,102,204}
\definecolor{keycolor}{RGB}{220,20,60}
\definecolor{solutioncolor}{RGB}{34,139,34}
\definecolor{mnemoniccolor}{RGB}{148,0,211}
\definecolor{codecolor}{RGB}{0,0,100}

% Spacing
\setlength{\parskip}{3pt}
\setlist[itemize]{nosep}
\setlist[enumerate]{nosep}

% Title formatting
\titleformat{\section}{\Large\bfseries\color{headcolor}}{\thesection}{1em}{}
\titleformat{\subsection}{\large\bfseries\color{headcolor}}{\thesubsection}{1em}{}

% Pandoc tightlist compatibility
\providecommand{\tightlist}{%
  \setlength{\itemsep}{0pt}\setlength{\parskip}{0pt}}

% Pandoc longtable compatibility
\newcounter{none}
\def\thenone{}


% content/resources/templates/english-boxes.tex
% This file is currently empty - it exists to maintain consistency with the import structure.
% Add custom environments here if needed in the future.


\begin{document}

\begin{center}
{\Huge\bfseries\color{headcolor} Environment and Sustainability Solutions}\\[5pt]
{\LARGE 4300003 -- Winter 2021}\\[3pt]
{\large Semester 1 Study Material}\\[3pt]
{\normalsize\textit{Detailed Solutions and Explanations}}
\end{center}

\vspace{10pt}

\subsection*{Question 1 (Any Seven) {[}14
marks{]}}\label{question-1-any-seven-14-marks}

\subsubsection{\texorpdfstring{1. \textbf{Define the terms: `Ecology'
and
`Ecosystem'.}}{1. Define the terms: `Ecology' and `Ecosystem'.}}\label{define-the-terms-ecology-and-ecosystem.}

\begin{solutionbox}
\textbf{Ecology} is the scientific study of
relationships between living organisms and their environment.
\textbf{Ecosystem} is a biological community of interacting organisms
and their physical environment functioning as a unit.


{\def\LTcaptype{none} % do not increment counter
\begin{longtable}[]{@{}
  >{\raggedright\arraybackslash}p{(\linewidth - 4\tabcolsep) * \real{0.2222}}
  >{\raggedright\arraybackslash}p{(\linewidth - 4\tabcolsep) * \real{0.4444}}
  >{\raggedright\arraybackslash}p{(\linewidth - 4\tabcolsep) * \real{0.3333}}@{}}
\toprule\noalign{}
\begin{minipage}[b]{\linewidth}\raggedright
Term
\end{minipage} & \begin{minipage}[b]{\linewidth}\raggedright
Definition
\end{minipage} & \begin{minipage}[b]{\linewidth}\raggedright
Example
\end{minipage} \\
\midrule\noalign{}
\endhead
\bottomrule\noalign{}
\endlastfoot
Ecology & Study of organism-environment relationships & Forest
ecology \\
Ecosystem & Living and non-living components interaction & Pond
ecosystem \\
\end{longtable}
}

\begin{itemize}
\tightlist
\item
  \textbf{Biotic components}: Living organisms in the system
\item
  \textbf{Abiotic components}: Non-living factors like air, water, soil
\end{itemize}

\end{solutionbox}
\begin{mnemonicbox}
``Every Component Lives Together'' (Ecology Creates
Living Together)

\subsubsection{\texorpdfstring{2. \textbf{Define the terms: `Pollution'
and
`Pollutant'.}}{2. Define the terms: `Pollution' and `Pollutant'.}}\label{define-the-terms-pollution-and-pollutant.}

\end{mnemonicbox}
\begin{solutionbox}
\textbf{Pollution} is the introduction of harmful
substances into the environment causing adverse effects.
\textbf{Pollutant} is any substance that causes pollution when present
in excessive amounts.


{\def\LTcaptype{none} % do not increment counter
\begin{longtable}[]{@{}lll@{}}
\toprule\noalign{}
Term & Definition & Types \\
\midrule\noalign{}
\endhead
\bottomrule\noalign{}
\endlastfoot
Pollution & Environmental contamination & Air, Water, Soil, Noise \\
Pollutant & Harmful substance & Physical, Chemical, Biological \\
\end{longtable}
}

\begin{itemize}
\tightlist
\item
  \textbf{Primary pollutants}: Directly emitted substances
\item
  \textbf{Secondary pollutants}: Formed by reactions in atmosphere
\end{itemize}

\end{solutionbox}
\begin{mnemonicbox}
``Pollution Produces Problems'' (Pollutants Produce
Problems)

\subsubsection{\texorpdfstring{3. \textbf{What is noise pollution? What
is unit of intensity of
sound?}}{3. What is noise pollution? What is unit of intensity of sound?}}\label{what-is-noise-pollution-what-is-unit-of-intensity-of-sound}

\end{mnemonicbox}
\begin{solutionbox}
\textbf{Noise pollution} is unwanted or excessive sound
that disrupts human activities and harms living beings. The unit of
sound intensity is \textbf{decibel (dB)}.


{\def\LTcaptype{none} % do not increment counter
\begin{longtable}[]{@{}lll@{}}
\toprule\noalign{}
Sound Level & Source & Effect \\
\midrule\noalign{}
\endhead
\bottomrule\noalign{}
\endlastfoot
30-40 dB & Library & Comfortable \\
60-70 dB & Traffic & Annoying \\
90+ dB & Industry & Harmful \\
\end{longtable}
}

\begin{itemize}
\tightlist
\item
  \textbf{Threshold of hearing}: 0 dB
\item
  \textbf{Threshold of pain}: 120 dB
\end{itemize}

\end{solutionbox}
\begin{mnemonicbox}
``Decibels Determine Damage'' (dB Determines Damage)

\subsubsection{\texorpdfstring{4. \textbf{What is solid waste
management? Give its
objectives.}}{4. What is solid waste management? Give its objectives.}}\label{what-is-solid-waste-management-give-its-objectives.}

\end{mnemonicbox}
\begin{solutionbox}
\textbf{Solid waste management} is systematic handling
of waste from generation to final disposal to minimize environmental
impact and protect public health.

\textbf{Objectives:}

\begin{itemize}
\tightlist
\item
  \textbf{Public health protection}: Prevent disease transmission
\item
  \textbf{Environmental protection}: Reduce pollution and contamination
\item
  \textbf{Resource recovery}: Recycle and reuse materials
\item
  \textbf{Cost effectiveness}: Economic waste handling
\end{itemize}

\end{solutionbox}
\begin{mnemonicbox}
``People Expect Resource Conservation'' (Protection,
Environment, Resource, Cost)

\subsubsection{\texorpdfstring{5. \textbf{Enlist types of solar
cells.}}{5. Enlist types of solar cells.}}\label{enlist-types-of-solar-cells.}

\end{mnemonicbox}
\begin{solutionbox}
Solar cells convert sunlight directly into electricity
through photovoltaic effect.


{\def\LTcaptype{none} % do not increment counter
\begin{longtable}[]{@{}llll@{}}
\toprule\noalign{}
Type & Efficiency & Cost & Application \\
\midrule\noalign{}
\endhead
\bottomrule\noalign{}
\endlastfoot
Monocrystalline & 15-20\% & High & Residential \\
Polycrystalline & 13-16\% & Medium & Commercial \\
Thin Film & 7-13\% & Low & Large scale \\
\end{longtable}
}

\begin{itemize}
\tightlist
\item
  \textbf{Silicon-based}: Most common type
\item
  \textbf{Non-silicon}: Emerging technologies
\end{itemize}

\end{solutionbox}
\begin{mnemonicbox}
``Most People Think'' (Mono, Poly, Thin-film)

\subsubsection{\texorpdfstring{6. \textbf{What is climate
change?}}{6. What is climate change?}}\label{what-is-climate-change}

\end{mnemonicbox}
\begin{solutionbox}
\textbf{Climate change} refers to long-term shifts in
global temperatures and weather patterns, primarily caused by human
activities and greenhouse gas emissions.

\textbf{Causes:}

\begin{itemize}
\tightlist
\item
  \textbf{Greenhouse gases}: CO₂, CH₄, N₂O emissions
\item
  \textbf{Deforestation}: Reduced carbon absorption
\item
  \textbf{Industrial activities}: Fossil fuel burning
\end{itemize}

\textbf{Effects:}

\begin{itemize}
\tightlist
\item
  \textbf{Rising temperatures}: Global warming
\item
  \textbf{Sea level rise}: Melting ice caps
\end{itemize}

\end{solutionbox}
\begin{mnemonicbox}
``Change Creates Consequences'' (Climate Change
Creates Consequences)

\subsubsection{\texorpdfstring{7. \textbf{What is
C.F.C?}}{7. What is C.F.C?}}\label{what-is-c.f.c}

\end{mnemonicbox}
\begin{solutionbox}
\textbf{CFC (Chlorofluorocarbon)} are synthetic
compounds containing carbon, fluorine, and chlorine atoms, previously
used in refrigeration and aerosols.

\textbf{Properties:}

\begin{itemize}
\tightlist
\item
  \textbf{Ozone depleting}: Destroys stratospheric ozone
\item
  \textbf{Greenhouse gas}: Contributes to global warming
\item
  \textbf{Stable compounds}: Long atmospheric lifetime
\item
  \textbf{Montreal Protocol}: International ban agreement
\end{itemize}

\end{solutionbox}
\begin{mnemonicbox}
``Chlorine Fluorine Carbon'' (CFC components)

\subsubsection{\texorpdfstring{8. \textbf{Give advantages of
ISO-14000.}}{8. Give advantages of ISO-14000.}}\label{give-advantages-of-iso-14000.}

\end{mnemonicbox}
\begin{solutionbox}
\textbf{ISO 14000} is international standard for
environmental management systems.

\textbf{Advantages:}

\begin{itemize}
\tightlist
\item
  \textbf{Environmental compliance}: Meet legal requirements
\item
  \textbf{Cost reduction}: Efficient resource use
\item
  \textbf{Market advantage}: Enhanced company image
\item
  \textbf{Risk management}: Prevent environmental incidents
\end{itemize}


{\def\LTcaptype{none} % do not increment counter
\begin{longtable}[]{@{}lll@{}}
\toprule\noalign{}
Benefit & Impact & Result \\
\midrule\noalign{}
\endhead
\bottomrule\noalign{}
\endlastfoot
Compliance & Legal safety & Avoid penalties \\
Efficiency & Resource saving & Cost reduction \\
Image & Market position & Competitive advantage \\
\end{longtable}
}

\end{solutionbox}
\begin{mnemonicbox}
``Companies Gain Market Recognition'' (Compliance,
Cost, Market, Risk)

\subsubsection{\texorpdfstring{9. \textbf{Enlist various Acts related to
environment in
India.}}{9. Enlist various Acts related to environment in India.}}\label{enlist-various-acts-related-to-environment-in-india.}

\end{mnemonicbox}
\begin{solutionbox}
India has comprehensive environmental legislation
framework.

\textbf{Major Acts:}

\begin{itemize}
\tightlist
\item
  \textbf{Air Act (1981)}: Air pollution control
\item
  \textbf{Water Act (1974)}: Water pollution prevention
\item
  \textbf{Environment Protection Act (1986)}: Comprehensive
  environmental law
\item
  \textbf{Wildlife Protection Act (1972)}: Biodiversity conservation
\item
  \textbf{Forest Conservation Act (1980)}: Forest protection
\end{itemize}

\end{solutionbox}
\begin{mnemonicbox}
``All Water Environments Wildlife Forests'' (AWEWF)

\subsubsection{\texorpdfstring{10. \textbf{Enlist various methods of
rainwater
harvesting.}}{10. Enlist various methods of rainwater harvesting.}}\label{enlist-various-methods-of-rainwater-harvesting.}

\end{mnemonicbox}
\begin{solutionbox}
\textbf{Rainwater harvesting} collects and stores
rainwater for future use.

\textbf{Methods:}

\begin{itemize}
\tightlist
\item
  \textbf{Rooftop harvesting}: Direct collection from roofs
\item
  \textbf{Surface runoff harvesting}: From ground surfaces
\item
  \textbf{Recharge pits}: Groundwater recharging
\item
  \textbf{Check dams}: Stream water collection
\end{itemize}


{\def\LTcaptype{none} % do not increment counter
\begin{longtable}[]{@{}lll@{}}
\toprule\noalign{}
Method & Application & Benefit \\
\midrule\noalign{}
\endhead
\bottomrule\noalign{}
\endlastfoot
Rooftop & Urban areas & Direct use \\
Surface & Rural areas & Large volume \\
Recharge & Water table & Groundwater \\
\end{longtable}
}

\end{solutionbox}
\begin{mnemonicbox}
``Roofs Surface Recharge Check'' (RSRC)

\end{mnemonicbox}
\subsection*{Question 2(a) {[}3 marks{]}}\label{question-2a-3-marks}

\subsubsection{\texorpdfstring{\textbf{Write short note on: Food
chain.}}{Write short note on: Food chain.}}\label{write-short-note-on-food-chain.}

\begin{solutionbox}
\textbf{Food chain} represents the flow of energy and
nutrients through different trophic levels in an ecosystem.

\begin{center}
\textbf{Mermaid Diagram (Code)}
\begin{verbatim}
{Shaded}
{Highlighting}[]
graph LR
    A[Producers{br/{}Plants] {-}{-}{} B[Primary Consumers{}br/{}Herbivores]}
    B {-{-}{} C[Secondary Consumers{}br/{}Carnivores]}
    C {-{-}{} D[Tertiary Consumers{}br/{}Top Predators]}
    D {-{-}{} E[Decomposers{}br/{}Bacteria/Fungi]}
{Highlighting}
{Shaded}
\end{verbatim}
\end{center}

\begin{itemize}
\tightlist
\item
  \textbf{Energy transfer}: Only 10\% passes to next level
\item
  \textbf{Biomass pyramid}: Decreases at higher levels
\end{itemize}

\end{solutionbox}
\begin{mnemonicbox}
``Plants Provide Primary Power'' (Producer to
Predator Path)

\subsubsection{OR}\label{or}

\subsubsection{\texorpdfstring{\textbf{Explain factors affecting
ecosystem.}}{Explain factors affecting ecosystem.}}\label{explain-factors-affecting-ecosystem.}

\end{mnemonicbox}
\begin{solutionbox}
Ecosystems are influenced by various biotic and abiotic
factors.

\textbf{Factors:}

\begin{itemize}
\tightlist
\item
  \textbf{Climate factors}: Temperature, rainfall, humidity
\item
  \textbf{Soil factors}: pH, nutrients, texture
\item
  \textbf{Biotic factors}: Species interactions, population density
\item
  \textbf{Human factors}: Pollution, habitat destruction
\end{itemize}


{\def\LTcaptype{none} % do not increment counter
\begin{longtable}[]{@{}lll@{}}
\toprule\noalign{}
Factor Type & Components & Impact \\
\midrule\noalign{}
\endhead
\bottomrule\noalign{}
\endlastfoot
Abiotic & Climate, Soil & Habitat conditions \\
Biotic & Organisms & Species interactions \\
Anthropogenic & Human activities & Ecosystem disruption \\
\end{longtable}
}

\end{solutionbox}
\begin{mnemonicbox}
``Climate Soil Biology Humans'' (CSBH)

\end{mnemonicbox}
\subsection*{Question 2(b) {[}3 marks{]}}\label{question-2b-3-marks}

\subsubsection{\texorpdfstring{\textbf{Write short note on: Virtual
water}}{Write short note on: Virtual water}}\label{write-short-note-on-virtual-water}

\begin{solutionbox}
\textbf{Virtual water} is the hidden water used in
production of goods and services, representing total water consumption
in supply chain.

\textbf{Examples:}

\begin{itemize}
\item
  \textbf{1 kg wheat}: 1,300 liters virtual water
\item
  \textbf{1 kg beef}: 15,400 liters virtual water
\item
  \textbf{1 cotton t-shirt}: 2,700 liters virtual water
\item
  \textbf{Water footprint}: Total virtual water consumption
\item
  \textbf{Trade implications}: Water-rich countries export virtual water
\end{itemize}

\end{solutionbox}
\begin{mnemonicbox}
``Virtual Water Worldwide'' (VWW)

\subsubsection{OR}\label{or-1}

\subsubsection{\texorpdfstring{\textbf{What is biodiversity? Give its
types.}}{What is biodiversity? Give its types.}}\label{what-is-biodiversity-give-its-types.}

\end{mnemonicbox}
\begin{solutionbox}
\textbf{Biodiversity} is the variety of life forms at
genetic, species, and ecosystem levels on Earth.

\textbf{Types:}

\begin{itemize}
\tightlist
\item
  \textbf{Genetic diversity}: Variation within species
\item
  \textbf{Species diversity}: Number of different species
\item
  \textbf{Ecosystem diversity}: Variety of habitats and communities
\end{itemize}

\begin{verbatim}
mindmap
  root((Biodiversity))
    Genetic
      DNA variation
      Population genetics
    Species
      Flora
      Fauna
    Ecosystem
      Terrestrial
      Aquatic
\end{verbatim}

\end{solutionbox}
\begin{mnemonicbox}
``Genes Species Ecosystems'' (GSE)

\end{mnemonicbox}
\subsection*{Question 2(c) {[}4 marks{]}}\label{question-2c-4-marks}

\subsubsection{\texorpdfstring{\textbf{Explain: Carbon
cycle}}{Explain: Carbon cycle}}\label{explain-carbon-cycle}

\begin{solutionbox}
\textbf{Carbon cycle} describes the movement of carbon
through Earth's atmosphere, land, water, and organisms.

\begin{center}
\textbf{Mermaid Diagram (Code)}
\begin{verbatim}
{Shaded}
{Highlighting}[]
graph LR
    A[Atmospheric CO₂] {-{-}{} B[Photosynthesis]}
    B {-{-}{} C[Plant Biomass]}
    C {-{-}{} D[Animal Consumption]}
    D {-{-}{} A}
    C {-{-}{} E[Decomposition]}
    E {-{-}{} A}
    F[Fossil Fuels] {-{-}{} A}
    A {-{-}{} G[Ocean Absorption]}
    G {-{-}{} H[Marine Life]}
{Highlighting}
{Shaded}
\end{verbatim}
\end{center}

\textbf{Processes:}

\begin{itemize}
\tightlist
\item
  \textbf{Photosynthesis}: CO₂ absorption by plants
\item
  \textbf{Respiration}: CO₂ release by organisms
\item
  \textbf{Decomposition}: Carbon return to atmosphere
\item
  \textbf{Ocean exchange}: CO₂ dissolution in seawater
\end{itemize}

\end{solutionbox}
\begin{mnemonicbox}
``Plants Breathe, Die, Ocean'' (PBDO)

\subsubsection{OR}\label{or-2}

\subsubsection{\texorpdfstring{\textbf{Draw and explain the hydrologic
cycle}}{Draw and explain the hydrologic cycle}}\label{draw-and-explain-the-hydrologic-cycle}

\end{mnemonicbox}
\begin{solutionbox}
\textbf{Hydrologic cycle} is the continuous movement of
water through atmosphere, land, and oceans.

\begin{center}
\textbf{Mermaid Diagram (Code)}
\begin{verbatim}
{Shaded}
{Highlighting}[]
graph LR
    A[Ocean] {-{-}{} B[Evaporation]}
    B {-{-}{} C[Water Vapor]}
    C {-{-}{} D[Condensation]}
    D {-{-}{} E[Clouds]}
    E {-{-}{} F[Precipitation]}
    F {-{-}{} G[Surface Runoff]}
    F {-{-}{} H[Infiltration]}
    G {-{-}{} A}
    H {-{-}{} I[Groundwater]}
    I {-{-}{} A}
{Highlighting}
{Shaded}
\end{verbatim}
\end{center}

\textbf{Processes:}

\begin{itemize}
\tightlist
\item
  \textbf{Evaporation}: Water to vapor conversion
\item
  \textbf{Condensation}: Vapor to liquid conversion
\item
  \textbf{Precipitation}: Rain, snow formation
\item
  \textbf{Infiltration}: Groundwater recharge
\end{itemize}

\end{solutionbox}
\begin{mnemonicbox}
``Every Cloud Produces Rain'' (ECPR)

\end{mnemonicbox}
\subsection*{Question 2(d) {[}4 marks{]}}\label{question-2d-4-marks}

\subsubsection{\texorpdfstring{\textbf{Enlist equipments used to control
air pollution and explain any
one.}}{Enlist equipments used to control air pollution and explain any one.}}\label{enlist-equipments-used-to-control-air-pollution-and-explain-any-one.}

\begin{solutionbox}
Air pollution control equipment removes pollutants from
industrial emissions.

\textbf{Equipment List:}

\begin{itemize}
\tightlist
\item
  \textbf{Cyclone separators}: Particulate removal
\item
  \textbf{Electrostatic precipitators}: Fine particle collection
\item
  \textbf{Bag filters}: Fabric filtration
\item
  \textbf{Scrubbers}: Gas absorption
\end{itemize}

\textbf{Electrostatic Precipitator:}

\begin{verbatim}
   +{-{-}{-}{-}{-}{-}{-}{-}{-}{-}+}
   |    +     |  High voltage electrode
   |    |     |
   |    v     |
   |  Dust    |  Charged particles
   |  +{-+{-}+   |}
   |  |   |   |
   |  v   v   |
   +{-{-}+{-}{-}{-}+{-}{-}{-}+  Collection plate ({-})}
      Clean gas out
\end{verbatim}

\begin{itemize}
\tightlist
\item
  \textbf{Charging}: Particles acquire electric charge
\item
  \textbf{Collection}: Charged particles attracted to plates
\item
  \textbf{Efficiency}: 99\% removal of fine particles
\end{itemize}

\end{solutionbox}
\begin{mnemonicbox}
``Charge Collect Clean'' (CCC)

\subsubsection{OR}\label{or-3}

\subsubsection{\texorpdfstring{\textbf{Enlist the types of environmental
pollution and give the effects of noise
pollution}}{Enlist the types of environmental pollution and give the effects of noise pollution}}\label{enlist-the-types-of-environmental-pollution-and-give-the-effects-of-noise-pollution}

\end{mnemonicbox}
\begin{solutionbox}
\textbf{Environmental pollution types:}

\begin{itemize}
\tightlist
\item
  \textbf{Air pollution}: Atmospheric contamination
\item
  \textbf{Water pollution}: Aquatic contamination
\item
  \textbf{Soil pollution}: Land contamination
\item
  \textbf{Noise pollution}: Sound contamination
\end{itemize}

\textbf{Noise Pollution Effects:}

\begin{itemize}
\tightlist
\item
  \textbf{Health effects}: Hearing loss, stress, hypertension
\item
  \textbf{Psychological effects}: Irritation, sleep disturbance
\item
  \textbf{Performance effects}: Reduced concentration, productivity
\item
  \textbf{Communication effects}: Speech interference
\end{itemize}


{\def\LTcaptype{none} % do not increment counter
\begin{longtable}[]{@{}lll@{}}
\toprule\noalign{}
Effect Type & Symptoms & Impact \\
\midrule\noalign{}
\endhead
\bottomrule\noalign{}
\endlastfoot
Physical & Hearing damage & Permanent loss \\
Mental & Stress, anxiety & Health issues \\
Social & Communication problems & Relationship strain \\
\end{longtable}
}

\end{solutionbox}
\begin{mnemonicbox}
``Air Water Soil Sound'' (AWSS)

\end{mnemonicbox}
\subsection*{Question 3(a) {[}3 marks{]}}\label{question-3a-3-marks}

\subsubsection{\texorpdfstring{\textbf{What is e-waste? Give effects of
e-waste on environment and
humans.}}{What is e-waste? Give effects of e-waste on environment and humans.}}\label{what-is-e-waste-give-effects-of-e-waste-on-environment-and-humans.}

\begin{solutionbox}
\textbf{E-waste} (Electronic waste) consists of
discarded electrical and electronic devices containing hazardous
materials.

\textbf{Environmental Effects:}

\begin{itemize}
\tightlist
\item
  \textbf{Soil contamination}: Heavy metals leaching
\item
  \textbf{Water pollution}: Toxic chemical runoff
\item
  \textbf{Air pollution}: Burning releases toxic fumes
\end{itemize}

\textbf{Human Effects:}

\begin{itemize}
\tightlist
\item
  \textbf{Health hazards}: Lead, mercury poisoning
\item
  \textbf{Respiratory problems}: Toxic gas inhalation
\item
  \textbf{Skin disorders}: Direct contact with chemicals
\end{itemize}


{\def\LTcaptype{none} % do not increment counter
\begin{longtable}[]{@{}lll@{}}
\toprule\noalign{}
Component & Hazard & Impact \\
\midrule\noalign{}
\endhead
\bottomrule\noalign{}
\endlastfoot
Lead & Neurotoxin & Brain damage \\
Mercury & Toxic metal & Kidney damage \\
Cadmium & Carcinogen & Cancer risk \\
\end{longtable}
}

\end{solutionbox}
\begin{mnemonicbox}
``Electronic Equipment Endangers Everyone'' (E4)

\subsubsection{OR}\label{or-4}

\subsubsection{\texorpdfstring{\textbf{What is plastic waste? Give
effects of plastic
waste.}}{What is plastic waste? Give effects of plastic waste.}}\label{what-is-plastic-waste-give-effects-of-plastic-waste.}

\end{mnemonicbox}
\begin{solutionbox}
\textbf{Plastic waste} consists of discarded plastic
materials that persist in environment due to non-biodegradable nature.

\textbf{Effects:}

\begin{itemize}
\tightlist
\item
  \textbf{Marine pollution}: Ocean plastic accumulation
\item
  \textbf{Wildlife impact}: Entanglement, ingestion by animals
\item
  \textbf{Soil degradation}: Reduced fertility and water infiltration
\item
  \textbf{Human health}: Microplastics in food chain
\end{itemize}

\textbf{Categories:}

\begin{itemize}
\tightlist
\item
  \textbf{Single-use plastics}: Bags, bottles, straws
\item
  \textbf{Packaging waste}: Food containers, wrappings
\item
  \textbf{Industrial plastic}: Manufacturing waste
\end{itemize}

\end{solutionbox}
\begin{mnemonicbox}
``Plastic Persists, Problems Persist'' (PPPP)

\end{mnemonicbox}
\subsection*{Question 3(b) {[}3 marks{]}}\label{question-3b-3-marks}

\subsubsection{\texorpdfstring{\textbf{Give main sources of solid
waste.}}{Give main sources of solid waste.}}\label{give-main-sources-of-solid-waste.}

\begin{solutionbox}
\textbf{Solid waste} originates from various human
activities and natural processes.

\textbf{Sources:}

\begin{itemize}
\tightlist
\item
  \textbf{Residential}: Household garbage, food waste
\item
  \textbf{Commercial}: Office waste, packaging materials
\item
  \textbf{Industrial}: Manufacturing waste, chemicals
\item
  \textbf{Agricultural}: Crop residues, animal waste
\item
  \textbf{Municipal}: Street sweeping, park maintenance
\end{itemize}


{\def\LTcaptype{none} % do not increment counter
\begin{longtable}[]{@{}lll@{}}
\toprule\noalign{}
Source & Waste Type & Management \\
\midrule\noalign{}
\endhead
\bottomrule\noalign{}
\endlastfoot
Domestic & Organic, Plastic & Collection \\
Industrial & Hazardous, Non-hazardous & Treatment \\
Agricultural & Biodegradable & Composting \\
\end{longtable}
}

\end{solutionbox}
\begin{mnemonicbox}
``Residential Commercial Industrial Agricultural
Municipal'' (RCIAM)

\subsubsection{OR}\label{or-5}

\subsubsection{\texorpdfstring{\textbf{Enlist various methods of solid
waste disposal and explain any
one.}}{Enlist various methods of solid waste disposal and explain any one.}}\label{enlist-various-methods-of-solid-waste-disposal-and-explain-any-one.}

\end{mnemonicbox}
\begin{solutionbox}
\textbf{Disposal Methods:}

\begin{itemize}
\tightlist
\item
  \textbf{Landfilling}: Controlled waste burial
\item
  \textbf{Incineration}: Waste burning with energy recovery
\item
  \textbf{Composting}: Organic waste decomposition
\item
  \textbf{Recycling}: Material recovery and reuse
\end{itemize}

\textbf{Sanitary Landfill:}

\begin{verbatim}
    Daily cover
    +{-{-}{-}{-}{-}{-}{-}{-}{-}{-}+}
    |  Waste   |  Compacted layers
    +{-{-}{-}{-}{-}{-}{-}{-}{-}{-}+  }
    |   Clay   |  Liner system
    +{-{-}{-}{-}{-}{-}{-}{-}{-}{-}+}
    | Drainage |  Leachate collection
    +{-{-}{-}{-}{-}{-}{-}{-}{-}{-}+}
\end{verbatim}

\begin{itemize}
\tightlist
\item
  \textbf{Design}: Engineered system with liners
\item
  \textbf{Operation}: Daily cover, compaction
\item
  \textbf{Environmental protection}: Leachate and gas control
\end{itemize}

\end{solutionbox}
\begin{mnemonicbox}
``Land Incinerate Compost Recycle'' (LICR)

\end{mnemonicbox}
\subsection*{Question 3(c) {[}4 marks{]}}\label{question-3c-4-marks}

\subsubsection{\texorpdfstring{\textbf{Explain the working of Liquid
Flat Plate Collector with a neat
sketch.}}{Explain the working of Liquid Flat Plate Collector with a neat sketch.}}\label{explain-the-working-of-liquid-flat-plate-collector-with-a-neat-sketch.}

\begin{solutionbox}
\textbf{Liquid Flat Plate Collector} converts solar
radiation into thermal energy for heating water.

\begin{verbatim}
    Glass cover
    +================+
    |  {  |  Air gap}
    +================+
    |  ||||||||||||  |  Absorber plate (black)
    |  [{-{-}{-}{-}{-}{-}{-}{-}{-}{-}]  |  Fluid tubes}
    +================+
    |  Insulation    |  Back insulation
    +================+
         \^{      \^{}}
    Cold water  Hot water
    inlet       outlet
\end{verbatim}

\textbf{Working:}

\begin{itemize}
\tightlist
\item
  \textbf{Solar absorption}: Black absorber plate captures solar energy
\item
  \textbf{Heat transfer}: Absorbed heat transfers to flowing liquid
\item
  \textbf{Circulation}: Heated liquid rises, cool liquid enters
\item
  \textbf{Insulation}: Minimizes heat losses
\end{itemize}

\textbf{Components:}

\begin{itemize}
\tightlist
\item
  \textbf{Transparent cover}: Reduces convection losses
\item
  \textbf{Absorber plate}: Maximum solar absorption
\item
  \textbf{Heat transfer fluid}: Water or antifreeze solution
\end{itemize}

\end{solutionbox}
\begin{mnemonicbox}
``Solar Absorption Creates Heat Transfer'' (SACHT)

\subsubsection{OR}\label{or-6}

\subsubsection{\texorpdfstring{\textbf{Write short note on solar
pond}}{Write short note on solar pond}}\label{write-short-note-on-solar-pond}

\end{mnemonicbox}
\begin{solutionbox}
\textbf{Solar pond} is a pool of saltwater that acts as
both solar collector and thermal storage system.

\textbf{Structure:}

\begin{itemize}
\tightlist
\item
  \textbf{Upper zone}: Low salt concentration
\item
  \textbf{Middle zone}: Increasing salt gradient
\item
  \textbf{Lower zone}: High salt concentration
\end{itemize}

\textbf{Working:}

\begin{itemize}
\tightlist
\item
  \textbf{Density gradient}: Prevents convection mixing
\item
  \textbf{Heat storage}: Bottom layer stores thermal energy
\item
  \textbf{Temperature}: Can reach 70-85°C at bottom
\end{itemize}

\textbf{Applications:}

\begin{itemize}
\tightlist
\item
  \textbf{Power generation}: Steam production
\item
  \textbf{Industrial heating}: Process heat supply
\item
  \textbf{Desalination}: Water purification
\end{itemize}

\end{solutionbox}
\begin{mnemonicbox}
``Salt Stores Solar Thermal'' (SSST)

\end{mnemonicbox}
\subsection*{Question 3(d) {[}4 marks{]}}\label{question-3d-4-marks}

\subsubsection{\texorpdfstring{\textbf{Explain Savonious wind mill with
a neat
sketch.}}{Explain Savonious wind mill with a neat sketch.}}\label{explain-savonious-wind-mill-with-a-neat-sketch.}

\begin{solutionbox}
\textbf{Savonius wind turbine} is a vertical axis wind
turbine with S-shaped rotor blades.

\begin{verbatim}
    Wind direction →
           |
       +{-{-}{-}{-}{-}{-}{-}+}
       |   S   |  S{-shaped blade}
       |  {-{-}{-}  |  }
       |       |
       +{-{-}{-}{-}{-}{-}{-}+}
           |
       Generator
\end{verbatim}

\textbf{Working:}

\begin{itemize}
\tightlist
\item
  \textbf{Drag principle}: Wind creates differential drag on blades
\item
  \textbf{Rotation}: S-shape causes continuous rotation
\item
  \textbf{Self-starting}: Starts at low wind speeds
\item
  \textbf{Vertical axis}: Independent of wind direction
\end{itemize}

\textbf{Advantages:}

\begin{itemize}
\tightlist
\item
  \textbf{Simple design}: Low maintenance requirements
\item
  \textbf{Low noise}: Quiet operation
\item
  \textbf{All wind directions}: Omnidirectional capability
\end{itemize}

\textbf{Disadvantages:}

\begin{itemize}
\tightlist
\item
  \textbf{Lower efficiency}: 20-30\% compared to HAWT
\item
  \textbf{Space requirement}: Larger area needed
\end{itemize}

\end{solutionbox}
\begin{mnemonicbox}
``S-Shape Starts Slowly'' (SSS)

\subsubsection{OR}\label{or-7}

\subsubsection{\texorpdfstring{\textbf{Give the comparison between
Horizontal Axis and Vertical Axis wind
mills.}}{Give the comparison between Horizontal Axis and Vertical Axis wind mills.}}\label{give-the-comparison-between-horizontal-axis-and-vertical-axis-wind-mills.}

\end{mnemonicbox}
\begin{solutionbox}
Wind turbines are classified based on rotor axis
orientation.

\textbf{Comparison Table:}

{\def\LTcaptype{none} % do not increment counter
\begin{longtable}[]{@{}lll@{}}
\toprule\noalign{}
Parameter & Horizontal Axis (HAWT) & Vertical Axis (VAWT) \\
\midrule\noalign{}
\endhead
\bottomrule\noalign{}
\endlastfoot
Efficiency & 35-45\% & 20-30\% \\
Wind direction & Must face wind & Any direction \\
Installation & Tower required & Ground level possible \\
Maintenance & Difficult access & Easy access \\
Noise & Higher & Lower \\
Cost & Higher & Lower \\
\end{longtable}
}

\textbf{HAWT Features:}

\begin{itemize}
\tightlist
\item
  \textbf{Upwind design}: Rotor faces wind
\item
  \textbf{Pitch control}: Blade angle adjustment
\item
  \textbf{Yaw system}: Wind direction tracking
\end{itemize}

\textbf{VAWT Features:}

\begin{itemize}
\tightlist
\item
  \textbf{Omnidirectional}: No wind tracking needed
\item
  \textbf{Ground installation}: Easier maintenance
\item
  \textbf{Lower wind speeds}: Better performance
\end{itemize}

\end{solutionbox}
\begin{mnemonicbox}
``Horizontal High, Vertical Versatile'' (HHVV)

\end{mnemonicbox}
\subsection*{Question 4(a) {[}3 marks{]}}\label{question-4a-3-marks}

\subsubsection{\texorpdfstring{\textbf{Give effects of climate
change.}}{Give effects of climate change.}}\label{give-effects-of-climate-change.}

\begin{solutionbox}
\textbf{Climate change} causes widespread environmental
and socio-economic impacts globally.

\textbf{Environmental Effects:}

\begin{itemize}
\tightlist
\item
  \textbf{Temperature rise}: Global average increase
\item
  \textbf{Sea level rise}: Thermal expansion and ice melting
\item
  \textbf{Weather extremes}: Intense storms, droughts, floods
\item
  \textbf{Ecosystem shifts}: Species migration and extinction
\end{itemize}

\textbf{Socio-economic Effects:}

\begin{itemize}
\tightlist
\item
  \textbf{Agricultural impact}: Crop yield changes
\item
  \textbf{Water resources}: Availability and quality issues
\item
  \textbf{Human health}: Heat stress, disease spread
\item
  \textbf{Economic losses}: Infrastructure damage
\end{itemize}


{\def\LTcaptype{none} % do not increment counter
\begin{longtable}[]{@{}lll@{}}
\toprule\noalign{}
Impact Category & Examples & Severity \\
\midrule\noalign{}
\endhead
\bottomrule\noalign{}
\endlastfoot
Environmental & Melting glaciers & High \\
Agricultural & Crop failure & Medium \\
Health & Heat waves & High \\
\end{longtable}
}

\end{solutionbox}
\begin{mnemonicbox}
``Temperature Sea Weather Ecosystem'' (TSWE)

\subsubsection{OR}\label{or-8}

\subsubsection{\texorpdfstring{\textbf{Write a short note on Green House
gases.}}{Write a short note on Green House gases.}}\label{write-a-short-note-on-green-house-gases.}

\end{mnemonicbox}
\begin{solutionbox}
\textbf{Greenhouse gases} trap heat in Earth's
atmosphere, causing global warming through greenhouse effect.

\textbf{Major Greenhouse Gases:}

\begin{itemize}
\tightlist
\item
  \textbf{Carbon dioxide (CO₂)}: 76\% of emissions
\item
  \textbf{Methane (CH₄)}: 16\% of emissions
\item
  \textbf{Nitrous oxide (N₂O)}: 6\% of emissions
\item
  \textbf{Fluorinated gases}: 2\% of emissions
\end{itemize}

\textbf{Sources:}

\begin{itemize}
\tightlist
\item
  \textbf{CO₂}: Fossil fuel burning, deforestation
\item
  \textbf{CH₄}: Agriculture, landfills, livestock
\item
  \textbf{N₂O}: Fertilizers, fossil fuel combustion
\end{itemize}

\textbf{Global Warming Potential:}

\begin{itemize}
\tightlist
\item
  \textbf{CO₂}: Reference (GWP = 1)
\item
  \textbf{CH₄}: 25 times CO₂
\item
  \textbf{N₂O}: 298 times CO₂
\end{itemize}

\end{solutionbox}
\begin{mnemonicbox}
``Carbon Methane Nitrous Fluorine'' (CMNF)

\end{mnemonicbox}
\subsection*{Question 4(b) {[}4 marks{]}}\label{question-4b-4-marks}

\subsubsection{\texorpdfstring{\textbf{Explain climate change
Management.}}{Explain climate change Management.}}\label{explain-climate-change-management.}

\begin{solutionbox}
\textbf{Climate change management} involves strategies
to reduce greenhouse gas emissions and adapt to climate impacts.

\textbf{Mitigation Strategies:}

\begin{itemize}
\tightlist
\item
  \textbf{Renewable energy}: Solar, wind, hydroelectric power
\item
  \textbf{Energy efficiency}: Improved building designs, LED lighting
\item
  \textbf{Carbon sequestration}: Forest conservation, tree planting
\item
  \textbf{Sustainable transport}: Electric vehicles, public transport
\end{itemize}

\textbf{Adaptation Strategies:}

\begin{itemize}
\tightlist
\item
  \textbf{Infrastructure resilience}: Flood defenses, drought-resistant
  crops
\item
  \textbf{Water management}: Rainwater harvesting, efficient irrigation
\item
  \textbf{Coastal protection}: Sea walls, mangrove restoration
\item
  \textbf{Emergency preparedness}: Early warning systems
\end{itemize}

\textbf{Policy Measures:}

\begin{itemize}
\tightlist
\item
  \textbf{Carbon pricing}: Tax on emissions
\item
  \textbf{Renewable energy targets}: Clean energy goals
\item
  \textbf{Building codes}: Energy efficiency standards
\end{itemize}

\end{solutionbox}
\begin{mnemonicbox}
``Mitigation Adaptation Policy'' (MAP)

\subsubsection{OR}\label{or-9}

\subsubsection{\texorpdfstring{\textbf{Give effects of ozone layer
depletion.}}{Give effects of ozone layer depletion.}}\label{give-effects-of-ozone-layer-depletion.}

\end{mnemonicbox}
\begin{solutionbox}
\textbf{Ozone layer depletion} reduces stratospheric
ozone, allowing harmful UV radiation to reach Earth.

\textbf{Effects on Humans:}

\begin{itemize}
\tightlist
\item
  \textbf{Skin cancer}: Increased UV-B radiation exposure
\item
  \textbf{Eye cataracts}: UV damage to eye lens
\item
  \textbf{Immune suppression}: Weakened immune system
\item
  \textbf{Premature aging}: Skin damage acceleration
\end{itemize}

\textbf{Effects on Environment:}

\begin{itemize}
\tightlist
\item
  \textbf{Crop damage}: Reduced agricultural productivity
\item
  \textbf{Marine ecosystem}: Phytoplankton reduction
\item
  \textbf{Material degradation}: Plastic and rubber damage
\item
  \textbf{Climate change}: Ozone as greenhouse gas
\end{itemize}


{\def\LTcaptype{none} % do not increment counter
\begin{longtable}[]{@{}lll@{}}
\toprule\noalign{}
UV Type & Wavelength & Effect \\
\midrule\noalign{}
\endhead
\bottomrule\noalign{}
\endlastfoot
UV-A & 320-400 nm & Skin aging \\
UV-B & 280-320 nm & Sunburn, cancer \\
UV-C & 200-280 nm & Blocked by ozone \\
\end{longtable}
}

\end{solutionbox}
\begin{mnemonicbox}
``Skin Eyes Immunity Climate'' (SEIC)

\end{mnemonicbox}
\subsection*{Question 4(c) {[}7 marks{]}}\label{question-4c-7-marks}

\subsubsection{\texorpdfstring{\textbf{Explain biogas plant with
sketch.}}{Explain biogas plant with sketch.}}\label{explain-biogas-plant-with-sketch.}

\begin{solutionbox}
\textbf{Biogas plant} produces methane-rich gas through
anaerobic digestion of organic waste.

\begin{verbatim}
         Gas outlet
             |
    +{-{-}{-}{-}{-}{-}{-}{-}{-}{-}{-}{-}{-}{-}{-}{-}+}
    |    Gas dome    |  Gas collection
    +{-{-}{-}{-}{-}{-}{-}{-}{-}{-}{-}{-}{-}{-}{-}{-}+}
    |                |
    |   Slurry       |  Digester tank
    |  (Anaerobic)   |  
    |                |
    +{-{-}{-}{-}{-}{-}{-}+{-}{-}{-}{-}{-}{-}{-}{-}+}
    Inlet   |        Outlet
    pipe    |        pipe
           /|{}
          / | {}
       Waste  Slurry
       input  outlet
\end{verbatim}

\textbf{Components:}

\begin{itemize}
\tightlist
\item
  \textbf{Digester tank}: Anaerobic fermentation chamber
\item
  \textbf{Gas dome}: Biogas collection and storage
\item
  \textbf{Inlet pipe}: Waste material feeding
\item
  \textbf{Outlet pipe}: Digested slurry removal
\end{itemize}

\textbf{Process:}

\begin{itemize}
\tightlist
\item
  \textbf{Hydrolysis}: Complex organics break down
\item
  \textbf{Acidogenesis}: Acid-forming bacteria action
\item
  \textbf{Methanogenesis}: Methane-producing bacteria
\item
  \textbf{Gas production}: 50-70\% methane, 30-40\% CO₂
\end{itemize}

\textbf{Operating Conditions:}

\begin{itemize}
\tightlist
\item
  \textbf{Temperature}: 30-40°C optimal
\item
  \textbf{pH}: 6.8-7.2 range
\item
  \textbf{Retention time}: 15-30 days
\item
  \textbf{C:N ratio}: 20-30:1 optimal
\end{itemize}

\textbf{Applications:}

\begin{itemize}
\tightlist
\item
  \textbf{Cooking fuel}: Household energy needs
\item
  \textbf{Lighting}: Gas lamp illumination
\item
  \textbf{Electricity}: Generator power
\item
  \textbf{Fertilizer}: Nutrient-rich slurry
\end{itemize}

\textbf{Advantages:}

\begin{itemize}
\tightlist
\item
  \textbf{Renewable energy}: Sustainable fuel source
\item
  \textbf{Waste management}: Organic waste utilization
\item
  \textbf{Environmental benefits}: Reduced methane emissions
\item
  \textbf{Economic benefits}: Cost savings on fuel
\end{itemize}

\end{solutionbox}
\begin{mnemonicbox}
``Biogas Benefits: Renewable Waste Environment
Economy'' (BRWEE)

\end{mnemonicbox}
\subsection*{Question 5(a) {[}4 marks{]}}\label{question-5a-4-marks}

\subsubsection{\texorpdfstring{\textbf{Write short note on global
warming.}}{Write short note on global warming.}}\label{write-short-note-on-global-warming.}

\begin{solutionbox}
\textbf{Global warming} refers to long-term increase in
Earth's average surface temperature due to human activities.

\textbf{Causes:}

\begin{itemize}
\tightlist
\item
  \textbf{Greenhouse gases}: CO₂, CH₄, N₂O emissions
\item
  \textbf{Deforestation}: Reduced carbon absorption
\item
  \textbf{Industrial activities}: Fossil fuel combustion
\item
  \textbf{Transportation}: Vehicle emissions
\end{itemize}

\textbf{Effects:}

\begin{itemize}
\tightlist
\item
  \textbf{Temperature rise}: 1.1°C since pre-industrial times
\item
  \textbf{Ice melting}: Arctic sea ice, glaciers shrinking
\item
  \textbf{Sea level rise}: Coastal flooding threat
\item
  \textbf{Weather changes}: Extreme events frequency
\end{itemize}

\textbf{Evidence:}

\begin{itemize}
\tightlist
\item
  \textbf{Temperature records}: Warmest years in recent decades
\item
  \textbf{Ice core data}: Historical CO₂ levels
\item
  \textbf{Satellite measurements}: Global temperature monitoring
\end{itemize}

\textbf{Solutions:}

\begin{itemize}
\tightlist
\item
  \textbf{Renewable energy}: Clean power sources
\item
  \textbf{Energy efficiency}: Reduced consumption
\item
  \textbf{Carbon capture}: Technology development
\item
  \textbf{International cooperation}: Paris Agreement
\end{itemize}

\end{solutionbox}
\begin{mnemonicbox}
``Greenhouse Gases Generate Global Change'' (GGGC)

\end{mnemonicbox}
\subsection*{Question 5(b) {[}4 marks{]}}\label{question-5b-4-marks}

\subsubsection{\texorpdfstring{\textbf{Explain 5R
concept.}}{Explain 5R concept.}}\label{explain-5r-concept.}

\begin{solutionbox}
\textbf{5R concept} is waste management hierarchy for
sustainable resource utilization.

\begin{center}
\textbf{Mermaid Diagram (Code)}
\begin{verbatim}
{Shaded}
{Highlighting}[]
graph TD
    A[5R Hierarchy] {-{-}{} B[Refuse]}
    A {-{-}{} C[Reduce]}
    A {-{-}{} D[Reuse]}
    A {-{-}{} E[Repurpose]}
    A {-{-}{} F[Recycle]}
{Highlighting}
{Shaded}
\end{verbatim}
\end{center}

\textbf{The 5 R's:}

\textbf{1. Refuse:}

\begin{itemize}
\tightlist
\item
  \textbf{Avoid unnecessary items}: Say no to single-use products
\item
  \textbf{Examples}: Plastic bags, straws, excessive packaging
\end{itemize}

\textbf{2. Reduce:}

\begin{itemize}
\tightlist
\item
  \textbf{Minimize consumption}: Use less resources
\item
  \textbf{Examples}: Energy conservation, water saving
\end{itemize}

\textbf{3. Reuse:}

\begin{itemize}
\tightlist
\item
  \textbf{Multiple use}: Extend product life
\item
  \textbf{Examples}: Glass jars as containers, paper both sides
\end{itemize}

\textbf{4. Repurpose:}

\begin{itemize}
\tightlist
\item
  \textbf{Creative reuse}: New function for old items
\item
  \textbf{Examples}: Tire planters, bottle bird feeders
\end{itemize}

\textbf{5. Recycle:}

\begin{itemize}
\tightlist
\item
  \textbf{Material recovery}: Process into new products
\item
  \textbf{Examples}: Paper, plastic, metal recycling
\end{itemize}

\textbf{Benefits:}

\begin{itemize}
\tightlist
\item
  \textbf{Waste reduction}: Less landfill burden
\item
  \textbf{Resource conservation}: Natural resource preservation
\item
  \textbf{Cost savings}: Economic benefits
\item
  \textbf{Environmental protection}: Pollution reduction
\end{itemize}

\end{solutionbox}
\begin{mnemonicbox}
``Refuse Reduce Reuse Repurpose Recycle'' (R5)

\end{mnemonicbox}
\subsection*{Question 5(c) {[}3 marks{]}}\label{question-5c-3-marks}

\subsubsection{\texorpdfstring{\textbf{Explain the benefits of Green
building.}}{Explain the benefits of Green building.}}\label{explain-the-benefits-of-green-building.}

\begin{solutionbox}
\textbf{Green building} incorporates sustainable design
and construction practices for environmental and human benefits.

\textbf{Environmental Benefits:}

\begin{itemize}
\tightlist
\item
  \textbf{Energy efficiency}: Reduced power consumption
\item
  \textbf{Water conservation}: Efficient water systems
\item
  \textbf{Waste reduction}: Construction and operational waste
  minimization
\end{itemize}

\textbf{Economic Benefits:}

\begin{itemize}
\tightlist
\item
  \textbf{Operating cost savings}: Lower utility bills
\item
  \textbf{Increased property value}: Market premium
\item
  \textbf{Tax incentives}: Government rebates
\end{itemize}

\textbf{Health Benefits:}

\begin{itemize}
\tightlist
\item
  \textbf{Indoor air quality}: Better ventilation systems
\item
  \textbf{Natural lighting}: Improved occupant comfort
\item
  \textbf{Toxic material reduction}: Healthier environment
\end{itemize}


{\def\LTcaptype{none} % do not increment counter
\begin{longtable}[]{@{}lll@{}}
\toprule\noalign{}
Benefit Type & Examples & Impact \\
\midrule\noalign{}
\endhead
\bottomrule\noalign{}
\endlastfoot
Environmental & Energy saving & 30-50\% reduction \\
Economic & Cost savings & 20\% operating costs \\
Health & Air quality & Productivity increase \\
\end{longtable}
}

\end{solutionbox}
\begin{mnemonicbox}
``Green Buildings Give Environmental Economic
Health'' (GBEEH)

\end{mnemonicbox}
\subsection*{Question 5(d) {[}3 marks{]}}\label{question-5d-3-marks}

\subsubsection{\texorpdfstring{\textbf{Enlist various Acts related to
environment in India and explain any
one.}}{Enlist various Acts related to environment in India and explain any one.}}\label{enlist-various-acts-related-to-environment-in-india-and-explain-any-one.}

\begin{solutionbox}
\textbf{Environmental Acts in India:}

\begin{itemize}
\tightlist
\item
  \textbf{Water (Prevention and Control of Pollution) Act, 1974}
\item
  \textbf{Air (Prevention and Control of Pollution) Act, 1981}
\item
  \textbf{Environment Protection Act, 1986}
\item
  \textbf{Wildlife Protection Act, 1972}
\item
  \textbf{Forest (Conservation) Act, 1980}
\item
  \textbf{Biodiversity Act, 2002}
\end{itemize}

\textbf{Environment Protection Act, 1986:} \textbf{Objectives:}

\begin{itemize}
\tightlist
\item
  \textbf{Comprehensive framework}: Overall environmental protection
\item
  \textbf{Pollution prevention}: Air, water, soil contamination control
\item
  \textbf{Standard setting}: Environmental quality standards
\item
  \textbf{Enforcement}: Penalties for violations
\end{itemize}

\textbf{Powers:}

\begin{itemize}
\tightlist
\item
  \textbf{Central government authority}: Environmental regulations
\item
  \textbf{Inspection rights}: Industrial facilities monitoring
\item
  \textbf{Closure orders}: Non-compliant industries
\item
  \textbf{Emergency measures}: Environmental hazards response
\end{itemize}

\textbf{Significance:}

\begin{itemize}
\tightlist
\item
  \textbf{Umbrella legislation}: Covers all environmental aspects
\item
  \textbf{Post-Bhopal disaster}: Response to industrial accidents
\end{itemize}

\end{solutionbox}
\begin{mnemonicbox}
``Water Air Environment Wildlife Forest
Biodiversity'' (WAEWFB)

\end{mnemonicbox}
\end{document}