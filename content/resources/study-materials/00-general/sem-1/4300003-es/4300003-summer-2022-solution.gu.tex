\documentclass[10pt,a4paper]{article}

% content/resources/templates/preamble.tex
\usepackage[margin=0.6in]{geometry}
\author{Milav Dabgar}
\usepackage{amsmath,amssymb,amsthm}
\usepackage{booktabs}
\usepackage{multirow}
\usepackage{xcolor}
\usepackage{tcolorbox}
\tcbuselibrary{breakable,skins}
\usepackage[colorlinks=true,linkcolor=blue]{hyperref}
\usepackage{titlesec}
\usepackage{enumitem}
\usepackage{tikz}
\usepackage{pgfplots}
\usepackage{circuitikz}
\usepackage[version=4]{mhchem}
\usepackage{longtable}
\usepackage{array}
\usepackage{float}
\usepackage{caption}
\usepackage{listings}

\lstset{
  basicstyle=\small\ttfamily,
  breaklines=true,
  breakatwhitespace=false,
  postbreak=\mbox{\textcolor{red}{$\hookrightarrow$}\space},
  float=false,
  numbers=left,
  numberstyle=\tiny\color{gray},
  numbersep=10pt,
  xleftmargin=2em,
  keywordstyle=\color{blue},
  commentstyle=\color{green!60!black},
  stringstyle=\color{purple},
  backgroundcolor=\color{gray!5},
  showstringspaces=false,
  tabsize=2,
  captionpos=b,
  keepspaces=true,
  columns=flexible
}

\pgfplotsset{compat=1.18}
\usetikzlibrary{shapes,arrows,positioning,calc,patterns,decorations.pathmorphing,decorations.markings,arrows.meta}

% Color scheme
\definecolor{headcolor}{RGB}{0,102,204}
\definecolor{keycolor}{RGB}{220,20,60}
\definecolor{solutioncolor}{RGB}{34,139,34}
\definecolor{mnemoniccolor}{RGB}{148,0,211}
\definecolor{codecolor}{RGB}{0,0,100}

% Spacing
\setlength{\parskip}{3pt}
\setlist[itemize]{nosep}
\setlist[enumerate]{nosep}

% Title formatting
\titleformat{\section}{\Large\bfseries\color{headcolor}}{\thesection}{1em}{}
\titleformat{\subsection}{\large\bfseries\color{headcolor}}{\thesubsection}{1em}{}

% Pandoc tightlist compatibility
\providecommand{\tightlist}{%
  \setlength{\itemsep}{0pt}\setlength{\parskip}{0pt}}

% Pandoc longtable compatibility
\newcounter{none}
\def\thenone{}


% content/resources/templates/gujarati-boxes.tex
\usepackage{fontspec}
\usepackage{polyglossia}

% Set Gujarati as main language (document is primarily in Gujarati)
% Note: gloss-gujarati.ldf doesn't exist in polyglossia, but it will use hyphenation patterns
\setdefaultlanguage{gujarati}
\setotherlanguage{english}

% Configure Gujarati font properly
% Use Language=Default to prevent polyglossia from trying to add language-specific features
% that don't exist for Gujarati, which causes "empty feature" warnings
\newfontfamily\gujaratifont[Script=Gujarati,AutoFakeBold=2.5,AutoFakeSlant=0.3]{Noto Sans Gujarati}
\setmainfont[Script=Gujarati,AutoFakeBold=2.5,AutoFakeSlant=0.3]{Noto Sans Gujarati}
% Use Noto Sans Gujarati for monospace to support Gujarati in text
\setmonofont[Scale=0.9]{Noto Sans Gujarati}

% Configure English to use the same font
\newfontfamily\englishfont[Script=Gujarati,AutoFakeBold=2.5,AutoFakeSlant=0.3]{Noto Sans Gujarati}

% Translations for polyglossia
\gappto\captionsgujarati{
  \renewcommand{\tablename}{કોષ્ટક}
  \renewcommand{\figurename}{આકૃતિ}
}

% Helper for TikZ nodes to ensure Gujarati font
\newcommand{\gu}[1]{{\gujaratifont #1}}

% Custom environments
\newtcolorbox{solutionbox}{
    breakable,
    enhanced,
    colback=solutioncolor!5!white,
    colframe=solutioncolor!75!black,
    fonttitle=\bfseries,
    title=જવાબ
}

\newtcolorbox{solutionboxnobreak}{
 colback=solutioncolor!5!white,
 colframe=solutioncolor!75!black,
 fonttitle=\bfseries,
 title=જવાબ
}

\newtcolorbox{keyformula}{
 breakable,
 enhanced,
 colback=keycolor!5!white,
 colframe=keycolor!75!black,
 fonttitle=\bfseries,
 title=રાસાયણિક સમીકરણ/સૂત્ર
}

\newtcolorbox{mnemonicbox}{
 breakable,
 enhanced,
 colback=mnemoniccolor!5!white,
 colframe=mnemoniccolor!75!black,
 fonttitle=\bfseries,
 title=મેમરી ટ્રીક
}


\begin{document}

\begin{center}
{\Huge\bfseries\color{headcolor} Environment and Sustainability (Gujarati)}\\[5pt]
{\LARGE 4300003 -- Summer 2022}\\[3pt]
{\large Semester 1 Study Material}\\[3pt]
{\normalsize\textit{Detailed Solutions and Explanations}}
\end{center}

\vspace{10pt}

\subsection*{પ્રશ્ન 1(a) {[}3
ગુણ{]}}\label{uxaaauxab0uxab6uxaa8-1a-3-uxa97uxaa3}

\textbf{ટૂંકનોધ લખો: પારિસ્થિતિક પિરામિડ.}

\textbf{જવાબ}:

\textbf{કોષ્ટક: પારિસ્થિતિક પિરામિડના પ્રકારો}

{\def\LTcaptype{none} % do not increment counter
\begin{longtable}[]{@{}
  >{\raggedright\arraybackslash}p{(\linewidth - 4\tabcolsep) * \real{0.2143}}
  >{\raggedright\arraybackslash}p{(\linewidth - 4\tabcolsep) * \real{0.4643}}
  >{\raggedright\arraybackslash}p{(\linewidth - 4\tabcolsep) * \real{0.3214}}@{}}
\toprule\noalign{}
\begin{minipage}[b]{\linewidth}\raggedright
પ્રકાર
\end{minipage} & \begin{minipage}[b]{\linewidth}\raggedright
વર્ણન
\end{minipage} & \begin{minipage}[b]{\linewidth}\raggedright
ઉદાહરણ
\end{minipage} \\
\midrule\noalign{}
\endhead
\bottomrule\noalign{}
\endlastfoot
\textbf{સંખ્યાનો પિરામિડ} & દરેક સ્તરે જીવોની સંખ્યા દર્શાવે છે & વૃક્ષો → જંતુઓ →
પક્ષીઓ \\
\textbf{બાયોમાસ પિરામિડ} & જીવોનું કુલ દળ દર્શાવે છે & ઉત્પાદક સ્તરે વધુ \\
\textbf{ઊર્જા પિરામિડ} & સ્તરોમાં ઊર્જાનો પ્રવાહ દર્શાવે છે & હંમેશા સીધો \\
\end{longtable}
}

\begin{itemize}
\tightlist
\item
  \textbf{ઊર્જા સ્થાનાંતરણ}: માત્ર 10\% ઊર્જા આગલા સ્તરમાં જાય છે
\item
  \textbf{પોષક સ્તરો}: ઉત્પાદકો, પ્રાથમિક ઉપભોક્તાઓ, ગૌણ ઉપભોક્તાઓ
\item
  \textbf{હંમેશા સીધો}: ઊર્જા પિરામિડ ક્યારેય ઊંધો નથી થતો
\end{itemize}

\textbf{મેમરી ટ્રીક:} ``સંખ્યા-બાયોમાસ-ઊર્જા ઉપર વહે છે''

\subsection*{પ્રશ્ન 1(b) {[}4
ગુણ{]}}\label{uxaaauxab0uxab6uxaa8-1b-4-uxa97uxaa3}

\textbf{વૈશ્વિક પારિસ્થિતિકીય ઊછાળ વિશે ટૂંકનોધ લખો.}

\textbf{જવાબ}:

વૈશ્વિક પારિસ્થિતિકીય ઊછાળ એ ત્યારે થાય છે જ્યારે માનવતાની માંગ પૃથ્વીની પુનઃઉત્પાદન
ક્ષમતા કરતાં વધી જાય છે.

\textbf{મુખ્ય ઘટકો:}

{\def\LTcaptype{none} % do not increment counter
\begin{longtable}[]{@{}ll@{}}
\toprule\noalign{}
પરિબળ & વર્ણન \\
\midrule\noalign{}
\endhead
\bottomrule\noalign{}
\endlastfoot
\textbf{પૃથ્વી ઓવરશૂટ દિવસ} & જે દિવસે વાર્ષિક સંસાધન વપરાશ પુનઃઉત્પાદન કરતાં વધે
છે \\
\textbf{પારિસ્થિતિક પદચિહ્ન} & કુદરતી સંસાધનો પર માનવીય માંગ \\
\textbf{બાયોકેપેસિટી} & સંસાધનો પુનઃઉત્પન્ન કરવાની પૃથ્વીની ક્ષમતા \\
\end{longtable}
}

\begin{itemize}
\tightlist
\item
  \textbf{હાલની સ્થિતિ}: વાર્ષિક 1.7 પૃથ્વી જેટલા સંસાધનોનો ઉપયોગ
\item
  \textbf{પરિણામો}: હવામાન પરિવર્તન, જૈવવિવિધતા નુકસાન, સંસાધન અવક્ષય
\item
  \textbf{ઉકેલો}: ટકાઉ વપરાશ, નવીકરણીય ઊર્જા અપનાવવી
\end{itemize}

\textbf{મેમરી ટ્રીક:} ``માંગ પુરવઠા કરતાં વધારે = ઊછાળ''

\subsection*{પ્રશ્ન 1(c) {[}7
ગુણ{]}}\label{uxaaauxab0uxab6uxaa8-1c-7-uxa97uxaa3}

\textbf{જૈવ-ભૂરાસાયણિક ચક્ર કોને કહે છે? કોઇ પણ બે ચક્ર વિશે વિગત માટે જણાવો.}

\textbf{જવાબ}:

જૈવ-ભૂરાસાયણિક ચક્રો એ કુદરતી પ્રક્રિયાઓ છે જે જૈવિક અને અજૈવિક ઘટકો દ્વારા આવશ્યક
તત્વોને પુનર્ચક્રિત કરે છે.

\textbf{કાર્બન ચક્ર:}

\begin{center}
\textbf{Mermaid Diagram (Code)}
\begin{verbatim}
{Shaded}
{Highlighting}[]
graph LR
    A[વાતાવરણ CO2] {-{-}{} B[છોડ પ્રકાશસંશ્લેષણ]}
    B {-{-}{} C[પ્રાણીઓ શ્વસન]}
    C {-{-}{} A}
    B {-{-}{} D[વિઘટન]}
    D {-{-}{} A}
    A {-{-}{} E[સમુદ્ર શોષણ]}
    E {-{-}{} A}
{Highlighting}
{Shaded}
\end{verbatim}
\end{center}

\textbf{નાઇટ્રોજન ચક્ર:}

{\def\LTcaptype{none} % do not increment counter
\begin{longtable}[]{@{}lll@{}}
\toprule\noalign{}
તબક્કો & પ્રક્રિયા & જીવતંત્ર \\
\midrule\noalign{}
\endhead
\bottomrule\noalign{}
\endlastfoot
\textbf{નાઇટ્રોજન સ્થિરીકરણ} & N2 → NH3 & રાયઝોબિયમ બેક્ટેરિયા \\
\textbf{નાઇટ્રિફિકેશન} & NH3 → NO3 & નાઇટ્રોસોમોનાસ, નાઇટ્રોબેક્ટર \\
\textbf{ડિનાઇટ્રિફિકેશન} & NO3 → N2 & ડિનાઇટ્રિફાઇંગ બેક્ટેરિયા \\
\end{longtable}
}

\begin{itemize}
\tightlist
\item
  \textbf{મહત્વ}: પ્રોટીન સંશ્લેષણ અને DNA રચના માટે આવશ્યક
\item
  \textbf{માનવીય અસર}: ખાતરો કુદરતી સંતુલન વિખેરે છે
\item
  \textbf{સંરક્ષણ}: રાસાયણિક ખાતરનો ઉપયોગ ઘટાડવો
\end{itemize}

\textbf{મેમરી ટ્રીક:} ``બેક્ટેરિયા નાઇટ્રોજન ઠીક કરે છે, છોડ વાપરે છે''

\subsection*{પ્રશ્ન 1(c) OR {[}7
ગુણ{]}}\label{uxaaauxab0uxab6uxaa8-1c-or-7-uxa97uxaa3}

\textbf{જંગલના નિસર્ગતંત્ર વિશે વિગત માટે જણાવો. વનનાશીકરણની અસરકારક પરિબળો અને
જંગલના નિસર્ગતંત્રનું સંરક્ષણ માટેના પરિબળો સમજાવો.}

\textbf{જવાબ}:

\textbf{જંગલ નિસર્ગતંત્રના ઘટકો:}

{\def\LTcaptype{none} % do not increment counter
\begin{longtable}[]{@{}ll@{}}
\toprule\noalign{}
ઘટક & ઉદાહરણો \\
\midrule\noalign{}
\endhead
\bottomrule\noalign{}
\endlastfoot
\textbf{ઉત્પાદકો} & વૃક્ષો, ઝાડીઓ, ઔષધીઓ \\
\textbf{પ્રાથમિક ઉપભોક્તાઓ} & હરણ, સસલાં, જંતુઓ \\
\textbf{ગૌણ ઉપભોક્તાઓ} & માંસાહારીઓ, પક્ષીઓ \\
\textbf{વિઘટકો} & બેક્ટેરિયા, ફૂગ \\
\end{longtable}
}

\textbf{વનનાશીકરણની અસરો:}

\begin{center}
\textbf{Mermaid Diagram (Code)}
\begin{verbatim}
{Shaded}
{Highlighting}[]
graph TD
    A[વનનાશીકરણ] {-{-}{} B[હવામાન પરિવર્તન]}
    A {-{-}{} C[જૈવવિવિધતા નુકસાન]}
    A {-{-}{} D[જમીન ધોવાણ]}
    A {-{-}{} E[જળચક્ર વિક્ષેપ]}
{Highlighting}
{Shaded}
\end{verbatim}
\end{center}

\textbf{સંરક્ષણ પદ્ધતિઓ:}

\begin{itemize}
\tightlist
\item
  \textbf{વનીકરણ}: નવા વિસ્તારોમાં વૃક્ષો લગાવવા
\item
  \textbf{પુનર્વનીકરણ}: વન નષ્ટ થયેલા વિસ્તારોમાં વૃક્ષો લગાવવા
\item
  \textbf{સંરક્ષિત વિસ્તારો}: રાષ્ટ્રીય ઉદ્યાનો અને અભયારણ્યો
\item
  \textbf{ટકાઉ કાપણી}: નિયંત્રિત લાકડા કાપણી પ્રથાઓ
\end{itemize}

\textbf{મેમરી ટ્રીક:} ``લગાવો, સંરક્ષિત કરો, ટકાઉપણાનો અભ્યાસ કરો''

\subsection*{પ્રશ્ન 2(a) {[}3
ગુણ{]}}\label{uxaaauxab0uxab6uxaa8-2a-3-uxa97uxaa3}

\textbf{પ્રદૂષણ અને પ્રદૂષક ની વ્યાખ્યા આપો.}

\textbf{જવાબ}:

\textbf{વ્યાખ્યાઓ:}

{\def\LTcaptype{none} % do not increment counter
\begin{longtable}[]{@{}ll@{}}
\toprule\noalign{}
શબ્દ & વ્યાખ્યા \\
\midrule\noalign{}
\endhead
\bottomrule\noalign{}
\endlastfoot
\textbf{પ્રદૂષણ} & પર્યાવરણમાં હાનિકારક પદાર્થોનો ઉમેરો \\
\textbf{પ્રદૂષક} & પર્યાવરણીય દૂષણ લાવનાર પદાર્થ \\
\end{longtable}
}

\begin{itemize}
\tightlist
\item
  \textbf{સ્ત્રોતો}: ઔદ્યોગિક, ઘરેલુ, કૃષિ પ્રવૃત્તિઓ
\item
  \textbf{પ્રકારો}: હવા, પાણી, જમીન, ધ્વનિ પ્રદૂષણ
\item
  \textbf{અસરો}: આરોગ્યની સમસ્યાઓ, પર્યાવરણતંત્રને નુકસાન
\end{itemize}

\textbf{મેમરી ટ્રીક:} ``પ્રદૂષકો પ્રદૂષણ લાવે છે''

\subsection*{પ્રશ્ન 2(b) {[}4
ગુણ{]}}\label{uxaaauxab0uxab6uxaa8-2b-4-uxa97uxaa3}

\textbf{હવાના પ્રદૂષણને નિયંત્રણ રાખવા માટે ગ્રેવિટી સેટલિંગ ચેમ્બર વિશે ટૂંકનોધ લખો.}

\textbf{જવાબ}:

\textbf{ગ્રેવિટી સેટલિંગ ચેમ્બર:}

\begin{verbatim}
+{-{-}{-}{-}{-}{-}{-}{-}{-}{-}{-}{-}{-}{-}{-}{-}{-}{-}+}
|  ગંદી હવા  {-{-}   |}
|                  |
|   કણો            |
|      ↓           |
|  સંગ્રહ ચેમ્બર    |
|                  |
|  સાફ હવા  {-{-}    |}
+{-{-}{-}{-}{-}{-}{-}{-}{-}{-}{-}{-}{-}{-}{-}{-}{-}{-}+}
\end{verbatim}

\textbf{કાર્યસિદ્ધાંત:}

{\def\LTcaptype{none} % do not increment counter
\begin{longtable}[]{@{}ll@{}}
\toprule\noalign{}
પરિમાણ & વર્ણન \\
\midrule\noalign{}
\endhead
\bottomrule\noalign{}
\endlastfoot
\textbf{પદ્ધતિ} & કણોનું ગુરુત્વાકર્ષણ સ્થાપન \\
\textbf{કાર્યક્ષમતા} & \textgreater50 μm કણો માટે 50-70\% \\
\textbf{વેગ} & ધીમો ગેસ વેગ સ્થાપનને મંજૂરી આપે છે \\
\end{longtable}
}

\begin{itemize}
\tightlist
\item
  \textbf{ઉપયોગો}: સિમેન્ટ, ખાણકામ, ધાતુવિદ્યા ઉદ્યોગો
\item
  \textbf{ફાયદા}: સરળ ડિઝાઇન, ઓછો જાળવણી ખર્ચ
\item
  \textbf{મર્યાદાઓ}: બારીક કણો માટે બિનઅસરકારક
\end{itemize}

\textbf{મેમરી ટ્રીક:} ``ગુરુત્વાકર્ષણ ભારે કણો સ્થાપિત કરે છે''

\subsection*{પ્રશ્ન 2(c) {[}7
ગુણ{]}}\label{uxaaauxab0uxab6uxaa8-2c-7-uxa97uxaa3}

\textbf{ઘન કચરાનું વ્યવસ્થાપન સમજાવો.}

\textbf{જવાબ}:

\textbf{ઘન કચરા વ્યવસ્થાપન શ્રેણી:}

\begin{center}
\textbf{Mermaid Diagram (Code)}
\begin{verbatim}
{Shaded}
{Highlighting}[]
graph LR
    A[ઘટાડવું] {-{-}{} B[પુનઃઉપયોગ]}
    B {-{-}{} C[પુનર્ચક્રણ]}
    C {-{-}{} D[પુનઃપ્રાપ્તિ]}
    D {-{-}{} E[નિકાલ]}
{Highlighting}
{Shaded}
\end{verbatim}
\end{center}

\textbf{વ્યવસ્થાપન પદ્ધતિઓ:}

{\def\LTcaptype{none} % do not increment counter
\begin{longtable}[]{@{}lll@{}}
\toprule\noalign{}
પદ્ધતિ & વર્ણન & ફાયદા \\
\midrule\noalign{}
\endhead
\bottomrule\noalign{}
\endlastfoot
\textbf{લેન્ડફિલ} & નિયંત્રિત દફન & સરળ, ખર્ચ-અસરકારક \\
\textbf{દહન} & ઉચ્ચ તાપમાનમાં બાળવું & વોલ્યુમ ઘટાડો \\
\textbf{ખાતર} & જૈવિક વિઘટન & પોષક તત્વોથી ભરપૂર ખાતર \\
\textbf{પુનર્ચક્રણ} & સામગ્રી પુનઃપ્રાપ્તિ & સંસાધન સંરક્ષણ \\
\end{longtable}
}

\textbf{ઘટકો:}

\begin{itemize}
\tightlist
\item
  \textbf{સંગ્રહ}: ઘર-ઘર પિકઅપ સિસ્ટમ
\item
  \textbf{પરિવહન}: કાર્યક્ષમ વાહન માર્ગ
\item
  \textbf{ઉપચાર}: વર્ગીકરણ, પ્રક્રિયા, નિકાલ
\item
  \textbf{મોનિટરિંગ}: નિયમિત ગુણવત્તા તપાસ
\end{itemize}

\textbf{મેમરી ટ્રીક:} ``ભેગું કરો, પરિવહન કરો, ઉપચાર કરો, મોનિટર કરો''

\subsection*{પ્રશ્ન 2(a) OR {[}3
ગુણ{]}}\label{uxaaauxab0uxab6uxaa8-2a-or-3-uxa97uxaa3}

\textbf{ઘોંઘાટની નિવારણ અસર જણાવો.}

\textbf{જવાબ}:

\textbf{ધ્વનિ પ્રદૂષણની અસરો:}

{\def\LTcaptype{none} % do not increment counter
\begin{longtable}[]{@{}ll@{}}
\toprule\noalign{}
પ્રકાર & અસરો \\
\midrule\noalign{}
\endhead
\bottomrule\noalign{}
\endlastfoot
\textbf{આરોગ્યની અસરો} & સાંભળવાની ખોટ, તાણ, હાઈ બ્લડ પ્રેશર \\
\textbf{મનોવૈજ્ઞાનિક} & ચિડાઈ, ઊંઘની અવ્યવસ્થા, ચિંતા \\
\textbf{પર્યાવરણીય} & વન્યજીવો વિક્ષેપ, પર્યાવરણતંત્ર નુકસાન \\
\end{longtable}
}

\begin{itemize}
\tightlist
\item
  \textbf{સ્ત્રોતો}: ટ્રાફિક, ઉદ્યોગો, બાંધકામ, એરક્રાફ્ટ
\item
  \textbf{માપદંડ}: ડેસિબલ (dB) સ્કેલ
\item
  \textbf{નિયંત્રણ}: ધ્વનિ અવરોધ, ધ્વનિ નિયમો
\end{itemize}

\textbf{મેમરી ટ્રીક:} ``ધ્વનિ આરોગ્ય અને વસવાટને હાનિ પહોંચાડે છે''

\subsection*{પ્રશ્ન 2(b) OR {[}4
ગુણ{]}}\label{uxaaauxab0uxab6uxaa8-2b-or-4-uxa97uxaa3}

\textbf{પાણીનું પ્રદૂષણ એટલે શું? પાણીના મુખ્ય પ્રદૂષકો જણાવો.}

\textbf{જવાબ}:

\textbf{પાણી પ્રદૂષણ વ્યાખ્યા:} હાનિકારક પદાર્થો દ્વારા જળાશયોનું દૂષણ જે તેને ઉપયોગ
માટે અનુપયુક્ત બનાવે છે.

\textbf{મુખ્ય જળ પ્રદૂષકો:}

{\def\LTcaptype{none} % do not increment counter
\begin{longtable}[]{@{}ll@{}}
\toprule\noalign{}
વર્ગ & ઉદાહરણો \\
\midrule\noalign{}
\endhead
\bottomrule\noalign{}
\endlastfoot
\textbf{રાસાયણિક} & ભારે ધાતુઓ, જંતુનાશકો, ખાતરો \\
\textbf{જૈવિક} & બેક્ટેરિયા, વાયરસ, પરજીવીઓ \\
\textbf{ભૌતિક} & છેતરી પાવેલા ઘન પદાર્થો, થર્મલ પ્રદૂષણ \\
\textbf{કિરણોત્સર્ગી} & પરમાણુ કચરા સામગ્રી \\
\end{longtable}
}

\begin{itemize}
\tightlist
\item
  \textbf{સ્ત્રોતો}: ઔદ્યોગિક વિસર્જન, ઘરેલુ ગંદુ પાણી, કૃષિ પ્રવાહ
\item
  \textbf{અસરો}: રોગ સંક્રમણ, પર્યાવરણતંત્ર વિક્ષેપ
\item
  \textbf{નિયંત્રણ}: ઉપચાર પ્લાન્ટ, પ્રદૂષણ નિવારણ
\end{itemize}

\textbf{મેમરી ટ્રીક:} ``રાસાયણિક, જૈવિક, ભૌતિક, કિરણોત્સર્ગી''

\subsection*{પ્રશ્ન 2(c) OR {[}7
ગુણ{]}}\label{uxaaauxab0uxab6uxaa8-2c-or-7-uxa97uxaa3}

\textbf{ઇ-વેસ્ટ એટલે શું? ઇ-વેસ્ટની પર્યાવરણ અને માનવ સ્વાસ્થ્ય પર અસર વિશે લખો તેના
રીસાયક્લિંગ વિશે સમજાવો.}

\textbf{જવાબ}:

\textbf{ઇ-વેસ્ટ વ્યાખ્યા:} ઇલેક્ટ્રોનિક વેસ્ટમાં કાઢી નાખવામાં આવેલા વિદ્યુત અને
ઇલેક્ટ્રોનિક ઉપકરણોનો સમાવેશ થાય છે.

\textbf{પર્યાવરણીય અસર:}

\begin{center}
\textbf{Mermaid Diagram (Code)}
\begin{verbatim}
{Shaded}
{Highlighting}[]
graph TD
    A[ઇ{-વેસ્ટ] {-}{-}{} B[જમીન દૂષણ]}
    A {-{-}{} C[પાણી પ્રદૂષણ]}
    A {-{-}{} D[હવા પ્રદૂષણ]}
    A {-{-}{} E[સંસાધન અવક્ષય]}
{Highlighting}
{Shaded}
\end{verbatim}
\end{center}

\textbf{આરોગ્યની અસર:}

{\def\LTcaptype{none} % do not increment counter
\begin{longtable}[]{@{}ll@{}}
\toprule\noalign{}
ઝેરી સામગ્રી & આરોગ્યની અસરો \\
\midrule\noalign{}
\endhead
\bottomrule\noalign{}
\endlastfoot
\textbf{સીસું} & ન્યુરસ સિસ્ટમને નુકસાન \\
\textbf{પારો} & મગજ અને કિડનીને નુકસાન \\
\textbf{કેડમિયમ} & કેન્સર, ફેફસાંને નુકસાન \\
\end{longtable}
}

\textbf{ઇ-વેસ્ટ રીસાયક્લિંગ પ્રક્રિયા:}

\begin{itemize}
\tightlist
\item
  \textbf{સંગ્રહ}: નિર્દિષ્ટ સંગ્રહ કેન્દ્રો
\item
  \textbf{ડિસમેન્ટલિંગ}: ઘટકોનું મેન્યુઅલ વિભાજન
\item
  \textbf{પુનઃપ્રાપ્તિ}: મૂલ્યવાન સામગ્રીઓનું નિષ્કર્ષણ
\item
  \textbf{નિકાલ}: ઝેરી પદાર્થોનું સુરક્ષિત સંચાલન
\end{itemize}

\textbf{મેમરી ટ્રીક:} ``એકત્ર કરો, ડિસમેન્ટલ કરો, પુનઃપ્રાપ્ત કરો, સુરક્ષિત નિકાલ
કરો''

\subsection*{પ્રશ્ન 3(a) {[}3
ગુણ{]}}\label{uxaaauxab0uxab6uxaa8-3a-3-uxa97uxaa3}

\textbf{BOD એટલે શું? BOD ની અગત્યતા સમજાવો.}

\textbf{જવાબ}:

\textbf{BOD (Biochemical Oxygen Demand):}

{\def\LTcaptype{none} % do not increment counter
\begin{longtable}[]{@{}
  >{\raggedright\arraybackslash}p{(\linewidth - 2\tabcolsep) * \real{0.4583}}
  >{\raggedright\arraybackslash}p{(\linewidth - 2\tabcolsep) * \real{0.5417}}@{}}
\toprule\noalign{}
\begin{minipage}[b]{\linewidth}\raggedright
પરિમાણ
\end{minipage} & \begin{minipage}[b]{\linewidth}\raggedright
વર્ણન
\end{minipage} \\
\midrule\noalign{}
\endhead
\bottomrule\noalign{}
\endlastfoot
\textbf{વ્યાખ્યા} & કાર્બનિક પદાર્થોને વિઘટિત કરવા માટે સૂક્ષ્મજીવાણુ દ્વારા જરૂરી
ઓક્સિજન \\
\textbf{એકમ} & mg/L અથવા ppm \\
\textbf{ટેસ્ટ સમયગાળો} & 20°C પર 5 દિવસ \\
\end{longtable}
}

\textbf{મહત્વ:}

\begin{itemize}
\tightlist
\item
  \textbf{પાણીની ગુણવત્તા}: કાર્બનિક પ્રદૂષણનું સ્તર દર્શાવે છે
\item
  \textbf{ઉપચાર કાર્યક્ષમતા}: ઉપચાર પ્લાન્ટની કામગીરી મોનિટર કરે છે
\item
  \textbf{પર્યાવરણીય આરોગ્ય}: જલીય પર્યાવરણતંત્રની સ્થિતિ આંકે છે
\end{itemize}

\textbf{મેમરી ટ્રીક:} ``બેક્ટેરિયા ઓક્સિજન માંગ પ્રદૂષણ માપે છે''

\subsection*{પ્રશ્ન 3(b) {[}4
ગુણ{]}}\label{uxaaauxab0uxab6uxaa8-3b-4-uxa97uxaa3}

\textbf{પરંપરાગત અને બિનપરંપરાગત ઊર્જાના સ્ત્રોતની સરખામણી કરો.}

\textbf{જવાબ}:

\textbf{ઊર્જા સ્ત્રોતોની સરખામણી:}

{\def\LTcaptype{none} % do not increment counter
\begin{longtable}[]{@{}lll@{}}
\toprule\noalign{}
પરિમાણ & પરંપરાગત & બિનપરંપરાગત \\
\midrule\noalign{}
\endhead
\bottomrule\noalign{}
\endlastfoot
\textbf{ઉદાહરણો} & કોલસો, તેલ, કુદરતી ગેસ & સૌર, પવન, બાયોમાસ \\
\textbf{ઉપલબ્ધતા} & મર્યાદિત અનામત & અમર્યાદિત/નવીકરણીય \\
\textbf{પર્યાવરણ} & ઉચ્ચ પ્રદૂષણ & પર્યાવરણ મૈત્રી \\
\textbf{કિંમત} & પ્રારંભે સસ્તી & ઉચ્ચ પ્રારંભિક કિંમત \\
\textbf{ટકાઉપણું} & બિન-ટકાઉ & ટકાઉ \\
\end{longtable}
}

\begin{itemize}
\tightlist
\item
  \textbf{પરંપરાગત}: ઝડપથી ઘટતા, ગ્રીનહાઉસ ગેસ લાવે છે
\item
  \textbf{બિનપરંપરાગત}: સ્વચ્છ, વિપુલ, ભવિષ્યનો ઊર્જા ઉકેલ
\item
  \textbf{સંક્રમણ}: નવીકરણીય ઊર્જા તરફ વૈશ્વિક પરિવર્તન
\end{itemize}

\textbf{મેમરી ટ્રીક:} ``પરંપરાગત પ્રદૂષિત કરે છે, નવીકરણીય ટકાવે છે''

\subsection*{પ્રશ્ન 3(c) {[}7
ગુણ{]}}\label{uxaaauxab0uxab6uxaa8-3c-7-uxa97uxaa3}

\textbf{પવનચક્કીનું વર્ગીકરણ કરી આડી ધરી વાળી પવનચક્કી વિશે સમજાવો.}

\textbf{જવાબ}:

\textbf{પવન ટર્બાઇનનું વર્ગીકરણ:}

\begin{center}
\textbf{Mermaid Diagram (Code)}
\begin{verbatim}
{Shaded}
{Highlighting}[]
graph TD
    A[પવન ટર્બાઇન] {-{-}{} B[આડી ધરી {-} HAWT]}
    A {-{-}{} C[ઊભી ધરી {-} VAWT]}
    B {-{-}{} D[અપવિન્ડ]}
    B {-{-}{} E[ડાઉનવિન્ડ]}
    C {-{-}{} F[ડેરિયસ]}
    C {-{-}{} G[સેવોનિયસ]}
{Highlighting}
{Shaded}
\end{verbatim}
\end{center}

\textbf{આડી ધરી પવન ટર્બાઇન (HAWT):}

\textbf{ઘટકો:}

{\def\LTcaptype{none} % do not increment counter
\begin{longtable}[]{@{}ll@{}}
\toprule\noalign{}
ઘટક & કાર્ય \\
\midrule\noalign{}
\endhead
\bottomrule\noalign{}
\endlastfoot
\textbf{રોટર બ્લેડ} & પવન ઊર્જાને ફરતી ગતિમાં રૂપાંતરિત કરે છે \\
\textbf{નેસેલ} & જનરેટર અને ગિયરબોક્સ રાખે છે \\
\textbf{ટાવર} & ઇષ્ટ ઊંચાઈ પર ટર્બાઇનને ટેકો આપે છે \\
\textbf{ફાઉન્ડેશન} & માળખાકીય સ્થિરતા પ્રદાન કરે છે \\
\end{longtable}
}

\textbf{કાર્યસિદ્ધાંત:}

\begin{itemize}
\tightlist
\item
  \textbf{પવનની દિશા}: રોટર અક્ષની સમાંતર
\item
  \textbf{બ્લેડ ડિઝાઇન}: એરોડાયનેમિક લિફ્ટ સિદ્ધાંત
\item
  \textbf{પાવર જનરેશન}: વેરિયેબલ સ્પીડ ઓપરેશન
\item
  \textbf{કાર્યક્ષમતા}: 35-45\% ઊર્જા રૂપાંતરણ
\end{itemize}

\textbf{ફાયદા:}

\begin{itemize}
\tightlist
\item
  \textbf{ઉચ્ચ કાર્યક્ષમતા}: વધુ સારો પાવર કોએફિશિઅન્ટ
\item
  \textbf{પરિપક્વ ટેકનોલોજી}: સુસ્થાપિત ડિઝાઇન
\item
  \textbf{ખર્ચ અસરકારક}: ઓછો જાળવણી ખર્ચ
\end{itemize}

\textbf{મેમરી ટ્રીક:} ``આડી ઉચ્ચ કાર્યક્ષમતા''

\subsection*{પ્રશ્ન 3(a) OR {[}3
ગુણ{]}}\label{uxaaauxab0uxab6uxaa8-3a-or-3-uxa97uxaa3}

\textbf{રીન્યુએબલ એનર્જીની જરૂરિયાત સમજાવો.}

\textbf{જવાબ}:

\textbf{નવીકરણીય ઊર્જાની જરૂરિયાત:}

{\def\LTcaptype{none} % do not increment counter
\begin{longtable}[]{@{}ll@{}}
\toprule\noalign{}
કારણ & વર્ણન \\
\midrule\noalign{}
\endhead
\bottomrule\noalign{}
\endlastfoot
\textbf{ઊર્જા સુરક્ષા} & આયાત પર નિર્ભરતા ઘટાડવી \\
\textbf{પર્યાવરણ સંરક્ષણ} & શૂન્ય કાર્બન ઉત્સર્જન \\
\textbf{આર્થિક ફાયદા} & રોજગાર સર્જન, ખર્ચ ઘટાડો \\
\end{longtable}
}

\begin{itemize}
\tightlist
\item
  \textbf{અશ્મિ ઇંધન અવક્ષય}: મર્યાદિત અનામત, વધતી કિંમતો
\item
  \textbf{હવામાન પરિવર્તન}: ગ્રીનહાઉસ ગેસ ઘટાડવાની તાત્કાલિક જરૂર
\item
  \textbf{ટકાઉ વિકાસ}: ભવિષ્યને સાક્ષાત્કાર કર્યા વગર વર્તમાન જરૂરિયાતો પૂરી
  કરવી
\end{itemize}

\textbf{મેમરી ટ્રીક:} ``સુરક્ષા, પર્યાવરણ, અર્થવ્યવસ્થાને નવીકરણીય જોઈએ''

\subsection*{પ્રશ્ન 3(b) OR {[}4
ગુણ{]}}\label{uxaaauxab0uxab6uxaa8-3b-or-4-uxa97uxaa3}

\textbf{ટૂંકનોધ લખો: ભૂ-થર્મલ ઊર્જા.}

\textbf{જવાબ}:

\textbf{ભૂ-થર્મલ ઊર્જા:}

પૃથ્વીની અંદરની સપાટીની નીચે સંગ્રહિત ગરમીની ઊર્જા જેનો પાવર જનરેશન માટે ઉપયોગ થાય
છે.

\textbf{પ્રકારો:}

{\def\LTcaptype{none} % do not increment counter
\begin{longtable}[]{@{}lll@{}}
\toprule\noalign{}
પ્રકાર & તાપમાન & ઉપયોગ \\
\midrule\noalign{}
\endhead
\bottomrule\noalign{}
\endlastfoot
\textbf{ઉચ્ચ તાપમાન} & \textgreater150°C & પાવર જનરેશન \\
\textbf{મધ્યમ તાપમાન} & 90-150°C & સીધું ગરમ કરવું \\
\textbf{નીચો તાપમાન} & \textless90°C & હીટ પમ્પ \\
\end{longtable}
}

\begin{itemize}
\tightlist
\item
  \textbf{સ્ત્રોતો}: ગરમ ઝરણા, ગિઝર, ભૂગર્ભ જળાશયો
\item
  \textbf{ફાયદા}: સતત ઉપલબ્ધતા, ઓછું ઉત્સર્જન
\item
  \textbf{ઉપયોગો}: વીજ ઉત્પાદન, સ્પેસ હીટિંગ, ઔદ્યોગિક પ્રક્રિયાઓ
\end{itemize}

\textbf{મેમરી ટ્રીક:} ``પૃથ્વીની ગરમી ઘરોને પાવર આપે છે''

\subsection*{પ્રશ્ન 3(c) OR {[}7
ગુણ{]}}\label{uxaaauxab0uxab6uxaa8-3c-or-7-uxa97uxaa3}

\textbf{સોલર ફોટો વોલ્ટેઇક સેલનો સિદ્ધાંત લખી કાર્યપદ્ધતિ સમજાવો. તેના ઉપયોગો
લખો.}

\textbf{જવાબ}:

\textbf{સોલર ફોટોવોલ્ટેઇક સેલ સિદ્ધાંત:}

ફોટોવોલ્ટેઇક અસરનો ઉપયોગ કરીને સૂર્યપ્રકાશને સીધા વીજળીમાં રૂપાંતરિત કરે છે.

\textbf{કાર્ય પ્રક્રિયા:}

\begin{center}
\textbf{Mermaid Diagram (Code)}
\begin{verbatim}
{Shaded}
{Highlighting}[]
graph LR
    A[સૂર્યપ્રકાશ] {-{-}{} B[સિલિકોન સેલ]}
    B {-{-}{} C[ઇલેક્ટ્રોન હલનચલન]}
    C {-{-}{} D[વિદ્યુત પ્રવાહ]}
    D {-{-}{} E[DC પાવર]}
    E {-{-}{} F[ઇન્વર્ટર]}
    F {-{-}{} G[AC પાવર]}
{Highlighting}
{Shaded}
\end{verbatim}
\end{center}

\textbf{સેલ માળખું:}

{\def\LTcaptype{none} % do not increment counter
\begin{longtable}[]{@{}lll@{}}
\toprule\noalign{}
સ્તર & સામગ્રી & કાર્ય \\
\midrule\noalign{}
\endhead
\bottomrule\noalign{}
\endlastfoot
\textbf{ઉપરનો સ્તર} & N-type સિલિકોન & વધારાના ઇલેક્ટ્રોન \\
\textbf{નીચેનો સ્તર} & P-type સિલિકોન & ઇલેક્ટ્રોન હોલ \\
\textbf{જંક્શન} & P-N જંક્શન & વિદ્યુત ક્ષેત્ર સર્જન \\
\end{longtable}
}

\textbf{કાર્ય પગલાં:}

\begin{itemize}
\tightlist
\item
  \textbf{ફોટોન શોષણ}: સિલિકોન દ્વારા પ્રકાશ ઊર્જા શોષાય છે
\item
  \textbf{ઇલેક્ટ્રોન ઉત્તેજના}: ઇલેક્ટ્રોન ઊર્જા મેળવે છે અને હલે છે
\item
  \textbf{પ્રવાહ જનરેશન}: ઇલેક્ટ્રોન પ્રવાહ વીજળી બનાવે છે
\item
  \textbf{બાહ્ય સર્કિટ}: લોડ દ્વારા પ્રવાહ વહે છે
\end{itemize}

\textbf{ઉપયોગો:}

\begin{itemize}
\tightlist
\item
  \textbf{રહેણાંક}: છતની સોલર સિસ્ટમ
\item
  \textbf{વ્યાપારિક}: સોલર ફાર્મ, સ્ટ્રીટ લાઇટિંગ
\item
  \textbf{ઔદ્યોગિક}: રિમોટ પાવર સપ્લાય, સેટેલાઇટ
\item
  \textbf{પરિવહન}: સોલર વાહનો, ચાર્જિંગ સ્ટેશન
\end{itemize}

\textbf{ફાયદા:}

\begin{itemize}
\tightlist
\item
  \textbf{સ્વચ્છ ઊર્જા}: ઓપરેશન દરમિયાન કોઈ ઉત્સર્જન નહીં
\item
  \textbf{ઓછી જાળવણી}: ન્યૂનતમ હલતા ભાગો
\item
  \textbf{મોડ્યુલર}: સ્કેલેબલ ઇન્સ્ટોલેશન
\end{itemize}

\textbf{મેમરી ટ્રીક:} ``સૂર્ય સિલિકોન પર પ્રહાર કરે છે, પ્રવાહ ચાલુ કરે છે''

\subsection*{પ્રશ્ન 4(a) {[}3
ગુણ{]}}\label{uxaaauxab0uxab6uxaa8-4a-3-uxa97uxaa3}

\textbf{ગ્રીન હાઉસ અસર સમજાવો.}

\textbf{જવાબ}:

\textbf{ગ્રીનહાઉસ અસર:}

કુદરતી પ્રક્રિયા જ્યાં ચોક્કસ ગેસો પૃથ્વીના વાતાવરણમાં ગરમીને ફસાવે છે.

\textbf{પદ્ધતિ:}

{\def\LTcaptype{none} % do not increment counter
\begin{longtable}[]{@{}ll@{}}
\toprule\noalign{}
પગલું & પ્રક્રિયા \\
\midrule\noalign{}
\endhead
\bottomrule\noalign{}
\endlastfoot
\textbf{સૌર કિરણોત્સર્ગ} & સૂર્યની ઊર્જા પૃથ્વી સુધી પહોંચે છે \\
\textbf{સપાટી શોષણ} & પૃથ્વી શોષે છે અને ગરમ થાય છે \\
\textbf{પુનઃકિરણોત્સર્ગ} & પૃથ્વી ઇન્ફ્રારેડ કિરણોત્સર્ગ બહાર કાઢે છે \\
\textbf{ગેસ ફસાવણી} & ગ્રીનહાઉસ ગેસો ગરમી ફસાવે છે \\
\end{longtable}
}

\begin{itemize}
\tightlist
\item
  \textbf{કુદરતી અસર}: જીવન માટે પૃથ્વીનું તાપમાન જાળવે છે
\item
  \textbf{વધારેલી અસર}: માનવીય પ્રવૃત્તિઓ ગ્રીનહાઉસ ગેસ વધારે છે
\item
  \textbf{પરિણામ}: ગ્લોબલ વોર્મિંગ અને હવામાન પરિવર્તન
\end{itemize}

\textbf{મેમરી ટ્રીક:} ``ગેસો ગરમી ફસાવે છે, પૃથ્વી ગરમ થાય છે''

\subsection*{પ્રશ્ન 4(b) {[}4
ગુણ{]}}\label{uxaaauxab0uxab6uxaa8-4b-4-uxa97uxaa3}

\textbf{જળવાયુ પરિવર્તન માટે આંતરરાષ્ટ્રીય કરાર વિશે જણાવો.}

\textbf{જવાબ}:

\textbf{આંતરરાષ્ટ્રીય હવામાન પ્રોટોકોલ:}

{\def\LTcaptype{none} % do not increment counter
\begin{longtable}[]{@{}lll@{}}
\toprule\noalign{}
પ્રોટોકોલ & વર્ષ & ઉદ્દેશ્ય \\
\midrule\noalign{}
\endhead
\bottomrule\noalign{}
\endlastfoot
\textbf{ક્યોટો પ્રોટોકોલ} & 1997 & ગ્રીનહાઉસ ગેસ ઉત્સર્જન ઘટાડવું \\
\textbf{પેરિસ એગ્રિમેન્ટ} & 2015 & ગ્લોબલ વોર્મિંગ 1.5°C સુધી મર્યાદિત કરવું \\
\textbf{મોન્ટ્રીયલ પ્રોટોકોલ} & 1987 & ઓઝોન સ્તરનું સંરક્ષણ \\
\end{longtable}
}

\textbf{મુખ્ય લક્ષણો:}

\begin{itemize}
\tightlist
\item
  \textbf{ઉત્સર્જન લક્ષ્યાંકો}: વિકસિત દેશો માટે બંધનકર્તા પ્રતિબદ્ધતાઓ
\item
  \textbf{સ્વચ્છ વિકાસ}: વિકાસશીલ રાષ્ટ્રોમાં ટેકનોલોજી ટ્રાન્સફર
\item
  \textbf{કાર્બન ટ્રેડિંગ}: બજાર-આધારિત ઉત્સર્જન ઘટાડાની પદ્ધતિઓ
\item
  \textbf{મોનિટરિંગ}: નિયમિત રિપોર્ટિંગ અને ચકાસણી સિસ્ટમ
\end{itemize}

\textbf{મેમરી ટ્રીક:} ``ક્યોટો, પેરિસ, મોન્ટ્રીયલ હવામાનનું સંરક્ષણ કરે છે''

\subsection*{પ્રશ્ન 4(c) {[}7
ગુણ{]}}\label{uxaaauxab0uxab6uxaa8-4c-7-uxa97uxaa3}

\textbf{બાયોગેસ પ્લાન્ટ આકૃતિ સાથે સમજાવો.}

\textbf{જવાબ}:

\textbf{બાયોગેસ પ્લાન્ટ:}

\begin{verbatim}
    ગેસ આઉટલેટ
        ↑
+{-{-}{-}[ગેસ હોલ્ડર]{-}{-}{-}+}
|                  |
|  સ્લરી ચેમ્બર     |
|                  |
+{-{-}{-}{-}{-}{-}{-}{-}+{-}{-}{-}{-}{-}{-}{-}{-}{-}+}
         |
    ઇનલેટ ટાંકી
         ↓
   કાર્બનિક કચરો
\end{verbatim}

\textbf{ઘટકો:}

{\def\LTcaptype{none} % do not increment counter
\begin{longtable}[]{@{}ll@{}}
\toprule\noalign{}
ઘટક & કાર્ય \\
\midrule\noalign{}
\endhead
\bottomrule\noalign{}
\endlastfoot
\textbf{ઇનલેટ ટાંકી} & કાર્બનિક કચરો મેળવે છે \\
\textbf{ડાઇજેસ્ટર} & એનેરોબિક વિઘટન થાય છે \\
\textbf{ગેસ હોલ્ડર} & ઉત્પન્ન થયેલ બાયોગેસ સંગ્રહ કરે છે \\
\textbf{આઉટલેટ} & વપરાયેલ સ્લરી કાઢે છે \\
\end{longtable}
}

\textbf{કાર્ય પ્રક્રિયા:}

\begin{itemize}
\tightlist
\item
  \textbf{લોડિંગ}: કાર્બનિક કચરો પાણી સાથે મિશ્રિત
\item
  \textbf{પાચન}: બેક્ટેરિયા કચરાને એનેરોબિક રીતે વિઘટિત કરે છે
\item
  \textbf{ગેસ ઉત્પાદન}: મિથેન અને CO2 ઉત્પન્ન થાય છે
\item
  \textbf{સંગ્રહ}: ગેસ હોલ્ડરમાં ઉપયોગ માટે સંગ્રહિત
\end{itemize}

\textbf{કાચો માલ:}

\begin{itemize}
\tightlist
\item
  \textbf{પ્રાણી કચરો}: ગાયનું છાણ, પોલ્ટ્રી ડ્રોપિંગ્સ
\item
  \textbf{છોડ કચરો}: કૃષિ અવશેષો, રસોડાનો કચરો
\item
  \textbf{પાણી}: યોગ્ય સુસંગતતા જાળવે છે
\end{itemize}

\textbf{ઉત્પાદનો:}

\begin{itemize}
\tightlist
\item
  \textbf{બાયોગેસ}: રસોઈ/ગરમ કરવા માટે 50-70\% મિથેન
\item
  \textbf{સ્લરી}: ઉત્તમ કાર્બનિક ખાતર
\end{itemize}

\textbf{ફાયદા:}

\begin{itemize}
\tightlist
\item
  \textbf{નવીકરણીય}: સતત ગેસ ઉત્પાદન
\item
  \textbf{કચરા વ્યવસ્થાપન}: કચરાને ઊર્જામાં રૂપાંતરિત કરે છે
\item
  \textbf{ગ્રામીણ વિકાસ}: ગામો માટે યોગ્ય
\end{itemize}

\textbf{મેમરી ટ્રીક:} ``કચરો અંદર, ગેસ બહાર, ખાતર બોનસ''

\subsection*{પ્રશ્ન 4(a) OR {[}3
ગુણ{]}}\label{uxaaauxab0uxab6uxaa8-4a-or-3-uxa97uxaa3}

\textbf{ટૂંકનોધ લખો: ગ્રીન હાઉસ ગેસો.}

\textbf{જવાબ}:

\textbf{ગ્રીનહાઉસ ગેસો:}

{\def\LTcaptype{none} % do not increment counter
\begin{longtable}[]{@{}lll@{}}
\toprule\noalign{}
ગેસ & સ્ત્રોત & યોગદાન \\
\midrule\noalign{}
\endhead
\bottomrule\noalign{}
\endlastfoot
\textbf{કાર્બન ડાયોક્સાઇડ} & અશ્મિ ઇંધન, વનનાશ & 76\% \\
\textbf{મિથેન} & કૃષિ, લેન્ડફિલ & 16\% \\
\textbf{નાઇટ્રસ ઓક્સાઇડ} & ખાતરો, દહન & 6\% \\
\textbf{ફ્લોરિનેટેડ ગેસો} & ઔદ્યોગિક પ્રક્રિયાઓ & 2\% \\
\end{longtable}
}

\begin{itemize}
\tightlist
\item
  \textbf{ગુણધર્મો}: ઇન્ફ્રારેડ કિરણોત્સર્ગ શોષે છે અને બહાર કાઢે છે
\item
  \textbf{અસર}: ગરમી ફસાવીને ગ્લોબલ વોર્મિંગ લાવે છે
\item
  \textbf{નિયંત્રણ}: ઉત્સર્જન ઘટાડવું, વિકલ્પોનો ઉપયોગ
\end{itemize}

\textbf{મેમરી ટ્રીક:} ``CO2, CH4, N2O, F-ગેસો પૃથ્વીને ગરમ કરે છે''

\subsection*{પ્રશ્ન 4(b) OR {[}4
ગુણ{]}}\label{uxaaauxab0uxab6uxaa8-4b-or-4-uxa97uxaa3}

\textbf{ઓઝોન સ્તરમાં બાકોરા સમજાવો.}

\textbf{જવાબ}:

\textbf{ઓઝોન સ્તર અવક્ષય:}

માનવીય પ્રવૃત્તિઓને કારણે સ્ટ્રેટોસ્ફિયરમાં ઓઝોન સાંદ્રતામાં ઘટાડો.

\textbf{કારણો:}

{\def\LTcaptype{none} % do not increment counter
\begin{longtable}[]{@{}lll@{}}
\toprule\noalign{}
પદાર્થ & સ્ત્રોત & અસર \\
\midrule\noalign{}
\endhead
\bottomrule\noalign{}
\endlastfoot
\textbf{CFCs} & રેફ્રિજરન્ટ્સ, એરોસોલ & ઓઝોન અણુઓ તોડે છે \\
\textbf{હેલોન} & ફાયર એક્સ્ટિંગ્યુશર & ઉત્પ્રેરક ઓઝોન વિનાશ \\
\textbf{મિથાઇલ બ્રોમાઇડ} & જંતુનાશકો & ઓઝોન સ્તર પાતળું થવું \\
\end{longtable}
}

\textbf{પ્રક્રિયા:}

\begin{itemize}
\tightlist
\item
  \textbf{UV વિભાજન}: UV કિરણોત્સર્ગ CFC અણુઓ તોડે છે
\item
  \textbf{ક્લોરિન મુક્તિ}: મુક્ત ક્લોરિન અણુઓ મુક્ત થાય છે
\item
  \textbf{ઓઝોન વિનાશ}: ક્લોરિન ઓઝોન અણુઓનો નાશ કરે છે
\item
  \textbf{સાંકળ પ્રતિક્રિયા}: એક CFC અણુ ઘણા ઓઝોન અણુઓનો નાશ કરે છે
\end{itemize}

\textbf{અસરો}: વધેલું UV કિરણોત્સર્ગ, ત્વચા કેન્સર, પાક નુકસાન

\textbf{મેમરી ટ્રીક:} ``CFCs ચઢે છે, ક્લોરિન ઓઝોન કાપે છે''

\subsection*{પ્રશ્ન 4(c) OR {[}7
ગુણ{]}}\label{uxaaauxab0uxab6uxaa8-4c-or-7-uxa97uxaa3}

\textbf{જળવાયુ પરિવર્તન એટલે શું? જળવાયુ પરિવર્તન માટે જવાબદાર પરિબળો સમજાવો.}

\textbf{જવાબ}:

\textbf{હવામાન પરિવર્તન વ્યાખ્યા:} વૈશ્વિક હવામાન પેટર્ન અને તાપમાનમાં લાંબા
ગાળાના પરિવર્તનો.

\textbf{કારણો:}

\begin{center}
\textbf{Mermaid Diagram (Code)}
\begin{verbatim}
{Shaded}
{Highlighting}[]
graph TD
    A[હવામાન પરિવર્તન કારણો] {-{-}{} B[કુદરતી]}
    A {-{-}{} C[માનવીય પ્રવૃત્તિઓ]}
    B {-{-}{} D[સૌર ભિન્નતાઓ]}
    B {-{-}{} E[જ્વાળામુખી વિસ્ફોટ]}
    C {-{-}{} F[ગ્રીનહાઉસ ગેસ ઉત્સર્જન]}
    C {-{-}{} G[વનનાશ]}
    C {-{-}{} H[ઔદ્યોગિક પ્રવૃત્તિઓ]}
{Highlighting}
{Shaded}
\end{verbatim}
\end{center}

\textbf{માનવીય કારણો:}

{\def\LTcaptype{none} % do not increment counter
\begin{longtable}[]{@{}ll@{}}
\toprule\noalign{}
પ્રવૃત્તિ & યોગદાન \\
\midrule\noalign{}
\endhead
\bottomrule\noalign{}
\endlastfoot
\textbf{અશ્મિ ઇંધન બર્નિંગ} & CO2 ઉત્સર્જનનું 65\% \\
\textbf{વનનાશ} & 15\% ઉત્સર્જન \\
\textbf{ઔદ્યોગિક પ્રક્રિયાઓ} & 20\% ઉત્સર્જન \\
\end{longtable}
}

\textbf{અસરો:}

\textbf{પર્યાવરણીય અસરો:}

\begin{itemize}
\tightlist
\item
  \textbf{તાપમાન વધારો}: વૈશ્વિક સરેરાશ તાપમાન વધારો
\item
  \textbf{સમુદ્રી સપાટી વધારો}: થર્મલ વિસ્તરણ અને બરફ પીગળવું
\item
  \textbf{હવામાન ચરમસીમાઓ}: વધુ વારંવાર દુષ્કાળ, પૂર
\end{itemize}

\textbf{જૈવિક અસરો:}

\begin{itemize}
\tightlist
\item
  \textbf{જાતિઓનું સ્થળાંતર}: પ્રાણીઓ ઠંડા પ્રદેશોમાં જતા રહે છે
\item
  \textbf{પર્યાવરણતંત્ર વિક્ષેપ}: ખોરાક સાંકળમાં ફેરફારો
\item
  \textbf{જૈવવિવિધતા નુકસાન}: જાતિઓના લુપ્ત થવાના દર વધે છે
\end{itemize}

\textbf{માનવીય અસરો:}

\begin{itemize}
\tightlist
\item
  \textbf{કૃષિ}: પાક ઉત્પાદનમાં ફેરફાર, ખોરાક સુરક્ષાની સમસ્યાઓ
\item
  \textbf{આરોગ્ય}: ગરમીનો તાણ, રોગ વેક્ટર ફેરફારો
\item
  \textbf{અર્થવ્યવસ્થા}: ઇન્ફ્રાસ્ટ્રક્ચર નુકસાન, અનુકૂલન ખર્ચ
\end{itemize}

\textbf{ઘટાડો વ્યૂહરચનાઓ:}

\begin{itemize}
\tightlist
\item
  \textbf{નવીકરણીય ઊર્જા}: અશ્મિ ઇંધનમાંથી સંક્રમણ
\item
  \textbf{ઊર્જા કાર્યક્ષમતા}: વપરાશ ઘટાડવો
\item
  \textbf{કાર્બન સિક્વેસ્ટ્રેશન}: વન સંરક્ષણ, વૃક્ષ રોપણી
\item
  \textbf{આંતરરાષ્ટ્રીય સહયોગ}: વૈશ્વિક કરારો અને નીતિઓ
\end{itemize}

\textbf{મેમરી ટ્રીક:} ``માનવીય ક્રિયાઓ પૃથ્વીને ગરમ કરે છે, દરેકને અસર થાય છે''

\subsection*{પ્રશ્ન 5(a) {[}3
ગુણ{]}}\label{uxaaauxab0uxab6uxaa8-5a-3-uxa97uxaa3}

\textbf{``ખેત તલાવડી'' વિશે સમજાવો.}

\textbf{જવાબ}:

\textbf{ખેત તલાવડી (ફાર્મ પોન્ડ):}

સિંચાઈ માટે કૃષિ ક્ષેત્રોમાં નાના જળ સંચય માળખું.

\textbf{લક્ષણો:}

{\def\LTcaptype{none} % do not increment counter
\begin{longtable}[]{@{}ll@{}}
\toprule\noalign{}
પરિમાણ & વર્ણન \\
\midrule\noalign{}
\endhead
\bottomrule\noalign{}
\endlastfoot
\textbf{માપ} & 20m x 20m x 3m ઊંડાઈ \\
\textbf{ક્ષમતા} & 1200 ઘન મીટર \\
\textbf{કિંમત} & સરકાર દ્વારા સબસિડી આપવામાં આવે છે \\
\end{longtable}
}

\begin{itemize}
\tightlist
\item
  \textbf{હેતુ}: વરસાદી પાણીનો સંગ્રહ, સૂકા સમયે સિંચાઈ
\item
  \textbf{ફાયદા}: વધેલી પાક ઉપજ, ભૂગર્ભજળ પુનર્ભરણ
\item
  \textbf{બાંધકામ}: પ્લાસ્ટિક શીટ અથવા સિમેન્ટથી લાઇન કરેલ
\end{itemize}

\textbf{મેમરી ટ્રીક:} ``ફાર્મ પોન્ડ પાકો માટે વરસાદ સંગ્રહ કરે છે''

\subsection*{પ્રશ્ન 5(b) {[}4
ગુણ{]}}\label{uxaaauxab0uxab6uxaa8-5b-4-uxa97uxaa3}

\textbf{ગ્રીન બિલ્ડિંગના ઉદ્દેશો અને તેના ફાયદા જણાવો.}

\textbf{જવાબ}:

\textbf{ગ્રીન બિલ્ડિંગ લક્ષ્યાંકો:}

{\def\LTcaptype{none} % do not increment counter
\begin{longtable}[]{@{}ll@{}}
\toprule\noalign{}
લક્ષ્ય & વર્ણન \\
\midrule\noalign{}
\endhead
\bottomrule\noalign{}
\endlastfoot
\textbf{ઊર્જા કાર્યક્ષમતા} & ઊર્જા વપરાશ ઘટાડવો \\
\textbf{જળ સંરક્ષણ} & પાણીનો ઉપયોગ ન્યૂનતમ કરવો \\
\textbf{સામગ્રી કાર્યક્ષમતા} & ટકાઉ સામગ્રીનો ઉપયોગ \\
\textbf{ઇન્ડોર પર્યાવરણ} & હવાની ગુણવત્તા સુધારવી \\
\end{longtable}
}

\textbf{ફાયદા:}

\begin{itemize}
\tightlist
\item
  \textbf{પર્યાવરણીય}: ઘટેલું કાર્બન ફૂટપ્રિન્ટ, કચરો ન્યૂનીકરણ
\item
  \textbf{આર્થિક}: ઓછા ઓપરેટિંગ ખર્ચ, વધેલી મિલકત કિંમત
\item
  \textbf{આરોગ્ય}: વધુ સારી ઇન્ડોર હવાની ગુણવત્તા, કુદરતી પ્રકાશ
\item
  \textbf{સામાજિક}: વધેલો રહેવાસીઓનો આરામ, ઉત્પાદકતા
\end{itemize}

\textbf{ગ્રીન બિલ્ડિંગ લક્ષણો:}

\begin{itemize}
\tightlist
\item
  \textbf{સોલર પેનલ}: નવીકરણીય ઊર્જા ઉત્પાદન
\item
  \textbf{વરસાદી પાણી સંચય}: જળ સંરક્ષણ
\item
  \textbf{ગ્રીન રૂફ}: ઇન્સ્યુલેશન અને હવા શુદ્ધિકરણ
\end{itemize}

\textbf{મેમરી ટ્રીક:} ``ગ્રીન લક્ષ્યાંકો: ઊર્જા, પાણી, સામગ્રી, પર્યાવરણ''

\subsection*{પ્રશ્ન 5(c) {[}7
ગુણ{]}}\label{uxaaauxab0uxab6uxaa8-5c-7-uxa97uxaa3}

\textbf{વરસાદના પાણીના સંચયની જુદી જુદી રીતો જણાવો.}

\textbf{જવાબ}:

\textbf{વરસાદી પાણી સંચયની પદ્ધતિઓ:}

\textbf{સપાટી પદ્ધતિઓ:}

\begin{center}
\textbf{Mermaid Diagram (Code)}
\begin{verbatim}
{Shaded}
{Highlighting}[]
graph TD
    A[વરસાદી પાણી સંચય] {-{-}{} B[સપાટી પદ્ધતિઓ]}
    A {-{-}{} C[ભૂગર્ભજળ પદ્ધતિઓ]}
    B {-{-}{} D[તળાવો અને ટાંકીઓ]}
    B {-{-}{} E[ચેક ડેમ]}
    C {-{-}{} F[પરકોલેશન ખાડાઓ]}
    C {-{-}{} G[રિચાર્જ વેલ]}
{Highlighting}
{Shaded}
\end{verbatim}
\end{center}

\textbf{વિગતવાર પદ્ધતિઓ:}

{\def\LTcaptype{none} % do not increment counter
\begin{longtable}[]{@{}lll@{}}
\toprule\noalign{}
પદ્ધતિ & વર્ણન & ઉપયોગ \\
\midrule\noalign{}
\endhead
\bottomrule\noalign{}
\endlastfoot
\textbf{છતની સંચય} & બિલ્ડિંગની છતમાંથી પાણી એકત્ર કરવું & શહેરી વિસ્તારો \\
\textbf{સપાટી પ્રવાહ} & જમીનની સપાટીમાંથી પાણી પકડવું & ગ્રામીણ વિસ્તારો \\
\textbf{ચેક ડેમ} & નાળાઓ આરપાર નાના અવરોધો & પર્વતીય પ્રદેશો \\
\textbf{પરકોલેશન ટાંકીઓ} & પાણીને ભૂગર્ભમાં જવા દેવાનું & ભૂગર્ભજળ પુનર્ભરણ \\
\end{longtable}
}

\textbf{સિસ્ટમના ઘટકો:}

\begin{itemize}
\tightlist
\item
  \textbf{કેચમેન્ટ એરિયા}: વરસાદી પાણી એકત્ર કરતી સપાટી
\item
  \textbf{કન્વેયન્સ સિસ્ટમ}: પરિવહન માટે ગટર, પાઇપ
\item
  \textbf{સ્ટોરેજ સિસ્ટમ}: પાણી રાખવા માટે ટાંકીઓ, તળાવો
\item
  \textbf{ફિલ્ટર સિસ્ટમ}: કચરો અને દૂષિત પદાર્થો કાઢવા
\end{itemize}

\textbf{છતની સંચય પ્રક્રિયા:}

\begin{itemize}
\tightlist
\item
  \textbf{સંગ્રહ}: છતની સપાટી પર વરસાદ પડે છે
\item
  \textbf{કન્વેયન્સ}: ગટર અને ડાઉનસ્પાઉટ દ્વારા પાણી વહે છે
\item
  \textbf{ફર્સ્ટ ફ્લશ}: પ્રારંભિક ગંદું પાણી દિવર્ટ કરવામાં આવે છે
\item
  \textbf{સ્ટોરેજ}: સાફ પાણી ટાંકીઓમાં સંગ્રહિત કરવામાં આવે છે
\item
  \textbf{વિતરણ}: વિવિધ હેતુઓ માટે પાણીનો ઉપયોગ
\end{itemize}

\textbf{ફાયદા:}

\begin{itemize}
\tightlist
\item
  \textbf{જળ સુરક્ષા}: બાહ્ય પુરવઠા પર નિર્ભરતા ઘટાડવી
\item
  \textbf{પૂર નિયંત્રણ}: સપાટી પ્રવાહ અને પૂર ઘટાડવો
\item
  \textbf{ભૂગર્ભજળ પુનર્ભરણ}: ભૂગર્ભ જળાશયો ફરીથી ભરવા
\item
  \textbf{ખર્ચ બચાવવો}: પાણીના બિલ ઘટાડવા
\end{itemize}

\textbf{ડિઝાઇન વિચારણાઓ:}

\begin{itemize}
\tightlist
\item
  \textbf{વરસાદ ડેટા}: વાર્ષિક વરસાદી પેટર્ન
\item
  \textbf{કેચમેન્ટ એરિયા}: ઉપલબ્ધ છત/જમીન વિસ્તાર
\item
  \textbf{સ્ટોરેજ ક્ષમતા}: માંગ અને પુરવઠાના આધારે
\item
  \textbf{પાણીની ગુણવત્તા}: ઉપચારની જરૂરિયાતો
\end{itemize}

\textbf{મેમરી ટ્રીક:} ``પકડો, પહોંચાડો, સંગ્રહ કરો, ફિલ્ટર કરો, વાપરો''

\subsection*{પ્રશ્ન 5(a) OR {[}3
ગુણ{]}}\label{uxaaauxab0uxab6uxaa8-5a-or-3-uxa97uxaa3}

\textbf{લાઇફ સાયકલ એનાલિસિસ (LCA) એટલે શું?}

\textbf{જવાબ}:

\textbf{લાઇફ સાયકલ એનાલિસિસ (LCA):}

ઉત્પાદનના સંપૂર્ણ જીવન ચક્ર દરમિયાન તેની પર્યાવરણીય અસરોનું વ્યવસ્થિત મૂલ્યાંકન.

\textbf{LCA તબક્કાઓ:}

{\def\LTcaptype{none} % do not increment counter
\begin{longtable}[]{@{}ll@{}}
\toprule\noalign{}
તબક્કો & વર્ણન \\
\midrule\noalign{}
\endhead
\bottomrule\noalign{}
\endlastfoot
\textbf{કાચો માલ} & સંસાધન નિષ્કર્ષણ \\
\textbf{ઉત્પાદન} & ઉત્પાદન પ્રક્રિયાઓ \\
\textbf{ઉપયોગ તબક્કો} & ઉત્પાદનનો ઉપયોગ \\
\textbf{જીવનનો અંત} & નિકાલ અથવા રીસાયક્લિંગ \\
\end{longtable}
}

\begin{itemize}
\tightlist
\item
  \textbf{હેતુ}: પર્યાવરણીય હોટસ્પોટ ઓળખવા, વિકલ્પોની સરખામણી કરવી
\item
  \textbf{ઉપયોગો}: ઉત્પાદન ડિઝાઇન, નીતિ નિર્ણયો, ઉપભોક્તા પસંદગી
\end{itemize}

\textbf{મેમરી ટ્રીક:} ``જીવન ચક્ર: કાચો, બનાવો, વાપરો, નિકાલ કરો''

\subsection*{પ્રશ્ન 5(b) OR {[}4
ગુણ{]}}\label{uxaaauxab0uxab6uxaa8-5b-or-4-uxa97uxaa3}

\textbf{જૈવ વૈવિધ્ય કાયદા, 2002 ની મુખ્ય લાક્ષણિકતા જણાવો.}

\textbf{જવાબ}:

\textbf{જૈવિક વિવિધતા કાયદો, 2002:}

\textbf{મુખ્ય લાક્ષણો:}

{\def\LTcaptype{none} % do not increment counter
\begin{longtable}[]{@{}ll@{}}
\toprule\noalign{}
લક્ષણ & વર્ણન \\
\midrule\noalign{}
\endhead
\bottomrule\noalign{}
\endlastfoot
\textbf{ત્રિ-સ્તરીય માળખું} & રાષ્ટ્રીય, રાજ્ય, સ્થાનિક જૈવવિવિધતા બોર્ડ \\
\textbf{પૂર્વ મંજૂરી} & બાયો-રિસોર્સ એક્સેસ માટે જરૂરી \\
\textbf{લાભ વહેંચણી} & સ્થાનિક સમુદાયો સાથે ન્યાયસંગત વહેંચણી \\
\textbf{બાયો-પાઇરસી નિવારણ} & પરંપરાગત જ્ઞાનનું સંરક્ષણ \\
\end{longtable}
}

\textbf{મુખ્ય જોગવાઈઓ:}

\begin{itemize}
\tightlist
\item
  \textbf{એક્સેસ નિયમન}: જૈવિક સંસાધનો પર નિયંત્રણ
\item
  \textbf{ટકાઉ ઉપયોગ}: ઉપયોગ દ્વારા સંરક્ષણ
\item
  \textbf{સમુદાયિક અધિકારો}: સ્થાનિક સમુદાયના યોગદાનને માન્યતા
\item
  \textbf{દંડ}: ઉલ્લંઘન માટે કડક સજા
\end{itemize}

\textbf{ઉદ્દેશ્યો}: સંરક્ષણ, ટકાઉ ઉપયોગ, ન્યાયસંગત લાભ વહેંચણી

\textbf{મેમરી ટ્રીક:} ``જૈવવિવિધતા કાયદો: એક્સેસ, લાભ, સંરક્ષણ, સુરક્ષા''

\subsection*{પ્રશ્ન 5(c) OR {[}7
ગુણ{]}}\label{uxaaauxab0uxab6uxaa8-5c-or-7-uxa97uxaa3}

\textbf{5R નો કોન્સેપ્ટ સમજાવો.}

\textbf{જવાબ}:

\textbf{5R કોન્સેપ્ટ:}

પર્યાવરણીય ટકાઉપણા માટે કચરા વ્યવસ્થાપન શ્રેણી.

\textbf{5Rs:}

\begin{center}
\textbf{Mermaid Diagram (Code)}
\begin{verbatim}
{Shaded}
{Highlighting}[]
graph TD
    A[5R શ્રેણી] {-{-}{} B[1. ઇનકાર]}
    A {-{-}{} C[2. ઘટાડો]}
    A {-{-}{} D[3. પુનઃઉપયોગ]}
    A {-{-}{} E[4. પુનઃહેતુ]}
    A {-{-}{} F[5. પુનર્ચક્રણ]}
{Highlighting}
{Shaded}
\end{verbatim}
\end{center}

\textbf{વિગતવાર સમજાવટ:}

{\def\LTcaptype{none} % do not increment counter
\begin{longtable}[]{@{}
  >{\raggedright\arraybackslash}p{(\linewidth - 6\tabcolsep) * \real{0.0857}}
  >{\raggedright\arraybackslash}p{(\linewidth - 6\tabcolsep) * \real{0.3429}}
  >{\raggedright\arraybackslash}p{(\linewidth - 6\tabcolsep) * \real{0.2857}}
  >{\raggedright\arraybackslash}p{(\linewidth - 6\tabcolsep) * \real{0.2857}}@{}}
\toprule\noalign{}
\begin{minipage}[b]{\linewidth}\raggedright
R
\end{minipage} & \begin{minipage}[b]{\linewidth}\raggedright
વ્યાખ્યા
\end{minipage} & \begin{minipage}[b]{\linewidth}\raggedright
ઉદાહરણો
\end{minipage} & \begin{minipage}[b]{\linewidth}\raggedright
ફાયદા
\end{minipage} \\
\midrule\noalign{}
\endhead
\bottomrule\noalign{}
\endlastfoot
\textbf{ઇનકાર (Refuse)} & બિનજરૂરી વસ્તુઓ ટાળવી & પ્લાસ્ટિક બેગ, ડિસ્પોઝેબલ &
કચરા ઉત્પાદન અટકાવવું \\
\textbf{ઘટાડવું (Reduce)} & વપરાશ ન્યૂનીકરણ & ઊર્જા, પાણી, સામગ્રી & સંસાધનની
માંગ ઓછી કરવી \\
\textbf{પુનઃઉપયોગ (Reuse)} & વસ્તુઓનો વારંવાર ઉપયોગ & કન્ટેનર, કપડાં & ઉત્પાદનનું
જીવન લંબાવવું \\
\textbf{પુનઃહેતુ (Repurpose)} & વસ્તુઓ માટે નવા ઉપયોગ શોધવા & ટાયર પ્લાન્ટર,
બોટલ હસ્તકલા & સર્જનાત્મક કચરો દિવર્ટ કરવું \\
\textbf{પુનર્ચક્રણ (Recycle)} & નવા ઉત્પાદનોમાં પ્રક્રિયા કરવી & કાગળ, પ્લાસ્ટિક,
ધાતુઓ & સામગ્રી પુનઃપ્રાપ્તિ \\
\end{longtable}
}

\textbf{અમલીકરણ વ્યૂહરચનાઓ:}

\textbf{વ્યક્તિગત સ્તરે:}

\begin{itemize}
\tightlist
\item
  \textbf{ઇનકાર}: સિંગલ-યુઝ પ્લાસ્ટિકને ના કહો
\item
  \textbf{ઘટાડો}: માત્ર જરૂરી વસ્તુઓ ખરીદો
\item
  \textbf{પુનઃઉપયોગ}: કન્ટેનર અને સામગ્રીનો પુનઃઉપયોગ કરો
\item
  \textbf{પુનઃહેતુ}: સર્જનાત્મક DIY પ્રોજેક્ટ્સ
\item
  \textbf{પુનર્ચક્રણ}: યોગ્ય વર્ગીકરણ અને નિકાલ
\end{itemize}

\textbf{સમુદાય સ્તરે:}

\begin{itemize}
\tightlist
\item
  \textbf{જાગૃતિ કાર્યક્રમો}: 5R સિદ્ધાંતો વિશે શિક્ષણ
\item
  \textbf{ઇન્ફ્રાસ્ટ્રક્ચર}: રીસાયક્લિંગ સુવિધાઓ અને સંગ્રહ સિસ્ટમ
\item
  \textbf{નીતિઓ}: કચરા ઘટાડવાને પ્રોત્સાહન આપનાર નિયમો
\item
  \textbf{પ્રોત્સાહન}: ટકાઉ પ્રથાઓ માટે પુરસ્કારો
\end{itemize}

\textbf{ઔદ્યોગિક સ્તરે:}

\begin{itemize}
\tightlist
\item
  \textbf{ટકાઉપણા માટે ડિઝાઇન}: લાંબા સમય સુધી ચાલતા ઉત્પાદનો
\item
  \textbf{સામગ્રી પસંદગી}: રીસાયકલ અને બાયોડિગ્રેડેબલ સામગ્રી
\item
  \textbf{પરિપત્ર અર્થવ્યવસ્થા}: બંધ-લૂપ ઉત્પાદન સિસ્ટમ
\item
  \textbf{વિસ્તૃત ઉત્પાદક જવાબદારી}: ઉત્પાદક જવાબદારી
\end{itemize}

\textbf{પર્યાવરણીય ફાયદા:}

\begin{itemize}
\tightlist
\item
  \textbf{સંસાધન સંરક્ષણ}: ઘટેલી કાચી સામગ્રી નિષ્કર્ષણ
\item
  \textbf{ઊર્જા બચત}: ઓછી ઉત્પાદન ઊર્જા જરૂરિયાતો
\item
  \textbf{પ્રદૂષણ ઘટાડો}: ઘટેલું કચરો ઉત્પાદન
\item
  \textbf{હવામાન સંરક્ષણ}: ઘટેલું ગ્રીનહાઉસ ગેસ ઉત્સર્જન
\end{itemize}

\textbf{આર્થિક ફાયદા:}

\begin{itemize}
\tightlist
\item
  \textbf{ખર્ચ બચત}: ઓછો નિકાલ અને સામગ્રી ખર્ચ
\item
  \textbf{નોકરી સર્જન}: રીસાયક્લિંગ અને પુનઃઉપયોગ ક્ષેત્રોમાં ગ્રીન જોબ્સ
\item
  \textbf{નવીનતા}: ટકાઉ તકનીકોનો વિકાસ
\item
  \textbf{બજાર તકો}: નવા બિઝનેસ મોડેલ
\end{itemize}

\textbf{સામાજિક ફાયદા:}

\begin{itemize}
\tightlist
\item
  \textbf{સમુદાય સંલગ્નતા}: સામૂહિક પર્યાવરણીય ક્રિયા
\item
  \textbf{આરોગ્ય સુધારણા}: સ્વચ્છ પર્યાવરણ
\item
  \textbf{શિક્ષણ}: પર્યાવરણીય જાગૃતિ અને જવાબદારી
\item
  \textbf{સાંસ્કૃતિક પરિવર્તન}: ટકાઉ જીવનશૈલી અપનાવવી
\end{itemize}

\textbf{પડકારો:}

\begin{itemize}
\tightlist
\item
  \textbf{વર્તન પરિવર્તન}: વપરાશની આદતો પર કાબુ મેળવવો
\item
  \textbf{ઇન્ફ્રાસ્ટ્રક્ચર}: પર્યાપ્ત રીસાયક્લિંગ સુવિધાઓ
\item
  \textbf{આર્થિક અવરોધો}: પ્રારંભિક રોકાણની જરૂરિયાતો
\item
  \textbf{નીતિ સમર્થન}: સરકારી નિયમો અને પ્રોત્સાહન
\end{itemize}

\textbf{સફળતાની વાર્તાઓ:}

\begin{itemize}
\tightlist
\item
  \textbf{ઝીરો વેસ્ટ શહેરો}: સાન ફ્રાન્સિસ્કો, કામિકાત્સુ
\item
  \textbf{કોર્પોરેટ પહેલ}: કંપની 5R કાર્યક્રમો
\item
  \textbf{શાળા કાર્યક્રમો}: વિદ્યાર્થી પર્યાવરણીય શિક્ષણ
\item
  \textbf{સમુદાય પ્રોજેક્ટ્સ}: સ્થાનિક કચરા ઘટાડવાના પ્રયાસો
\end{itemize}

\textbf{મેમરી ટ્રીક:} ``ખરેખર ઘટાડો પુનઃઉપયોગ પુનઃહેતુ પુનર્ચક્રણ''

\begin{center}\rule{0.5\linewidth}{0.5pt}\end{center}

\end{document}