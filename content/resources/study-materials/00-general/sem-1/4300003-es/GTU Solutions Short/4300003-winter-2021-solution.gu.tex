\documentclass{article}

% content/resources/templates/preamble.tex
\usepackage[margin=0.6in]{geometry}
\author{Milav Dabgar}
\usepackage{amsmath,amssymb,amsthm}
\usepackage{booktabs}
\usepackage{multirow}
\usepackage{xcolor}
\usepackage{tcolorbox}
\tcbuselibrary{breakable,skins}
\usepackage[colorlinks=true,linkcolor=blue]{hyperref}
\usepackage{titlesec}
\usepackage{enumitem}
\usepackage{tikz}
\usepackage{pgfplots}
\usepackage{circuitikz}
\usepackage[version=4]{mhchem}
\usepackage{longtable}
\usepackage{array}
\usepackage{float}
\usepackage{caption}
\usepackage{listings}

\lstset{
  basicstyle=\small\ttfamily,
  breaklines=true,
  breakatwhitespace=false,
  postbreak=\mbox{\textcolor{red}{$\hookrightarrow$}\space},
  float=false,
  numbers=left,
  numberstyle=\tiny\color{gray},
  numbersep=10pt,
  xleftmargin=2em,
  keywordstyle=\color{blue},
  commentstyle=\color{green!60!black},
  stringstyle=\color{purple},
  backgroundcolor=\color{gray!5},
  showstringspaces=false,
  tabsize=2,
  captionpos=b,
  keepspaces=true,
  columns=flexible
}

\pgfplotsset{compat=1.18}
\usetikzlibrary{shapes,arrows,positioning,calc,patterns,decorations.pathmorphing,decorations.markings,arrows.meta}

% Color scheme
\definecolor{headcolor}{RGB}{0,102,204}
\definecolor{keycolor}{RGB}{220,20,60}
\definecolor{solutioncolor}{RGB}{34,139,34}
\definecolor{mnemoniccolor}{RGB}{148,0,211}
\definecolor{codecolor}{RGB}{0,0,100}

% Spacing
\setlength{\parskip}{3pt}
\setlist[itemize]{nosep}
\setlist[enumerate]{nosep}

% Title formatting
\titleformat{\section}{\Large\bfseries\color{headcolor}}{\thesection}{1em}{}
\titleformat{\subsection}{\large\bfseries\color{headcolor}}{\thesubsection}{1em}{}

% Pandoc tightlist compatibility
\providecommand{\tightlist}{%
  \setlength{\itemsep}{0pt}\setlength{\parskip}{0pt}}

% Pandoc longtable compatibility
\newcounter{none}
\def\thenone{}


% content/resources/templates/gujarati-boxes.tex
\usepackage{fontspec}
\usepackage{polyglossia}

% Set Gujarati as main language (document is primarily in Gujarati)
% Note: gloss-gujarati.ldf doesn't exist in polyglossia, but it will use hyphenation patterns
\setdefaultlanguage{gujarati}
\setotherlanguage{english}

% Configure Gujarati font properly
% Use Language=Default to prevent polyglossia from trying to add language-specific features
% that don't exist for Gujarati, which causes "empty feature" warnings
\newfontfamily\gujaratifont[Script=Gujarati,AutoFakeBold=2.5,AutoFakeSlant=0.3]{Noto Sans Gujarati}
\setmainfont[Script=Gujarati,AutoFakeBold=2.5,AutoFakeSlant=0.3]{Noto Sans Gujarati}
% Use Noto Sans Gujarati for monospace to support Gujarati in text
\setmonofont[Scale=0.9]{Noto Sans Gujarati}

% Configure English to use the same font
\newfontfamily\englishfont[Script=Gujarati,AutoFakeBold=2.5,AutoFakeSlant=0.3]{Noto Sans Gujarati}

% Translations for polyglossia
\gappto\captionsgujarati{
  \renewcommand{\tablename}{કોષ્ટક}
  \renewcommand{\figurename}{આકૃતિ}
}

% Helper for TikZ nodes to ensure Gujarati font
\newcommand{\gu}[1]{{\gujaratifont #1}}

% Custom environments
\newtcolorbox{solutionbox}{
    breakable,
    enhanced,
    colback=solutioncolor!5!white,
    colframe=solutioncolor!75!black,
    fonttitle=\bfseries,
    title=જવાબ
}

\newtcolorbox{solutionboxnobreak}{
 colback=solutioncolor!5!white,
 colframe=solutioncolor!75!black,
 fonttitle=\bfseries,
 title=જવાબ
}

\newtcolorbox{keyformula}{
 breakable,
 enhanced,
 colback=keycolor!5!white,
 colframe=keycolor!75!black,
 fonttitle=\bfseries,
 title=રાસાયણિક સમીકરણ/સૂત્ર
}

\newtcolorbox{mnemonicbox}{
 breakable,
 enhanced,
 colback=mnemoniccolor!5!white,
 colframe=mnemoniccolor!75!black,
 fonttitle=\bfseries,
 title=મેમરી ટ્રીક
}


% Custom commands for GTU solutions
% This file defines semantic commands for consistent formatting

% Question command with automatic formatting
\newcommand{\question}[2]{%
  \section*{Question #1}%
  \textbf{#2}%
}

% OR question variant
\newcommand{\questionor}[2]{%
  \section*{Question #1 OR}%
  \textbf{#2}%
}

% Proper table environment with caption
\newenvironment{answertable}[1]{%
  \begin{table}[htbp]
  \centering
  \caption{#1}
}{%
  \end{table}
}

% Proper figure environment for diagrams
\newenvironment{answerdiagram}[1]{%
  \begin{figure}[htbp]
  \centering
  \caption{#1}
}{%
  \end{figure}
}

% Semantic markup for key terms
\newcommand{\keyword}[1]{\textbf{#1}}
\newcommand{\code}[1]{\texttt{#1}}
\newcommand{\classname}[1]{\texttt{#1}}
\newcommand{\methodname}[1]{\texttt{#1}}

% Proper quotation marks
\newcommand{\mnemonic}[1]{``#1''}


\title{પર્યાવરણ અને ટકાઉપણું (4300003) - શિયાળુ 2021 ઉકેલ}
\date{March 24, 2022}

\begin{document}
\maketitle

\questionmarks{1}{}{14}
\textbf{કોઈપણ સાત પ્રશ્નોના જવાબ આપો.}

\subsubsection{1. 'પરિસ્થિતિશાસ્ત્ર' અને 'નિવસનતંત્ર' ની વ્યાખ્યા આપો.}

\begin{solutionbox}
    \textbf{જવાબ:}
    \textbf{Ecology} એ જીવિત જીવોના તેમના પર્યાવરણ સાથેના સંબંધોનો વૈજ્ઞાનિક અભ્યાસ છે. \textbf{Ecosystem} એ એકમ તરીકે કામ કરતા જીવો અને તેમના ભૌતિક પર્યાવરણનો જૈવિક સમુદાય છે.

    \begin{answertable}{શબ્દો અને વ્યાખ્યાઓ}
    \begin{tabulary}{\linewidth}{L L L}
        \toprule
        \textbf{શબ્દ} & \textbf{વ્યાખ્યા} & \textbf{ઉદાહરણ} \\
        \midrule
        \textbf{Ecology} & જીવ-પર્યાવરણ સંબંધોનો અભ્યાસ & વન ecology \\
        \textbf{Ecosystem} & જીવંત અને નિર્જીવ ઘટકોની પરસ્પર ક્રિયા & તળાવનું ecosystem \\
        \bottomrule
    \end{tabulary}
    \end{answertable}

    \begin{itemize}
        \item \textbf{જૈવિક ઘટકો}: તંત્રમાં જીવંત જીવો
        \item \textbf{અજૈવિક ઘટકો}: હવા, પાણી, માટી જેવા નિર્જીવ પરિબળો
    \end{itemize}

    \begin{mnemonicbox}દરેક ઘટક એકસાથે રહે છે (Ecology Creates Living Together)\end{mnemonicbox}
\end{solutionbox}

\subsubsection{2. 'પ્રદૂષણ' અને 'પ્રદૂષક' ની વ્યાખ્યા આપો.}

\begin{solutionbox}
    \textbf{જવાબ:}
    \textbf{Pollution} એ પર્યાવરણમાં હાનિકારક પદાર્થોનો પ્રવેશ છે જે પ્રતિકૂળ અસરો લાવે છે. \textbf{Pollutant} એ કોઈપણ પદાર્થ છે જે વધારે માત્રામાં હાજર હોય ત્યારે પ્રદૂષણ લાવે છે.

    \begin{answertable}{પ્રદૂષણ શબ્દો}
    \begin{tabulary}{\linewidth}{L L L}
        \toprule
        \textbf{શબ્દ} & \textbf{વ્યાખ્યા} & \textbf{પ્રકારો} \\
        \midrule
        \textbf{Pollution} & પર્યાવરણીય દૂષણ & હવા, પાણી, માટી, અવાજ \\
        \textbf{Pollutant} & હાનિકારક પદાર્થ & ભૌતિક, રાસાયણિક, જૈવિક \\
        \bottomrule
    \end{tabulary}
    \end{answertable}

    \begin{itemize}
        \item \textbf{પ્રાથમિક પ્રદૂષકો}: સીધા વિસર્જિત પદાર્થો
        \item \textbf{ગૌણ પ્રદૂષકો}: વાતાવરણમાં પ્રતિક્રિયાઓથી બનેલા
    \end{itemize}

    \begin{mnemonicbox}પ્રદૂષણ સમસ્યાઓ પેદા કરે છે (Pollution Produces Problems)\end{mnemonicbox}
\end{solutionbox}

\subsubsection{3. 'અવાજનું પ્રદૂષણ' એટલે શું? ધ્વનિની તીવ્રતાનો એકમ શું છે?}

\begin{solutionbox}
    \textbf{જવાબ:}
    \textbf{Noise pollution} એ અનિચ્છિત અથવા વધુ પડતો અવાજ છે જે માનવીય પ્રવૃત્તિઓને ખલેલ પહોંચાડે છે. ધ્વનિની તીવ્રતાનો એકમ \textbf{decibel (dB)} છે.

    \begin{answertable}{અવાજ સ્તર}
    \begin{tabulary}{\linewidth}{L L L}
        \toprule
        \textbf{અવાજનું સ્તર} & \textbf{સ્રોત} & \textbf{અસર} \\
        \midrule
        30-40 dB & પુસ્તકાલય & આરામદાયક \\
        60-70 dB & ટ્રાફિક & હેરાનીજનક \\
        90+ dB & ઉદ્યોગ & હાનિકારક \\
        \bottomrule
    \end{tabulary}
    \end{answertable}

    \begin{itemize}
        \item \textbf{સાંભળવાની સીમા}: 0 dB
        \item \textbf{પીડાની સીમા}: 120 dB
    \end{itemize}

    \begin{mnemonicbox}Decibel નુકસાન નક્કી કરે છે (dB Determines Damage)\end{mnemonicbox}
\end{solutionbox}

\subsubsection{4. ઘન કચરાનું વ્યવસ્થાપન શું છે? તેના હેતુઓ જણાવો.}

\begin{solutionbox}
    \textbf{જવાબ:}
    \textbf{Solid waste management} એ પર્યાવરણીય અસર ઘટાડવા અને જાહેર આરોગ્યની સુરક્ષા માટે કચરાના ઉત્પાદનથી અંતિમ નિકાલ સુધીનું વ્યવસ્થિત સંચાલન છે.

    \textbf{હેતુઓ:}
    \begin{itemize}
        \item \textbf{જાહેર આરોગ્ય સંરક્ષણ}: રોગ પ્રસારણ અટકાવવું
        \item \textbf{પર્યાવરણ સંરક્ષણ}: પ્રદૂષણ અને દૂષણ ઘટાડવું
        \item \textbf{સંસાધન પુનઃપ્રાપ્તિ}: સામગ્રીનું પુનઃઉપયોગ અને રીસાયકલિંગ
        \item \textbf{ખર્ચ અસરકારકતા}: આર્થિક કચરા નિયંત્રણ
    \end{itemize}

    \begin{mnemonicbox}લોકો સંસાધન સંરક્ષણની અપેક્ષા રાખે છે (Protection, Environment, Resource, Cost)\end{mnemonicbox}
\end{solutionbox}

\subsubsection{5. સોલાર સેલના પ્રકારો સમજાવો.}

\begin{solutionbox}
    \textbf{જવાબ:}
    Solar cells સૂર્યપ્રકાશને photovoltaic effect દ્વારા સીધી વીજળીમાં રૂપાંતરિત કરે છે.

    \begin{answertable}{સોલાર સેલના પ્રકારો}
    \begin{tabulary}{\linewidth}{L L L L}
        \toprule
        \textbf{પ્રકાર} & \textbf{કાર્યક્ષમતા} & \textbf{કિંમત} & \textbf{ઉપયોગ} \\
        \midrule
        \textbf{Monocrystalline} & 15-20\% & વધુ & આવાસીય \\
        \textbf{Polycrystalline} & 13-16\% & મધ્યમ & વ્યાવસાયિક \\
        \textbf{Thin Film} & 7-13\% & ઓછી & વિશાળ પ્રમાણ \\
        \bottomrule
    \end{tabulary}
    \end{answertable}

    \begin{itemize}
        \item \textbf{Silicon-based}: સૌથી સામાન્ય પ્રકાર
        \item \textbf{Non-silicon}: ઉદીયમાન તકનીકો
    \end{itemize}

    \begin{mnemonicbox}મોટાભાગના લોકો વિચારે છે (Mono, Poly, Thin-film)\end{mnemonicbox}
\end{solutionbox}

\subsubsection{6. 'આબોહવા (જલવાયુ) પરિવર્તન' શું છે?}

\begin{solutionbox}
    \textbf{જવાબ:}
    \textbf{Climate change} એ મુખ્યત: માનવીય પ્રવૃત્તિઓ અને greenhouse gas ઉત્સર્જનને કારણે વૈશ્વિક તાપમાન અને હવામાન પેટર્નમાં લાંબા ગાળાના ફેરફારોનો સંદર્ભ આપે છે.

    \textbf{કારણો:}
    \begin{itemize}
        \item \textbf{Greenhouse gases}: CO\textsubscript{2}, CH\textsubscript{4}, N\textsubscript{2}O ઉત્સર્જન
        \item \textbf{વનનાશ}: કાર્બન શોષણમાં ઘટાડો
        \item \textbf{ઔદ્યોગિક પ્રવૃત્તિઓ}: અશ્મિભૂત ઇંધનનું બર્નિંગ
    \end{itemize}

    \textbf{અસરો:}
    \begin{itemize}
        \item \textbf{વધતું તાપમાન}: વૈશ્વિક ઉષ્ણતા
        \item \textbf{દરિયાઈ સ્તરમાં વધારો}: બરફ પીગળવાથી
    \end{itemize}

    \begin{mnemonicbox}પરિવર્તન પરિણામો બનાવે છે (Change Creates Consequences)\end{mnemonicbox}
\end{solutionbox}

\subsubsection{7. C.F.C શું છે?}

\begin{solutionbox}
    \textbf{જવાબ:}
    \textbf{CFC (Chlorofluorocarbon)} એ કાર્બન, ફ્લોરિન અને ક્લોરિન અણુઓ ધરાવતા કૃત્રિમ સંયોજનો છે, જે અગાઉ refrigeration અને aerosols માં વપરાતા હતા.

    \textbf{ગુણધર્મો:}
    \begin{itemize}
        \item \textbf{ઓઝોન નાશક}: stratospheric ozone નાશ કરે છે
        \item \textbf{Greenhouse gas}: વૈશ્વિક ઉષ્ણતામાં યોગદાન
        \item \textbf{સ્થિર સંયોજનો}: લાંબા વાતાવરણીય આયુષ્ય
        \item \textbf{Montreal Protocol}: આંતરરાષ્ટ્રીય પ્રતિબંધ કરાર
    \end{itemize}

    \begin{mnemonicbox}ક્લોરિન ફ્લોરિન કાર્બન (CFC ઘટકો)\end{mnemonicbox}
\end{solutionbox}

\subsubsection{8. ISO-14000 ના ફાયદા આપો.}

\begin{solutionbox}
    \textbf{જવાબ:}
    \textbf{ISO 14000} પર્યાવરણીય વ્યવસ્થાપન પ્રણાલીઓ માટેનું આંતરરાષ્ટ્રીય ધોરણ છે.

    \textbf{ફાયદા:}
    \begin{itemize}
        \item \textbf{પર્યાવરણીય અનુપાલન}: કાનૂની જરૂરિયાતોની પૂર્તિ
        \item \textbf{ખર્ચ ઘટાડો}: કુશળ સંસાધન ઉપયોગ
        \item \textbf{બજાર ફાયદો}: કંપનીની છબીમાં સુધારો
        \item \textbf{જોખમ વ્યવસ્થાપન}: પર્યાવરણીય દુર્ઘટનાઓ અટકાવવી
    \end{itemize}

    \begin{answertable}{ISO-14000 ના ફાયદા}
    \begin{tabulary}{\linewidth}{L L L}
        \toprule
        \textbf{ફાયદો} & \textbf{અસર} & \textbf{પરિણામ} \\
        \midrule
        \textbf{અનુપાલન} & કાનૂની સુરક્ષા & દંડ ટાળવો \\
        \textbf{કુશળતા} & સંસાધન બચત & ખર્ચ ઘટાડો \\
        \textbf{છબી} & બજાર સ્થિતિ & સ્પર્ધાત્મક ફાયદો \\
        \bottomrule
    \end{tabulary}
    \end{answertable}

    \begin{mnemonicbox}કંપનીઓ બજાર માન્યતા મેળવે છે (Compliance, Cost, Market, Risk)\end{mnemonicbox}
\end{solutionbox}

\subsubsection{9. ભારતમાં પર્યાવરણ સંબંધિત વિવિધ કાયદાઓની યાદી બનાવો.}

\begin{solutionbox}
    \textbf{જવાબ:}
    ભારતમાં વ્યાપક પર્યાવરણીય કાયદાકીય માળખું છે.

    \textbf{મુખ્ય કાયદાઓ:}
    \begin{itemize}
        \item \textbf{Air Act (1981)}: હવા પ્રદૂષણ નિયંત્રણ
        \item \textbf{Water Act (1974)}: પાણી પ્રદૂષણ અટકાવવા
        \item \textbf{Environment Protection Act (1986)}: વ્યાપક પર્યાવરણીય કાયદો
        \item \textbf{Wildlife Protection Act (1972)}: જૈવવિવિધતા સંરક્ષણ
        \item \textbf{Forest Conservation Act (1980)}: વન સંરક્ષણ
    \end{itemize}

    \begin{mnemonicbox}તમામ પાણી પર્યાવરણ વન્યજીવ વન (AWEWF)\end{mnemonicbox}
\end{solutionbox}

\subsubsection{10. વરસાદના પાણીના સંચયની વિવિધ પદ્ધતિઓની યાદી બનાવો.}

\begin{solutionbox}
    \textbf{જવાબ:}
    \textbf{Rainwater harvesting} ભવિષ્યના ઉપયોગ માટે વરસાદી પાણીનું સંગ્રહ અને સંચય કરે છે.

    \textbf{પદ્ધતિઓ:}
    \begin{itemize}
        \item \textbf{છતથી સંચય}: છતમાંથી સીધો સંગ્રહ
        \item \textbf{સપાટીની વહેણ સંચય}: જમીનની સપાટીમાંથી
        \item \textbf{રિચાર્જ પિટ્સ}: ભૂગર્ભજળ રિચાર્જિંગ
        \item \textbf{ચેક ડેમ}: નદીના પાણીનો સંગ્રહ
    \end{itemize}

    \begin{answertable}{સંચય પદ્ધતિઓ}
    \begin{tabulary}{\linewidth}{L L L}
        \toprule
        \textbf{પદ્ધતિ} & \textbf{ઉપયોગ} & \textbf{ફાયદો} \\
        \midrule
        \textbf{છત} & શહેરી વિસ્તારો & સીધો ઉપયોગ \\
        \textbf{સપાટી} & ગ્રામીણ વિસ્તારો & મોટી માત્રા \\
        \textbf{રિચાર્જ} & પાણીનું સ્તર & ભૂગર્ભજળ \\
        \bottomrule
    \end{tabulary}
    \end{answertable}

    \begin{mnemonicbox}છત સપાટી રિચાર્જ ચેક (RSRC)\end{mnemonicbox}
\end{solutionbox}

\questionmarks{2}{a}{3}
\textbf{ટૂંક નોંધ લખો: ફૂડ ચેઇન.}

\begin{solutionbox}
    \textbf{જવાબ:}
    \textbf{Food chain} ecosystem માં વિવિધ trophic levels દ્વારા ઊર્જા અને પોષકતત્વોના પ્રવાહને દર્શાવે છે.

    \begin{center}
    \begin{tikzpicture}[node distance=1.5cm, auto]
        \node (prod) [gtu block, align=center] {ઉત્પાદકો\\છોડ};
        \node (primary) [gtu block, right=of prod, align=center] {પ્રાથમિક ઉપભોક્તા\\શાકાહારી};
        \node (secondary) [gtu block, right=of primary, align=center] {ગૌણ ઉપભોક્તા\\માંસાહારી};
        \node (tertiary) [gtu block, below=of secondary, align=center] {તૃતીય ઉપભોક્તા\\ટોપ શિકારી};
        \node (decomp) [gtu block, below=of primary, align=center] {વિઘટનકર્તા\\બેક્ટેરિયા/ફૂગ};

        \draw [gtu arrow] (prod) -- (primary);
        \draw [gtu arrow] (primary) -- (secondary);
        \draw [gtu arrow] (secondary) -- (tertiary);
        \draw [gtu arrow] (tertiary) -- (decomp);
        \draw [gtu arrow] (primary) -- (decomp);
        \draw [gtu arrow] (prod) -- (decomp);
    \end{tikzpicture}
    \end{center}

    \begin{itemize}
        \item \textbf{ઊર્જા સ્થાનાંતરણ}: આગલા સ્તરે માત્ર 10\% જાય છે
        \item \textbf{Biomass પિરામિડ}: ઉચ્ચ સ્તરે ઘટતું જાય છે
    \end{itemize}

    \begin{mnemonicbox}છોડ પ્રાથમિક શક્તિ પૂરી પાડે છે (Producer to Predator Path)\end{mnemonicbox}
\end{solutionbox}

\questionmarks{2}{a}{3}
\textbf{Ecosystem ને અસર કરતાં ઘટકો સમજાવો.}

\begin{solutionbox}
    \textbf{જવાબ:}
    Ecosystems વિવિધ જૈવિક અને અજૈવિક ઘટકોથી પ્રભાવિત થાય છે.

    \textbf{ઘટકો:}
    \begin{itemize}
        \item \textbf{આબોહવા ઘટકો}: તાપમાન, વરસાદ, ભેજ
        \item \textbf{માટીના ઘટકો}: pH, પોષકતત્વો, રચના
        \item \textbf{જૈવિક ઘટકો}: જાતિઓના સંબંધો, વસ્તીની ઘનતા
        \item \textbf{માનવીય ઘટકો}: પ્રદૂષણ, નિવાસસ્થાન નાશ
    \end{itemize}

    \begin{answertable}{Ecosystem ના ઘટકો}
    \begin{tabulary}{\linewidth}{L L L}
        \toprule
        \textbf{ઘટકનો પ્રકાર} & \textbf{ઘટકો} & \textbf{અસર} \\
        \midrule
        \textbf{અજૈવિક} & આબોહવા, માટી & નિવાસસ્થાનની સ્થિતિ \\
        \textbf{જૈવિક} & જીવો & જાતિઓના સંબંધો \\
        \textbf{માનવજન્ય} & માનવીય પ્રવૃત્તિઓ & Ecosystem ખલેલ \\
        \bottomrule
    \end{tabulary}
    \end{answertable}

    \begin{mnemonicbox}આબોહવા માટી જીવવિજ્ઞાન માનવો (CSBH)\end{mnemonicbox}
\end{solutionbox}

\questionmarks{2}{b}{3}
\textbf{ટૂંક નોંધ લખો: કાલ્પનિક જળ}

\begin{solutionbox}
    \textbf{જવાબ:}
    \textbf{Virtual water} એ માલ અને સેવાઓના ઉત્પાદનમાં વપરાતું છુપાયેલું પાણી છે, જે supply chain માં કુલ પાણીના વપરાશને દર્શાવે છે.

    \textbf{ઉદાહરણો:}
    \begin{itemize}
        \item \textbf{1 kg ઘઉં}: 1,300 લિટર virtual water
        \item \textbf{1 kg બીફ}: 15,400 લિટર virtual water
        \item \textbf{1 કપાસનું t-shirt}: 2,700 લિટર virtual water
    \end{itemize}

    \begin{itemize}
        \item \textbf{Water footprint}: કુલ virtual water વપરાશ
        \item \textbf{વેપારની અસરો}: પાણીથી સમૃદ્ધ દેશો virtual water નિકાસ કરે છે
    \end{itemize}

    \begin{mnemonicbox}વર્ચ્યુઅલ વોટર વર્લ્ડવાઇડ (VWW)\end{mnemonicbox}
\end{solutionbox}

\questionmarks{2}{b}{3}
\textbf{'જૈવ-વૈવિધ્ય' એટલે શું? જૈવ-વૈવિધ્યના પ્રકારો જણાવો.}

\begin{solutionbox}
    \textbf{જવાબ:}
    \textbf{Biodiversity} એ પૃથ્વી પર આનુવંશિક, જાતિઓ અને ecosystem સ્તરે જીવન સ્વરૂપોની વિવિધતા છે.

    \textbf{પ્રકારો:}
    \begin{itemize}
        \item \textbf{આનુવંશિક વિવિધતા}: જાતિઓની અંદર વિવિધતા
        \item \textbf{જાતિઓ વિવિધતા}: વિવિધ જાતિઓની સંખ્યા
        \item \textbf{Ecosystem વિવિધતા}: નિવાસસ્થાન અને સમુદાયોની વિવિધતા
    \end{itemize}

    \begin{center}
    \begin{tikzpicture}[node distance=1.5cm, auto, align=center]
        \node (root) [gtu block] {Biodiversity};
        \node (genetic) [gtu block, below left=of root, xshift=-1cm] {આનુવંશિક};
        \node (species) [gtu block, below=of root] {જાતિઓ};
        \node (ecosystem) [gtu block, below right=of root, xshift=1cm] {Ecosystem};

        \node (dna) [gtu state, below=of genetic] {DNA વિવિધતા};
        \node (pop) [gtu state, below=of dna] {વસ્તી\\આનુવંશિકતા};
        
        \node (flora) [gtu state, below=of species] {વનસ્પતિ};
        \node (fauna) [gtu state, below=of flora] {પ્રાણી};
        
        \node (terr) [gtu state, below=of ecosystem] {જમીની};
        \node (aqua) [gtu state, below=of terr] {જળીય};

        \draw [gtu arrow] (root) -- (genetic);
        \draw [gtu arrow] (root) -- (species);
        \draw [gtu arrow] (root) -- (ecosystem);
        \draw [gtu arrow] (genetic) -- (dna);
        \draw [gtu arrow] (dna) -- (pop);
        \draw [gtu arrow] (species) -- (flora);
        \draw [gtu arrow] (flora) -- (fauna);
        \draw [gtu arrow] (ecosystem) -- (terr);
        \draw [gtu arrow] (terr) -- (aqua);
    \end{tikzpicture}
    \end{center}

    \begin{mnemonicbox}જીન્સ જાતિઓ Ecosystems (GSE)\end{mnemonicbox}
\end{solutionbox}

\questionmarks{2}{c}{4}
\textbf{કાર્બનચક્ર સમજાવો.}

\begin{solutionbox}
    \textbf{જવાબ:}
    \textbf{Carbon cycle} પૃથ્વીના વાતાવરણ, જમીન, પાણી અને જીવોમાં કાર્બનની હિલચાલનું વર્ણન કરે છે.

    \begin{center}
    \begin{tikzpicture}[node distance=2cm, auto, align=center]
        \node (atmos) [gtu block] {વાતાવરણીય CO\textsubscript{2}};
        \node (plants) [gtu block, below left=of atmos] {પ્રકાશસંશ્લેષણ\\(છોડ)};
        \node (animals) [gtu block, below=of plants] {પ્રાણીઓનું સેવન};
        \node (decomp) [gtu block, right=of animals] {વિઘટન};
        \node (ocean) [gtu block, below right=of atmos] {સમુદ્રી શોષણ};
        \node (fossil) [gtu block, below=of ocean] {અશ્મિભૂત ઇંધન};

        \draw [gtu arrow] (atmos) -- (plants);
        \draw [gtu arrow] (plants) -- (animals);
        \draw [gtu arrow] (animals) -- (decomp);
        \draw [gtu arrow] (plants) -- (decomp);
        \draw [gtu arrow] (decomp) -- (atmos);
        \draw [gtu arrow, <->] (atmos) -- (ocean);
        \draw [gtu arrow] (fossil) -- (atmos);
    \end{tikzpicture}
    \end{center}

    \textbf{પ્રક્રિયાઓ:}
    \begin{itemize}
        \item \textbf{પ્રકાશસંશ્લેષણ}: છોડ દ્વારા CO\textsubscript{2} શોષણ
        \item \textbf{શ્વસન}: જીવો દ્વારા CO\textsubscript{2} છોડવું
        \item \textbf{વિઘટન}: વાતાવરણમાં કાર્બન પરત આવવું
        \item \textbf{સમુદ્રી આપલે}: દરિયાઈ પાણીમાં CO\textsubscript{2} ઓગળવું
    \end{itemize}

    \begin{mnemonicbox}છોડ શ્વાસ લે છે, મરે છે, સમુદ્ર (PBDO)\end{mnemonicbox}
\end{solutionbox}

\questionmarks{2}{c}{4}
\textbf{જલીયચક્ર દોરો અને સમજાવો}

\begin{solutionbox}
    \textbf{જવાબ:}
    \textbf{Hydrologic cycle} એ વાતાવરણ, જમીન અને મહાસાગરોમાં પાણીની સતત હિલચાલ છે.

    \begin{center}
    \begin{tikzpicture}[node distance=2cm, auto, align=center]
        \node (ocean) [gtu block] {સમુદ્ર};
        \node (vapor) [gtu block, above=of ocean] {જળવાષ્પ};
        \node (clouds) [gtu block, right=of vapor] {વાદળો};
        \node (land) [gtu block, right=of ocean] {જમીનની સપાટી};
        \node (ground) [gtu block, below=of land] {ભૂગર્ભજળ};

        \draw [gtu arrow] (ocean) -- node[left] {બાષ્પીભવન} (vapor);
        \draw [gtu arrow] (vapor) -- node[above] {ઘનીકરણ} (clouds);
        \draw [gtu arrow] (clouds) -- node[right] {વરસાદ} (land);
        \draw [gtu arrow] (land) -- node[above] {વહેણ} (ocean);
        \draw [gtu arrow] (land) -- node[right] {ઘૂસણખોરી} (ground);
        \draw [gtu arrow] (ground) -- (ocean);
    \end{tikzpicture}
    \end{center}

    \textbf{પ્રક્રિયાઓ:}
    \begin{itemize}
        \item \textbf{બાષ્પીભવન}: પાણીથી વાષ્પમાં રૂપાંતર
        \item \textbf{ઘનીકરણ}: વાષ્પથી પ્રવાહીમાં રૂપાંતર
        \item \textbf{વરસાદ}: વરસાદ, બરફનું નિર્માણ
        \item \textbf{ઘૂસણખોરી}: ભૂગર્ભજળ રિચાર્જ
    \end{itemize}

    \begin{mnemonicbox}દરેક વાદળ વરસાદ લાવે છે (ECPR)\end{mnemonicbox}
\end{solutionbox}

\questionmarks{2}{d}{4}
\textbf{હવાના પ્રદૂષણને નિયંત્રણમાં વપરાતા સાધનો જણાવો અને કોઈ એક સમજાવો.}

\begin{solutionbox}
    \textbf{જવાબ:}
    હવા પ્રદૂષણ નિયંત્રણ સાધનો ઔદ્યોગિક ઉત્સર્જનમાંથી પ્રદૂષકો દૂર કરે છે.

    \textbf{સાધનોની યાદી:}
    \begin{itemize}
        \item \textbf{Cyclone separators}: કણીય દૂરીકરણ
        \item \textbf{Electrostatic precipitators}: ઝીણા કણોનો સંગ્રહ
        \item \textbf{Bag filters}: કાપડ ગાળક
        \item \textbf{Scrubbers}: ગેસ શોષણ
    \end{itemize}

    \textbf{Electrostatic Precipitator:}
    \begin{center}
    \begin{tikzpicture}[scale=0.8]
        % Box
        \draw[thick] (0,0) rectangle (6,4);
        
        % Discharge electrodes
        \foreach \x in {1,2,3,4,5} {
            \draw[thick] (\x,3.5) -- (\x,0.5);
            \node at (\x,2) {+};
        }
        
        % Collection plates
        \draw[thick] (0.5,0) -- (0.5,4);
        \draw[thick] (5.5,0) -- (5.5,4);
        \node at (0.2, 2) {-};
        \node at (5.8, 2) {-};
        
        % Flow
        \draw[thick, ->] (-1,2) -- (0,2) node[midway, above] {દૂષિત ગેસ};
        \draw[thick, ->] (6,2) -- (7,2) node[midway, above] {સાફ ગેસ};
        
        % Hoppers
        \draw (1,0) -- (1.5,-1) -- (2,0);
        \draw (3,0) -- (3.5,-1) -- (4,0);
        \node at (2.5,-1.5) {ધૂળ હોપર};
    \end{tikzpicture}
    \end{center}

    \begin{itemize}
        \item \textbf{ચાર્જિંગ}: કણો વિદ્યુત ચાર્જ મેળવે છે
        \item \textbf{સંગ્રહ}: ચાર્જ થયેલા કણો પ્લેટ્સ તરફ આકર્ષાય છે
        \item \textbf{કાર્યક્ષમતા}: ઝીણા કણોનું 99\% દૂરીકરણ
    \end{itemize}

    \begin{mnemonicbox}ચાર્જ કલેક્ટ ક્લીન (CCC)\end{mnemonicbox}
\end{solutionbox}

\questionmarks{2}{d}{4}
\textbf{પર્યાવરણીય પ્રદૂષણના પ્રકારો જણાવો અને અવાજના પ્રદૂષણની અસરો જણાવો}

\begin{solutionbox}
    \textbf{જવાબ:}
    \textbf{પર્યાવરણીય પ્રદૂષણના પ્રકારો:}
    \begin{itemize}
        \item \textbf{હવા પ્રદૂષણ}: વાતાવરણીય દૂષણ
        \item \textbf{પાણી પ્રદૂષણ}: જળીય દૂષણ
        \item \textbf{માટી પ્રદૂષણ}: જમીનનું દૂષણ
        \item \textbf{અવાજ પ્રદૂષણ}: ધ્વનિ દૂષણ
    \end{itemize}

    \textbf{Noise Pollution ની અસરો:}
    \begin{itemize}
        \item \textbf{આરોગ્યની અસરો}: સાંભળવાની ખોટ, તણાવ, હાયપરટેન્શન
        \item \textbf{માનસિક અસરો}: હેરાનગતિ, ઊંઘનો ખલેલ
        \item \textbf{કામગીરીની અસરો}: ધ્યાન ઘટવું, ઉત્પાદકતા ઘટવી
        \item \textbf{વાતચીતની અસરો}: બોલચાલમાં અવરોધ
    \end{itemize}

    \begin{answertable}{અવાજ પ્રદૂષણ અસરો}
    \begin{tabulary}{\linewidth}{L L L}
        \toprule
        \textbf{અસરનો પ્રકાર} & \textbf{લક્ષણો} & \textbf{અસર} \\
        \midrule
        \textbf{શારીરિક} & સાંભળવાનું નુકસાન & કાયમી ખોટ \\
        \textbf{માનસિક} & તણાવ, ચિંતા & આરોગ્ય સમસ્યાઓ \\
        \textbf{સામાજિક} & વાતચીતની સમસ્યાઓ & સંબંધોમાં તણાવ \\
        \bottomrule
    \end{tabulary}
    \end{answertable}

    \begin{mnemonicbox}હવા પાણી માટી અવાજ (AWSS)\end{mnemonicbox}
\end{solutionbox}

\questionmarks{3}{a}{3}
\textbf{E-વેસ્ટ શું છે? પર્યાવરણ અને માનવીઓ ઉપર E-વેસ્ટની અસરો જણાવો.}

\begin{solutionbox}
    \textbf{જવાબ:}
    \textbf{E-waste} (Electronic waste) એ હાનિકારક સામગ્રી ધરાવતા ફેંકાયેલા વિદ્યુત અને ઇલેક્ટ્રોનિક ઉપકરણોનો સમાવેશ થાય છે.

    \textbf{પર્યાવરણીય અસરો:}
    \begin{itemize}
        \item \textbf{માટીનું દૂષણ}: ભારે ધાતુઓનું લીકેજ
        \item \textbf{પાણીનું પ્રદૂષણ}: ઝેરી રસાયણોનો વહેણ
        \item \textbf{હવાનું પ્રદૂષણ}: બર્નિંગથી ઝેરી ધુમાડો
    \end{itemize}

    \textbf{માનવીય અસરો:}
    \begin{itemize}
        \item \textbf{આરોગ્ય જોખમો}: લીડ, મર્ક્યુરી વિષાક્તતા
        \item \textbf{શ્વસનની સમસ્યાઓ}: ઝેરી વાયુનો શ્વાસ
        \item \textbf{ચામડીના રોગો}: રસાયણો સાથે સીધો સંપર્ક
    \end{itemize}

    \begin{answertable}{E-waste જોખમો}
    \begin{tabulary}{\linewidth}{L L L}
        \toprule
        \textbf{ઘટક} & \textbf{જોખમ} & \textbf{અસર} \\
        \midrule
        \textbf{લીડ} & ન્યુરોટોક્સિન & મગજનું નુકસાન \\
        \textbf{મર્ક્યુરી} & ઝેરી ધાતુ & કિડનીનું નુકસાન \\
        \textbf{કેડમિયમ} & કેન્સરકારક & કેન્સરનું જોખમ \\
        \bottomrule
    \end{tabulary}
    \end{answertable}

    \begin{mnemonicbox}ઇલેક્ટ્રોનિક સાધનો દરેકને જોખમમાં મૂકે છે (E4)\end{mnemonicbox}
\end{solutionbox}

\questionmarks{3}{a}{3}
\textbf{પ્લાસ્ટિક કચરો શું છે? પ્લાસ્ટિકના કચરાથી થતી અસરો જણાવો.}

\begin{solutionbox}
    \textbf{જવાબ:}
    \textbf{Plastic waste} એ બાયોડિગ્રેડેબલ ન હોવાના કારણે પર્યાવરણમાં ટકી રહેતા ફેંકાયેલા પ્લાસ્ટિક સામગ્રીનો સમાવેશ થાય છે.

    \textbf{અસરો:}
    \begin{itemize}
        \item \textbf{દરિયાઈ પ્રદૂષણ}: સમુદ્રમાં પ્લાસ્ટિકનો સંચય
        \item \textbf{વન્યજીવોની અસર}: પ્રાણીઓને ફસાવવું, ગળવું
        \item \textbf{માટીનું ક્ષીણીકરણ}: ફળદ્રુપતા અને પાણી ઘૂસણમાં ઘટાડો
        \item \textbf{માનવ આરોગ્ય}: ખોરાકના ચેઇનમાં માઇક્રોપ્લાસ્ટિક
    \end{itemize}

    \textbf{વર્ગીકરણ:}
    \begin{itemize}
        \item \textbf{એક વારનો ઉપયોગ}: બેગ, બોટલ, સ્ટ્રો
        \item \textbf{પેકેજિંગ વેસ્ટ}: ખોરાકના કન્ટેનર, આવરણ
        \item \textbf{ઔદ્યોગિક પ્લાસ્ટિક}: ઉત્પાદનનો કચરો
    \end{itemize}

    \begin{mnemonicbox}પ્લાસ્ટિક ટકે છે, સમસ્યાઓ ટકે છે (PPPP)\end{mnemonicbox}
\end{solutionbox}

\questionmarks{3}{b}{3}
\textbf{ઘન કચરાના મુખ્ય સ્ત્રોતો આપો.}

\begin{solutionbox}
    \textbf{જવાબ:}
    \textbf{Solid waste} વિવિધ માનવીય પ્રવૃત્તિઓ અને કુદરતી પ્રક્રિયાઓમાંથી ઉત્પન્ન થાય છે.

    \textbf{સ્ત્રોતો:}
    \begin{itemize}
        \item \textbf{આવાસીય}: ઘરેલું કચરો, ખોરાકનો કચરો
        \item \textbf{વ્યાવસાયિક}: ઓફિસ વેસ્ટ, પેકેજિંગ સામગ્રી
        \item \textbf{ઔદ્યોગિક}: ઉત્પાદન કચરો, રસાયણો
        \item \textbf{કૃષિ}: પાકના અવશેષો, પ્રાણીઓનો કચરો
        \item \textbf{મ્યુનિસિપલ}: રસ્તાની સફાઈ, પાર્કની જાળવણી
    \end{itemize}

    \begin{answertable}{ઘન કચરાના સ્ત્રોતો}
    \begin{tabulary}{\linewidth}{L L L}
        \toprule
        \textbf{સ્ત્રોત} & \textbf{કચરાનો પ્રકાર} & \textbf{વ્યવસ્થાપન} \\
        \midrule
        \textbf{ઘરેલું} & કાર્બનિક, પ્લાસ્ટિક & સંગ્રહ \\
        \textbf{ઔદ્યોગિક} & જોખમી, બિન-જોખમી & સારવાર \\
        \textbf{કૃષિ} & બાયોડિગ્રેડેબલ & કમ્પોસ્ટિંગ \\
        \bottomrule
    \end{tabulary}
    \end{answertable}

    \begin{mnemonicbox}આવાસીય વ્યાવસાયિક ઔદ્યોગિક કૃષિ મ્યુનિસિપલ (RCIAM)\end{mnemonicbox}
\end{solutionbox}

\questionmarks{3}{b}{3}
\textbf{ઘન કચરાના નિકાલની વિવિધ પદ્ધતિઓ જણાવો અને કોઈપણ એકને સમજાવો.}

\begin{solutionbox}
    \textbf{જવાબ:}
    \textbf{નિકાલની પદ્ધતિઓ:}
    \begin{itemize}
        \item \textbf{લેન્ડફિલિંગ}: નિયંત્રિત કચરાનું દફનાવવું
        \item \textbf{ઇન્સિનરેશન}: ઊર્જા પુનઃપ્રાપ્તિ સાથે કચરો બાળવો
        \item \textbf{કમ્પોસ્ટિંગ}: કાર્બનિક કચરાનું વિઘટન
        \item \textbf{રીસાયક્લિંગ}: સામગ્રી પુનઃપ્રાપ્તિ અને પુનઃઉપયોગ
    \end{itemize}

    \textbf{Sanitary Landfill:}
    \begin{center}
    \begin{tikzpicture}[scale=0.8]
        \draw[thick] (0,0) -- (8,0) -- (8,4) -- (0,4) -- cycle;
        \foreach \y in {0.5, 1.5, 2.5} {
            \draw[fill=gray!20] (0.5, \y) rectangle (7.5, \y+0.8);
            \node at (4, \y+0.4) {કચરાનું સ્તર};
        }
        \draw[dashed] (0.5, 0.2) -- (7.5, 0.2);
        \node at (4, 0.1) {લાઈનર અને ડ્રેનેજ};
        \node at (4, 3.8) {દૈનિક માટીનું આવરણ};
    \end{tikzpicture}
    \end{center}

    \begin{itemize}
        \item \textbf{ડિઝાઇન}: લાઈનર સાથે એન્જિનિયર્ડ સિસ્ટમ
        \item \textbf{ઓપરેશન}: દૈનિક આવરણ, સંકુચન
        \item \textbf{પર્યાવરણ સંરક્ષણ}: લીચેટ અને ગેસ નિયંત્રણ
    \end{itemize}

    \begin{mnemonicbox}લેન્ડ ઇન્સિનરેટ કમ્પોસ્ટ રીસાયકલ (LICR)\end{mnemonicbox}
\end{solutionbox}

\questionmarks{3}{c}{4}
\textbf{પ્રવાહી ફ્લેટ પ્લેટ કલેક્ટરનું કાર્ય સ્વચ્છ આકૃતિ સાથે સમજાવો.}

\begin{solutionbox}
    \textbf{જવાબ:}
    \textbf{Liquid Flat Plate Collector} પાણી ગરમ કરવા માટે સૌર કિરણોત્સર્ગને ઉષ્મીય ઊર્જામાં રૂપાંતરિત કરે છે.

    \begin{center}
    \begin{tikzpicture}[scale=0.8]
        % Frame
        \draw[thick] (0,0) rectangle (6,4);
        % Tubes
        \foreach \x in {1,2,3,4,5} {
            \draw[thick, double] (\x,0.5) -- (\x,3.5);
        }
        % Headers
        \draw[thick, double] (0.5,0.5) -- (5.5,0.5);
        \draw[thick, double] (0.5,3.5) -- (5.5,3.5);
        
        % Labels
        \node at (3,4.2) {કાચનું આવરણ};
        \node at (3,-0.5) {ઇન્સ્યુલેશન બેકિંગ};
        \draw[->] (-1,0.5) -- (0.5,0.5) node[midway, above] {ઠંડું ઇન};
        \draw[->] (5.5,3.5) -- (7,3.5) node[midway, above] {ગરમ આઉટ};
    \end{tikzpicture}
    \end{center}

    \textbf{કાર્યપ્રણાલી:}
    \begin{itemize}
        \item \textbf{સૌર શોષણ}: કાળી શોષક પ્લેટ સૌર ઊર્જા કેપ્ચર કરે છે
        \item \textbf{ગરમી સ્થાનાંતરણ}: શોષાયેલી ગરમી વહેતા પ્રવાહીમાં સ્થાનાંતરિત થાય છે
        \item \textbf{પરિભ્રમણ}: ગરમ પ્રવાહી ઉપર આવે છે, ઠંડો પ્રવાહી અંદર જાય છે
        \item \textbf{ઇન્સ્યુલેશન}: ગરમીના નુકસાનને ન્યૂનતમ કરે છે
    \end{itemize}

    \textbf{ઘટકો:}
    \begin{itemize}
        \item \textbf{પારદર્શક આવરણ}: કન્વેક્શન લોસ ઘટાડે છે
        \item \textbf{શોષક પ્લેટ}: મહત્તમ સૌર શોષણ
        \item \textbf{હીટ ટ્રાન્સફર ફ્લુઇડ}: પાણી અથવા એન્ટિફ્રીઝ સોલ્યુશન
    \end{itemize}

    \begin{mnemonicbox}સૌર શોષણ ગરમી સ્થાનાંતરણ બનાવે છે (SACHT)\end{mnemonicbox}
\end{solutionbox}

\questionmarks{3}{c}{4}
\textbf{સોલાર પોન્ડ પર ટૂંક નોંધ લખો}

\begin{solutionbox}
    \textbf{જવાબ:}
    \textbf{Solar pond} એ મીઠાપાણીનું પૂલ છે જે સૌર કલેક્ટર અને ઉષ્મીય સ્ટોરેજ સિસ્ટમ બંને તરીકે કામ કરે છે.

    \textbf{રચના:}
    \begin{itemize}
        \item \textbf{ઉપરનો ઝોન}: ઓછી મીઠાની સાંદ્રતા
        \item \textbf{મધ્યમ ઝોન}: વધતી મીઠાની ગ્રેડિએન્ટ
        \item \textbf{નીચેનો ઝોન}: વધુ મીઠાની સાંદ્રતા
    \end{itemize}

    \textbf{કાર્યપ્રણાલી:}
    \begin{itemize}
        \item \textbf{ઘનતા ગ્રેડિએન્ટ}: કન્વેક્શન મિશ્રણ અટકાવે છે
        \item \textbf{ગરમી સ્ટોરેજ}: નીચેનો સ્તર ઉષ્મીય ઊર્જા સંગ્રહ કરે છે
        \item \textbf{તાપમાન}: તળિયે 70-85$^{\circ}$C સુધી પહોંચી શકે છે
    \end{itemize}

    \textbf{ઉપયોગો:}
    \begin{itemize}
        \item \textbf{વીજ ઉત્પાદન}: વરાળ ઉત્પાદન
        \item \textbf{ઔદ્યોગિક ગરમી}: પ્રોસેસ હીટ સપ્લાય
        \item \textbf{ડિસેલિનેશન}: પાણીની શુદ્ધિકરણ
    \end{itemize}

    \begin{mnemonicbox}મીઠું સૌર ઉષ્મીય સંગ્રહ કરે છે (SSST)\end{mnemonicbox}
\end{solutionbox}

\questionmarks{3}{d}{4}
\textbf{સેવોનિયસ પવનચક્કી સ્વચ્છ આકૃતિ સાથે સમજાવો.}

\begin{solutionbox}
    \textbf{Savonius wind turbine} એ S-આકારના રોટર બ્લેડ સાથેનું વર્ટિકલ એક્સિસ વિન્ડ ટર્બાઇન છે.

    \begin{center}
    \begin{tikzpicture}[scale=0.8]
        % Axis
        \draw[thick] (4,0) -- (4,5);
        \node at (4,-0.5) {જનરેટર};
        \draw[fill=black] (3.8,-0.2) rectangle (4.2,0);
        
        % S-Rotor (Top view projection)
        \draw[thick] (2,4) arc (180:360:1);
        \draw[thick] (4,4) arc (0:180:1);
        \node at (4, 4.5) {S-આકારનો રોટર};
        
        % Wind
        \draw[thick, ->] (0,2.5) -- (2,2.5) node[midway, above] {પવનની દિશા};
        
        % Rotation
        \draw[->, dashed] (3, 5.2) arc (150:30:1);
        \node at (4, 5.5) {પરિભ્રમણ};
    \end{tikzpicture}
    \end{center}

    \textbf{કાર્યપ્રણાલી:}
    \begin{itemize}
        \item \textbf{ડ્રેગ સિદ્ધાંત}: પવન બ્લેડ પર વિભેદક ડ્રેગ બનાવે છે
        \item \textbf{પરિભ્રમણ}: S-આકાર સતત પરિભ્રમણ બનાવે છે
        \item \textbf{સેલ્ફ-સ્ટાર્ટિંગ}: ઓછી પવનની ઝડપે શરૂ થાય છે
        \item \textbf{વર્ટિકલ એક્સિસ}: પવનની દિશાથી સ્વતંત્ર
    \end{itemize}

    \textbf{ફાયદા:}
    \begin{itemize}
        \item \textbf{સરળ ડિઝાઇન}: ઓછી જાળવણીની જરૂરિયાતો
        \item \textbf{ઓછો અવાજ}: શાંત ઓપરેશન
        \item \textbf{બધી પવન દિશાઓ}: સર્વદિશીય ક્ષમતા
    \end{itemize}

    \textbf{ગેરફાયદા:}
    \begin{itemize}
        \item \textbf{ઓછી કાર્યક્ષમતા}: HAWT ની સરખામણીમાં 20-30\%
        \item \textbf{જગ્યાની જરૂરિયાત}: મોટા વિસ્તારની જરૂર
    \end{itemize}

    \begin{mnemonicbox}S-આકાર ધીમે ધીમે શરૂ થાય છે (SSS)\end{mnemonicbox}
\end{solutionbox}

\questionmarks{3}{d}{4}
\textbf{આડી અરીવાળી તથા ઊભી અરીવાળી પવનચક્કીની તુલના કરો.}

\begin{solutionbox}
    વિન્ડ ટર્બાઇનનું રોટર એક્સિસ ઓરિએન્ટેશનના આધારે વર્ગીકરણ થાય છે.

    \begin{answertable}{HAWT અને VAWT તુલના}
    \begin{tabulary}{\linewidth}{L L L}
        \toprule
        \textbf{પરિમાણ} & \textbf{આડી અરી (HAWT)} & \textbf{ઊભી અરી (VAWT)} \\
        \midrule
        \textbf{કાર્યક્ષમતા} & 35-45\% & 20-30\% \\
        \textbf{પવનની દિશા} & પવન સામે મુંહ & કોઈપણ દિશા \\
        \textbf{સ્થાપના} & ટાવર જરૂરી & જમીન સ્તરે શક્ય \\
        \textbf{જાળવણી} & મુશ્કેલ પહોંચ & સરળ પહોંચ \\
        \textbf{અવાજ} & વધુ & ઓછો \\
        \textbf{કિંમત} & વધુ & ઓછી \\
        \bottomrule
    \end{tabulary}
    \end{answertable}

    \textbf{HAWT ફીચર્સ:}
    \begin{itemize}
        \item \textbf{અપવિન્ડ ડિઝાઇન}: રોટર પવનનો સામનો કરે છે
        \item \textbf{પિચ કન્ટ્રોલ}: બ્લેડ એંગલ એડજસ્ટમેન્ટ
        \item \textbf{યૉ સિસ્ટમ}: પવનની દિશા ટ્રેકિંગ
    \end{itemize}

    \textbf{VAWT ફીચર્સ:}
    \begin{itemize}
        \item \textbf{સર્વદિશીય}: પવન ટ્રેકિંગની જરૂર નથી
        \item \textbf{જમીન સ્થાપના}: સરળ જાળવણી
        \item \textbf{ઓછી પવનની ઝડપ}: વધુ સારી કામગીરી
    \end{itemize}

    \begin{mnemonicbox}આડી ઉચ્ચ, ઊભી વર્સેટાઇલ (HHVV)\end{mnemonicbox}
\end{solutionbox}

\questionmarks{4}{a}{3}
\textbf{આબોહવા (જલવાયુ) પરિવર્તનની અસરો જણાવો.}

\begin{solutionbox}
    \textbf{જવાબ:}
    \textbf{Climate change} વૈશ્વિક સ્તરે વ્યાપક પર્યાવરણીય અને સામાજિક-આર્થિક અસરો લાવે છે.

    \textbf{પર્યાવરણીય અસરો:}
    \begin{itemize}
        \item \textbf{તાપમાનમાં વૃદ્ધિ}: વૈશ્વિક સરેરાશ વધારો
        \item \textbf{દરિયાઈ સ્તરમાં વૃદ્ધિ}: ઉષ્મીય વિસ્તરણ અને બરફ પીગળવાથી
        \item \textbf{હવામાનની ચરમસીમાઓ}: તીવ્ર તોફાન, દુષ્કાળ, પૂર
        \item \textbf{ઇકોસિસ્ટમ ફેરફાર}: જાતિઓનું સ્થળાંતર અને લુપ્ત થવું
    \end{itemize}

    \textbf{સામાજિક-આર્થિક અસરો:}
    \begin{itemize}
        \item \textbf{કૃષિ અસર}: પાકના ઉત્પાદનમાં બદલાવ
        \item \textbf{પાણીના સંસાધનો}: ઉપલબ્ધતા અને ગુણવત્તાની સમસ્યાઓ
        \item \textbf{માનવ આરોગ્ય}: ગરમીનો તાણ, રોગનો ફેલાવો
        \item \textbf{આર્થિક નુકસાન}: ઇન્ફ્રાસ્ટ્રક્ચરનું નુકસાન
    \end{itemize}

    \begin{answertable}{આબોહવા પરિવર્તન અસરો}
    \begin{tabulary}{\linewidth}{L L L}
        \toprule
        \textbf{અસરનો વર્ગ} & \textbf{ઉદાહરણો} & \textbf{ગંભીરતા} \\
        \midrule
        \textbf{પર્યાવરણીય} & ગ્લેશિયર પીગળવા & ઉચ્ચ \\
        \textbf{કૃષિ} & પાકની નિષ્ફળતા & મધ્યમ \\
        \textbf{આરોગ્ય} & ગરમીના લહેરા & ઉચ્ચ \\
        \bottomrule
    \end{tabulary}
    \end{answertable}

    \begin{mnemonicbox}તાપમાન સમુદ્ર હવામાન ઇકોસિસ્ટમ (TSWE)\end{mnemonicbox}
\end{solutionbox}

\questionmarks{4}{a}{3}
\textbf{ગ્રીન હાઉસ વાયુઓ પર ટૂંક નોંધ લખો.}

\begin{solutionbox}
    \textbf{જવાબ:}
    \textbf{Greenhouse gases} પૃથ્વીના વાતાવરણમાં ગરમી અટકાવે છે, જે greenhouse effect દ્વારા વૈશ્વિક ઉષ્ણતા લાવે છે.

    \textbf{મુખ્ય Greenhouse Gases:}
    \begin{itemize}
        \item \textbf{કાર્બન ડાયોક્સાઇડ (CO\textsubscript{2})}: ઉત્સર્જનના 76\%
        \item \textbf{મિથેન (CH\textsubscript{4})}: ઉત્સર્જનના 16\%
        \item \textbf{નાઇટ્રસ ઓક્સાઇડ (N\textsubscript{2}O)}: ઉત્સર્જનના 6\%
        \item \textbf{ફ્લોરિનેટેડ ગેસીસ}: ઉત્સર્જનના 2\%
    \end{itemize}

    \textbf{સ્ત્રોતો:}
    \begin{itemize}
        \item \textbf{CO\textsubscript{2}}: અશ્મિભૂત ઇંધનનું બર્નિંગ, વનનાશ
        \item \textbf{CH\textsubscript{4}}: કૃષિ, લેન્ડફિલ, પશુધન
        \item \textbf{N\textsubscript{2}O}: ખાતર, અશ્મિભૂત ઇંધન દહન
    \end{itemize}

    \textbf{વૈશ્વિક ઉષ્ણતા ક્ષમતા:}
    \begin{itemize}
        \item \textbf{CO\textsubscript{2}}: સંદર્ભ (GWP = 1)
        \item \textbf{CH\textsubscript{4}}: CO\textsubscript{2} કરતાં 25 ગણી
        \item \textbf{N\textsubscript{2}O}: CO\textsubscript{2} કરતાં 298 ગણી
    \end{itemize}

    \begin{mnemonicbox}કાર્બન મિથેન નાઇટ્રસ ફ્લોરિન (CMNF)\end{mnemonicbox}
\end{solutionbox}

\questionmarks{4}{b}{4}
\textbf{આબોહવા (જલવાયુ) પરિવર્તન સંચાલન સમજાવો.}

\begin{solutionbox}
    \textbf{જવાબ:}
    \textbf{Climate change management} માં greenhouse gas ઉત્સર્જન ઘટાડવા અને આબોહવાની અસરોને અનુકૂળ થવાની વ્યૂહરચનાઓનો સમાવેશ થાય છે.

    \textbf{શમન વ્યૂહરચનાઓ:}
    \begin{itemize}
        \item \textbf{નવીકરણીય ઊર્જા}: સૌર, પવન, હાઇડ્રોઇલેક્ટ્રિક પાવર
        \item \textbf{ઊર્જા કાર્યક્ષમતા}: સુધારેલી બિલ્ડિંગ ડિઝાઇન, LED લાઇટિંગ
        \item \textbf{કાર્બન સિક્વેસ્ટ્રેશન}: વન સંરક્ષણ, વૃક્ષ વાવેતર
        \item \textbf{ટકાઉ પરિવહન}: ઇલેક્ટ્રિક વાહનો, જાહેર પરિવહન
    \end{itemize}

    \textbf{અનુકૂલન વ્યૂહરચનાઓ:}
    \begin{itemize}
        \item \textbf{ઇન્ફ્રાસ્ટ્રક્ચર સ્થિતિસ્થાપકતા}: પૂર સંરક્ષણ, દુષ્કાળ-પ્રતિરોધી પાકો
        \item \textbf{પાણી વ્યવસ્થાપન}: વરસાદી પાણીનો સંગ્રહ, કુશળ સિંચાઈ
        \item \textbf{દરિયાકાંઠા સંરક્ષણ}: દરિયાઈ દિવાલો, મેન્ગ્રોવ પુનઃસ્થાપન
        \item \textbf{કટોકટીની તૈયારી}: પ્રારંભિક ચેતવણી પ્રણાલીઓ
    \end{itemize}

    \textbf{નીતિગત પગલાં:}
    \begin{itemize}
        \item \textbf{કાર્બન કિંમત}: ઉત્સર્જન પર કર
        \item \textbf{નવીકરણીય ઊર્જા લક્ષ્યો}: સ્વચ્છ ઊર્જા લક્ષ્યો
        \item \textbf{બિલ્ડિંગ કોડ}: ઊર્જા કાર્યક્ષમતા માનદંડો
    \end{itemize}

    \begin{mnemonicbox}શમન અનુકૂલન નીતિ (MAP)\end{mnemonicbox}
\end{solutionbox}

\questionmarks{4}{b}{4}
\textbf{ઓઝોન સ્તરની ક્ષતિની અસરો જણાવો.}

\begin{solutionbox}
    \textbf{જવાબ:}
    \textbf{Ozone layer depletion} stratospheric ozone ઘટાડે છે, જે હાનિકારક UV કિરણોત્સર્ગને પૃથ્વી પર પહોંચવા દે છે.

    \textbf{માનવો પર અસરો:}
    \begin{itemize}
        \item \textbf{ચામડીનું કેન્સર}: વધેલા UV-B કિરણોત્સર્ગના સંપર્કથી
        \item \textbf{આંખનું મોતિયો}: આંખના લેન્સને UV નુકસાન
        \item \textbf{રોગપ્રતિકારક શક્તિ ઘટવી}: નબળી રોગપ્રતિકારક પ્રણાલી
        \item \textbf{અકાળે વૃદ્ધાવસ્થા}: ચામડીના નુકસાનને વેગ આપવો
    \end{itemize}

    \textbf{પર્યાવરણ પર અસરો:}
    \begin{itemize}
        \item \textbf{પાકનું નુકસાન}: કૃષિ ઉત્પાદકતામાં ઘટાડો
        \item \textbf{દરિયાઈ ઇકોસિસ્ટમ}: ફાયટોપ્લાંકટોનમાં ઘટાડો
        \item \textbf{સામગ્રીનું ક્ષીણીકરણ}: પ્લાસ્ટિક અને રબરનું નુકસાન
        \item \textbf{આબોહવા પરિવર્તન}: greenhouse gas તરીકે ઓઝોન
    \end{itemize}

    \begin{answertable}{UV કિરણોત્સર્ગ અસરો}
    \begin{tabulary}{\linewidth}{L L L}
        \toprule
        \textbf{UV પ્રકાર} & \textbf{તરંગલંબાઈ} & \textbf{અસર} \\
        \midrule
        \textbf{UV-A} & 320-400 nm & ચામડીનું વૃદ્ધાવસ્થા \\
        \textbf{UV-B} & 280-320 nm & સનબર્ન, કેન્સર \\
        \textbf{UV-C} & 200-280 nm & ઓઝોન દ્વારા અવરોધ \\
        \bottomrule
    \end{tabulary}
    \end{answertable}

    \begin{mnemonicbox}ચામડી આંખો રોગપ્રતિકારક આબોહવા (SEIC)\end{mnemonicbox}
\end{solutionbox}

\questionmarks{4}{c}{7}
\textbf{બાયોગેસ પ્લાન્ટને આકૃતિ સાથે સમજાવો.}

\begin{solutionbox}
    \textbf{જવાબ:}
    \textbf{Biogas plant} કાર્બનિક કચરાના anaerobic digestion દ્વારા મિથેન-સમૃદ્ધ ગેસ ઉત્પન્ન કરે છે.

    \begin{center}
    \begin{tikzpicture}[scale=0.8]
        % Tank
        \draw[thick] (2,0) rectangle (6,4);
        \node at (4,2) {ડાયજેસ્ટર (સ્લરી)};
        
        % Dome
        \draw[thick] (2,4) arc (180:0:2);
        \node at (4,4.5) {ગેસ હોલ્ડર};
        \draw[thick] (4,6) -- (4,6.5) node[above] {ગેસ આઉટલેટ};
        
        % Inlet
        \draw[thick] (0,3) -- (1,1) -- (2,1);
        \draw[thick] (0,3.5) -- (1,1.5) -- (2,1.5);
        \node at (0,3.8) {ઇનલેટ (કચરો)};
        
        % Outlet
        \draw[thick] (6,1) -- (7,1) -- (8,2);
        \draw[thick] (6,1.5) -- (7,1.5) -- (8,2.5);
        \node at (8,2.8) {આઉટલેટ (ખાતર)};
    \end{tikzpicture}
    \end{center}

    \textbf{ઘટકો:}
    \begin{itemize}
        \item \textbf{ડાયજેસ્ટર ટેન્ક}: Anaerobic fermentation ચેમ્બર
        \item \textbf{ગેસ ડોમ}: બાયોગેસ સંગ્રહ અને સ્ટોરેજ
        \item \textbf{ઇનલેટ પાઇપ}: કચરા સામગ્રીનું ફીડિંગ
        \item \textbf{આઉટલેટ પાઇપ}: પચેલા સ્લરીને દૂર કરવું
    \end{itemize}

    \textbf{પ્રક્રિયા:}
    \begin{itemize}
        \item \textbf{હાઇડ્રોલિસિસ}: જટિલ કાર્બનિક પદાર્થો તૂટે છે
        \item \textbf{એસિડોજેનેસિસ}: એસિડ બનાવતા બેક્ટેરિયાની ક્રિયા
        \item \textbf{મિથેનોજેનેસિસ}: મિથેન ઉત્પન્ન કરતા બેક્ટેરિયા
        \item \textbf{ગેસ ઉત્પાદન}: 50-70\% મિથેન, 30-40\% CO\textsubscript{2}
    \end{itemize}

    \textbf{ઓપરેટિંગ પરિસ્થિતિઓ:}
    \begin{itemize}
        \item \textbf{તાપમાન}: 30-40{\circ} શ્રેષ્ઠ
        \item \textbf{pH}: 6.8-7.2 રેન્જ
        \item \textbf{રીટેન્શન ટાઇમ}: 15-30 દિવસ
        \item \textbf{C:N રેશિયો}: 20-30:1 શ્રેષ્ઠ
    \end{itemize}

    \textbf{ઉપયોગો:}
    \begin{itemize}
        \item \textbf{રસોઈ ઇંધન}: ઘરેલું ઊર્જાની જરૂરિયાતો
        \item \textbf{પ્રકાશ}: ગેસ લેમ્પ રોશની
        \item \textbf{વીજળી}: જનરેટર પાવર
        \item \textbf{ખાતર}: પોષક તત્વોથી સમૃદ્ધ સ્લરી
    \end{itemize}

    \textbf{ફાયદા:}
    \begin{itemize}
        \item \textbf{નવીકરણીય ઊર્જા}: ટકાઉ ઇંધન સ્ત્રોત
        \item \textbf{કચરા વ્યવસ્થાપન}: કાર્બનિક કચરાનો ઉપયોગ
        \item \textbf{પર્યાવરણીય ફાયદા}: મિથેન ઉત્સર્જનમાં ઘટાડો
        \item \textbf{આર્થિક ફાયદા}: ઇંધન પર ખર્ચ બચત
    \end{itemize}

    \begin{mnemonicbox}બાયોગેસ ફાયદા: નવીકરણીય કચરા પર્યાવરણ અર્થતંત્ર (BRWEE)\end{mnemonicbox}
\end{solutionbox}

\questionmarks{5}{a}{4}
\textbf{'ગ્લોબલ વોર્મિંગ' પર ટૂંક નોંધ લખો.}

\begin{solutionbox}
    \textbf{જવાબ:}
    \textbf{Global warming} માનવીય પ્રવૃત્તિઓને કારણે પૃથ્વીના સરેરાશ સપાટીના તાપમાનમાં લાંબા ગાળાના વધારાનો સંદર્ભ આપે છે.

    \textbf{કારણો:}
    \begin{itemize}
        \item \textbf{Greenhouse gases}: CO\textsubscript{2}, CH\textsubscript{4}, N\textsubscript{2}O ઉત્સર્જન
        \item \textbf{વનનાશ}: કાર્બન શોષણમાં ઘટાડો
        \item \textbf{ઔદ્યોગિક પ્રવૃત્તિઓ}: અશ્મિભૂત ઇંધન દહન
        \item \textbf{પરિવહન}: વાહન ઉત્સર્જન
    \end{itemize}

    \textbf{અસરો:}
    \begin{itemize}
        \item \textbf{તાપમાન વૃદ્ધિ}: પૂર્વ-ઔદ્યોગિક સમયથી 1.1{\circ}
        \item \textbf{બરફ પીગળવો}: આર્કટિક દરિયાઈ બરફ, ગ્લેશિયર સંકુચિત થવા
        \item \textbf{દરિયાઈ સ્તર વૃદ્ધિ}: દરિયાકાંઠાના પૂરનું જોખમ
        \item \textbf{હવામાન ફેરફાર}: ચરમ ઘટનાઓની આવૃત્તિ
    \end{itemize}

    \textbf{પુરાવા:}
    \begin{itemize}
        \item \textbf{તાપમાનના રેકોર્ડ}: તાજેતરના દાયકાઓમાં સૌથી ગરમ વર્ષો
        \item \textbf{બરફ કોર ડેટા}: ઐતિહાસિક CO\textsubscript{2} સ્તરો
        \item \textbf{સેટેલાઇટ માપ}: વૈશ્વિક તાપમાન મોનિટરિંગ
    \end{itemize}

    \textbf{ઉકેલો:}
    \begin{itemize}
        \item \textbf{નવીકરણીય ઊર્જા}: સ્વચ્છ પાવર સ્ત્રોતો
        \item \textbf{ઊર્જા કાર્યક્ષમતા}: ઘટતો વપરાશ
        \item \textbf{કાર્બન કેપ્ચર}: ટેકનોલોજી ડેવલપમેન્ટ
        \item \textbf{આંતરરાષ્ટ્રીય સહયોગ}: પેરિસ એગ્રીમેન્ટ
    \end{itemize}

    \begin{mnemonicbox}Greenhouse ગેસીસ વૈશ્વિક ફેરફાર બનાવે છે (GGGC)\end{mnemonicbox}
\end{solutionbox}

\questionmarks{5}{b}{4}
\textbf{'5R નો કન્સેપ્ટ' સમજાવો.}

\begin{solutionbox}
    \textbf{જવાબ:}
    \textbf{5R concept} ટકાઉ સંસાધન ઉપયોગ માટે કચરા વ્યવસ્થાપન પદાનુક્રમ છે.

    \begin{center}
    \begin{tikzpicture}[node distance=1.5cm, auto]
        \node (root) [gtu block] {5R પદાનુક્રમ};
        \node (r1) [gtu block, below=of root] {Refuse};
        \node (r2) [gtu block, below=of r1] {Reduce};
        \node (r3) [gtu block, below=of r2] {Reuse};
        \node (r4) [gtu block, below=of r3] {Repurpose};
        \node (r5) [gtu block, below=of r4] {Recycle};

        \draw [gtu arrow] (root) -- (r1);
        \draw [gtu arrow] (r1) -- (r2);
        \draw [gtu arrow] (r2) -- (r3);
        \draw [gtu arrow] (r3) -- (r4);
        \draw [gtu arrow] (r4) -- (r5);
    \end{tikzpicture}
    \end{center}

    \textbf{5 R's:}

    \textbf{1. Refuse:}
    \begin{itemize}
        \item \textbf{બિનજરૂરી વસ્તુઓ ટાળો}: એકવાર વપરાશની વસ્તુઓને ના કહો
        \item \textbf{ઉદાહરણો}: પ્લાસ્ટિક બેગ, સ્ટ્રો, વધુ પડતું પેકેજિંગ
    \end{itemize}

    \textbf{2. Reduce:}
    \begin{itemize}
        \item \textbf{વપરાશ ઓછો કરો}: ઓછા સંસાધનોનો ઉપયોગ
        \item \textbf{ઉદાહરણો}: ઊર્જા સંરક્ષણ, પાણી બચાવવું
    \end{itemize}

    \textbf{3. Reuse:}
    \begin{itemize}
        \item \textbf{અનેક વાર ઉપયોગ}: ઉત્પાદનનું જીવન વધારવું
        \item \textbf{ઉદાહરણો}: કાચના જાર કન્ટેનર તરીકે, કાગળ બંને બાજુ
    \end{itemize}

    \textbf{4. Repurpose:}
    \begin{itemize}
        \item \textbf{સર્જનાત્મક પુનઃઉપયોગ}: જૂની વસ્તુઓ માટે નવું કાર્ય
        \item \textbf{ઉદાહરણો}: ટાયર પ્લાન્ટર, બોટલ પક્ષી ફીડર
    \end{itemize}

    \textbf{5. Recycle:}
    \begin{itemize}
        \item \textbf{સામગ્રી પુનઃપ્રાપ્તિ}: નવા ઉત્પાદનોમાં પ્રક્રિયા
        \item \textbf{ઉદાહરણો}: કાગળ, પ્લાસ્ટિક, ધાતુ રીસાયક્લિંગ
    \end{itemize}

    \textbf{ફાયદા:}
    \begin{itemize}
        \item \textbf{કચરા ઘટાડો}: લેન્ડફિલ પર ઓછો બોજ
        \item \textbf{સંસાધન સંરક્ષણ}: કુદરતી સંસાધન સાચવણી
        \item \textbf{ખર્ચ બચત}: આર્થિક ફાયદા
        \item \textbf{પર્યાવરણ સંરક્ષણ}: પ્રદૂષણ ઘટાડો
    \end{itemize}

    \begin{mnemonicbox}રિફ્યુઝ રિડ્યુસ રીયુઝ રિપર્પઝ રીસાયકલ (R5)\end{mnemonicbox}
\end{solutionbox}

\questionmarks{5}{c}{3}
\textbf{ગ્રીન બિલ્ડિંગના ફાયદા સમજાવો.}

\begin{solutionbox}
    \textbf{જવાબ:}
    \textbf{Green building} પર્યાવરણીય અને માનવીય ફાયદા માટે ટકાઉ ડિઝાઇન અને બાંધકામ પ્રથાઓનો સમાવેશ કરે છે.

    \textbf{પર્યાવરણીય ફાયદા:}
    \begin{itemize}
        \item \textbf{ઊર્જા કાર્યક્ષમતા}: ઘટતો વીજ વપરાશ
        \item \textbf{પાણી સંરક્ષણ}: કુશળ પાણી પ્રણાલીઓ
        \item \textbf{કચરા ઘટાડો}: બાંધકામ અને ઓપરેશનલ કચરા ઓછા કરવા
    \end{itemize}

    \textbf{આર્થિક ફાયદા:}
    \begin{itemize}
        \item \textbf{ઓપરેટિંગ કોસ્ટ બચત}: ઓછા યુટિલિટી બિલ
        \item \textbf{મિલકતના ભાવમાં વધારો}: બજાર પ્રીમિયમ
        \item \textbf{ટેક્સ પ્રોત્સાહન}: સરકારી રિબેટ
    \end{itemize}

    \textbf{આરોગ્ય ફાયદા:}
    \begin{itemize}
        \item \textbf{ઇન્ડોર એર ક્વોલિટી}: વધુ સારી વેન્ટિલેશન સિસ્ટમ
        \item \textbf{કુદરતી લાઇટિંગ}: વધુ સારી રહેવાસીઓની સગવડતા
        \item \textbf{ઝેરી સામગ્રી ઘટાડો}: વધુ સ્વસ્થ વાતાવરણ
    \end{itemize}

    \begin{answertable}{ગ્રીન બિલ્ડિંગ ફાયદા}
    \begin{tabulary}{\linewidth}{L L L}
        \toprule
        \textbf{ફાયદાનો પ્રકાર} & \textbf{ઉદાહરણો} & \textbf{અસર} \\
        \midrule
        \textbf{પર્યાવરણીય} & ઊર્જા બચત & 30-50\% ઘટાડો \\
        \textbf{આર્થિક} & કોસ્ટ બચત & 20\% ઓપરેટિંગ કોસ્ટ \\
        \textbf{આરોગ્ય} & હવાની ગુણવત્તા & ઉત્પાદકતા વધારો \\
        \bottomrule
    \end{tabulary}
    \end{answertable}

    \begin{mnemonicbox}ગ્રીન બિલ્ડિંગ્સ પર્યાવરણીય આર્થિક આરોગ્ય આપે છે (GBEEH)\end{mnemonicbox}
\end{solutionbox}

\questionmarks{5}{d}{3}
\textbf{ભારતમાં પર્યાવરણ સંબંધિત વિવિધ કાયદાઓ જણાવો અને કોઈપણ એક સમજાવો.}

\begin{solutionbox}
    \textbf{જવાબ:}
    \textbf{ભારતમાં પર્યાવરણીય કાયદાઓ:}
    \begin{itemize}
        \item \textbf{Water (Prevention and Control of Pollution) Act, 1974}
        \item \textbf{Air (Prevention and Control of Pollution) Act, 1981}
        \item \textbf{Environment Protection Act, 1986}
        \item \textbf{Wildlife Protection Act, 1972}
        \item \textbf{Forest (Conservation) Act, 1980}
        \item \textbf{Biodiversity Act, 2002}
    \end{itemize}

    \textbf{Environment Protection Act, 1986:}
    
    \textbf{હેતુઓ:}
    \begin{itemize}
        \item \textbf{વ્યાપક માળખું}: એકંદર પર્યાવરણ સંરક્ષણ
        \item \textbf{પ્રદૂષણ નિવારણ}: હવા, પાણી, માટી દૂષણ નિયંત્રણ
        \item \textbf{માનદંડ સેટિંગ}: પર્યાવરણીય ગુણવત્તા માનદંડો
        \item \textbf{અમલીકરણ}: ઉલ્લંઘન માટે દંડ
    \end{itemize}

    \textbf{શક્તિઓ:}
    \begin{itemize}
        \item \textbf{કેન્દ્ર સરકાર સત્તા}: પર્યાવરણીય નિયમો
        \item \textbf{નિરીક્ષણ અધિકારો}: ઔદ્યોગિક સુવિધાઓની દેખરેખ
        \item \textbf{બંધ કરવાના આદેશો}: બિન-અનુપાલન કરતા ઉદ્યોગો
        \item \textbf{કટોકટીના પગલાં}: પર્યાવરણીય સંકટોનો પ્રતિસાદ
    \end{itemize}

    \textbf{મહત્વ:}
    \begin{itemize}
        \item \textbf{છત્ર કાયદો}: બધા પર્યાવરણીય પાસાઓને આવરે છે
        \item \textbf{ભોપાલ દુર્ઘટના પછી}: ઔદ્યોગિક અકસ્માતોનો પ્રતિસાદ
    \end{itemize}

    \begin{mnemonicbox}પાણી હવા પર્યાવરણ વન્યજીવ વન જૈવવિવિધતા (WAEWFB)\end{mnemonicbox}
\end{solutionbox}

\end{document}
