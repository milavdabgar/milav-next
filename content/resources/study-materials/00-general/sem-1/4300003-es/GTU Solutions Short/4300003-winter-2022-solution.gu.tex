\documentclass{article}

% content/resources/templates/preamble.tex
\usepackage[margin=0.6in]{geometry}
\author{Milav Dabgar}
\usepackage{amsmath,amssymb,amsthm}
\usepackage{booktabs}
\usepackage{multirow}
\usepackage{xcolor}
\usepackage{tcolorbox}
\tcbuselibrary{breakable,skins}
\usepackage[colorlinks=true,linkcolor=blue]{hyperref}
\usepackage{titlesec}
\usepackage{enumitem}
\usepackage{tikz}
\usepackage{pgfplots}
\usepackage{circuitikz}
\usepackage[version=4]{mhchem}
\usepackage{longtable}
\usepackage{array}
\usepackage{float}
\usepackage{caption}
\usepackage{listings}

\lstset{
  basicstyle=\small\ttfamily,
  breaklines=true,
  breakatwhitespace=false,
  postbreak=\mbox{\textcolor{red}{$\hookrightarrow$}\space},
  float=false,
  numbers=left,
  numberstyle=\tiny\color{gray},
  numbersep=10pt,
  xleftmargin=2em,
  keywordstyle=\color{blue},
  commentstyle=\color{green!60!black},
  stringstyle=\color{purple},
  backgroundcolor=\color{gray!5},
  showstringspaces=false,
  tabsize=2,
  captionpos=b,
  keepspaces=true,
  columns=flexible
}

\pgfplotsset{compat=1.18}
\usetikzlibrary{shapes,arrows,positioning,calc,patterns,decorations.pathmorphing,decorations.markings,arrows.meta}

% Color scheme
\definecolor{headcolor}{RGB}{0,102,204}
\definecolor{keycolor}{RGB}{220,20,60}
\definecolor{solutioncolor}{RGB}{34,139,34}
\definecolor{mnemoniccolor}{RGB}{148,0,211}
\definecolor{codecolor}{RGB}{0,0,100}

% Spacing
\setlength{\parskip}{3pt}
\setlist[itemize]{nosep}
\setlist[enumerate]{nosep}

% Title formatting
\titleformat{\section}{\Large\bfseries\color{headcolor}}{\thesection}{1em}{}
\titleformat{\subsection}{\large\bfseries\color{headcolor}}{\thesubsection}{1em}{}

% Pandoc tightlist compatibility
\providecommand{\tightlist}{%
  \setlength{\itemsep}{0pt}\setlength{\parskip}{0pt}}

% Pandoc longtable compatibility
\newcounter{none}
\def\thenone{}


% content/resources/templates/gujarati-boxes.tex
\usepackage{fontspec}
\usepackage{polyglossia}

% Set Gujarati as main language (document is primarily in Gujarati)
% Note: gloss-gujarati.ldf doesn't exist in polyglossia, but it will use hyphenation patterns
\setdefaultlanguage{gujarati}
\setotherlanguage{english}

% Configure Gujarati font properly
% Use Language=Default to prevent polyglossia from trying to add language-specific features
% that don't exist for Gujarati, which causes "empty feature" warnings
\newfontfamily\gujaratifont[Script=Gujarati,AutoFakeBold=2.5,AutoFakeSlant=0.3]{Noto Sans Gujarati}
\setmainfont[Script=Gujarati,AutoFakeBold=2.5,AutoFakeSlant=0.3]{Noto Sans Gujarati}
% Use Noto Sans Gujarati for monospace to support Gujarati in text
\setmonofont[Scale=0.9]{Noto Sans Gujarati}

% Configure English to use the same font
\newfontfamily\englishfont[Script=Gujarati,AutoFakeBold=2.5,AutoFakeSlant=0.3]{Noto Sans Gujarati}

% Translations for polyglossia
\gappto\captionsgujarati{
  \renewcommand{\tablename}{કોષ્ટક}
  \renewcommand{\figurename}{આકૃતિ}
}

% Helper for TikZ nodes to ensure Gujarati font
\newcommand{\gu}[1]{{\gujaratifont #1}}

% Custom environments
\newtcolorbox{solutionbox}{
    breakable,
    enhanced,
    colback=solutioncolor!5!white,
    colframe=solutioncolor!75!black,
    fonttitle=\bfseries,
    title=જવાબ
}

\newtcolorbox{solutionboxnobreak}{
 colback=solutioncolor!5!white,
 colframe=solutioncolor!75!black,
 fonttitle=\bfseries,
 title=જવાબ
}

\newtcolorbox{keyformula}{
 breakable,
 enhanced,
 colback=keycolor!5!white,
 colframe=keycolor!75!black,
 fonttitle=\bfseries,
 title=રાસાયણિક સમીકરણ/સૂત્ર
}

\newtcolorbox{mnemonicbox}{
 breakable,
 enhanced,
 colback=mnemoniccolor!5!white,
 colframe=mnemoniccolor!75!black,
 fonttitle=\bfseries,
 title=મેમરી ટ્રીક
}


% Custom commands for GTU solutions
% This file defines semantic commands for consistent formatting

% Question command with automatic formatting
\newcommand{\question}[2]{%
  \section*{Question #1}%
  \textbf{#2}%
}

% OR question variant
\newcommand{\questionor}[2]{%
  \section*{Question #1 OR}%
  \textbf{#2}%
}

% Proper table environment with caption
\newenvironment{answertable}[1]{%
  \begin{table}[htbp]
  \centering
  \caption{#1}
}{%
  \end{table}
}

% Proper figure environment for diagrams
\newenvironment{answerdiagram}[1]{%
  \begin{figure}[htbp]
  \centering
  \caption{#1}
}{%
  \end{figure}
}

% Semantic markup for key terms
\newcommand{\keyword}[1]{\textbf{#1}}
\newcommand{\code}[1]{\texttt{#1}}
\newcommand{\classname}[1]{\texttt{#1}}
\newcommand{\methodname}[1]{\texttt{#1}}

% Proper quotation marks
\newcommand{\mnemonic}[1]{``#1''}


\title{પર્યાવરણ અને ટકાઉપણું (4300003) - શિયાળો 2022 ઉકેલ}
\date{March 01, 2023}

\begin{document}
\maketitle

\questionmarks{1}{a}{3}
\textbf{વૈશ્વિક પર્યાવરણીય ઉછાળ ક્યારે થાય છે? કારણો સાથે સમજાવો.}

\begin{solutionbox}
    \textbf{જવાબ:}

    \begin{answertable}{પર્યાવરણીય ઉછાળની શરતો}
    \begin{tabulary}{\linewidth}{L L L}
        \toprule
        \textbf{શરત} & \textbf{વર્ણન} & \textbf{અસર} \\
        \midrule
        \textbf{સંસાધન ઘટાડો} & વપરાશ પુનઃજનન દર કરતા વધારે & ખાધ સંચય \\
        \textbf{વસ્તી દબાણ} & માનવ માંગ વહન ક્ષમતા કરતા વધારે & સંસાધન અછત \\
        \textbf{કચરાનો સંગ્રહ} & ઉત્પાદન શોષણ ક્ષમતા કરતા વધારે & પર્યાવરણ અધોગતિ \\
        \bottomrule
    \end{tabulary}
    \end{answertable}

    \textbf{પર્યાવરણીય ઉછાળ} ત્યારે થાય છે જ્યારે માનવતાનું પર્યાવરણીય પદચિહ્ન પૃથ્વીની જૈવિક ક્ષમતા કરતા વધી જાય છે.

    \textbf{મુખ્ય કારણો}:
    \begin{itemize}
        \item \textbf{વસ્તી વૃદ્ધિ}: માનવ સંખ્યામાં વધારો
        \item \textbf{વપરાશની પદ્ધતિ}: વ્યક્તિ દીઠ ઊંચો સંસાધન ઉપયોગ
        \item \textbf{ટેકનોલોજીની અસર}: બિનકાર્યક્ષમ સંસાધન ઉપયોગ
    \end{itemize}

    \begin{mnemonicbox}POP-CON-TECH (Population-Consumption-Technology)\end{mnemonicbox}
\end{solutionbox}

\questionmarks{1}{b}{4}
\textbf{આકૃતિની મદદથી પોષણ કડી સમજાવો.}

\begin{solutionbox}
    \textbf{જવાબ:}

    \begin{center}
    \begin{tikzpicture}[node distance=1.5cm, auto]
        \node (a) [gtu block] {સૂર્ય ઉર્જા};
        \node (b) [gtu block, right=of a] {ઉત્પાદક:\\લીલા છોડ};
        \node (c) [gtu block, right=of b] {પ્રાથમિક ઉપભોક્તા:\\શાકાહારી};
        \node (d) [gtu block, below=of c] {ગૌણ ઉપભોક્તા:\\માંસાહારી};
        \node (e) [gtu block, left=of d] {તૃતીય ઉપભોક્તા:\\શિકારી};
        \node (f) [gtu block, left=of e] {અપઘટક:\\બેક્ટેરિયા/ફૂગ};
        \node (g) [gtu block, above=of f] {માટીમાં પોષક તત્વો};

        \draw [gtu arrow] (a) -- (b);
        \draw [gtu arrow] (b) -- (c);
        \draw [gtu arrow] (c) -- (d);
        \draw [gtu arrow] (d) -- (e);
        \draw [gtu arrow] (e) -- (f);
        \draw [gtu arrow] (f) -- (g);
        \draw [gtu arrow] (g) -- (b);
    \end{tikzpicture}
    \end{center}

    \textbf{પોષણ કડી} એ ઇકોસિસ્ટમમાં એક ટ્રોફિક સ્તરથી બીજા સ્તરમાં ઉર્જા સ્થાનાંતરણનો રેખીય ક્રમ દર્શાવે છે.

    \textbf{ઘટકો}:
    \begin{itemize}
        \item \textbf{ઉત્પાદકો}: સૂર્ય ઉર્જાને રાસાયણિક ઉર્જામાં રૂપાંતરિત કરે છે
        \item \textbf{પ્રાથમિક ઉપભોક્તા}: ઉત્પાદકોને ખાય છે (શાકાહારી)
        \item \textbf{ગૌણ ઉપભોક્તા}: પ્રાથમિક ઉપભોક્તાને ખાય છે (માંસાહારી)
        \item \textbf{અપઘટક}: મૃત જીવોને વિઘટિત કરે છે
    \end{itemize}

    \textbf{ઉર્જા પ્રવાહ}: સૂર્યથી ટોચના શિકારી સુધી એક દિશામાં 10\% કાર્યક્ષમતા સાથે.

    \begin{mnemonicbox}PPSD (Producer-Primary-Secondary-Decomposer)\end{mnemonicbox}
\end{solutionbox}

\questionmarks{1}{c}{7}
\textbf{કાર્બન ચક્ર પર ટૂંકી નોંધ લખો.}

\begin{solutionbox}
    \textbf{જવાબ:}

    \begin{center}
    \begin{tikzpicture}[node distance=2cm, auto]
        \node (a) [gtu block] {વાતાવરણીય CO$_2$};
        \node (b) [gtu block, below left=of a] {પ્રકાશસંશ્લેષણ};
        \node (c) [gtu block, below=of b] {છોડનું બાયોમાસ};
        \node (d) [gtu block, right=of c] {પ્રાણીઓનો વપરાશ};
        \node (e) [gtu block, above right=of d] {શ્વસન};
        \node (f) [gtu block, below=of c] {અપઘટન};
        \node (g) [gtu block, left=of a] {સમુદ્રમાં વિસર્જન};
        \node (h) [gtu block, below=of g] {સમુદ્રી જીવન};
        \node (i) [gtu block, right=of a] {અશ્મિભૂત ઇંધણ દહન};

        \draw [gtu arrow] (a) -- (b);
        \draw [gtu arrow] (b) -- (c);
        \draw [gtu arrow] (c) -- (d);
        \draw [gtu arrow] (d) -- (e);
        \draw [gtu arrow] (e) -- (a);
        \draw [gtu arrow] (c) -- (f);
        \draw [gtu arrow] (f) -- (a);
        \draw [gtu arrow] (a) -- (g);
        \draw [gtu arrow] (g) -- (h);
        \draw [gtu arrow] (h) to [bend left] (a);
        \draw [gtu arrow] (i) -- (a);
    \end{tikzpicture}
    \end{center}

    \textbf{કાર્બન ચક્ર} એ જૈવ-ભૂ-રાસાયણિક પ્રક્રિયા છે જેમાં કાર્બન વાતાવરણ, જીવમંડળ, જળમંડળ અને ભૂમંડળમાં ફરે છે.

    \textbf{મુખ્ય પ્રક્રિયાઓ}:
    \begin{itemize}
        \item \textbf{પ્રકાશસંશ્લેષણ}: છોડ વાતાવરણમાંથી CO$_2$ શોષે છે
        \item \textbf{શ્વસન}: જીવો CO$_2$ પાછું વાતાવરણમાં છોડે છે
        \item \textbf{અપઘટન}: મૃત કાર્બનિક પદાર્થ સંગ્રહિત કાર્બન મુક્ત કરે છે
        \item \textbf{સમુદ્રી વિનિમય}: CO$_2$ સમુદ્રના પાણીમાં ઓગળીને કાર્બોનિક એસિડ બનાવે છે
    \end{itemize}

    \textbf{માનવીય પ્રભાવ}:
    \begin{itemize}
        \item \textbf{અશ્મિભૂત ઇંધણ દહન}: વાતાવરણીય CO$_2$ વધારે છે
        \item \textbf{વનનાશ}: કાર્બન પ્રતિબંધની ક્ષમતા ઘટાડે છે
        \item \textbf{ઔદ્યોગિક પ્રક્રિયાઓ}: વધારાના કાર્બન ઉત્સર્જન
    \end{itemize}

    \textbf{પર્યાવરણીય મહત્વ}: વાતાવરણીય CO$_2$ સંતુલન જાળવે છે, વૈશ્વિક તાપમાન નિયંત્રિત કરે છે, જીવન પ્રક્રિયાઓને આધાર આપે છે.

    \begin{mnemonicbox}PRDO-FDI (Photosynthesis-Respiration-Decomposition-Ocean, Fossil-Deforestation-Industry)\end{mnemonicbox}
\end{solutionbox}

\questionmarks{1}{c}{7}
\textbf{જળીય નિવસનતંત્રનું વર્ગીકરણ કરો. દરિયાઈ નિવસનતંત્ર સમજાવો.}

\begin{solutionbox}
    \textbf{જવાબ:}

    \begin{answertable}{જળીય નિવસનતંત્ર વર્ગીકરણ}
    \begin{tabulary}{\linewidth}{L L L}
        \toprule
        \textbf{પ્રકાર} & \textbf{લાક્ષણિકતાઓ} & \textbf{ઉદાહરણો} \\
        \midrule
        \textbf{તાજા પાણીનું} & ઓછું મીઠું (<1\%) & નદીઓ, તળાવો, તાલાવો \\
        \textbf{દરિયાઈ} & વધારે મીઠું (3.5\%) & મહાસાગરો, સમુદ્રો \\
        \textbf{ખારા} & મિશ્રિત તાજા-ખારા પાણી & નદીમુખો, લગૂન \\
        \bottomrule
    \end{tabulary}
    \end{answertable}

    \textbf{દરિયાઈ નિવસનતંત્રના ઘટકો}:

    \begin{center}
    \begin{tikzpicture}[node distance=1.5cm, auto]
        \node (root) [gtu root] {દરિયાઈ નિવસનતંત્ર};
        \node (pelagic) [gtu child, below left=of root] {પેલેજિક ઝોન};
        \node (benthic) [gtu child, below right=of root] {બેન્થિક ઝોન};
        
        \node (photic) [gtu block, below=of pelagic, xshift=-1cm] {ફોટિક ઝોન:\\0-200m};
        \node (aphotic) [gtu block, below=of pelagic, xshift=1cm] {એફોટિક ઝોન:\\>200m};
        
        \node (shelf) [gtu block, below=of benthic, xshift=-1cm] {ખંડીય શેલ્ફ};
        \node (deep) [gtu block, below=of benthic, xshift=1cm] {ઊંડા સમુદ્રનું તળ};

        \draw [gtu arrow] (root) -- (pelagic);
        \draw [gtu arrow] (root) -- (benthic);
        \draw [gtu arrow] (pelagic) -- (photic);
        \draw [gtu arrow] (pelagic) -- (aphotic);
        \draw [gtu arrow] (benthic) -- (shelf);
        \draw [gtu arrow] (benthic) -- (deep);
    \end{tikzpicture}
    \end{center}

    \textbf{દરિયાઈ નિવસનતંત્ર} પૃથ્વીની સપાટીના 71\% ભાગને આવરી લે છે, જેમાં જટિલ ખાદ્ય જાળ સાથે ખારા પાણીના મોટા વિસ્તારો છે.

    \textbf{ઝોન}:
    \begin{itemize}
        \item \textbf{પેલેજિક}: ખુલ્લા પાણીનો સ્તંભ જેમાં પ્લાન્કટન, માછલીઓ
        \item \textbf{બેન્થિક}: સમુદ્રનું તળ જેમાં તળિયે રહેતા જીવો
        \item \textbf{આંતરજોવારી}: ભરતી-ઓટના વચ્ચેનો કિનારાનો વિસ્તાર
    \end{itemize}

    \textbf{મહત્વ}:
    \begin{itemize}
        \item \textbf{આબોહવા નિયંત્રણ}: સમુદ્રી પ્રવાહો વૈશ્વિક તાપમાન નિયંત્રિત કરે છે
        \item \textbf{ઓક્સિજન ઉત્પાદન}: દરિયાઈ ફાયટોપ્લાન્કટન વાતાવરણીય ઓક્સિજનના 50\% ઉત્પાદન કરે છે
        \item \textbf{આર્થિક મૂલ્ય}: મત્સ્યવ્યવસાય, પરિવહન, પર્યટન
    \end{itemize}

    \begin{mnemonicbox}PBI-COE (Pelagic-Benthic-Intertidal, Climate-Oxygen-Economy)\end{mnemonicbox}
\end{solutionbox}

\questionmarks{2}{a}{3}
\textbf{પૃથ્વીની વહન ક્ષમતા એટલે શું?}

\begin{solutionbox}
    \textbf{જવાબ:}

    \begin{answertable}{વહન ક્ષમતાના કારકો}
    \begin{tabulary}{\linewidth}{L L L}
        \toprule
        \textbf{કારક} & \textbf{વર્ણન} & \textbf{મર્યાદા} \\
        \midrule
        \textbf{સંસાધનો} & ઉપલબ્ધ જમીન, પાણી, ખનિજો & મર્યાદિત \\
        \textbf{ખાદ્ય ઉત્પાદન} & કૃષિ ક્ષમતા & માટી દ્વારા મર્યાદિત \\
        \textbf{કચરા શોષણ} & ઇકોસિસ્ટમની કચરા પ્રક્રિયા & સંતૃપ્તિ બિંદુ \\
        \bottomrule
    \end{tabulary}
    \end{answertable}

    \textbf{વહન ક્ષમતા} એ પર્યાવરણને અધોગતિ કર્યા વિના અનિશ્ચિત સમય સુધી ટકાવી શકાય તેવી મહત્તમ વસ્તી માપ છે.

    \textbf{પૃથ્વીની વહન ક્ષમતા} આ પર આધાર રાખે છે:
    \begin{itemize}
        \item \textbf{સંસાધન ઉપલબ્ધતા}: તાજું પાણી, ખેતીલાયક જમીન, ઉર્જા સ્રોતો
        \item \textbf{ટેકનોલોજી સ્તર}: સંસાધન ઉપયોગની કાર્યક્ષમતા
        \item \textbf{વપરાશની પદ્ધતિ}: વ્યક્તિ દીઠ સંસાધન માંગ
    \end{itemize}

    \textbf{વર્તમાન અંદાજ}: વપરાશ સ્તર અને તકનીકી પ્રગતિના આધારે 4-16 અબજ લોકો.

    \begin{mnemonicbox}RTC (Resources-Technology-Consumption)\end{mnemonicbox}
\end{solutionbox}

\questionmarks{2}{b}{4}
\textbf{આહાર જાળ એ પોષણ કડી સાથે કેવી રીતે સંબંધિત છે?}

\begin{solutionbox}
    \textbf{જવાબ:}

    \begin{center}
    \begin{tikzpicture}[node distance=1.5cm, auto]
        \node (grass) [gtu block] {ઘાસ};
        \node (rabbit) [gtu block, right=of grass] {સસલું};
        \node (deer) [gtu block, below=of grass] {હરણ};
        \node (fox) [gtu block, right=of rabbit] {શિયાળ};
        \node (hawk) [gtu block, below=of fox] {બાજ};
        \node (wolf) [gtu block, right=of deer] {વરુ};
        \node (decomp) [gtu block, right=of fox] {અપઘટક};

        \draw [gtu arrow] (grass) -- (rabbit);
        \draw [gtu arrow] (grass) -- (deer);
        \draw [gtu arrow] (rabbit) -- (fox);
        \draw [gtu arrow] (rabbit) -- (hawk);
        \draw [gtu arrow] (deer) -- (fox);
        \draw [gtu arrow] (deer) -- (wolf);
        \draw [gtu arrow] (fox) -- (decomp);
        \draw [gtu arrow] (hawk) -- (decomp);
        \draw [gtu arrow] (wolf) -- (decomp);
    \end{tikzpicture}
    \end{center}

    \textbf{આહાર જાળ} એ ઇકોસિસ્ટમમાં જટિલ ખાદ્ય સંબંધો દર્શાવતા બહુવિધ પોષણ કડીઓનું પરસ્પર જોડાયેલું જાળ છે.

    \textbf{આહાર જાળ અને પોષણ કડી વચ્ચેનો સંબંધ}:
    \begin{itemize}
        \item \textbf{પોષણ કડી}: ઉર્જા સ્થાનાંતરણનો રેખીય ક્રમ
        \item \textbf{આહાર જાળ}: બહુવિધ પરસ્પર જોડાયેલી પોષણ કડીઓ
        \item \textbf{જટિલતા}: આહાર જાળ વાસ્તવિક ઇકોસિસ્ટમ ક્રિયાપ્રતિક્રિયા દર્શાવે છે
        \item \textbf{સ્થિરતા}: બહુવિધ માર્ગો ઇકોસિસ્ટમ પ્રતિરોધક ક્ષમતા પ્રદાન કરે છે
    \end{itemize}

    \textbf{મુખ્ય તફાવતો}:
    \begin{itemize}
        \item \textbf{માળખું}: કડી રેખીય, જાળ નેટવર્ક આધારિત
        \item \textbf{ઉર્જા પ્રવાહ}: કડી એક માર્ગ, જાળ બહુવિધ માર્ગો
        \item \textbf{પ્રજાતિ ક્રિયાપ્રતિક્રિયા}: જાળ સર્વભક્ષીતા અને વૈકલ્પિક ખાદ્ય દર્શાવે છે
    \end{itemize}

    \begin{mnemonicbox}LNCR (Linear-Network, Chain-Resilience)\end{mnemonicbox}
\end{solutionbox}

\questionmarks{2}{c}{7}
\textbf{હવા પ્રદૂષણ પર નોંધ લખો.}

\begin{solutionbox}
    \textbf{જવાબ:}

    \begin{answertable}{હવા પ્રદૂષણના સ્રોતો અને અસરો}
    \begin{tabulary}{\linewidth}{L L L}
        \toprule
        \textbf{પ્રદૂષક} & \textbf{સ્રોત} & \textbf{આરોગ્ય અસર} \\
        \midrule
        \textbf{PM2.5/PM10} & વાહનો, ઉદ્યોગો & શ્વસન રોગો \\
        \textbf{SO$_2$} & કોલસાનું દહન & એસિડ વરસાદ, અસ્થમા \\
        \textbf{NO$_x$} & વાહન એક્ઝોસ્ટ & સ્મોગ રચના \\
        \textbf{CO} & અપૂર્ણ દહન & ઓક્સિજનની ઉણપ \\
        \bottomrule
    \end{tabulary}
    \end{answertable}

    \textbf{હવા પ્રદૂષણ} એ વાતાવરણમાં હાનિકારક પદાર્થોથી થતું દૂષણ છે જે માનવ આરોગ્ય અને પર્યાવરણ પર નકારાત્મક અસર કરે છે.

    \textbf{સ્રોત પ્રમાણે વર્ગીકરણ}:
    \begin{itemize}
        \item \textbf{પ્રાથમિક પ્રદૂષક}: સીધું ઉત્સર્જિત (CO, SO$_2$, કણો)
        \item \textbf{ગૌણ પ્રદૂષક}: રાસાયણિક પ્રતિક્રિયા દ્વારા રચાય (ઓઝોન, એસિડ વરસાદ)
    \end{itemize}

    \textbf{મુખ્ય સ્રોતો}:
    \begin{itemize}
        \item \textbf{ગતિશીલ સ્રોતો}: વાહનો, વિમાન, જહાજો
        \item \textbf{સ્થિર સ્રોતો}: પાવર પ્લાન્ટ, ઉદ્યોગો, રહેણાંક હોટિંગ
        \item \textbf{કુદરતી સ્રોતો}: જ્વાળામુખી વિસ્ફોટ, જંગલી આગ, ધૂળના તોફાન
    \end{itemize}

    \textbf{નિયંત્રણ પગલાં}:
    \begin{itemize}
        \item \textbf{તકનીકી}: કેટેલિટિક કન્વર્ટર, સ્ક્રબર, ફિલ્ટર
        \item \textbf{નિયમનકારી}: ઉત્સર્જન ધોરણો, ઇંધણ ગુણવત્તા નિયમો
        \item \textbf{વૈકલ્પિક ઊર્જા}: નવીકરણીય સ્રોતો, ઇલેક્ટ્રિક વાહનો
    \end{itemize}

    \textbf{આરોગ્ય અસરો}: શ્વસન રોગો, હૃદયરોગ સમસ્યાઓ, કેન્સર, આયુષ્યમાં ઘટાડો.

    \textbf{પર્યાવરણીય અસરો}: એસિડ વરસાદ, ઓઝોન ઘટાડો, આબોહવા પરિવર્તન, દૃશ્યતામાં ઘટાડો.

    \begin{mnemonicbox}PSMT-RE-HE (Primary-Secondary-Mobile-stationary-Technological-Regulatory-Health-Environment)\end{mnemonicbox}
\end{solutionbox}

\questionmarks{2}{a}{3}
\textbf{પ્લાસ્ટિક કચરાની પર્યાવરણ પર ખરાબ અસરો સમજાવો.}

\begin{solutionbox}
    \textbf{જવાબ:}

    \begin{answertable}{પ્લાસ્ટિક કચરાની પર્યાવરણીય અસરો}
    \begin{tabulary}{\linewidth}{L L L}
        \toprule
        \textbf{અસરનું ક્ષેત્ર} & \textbf{અસર} & \textbf{સમયગાળો} \\
        \midrule
        \textbf{દરિયાઈ જીવન} & ફસાવટ, ગળવું & કાયમી \\
        \textbf{માટી} & માઇક્રોપ્લાસ્ટિક દૂષણ & 500+ વર્ષો \\
        \textbf{ખાદ્ય શૃંખલા} & બાયોએક્યુમ્યુલેશન & પેઢીદર પેઢી \\
        \bottomrule
    \end{tabulary}
    \end{answertable}

    \textbf{પ્લાસ્ટિક કચરો} તેની બિન-બાયોડિગ્રેડેબલ પ્રકૃતિને કારણે ગંભીર પર્યાવરણીય અધોગતિનું કારણ બને છે.

    \textbf{પર્યાવરણીય અસરો}:
    \begin{itemize}
        \item \textbf{દરિયાઈ પ્રદૂષણ}: સમુદ્રમાં પ્લાસ્ટિક દરિયાઈ પ્રાણીઓને ફસાવટ અને ગળવાથી મારી નાખે છે
        \item \textbf{માટી દૂષણ}: માઇક્રોપ્લાસ્ટિક માટીની ફળદ્રુપતા અને પાકની વૃદ્ધિને અસર કરે છે
        \item \textbf{ખાદ્ય શૃંખલા વિક્ષેપ}: પ્લાસ્ટિકના કણો જીવોમાં સંચિત થાય છે
    \end{itemize}

    \textbf{લાંબાગાળાની અસરો}: કાયમી કાર્બનિક પ્રદૂષક, આવાસનો વિનાશ, ઇકોસિસ્ટમ અસંતુલન.

    \begin{mnemonicbox}MSF (Marine-Soil-Foodchain)\end{mnemonicbox}
\end{solutionbox}

\questionmarks{2}{b}{4}
\textbf{દૂષિત પાણીના લક્ષણો કયા છે? જળ પ્રદૂષણના મુખ્ય સ્રોતોની યાદી બનાવો.}

\begin{solutionbox}
    \textbf{જવાબ:}

    \begin{answertable}{જળ પ્રદૂષણના સૂચકો અને સ્રોતો}
    \begin{tabulary}{\linewidth}{L L L}
        \toprule
        \textbf{લક્ષણો} & \textbf{માપન} & \textbf{સ્રોતો} \\
        \midrule
        \textbf{ઊંચું BOD/COD} & >5 mg/L & ઔદ્યોગિક ડિસ્ચાર્જ \\
        \textbf{ટર્બિડિટી} & ધૂંધળાપણું & કૃષિ અપવાહ \\
        \textbf{pH ફેરફાર} & <6.5 અથવા >8.5 & એસિડ ખાણ ડ્રેનેજ \\
        \textbf{દુર્ગંધ} & H$_2$S ગંધ & ગટર ડિસ્ચાર્જ \\
        \bottomrule
    \end{tabulary}
    \end{answertable}

    \textbf{દૂષિત પાણીના લક્ષણો}:
    \begin{itemize}
        \item \textbf{ભૌતિક}: રંગ ફેરફાર, ટર્બિડિટી, તરતા કચરા, ગંધ
        \item \textbf{રાસાયણિક}: ઊંચું BOD/COD, pH વિચલન, ભારે ધાતુઓ, ઝેરી સંયોજનો
        \item \textbf{જૈવિક}: રોગકારક સૂક્ષ્મજીવો, એલ્ગલ બ્લૂમ, માછલીઓનું મૃત્યુ
    \end{itemize}

    \textbf{મુખ્ય સ્રોતો}:
    \begin{itemize}
        \item \textbf{બિંદુ સ્રોતો}: ઔદ્યોગિક ડિસ્ચાર્જ, ગટર આઉટફોલ, કેન્દ્રિત પ્રાણી ખવડાવવું
        \item \textbf{બિન-બિંદુ સ્રોતો}: કૃષિ અપવાહ, શહેરી વરસાદી પાણી, વાતાવરણીય નિક્ષેપ
    \end{itemize}

    \begin{mnemonicbox}PCB-PIN (Physical-Chemical-Biological, Point-Non-point)\end{mnemonicbox}
\end{solutionbox}

\questionmarks{2}{c}{7}
\textbf{ઈ-કચરો શું છે? ઈ-કચરાને પુન:ઉપયોગી કેવી રીતે બનાવી શકાય?}

\begin{solutionbox}
    \textbf{જવાબ:}

    \begin{answertable}{ઈ-કચરાનું વર્ગીકરણ}
    \begin{tabulary}{\linewidth}{L L L}
        \toprule
        \textbf{શ્રેણી} & \textbf{ઉદાહરણો} & \textbf{હાનિકારક ઘટકો} \\
        \midrule
        \textbf{મોટા ઉપકરણો} & રેફ્રિજરેટર, વોશિંગ મશીન & CFCs, ભારે ધાતુઓ \\
        \textbf{નાના ઉપકરણો} & માઇક્રોવેવ, વેક્યુમ ક્લીનર & પ્લાસ્ટિક, ધાતુઓ \\
        \textbf{IT સાધનો} & કમ્પ્યુટર, પ્રિંટર & લેડ, પારો, કેડમિયમ \\
        \textbf{ઉપભોક્તા ઇલેક્ટ્રોનિક્સ} & TV, મોબાઇલ ફોન & દુર્લભ પૃથ્વી તત્વો \\
        \bottomrule
    \end{tabulary}
    \end{answertable}

    \textbf{ઈ-કચરાનું વર્ગીકરણ}:
    \begin{itemize}
        \item \textbf{સફેદ સામાન}: મોટા ઘરેલું ઉપકરણો
        \item \textbf{બ્રાઉન સામાન}: મનોરંજન ઇલેક્ટ્રોનિક્સ
        \item \textbf{ગ્રે સામાન}: IT અને ટેલિકોમ્યુનિકેશન સાધનો
        \item \textbf{ગ્રીન સામાન}: નવીકરણીય ઊર્જા સાધનો
    \end{itemize}

    \textbf{ઈ-કચરા રિસાયકલિંગ પ્રક્રિયા}:

    \begin{center}
    \begin{tikzpicture}[node distance=1.5cm, auto]
        \node (a) [gtu block] {સંગ્રહ};
        \node (b) [gtu block, right=of a] {વર્ગીકરણ};
        \node (c) [gtu block, right=of b] {વિઘટન};
        \node (d) [gtu block, below=of c] {કાપવું};
        \node (e) [gtu block, left=of d] {વિભાજન};
        \node (f) [gtu block, left=of e] {સામગ્રી\\પુનઃપ્રાપ્તિ};
        \node (g) [gtu block, below=of f] {શુદ્ધિકરણ};
        \node (h) [gtu block, right=of g] {નવા ઉત્પાદનો};

        \draw [gtu arrow] (a) -- (b);
        \draw [gtu arrow] (b) -- (c);
        \draw [gtu arrow] (c) -- (d);
        \draw [gtu arrow] (d) -- (e);
        \draw [gtu arrow] (e) -- (f);
        \draw [gtu arrow] (f) -- (g);
        \draw [gtu arrow] (g) -- (h);
    \end{tikzpicture}
    \end{center}

    \textbf{રિસાયકલિંગ પદ્ધતિઓ}:
    \begin{itemize}
        \item \textbf{યાંત્રિક}: સામગ્રીનું ભૌતિક વિભાજન
        \item \textbf{ધાતુશાસ્ત્રીય}: ધાતુ પુનઃપ્રાપ્તિ માટે ઊંચા તાપમાનની પ્રક્રિયા
        \item \textbf{રાસાયણિક}: કિંમતી ધાતુઓ માટે લીચિંગ પ્રક્રિયાઓ
    \end{itemize}

    \textbf{પડકારો}: હાનિકારક સામગ્રી હેન્ડલિંગ, જટિલ રચના, આર્થિક વ્યવહાર્યતા.

    \textbf{ફાયદાઓ}: સંસાધન સંરક્ષણ, પ્રદૂષણ નિવારણ, રોજગાર સર્જન, ખાણકામની જરૂરિયાત ઘટાડવી.

    \begin{mnemonicbox}WBGG-CSDSMR (White-Brown-Gray-Green, Collection-Sorting-Dismantling-Shredding-Separation-Material-Refining)\end{mnemonicbox}
\end{solutionbox}

\questionmarks{3}{a}{3}
\textbf{BOD અને COD વચ્ચેનો તફાવત લખો.}

\begin{solutionbox}
    \textbf{જવાબ:}

    \begin{answertable}{BOD વિ COD સરખામણી}
    \begin{tabulary}{\linewidth}{L L L}
        \toprule
        \textbf{પેરામીટર} & \textbf{BOD} & \textbf{COD} \\
        \midrule
        \textbf{પૂર્ણ સ્વરૂપ} & બાયોકેમિકલ ઓક્સિજન ડિમાન્ડ & કેમિકલ ઓક્સિજન ડિમાન્ડ \\
        \textbf{ટેસ્ટ સમયગાળો} & 5 દિવસ & 2-3 કલાક \\
        \textbf{ઓક્સિડેશન પ્રકાર} & જૈવિક & રાસાયણિક \\
        \textbf{અપઘટન} & ફક્ત બાયોડિગ્રેડેબલ કાર્બનિક & બધા કાર્બનિક સંયોજનો \\
        \bottomrule
    \end{tabulary}
    \end{answertable}

    \textbf{BOD (બાયોકેમિકલ ઓક્સિજન ડિમાન્ડ)}:
    \begin{itemize}
        \item સૂક્ષ્મજીવો દ્વારા વપરાતી ઓક્સિજન માપે છે
        \item બાયોડિગ્રેડેબલ કાર્બનિક પ્રદૂષણ દર્શાવે છે
        \item માનક ટેસ્ટ: 20\textdegree{}C પર 5 દિવસ
    \end{itemize}

    \textbf{COD (કેમિકલ ઓક્સિજન ડિમાન્ડ)}:
    \begin{itemize}
        \item રાસાયણિક ઓક્સિડેશન માટે જરૂરી ઓક્સિજન માપે છે
        \item કુલ કાર્બનિક પ્રદૂષણ દર્શાવે છે
        \item મજબૂત ઓક્સિડાઇઝિંગ એજન્ટ વાપરે છે (પોટેશિયમ ડાઇક્રોમેટ)
    \end{itemize}

    \begin{mnemonicbox}BTCD (Biological-Time-Chemical-Degradation)\end{mnemonicbox}
\end{solutionbox}

\questionmarks{3}{b}{4}
\textbf{ઘન કચરાનું વર્ગીકરણ કરો.}

\begin{solutionbox}
    \textbf{જવાબ:}

    \begin{answertable}{ઘન કચરાનું વર્ગીકરણ}
    \begin{tabulary}{\linewidth}{L L L}
        \toprule
        \textbf{વર્ગીકરણ} & \textbf{પ્રકાર} & \textbf{ઉદાહરણો} \\
        \midrule
        \textbf{સ્રોત દ્વારા} & મ્યુનિસિપલ, ઔદ્યોગિક, કૃષિ & ઘરેલું, ફેક્ટરી, ખેતીનો કચરો \\
        \textbf{રચના દ્વારા} & કાર્બનિક, અકાર્બનિક & ખાદ્ય કચરો, પ્લાસ્ટિક \\
        \textbf{જોખમ દ્વારા} & હાનિકારક, બિન-હાનિકારક & તબીબી, કાગળ \\
        \bottomrule
    \end{tabulary}
    \end{answertable}

    \textbf{ઘન કચરાનું વર્ગીકરણ}:

    \begin{center}
    \begin{tikzpicture}[node distance=1.5cm, auto]
        \node (root) [gtu root] {ઘન કચરો};
        \node (mun) [gtu child, below left=of root, xshift=-2cm] {મ્યુનિસિપલ\\ઘન કચરો};
        \node (ind) [gtu child, left=of mun] {ઔદ્યોગિક કચરો};
        \node (haz) [gtu child, below right=of root, xshift=2cm] {હાનિકારક કચરો};
        \node (agri) [gtu child, right=of haz] {કૃષિ કચરો};
        
        \node (org) [gtu block, below=of mun] {કાર્બનિક:\\50-60\%};
        \node (rec) [gtu block, left=of org] {રિસાયક્લેબલ:\\20-30\%};
        \node (inert) [gtu block, right=of org] {જડ:\\10-20\%};

        \draw [gtu arrow] (root) -- (mun);
        \draw [gtu arrow] (root) -- (ind);
        \draw [gtu arrow] (root) -- (haz);
        \draw [gtu arrow] (root) -- (agri);
        \draw [gtu arrow] (mun) -- (org);
        \draw [gtu arrow] (mun) -- (rec);
        \draw [gtu arrow] (mun) -- (inert);
    \end{tikzpicture}
    \end{center}

    \textbf{સ્રોત દ્વારા}:
    \begin{itemize}
        \item \textbf{મ્યુનિસિપલ}: રહેણાંક, વ્યાપારી, સંસ્થાકીય કચરો
        \item \textbf{ઔદ્યોગિક}: ઉત્પાદન, પ્રક્રિયાકરણ ઉપ-ઉત્પાદનો
        \item \textbf{કૃષિ}: પાક અવશેષો, પ્રાણીઓનો કચરો
    \end{itemize}

    \textbf{રચના દ્વારા}: કાર્બનિક (બાયોડિગ્રેડેબલ), અકાર્બનિક (બિન-બાયોડિગ્રેડેબલ), રિસાયક્લેબલ સામગ્રી.

    \textbf{વ્યવસ્થાપન હાયરાર્કી}: ઘટાડો, પુનઃઉપયોગ, રિસાયકલ, પુનઃપ્રાપ્તિ, નિકાલ.

    \begin{mnemonicbox}MIA-OIR (Municipal-Industrial-Agricultural, Organic-Inorganic-Recyclable)\end{mnemonicbox}
\end{solutionbox}

\questionmarks{3}{c}{7}
\textbf{આકૃતિની મદદથી સોલર ફોટોવોલ્ટેઇક સિસ્ટમ સમજાવો.}

\begin{solutionbox}
    \textbf{જવાબ:}

    \begin{center}
    \begin{tikzpicture}[node distance=1.5cm, auto]
        \node (solar) [gtu start] {સૂર્ય કિરણોત્સર્ગ};
        \node (panel) [gtu block, right=of solar] {PV પેનલ};
        \node (dc) [gtu block, right=of panel] {DC પાવર};
        \node (inv) [gtu block, right=of dc] {ઇન્વર્ટર};
        \node (ac) [gtu block, right=of inv] {AC પાવર};
        \node (load) [gtu block, right=of ac] {લોડ/ગ્રિડ};
        
        \node (bat) [gtu block, below=of dc] {બેટરી};
        \node (cc) [gtu block, left=of bat] {ચાર્જ કંટ્રોલર};

        \draw [gtu arrow] (solar) -- (panel);
        \draw [gtu arrow] (panel) -- (dc);
        \draw [gtu arrow] (dc) -- (inv);
        \draw [gtu arrow] (inv) -- (ac);
        \draw [gtu arrow] (ac) -- (load);
        
        \draw [gtu arrow] (dc) -- (bat);
        \draw [gtu arrow] (bat) -- (dc);
        \draw [gtu arrow] (dc) to [bend left] (cc);
        \draw [gtu arrow] (cc) -- (bat);
        \draw [gtu arrow] (ac) to [bend left] (bat);
    \end{tikzpicture}
    \end{center}

    \textbf{સોલર ફોટોવોલ્ટેઇક સિસ્ટમ} સેમિકન્ડક્ટર સામગ્રીનો ઉપયોગ કરીને સૂર્યપ્રકાશને સીધા વીજળીમાં રૂપાંતરિત કરે છે.

    \textbf{ઘટકો}:
    \begin{itemize}
        \item \textbf{PV મોડ્યુલ}: સિલિકોન સેલ્સ પ્રકાશને DC વીજળીમાં રૂપાંતરિત કરે છે
        \item \textbf{ઇન્વર્ટર}: DC ને AC પાવરમાં રૂપાંતરિત કરે છે
        \item \textbf{બેટરી સ્ટોરેજ}: વધારાની ઊર્જા પછીના ઉપયોગ માટે સંગ્રહિત કરે છે
        \item \textbf{ચાર્જ કંટ્રોલર}: બેટરી ચાર્જિંગને નિયંત્રિત કરે છે
        \item \textbf{મોનિટરિંગ સિસ્ટમ}: પ્રદર્શન અને ખામીઓને ટ્રેક કરે છે
    \end{itemize}

    \textbf{કાર્યિંગ સિદ્ધાંત}:
    \begin{enumerate}
        \item \textbf{ફોટોવોલ્ટેઇક અસર}: સોલર સેલ્સ ફોટોન્સને શોષે છે
        \item \textbf{ઇલેક્ટ્રોન ઉત્તેજના}: ઇલેક્ટ્રોન-હોલ જોડી બનાવે છે
        \item \textbf{કરંટ જનરેશન}: ઇલેક્ટ્રોન પ્રવાહ DC કરંટ બનાવે છે
        \item \textbf{પાવર કંડિશનિંગ}: ઇન્વર્ટર DC ને AC માં રૂપાંતરિત કરે છે
    \end{enumerate}

    \textbf{પ્રકારો}:
    \begin{itemize}
        \item \textbf{ગ્રિડ-કનેક્ટેડ}: યુટિલિટી ગ્રિડ સાથે સમન્વયિત
        \item \textbf{સ્ટેન્ડ-એલોન}: બેટરી બેકઅપ સાથે સ્વતંત્ર સિસ્ટમ
        \item \textbf{હાઇબ્રિડ}: ગ્રિડ-કનેક્ટેડ અને બેટરી સ્ટોરેજનું સંયોજન
    \end{itemize}

    \textbf{ઉપયોગો}: રહેણાંક છત, વ્યાપારી ઇમારતો, યુટિલિટી-સ્કેલ પાવર પ્લાન્ટ, દૂરના વિસ્તારોમાં વીજકરણ.

    \textbf{ફાયદાઓ}: સ્વચ્છ ઊર્જા, ઓછા જાળવણી, મોડ્યુલર ડિઝાઇન, લાંબી આયુષ્ય (25+ વર્ષ).

    \begin{mnemonicbox}PIBCM-PECG (Panel-Inverter-Battery-Controller-Monitor, Photovoltaic-Electron-Current-Grid)\end{mnemonicbox}
\end{solutionbox}

\questionmarks{3}{a}{3}
\textbf{પરંપરાગત અને બિન-પરંપરાગત ઉર્જા સ્ત્રોતોની તુલના કરો.}

\begin{solutionbox}
    \textbf{જવાબ:}

    \begin{answertable}{ઉર્જા સ્ત્રોતોની સરખામણી}
    \begin{tabulary}{\linewidth}{L L L}
        \toprule
        \textbf{પાસું} & \textbf{પરંપરાગત} & \textbf{બિન-પરંપરાગત} \\
        \midrule
        \textbf{ઉપલબ્ધતા} & મર્યાદિત ભંડાર & અમર્યાદિત/નવીકરણીય \\
        \textbf{પર્યાવરણીય અસર} & ઉચ્ચ પ્રદૂષણ & સ્વચ્છ/ન્યૂનતમ અસર \\
        \textbf{ખર્ચ} & શરૂઆતમાં ઓછો & ઝડપથી ઘટી રહ્યો છે \\
        \bottomrule
    \end{tabulary}
    \end{answertable}

    \textbf{પરંપરાગત ઉર્જા સ્ત્રોતો}: કોલસો, તેલ, કુદરતી ગેસ, પરમાણુ શક્તિ - પર્યાવરણીય ચિંતાઓ સાથે મર્યાદિત સંસાધનો.

    \textbf{બિન-પરંપરાગત ઉર્જા સ્ત્રોતો}: સૌર, પવન, હાઇડ્રો, બાયોમાસ - ટકાઉ લાક્ષણિકતાઓ સાથે નવીકરણીય સંસાધનો.

    \textbf{મુખ્ય તફાવતો}: અવક્ષય વિ નવીકરણીય, પ્રદૂષણ વિ સ્વચ્છ, સ્થાપિત વિ ઉભરતી તકનીક.

    \begin{mnemonicbox}AEC (Availability-Environmental-Cost)\end{mnemonicbox}
\end{solutionbox}

\questionmarks{3}{b}{4}
\textbf{કુદરતી પરિભ્રમણ સોલર વોટર હીટરનું કાર્ય સમજાવો.}

\begin{solutionbox}
    \textbf{જવાબ:}

    \begin{center}
    \begin{tikzpicture}[node distance=1.5cm, auto]
        \node (tank) [gtu block, minimum width=3cm] {સોલર ટેન્ક\\(ગરમ પાણી)};
        \node (collector) [gtu block, below=of tank, yshift=-1cm, minimum width=3cm] {સોલર કલેક્ટર\\(ઠંડુ પાણી)};
        
        \draw [gtu arrow] (collector.north east) -- (tank.south east) node[midway, right] {ગરમ પાણી ઉપર ચઢે};
        \draw [gtu arrow] (tank.south west) -- (collector.north west) node[midway, left] {ઠંડુ પાણી નીચે ઉતરે};
        
        \node [text width=3cm, align=center, below of=collector] {થર્મોસિફોન સિદ્ધાંત};
    \end{tikzpicture}
    \end{center}

    \textbf{કુદરતી પરિભ્રમણ સોલર વોટર હીટર} બાહ્ય પંપ વિના પાણીના પરિભ્રમણ માટે થર્મોસિફોન સિદ્ધાંતનો ઉપયોગ કરે છે.

    \textbf{કાર્યકારી સિદ્ધાંત}:
    \begin{itemize}
        \item \textbf{સૌર સંગ્રહ}: કલેક્ટર સૂર્ય કિરણોત્સર્ગ શોષે છે, પાણી ગરમ કરે છે
        \item \textbf{ઘનતા તફાવત}: ગરમ પાણી ઓછું ઘટ્ટ બને છે, આપમેળે ઉપર ચઢે છે
        \item \textbf{પરિભ્રમણ}: ઠંડુ પાણી ટેન્કના તળિયેથી કલેક્ટર તરફ વહે છે
        \item \textbf{સંગ્રહ}: ગરમ પાણી ઇન્સ્યુલેટેડ સ્ટોરેજ ટેન્કમાં જમા થાય છે
    \end{itemize}

    \textbf{ઘટકો}: ફ્લેટ પ્લેટ કલેક્ટર, ઇન્સ્યુલેટેડ સ્ટોરેજ ટેન્ક, કનેક્ટિંગ પાઈપો, સેફ્ટી વાલ્વ.

    \textbf{ફાયદાઓ}: વીજળીની જરૂર નથી, સરળ ડિઝાઇન, ઓછો જાળવણી ખર્ચ, ખર્ચ-અસરકારક.

    \begin{mnemonicbox}SDCS (Solar-Density-Circulation-Storage)\end{mnemonicbox}
\end{solutionbox}

\questionmarks{3}{c}{7}
\textbf{હોરીઝોન્ટલ એક્સિસ વિન્ડ ટર્બાઇનનો કાર્યકારી સિદ્ધાંત સમજાવો.}

\begin{solutionbox}
    \textbf{જવાબ:}

    \begin{center}
    \begin{tikzpicture}[node distance=1.5cm, auto]
        \node (wind) [gtu start] {પવન ઉર્જા};
        \node (blades) [gtu block, right=of wind] {રોટર બ્લેડ};
        \node (shaft) [gtu block, right=of blades] {શાફ્ટ\\પરિભ્રમણ};
        \node (gear) [gtu block, right=of shaft] {ગિયરબોક્સ};
        \node (gen) [gtu block, right=of gear] {જનરેટર};
        \node (power) [gtu block, right=of gen] {વિદ્યુત\\પાવર};
        
        \node (nacelle) [gtu block, below=of gear] {નેસેલ};
        \node (tower) [gtu block, below=of nacelle] {ટાવર};

        \draw [gtu arrow] (wind) -- (blades);
        \draw [gtu arrow] (blades) -- (shaft);
        \draw [gtu arrow] (shaft) -- (gear);
        \draw [gtu arrow] (gear) -- (gen);
        \draw [gtu arrow] (gen) -- (power);
        
        \draw [dashed] (nacelle) -- (blades);
        \draw [dashed] (nacelle) -- (gear);
        \draw [dashed] (nacelle) -- (gen);
        \draw [gtu arrow] (tower) -- (nacelle);
    \end{tikzpicture}
    \end{center}

    \textbf{હોરીઝોન્ટલ એક્સિસ વિન્ડ ટર્બાઇન (HAWT)} એરોડાયનેમિક લિફ્ટ સિદ્ધાંતનો ઉપયોગ કરીને પવનની ગતિ ઉર્જાને વિદ્યુત ઉર્જામાં રૂપાંતરિત કરે છે.

    \textbf{કાર્યકારી સિદ્ધાંત}:
    \begin{enumerate}
        \item \textbf{પવન કેપ્ચર}: રોટર બ્લેડ એરોડાયનેમિક પ્રોફાઇલ સાથે ડિઝાઇન કરવામાં આવ્યા છે
        \item \textbf{લિફ્ટ જનરેશન}: બ્લેડ સપાટી પર દબાણ તફાવત લિફ્ટ બળ બનાવે છે
        \item \textbf{પરિભ્રમણ}: લિફ્ટ બળ રોટરને આડી ધરી પર ફેરવે છે
        \item \textbf{સ્પીડ રૂપાંતરણ}: ગિયરબોક્સ પરિભ્રમણ ગતિ 30-50 rpm થી વધારીને 1500 rpm કરે છે
        \item \textbf{પાવર જનરેશન}: હાઇ-સ્પીડ પરિભ્રમણ ઇલેક્ટ્રિકલ જનરેટર ચલાવે છે
    \end{enumerate}

    \textbf{ઘટકો}:
    \begin{itemize}
        \item \textbf{રોટર એસેમ્બલી}: 2-3 બ્લેડ, હબ, પિચ કંટ્રોલ સિસ્ટમ
        \item \textbf{નેસેલ}: ગિયરબોક્સ, જનરેટર, કંટ્રોલ સિસ્ટમ્સ ઘરે છે
        \item \textbf{ટાવર}: નેસેલને શ્રેષ્ઠ ઊંચાઈ પર સપોર્ટ કરે છે (50-120m)
        \item \textbf{ફાઉન્ડેશન}: માળખાકીય સ્થિરતા માટે કોંક્રિટ બેઝ
    \end{itemize}

    \textbf{કંટ્રોલ સિસ્ટમ્સ}:
    \begin{itemize}
        \item \textbf{યૉ સિસ્ટમ}: ટર્બાઇનને પવનની દિશામાં ગોઠવે છે
        \item \textbf{પિચ કંટ્રોલ}: શ્રેષ્ઠ પવન કેપ્ચર માટે બ્લેડ એંગલ એડજસ્ટ કરે છે
        \item \textbf{બ્રેક સિસ્ટમ}: ઇમરજન્સી સ્ટોપિંગ મિકેનિઝમ
    \end{itemize}

    \textbf{ફાયદાઓ}: ઉચ્ચ કાર્યક્ષમતા (35-45\%), સાબિત તકનીક, સ્કેલની અર્થવ્યવસ્થા.
    \textbf{ગેરફાયદા}: દ્રશ્ય અસર, અવાજ, પક્ષી ટકરાવ, પવન પરિવર્તનશીલતા.

    \textbf{પાવર ગણતરી}: $P = 0.5 \times \rho \times A \times V^3 \times C_p$
    જ્યાં: $\rho$ = હવાની ઘનતા, $A$ = સ્વેપ્ટ એરિયા, $V$ = પવન ગતિ, $C_p$ = પાવર ગુણાંક

    \begin{mnemonicbox}WLRSG-RNTP-YPB (Wind-Lift-Rotation-Speed-Generation, Rotor-Nacelle-Tower-Foundation, Yaw-Pitch-Brake)\end{mnemonicbox}
\end{solutionbox}

\questionmarks{4}{a}{3}
\textbf{ભરતી ઉર્જાના ફાયદા અને ગેરફાયદા લખો.}

\begin{solutionbox}
    \textbf{જવાબ:}

    \begin{answertable}{ભરતી ઉર્જાના ફાયદા અને ગેરફાયદા}
    \begin{tabulary}{\linewidth}{L L}
        \toprule
        \textbf{ફાયદા} & \textbf{ગેરફાયદા} \\
        \midrule
        આગાહી કરી શકાય તેવી ઉર્જા & મર્યાદિત યોગ્ય સ્થાનો \\
        કોઈ ગ્રીનહાઉસ ગેસ ઉત્સર્જન નથી & શરૂઆતનો ઉચ્ચ મૂડી ખર્ચ \\
        લાંબું આયુષ્ય (100+ વર્ષ) & દરિયાઈ જીવન પર પર્યાવરણીય અસર \\
        \bottomrule
    \end{tabulary}
    \end{answertable}

    \textbf{ભરતી ઉર્જા} પૃથ્વી, ચંદ્ર અને સૂર્ય વચ્ચેના ગુરુત્વાકર્ષણ બળોનો ઉપયોગ કરીને વીજળી ઉત્પન્ન કરે છે.

    \textbf{ફાયદા}:
    \begin{itemize}
        \item \textbf{વિશ્વસનીયતા}: અત્યંત અનુમાનિત ભરતી ચક્ર
        \item \textbf{સ્વચ્છ ઉર્જા}: શૂન્ય ઓપરેશનલ ઉત્સર્જન
        \item \textbf{ટકાઉપણું}: ઇન્ફ્રાસ્ટ્રક્ચર દાયકાઓ સુધી ચાલે છે
    \end{itemize}

    \textbf{ગેરફાયદા}:
    \begin{itemize}
        \item \textbf{ભૌગોલિક મર્યાદાઓ}: ચોક્કસ દરિયાકાંઠાની પરિસ્થિતિઓ જરૂરી
        \item \textbf{ઊંચો ખર્ચ}: ઇન્સ્ટોલેશન અને જાળવણી ખર્ચાળ
        \item \textbf{પારિસ્થિતિક અસર}: દરિયાઈ ઇકોસિસ્ટમને અસર કરે છે
    \end{itemize}

    \begin{mnemonicbox}RCD-GHE (Reliable-Clean-Durable, Geographic-High cost-Ecological)\end{mnemonicbox}
\end{solutionbox}

\questionmarks{4}{b}{4}
\textbf{બાયોગેસ પ્લાન્ટનો કાર્યકારી સિદ્ધાંત સમજાવો.}

\begin{solutionbox}
    \textbf{જવાબ:}

    \begin{center}
    \begin{tikzpicture}[node distance=1.5cm, auto]
        \node (input) [gtu start] {કાર્બનિક કચરો\\ઇનપુટ};
        \node (mix) [gtu block, right=of input] {મિક્સિંગ ટેન્ક};
        \node (digest) [gtu block, right=of mix] {ડાયજેસ્ટર ટેન્ક};
        \node (gas) [gtu block, above=of digest] {ગેસ સંગ્રહ};
        \node (slurry) [gtu block, below=of digest] {સ્લરી આઉટપુટ};
        \node (storage) [gtu block, right=of gas] {બાયોગેસ\\સંગ્રહ};
        \node (use) [gtu block, right=of storage] {અંતિમ ઉપયોગ};

        \draw [gtu arrow] (input) -- (mix);
        \draw [gtu arrow] (mix) -- (digest);
        \draw [gtu arrow] (digest) -- (gas);
        \draw [gtu arrow] (digest) -- (slurry);
        \draw [gtu arrow] (gas) -- (storage);
        \draw [gtu arrow] (storage) -- (use);
    \end{tikzpicture}
    \end{center}

    \textbf{બાયોગેસ પ્લાન્ટ} કાર્બનિક કચરાના અજારક વિઘટન દ્વારા મિથેનથી ભરપૂર ગેસ ઉત્પન્ન કરે છે.

    \textbf{કાર્યકારી સિદ્ધાંત}:
    \begin{enumerate}
        \item \textbf{ફીડ તૈયારી}: કાર્બનિક કચરો પાણી સાથે મિશ્ર કરવામાં આવે છે (1:1 ગુણોત્તર)
        \item \textbf{અજારક વિઘટન}: બેક્ટેરિયા ઓક્સિજન મુક્ત વાતાવરણમાં કાર્બનિક પદાર્થોને તોડે છે
        \item \textbf{ગેસ ઉત્પાદન}: મિથેન (50-70\%) અને CO$_2$ (30-40\%) ઉત્પન્ન થાય છે
        \item \textbf{ગેસ સંગ્રહ}: બાયોગેસ ગેસ હોલ્ડર ડોમમાં એકત્રિત થાય છે
    \end{enumerate}

    \textbf{પ્રક્રિયા તબક્કાઓ}:
    \begin{itemize}
        \item \textbf{હાઇડ્રોલિસિસ}: જટિલ કાર્બનિક પદાર્થો સરળ સંયોજનોમાં તૂટી જાય છે
        \item \textbf{એસિડોજેનેસિસ}: કાર્બનિક એસિડ રચના
        \item \textbf{મિથેનોજેનેસિસ}: મિથેનોજેનિક બેક્ટેરિયા દ્વારા મિથેન ઉત્પાદન
    \end{itemize}

    \textbf{શ્રેષ્ઠ સ્થિતિ}: તાપમાન 35-40\textdegree{}C, pH 6.8-7.2, રીટેન્શન સમય 15-30 દિવસ.

    \begin{mnemonicbox}FAGH-HAM (Feed-Anaerobic-Gas-Holder, Hydrolysis-Acidogenesis-Methanogenesis)\end{mnemonicbox}
\end{solutionbox}

\questionmarks{4}{c}{7}
\textbf{ગ્રીન હાઉસ અસર સમજાવો.}

\begin{solutionbox}
    \textbf{જવાબ:}

    \begin{center}
    \begin{tikzpicture}[node distance=1.5cm, auto]
        \node (solar) [gtu start] {સૂર્ય કિરણોત્સર્ગ};
        \node (earth) [gtu block, right=of solar] {પૃથ્વીની સપાટી};
        \node (absorb) [gtu block, right=of earth] {ગરમીનું\\શોષણ};
        \node (ir) [gtu block, below=of absorb] {ઇન્ફ્રારેડ\\રેડિયેશન};
        \node (gas) [gtu block, left=of ir] {ગ્રીનહાઉસ\\વાયુઓ};
        \node (trap) [gtu block, left=of gas] {ગરમી ફસાવવી};
        \node (rerad) [gtu block, below=of trap] {પૃથ્વી પર\\પુનઃવિકિરણ};
        \node (warm) [gtu block, right=of rerad] {વૈશ્વિક તાપમાન\\વધારો};

        \draw [gtu arrow] (solar) -- (earth);
        \draw [gtu arrow] (earth) -- (absorb);
        \draw [gtu arrow] (absorb) -- (ir);
        \draw [gtu arrow] (ir) -- (gas);
        \draw [gtu arrow] (gas) -- (trap);
        \draw [gtu arrow] (trap) -- (rerad);
        \draw [gtu arrow] (rerad) -- (warm);
    \end{tikzpicture}
    \end{center}

    \textbf{ગ્રીનહાઉસ અસર} એ પ્રક્રિયા છે જ્યાં વાતાવરણીય વાયુઓ સૂર્યમાંથી ગરમીને પકડી રાખે છે, પૃથ્વીની સપાટીને સામાન્ય તાપમાન કરતા વધુ ગરમ કરે છે.

    \textbf{કુદરતી ગ્રીનહાઉસ અસર}:
    \begin{itemize}
        \item \textbf{સૂર્ય કિરણોત્સર્ગ}: સૂર્ય ટૂંકા તરંગોનું રેડિયેશન ફેંકે છે (દૃશ્યમાન પ્રકાશ)
        \item \textbf{સપાટી શોષણ}: પૃથ્વી સૌર ઊર્જા શોષે છે, ગરમ થાય છે
        \item \textbf{ગરમી પુનઃઉત્સર્જન}: પૃથ્વી લાંબા તરંગોનું ઇન્ફ્રારેડ રેડિયેશન ફેંકે છે
        \item \textbf{વાયુ શોષણ}: ગ્રીનહાઉસ વાયુઓ ઇન્ફ્રારેડ રેડિયેશન શોષે છે
        \item \textbf{ગરમી જાળવી રાખવી}: ફસાયેલી ગરમી નીચલા વાતાવરણને ગરમ કરે છે
    \end{itemize}

    \textbf{ગ્રીનહાઉસ વાયુઓ અને ફાળો}:
    \begin{itemize}
        \item \textbf{કાર્બન ડાયોક્સાઇડ (CO$_2$)}: 76\% - અશ્મિભૂત ઇંધણ દહન, વનનાશ
        \item \textbf{મિથેન (CH$_4$)}: 16\% - કૃષિ, લેન્ડફિલ્સ, પશુધન
        \item \textbf{નાઈટ્રસ ઓક્સાઇડ (N$_2$O)}: 6\% - ખાતરો, અશ્મિભૂત ઇંધણ દહન
        \item \textbf{ફ્લોરિનેટેડ વાયુઓ}: 2\% - ઔદ્યોગિક પ્રક્રિયાઓ, રેફ્રિજરેશન
    \end{itemize}

    \textbf{વધારેલી ગ્રીનહાઉસ અસર}: માનવ પ્રવૃત્તિઓ ગ્રીનહાઉસ વાયુઓની સાંદ્રતા વધારે છે, ગરમી ફસાવવાની તીવ્રતા વધારે છે.

    \textbf{પરિણામો}:
    \begin{itemize}
        \item \textbf{વૈશ્વિક તાપમાન વધારો}: પૂર્વ-ઔદ્યોગિક સમયથી સરેરાશ 1.1\textdegree{}C વધારો
        \item \textbf{આબોહવા પરિવર્તન}: વરસાદની પદ્ધતિઓમાં ફેરફાર, આત્યંતિક હવામાન ઘટનાઓ
        \item \textbf{સમુદ્ર સપાટીમાં વધારો}: થર્મલ વિસ્તરણ અને બરફના આવરણનું પીગળવું
        \item \textbf{ઇકોસિસ્ટમ વિક્ષેપ}: પ્રજાતિ સ્થાનાંતરણ, કોરલ બ્લીચિંગ, જંગલી આગ
    \end{itemize}

    \textbf{શમન વ્યૂહરચનાઓ}:
    \begin{itemize}
        \item \textbf{નવીકરણીય ઊર્જા}: અશ્મિભૂત ઇંધણ પર નિર્ભરતા ઘટાડવી
        \item \textbf{ઊર્જા કાર્યક્ષમતા}: ટેકનોલોજી અને પદ્ધતિઓમાં સુધારો
        \item \textbf{કાર્બન સિક્વેસ્ટ્રેશન}: જંગલ પુનઃસ્થાપન, કાર્બન કેપ્ચર સ્ટોરેજ
        \item \textbf{આંતરરાષ્ટ્રીય સહકાર}: પેરિસ કરાર, ઉત્સર્જન ઘટાડવાના લક્ષ્યો
    \end{itemize}

    \begin{mnemonicbox}SSAHR-CMNO-GTSE-RECC (Solar-Surface-Absorption-Heat-Radiation, CO2-Methane-Nitrous-Other, Global-Temperature-Sea-Ecosystem, Renewable-Efficiency-Carbon-Cooperation)\end{mnemonicbox}
\end{solutionbox}

\questionmarks{4}{a}{3}
\textbf{આબોહવા પરિવર્તન એટલે શું?}

\begin{solutionbox}
    \textbf{જવાબ:}

    \begin{answertable}{આબોહવા પરિવર્તનના સૂચકો}
    \begin{tabulary}{\linewidth}{L L L}
        \toprule
        \textbf{સૂચક} & \textbf{ફેરફાર} & \textbf{પુરાવા} \\
        \midrule
        \textbf{તાપમાન} & 1880 થી +1.1\textdegree{}C & વૈશ્વિક તાપમાન રેકોર્ડ \\
        \textbf{સમુદ્ર સપાટી} & 1900 થી +21 cm & સેટેલાઇટ માપન \\
        \textbf{આર્કટિક બરફ} & દાયકા દીઠ -13\% & સેટેલાઇટ ઇમેજરી \\
        \bottomrule
    \end{tabulary}
    \end{answertable}

    \textbf{આબોહવા પરિવર્તન} એ વૈશ્વિક તાપમાન અને હવામાન પદ્ધતિઓમાં લાંબા ગાળાના ફેરફારો છે, જે મુખ્યત્વે 20મી સદીના મધ્યથી માનવ પ્રવૃત્તિઓને કારણે થાય છે.

    \textbf{મુખ્ય લાક્ષણિકતાઓ}:
    \begin{itemize}
        \item \textbf{તાપમાન વધારો}: વૈશ્વિક સરેરાશ તાપમાનમાં વધારો
        \item \textbf{હવામાન ચરમસીમા}: વધુ વારંવાર વાવાઝોડા, દુષ્કાળ, પૂર
        \item \textbf{ઇકોસિસ્ટમ ફેરફારો}: પ્રજાતિ સ્થાનાંતરણ, આવાસ નુકશાન
    \end{itemize}

    \textbf{પ્રાથમિક કારણ}: અશ્મિભૂત ઇંધણ દહન, વનનાશ, ઔદ્યોગિક પ્રક્રિયાઓમાંથી ગ્રીનહાઉસ વાયુ ઉત્સર્જનમાં વધારો.

    \begin{mnemonicbox}TSE (Temperature-Sea level-Ecosystem)\end{mnemonicbox}
\end{solutionbox}

\questionmarks{4}{b}{4}
\textbf{ગ્લોબલ વોર્મિંગને નિયંત્રિત કરવાના કેટલાક પગલાં લખો.}

\begin{solutionbox}
    \textbf{જવાબ:}

    \begin{answertable}{ગ્લોબલ વોર્મિંગ નિયંત્રણ પગલાં}
    \begin{tabulary}{\linewidth}{L L L}
        \toprule
        \textbf{શ્રેણી} & \textbf{પગલાં} & \textbf{અસર} \\
        \midrule
        \textbf{ઊર્જા} & નવીકરણીય સ્રોતો, કાર્યક્ષમતા & CO$_2$ ઉત્સર્જન ઘટાડો \\
        \textbf{પરિવહન} & ઇલેક્ટ્રિક વાહનો, જાહેર પરિવહન & ઇંધણ વપરાશ ઓછો કરવો \\
        \textbf{ઉદ્યોગ} & સ્વચ્છ ટેકનોલોજી, કાર્બન કેપ્ચર & ઉત્સર્જન ઘટાડો \\
        \textbf{વ્યક્તિગત} & ઊર્જા સંરક્ષણ, જીવનશૈલી ફેરફારો & સંચિત અસર \\
        \bottomrule
    \end{tabulary}
    \end{answertable}

    \textbf{નિયંત્રણ પગલાં}:

    \textbf{સરકારી સ્તર}:
    \begin{itemize}
        \item \textbf{નીતિ માળખા}: કાર્બન ભાવો, ઉત્સર્જન ધોરણો
        \item \textbf{નવીકરણીય ઊર્જા}: સૌર, પવન ઉર્જા પ્રોત્સાહન
        \item \textbf{જાહેર પરિવહન}: માસ ટ્રાન્ઝિટ સિસ્ટમ વિકાસ
    \end{itemize}

    \textbf{ઔદ્યોગિક સ્તર}:
    \begin{itemize}
        \item \textbf{સ્વચ્છ ટેકનોલોજી}: કાર્યક્ષમ પ્રક્રિયાઓ, કચરો ઘટાડો
        \item \textbf{કાર્બન કેપ્ચર}: સંગ્રહ અને ઉપયોગ ટેકનોલોજી
        \item \textbf{ટકાઉ પદ્ધતિઓ}: ગ્રીન મેન્યુફેક્ચરિંગ, પરિપત્ર અર્થતંત્ર
    \end{itemize}

    \textbf{વ્યક્તિગત સ્તર}:
    \begin{itemize}
        \item \textbf{ઊર્જા સંરક્ષણ}: LED લાઇટ્સ, કાર્યક્ષમ ઉપકરણો
        \item \textbf{પરિવહન}: ચાલવું, સાયકલિંગ, કારપૂલિંગ
        \item \textbf{જીવનશૈલી ફેરફારો}: ઓછો વપરાશ, રિસાયક્લિંગ
    \end{itemize}

    \begin{mnemonicbox}PRT-CCS-ECL (Policy-Renewable-Transport, Carbon-Clean-Sustainable, Energy-Communication-Lifestyle)\end{mnemonicbox}
\end{solutionbox}

\questionmarks{4}{c}{7}
\textbf{વૈશ્વિક સ્તરે આબોહવા પરિવર્તનને ઘટાડવા માટે કયા મહત્વપૂર્ણ કરારો છે?}

\begin{solutionbox}
    \textbf{જવાબ:}

    \begin{answertable}{મુખ્ય આબોહવા કરારો}
    \begin{tabulary}{\linewidth}{L L L}
        \toprule
        \textbf{કરાર} & \textbf{વર્ષ} & \textbf{મુખ્ય લક્ષણો} \\
        \midrule
        \textbf{UNFCCC} & 1992 & ફ્રેમવર્ક સંમેલન \\
        \textbf{ક્યોટો પ્રોટોકોલ} & 1997 & બંધનકર્તા ઉત્સર્જન લક્ષ્યો \\
        \textbf{પેરિસ કરાર} & 2015 & વૈશ્વિક તાપમાન મર્યાદા \\
        \bottomrule
    \end{tabulary}
    \end{answertable}

    \textbf{મહત્વપૂર્ણ વૈશ્વિક આબોહવા કરારો}:

    \textbf{1. યુનાઇટેડ નેશન્સ ફ્રેમવર્ક કન્વેન્શન ઓન ક્લાઇમેટ ચેન્જ (UNFCCC) - 1992}:
    \begin{itemize}
        \item \textbf{ઉદ્દેશ્ય}: ગ્રીનહાઉસ વાયુની સાંદ્રતા સ્થિર કરવી
        \item \textbf{સિદ્ધાંતો}: સામાન્ય પરંતુ વિભિન્ન જવાબદારીઓ
        \item \textbf{ફ્રેમવર્ક}: ભવિષ્યની આબોહવા વાટાઘાટો માટે પાયો
    \end{itemize}

    \textbf{2. ક્યોટો પ્રોટોકોલ - 1997}:
    \begin{itemize}
        \item \textbf{બંધનકર્તા લક્ષ્યો}: વિકસિત દેશો ઉત્સર્જનમાં 5.2\% ઘટાડો (1990 સ્તર)
        \item \textbf{લવચીક મિકેનિઝમ્સ}: ઉત્સર્જન ટ્રેડિંગ, સ્વચ્છ વિકાસ મિકેનિઝમ
        \item \textbf{પ્રતિબદ્ધતા સમયગાળો}: પ્રથમ (2008-2012), બીજો (2013-2020)
    \end{itemize}

    \textbf{3. પેરિસ કરાર - 2015}:
    \begin{itemize}
        \item \textbf{તાપમાન લક્ષ્ય}: ગ્લોબલ વોર્મિંગને 2\textdegree{}C થી નીચે, પ્રાધાન્ય 1.5\textdegree{}C સુધી મર્યાદિત કરવું
        \item \textbf{રાષ્ટ્રીય નિર્ધારિત યોગદાન (NDCs)}: દેશો પોતાના લક્ષ્યો નક્કી કરે છે
        \item \textbf{સમીક્ષા મિકેનિઝમ}: પાંચ વર્ષનું મૂલ્યાંકન અને સુધારણા ચક્ર
        \item \textbf{આબોહવા ધિરાણ}: વિકાસશીલ દેશો માટે વાર્ષિક \$100 અબજ
    \end{itemize}

    \textbf{4. અન્ય મહત્વપૂર્ણ કરારો}:
    \begin{itemize}
        \item \textbf{મોન્ટ્રીયલ પ્રોટોકોલ (1987)}: ઓઝોન સ્તર સંરક્ષણ, પરોક્ષ આબોહવા લાભો
        \item \textbf{કોપનહેગન એકોર્ડ (2009)}: ઉત્સર્જન ઘટાડવા પર રાજકીય કરાર
        \item \textbf{દોહા સુધારો (2012)}: ક્યોટો પ્રોટોકોલ પ્રતિબદ્ધતાઓ વિસ્તૃત
    \end{itemize}

    \textbf{અમલીકરણ પડકારો}:
    \begin{itemize}
        \item \textbf{પાલન}: સ્વૈચ્છિક વિ ફરજિયાત પ્રતિબદ્ધતાઓ
        \item \textbf{ધિરાણ}: શમન અને અનુકૂલન માટે પર્યાપ્ત ભંડોળ
        \item \textbf{ટેકનોલોજી સ્થાનાંતરણ}: વિકાસશીલ દેશો માટે સ્વચ્છ ટેકનોલોજી એક્સેસ
        \item \textbf{મોનિટરિંગ}: પારદર્શક રિપોર્ટિંગ અને ચકાસણી સિસ્ટમો
    \end{itemize}

    \textbf{તાજેતરના વિકાસ}:
    \begin{itemize}
        \item \textbf{આર્ટિકલ 6 નિયમો}: પેરિસ કરાર હેઠળ આંતરરાષ્ટ્રીય કાર્બન બજારો
        \item \textbf{નુકસાન અને વળતર}: આબોહવા-સંવેદનશીલ દેશો માટે સપોર્ટ
        \item \textbf{નેટ-ઝીરો પ્રતિબદ્ધતાઓ}: દેશો કાર્બન તટસ્થતાની પ્રતિજ્ઞા લેતા
    \end{itemize}

    \begin{mnemonicbox}UKPOM-CDOG-TFMC (UNFCCC-Kyoto-Paris-Other-Montreal, Copenhagen-Doha-Other-Goals, Technology-Finance-Monitoring-Commitments)\end{mnemonicbox}
\end{solutionbox}

\questionmarks{5}{a}{3}
\textbf{ઓઝોન સ્તરના અવક્ષયની અસરો સમજાવો.}

\begin{solutionbox}
    \textbf{જવાબ:}

    \begin{answertable}{ઓઝોન અવક્ષય અસરો}
    \begin{tabulary}{\linewidth}{L L L}
        \toprule
        \textbf{અસરનું ક્ષેત્ર} & \textbf{અસર} & \textbf{પરિણામ} \\
        \midrule
        \textbf{માનવ આરોગ્ય} & વધેલ UV-B રેડિયેશન & સ્કિન કેન્સર, મોતિયા \\
        \textbf{પર્યાવરણ} & ઇકોસિસ્ટમ વિક્ષેપ & દરિયાઈ ખાદ્ય શૃંખલા નુકસાન \\
        \textbf{કૃષિ} & પાક નુકસાન & ઓછું ખાદ્ય ઉત્પાદન \\
        \bottomrule
    \end{tabulary}
    \end{answertable}

    \textbf{ઓઝોન સ્તર અવક્ષય} પૃથ્વીની સપાટી પર અલ્ટ્રાવાયોલેટ-બી (UV-B) રેડિયેશનમાં વધારો કરે છે.

    \textbf{અસરો}:
    \begin{itemize}
        \item \textbf{માનવ આરોગ્ય}: ઉચ્ચ ત્વચા કેન્સર દર, આંખ નુકસાન, રોગપ્રતિકારક શક્તિ દબાવવી
        \item \textbf{દરિયાઈ ઇકોસિસ્ટમ્સ}: ફાયટોપ્લાન્કટન ઘટાડો દરિયાઈ ખાદ્ય શૃંખલાને અસર કરે છે
        \item \textbf{કૃષિ અસર}: પાક ઉપજમાં ઘટાડો, છોડ વૃદ્ધિ અવરોધ
    \end{itemize}

    \textbf{કારણ}: ક્લોરોફ્લોરોકાર્બન્સ (CFCs) સ્ટ્રેટોસ્ફિયરમાં ઓઝોન પરમાણુઓનો નાશ કરે છે.

    \begin{mnemonicbox}HMA (Human-Marine-Agricultural)\end{mnemonicbox}
\end{solutionbox}

\questionmarks{5}{b}{4}
\textbf{ગ્રીનહાઉસ વાયુઓ પર ટૂંકી નોંધ લખો.}

\begin{solutionbox}
    \textbf{જવાબ:}

    \begin{answertable}{મુખ્ય ગ્રીનહાઉસ વાયુઓ}
    \begin{tabulary}{\linewidth}{L L L}
        \toprule
        \textbf{ગેસ} & \textbf{સ્રોતો} & \textbf{વૈશ્વિક વોર્મિંગ ક્ષમતા} \\
        \midrule
        \textbf{CO$_2$} & અશ્મિભૂત ઇંધણ, વનનાશ & 1 (સંદર્ભ) \\
        \textbf{CH$_4$} & કૃષિ, લેન્ડફિલ્સ & 25 ગણું CO$_2$ \\
        \textbf{N$_2$O} & ખાતરો, દહન & 298 ગણું CO$_2$ \\
        \textbf{F-વાયુઓ} & ઔદ્યોગિક પ્રક્રિયાઓ & 1,000-20,000 ગણું CO$_2$ \\
        \bottomrule
    \end{tabulary}
    \end{answertable}

    \textbf{ગ્રીનહાઉસ વાયુઓ} વાતાવરણીય સંયોજનો છે જે પૃથ્વીની સપાટીથી વિકિરણ ગરમીને પકડી રાખે છે.

    \textbf{મુખ્ય ગ્રીનહાઉસ વાયુઓ}:
    \begin{itemize}
        \item \textbf{કાર્બન ડાયોક્સાઇડ (CO$_2$)}: સૌથી વધુ વિપુલ, અશ્મિભૂત ઇંધણ દહનથી
        \item \textbf{મિથેન (CH$_4$)}: શક્તિશાળી પરંતુ ટૂંકા આયુષ્યવાળું, કૃષિમાંથી
        \item \textbf{નાઈટ્રસ ઓક્સાઇડ (N$_2$O)}: લાંબું આયુષ્ય, ખાતરો અને ઉદ્યોગમાંથી
        \item \textbf{ફ્લોરિનેટેડ વાયુઓ}: ખૂબ શક્તિશાળી, રેફ્રિજરેશન industrialદ્યોગિક ઉપયોગોમાંથી
    \end{itemize}

    \textbf{ગુણધર્મો}: ઇન્ફ્રારેડ રેડિયેશન શોષે છે, દૃશ્યમાન પ્રકાશ માટે પારદર્શક, વિવિધ વાતાવરણીય આયુષ્ય.

    \textbf{વૈશ્વિક વોર્મિંગ ક્ષમતા}: ચોક્કસ સમયગાળામાં CO$_2$ ની સાપેક્ષ ગરમી પકડવાની ક્ષમતા માપે છે.

    \begin{mnemonicbox}CMNF (Carbon dioxide-Methane-Nitrous oxide-Fluorinated gases)\end{mnemonicbox}
\end{solutionbox}

\questionmarks{5}{c}{7}
\textbf{5R નો ખ્યાલ સમજાવો.}

\begin{solutionbox}
    \textbf{જવાબ:}

    \begin{center}
    \begin{tikzpicture}[node distance=1.5cm, auto]
        \node (a) [gtu start] {5R ખ્યાલ};
        \node (b) [gtu block, below left=of a] {ના પાડવી (Refuse)};
        \node (c) [gtu block, left=of b] {ઘટાડો (Reduce)};
        \node (d) [gtu block, right=of b] {પુનઃઉપયોગ (Reuse)};
        \node (e) [gtu block, right=of d] {પુનઃહેતુ (Repurpose)};
        \node (f) [gtu block, right=of e] {રિસાયકલ (Recycle)};

        % Explanations
        \node (b1) [gtu block, below=of b] {બિનજરૂરી\\ટાળો};
        \node (c1) [gtu block, below=of c] {વપરાશ\\ઘટાડો};
        \node (d1) [gtu block, below=of d] {બહુવિધ વખત\\વાપરો};
        \node (e1) [gtu block, below=of e] {નવો ઉપયોગ\\શોધો};
        \node (f1) [gtu block, below=of f] {નવા ઉત્પાદનોમાં\\પ્રક્રિયા};

        \draw [gtu arrow] (a) -- (b);
        \draw [gtu arrow] (a) -- (c);
        \draw [gtu arrow] (a) -- (d);
        \draw [gtu arrow] (a) -- (e);
        \draw [gtu arrow] (a) -- (f);
        
        \draw [gtu arrow] (b) -- (b1);
        \draw [gtu arrow] (c) -- (c1);
        \draw [gtu arrow] (d) -- (d1);
        \draw [gtu arrow] (e) -- (e1);
        \draw [gtu arrow] (f) -- (f1);
    \end{tikzpicture}
    \end{center}

    \textbf{5R ખ્યાલ} કચરા વ્યવસ્થાપન હાયરાર્કી છે જે કચરા નિવારણ અને સંસાધન સંરક્ષણને પ્રાથમિકતા આપે છે.

    \textbf{પ્રાથમિકતાના ક્રમમાં પાંચ R}:

    \textbf{1. ના પાડવી (Refuse)}:
    \begin{itemize}
        \item \textbf{વ્યાખ્યા}: બિનજરૂરી વસ્તુઓ સ્વીકારવાનું ટાળો
        \item \textbf{ઉદાહરણો}: સિંગલ-યુઝ પ્લાસ્ટિક, પ્રમોશનલ ફ્રીબીઝ, અતિશય પેકેજિંગ
        \item \textbf{અસર}: સ્રોત પર કચરો ઉત્પાદન અટકાવે છે
    \end{itemize}

    \textbf{2. ઘટાડો (Reduce)}:
    \begin{itemize}
        \item \textbf{વ્યાખ્યા}: વપરાશ અને કચરા ઉત્પાદન ઓછું કરવું
        \item \textbf{ઉદાહરણો}: ફક્ત જરૂરી વસ્તુઓ ખરીદો, ટકાઉ ઉત્પાદનો પસંદ કરો, ઊર્જા સંરક્ષણ
        \item \textbf{અસર}: સંસાધન નિષ્કર્ષણ અને કચરાનું પ્રમાણ ઘટાડે છે
    \end{itemize}

    \textbf{3. પુનઃઉપયોગ (Reuse)}:
    \begin{itemize}
        \item \textbf{વ્યાખ્યા}: વસ્તુઓનો તેમના મૂળ સ્વરૂપમાં બહુવિધ વખત ઉપયોગ કરો
        \item \textbf{ઉદાહરણો}: સંગ્રહ માટે કાચની બરણીઓ, કપડાં દાન, ફર્નિચર પુનઃહેતુ
        \item \textbf{અસર}: ઉત્પાદન આયુષ્ય વધારે છે, બદલવાની જરૂરિયાતો ઘટાડે છે
    \end{itemize}

    \textbf{4. પુનઃહેતુ (Repurpose)}:
    \begin{itemize}
        \item \textbf{વ્યાખ્યા}: ફેંકી દેવાને બદલે વસ્તુઓ માટે નવી એપ્લિકેશનો શોધો
        \item \textbf{ઉદાહરણો}: ટાયર પ્લાન્ટર્સ, બોટલ વાઝ, કાર્ડબોર્ડ આયોજકો
        \item \textbf{અસર}: સર્જનાત્મક કચરો વાળવો, કલાત્મક મૂલ્ય ઉમેરો
    \end{itemize}

    \textbf{5. રિસાયકલ (Recycle)}:
    \begin{itemize}
        \item \textbf{વ્યાખ્યા}: કચરા સામગ્રીને નવા ઉત્પાદનોમાં પ્રક્રિયા કરવી
        \item \textbf{ઉદાહરણો}: કાગળ રિસાયક્લિંગ, ધાતુ પુનઃપ્રાપ્તિ, પ્લાસ્ટિક પુનઃપ્રક્રિયા
        \item \textbf{અસર}: સંસાધન પુનઃપ્રાપ્તિ, લેન્ડફિલ બોજ ઘટાડો
    \end{itemize}

    \textbf{5R અભિગમના ફાયદા}:
    \begin{itemize}
        \item \textbf{પર્યાવરણીય}: પ્રદૂષણ ઘટાડો, સંસાધન સંરક્ષણ, ઇકોસિસ્ટમ સંરક્ષણ
        \item \textbf{આર્થિક}: ખર્ચ બચત, રિસાયક્લિંગ ઉદ્યોગમાં રોજગાર સર્જન
        \item \textbf{સામાજિક}: સમુદાય જાગૃતિ, ટકાઉ જીવનશૈલી પ્રોત્સાહન
    \end{itemize}

    \textbf{અમલીકરણ હાયરાર્કી}: પહેલા ના પાડવા અને ઘટાડવા પર ધ્યાન કેન્દ્રિત કરો (નિવારણ), પછી પુનઃઉપયોગ અને પુનઃહેતુ (કચરો વાળવો), છેલ્લે રિસાયકલ (કચરા પ્રક્રિયા).

    \textbf{પડકારો}: વર્તણૂકીય ફેરફાર જરૂરિયાતો, ઇન્ફ્રાસ્ટ્રક્ચર વિકાસ, આર્થિક પ્રોત્સાહનો ગોઠવણી.

    \begin{mnemonicbox}Real Recycling Requires Refusing Rubbish (Refuse-Reduce-Reuse-Repurpose-Recycle)\end{mnemonicbox}
\end{solutionbox}

\questionmarks{5}{a}{3}
\textbf{વર્ષ 1972ના વન્ય જીવન સુરક્ષા કાયદાના મુખ્ય લક્ષણો જણાવો.}

\begin{solutionbox}
    \textbf{જવાબ:}

    \begin{answertable}{વન્ય જીવન સુરક્ષા કાયદા 1972ના લક્ષણો}
    \begin{tabulary}{\linewidth}{L L L}
        \toprule
        \textbf{લક્ષણ} & \textbf{વર્ણન} & \textbf{દંડ} \\
        \midrule
        \textbf{સુરક્ષિત પ્રજાતિઓ} & અનુસૂચિત પ્રાણીઓ/છોડ & દંડ + જેલ \\
        \textbf{શિકાર પ્રતિબંધ} & શિકાર પર પ્રતિબંધ & 7 વર્ષ સુધી જેલ \\
        \textbf{વેપાર નિયમન} & વન્યજીવન ઉત્પાદન વેપાર નિયંત્રણ & જપ્તી + દંડ \\
        \bottomrule
    \end{tabulary}
    \end{answertable}

    \textbf{વન્ય જીવન સુરક્ષા કાયદો, 1972} ભારતમાં વન્યજીવનના સંરક્ષણ માટે કાયદાકીય માળખું પૂરું પાડે છે.

    \textbf{મુખ્ય લક્ષણો}:
    \begin{itemize}
        \item \textbf{પ્રજાતિ સંરક્ષણ}: સુરક્ષા સ્તર દ્વારા પ્રજાતિઓને વર્ગીકૃત કરતા છ અનુસૂચિઓ
        \item \textbf{શિકાર પ્રતિબંધ}: સુરક્ષિત પ્રજાતિઓના શિકાર પર સંપૂર્ણ પ્રતિબંધ
        \item \textbf{આવાસ સંરક્ષણ}: સુરક્ષિત વિસ્તારોનું હોદ્દો અને સંચાલન
        \item \textbf{વેપાર નિયંત્રણ}: વન્યજીવન ઉત્પાદન વાણિજ્યનું નિયમન
    \end{itemize}

    \textbf{અમલીકરણ}: વન્યજીવન ગુના નિયંત્રણ બ્યુરો, વન વિભાગો, વન્યજીવન ગુનાઓ માટે વિશેષ અદાલતો.

    \textbf{સુધારા}: નવી પ્રજાતિઓ શામેલ કરવા અને જોગવાઈઓને મજબૂત કરવા નિયમિત અપડેટ્સ.

    \begin{mnemonicbox}SHTE (Species-Hunting-Trade-Enforcement)\end{mnemonicbox}
\end{solutionbox}

\questionmarks{5}{b}{4}
\textbf{ભારતની પર્યાવરણીય નીતિઓ કઈ કઈ છે?}

\begin{solutionbox}
    \textbf{જવાબ:}

    \begin{answertable}{ભારતમાં મુખ્ય પર્યાવરણીય નીતિઓ}
    \begin{tabulary}{\linewidth}{L L L}
        \toprule
        \textbf{નીતિ} & \textbf{વર્ષ} & \textbf{ફોકસ વિસ્તાર} \\
        \midrule
        \textbf{રાષ્ટ્રીય પર્યાવરણ નીતિ} & 2006 & વ્યાપક માળખું \\
        \textbf{રાષ્ટ્રીય જળ નીતિ} & 2012 & જળ સંસાધન વ્યવસ્થાપન \\
        \textbf{રાષ્ટ્રીય વન નીતિ} & 1988 & વન સંરક્ષણ \\
        \textbf{આબોહવા પરિવર્તન પર રાષ્ટ્રીય કાર્ય યોજના} & 2008 & આબોહવા પરિવર્તન શમન \\
        \bottomrule
    \end{tabulary}
    \end{answertable}

    \textbf{મુખ્ય પર્યાવરણીય નીતિઓ}:

    \textbf{રાષ્ટ્રીય પર્યાવરણ નીતિ (2006)}:
    \begin{itemize}
        \item \textbf{ઉદ્દેશ્ય}: પર્યાવરણ સુરક્ષા સાથે ટકાઉ વિકાસ
        \item \textbf{સિદ્ધાંતો}: પ્રદૂષક ચૂકવે છે, સાવચેતીભર્યો અભિગમ
        \item \textbf{અમલીકરણ}: ક્ષેત્રોમાં એકીકરણ
    \end{itemize}

    \textbf{ક્ષેત્રીય નીતિઓ}:
    \begin{itemize}
        \item \textbf{રાષ્ટ્રીય જળ નીતિ}: સંકલિત જળ સંસાધન વ્યવસ્થાપન
        \item \textbf{રાષ્ટ્રીય વન નીતિ}: 33\% વન કવર લક્ષ્ય
        \item \textbf{રાષ્ટ્રીય સૌર મિશન}: નવીકરણીય ઊર્જા પ્રોત્સાહન
        \item \textbf{કચરા વ્યવસ્થાપન નિયમો}: ઘન કચરો, ઈ-કચરો, પ્લાસ્ટિક કચરા વ્યવસ્થાપન
    \end{itemize}

    \textbf{નિયમનકારી માળખું}: પર્યાવરણ સુરક્ષા કાયદો, જળ કાયદો, હવા કાયદો, વન સંરક્ષણ કાયદો.

    \begin{mnemonicbox}NWFS (National-Water-Forest-Solar)\end{mnemonicbox}
\end{solutionbox}

\questionmarks{5}{c}{7}
\textbf{વરસાદી પાણીના સંગ્રહ વિશે વિગતવાર સમજાવો.}

\begin{solutionbox}
    \textbf{જવાબ:}

    \begin{center}
    \begin{tikzpicture}[node distance=1.5cm, auto]
        \node (rain) [gtu start] {વરસાદ};
        \node (catch) [gtu block, right=of rain] {કેચમેન્ટ વિસ્તાર};
        \node (coll) [gtu block, right=of catch] {સંગ્રહ સિસ્ટમ};
        \node (flush) [gtu block, right=of coll] {પ્રથમ ફ્લશ\\ડાયવર્ટર};
        \node (filt) [gtu block, below=of flush] {ફિલ્ટરેશન};
        \node (tank) [gtu block, left=of filt] {સ્ટોરેજ ટેન્ક};
        \node (dist) [gtu block, left=of tank] {વિતરણ};
        
        \node (recharge) [gtu block, below=of coll] {રિચાર્જ ખાડો};
        \node (ground) [gtu block, below=of recharge] {ભૂગર્ભજળ};

        \draw [gtu arrow] (rain) -- (catch);
        \draw [gtu arrow] (catch) -- (coll);
        \draw [gtu arrow] (coll) -- (flush);
        \draw [gtu arrow] (flush) -- (filt);
        \draw [gtu arrow] (filt) -- (tank);
        \draw [gtu arrow] (tank) -- (dist);
        
        \draw [gtu arrow] (coll) -- (recharge);
        \draw [gtu arrow] (recharge) -- (ground);
    \end{tikzpicture}
    \end{center}

    \textbf{વરસાદી પાણીનો સંગ્રહ} ફાયદાકારક હેતુઓ માટે વરસાદી પાણીનો સંગ્રહ, સંગ્રહ અને ઉપયોગ છે.

    \textbf{વરસાદી પાણી સંગ્રહ સિસ્ટમના ઘટકો}:

    \textbf{1. કેચમેન્ટ વિસ્તાર}:
    \begin{itemize}
        \item \textbf{કાર્ય}: વરસાદ સંગ્રહ માટે સપાટી (છત, ખુલ્લા વિસ્તારો)
        \item \textbf{સામગ્રી}: સ્વચ્છ, બિન-ઝેરી હોવી જોઈએ (એસ્બેસ્ટોસ, લીડ-પેઇન્ટેડ સપાટી ટાળો)
        \item \textbf{ગણતરી}: સંગ્રહ = કેચમેન્ટ વિસ્તાર $\times$ વરસાદ $\times$ રનઓફ ગુણાંક
    \end{itemize}

    \textbf{2. સંગ્રહ અને વહન સિસ્ટમ}:
    \begin{itemize}
        \item \textbf{ગટર (Gutters)}: કેચમેન્ટ સપાટી પરથી ચેનલ પાણી
        \item \textbf{ડાઉનસ્પાઉટ્સ}: ગટરમાંથી પાણી લઈ જતી ઊભી પાઈપો
        \item \textbf{પરિવહન}: વિવિધ ઘટકોને જોડતી પાઈપો
    \end{itemize}

    \textbf{3. પ્રથમ ફ્લશ ડાયવર્ટર}:
    \begin{itemize}
        \item \textbf{હેતુ}: કાટમાળ ધરાવતું પ્રારંભિક ગંદુ પાણી દૂર કરે છે
        \item \textbf{પ્રકારો}: મેન્યુઅલ વાલ્વ, ઓટોમેટિક ડાયવર્ટર, ફ્લોટિંગ બોલ સિસ્ટમ
        \item \textbf{ક્ષમતા}: સામાન્ય રીતે 100 ચો.મી. છત વિસ્તાર દીઠ 10-15 લિટર
    \end{itemize}

    \textbf{4. ફિલ્ટરેશન સિસ્ટમ}:
    \begin{itemize}
        \item \textbf{બરછટ ફિલ્ટર (Coarse filter)}: પાંદડા, કાટમાળ દૂર કરે છે (મેશ સ્ક્રીન)
        \item \textbf{ફાઇન ફિલ્ટર (Fine filter)}: રેતી, કાંકરી, સક્રિય કાર્બન
        \item \textbf{ધીમું રેતી ફિલ્ટર}: પીવાના પાણી માટે જૈવિક સારવાર
    \end{itemize}

    \textbf{5. સંગ્રહ સિસ્ટમ}:
    \begin{itemize}
        \item \textbf{સપાટી સંગ્રહ}: જમીન ઉપર ટેન્કો, જળાશયો
        \item \textbf{ભૂગર્ભ સંગ્રહ}: જમીન નીચે સંપ, ટાંકીઓ
        \item \textbf{સામગ્રી}: ફેરોસીમેન્ટ, પ્લાસ્ટિક, કોંક્રિટ, ફાઇબરગ્લાસ
    \end{itemize}

    \textbf{વરસાદી પાણી સંગ્રહના પ્રકારો}:

    \textbf{A. રૂફટોપ હાર્વેસ્ટિંગ}:
    \begin{itemize}
        \item \textbf{સીધો સંગ્રહ}: તાત્કાલિક ઉપયોગ માટે ટેન્કોમાં સંગ્રહિત વરસાદી પાણી
        \item \textbf{પરોક્ષ રિચાર્જ}: ભૂગર્ભજળ રિચાર્જ કરવા માટે સીધું પાણી
    \end{itemize}

    \textbf{B. સપાટી જળ સંગ્રહ}:
    \begin{itemize}
        \item \textbf{ચેક ડેમ}: પ્રવાહોમાં નાના અવરોધો
        \item \textbf{પરકોલેશન ટેન્ક}: કૃત્રિમ રિચાર્જ માળખાઓ
        \item \textbf{કોન્ટૂર બંડિંગ}: પાણી સંગ્રહ સાથે જમીન સંરક્ષણ
    \end{itemize}

    \textbf{ફાયદાઓ}:
    \begin{itemize}
        \item \textbf{જળ સુરક્ષા}: બાહ્ય જળ સ્રોતો પર નિર્ભરતા ઘટાડે છે
        \item \textbf{ભૂગર્ભજળ રિચાર્જ}: જળ સ્તરનો ઘટાડો અટકાવે છે
        \item \textbf{પૂર નિયંત્રણ}: સપાટીનો રનઓફ અને શહેરી પૂર ઘટાડે છે
        \item \textbf{ગુણવત્તા સુધારણા}: સામાન્ય રીતે પ્રદૂષિત વિસ્તારોમાં ભૂગર્ભજળ કરતા સારું
        \item \textbf{ખર્ચ-અસરકારક}: પાણી પુરવઠા યોજનાઓ કરતા ઓછું
        \item \textbf{ઊર્જા બચત}: પમ્પિંગ જરૂરિયાતો ઘટાડે છે
    \end{itemize}

    \textbf{ડિઝાઇન વિચારણાઓ}:
    \begin{itemize}
        \item \textbf{વરસાદની પદ્ધતિ}: મોસમી વિતરણ, તીવ્રતા
        \item \textbf{પાણીની માંગ}: ઘરની જરૂરિયાતો, વપરાશની પદ્ધતિઓ
        \item \textbf{સંગ્રહ ક્ષમતા}: સૂકા સમયગાળાના આધારે
        \item \textbf{ગુણવત્તા જરૂરિયાતો}: પીવાલાયક વિ બિન-પીવાલાયક ઉપયોગ
        \item \textbf{સાઇટ શરતો}: જગ્યા ઉપલબ્ધતા, જમીનની અભેદ્યતા
    \end{itemize}

    \textbf{જાળવણી જરૂરિયાતો}:
    \begin{itemize}
        \item \textbf{નિયમિત સફાઈ}: ગટરો, ફિલ્ટર, સ્ટોરેજ ટેન્ક
        \item \textbf{છત જાળવણી}: દૂષિત સ્રોતો અટકાવો
        \item \textbf{સિસ્ટમ નિરીક્ષણ}: લીક, અવરોધો માટે તપાસો
        \item \textbf{પાણીની ગુણવત્તા પરીક્ષણ}: પીવાલાયક ઉપયોગ માટે સમયાંતરે વિશ્લેષણ
    \end{itemize}

    \textbf{સરકારી પહેલ}:
    \begin{itemize}
        \item \textbf{બિલ્ડિંગ કોડ્સ}: નવા બાંધકામોમાં ફરજિયાત વરસાદી પાણીનો સંગ્રહ
        \item \textbf{સબસિડી}: ઇન્સ્ટોલેશન માટે નાણાકીય પ્રોત્સાહનો
        \item \textbf{જાગૃતિ કાર્યક્રમો}: સમુદાય શિક્ષણ અને તાલીમ
        \item \textbf{તકનીકી સપોર્ટ}: ડિઝાઇન માર્ગદર્શિકા, અમલીકરણ સહાય
    \end{itemize}

    \textbf{પડકારો}:
    \begin{itemize}
        \item \textbf{શરૂઆતનો ખર્ચ}: સંપૂર્ણ સિસ્ટમ માટે સેટઅપ ખર્ચ
        \item \textbf{જાળવણી}: નિયમિત જાળવણી જરૂરિયાતો
        \item \textbf{જગ્યા જરૂરિયાતો}: સ્ટોરેજ ટેન્ક માટે જગ્યાની જરૂરિયાત
        \item \textbf{મોસમી ઉપલબ્ધતા}: ચોમાસાની પદ્ધતિઓ પર નિર્ભરતા
        \item \textbf{ગુણવત્તા ચિંતાઓ}: સંભવિત દૂષણ સમસ્યાઓ
    \end{itemize}

    \textbf{ગણતરી ઉદાહરણ}:
    \begin{itemize}
        \item છત વિસ્તાર: 100 ચો.મી.
        \item વાર્ષિક વરસાદ: 1000 mm
        \item રનઓફ ગુણાંક: 0.8
        \item હાર્વેસ્ટેબલ પાણી = 100 $\times$ 1 $\times$ 0.8 = 80,000 લિટર/વર્ષ
    \end{itemize}

    \begin{mnemonicbox}CCFFS-RSBD-WGFQC-RCSMQ (Catchment-Collection-Flush-Filter-Storage, Rooftop-Surface-Benefits-Design, Water-Groundwater-Flood-Quality-Cost, Regular-Check-System-Maintenance-Quality)\end{mnemonicbox}
\end{solutionbox}

\end{document}
