\documentclass{article}

% content/resources/templates/preamble.tex
\usepackage[margin=0.6in]{geometry}
\author{Milav Dabgar}
\usepackage{amsmath,amssymb,amsthm}
\usepackage{booktabs}
\usepackage{multirow}
\usepackage{xcolor}
\usepackage{tcolorbox}
\tcbuselibrary{breakable,skins}
\usepackage[colorlinks=true,linkcolor=blue]{hyperref}
\usepackage{titlesec}
\usepackage{enumitem}
\usepackage{tikz}
\usepackage{pgfplots}
\usepackage{circuitikz}
\usepackage[version=4]{mhchem}
\usepackage{longtable}
\usepackage{array}
\usepackage{float}
\usepackage{caption}
\usepackage{listings}

\lstset{
  basicstyle=\small\ttfamily,
  breaklines=true,
  breakatwhitespace=false,
  postbreak=\mbox{\textcolor{red}{$\hookrightarrow$}\space},
  float=false,
  numbers=left,
  numberstyle=\tiny\color{gray},
  numbersep=10pt,
  xleftmargin=2em,
  keywordstyle=\color{blue},
  commentstyle=\color{green!60!black},
  stringstyle=\color{purple},
  backgroundcolor=\color{gray!5},
  showstringspaces=false,
  tabsize=2,
  captionpos=b,
  keepspaces=true,
  columns=flexible
}

\pgfplotsset{compat=1.18}
\usetikzlibrary{shapes,arrows,positioning,calc,patterns,decorations.pathmorphing,decorations.markings,arrows.meta}

% Color scheme
\definecolor{headcolor}{RGB}{0,102,204}
\definecolor{keycolor}{RGB}{220,20,60}
\definecolor{solutioncolor}{RGB}{34,139,34}
\definecolor{mnemoniccolor}{RGB}{148,0,211}
\definecolor{codecolor}{RGB}{0,0,100}

% Spacing
\setlength{\parskip}{3pt}
\setlist[itemize]{nosep}
\setlist[enumerate]{nosep}

% Title formatting
\titleformat{\section}{\Large\bfseries\color{headcolor}}{\thesection}{1em}{}
\titleformat{\subsection}{\large\bfseries\color{headcolor}}{\thesubsection}{1em}{}

% Pandoc tightlist compatibility
\providecommand{\tightlist}{%
  \setlength{\itemsep}{0pt}\setlength{\parskip}{0pt}}

% Pandoc longtable compatibility
\newcounter{none}
\def\thenone{}


% content/resources/templates/english-boxes.tex

% Custom environments
\newtcolorbox{solutionbox}{
 breakable,
 enhanced,
 colback=solutioncolor!5!white,
 colframe=solutioncolor!75!black,
 fonttitle=\bfseries,
 title=Solution
}

\newtcolorbox{solutionboxnobreak}{
 colback=solutioncolor!5!white,
 colframe=solutioncolor!75!black,
 fonttitle=\bfseries,
 title=Solution
}

\newtcolorbox{keyformula}{
 breakable,
 enhanced,
 colback=keycolor!5!white,
 colframe=keycolor!75!black,
 fonttitle=\bfseries,
 title=Key Formula
}

\newtcolorbox{mnemonicboxenv}{
 breakable,
 enhanced,
 colback=mnemoniccolor!5!white,
 colframe=mnemoniccolor!75!black,
 fonttitle=\bfseries,
 title=Mnemonic
}

\newcommand{\mnemonicbox}[1]{%
  \begin{mnemonicboxenv}
    #1
  \end{mnemonicboxenv}
}


% Custom commands for GTU solutions
% This file defines semantic commands for consistent formatting

% Question command with automatic formatting
\newcommand{\question}[2]{%
  \section*{Question #1}%
  \textbf{#2}%
}

% OR question variant
\newcommand{\questionor}[2]{%
  \section*{Question #1 OR}%
  \textbf{#2}%
}

% Proper table environment with caption
\newenvironment{answertable}[1]{%
  \begin{table}[htbp]
  \centering
  \caption{#1}
}{%
  \end{table}
}

% Proper figure environment for diagrams
\newenvironment{answerdiagram}[1]{%
  \begin{figure}[htbp]
  \centering
  \caption{#1}
}{%
  \end{figure}
}

% Semantic markup for key terms
\newcommand{\keyword}[1]{\textbf{#1}}
\newcommand{\code}[1]{\texttt{#1}}
\newcommand{\classname}[1]{\texttt{#1}}
\newcommand{\methodname}[1]{\texttt{#1}}

% Proper quotation marks
\newcommand{\mnemonic}[1]{``#1''}


\title{Environment and Sustainability (4300003) - Winter 2022 Solution}
\date{March 01, 2023}

\begin{document}
\maketitle

\questionmarks{1}{a}{3}
\textbf{When ecological overshoot occurs? Explain with reasons.}

\begin{solutionbox}
    \textbf{Answer:}

    \begin{answertable}{Ecological Overshoot Conditions}
    \begin{tabulary}{\linewidth}{L L L}
        \toprule
        \textbf{Condition} & \textbf{Description} & \textbf{Impact} \\
        \midrule
        \textbf{Resource depletion} & Consumption exceeds regeneration rate & Deficit accumulation \\
        \textbf{Population pressure} & Human demand surpasses carrying capacity & Resource scarcity \\
        \textbf{Waste accumulation} & Production exceeds absorption capacity & Environmental degradation \\
        \bottomrule
    \end{tabulary}
    \end{answertable}

    \textbf{Ecological overshoot} occurs when humanity's ecological footprint exceeds Earth's biocapacity. This happens when we consume resources faster than nature can regenerate them and produce waste faster than ecosystems can absorb it.

    \textbf{Key reasons include}:
    \begin{itemize}
        \item \textbf{Population growth}: Increasing human numbers
        \item \textbf{Consumption patterns}: High per-capita resource use
        \item \textbf{Technology impact}: Inefficient resource utilization
    \end{itemize}

    \begin{mnemonicbox}POP-CON-TECH (Population-Consumption-Technology)\end{mnemonicbox}
\end{solutionbox}

\questionmarks{1}{b}{4}
\textbf{Explain food chain using diagram.}

\begin{solutionbox}
    \textbf{Answer:}

    \begin{center}
    \begin{tikzpicture}[node distance=1.5cm, auto]
        \node (a) [gtu block] {Sun Energy};
        \node (b) [gtu block, right=of a] {Producer:\\Green Plants};
        \node (c) [gtu block, right=of b] {Primary Consumer:\\Herbivores};
        \node (d) [gtu block, below=of c] {Secondary Consumer:\\Carnivores};
        \node (e) [gtu block, left=of d] {Tertiary Consumer:\\Top Predators};
        \node (f) [gtu block, left=of e] {Decomposer:\\Bacteria/Fungi};
        \node (g) [gtu block, above=of f] {Nutrients to Soil};

        \draw [gtu arrow] (a) -- (b);
        \draw [gtu arrow] (b) -- (c);
        \draw [gtu arrow] (c) -- (d);
        \draw [gtu arrow] (d) -- (e);
        \draw [gtu arrow] (e) -- (f);
        \draw [gtu arrow] (f) -- (g);
        \draw [gtu arrow] (g) -- (b);
    \end{tikzpicture}
    \end{center}

    \textbf{Food chain} represents the linear sequence of energy transfer from one trophic level to another in an ecosystem.

    \textbf{Components}:
    \begin{itemize}
        \item \textbf{Producers}: Convert solar energy to chemical energy
        \item \textbf{Primary consumers}: Feed on producers (herbivores)
        \item \textbf{Secondary consumers}: Feed on primary consumers (carnivores)
        \item \textbf{Decomposers}: Break down dead organisms
    \end{itemize}

    \textbf{Energy flow}: Unidirectional from sun to top predators with 10\% efficiency between levels.

    \begin{mnemonicbox}PPSD (Producer-Primary-Secondary-Decomposer)\end{mnemonicbox}
\end{solutionbox}

\questionmarks{1}{c}{7}
\textbf{Write a note on: carbon cycle.}

\begin{solutionbox}
    \textbf{Answer:}

    \begin{center}
    \begin{tikzpicture}[node distance=2cm, auto]
        \node (a) [gtu block] {Atmospheric CO$_2$};
        \node (b) [gtu block, below left=of a] {Photosynthesis\\(Plants)};
        \node (c) [gtu block, below=of b] {Biomass};
        \node (d) [gtu block, right=of c] {Animals};
        \node (e) [gtu block, above right=of d] {Respiration};
        \node (f) [gtu block, below=of c] {Decomposition};
        \node (g) [gtu block, left=of a] {Ocean Dissolution};
        \node (h) [gtu block, below=of g] {Marine Life};
        \node (i) [gtu block, right=of a] {Fossil Fuel\\Burning};

        \draw [gtu arrow] (a) -- (b);
        \draw [gtu arrow] (b) -- (c);
        \draw [gtu arrow] (c) -- (d);
        \draw [gtu arrow] (d) -- (e);
        \draw [gtu arrow] (e) -- (a);
        \draw [gtu arrow] (c) -- (f);
        \draw [gtu arrow] (f) -- (a);
        \draw [gtu arrow] (a) -- (g);
        \draw [gtu arrow] (g) -- (h);
        \draw [gtu arrow] (h) to [bend left] (a);
        \draw [gtu arrow] (i) -- (a);
    \end{tikzpicture}
    \end{center}

    \textbf{Carbon cycle} is the biogeochemical process where carbon moves through atmosphere, biosphere, hydrosphere, and geosphere.

    \textbf{Major processes}:
    \begin{itemize}
        \item \textbf{Photosynthesis}: Plants absorb CO$_2$ from atmosphere
        \item \textbf{Respiration}: Organisms release CO$_2$ back to atmosphere
        \item \textbf{Decomposition}: Dead organic matter releases stored carbon
        \item \textbf{Ocean exchange}: CO$_2$ dissolves in seawater forming carbonic acid
    \end{itemize}

    \textbf{Human impact}:
    \begin{itemize}
        \item \textbf{Fossil fuel combustion}: Increases atmospheric CO$_2$
        \item \textbf{Deforestation}: Reduces carbon sequestration capacity
        \item \textbf{Industrial processes}: Additional carbon emissions
    \end{itemize}

    \textbf{Environmental significance}: Maintains atmospheric CO$_2$ balance, regulates global temperature, supports life processes.

    \begin{mnemonicbox}PRDO-FDI (Photosynthesis-Respiration-Decomposition-Ocean, Fossil-Deforestation-Industry)\end{mnemonicbox}
\end{solutionbox}

\questionmarks{1}{c}{7}
\textbf{Classify aquatic ecosystem. Explain marine ecosystem.}

\begin{solutionbox}
    \textbf{Answer:}

    \begin{answertable}{Aquatic Ecosystem Classification}
    \begin{tabulary}{\linewidth}{L L L}
        \toprule
        \textbf{Type} & \textbf{Characteristics} & \textbf{Examples} \\
        \midrule
        \textbf{Freshwater} & Low salt content (<1\%) & Rivers, lakes, ponds \\
        \textbf{Marine} & High salt content (3.5\%) & Oceans, seas \\
        \textbf{Brackish} & Mixed fresh-salt water & Estuaries, lagoons \\
        \bottomrule
    \end{tabulary}
    \end{answertable}

    \textbf{Marine Ecosystem Components}:

    \begin{center}
    \begin{tikzpicture}[node distance=1.5cm, auto]
        \node (root) [gtu root] {Marine Ecosystem};
        \node (pelagic) [gtu child, below left=of root] {Pelagic Zone};
        \node (benthic) [gtu child, below right=of root] {Benthic Zone};
        
        \node (photic) [gtu block, below=of pelagic, xshift=-1cm] {Photic Zone:\\0-200m};
        \node (aphotic) [gtu block, below=of pelagic, xshift=1cm] {Aphotic Zone:\\>200m};
        
        \node (shelf) [gtu block, below=of benthic, xshift=-1cm] {Continental\\Shelf};
        \node (deep) [gtu block, below=of benthic, xshift=1cm] {Deep Ocean\\Floor};

        \draw [gtu arrow] (root) -- (pelagic);
        \draw [gtu arrow] (root) -- (benthic);
        \draw [gtu arrow] (pelagic) -- (photic);
        \draw [gtu arrow] (pelagic) -- (aphotic);
        \draw [gtu arrow] (benthic) -- (shelf);
        \draw [gtu arrow] (benthic) -- (deep);
    \end{tikzpicture}
    \end{center}

    \textbf{Marine ecosystem} covers 71\% of Earth's surface, containing saltwater bodies with complex food webs.

    \textbf{Zones}:
    \begin{itemize}
        \item \textbf{Pelagic}: Open water column with plankton, fish
        \item \textbf{Benthic}: Ocean floor with bottom-dwelling organisms
        \item \textbf{Intertidal}: Shore area between high and low tides
    \end{itemize}

    \textbf{Importance}:
    \begin{itemize}
        \item \textbf{Climate regulation}: Ocean currents moderate global temperature
        \item \textbf{Oxygen production}: Marine phytoplankton produce 50\% of atmospheric oxygen
        \item \textbf{Economic value}: Fisheries, transportation, tourism
    \end{itemize}

    \begin{mnemonicbox}PBI-COE (Pelagic-Benthic-Intertidal, Climate-Oxygen-Economy)\end{mnemonicbox}
\end{solutionbox}

\questionmarks{2}{a}{3}
\textbf{What is carrying capacity of earth?}

\begin{solutionbox}
    \textbf{Answer:}

    \begin{answertable}{Carrying Capacity Factors}
    \begin{tabulary}{\linewidth}{L L L}
        \toprule
        \textbf{Factor} & \textbf{Description} & \textbf{Limit} \\
        \midrule
        \textbf{Resources} & Available land, water, minerals & Finite \\
        \textbf{Food production} & Agricultural capacity & Limited by soil \\
        \textbf{Waste absorption} & Ecosystem's waste processing & Saturation point \\
        \bottomrule
    \end{tabulary}
    \end{answertable}

    \textbf{Carrying capacity} is the maximum population size an environment can sustain indefinitely without degrading the environment.

    \textbf{Earth's carrying capacity} depends on:
    \begin{itemize}
        \item \textbf{Resource availability}: Fresh water, arable land, energy sources
        \item \textbf{Technology level}: Efficiency of resource utilization
        \item \textbf{Consumption patterns}: Per-capita resource demand
    \end{itemize}

    \textbf{Current estimates}: Range from 4-16 billion people based on consumption levels and technological advancement.

    \begin{mnemonicbox}RTC (Resources-Technology-Consumption)\end{mnemonicbox}
\end{solutionbox}

\questionmarks{2}{b}{4}
\textbf{How food web relates to food chain?}

\begin{solutionbox}
    \textbf{Answer:}

    \begin{center}
    \begin{tikzpicture}[node distance=1.5cm, auto]
        \node (grass) [gtu block] {Grass};
        \node (rabbit) [gtu block, right=of grass] {Rabbit};
        \node (deer) [gtu block, below=of grass] {Deer};
        \node (fox) [gtu block, right=of rabbit] {Fox};
        \node (hawk) [gtu block, below=of fox] {Hawk};
        \node (wolf) [gtu block, right=of deer] {Wolf};
        \node (decomp) [gtu block, right=of fox] {Decomposers};

        \draw [gtu arrow] (grass) -- (rabbit);
        \draw [gtu arrow] (grass) -- (deer);
        \draw [gtu arrow] (rabbit) -- (fox);
        \draw [gtu arrow] (rabbit) -- (hawk);
        \draw [gtu arrow] (deer) -- (fox);
        \draw [gtu arrow] (deer) -- (wolf);
        \draw [gtu arrow] (fox) -- (decomp);
        \draw [gtu arrow] (hawk) -- (decomp);
        \draw [gtu arrow] (wolf) -- (decomp);
    \end{tikzpicture}
    \end{center}

    \textbf{Food web} is an interconnected network of multiple food chains showing complex feeding relationships in an ecosystem.

    \textbf{Relationship between food web and food chain}:
    \begin{itemize}
        \item \textbf{Food chain}: Linear sequence of energy transfer
        \item \textbf{Food web}: Multiple interconnected food chains
        \item \textbf{Complexity}: Food webs show realistic ecosystem interactions
        \item \textbf{Stability}: Multiple pathways provide ecosystem resilience
    \end{itemize}

    \textbf{Key differences}:
    \begin{itemize}
        \item \textbf{Structure}: Chain is linear, web is network-based
        \item \textbf{Energy flow}: Chain shows single pathway, web shows multiple routes
        \item \textbf{Species interaction}: Web demonstrates omnivory and alternative feeding
    \end{itemize}

    \begin{mnemonicbox}LNCR (Linear-Network, Chain-Resilience)\end{mnemonicbox}
\end{solutionbox}

\questionmarks{2}{c}{7}
\textbf{Write a note on: air pollution}

\begin{solutionbox}
    \textbf{Answer:}

    \begin{answertable}{Air Pollution Sources and Effects}
    \begin{tabulary}{\linewidth}{L L L}
        \toprule
        \textbf{Pollutant} & \textbf{Source} & \textbf{Health Effect} \\
        \midrule
        \textbf{PM2.5/PM10} & Vehicles, industries & Respiratory diseases \\
        \textbf{SO$_2$} & Coal burning & Acid rain, asthma \\
        \textbf{NO$_x$} & Vehicle exhaust & Smog formation \\
        \textbf{CO} & Incomplete combustion & Oxygen deficiency \\
        \bottomrule
    \end{tabulary}
    \end{answertable}

    \textbf{Air pollution} is contamination of atmosphere by harmful substances that cause adverse effects on human health and environment.

    \textbf{Classification by source}:
    \begin{itemize}
        \item \textbf{Primary pollutants}: Directly emitted (CO, SO$_2$, particulates)
        \item \textbf{Secondary pollutants}: Formed through chemical reactions (ozone, acid rain)
    \end{itemize}

    \textbf{Major sources}:
    \begin{itemize}
        \item \textbf{Mobile sources}: Vehicles, aircraft, ships
        \item \textbf{Stationary sources}: Power plants, industries, residential heating
        \item \textbf{Natural sources}: Volcanic eruptions, forest fires, dust storms
    \end{itemize}

    \textbf{Control measures}:
    \begin{itemize}
        \item \textbf{Technological}: Catalytic converters, scrubbers, filters
        \item \textbf{Regulatory}: Emission standards, fuel quality norms
        \item \textbf{Alternative energy}: Renewable sources, electric vehicles
    \end{itemize}

    \textbf{Health impacts}: Respiratory diseases, cardiovascular problems, cancer, reduced life expectancy.

    \textbf{Environmental effects}: Acid rain, ozone depletion, climate change, visibility reduction.

    \begin{mnemonicbox}PSMT-RE-HE (Primary-Secondary-Mobile-stationary-Technological-Regulatory-Health-Environment)\end{mnemonicbox}
\end{solutionbox}

\questionmarks{2}{a}{3}
\textbf{Explain bad effects of plastic waste on environment.}

\begin{solutionbox}
    \textbf{Answer:}

    \begin{answertable}{Plastic Waste Environmental Effects}
    \begin{tabulary}{\linewidth}{L L L}
        \toprule
        \textbf{Impact Area} & \textbf{Effect} & \textbf{Duration} \\
        \midrule
        \textbf{Marine life} & Entanglement, ingestion & Persistent \\
        \textbf{Soil} & Microplastic contamination & 500+ years \\
        \textbf{Food chain} & Bioaccumulation & Generational \\
        \bottomrule
    \end{tabulary}
    \end{answertable}

    \textbf{Plastic waste} causes severe environmental degradation due to its non-biodegradable nature.

    \textbf{Environmental effects}:
    \begin{itemize}
        \item \textbf{Marine pollution}: Ocean plastic kills marine animals through entanglement and ingestion
        \item \textbf{Soil contamination}: Microplastics affect soil fertility and crop growth
        \item \textbf{Food chain disruption}: Plastic particles accumulate in organisms
    \end{itemize}

    \textbf{Long-term impacts}: Persistent organic pollutants, habitat destruction, ecosystem imbalance.

    \begin{mnemonicbox}MSF (Marine-Soil-Foodchain)\end{mnemonicbox}
\end{solutionbox}

\questionmarks{2}{b}{4}
\textbf{Which are signs of polluted water? List major sources of water pollution.}

\begin{solutionbox}
    \textbf{Answer:}

    \begin{answertable}{Water Pollution Indicators and Sources}
    \begin{tabulary}{\linewidth}{L L L}
        \toprule
        \textbf{Signs} & \textbf{Measurement} & \textbf{Sources} \\
        \midrule
        \textbf{High BOD/COD} & >5 mg/L & Industrial discharge \\
        \textbf{Turbidity} & Cloudiness & Agricultural runoff \\
        \textbf{pH changes} & <6.5 or >8.5 & Acid mine drainage \\
        \textbf{Foul odor} & H$_2$S smell & Sewage discharge \\
        \bottomrule
    \end{tabulary}
    \end{answertable}

    \textbf{Signs of polluted water}:
    \begin{itemize}
        \item \textbf{Physical}: Color change, turbidity, floating debris, odor
        \item \textbf{Chemical}: High BOD/COD, pH deviation, heavy metals, toxic compounds
        \item \textbf{Biological}: Pathogenic microorganisms, algal blooms, fish kills
    \end{itemize}

    \textbf{Major sources}:
    \begin{itemize}
        \item \textbf{Point sources}: Industrial discharge, sewage outfalls, concentrated animal feeding
        \item \textbf{Non-point sources}: Agricultural runoff, urban stormwater, atmospheric deposition
    \end{itemize}

    \begin{mnemonicbox}PCB-PIN (Physical-Chemical-Biological, Point-Non-point)\end{mnemonicbox}
\end{solutionbox}

\questionmarks{2}{c}{7}
\textbf{Classify e-waste. How e-waste is recycled?}

\begin{solutionbox}
    \textbf{Answer:}

    \begin{answertable}{E-waste Classification}
    \begin{tabulary}{\linewidth}{L L L}
        \toprule
        \textbf{Category} & \textbf{Examples} & \textbf{Hazardous Components} \\
        \midrule
        \textbf{Large appliances} & Refrigerators, washing machines & CFCs, heavy metals \\
        \textbf{Small appliances} & Microwaves, vacuum cleaners & Plastics, metals \\
        \textbf{IT equipment} & Computers, printers & Lead, mercury, cadmium \\
        \textbf{Consumer electronics} & TVs, mobile phones & Rare earth elements \\
        \bottomrule
    \end{tabulary}
    \end{answertable}

    \textbf{E-waste classification}:
    \begin{itemize}
        \item \textbf{White goods}: Large household appliances
        \item \textbf{Brown goods}: Entertainment electronics
        \item \textbf{Gray goods}: IT and telecommunication equipment
        \item \textbf{Green goods}: Renewable energy equipment
    \end{itemize}

    \textbf{E-waste recycling process}:

    \begin{center}
    \begin{tikzpicture}[node distance=1.5cm, auto]
        \node (a) [gtu block] {Collection};
        \node (b) [gtu block, right=of a] {Sorting};
        \node (c) [gtu block, right=of b] {Dismantling};
        \node (d) [gtu block, below=of c] {Shredding};
        \node (e) [gtu block, left=of d] {Separation};
        \node (f) [gtu block, left=of e] {Material\\Recovery};
        \node (g) [gtu block, below=of f] {Refining};
        \node (h) [gtu block, right=of g] {New Products};

        \draw [gtu arrow] (a) -- (b);
        \draw [gtu arrow] (b) -- (c);
        \draw [gtu arrow] (c) -- (d);
        \draw [gtu arrow] (d) -- (e);
        \draw [gtu arrow] (e) -- (f);
        \draw [gtu arrow] (f) -- (g);
        \draw [gtu arrow] (g) -- (h);
    \end{tikzpicture}
    \end{center}

    \textbf{Recycling methods}:
    \begin{itemize}
        \item \textbf{Mechanical}: Physical separation of materials
        \item \textbf{Metallurgical}: High-temperature processing for metal recovery
        \item \textbf{Chemical}: Leaching processes for precious metals
    \end{itemize}

    \textbf{Challenges}: Hazardous material handling, complex composition, economic viability.

    \textbf{Benefits}: Resource conservation, pollution prevention, job creation, reduced mining needs.

    \begin{mnemonicbox}WBGG-CSDSMR (White-Brown-Gray-Green, Collection-Sorting-Dismantling-Shredding-Separation-Material-Refining)\end{mnemonicbox}
\end{solutionbox}

\questionmarks{3}{a}{3}
\textbf{Distinguish BOD and COD.}

\begin{solutionbox}
    \textbf{Answer:}

    \begin{answertable}{BOD vs COD Comparison}
    \begin{tabulary}{\linewidth}{L L L}
        \toprule
        \textbf{Parameter} & \textbf{BOD} & \textbf{COD} \\
        \midrule
        \textbf{Full form} & Biochemical Oxygen Demand & Chemical Oxygen Demand \\
        \textbf{Test duration} & 5 days & 2-3 hours \\
        \textbf{Oxidation type} & Biological & Chemical \\
        \textbf{Degradation} & Biodegradable organics only & All organic compounds \\
        \bottomrule
    \end{tabulary}
    \end{answertable}

    \textbf{BOD (Biochemical Oxygen Demand)}:
    \begin{itemize}
        \item Measures oxygen consumed by microorganisms
        \item Indicates biodegradable organic pollution
        \item Standard test: 5 days at 20\textdegree{}C
    \end{itemize}

    \textbf{COD (Chemical Oxygen Demand)}:
    \begin{itemize}
        \item Measures oxygen required for chemical oxidation
        \item Indicates total organic pollution
        \item Uses strong oxidizing agents (potassium dichromate)
    \end{itemize}

    \begin{mnemonicbox}BTCD (Biological-Time-Chemical-Degradation)\end{mnemonicbox}
\end{solutionbox}

\questionmarks{3}{b}{4}
\textbf{Classify solid waste.}

\begin{solutionbox}
    \textbf{Answer:}

    \begin{answertable}{Solid Waste Classification}
    \begin{tabulary}{\linewidth}{L L L}
        \toprule
        \textbf{Classification} & \textbf{Type} & \textbf{Examples} \\
        \midrule
        \textbf{By source} & Municipal, Industrial, Agricultural & Household, Factory, Farm waste \\
        \textbf{By composition} & Organic, Inorganic & Food waste, Plastics \\
        \textbf{By hazard} & Hazardous, Non-hazardous & Medical, Paper \\
        \bottomrule
    \end{tabulary}
    \end{answertable}

    \textbf{Solid waste classification}:

    \begin{center}
    \begin{tikzpicture}[node distance=1.5cm, auto]
        \node (root) [gtu root] {Solid Waste};
        \node (mun) [gtu child, below left=of root, xshift=-2cm] {Municipal Solid\\Waste};
        \node (ind) [gtu child, left=of mun] {Industrial Waste};
        \node (haz) [gtu child, below right=of root, xshift=2cm] {Hazardous Waste};
        \node (agri) [gtu child, right=of haz] {Agricultural Waste};
        
        \node (org) [gtu block, below=of mun] {Organic:\\50-60\%};
        \node (rec) [gtu block, left=of org] {Recyclables:\\20-30\%};
        \node (inert) [gtu block, right=of org] {Inert:\\10-20\%};

        \draw [gtu arrow] (root) -- (mun);
        \draw [gtu arrow] (root) -- (ind);
        \draw [gtu arrow] (root) -- (haz);
        \draw [gtu arrow] (root) -- (agri);
        \draw [gtu arrow] (mun) -- (org);
        \draw [gtu arrow] (mun) -- (rec);
        \draw [gtu arrow] (mun) -- (inert);
    \end{tikzpicture}
    \end{center}

    \textbf{By source}:
    \begin{itemize}
        \item \textbf{Municipal}: Residential, commercial, institutional waste
        \item \textbf{Industrial}: Manufacturing, processing byproducts
        \item \textbf{Agricultural}: Crop residues, animal waste
    \end{itemize}

    \textbf{By composition}: Organic (biodegradable), inorganic (non-biodegradable), recyclable materials.

    \textbf{Management hierarchy}: Reduce, reuse, recycle, recover, dispose.

    \begin{mnemonicbox}MIA-OIR (Municipal-Industrial-Agricultural, Organic-Inorganic-Recyclable)\end{mnemonicbox}
\end{solutionbox}

\questionmarks{3}{c}{7}
\textbf{With the use of diagram explain solar photovoltaic System.}

\begin{solutionbox}
    \textbf{Answer:}

    \begin{center}
    \begin{tikzpicture}[node distance=1.5cm, auto]
        \node (solar) [gtu start] {Solar Radiation};
        \node (panel) [gtu block, right=of solar] {PV Panel};
        \node (dc) [gtu block, right=of panel] {DC Power};
        \node (inv) [gtu block, right=of dc] {Inverter};
        \node (ac) [gtu block, right=of inv] {AC Power};
        \node (load) [gtu block, right=of ac] {Load/Grid};
        
        \node (bat) [gtu block, below=of dc] {Battery};
        \node (cc) [gtu block, left=of bat] {Charge\\Controller};

        \draw [gtu arrow] (solar) -- (panel);
        \draw [gtu arrow] (panel) -- (dc);
        \draw [gtu arrow] (dc) -- (inv);
        \draw [gtu arrow] (inv) -- (ac);
        \draw [gtu arrow] (ac) -- (load);
        
        \draw [gtu arrow] (dc) -- (bat);
        \draw [gtu arrow] (bat) -- (dc); % Bidirectional simplified
        \draw [gtu arrow] (dc) to [bend left] (cc);
        \draw [gtu arrow] (cc) -- (bat);
        % Connecting AC to battery is unusual in simple diagrams unless via inverter/charger, 
        % sticking to faithful implementation of mermaid logic: E --> G (AC to Battery? Likely charger)
        % Mermaid said: E --> G (AC -> Battery). A bit odd, usually Grid charges battery.
        \draw [gtu arrow] (ac) to [bend left] (bat);
    \end{tikzpicture}
    \end{center}

    \textbf{Solar Photovoltaic System} converts sunlight directly into electricity using semiconductor materials.

    \textbf{Components}:
    \begin{itemize}
        \item \textbf{PV modules}: Silicon cells convert light to DC electricity
        \item \textbf{Inverter}: Converts DC to AC power
        \item \textbf{Battery storage}: Stores excess energy for later use
        \item \textbf{Charge controller}: Regulates battery charging
        \item \textbf{Monitoring system}: Tracks performance and faults
    \end{itemize}

    \textbf{Working principle}:
    \begin{enumerate}
        \item \textbf{Photovoltaic effect}: Solar cells absorb photons
        \item \textbf{Electron excitation}: Creates electron-hole pairs
        \item \textbf{Current generation}: Electrons flow creating DC current
        \item \textbf{Power conditioning}: Inverter converts DC to AC
    \end{enumerate}

    \textbf{Types}:
    \begin{itemize}
        \item \textbf{Grid-connected}: Synchronized with utility grid
        \item \textbf{Stand-alone}: Independent systems with battery backup
        \item \textbf{Hybrid}: Combination of grid-connected and battery storage
    \end{itemize}

    \textbf{Applications}: Residential rooftops, commercial buildings, utility-scale power plants, remote area electrification.

    \textbf{Advantages}: Clean energy, low maintenance, modular design, long lifespan (25+ years).

    \begin{mnemonicbox}PIBCM-PECG (Panel-Inverter-Battery-Controller-Monitor, Photovoltaic-Electron-Current-Grid)\end{mnemonicbox}
\end{solutionbox}

\questionmarks{3}{a}{3}
\textbf{Compare conventional and non-conventional energy sources.}

\begin{solutionbox}
    \textbf{Answer:}

    \begin{answertable}{Energy Sources Comparison}
    \begin{tabulary}{\linewidth}{L L L}
        \toprule
        \textbf{Aspect} & \textbf{Conventional} & \textbf{Non-conventional} \\
        \midrule
        \textbf{Availability} & Limited reserves & Unlimited/renewable \\
        \textbf{Environmental impact} & High pollution & Clean/minimal impact \\
        \textbf{Cost} & Initially lower & Decreasing rapidly \\
        \bottomrule
    \end{tabulary}
    \end{answertable}

    \textbf{Conventional energy sources}: Coal, oil, natural gas, nuclear power - finite resources with environmental concerns.

    \textbf{Non-conventional energy sources}: Solar, wind, hydro, biomass - renewable resources with sustainable characteristics.

    \textbf{Key differences}: Depletion vs renewable, pollution vs clean, established vs emerging technology.

    \begin{mnemonicbox}AEC (Availability-Environmental-Cost)\end{mnemonicbox}
\end{solutionbox}

\questionmarks{3}{b}{4}
\textbf{Explain working of natural circulation solar water heater.}

\begin{solutionbox}
    \textbf{Answer:}

    \begin{center}
    \begin{tikzpicture}[node distance=1.5cm, auto]
        \node (tank) [gtu block, minimum width=3cm] {Solar Tank\\(Hot Water)};
        \node (collector) [gtu block, below=of tank, yshift=-1cm, minimum width=3cm] {Solar Collector\\(Cold Water)};
        
        \draw [gtu arrow] (collector.north east) -- (tank.south east) node[midway, right] {Hot Water Rises};
        \draw [gtu arrow] (tank.south west) -- (collector.north west) node[midway, left] {Cold Water Sinks};
        
        \node [text width=3cm, align=center, below of=collector] {Thermosiphon Principle};
    \end{tikzpicture}
    \end{center}

    \textbf{Natural circulation solar water heater} uses thermosiphon principle for water circulation without external pumps.

    \textbf{Working principle}:
    \begin{itemize}
        \item \textbf{Solar collection}: Collector absorbs solar radiation, heating water
        \item \textbf{Density difference}: Hot water becomes less dense, rises naturally
        \item \textbf{Circulation}: Cold water from tank bottom flows to collector
        \item \textbf{Storage}: Hot water accumulates in insulated storage tank
    \end{itemize}

    \textbf{Components}: Flat plate collector, insulated storage tank, connecting pipes, safety valves.

    \textbf{Advantages}: No electricity required, simple design, low maintenance, cost-effective.

    \begin{mnemonicbox}SDCS (Solar-Density-Circulation-Storage)\end{mnemonicbox}
\end{solutionbox}

\questionmarks{3}{c}{7}
\textbf{Explain working principle of horizontal axis wind turbine.}

\begin{solutionbox}
    \textbf{Answer:}

    \begin{center}
    \begin{tikzpicture}[node distance=1.5cm, auto]
        \node (wind) [gtu start] {Wind Energy};
        \node (blades) [gtu block, right=of wind] {Rotor Blades};
        \node (shaft) [gtu block, right=of blades] {Shaft\\Rotation};
        \node (gear) [gtu block, right=of shaft] {Gearbox};
        \node (gen) [gtu block, right=of gear] {Generator};
        \node (power) [gtu block, right=of gen] {Electrical\\Power};
        
        \node (nacelle) [gtu block, below=of gear] {Nacelle};
        \node (tower) [gtu block, below=of nacelle] {Tower};

        \draw [gtu arrow] (wind) -- (blades);
        \draw [gtu arrow] (blades) -- (shaft);
        \draw [gtu arrow] (shaft) -- (gear);
        \draw [gtu arrow] (gear) -- (gen);
        \draw [gtu arrow] (gen) -- (power);
        
        \draw [dashed] (nacelle) -- (blades); % Structural link
        \draw [dashed] (nacelle) -- (gear);
        \draw [dashed] (nacelle) -- (gen);
        \draw [gtu arrow] (tower) -- (nacelle);
    \end{tikzpicture}
    \end{center}

    \textbf{Horizontal Axis Wind Turbine (HAWT)} converts kinetic energy of wind into electrical energy using aerodynamic lift principle.

    \textbf{Working principle}:
    \begin{enumerate}
        \item \textbf{Wind capture}: Rotor blades designed with aerodynamic profile
        \item \textbf{Lift generation}: Pressure difference across blade surfaces creates lift force
        \item \textbf{Rotation}: Lift force causes rotor to rotate around horizontal axis
        \item \textbf{Speed conversion}: Gearbox increases rotational speed from 30-50 rpm to 1500 rpm
        \item \textbf{Power generation}: High-speed rotation drives electrical generator
    \end{enumerate}

    \textbf{Components}:
    \begin{itemize}
        \item \textbf{Rotor assembly}: 2-3 blades, hub, pitch control system
        \item \textbf{Nacelle}: Houses gearbox, generator, control systems
        \item \textbf{Tower}: Supports nacelle at optimal height (50-120m)
        \item \textbf{Foundation}: Concrete base for structural stability
    \end{itemize}

    \textbf{Control systems}:
    \begin{itemize}
        \item \textbf{Yaw system}: Orients turbine to face wind direction
        \item \textbf{Pitch control}: Adjusts blade angle for optimal wind capture
        \item \textbf{Brake system}: Emergency stopping mechanism
    \end{itemize}

    \textbf{Advantages}: High efficiency (35-45\%), proven technology, economies of scale.
    \textbf{Disadvantages}: Visual impact, noise, bird strikes, wind variability.

    \textbf{Power calculation}: $P = 0.5 \times \rho \times A \times V^3 \times C_p$
    Where: $\rho$ = air density, $A$ = swept area, $V$ = wind speed, $C_p$ = power coefficient

    \begin{mnemonicbox}WLRSG-RNTP-YPB (Wind-Lift-Rotation-Speed-Generation, Rotor-Nacelle-Tower-Foundation, Yaw-Pitch-Brake)\end{mnemonicbox}
\end{solutionbox}

\questionmarks{4}{a}{3}
\textbf{Write advantages and disadvantages of tidal energy.}

\begin{solutionbox}
    \textbf{Answer:}

    \begin{answertable}{Tidal Energy Pros and Cons}
    \begin{tabulary}{\linewidth}{L L}
        \toprule
        \textbf{Advantages} & \textbf{Disadvantages} \\
        \midrule
        Predictable energy source & Limited suitable locations \\
        No greenhouse gas emissions & High initial capital cost \\
        Long lifespan (100+ years) & Environmental impact on marine life \\
        \bottomrule
    \end{tabulary}
    \end{answertable}

    \textbf{Tidal energy} harnesses gravitational forces between Earth, moon, and sun to generate electricity.

    \textbf{Advantages}:
    \begin{itemize}
        \item \textbf{Reliability}: Highly predictable tidal cycles
        \item \textbf{Clean energy}: Zero operational emissions
        \item \textbf{Durability}: Infrastructure lasts decades
    \end{itemize}

    \textbf{Disadvantages}:
    \begin{itemize}
        \item \textbf{Geographic limitations}: Requires specific coastal conditions
        \item \textbf{High costs}: Expensive installation and maintenance
        \item \textbf{Ecological impact}: Affects marine ecosystems
    \end{itemize}

    \begin{mnemonicbox}RCD-GHE (Reliable-Clean-Durable, Geographic-High cost-Ecological)\end{mnemonicbox}
\end{solutionbox}

\questionmarks{4}{b}{4}
\textbf{Explain working principle of biogas plant.}

\begin{solutionbox}
    \textbf{Answer:}

    \begin{center}
    \begin{tikzpicture}[node distance=1.5cm, auto]
        \node (input) [gtu start] {Organic Waste\\Input};
        \node (mix) [gtu block, right=of input] {Mixing Tank};
        \node (digest) [gtu block, right=of mix] {Digester Tank};
        \node (gas) [gtu block, above=of digest] {Gas Collection};
        \node (slurry) [gtu block, below=of digest] {Slurry Output};
        \node (storage) [gtu block, right=of gas] {Biogas\\Storage};
        \node (use) [gtu block, right=of storage] {End Use};

        \draw [gtu arrow] (input) -- (mix);
        \draw [gtu arrow] (mix) -- (digest);
        \draw [gtu arrow] (digest) -- (gas);
        \draw [gtu arrow] (digest) -- (slurry);
        \draw [gtu arrow] (gas) -- (storage);
        \draw [gtu arrow] (storage) -- (use);
    \end{tikzpicture}
    \end{center}

    \textbf{Biogas plant} produces methane-rich gas through anaerobic digestion of organic waste materials.

    \textbf{Working principle}:
    \begin{enumerate}
        \item \textbf{Feed preparation}: Organic waste mixed with water (1:1 ratio)
        \item \textbf{Anaerobic digestion}: Bacteria break down organic matter in oxygen-free environment
        \item \textbf{Gas production}: Methane (50-70\%) and CO$_2$ (30-40\%) generated
        \item \textbf{Gas collection}: Biogas collected in gas holder dome
    \end{enumerate}

    \textbf{Process stages}:
    \begin{itemize}
        \item \textbf{Hydrolysis}: Complex organics broken into simple compounds
        \item \textbf{Acidogenesis}: Organic acids formation
        \item \textbf{Methanogenesis}: Methane production by methanogenic bacteria
    \end{itemize}

    \textbf{Optimal conditions}: Temperature 35-40\textdegree{}C, pH 6.8-7.2, retention time 15-30 days.

    \begin{mnemonicbox}FAGH-HAM (Feed-Anaerobic-Gas-Holder, Hydrolysis-Acidogenesis-Methanogenesis)\end{mnemonicbox}
\end{solutionbox}

\questionmarks{4}{c}{7}
\textbf{Explain green house effect.}

\begin{solutionbox}
    \textbf{Answer:}

    \begin{center}
    \begin{tikzpicture}[node distance=1.5cm, auto]
        \node (solar) [gtu start] {Solar Radiation};
        \node (earth) [gtu block, right=of solar] {Earth's Surface};
        \node (absorb) [gtu block, right=of earth] {Heat\\Absorption};
        \node (ir) [gtu block, below=of absorb] {Infrared\\Radiation};
        \node (gas) [gtu block, left=of ir] {Greenhouse\\Gases};
        \node (trap) [gtu block, left=of gas] {Heat Trapping};
        \node (rerad) [gtu block, below=of trap] {Re-radiation\\to Earth};
        \node (warm) [gtu block, right=of rerad] {Global\\Warming};

        \draw [gtu arrow] (solar) -- (earth);
        \draw [gtu arrow] (earth) -- (absorb);
        \draw [gtu arrow] (absorb) -- (ir);
        \draw [gtu arrow] (ir) -- (gas);
        \draw [gtu arrow] (gas) -- (trap);
        \draw [gtu arrow] (trap) -- (rerad);
        \draw [gtu arrow] (rerad) -- (warm);
    \end{tikzpicture}
    \end{center}

    \textbf{Greenhouse effect} is the process where atmospheric gases trap heat from sun, warming Earth's surface beyond normal temperature.

    \textbf{Natural greenhouse effect}:
    \begin{itemize}
        \item \textbf{Solar radiation}: Sun emits short-wave radiation (visible light)
        \item \textbf{Surface absorption}: Earth absorbs solar energy, heats up
        \item \textbf{Heat re-emission}: Earth emits long-wave infrared radiation
        \item \textbf{Gas absorption}: Greenhouse gases absorb infrared radiation
        \item \textbf{Heat retention}: Trapped heat warms lower atmosphere
    \end{itemize}

    \textbf{Greenhouse gases and contributions}:
    \begin{itemize}
        \item \textbf{Carbon dioxide (CO$_2$)}: 76\% - fossil fuel burning, deforestation
        \item \textbf{Methane (CH$_4$)}: 16\% - agriculture, landfills, livestock
        \item \textbf{Nitrous oxide (N$_2$O)}: 6\% - fertilizers, fossil fuel combustion
        \item \textbf{Fluorinated gases}: 2\% - industrial processes, refrigeration
    \end{itemize}

    \textbf{Enhanced greenhouse effect}: Human activities increase greenhouse gas concentrations, intensifying heat trapping.

    \textbf{Consequences}:
    \begin{itemize}
        \item \textbf{Global temperature rise}: Average 1.1\textdegree{}C increase since pre-industrial times
        \item \textbf{Climate change}: Altered precipitation patterns, extreme weather events
        \item \textbf{Sea level rise}: Thermal expansion and ice sheet melting
        \item \textbf{Ecosystem disruption}: Species migration, coral bleaching, forest fires
    \end{itemize}

    \textbf{Mitigation strategies}:
    \begin{itemize}
        \item \textbf{Renewable energy}: Reduce fossil fuel dependence
        \item \textbf{Energy efficiency}: Improve technology and practices
        \item \textbf{Carbon sequestration}: Forest restoration, carbon capture storage
        \item \textbf{International cooperation}: Paris Agreement, emission reduction targets
    \end{itemize}

    \begin{mnemonicbox}SSAHR-CMNO-GTSE-RECC (Solar-Surface-Absorption-Heat-Radiation, CO2-Methane-Nitrous-Other, Global-Temperature-Sea-Ecosystem, Renewable-Efficiency-Carbon-Cooperation)\end{mnemonicbox}
\end{solutionbox}

\questionmarks{4}{a}{3}
\textbf{What is climate change?}

\begin{solutionbox}
    \textbf{Answer:}

    \begin{answertable}{Climate Change Indicators}
    \begin{tabulary}{\linewidth}{L L L}
        \toprule
        \textbf{Indicator} & \textbf{Change} & \textbf{Evidence} \\
        \midrule
        \textbf{Temperature} & +1.1\textdegree{}C since 1880 & Global temperature records \\
        \textbf{Sea level} & +21 cm since 1900 & Satellite measurements \\
        \textbf{Arctic ice} & -13\% per decade & Satellite imagery \\
        \bottomrule
    \end{tabulary}
    \end{answertable}

    \textbf{Climate change} refers to long-term shifts in global temperatures and weather patterns, primarily caused by human activities since mid-20th century.

    \textbf{Key characteristics}:
    \begin{itemize}
        \item \textbf{Temperature rise}: Global average temperature increase
        \item \textbf{Weather extremes}: More frequent hurricanes, droughts, floods
        \item \textbf{Ecosystem changes}: Species migration, habitat loss
    \end{itemize}

    \textbf{Primary cause}: Increased greenhouse gas emissions from fossil fuel burning, deforestation, industrial processes.

    \begin{mnemonicbox}TSE (Temperature-Sea level-Ecosystem)\end{mnemonicbox}
\end{solutionbox}

\questionmarks{4}{b}{4}
\textbf{Write some of measures to control global warming.}

\begin{solutionbox}
    \textbf{Answer:}

    \begin{answertable}{Global Warming Control Measures}
    \begin{tabulary}{\linewidth}{L L L}
        \toprule
        \textbf{Category} & \textbf{Measures} & \textbf{Impact} \\
        \midrule
        \textbf{Energy} & Renewable sources, efficiency & Reduce CO$_2$ emissions \\
        \textbf{Transport} & Electric vehicles, public transport & Lower fuel consumption \\
        \textbf{Industry} & Clean technology, carbon capture & Emission reduction \\
        \textbf{Individual} & Energy conservation, lifestyle changes & Cumulative effect \\
        \bottomrule
    \end{tabulary}
    \end{answertable}

    \textbf{Control measures}:

    \textbf{Government level}:
    \begin{itemize}
        \item \textbf{Policy frameworks}: Carbon pricing, emission standards
        \item \textbf{Renewable energy}: Solar, wind power promotion
        \item \textbf{Public transport}: Mass transit system development
    \end{itemize}

    \textbf{Industrial level}:
    \begin{itemize}
        \item \textbf{Clean technology}: Efficient processes, waste reduction
        \item \textbf{Carbon capture}: Storage and utilization technologies
        \item \textbf{Sustainable practices}: Green manufacturing, circular economy
    \end{itemize}

    \textbf{Individual level}:
    \begin{itemize}
        \item \textbf{Energy conservation}: LED lights, efficient appliances
        \item \textbf{Transportation}: Walking, cycling, carpooling
        \item \textbf{Lifestyle changes}: Reduced consumption, recycling
    \end{itemize}

    \begin{mnemonicbox}PRT-CCS-ECL (Policy-Renewable-Transport, Carbon-Clean-Sustainable, Energy-Communication-Lifestyle)\end{mnemonicbox}
\end{solutionbox}

\questionmarks{4}{c}{7}
\textbf{Which are some important agreements for mitigating climate change at global level?}

\begin{solutionbox}
    \textbf{Answer:}

    \begin{answertable}{Major Climate Agreements}
    \begin{tabulary}{\linewidth}{L L L}
        \toprule
        \textbf{Agreement} & \textbf{Year} & \textbf{Key Features} \\
        \midrule
        \textbf{UNFCCC} & 1992 & Framework convention \\
        \textbf{Kyoto Protocol} & 1997 & Binding emission targets \\
        \textbf{Paris Agreement} & 2015 & Global temperature limit \\
        \bottomrule
    \end{tabulary}
    \end{answertable}

    \textbf{Important global climate agreements}:

    \textbf{1. United Nations Framework Convention on Climate Change (UNFCCC) - 1992}:
    \begin{itemize}
        \item \textbf{Objective}: Stabilize greenhouse gas concentrations
        \item \textbf{Principles}: Common but differentiated responsibilities
        \item \textbf{Framework}: Foundation for future climate negotiations
    \end{itemize}

    \textbf{2. Kyoto Protocol - 1997}:
    \begin{itemize}
        \item \textbf{Binding targets}: Developed countries reduce emissions by 5.2\% (1990 levels)
        \item \textbf{Flexible mechanisms}: Emissions trading, clean development mechanism
        \item \textbf{Commitment periods}: First (2008-2012), Second (2013-2020)
    \end{itemize}

    \textbf{3. Paris Agreement - 2015}:
    \begin{itemize}
        \item \textbf{Temperature goal}: Limit global warming to well below 2\textdegree{}C, preferably 1.5\textdegree{}C
        \item \textbf{Nationally Determined Contributions (NDCs)}: Countries set own targets
        \item \textbf{Review mechanism}: Five-year assessment and enhancement cycles
        \item \textbf{Climate finance}: \$100 billion annually for developing countries
    \end{itemize}

    \textbf{4. Other significant agreements}:
    \begin{itemize}
        \item \textbf{Montreal Protocol (1987)}: Ozone layer protection, indirect climate benefits
        \item \textbf{Copenhagen Accord (2009)}: Political agreement on emission reductions
        \item \textbf{Doha Amendment (2012)}: Extended Kyoto Protocol commitments
    \end{itemize}

    \textbf{Implementation challenges}:
    \begin{itemize}
        \item \textbf{Compliance}: Voluntary vs mandatory commitments
        \item \textbf{Financing}: Adequate funding for mitigation and adaptation
        \item \textbf{Technology transfer}: Clean technology access for developing countries
        \item \textbf{Monitoring}: Transparent reporting and verification systems
    \end{itemize}

    \textbf{Recent developments}:
    \begin{itemize}
        \item \textbf{Article 6 rules}: International carbon markets under Paris Agreement
        \item \textbf{Loss and damage}: Support for climate-vulnerable countries
        \item \textbf{Net-zero commitments}: Countries pledging carbon neutrality
    \end{itemize}

    \begin{mnemonicbox}UKPOM-CDOG-TFMC (UNFCCC-Kyoto-Paris-Other-Montreal, Copenhagen-Doha-Other-Goals, Technology-Finance-Monitoring-Commitments)\end{mnemonicbox}
\end{solutionbox}

\questionmarks{5}{a}{3}
\textbf{Explain effects of ozone layer depletion.}

\begin{solutionbox}
    \textbf{Answer:}

    \begin{answertable}{Ozone Depletion Effects}
    \begin{tabulary}{\linewidth}{L L L}
        \toprule
        \textbf{Impact Area} & \textbf{Effect} & \textbf{Consequence} \\
        \midrule
        \textbf{Human health} & Increased UV-B radiation & Skin cancer, cataracts \\
        \textbf{Environment} & Ecosystem disruption & Marine food chain damage \\
        \textbf{Agriculture} & Crop damage & Reduced food production \\
        \bottomrule
    \end{tabulary}
    \end{answertable}

    \textbf{Ozone layer depletion} results in increased ultraviolet-B (UV-B) radiation reaching Earth's surface.

    \textbf{Effects}:
    \begin{itemize}
        \item \textbf{Human health}: Higher skin cancer rates, eye damage, immune system suppression
        \item \textbf{Marine ecosystems}: Phytoplankton reduction affects ocean food chains
        \item \textbf{Agricultural impact}: Reduced crop yields, plant growth inhibition
    \end{itemize}

    \textbf{Cause}: Chlorofluorocarbons (CFCs) destroy ozone molecules in stratosphere.

    \begin{mnemonicbox}HMA (Human-Marine-Agricultural)\end{mnemonicbox}
\end{solutionbox}

\questionmarks{5}{b}{4}
\textbf{Write short note on greenhouse gases.}

\begin{solutionbox}
    \textbf{Answer:}

    \begin{answertable}{Major Greenhouse Gases}
    \begin{tabulary}{\linewidth}{L L L}
        \toprule
        \textbf{Gas} & \textbf{Sources} & \textbf{Global Warming Potential} \\
        \midrule
        \textbf{CO$_2$} & Fossil fuels, deforestation & 1 (reference) \\
        \textbf{CH$_4$} & Agriculture, landfills & 25 times CO$_2$ \\
        \textbf{N$_2$O} & Fertilizers, combustion & 298 times CO$_2$ \\
        \textbf{F-gases} & Industrial processes & 1,000-20,000 times CO$_2$ \\
        \bottomrule
    \end{tabulary}
    \end{answertable}

    \textbf{Greenhouse gases} are atmospheric compounds that trap heat radiated from Earth's surface.

    \textbf{Major greenhouse gases}:
    \begin{itemize}
        \item \textbf{Carbon dioxide (CO$_2$)}: Most abundant, from fossil fuel burning
        \item \textbf{Methane (CH$_4$)}: Potent but shorter-lived, from agriculture
        \item \textbf{Nitrous oxide (N$_2$O)}: Long-lived, from fertilizers and industry
        \item \textbf{Fluorinated gases}: Very potent, from refrigeration and industrial uses
    \end{itemize}

    \textbf{Properties}: Absorb infrared radiation, transparent to visible light, varying atmospheric lifespans.

    \textbf{Global warming potential}: Measures heat-trapping capacity relative to CO$_2$ over specific time periods.

    \begin{mnemonicbox}CMNF (Carbon dioxide-Methane-Nitrous oxide-Fluorinated gases)\end{mnemonicbox}
\end{solutionbox}

\questionmarks{5}{c}{7}
\textbf{Explain Concept of 5R.}

\begin{solutionbox}
    \textbf{Answer:}

    \begin{center}
    \begin{tikzpicture}[node distance=1.5cm, auto]
        \node (a) [gtu start] {5R Concept};
        \node (b) [gtu block, below left=of a] {Refuse};
        \node (c) [gtu block, left=of b] {Reduce};
        \node (d) [gtu block, right=of b] {Reuse};
        \node (e) [gtu block, right=of d] {Repurpose};
        \node (f) [gtu block, right=of e] {Recycle};

        % Explanations
        \node (b1) [gtu block, below=of b] {Avoid\\Unnecessary};
        \node (c1) [gtu block, below=of c] {Minimize\\Consumption};
        \node (d1) [gtu block, below=of d] {Use\\Multiple Times};
        \node (e1) [gtu block, below=of e] {Find New\\Uses};
        \node (f1) [gtu block, below=of f] {Process to\\New Products};

        \draw [gtu arrow] (a) -- (b);
        \draw [gtu arrow] (a) -- (c);
        \draw [gtu arrow] (a) -- (d);
        \draw [gtu arrow] (a) -- (e);
        \draw [gtu arrow] (a) -- (f);
        
        \draw [gtu arrow] (b) -- (b1);
        \draw [gtu arrow] (c) -- (c1);
        \draw [gtu arrow] (d) -- (d1);
        \draw [gtu arrow] (e) -- (e1);
        \draw [gtu arrow] (f) -- (f1);
    \end{tikzpicture}
    \end{center}

    \textbf{5R Concept} is a waste management hierarchy that prioritizes waste prevention and resource conservation.

    \textbf{The Five R's in order of priority}:

    \textbf{1. Refuse}:
    \begin{itemize}
        \item \textbf{Definition}: Avoid accepting unnecessary items
        \item \textbf{Examples}: Single-use plastics, promotional freebies, excessive packaging
        \item \textbf{Impact}: Prevents waste generation at source
    \end{itemize}

    \textbf{2. Reduce}:
    \begin{itemize}
        \item \textbf{Definition}: Minimize consumption and waste production
        \item \textbf{Examples}: Buy only needed items, choose durable products, energy conservation
        \item \textbf{Impact}: Decreases resource extraction and waste volume
    \end{itemize}

    \textbf{3. Reuse}:
    \begin{itemize}
        \item \textbf{Definition}: Use items multiple times in their original form
        \item \textbf{Examples}: Glass jars for storage, clothing donation, furniture repurposing
        \item \textbf{Impact}: Extends product lifespan, reduces replacement needs
    \end{itemize}

    \textbf{4. Repurpose}:
    \begin{itemize}
        \item \textbf{Definition}: Find new applications for items instead of discarding
        \item \textbf{Examples}: Tire planters, bottle vases, cardboard organizers
        \item \textbf{Impact}: Creative waste diversion, artistic value addition
    \end{itemize}

    \textbf{5. Recycle}:
    \begin{itemize}
        \item \textbf{Definition}: Process waste materials into new products
        \item \textbf{Examples}: Paper recycling, metal recovery, plastic reprocessing
        \item \textbf{Impact}: Resource recovery, reduced landfill burden
    \end{itemize}

    \textbf{Benefits of 5R approach}:
    \begin{itemize}
        \item \textbf{Environmental}: Reduced pollution, resource conservation, ecosystem protection
        \item \textbf{Economic}: Cost savings, job creation in recycling industry
        \item \textbf{Social}: Community awareness, sustainable lifestyle promotion
    \end{itemize}

    \textbf{Implementation hierarchy}: Focus on refuse and reduce first (prevention), then reuse and repurpose (waste diversion), finally recycle (waste processing).

    \textbf{Challenges}: Behavioral change requirements, infrastructure development, economic incentives alignment.

    \begin{mnemonicbox}Real Recycling Requires Refusing Rubbish (Refuse-Reduce-Reuse-Repurpose-Recycle)\end{mnemonicbox}
\end{solutionbox}

\questionmarks{5}{a}{3}
\textbf{Write salient features of wild life protection act, 1972.}

\begin{solutionbox}
    \textbf{Answer:}

    \begin{answertable}{Wildlife Protection Act 1972 Features}
    \begin{tabulary}{\linewidth}{L L L}
        \toprule
        \textbf{Feature} & \textbf{Description} & \textbf{Penalty} \\
        \midrule
        \textbf{Protected species} & Scheduled animals/plants & Fine + imprisonment \\
        \textbf{Hunting ban} & Prohibition of hunting & Up to 7 years jail \\
        \textbf{Trade regulation} & Wildlife product trade control & Confiscation + fine \\
        \bottomrule
    \end{tabulary}
    \end{answertable}

    \textbf{Wildlife Protection Act, 1972} provides legal framework for conservation of wildlife in India.

    \textbf{Salient features}:
    \begin{itemize}
        \item \textbf{Species protection}: Six schedules categorizing species by protection level
        \item \textbf{Hunting prohibition}: Complete ban on hunting of protected species
        \item \textbf{Habitat conservation}: Protected areas designation and management
        \item \textbf{Trade control}: Regulation of wildlife product commerce
    \end{itemize}

    \textbf{Enforcement}: Wildlife Crime Control Bureau, forest departments, special courts for wildlife offenses.

    \textbf{Amendments}: Regular updates to include new species and strengthen provisions.

    \begin{mnemonicbox}SHTE (Species-Hunting-Trade-Enforcement)\end{mnemonicbox}
\end{solutionbox}

\questionmarks{5}{b}{4}
\textbf{Which are the environmental policies in India?}

\begin{solutionbox}
    \textbf{Answer:}

    \begin{answertable}{Major Environmental Policies in India}
    \begin{tabulary}{\linewidth}{L L L}
        \toprule
        \textbf{Policy} & \textbf{Year} & \textbf{Focus Area} \\
        \midrule
        \textbf{National Environment Policy} & 2006 & Comprehensive framework \\
        \textbf{National Water Policy} & 2012 & Water resource management \\
        \textbf{National Forest Policy} & 1988 & Forest conservation \\
        \textbf{National Action Plan on Climate Change} & 2008 & Climate change mitigation \\
        \bottomrule
    \end{tabulary}
    \end{answertable}

    \textbf{Major environmental policies}:

    \textbf{National Environment Policy (2006)}:
    \begin{itemize}
        \item \textbf{Objective}: Sustainable development with environmental protection
        \item \textbf{Principles}: Polluter pays, precautionary approach
        \item \textbf{Implementation}: Integration across sectors
    \end{itemize}

    \textbf{Sectoral policies}:
    \begin{itemize}
        \item \textbf{National Water Policy}: Integrated water resource management
        \item \textbf{National Forest Policy}: 33\% forest cover target
        \item \textbf{National Solar Mission}: Renewable energy promotion
        \item \textbf{Waste Management Rules}: Solid waste, e-waste, plastic waste management
    \end{itemize}

    \textbf{Regulatory framework}: Environment Protection Act, Water Act, Air Act, Forest Conservation Act.

    \begin{mnemonicbox}NWFS (National-Water-Forest-Solar)\end{mnemonicbox}
\end{solutionbox}

\questionmarks{5}{c}{7}
\textbf{Explain rainwater harvesting in detail.}

\begin{solutionbox}
    \textbf{Answer:}

    \begin{center}
    \begin{tikzpicture}[node distance=1.5cm, auto]
        \node (rain) [gtu start] {Rainfall};
        \node (catch) [gtu block, right=of rain] {Catchment Area};
        \node (coll) [gtu block, right=of catch] {Collection System};
        \node (flush) [gtu block, right=of coll] {First Flush\\Diverter};
        \node (filt) [gtu block, below=of flush] {Filtration};
        \node (tank) [gtu block, left=of filt] {Storage Tank};
        \node (dist) [gtu block, left=of tank] {Distribution};
        
        \node (recharge) [gtu block, below=of coll] {Recharge Pit};
        \node (ground) [gtu block, below=of recharge] {Groundwater};

        \draw [gtu arrow] (rain) -- (catch);
        \draw [gtu arrow] (catch) -- (coll);
        \draw [gtu arrow] (coll) -- (flush);
        \draw [gtu arrow] (flush) -- (filt);
        \draw [gtu arrow] (filt) -- (tank);
        \draw [gtu arrow] (tank) -- (dist);
        
        \draw [gtu arrow] (coll) -- (recharge);
        \draw [gtu arrow] (recharge) -- (ground);
    \end{tikzpicture}
    \end{center}

    \textbf{Rainwater harvesting} is the collection, storage, and utilization of rainwater for beneficial purposes.

    \textbf{Components of rainwater harvesting system}:

    \textbf{1. Catchment area}:
    \begin{itemize}
        \item \textbf{Function}: Surface for rain collection (rooftops, open areas)
        \item \textbf{Material}: Should be clean, non-toxic (avoid asbestos, lead-painted surfaces)
        \item \textbf{Calculation}: Collection = Catchment area $\times$ Rainfall $\times$ Runoff coefficient
    \end{itemize}

    \textbf{2. Collection and conveyance system}:
    \begin{itemize}
        \item \textbf{Gutters}: Channel water from catchment surface
        \item \textbf{Downspouts}: Vertical pipes carrying water from gutters
        \item \textbf{Transportation}: Pipes connecting different components
    \end{itemize}

    \textbf{3. First flush diverter}:
    \begin{itemize}
        \item \textbf{Purpose}: Removes initial dirty water containing debris
        \item \textbf{Types}: Manual valve, automatic diverter, floating ball system
        \item \textbf{Capacity}: Usually 10-15 liters per 100 sq.m of roof area
    \end{itemize}

    \textbf{4. Filtration system}:
    \begin{itemize}
        \item \textbf{Coarse filter}: Removes leaves, debris (mesh screen)
        \item \textbf{Fine filter}: Sand, gravel, activated carbon
        \item \textbf{Slow sand filter}: Biological treatment for drinking water
    \end{itemize}

    \textbf{5. Storage system}:
    \begin{itemize}
        \item \textbf{Surface storage}: Tanks, reservoirs above ground
        \item \textbf{Underground storage}: Sumps, cisterns below ground
        \item \textbf{Material}: Ferrocement, plastic, concrete, fiberglass
    \end{itemize}

    \textbf{Types of rainwater harvesting}:

    \textbf{A. Rooftop harvesting}:
    \begin{itemize}
        \item \textbf{Direct storage}: Rainwater stored in tanks for immediate use
        \item \textbf{Indirect recharge}: Water directed to recharge groundwater
    \end{itemize}

    \textbf{B. Surface water harvesting}:
    \begin{itemize}
        \item \textbf{Check dams}: Small barriers across streams
        \item \textbf{Percolation tanks}: Artificial recharge structures
        \item \textbf{Contour bunding}: Soil conservation with water harvesting
    \end{itemize}

    \textbf{Benefits}:
    \begin{itemize}
        \item \textbf{Water security}: Reduces dependence on external water sources
        \item \textbf{Groundwater recharge}: Prevents water table decline
        \item \textbf{Flood control}: Reduces surface runoff and urban flooding
        \item \textbf{Quality improvement}: Generally better than groundwater in polluted areas
        \item \textbf{Cost-effective}: Lower than water supply schemes
        \item \textbf{Energy saving}: Reduces pumping requirements
    \end{itemize}

    \textbf{Design considerations}:
    \begin{itemize}
        \item \textbf{Rainfall pattern}: Seasonal distribution, intensity
        \item \textbf{Water demand}: Household requirements, usage patterns
        \item \textbf{Storage capacity}: Based on dry period duration
        \item \textbf{Quality requirements}: Potable vs non-potable use
        \item \textbf{Site conditions}: Space availability, soil permeability
    \end{itemize}

    \textbf{Maintenance requirements}:
    \begin{itemize}
        \item \textbf{Regular cleaning}: Gutters, filters, storage tanks
        \item \textbf{Roof maintenance}: Prevent contamination sources
        \item \textbf{System inspection}: Check for leaks, blockages
        \item \textbf{Water quality testing}: Periodic analysis for potable use
    \end{itemize}

    \textbf{Government initiatives}:
    \begin{itemize}
        \item \textbf{Building codes}: Mandatory rainwater harvesting in new constructions
        \item \textbf{Subsidies}: Financial incentives for installation
        \item \textbf{Awareness programs}: Community education and training
        \item \textbf{Technical support}: Design guidelines, implementation assistance
    \end{itemize}

    \textbf{Challenges}:
    \begin{itemize}
        \item \textbf{Initial cost}: Setup expenses for complete system
        \item \textbf{Maintenance}: Regular upkeep requirements
        \item \textbf{Space requirements}: Storage tank space needs
        \item \textbf{Seasonal availability}: Dependence on monsoon patterns
        \item \textbf{Quality concerns}: Potential contamination issues
    \end{itemize}

    \textbf{Calculation example}:
    \begin{itemize}
        \item Roof area: 100 sq.m
        \item Annual rainfall: 1000 mm
        \item Runoff coefficient: 0.8
        \item Harvestable water = 100 $\times$ 1 $\times$ 0.8 = 80,000 liters/year
    \end{itemize}

    \begin{mnemonicbox}CCFFS-RSBD-WGFQC-RCSMQ (Catchment-Collection-Flush-Filter-Storage, Rooftop-Surface-Benefits-Design, Water-Groundwater-Flood-Quality-Cost, Regular-Check-System-Maintenance-Quality)\end{mnemonicbox}
\end{solutionbox}

\end{document}
