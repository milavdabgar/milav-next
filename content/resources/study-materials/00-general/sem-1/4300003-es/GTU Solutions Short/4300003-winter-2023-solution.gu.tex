\documentclass{article}

% content/resources/templates/preamble.tex
\usepackage[margin=0.6in]{geometry}
\author{Milav Dabgar}
\usepackage{amsmath,amssymb,amsthm}
\usepackage{booktabs}
\usepackage{multirow}
\usepackage{xcolor}
\usepackage{tcolorbox}
\tcbuselibrary{breakable,skins}
\usepackage[colorlinks=true,linkcolor=blue]{hyperref}
\usepackage{titlesec}
\usepackage{enumitem}
\usepackage{tikz}
\usepackage{pgfplots}
\usepackage{circuitikz}
\usepackage[version=4]{mhchem}
\usepackage{longtable}
\usepackage{array}
\usepackage{float}
\usepackage{caption}
\usepackage{listings}

\lstset{
  basicstyle=\small\ttfamily,
  breaklines=true,
  breakatwhitespace=false,
  postbreak=\mbox{\textcolor{red}{$\hookrightarrow$}\space},
  float=false,
  numbers=left,
  numberstyle=\tiny\color{gray},
  numbersep=10pt,
  xleftmargin=2em,
  keywordstyle=\color{blue},
  commentstyle=\color{green!60!black},
  stringstyle=\color{purple},
  backgroundcolor=\color{gray!5},
  showstringspaces=false,
  tabsize=2,
  captionpos=b,
  keepspaces=true,
  columns=flexible
}

\pgfplotsset{compat=1.18}
\usetikzlibrary{shapes,arrows,positioning,calc,patterns,decorations.pathmorphing,decorations.markings,arrows.meta}

% Color scheme
\definecolor{headcolor}{RGB}{0,102,204}
\definecolor{keycolor}{RGB}{220,20,60}
\definecolor{solutioncolor}{RGB}{34,139,34}
\definecolor{mnemoniccolor}{RGB}{148,0,211}
\definecolor{codecolor}{RGB}{0,0,100}

% Spacing
\setlength{\parskip}{3pt}
\setlist[itemize]{nosep}
\setlist[enumerate]{nosep}

% Title formatting
\titleformat{\section}{\Large\bfseries\color{headcolor}}{\thesection}{1em}{}
\titleformat{\subsection}{\large\bfseries\color{headcolor}}{\thesubsection}{1em}{}

% Pandoc tightlist compatibility
\providecommand{\tightlist}{%
  \setlength{\itemsep}{0pt}\setlength{\parskip}{0pt}}

% Pandoc longtable compatibility
\newcounter{none}
\def\thenone{}


% content/resources/templates/gujarati-boxes.tex
\usepackage{fontspec}
\usepackage{polyglossia}

% Set Gujarati as main language (document is primarily in Gujarati)
% Note: gloss-gujarati.ldf doesn't exist in polyglossia, but it will use hyphenation patterns
\setdefaultlanguage{gujarati}
\setotherlanguage{english}

% Configure Gujarati font properly
% Use Language=Default to prevent polyglossia from trying to add language-specific features
% that don't exist for Gujarati, which causes "empty feature" warnings
\newfontfamily\gujaratifont[Script=Gujarati,AutoFakeBold=2.5,AutoFakeSlant=0.3]{Noto Sans Gujarati}
\setmainfont[Script=Gujarati,AutoFakeBold=2.5,AutoFakeSlant=0.3]{Noto Sans Gujarati}
% Use Noto Sans Gujarati for monospace to support Gujarati in text
\setmonofont[Scale=0.9]{Noto Sans Gujarati}

% Configure English to use the same font
\newfontfamily\englishfont[Script=Gujarati,AutoFakeBold=2.5,AutoFakeSlant=0.3]{Noto Sans Gujarati}

% Translations for polyglossia
\gappto\captionsgujarati{
  \renewcommand{\tablename}{કોષ્ટક}
  \renewcommand{\figurename}{આકૃતિ}
}

% Helper for TikZ nodes to ensure Gujarati font
\newcommand{\gu}[1]{{\gujaratifont #1}}

% Custom environments
\newtcolorbox{solutionbox}{
    breakable,
    enhanced,
    colback=solutioncolor!5!white,
    colframe=solutioncolor!75!black,
    fonttitle=\bfseries,
    title=જવાબ
}

\newtcolorbox{solutionboxnobreak}{
 colback=solutioncolor!5!white,
 colframe=solutioncolor!75!black,
 fonttitle=\bfseries,
 title=જવાબ
}

\newtcolorbox{keyformula}{
 breakable,
 enhanced,
 colback=keycolor!5!white,
 colframe=keycolor!75!black,
 fonttitle=\bfseries,
 title=રાસાયણિક સમીકરણ/સૂત્ર
}

\newtcolorbox{mnemonicbox}{
 breakable,
 enhanced,
 colback=mnemoniccolor!5!white,
 colframe=mnemoniccolor!75!black,
 fonttitle=\bfseries,
 title=મેમરી ટ્રીક
}


% Custom commands for GTU solutions
% This file defines semantic commands for consistent formatting

% Question command with automatic formatting
\newcommand{\question}[2]{%
  \section*{Question #1}%
  \textbf{#2}%
}

% OR question variant
\newcommand{\questionor}[2]{%
  \section*{Question #1 OR}%
  \textbf{#2}%
}

% Proper table environment with caption
\newenvironment{answertable}[1]{%
  \begin{table}[htbp]
  \centering
  \caption{#1}
}{%
  \end{table}
}

% Proper figure environment for diagrams
\newenvironment{answerdiagram}[1]{%
  \begin{figure}[htbp]
  \centering
  \caption{#1}
}{%
  \end{figure}
}

% Semantic markup for key terms
\newcommand{\keyword}[1]{\textbf{#1}}
\newcommand{\code}[1]{\texttt{#1}}
\newcommand{\classname}[1]{\texttt{#1}}
\newcommand{\methodname}[1]{\texttt{#1}}

% Proper quotation marks
\newcommand{\mnemonic}[1]{``#1''}


\title{પર્યાવરણ અને સ્થિરતા (4300003) - શિયાળુ 2023 ઉકેલ}
\date{જાન્યુઆરી 11, 2024}

\begin{document}
\maketitle

\questionmarks{1(અ)}{3}{ઇકોલોજીકલ ફૂટપ્રિન્ટ સમજાવો.}

\begin{solutionbox}
ઇકોલોજીકલ ફૂટપ્રિન્ટ એ વ્યક્તિઓ, સમુદાયો અથવા દેશો દ્વારા પ્રકૃતિ પરની માંગને જૈવિક રીતે ઉત્પાદક જમીન અને પાણીના વિસ્તારના સંદર્ભમાં માપે છે.

\begin{center}
\captionof{table}{ઇકોલોજીકલ ફૂટપ્રિન્ટના ઘટકો}
\begin{tabulary}{\linewidth}{|L|L|}
\hline
\textbf{ઘટક} & \textbf{વર્ણન} \\ \hline
\textbf{કાર્બન ફૂટપ્રિન્ટ} & CO\textsubscript{2} ઉત્સર્જન શોષવા માટે જરૂરી જમીન \\ \hline
\textbf{કૃષિ જમીન} & ખોરાક ઉત્પાદન માટે વિસ્તાર \\ \hline
\textbf{ચરાઈ જમીન} & પશુધન માટે વિસ્તાર \\ \hline
\textbf{વન ઉત્પાદનો} & લાકડા અને કાગળ માટે વિસ્તાર \\ \hline
\textbf{નિર્મિત જમીન} & આધારભૂત સુવિધાઓ અને શહેરી વિસ્તારો \\ \hline
\end{tabulary}
\end{center}

\begin{itemize}
    \item \keyword{વૈશ્વિક હેક્ટર}: માપન માટે માનક એકમ
    \item \keyword{ઓવરશૂટ}: જ્યારે ફૂટપ્રિન્ટ બાયોકેપેસિટી કરતાં વધે
    \item \keyword{ટકાઉપણું}: વપરાશ અને પુનઃઉત્પાદન વચ્ચે સંતુલન
\end{itemize}
\end{solutionbox}

\begin{mnemonicbox}
\mnemonic{CGFBB: Carbon, Cropland, Grazing, Forest, Built-up}
\end{mnemonicbox}

\questionmarks{1(બ)}{4}{એલ્ટોનિયન પિરામિડ સમજાવો.}

\begin{solutionbox}
એલ્ટોનિયન પિરામિડ (સંખ્યાનો પિરામિડ) ઇકોસિસ્ટમમાં દરેક પોષક સ્તરે જીવોની સંખ્યા દર્શાવે છે, જે ચાર્લ્સ એલ્ટન દ્વારા પ્રસ્તાવિત કરવામાં આવ્યો હતો.

\begin{center}
\begin{tikzpicture}[node distance=0cm, outer sep=0pt]
    \node [draw, trapezium, trapezium angle=70, minimum width=2cm, minimum height=1cm, fill=red!10, align=center] (T) at (0,3) {Tertiary Consumers\\(થોડા - 10)};
    \node [draw, trapezium, trapezium angle=70, minimum width=4cm, minimum height=1cm, fill=orange!10, align=center] (S) at (0,2) {Secondary Consumers\\(મધ્યમ - 100)};
    \node [draw, trapezium, trapezium angle=70, minimum width=6cm, minimum height=1cm, fill=yellow!10, align=center] (P) at (0,1) {Primary Consumers\\(ઘણા - 1000)};
    \node [draw, trapezium, trapezium angle=70, minimum width=8cm, minimum height=1cm, fill=green!10, align=center] (Pr) at (0,0) {Producers\\(સૌથી વધુ - 10000)};
\end{tikzpicture}
\captionof{figure}{એલ્ટોનિયન સંખ્યાનો પિરામિડ}
\end{center}

\begin{center}
\captionof{table}{પિરામિડના પ્રકારો}
\begin{tabulary}{\linewidth}{|L|L|L|}
\hline
\textbf{પ્રકાર} & \textbf{આધાર} & \textbf{આકાર} \\ \hline
\textbf{સંખ્યા} & વ્યક્તિગત ગણતરી & સામાન્ય રીતે સીધો \\ \hline
\textbf{બાયોમાસ} & કુલ વજન & ઊંધો પણ હોઈ શકે \\ \hline
\textbf{ઊર્જા} & ઊર્જા પ્રવાહ & હંમેશા સીધો \\ \hline
\end{tabulary}
\end{center}

\begin{itemize}
    \item \keyword{પોષક સ્તરો}: ખોરાક શૃંખલામાં ખોરાકની સ્થિતિ
    \item \keyword{10\% નિયમ}: માત્ર 10\% ઊર્જા આગલા સ્તરે સ્થાનાંતરિત થાય
    \item \keyword{અપવાદો}: વૃક્ષ ઇકોસિસ્ટમ ઊંધો સંખ્યા પિરામિડ દર્શાવે
\end{itemize}
\end{solutionbox}

\begin{mnemonicbox}
\mnemonic{ELTON: Energy Loss Through Organism Numbers}
\end{mnemonicbox}

\questionmarks{1(ક)}{7}{ઇકો-સિસ્ટમ તેના વર્ગીકરણ અને ઘટક સાથે સમજાવો.}

\begin{solutionbox}
ઇકોસિસ્ટમ એ પ્રકૃતિની એક કાર્યાત્મક એકમ છે જ્યાં જીવંત સજીવો એકબીજા સાથે અને તેમના ભૌતિક વાતાવરણ સાથે ક્રિયાપ્રતિક્રિયા કરે છે, જેમાં ઊર્જા પ્રવાહ અને પોષક ચક્રણ સામેલ છે.

\begin{center}
\captionof{table}{ઇકોસિસ્ટમના ઘટકો}
\begin{tabulary}{\linewidth}{|L|L|L|}
\hline
\textbf{ઘટક} & \textbf{પ્રકાર} & \textbf{ઉદાહરણો} \\ \hline
\textbf{અજૈવિક} & નિર્જીવ & હવા, પાણી, માટી, આબોહવા \\ \hline
\textbf{જૈવિક} & સજીવ & છોડ, પ્રાણીઓ, સૂક્ષ્મજીવો \\ \hline
\textbf{ઉત્પાદકો} & સ્વપોષક & લીલા છોડ, શેવાળ \\ \hline
\textbf{ઉપભોક્તાઓ} & પરપોષક & શાકાહારી, માંસાહારી, સર્વાહારી \\ \hline
\textbf{વિઘટનકર્તા} & પુનર્ચક્રીકરણકર્તા & બેક્ટેરિયા, ફૂગ \\ \hline
\end{tabulary}
\end{center}

\textbf{ઇકોસિસ્ટમનું વર્ગીકરણ:}

\textbf{કુદરતી ઇકોસિસ્ટમ:}
\begin{itemize}
    \item \keyword{સ્થલીય}: જંગલ, ઘાસના મેદાનો, રણ
    \item \keyword{જળીય}: તાજા પાણી (તળાવ, નદી), દરિયાઈ (મહાસાગર, સમુદ્ર)
\end{itemize}

\textbf{કૃત્રિમ ઇકોસિસ્ટમ:}
\begin{itemize}
    \item \keyword{કૃષિ}: પાકના ખેતરો, બગીચાઓ
    \item \keyword{શહેરી}: ઉદ્યાનો, કૃત્રિમ તળાવો
\end{itemize}

\begin{center}
\begin{tikzpicture}[node distance=1.5cm, auto]
    \node [gtu block] (Sun) {સૂર્ય};
    \node [gtu block, right=1.5cm of Sun] (Prod) {ઉત્પાદકો};
    \node [gtu block, right=1.5cm of Prod] (PC) {પ્રાથમિક\\ઉપભોક્તાઓ};
    \node [gtu block, right=1.5cm of PC] (SC) {ગૌણ\\ઉપભોક્તાઓ};
    \node [gtu block, right=1.5cm of SC] (TC) {તૃતીયક\\ઉપભોક્તાઓ};
    \node [gtu block, below=1.5cm of PC] (Dec) {વિઘટનકર્તા};

    \path [gtu arrow] (Sun) -- (Prod);
    \path [gtu arrow] (Prod) -- (PC);
    \path [gtu arrow] (PC) -- (SC);
    \path [gtu arrow] (SC) -- (TC);
    \path [gtu arrow] (Prod) -- (Dec);
    \path [gtu arrow] (PC) -- (Dec);
    \path [gtu arrow] (SC) -- (Dec);
    \path [gtu arrow] (TC) -- (Dec);
    \path [gtu arrow] (Dec) -| (Prod);
\end{tikzpicture}
\captionof{figure}{ઇકોસિસ્ટમમાં ઊર્જા પ્રવાહ}
\end{center}

\begin{itemize}
    \item \keyword{ઊર્જા પ્રવાહ}: સૂર્યથી વિઘટનકર્તા સુધી એક દિશામાં
    \item \keyword{પોષક ચક્રણ}: તત્વોની ચક્રીય હિલચાલ
    \item \keyword{ખોરાક શૃંખલા}: રેખીય ઊર્જા સ્થાનાંતરણ
    \item \keyword{ખોરાક જાળ}: પરસ્પર જોડાયેલી ખોરાક શૃંખલાઓ
\end{itemize}
\end{solutionbox}

\begin{mnemonicbox}
\mnemonic{PEACE: Producers, Energy, Animals, Cycles, Environment}
\end{mnemonicbox}

\questionmarks{1(ક અથવા)}{7}{નાઈટ્રોજન ચક્ર સમજાવો.}

\begin{solutionbox}
નાઈટ્રોજન ચક્ર એ બાયોજિયોકેમિકલ ચક્ર છે જે વાતાવરણ, સ્થલીય અને જળીય પ્રણાલીઓમાં ફરતા વખતે નાઈટ્રોજન સંયોજનોને વિવિધ રાસાયણિક સ્વરૂપોમાં રૂપાંતરિત કરે છે.

\begin{center}
\begin{tikzpicture}[node distance=2cm, auto]
    \node [gtu block] (N2) {વાતાવરણીય N\textsubscript{2}};
    \node [gtu block, below right=2cm of N2] (NH3) {અમોનિયા NH\textsubscript{3}};
    \node [gtu block, below=1.5cm of NH3] (NO2) {નાઈટ્રાઈટ NO\textsubscript{2}\textsuperscript{-}};
    \node [gtu block, left=2cm of NO2] (NO3) {નાઈટ્રેટ NO\textsubscript{3}\textsuperscript{-}};
    \node [gtu block, above left=1.5cm of NO3] (Bio) {વનસ્પતિ અને\\પ્રાણી બાયોમાસ};
    
    % Fixation
    \path [gtu arrow] (N2) -- node[midway, right] {સ્થિરીકરણ} (NH3);
    % Nitrification
    \path [gtu arrow] (NH3) -- node[midway, right] {નાઈટ્રિફિકેશન} (NO2);
    \path [gtu arrow] (NO2) -- (NO3);
    % Assimilation
    \path [gtu arrow] (NO3) -- node[midway, left] {આત્મસાત્કરણ} (Bio);
    % Decomposition
    \path [gtu arrow] (Bio) -- node[midway, above] {વિઘટન} (NH3);
    % Denitrification
    \path [gtu arrow] (NO3) -- node[midway, left] {ડી-નાઈટ્રિફિકેશન} (N2);
\end{tikzpicture}
\captionof{figure}{નાઈટ્રોજન ચક્ર}
\end{center}

\begin{center}
\captionof{table}{નાઈટ્રોજન ચક્રની પ્રક્રિયાઓ}
\begin{tabulary}{\linewidth}{|L|L|L|}
\hline
\textbf{પ્રક્રિયા} & \textbf{રૂપાંતરણ} & \textbf{સજીવો} \\ \hline
\textbf{સ્થિરીકરણ} & N\textsubscript{2} $\rightarrow$ NH\textsubscript{3} & રાઈઝોબિયમ, એઝોટોબેક્ટર \\ \hline
\textbf{નાઈટ્રિફિકેશન} & NH\textsubscript{3} $\rightarrow$ NO\textsubscript{2}\textsuperscript{-} $\rightarrow$ NO\textsubscript{3}\textsuperscript{-} & નાઈટ્રોસોમોનાસ, નાઈટ્રોબેક્ટર \\ \hline
\textbf{આત્મસાત્કરણ} & NO\textsubscript{3}\textsuperscript{-} $\rightarrow$ પ્રોટીન & છોડવા \\ \hline
\textbf{વિઘટન} & પ્રોટીન $\rightarrow$ NH\textsubscript{3} & બેક્ટેરિયા, ફૂગ \\ \hline
\textbf{ડી-નાઈટ્રિફિકેશન} & NO\textsubscript{3}\textsuperscript{-} $\rightarrow$ N\textsubscript{2} & એનેરોબિક બેક્ટેરિયા \\ \hline
\end{tabulary}
\end{center}

\begin{itemize}
    \item \keyword{જૈવિક સ્થિરીકરણ}: કુલ સ્થિરીકરણનો 80\%
    \item \keyword{ઔદ્યોગિક સ્થિરીકરણ}: ખાતર માટે હેબર પ્રક્રિયા
    \item \keyword{વીજળી}: કુદરતી વાતાવરણીય સ્થિરીકરણ
    \item \keyword{પ્રદૂષણ}: વધારાના નાઈટ્રેટ યુટ્રોફિકેશન કારણે
\end{itemize}
\end{solutionbox}

\begin{mnemonicbox}
\mnemonic{FNADD: Fixation, Nitrification, Assimilation, Decomposition, Denitrification}
\end{mnemonicbox}

\questionmarks{2(અ)}{3}{વેસ્ટ વોટર ક્વોલિટી પેરામીટરની યાદી બનાવો.}

\begin{solutionbox}
\begin{center}
\captionof{table}{વેસ્ટ વોટર ક્વોલિટી પેરામીટર}
\begin{tabulary}{\linewidth}{|L|L|L|}
\hline
\textbf{ભૌતિક} & \textbf{રાસાયણિક} & \textbf{જૈવિક} \\ \hline
\textbf{ટર્બિડિટી} & \textbf{BOD} & \textbf{કોલિફોર્મ ગણતરી} \\ \hline
\textbf{રંગ} & \textbf{COD} & \textbf{પેથોજેનિક બેક્ટેરિયા} \\ \hline
\textbf{ગંધ} & \textbf{pH} & \textbf{શેવાળ} \\ \hline
\textbf{તાપમાન} & \textbf{DO} & \textbf{વાયરસ} \\ \hline
\textbf{કુલ ઘન પદાર્થો} & \textbf{અમોનિયા} & \textbf{પ્રોટોઝોઆ} \\ \hline
\end{tabulary}
\end{center}

\begin{itemize}
    \item \keyword{પ્રાથમિક પેરામીટર}: BOD, COD, pH, સસ્પેન્ડેડ સોલિડ્સ
    \item \keyword{ગૌણ પેરામીટર}: ભારે ધાતુઓ, પોષક તત્વો
    \item \keyword{સૂચક સજીવો}: મળના દૂષણ માટે E.coli
\end{itemize}
\end{solutionbox}

\begin{mnemonicbox}
\mnemonic{PCB: Physical, Chemical, Biological parameters}
\end{mnemonicbox}

\questionmarks{2(બ)}{4}{ઈ-કચરાનું વર્ગીકરણ અને અસરો સમજાવો.}

\begin{solutionbox}
ઈલેક્ટ્રોનિક કચરો (ઈ-વેસ્ટ) એ હાનિકારક સામગ્રી ધરાવતા છોડી દેવાયેલા વિદ્યુત અને ઈલેક્ટ્રોનિક સાધનોનો સંદર્ભ આપે છે.

\begin{center}
\captionof{table}{ઈ-વેસ્ટ વર્ગીકરણ}
\begin{tabulary}{\linewidth}{|L|L|L|}
\hline
\textbf{કેટેગરી} & \textbf{ઉદાહરણો} & \textbf{હાનિકારક સામગ્રી} \\ \hline
\textbf{મોટા ઉપકરણો} & રેફ્રિજરેટર, વોશિંગ મશીન & CFCs, ભારે ધાતુઓ \\ \hline
\textbf{નાના ઉપકરણો} & માઈક્રોવેવ, ટોસ્ટર & લીડ, મર્ક્યુરી \\ \hline
\textbf{IT સાધનો} & કમ્પ્યુટર, પ્રિન્ટર & કેડમિયમ, ક્રોમિયમ \\ \hline
\textbf{ટેલિકોમ સાધનો} & મોબાઈલ ફોન, કેબલ & બેરિલિયમ, ફ્લેમ રિટાર્ડન્ટ \\ \hline
\textbf{કન્ઝ્યુમર ઈલેક્ટ્રોનિક્સ} & ટીવી, રેડિયો & પોલિવિનાઈલ ક્લોરાઈડ (PVC) \\ \hline
\end{tabulary}
\end{center}

\textbf{ઈ-વેસ્ટની અસરો:}
\begin{itemize}
    \item \keyword{પર્યાવરણીય}: માટી અને પાણીનું પ્રદૂષણ, હવાનું દૂષણ
    \item \keyword{આરોગ્ય}: કેન્સર, ન્યુરોલોજિકલ વિકાર, શ્વસન સમસ્યાઓ
    \item \keyword{સંસાધન ક્ષય}: સોના, ચાંદી જેવી મૂલ્યવાન ધાતુઓનું નુકસાન
    \item \keyword{ઇકોસિસ્ટમ નુકસાન}: ખોરાક શૃંખલામાં બાયોએક્યુમ્યુલેશન
\end{itemize}
\end{solutionbox}

\begin{mnemonicbox}
\mnemonic{LSITC: Large, Small, IT, Telecom, Consumer electronics}
\end{mnemonicbox}

\questionmarks{2(ક)}{7}{ઈલેક્ટ્રોસ્ટેટિક પ્રીસીપીટેટર સમજાવો.}

\begin{solutionbox}
ઈલેક્ટ્રોસ્ટેટિક પ્રીસીપીટેટર (ESP) એ હવા પ્રદૂષણ નિયંત્રણ ઉપકરણ છે જે વિદ્યુત ચાર્જનો ઉપયોગ કરીને ઔદ્યોગિક ગેસ પ્રવાહમાંથી કણોનો દ્રવ્ય દૂર કરે છે.

\begin{center}
\begin{tikzpicture}[node distance=1cm, auto]
    \node [gtu block, minimum width=6cm, minimum height=3cm] (Box) {};
    \node [left=1cm of Box.west] (In) {ગંદો ગેસ ઈનપુટ};
    \node [right=1cm of Box.east] (Out) {સાફ ગેસ આઉટપુટ};
    
    \node [draw, circle, fill=black, inner sep=1pt] (Elec1) at (Box.center) {};
    \node [above=0.2cm of Elec1] {+ ઇલેક્ટ્રોડ};
    
    \draw [thick] (Box.north west) -- (Box.north east);
    \draw [thick] (Box.south west) -- (Box.south east);
    
    \node [above=0.1cm of Box.south] {- કલેક્શન પ્લેટ};
    
    \node [draw, trapezium, trapezium angle=60, shape border rotate=180, minimum width=2cm, below=0cm of Box] (Hopper) {ધૂળ કલેક્શન હોપર};
    
    \draw [gtu arrow] (In) -- (Box.west);
    \draw [gtu arrow] (Box.east) -- (Out);
\end{tikzpicture}
\captionof{figure}{ESP કામગીરીનો સિદ્ધાંત}
\end{center}

\begin{center}
\captionof{table}{ESP ઘટકો અને કાર્યો}
\begin{tabulary}{\linewidth}{|L|L|L|}
\hline
\textbf{ઘટક} & \textbf{કાર્ય} & \textbf{સામગ્રી} \\ \hline
\textbf{ડિસચાર્જ ઈલેક્ટ્રોડ} & કોરોના ડિસચાર્જ બનાવે & ટંગસ્ટન વાયર \\ \hline
\textbf{કલેક્શન પ્લેટ} & ચાર્જ કરેલા કણોને આકર્ષે & સ્ટીલ પ્લેટ્સ \\ \hline
\textbf{હાઈ વોલ્ટેજ સપ્લાઈ} & 30-100 kV DC પ્રદાન કરે & ટ્રાન્સફોર્મર-રેક્ટિફાયર \\ \hline
\textbf{રેપર સિસ્ટમ} & એકત્રિત ધૂળ દૂર કરે & યાંત્રિક વાઈબ્રેટર \\ \hline
\textbf{હોપર} & પડેલા કણો એકત્રિત કરે & સ્ટીલ કન્ટેનર \\ \hline
\end{tabulary}
\end{center}

\textbf{કામકાજનો સિદ્ધાંત:}
\begin{enumerate}
    \item \keyword{આયનીકરણ}: હાઈ વોલ્ટેજ કોરોના ડિસચાર્જ બનાવે
    \item \keyword{ચાર્જિંગ}: કણો નકારાત્મક ચાર્જ મેળવે
    \item \keyword{કલેક્શન}: ચાર્જ કરેલા કણો સકારાત્મક પ્લેટ્સ તરફ જાય
    \item \keyword{દૂર કરવું}: રેપિંગ એકત્રિત ધૂળને છૂટી કરે
\end{enumerate}

\textbf{ઉપયોગો:}
\begin{itemize}
    \item \keyword{પાવર પ્લાન્ટ્સ}: કોલસાથી ચાલતા બોઈલર
    \item \keyword{સિમેન્ટ ઉદ્યોગ}: ભઠ્ઠાના ગેસ સફાઈ
    \item \keyword{સ્ટીલ ઉદ્યોગ}: બ્લાસ્ટ ફર્નેસ ગેસ
    \item \keyword{કેમિકલ પ્લાન્ટ્સ}: પ્રોસેસ ગેસ ટ્રીટમેન્ટ
\end{itemize}

\textbf{ફાયદાઓ:}
\begin{itemize}
    \item \keyword{ઉચ્ચ કાર્યક્ષમતા}: બારીક કણો માટે 99\%+ દૂર કરવું
    \item \keyword{ઓછું પ્રેશર ડ્રોપ}: ઊર્જા કાર્યક્ષમ કામગીરી
    \item \keyword{ઉચ્ચ તાપમાન સંભાળે}: 400\textdegree C સુધી
\end{itemize}
\end{solutionbox}

\begin{mnemonicbox}
\mnemonic{CHARGE: Corona, High-voltage, Attract, Rapper, Gas, Efficiency}
\end{mnemonicbox}

\questionmarks{2(અ અથવા)}{3}{સમજાવો (1) BOD (2) COD}

\begin{solutionbox}
\begin{center}
\captionof{table}{BOD vs COD}
\begin{tabulary}{\linewidth}{|L|L|L|}
\hline
\textbf{પેરામીટર} & \textbf{BOD} & \textbf{COD} \\ \hline
\textbf{પૂરું નામ} & બાયોકેમિકલ ઓક્સિજન ડિમાન્ડ & કેમિકલ ઓક્સિજન ડિમાન્ડ \\ \hline
\textbf{પદ્ધતિ} & જૈવિક ઓક્સિડેશન & રાસાયણિક ઓક્સિડેશન \\ \hline
\textbf{સમય} & 20\textdegree C પર 5 દિવસ & 2-3 કલાક \\ \hline
\textbf{ઓક્સિડાઈઝિંગ એજન્ટ} & સૂક્ષ્મજીવો & પોટેશિયમ ડાઈક્રોમેટ \\ \hline
\end{tabulary}
\end{center}

\textbf{(1) BOD (બાયોકેમિકલ ઓક્સિજન ડિમાન્ડ):}
\begin{itemize}
    \item \keyword{વ્યાખ્યા}: કાર્બનિક પદાર્થને વિઘટન કરવા માટે સૂક્ષ્મજીવો દ્વારા જરૂરી ઓક્સિજન
    \item \keyword{પ્રમાણભૂત પરિસ્થિતિઓ}: 5 દિવસ, 20\textdegree C, અંધકારની સ્થિતિ
    \item \keyword{એકમો}: mg/L અથવા ppm
\end{itemize}

\textbf{(2) COD (કેમિકલ ઓક્સિજન ડિમાન્ડ):}
\begin{itemize}
    \item \keyword{વ્યાખ્યા}: કાર્બનિક પદાર્થને રાસાયણિક રીતે ઓક્સિડાઈઝ કરવા માટે ઓક્સિજન સમકક્ષ
    \item \keyword{ઓક્સિડાઈઝિંગ એજન્ટ}: અમ્લીય માધ્યમમાં K\textsubscript{2}Cr\textsubscript{2}O\textsubscript{7}
    \item \keyword{BOD કરતાં ઊંચું}: બિન-બાયોડિગ્રેડેબલ સંયોજનો સામેલ
\end{itemize}
\end{solutionbox}

\begin{mnemonicbox}
\mnemonic{BTCO: Biological Time, Chemical Oxidation}
\end{mnemonicbox}

\questionmarks{2(બ અથવા)}{4}{ઇ-કચરાનું રિસાયકલ સમજાવો.}

\begin{solutionbox}
ઇ-વેસ્ટ રિસાયક્લિંગ એ હાનિકારક પદાર્થોના સુરક્ષિત નિકાલ સાથે ઇલેક્ટ્રોનિક કચરામાંથી મૂલ્યવાન સામગ્રી પુનઃપ્રાપ્ત કરવાની પ્રક્રિયા છે.

\begin{center}
\captionof{table}{ઇ-વેસ્ટ રિસાયક્લિંગ પ્રક્રિયા}
\begin{tabulary}{\linewidth}{|L|L|L|}
\hline
\textbf{તબક્કો} & \textbf{પ્રક્રિયા} & \textbf{પુનઃપ્રાપ્તિ} \\ \hline
\textbf{કલેક્શન} & ઘરો, ઓફિસોમાંથી એકત્રીકરણ & સંપૂર્ણ ઉપકરણો \\ \hline
\textbf{ડિસમેન્ટલિંગ} & ઘટકોનું મેન્યુઅલ વિભાજન & પ્લાસ્ટિક, ધાતુઓ, સર્કિટ બોર્ડ \\ \hline
\textbf{શ્રેડિંગ} & યાંત્રિક કદ ઘટાડો & મિશ્ર સામગ્રી પ્રવાહ \\ \hline
\textbf{વિભાજન} & ચુંબકીય, ઘનતા, ઓપ્ટિકલ સોર્ટિંગ & ફેરસ, નોન-ફેરસ ધાતુઓ \\ \hline
\textbf{શુદ્ધિકરણ} & રાસાયણિક પ્રક્રિયા & શુદ્ધ ધાતુઓ (Au, Ag, Cu, Pd) \\ \hline
\end{tabulary}
\end{center}

\textbf{રિસાયક્લિંગ પદ્ધતિઓ:}
\begin{itemize}
    \item \keyword{યાંત્રિક}: ભૌતિક વિભાજન અને કદ ઘટાડો
    \item \keyword{પાયરોમેટલર્જી}: ઉચ્ચ તાપમાન ધાતુ પુનઃપ્રાપ્તિ
    \item \keyword{હાઇડ્રોમેટલર્જી}: રાસાયણિક લીચિંગ પ્રક્રિયાઓ
    \item \keyword{બાયોટેકનોલોજી}: સૂક્ષ્મજીવીય ધાતુ નિષ્કર્ષણ
\end{itemize}

\textbf{ફાયદાઓ:}
\begin{itemize}
    \item \keyword{સંસાધન સંરક્ષણ}: કિંમતી ધાતુઓની પુનઃપ્રાપ્તિ
    \item \keyword{પર્યાવરણ સંરક્ષણ}: માટી અને પાણીનું દૂષણ અટકાવે
    \item \keyword{આર્થિક મૂલ્ય}: નોકરીઓ સર્જન અને આવક ઉત્પાદન
    \item \keyword{ઊર્જા બચત}: પ્રાથમિક ઉત્પાદન કરતાં ઓછી ઊર્જા
\end{itemize}
\end{solutionbox}

\begin{mnemonicbox}
\mnemonic{CDSPR: Collection, Dismantling, Shredding, Separation, Refining}
\end{mnemonicbox}

\questionmarks{2(ક અથવા)}{7}{પ્રદૂષણ અને તેના સ્ત્રોતને વ્યાખ્યાયિત કરો. પ્રદૂષકોનું વર્ગીકરણ સમજાવો.}

\begin{solutionbox}
\textbf{વ્યાખ્યા:} પ્રદૂષણ એ પર્યાવરણમાં હાનિકારક પદાર્થો અથવા ઊર્જાનો પ્રવેશ છે, જે હવા, પાણી, માટી અથવા સજીવોમાં પ્રતિકૂળ ફેરફારોનું કારણ બને છે.

\begin{center}
\captionof{table}{પ્રદૂષણના સ્ત્રોતો}
\begin{tabulary}{\linewidth}{|L|L|L|}
\hline
\textbf{સ્ત્રોત પ્રકાર} & \textbf{ઉદાહરણો} & \textbf{બહાર પાડવામાં આવતા પ્રદૂષકો} \\ \hline
\textbf{પોઈન્ટ સોર્સ} & ઔદ્યોગિક ચીમની, ગટર આઉટફોલ & ચોક્કસ સ્થાન ડિસચાર્જ \\ \hline
\textbf{નોન-પોઈન્ટ સોર્સ} & કૃષિ રનઓફ, શહેરી વરસાદી પાણી & ફેલાયેલા વિસ્તારનું પ્રદૂષણ \\ \hline
\textbf{મોબાઈલ સોર્સ} & વાહનો, જહાજો, વિમાનો & એક્ઝોસ્ટ એમિશન \\ \hline
\textbf{સ્ટેશનરી સોર્સ} & પાવર પ્લાન્ટ, ફેક્ટરીઓ & સ્ટેક એમિશન \\ \hline
\end{tabulary}
\end{center}

\textbf{પ્રદૂષકોનું વર્ગીકરણ:}

\textbf{1. પ્રકૃતિ અનુસાર:}
\begin{center}
\captionof{table}{પ્રકૃતિ અનુસાર પ્રદૂષક વર્ગીકરણ}
\begin{tabulary}{\linewidth}{|L|L|L|}
\hline
\textbf{પ્રકાર} & \textbf{લાક્ષણિકતાઓ} & \textbf{ઉદાહરણો} \\ \hline
\textbf{બાયોડિગ્રેડેબલ} & કુદરતી રીતે વિઘટિત થાય & કાર્બનિક કચરો, ગટરનું પાણી \\ \hline
\textbf{નોન-બાયોડિગ્રેડેબલ} & પર્યાવરણમાં ટકી રહે & પ્લાસ્ટિક, ભારે ધાતુઓ \\ \hline
\textbf{ધીમે વિઘટિત થતા} & વર્ષો સુધી વિઘટિત થાય & જંતુનાશકો, કિરણોત્સર્ગી સામગ્રી \\ \hline
\end{tabulary}
\end{center}

\textbf{2. સ્વરૂપ અનુસાર:}
\begin{itemize}
    \item \keyword{પ્રાથમિક}: સીધા ઉત્સર્જિત (SO\textsubscript{2}, CO, કણો)
    \item \keyword{ગૌણ}: પ્રતિક્રિયાઓ દ્વારા રચાય (O\textsubscript{3}, અમ્લ વરસાદ, ધુમ્મસ)
\end{itemize}

\textbf{3. સ્ત્રોત અનુસાર:}
\begin{itemize}
    \item \keyword{કુદરતી}: જ્વાળામુખી વિસ્ફોટ, જંગલની આગ
    \item \keyword{માનવજન્ય}: માનવ પ્રવૃત્તિઓ, ઔદ્યોગિક પ્રક્રિયાઓ
\end{itemize}

\begin{center}
\begin{tikzpicture}[node distance=1.5cm, auto]
    \node [gtu block] (P) {પ્રદૂષકો};
    \node [gtu block, below left=1.5cm of P] (N) {પ્રકૃતિ અનુસાર};
    \node [gtu block, below=1.5cm of P] (F) {સ્વરૂપ અનુસાર};
    \node [gtu block, below right=1.5cm of P] (S) {સ્ત્રોત અનુસાર};
    
    \node [gtu block, below left=0.5cm of N] (Bio) {બાયોડિગ્રેડેબલ};
    \node [gtu block, below right=0.5cm of N] (Non) {નોન-બાયોડિગ્રેડેબલ};
    
    \node [gtu block, below left=0.5cm of F] (Pri) {પ્રાથમિક};
    \node [gtu block, below right=0.5cm of F] (Sec) {ગૌણ};
    
    \node [gtu block, below left=0.5cm of S] (Nat) {કુદરતી};
    \node [gtu block, below right=0.5cm of S] (Anth) {માનવજન્ય};
    
    \path [gtu arrow] (P) -- (N);
    \path [gtu arrow] (P) -- (F);
    \path [gtu arrow] (P) -- (S);
    
    \path [gtu arrow] (N) -- (Bio);
    \path [gtu arrow] (N) -- (Non);
    \path [gtu arrow] (F) -- (Pri);
    \path [gtu arrow] (F) -- (Sec);
    \path [gtu arrow] (S) -- (Nat);
    \path [gtu arrow] (S) -- (Anth);
\end{tikzpicture}
\captionof{figure}{પ્રદૂષણ વર્ગીકરણ}
\end{center}

\textbf{પ્રદૂષણની અસરો:}
\begin{itemize}
    \item \keyword{પર્યાવરણીય}: ઇકોસિસ્ટમ વિક્ષેપ, પ્રજાતિઓનું લુપ્ત થવું
    \item \keyword{આરોગ્ય}: શ્વસન રોગો, કેન્સર, આનુવંશિક વિકાર
    \item \keyword{આર્થિક}: આરોગ્ય સંભાળના ખર્ચ, ઘટતી ઉત્પાદકતા
    \item \keyword{સામાજિક}: જીવનની ગુણવત્તામાં ઘટાડો
\end{itemize}
\end{solutionbox}

\begin{mnemonicbox}
\mnemonic{BNS-PFC: Biodegradable, Non-biodegradable, Slowly degradable - Primary, Form, Classification}
\end{mnemonicbox}


\questionmarks{3(અ)}{3}{સૌર કોષની કામગીરી જણાવો.}

\begin{solutionbox}
સૌર કોષ અર્ધવાહક સામગ્રીનો ઉપયોગ કરીને ફોટોવોલ્ટેઇક અસર દ્વારા પ્રકાશ ઊર્જાને સીધી વિદ્યુત ઊર્જામાં રૂપાંતરિત કરે છે.

\begin{center}
\captionof{table}{સૌર કોષની કામગીરી પ્રક્રિયા}
\begin{tabulary}{\linewidth}{|L|L|L|}
\hline
\textbf{પગલું} & \textbf{પ્રક્રિયા} & \textbf{પરિણામ} \\ \hline
\textbf{ફોટોન શોષણ} & પ્રકાશ અર્ધવાહક પર પડે & ઇલેક્ટ્રોન ઉત્તેજના \\ \hline
\textbf{ઇલેક્ટ્રોન-હોલ ઉત્પાદન} & ઊર્જા બોન્ડ તોડે & મુક્ત ચાર્જ વાહકો \\ \hline
\textbf{ચાર્જ વિભાજન} & આંતરિક વિદ્યુત ક્ષેત્ર & ઇલેક્ટ્રોન n-બાજુ, હોલ p-બાજુ \\ \hline
\textbf{કરંટ કલેક્શન} & બાહ્ય સર્કિટ જોડાણ & વિદ્યુત પ્રવાહ \\ \hline
\end{tabulary}
\end{center}

\begin{itemize}
    \item \keyword{p-n જંક્શન}: આંતરિક વિદ્યુત ક્ષેત્ર બનાવે
    \item \keyword{ડિપ્લેશન રીજન}: ચાર્જ વિભાજન સાથેનો વિસ્તાર
    \item \keyword{બાહ્ય લોડ}: વિદ્યુત સર્કિટ પૂર્ણ કરે
\end{itemize}
\end{solutionbox}

\begin{mnemonicbox}
\mnemonic{PECS: Photon, Electron, Charge, Separation}
\end{mnemonicbox}

\questionmarks{3(બ)}{4}{આડી ધરી અને ઉભી ધરી વિન્ડ મિલ્સ વચ્ચેની સરખામણી આપો.}

\begin{solutionbox}
\begin{center}
\captionof{table}{HAWT vs VAWT સરખામણી}
\begin{tabulary}{\linewidth}{|L|L|L|}
\hline
\textbf{પેરામીટર} & \textbf{આડી ધરી (HAWT)} & \textbf{ઉભી ધરી (VAWT)} \\ \hline
\textbf{બ્લેડ અભિગમ} & આડા ભ્રમણ & ઉભા ભ્રમણ \\ \hline
\textbf{પવનની દિશા} & પવનનો સામનો કરવો જોઈએ & કોઈપણ દિશાથી સ્વીકારે \\ \hline
\textbf{કાર્યક્ષમતા} & ઊંચી (35-45\%) & નીચી (20-35\%) \\ \hline
\textbf{ઊંચાઈ} & ટાવર પર માઉન્ટ, ઊંચું & જમીન સ્તરે સ્થાપના \\ \hline
\textbf{જાળવણી} & મુશ્કેલ, ઊંચી ઊંચાઈ & સરળ, જમીન સુલભ \\ \hline
\textbf{અવાજ} & મધ્યમ & ઓછો \\ \hline
\textbf{કિંમત} & પ્રારંભિક ઊંચી & ઓછી સ્થાપના \\ \hline
\textbf{પાવર આઉટપુટ} & મોટા પાયે ઊંચું & નાના પાયે યોગ્ય \\ \hline
\end{tabulary}
\end{center}

\textbf{ફાયદાઓ:}
\begin{itemize}
    \item \keyword{HAWT}: ઊંચી કાર્યક્ષમતા, સાબિત ટેકનોલોજી, બહેતર પાવર-ટુ-વેઈટ રેશિયો
    \item \keyword{VAWT}: સર્વદિશીય, સરળ જાળવણી, શાંત કામગીરી, શહેરી મિત્ર
\end{itemize}

\textbf{ઉપયોગો:}
\begin{itemize}
    \item \keyword{HAWT}: મોટા વિન્ડ ફાર્મ, યુટિલિટી-સ્કેલ પાવર જનરેશન
    \item \keyword{VAWT}: શહેરી વિસ્તારો, નાના પાયાના ઉપયોગો, વિતરિત જનરેશન
\end{itemize}
\end{solutionbox}

\begin{mnemonicbox}
\mnemonic{HEAVEN: Height, Efficiency, Accessibility, Versatility, Economics, Noise}
\end{mnemonicbox}

\questionmarks{3(ક)}{7}{બાયોગેસ પ્લાન્ટનું બાંધકામ અને કાર્ય આકૃતી સાથે સમજાવો.}

\begin{solutionbox}
બાયોગેસ પ્લાન્ટ મેથેનોજેનિક બેક્ટેરિયા દ્વારા કાર્બનિક કચરા સામગ્રીના એનેરોબિક પાચન દ્વારા મેથેન-સમૃદ્ધ ગેસ ઉત્પન્ન કરે છે.

\begin{center}
\begin{tikzpicture}[node distance=1.5cm, auto]
    \node [gtu block, minimum width=3cm, minimum height=2cm] (Digester) {ડાયજેસ્ટર};
    \node [gtu block, above=1cm of Digester] (GasHolder) {ગેસ હોલ્ડર};
    \node [left=1.5cm of Digester] (Inlet) {ફીડ ઇનલેટ};
    \node [right=1.5cm of Digester] (Outlet) {સ્લરી આઉટલેટ};
    \node [above=1cm of GasHolder] (GasOut) {ગેસ આઉટલેટ};
    
    \node [draw, dashed, fit=(Digester) (GasHolder), label=below:ભૂગર્ભ ચેમ્બર] (Chamber) {};
    
    \path [gtu arrow] (Inlet) -- (Digester);
    \path [gtu arrow] (Digester) -- (Outlet);
    \path [gtu arrow] (Digester) -- (GasHolder);
    \path [gtu arrow] (GasHolder) -- (GasOut);
\end{tikzpicture}
\captionof{figure}{બાયોગેસ પ્લાન્ટની રેખાકૃતિ}
\end{center}

\begin{center}
\captionof{table}{બાયોગેસ પ્લાન્ટના ઘટકો}
\begin{tabulary}{\linewidth}{|L|L|L|}
\hline
\textbf{ઘટક} & \textbf{કાર્ય} & \textbf{સામગ્રી} \\ \hline
\textbf{ડાયજેસ્ટર} & એનેરોબિક ફર્મેન્ટેશન ચેમ્બર & કોંક્રીટ/સ્ટીલ \\ \hline
\textbf{ગેસ હોલ્ડર} & ગેસ સ્ટોરેજ અને પ્રેશર રેગ્યુલેશન & સ્ટીલ/પ્લાસ્ટિક \\ \hline
\textbf{ઇનલેટ ચેમ્બર} & ફીડ સામગ્રી પ્રવેશ & ચણતર \\ \hline
\textbf{આઉટલેટ ચેમ્બર} & સ્લરી ડિસચાર્જ & ચણતર \\ \hline
\textbf{મિક્સિંગ ટેન્ક} & કાચી સામગ્રી તૈયારી & કોંક્રીટ \\ \hline
\end{tabulary}
\end{center}

\textbf{બાંધકામની વિગતો:}

\textbf{ભૂગર્ભ ડાયજેસ્ટર:}
\begin{itemize}
    \item \keyword{આકાર}: બેલનાકાર અથવા ગુંબજ આકાર
    \item \keyword{ક્ષમતા}: ઘરેલુ પ્લાન્ટ માટે 10-100 m\textsuperscript{3}
    \item \keyword{દિવાલની જાડાઈ}: 10-15 સેમી કોંક્રીટ
    \item \keyword{ઇન્સ્યુલેશન}: ગરમીનું નુકસાન અટકાવે
\end{itemize}

\textbf{કામકાજની પ્રક્રિયા:}

\begin{center}
\captionof{table}{બાયોગેસ ઉત્પાદનના તબક્કાઓ}
\begin{tabulary}{\linewidth}{|L|L|L|L|}
\hline
\textbf{તબક્કો} & \textbf{પ્રક્રિયા} & \textbf{અવધિ} & \textbf{ઉત્પાદનો} \\ \hline
\textbf{હાઇડ્રોલિસિસ} & મોટા અણુઓનું વિભાજન & 1-3 દિવસ & સાદી શર્કરા, એમિનો એસિડ \\ \hline
\textbf{એસિડોજેનેસિસ} & એસિડ રચના & 3-7 દિવસ & કાર્બનિક એસિડ, આલ્કોહોલ \\ \hline
\textbf{મેથેનોજેનેસિસ} & મેથેન ઉત્પાદન & 15-30 દિવસ & CH\textsubscript{4} (60\%), CO\textsubscript{2} (40\%) \\ \hline
\end{tabulary}
\end{center}

\textbf{ઓપરેટિંગ પરિસ્થિતિઓ:}
\begin{itemize}
    \item \keyword{તાપમાન}: 30-40\textdegree C (મેસોફિલિક)
    \item \keyword{pH}: 6.8-7.2 (તટસ્થ)
    \item \keyword{C:N રેશિયો}: 25-30:1 શ્રેષ્ઠ
    \item \keyword{રિટેન્શન ટાઈમ}: 20-30 દિવસ
\end{itemize}

\textbf{ઉપયોગો:}
\begin{itemize}
    \item \keyword{રસોઈ}: સ્વચ્છ બર્નિંગ ઇંધન
    \item \keyword{લાઈટિંગ}: ગેસ લેમ્પ
    \item \keyword{હીટિંગ}: સ્પેસ અને વોટર હીટિંગ
    \item \keyword{વિજળી}: જનરેટર સેટ
\end{itemize}

\textbf{ફાયદાઓ:}
\begin{itemize}
    \item \keyword{નવીકરણીય ઊર્જા}: ટકાઉ ઇંધન સ્ત્રોત
    \item \keyword{કચરા વ્યવસ્થાપન}: કાર્બનિક કચરાનો નિકાલ
    \item \keyword{ખાતર ઉત્પાદન}: પોષક તત્વોથી ભરપૂર સ્લરી
    \item \keyword{પર્યાવરણીય ફાયદાઓ}: ગ્રીનહાઉસ ગેસ ઘટાડે
\end{itemize}
\end{solutionbox}

\begin{mnemonicbox}
\mnemonic{BIGHM: Biological, Input, Gas, Holder, Methane}
\end{mnemonicbox}

\questionmarks{3(અ અથવા)}{3}{ફ્લેટ પ્લેટ કલેક્ટરના ફાયદાઓની યાદી બનાવો.}

\begin{solutionbox}
\begin{center}
\captionof{table}{ફ્લેટ પ્લેટ કલેક્ટરના ફાયદાઓ}
\begin{tabulary}{\linewidth}{|L|L|}
\hline
\textbf{કેટેગરી} & \textbf{ફાયદાઓ} \\ \hline
\textbf{તકનીકી} & સાદી ડિઝાઈન, કોઈ હિલતા ભાગો નથી, ઓછી જાળવણી \\ \hline
\textbf{આર્થિક} & ઓછી કિંમત, મોટા પાયે ઉત્પાદન શક્ય \\ \hline
\textbf{ઓપરેશનલ} & વેરવિખેર પ્રકાશ સાથે કામ કરે, સીધા અને પરોક્ષ બંને રેડિએશન સંભાળે \\ \hline
\textbf{ટકાઉપણું} & લાંબું જીવન (15-20 વર્ષ), હવામાન પ્રતિરોધક \\ \hline
\textbf{વર્સેટિલિટી} & બહુવિધ ઉપયોગો, મોડ્યુલર ઇન્સ્ટોલેશન \\ \hline
\end{tabulary}
\end{center}

\textbf{મુખ્ય ફાયદાઓ:}
\begin{itemize}
    \item \keyword{વિશ્વસનીયતા}: જટિલ મિકેનિઝમ અથવા નિયંત્રણોની જરૂર નથી
    \item \keyword{કાર્યક્ષમતા}: શ્રેષ્ઠ પરિસ્થિતિઓમાં 40-60\% થર્મલ કાર્યક્ષમતા
    \item \keyword{ઇન્સ્ટોલેશન}: છત અથવા જમીન પર સરળ માઉન્ટિંગ
\end{itemize}
\end{solutionbox}

\begin{mnemonicbox}
\mnemonic{TEODV: Technical, Economic, Operational, Durability, Versatility}
\end{mnemonicbox}

\questionmarks{3(બ અથવા)}{4}{પવન ચક્કી ક્ષેત્ર શું છે? તેના ફાયદાઓની યાદી આપો.}

\begin{solutionbox}
\textbf{વ્યાખ્યા:} વિન્ડ ફાર્મ એ વ્યાવસાયિક વિજળી ઉત્પાદન માટે એક જ સ્થાને સ્થાપિત વિન્ડ ટર્બાઇનનું જૂથ છે, જે ટ્રાન્સમિશન લાઇન દ્વારા વિદ્યુત ગ્રિડ સાથે જોડાયેલ હોય છે.

\begin{center}
\captionof{table}{વિન્ડ ફાર્મના ફાયદાઓ}
\begin{tabulary}{\linewidth}{|L|L|}
\hline
\textbf{કેટેગરી} & \textbf{ફાયદાઓ} \\ \hline
\textbf{પર્યાવરણીય} & સ્વચ્છ ઊર્જા, શૂન્ય ઉત્સર્જન, કાર્બન ફૂટપ્રિન્ટ ઘટાડે \\ \hline
\textbf{આર્થિક} & નોકરીઓ સર્જન, ઓછા ઓપરેટિંગ ખર્ચ, જમીન માલિકો માટે આવક \\ \hline
\textbf{તકનીકી} & સ્કેલેબલ ક્ષમતા, ગ્રિડ સ્થિરતા, ઊર્જા સ્વતંત્રતા \\ \hline
\textbf{સામાજિક} & ગ્રામીણ વિકાસ, સમુદાયિક ફાયદાઓ, શૈક્ષણિક તકો \\ \hline
\end{tabulary}
\end{center}

\textbf{વિશિષ્ટ ફાયદાઓ:}
\begin{itemize}
    \item \keyword{જમીનના ઉપયોગની કાર્યક્ષમતા}: ટર્બાઇન વચ્ચે ખેતી ચાલુ રાખી શકાય
    \item \keyword{ઝડપી ઇન્સ્ટોલેશન}: પરંપરાગત પાવર પ્લાન્ટ કરતાં ઝડપી
    \item \keyword{અનુમાનિત કિંમતો}: નિશ્ચિત ઇંધન કિંમત (પવન મફત છે)
    \item \keyword{મોડ્યુલર વિસ્તરણ}: ક્ષમતા ક્રમશઃ વધારી શકાય
\end{itemize}

\textbf{ઉપયોગો:}
\begin{itemize}
    \item \keyword{ઓનશોર}: જમીન આધારિત ઇન્સ્ટોલેશન
    \item \keyword{ઓફશોર}: વધુ પવનની ઝડપ માટે સમુદ્ર આધારિત
    \item \keyword{વિતરિત}: નાના પાયાના સમુદાયિક પ્રોજેક્ટ્સ
\end{itemize}
\end{solutionbox}

\begin{mnemonicbox}
\mnemonic{ECTS: Environmental, Economic, Technical, Social benefits}
\end{mnemonicbox}

\questionmarks{3(ક અથવા)}{7}{ટૂંકમાં સમજાવો (1) ભૂઉષ્મીય ઊર્જા (2) ભરતી ઊર્જા}

\begin{solutionbox}
\textbf{(1) ભૂઉષ્મીય ઊર્જા:}

ભૂઉષ્મીય ઊર્જા વિજળી ઉત્પાદન અને સીધા હીટિંગ ઉપયોગો માટે પૃથ્વીના આંતરિક ગરમીનો ઉપયોગ કરે છે.

\begin{center}
\captionof{table}{ભૂઉષ્મીય ઊર્જા સિસ્ટમ}
\begin{tabulary}{\linewidth}{|L|L|L|}
\hline
\textbf{પ્રકાર} & \textbf{તાપમાન} & \textbf{ઉપયોગો} \\ \hline
\textbf{ઉચ્ચ તાપમાન} & >150\textdegree C & વિજળી ઉત્પાદન \\ \hline
\textbf{મધ્યમ તાપમાન} & 90-150\textdegree C & સીધું હીટિંગ, કૂલિંગ \\ \hline
\textbf{નીચા તાપમાન} & <90\textdegree C & હીટ પંપ, કૃષિ \\ \hline
\end{tabulary}
\end{center}

\textbf{કાર્યસિદ્ધાંત:}
\begin{itemize}
    \item \keyword{ગરમીનો સ્ત્રોત}: પૃથ્વીના કોરમાં કિરણોત્સર્ગી ક્ષય
    \item \keyword{નિષ્કર્ષણ}: ગરમ પાણી/વરાળ મેળવવા માટે કૂવા ખોદવા
    \item \keyword{રૂપાંતરણ}: વરાળ વિજળી માટે ટર્બાઇન ચલાવે
    \item \keyword{રી-ઇન્જેક્શન}: પાણી રિઝર્વોયરમાં પાછું મોકલવું
\end{itemize}

\textbf{(2) ભરતી ઊર્જા:}

ભરતી ઊર્જા અનુમાનિત ભરતીની હિલચાલનો ઉપયોગ કરીને સમુદ્રી ભરતીની ગતિશીલ અને સ્થિતિશીલ ઊર્જાને વિજળીમાં રૂપાંતરિત કરે છે.

\begin{center}
\captionof{table}{ભરતી ઊર્જા તકનીકો}
\begin{tabulary}{\linewidth}{|L|L|L|}
\hline
\textbf{તકનીક} & \textbf{સિદ્ધાંત} & \textbf{ઇન્સ્ટોલેશન} \\ \hline
\textbf{ટાઇડલ બેરેજ} & ભરતીની શ્રેણીની સ્થિતિશીલ ઊર્જા & નદીમુખ પર ડેમ \\ \hline
\textbf{ટાઇડલ સ્ટ્રીમ} & ભરતીના પ્રવાહની ગતિશીલ ઊર્જા & પાણીની અંદર ટર્બાઇન \\ \hline
\textbf{ટાઇડલ લેગૂન} & કૃત્રિમ બંધ વિસ્તાર & બ્રેકવોટર બાંધકામ \\ \hline
\end{tabulary}
\end{center}

\textbf{ફાયદાઓ:}
\begin{itemize}
    \item \keyword{ભૂઉષ્મીય}: બેઝલોડ પાવર, ઓછા ઉત્સર્જન, નાનું ફૂટપ્રિન્ટ, વિશ્વસનીય
    \item \keyword{ભરતી}: અનુમાનિત, ઉચ્ચ ઊર્જા ઘનતા, લાંબું જીવનકાળ, ઇંધન ખર્ચ નહીં
\end{itemize}

\textbf{પડકારો:}
\begin{itemize}
    \item \keyword{ભૂઉષ્મીય}: સ્થાન વિશિષ્ટ, ઉચ્ચ પ્રારંભિક કિંમત, પ્રેરિત ભૂકંપ
    \item \keyword{ભરતી}: ઉચ્ચ મૂડી ખર્ચ, પર્યાવરણીય અસર, મર્યાદિત સ્થાનો
\end{itemize}
\end{solutionbox}

\begin{mnemonicbox}
\mnemonic{GT-POWER: Geothermal Temperature, Tidal Predictable Ocean Water Energy Resource}
\end{mnemonicbox}

\questionmarks{4(અ)}{3}{નવીનીકરણીય ઊર્જાની જરૂરિયાત વ્યાખ્યાયિત કરો}

\begin{solutionbox}
\begin{center}
\captionof{table}{નવીનીકરણીય ઊર્જાની જરૂરિયાત}
\begin{tabulary}{\linewidth}{|L|L|}
\hline
\textbf{ચાલક} & \textbf{કારણો} \\ \hline
\textbf{પર્યાવરણીય} & આબોહવા પરિવર્તન ઘટાડો, પ્રદૂષણ ઘટાડો \\ \hline
\textbf{આર્થિક} & ઊર્જા સુરક્ષા, કિંમત સ્થિરતા, નોકરીઓ સર્જન \\ \hline
\textbf{તકનીકી} & અવશેષ ઇંધણોનો ક્ષય, તકનીકી પ્રગતિ \\ \hline
\textbf{સામાજિક} & ગ્રામીણ વિકાસ, આરોગ્યને ફાયદાઓ, ઊર્જા પહોંચ \\ \hline
\end{tabulary}
\end{center}

\textbf{મુખ્ય જરૂરિયાતો:}
\begin{itemize}
    \item \keyword{આબોહવા પ્રતિબદ્ધતાઓ}: પેરિસ એગ્રીમેન્ટ લક્ષ્યો પૂરા કરવા
    \item \keyword{ઊર્જા સ્વતંત્રતા}: આયાત નિર્ભરતા ઘટાડવી
    \item \keyword{ટકાઉ વિકાસ}: લાંબાગાળાની ઊર્જા સુરક્ષા
\end{itemize}
\end{solutionbox}

\begin{mnemonicbox}
\mnemonic{EETS: Environmental, Economic, Technical, Social needs}
\end{mnemonicbox}

\questionmarks{4(બ)}{4}{ઓઝોન સ્તરના અવક્ષયને સમજાવો.}

\begin{solutionbox}
ઓઝોન સ્તરનો અવક્ષય માનવ નિર્મિત રસાયણો, ખાસ કરીને ક્લોરોફ્લોરોકાર્બન (CFCs) ને કારણે સ્ટ્રેટોસ્ફિયરમાં ઓઝોન સાંદ્રતાનો ઘટાડો છે.

\begin{center}
\captionof{table}{ઓઝોન અવક્ષય પ્રક્રિયા}
\begin{tabulary}{\linewidth}{|L|L|L|}
\hline
\textbf{તબક્કો} & \textbf{પ્રક્રિયા} & \textbf{રાસાયણિક પ્રતિક્રિયા} \\ \hline
\textbf{CFC મુક્તિ} & ઔદ્યોગિક ઉત્સર્જન & CFCs સ્ટ્રેટોસ્ફિયરમાં ઉગે \\ \hline
\textbf{UV વિભાજન} & ફોટોડિસોસિએશન & CFC + UV $\rightarrow$ Cl + અન્ય ઉત્પાદનો \\ \hline
\textbf{ઓઝોન વિનાશ} & કેટેલિટિક ચક્ર & Cl + O\textsubscript{3} $\rightarrow$ ClO + O\textsubscript{2} \\ \hline
\textbf{શૃંખલા પ્રતિક્રિયા} & સતત પ્રક્રિયા & ClO + O $\rightarrow$ Cl + O\textsubscript{2} \\ \hline
\end{tabulary}
\end{center}

\textbf{કારણો:}
\begin{itemize}
    \item \keyword{પ્રાથમિક}: CFCs, હેલોન્સ, મેથાઈલ બ્રોમાઈડ
    \item \keyword{ગૌણ}: HCFCs, નાઈટ્રસ ઓક્સાઈડ, કાર્બન ટેટ્રાક્લોરાઈડ
\end{itemize}

\textbf{અસરો:}
\begin{itemize}
    \item \keyword{વધેલ UV-B રેડિએશન}: ત્વચા કેન્સર, મોતિયો
    \item \keyword{પર્યાવરણીય અસર}: પાકની ઉપજ ઘટાડો, દરિયાઈ ઇકોસિસ્ટમ નુકસાન
    \item \keyword{આબોહવા અસરો}: બદલાયેલ વાતાવરણીય પરિભ્રમણ
\end{itemize}

\textbf{ઉકેલો:}
\begin{itemize}
    \item \keyword{મોન્ટ્રીલ પ્રોટોકોલ}: આંતરરાષ્ટ્રીય એગ્રીમેન્ટ (1987)
    \item \keyword{CFC ફેઝ-આઉટ}: ઓઝોન-ફ્રેન્ડલી વિકલ્પો સાથે બદલવું
    \item \keyword{HCFC સંક્રમણ}: અસ્થાયી વિકલ્પો તબક્કાવાર બંધ
\end{itemize}
\end{solutionbox}

\begin{mnemonicbox}
\mnemonic{CURE: CFCs, UV, Reactions, Effects}
\end{mnemonicbox}

\questionmarks{4(ક)}{7}{સમજાવો: (1) ગ્રીનહાઉસ અસર (2) આબોહવા પરિવર્તન વ્યવસ્થાપન}

\begin{solutionbox}
\textbf{(1) ગ્રીનહાઉસ અસર:}

કુદરતી પ્રક્રિયા જ્યાં ચોક્કસ વાતાવરણીય ગેસો સૂર્યથી ગરમીને ફસાવે છે, જીવન માટે યોગ્ય પૃથ્વીનું તાપમાન જાળવે છે.

\begin{center}
\begin{tikzpicture}[node distance=1.5cm, auto]
    \node [gtu block] (Sun) {સૌર કિરણોત્સર્ગ};
    \node [gtu block, below=1.5cm of Sun] (Earth) {પૃથ્વીની સપાટી};
    \node [gtu block, right=1.5cm of Earth] (Heat) {ગરમી કિરણોત્સર્ગ};
    \node [gtu block, above=1.5cm of Heat] (GHG) {ગ્રીનહાઉસ ગેસો};
    \node [gtu block, right=1.5cm of GHG] (Trap) {ગરમી ફસાઈ};
    \node [gtu block, right=1.5cm of Earth] (Back) at (6,0) {પૃથ્વી પર પાછી\\કિરણોત્સર્ગ};
    
    \path [gtu arrow] (Sun) -- (Earth);
    \path [gtu arrow] (Earth) -- (Heat);
    \path [gtu arrow] (Heat) -- (GHG);
    \path [gtu arrow] (GHG) -- (Trap);
    \path [gtu arrow] (Trap) -- (Back);
    \path [gtu arrow] (Back) -- (Earth);
\end{tikzpicture}
\captionof{figure}{ગ્રીનહાઉસ અસર}
\end{center}

\begin{center}
\captionof{table}{ગ્રીનહાઉસ ગેસો}
\begin{tabulary}{\linewidth}{|L|L|L|L|}
\hline
\textbf{ગેસ} & \textbf{સ્ત્રોતો} & \textbf{યોગદાન} & \textbf{જીવનકાળ} \\ \hline
\textbf{CO\textsubscript{2}} & અવશેષ ઇંધણ, વનનાશ & 76\% & 300-1000 વર્ષ \\ \hline
\textbf{CH\textsubscript{4}} & કૃષિ, લેન્ડફિલ & 16\% & 12 વર્ષ \\ \hline
\textbf{N\textsubscript{2}O} & ખાતર, દહન & 6\% & 120 વર્ષ \\ \hline
\textbf{F-ગેસો} & ઔદ્યોગિક પ્રક્રિયાઓ & 2\% & વિવિધ \\ \hline
\end{tabulary}
\end{center}

\textbf{વધેલી ગ્રીનહાઉસ અસર:}
\begin{itemize}
    \item \keyword{કારણ}: માનવ પ્રવૃત્તિઓથી વધેલ GHG સાંદ્રતા
    \item \keyword{પરિણામ}: વૈશ્વિક તાપમાન વધારો, આબોહવા પરિવર્તન
    \item \keyword{ફીડબેક લૂપ્સ}: ગરમ થવાની અસરોને વધારે
\end{itemize}

\textbf{(2) આબોહવા પરિવર્તન વ્યવસ્થાપન:}

શમન અને અનુકૂલન વ્યૂહરચના દ્વારા આબોહવા પરિવર્તનને સંબોધવા માટે વ્યાપક અભિગમ.

\begin{center}
\captionof{table}{આબોહવા પરિવર્તન વ્યવસ્થાપન વ્યૂહરચનાઓ}
\begin{tabulary}{\linewidth}{|L|L|L|}
\hline
\textbf{વ્યૂહરચના} & \textbf{અભિગમ} & \textbf{ઉદાહરણો} \\ \hline
\textbf{શમન} & GHG ઉત્સર્જન ઘટાડો & નવીકરણીય ઊર્જા, ઊર્જા કાર્યક્ષમતા \\ \hline
\textbf{અનુકૂલન} & આબોહવા અસરોને સમાયોજન & સીવોલ, દુષ્કાળ પ્રતિરોધી પાકો \\ \hline
\textbf{ટેકનોલોજી} & નવાચાર ઉકેલો & કાર્બન કેપ્ચર, સ્માર્ટ ગ્રિડ \\ \hline
\textbf{નીતિ} & નિયમનકારી ફ્રેમવર્ક & કાર્બન પ્રાઈસિંગ, ઉત્સર્જન ધોરણો \\ \hline
\textbf{આંતરરાષ્ટ્રીય} & વૈશ્વિક સહયોગ & પેરિસ એગ્રીમેન્ટ, આબોહવા ફાઈનાન્સ \\ \hline
\end{tabulary}
\end{center}

\textbf{શમન પગલાં:}
\begin{itemize}
    \item \keyword{ઊર્જા ક્ષેત્ર}: નવીકરણીય ઊર્જા જમાવટ, કાર્યક્ષમતા સુધારા
    \item \keyword{પરિવહન}: ઇલેક્ટ્રિક વાહનો, સાર્વજનિક પરિવહન, બાયોફ્યુઅલ
    \item \keyword{ઉદ્યોગ}: પ્રક્રિયા ઓપ્ટિમાઇઝેશન, લો-કાર્બન ટેકનોલોજી
    \item \keyword{ઇમારતો}: ગ્રીન કન્સ્ટ્રક્શન, સ્માર્ટ સિસ્ટમ
    \item \keyword{કૃષિ}: ટકાઉ પ્રથાઓ, ઘટાડેલ ઉત્સર્જન
\end{itemize}

\textbf{અનુકૂલન પગલાં:}
\begin{itemize}
    \item \keyword{ઇન્ફ્રાસ્ટ્રક્ચર}: આબોહવા-પ્રત્યાસ્થ ડિઝાઇન, પૂર સંરક્ષણ
    \item \keyword{ઇકોસિસ્ટમ}: સંરક્ષણ, પુનઃસ્થાપન, કોરિડોર
    \item \keyword{પાણીના સંસાધનો}: કાર્યક્ષમ ઉપયોગ, સંગ્રહ, ગુણવત્તા વ્યવસ્થાપન
    \item \keyword{આરોગ્ય}: રોગ સર્વેલન્સ, ગરમીની લહેર તૈયારી
\end{itemize}

\textbf{વ્યવસ્થાપન ફ્રેમવર્ક:}
\begin{enumerate}
    \item \keyword{મૂલ્યાંકન}: આબોહવા જોખમ અને નબળાઈ વિશ્લેષણ
    \item \keyword{આયોજન}: એકીકૃત વ્યૂહરચના અને કાર્ય યોજનાઓ
    \item \keyword{અમલીકરણ}: પ્રોજેક્ટ અમલ અને મોનિટરિંગ
    \item \keyword{મૂલ્યાંકન}: પ્રદર્શન મૂલ્યાંકન અને ગોઠવણ
\end{enumerate}
\end{solutionbox}

\begin{mnemonicbox}
\mnemonic{GEMMA: Gases, Enhanced, Mitigation, Management, Adaptation}
\end{mnemonicbox}

\questionmarks{4(અ અથવા)}{3}{આબોહવા પરિવર્તનને અસર કરતા પરિબળોની ચર્ચા કરો.}

\begin{solutionbox}
\begin{center}
\captionof{table}{આબોહવા પરિવર્તન પરિબળો}
\begin{tabulary}{\linewidth}{|L|L|L|}
\hline
\textbf{પરિબળ પ્રકાર} & \textbf{ઉદાહરણો} & \textbf{અસર} \\ \hline
\textbf{કુદરતી} & સૌર વેરિએશન, જ્વાળામુખી વિસ્ફોટ & નજીવો પ્રભાવ \\ \hline
\textbf{માનવજન્ય} & GHG ઉત્સર્જન, જમીન ઉપયોગ પરિવર્તન & મુખ્ય ચાલક \\ \hline
\textbf{ફીડબેક} & બરફ-એલ્બેડો, પાણીની વરાળ & વિસ્તૃતીકરણ \\ \hline
\end{tabulary}
\end{center}

\textbf{મુખ્ય પરિબળો:}
\begin{itemize}
    \item \keyword{ગ્રીનહાઉસ ગેસ સાંદ્રતા}: ગરમ થવાનો પ્રાથમિક ચાલક
    \item \keyword{એરોસોલ્સ}: ઠંડક અસર, કેટલાક ગરમ થવાને છુપાવે
    \item \keyword{જમીન ઉપયોગ પરિવર્તન}: વનનાશ, શહેરીકરણ અસરો
\end{itemize}
\end{solutionbox}

\begin{mnemonicbox}
\mnemonic{NAF: Natural, Anthropogenic, Feedback factors}
\end{mnemonicbox}

\questionmarks{4(બ અથવા)}{4}{ક્લાઈમેટ ચેન્જ સમજાવો}

\begin{solutionbox}
આબોહવા પરિવર્તન 20મી સદીના મધ્યથી મુખ્યત્વે માનવ પ્રવૃત્તિઓને કારણે વૈશ્વિક તાપમાન અને હવામાન પેટર્નમાં લાંબાગાળાના ફેરફારોનો સંદર્ભ આપે છે.

\begin{center}
\captionof{table}{આબોહવા પરિવર્તન સૂચકાંકો}
\begin{tabulary}{\linewidth}{|L|L|L|}
\hline
\textbf{સૂચકાંક} & \textbf{અવલોકિત ફેરફારો} & \textbf{વલણ} \\ \hline
\textbf{તાપમાન} & 1880 થી +1.1\textdegree C & વધતું \\ \hline
\textbf{સમુદ્ર સ્તર} & 1880 થી 21-24 સેમી & વધતું \\ \hline
\textbf{આર્કટિક બરફ} & દર દાયકાએ 13\% નુકસાન & ઘટતું \\ \hline
\textbf{વરસાદ} & પ્રાદેશિક વિવિધતાઓ & બદલાતા પેટર્ન \\ \hline
\end{tabulary}
\end{center}

\textbf{કારણો:}
\begin{itemize}
    \item \keyword{પ્રાથમિક}: અવશેષ ઇંધણોથી ગ્રીનહાઉસ ગેસ ઉત્સર્જન
    \item \keyword{ગૌણ}: વનનાશ, ઔદ્યોગિક પ્રક્રિયાઓ, કૃષિ
\end{itemize}

\textbf{અસરો:}
\begin{itemize}
    \item \keyword{ભૌતિક}: આત્યંતિક હવામાન, સમુદ્ર સ્તર વધારો, બરફ નુકસાન
    \item \keyword{જૈવિક}: પ્રજાતિઓનું સ્થળાંતર, ઇકોસિસ્ટમ વિક્ષેપ
    \item \keyword{માનવ}: ખોરાક સુરક્ષા, પાણીના સંસાધનો, આરોગ્ય
\end{itemize}

\textbf{પુરાવા:}
\begin{itemize}
    \item \keyword{તાપમાન રેકોર્ડ}: વૈશ્વિક ગરમ થવાનો વલણ
    \item \keyword{બરફના કોર ડેટા}: ઐતિહાસિક CO\textsubscript{2} સ્તર
    \item \keyword{સેટેલાઇટ અવલોકનો}: બરફની ચાદરમાં ફેરફાર
\end{itemize}
\end{solutionbox}

\begin{mnemonicbox}
\mnemonic{CHIP: Causes, Human impacts, Indicators, Physical evidence}
\end{mnemonicbox}

\questionmarks{4(ક અથવા)}{7}{ગ્લોબલ વોર્મિંગ પર ટૂંકી નોંધ લખો.}

\begin{solutionbox}
ગ્લોબલ વોર્મિંગ એ માનવ પ્રવૃત્તિઓથી વધેલી ગ્રીનહાઉસ અસરને કારણે પૃથ્વીના સરેરાશ સપાટીના તાપમાનમાં લાંબાગાળાનો વધારો છે.

\begin{center}
\captionof{table}{ગ્લોબલ વોર્મિંગના ઘટકો}
\begin{tabulary}{\linewidth}{|L|L|L|}
\hline
\textbf{પાસું} & \textbf{વિગતો} & \textbf{અસર} \\ \hline
\textbf{વ્યાખ્યા} & વૈશ્વિક સરેરાશ તાપમાનમાં વધારો & પૂર્વ-ઔદ્યોગિક કાળથી +1.1\textdegree C \\ \hline
\textbf{પ્રાથમિક કારણ} & અવશેષ ઇંધણોથી CO\textsubscript{2} ઉત્સર્જન & 410+ ppm વાતાવરણીય CO\textsubscript{2} \\ \hline
\textbf{સમયરેખા} & 1950 ના દાયકાથી ઝડપી & 10,000 વર્ષમાં સૌથી ઝડપી ગરમ થવું \\ \hline
\textbf{પ્રાદેશિક વિવિધતા} & આર્કટિક ગરમ થવું વૈશ્વિક સરેરાશ કરતાં 2x & ધ્રુવીય વિસ્તૃતીકરણ \\ \hline
\end{tabulary}
\end{center}

\textbf{ગ્લોબલ વોર્મિંગના કારણો:}
\begin{center}
\captionof{table}{ઉત્સર્જન સ્ત્રોતો}
\begin{tabulary}{\linewidth}{|L|L|L|}
\hline
\textbf{ક્ષેત્ર} & \textbf{યોગદાન} & \textbf{મુખ્ય પ્રવૃત્તિઓ} \\ \hline
\textbf{ઊર્જા} & 73\% & વિજળી, ગરમી, પરિવહન \\ \hline
\textbf{કૃષિ} & 18\% & પશુધન, ચોખાની ખેતી \\ \hline
\textbf{ઔદ્યોગિક} & 5\% & સિમેન્ટ, સ્ટીલ, રસાયણો \\ \hline
\textbf{કચરો} & 3\% & લેન્ડફિલ, ગંદા પાણી \\ \hline
\textbf{જમીન ઉપયોગ} & 1\% & વનનાશ, વિકાસ \\ \hline
\end{tabulary}
\end{center}

\textbf{પરિણામો:}
\begin{itemize}
    \item \keyword{ભૌતિક અસરો}: સમુદ્ર સ્તર વધારો, ગ્લેશિયર પીછેહઠ, પર્માફ્રોસ્ટ પીગળવું
    \item \keyword{હવામાન પેટર્ન}: વધુ વારંવાર ગરમીની લહેરો, બદલાયેલ વરસાદ
    \item \keyword{ઇકોસિસ્ટમ અસરો}: પ્રજાતિઓનું લુપ્ત થવું, વસવાટ નુકસાન, કોરલ બ્લીચિંગ
    \item \keyword{માનવ અસરો}: કૃષિ વિક્ષેપ, પાણીની અછત, આરોગ્ય જોખમો
\end{itemize}

\textbf{ફીડબેક મિકેનિઝમ:}
\begin{itemize}
    \item \keyword{બરફ-એલ્બેડો ફીડબેક}: ઓછું બરફ $\rightarrow$ વધુ ગરમી શોષણ
    \item \keyword{પાણીની વરાળ ફીડબેક}: ગરમ હવા વધુ ભેજ ધરાવે
    \item \keyword{પર્માફ્રોસ્ટ ફીડબેક}: પીગળવાથી સંગ્રહિત કાર્બન મુક્ત થાય
\end{itemize}

\textbf{ઉકેલો:}
\begin{itemize}
    \item \keyword{શમન}: ગ્રીનહાઉસ ગેસ ઉત્સર્જન ઘટાડવું
    \item \keyword{નવીકરણીય ઊર્જા}: સૌર, પવન, હાઇડ્રોઇલેક્ટ્રિક પાવર
    \item \keyword{ઊર્જા કાર્યક્ષમતા}: ઇમારતો, પરિવહન, ઉદ્યોગ
    \item \keyword{કાર્બન સીક્વેસ્ટ્રેશન}: જંગલો, માટી, તકનીકી કેપ્ચર
    \item \keyword{નીતિ પગલાં}: કાર્બન પ્રાઇસિંગ, નિયમો, પ્રોત્સાહનો
\end{itemize}

\textbf{આંતરરાષ્ટ્રીય પ્રતિસાદ:}
\begin{itemize}
    \item \keyword{UNFCCC}: આબોહવા પરિવર્તન પર ફ્રેમવર્ક કન્વેન્શન
    \item \keyword{ક્યોટો પ્રોટોકોલ}: પ્રથમ બંધનકર્તા ઉત્સર્જન ઘટાડા કરાર
    \item \keyword{પેરિસ એગ્રીમેન્ટ}: વર્તમાન વૈશ્વિક આબોહવા સમજૂતી (2015)
    \item \keyword{IPCC રિપોર્ટ્સ}: વૈજ્ઞાનિક મૂલ્યાંકન અને માર્ગદર્શન
\end{itemize}

\textbf{ભાવિ અનુમાનો:}
\begin{itemize}
    \item \keyword{તાપમાન વધારો}: ઉત્સર્જનના આધારે 2100 સુધીમાં 1.5-4.5\textdegree C
    \item \keyword{સમુદ્ર સ્તર વધારો}: 2100 સુધીમાં 0.43-2.84 મીટર
    \item \keyword{ટિપિંગ પોઇન્ટ્સ}: આબોહવા પ્રણાલીમાં અપરિવર્તનીય ફેરફારો
\end{itemize}
\end{solutionbox}

\begin{mnemonicbox}
\mnemonic{GWCF: Global Warming Causes Consequences Feedback}
\end{mnemonicbox}

\questionmarks{5(અ)}{3}{"ઇકો ટુરીઝમ" ની વિભાવના સમજાવો}

\begin{solutionbox}
ઇકો-ટુરીઝમ એ કુદરતી વિસ્તારોમાં જવાબદાર મુસાફરી છે જે પર્યાવરણનું સંરક્ષણ કરે છે, સ્થાનિક લોકોના કલ્યાણને ટકાવી રાખે છે, અને અર્થઘટન અને શિક્ષણ સામેલ કરે છે.

\begin{center}
\captionof{table}{ઇકો-ટુરીઝમના સિદ્ધાંતો}
\begin{tabulary}{\linewidth}{|L|L|}
\hline
\textbf{સિદ્ધાંત} & \textbf{વર્ણન} \\ \hline
\textbf{સંરક્ષણ} & કુદરતી વસવાટ અને વન્યજીવનનું સંરક્ષણ \\ \hline
\textbf{સમુદાય} & સ્થાનિક સમુદાયોને આર્થિક ફાયદો \\ \hline
\textbf{શિક્ષણ} & પર્યાવરણીય જાગૃતિ અને શિક્ષણ \\ \hline
\textbf{ટકાઉપણું} & લાંબાગાળાનું પર્યાવરણ સંરક્ષણ \\ \hline
\textbf{જવાબદારી} & નકારાત્મક અસરો ઘટાડવી \\ \hline
\end{tabulary}
\end{center}

\begin{itemize}
    \item \keyword{પ્રકૃતિ આધારિત}: કુદરતી વાતાવરણ પર ધ્યાન
    \item \keyword{ઓછી અસર}: ન્યૂનતમ પર્યાવરણીય વિક્ષેપ
    \item \keyword{સાંસ્કૃતિક આદર}: સ્થાનિક પરંપરાઓ અને રિવાજોનું મૂલ્ય
\end{itemize}
\end{solutionbox}

\begin{mnemonicbox}
\mnemonic{ECERS: Environment, Community, Education, Responsibility, Sustainability}
\end{mnemonicbox}

\questionmarks{5(બ)}{4}{પરંપરાગત અને બિનપરંપરાગત ઉર્જા સ્ત્રોતની સરખામણી.}

\begin{solutionbox}
\begin{center}
\captionof{table}{પરંપરાગત વિ બિનપરંપરાગત ઉર્જા સ્ત્રોતો}
\begin{tabulary}{\linewidth}{|L|L|L|}
\hline
\textbf{પેરામીટર} & \textbf{પરંપરાગત} & \textbf{બિનપરંપરાગત} \\ \hline
\textbf{ઉદાહરણો} & કોલસો, તેલ, કુદરતી ગેસ, ન્યુક્લિયર & સૌર, પવન, હાઇડ્રો, બાયોમાસ \\ \hline
\textbf{ઉપલબ્ધતા} & મર્યાદિત ભંડાર & વિપુલ અને નવીકરણીય \\ \hline
\textbf{પર્યાવરણીય અસર} & ઉચ્ચ પ્રદૂષણ, CO\textsubscript{2} ઉત્સર્જન & સ્વચ્છ, ન્યૂનતમ ઉત્સર્જન \\ \hline
\textbf{કિંમત} & શરૂઆતમાં ઓછી, વધતી કિંમતો & ઉચ્ચ પ્રારંભિક, ઘટતી કિંમતો \\ \hline
\textbf{ટેકનોલોજી} & પરિપક્વ, સ્થાપિત & વિકસતી, સુધરતી \\ \hline
\textbf{વિશ્વસનીયતા} & સતત પુરવઠો & હવામાન આધારિત \\ \hline
\textbf{ઇન્ફ્રાસ્ટ્રક્ચર} & સુસ્થાપિત & વિકાસ જરૂરી \\ \hline
\textbf{ક્ષય} & ખતમ થતા સંસાધનો & અખૂટ સ્ત્રોતો \\ \hline
\end{tabulary}
\end{center}

\textbf{ફાયદાઓ:}
\begin{itemize}
    \item \keyword{પરંપરાગત}: વિશ્વસનીય પુરવઠો, સ્થાપિત ઇન્ફ્રાસ્ટ્રક્ચર, ઉચ્ચ ઊર્જા ઘનતા
    \item \keyword{બિનપરંપરાગત}: ટકાઉ, સ્વચ્છ, નોકરીઓ સર્જન, ઊર્જા સ્વતંત્રતા
\end{itemize}

\textbf{પડકારો:}
\begin{itemize}
    \item \keyword{પરંપરાગત}: પર્યાવરણ નુકસાન, કિંમત અસ્થિરતા, મર્યાદિત સંસાધનો
    \item \keyword{બિનપરંપરાગત}: તૂટક તૂટક, સંગ્રહની જરૂર, પ્રારંભિક રોકાણ
\end{itemize}
\end{solutionbox}

\begin{mnemonicbox}
\mnemonic{CATERED: Conventional Available Technology Established Reliable Environmental Depletion}
\end{mnemonicbox}

\questionmarks{5(ક)}{7}{સમજાવો (1) પાણી અધિનિયમ, 1974 (2) પર્યાવરણ અધિનિયમ, 1986}

\begin{solutionbox}
\textbf{(1) પાણી (પ્રદૂષણ નિવારણ અને નિયંત્રણ) અધિનિયમ, 1974:}

ભારતમાં પાણીના પ્રદૂષણને અટકાવવા અને નિયંત્રિત કરવા અને પાણીની સ્વચ્છતા જાળવવા/પુનઃસ્થાપિત કરવા માટે વ્યાપક કાયદો.

\begin{center}
\captionof{table}{પાણી અધિનિયમ 1974 - મુખ્ય જોગવાઈઓ}
\begin{tabulary}{\linewidth}{|L|L|}
\hline
\textbf{પાસું} & \textbf{વિગતો} \\ \hline
\textbf{ઉદ્દેશ્ય} & પાણીના પ્રદૂષણને અટકાવવું અને નિયંત્રિત કરવું \\ \hline
\textbf{સત્તા} & કેન્દ્રીય અને રાજ્ય પ્રદૂષણ નિયંત્રણ બોર્ડ \\ \hline
\textbf{કવરેજ} & તમામ જળ સ્ત્રોતો - નદીઓ, પ્રવાહો, કૂવા, ભૂગર્ભજળ \\ \hline
\textbf{દંડ} & ઉલ્લંઘન માટે દંડ અને કેદ \\ \hline
\end{tabulary}
\end{center}

\textbf{મુખ્ય વિશેષતાઓ:}
\begin{itemize}
    \item \keyword{પ્રદૂષણ નિયંત્રણ બોર્ડ}: કેન્દ્રીય અને રાજ્ય સ્તરે સ્થાપના
    \item \keyword{સંમતિ મિકેનિઝમ}: ઉદ્યોગો માટે નો-ઓબ્જેક્શન સર્ટિફિકેટ
    \item \keyword{ધોરણો}: પાણીની ગુણવત્તા ધોરણો અને વહેતા પાણીની મર્યાદાઓ
    \item \keyword{મોનિટરિંગ}: જળ સ્ત્રોતોની નિયમિત તપાસ અને નમૂના લેવું
    \item \keyword{કટોકટીની જોગવાઈઓ}: પ્રદૂષણની કટોકટીઓ સંભાળવાની સત્તા
\end{itemize}

\textbf{બોર્ડની સત્તાઓ:}
\begin{itemize}
    \item \keyword{આયોજન}: પ્રદૂષણ નિવારણ અને નિયંત્રણ કાર્યક્રમો
    \item \keyword{ધોરણ સેટિંગ}: પાણીની ગુણવત્તા અને ડિસચાર્જ ધોરણો
    \item \keyword{સંમતિ આપવી}: કચરો છોડવાની પરવાનગી
    \item \keyword{મોનિટરિંગ}: પાણીની ગુણવત્તા દેખરેખ
    \item \keyword{અમલીકરણ}: ઉલ્લંઘનકર્તાઓ સામે કાનૂની કાર્યવાહી
\end{itemize}

\textbf{(2) પર્યાવરણ (સંરક્ષણ) અધિનિયમ, 1986:}

ભારતમાં પર્યાવરણ સંરક્ષણ અને સુધારા માટે ફ્રેમવર્ક પૂરો પાડતો છત્ર કાયદો, ભોપાલ ગેસ દુર્ઘટના પછી ઘડવામાં આવ્યો.

\begin{center}
\captionof{table}{પર્યાવરણ અધિનિયમ 1986 - મુખ્ય જોગવાઈઓ}
\begin{tabulary}{\linewidth}{|L|L|}
\hline
\textbf{પાસું} & \textbf{વિગતો} \\ \hline
\textbf{ઉદ્દેશ્ય} & વ્યાપક પર્યાવરણ સંરક્ષણ \\ \hline
\textbf{વ્યાપ્તિ} & હવા, પાણી, જમીન પ્રદૂષણ અને જોખમી પદાર્થો \\ \hline
\textbf{સત્તા} & કેન્દ્ર સરકાર અને નિયુક્ત એજન્સીઓ \\ \hline
\textbf{દંડ} & 5 વર્ષ સુધીની કેદ અને/અથવા ₹1 લાખ સુધીનો દંડ \\ \hline
\end{tabulary}
\end{center}

\textbf{મુખ્ય વિશેષતાઓ:}
\begin{itemize}
    \item \keyword{સામાન્ય સત્તાઓ}: પર્યાવરણ સંરક્ષણ માટે કેન્દ્ર સરકારની સત્તા
    \item \keyword{ધોરણો}: હવા, પાણી, માટી માટે પર્યાવરણીય ગુણવત્તા ધોરણો
    \item \keyword{અસર મૂલ્યાંકન}: પ્રોજેક્ટ્સ માટે પર્યાવરણીય મંજૂરી
    \item \keyword{જોખમી પદાર્થો}: હેન્ડલિંગ અને નિકાલનું નિયમન
    \item \keyword{જનભાગીદારી}: માહિતી અને ભાગીદારીનો અધિકાર
\end{itemize}

\textbf{મહત્વના નિયમો:}
\begin{itemize}
    \item \keyword{EIA નોટિફિકેશન 2006}: પર્યાવરણીય અસર મૂલ્યાંકન
    \item \keyword{હેજાર્ડસ વેસ્ટ રૂલ્સ}: વ્યવસ્થાપન અને હેન્ડલિંગ
    \item \keyword{અવાજ પ્રદૂષણ નિયમો}: આસપાસના અવાજના ધોરણો
    \item \keyword{કોસ્ટલ રેગ્યુલેશન ઝોન}: દરિયાકાંઠાના વિસ્તારનું સંરક્ષણ
\end{itemize}

\textbf{સરખામણી:}
\begin{center}
\captionof{table}{પાણી અધિનિયમ વિ પર્યાવરણ અધિનિયમ}
\begin{tabulary}{\linewidth}{|L|L|L|}
\hline
\textbf{પાસું} & \textbf{પાણી અધિનિયમ 1974} & \textbf{પર્યાવરણ અધિનિયમ 1986} \\ \hline
\textbf{વ્યાપ્તિ} & માત્ર પાણી પ્રદૂષણ & તમામ પર્યાવરણીય માધ્યમો \\ \hline
\textbf{અભિગમ} & ક્ષેત્રીય & વ્યાપક \\ \hline
\textbf{અમલીકરણ} & PCBs & કેન્દ્ર સરકાર \\ \hline
\textbf{દંડ} & મધ્યમ & કડક \\ \hline
\end{tabulary}
\end{center}

\textbf{અમલીકરણ મિકેનિઝમ:}
\begin{itemize}
    \item \keyword{મોનિટરિંગ}: નિયમિત તપાસ અને અનુપાલન તપાસ
    \item \keyword{કાનૂની કાર્યવાહી}: ઉલ્લંઘનકર્તાઓની કાર્યવાહી
    \item \keyword{બંધ કરવાના આદેશો}: પ્રદૂષક એકમો બંધ કરવા
    \item \keyword{વળતર}: પર્યાવરણીય નુકસાનનું મૂલ્યાંકન
\end{itemize}
\end{solutionbox}

\begin{mnemonicbox}
\mnemonic{WEPCA: Water Environmental Protection Comprehensive Act}
\end{mnemonicbox}

\questionmarks{5(અ અથવા)}{3}{"કાર્બન ક્રેડિટ" ખ્યાલ સમજાવો}

\begin{solutionbox}
કાર્બન ક્રેડિટ એ ઉત્સર્જન ઘટાડા અથવા કાર્બન સીક્વેસ્ટ્રેશન પ્રોજેક્ટ્સ દ્વારા વાતાવરણમાંથી એક ટન CO\textsubscript{2} સમકક્ષ ઘટાડેલ અથવા દૂર કરેલનું પ્રતિનિધિત્વ કરતું વેપારીલાયક પ્રમાણપત્ર છે.

\begin{center}
\captionof{table}{કાર્બન ક્રેડિટ મિકેનિઝમ}
\begin{tabulary}{\linewidth}{|L|L|}
\hline
\textbf{ઘટક} & \textbf{વર્ણન} \\ \hline
\textbf{એકમ} & 1 ક્રેડિટ = 1 ટન CO\textsubscript{2} સમકક્ષ \\ \hline
\textbf{ઉત્પાદન} & ઉત્સર્જન ઘટાડા/દૂર કરવાના પ્રોજેક્ટ્સ \\ \hline
\textbf{વેપાર} & કાર્બન બજારોમાં ખરીદી/વેચાણ \\ \hline
\textbf{ચકાસણી} & તૃતીય-પક્ષ માન્યતા જરૂરી \\ \hline
\end{tabulary}
\end{center}

\begin{itemize}
    \item \keyword{CDM}: ક્યોટો પ્રોટોકોલ હેઠળ ક્લીન ડેવલપમેન્ટ મિકેનિઝમ
    \item \keyword{સ્વૈચ્છિક બજારો}: ખાનગી ક્ષેત્રની પહેલ
    \item \keyword{અનુપાલન બજારો}: નિયમનકારી જરૂરિયાતો
\end{itemize}
\end{solutionbox}

\begin{mnemonicbox}
\mnemonic{CUTV: Credit Unit Trading Verification}
\end{mnemonicbox}

\questionmarks{5(બ અથવા)}{4}{"સોલિડ વેસ્ટ મેનેજમેન્ટ" ટૂંકમાં સમજાવો}

\begin{solutionbox}
ઘન કચરા વ્યવસ્થાપન એ માનવ પ્રવૃત્તિઓ દ્વારા છોડી દેવાયેલી ઘન સામગ્રીનું વ્યવસ્થિત એકત્રીકરણ, પરિવહન, પ્રક્રિયા, રિસાયક્લિંગ અને નિકાલ છે.

\begin{center}
\captionof{table}{ઘન કચરા વ્યવસ્થાપન હાયરાર્કી}
\begin{tabulary}{\linewidth}{|L|L|L|}
\hline
\textbf{પ્રાથમિકતા} & \textbf{પદ્ધતિ} & \textbf{વર્ણન} \\ \hline
\textbf{1મી} & \textbf{ઘટાડવું} & કચરાનું ઉત્પાદન ઘટાડવું \\ \hline
\textbf{2જી} & \textbf{પુનઃઉપયોગ} & વસ્તુઓનો બહુવિધ વાર ઉપયોગ \\ \hline
\textbf{3જી} & \textbf{રિસાયકલ} & કચરાને નવા ઉત્પાદનોમાં રૂપાંતરિત કરવું \\ \hline
\textbf{4થી} & \textbf{પુનઃપ્રાપ્તિ} & કચરામાંથી ઊર્જા પુનઃપ્રાપ્તિ \\ \hline
\textbf{5મી} & \textbf{નિકાલ} & સુરક્ષિત લેન્ડફિલિંગ \\ \hline
\end{tabulary}
\end{center}

\textbf{વ્યવસ્થાપન પ્રક્રિયા:}
\begin{itemize}
    \item \keyword{એકત્રીકરણ}: ઘરે-ઘરે પિકઅપ, સ્ત્રોતે વિભાજન
    \item \keyword{પરિવહન}: ટ્રાન્સફર સ્ટેશન, બલ્ક ટ્રાન્સપોર્ટ
    \item \keyword{ટ્રીટમેન્ટ}: કમ્પોસ્ટિંગ, રિસાયક્લિંગ, ઇન્સિનરેશન
    \item \keyword{નિકાલ}: સેનિટરી લેન્ડફિલ, વેસ્ટ-ટુ-એનર્જી
\end{itemize}

\textbf{ટેકનોલોજીઓ:}
\begin{itemize}
    \item \keyword{કમ્પોસ્ટિંગ}: કાર્બનિક કચરાનું વિઘટન
    \item \keyword{ઇન્સિનરેશન}: ઊર્જા પુનઃપ્રાપ્તિ સાથે ઉચ્ચ તાપમાન બર્નિંગ
    \item \keyword{એનેરોબિક પાચન}: કાર્બનિક કચરામાંથી બાયોગેસ ઉત્પાદન
    \item \keyword{મટેરિયલ રિકવરી}: સામગ્રીનું વિભાજન અને રિસાયક્લિંગ
\end{itemize}

\textbf{પડકારો:}
\begin{itemize}
    \item \keyword{વધતી માત્રા}: વસ્તી અને વપરાશ વૃદ્ધિ
    \item \keyword{મિશ્ર કચરો}: સ્ત્રોતે વિભાજનનો અભાવ
    \item \keyword{ઇન્ફ્રાસ્ટ્રક્ચર}: અપૂરતી એકત્રીકરણ અને ટ્રીટમેન્ટ સુવિધાઓ
    \item \keyword{ફાઇનાન્સિંગ}: ઉચ્ચ મૂડી અને ઓપરેશનલ ખર્ચ
\end{itemize}
\end{solutionbox}

\begin{mnemonicbox}
\mnemonic{CTTD: Collection, Transportation, Treatment, Disposal}
\end{mnemonicbox}

\questionmarks{5(ક અથવા)}{7}{"5R" ની વિભાવના સમજાવો.}

\begin{solutionbox}
5R વિભાવના એ વ્યાપક કચરા વ્યવસ્થાપન હાયરાર્કી છે જે પાંચ પરસ્પર જોડાયેલ વ્યૂહરચનાઓ દ્વારા ટકાઉ વપરાશ અને કચરા ઘટાડાને પ્રોત્સાહન આપે છે.

\begin{center}
\captionof{table}{5R કચરા વ્યવસ્થાપન હાયરાર્કી}
\begin{tabulary}{\linewidth}{|L|L|L|L|}
\hline
\textbf{R} & \textbf{વ્યૂહરચના} & \textbf{વ્યાખ્યા} & \textbf{ઉદાહરણો} \\ \hline
\textbf{1. નકારવું} & બિનજરૂરી વસ્તુઓ નકારવી & કચરો બનાવતા ઉત્પાદનોથી બચવું & પ્લાસ્ટિક બેગ, ડિસ્પોઝેબલ વસ્તુઓને ના કહેવું \\ \hline
\textbf{2. ઘટાડવું} & વપરાશ ઘટાડવો & સંસાધનોનો ઓછો ઉપયોગ & માત્ર જરૂરી વસ્તુઓ ખરીદવી, ટકાઉ ઉત્પાદનો પસંદ કરવા \\ \hline
\textbf{3. પુનઃઉપયોગ} & વસ્તુઓનો બહુવિધ વાર ઉપયોગ & ઉત્પાદનનું જીવનકાળ વધારવું & કન્ટેનરનો પુનઃઉપયોગ, જૂના કપડા દાન કરવા \\ \hline
\textbf{4. પુનર્નિર્દેશન} & સર્જનાત્મક વૈકલ્પિક ઉપયોગો & કચરાને ઉપયોગી વસ્તુઓમાં રૂપાંતરિત કરવું & બોટલને પ્લાન્ટર બનાવવા, ટાયરને ઝૂલા બનાવવા \\ \hline
\textbf{5. રિસાયકલ} & કચરાને નવા ઉત્પાદનોમાં પ્રક્રિયા કરવી & સામગ્રી પુનઃપ્રાપ્તિ અને પુનઃપ્રક્રિયા & કાગળ, પ્લાસ્ટિક, ધાતુ રિસાયક્લિંગ \\ \hline
\end{tabulary}
\end{center}

\textbf{વિગતવાર સમજૂતી:}

\textbf{1. નકારવું:}
\begin{itemize}
    \item \keyword{વિભાવના}: કચરા સામે પ્રથમ સંરક્ષણ રેખા
    \item \keyword{અમલીકરણ}: ઉપભોક્તાની પસંદગી અને જાગૃતિ
    \item \keyword{અસર}: સ્ત્રોતે કચરાનું ઉત્પાદન અટકાવે
    \item \keyword{ઉદાહરણો}: સિંગલ-યુઝ પ્લાસ્ટિક નકારવા, બિનજરૂરી પેકેજિંગ
\end{itemize}

\textbf{2. ઘટાડવું:}
\begin{itemize}
    \item \keyword{વિભાવના}: સંસાધન વપરાશ અને કચરા ઉત્પાદન ઘટાડવું
    \item \keyword{વ્યૂહરચના}: કાર્યક્ષમ ઉપયોગ, ટકાઉપણાં પર ધ્યાન, શેરિંગ ઇકોનોમી
    \item \keyword{ફાયદાઓ}: ઓછું પર્યાવરણીય ફૂટપ્રિન્ટ, ખર્ચ બચત
    \item \keyword{ઉપયોગો}: ઊર્જા કાર્યક્ષમતા, પાણી સંરક્ષણ, ન્યૂનતમ પેકેજિંગ
\end{itemize}

\textbf{3. પુનઃઉપયોગ:}
\begin{itemize}
    \item \keyword{વિભાવના}: પુનઃપ્રક્રિયા વિના ઉત્પાદનનું જીવન વધારવું
    \item \keyword{પદ્ધતિઓ}: સીધો પુનઃઉપયોગ, સમારકામ અને જાળવણી, પુનર્વિતરણ
    \item \keyword{ફાયદાઓ}: ઊર્જા બચત, આર્થિક ફાયદાઓ, સર્જનાત્મકતા
    \item \keyword{ઉદાહરણો}: સંગ્રહ માટે કાચના જાર, ફર્નિચર પુનઃસ્થાપન
\end{itemize}

\textbf{4. પુનર્નિર્દેશન:}
\begin{itemize}
    \item \keyword{વિભાવના}: વિવિધ કાર્યો માટે સર્જનાત્મક રૂપાંતરણ
    \item \keyword{નવાચાર}: ડિઝાઇન વિચારસરણી અને સર્જનાત્મકતા
    \item \keyword{સમુદાયિક પાસું}: મેકર સ્પેસ, DIY સંસ્કૃતિ
    \item \keyword{પર્યાવરણીય ફાયદો}: લેન્ડફિલમાંથી કચરો વાળવું
\end{itemize}

\textbf{5. રિસાયકલ:}
\begin{itemize}
    \item \keyword{વિભાવના}: સામગ્રી પુનઃપ્રાપ્તિ અને પુનઃપ્રક્રિયા
    \item \keyword{પ્રકારો}: યાંત્રિક, રાસાયણિક, જૈવિક રિસાયક્લિંગ
    \item \keyword{ઇન્ફ્રાસ્ટ્રક્ચર}: એકત્રીકરણ, સોર્ટિંગ, પ્રક્રિયા સુવિધાઓ
    \item \keyword{બજારો}: રિસાયકલ કરેલી સામગ્રી માટે અંત-ઉપયોગ ઉપયોગો
\end{itemize}

\textbf{અમલીકરણ ફ્રેમવર્ક:}
\begin{center}
\captionof{table}{5R અમલીકરણ સ્તરો}
\begin{tabulary}{\linewidth}{|L|L|L|L|}
\hline
\textbf{સ્તર} & \textbf{હિસ્સેદારો} & \textbf{ક્રિયાઓ} & \textbf{પરિણામો} \\ \hline
\textbf{વ્યક્તિગત} & ઉપભોક્તાઓ, પરિવારો & સભાન પસંદગીઓ, જીવનશૈલી ફેરફારો & ઘટાડેલ વ્યક્તિગત ફૂટપ્રિન્ટ \\ \hline
\textbf{સમુદાય} & પડોશીઓ, શાળાઓ & સ્થાનિક કાર્યક્રમો, જાગૃતિ અભિયાન & સમુદાયિક જોડાણ \\ \hline
\textbf{વ્યવસાય} & કંપનીઓ, ઉદ્યોગો & સર્ક્યુલર ઇકોનોમી, ટકાઉ ડિઝાઇન & સંસાધન કાર્યક્ષમતા \\ \hline
\textbf{સરકાર} & નીતિ ઘડવૈયાઓ, નિયમનકારો & નિયમો, પ્રોત્સાહનો, ઇન્ફ્રાસ્ટ્રક્ચર & સિસ્ટમ-વ્યાપી ફેરફાર \\ \hline
\end{tabulary}
\end{center}

\textbf{5R અભિગમના ફાયદાઓ:}
\begin{itemize}
    \item \keyword{પર્યાવરણીય}: ઘટાડેલ પ્રદૂષણ, સંસાધન સંરક્ષણ, આબોહવા સંરક્ષણ
    \item \keyword{આર્થિક}: ખર્ચ બચત, નોકરીઓ સર્જન, નવી વ્યવસાયિક તકો
    \item \keyword{સામાજિક}: સમુદાયિક જોડાણ, શિક્ષણ, વર્તન પરિવર્તન
    \item \keyword{સંસાધન સુરક્ષા}: કુમારી સામગ્રી પર ઘટાડેલ નિર્ભરતા
\end{itemize}

\textbf{પડકારો:}
\begin{itemize}
    \item \keyword{ઉપભોક્તા વર્તન}: સ્થાપિત આદતો અને પસંદગીઓ બદલવી
    \item \keyword{ઇન્ફ્રાસ્ટ્રક્ચર}: પૂરતી એકત્રીકરણ અને પ્રક્રિયા સુવિધાઓ
    \item \keyword{અર્થશાસ્ત્ર}: રિસાયકલ કરેલા ઉત્પાદનોની બજાર વ્યવહાર્યતા
    \item \keyword{નીતિ સમર્થન}: સહાયક નિયમો અને આર્થિક સાધનો
\end{itemize}

\textbf{સફળતાના પરિબળો:}
\begin{itemize}
    \item \keyword{શિક્ષણ}: જાગૃતિ અને ક્ષમતા નિર્માણ કાર્યક્રમો
    \item \keyword{ઇન્ફ્રાસ્ટ્રક્ચર}: પૂરતી કચરા વ્યવસ્થાપન પ્રણાલી
    \item \keyword{નીતિ}: સહાયક નિયમો અને આર્થિક સાધનો
    \item \keyword{ટેકનોલોજી}: કચરા પ્રક્રિયા અને ઉત્પાદન ડિઝાઇનમાં નવાચાર
    \item \keyword{સહયોગ}: બહુ-હિસ્સેદાર ભાગીદારી
\end{itemize}

\textbf{સર્ક્યુલર ઇકોનોમી કનેક્શન:}
5R વિભાવના સર્ક્યુલર ઇકોનોમી સિદ્ધાંતોનો પાયો બનાવે છે, જ્યાં કચરો નવા ઉત્પાદન ચક્ર માટે ઇનપુટ બને છે, સંસાધન નિષ્કર્ષણ અને પર્યાવરણીય અસર ઘટાડે છે.

\textbf{માપ અને મોનિટરિંગ:}
\begin{itemize}
    \item \keyword{કચરા ઘટાડાના મેટ્રિક્સ}: નિકાલમાંથી વાળેલી માત્રા
    \item \keyword{સામગ્રી પુનઃપ્રાપ્તિ દરો}: રિસાયકલ/પુનઃઉપયોગ કરેલા કચરાની ટકાવારી
    \item \keyword{પર્યાવરણીય સૂચકાંકો}: કાર્બન ફૂટપ્રિન્ટ, સંસાધન વપરાશ
    \item \keyword{આર્થિક મેટ્રિક્સ}: ખર્ચ બચત, નોકરીઓ સર્જન, આવક ઉત્પાદન
\end{itemize}

\textbf{વૈશ્વિક ઉદાહરણો:}
\begin{itemize}
    \item \keyword{ઝીરો વેસ્ટ શહેરો}: સાન ફ્રાન્સિસ્કો, લજુબલજાના, કામીકાત્સુ
    \item \keyword{વિસ્તૃત ઉત્પાદક જવાબદારી}: EU પેકેજિંગ નિયમો
    \item \keyword{ડિપોઝિટ સિસ્ટમ}: જર્મની, કેનાડામાં બોટલ રિટર્ન કાર્યક્રમો
    \item \keyword{શેરિંગ ઇકોનોમી}: ટૂલ લાઇબ્રેરી, કપડા સ્વેપ, રિપેર કેફે
\end{itemize}

\textbf{ભાવિ દિશાઓ:}
\begin{itemize}
    \item \keyword{ડિજિટલ પ્લેટફોર્મ}: કચરા ઘટાડા અને શેરિંગ માટે એપ્સ
    \item \keyword{એડવાન્સ્ડ રિસાયક્લિંગ}: કેમિકલ રિસાયક્લિંગ, AI-પાવર્ડ સોર્ટિંગ
    \item \keyword{બાયોપ્લાસ્ટિક્સ}: પરંપરાગત પ્લાસ્ટિકના બાયોડિગ્રેડેબલ વિકલ્પો
    \item \keyword{નીતિ ઉત્ક્રાંતિ}: સમારકામનો અધિકાર, વિસ્તૃત ઉત્પાદક જવાબદારી
\end{itemize}
\end{solutionbox}

\begin{mnemonicbox}
\mnemonic{R5-POWER: Refuse, Reduce, Reuse, Repurpose, Recycle - Protect Our World's Environmental Resources}
\end{mnemonicbox}

\end{document}

