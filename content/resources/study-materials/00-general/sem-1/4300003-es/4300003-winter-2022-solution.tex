\documentclass[10pt,a4paper]{article}

% content/resources/templates/preamble.tex
\usepackage[margin=0.6in]{geometry}
\author{Milav Dabgar}
\usepackage{amsmath,amssymb,amsthm}
\usepackage{booktabs}
\usepackage{multirow}
\usepackage{xcolor}
\usepackage{tcolorbox}
\tcbuselibrary{breakable,skins}
\usepackage[colorlinks=true,linkcolor=blue]{hyperref}
\usepackage{titlesec}
\usepackage{enumitem}
\usepackage{tikz}
\usepackage{pgfplots}
\usepackage{circuitikz}
\usepackage[version=4]{mhchem}
\usepackage{longtable}
\usepackage{array}
\usepackage{float}
\usepackage{caption}
\usepackage{listings}

\lstset{
  basicstyle=\small\ttfamily,
  breaklines=true,
  breakatwhitespace=false,
  postbreak=\mbox{\textcolor{red}{$\hookrightarrow$}\space},
  float=false,
  numbers=left,
  numberstyle=\tiny\color{gray},
  numbersep=10pt,
  xleftmargin=2em,
  keywordstyle=\color{blue},
  commentstyle=\color{green!60!black},
  stringstyle=\color{purple},
  backgroundcolor=\color{gray!5},
  showstringspaces=false,
  tabsize=2,
  captionpos=b,
  keepspaces=true,
  columns=flexible
}

\pgfplotsset{compat=1.18}
\usetikzlibrary{shapes,arrows,positioning,calc,patterns,decorations.pathmorphing,decorations.markings,arrows.meta}

% Color scheme
\definecolor{headcolor}{RGB}{0,102,204}
\definecolor{keycolor}{RGB}{220,20,60}
\definecolor{solutioncolor}{RGB}{34,139,34}
\definecolor{mnemoniccolor}{RGB}{148,0,211}
\definecolor{codecolor}{RGB}{0,0,100}

% Spacing
\setlength{\parskip}{3pt}
\setlist[itemize]{nosep}
\setlist[enumerate]{nosep}

% Title formatting
\titleformat{\section}{\Large\bfseries\color{headcolor}}{\thesection}{1em}{}
\titleformat{\subsection}{\large\bfseries\color{headcolor}}{\thesubsection}{1em}{}

% Pandoc tightlist compatibility
\providecommand{\tightlist}{%
  \setlength{\itemsep}{0pt}\setlength{\parskip}{0pt}}

% Pandoc longtable compatibility
\newcounter{none}
\def\thenone{}


% content/resources/templates/english-boxes.tex
% This file is currently empty - it exists to maintain consistency with the import structure.
% Add custom environments here if needed in the future.


\begin{document}

\begin{center}
{\Huge\bfseries\color{headcolor} Environment and Sustainability Solutions}\\[5pt]
{\LARGE 4300003 -- Winter 2022}\\[3pt]
{\large Semester 1 Study Material}\\[3pt]
{\normalsize\textit{Detailed Solutions and Explanations}}
\end{center}

\vspace{10pt}

\subsection*{Question 1(a) {[}3 marks{]}}\label{question-1a-3-marks}

\textbf{When ecological overshoot occurs? Explain with reasons.}

\begin{solutionbox}


{\def\LTcaptype{none} % do not increment counter
\vspace{-5pt}
\captionof{table}{Ecological Overshoot Conditions}
\vspace{-10pt}
\begin{longtable}[]{@{}
  >{\raggedright\arraybackslash}p{(\linewidth - 4\tabcolsep) * \real{0.3333}}
  >{\raggedright\arraybackslash}p{(\linewidth - 4\tabcolsep) * \real{0.3939}}
  >{\raggedright\arraybackslash}p{(\linewidth - 4\tabcolsep) * \real{0.2727}}@{}}
\toprule\noalign{}
\begin{minipage}[b]{\linewidth}\raggedright
Condition
\end{minipage} & \begin{minipage}[b]{\linewidth}\raggedright
Description
\end{minipage} & \begin{minipage}[b]{\linewidth}\raggedright
Impact
\end{minipage} \\
\midrule\noalign{}
\endhead
\bottomrule\noalign{}
\endlastfoot
Resource depletion & Consumption exceeds regeneration rate & Deficit
accumulation \\
Population pressure & Human demand surpasses carrying capacity &
Resource scarcity \\
Waste accumulation & Production exceeds absorption capacity &
Environmental degradation \\
\end{longtable}
}

\textbf{Ecological overshoot} occurs when humanity's ecological
footprint exceeds Earth's biocapacity. This happens when we consume
resources faster than nature can regenerate them and produce waste
faster than ecosystems can absorb it.

\textbf{Key reasons include}:

\begin{itemize}
\tightlist
\item
  \textbf{Population growth}: Increasing human numbers
\item
  \textbf{Consumption patterns}: High per-capita resource use
\item
  \textbf{Technology impact}: Inefficient resource utilization
\end{itemize}

\end{solutionbox}
\begin{mnemonicbox}
``POP-CON-TECH'' (Population-Consumption-Technology)

\end{mnemonicbox}
\subsection*{Question 1(b) {[}4 marks{]}}\label{question-1b-4-marks}

\textbf{Explain food chain using diagram.}

\begin{solutionbox}

\begin{center}
\textbf{Mermaid Diagram (Code)}
\begin{verbatim}
{Shaded}
{Highlighting}[]
graph LR
    A[Sun Energy] {-{-}{} B[Producer: Green Plants]}
    B {-{-}{} C[Primary Consumer: Herbivores]}
    C {-{-}{} D[Secondary Consumer: Carnivores]}
    D {-{-}{} E[Tertiary Consumer: Top Predators]}
    E {-{-}{} F[Decomposer: Bacteria/Fungi]}
    F {-{-}{} G[Nutrients to Soil]}
    G {-{-}{} B}
{Highlighting}
{Shaded}
\end{verbatim}
\end{center}

\textbf{Food chain} represents the linear sequence of energy transfer
from one trophic level to another in an ecosystem.

\textbf{Components}:

\begin{itemize}
\tightlist
\item
  \textbf{Producers}: Convert solar energy to chemical energy
\item
  \textbf{Primary consumers}: Feed on producers (herbivores)
\item
  \textbf{Secondary consumers}: Feed on primary consumers (carnivores)
\item
  \textbf{Decomposers}: Break down dead organisms
\end{itemize}

\textbf{Energy flow}: Unidirectional from sun to top predators with 10\%
efficiency between levels.

\end{solutionbox}
\begin{mnemonicbox}
``PPSD'' (Producer-Primary-Secondary-Decomposer)

\end{mnemonicbox}
\subsection*{Question 1(c) {[}7 marks{]}}\label{question-1c-7-marks}

\textbf{Write a note on: carbon cycle.}

\begin{solutionbox}

\begin{center}
\textbf{Mermaid Diagram (Code)}
\begin{verbatim}
{Shaded}
{Highlighting}[]
graph LR
    A[Atmospheric CO2] {-{-}{} B[Photosynthesis]}
    B {-{-}{} C[Plant Biomass]}
    C {-{-}{} D[Animal Consumption]}
    D {-{-}{} E[Respiration]}
    E {-{-}{} A}
    C {-{-}{} F[Decomposition]}
    F {-{-}{} A}
    A {-{-}{} G[Ocean Dissolution]}
    G {-{-}{} H[Marine Life]}
    H {-{-}{} A}
    I[Fossil Fuel Burning] {-{-}{} A}
{Highlighting}
{Shaded}
\end{verbatim}
\end{center}

\textbf{Carbon cycle} is the biogeochemical process where carbon moves
through atmosphere, biosphere, hydrosphere, and geosphere.

\textbf{Major processes}:

\begin{itemize}
\tightlist
\item
  \textbf{Photosynthesis}: Plants absorb CO2 from atmosphere
\item
  \textbf{Respiration}: Organisms release CO2 back to atmosphere
\item
  \textbf{Decomposition}: Dead organic matter releases stored carbon
\item
  \textbf{Ocean exchange}: CO2 dissolves in seawater forming carbonic
  acid
\end{itemize}

\textbf{Human impact}:

\begin{itemize}
\tightlist
\item
  \textbf{Fossil fuel combustion}: Increases atmospheric CO2
\item
  \textbf{Deforestation}: Reduces carbon sequestration capacity
\item
  \textbf{Industrial processes}: Additional carbon emissions
\end{itemize}

\textbf{Environmental significance}: Maintains atmospheric CO2 balance,
regulates global temperature, supports life processes.

\end{solutionbox}
\begin{mnemonicbox}
``PRDO-FDI''
(Photosynthesis-Respiration-Decomposition-Ocean,
Fossil-Deforestation-Industry)

\end{mnemonicbox}
\subsection*{Question 1(c) OR {[}7
marks{]}}\label{question-1c-or-7-marks}

\textbf{Classify aquatic ecosystem. Explain marine ecosystem.}

\begin{solutionbox}


{\def\LTcaptype{none} % do not increment counter
\vspace{-5pt}
\captionof{table}{Aquatic Ecosystem Classification}
\vspace{-10pt}
\begin{longtable}[]{@{}lll@{}}
\toprule\noalign{}
Type & Characteristics & Examples \\
\midrule\noalign{}
\endhead
\bottomrule\noalign{}
\endlastfoot
Freshwater & Low salt content (\textless1\%) & Rivers, lakes, ponds \\
Marine & High salt content (3.5\%) & Oceans, seas \\
Brackish & Mixed fresh-salt water & Estuaries, lagoons \\
\end{longtable}
}

\textbf{Marine Ecosystem Components}:

\begin{center}
\textbf{Mermaid Diagram (Code)}
\begin{verbatim}
{Shaded}
{Highlighting}[]
graph TD
    A[Marine Ecosystem] {-{-}{} B[Pelagic Zone]}
    A {-{-}{} C[Benthic Zone]}
    B {-{-}{} D[Photic Zone: 0{-}200m]}
    B {-{-}{} E[Aphotic Zone: {}200m]}
    C {-{-}{} F[Continental Shelf]}
    C {-{-}{} G[Deep Ocean Floor]}
{Highlighting}
{Shaded}
\end{verbatim}
\end{center}

\textbf{Marine ecosystem} covers 71\% of Earth's surface, containing
saltwater bodies with complex food webs.

\textbf{Zones}:

\begin{itemize}
\tightlist
\item
  \textbf{Pelagic}: Open water column with plankton, fish
\item
  \textbf{Benthic}: Ocean floor with bottom-dwelling organisms
\item
  \textbf{Intertidal}: Shore area between high and low tides
\end{itemize}

\textbf{Importance}:

\begin{itemize}
\tightlist
\item
  \textbf{Climate regulation}: Ocean currents moderate global
  temperature
\item
  \textbf{Oxygen production}: Marine phytoplankton produce 50\% of
  atmospheric oxygen
\item
  \textbf{Economic value}: Fisheries, transportation, tourism
\end{itemize}

\end{solutionbox}
\begin{mnemonicbox}
``PBI-COE'' (Pelagic-Benthic-Intertidal,
Climate-Oxygen-Economy)

\end{mnemonicbox}
\subsection*{Question 2(a) {[}3 marks{]}}\label{question-2a-3-marks}

\textbf{What is carrying capacity of earth?}

\begin{solutionbox}


{\def\LTcaptype{none} % do not increment counter
\vspace{-5pt}
\captionof{table}{Carrying Capacity Factors}
\vspace{-10pt}
\begin{longtable}[]{@{}lll@{}}
\toprule\noalign{}
Factor & Description & Limit \\
\midrule\noalign{}
\endhead
\bottomrule\noalign{}
\endlastfoot
Resources & Available land, water, minerals & Finite \\
Food production & Agricultural capacity & Limited by soil \\
Waste absorption & Ecosystem's waste processing & Saturation point \\
\end{longtable}
}

\textbf{Carrying capacity} is the maximum population size an environment
can sustain indefinitely without degrading the environment.

\textbf{Earth's carrying capacity} depends on:

\begin{itemize}
\tightlist
\item
  \textbf{Resource availability}: Fresh water, arable land, energy
  sources
\item
  \textbf{Technology level}: Efficiency of resource utilization
\item
  \textbf{Consumption patterns}: Per-capita resource demand
\end{itemize}

\textbf{Current estimates}: Range from 4-16 billion people based on
consumption levels and technological advancement.

\end{solutionbox}
\begin{mnemonicbox}
``RTC'' (Resources-Technology-Consumption)

\end{mnemonicbox}
\subsection*{Question 2(b) {[}4 marks{]}}\label{question-2b-4-marks}

\textbf{How food web relates to food chain?}

\begin{solutionbox}

\begin{center}
\textbf{Mermaid Diagram (Code)}
\begin{verbatim}
{Shaded}
{Highlighting}[]
graph LR
    A[Grass] {-{-}{} B[Rabbit]}
    A {-{-}{} C[Deer]}
    B {-{-}{} D[Fox]}
    C {-{-}{} D}
    B {-{-}{} E[Hawk]}
    C {-{-}{} F[Wolf]}
    D {-{-}{} G[Decomposers]}
    E {-{-}{} G}
    F {-{-}{} G}
{Highlighting}
{Shaded}
\end{verbatim}
\end{center}

\textbf{Food web} is an interconnected network of multiple food chains
showing complex feeding relationships in an ecosystem.

\textbf{Relationship between food web and food chain}:

\begin{itemize}
\tightlist
\item
  \textbf{Food chain}: Linear sequence of energy transfer
\item
  \textbf{Food web}: Multiple interconnected food chains
\item
  \textbf{Complexity}: Food webs show realistic ecosystem interactions
\item
  \textbf{Stability}: Multiple pathways provide ecosystem resilience
\end{itemize}

\textbf{Key differences}:

\begin{itemize}
\tightlist
\item
  \textbf{Structure}: Chain is linear, web is network-based
\item
  \textbf{Energy flow}: Chain shows single pathway, web shows multiple
  routes
\item
  \textbf{Species interaction}: Web demonstrates omnivory and
  alternative feeding
\end{itemize}

\end{solutionbox}
\begin{mnemonicbox}
``LNCR'' (Linear-Network, Chain-Resilience)

\end{mnemonicbox}
\subsection*{Question 2(c) {[}7 marks{]}}\label{question-2c-7-marks}

\textbf{Write a note on: air pollution}

\begin{solutionbox}


{\def\LTcaptype{none} % do not increment counter
\vspace{-5pt}
\captionof{table}{Air Pollution Sources and Effects}
\vspace{-10pt}
\begin{longtable}[]{@{}lll@{}}
\toprule\noalign{}
Pollutant & Source & Health Effect \\
\midrule\noalign{}
\endhead
\bottomrule\noalign{}
\endlastfoot
PM2.5/PM10 & Vehicles, industries & Respiratory diseases \\
SO2 & Coal burning & Acid rain, asthma \\
NOx & Vehicle exhaust & Smog formation \\
CO & Incomplete combustion & Oxygen deficiency \\
\end{longtable}
}

\textbf{Air pollution} is contamination of atmosphere by harmful
substances that cause adverse effects on human health and environment.

\textbf{Classification by source}:

\begin{itemize}
\tightlist
\item
  \textbf{Primary pollutants}: Directly emitted (CO, SO2, particulates)
\item
  \textbf{Secondary pollutants}: Formed through chemical reactions
  (ozone, acid rain)
\end{itemize}

\textbf{Major sources}:

\begin{itemize}
\tightlist
\item
  \textbf{Mobile sources}: Vehicles, aircraft, ships
\item
  \textbf{Stationary sources}: Power plants, industries, residential
  heating
\item
  \textbf{Natural sources}: Volcanic eruptions, forest fires, dust
  storms
\end{itemize}

\textbf{Control measures}:

\begin{itemize}
\tightlist
\item
  \textbf{Technological}: Catalytic converters, scrubbers, filters
\item
  \textbf{Regulatory}: Emission standards, fuel quality norms
\item
  \textbf{Alternative energy}: Renewable sources, electric vehicles
\end{itemize}

\textbf{Health impacts}: Respiratory diseases, cardiovascular problems,
cancer, reduced life expectancy.

\textbf{Environmental effects}: Acid rain, ozone depletion, climate
change, visibility reduction.

\end{solutionbox}
\begin{mnemonicbox}
``PSMT-RE-HE''
(Primary-Secondary-Mobile-stationary-Technological-Regulatory-Health-Environment)

\end{mnemonicbox}
\subsection*{Question 2(a) OR {[}3
marks{]}}\label{question-2a-or-3-marks}

\textbf{Explain bad effects of plastic waste on environment.}

\begin{solutionbox}


{\def\LTcaptype{none} % do not increment counter
\vspace{-5pt}
\captionof{table}{Plastic Waste Environmental Effects}
\vspace{-10pt}
\begin{longtable}[]{@{}lll@{}}
\toprule\noalign{}
Impact Area & Effect & Duration \\
\midrule\noalign{}
\endhead
\bottomrule\noalign{}
\endlastfoot
Marine life & Entanglement, ingestion & Persistent \\
Soil & Microplastic contamination & 500+ years \\
Food chain & Bioaccumulation & Generational \\
\end{longtable}
}

\textbf{Plastic waste} causes severe environmental degradation due to
its non-biodegradable nature.

\textbf{Environmental effects}:

\begin{itemize}
\tightlist
\item
  \textbf{Marine pollution}: Ocean plastic kills marine animals through
  entanglement and ingestion
\item
  \textbf{Soil contamination}: Microplastics affect soil fertility and
  crop growth
\item
  \textbf{Food chain disruption}: Plastic particles accumulate in
  organisms
\end{itemize}

\textbf{Long-term impacts}: Persistent organic pollutants, habitat
destruction, ecosystem imbalance.

\end{solutionbox}
\begin{mnemonicbox}
``MSF'' (Marine-Soil-Foodchain)

\end{mnemonicbox}
\subsection*{Question 2(b) OR {[}4
marks{]}}\label{question-2b-or-4-marks}

\textbf{Which are signs of polluted water? List major sources of water
pollution.}

\begin{solutionbox}


{\def\LTcaptype{none} % do not increment counter
\vspace{-5pt}
\captionof{table}{Water Pollution Indicators and Sources}
\vspace{-10pt}
\begin{longtable}[]{@{}lll@{}}
\toprule\noalign{}
Signs & Measurement & Sources \\
\midrule\noalign{}
\endhead
\bottomrule\noalign{}
\endlastfoot
High BOD/COD & \textgreater5 mg/L & Industrial discharge \\
Turbidity & Cloudiness & Agricultural runoff \\
pH changes & \textless6.5 or \textgreater8.5 & Acid mine drainage \\
Foul odor & H2S smell & Sewage discharge \\
\end{longtable}
}

\textbf{Signs of polluted water}:

\begin{itemize}
\tightlist
\item
  \textbf{Physical}: Color change, turbidity, floating debris, odor
\item
  \textbf{Chemical}: High BOD/COD, pH deviation, heavy metals, toxic
  compounds
\item
  \textbf{Biological}: Pathogenic microorganisms, algal blooms, fish
  kills
\end{itemize}

\textbf{Major sources}:

\begin{itemize}
\tightlist
\item
  \textbf{Point sources}: Industrial discharge, sewage outfalls,
  concentrated animal feeding
\item
  \textbf{Non-point sources}: Agricultural runoff, urban stormwater,
  atmospheric deposition
\end{itemize}

\end{solutionbox}
\begin{mnemonicbox}
``PCB-PIN'' (Physical-Chemical-Biological,
Point-Non-point)

\end{mnemonicbox}
\subsection*{Question 2(c) OR {[}7
marks{]}}\label{question-2c-or-7-marks}

\textbf{Classify e-waste. How e-waste is recycled?}

\begin{solutionbox}


{\def\LTcaptype{none} % do not increment counter
\vspace{-5pt}
\captionof{table}{E-waste Classification}
\vspace{-10pt}
\begin{longtable}[]{@{}
  >{\raggedright\arraybackslash}p{(\linewidth - 4\tabcolsep) * \real{0.2439}}
  >{\raggedright\arraybackslash}p{(\linewidth - 4\tabcolsep) * \real{0.2439}}
  >{\raggedright\arraybackslash}p{(\linewidth - 4\tabcolsep) * \real{0.5122}}@{}}
\toprule\noalign{}
\begin{minipage}[b]{\linewidth}\raggedright
Category
\end{minipage} & \begin{minipage}[b]{\linewidth}\raggedright
Examples
\end{minipage} & \begin{minipage}[b]{\linewidth}\raggedright
Hazardous Components
\end{minipage} \\
\midrule\noalign{}
\endhead
\bottomrule\noalign{}
\endlastfoot
Large appliances & Refrigerators, washing machines & CFCs, heavy
metals \\
Small appliances & Microwaves, vacuum cleaners & Plastics, metals \\
IT equipment & Computers, printers & Lead, mercury, cadmium \\
Consumer electronics & TVs, mobile phones & Rare earth elements \\
\end{longtable}
}

\textbf{E-waste classification}:

\begin{itemize}
\tightlist
\item
  \textbf{White goods}: Large household appliances
\item
  \textbf{Brown goods}: Entertainment electronics
\item
  \textbf{Gray goods}: IT and telecommunication equipment
\item
  \textbf{Green goods}: Renewable energy equipment
\end{itemize}

\textbf{E-waste recycling process}:

\begin{center}
\textbf{Mermaid Diagram (Code)}
\begin{verbatim}
{Shaded}
{Highlighting}[]
graph LR
    A[Collection] {-{-}{} B[Sorting]}
    B {-{-}{} C[Dismantling]}
    C {-{-}{} D[Shredding]}
    D {-{-}{} E[Separation]}
    E {-{-}{} F[Material Recovery]}
    F {-{-}{} G[Refining]}
    G {-{-}{} H[New Products]}
{Highlighting}
{Shaded}
\end{verbatim}
\end{center}

\textbf{Recycling methods}:

\begin{itemize}
\tightlist
\item
  \textbf{Mechanical}: Physical separation of materials
\item
  \textbf{Metallurgical}: High-temperature processing for metal recovery
\item
  \textbf{Chemical}: Leaching processes for precious metals
\end{itemize}

\textbf{Challenges}: Hazardous material handling, complex composition,
economic viability.

\textbf{Benefits}: Resource conservation, pollution prevention, job
creation, reduced mining needs.

\end{solutionbox}
\begin{mnemonicbox}
``WBGG-CSDSMR'' (White-Brown-Gray-Green,
Collection-Sorting-Dismantling-Shredding-Separation-Material-Refining)

\end{mnemonicbox}
\subsection*{Question 3(a) {[}3 marks{]}}\label{question-3a-3-marks}

\textbf{Distinguish BOD and COD.}

\begin{solutionbox}


{\def\LTcaptype{none} % do not increment counter
\vspace{-5pt}
\captionof{table}{BOD vs COD Comparison}
\vspace{-10pt}
\begin{longtable}[]{@{}lll@{}}
\toprule\noalign{}
Parameter & BOD & COD \\
\midrule\noalign{}
\endhead
\bottomrule\noalign{}
\endlastfoot
Full form & Biochemical Oxygen Demand & Chemical Oxygen Demand \\
Test duration & 5 days & 2-3 hours \\
Oxidation type & Biological & Chemical \\
Degradation & Biodegradable organics only & All organic compounds \\
\end{longtable}
}

\textbf{BOD (Biochemical Oxygen Demand)}:

\begin{itemize}
\tightlist
\item
  Measures oxygen consumed by microorganisms
\item
  Indicates biodegradable organic pollution
\item
  Standard test: 5 days at 20°C
\end{itemize}

\textbf{COD (Chemical Oxygen Demand)}:

\begin{itemize}
\tightlist
\item
  Measures oxygen required for chemical oxidation
\item
  Indicates total organic pollution
\item
  Uses strong oxidizing agents (potassium dichromate)
\end{itemize}

\end{solutionbox}
\begin{mnemonicbox}
``BTCD'' (Biological-Time-Chemical-Degradation)

\end{mnemonicbox}
\subsection*{Question 3(b) {[}4 marks{]}}\label{question-3b-4-marks}

\textbf{Classify solid waste.}

\begin{solutionbox}


{\def\LTcaptype{none} % do not increment counter
\vspace{-5pt}
\captionof{table}{Solid Waste Classification}
\vspace{-10pt}
\begin{longtable}[]{@{}
  >{\raggedright\arraybackslash}p{(\linewidth - 4\tabcolsep) * \real{0.5000}}
  >{\raggedright\arraybackslash}p{(\linewidth - 4\tabcolsep) * \real{0.1875}}
  >{\raggedright\arraybackslash}p{(\linewidth - 4\tabcolsep) * \real{0.3125}}@{}}
\toprule\noalign{}
\begin{minipage}[b]{\linewidth}\raggedright
Classification
\end{minipage} & \begin{minipage}[b]{\linewidth}\raggedright
Type
\end{minipage} & \begin{minipage}[b]{\linewidth}\raggedright
Examples
\end{minipage} \\
\midrule\noalign{}
\endhead
\bottomrule\noalign{}
\endlastfoot
By source & Municipal, Industrial, Agricultural & Household, Factory,
Farm waste \\
By composition & Organic, Inorganic & Food waste, Plastics \\
By hazard & Hazardous, Non-hazardous & Medical, Paper \\
\end{longtable}
}

\textbf{Solid waste classification}:

\begin{center}
\textbf{Mermaid Diagram (Code)}
\begin{verbatim}
{Shaded}
{Highlighting}[]
graph TD
    A[Solid Waste] {-{-}{} B[Municipal Solid Waste]}
    A {-{-}{} C[Industrial Waste]}
    A {-{-}{} D[Hazardous Waste]}
    A {-{-}{} E[Agricultural Waste]}
    B {-{-}{} F[Organic: 50{-}60\%]}
    B {-{-}{} G[Recyclables: 20{-}30\%]}
    B {-{-}{} H[Inert: 10{-}20\%]}
{Highlighting}
{Shaded}
\end{verbatim}
\end{center}

\textbf{By source}:

\begin{itemize}
\tightlist
\item
  \textbf{Municipal}: Residential, commercial, institutional waste
\item
  \textbf{Industrial}: Manufacturing, processing byproducts
\item
  \textbf{Agricultural}: Crop residues, animal waste
\end{itemize}

\textbf{By composition}: Organic (biodegradable), inorganic
(non-biodegradable), recyclable materials.

\textbf{Management hierarchy}: Reduce, reuse, recycle, recover, dispose.

\end{solutionbox}
\begin{mnemonicbox}
``MIA-OIR'' (Municipal-Industrial-Agricultural,
Organic-Inorganic-Recyclable)

\end{mnemonicbox}
\subsection*{Question 3(c) {[}7 marks{]}}\label{question-3c-7-marks}

\textbf{With the use of diagram explain solar photovoltaic System.}

\begin{solutionbox}

\begin{center}
\textbf{Mermaid Diagram (Code)}
\begin{verbatim}
{Shaded}
{Highlighting}[]
graph LR
    A[Solar Radiation] {-{-}{} B[PV Panel]}
    B {-{-}{} C[DC Power]}
    C {-{-}{} D[Inverter]}
    D {-{-}{} E[AC Power]}
    E {-{-}{} F[Load/Grid]}
    G[Battery] {-{-}{} C}
    E {-{-}{} G}
    H[Charge Controller] {-{-}{} G}
    C {-{-}{} H}
{Highlighting}
{Shaded}
\end{verbatim}
\end{center}

\textbf{Solar Photovoltaic System} converts sunlight directly into
electricity using semiconductor materials.

\textbf{Components}:

\begin{itemize}
\tightlist
\item
  \textbf{PV modules}: Silicon cells convert light to DC electricity
\item
  \textbf{Inverter}: Converts DC to AC power
\item
  \textbf{Battery storage}: Stores excess energy for later use
\item
  \textbf{Charge controller}: Regulates battery charging
\item
  \textbf{Monitoring system}: Tracks performance and faults
\end{itemize}

\textbf{Working principle}:

\begin{enumerate}
\def\labelenumi{\arabic{enumi}.}
\tightlist
\item
  \textbf{Photovoltaic effect}: Solar cells absorb photons
\item
  \textbf{Electron excitation}: Creates electron-hole pairs
\item
  \textbf{Current generation}: Electrons flow creating DC current
\item
  \textbf{Power conditioning}: Inverter converts DC to AC
\end{enumerate}

\textbf{Types}:

\begin{itemize}
\tightlist
\item
  \textbf{Grid-connected}: Synchronized with utility grid
\item
  \textbf{Stand-alone}: Independent systems with battery backup
\item
  \textbf{Hybrid}: Combination of grid-connected and battery storage
\end{itemize}

\textbf{Applications}: Residential rooftops, commercial buildings,
utility-scale power plants, remote area electrification.

\textbf{Advantages}: Clean energy, low maintenance, modular design, long
lifespan (25+ years).

\end{solutionbox}
\begin{mnemonicbox}
``PIBCM-PECG''
(Panel-Inverter-Battery-Controller-Monitor,
Photovoltaic-Electron-Current-Grid)

\end{mnemonicbox}
\subsection*{Question 3(a) OR {[}3
marks{]}}\label{question-3a-or-3-marks}

\textbf{Compare conventional and non-conventional energy sources.}

\begin{solutionbox}


{\def\LTcaptype{none} % do not increment counter
\vspace{-5pt}
\captionof{table}{Energy Sources Comparison}
\vspace{-10pt}
\begin{longtable}[]{@{}lll@{}}
\toprule\noalign{}
Aspect & Conventional & Non-conventional \\
\midrule\noalign{}
\endhead
\bottomrule\noalign{}
\endlastfoot
Availability & Limited reserves & Unlimited/renewable \\
Environmental impact & High pollution & Clean/minimal impact \\
Cost & Initially lower & Decreasing rapidly \\
\end{longtable}
}

\textbf{Conventional energy sources}: Coal, oil, natural gas, nuclear
power - finite resources with environmental concerns.

\textbf{Non-conventional energy sources}: Solar, wind, hydro, biomass -
renewable resources with sustainable characteristics.

\textbf{Key differences}: Depletion vs renewable, pollution vs clean,
established vs emerging technology.

\end{solutionbox}
\begin{mnemonicbox}
``AEC'' (Availability-Environmental-Cost)

\end{mnemonicbox}
\subsection*{Question 3(b) OR {[}4
marks{]}}\label{question-3b-or-4-marks}

\textbf{Explain working of natural circulation solar water heater.}

\begin{solutionbox}

\begin{verbatim}
    +{-{-}{-}{-}{-}{-}{-}{-}{-}{-}{-}{-}{-}{-}{-}{-}{-}{-}+}
    |   Solar Tank     |
    |   (Hot Water)    |
    +{-{-}{-}{-}{-}{-}{-}{-}+{-}{-}{-}{-}{-}{-}{-}{-}{-}+}
             |
    +{-{-}{-}{-}{-}{-}{-}{-}v{-}{-}{-}{-}{-}{-}{-}{-}{-}+}
    |  Solar Collector |
    |     (Cold Water) |
    +{-{-}{-}{-}{-}{-}{-}{-}{-}{-}{-}{-}{-}{-}{-}{-}{-}{-}+}
\end{verbatim}

\textbf{Natural circulation solar water heater} uses thermosiphon
principle for water circulation without external pumps.

\textbf{Working principle}:

\begin{itemize}
\tightlist
\item
  \textbf{Solar collection}: Collector absorbs solar radiation, heating
  water
\item
  \textbf{Density difference}: Hot water becomes less dense, rises
  naturally
\item
  \textbf{Circulation}: Cold water from tank bottom flows to collector
\item
  \textbf{Storage}: Hot water accumulates in insulated storage tank
\end{itemize}

\textbf{Components}: Flat plate collector, insulated storage tank,
connecting pipes, safety valves.

\textbf{Advantages}: No electricity required, simple design, low
maintenance, cost-effective.

\end{solutionbox}
\begin{mnemonicbox}
``SDCS'' (Solar-Density-Circulation-Storage)

\end{mnemonicbox}
\subsection*{Question 3(c) OR {[}7
marks{]}}\label{question-3c-or-7-marks}

\textbf{Explain working principle of horizontal axis wind turbine.}

\begin{solutionbox}

\begin{center}
\textbf{Mermaid Diagram (Code)}
\begin{verbatim}
{Shaded}
{Highlighting}[]
graph LR
    A[Wind Energy] {-{-}{} B[Rotor Blades]}
    B {-{-}{} C[Shaft Rotation]}
    C {-{-}{} D[Gearbox]}
    D {-{-}{} E[Generator]}
    E {-{-}{} F[Electrical Power]}
    G[Nacelle] {-{-}{} B}
    H[Tower] {-{-}{} G}
{Highlighting}
{Shaded}
\end{verbatim}
\end{center}

\textbf{Horizontal Axis Wind Turbine (HAWT)} converts kinetic energy of
wind into electrical energy using aerodynamic lift principle.

\textbf{Working principle}:

\begin{enumerate}
\def\labelenumi{\arabic{enumi}.}
\tightlist
\item
  \textbf{Wind capture}: Rotor blades designed with aerodynamic profile
\item
  \textbf{Lift generation}: Pressure difference across blade surfaces
  creates lift force
\item
  \textbf{Rotation}: Lift force causes rotor to rotate around horizontal
  axis
\item
  \textbf{Speed conversion}: Gearbox increases rotational speed from
  30-50 rpm to 1500 rpm
\item
  \textbf{Power generation}: High-speed rotation drives electrical
  generator
\end{enumerate}

\textbf{Components}:

\begin{itemize}
\tightlist
\item
  \textbf{Rotor assembly}: 2-3 blades, hub, pitch control system
\item
  \textbf{Nacelle}: Houses gearbox, generator, control systems
\item
  \textbf{Tower}: Supports nacelle at optimal height (50-120m)
\item
  \textbf{Foundation}: Concrete base for structural stability
\end{itemize}

\textbf{Control systems}:

\begin{itemize}
\tightlist
\item
  \textbf{Yaw system}: Orients turbine to face wind direction
\item
  \textbf{Pitch control}: Adjusts blade angle for optimal wind capture
\item
  \textbf{Brake system}: Emergency stopping mechanism
\end{itemize}

\textbf{Advantages}: High efficiency (35-45\%), proven technology,
economies of scale. \textbf{Disadvantages}: Visual impact, noise, bird
strikes, wind variability.

\textbf{Power calculation}: P = 0.5 × ρ × A × V³ × Cp Where: ρ = air
density, A = swept area, V = wind speed, Cp = power coefficient

\end{solutionbox}
\begin{mnemonicbox}
``WLRSG-RNTP-YPB''
(Wind-Lift-Rotation-Speed-Generation, Rotor-Nacelle-Tower-Foundation,
Yaw-Pitch-Brake)

\end{mnemonicbox}
\subsection*{Question 4(a) {[}3 marks{]}}\label{question-4a-3-marks}

\textbf{Write advantages and disadvantages of tidal energy.}

\begin{solutionbox}


{\def\LTcaptype{none} % do not increment counter
\vspace{-5pt}
\captionof{table}{Tidal Energy Pros and Cons}
\vspace{-10pt}
\begin{longtable}[]{@{}ll@{}}
\toprule\noalign{}
Advantages & Disadvantages \\
\midrule\noalign{}
\endhead
\bottomrule\noalign{}
\endlastfoot
Predictable energy source & Limited suitable locations \\
No greenhouse gas emissions & High initial capital cost \\
Long lifespan (100+ years) & Environmental impact on marine life \\
\end{longtable}
}

\textbf{Tidal energy} harnesses gravitational forces between Earth,
moon, and sun to generate electricity.

\textbf{Advantages}:

\begin{itemize}
\tightlist
\item
  \textbf{Reliability}: Highly predictable tidal cycles
\item
  \textbf{Clean energy}: Zero operational emissions
\item
  \textbf{Durability}: Infrastructure lasts decades
\end{itemize}

\textbf{Disadvantages}:

\begin{itemize}
\tightlist
\item
  \textbf{Geographic limitations}: Requires specific coastal conditions
\item
  \textbf{High costs}: Expensive installation and maintenance
\item
  \textbf{Ecological impact}: Affects marine ecosystems
\end{itemize}

\end{solutionbox}
\begin{mnemonicbox}
``RCD-GHE'' (Reliable-Clean-Durable, Geographic-High
cost-Ecological)

\end{mnemonicbox}
\subsection*{Question 4(b) {[}4 marks{]}}\label{question-4b-4-marks}

\textbf{Explain working principle of biogas plant.}

\begin{solutionbox}

\begin{center}
\textbf{Mermaid Diagram (Code)}
\begin{verbatim}
{Shaded}
{Highlighting}[]
graph LR
    A[Organic Waste Input] {-{-}{} B[Mixing Tank]}
    B {-{-}{} C[Digester Tank]}
    C {-{-}{} D[Gas Collection]}
    C {-{-}{} E[Slurry Output]}
    D {-{-}{} F[Biogas Storage]}
    F {-{-}{} G[End Use]}
{Highlighting}
{Shaded}
\end{verbatim}
\end{center}

\textbf{Biogas plant} produces methane-rich gas through anaerobic
digestion of organic waste materials.

\textbf{Working principle}:

\begin{enumerate}
\def\labelenumi{\arabic{enumi}.}
\tightlist
\item
  \textbf{Feed preparation}: Organic waste mixed with water (1:1 ratio)
\item
  \textbf{Anaerobic digestion}: Bacteria break down organic matter in
  oxygen-free environment
\item
  \textbf{Gas production}: Methane (50-70\%) and CO2 (30-40\%) generated
\item
  \textbf{Gas collection}: Biogas collected in gas holder dome
\end{enumerate}

\textbf{Process stages}:

\begin{itemize}
\tightlist
\item
  \textbf{Hydrolysis}: Complex organics broken into simple compounds
\item
  \textbf{Acidogenesis}: Organic acids formation
\item
  \textbf{Methanogenesis}: Methane production by methanogenic bacteria
\end{itemize}

\textbf{Optimal conditions}: Temperature 35-40°C, pH 6.8-7.2, retention
time 15-30 days.

\end{solutionbox}
\begin{mnemonicbox}
``FAGH-HAM'' (Feed-Anaerobic-Gas-Holder,
Hydrolysis-Acidogenesis-Methanogenesis)

\end{mnemonicbox}
\subsection*{Question 4(c) {[}7 marks{]}}\label{question-4c-7-marks}

\textbf{Explain green house effect.}

\begin{solutionbox}

\begin{center}
\textbf{Mermaid Diagram (Code)}
\begin{verbatim}
{Shaded}
{Highlighting}[]
graph LR
    A[Solar Radiation] {-{-}{} B[Earth{}s Surface]}
    B {-{-}{} C[Heat Absorption]}
    C {-{-}{} D[Infrared Radiation]}
    D {-{-}{} E[Greenhouse Gases]}
    E {-{-}{} F[Heat Trapping]}
    F {-{-}{} G[Re{-}radiation to Earth]}
    G {-{-}{} H[Global Warming]}
{Highlighting}
{Shaded}
\end{verbatim}
\end{center}

\textbf{Greenhouse effect} is the process where atmospheric gases trap
heat from sun, warming Earth's surface beyond normal temperature.

\textbf{Natural greenhouse effect}:

\begin{itemize}
\tightlist
\item
  \textbf{Solar radiation}: Sun emits short-wave radiation (visible
  light)
\item
  \textbf{Surface absorption}: Earth absorbs solar energy, heats up
\item
  \textbf{Heat re-emission}: Earth emits long-wave infrared radiation
\item
  \textbf{Gas absorption}: Greenhouse gases absorb infrared radiation
\item
  \textbf{Heat retention}: Trapped heat warms lower atmosphere
\end{itemize}

\textbf{Greenhouse gases and contributions}:

\begin{itemize}
\tightlist
\item
  \textbf{Carbon dioxide (CO2)}: 76\% - fossil fuel burning,
  deforestation
\item
  \textbf{Methane (CH4)}: 16\% - agriculture, landfills, livestock
\item
  \textbf{Nitrous oxide (N2O)}: 6\% - fertilizers, fossil fuel
  combustion
\item
  \textbf{Fluorinated gases}: 2\% - industrial processes, refrigeration
\end{itemize}

\textbf{Enhanced greenhouse effect}: Human activities increase
greenhouse gas concentrations, intensifying heat trapping.

\textbf{Consequences}:

\begin{itemize}
\tightlist
\item
  \textbf{Global temperature rise}: Average 1.1°C increase since
  pre-industrial times
\item
  \textbf{Climate change}: Altered precipitation patterns, extreme
  weather events
\item
  \textbf{Sea level rise}: Thermal expansion and ice sheet melting
\item
  \textbf{Ecosystem disruption}: Species migration, coral bleaching,
  forest fires
\end{itemize}

\textbf{Mitigation strategies}:

\begin{itemize}
\tightlist
\item
  \textbf{Renewable energy}: Reduce fossil fuel dependence
\item
  \textbf{Energy efficiency}: Improve technology and practices
\item
  \textbf{Carbon sequestration}: Forest restoration, carbon capture
  storage
\item
  \textbf{International cooperation}: Paris Agreement, emission
  reduction targets
\end{itemize}

\end{solutionbox}
\begin{mnemonicbox}
``SSAHR-CMNO-GTSE-RECC''
(Solar-Surface-Absorption-Heat-Radiation, CO2-Methane-Nitrous-Other,
Global-Temperature-Sea-Ecosystem,
Renewable-Efficiency-Carbon-Cooperation)

\end{mnemonicbox}
\subsection*{Question 4(a) OR {[}3
marks{]}}\label{question-4a-or-3-marks}

\textbf{What is climate change?}

\begin{solutionbox}


{\def\LTcaptype{none} % do not increment counter
\vspace{-5pt}
\captionof{table}{Climate Change Indicators}
\vspace{-10pt}
\begin{longtable}[]{@{}lll@{}}
\toprule\noalign{}
Indicator & Change & Evidence \\
\midrule\noalign{}
\endhead
\bottomrule\noalign{}
\endlastfoot
Temperature & +1.1°C since 1880 & Global temperature records \\
Sea level & +21 cm since 1900 & Satellite measurements \\
Arctic ice & -13\% per decade & Satellite imagery \\
\end{longtable}
}

\textbf{Climate change} refers to long-term shifts in global
temperatures and weather patterns, primarily caused by human activities
since mid-20th century.

\textbf{Key characteristics}:

\begin{itemize}
\tightlist
\item
  \textbf{Temperature rise}: Global average temperature increase
\item
  \textbf{Weather extremes}: More frequent hurricanes, droughts, floods
\item
  \textbf{Ecosystem changes}: Species migration, habitat loss
\end{itemize}

\textbf{Primary cause}: Increased greenhouse gas emissions from fossil
fuel burning, deforestation, industrial processes.

\end{solutionbox}
\begin{mnemonicbox}
``TSE'' (Temperature-Sea level-Ecosystem)

\end{mnemonicbox}
\subsection*{Question 4(b) OR {[}4
marks{]}}\label{question-4b-or-4-marks}

\textbf{Write some of measures to control global warming.}

\begin{solutionbox}


{\def\LTcaptype{none} % do not increment counter
\vspace{-5pt}
\captionof{table}{Global Warming Control Measures}
\vspace{-10pt}
\begin{longtable}[]{@{}
  >{\raggedright\arraybackslash}p{(\linewidth - 4\tabcolsep) * \real{0.3448}}
  >{\raggedright\arraybackslash}p{(\linewidth - 4\tabcolsep) * \real{0.3448}}
  >{\raggedright\arraybackslash}p{(\linewidth - 4\tabcolsep) * \real{0.3103}}@{}}
\toprule\noalign{}
\begin{minipage}[b]{\linewidth}\raggedright
Category
\end{minipage} & \begin{minipage}[b]{\linewidth}\raggedright
Measures
\end{minipage} & \begin{minipage}[b]{\linewidth}\raggedright
Impact
\end{minipage} \\
\midrule\noalign{}
\endhead
\bottomrule\noalign{}
\endlastfoot
Energy & Renewable sources, efficiency & Reduce CO2 emissions \\
Transport & Electric vehicles, public transport & Lower fuel
consumption \\
Industry & Clean technology, carbon capture & Emission reduction \\
Individual & Energy conservation, lifestyle changes & Cumulative
effect \\
\end{longtable}
}

\textbf{Control measures}:

\textbf{Government level}:

\begin{itemize}
\tightlist
\item
  \textbf{Policy frameworks}: Carbon pricing, emission standards
\item
  \textbf{Renewable energy}: Solar, wind power promotion
\item
  \textbf{Public transport}: Mass transit system development
\end{itemize}

\textbf{Industrial level}:

\begin{itemize}
\tightlist
\item
  \textbf{Clean technology}: Efficient processes, waste reduction
\item
  \textbf{Carbon capture}: Storage and utilization technologies
\item
  \textbf{Sustainable practices}: Green manufacturing, circular economy
\end{itemize}

\textbf{Individual level}:

\begin{itemize}
\tightlist
\item
  \textbf{Energy conservation}: LED lights, efficient appliances
\item
  \textbf{Transportation}: Walking, cycling, carpooling
\item
  \textbf{Lifestyle changes}: Reduced consumption, recycling
\end{itemize}

\end{solutionbox}
\begin{mnemonicbox}
``PRT-CCS-ECL'' (Policy-Renewable-Transport,
Carbon-Clean-Sustainable, Energy-Communication-Lifestyle)

\end{mnemonicbox}
\subsection*{Question 4(c) OR {[}7
marks{]}}\label{question-4c-or-7-marks}

\textbf{Which are some important agreements for mitigating climate
change at global level?}

\begin{solutionbox}


{\def\LTcaptype{none} % do not increment counter
\vspace{-5pt}
\captionof{table}{Major Climate Agreements}
\vspace{-10pt}
\begin{longtable}[]{@{}lll@{}}
\toprule\noalign{}
Agreement & Year & Key Features \\
\midrule\noalign{}
\endhead
\bottomrule\noalign{}
\endlastfoot
UNFCCC & 1992 & Framework convention \\
Kyoto Protocol & 1997 & Binding emission targets \\
Paris Agreement & 2015 & Global temperature limit \\
\end{longtable}
}

\textbf{Important global climate agreements}:

\textbf{1. United Nations Framework Convention on Climate Change
(UNFCCC) - 1992}:

\begin{itemize}
\tightlist
\item
  \textbf{Objective}: Stabilize greenhouse gas concentrations
\item
  \textbf{Principles}: Common but differentiated responsibilities
\item
  \textbf{Framework}: Foundation for future climate negotiations
\end{itemize}

\textbf{2. Kyoto Protocol - 1997}:

\begin{itemize}
\tightlist
\item
  \textbf{Binding targets}: Developed countries reduce emissions by
  5.2\% (1990 levels)
\item
  \textbf{Flexible mechanisms}: Emissions trading, clean development
  mechanism
\item
  \textbf{Commitment periods}: First (2008-2012), Second (2013-2020)
\end{itemize}

\textbf{3. Paris Agreement - 2015}:

\begin{itemize}
\tightlist
\item
  \textbf{Temperature goal}: Limit global warming to well below 2°C,
  preferably 1.5°C
\item
  \textbf{Nationally Determined Contributions (NDCs)}: Countries set own
  targets
\item
  \textbf{Review mechanism}: Five-year assessment and enhancement cycles
\item
  \textbf{Climate finance}: \$100 billion annually for developing
  countries
\end{itemize}

\textbf{4. Other significant agreements}:

\begin{itemize}
\tightlist
\item
  \textbf{Montreal Protocol (1987)}: Ozone layer protection, indirect
  climate benefits
\item
  \textbf{Copenhagen Accord (2009)}: Political agreement on emission
  reductions
\item
  \textbf{Doha Amendment (2012)}: Extended Kyoto Protocol commitments
\end{itemize}

\textbf{Implementation challenges}:

\begin{itemize}
\tightlist
\item
  \textbf{Compliance}: Voluntary vs mandatory commitments
\item
  \textbf{Financing}: Adequate funding for mitigation and adaptation
\item
  \textbf{Technology transfer}: Clean technology access for developing
  countries
\item
  \textbf{Monitoring}: Transparent reporting and verification systems
\end{itemize}

\textbf{Recent developments}:

\begin{itemize}
\tightlist
\item
  \textbf{Article 6 rules}: International carbon markets under Paris
  Agreement
\item
  \textbf{Loss and damage}: Support for climate-vulnerable countries
\item
  \textbf{Net-zero commitments}: Countries pledging carbon neutrality
\end{itemize}

\end{solutionbox}
\begin{mnemonicbox}
``UKPOM-CDOG-TFMC''
(UNFCCC-Kyoto-Paris-Other-Montreal, Copenhagen-Doha-Other-Goals,
Technology-Finance-Monitoring-Commitments)

\end{mnemonicbox}
\subsection*{Question 5(a) {[}3 marks{]}}\label{question-5a-3-marks}

\textbf{Explain effects of ozone layer depletion.}

\begin{solutionbox}


{\def\LTcaptype{none} % do not increment counter
\vspace{-5pt}
\captionof{table}{Ozone Depletion Effects}
\vspace{-10pt}
\begin{longtable}[]{@{}lll@{}}
\toprule\noalign{}
Impact Area & Effect & Consequence \\
\midrule\noalign{}
\endhead
\bottomrule\noalign{}
\endlastfoot
Human health & Increased UV-B radiation & Skin cancer, cataracts \\
Environment & Ecosystem disruption & Marine food chain damage \\
Agriculture & Crop damage & Reduced food production \\
\end{longtable}
}

\textbf{Ozone layer depletion} results in increased ultraviolet-B (UV-B)
radiation reaching Earth's surface.

\textbf{Effects}:

\begin{itemize}
\tightlist
\item
  \textbf{Human health}: Higher skin cancer rates, eye damage, immune
  system suppression
\item
  \textbf{Marine ecosystems}: Phytoplankton reduction affects ocean food
  chains
\item
  \textbf{Agricultural impact}: Reduced crop yields, plant growth
  inhibition
\end{itemize}

\textbf{Cause}: Chlorofluorocarbons (CFCs) destroy ozone molecules in
stratosphere.

\end{solutionbox}
\begin{mnemonicbox}
``HMA'' (Human-Marine-Agricultural)

\end{mnemonicbox}
\subsection*{Question 5(b) {[}4 marks{]}}\label{question-5b-4-marks}

\textbf{Write short note on greenhouse gases.}

\begin{solutionbox}


{\def\LTcaptype{none} % do not increment counter
\vspace{-5pt}
\captionof{table}{Major Greenhouse Gases}
\vspace{-10pt}
\begin{longtable}[]{@{}lll@{}}
\toprule\noalign{}
Gas & Sources & Global Warming Potential \\
\midrule\noalign{}
\endhead
\bottomrule\noalign{}
\endlastfoot
CO2 & Fossil fuels, deforestation & 1 (reference) \\
CH4 & Agriculture, landfills & 25 times CO2 \\
N2O & Fertilizers, combustion & 298 times CO2 \\
F-gases & Industrial processes & 1,000-20,000 times CO2 \\
\end{longtable}
}

\textbf{Greenhouse gases} are atmospheric compounds that trap heat
radiated from Earth's surface.

\textbf{Major greenhouse gases}:

\begin{itemize}
\tightlist
\item
  \textbf{Carbon dioxide (CO2)}: Most abundant, from fossil fuel burning
\item
  \textbf{Methane (CH4)}: Potent but shorter-lived, from agriculture
\item
  \textbf{Nitrous oxide (N2O)}: Long-lived, from fertilizers and
  industry
\item
  \textbf{Fluorinated gases}: Very potent, from refrigeration and
  industrial uses
\end{itemize}

\textbf{Properties}: Absorb infrared radiation, transparent to visible
light, varying atmospheric lifespans.

\textbf{Global warming potential}: Measures heat-trapping capacity
relative to CO2 over specific time periods.

\end{solutionbox}
\begin{mnemonicbox}
``CMNF'' (Carbon dioxide-Methane-Nitrous
oxide-Fluorinated gases)

\end{mnemonicbox}
\subsection*{Question 5(c) {[}7 marks{]}}\label{question-5c-7-marks}

\textbf{Explain Concept of 5R.}

\begin{solutionbox}

\begin{center}
\textbf{Mermaid Diagram (Code)}
\begin{verbatim}
{Shaded}
{Highlighting}[]
graph TD
    A[5R Concept] {-{-}{} B[Refuse]}
    A {-{-}{} C[Reduce]}
    A {-{-}{} D[Reuse]}
    A {-{-}{} E[Repurpose]}
    A {-{-}{} F[Recycle]}
    B {-{-}{} G[Avoid unnecessary items]}
    C {-{-}{} H[Minimize consumption]}
    D {-{-}{} I[Use items multiple times]}
    E {-{-}{} J[Find new uses]}
    F {-{-}{} K[Process into new products]}
{Highlighting}
{Shaded}
\end{verbatim}
\end{center}

\textbf{5R Concept} is a waste management hierarchy that prioritizes
waste prevention and resource conservation.

\textbf{The Five R's in order of priority}:

\textbf{1. Refuse}:

\begin{itemize}
\tightlist
\item
  \textbf{Definition}: Avoid accepting unnecessary items
\item
  \textbf{Examples}: Single-use plastics, promotional freebies,
  excessive packaging
\item
  \textbf{Impact}: Prevents waste generation at source
\end{itemize}

\textbf{2. Reduce}:

\begin{itemize}
\tightlist
\item
  \textbf{Definition}: Minimize consumption and waste production
\item
  \textbf{Examples}: Buy only needed items, choose durable products,
  energy conservation
\item
  \textbf{Impact}: Decreases resource extraction and waste volume
\end{itemize}

\textbf{3. Reuse}:

\begin{itemize}
\tightlist
\item
  \textbf{Definition}: Use items multiple times in their original form
\item
  \textbf{Examples}: Glass jars for storage, clothing donation,
  furniture repurposing
\item
  \textbf{Impact}: Extends product lifespan, reduces replacement needs
\end{itemize}

\textbf{4. Repurpose}:

\begin{itemize}
\tightlist
\item
  \textbf{Definition}: Find new applications for items instead of
  discarding
\item
  \textbf{Examples}: Tire planters, bottle vases, cardboard organizers
\item
  \textbf{Impact}: Creative waste diversion, artistic value addition
\end{itemize}

\textbf{5. Recycle}:

\begin{itemize}
\tightlist
\item
  \textbf{Definition}: Process waste materials into new products
\item
  \textbf{Examples}: Paper recycling, metal recovery, plastic
  reprocessing
\item
  \textbf{Impact}: Resource recovery, reduced landfill burden
\end{itemize}

\textbf{Benefits of 5R approach}:

\begin{itemize}
\tightlist
\item
  \textbf{Environmental}: Reduced pollution, resource conservation,
  ecosystem protection
\item
  \textbf{Economic}: Cost savings, job creation in recycling industry
\item
  \textbf{Social}: Community awareness, sustainable lifestyle promotion
\end{itemize}

\textbf{Implementation hierarchy}: Focus on refuse and reduce first
(prevention), then reuse and repurpose (waste diversion), finally
recycle (waste processing).

\textbf{Challenges}: Behavioral change requirements, infrastructure
development, economic incentives alignment.

\end{solutionbox}
\begin{mnemonicbox}
``Real Recycling Requires Refusing Rubbish''
(Refuse-Reduce-Reuse-Repurpose-Recycle)

\end{mnemonicbox}
\subsection*{Question 5(a) OR {[}3
marks{]}}\label{question-5a-or-3-marks}

\textbf{Write salient features of wild life protection act, 1972.}

\begin{solutionbox}


{\def\LTcaptype{none} % do not increment counter
\vspace{-5pt}
\captionof{table}{Wildlife Protection Act 1972 Features}
\vspace{-10pt}
\begin{longtable}[]{@{}
  >{\raggedright\arraybackslash}p{(\linewidth - 4\tabcolsep) * \real{0.2903}}
  >{\raggedright\arraybackslash}p{(\linewidth - 4\tabcolsep) * \real{0.4194}}
  >{\raggedright\arraybackslash}p{(\linewidth - 4\tabcolsep) * \real{0.2903}}@{}}
\toprule\noalign{}
\begin{minipage}[b]{\linewidth}\raggedright
Feature
\end{minipage} & \begin{minipage}[b]{\linewidth}\raggedright
Description
\end{minipage} & \begin{minipage}[b]{\linewidth}\raggedright
Penalty
\end{minipage} \\
\midrule\noalign{}
\endhead
\bottomrule\noalign{}
\endlastfoot
Protected species & Scheduled animals/plants & Fine + imprisonment \\
Hunting ban & Prohibition of hunting & Up to 7 years jail \\
Trade regulation & Wildlife product trade control & Confiscation +
fine \\
\end{longtable}
}

\textbf{Wildlife Protection Act, 1972} provides legal framework for
conservation of wildlife in India.

\textbf{Salient features}:

\begin{itemize}
\tightlist
\item
  \textbf{Species protection}: Six schedules categorizing species by
  protection level
\item
  \textbf{Hunting prohibition}: Complete ban on hunting of protected
  species
\item
  \textbf{Habitat conservation}: Protected areas designation and
  management
\item
  \textbf{Trade control}: Regulation of wildlife product commerce
\end{itemize}

\textbf{Enforcement}: Wildlife Crime Control Bureau, forest departments,
special courts for wildlife offenses.

\textbf{Amendments}: Regular updates to include new species and
strengthen provisions.

\end{solutionbox}
\begin{mnemonicbox}
``SHTE'' (Species-Hunting-Trade-Enforcement)

\end{mnemonicbox}
\subsection*{Question 5(b) OR {[}4
marks{]}}\label{question-5b-or-4-marks}

\textbf{Which are the environmental policies in India?}

\begin{solutionbox}


{\def\LTcaptype{none} % do not increment counter
\vspace{-5pt}
\captionof{table}{Major Environmental Policies in India}
\vspace{-10pt}
\begin{longtable}[]{@{}
  >{\raggedright\arraybackslash}p{(\linewidth - 4\tabcolsep) * \real{0.3077}}
  >{\raggedright\arraybackslash}p{(\linewidth - 4\tabcolsep) * \real{0.2308}}
  >{\raggedright\arraybackslash}p{(\linewidth - 4\tabcolsep) * \real{0.4615}}@{}}
\toprule\noalign{}
\begin{minipage}[b]{\linewidth}\raggedright
Policy
\end{minipage} & \begin{minipage}[b]{\linewidth}\raggedright
Year
\end{minipage} & \begin{minipage}[b]{\linewidth}\raggedright
Focus Area
\end{minipage} \\
\midrule\noalign{}
\endhead
\bottomrule\noalign{}
\endlastfoot
National Environment Policy & 2006 & Comprehensive framework \\
National Water Policy & 2012 & Water resource management \\
National Forest Policy & 1988 & Forest conservation \\
National Action Plan on Climate Change & 2008 & Climate change
mitigation \\
\end{longtable}
}

\textbf{Major environmental policies}:

\textbf{National Environment Policy (2006)}:

\begin{itemize}
\tightlist
\item
  \textbf{Objective}: Sustainable development with environmental
  protection
\item
  \textbf{Principles}: Polluter pays, precautionary approach
\item
  \textbf{Implementation}: Integration across sectors
\end{itemize}

\textbf{Sectoral policies}:

\begin{itemize}
\tightlist
\item
  \textbf{National Water Policy}: Integrated water resource management
\item
  \textbf{National Forest Policy}: 33\% forest cover target
\item
  \textbf{National Solar Mission}: Renewable energy promotion
\item
  \textbf{Waste Management Rules}: Solid waste, e-waste, plastic waste
  management
\end{itemize}

\textbf{Regulatory framework}: Environment Protection Act, Water Act,
Air Act, Forest Conservation Act.

\end{solutionbox}
\begin{mnemonicbox}
``NWFS'' (National-Water-Forest-Solar)

\end{mnemonicbox}
\subsection*{Question 5(c) OR {[}7
marks{]}}\label{question-5c-or-7-marks}

\textbf{Explain rainwater harvesting in detail.}

\begin{solutionbox}

\begin{center}
\textbf{Mermaid Diagram (Code)}
\begin{verbatim}
{Shaded}
{Highlighting}[]
graph LR
    A[Rainfall] {-{-}{} B[Catchment Area]}
    B {-{-}{} C[Collection System]}
    C {-{-}{} D[First Flush Diverter]}
    D {-{-}{} E[Filtration]}
    E {-{-}{} F[Storage Tank]}
    F {-{-}{} G[Distribution]}
    H[Recharge Pit] {-{-}{} I[Groundwater]}
    C {-{-}{} H}
{Highlighting}
{Shaded}
\end{verbatim}
\end{center}

\textbf{Rainwater harvesting} is the collection, storage, and
utilization of rainwater for beneficial purposes.

\textbf{Components of rainwater harvesting system}:

\textbf{1. Catchment area}:

\begin{itemize}
\tightlist
\item
  \textbf{Function}: Surface for rain collection (rooftops, open areas)
\item
  \textbf{Material}: Should be clean, non-toxic (avoid asbestos,
  lead-painted surfaces)
\item
  \textbf{Calculation}: Collection = Catchment area × Rainfall × Runoff
  coefficient
\end{itemize}

\textbf{2. Collection and conveyance system}:

\begin{itemize}
\tightlist
\item
  \textbf{Gutters}: Channel water from catchment surface
\item
  \textbf{Downspouts}: Vertical pipes carrying water from gutters
\item
  \textbf{Transportation}: Pipes connecting different components
\end{itemize}

\textbf{3. First flush diverter}:

\begin{itemize}
\tightlist
\item
  \textbf{Purpose}: Removes initial dirty water containing debris
\item
  \textbf{Types}: Manual valve, automatic diverter, floating ball system
\item
  \textbf{Capacity}: Usually 10-15 liters per 100 sq.m of roof area
\end{itemize}

\textbf{4. Filtration system}:

\begin{itemize}
\tightlist
\item
  \textbf{Coarse filter}: Removes leaves, debris (mesh screen)
\item
  \textbf{Fine filter}: Sand, gravel, activated carbon
\item
  \textbf{Slow sand filter}: Biological treatment for drinking water
\end{itemize}

\textbf{5. Storage system}:

\begin{itemize}
\tightlist
\item
  \textbf{Surface storage}: Tanks, reservoirs above ground
\item
  \textbf{Underground storage}: Sumps, cisterns below ground
\item
  \textbf{Material}: Ferrocement, plastic, concrete, fiberglass
\end{itemize}

\textbf{Types of rainwater harvesting}:

\textbf{A. Rooftop harvesting}:

\begin{itemize}
\tightlist
\item
  \textbf{Direct storage}: Rainwater stored in tanks for immediate use
\item
  \textbf{Indirect recharge}: Water directed to recharge groundwater
\end{itemize}

\textbf{B. Surface water harvesting}:

\begin{itemize}
\tightlist
\item
  \textbf{Check dams}: Small barriers across streams
\item
  \textbf{Percolation tanks}: Artificial recharge structures
\item
  \textbf{Contour bunding}: Soil conservation with water harvesting
\end{itemize}

\textbf{Benefits}:

\begin{itemize}
\tightlist
\item
  \textbf{Water security}: Reduces dependence on external water sources
\item
  \textbf{Groundwater recharge}: Prevents water table decline
\item
  \textbf{Flood control}: Reduces surface runoff and urban flooding
\item
  \textbf{Quality improvement}: Generally better than groundwater in
  polluted areas
\item
  \textbf{Cost-effective}: Lower than water supply schemes
\item
  \textbf{Energy saving}: Reduces pumping requirements
\end{itemize}

\textbf{Design considerations}:

\begin{itemize}
\tightlist
\item
  \textbf{Rainfall pattern}: Seasonal distribution, intensity
\item
  \textbf{Water demand}: Household requirements, usage patterns
\item
  \textbf{Storage capacity}: Based on dry period duration
\item
  \textbf{Quality requirements}: Potable vs non-potable use
\item
  \textbf{Site conditions}: Space availability, soil permeability
\end{itemize}

\textbf{Maintenance requirements}:

\begin{itemize}
\tightlist
\item
  \textbf{Regular cleaning}: Gutters, filters, storage tanks
\item
  \textbf{Roof maintenance}: Prevent contamination sources
\item
  \textbf{System inspection}: Check for leaks, blockages
\item
  \textbf{Water quality testing}: Periodic analysis for potable use
\end{itemize}

\textbf{Government initiatives}:

\begin{itemize}
\tightlist
\item
  \textbf{Building codes}: Mandatory rainwater harvesting in new
  constructions
\item
  \textbf{Subsidies}: Financial incentives for installation
\item
  \textbf{Awareness programs}: Community education and training
\item
  \textbf{Technical support}: Design guidelines, implementation
  assistance
\end{itemize}

\textbf{Challenges}:

\begin{itemize}
\tightlist
\item
  \textbf{Initial cost}: Setup expenses for complete system
\item
  \textbf{Maintenance}: Regular upkeep requirements
\item
  \textbf{Space requirements}: Storage tank space needs
\item
  \textbf{Seasonal availability}: Dependence on monsoon patterns
\item
  \textbf{Quality concerns}: Potential contamination issues
\end{itemize}

\textbf{Calculation example}:

\begin{itemize}
\tightlist
\item
  Roof area: 100 sq.m
\item
  Annual rainfall: 1000 mm
\item
  Runoff coefficient: 0.8
\item
  Harvestable water = 100 × 1 × 0.8 = 80,000 liters/year
\end{itemize}

\end{solutionbox}
\begin{mnemonicbox}
``CCFFS-RSBD-WGFQC-RCSMQ''
(Catchment-Collection-Flush-Filter-Storage,
Rooftop-Surface-Benefits-Design, Water-Groundwater-Flood-Quality-Cost,
Regular-Check-System-Maintenance-Quality)

\end{mnemonicbox}
\end{document}