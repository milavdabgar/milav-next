\documentclass[10pt,a4paper]{article}

% content/resources/templates/preamble.tex
\usepackage[margin=0.6in]{geometry}
\author{Milav Dabgar}
\usepackage{amsmath,amssymb,amsthm}
\usepackage{booktabs}
\usepackage{multirow}
\usepackage{xcolor}
\usepackage{tcolorbox}
\tcbuselibrary{breakable,skins}
\usepackage[colorlinks=true,linkcolor=blue]{hyperref}
\usepackage{titlesec}
\usepackage{enumitem}
\usepackage{tikz}
\usepackage{pgfplots}
\usepackage{circuitikz}
\usepackage[version=4]{mhchem}
\usepackage{longtable}
\usepackage{array}
\usepackage{float}
\usepackage{caption}
\usepackage{listings}

\lstset{
  basicstyle=\small\ttfamily,
  breaklines=true,
  breakatwhitespace=false,
  postbreak=\mbox{\textcolor{red}{$\hookrightarrow$}\space},
  float=false,
  numbers=left,
  numberstyle=\tiny\color{gray},
  numbersep=10pt,
  xleftmargin=2em,
  keywordstyle=\color{blue},
  commentstyle=\color{green!60!black},
  stringstyle=\color{purple},
  backgroundcolor=\color{gray!5},
  showstringspaces=false,
  tabsize=2,
  captionpos=b,
  keepspaces=true,
  columns=flexible
}

\pgfplotsset{compat=1.18}
\usetikzlibrary{shapes,arrows,positioning,calc,patterns,decorations.pathmorphing,decorations.markings,arrows.meta}

% Color scheme
\definecolor{headcolor}{RGB}{0,102,204}
\definecolor{keycolor}{RGB}{220,20,60}
\definecolor{solutioncolor}{RGB}{34,139,34}
\definecolor{mnemoniccolor}{RGB}{148,0,211}
\definecolor{codecolor}{RGB}{0,0,100}

% Spacing
\setlength{\parskip}{3pt}
\setlist[itemize]{nosep}
\setlist[enumerate]{nosep}

% Title formatting
\titleformat{\section}{\Large\bfseries\color{headcolor}}{\thesection}{1em}{}
\titleformat{\subsection}{\large\bfseries\color{headcolor}}{\thesubsection}{1em}{}

% Pandoc tightlist compatibility
\providecommand{\tightlist}{%
  \setlength{\itemsep}{0pt}\setlength{\parskip}{0pt}}

% Pandoc longtable compatibility
\newcounter{none}
\def\thenone{}


% content/resources/templates/gujarati-boxes.tex
\usepackage{fontspec}
\usepackage{polyglossia}

% Set Gujarati as main language (document is primarily in Gujarati)
% Note: gloss-gujarati.ldf doesn't exist in polyglossia, but it will use hyphenation patterns
\setdefaultlanguage{gujarati}
\setotherlanguage{english}

% Configure Gujarati font properly
% Use Language=Default to prevent polyglossia from trying to add language-specific features
% that don't exist for Gujarati, which causes "empty feature" warnings
\newfontfamily\gujaratifont[Script=Gujarati,AutoFakeBold=2.5,AutoFakeSlant=0.3]{Noto Sans Gujarati}
\setmainfont[Script=Gujarati,AutoFakeBold=2.5,AutoFakeSlant=0.3]{Noto Sans Gujarati}
% Use Noto Sans Gujarati for monospace to support Gujarati in text
\setmonofont[Scale=0.9]{Noto Sans Gujarati}

% Configure English to use the same font
\newfontfamily\englishfont[Script=Gujarati,AutoFakeBold=2.5,AutoFakeSlant=0.3]{Noto Sans Gujarati}

% Translations for polyglossia
\gappto\captionsgujarati{
  \renewcommand{\tablename}{કોષ્ટક}
  \renewcommand{\figurename}{આકૃતિ}
}

% Helper for TikZ nodes to ensure Gujarati font
\newcommand{\gu}[1]{{\gujaratifont #1}}

% Custom environments
\newtcolorbox{solutionbox}{
    breakable,
    enhanced,
    colback=solutioncolor!5!white,
    colframe=solutioncolor!75!black,
    fonttitle=\bfseries,
    title=જવાબ
}

\newtcolorbox{solutionboxnobreak}{
 colback=solutioncolor!5!white,
 colframe=solutioncolor!75!black,
 fonttitle=\bfseries,
 title=જવાબ
}

\newtcolorbox{keyformula}{
 breakable,
 enhanced,
 colback=keycolor!5!white,
 colframe=keycolor!75!black,
 fonttitle=\bfseries,
 title=રાસાયણિક સમીકરણ/સૂત્ર
}

\newtcolorbox{mnemonicbox}{
 breakable,
 enhanced,
 colback=mnemoniccolor!5!white,
 colframe=mnemoniccolor!75!black,
 fonttitle=\bfseries,
 title=મેમરી ટ્રીક
}


\begin{document}

\begin{center}
{\Huge\bfseries\color{headcolor} Environment and Sustainability (Gujarati)}\\[5pt]
{\LARGE 4300003 -- Winter 2023}\\[3pt]
{\large Semester 1 Study Material}\\[3pt]
{\normalsize\textit{Detailed Solutions and Explanations}}
\end{center}

\vspace{10pt}

\subsection*{પ્રશ્ન 1(અ) {[}03
ગુણ{]}}\label{uxaaauxab0uxab6uxaa8-1uxa85-03-uxa97uxaa3}

\textbf{ઇકોલોજીકલ ફૂટપ્રિન્ટ સમજાવો.}

\begin{solutionbox}

ઇકોલોજીકલ ફૂટપ્રિન્ટ એ વ્યક્તિઓ, સમુદાયો અથવા દેશો દ્વારા પ્રકૃતિ પરની માંગને જૈવિક
રીતે ઉત્પાદક જમીન અને પાણીના વિસ્તારના સંદર્ભમાં માપે છે.


{\def\LTcaptype{none} % do not increment counter
\vspace{-5pt}
\captionof{table}{ઇકોલોજીકલ ફૂટપ્રિન્ટના ઘટકો}
\vspace{-10pt}
\begin{longtable}[]{@{}ll@{}}
\toprule\noalign{}
ઘટક & વર્ણન \\
\midrule\noalign{}
\endhead
\bottomrule\noalign{}
\endlastfoot
\textbf{કાર્બન ફૂટપ્રિન્ટ} & CO₂ ઉત્સર્જન શોષવા માટે જરૂરી જમીન \\
\textbf{કૃષિ જમીન} & ખોરાક ઉત્પાદન માટે વિસ્તાર \\
\textbf{ચરાઈ જમીન} & પશુધન માટે વિસ્તાર \\
\textbf{વન ઉત્પાદનો} & લાકડા અને કાગળ માટે વિસ્તાર \\
\textbf{નિર્મિત જમીન} & આધારભૂત સુવિધાઓ અને શહેરી વિસ્તારો \\
\end{longtable}
}

\begin{itemize}
\tightlist
\item
  \textbf{વૈશ્વિક હેક્ટર}: માપન માટે માનક એકમ
\item
  \textbf{ઓવરશૂટ}: જ્યારે ફૂટપ્રિન્ટ બાયોકેપેસિટી કરતાં વધે
\item
  \textbf{ટકાઉપણું}: વપરાશ અને પુનઃઉત્પાદન વચ્ચે સંતુલન
\end{itemize}

\end{solutionbox}
\begin{mnemonicbox}
``CGFBB'' - Carbon, Cropland, Grazing, Forest,
Built-up

\begin{center}\rule{0.5\linewidth}{0.5pt}\end{center}

\end{mnemonicbox}
\subsection*{પ્રશ્ન 1(બ) {[}04
ગુણ{]}}\label{uxaaauxab0uxab6uxaa8-1uxaac-04-uxa97uxaa3}

\textbf{એલ્ટોનિયન પિરામિડ સમજાવો.}

\begin{solutionbox}

એલ્ટોનિયન પિરામિડ (સંખ્યાનો પિરામિડ) ઇકોસિસ્ટમમાં દરેક પોષક સ્તરે જીવોની સંખ્યા
દર્શાવે છે, જે ચાર્લ્સ એલ્ટન દ્વારા પ્રસ્તાવિત કરવામાં આવ્યો હતો.

\textbf{આકૃતિ:}

\begin{verbatim}
Tertiary Consumers
(થોડા {- 10)}
         
Secondary Consumers  
(મધ્યમ {- 100)}
      
Primary Consumers
(ઘણા {- 1000)}
     
Producers
(સૌથી વધુ {- 10000)}
\end{verbatim}


{\def\LTcaptype{none} % do not increment counter
\vspace{-5pt}
\captionof{table}{પિરામિડના પ્રકારો}
\vspace{-10pt}
\begin{longtable}[]{@{}lll@{}}
\toprule\noalign{}
પ્રકાર & આધાર & આકાર \\
\midrule\noalign{}
\endhead
\bottomrule\noalign{}
\endlastfoot
\textbf{સંખ્યા} & વ્યક્તિગત ગણતરી & સામાન્ય રીતે સીધો \\
\textbf{બાયોમાસ} & કુલ વજન & ઊંધો પણ હોઈ શકે \\
\textbf{ઊર્જા} & ઊર્જા પ્રવાહ & હંમેશા સીધો \\
\end{longtable}
}

\begin{itemize}
\tightlist
\item
  \textbf{પોષક સ્તરો}: ખોરાક શૃંખલામાં ખોરાકની સ્થિતિ
\item
  \textbf{10\% નિયમ}: માત્ર 10\% ઊર્જા આગલા સ્તરે સ્થાનાંતરિત થાય
\item
  \textbf{અપવાદો}: વૃક્ષ ઇકોસિસ્ટમ ઊંધો સંખ્યા પિરામિડ દર્શાવે
\end{itemize}

\end{solutionbox}
\begin{mnemonicbox}
``ELTON'' - Energy Loss Through Organism Numbers

\begin{center}\rule{0.5\linewidth}{0.5pt}\end{center}

\end{mnemonicbox}
\subsection*{પ્રશ્ન 1(ક) {[}07
ગુણ{]}}\label{uxaaauxab0uxab6uxaa8-1uxa95-07-uxa97uxaa3}

\textbf{ઇકો-સિસ્ટમ તેના વર્ગીકરણ અને ઘટક સાથે સમજાવો.}

\begin{solutionbox}

ઇકોસિસ્ટમ એ પ્રકૃતિની એક કાર્યાત્મક એકમ છે જ્યાં જીવંત સજીવો એકબીજા સાથે અને તેમના
ભૌતિક વાતાવરણ સાથે ક્રિયાપ્રતિક્રિયા કરે છે, જેમાં ઊર્જા પ્રવાહ અને પોષક ચક્રણ સામેલ
છે.


{\def\LTcaptype{none} % do not increment counter
\vspace{-5pt}
\captionof{table}{ઇકોસિસ્ટમના ઘટકો}
\vspace{-10pt}
\begin{longtable}[]{@{}lll@{}}
\toprule\noalign{}
ઘટક & પ્રકાર & ઉદાહરણો \\
\midrule\noalign{}
\endhead
\bottomrule\noalign{}
\endlastfoot
\textbf{અજૈવિક} & નિર્જીવ & હવા, પાણી, માટી, આબોહવા \\
\textbf{જૈવિક} & સજીવ & છોડ, પ્રાણીઓ, સૂક્ષ્મજીવો \\
\textbf{ઉત્પાદકો} & સ્વપોષક & લીલા છોડ, શેવાળ \\
\textbf{ઉપભોક્તાઓ} & પરપોષક & શાકાહારી, માંસાહારી, સર્વાહારી \\
\textbf{વિઘટનકર્તા} & પુનર્ચક્રીકરણકર્તા & બેક્ટેરિયા, ફૂગ \\
\end{longtable}
}

\textbf{ઇકોસિસ્ટમનું વર્ગીકરણ:}

\textbf{કુદરતી ઇકોસિસ્ટમ:}

\begin{itemize}
\tightlist
\item
  \textbf{સ્થલીય}: જંગલ, ઘાસના મેદાનો, રણ
\item
  \textbf{જળીય}: તાજા પાણી (તળાવ, નદી), દરિયાઈ (મહાસાગર, સમુદ્ર)
\end{itemize}

\textbf{કૃત્રિમ ઇકોસિસ્ટમ:}

\begin{itemize}
\tightlist
\item
  \textbf{કૃષિ}: પાકના ખેતરો, બગીચાઓ
\item
  \textbf{શહેરી}: ઉદ્યાનો, કૃત્રિમ તળાવો
\end{itemize}

\textbf{આકૃતિ: ઊર્જા પ્રવાહ}

\begin{verbatim}
flowchart LR
    A[સૂર્ય] {-{-} B[ઉત્પાદકો]}
    B {-{-} C[પ્રાથમિક ઉપભોક્તાઓ]}
    C {-{-} D[ગૌણ ઉપભોક્તાઓ]}
    D {-{-} E[તૃતીયક ઉપભોક્તાઓ]}
    F[વિઘટનકર્તા] {-{-} B}
    C {-{-} F}
    D {-{-} F}
    E {-{-} F}
\end{verbatim}

\begin{itemize}
\tightlist
\item
  \textbf{ઊર્જા પ્રવાહ}: સૂર્યથી વિઘટનકર્તા સુધી એક દિશામાં
\item
  \textbf{પોષક ચક્રણ}: તત્વોની ચક્રીય હિલચાલ
\item
  \textbf{ખોરાક શૃંખલા}: રેખીય ઊર્જા સ્થાનાંતરણ
\item
  \textbf{ખોરાક જાળ}: પરસ્પર જોડાયેલી ખોરાક શૃંખલાઓ
\end{itemize}

\end{solutionbox}
\begin{mnemonicbox}
``PEACE'' - Producers, Energy, Animals, Cycles,
Environment

\begin{center}\rule{0.5\linewidth}{0.5pt}\end{center}

\end{mnemonicbox}
\subsection*{પ્રશ્ન 1(ક અથવા) {[}07
ગુણ{]}}\label{uxaaauxab0uxab6uxaa8-1uxa95-uxa85uxaa5uxab5-07-uxa97uxaa3}

\textbf{નાઈટ્રોજન ચક્ર સમજાવો.}

\begin{solutionbox}

નાઈટ્રોજન ચક્ર એ બાયોજિયોકેમિકલ ચક્ર છે જે વાતાવરણ, સ્થલીય અને જળીય પ્રણાલીઓમાં
ફરતા વખતે નાઈટ્રોજન સંયોજનોને વિવિધ રાસાયણિક સ્વરૂપોમાં રૂપાંતરિત કરે છે.

\textbf{આકૃતિ: નાઈટ્રોજન ચક્ર}

\begin{verbatim}
flowchart LR
    A[વાતાવરણીય N₂] {-{-} B[નાઈટ્રોજન સ્થિરીકરણ]}
    B {-{-} C[અમોનિયા NH₃]}
    C {-{-} D[નાઈટ્રિફિકેશન]}
    D {-{-} E[નાઈટ્રાઈટ NO₂⁻]}
    E {-{-} F[નાઈટ્રેટ NO₃⁻]}
    F {-{-} G[વનસ્પતિ શોષણ]}
    G {-{-} H[પ્રાણી વપરાશ]}
    H {-{-} I[વિઘટન]}
    I {-{-} C}
    F {-{-} J[ડી{-}નાઈટ્રિફિકેશન]}
    J {-{-} A}
\end{verbatim}


{\def\LTcaptype{none} % do not increment counter
\vspace{-5pt}
\captionof{table}{નાઈટ્રોજન ચક્રની પ્રક્રિયાઓ}
\vspace{-10pt}
\begin{longtable}[]{@{}lll@{}}
\toprule\noalign{}
પ્રક્રિયા & રૂપાંતરણ & સજીવો \\
\midrule\noalign{}
\endhead
\bottomrule\noalign{}
\endlastfoot
\textbf{સ્થિરીકરણ} & N₂ → NH₃ & રાઈઝોબિયમ, એઝોટોબેક્ટર \\
\textbf{નાઈટ્રિફિકેશન} & NH₃ → NO₂⁻ → NO₃⁻ & નાઈટ્રોસોમોનાસ, નાઈટ્રોબેક્ટર \\
\textbf{આત્મસાત્કરણ} & NO₃⁻ → પ્રોટીન & છોડવા \\
\textbf{વિઘટન} & પ્રોટીન → NH₃ & બેક્ટેરિયા, ફૂગ \\
\textbf{ડી-નાઈટ્રિફિકેશન} & NO₃⁻ → N₂ & એનેરોબિક બેક્ટેરિયા \\
\end{longtable}
}

\begin{itemize}
\tightlist
\item
  \textbf{જૈવિક સ્થિરીકરણ}: કુલ સ્થિરીકરણનો 80\%
\item
  \textbf{ઔદ્યોગિક સ્થિરીકરણ}: ખાતર માટે હેબર પ્રક્રિયા
\item
  \textbf{વીજળી}: કુદરતી વાતાવરણીય સ્થિરીકરણ
\item
  \textbf{પ્રદૂષણ}: વધારાના નાઈટ્રેટ યુટ્રોફિકેશન કારણે
\end{itemize}

\end{solutionbox}
\begin{mnemonicbox}
``FNADD'' - Fixation, Nitrification, Assimilation,
Decomposition, Denitrification

\begin{center}\rule{0.5\linewidth}{0.5pt}\end{center}

\end{mnemonicbox}
\subsection*{પ્રશ્ન 2(અ) {[}03
ગુણ{]}}\label{uxaaauxab0uxab6uxaa8-2uxa85-03-uxa97uxaa3}

\textbf{વેસ્ટ વોટર ક્વોલિટી પેરામીટરની યાદી બનાવો.}

\begin{solutionbox}


{\def\LTcaptype{none} % do not increment counter
\vspace{-5pt}
\captionof{table}{વેસ્ટ વોટર ક્વોલિટી પેરામીટર}
\vspace{-10pt}
\begin{longtable}[]{@{}lll@{}}
\toprule\noalign{}
ભૌતિક & રાસાયણિક & જૈવિક \\
\midrule\noalign{}
\endhead
\bottomrule\noalign{}
\endlastfoot
\textbf{ટર્બિડિટી} & \textbf{BOD} & \textbf{કોલિફોર્મ ગણતરી} \\
\textbf{રંગ} & \textbf{COD} & \textbf{પેથોજેનિક બેક્ટેરિયા} \\
\textbf{ગંધ} & \textbf{pH} & \textbf{શેવાળ} \\
\textbf{તાપમાન} & \textbf{DO} & \textbf{વાયરસ} \\
\textbf{કુલ ઘન પદાર્થો} & \textbf{અમોનિયા} & \textbf{પ્રોટોઝોઆ} \\
\end{longtable}
}

\begin{itemize}
\tightlist
\item
  \textbf{પ્રાથમિક પેરામીટર}: BOD, COD, pH, સસ્પેન્ડેડ સોલિડ્સ
\item
  \textbf{ગૌણ પેરામીટર}: ભારે ધાતુઓ, પોષક તત્વો
\item
  \textbf{સૂચક સજીવો}: મળના દૂષણ માટે E.coli
\end{itemize}

\end{solutionbox}
\begin{mnemonicbox}
``PCB'' - Physical, Chemical, Biological parameters

\begin{center}\rule{0.5\linewidth}{0.5pt}\end{center}

\end{mnemonicbox}
\subsection*{પ્રશ્ન 2(બ) {[}04
ગુણ{]}}\label{uxaaauxab0uxab6uxaa8-2uxaac-04-uxa97uxaa3}

\textbf{ઈ-કચરાનું વર્ગીકરણ અને અસરો સમજાવો.}

\begin{solutionbox}

ઈલેક્ટ્રોનિક કચરો (ઈ-વેસ્ટ) એ હાનિકારક સામગ્રી ધરાવતા છોડી દેવાયેલા વિદ્યુત અને
ઈલેક્ટ્રોનિક સાધનોનો સંદર્ભ આપે છે.


{\def\LTcaptype{none} % do not increment counter
\vspace{-5pt}
\captionof{table}{ઈ-વેસ્ટ વર્ગીકરણ}
\vspace{-10pt}
\begin{longtable}[]{@{}lll@{}}
\toprule\noalign{}
કેટેગરી & ઉદાહરણો & હાનિકારક સામગ્રી \\
\midrule\noalign{}
\endhead
\bottomrule\noalign{}
\endlastfoot
\textbf{મોટા ઉપકરણો} & રેફ્રિજરેટર, વોશિંગ મશીન & CFCs, ભારે ધાતુઓ \\
\textbf{નાના ઉપકરણો} & માઈક્રોવેવ, ટોસ્ટર & લીડ, મર્ક્યુરી \\
\textbf{IT સાધનો} & કમ્પ્યુટર, પ્રિન્ટર & કેડમિયમ, ક્રોમિયમ \\
\textbf{ટેલિકોમ સાધનો} & મોબાઈલ ફોન, કેબલ & બેરિલિયમ, ફ્લેમ રિટાર્ડન્ટ \\
\textbf{કન્ઝ્યુમર ઈલેક્ટ્રોનિક્સ} & ટીવી, રેડિયો & પોલિવિનાઈલ ક્લોરાઈડ (PVC) \\
\end{longtable}
}

\textbf{ઈ-વેસ્ટની અસરો:}

\begin{itemize}
\tightlist
\item
  \textbf{પર્યાવરણીય}: માટી અને પાણીનું પ્રદૂષણ, હવાનું દૂષણ
\item
  \textbf{આરોગ્ય}: કેન્સર, ન્યુરોલોજિકલ વિકાર, શ્વસન સમસ્યાઓ
\item
  \textbf{સંસાધન ક્ષય}: સોના, ચાંદી જેવી મૂલ્યવાન ધાતુઓનું નુકસાન
\item
  \textbf{ઇકોસિસ્ટમ નુકસાન}: ખોરાક શૃંખલામાં બાયોએક્યુમ્યુલેશન
\end{itemize}

\end{solutionbox}
\begin{mnemonicbox}
``LSITC'' - Large, Small, IT, Telecom, Consumer
electronics

\begin{center}\rule{0.5\linewidth}{0.5pt}\end{center}

\end{mnemonicbox}
\subsection*{પ્રશ્ન 2(ક) {[}07
ગુણ{]}}\label{uxaaauxab0uxab6uxaa8-2uxa95-07-uxa97uxaa3}

\textbf{ઈલેક્ટ્રોસ્ટેટિક પ્રીસીપીટેટર સમજાવો.}

\begin{solutionbox}

ઈલેક્ટ્રોસ્ટેટિક પ્રીસીપીટેટર (ESP) એ હવા પ્રદૂષણ નિયંત્રણ ઉપકરણ છે જે વિદ્યુત ચાર્જનો
ઉપયોગ કરીને ઔદ્યોગિક ગેસ પ્રવાહમાંથી કણોનો દ્રવ્ય દૂર કરે છે.

\textbf{આકૃતિ: ESP કામગીરી}

\begin{verbatim}
ગંદો ગેસ →  |─────────────────| → સાફ ગેસ
ઇનપુટ       | + ઇલેક્ટ્રોડ    |   આઉટપુટ
             |                 |
             | {- કલેક્શન      |}
             |   પ્લેટ         |
             |                 |
             | ધૂળ કલેક્શન    |
             | હોપર          |
             |\_\_\_\_\_\_\_\_\_\_\_\_\_\_\_\_\_|
\end{verbatim}


{\def\LTcaptype{none} % do not increment counter
\vspace{-5pt}
\captionof{table}{ESP ઘટકો અને કાર્યો}
\vspace{-10pt}
\begin{longtable}[]{@{}lll@{}}
\toprule\noalign{}
ઘટક & કાર્ય & સામગ્રી \\
\midrule\noalign{}
\endhead
\bottomrule\noalign{}
\endlastfoot
\textbf{ડિસચાર્જ ઈલેક્ટ્રોડ} & કોરોના ડિસચાર્જ બનાવે & ટંગસ્ટન વાયર \\
\textbf{કલેક્શન પ્લેટ} & ચાર્જ કરેલા કણોને આકર્ષે & સ્ટીલ પ્લેટ્સ \\
\textbf{હાઈ વોલ્ટેજ સપ્લાઈ} & 30-100 kV DC પ્રદાન કરે &
ટ્રાન્સફોર્મર-રેક્ટિફાયર \\
\textbf{રેપર સિસ્ટમ} & એકત્રિત ધૂળ દૂર કરે & યાંત્રિક વાઈબ્રેટર \\
\textbf{હોપર} & પડેલા કણો એકત્રિત કરે & સ્ટીલ કન્ટેનર \\
\end{longtable}
}

\textbf{કામકાજનો સિદ્ધાંત:}

\begin{enumerate}
\def\labelenumi{\arabic{enumi}.}
\tightlist
\item
  \textbf{આયનીકરણ}: હાઈ વોલ્ટેજ કોરોના ડિસચાર્જ બનાવે
\item
  \textbf{ચાર્જિંગ}: કણો નકારાત્મક ચાર્જ મેળવે
\item
  \textbf{કલેક્શન}: ચાર્જ કરેલા કણો સકારાત્મક પ્લેટ્સ તરફ જાય
\item
  \textbf{દૂર કરવું}: રેપિંગ એકત્રિત ધૂળને છૂટી કરે
\end{enumerate}

\textbf{ઉપયોગો:}

\begin{itemize}
\tightlist
\item
  \textbf{પાવર પ્લાન્ટ્સ}: કોલસાથી ચાલતા બોઈલર
\item
  \textbf{સિમેન્ટ ઉદ્યોગ}: ભઠ્ઠાના ગેસ સફાઈ
\item
  \textbf{સ્ટીલ ઉદ્યોગ}: બ્લાસ્ટ ફર્નેસ ગેસ
\item
  \textbf{કેમિકલ પ્લાન્ટ્સ}: પ્રોસેસ ગેસ ટ્રીટમેન્ટ
\end{itemize}

\textbf{ફાયદાઓ:}

\begin{itemize}
\tightlist
\item
  \textbf{ઉચ્ચ કાર્યક્ષમતા}: બારીક કણો માટે 99\%+ દૂર કરવું
\item
  \textbf{ઓછું પ્રેશર ડ્રોપ}: ઊર્જા કાર્યક્ષમ કામગીરી
\item
  \textbf{ઉચ્ચ તાપમાન સંભાળે}: 400°C સુધી
\end{itemize}

\end{solutionbox}
\begin{mnemonicbox}
``CHARGE'' - Corona, High-voltage, Attract, Rapper,
Gas, Efficiency

\begin{center}\rule{0.5\linewidth}{0.5pt}\end{center}

\end{mnemonicbox}
\subsection*{પ્રશ્ન 2(અ અથવા) {[}03
ગુણ{]}}\label{uxaaauxab0uxab6uxaa8-2uxa85-uxa85uxaa5uxab5-03-uxa97uxaa3}

\textbf{સમજાવો (1) BOD (2) COD}

\begin{solutionbox}


{\def\LTcaptype{none} % do not increment counter
\vspace{-5pt}
\captionof{table}{BOD vs COD}
\vspace{-10pt}
\begin{longtable}[]{@{}lll@{}}
\toprule\noalign{}
પેરામીટર & BOD & COD \\
\midrule\noalign{}
\endhead
\bottomrule\noalign{}
\endlastfoot
\textbf{પૂરું નામ} & બાયોકેમિકલ ઓક્સિજન ડિમાન્ડ & કેમિકલ ઓક્સિજન ડિમાન્ડ \\
\textbf{પદ્ધતિ} & જૈવિક ઓક્સિડેશન & રાસાયણિક ઓક્સિડેશન \\
\textbf{સમય} & 20°C પર 5 દિવસ & 2-3 કલાક \\
\textbf{ઓક્સિડાઈઝિંગ એજન્ટ} & સૂક્ષ્મજીવો & પોટેશિયમ ડાઈક્રોમેટ \\
\end{longtable}
}

\textbf{(1) BOD (બાયોકેમિકલ ઓક્સિજન ડિમાન્ડ):}

\begin{itemize}
\tightlist
\item
  \textbf{વ્યાખ્યા}: કાર્બનિક પદાર્થને વિઘટન કરવા માટે સૂક્ષ્મજીવો દ્વારા જરૂરી
  ઓક્સિજન
\item
  \textbf{પ્રમાણભૂત પરિસ્થિતિઓ}: 5 દિવસ, 20°C, અંધકારની સ્થિતિ
\item
  \textbf{એકમો}: mg/L અથવા ppm
\end{itemize}

\textbf{(2) COD (કેમિકલ ઓક્સિજન ડિમાન્ડ):}

\begin{itemize}
\tightlist
\item
  \textbf{વ્યાખ્યા}: કાર્બનિક પદાર્થને રાસાયણિક રીતે ઓક્સિડાઈઝ કરવા માટે ઓક્સિજન
  સમકક્ષ
\item
  \textbf{ઓક્સિડાઈઝિંગ એજન્ટ}: અમ્લીય માધ્યમમાં K₂Cr₂O₇
\item
  \textbf{BOD કરતાં ઊંચું}: બિન-બાયોડિગ્રેડેબલ સંયોજનો સામેલ
\end{itemize}

\end{solutionbox}
\begin{mnemonicbox}
``BTCO'' - Biological Time, Chemical Oxidation

\begin{center}\rule{0.5\linewidth}{0.5pt}\end{center}

\end{mnemonicbox}
\subsection*{પ્રશ્ન 2(બ અથવા) {[}04
ગુણ{]}}\label{uxaaauxab0uxab6uxaa8-2uxaac-uxa85uxaa5uxab5-04-uxa97uxaa3}

\textbf{ઇ-કચરાનું રિસાયકલ સમજાવો.}

\begin{solutionbox}

ઇ-વેસ્ટ રિસાયક્લિંગ એ હાનિકારક પદાર્થોના સુરક્ષિત નિકાલ સાથે ઇલેક્ટ્રોનિક કચરામાંથી
મૂલ્યવાન સામગ્રી પુનઃપ્રાપ્ત કરવાની પ્રક્રિયા છે.


{\def\LTcaptype{none} % do not increment counter
\vspace{-5pt}
\captionof{table}{ઇ-વેસ્ટ રિસાયક્લિંગ પ્રક્રિયા}
\vspace{-10pt}
\begin{longtable}[]{@{}lll@{}}
\toprule\noalign{}
તબક્કો & પ્રક્રિયા & પુનઃપ્રાપ્તિ \\
\midrule\noalign{}
\endhead
\bottomrule\noalign{}
\endlastfoot
\textbf{કલેક્શન} & ઘરો, ઓફિસોમાંથી એકત્રીકરણ & સંપૂર્ણ ઉપકરણો \\
\textbf{ડિસમેન્ટલિંગ} & ઘટકોનું મેન્યુઅલ વિભાજન & પ્લાસ્ટિક, ધાતુઓ, સર્કિટ બોર્ડ \\
\textbf{શ્રેડિંગ} & યાંત્રિક કદ ઘટાડો & મિશ્ર સામગ્રી પ્રવાહ \\
\textbf{વિભાજન} & ચુંબકીય, ઘનતા, ઓપ્ટિકલ સોર્ટિંગ & ફેરસ, નોન-ફેરસ ધાતુઓ \\
\textbf{શુદ્ધિકરણ} & રાસાયણિક પ્રક્રિયા & શુદ્ધ ધાતુઓ (Au, Ag, Cu, Pd) \\
\end{longtable}
}

\textbf{રિસાયક્લિંગ પદ્ધતિઓ:}

\begin{itemize}
\tightlist
\item
  \textbf{યાંત્રિક}: ભૌતિક વિભાજન અને કદ ઘટાડો
\item
  \textbf{પાયરોમેટલર્જી}: ઉચ્ચ તાપમાન ધાતુ પુનઃપ્રાપ્તિ
\item
  \textbf{હાઇડ્રોમેટલર્જી}: રાસાયણિક લીચિંગ પ્રક્રિયાઓ
\item
  \textbf{બાયોટેકનોલોજી}: સૂક્ષ્મજીવીય ધાતુ નિષ્કર્ષણ
\end{itemize}

\textbf{ફાયદાઓ:}

\begin{itemize}
\tightlist
\item
  \textbf{સંસાધન સંરક્ષણ}: કિંમતી ધાતુઓની પુનઃપ્રાપ્તિ
\item
  \textbf{પર્યાવરણ સંરક્ષણ}: માટી અને પાણીનું દૂષણ અટકાવે
\item
  \textbf{આર્થિક મૂલ્ય}: નોકરીઓ સર્જન અને આવક ઉત્પાદન
\item
  \textbf{ઊર્જા બચત}: પ્રાથમિક ઉત્પાદન કરતાં ઓછી ઊર્જા
\end{itemize}

\end{solutionbox}
\begin{mnemonicbox}
``CDSPR'' - Collection, Dismantling, Shredding,
Separation, Refining

\begin{center}\rule{0.5\linewidth}{0.5pt}\end{center}

\end{mnemonicbox}
\subsection*{પ્રશ્ન 2(ક અથવા) {[}07
ગુણ{]}}\label{uxaaauxab0uxab6uxaa8-2uxa95-uxa85uxaa5uxab5-07-uxa97uxaa3}

\textbf{પ્રદૂષણ અને તેના સ્ત્રોતને વ્યાખ્યાયિત કરો. પ્રદૂષકોનું વર્ગીકરણ સમજાવો.}

\begin{solutionbox}

\textbf{વ્યાખ્યા:} પ્રદૂષણ એ પર્યાવરણમાં હાનિકારક પદાર્થો અથવા ઊર્જાનો પ્રવેશ છે, જે
હવા, પાણી, માટી અથવા સજીવોમાં પ્રતિકૂળ ફેરફારોનું કારણ બને છે.


{\def\LTcaptype{none} % do not increment counter
\vspace{-5pt}
\captionof{table}{પ્રદૂષણના સ્ત્રોતો}
\vspace{-10pt}
\begin{longtable}[]{@{}
  >{\raggedright\arraybackslash}p{(\linewidth - 4\tabcolsep) * \real{0.3095}}
  >{\raggedright\arraybackslash}p{(\linewidth - 4\tabcolsep) * \real{0.2381}}
  >{\raggedright\arraybackslash}p{(\linewidth - 4\tabcolsep) * \real{0.4524}}@{}}
\toprule\noalign{}
\begin{minipage}[b]{\linewidth}\raggedright
સ્ત્રોત પ્રકાર
\end{minipage} & \begin{minipage}[b]{\linewidth}\raggedright
ઉદાહરણો
\end{minipage} & \begin{minipage}[b]{\linewidth}\raggedright
બહાર પાડવામાં આવતા પ્રદૂષકો
\end{minipage} \\
\midrule\noalign{}
\endhead
\bottomrule\noalign{}
\endlastfoot
\textbf{પોઈન્ટ સોર્સ} & ઔદ્યોગિક ચીમની, ગટર આઉટફોલ & ચોક્કસ સ્થાન ડિસચાર્જ \\
\textbf{નોન-પોઈન્ટ સોર્સ} & કૃષિ રનઓફ, શહેરી વરસાદી પાણી & ફેલાયેલા વિસ્તારનું
પ્રદૂષણ \\
\textbf{મોબાઈલ સોર્સ} & વાહનો, જહાજો, વિમાનો & એક્ઝોસ્ટ એમિશન \\
\textbf{સ્ટેશનરી સોર્સ} & પાવર પ્લાન્ટ, ફેક્ટરીઓ & સ્ટેક એમિશન \\
\end{longtable}
}

\textbf{પ્રદૂષકોનું વર્ગીકરણ:}

\textbf{1. પ્રકૃતિ અનુસાર:}


{\def\LTcaptype{none} % do not increment counter
\vspace{-5pt}
\captionof{table}{પ્રકૃતિ અનુસાર પ્રદૂષક વર્ગીકરણ}
\vspace{-10pt}
\begin{longtable}[]{@{}
  >{\raggedright\arraybackslash}p{(\linewidth - 4\tabcolsep) * \real{0.1818}}
  >{\raggedright\arraybackslash}p{(\linewidth - 4\tabcolsep) * \real{0.5152}}
  >{\raggedright\arraybackslash}p{(\linewidth - 4\tabcolsep) * \real{0.3030}}@{}}
\toprule\noalign{}
\begin{minipage}[b]{\linewidth}\raggedright
પ્રકાર
\end{minipage} & \begin{minipage}[b]{\linewidth}\raggedright
લાક્ષણિકતાઓ
\end{minipage} & \begin{minipage}[b]{\linewidth}\raggedright
ઉદાહરણો
\end{minipage} \\
\midrule\noalign{}
\endhead
\bottomrule\noalign{}
\endlastfoot
\textbf{બાયોડિગ્રેડેબલ} & કુદરતી રીતે વિઘટિત થાય & કાર્બનિક કચરો, ગટરનું પાણી \\
\textbf{નોન-બાયોડિગ્રેડેબલ} & પર્યાવરણમાં ટકી રહે & પ્લાસ્ટિક, ભારે ધાતુઓ \\
\textbf{ધીમે વિઘટિત થતા} & વર્ષો સુધી વિઘટિત થાય & જંતુનાશકો, કિરણોત્સર્ગી
સામગ્રી \\
\end{longtable}
}

\textbf{2. સ્વરૂપ અનુસાર:}

\begin{itemize}
\tightlist
\item
  \textbf{પ્રાથમિક}: સીધા ઉત્સર્જિત (SO₂, CO, કણો)
\item
  \textbf{ગૌણ}: પ્રતિક્રિયાઓ દ્વારા રચાય (O₃, અમ્લ વરસાદ, ધુમ્મસ)
\end{itemize}

\textbf{3. સ્ત્રોત અનુસાર:}

\begin{itemize}
\tightlist
\item
  \textbf{કુદરતી}: જ્વાળામુખી વિસ્ફોટ, જંગલની આગ
\item
  \textbf{માનવજન્ય}: માનવ પ્રવૃત્તિઓ, ઔદ્યોગિક પ્રક્રિયાઓ
\end{itemize}

\textbf{આકૃતિ: પ્રદૂષણ વર્ગીકરણ}

\begin{center}
\textbf{Mermaid Diagram (Code)}
\begin{verbatim}
{Shaded}
{Highlighting}[]
graph TD
    A[પ્રદૂષકો] {-{-}{} B[પ્રકૃતિ અનુસાર]}
    A {-{-}{} C[સ્વરૂપ અનુસાર]}
    A {-{-}{} D[સ્ત્રોત અનુસાર]}
    B {-{-}{} E[બાયોડિગ્રેડેબલ]}
    B {-{-}{} F[નોન{-}બાયોડિગ્રેડેબલ]}
    C {-{-}{} G[પ્રાથમિક]}
    C {-{-}{} H[ગૌણ]}
    D {-{-}{} I[કુદરતી]}
    D {-{-}{} J[માનવજન્ય]}
{Highlighting}
{Shaded}
\end{verbatim}
\end{center}

\textbf{પ્રદૂષણની અસરો:}

\begin{itemize}
\tightlist
\item
  \textbf{પર્યાવરણીય}: ઇકોસિસ્ટમ વિક્ષેપ, પ્રજાતિઓનું લુપ્ત થવું
\item
  \textbf{આરોગ્ય}: શ્વસન રોગો, કેન્સર, આનુવંશિક વિકાર
\item
  \textbf{આર્થિક}: આરોગ્ય સંભાળના ખર્ચ, ઘટતી ઉત્પાદકતા
\item
  \textbf{સામાજિક}: જીવનની ગુણવત્તામાં ઘટાડો
\end{itemize}

\end{solutionbox}
\begin{mnemonicbox}
``BNS-PFC'' - Biodegradable, Non-biodegradable,
Slowly degradable - Primary, Form, Classification

\begin{center}\rule{0.5\linewidth}{0.5pt}\end{center}

\end{mnemonicbox}
\subsection*{પ્રશ્ન 3(અ) {[}03
ગુણ{]}}\label{uxaaauxab0uxab6uxaa8-3uxa85-03-uxa97uxaa3}

\textbf{સૌર કોષની કામગીરી જણાવો.}

\begin{solutionbox}

સૌર કોષ અર્ધવાહક સામગ્રીનો ઉપયોગ કરીને ફોટોવોલ્ટેઇક અસર દ્વારા પ્રકાશ ઊર્જાને
સીધી વિદ્યુત ઊર્જામાં રૂપાંતરિત કરે છે.


{\def\LTcaptype{none} % do not increment counter
\vspace{-5pt}
\captionof{table}{સૌર કોષની કામગીરી પ્રક્રિયા}
\vspace{-10pt}
\begin{longtable}[]{@{}lll@{}}
\toprule\noalign{}
પગલું & પ્રક્રિયા & પરિણામ \\
\midrule\noalign{}
\endhead
\bottomrule\noalign{}
\endlastfoot
\textbf{ફોટોન શોષણ} & પ્રકાશ અર્ધવાહક પર પડે & ઇલેક્ટ્રોન ઉત્તેજના \\
\textbf{ઇલેક્ટ્રોન-હોલ ઉત્પાદન} & ઊર્જા બોન્ડ તોડે & મુક્ત ચાર્જ વાહકો \\
\textbf{ચાર્જ વિભાજન} & આંતરિક વિદ્યુત ક્ષેત્ર & ઇલેક્ટ્રોન n-બાજુ, હોલ p-બાજુ \\
\textbf{કરંટ કલેક્શન} & બાહ્ય સર્કિટ જોડાણ & વિદ્યુત પ્રવાહ \\
\end{longtable}
}

\begin{itemize}
\tightlist
\item
  \textbf{p-n જંક્શન}: આંતરિક વિદ્યુત ક્ષેત્ર બનાવે
\item
  \textbf{ડિપ્લેશન રીજન}: ચાર્જ વિભાજન સાથેનો વિસ્તાર
\item
  \textbf{બાહ્ય લોડ}: વિદ્યુત સર્કિટ પૂર્ણ કરે
\end{itemize}

\end{solutionbox}
\begin{mnemonicbox}
``PECS'' - Photon, Electron, Charge, Separation

\begin{center}\rule{0.5\linewidth}{0.5pt}\end{center}

\end{mnemonicbox}
\subsection*{પ્રશ્ન 3(બ) {[}04
ગુણ{]}}\label{uxaaauxab0uxab6uxaa8-3uxaac-04-uxa97uxaa3}

\textbf{આડી ધરી અને ઉભી ધરી વિન્ડ મિલ્સ વચ્ચેની સરખામણી આપો.}

\begin{solutionbox}


{\def\LTcaptype{none} % do not increment counter
\vspace{-5pt}
\captionof{table}{HAWT vs VAWT સરખામણી}
\vspace{-10pt}
\begin{longtable}[]{@{}lll@{}}
\toprule\noalign{}
પેરામીટર & આડી ધરી (HAWT) & ઉભી ધરી (VAWT) \\
\midrule\noalign{}
\endhead
\bottomrule\noalign{}
\endlastfoot
\textbf{બ્લેડ અભિગમ} & આડા ભ્રમણ & ઉભા ભ્રમણ \\
\textbf{પવનની દિશા} & પવનનો સામનો કરવો જોઈએ & કોઈપણ દિશાથી સ્વીકારે \\
\textbf{કાર્યક્ષમતા} & ઊંચી (35-45\%) & નીચી (20-35\%) \\
\textbf{ઊંચાઈ} & ટાવર પર માઉન્ટ, ઊંચું & જમીન સ્તરે સ્થાપના \\
\textbf{જાળવણી} & મુશ્કેલ, ઊંચી ઊંચાઈ & સરળ, જમીન સુલભ \\
\textbf{અવાજ} & મધ્યમ & ઓછો \\
\textbf{કિંમત} & પ્રારંભિક ઊંચી & ઓછી સ્થાપના \\
\textbf{પાવર આઉટપુટ} & મોટા પાયે ઊંચું & નાના પાયે યોગ્ય \\
\end{longtable}
}

\textbf{ફાયદાઓ:} \textbf{HAWT}: ઊંચી કાર્યક્ષમતા, સાબિત ટેકનોલોજી, બહેતર
પાવર-ટુ-વેઈટ રેશિયો \textbf{VAWT}: સર્વદિશીય, સરળ જાળવણી, શાંત કામગીરી, શહેરી
મિત્ર

\textbf{ઉપયોગો:} \textbf{HAWT}: મોટા વિન્ડ ફાર્મ, યુટિલિટી-સ્કેલ પાવર જનરેશન
\textbf{VAWT}: શહેરી વિસ્તારો, નાના પાયાના ઉપયોગો, વિતરિત જનરેશન

\end{solutionbox}
\begin{mnemonicbox}
``HEAVEN'' - Height, Efficiency, Accessibility,
Versatility, Economics, Noise

\begin{center}\rule{0.5\linewidth}{0.5pt}\end{center}

\end{mnemonicbox}
\subsection*{પ્રશ્ન 3(ક) {[}07
ગુણ{]}}\label{uxaaauxab0uxab6uxaa8-3uxa95-07-uxa97uxaa3}

\textbf{બાયોગેસ પ્લાન્ટનું બાંધકામ અને કાર્ય આકૃતી સાથે સમજાવો.}

\begin{solutionbox}

બાયોગેસ પ્લાન્ટ મેથેનોજેનિક બેક્ટેરિયા દ્વારા કાર્બનિક કચરા સામગ્રીના એનેરોબિક પાચન
દ્વારા મેથેન-સમૃદ્ધ ગેસ ઉત્પન્ન કરે છે.

\textbf{આકૃતિ: બાયોગેસ પ્લાન્ટ}

\begin{verbatim}
                ગેસ આઉટલેટ
                    ↑
    ફીડ ઇનલેટ → [ડાયજેસ્ટર] → સ્લરી આઉટલેટ
                    ↓
               ગેસ હોલ્ડર
                    ↑
              ભૂગર્ભ ચેમ્બર
\end{verbatim}


{\def\LTcaptype{none} % do not increment counter
\vspace{-5pt}
\captionof{table}{બાયોગેસ પ્લાન્ટના ઘટકો}
\vspace{-10pt}
\begin{longtable}[]{@{}lll@{}}
\toprule\noalign{}
ઘટક & કાર્ય & સામગ્રી \\
\midrule\noalign{}
\endhead
\bottomrule\noalign{}
\endlastfoot
\textbf{ડાયજેસ્ટર} & એનેરોબિક ફર્મેન્ટેશન ચેમ્બર & કોંક્રીટ/સ્ટીલ \\
\textbf{ગેસ હોલ્ડર} & ગેસ સ્ટોરેજ અને પ્રેશર રેગ્યુલેશન & સ્ટીલ/પ્લાસ્ટિક \\
\textbf{ઇનલેટ ચેમ્બર} & ફીડ સામગ્રી પ્રવેશ & ચણતર \\
\textbf{આઉટલેટ ચેમ્બર} & સ્લરી ડિસચાર્જ & ચણતર \\
\textbf{મિક્સિંગ ટેન્ક} & કાચી સામગ્રી તૈયારી & કોંક્રીટ \\
\end{longtable}
}

\textbf{બાંધકામની વિગતો:}

\textbf{ભૂગર્ભ ડાયજેસ્ટર:}

\begin{itemize}
\tightlist
\item
  \textbf{આકાર}: બેલનાકાર અથવા ગુંબજ આકાર
\item
  \textbf{ક્ષમતા}: ઘરેલુ પ્લાન્ટ માટે 10-100 m³
\item
  \textbf{દિવાલની જાડાઈ}: 10-15 સેમી કોંક્રીટ
\item
  \textbf{ઇન્સ્યુલેશન}: ગરમીનું નુકસાન અટકાવે
\end{itemize}

\textbf{કામકાજની પ્રક્રિયા:}


{\def\LTcaptype{none} % do not increment counter
\vspace{-5pt}
\captionof{table}{બાયોગેસ ઉત્પાદનના તબક્કાઓ}
\vspace{-10pt}
\begin{longtable}[]{@{}
  >{\raggedright\arraybackslash}p{(\linewidth - 6\tabcolsep) * \real{0.1944}}
  >{\raggedright\arraybackslash}p{(\linewidth - 6\tabcolsep) * \real{0.2500}}
  >{\raggedright\arraybackslash}p{(\linewidth - 6\tabcolsep) * \real{0.2778}}
  >{\raggedright\arraybackslash}p{(\linewidth - 6\tabcolsep) * \real{0.2778}}@{}}
\toprule\noalign{}
\begin{minipage}[b]{\linewidth}\raggedright
તબક્કો
\end{minipage} & \begin{minipage}[b]{\linewidth}\raggedright
પ્રક્રિયા
\end{minipage} & \begin{minipage}[b]{\linewidth}\raggedright
અવધિ
\end{minipage} & \begin{minipage}[b]{\linewidth}\raggedright
ઉત્પાદનો
\end{minipage} \\
\midrule\noalign{}
\endhead
\bottomrule\noalign{}
\endlastfoot
\textbf{હાઇડ્રોલિસિસ} & મોટા અણુઓનું વિભાજન & 1-3 દિવસ & સાદી શર્કરા, એમિનો
એસિડ \\
\textbf{એસિડોજેનેસિસ} & એસિડ રચના & 3-7 દિવસ & કાર્બનિક એસિડ, આલ્કોહોલ \\
\textbf{મેથેનોજેનેસિસ} & મેથેન ઉત્પાદન & 15-30 દિવસ & CH₄ (60\%), CO₂ (40\%) \\
\end{longtable}
}

\textbf{ઓપરેટિંગ પરિસ્થિતિઓ:}

\begin{itemize}
\tightlist
\item
  \textbf{તાપમાન}: 30-40°C (મેસોફિલિક)
\item
  \textbf{pH}: 6.8-7.2 (તટસ્થ)
\item
  \textbf{C:N રેશિયો}: 25-30:1 શ્રેષ્ઠ
\item
  \textbf{રિટેન્શન ટાઈમ}: 20-30 દિવસ
\end{itemize}

\textbf{ઉપયોગો:}

\begin{itemize}
\tightlist
\item
  \textbf{રસોઈ}: સ્વચ્છ બર્નિંગ ઇંધન
\item
  \textbf{લાઈટિંગ}: ગેસ લેમ્પ
\item
  \textbf{હીટિંગ}: સ્પેસ અને વોટર હીટિંગ
\item
  \textbf{વિજળી}: જનરેટર સેટ
\end{itemize}

\textbf{ફાયદાઓ:}

\begin{itemize}
\tightlist
\item
  \textbf{નવીકરણીય ઊર્જા}: ટકાઉ ઇંધન સ્ત્રોત
\item
  \textbf{કચરા વ્યવસ્થાપન}: કાર્બનિક કચરાનો નિકાલ
\item
  \textbf{ખાતર ઉત્પાદન}: પોષક તત્વોથી ભરપૂર સ્લરી
\item
  \textbf{પર્યાવરણીય ફાયદાઓ}: ગ્રીનહાઉસ ગેસ ઘટાડે
\end{itemize}

\end{solutionbox}
\begin{mnemonicbox}
``BIGHM'' - Biological, Input, Gas, Holder, Methane

\begin{center}\rule{0.5\linewidth}{0.5pt}\end{center}

\end{mnemonicbox}
\subsection*{પ્રશ્ન 3(અ અથવા) {[}03
ગુણ{]}}\label{uxaaauxab0uxab6uxaa8-3uxa85-uxa85uxaa5uxab5-03-uxa97uxaa3}

\textbf{ફ્લેટ પ્લેટ કલેક્ટરના ફાયદાઓની યાદી બનાવો.}

\begin{solutionbox}


{\def\LTcaptype{none} % do not increment counter
\vspace{-5pt}
\captionof{table}{ફ્લેટ પ્લેટ કલેક્ટરના ફાયદાઓ}
\vspace{-10pt}
\begin{longtable}[]{@{}ll@{}}
\toprule\noalign{}
કેટેગરી & ફાયદાઓ \\
\midrule\noalign{}
\endhead
\bottomrule\noalign{}
\endlastfoot
\textbf{તકનીકી} & સાદી ડિઝાઈન, કોઈ હિલતા ભાગો નથી, ઓછી જાળવણી \\
\textbf{આર્થિક} & ઓછી કિંમત, મોટા પાયે ઉત્પાદન શક્ય \\
\textbf{ઓપરેશનલ} & વેરવિખેર પ્રકાશ સાથે કામ કરે, સીધા અને પરોક્ષ બંને રેડિએશન
સંભાળે \\
\textbf{ટકાઉપણું} & લાંબું જીવન (15-20 વર્ષ), હવામાન પ્રતિરોધક \\
\textbf{વર્સેટિલિટી} & બહુવિધ ઉપયોગો, મોડ્યુલર ઇન્સ્ટોલેશન \\
\end{longtable}
}

\textbf{મુખ્ય ફાયદાઓ:}

\begin{itemize}
\tightlist
\item
  \textbf{વિશ્વસનીયતા}: જટિલ મિકેનિઝમ અથવા નિયંત્રણોની જરૂર નથી
\item
  \textbf{કાર્યક્ષમતા}: શ્રેષ્ઠ પરિસ્થિતિઓમાં 40-60\% થર્મલ કાર્યક્ષમતા
\item
  \textbf{ઇન્સ્ટોલેશન}: છત અથવા જમીન પર સરળ માઉન્ટિંગ
\end{itemize}

\end{solutionbox}
\begin{mnemonicbox}
``TEODV'' - Technical, Economic, Operational,
Durability, Versatility

\begin{center}\rule{0.5\linewidth}{0.5pt}\end{center}

\end{mnemonicbox}
\subsection*{પ્રશ્ન 3(બ અથવા) {[}04
ગુણ{]}}\label{uxaaauxab0uxab6uxaa8-3uxaac-uxa85uxaa5uxab5-04-uxa97uxaa3}

\textbf{પવન ચક્કી ક્ષેત્ર શું છે? તેના ફાયદાઓની યાદી આપો.}

\begin{solutionbox}

\textbf{વ્યાખ્યા:} વિન્ડ ફાર્મ એ વ્યાવસાયિક વિજળી ઉત્પાદન માટે એક જ સ્થાને સ્થાપિત
વિન્ડ ટર્બાઇનનું જૂથ છે, જે ટ્રાન્સમિશન લાઇન દ્વારા વિદ્યુત ગ્રિડ સાથે જોડાયેલ હોય છે.


{\def\LTcaptype{none} % do not increment counter
\vspace{-5pt}
\captionof{table}{વિન્ડ ફાર્મના ફાયદાઓ}
\vspace{-10pt}
\begin{longtable}[]{@{}ll@{}}
\toprule\noalign{}
કેટેગરી & ફાયદાઓ \\
\midrule\noalign{}
\endhead
\bottomrule\noalign{}
\endlastfoot
\textbf{પર્યાવરણીય} & સ્વચ્છ ઊર્જા, શૂન્ય ઉત્સર્જન, કાર્બન ફૂટપ્રિન્ટ ઘટાડે \\
\textbf{આર્થિક} & નોકરીઓ સર્જન, ઓછા ઓપરેટિંગ ખર્ચ, જમીન માલિકો માટે આવક \\
\textbf{તકનીકી} & સ્કેલેબલ ક્ષમતા, ગ્રિડ સ્થિરતા, ઊર્જા સ્વતંત્રતા \\
\textbf{સામાજિક} & ગ્રામીણ વિકાસ, સમુદાયિક ફાયદાઓ, શૈક્ષણિક તકો \\
\end{longtable}
}

\textbf{વિશિષ્ટ ફાયદાઓ:}

\begin{itemize}
\tightlist
\item
  \textbf{જમીનના ઉપયોગની કાર્યક્ષમતા}: ટર્બાઇન વચ્ચે ખેતી ચાલુ રાખી શકાય
\item
  \textbf{ઝડપી ઇન્સ્ટોલેશન}: પરંપરાગત પાવર પ્લાન્ટ કરતાં ઝડપી
\item
  \textbf{અનુમાનિત કિંમતો}: નિશ્ચિત ઇંધન કિંમત (પવન મફત છે)
\item
  \textbf{મોડ્યુલર વિસ્તરણ}: ક્ષમતા ક્રમશઃ વધારી શકાય
\end{itemize}

\textbf{ઉપયોગો:}

\begin{itemize}
\tightlist
\item
  \textbf{ઓનશોર}: જમીન આધારિત ઇન્સ્ટોલેશન
\item
  \textbf{ઓફશોર}: વધુ પવનની ઝડપ માટે સમુદ્ર આધારિત
\item
  \textbf{વિતરિત}: નાના પાયાના સમુદાયિક પ્રોજેક્ટ્સ
\end{itemize}

\end{solutionbox}
\begin{mnemonicbox}
``ECTS'' - Environmental, Economic, Technical,
Social benefits

\begin{center}\rule{0.5\linewidth}{0.5pt}\end{center}

\end{mnemonicbox}
\subsection*{પ્રશ્ન 3(ક અથવા) {[}07
ગુણ{]}}\label{uxaaauxab0uxab6uxaa8-3uxa95-uxa85uxaa5uxab5-07-uxa97uxaa3}

\textbf{ટૂંકમાં સમજાવો (1) ભૂઉષ્મીય ઊર્જા (2) ભરતી ઊર્જા}

\begin{solutionbox}

\textbf{(1) ભૂઉષ્મીય ઊર્જા:}

ભૂઉષ્મીય ઊર્જા વિજળી ઉત્પાદન અને સીધા હીટિંગ ઉપયોગો માટે પૃથ્વીના આંતરિક ગરમીનો
ઉપયોગ કરે છે.


{\def\LTcaptype{none} % do not increment counter
\vspace{-5pt}
\captionof{table}{ભૂઉષ્મીય ઊર્જા સિસ્ટમ}
\vspace{-10pt}
\begin{longtable}[]{@{}lll@{}}
\toprule\noalign{}
પ્રકાર & તાપમાન & ઉપયોગો \\
\midrule\noalign{}
\endhead
\bottomrule\noalign{}
\endlastfoot
\textbf{ઉચ્ચ તાપમાન} & \textgreater150°C & વિજળી ઉત્પાદન \\
\textbf{મધ્યમ તાપમાન} & 90-150°C & સીધું હીટિંગ, કૂલિંગ \\
\textbf{નીચા તાપમાન} & \textless90°C & હીટ પંપ, કૃષિ \\
\end{longtable}
}

\textbf{કાર્યસિદ્ધાંત:}

\begin{itemize}
\tightlist
\item
  \textbf{ગરમીનો સ્ત્રોત}: પૃથ્વીના કોરમાં કિરણોત્સર્ગી ક્ષય
\item
  \textbf{નિષ્કર્ષણ}: ગરમ પાણી/વરાળ મેળવવા માટે કૂવા ખોદવા
\item
  \textbf{રૂપાંતરણ}: વરાળ વિજળી માટે ટર્બાઇન ચલાવે
\item
  \textbf{રી-ઇન્જેક્શન}: પાણી રિઝર્વોયરમાં પાછું મોકલવું
\end{itemize}

\textbf{(2) ભરતી ઊર્જા:}

ભરતી ઊર્જા અનુમાનિત ભરતીની હિલચાલનો ઉપયોગ કરીને સમુદ્રી ભરતીની ગતિશીલ અને
સ્થિતિશીલ ઊર્જાને વિજળીમાં રૂપાંતરિત કરે છે.


{\def\LTcaptype{none} % do not increment counter
\vspace{-5pt}
\captionof{table}{ભરતી ઊર્જા તકનીકો}
\vspace{-10pt}
\begin{longtable}[]{@{}lll@{}}
\toprule\noalign{}
તકનીક & સિદ્ધાંત & ઇન્સ્ટોલેશન \\
\midrule\noalign{}
\endhead
\bottomrule\noalign{}
\endlastfoot
\textbf{ટાઇડલ બેરેજ} & ભરતીની શ્રેણીની સ્થિતિશીલ ઊર્જા & નદીમુખ પર ડેમ \\
\textbf{ટાઇડલ સ્ટ્રીમ} & ભરતીના પ્રવાહની ગતિશીલ ઊર્જા & પાણીની અંદર ટર્બાઇન \\
\textbf{ટાઇડલ લેગૂન} & કૃત્રિમ બંધ વિસ્તાર & બ્રેકવોટર બાંધકામ \\
\end{longtable}
}

\textbf{ફાયદાઓ:} \textbf{ભૂઉષ્મીય}: બેઝલોડ પાવર, ઓછા ઉત્સર્જન, નાનું ફૂટપ્રિન્ટ,
વિશ્વસનીય \textbf{ભરતી}: અનુમાનિત, ઉચ્ચ ઊર્જા ઘનતા, લાંબું જીવનકાળ, ઇંધન ખર્ચ નહીં

\textbf{પડકારો:} \textbf{ભૂઉષ્મીય}: સ્થાન વિશિષ્ટ, ઉચ્ચ પ્રારંભિક કિંમત, પ્રેરિત
ભૂકંપ \textbf{ભરતી}: ઉચ્ચ મૂડી ખર્ચ, પર્યાવરણીય અસર, મર્યાદિત સ્થાનો

\end{solutionbox}
\begin{mnemonicbox}
``GT-POWER'' - Geothermal Temperature, Tidal
Predictable Ocean Water Energy Resource

\begin{center}\rule{0.5\linewidth}{0.5pt}\end{center}

\end{mnemonicbox}
\subsection*{પ્રશ્ન 4(અ) {[}03
ગુણ{]}}\label{uxaaauxab0uxab6uxaa8-4uxa85-03-uxa97uxaa3}

\textbf{નવીનીકરણીય ઊર્જાની જરૂરિયાત વ્યાખ્યાયિત કરો}

\begin{solutionbox}


{\def\LTcaptype{none} % do not increment counter
\vspace{-5pt}
\captionof{table}{નવીનીકરણીય ઊર્જાની જરૂરિયાત}
\vspace{-10pt}
\begin{longtable}[]{@{}ll@{}}
\toprule\noalign{}
ચાલક & કારણો \\
\midrule\noalign{}
\endhead
\bottomrule\noalign{}
\endlastfoot
\textbf{પર્યાવરણીય} & આબોહવા પરિવર્તન ઘટાડો, પ્રદૂષણ ઘટાડો \\
\textbf{આર્થિક} & ઊર્જા સુરક્ષા, કિંમત સ્થિરતા, નોકરીઓ સર્જન \\
\textbf{તકનીકી} & અવશેષ ઇંધણોનો ક્ષય, તકનીકી પ્રગતિ \\
\textbf{સામાજિક} & ગ્રામીણ વિકાસ, આરોગ્યને ફાયદાઓ, ઊર્જા પહોંચ \\
\end{longtable}
}

\textbf{મુખ્ય જરૂરિયાતો:}

\begin{itemize}
\tightlist
\item
  \textbf{આબોહવા પ્રતિબદ્ધતાઓ}: પેરિસ એગ્રીમેન્ટ લક્ષ્યો પૂરા કરવા
\item
  \textbf{ઊર્જા સ્વતંત્રતા}: આયાત નિર્ભરતા ઘટાડવી
\item
  \textbf{ટકાઉ વિકાસ}: લાંબાગાળાની ઊર્જા સુરક્ષા
\end{itemize}

\end{solutionbox}
\begin{mnemonicbox}
``EETS'' - Environmental, Economic, Technical,
Social needs

\begin{center}\rule{0.5\linewidth}{0.5pt}\end{center}

\end{mnemonicbox}
\subsection*{પ્રશ્ન 4(બ) {[}04
ગુણ{]}}\label{uxaaauxab0uxab6uxaa8-4uxaac-04-uxa97uxaa3}

\textbf{ઓઝોન સ્તરના અવક્ષયને સમજાવો.}

\begin{solutionbox}

ઓઝોન સ્તરનો અવક્ષય માનવ નિર્મિત રસાયણો, ખાસ કરીને ક્લોરોફ્લોરોકાર્બન (CFCs) ને
કારણે સ્ટ્રેટોસ્ફિયરમાં ઓઝોન સાંદ્રતાનો ઘટાડો છે.


{\def\LTcaptype{none} % do not increment counter
\vspace{-5pt}
\captionof{table}{ઓઝોન અવક્ષય પ્રક્રિયા}
\vspace{-10pt}
\begin{longtable}[]{@{}lll@{}}
\toprule\noalign{}
તબક્કો & પ્રક્રિયા & રાસાયણિક પ્રતિક્રિયા \\
\midrule\noalign{}
\endhead
\bottomrule\noalign{}
\endlastfoot
\textbf{CFC મુક્તિ} & ઔદ્યોગિક ઉત્સર્જન & CFCs સ્ટ્રેટોસ્ફિયરમાં ઉગે \\
\textbf{UV વિભાજન} & ફોટોડિસોસિએશન & CFC + UV → Cl + અન્ય ઉત્પાદનો \\
\textbf{ઓઝોન વિનાશ} & કેટેલિટિક ચક્ર & Cl + O₃ → ClO + O₂ \\
\textbf{શૃંખલા પ્રતિક્રિયા} & સતત પ્રક્રિયા & ClO + O → Cl + O₂ \\
\end{longtable}
}

\textbf{કારણો:}

\begin{itemize}
\tightlist
\item
  \textbf{પ્રાથમિક}: CFCs, હેલોન્સ, મેથાઈલ બ્રોમાઈડ
\item
  \textbf{ગૌણ}: HCFCs, નાઈટ્રસ ઓક્સાઈડ, કાર્બન ટેટ્રાક્લોરાઈડ
\end{itemize}

\textbf{અસરો:}

\begin{itemize}
\tightlist
\item
  \textbf{વધેલ UV-B રેડિએશન}: ત્વચા કેન્સર, મોતિયો
\item
  \textbf{પર્યાવરણીય અસર}: પાકની ઉપજ ઘટાડો, દરિયાઈ ઇકોસિસ્ટમ નુકસાન
\item
  \textbf{આબોહવા અસરો}: બદલાયેલ વાતાવરણીય પરિભ્રમણ
\end{itemize}

\textbf{ઉકેલો:}

\begin{itemize}
\tightlist
\item
  \textbf{મોન્ટ્રીલ પ્રોટોકોલ}: આંતરરાષ્ટ્રીય એગ્રીમેન્ટ (1987)
\item
  \textbf{CFC ફેઝ-આઉટ}: ઓઝોન-ફ્રેન્ડલી વિકલ્પો સાથે બદલવું
\item
  \textbf{HCFC સંક્રમણ}: અસ્થાયી વિકલ્પો તબક્કાવાર બંધ
\end{itemize}

\end{solutionbox}
\begin{mnemonicbox}
``CURE'' - CFCs, UV, Reactions, Effects

\begin{center}\rule{0.5\linewidth}{0.5pt}\end{center}

\end{mnemonicbox}
\subsection*{પ્રશ્ન 4(ક) {[}07
ગુણ{]}}\label{uxaaauxab0uxab6uxaa8-4uxa95-07-uxa97uxaa3}

\textbf{સમજાવો: (1) ગ્રીનહાઉસ અસર (2) આબોહવા પરિવર્તન વ્યવસ્થાપન}

\begin{solutionbox}

\textbf{(1) ગ્રીનહાઉસ અસર:}

કુદરતી પ્રક્રિયા જ્યાં ચોક્કસ વાતાવરણીય ગેસો સૂર્યથી ગરમીને ફસાવે છે, જીવન માટે યોગ્ય
પૃથ્વીનું તાપમાન જાળવે છે.

\textbf{આકૃતિ: ગ્રીનહાઉસ અસર}

\begin{verbatim}
flowchart LR
    A[સૌર કિરણોત્સર્ગ] {-{-} B[પૃથ્વીની સપાટી]}
    B {-{-} C[ગરમી કિરણોત્સર્ગ]}
    C {-{-} D[ગ્રીનહાઉસ ગેસો]}
    D {-{-} E[ગરમી ફસાઈ]}
    E {-{-} F[પૃથ્વી પર પાછી કિરણોત્સર્ગ]}
    F {-{-} B}
\end{verbatim}


{\def\LTcaptype{none} % do not increment counter
\vspace{-5pt}
\captionof{table}{ગ્રીનહાઉસ ગેસો}
\vspace{-10pt}
\begin{longtable}[]{@{}llll@{}}
\toprule\noalign{}
ગેસ & સ્ત્રોતો & યોગદાન & જીવનકાળ \\
\midrule\noalign{}
\endhead
\bottomrule\noalign{}
\endlastfoot
\textbf{CO₂} & અવશેષ ઇંધણ, વનનાશ & 76\% & 300-1000 વર્ષ \\
\textbf{CH₄} & કૃષિ, લેન્ડફિલ & 16\% & 12 વર્ષ \\
\textbf{N₂O} & ખાતર, દહન & 6\% & 120 વર્ષ \\
\textbf{F-ગેસો} & ઔદ્યોગિક પ્રક્રિયાઓ & 2\% & વિવિધ \\
\end{longtable}
}

\textbf{વધેલી ગ્રીનહાઉસ અસર:}

\begin{itemize}
\tightlist
\item
  \textbf{કારણ}: માનવ પ્રવૃત્તિઓથી વધેલ GHG સાંદ્રતા
\item
  \textbf{પરિણામ}: વૈશ્વિક તાપમાન વધારો, આબોહવા પરિવર્તન
\item
  \textbf{ફીડબેક લૂપ્સ}: ગરમ થવાની અસરોને વધારે
\end{itemize}

\textbf{(2) આબોહવા પરિવર્તન વ્યવસ્થાપન:}

શમન અને અનુકૂલન વ્યૂહરચના દ્વારા આબોહવા પરિવર્તનને સંબોધવા માટે વ્યાપક અભિગમ.


{\def\LTcaptype{none} % do not increment counter
\vspace{-5pt}
\captionof{table}{આબોહવા પરિવર્તન વ્યવસ્થાપન વ્યૂહરચનાઓ}
\vspace{-10pt}
\begin{longtable}[]{@{}lll@{}}
\toprule\noalign{}
વ્યૂહરચના & અભિગમ & ઉદાહરણો \\
\midrule\noalign{}
\endhead
\bottomrule\noalign{}
\endlastfoot
\textbf{શમન} & GHG ઉત્સર્જન ઘટાડો & નવીકરણીય ઊર્જા, ઊર્જા કાર્યક્ષમતા \\
\textbf{અનુકૂલન} & આબોહવા અસરોને સમાયોજન & સીવોલ, દુષ્કાળ પ્રતિરોધી પાકો \\
\textbf{ટેકનોલોજી} & નવાચાર ઉકેલો & કાર્બન કેપ્ચર, સ્માર્ટ ગ્રિડ \\
\textbf{નીતિ} & નિયમનકારી ફ્રેમવર્ક & કાર્બન પ્રાઈસિંગ, ઉત્સર્જન ધોરણો \\
\textbf{આંતરરાષ્ટ્રીય} & વૈશ્વિક સહયોગ & પેરિસ એગ્રીમેન્ટ, આબોહવા ફાઈનાન્સ \\
\end{longtable}
}

\textbf{શમન પગલાં:}

\begin{itemize}
\tightlist
\item
  \textbf{ઊર્જા ક્ષેત્ર}: નવીકરણીય ઊર્જા જમાવટ, કાર્યક્ષમતા સુધારા
\item
  \textbf{પરિવહન}: ઇલેક્ટ્રિક વાહનો, સાર્વજનિક પરિવહન, બાયોફ્યુઅલ
\item
  \textbf{ઉદ્યોગ}: પ્રક્રિયા ઓપ્ટિમાઇઝેશન, લો-કાર્બન ટેકનોલોજી
\item
  \textbf{ઇમારતો}: ગ્રીન કન્સ્ટ્રક્શન, સ્માર્ટ સિસ્ટમ
\item
  \textbf{કૃષિ}: ટકાઉ પ્રથાઓ, ઘટાડેલ ઉત્સર્જન
\end{itemize}

\textbf{અનુકૂલન પગલાં:}

\begin{itemize}
\tightlist
\item
  \textbf{ઇન્ફ્રાસ્ટ્રક્ચર}: આબોહવા-પ્રત્યાસ્થ ડિઝાઇન, પૂર સંરક્ષણ
\item
  \textbf{ઇકોસિસ્ટમ}: સંરક્ષણ, પુનઃસ્થાપન, કોરિડોર
\item
  \textbf{પાણીના સંસાધનો}: કાર્યક્ષમ ઉપયોગ, સંગ્રહ, ગુણવત્તા વ્યવસ્થાપન
\item
  \textbf{આરોગ્ય}: રોગ સર્વેલન્સ, ગરમીની લહેર તૈયારી
\end{itemize}

\textbf{વ્યવસ્થાપન ફ્રેમવર્ક:}

\begin{enumerate}
\def\labelenumi{\arabic{enumi}.}
\tightlist
\item
  \textbf{મૂલ્યાંકન}: આબોહવા જોખમ અને નબળાઈ વિશ્લેષણ
\item
  \textbf{આયોજન}: એકીકૃત વ્યૂહરચના અને કાર્ય યોજનાઓ
\item
  \textbf{અમલીકરણ}: પ્રોજેક્ટ અમલ અને મોનિટરિંગ
\item
  \textbf{મૂલ્યાંકન}: પ્રદર્શન મૂલ્યાંકન અને ગોઠવણ
\end{enumerate}

\end{solutionbox}
\begin{mnemonicbox}
``GEMMA'' - Gases, Enhanced, Mitigation, Management,
Adaptation

\begin{center}\rule{0.5\linewidth}{0.5pt}\end{center}

\end{mnemonicbox}
\subsection*{પ્રશ્ન 4(અ અથવા) {[}03
ગુણ{]}}\label{uxaaauxab0uxab6uxaa8-4uxa85-uxa85uxaa5uxab5-03-uxa97uxaa3}

\textbf{આબોહવા પરિવર્તનને અસર કરતા પરિબળોની ચર્ચા કરો.}

\begin{solutionbox}


{\def\LTcaptype{none} % do not increment counter
\vspace{-5pt}
\captionof{table}{આબોહવા પરિવર્તન પરિબળો}
\vspace{-10pt}
\begin{longtable}[]{@{}lll@{}}
\toprule\noalign{}
પરિબળ પ્રકાર & ઉદાહરણો & અસર \\
\midrule\noalign{}
\endhead
\bottomrule\noalign{}
\endlastfoot
\textbf{કુદરતી} & સૌર વેરિએશન, જ્વાળામુખી વિસ્ફોટ & નજીવો પ્રભાવ \\
\textbf{માનવજન્ય} & GHG ઉત્સર્જન, જમીન ઉપયોગ પરિવર્તન & મુખ્ય ચાલક \\
\textbf{ફીડબેક} & બરફ-એલ્બેડો, પાણીની વરાળ & વિસ્તૃતીકરણ \\
\end{longtable}
}

\textbf{મુખ્ય પરિબળો:}

\begin{itemize}
\tightlist
\item
  \textbf{ગ્રીનહાઉસ ગેસ સાંદ્રતા}: ગરમ થવાનો પ્રાથમિક ચાલક
\item
  \textbf{એરોસોલ્સ}: ઠંડક અસર, કેટલાક ગરમ થવાને છુપાવે
\item
  \textbf{જમીન ઉપયોગ પરિવર્તન}: વનનાશ, શહેરીકરણ અસરો
\end{itemize}

\end{solutionbox}
\begin{mnemonicbox}
``NAF'' - Natural, Anthropogenic, Feedback factors

\begin{center}\rule{0.5\linewidth}{0.5pt}\end{center}

\end{mnemonicbox}
\subsection*{પ્રશ્ન 4(બ અથવા) {[}04
ગુણ{]}}\label{uxaaauxab0uxab6uxaa8-4uxaac-uxa85uxaa5uxab5-04-uxa97uxaa3}

\textbf{ક્લાઈમેટ ચેન્જ સમજાવો}

\begin{solutionbox}

આબોહવા પરિવર્તન 20મી સદીના મધ્યથી મુખ્યત્વે માનવ પ્રવૃત્તિઓને કારણે વૈશ્વિક તાપમાન અને
હવામાન પેટર્નમાં લાંબાગાળાના ફેરફારોનો સંદર્ભ આપે છે.


{\def\LTcaptype{none} % do not increment counter
\vspace{-5pt}
\captionof{table}{આબોહવા પરિવર્તન સૂચકાંકો}
\vspace{-10pt}
\begin{longtable}[]{@{}lll@{}}
\toprule\noalign{}
સૂચકાંક & અવલોકિત ફેરફારો & વલણ \\
\midrule\noalign{}
\endhead
\bottomrule\noalign{}
\endlastfoot
\textbf{તાપમાન} & 1880 થી +1.1°C & વધતું \\
\textbf{સમુદ્ર સ્તર} & 1880 થી 21-24 સેમી & વધતું \\
\textbf{આર્કટિક બરફ} & દર દાયકાએ 13\% નુકસાન & ઘટતું \\
\textbf{વરસાદ} & પ્રાદેશિક વિવિધતાઓ & બદલાતા પેટર્ન \\
\end{longtable}
}

\textbf{કારણો:}

\begin{itemize}
\tightlist
\item
  \textbf{પ્રાથમિક}: અવશેષ ઇંધણોથી ગ્રીનહાઉસ ગેસ ઉત્સર્જન
\item
  \textbf{ગૌણ}: વનનાશ, ઔદ્યોગિક પ્રક્રિયાઓ, કૃષિ
\end{itemize}

\textbf{અસરો:}

\begin{itemize}
\tightlist
\item
  \textbf{ભૌતિક}: આત્યંતિક હવામાન, સમુદ્ર સ્તર વધારો, બરફ નુકસાન
\item
  \textbf{જૈવિક}: પ્રજાતિઓનું સ્થળાંતર, ઇકોસિસ્ટમ વિક્ષેપ
\item
  \textbf{માનવ}: ખોરાક સુરક્ષા, પાણીના સંસાધનો, આરોગ્ય
\end{itemize}

\textbf{પુરાવા:}

\begin{itemize}
\tightlist
\item
  \textbf{તાપમાન રેકોર્ડ}: વૈશ્વિક ગરમ થવાનો વલણ
\item
  \textbf{બરફના કોર ડેટા}: ઐતિહાસિક CO₂ સ્તર
\item
  \textbf{સેટેલાઇટ અવલોકનો}: બરફની ચાદરમાં ફેરફાર
\end{itemize}

\end{solutionbox}
\begin{mnemonicbox}
``CHIP'' - Causes, Human impacts, Indicators,
Physical evidence

\begin{center}\rule{0.5\linewidth}{0.5pt}\end{center}

\end{mnemonicbox}
\subsection*{પ્રશ્ન 4(ક અથવા) {[}07
ગુણ{]}}\label{uxaaauxab0uxab6uxaa8-4uxa95-uxa85uxaa5uxab5-07-uxa97uxaa3}

\textbf{ગ્લોબલ વોર્મિંગ પર ટૂંકી નોંધ લખો.}

\begin{solutionbox}

ગ્લોબલ વોર્મિંગ એ માનવ પ્રવૃત્તિઓથી વધેલી ગ્રીનહાઉસ અસરને કારણે પૃથ્વીના સરેરાશ
સપાટીના તાપમાનમાં લાંબાગાળાનો વધારો છે.


{\def\LTcaptype{none} % do not increment counter
\vspace{-5pt}
\captionof{table}{ગ્લોબલ વોર્મિંગના ઘટકો}
\vspace{-10pt}
\begin{longtable}[]{@{}
  >{\raggedright\arraybackslash}p{(\linewidth - 4\tabcolsep) * \real{0.3200}}
  >{\raggedright\arraybackslash}p{(\linewidth - 4\tabcolsep) * \real{0.3600}}
  >{\raggedright\arraybackslash}p{(\linewidth - 4\tabcolsep) * \real{0.3200}}@{}}
\toprule\noalign{}
\begin{minipage}[b]{\linewidth}\raggedright
પાસું
\end{minipage} & \begin{minipage}[b]{\linewidth}\raggedright
વિગતો
\end{minipage} & \begin{minipage}[b]{\linewidth}\raggedright
અસર
\end{minipage} \\
\midrule\noalign{}
\endhead
\bottomrule\noalign{}
\endlastfoot
\textbf{વ્યાખ્યા} & વૈશ્વિક સરેરાશ તાપમાનમાં વધારો & પૂર્વ-ઔદ્યોગિક કાળથી
+1.1°C \\
\textbf{પ્રાથમિક કારણ} & અવશેષ ઇંધણોથી CO₂ ઉત્સર્જન & 410+ ppm વાતાવરણીય
CO₂ \\
\textbf{સમયરેખા} & 1950 ના દાયકાથી ઝડપી & 10,000 વર્ષમાં સૌથી ઝડપી ગરમ
થવું \\
\textbf{પ્રાદેશિક વિવિધતા} & આર્કટિક ગરમ થવું વૈશ્વિક સરેરાશ કરતાં 2x & ધ્રુવીય
વિસ્તૃતીકરણ \\
\end{longtable}
}

\textbf{ગ્લોબલ વોર્મિંગના કારણો:}


{\def\LTcaptype{none} % do not increment counter
\vspace{-5pt}
\captionof{table}{ઉત્સર્જન સ્ત્રોતો}
\vspace{-10pt}
\begin{longtable}[]{@{}lll@{}}
\toprule\noalign{}
ક્ષેત્ર & યોગદાન & મુખ્ય પ્રવૃત્તિઓ \\
\midrule\noalign{}
\endhead
\bottomrule\noalign{}
\endlastfoot
\textbf{ઊર્જા} & 73\% & વિજળી, ગરમી, પરિવહન \\
\textbf{કૃષિ} & 18\% & પશુધન, ચોખાની ખેતી \\
\textbf{ઔદ્યોગિક} & 5\% & સિમેન્ટ, સ્ટીલ, રસાયણો \\
\textbf{કચરો} & 3\% & લેન્ડફિલ, ગંદા પાણી \\
\textbf{જમીન ઉપયોગ} & 1\% & વનનાશ, વિકાસ \\
\end{longtable}
}

\textbf{પરિણામો:}

\begin{itemize}
\tightlist
\item
  \textbf{ભૌતિક અસરો}: સમુદ્ર સ્તર વધારો, ગ્લેશિયર પીછેહઠ, પર્માફ્રોસ્ટ પીગળવું
\item
  \textbf{હવામાન પેટર્ન}: વધુ વારંવાર ગરમીની લહેરો, બદલાયેલ વરસાદ
\item
  \textbf{ઇકોસિસ્ટમ અસરો}: પ્રજાતિઓનું લુપ્ત થવું, વસવાટ નુકસાન, કોરલ બ્લીચિંગ
\item
  \textbf{માનવ અસરો}: કૃષિ વિક્ષેપ, પાણીની અછત, આરોગ્ય જોખમો
\end{itemize}

\textbf{ફીડબેક મિકેનિઝમ:}

\begin{itemize}
\tightlist
\item
  \textbf{બરફ-એલ્બેડો ફીડબેક}: ઓછું બરફ → વધુ ગરમી શોષણ
\item
  \textbf{પાણીની વરાળ ફીડબેક}: ગરમ હવા વધુ ભેજ ધરાવે
\item
  \textbf{પર્માફ્રોસ્ટ ફીડબેક}: પીગળવાથી સંગ્રહિત કાર્બન મુક્ત થાય
\end{itemize}

\textbf{ઉકેલો:}

\begin{itemize}
\tightlist
\item
  \textbf{શમન}: ગ્રીનહાઉસ ગેસ ઉત્સર્જન ઘટાડવું
\item
  \textbf{નવીકરણીય ઊર્જા}: સૌર, પવન, હાઇડ્રોઇલેક્ટ્રિક પાવર
\item
  \textbf{ઊર્જા કાર્યક્ષમતા}: ઇમારતો, પરિવહન, ઉદ્યોગ
\item
  \textbf{કાર્બન સીક્વેસ્ટ્રેશન}: જંગલો, માટી, તકનીકી કેપ્ચર
\item
  \textbf{નીતિ પગલાં}: કાર્બન પ્રાઇસિંગ, નિયમો, પ્રોત્સાહનો
\end{itemize}

\textbf{આંતરરાષ્ટ્રીય પ્રતિસાદ:}

\begin{itemize}
\tightlist
\item
  \textbf{UNFCCC}: આબોહવા પરિવર્તન પર ફ્રેમવર્ક કન્વેન્શન
\item
  \textbf{ક્યોટો પ્રોટોકોલ}: પ્રથમ બંધનકર્તા ઉત્સર્જન ઘટાડા કરાર
\item
  \textbf{પેરિસ એગ્રીમેન્ટ}: વર્તમાન વૈશ્વિક આબોહવા સમજૂતી (2015)
\item
  \textbf{IPCC રિપોર્ટ્સ}: વૈજ્ઞાનિક મૂલ્યાંકન અને માર્ગદર્શન
\end{itemize}

\textbf{ભાવિ અનુમાનો:}

\begin{itemize}
\tightlist
\item
  \textbf{તાપમાન વધારો}: ઉત્સર્જનના આધારે 2100 સુધીમાં 1.5-4.5°C
\item
  \textbf{સમુદ્ર સ્તર વધારો}: 2100 સુધીમાં 0.43-2.84 મીટર
\item
  \textbf{ટિપિંગ પોઇન્ટ્સ}: આબોહવા પ્રણાલીમાં અપરિવર્તનીય ફેરફારો
\end{itemize}

\end{solutionbox}
\begin{mnemonicbox}
``GWCF'' - Global Warming Causes Consequences
Feedback

\begin{center}\rule{0.5\linewidth}{0.5pt}\end{center}

\end{mnemonicbox}
\subsection*{પ્રશ્ન 5(અ) {[}03
ગુણ{]}}\label{uxaaauxab0uxab6uxaa8-5uxa85-03-uxa97uxaa3}

\textbf{``ઇકો ટુરીઝમ'' ની વિભાવના સમજાવો}

\begin{solutionbox}

ઇકો-ટુરીઝમ એ કુદરતી વિસ્તારોમાં જવાબદાર મુસાફરી છે જે પર્યાવરણનું સંરક્ષણ કરે છે,
સ્થાનિક લોકોના કલ્યાણને ટકાવી રાખે છે, અને અર્થઘટન અને શિક્ષણ સામેલ કરે છે.


{\def\LTcaptype{none} % do not increment counter
\vspace{-5pt}
\captionof{table}{ઇકો-ટુરીઝમના સિદ્ધાંતો}
\vspace{-10pt}
\begin{longtable}[]{@{}ll@{}}
\toprule\noalign{}
સિદ્ધાંત & વર્ણન \\
\midrule\noalign{}
\endhead
\bottomrule\noalign{}
\endlastfoot
\textbf{સંરક્ષણ} & કુદરતી વસવાટ અને વન્યજીવનનું સંરક્ષણ \\
\textbf{સમુદાય} & સ્થાનિક સમુદાયોને આર્થિક ફાયદો \\
\textbf{શિક્ષણ} & પર્યાવરણીય જાગૃતિ અને શિક્ષણ \\
\textbf{ટકાઉપણું} & લાંબાગાળાનું પર્યાવરણ સંરક્ષણ \\
\textbf{જવાબદારી} & નકારાત્મક અસરો ઘટાડવી \\
\end{longtable}
}

\begin{itemize}
\tightlist
\item
  \textbf{પ્રકૃતિ આધારિત}: કુદરતી વાતાવરણ પર ધ્યાન
\item
  \textbf{ઓછી અસર}: ન્યૂનતમ પર્યાવરણીય વિક્ષેપ
\item
  \textbf{સાંસ્કૃતિક આદર}: સ્થાનિક પરંપરાઓ અને રિવાજોનું મૂલ્ય
\end{itemize}

\end{solutionbox}
\begin{mnemonicbox}
``ECERS'' - Environment, Community, Education,
Responsibility, Sustainability

\begin{center}\rule{0.5\linewidth}{0.5pt}\end{center}

\end{mnemonicbox}
\subsection*{પ્રશ્ન 5(બ) {[}04
ગુણ{]}}\label{uxaaauxab0uxab6uxaa8-5uxaac-04-uxa97uxaa3}

\textbf{પરંપરાગત અને બિનપરંપરાગત ઉર્જા સ્ત્રોતની સરખામણી.}

\begin{solutionbox}


{\def\LTcaptype{none} % do not increment counter
\vspace{-5pt}
\captionof{table}{પરંપરાગત વિ બિનપરંપરાગત ઉર્જા સ્ત્રોતો}
\vspace{-10pt}
\begin{longtable}[]{@{}
  >{\raggedright\arraybackslash}p{(\linewidth - 4\tabcolsep) * \real{0.2558}}
  >{\raggedright\arraybackslash}p{(\linewidth - 4\tabcolsep) * \real{0.3256}}
  >{\raggedright\arraybackslash}p{(\linewidth - 4\tabcolsep) * \real{0.4186}}@{}}
\toprule\noalign{}
\begin{minipage}[b]{\linewidth}\raggedright
પેરામીટર
\end{minipage} & \begin{minipage}[b]{\linewidth}\raggedright
પરંપરાગત
\end{minipage} & \begin{minipage}[b]{\linewidth}\raggedright
બિનપરંપરાગત
\end{minipage} \\
\midrule\noalign{}
\endhead
\bottomrule\noalign{}
\endlastfoot
\textbf{ઉદાહરણો} & કોલસો, તેલ, કુદરતી ગેસ, ન્યુક્લિયર & સૌર, પવન, હાઇડ્રો,
બાયોમાસ \\
\textbf{ઉપલબ્ધતા} & મર્યાદિત ભંડાર & વિપુલ અને નવીકરણીય \\
\textbf{પર્યાવરણીય અસર} & ઉચ્ચ પ્રદૂષણ, CO₂ ઉત્સર્જન & સ્વચ્છ, ન્યૂનતમ ઉત્સર્જન \\
\textbf{કિંમત} & શરૂઆતમાં ઓછી, વધતી કિંમતો & ઉચ્ચ પ્રારંભિક, ઘટતી કિંમતો \\
\textbf{ટેકનોલોજી} & પરિપક્વ, સ્થાપિત & વિકસતી, સુધરતી \\
\textbf{વિશ્વસનીયતા} & સતત પુરવઠો & હવામાન આધારિત \\
\textbf{ઇન્ફ્રાસ્ટ્રક્ચર} & સુસ્થાપિત & વિકાસ જરૂરી \\
\textbf{ક્ષય} & ખતમ થતા સંસાધનો & અખૂટ સ્ત્રોતો \\
\end{longtable}
}

\textbf{ફાયદાઓ:} \textbf{પરંપરાગત}: વિશ્વસનીય પુરવઠો, સ્થાપિત ઇન્ફ્રાસ્ટ્રક્ચર,
ઉચ્ચ ઊર્જા ઘનતા \textbf{બિનપરંપરાગત}: ટકાઉ, સ્વચ્છ, નોકરીઓ સર્જન, ઊર્જા સ્વતંત્રતા

\textbf{પડકારો:} \textbf{પરંપરાગત}: પર્યાવરણ નુકસાન, કિંમત અસ્થિરતા, મર્યાદિત
સંસાધનો \textbf{બિનપરંપરાગત}: તૂટક તૂટક, સંગ્રહની જરૂર, પ્રારંભિક રોકાણ

\end{solutionbox}
\begin{mnemonicbox}
``CATERED'' - Conventional Available Technology
Established Reliable Environmental Depletion

\begin{center}\rule{0.5\linewidth}{0.5pt}\end{center}

\end{mnemonicbox}
\subsection*{પ્રશ્ન 5(ક) {[}07
ગુણ{]}}\label{uxaaauxab0uxab6uxaa8-5uxa95-07-uxa97uxaa3}

\textbf{સમજાવો (1) પાણી અધિનિયમ, 1974 (2) પર્યાવરણ અધિનિયમ, 1986}

\begin{solutionbox}

\textbf{(1) પાણી (પ્રદૂષણ નિવારણ અને નિયંત્રણ) અધિનિયમ, 1974:}

ભારતમાં પાણીના પ્રદૂષણને અટકાવવા અને નિયંત્રિત કરવા અને પાણીની સ્વચ્છતા
જાળવવા/પુનઃસ્થાપિત કરવા માટે વ્યાપક કાયદો.


{\def\LTcaptype{none} % do not increment counter
\vspace{-5pt}
\captionof{table}{પાણી અધિનિયમ 1974 - મુખ્ય જોગવાઈઓ}
\vspace{-10pt}
\begin{longtable}[]{@{}ll@{}}
\toprule\noalign{}
પાસું & વિગતો \\
\midrule\noalign{}
\endhead
\bottomrule\noalign{}
\endlastfoot
\textbf{ઉદ્દેશ્ય} & પાણીના પ્રદૂષણને અટકાવવું અને નિયંત્રિત કરવું \\
\textbf{સત્તા} & કેન્દ્રીય અને રાજ્ય પ્રદૂષણ નિયંત્રણ બોર્ડ \\
\textbf{કવરેજ} & તમામ જળ સ્ત્રોતો - નદીઓ, પ્રવાહો, કૂવા, ભૂગર્ભજળ \\
\textbf{દંડ} & ઉલ્લંઘન માટે દંડ અને કેદ \\
\end{longtable}
}

\textbf{મુખ્ય વિશેષતાઓ:}

\begin{itemize}
\tightlist
\item
  \textbf{પ્રદૂષણ નિયંત્રણ બોર્ડ}: કેન્દ્રીય અને રાજ્ય સ્તરે સ્થાપના
\item
  \textbf{સંમતિ મિકેનિઝમ}: ઉદ્યોગો માટે નો-ઓબ્જેક્શન સર્ટિફિકેટ
\item
  \textbf{ધોરણો}: પાણીની ગુણવત્તા ધોરણો અને વહેતા પાણીની મર્યાદાઓ
\item
  \textbf{મોનિટરિંગ}: જળ સ્ત્રોતોની નિયમિત તપાસ અને નમૂના લેવું
\item
  \textbf{કટોકટીની જોગવાઈઓ}: પ્રદૂષણની કટોકટીઓ સંભાળવાની સત્તા
\end{itemize}

\textbf{બોર્ડની સત્તાઓ:}

\begin{itemize}
\tightlist
\item
  \textbf{આયોજન}: પ્રદૂષણ નિવારણ અને નિયંત્રણ કાર્યક્રમો
\item
  \textbf{ધોરણ સેટિંગ}: પાણીની ગુણવત્તા અને ડિસચાર્જ ધોરણો
\item
  \textbf{સંમતિ આપવી}: કચરો છોડવાની પરવાનગી
\item
  \textbf{મોનિટરિંગ}: પાણીની ગુણવત્તા દેખરેખ
\item
  \textbf{અમલીકરણ}: ઉલ્લંઘનકર્તાઓ સામે કાનૂની કાર્યવાહી
\end{itemize}

\textbf{(2) પર્યાવરણ (સંરક્ષણ) અધિનિયમ, 1986:}

ભારતમાં પર્યાવરણ સંરક્ષણ અને સુધારા માટે ફ્રેમવર્ક પૂરો પાડતો છત્ર કાયદો, ભોપાલ গેસ
દુર્ઘટના પછી ઘડવામાં આવ્યો.


{\def\LTcaptype{none} % do not increment counter
\vspace{-5pt}
\captionof{table}{પર્યાવરણ અધિનિયમ 1986 - મુખ્ય જોગવાઈઓ}
\vspace{-10pt}
\begin{longtable}[]{@{}ll@{}}
\toprule\noalign{}
પાસું & વિગતો \\
\midrule\noalign{}
\endhead
\bottomrule\noalign{}
\endlastfoot
\textbf{ઉદ્દેશ્ય} & વ્યાપક પર્યાવરણ સંરક્ષણ \\
\textbf{વ્યાપ્તિ} & હવા, પાણી, જમીન પ્રદૂષણ અને જોખમી પદાર્થો \\
\textbf{સત્તા} & કેન્દ્ર સરકાર અને નિયુક્ત એજન્સીઓ \\
\textbf{દંડ} & 5 વર્ષ સુધીની કેદ અને/અથવા ₹1 લાખ સુધીનો દંડ \\
\end{longtable}
}

\textbf{મુખ્ય વિશેષતાઓ:}

\begin{itemize}
\tightlist
\item
  \textbf{સામાન્ય સત્તાઓ}: પર્યાવરણ સંરક્ષણ માટે કેન્દ્ર સરકારની સત્તા
\item
  \textbf{ધોરણો}: હવા, પાણી, માટી માટે પર્યાવરણીય ગુણવત્તા ધોરણો
\item
  \textbf{અસર મૂલ્યાંકન}: પ્રોજેક્ટ્સ માટે પર્યાવરણીય મંજૂરી
\item
  \textbf{જોખમી પદાર્થો}: હેન્ડલિંગ અને નિકાલનું નિયમન
\item
  \textbf{જનભાગીદારી}: માહિતી અને ભાગીદારીનો અધિકાર
\end{itemize}

\textbf{મહત્વના નિયમો:}

\begin{itemize}
\tightlist
\item
  \textbf{EIA નોટિફિકેશન 2006}: પર્યાવરણીય અસર મૂલ્યાંકન
\item
  \textbf{હેજાર્ડસ વેસ્ટ રૂલ્સ}: વ્યવસ્થાપન અને હેન્ડલિંગ
\item
  \textbf{અવાજ પ્રદૂષણ નિયમો}: આસપાસના અવાજના ધોરણો
\item
  \textbf{કોસ્ટલ રેગ્યુલેશન ઝોન}: દરિયાકાંઠાના વિસ્તારનું સંરક્ષણ
\end{itemize}

\textbf{સરખામણી:}


{\def\LTcaptype{none} % do not increment counter
\vspace{-5pt}
\captionof{table}{પાણી અધિનિયમ વિ પર્યાવરણ અધિનિયમ}
\vspace{-10pt}
\begin{longtable}[]{@{}lll@{}}
\toprule\noalign{}
પાસું & પાણી અધિનિયમ 1974 & પર્યાવરણ અધિનિયમ 1986 \\
\midrule\noalign{}
\endhead
\bottomrule\noalign{}
\endlastfoot
\textbf{વ્યાપ્તિ} & માત્ર પાણી પ્રદૂષણ & તમામ પર્યાવરણીય માધ્યમો \\
\textbf{અભિગમ} & ક્ષેત્રીય & વ્યાપક \\
\textbf{અમલીકરણ} & PCBs & કેન્દ્ર સરકાર \\
\textbf{દંડ} & મધ્યમ & કડક \\
\end{longtable}
}

\textbf{અમલીકરણ મિકેનિઝમ:}

\begin{itemize}
\tightlist
\item
  \textbf{મોનિટરિંગ}: નિયમિત તપાસ અને અનુપાલન તપાસ
\item
  \textbf{કાનૂની કાર્યવાહી}: ઉલ્લંઘનકર્તાઓની કાર્યવાહી
\item
  \textbf{બંધ કરવાના આદેશો}: પ્રદૂષક એકમો બંધ કરવા
\item
  \textbf{વળતર}: પર્યાવરણીય નુકસાનનું મૂલ્યાંકન
\end{itemize}

\end{solutionbox}
\begin{mnemonicbox}
``WEPCA'' - Water Environmental Protection
Comprehensive Act

\begin{center}\rule{0.5\linewidth}{0.5pt}\end{center}

\end{mnemonicbox}
\subsection*{પ્રશ્ન 5(અ અથવા) {[}03
ગુણ{]}}\label{uxaaauxab0uxab6uxaa8-5uxa85-uxa85uxaa5uxab5-03-uxa97uxaa3}

\textbf{``કાર્બન ક્રેડિટ'' ખ્યાલ સમજાવો}

\begin{solutionbox}

કાર્બન ક્રેડિટ એ ઉત્સર્જન ઘટાડા અથવા કાર્બન સીક્વેસ્ટ્રેશન પ્રોજેક્ટ્સ દ્વારા
વાતાવરણમાંથી એક ટન CO₂ સમકક્ષ ઘટાડેલ અથવા દૂર કરેલનું પ્રતિનિધિત્વ કરતું વેપારીલાયક
પ્રમાણપત્ર છે.


{\def\LTcaptype{none} % do not increment counter
\vspace{-5pt}
\captionof{table}{કાર્બન ક્રેડિટ મિકેનિઝમ}
\vspace{-10pt}
\begin{longtable}[]{@{}ll@{}}
\toprule\noalign{}
ઘટક & વર્ણન \\
\midrule\noalign{}
\endhead
\bottomrule\noalign{}
\endlastfoot
\textbf{એકમ} & 1 ક્રેડિટ = 1 ટન CO₂ સમકક્ષ \\
\textbf{ઉત્પાદન} & ઉત્સર્જન ઘટાડા/દૂર કરવાના પ્રોજેક્ટ્સ \\
\textbf{વેપાર} & કાર્બન બજારોમાં ખરીદી/વેચાણ \\
\textbf{ચકાસણી} & તૃતીય-પક્ષ માન્યતા જરૂરી \\
\end{longtable}
}

\begin{itemize}
\tightlist
\item
  \textbf{CDM}: ક્યોટો પ્રોટોકોલ હેઠળ ક્લીન ડેવલપમેન્ટ મિકેનિઝમ
\item
  \textbf{સ્વૈચ્છિક બજારો}: ખાનગી ક્ષેત્રની પહેલ
\item
  \textbf{અનુપાલન બજારો}: નિયમનકારી જરૂરિયાતો
\end{itemize}

\end{solutionbox}
\begin{mnemonicbox}
``CUTV'' - Credit Unit Trading Verification

\begin{center}\rule{0.5\linewidth}{0.5pt}\end{center}

\end{mnemonicbox}
\subsection*{પ્રશ્ન 5(બ અથવા) {[}04
ગુણ{]}}\label{uxaaauxab0uxab6uxaa8-5uxaac-uxa85uxaa5uxab5-04-uxa97uxaa3}

\textbf{``સોલિડ વેસ્ટ મેનેજમેન્ટ'' ટૂંકમાં સમજાવો}

\begin{solutionbox}

ઘન કચરા વ્યવસ્થાપન એ માનવ પ્રવૃત્તિઓ દ્વારા છોડી દેવાયેલી ઘન સામગ્રીનું વ્યવસ્થિત
એકત્રીકરણ, પરિવહન, પ્રક્રિયા, રિસાયક્લિંગ અને નિકાલ છે.


{\def\LTcaptype{none} % do not increment counter
\vspace{-5pt}
\captionof{table}{ઘન કચરા વ્યવસ્થાપન હાયરાર્કી}
\vspace{-10pt}
\begin{longtable}[]{@{}lll@{}}
\toprule\noalign{}
પ્રાથમિકતા & પદ્ધતિ & વર્ણન \\
\midrule\noalign{}
\endhead
\bottomrule\noalign{}
\endlastfoot
\textbf{1મી} & \textbf{ઘટાડવું} & કચરાનું ઉત્પાદન ઘટાડવું \\
\textbf{2જી} & \textbf{પુનઃઉપયોગ} & વસ્તુઓનો બહુવિધ વાર ઉપયોગ \\
\textbf{3જી} & \textbf{રિસાયકલ} & કચરાને નવા ઉત્પાદનોમાં રૂપાંતરિત કરવું \\
\textbf{4થી} & \textbf{પુનઃપ્રાપ્તિ} & કચરામાંથી ઊર્જા પુનઃપ્રાપ્તિ \\
\textbf{5મી} & \textbf{નિકાલ} & સુરક્ષિત લેન્ડફિલિંગ \\
\end{longtable}
}

\textbf{વ્યવસ્થાપન પ્રક્રિયા:}

\begin{itemize}
\tightlist
\item
  \textbf{એકત્રીકરણ}: ઘરે-ઘરે પિકઅપ, સ્ત્રોતે વિભાજન
\item
  \textbf{પરિવહન}: ટ્રાન્સફર સ્ટેશન, બલ્ક ટ્રાન્સપોર્ટ
\item
  \textbf{ટ્રીટમેન્ટ}: કમ્પોસ્ટિંગ, રિસાયક્લિંગ, ઇન્સિનરેશન
\item
  \textbf{નિકાલ}: સેનિટરી લેન્ડફિલ, વેસ્ટ-ટુ-એનર્જી
\end{itemize}

\textbf{ટેકનોલોજીઓ:}

\begin{itemize}
\tightlist
\item
  \textbf{કમ્પોસ્ટિંગ}: કાર્બનિક કચરાનું વિઘટન
\item
  \textbf{ઇન્સિનરેશન}: ઊર્જા પુનઃપ્રાપ્તિ સાથે ઉચ્ચ તાપમાન બર્નિંગ
\item
  \textbf{એનેરોબિક પાચન}: કાર્બનિક કચરામાંથી બાયોગેસ ઉત્પાદન
\item
  \textbf{મટેરિયલ રિકવરી}: સામગ્રીનું વિભાજન અને રિસાયક્લિંગ
\end{itemize}

\textbf{પડકારો:}

\begin{itemize}
\tightlist
\item
  \textbf{વધતી માત્રા}: વસ્તી અને વપરાશ વૃદ્ધિ
\item
  \textbf{મિશ્ર કચરો}: સ્ત્રોતે વિભાજનનો અભાવ
\item
  \textbf{ઇન્ફ્રાસ્ટ્રક્ચર}: અપૂરતી એકત્રીકરણ અને ટ્રીટમેન્ટ સુવિધાઓ
\item
  \textbf{ફાઇનાન્સિંગ}: ઉચ્ચ મૂડી અને ઓપરેશનલ ખર્ચ
\end{itemize}

\end{solutionbox}
\begin{mnemonicbox}
``CTTD'' - Collection, Transportation, Treatment,
Disposal

\begin{center}\rule{0.5\linewidth}{0.5pt}\end{center}

\end{mnemonicbox}
\subsection*{પ્રશ્ન 5(ક અથવા) {[}07
ગુણ{]}}\label{uxaaauxab0uxab6uxaa8-5uxa95-uxa85uxaa5uxab5-07-uxa97uxaa3}

\textbf{``5R'' ની વિભાવના સમજાવો.}

\begin{solutionbox}

5R વિભાવના એ વ્યાપક કચરા વ્યવસ્થાપન હાયરાર્કી છે જે પાંચ પરસ્પર જોડાયેલ વ્યૂહરચનાઓ
દ્વારા ટકાઉ વપરાશ અને કચરા ઘટાડાને પ્રોત્સાહન આપે છે.


{\def\LTcaptype{none} % do not increment counter
\vspace{-5pt}
\captionof{table}{5R કચરા વ્યવસ્થાપન હાયરાર્કી}
\vspace{-10pt}
\begin{longtable}[]{@{}
  >{\raggedright\arraybackslash}p{(\linewidth - 6\tabcolsep) * \real{0.0857}}
  >{\raggedright\arraybackslash}p{(\linewidth - 6\tabcolsep) * \real{0.2857}}
  >{\raggedright\arraybackslash}p{(\linewidth - 6\tabcolsep) * \real{0.3429}}
  >{\raggedright\arraybackslash}p{(\linewidth - 6\tabcolsep) * \real{0.2857}}@{}}
\toprule\noalign{}
\begin{minipage}[b]{\linewidth}\raggedright
R
\end{minipage} & \begin{minipage}[b]{\linewidth}\raggedright
વ્યૂહરચના
\end{minipage} & \begin{minipage}[b]{\linewidth}\raggedright
વ્યાખ્યા
\end{minipage} & \begin{minipage}[b]{\linewidth}\raggedright
ઉદાહરણો
\end{minipage} \\
\midrule\noalign{}
\endhead
\bottomrule\noalign{}
\endlastfoot
\textbf{1. નકારવું} & બિનજરૂરી વસ્તુઓ નકારવી & કચરો બનાવતા ઉત્પાદનોથી બચવું &
પ્લાસ્ટિક બેગ, ડિસ્પોઝેબલ વસ્તુઓને ના કહેવું \\
\textbf{2. ઘટાડવું} & વપરાશ ઘટાડવો & સંસાધનોનો ઓછો ઉપયોગ & માત્ર જરૂરી વસ્તુઓ
ખરીદવી, ટકાઉ ઉત્પાદનો પસંદ કરવા \\
\textbf{3. પુનઃઉપયોગ} & વસ્તુઓનો બહુવિધ વાર ઉપયોગ & ઉત્પાદનનું જીવનકાળ વધારવું &
કન્ટેનરનો પુનઃઉપયોગ, જૂના કપડા દાન કરવા \\
\textbf{4. પુનર્નિર્દેશન} & સર્જનાત્મક વૈકલ્પિક ઉપયોગો & કચરાને ઉપયોગી વસ્તુઓમાં
રૂપાંતરિત કરવું & બોટલને પ્લાન્ટર બનાવવા, ટાયરને ઝૂલા બનાવવા \\
\textbf{5. રિસાયકલ} & કચરાને નવા ઉત્પાદનોમાં પ્રક્રિયા કરવી & સામગ્રી
પુનઃપ્રાપ્તિ અને પુનઃપ્રક્રિયા & કાગળ, પ્લાસ્ટિક, ધાતુ રિસાયક્લિંગ \\
\end{longtable}
}

\textbf{વિગતવાર સમજૂતી:}

\textbf{1. નકારવું:}

\begin{itemize}
\tightlist
\item
  \textbf{વિભાવના}: કચરા સામે પ્રથમ સંરક્ષણ રેખા
\item
  \textbf{અમલીકરણ}: ઉપભોક્તાની પસંદગી અને જાગૃતિ
\item
  \textbf{અસર}: સ્ત્રોતે કચરાનું ઉત્પાદન અટકાવે
\item
  \textbf{ઉદાહરણો}: સિંગલ-યુઝ પ્લાસ્ટિક નકારવા, બિનજરૂરી પેકેજિંગ
\end{itemize}

\textbf{2. ઘટાડવું:}

\begin{itemize}
\tightlist
\item
  \textbf{વિભાવના}: સંસાધન વપરાશ અને કચરા ઉત્પાદન ઘટાડવું
\item
  \textbf{વ્યૂહરચના}: કાર્યક્ષમ ઉપયોગ, ટકાઉપણાં પર ધ્યાન, શેરિંગ ઇકોનોમી
\item
  \textbf{ફાયદાઓ}: ઓછું પર્યાવરણીય ફૂટપ્રિન્ટ, ખર્ચ બચત
\item
  \textbf{ઉપયોગો}: ઊર્જા કાર્યક્ષમતા, પાણી સંરક્ષણ, ન્યૂનતમ પેકેજિંગ
\end{itemize}

\textbf{3. પુનઃઉપયોગ:}

\begin{itemize}
\tightlist
\item
  \textbf{વિભાવના}: પુનઃપ્રક્રિયા વિના ઉત્પાદનનું જીવન વધારવું
\item
  \textbf{પદ્ધતિઓ}: સીધો પુનઃઉપયોગ, સમારકામ અને જાળવણી, પુનર્વિતરણ
\item
  \textbf{ફાયદાઓ}: ઊર્જા બચત, આર્થિક ફાયદાઓ, સર્જનાત્મકતા
\item
  \textbf{ઉદાહરણો}: સંગ્રહ માટે કાચના જાર, ફર્નિચર પુનઃસ્થાપન
\end{itemize}

\textbf{4. પુનર્નિર્દેશન:}

\begin{itemize}
\tightlist
\item
  \textbf{વિભાવના}: વિવિધ કાર્યો માટે સર્જનાત્મક રૂપાંતરણ
\item
  \textbf{નવાચાર}: ડિઝાઇન વિચારસરણી અને સર્જનાત્મકતા
\item
  \textbf{સમુદાયિક પાસું}: મેકર સ્પેસ, DIY સંસ્કૃતિ
\item
  \textbf{પર્યાવરણીય ફાયદો}: લેન્ડફિલમાંથી કચરો વાળવું
\end{itemize}

\textbf{5. રિસાયકલ:}

\begin{itemize}
\tightlist
\item
  \textbf{વિભાવના}: સામગ્રી પુનઃપ્રાપ્તિ અને પુનઃપ્રક્રિયા
\item
  \textbf{પ્રકારો}: યાંત્રિક, રાસાયણિક, જૈવિક રિસાયક્લિંગ
\item
  \textbf{ઇન્ફ્રાસ્ટ્રક્ચર}: એકત્રીકરણ, સોર્ટિંગ, પ્રક્રિયા સુવિધાઓ
\item
  \textbf{બજારો}: રિસાયકલ કરેલી સામગ્રી માટે અંત-ઉપયોગ ઉપયોગો
\end{itemize}

\textbf{અમલીકરણ ફ્રેમવર્ક:}


{\def\LTcaptype{none} % do not increment counter
\vspace{-5pt}
\captionof{table}{5R અમલીકરણ સ્તરો}
\vspace{-10pt}
\begin{longtable}[]{@{}
  >{\raggedright\arraybackslash}p{(\linewidth - 6\tabcolsep) * \real{0.1750}}
  >{\raggedright\arraybackslash}p{(\linewidth - 6\tabcolsep) * \real{0.3500}}
  >{\raggedright\arraybackslash}p{(\linewidth - 6\tabcolsep) * \real{0.2250}}
  >{\raggedright\arraybackslash}p{(\linewidth - 6\tabcolsep) * \real{0.2500}}@{}}
\toprule\noalign{}
\begin{minipage}[b]{\linewidth}\raggedright
સ્તર
\end{minipage} & \begin{minipage}[b]{\linewidth}\raggedright
હિસ્સેદારો
\end{minipage} & \begin{minipage}[b]{\linewidth}\raggedright
ક્રિયાઓ
\end{minipage} & \begin{minipage}[b]{\linewidth}\raggedright
પરિણામો
\end{minipage} \\
\midrule\noalign{}
\endhead
\bottomrule\noalign{}
\endlastfoot
\textbf{વ્યક્તિગત} & ઉપભોક્તાઓ, પરિવારો & સભાન પસંદગીઓ, જીવનશૈલી ફેરફારો &
ઘટાડેલ વ્યક્તિગત ફૂટપ્રિન્ટ \\
\textbf{સમુદાય} & પડોશીઓ, શાળાઓ & સ્થાનિક કાર્યક્રમો, જાગૃતિ અભિયાન & સમુદાયિક
જોડાણ \\
\textbf{વ્યવસાય} & કંપનીઓ, ઉદ્યોગો & સર્ક્યુલર ઇકોનોમી, ટકાઉ ડિઝાઇન & સંસાધન
કાર્યક્ષમતા \\
\textbf{સરકાર} & નીતિ ઘડવૈયાઓ, નિયમનકારો & નિયમો, પ્રોત્સાહનો, ઇન્ફ્રાસ્ટ્રક્ચર
& સિસ્ટમ-વ્યાપી ફેરફાર \\
\end{longtable}
}

\textbf{5R અભિગમના ફાયદાઓ:}

\begin{itemize}
\tightlist
\item
  \textbf{પર્યાવરણીય}: ઘટાડેલ પ્રદૂષણ, સંસાધન સંરક્ષણ, આબોહવા સંરક્ષણ
\item
  \textbf{આર્થિક}: ખર્ચ બચત, નોકરીઓ સર્જન, નવી વ્યવસાયિક તકો
\item
  \textbf{સામાજિક}: સમુદાયિક જોડાણ, શિક્ષણ, વર્તન પરિવર્તન
\item
  \textbf{સંસાધન સુરક્ષા}: કુમારી સામગ્રી પર ઘટાડેલ નિર્ભરતા
\end{itemize}

\textbf{પડકારો:}

\begin{itemize}
\tightlist
\item
  \textbf{ઉપભોક્તા વર્તન}: સ્થાપિત આદતો અને પસંદગીઓ બદલવી
\item
  \textbf{ઇન્ફ્રાસ્ટ્રક્ચર}: પૂરતી એકત્રીકરણ અને પ્રક્રિયા સુવિધાઓ
\item
  \textbf{અર્થશાસ્ત્ર}: રિસાયકલ કરેલા ઉત્પાદનોની બજાર વ્યવહાર્યતા
\item
  \textbf{નીતિ સમર્થન}: સહાયક નિયમો અને આર્થિક સાધનો
\end{itemize}

\textbf{સફળતાના પરિબળો:}

\begin{itemize}
\tightlist
\item
  \textbf{શિક્ષણ}: જાગૃતિ અને ક્ષમતા નિર્માણ કાર્યક્રમો
\item
  \textbf{ઇન્ફ્રાસ્ટ્રક્ચર}: પૂરતી કચરા વ્યવસ્થાપન પ્રણાલી
\item
  \textbf{નીતિ}: સહાયક નિયમો અને આર્થિક સાધનો
\item
  \textbf{ટેકનોલોજી}: કચરા પ્રક્રિયા અને ઉત્પાદન ડિઝાઇનમાં નવાચાર
\item
  \textbf{સહયોગ}: બહુ-હિસ્સેદાર ભાગીદારી
\end{itemize}

\textbf{સર્ક્યુલર ઇકોનોમી કનેક્શન:} 5R વિભાવના સર્ક્યુલર ઇકોનોમી સિદ્ધાંતોનો પાયો
બનાવે છે, જ્યાં કચરો નવા ઉત્પાદન ચક્ર માટે ઇનપુટ બને છે, સંસાધન નિષ્કર્ષણ અને
પર્યાવરણીય અસર ઘટાડે છે.

\textbf{માપ અને મોનિટરિંગ:}

\begin{itemize}
\tightlist
\item
  \textbf{કચરા ઘટાડાના મેટ્રિક્સ}: નિકાલમાંથી વાળેલી માત્રા
\item
  \textbf{સામગ્રી પુનઃપ્રાપ્તિ દરો}: રિસાયકલ/પુનઃઉપયોગ કરેલા કચરાની ટકાવારી
\item
  \textbf{પર્યાવરણીય સૂચકાંકો}: કાર્બન ફૂટપ્રિન્ટ, સંસાધન વપરાશ
\item
  \textbf{આર્થિક મેટ્રિક્સ}: ખર્ચ બચત, નોકરીઓ સર્જન, આવક ઉત્પાદન
\end{itemize}

\textbf{વૈશ્વિક ઉદાહરણો:}

\begin{itemize}
\tightlist
\item
  \textbf{ઝીરો વેસ્ટ શહેરો}: સાન ફ્રાન્સિસ્કો, લજુબલજાના, કામીકાત્સુ
\item
  \textbf{વિસ્તૃત ઉત્પાદક જવાબદારી}: EU પેકેજિંગ નિયમો
\item
  \textbf{ડિપોઝિટ સિસ્ટમ}: જર્મની, કેનાડામાં બોટલ રિટર્ન કાર્યક્રમો
\item
  \textbf{શેરિંગ ઇકોનોમી}: ટૂલ લાઇબ્રેરી, કપડા સ્વેપ, રિપેર કેફે
\end{itemize}

\textbf{ભાવિ દિશાઓ:}

\begin{itemize}
\tightlist
\item
  \textbf{ડિજિટલ પ્લેટફોર્મ}: કચરા ઘટાડા અને શેરિંગ માટે એપ્સ
\item
  \textbf{એડવાન્સ્ડ રિસાયક્લિંગ}: કેમિકલ રિસાયક્લિંગ, AI-પાવર્ડ સોર્ટિંગ
\item
  \textbf{બાયોપ્લાસ્ટિક્સ}: પરંપરાગત પ્લાસ્ટિકના બાયોડિગ્રેડેબલ વિકલ્પો
\item
  \textbf{નીતિ ઉત્ક્રાંતિ}: સમારકામનો અધિકાર, વિસ્તૃત ઉત્પાદક જવાબદારી
\end{itemize}

\end{solutionbox}
\begin{mnemonicbox}
``R5-POWER'' - Refuse, Reduce, Reuse, Repurpose,
Recycle - Protect Our World's Environmental Resources

\end{mnemonicbox}
\end{document}