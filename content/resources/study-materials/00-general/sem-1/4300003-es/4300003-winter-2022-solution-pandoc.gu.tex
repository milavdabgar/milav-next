\documentclass[10pt,a4paper]{article}

% content/resources/templates/preamble.tex
\usepackage[margin=0.6in]{geometry}
\author{Milav Dabgar}
\usepackage{amsmath,amssymb,amsthm}
\usepackage{booktabs}
\usepackage{multirow}
\usepackage{xcolor}
\usepackage{tcolorbox}
\tcbuselibrary{breakable,skins}
\usepackage[colorlinks=true,linkcolor=blue]{hyperref}
\usepackage{titlesec}
\usepackage{enumitem}
\usepackage{tikz}
\usepackage{pgfplots}
\usepackage{circuitikz}
\usepackage[version=4]{mhchem}
\usepackage{longtable}
\usepackage{array}
\usepackage{float}
\usepackage{caption}
\usepackage{listings}

\lstset{
  basicstyle=\small\ttfamily,
  breaklines=true,
  breakatwhitespace=false,
  postbreak=\mbox{\textcolor{red}{$\hookrightarrow$}\space},
  float=false,
  numbers=left,
  numberstyle=\tiny\color{gray},
  numbersep=10pt,
  xleftmargin=2em,
  keywordstyle=\color{blue},
  commentstyle=\color{green!60!black},
  stringstyle=\color{purple},
  backgroundcolor=\color{gray!5},
  showstringspaces=false,
  tabsize=2,
  captionpos=b,
  keepspaces=true,
  columns=flexible
}

\pgfplotsset{compat=1.18}
\usetikzlibrary{shapes,arrows,positioning,calc,patterns,decorations.pathmorphing,decorations.markings,arrows.meta}

% Color scheme
\definecolor{headcolor}{RGB}{0,102,204}
\definecolor{keycolor}{RGB}{220,20,60}
\definecolor{solutioncolor}{RGB}{34,139,34}
\definecolor{mnemoniccolor}{RGB}{148,0,211}
\definecolor{codecolor}{RGB}{0,0,100}

% Spacing
\setlength{\parskip}{3pt}
\setlist[itemize]{nosep}
\setlist[enumerate]{nosep}

% Title formatting
\titleformat{\section}{\Large\bfseries\color{headcolor}}{\thesection}{1em}{}
\titleformat{\subsection}{\large\bfseries\color{headcolor}}{\thesubsection}{1em}{}

% Pandoc tightlist compatibility
\providecommand{\tightlist}{%
  \setlength{\itemsep}{0pt}\setlength{\parskip}{0pt}}

% Pandoc longtable compatibility
\newcounter{none}
\def\thenone{}


% content/resources/templates/gujarati-boxes.tex
\usepackage{fontspec}
\usepackage{polyglossia}

% Set Gujarati as main language (document is primarily in Gujarati)
% Note: gloss-gujarati.ldf doesn't exist in polyglossia, but it will use hyphenation patterns
\setdefaultlanguage{gujarati}
\setotherlanguage{english}

% Configure Gujarati font properly
% Use Language=Default to prevent polyglossia from trying to add language-specific features
% that don't exist for Gujarati, which causes "empty feature" warnings
\newfontfamily\gujaratifont[Script=Gujarati,AutoFakeBold=2.5,AutoFakeSlant=0.3]{Noto Sans Gujarati}
\setmainfont[Script=Gujarati,AutoFakeBold=2.5,AutoFakeSlant=0.3]{Noto Sans Gujarati}
% Use Noto Sans Gujarati for monospace to support Gujarati in text
\setmonofont[Scale=0.9]{Noto Sans Gujarati}

% Configure English to use the same font
\newfontfamily\englishfont[Script=Gujarati,AutoFakeBold=2.5,AutoFakeSlant=0.3]{Noto Sans Gujarati}

% Translations for polyglossia
\gappto\captionsgujarati{
  \renewcommand{\tablename}{કોષ્ટક}
  \renewcommand{\figurename}{આકૃતિ}
}

% Helper for TikZ nodes to ensure Gujarati font
\newcommand{\gu}[1]{{\gujaratifont #1}}

% Custom environments
\newtcolorbox{solutionbox}{
    breakable,
    enhanced,
    colback=solutioncolor!5!white,
    colframe=solutioncolor!75!black,
    fonttitle=\bfseries,
    title=જવાબ
}

\newtcolorbox{solutionboxnobreak}{
 colback=solutioncolor!5!white,
 colframe=solutioncolor!75!black,
 fonttitle=\bfseries,
 title=જવાબ
}

\newtcolorbox{keyformula}{
 breakable,
 enhanced,
 colback=keycolor!5!white,
 colframe=keycolor!75!black,
 fonttitle=\bfseries,
 title=રાસાયણિક સમીકરણ/સૂત્ર
}

\newtcolorbox{mnemonicbox}{
 breakable,
 enhanced,
 colback=mnemoniccolor!5!white,
 colframe=mnemoniccolor!75!black,
 fonttitle=\bfseries,
 title=મેમરી ટ્રીક
}


\begin{document}

\begin{center}
{\Huge\bfseries\color{headcolor} Environment and Sustainability (Gujarati)}\\[5pt]
{\LARGE 4300003 -- Winter 2022}\\[3pt]
{\large Semester 1 Study Material}\\[3pt]
{\normalsize\textit{Detailed Solutions and Explanations}}
\end{center}

\vspace{10pt}

\subsection*{પ્રશ્ન 1(a) [3
ગુણ]}\label{q1a}

\textbf{વૈશ્વિક પર્યાવરણીય ઉછાળ ક્યારે થાય છે? કારણો સાથે સમજાવો.}

\begin{solutionbox}


\vspace{-5pt}
\captionof{table}{પર્યાવરણીય ઉછાળની શરતો}
\vspace{-10pt}
\begin{longtable}[]{@{}lll@{}}
\toprule\noalign{}
શરત & વર્ણન & અસર \\
\midrule\noalign{}
\endhead
\bottomrule\noalign{}
\endlastfoot
સંસાધન ઘટાડો & વપરાશ પુનઃજનન દર કરતા વધારે & ખાધ સંચય \\
વસ્તી દબાણ & માનવ માંગ વહન ક્ષમતા કરતા વધારે & સંસાધન અછત \\
કચરાનો સંગ્રહ & ઉત્પાદન શોષણ ક્ષમતા કરતા વધારે & પર્યાવરણ અધોગતિ \\
\end{longtable}

\textbf{પર્યાવરણીય ઉછાળ} ત્યારે થાય છે જ્યારે માનવતાનું પર્યાવરણીય પદચિહ્ન પૃથ્વીની
જૈવિક ક્ષમતા કરતા વધી જાય છે.

\textbf{મુખ્ય કારણો}:

\begin{itemize}
\tightlist
\item
  \textbf{વસ્તી વૃદ્ધિ}: માનવ સંખ્યામાં વધારો
\item
  \textbf{વપરાશની પદ્ધતિ}: વ્યક્તિ દીઠ ઊંચો સંસાધન ઉપયોગ
\item
  \textbf{ટેકનોલોજીની અસર}: બિનકાર્યક્ષમ સંસાધન ઉપયોગ
\end{itemize}

\end{solutionbox}
\begin{mnemonicbox}
``POP-CON-TECH''
(Population-Consumption-Technology)

\end{mnemonicbox}
\subsection*{પ્રશ્ન 1(b) [4
ગુણ]}\label{q1b}

\textbf{આકૃતિની મદદથી પોષણ કડી સમજાવો.}

\begin{solutionbox}

\begin{center}
\textbf{Mermaid Diagram (Code)}
\begin{verbatim}
{Shaded}
{Highlighting}[]
graph LR
    A[સૂર્ય ઉર્જા] {-{-}{} B[ઉત્પાદક: લીલા છોડ]}
    B {-{-}{} C[પ્રાથમિક ઉપભોક્તા: શાકાહારી]}
    C {-{-}{} D[ગૌણ ઉપભોક્તા: માંસાહારી]}
    D {-{-}{} E[તૃતીય ઉપભોક્તા: શિકારી]}
    E {-{-}{} F[અપઘટક: બેક્ટેરિયા/ફૂગ]}
    F {-{-}{} G[માટીમાં પોષક તત્વો]}
    G {-{-}{} B}
{Highlighting}
{Shaded}
\end{verbatim}
\end{center}

\textbf{પોષણ કડી} એ ઇકોસિસ્ટમમાં એક ટ્રોફિક સ્તરથી બીજા સ્તરમાં ઉર્જા
સ્થાનાંતરણનો રેખીય ક્રમ દર્શાવે છે.

\textbf{ઘટકો}:

\begin{itemize}
\tightlist
\item
  \textbf{ઉત્પાદકો}: સૂર્ય ઉર્જાને રાસાયણિક ઉર્જામાં રૂપાંતરિત કરે છે
\item
  \textbf{પ્રાથમિક ઉપભોક્તા}: ઉત્પાદકોને ખાય છે (શાકાહારી)
\item
  \textbf{ગૌણ ઉપભોક્તા}: પ્રાથમિક ઉપભોક્તાને ખાય છે (માંસાહારી)
\item
  \textbf{અપઘટક}: મૃત જીવોને વિઘટિત કરે છે
\end{itemize}

\textbf{ઉર્જા પ્રવાહ}: સૂર્યથી ટોચના શિકારી સુધી એક દિશામાં 10\% કાર્યક્ષમતા
સાથે.

\end{solutionbox}
\begin{mnemonicbox}
``PPSD'' (Producer-Primary-Secondary-Decomposer)

\end{mnemonicbox}
\subsection*{પ્રશ્ન 1(c) [7
ગુણ]}\label{q1c}

\textbf{કાર્બન ચક્ર પર ટૂંકી નોંધ લખો.}

\begin{solutionbox}

\begin{center}
\textbf{Mermaid Diagram (Code)}
\begin{verbatim}
{Shaded}
{Highlighting}[]
graph LR
    A[વાતાવરણીય CO2] {-{-}{} B[પ્રકાશસંશ્લેષણ]}
    B {-{-}{} C[છોડનું બાયોમાસ]}
    C {-{-}{} D[પ્રાણીઓનો વપરાશ]}
    D {-{-}{} E[શ્વસન]}
    E {-{-}{} A}
    C {-{-}{} F[અપઘટન]}
    F {-{-}{} A}
    A {-{-}{} G[સમુદ્રમાં વિસર્જન]}
    G {-{-}{} H[સમુદ્રી જીવન]}
    H {-{-}{} A}
    I[અશ્મિભૂત ઇંધણ દહન] {-{-}{} A}
{Highlighting}
{Shaded}
\end{verbatim}
\end{center}

\textbf{કાર્બન ચક્ર} એ જૈવ-ભૂ-રાસાયણિક પ્રક્રિયા છે જેમાં કાર્બન વાતાવરણ, જીવમંડળ,
જળમંડળ અને ભૂમંડળમાં ફરે છે.

\textbf{મુખ્ય પ્રક્રિયાઓ}:

\begin{itemize}
\tightlist
\item
  \textbf{પ્રકાશસંશ્લેષણ}: છોડ વાતાવરણમાંથી CO2 શોષે છે
\item
  \textbf{શ્વસન}: જીવો CO2 પાછું વાતાવરણમાં છોડે છે
\item
  \textbf{અપઘટન}: મૃત કાર્બનિક પદાર્થ સંગ્રહિત કાર્બન મુક્ત કરે છે
\item
  \textbf{સમુદ્રી વિનિમય}: CO2 સમુદ્રના પાણીમાં ઓગળીને કાર્બોનિક એસિડ બનાવે છે
\end{itemize}

\textbf{માનવીય પ્રભાવ}:

\begin{itemize}
\tightlist
\item
  \textbf{અશ્મિભૂત ઇંધણ દહન}: વાતાવરણીય CO2 વધારે છે
\item
  \textbf{વનનાશ}: કાર્બન પ્રતિબંધની ક્ષમતા ઘટાડે છે
\item
  \textbf{ઔદ્યોગિક પ્રક્રિયાઓ}: વધારાના કાર્બન ઉત્સર્જન
\end{itemize}

\textbf{પર્યાવરણીય મહત્વ}: વાતાવરણીય CO2 સંતુલન જાળવે છે, વૈશ્વિક તાપમાન નિયંત્રિત
કરે છે, જીવન પ્રક્રિયાઓને આધાર આપે છે.

\end{solutionbox}
\begin{mnemonicbox}
``PRDO-FDI''
(Photosynthesis-Respiration-Decomposition-Ocean,
Fossil-Deforestation-Industry)

\end{mnemonicbox}
\subsection*{પ્રશ્ન 1(c) અથવા [7
ગુણ]}\label{q1c}

\textbf{જળીય નિવસનતંત્રનું વર્ગીકરણ કરો. દરિયાઈ નિવસનતંત્ર સમજાવો.}

\begin{solutionbox}


\vspace{-5pt}
\captionof{table}{જળીય નિવસનતંત્ર વર્ગીકરણ}
\vspace{-10pt}
\begin{longtable}[]{@{}lll@{}}
\toprule\noalign{}
પ્રકાર & લાક્ષણિકતાઓ & ઉદાહરણો \\
\midrule\noalign{}
\endhead
\bottomrule\noalign{}
\endlastfoot
તાજા પાણીનું & ઓછું મીઠું (\textless1\%) & નદીઓ, તળાવો, તાલાવો \\
દરિયાઈ & વધારે મીઠું (3.5\%) & મહાસાગરો, સમુદ્રો \\
ખારા & મિશ્રિત તાજા-ખારા પાણી & નદીમુખો, લગૂન \\
\end{longtable}

\textbf{દરિયાઈ નિવસનતંત્રના ઘટકો}:

\begin{center}
\textbf{Mermaid Diagram (Code)}
\begin{verbatim}
{Shaded}
{Highlighting}[]
graph TD
    A[દરિયાઈ નિવસનતંત્ર] {-{-}{} B[પેલેજિક ઝોન]}
    A {-{-}{} C[બેન્થિક ઝોન]}
    B {-{-}{} D[ફોટિક ઝોન: 0{-}200m]}
    B {-{-}{} E[એફોટિક ઝોન: {}200m]}
    C {-{-}{} F[ખંડીય શેલ્ફ]}
    C {-{-}{} G[ઊંડા સમુદ્રનું તળ]}
{Highlighting}
{Shaded}
\end{verbatim}
\end{center}

\textbf{દરિયાઈ નિવસનતંત્ર} પૃથ્વીની સપાટીના 71\% ભાગને આવરી લે છે, જેમાં જટિલ
ખાદ્ય જાળ સાથે ખારા પાણીના મોટા વિસ્તારો છે.

\textbf{ઝોન}:

\begin{itemize}
\tightlist
\item
  \textbf{પેલેજિક}: ખુલ્લા પાણીનો સ્તંભ જેમાં પ્લાન્કટન, માછલીઓ
\item
  \textbf{બેન્થિક}: સમુદ્રનું તળ જેમાં તળિયે રહેતા જીવો
\item
  \textbf{આંતરજોવારી}: ભરતી-ઓટના વચ્ચેનો કિનારાનો વિસ્તાર
\end{itemize}

\textbf{મહત્વ}:

\begin{itemize}
\tightlist
\item
  \textbf{આબોહવા નિયંત્રણ}: સમુદ્રી પ્રવાહો વૈશ્વિક તાપમાન નિયંત્રિત કરે છે
\item
  \textbf{ઓક્સિજન ઉત્પાદન}: દરિયાઈ ફાયટોપ્લાન્કટન વાતાવરણીય ઓક્સિજનના 50\%
  ઉત્પાદન કરે છે
\item
  \textbf{આર્થિક મૂલ્ય}: મત્સ્યવ્યવસાય, પરિવહન, પર્યટન
\end{itemize}

\end{solutionbox}
\begin{mnemonicbox}
``PBI-COE'' (Pelagic-Benthic-Intertidal,
Climate-Oxygen-Economy)

\end{mnemonicbox}
\subsection*{પ્રશ્ન 2(a) [3
ગુણ]}\label{q2a}

\textbf{પૃથ્વીની વહન ક્ષમતા એટલે શું?}

\begin{solutionbox}


\vspace{-5pt}
\captionof{table}{વહન ક્ષમતાના કારકો}
\vspace{-10pt}
\begin{longtable}[]{@{}lll@{}}
\toprule\noalign{}
કારક & વર્ણન & મર્યાદા \\
\midrule\noalign{}
\endhead
\bottomrule\noalign{}
\endlastfoot
સંસાધનો & ઉપલબ્ધ જમીન, પાણી, ખનિજો & મર્યાદિત \\
ખાદ્ય ઉત્પાદન & કૃષિ ક્ષમતા & માટી દ્વારા મર્યાદિત \\
કચરા શોષણ & ઇકોસિસ્ટમની કચરા પ્રક્રિયા & સંતૃપ્તિ બિંદુ \\
\end{longtable}

\textbf{વહન ક્ષમતા} એ પર્યાવરણને અધોગતિ કર્યા વિના અનિશ્ચિત સમય સુધી ટકાવી
શકાય તેવી મહત્તમ વસ્તી માપ છે.

\textbf{પૃથ્વીની વહન ક્ષમતા} આ પર આધાર રાખે છે:

\begin{itemize}
\tightlist
\item
  \textbf{સંસાધન ઉપલબ્ધતા}: તાજું પાણી, ખેતીલાયક જમીન, ઉર્જા સ્રોતો
\item
  \textbf{ટેકનોલોજી સ્તર}: સંસાધન ઉપયોગની કાર્યક્ષમતા
\item
  \textbf{વપરાશની પદ્ધતિ}: વ્યક્તિ દીઠ સંસાધન માંગ
\end{itemize}

\textbf{વર્તમાન અંદાજ}: વપરાશ સ્તર અને તકનીકી પ્રગતિના આધારે 4-16 અબજ લોકો.

\end{solutionbox}
\begin{mnemonicbox}
``RTC'' (Resources-Technology-Consumption)

\end{mnemonicbox}
\subsection*{પ્રશ્ન 2(b) [4
ગુણ]}\label{q2b}

\textbf{આહાર જાળ એ પોષણ કડી સાથે કેવી રીતે સંબંધિત છે?}

\begin{solutionbox}

\begin{center}
\textbf{Mermaid Diagram (Code)}
\begin{verbatim}
{Shaded}
{Highlighting}[]
graph LR
    A[ઘાસ] {-{-}{} B[સસલું]}
    A {-{-}{} C[હરણ]}
    B {-{-}{} D[શિયાળ]}
    C {-{-}{} D}
    B {-{-}{} E[બાજ]}
    C {-{-}{} F[વરુ]}
    D {-{-}{} G[અપઘટક]}
    E {-{-}{} G}
    F {-{-}{} G}
{Highlighting}
{Shaded}
\end{verbatim}
\end{center}

\textbf{આહાર જાળ} એ ઇકોસિસ્ટમમાં જટિલ ખાદ્ય સંબંધો દર્શાવતા બહુવિધ પોષણ કડીઓનું
પરસ્પર જોડાયેલું જાળ છે.

\textbf{આહાર જાળ અને પોષણ કડી વચ્ચેનો સંબંધ}:

\begin{itemize}
\tightlist
\item
  \textbf{પોષણ કડી}: ઉર્જા સ્થાનાંતરણનો રેખીય ક્રમ
\item
  \textbf{આહાર જાળ}: બહુવિધ પરસ્પર જોડાયેલી પોષણ કડીઓ
\item
  \textbf{જટિલતા}: આહાર જાળ વાસ્તવિક ઇકોસિસ્ટમ ક્રિયાપ્રતિક્રિયા દર્શાવે છે
\item
  \textbf{સ્થિરતા}: બહુવિધ માર્ગો ઇકોસિસ્ટમ પ્રતિરોધક ક્ષમતા પ્રદાન કરે છે
\end{itemize}

\textbf{મુખ્ય તફાવતો}:

\begin{itemize}
\tightlist
\item
  \textbf{માળખું}: કડી રેખીય, જાળ નેટવર્ક આધારિત
\item
  \textbf{ઉર્જા પ્રવાહ}: કડી એક માર્ગ, જાળ બહુવિધ માર્ગો
\item
  \textbf{પ્રજાતિ ક્રિયાપ્રતિક્રિયા}: જાળ સર્વભક્ષીતા અને વૈકલ્પિક ખાદ્ય દર્શાવે છે
\end{itemize}

\end{solutionbox}
\begin{mnemonicbox}
``LNCR'' (Linear-Network, Chain-Resilience)

\end{mnemonicbox}
\subsection*{પ્રશ્ન 2(c) [7
ગુણ]}\label{q2c}

\textbf{હવા પ્રદૂષણ પર નોંધ લખો.}

\begin{solutionbox}


\vspace{-5pt}
\captionof{table}{હવા પ્રદૂષણના સ્રોતો અને અસરો}
\vspace{-10pt}
\begin{longtable}[]{@{}lll@{}}
\toprule\noalign{}
પ્રદૂષક & સ્રોત & આરોગ્ય અસર \\
\midrule\noalign{}
\endhead
\bottomrule\noalign{}
\endlastfoot
PM2.5/PM10 & વાહનો, ઉદ્યોગો & શ્વસન રોગો \\
SO2 & કોલસાનું દહન & એસિડ વરસાદ, અસ્થમા \\
NOx & વાહન એક્ઝોસ્ટ & સ્મોગ રચના \\
CO & અપૂર્ણ દહન & ઓક્સિજનની ઉણપ \\
\end{longtable}

\textbf{હવા પ્રદૂષણ} એ વાતાવરણમાં હાનિકારક પદાર્થોથી થતું દૂષણ છે જે માનવ આરોગ્ય
અને પર્યાવરણ પર નકારાત્મક અસર કરે છે.

\textbf{સ્રોત પ્રમાણે વર્ગીકરણ}:

\begin{itemize}
\tightlist
\item
  \textbf{પ્રાથમિક પ્રદૂષક}: સીધું ઉત્સર્જિત (CO, SO2, કણો)
\item
  \textbf{ગૌણ પ્રદૂષક}: રાસાયણિક પ્રતિક્રિયા દ્વારા રચાય (ઓઝોન, એસિડ વરસાદ)
\end{itemize}

\textbf{મુખ્ય સ્રોતો}:

\begin{itemize}
\tightlist
\item
  \textbf{ગતિશીલ સ્રોતો}: વાહનો, વિમાન, જહાજો
\item
  \textbf{સ્થિર સ્રોતો}: પાવર પ્લાન્ટ, ઉદ્યોગો, રહેણાંક હોટિંગ
\item
  \textbf{કુદરતી સ્રોતો}: જ્વાળામુખી વિસ્ફોટ, જંગલી આગ, ધૂળના તોફાન
\end{itemize}

\textbf{નિયંત્રણ પગલાં}:

\begin{itemize}
\tightlist
\item
  \textbf{તકનીકી}: કેટેલિટિક કન્વર્ટર, સ્ક્રબર, ફિલ્ટર
\item
  \textbf{નિયમનકારી}: ઉત્સર્જન ધોરણો, ઇંધણ ગુણવત્તા નિયમો
\item
  \textbf{વૈકલ્પિક ઊર્જા}: નવીકરણીય સ્રોતો, ઇલેક્ટ્રિક વાહનો
\end{itemize}

\textbf{આરોગ્ય અસરો}: શ્વસન રોગો, હૃદયરોગ સમસ્યાઓ, કેન્સર, આયુષ્યમાં ઘટાડો.

\textbf{પર્યાવરણીય અસરો}: એસિડ વરસાદ, ઓઝોન ઘટાડો, આબોહવા પરિવર્તન, દૃશ્યતામાં
ઘટાડો.

\end{solutionbox}
\begin{mnemonicbox}
``PSMT-RE-HE''
(Primary-Secondary-Mobile-stationary-Technological-Regulatory-Health-Environment)

\end{mnemonicbox}
\subsection*{પ્રશ્ન 2(a) અથવા [3
ગુણ]}\label{q2a}

\textbf{પ્લાસ્ટિક કચરાની પર્યાવરણ પર ખરાબ અસરો સમજાવો.}

\begin{solutionbox}


\vspace{-5pt}
\captionof{table}{પ્લાસ્ટિક કચરાની પર્યાવરણીય અસરો}
\vspace{-10pt}
\begin{longtable}[]{@{}lll@{}}
\toprule\noalign{}
અસરનું ક્ષેત્ર & અસર & સમયગાળો \\
\midrule\noalign{}
\endhead
\bottomrule\noalign{}
\endlastfoot
દરિયાઈ જીવન & ફસાવટ, ગળવું & કાયમી \\
માટી & માઇક્રોપ્લાસ્ટિક દૂષણ & 500+ વર્ષો \\
ખાદ્ય શૃંખલા & બાયોએક્યુમ્યુલેશન & પેઢીદર પેઢી \\
\end{longtable}

\textbf{પ્લાસ્ટિક કચરો} તેની બિન-બાયોડિગ્રેડેબલ પ્રકૃતિને કારણે ગંભીર પર્યાવરણીય
અધોગતિનું કારણ બને છે.

\textbf{પર્યાવરણીય અસરો}:

\begin{itemize}
\tightlist
\item
  \textbf{દરિયાઈ પ્રદૂષણ}: સમુદ્રમાં પ્લાસ્ટિક દરિયાઈ પ્રાણીઓને ફસાવટ અને ગળવાથી
  મારી નાખે છે
\item
  \textbf{માટી દૂષણ}: માઇક્રોપ્લાસ્ટિક માટીની ફળદ્રુપતા અને પાકની વૃદ્ધિને અસર કરે
  છે
\item
  \textbf{ખાદ્ય શૃંખલા વિક્ષેપ}: પ્લાસ્ટિકના કણો જીવોમાં સંચિત થાય છે
\end{itemize}

\textbf{લાંબાગાળાની અસરો}: કાયમી કાર્બનિક પ્રદૂષક, આવાસનો વિનાશ, ઇકોસિસ્ટમ
અસંતુલન.

\end{solutionbox}
\begin{mnemonicbox}
``MSF'' (Marine-Soil-Foodchain)

\end{mnemonicbox}
\subsection*{પ્રશ્ન 2(b) અથવા [4
ગુણ]}\label{q2b}

\textbf{દૂષિત પાણીના લક્ષણો કયા છે? જળ પ્રદૂષણના મુખ્ય સ્રોતોની યાદી બનાવો.}

\begin{solutionbox}


\vspace{-5pt}
\captionof{table}{જળ પ્રદૂષણના સૂચકો અને સ્રોતો}
\vspace{-10pt}
\begin{longtable}[]{@{}lll@{}}
\toprule\noalign{}
લક્ષણો & માપન & સ્રોતો \\
\midrule\noalign{}
\endhead
\bottomrule\noalign{}
\endlastfoot
ઊંચું BOD/COD & \textgreater5 mg/L & ઔદ્યોગિક ડિસ્ચાર્જ \\
ટર્બિડિટી & ધૂંધળાપણું & કૃષિ અપવાહ \\
pH ફેરફાર & \textless6.5 અથવા \textgreater8.5 & એસિડ ખાણ ડ્રેનેજ \\
દુર્ગંધ & H2S ગંધ & ગટર ડિસ્ચાર્જ \\
\end{longtable}

\textbf{દૂષિત પાણીના લક્ષણો}:

\begin{itemize}
\tightlist
\item
  \textbf{ભૌતિક}: રંગ ફેરફાર, ટર્બિડિટી, તરતા કચરા, ગંધ
\item
  \textbf{રાસાયણિક}: ઊંચું BOD/COD, pH વિચલન, ભારે ધાતુઓ, ઝેરી સંયોજનો
\item
  \textbf{જૈવિક}: રોગકારક સૂક્ષ્મજીવો, એલ્ગલ બ્લૂમ, માછલીઓનું મૃત્યુ
\end{itemize}

\textbf{મુખ્ય સ્રોતો}:

\begin{itemize}
\tightlist
\item
  \textbf{બિંદુ સ્રોતો}: ઔદ્યોગિક ડિસ્ચાર્જ, ગટર આઉટફોલ, કેન્દ્રિત પ્રાણી ખવડાવવું
\item
  \textbf{બિન-બિંદુ સ્રોતો}: કૃષિ અપવાહ, શહેરી વરસાદી પાણી, વાતાવરણીય નિક્ષેપ
\end{itemize}

\end{solutionbox}
\begin{mnemonicbox}
``PCB-PIN'' (Physical-Chemical-Biological,
Point-Non-point)

\end{mnemonicbox}
\subsection*{પ્રશ્ન 2(c) અથવા [7
ગુણ]}\label{q2c}

\textbf{ઈ-કચરો શું છે? ઈ-કચરાને પુન:ઉપયોગી કેવી રીતે બનાવી શકાય?}

\begin{solutionbox}


\vspace{-5pt}
\captionof{table}{ઈ-કચરાનું વર્ગીકરણ}
\vspace{-10pt}
\begin{longtable}[]{@{}lll@{}}
\toprule\noalign{}
શ્રેણી & ઉદાહરણો & હાનિકારક ઘટકો \\
\midrule\noalign{}
\endhead
\bottomrule\noalign{}
\endlastfoot
મોટા ઉપકરણો & રેફ્રિજરેટર, વોશિંગ મશીન & CFCs, ભારે ધાતુઓ \\
નાના ઉપકરણો & માઇક્રોવેવ, વેક્યુમ ક્લીનર & પ્લાસ્ટિક, ધાતુઓ \\
IT સાધનો & કમ્પ્યુટર, પ્રિંટર & લેડ, પારો, કેડમિયમ \\
ઉપભોક્તા ઇલેક્ટ્રોનિક્સ & TV, મોબાઇલ ફોન & દુર્લભ પૃથ્વી તત્વો \\
\end{longtable}

\textbf{ઈ-કચરાનું વર્ગીકરણ}:

\begin{itemize}
\tightlist
\item
  \textbf{સફેદ સામાન}: મોટા ઘરેલું ઉપકરણો
\item
  \textbf{બ્રાઉન સામાન}: મનોરંજન ઇલેક્ટ્રોનિક્સ
\item
  \textbf{ગ્રે સામાન}: IT અને ટેલિકોમ્યુનિકેશન સાધનો
\item
  \textbf{ગ્રીન સામાન}: નવીકરણીય ઊર્જા સાધનો
\end{itemize}

\textbf{ઈ-કચરા રિસાયકલિંગ પ્રક્રિયા}:

\begin{center}
\textbf{Mermaid Diagram (Code)}
\begin{verbatim}
{Shaded}
{Highlighting}[]
graph LR
    A[સંગ્રહ] {-{-}{} B[વર્ગીકરણ]}
    B {-{-}{} C[વિઘટન]}
    C {-{-}{} D[કાપવું]}
    D {-{-}{} E[વિભાજન]}
    E {-{-}{} F[સામગ્રી પુનઃપ્રાપ્તિ]}
    F {-{-}{} G[શુદ્ધિકરણ]}
    G {-{-}{} H[નવા ઉત્પાદનો]}
{Highlighting}
{Shaded}
\end{verbatim}
\end{center}

\textbf{રિસાયકલિંગ પદ્ધતિઓ}:

\begin{itemize}
\tightlist
\item
  \textbf{યાંત્રિક}: સામગ્રીનું ભૌતિક વિભાજન
\item
  \textbf{ધાતુશાસ્ત્રીય}: ધાતુ પુનઃપ્રાપ્તિ માટે ઊંચા તાપમાનની પ્રક્રિયા
\item
  \textbf{રાસાયણિક}: કિંમતી ધાતુઓ માટે લીચિંગ પ્રક્રિયાઓ
\end{itemize}

\textbf{પડકારો}: હાનિકારક સામગ્રી હેન્ડલિંગ, જટિલ રચના, આર્થિક વ્યવહાર્યતા.

\textbf{ફાયદાઓ}: સંસાધન સંરક્ષણ, પ્રદૂષણ નિવારણ, રોજગાર સર્જન, ખાણકામની
જરૂરિયાત ઘટાડવી.

\end{solutionbox}
\begin{mnemonicbox}
``WBGG-CSDSMR'' (White-Brown-Gray-Green,
Collection-Sorting-Dismantling-Shredding-Separation-Material-Refining)

\end{mnemonicbox}
\subsection*{પ્રશ્ન 3(a) [3
ગુણ]}\label{q3a}

\textbf{BOD અને COD વચ્ચેનો તફાવત લખો.}

\begin{solutionbox}


\vspace{-5pt}
\captionof{table}{BOD વિ COD સરખામણી}
\vspace{-10pt}
\begin{longtable}[]{@{}lll@{}}
\toprule\noalign{}
પેરામીટર & BOD & COD \\
\midrule\noalign{}
\endhead
\bottomrule\noalign{}
\endlastfoot
પૂર્ણ સ્વરૂપ & બાયોકેમિકલ ઓક્સિજન ડિમાન્ડ & કેમિકલ ઓક્સિજન ડિમાન્ડ \\
ટેસ્ટ સમયગાળો & 5 દિવસ & 2-3 કલાક \\
ઓક્સિડેશન પ્રકાર & જૈવિક & રાસાયણિક \\
અપઘટન & ફક્ત બાયોડિગ્રેડેબલ કાર્બનિક & બધા કાર્બનિક સંયોજનો \\
\end{longtable}

\textbf{BOD (બાયોકેમિકલ ઓક્સિજન ડિમાન્ડ)}:

\begin{itemize}
\tightlist
\item
  સૂક્ષ્મજીવો દ્વારા વપરાતી ઓક્સિજન માપે છે
\item
  બાયોડિગ્રેડેબલ કાર્બનિક પ્રદૂષણ દર્શાવે છે
\item
  માનક ટેસ્ટ: 20^\circC પર 5 દિવસ
\end{itemize}

\textbf{COD (કેમિકલ ઓક્સિજન ડિમાન્ડ)}:

\begin{itemize}
\tightlist
\item
  રાસાયણિક ઓક્સિડેશન માટે જરૂરી ઓક્સિજન માપે છે
\item
  કુલ કાર્બનિક પ્રદૂષણ દર્શાવે છે
\item
  મજબૂત ઓક્સિડાઇઝિંગ એજન્ટ વાપરે છે (પોટેશિયમ ડાઇક્રોમેટ)
\end{itemize}

\end{solutionbox}
\begin{mnemonicbox}
``BTCD'' (Biological-Time-Chemical-Degradation)

\end{mnemonicbox}
\subsection*{પ્રશ્ન 3(b) [4
ગુણ]}\label{q3b}

\textbf{ઘન કચરાનું વર્ગીકરણ કરો.}

\begin{solutionbox}


\vspace{-5pt}
\captionof{table}{ઘન કચરાનું વર્ગીકરણ}
\vspace{-10pt}
\begin{longtable}[]{@{}lll@{}}
\toprule\noalign{}
વર્ગીકરણ & પ્રકાર & ઉદાહરણો \\
\midrule\noalign{}
\endhead
\bottomrule\noalign{}
\endlastfoot
સ્રોત દ્વારા & મ્યુનિસિપલ, ઔદ્યોગિક, કૃષિ & ઘરેલું, ફેક્ટરી, ખેતીનો કચરો \\
રચના દ્વારા & કાર્બનિક, અકાર્બનિક & ખાદ્ય કચરો, પ્લાસ્ટિક \\
જોખમ દ્વારા & હાનિકારક, બિન-હાનિકારક & તબીબી, કાગળ \\
\end{longtable}

\textbf{ઘન કચરાનું વર્ગીકરણ}:

\begin{center}
\textbf{Mermaid Diagram (Code)}
\begin{verbatim}
{Shaded}
{Highlighting}[]
graph TD
    A[ઘન કચરો] {-{-}{} B[મ્યુનિસિપલ ઘન કચરો]}
    A {-{-}{} C[ઔદ્યોગિક કચરો]}
    A {-{-}{} D[હાનિકારક કચરો]}
    A {-{-}{} E[કૃષિ કચરો]}
    B {-{-}{} F[કાર્બનિક: 50{-}60\%]}
    B {-{-}{} G[રિસાયક્લેબલ: 20{-}30\%]}
    B {-{-}{} H[જડ: 10{-}20\%]}
{Highlighting}
{Shaded}
\end{verbatim}
\end{center}

\textbf{સ્રોત દ્વારા}:

\begin{itemize}
\tightlist
\item
  \textbf{મ્યુનિસિપલ}: રહેણાંક, વ્યાપારી, સંસ્થાકીય કચરો
\item
  \textbf{ઔદ્યોગિક}: ઉત્પાદન, પ્રક્રિયાકરણ ઉપ-ઉત્પાદનો
\item
  \textbf{કૃષિ}: પાક અવશેષો, પ્રાણીઓનો કચરો
\end{itemize}

\textbf{રચના દ્વારા}: કાર્બનિક (બાયોડિગ્રેડેબલ), અકાર્બનિક (બિન-બાયોડિગ્રેડેબલ),
રિસાયક્લેબલ સામગ્રી.

\textbf{વ્યવસ્થાપન હાયરાર્કી}: ઘટાડો, પુનઃઉપયોગ, રિસાયકલ, પુનઃપ્રાપ્તિ, નિકાલ.

\end{solutionbox}
\begin{mnemonicbox}
``MIA-OIR'' (Municipal-Industrial-Agricultural,
Organic-Inorganic-Recyclable)

\end{mnemonicbox}
\subsection*{પ્રશ્ન 3(c) [7
ગુણ]}\label{q3c}

\textbf{આકૃતિની મદદથી સોલર ફોટોવોલ્ટેઇક સિસ્ટમ સમજાવો.}

\begin{solutionbox}

\begin{center}
\textbf{Mermaid Diagram (Code)}
\begin{verbatim}
{Shaded}
{Highlighting}[]
graph LR
    A[સૂર્ય કિરણોત્સર્ગ] {-{-}{} B[PV પેનલ]}
    B {-{-}{} C[DC પાવર]}
    C {-{-}{} D[ઇન્વર્ટર]}
    D {-{-}{} E[AC પાવર]}
    E {-{-}{} F[લોડ/ગ્રિડ]}
    G[બેટરી] {-{-}{} C}
    E {-{-}{} G}
    H[ચાર્જ કંટ્રોલર] {-{-}{} G}
    C {-{-}{} H}
{Highlighting}
{Shaded}
\end{verbatim}
\end{center}

\textbf{સોલર ફોટોવોલ્ટેઇક સિસ્ટમ} સેમિકન્ડક્ટર સામગ્રીનો ઉપયોગ કરીને સૂર્યપ્રકાશને
સીધા વીજળીમાં રૂપાંતરિત કરે છે.

\textbf{ઘટકો}:

\begin{itemize}
\tightlist
\item
  \textbf{PV મોડ્યુલ}: સિલિકોન સેલ્સ પ્રકાશને DC વીજળીમાં રૂપાંતરિત કરે છે
\item
  \textbf{ઇન્વર્ટર}: DC ને AC પાવરમાં રૂપાંતરિત કરે છે
\item
  \textbf{બેટરી સ્ટોરેજ}: વધારાની ઊર્જા પછીના ઉપયોગ માટે સંગ્રહિત કરે છે
\item
  \textbf{ચાર્જ કંટ્રોલર}: બેટરી ચાર્જિંગને નિયંત્રિત કરે છે
\item
  \textbf{મોનિટરિંગ સિસ્ટમ}: પ્રદર્શન અને ખામીઓને ટ્રેક કરે છે
\end{itemize}

\textbf{કાર્યિંગ સિદ્ધાંત}:

\begin{enumerate}
\tightlist
\item
  \textbf{ફોટોવોલ્ટેઇક અસર}: સોલર સેલ્સ ફોટોન્સને શોષે છે
\item
  \textbf{ઇલેક્ટ્રોન ઉત્તેજના}: ઇલેક્ટ્રોન-હોલ જોડી બનાવે છે
\item
  \textbf{કરંટ જનરેશન}: ઇલેક્ટ્રોન પ્રવાહ DC કરંટ બનાવે છે
\item
  \textbf{પાવર કંડિશનિંગ}: ઇન્વર્ટર DC ને AC માં રૂપાંતરિત કરે છે
\end{enumerate}

\textbf{પ્રકારો}:

\begin{itemize}
\tightlist
\item
  \textbf{ગ્રિડ-કનેક્ટેડ}: યુટિલિટી ગ્રિડ સાથે સમન્વયિત
\item
  \textbf{સ્ટેન્ડ-એલોન}: બેટરી બેકઅપ સાથે સ્વતંત્ર સિસ્ટમ
\item
  \textbf{હાઇબ્રિડ}: ગ્રિડ-કનેક્ટેડ અને બેટરી સ્ટોરેજનું સંયોજન
\end{itemize}

\textbf{ઉપયોગો}: રહેણાંક છત, વ્યાપારી ઇમારતો, યુટિલિટી-સ્કેલ પાવર પ્લાન્ટ, દૂરના
વિસ્તારોમાં વીજકરણ.

\textbf{ફાયદાઓ}: સ્વચ્છ ઊર્જા, ઓછા જાળવણી, મોડ્યુલર ડિઝાઇન, લાંબી આયુષ્ય (25+
વર્ષ).

\end{solutionbox}
\begin{mnemonicbox}
``PIBCM-PECG''
(Panel-Inverter-Battery-Controller-Monitor,
Photovoltaic-Electron-Current-Grid)

\end{mnemonicbox}
\subsection*{પ્રશ્ન 3(a) અથવા [3
ગુણ]}\label{q3a}

\textbf{પરંપરાગત અને બિનપરંપરાગત ઊર્જા સ્રોતોની સરખામણી કરો.}

\begin{solutionbox}


\vspace{-5pt}
\captionof{table}{ઊર્જા સ્રોતોની સરખામણી}
\vspace{-10pt}
\begin{longtable}[]{@{}lll@{}}
\toprule\noalign{}
પાસું & પરંપરાગત & બિનપરંપરાગત \\
\midrule\noalign{}
\endhead
\bottomrule\noalign{}
\endlastfoot
ઉપલબ્ધતા & મર્યાદિત ભંડાર & અમર્યાદિત/નવીકરણીય \\
પર્યાવરણીય અસર & વધારે પ્રદૂષણ & સ્વચ્છ/ન્યૂનતમ અસર \\
કિંમત & શુરુઆતમાં ઓછી & ઝડપથી ઘટતી \\
\end{longtable}

\textbf{પરંપરાગત ઊર્જા સ્રોતો}: કોલસો, તેલ, કુદરતી ગેસ, પરમાણુ શક્તિ - મર્યાદિત
સંસાધનો પર્યાવરણીય ચિંતાઓ સાથે.

\textbf{બિનપરંપરાગત ઊર્જા સ્રોતો}: સૌર, પવન, હાઇડ્રો, બાયોમાસ - ટકાઉ
લાક્ષણિકતાઓ સાથે નવીકરણીય સંસાધનો.

\textbf{મુખ્ય તફાવતો}: ઘટાડો વિ નવીકરણીય, પ્રદૂષણ વિ સ્વચ્છ, સ્થાપિત વિ ઉભરતી
ટેકનોલોજી.

\end{solutionbox}
\begin{mnemonicbox}
``AEC'' (Availability-Environmental-Cost)

\end{mnemonicbox}
\subsection*{પ્રશ્ન 3(b) અથવા [4
ગુણ]}\label{q3b}

\textbf{કુદરતી પરિભ્રમણ આધારિત સોલર વોટર હીટરનું કાર્યિંગ સમજાવો.}

\begin{solutionbox}

\begin{verbatim}
    +{-{-}{-}{-}{-}{-}{-}{-}{-}{-}{-}{-}{-}{-}{-}{-}{-}{-}+}
    |   સોલર ટાંકી      |
    |   (ગરમ પાણી)     |
    +{-{-}{-}{-}{-}{-}{-}{-}+{-}{-}{-}{-}{-}{-}{-}{-}{-}+}
             |
    +{-{-}{-}{-}{-}{-}{-}{-}v{-}{-}{-}{-}{-}{-}{-}{-}{-}+}
    |  સોલર કલેક્ટર   |
    |     (ઠંડું પાણી)  |
    +{-{-}{-}{-}{-}{-}{-}{-}{-}{-}{-}{-}{-}{-}{-}{-}{-}{-}+}
\end{verbatim}

\textbf{કુદરતી પરિભ્રમણ સોલર વોટર હીટર} બાહ્ય પંપ વિના પાણીના પરિભ્રમણ માટે
થર્મોસાઇફોન સિદ્ધાંતનો ઉપયોગ કરે છે.

\textbf{કાર્યિંગ સિદ્ધાંત}:

\begin{itemize}
\tightlist
\item
  \textbf{સોલર કલેક્શન}: કલેક્ટર સૂર્ય કિરણોત્સર્ગ શોષીને પાણીને ગરમ કરે છે
\item
  \textbf{ઘનતાનો તફાવત}: ગરમ પાણી ઓછું ઘન બને છે, કુદરતી રીતે ઉપર આવે છે
\item
  \textbf{પરિભ્રમણ}: ટાંકીના તળિયેથી ઠંડું પાણી કલેક્ટરમાં વહે છે
\item
  \textbf{સંગ્રહ}: ગરમ પાણી ઇન્સ્યુલેટેડ સ્ટોરેજ ટાંકીમાં એકત્રિત થાય છે
\end{itemize}

\textbf{ઘટકો}: ફ્લેટ પ્લેટ કલેક્ટર, ઇન્સ્યુલેટેડ સ્ટોરેજ ટાંકી, જોડાણ પાઇપ, સેફ્ટી વાલ્વ.

\textbf{ફાયદાઓ}: વીજળીની જરૂર નથી, સરળ ડિઝાઇન, ઓછી જાળવણી, ખર્ચ-અસરકારક.

\end{solutionbox}
\begin{mnemonicbox}
``SDCS'' (Solar-Density-Circulation-Storage)

\end{mnemonicbox}
\subsection*{પ્રશ્ન 3(c) અથવા [7
ગુણ]}\label{q3c}

\textbf{હોરિઝોન્ટલ એક્સિસ વિન્ડ ટર્બાઇનનો કાર્યસિદ્ધાંત સમજાવો.}

\begin{solutionbox}

\begin{center}
\textbf{Mermaid Diagram (Code)}
\begin{verbatim}
{Shaded}
{Highlighting}[]
graph LR
    A[પવન ઊર્જા] {-{-}{} B[રોટર બ્લેડ]}
    B {-{-}{} C[શાફ્ટ રોટેશન]}
    C {-{-}{} D[ગિયરબોક્સ]}
    D {-{-}{} E[જનરેટર]}
    E {-{-}{} F[વિદ્યુત શક્તિ]}
    G[નેસેલ] {-{-}{} B}
    H[ટાવર] {-{-}{} G}
{Highlighting}
{Shaded}
\end{verbatim}
\end{center}

\textbf{હોરિઝોન્ટલ એક્સિસ વિન્ડ ટર્બાઇન (HAWT)} એરોડાયનેમિક લિફ્ટ સિદ્ધાંતનો
ઉપયોગ કરીને પવનની ગતિ ઊર્જાને વિદ્યુત ઊર્જામાં રૂપાંતરિત કરે છે.

\textbf{કાર્યિંગ સિદ્ધાંત}:

\begin{enumerate}
\tightlist
\item
  \textbf{પવન કેપ્ચર}: રોટર બ્લેડ એરોડાયનેમિક પ્રોફાઇલ સાથે ડિઝાઇન કરેલા
\item
  \textbf{લિફ્ટ જનરેશન}: બ્લેડ સપાટીઓ પર દબાણનો તફાવત લિફ્ટ બળ બનાવે છે
\item
  \textbf{રોટેશન}: લિફ્ટ બળ રોટરને આડી ધરી આસપાસ ફેરવે છે
\item
  \textbf{સ્પીડ કન્વર્ઝન}: ગિયરબોક્સ રોટેશનલ સ્પીડ 30-50 rpm થી 1500 rpm સુધી
  વધારે છે
\item
  \textbf{પાવર જનરેશન}: ઊંચી સ્પીડ રોટેશન વિદ્યુત જનરેટર ચલાવે છે
\end{enumerate}

\textbf{ઘટકો}:

\begin{itemize}
\tightlist
\item
  \textbf{રોટર એસેમ્બલી}: 2-3 બ્લેડ, હબ, પિચ કંટ્રોલ સિસ્ટમ
\item
  \textbf{નેસેલ}: ગિયરબોક્સ, જનરેટર, કંટ્રોલ સિસ્ટમ્સ હાઉસ કરે છે
\item
  \textbf{ટાવર}: ઓપ્ટિમલ ઊંચાઈ (50-120m) પર નેસેલને સપોર્ટ કરે છે
\item
  \textbf{ફાઉન્ડેશન}: માળખાકીય સ્થિરતા માટે કોંક્રિટ બેઝ
\end{itemize}

\textbf{કંટ્રોલ સિસ્ટમ્સ}:

\begin{itemize}
\tightlist
\item
  \textbf{યાવ સિસ્ટમ}: ટર્બાઇનને પવનની દિશા તરફ ઓરિએન્ટ કરે છે
\item
  \textbf{પિચ કંટ્રોલ}: ઓપ્ટિમલ પવન કેપ્ચર માટે બ્લેડ એંગલ એડજસ્ટ કરે છે
\item
  \textbf{બ્રેક સિસ્ટમ}: ઈમર્જન્સી સ્ટોપિંગ મેકેનિઝમ
\end{itemize}

\textbf{ફાયદાઓ}: ઊંચી કાર્યક્ષમતા (35-45\%), સાબિત ટેકનોલોજી, સ્કેલની
અર્થવ્યવસ્થા. \textbf{ગેરફાયદાઓ}: વિઝ્યુઅલ ઈમ્પેક્ટ, ઘોંઘાટ, પક્ષીઓની અથડામણ, પવનની
પરિવર્તનશીલતા.

\textbf{પાવર કેલ્ક્યુલેશન}: P = 0.5 \times ρ \times A \times V^{3} \times Cp જ્યાં: ρ = હવાની ઘનતા, A
= સ્વેપ્ટ એરિયા,

V = પવનની ઝડપ, Cp = પાવર કોએફિશિયન્ટ


\end{solutionbox}
\begin{mnemonicbox}
``WLRSG-RNTP-YPB''
(Wind-Lift-Rotation-Speed-Generation, Rotor-Nacelle-Tower-Foundation,
Yaw-Pitch-Brake)

\end{mnemonicbox}
\subsection*{પ્રશ્ન 4(a) [3
ગુણ]}\label{q4a}

\textbf{ભરતી ઊર્જાના લાભ અને ગેરલાભ જણાવો.}

\begin{solutionbox}


\vspace{-5pt}
\captionof{table}{ભરતી ઊર્જાના ફાયદા અને ગેરફાયદા}
\vspace{-10pt}
\begin{longtable}[]{@{}ll@{}}
\toprule\noalign{}
ફાયદાઓ & ગેરફાયદાઓ \\
\midrule\noalign{}
\endhead
\bottomrule\noalign{}
\endlastfoot
અનુમાનિત ઊર્જા સ્રોત & મર્યાદિત યોગ્ય સ્થાનો \\
ગ્રીનહાઉસ ગેસ ઉત્સર્જન નથી & ઊંચી પ્રારંભિક મૂડી કિંમત \\
લાંબી આયુષ્ય (100+ વર્ષ) & દરિયાઈ જીવન પર પર્યાવરણીય અસર \\
\end{longtable}

\textbf{ભરતી ઊર્જા} પૃથ્વી, ચંદ્ર અને સૂર્ય વચ્ચેના ગુરુત્વાકર્ષણ બળોનો ઉપયોગ કરીને
વીજળી ઉત્પન્ન કરે છે.

\textbf{ફાયદાઓ}:

\begin{itemize}
\tightlist
\item
  \textbf{વિશ્વસનીયતા}: અત્યંત અનુમાનિત ભરતી ચક્ર
\item
  \textbf{સ્વચ્છ ઊર્જા}: શૂન્ય ઓપરેશનલ ઉત્સર્જન
\item
  \textbf{ટકાઉપણું}: ઇન્ફ્રાસ્ટ્રક્ચર દાયકાઓ ટકે છે
\end{itemize}

\textbf{ગેરફાયદાઓ}:

\begin{itemize}
\tightlist
\item
  \textbf{ભૌગોલિક મર્યાદાઓ}: ચોક્કસ કિનારાકીય પરિસ્થિતિઓની જરૂર
\item
  \textbf{ઊંચી કિંમતો}: મોંઘું ઇન્સ્ટોલેશન અને જાળવણી
\item
  \textbf{ઇકોલોજિકલ ઈમ્પેક્ટ}: દરિયાઈ ઇકોસિસ્ટમ્સને અસર કરે છે
\end{itemize}

\end{solutionbox}
\begin{mnemonicbox}
``RCD-GHE'' (Reliable-Clean-Durable,
Geographic-High cost-Ecological)

\end{mnemonicbox}
\subsection*{પ્રશ્ન 4(b) [4
ગુણ]}\label{q4b}

\textbf{બાયોગેસ પ્લાન્ટનો કાર્યસિદ્ધાંત સમજાવો.}

\begin{solutionbox}

\begin{center}
\textbf{Mermaid Diagram (Code)}
\begin{verbatim}
{Shaded}
{Highlighting}[]
graph LR
    A[કાર્બનિક કચરો ઇનપુટ] {-{-}{} B[મિક્સિંગ ટાંકી]}
    B {-{-}{} C[ડાયજેસ્ટર ટાંકી]}
    C {-{-}{} D[ગેસ કલેક્શન]}
    C {-{-}{} E[સ્લરી આઉટપુટ]}
    D {-{-}{} F[બાયોગેસ સ્ટોરેજ]}
    F {-{-}{} G[અંતિમ ઉપયોગ]}
{Highlighting}
{Shaded}
\end{verbatim}
\end{center}

\textbf{બાયોગેસ પ્લાન્ટ} કાર્બનિક કચરા સામગ્રીના એનેરોબિક ડાયજેસ્શન દ્વારા મિથેન
સમૃદ્ધ ગેસ ઉત્પન્ન કરે છે.

\textbf{કાર્યિંગ સિદ્ધાંત}:

\begin{enumerate}
\tightlist
\item
  \textbf{ફીડ તૈયારી}: કાર્બનિક કચરો પાણી સાથે મિક્સ (1:1 રેશિયો)
\item
  \textbf{એનેરોબિક ડાયજેસ્શન}: ઓક્સિજન-મુક્ત વાતાવરણમાં બેક્ટેરિયા કાર્બનિક પદાર્થને
  તોડે છે
\item
  \textbf{ગેસ ઉત્પાદન}: મિથેન (50-70\%) અને CO2 (30-40\%) ઉત્પન્ન થાય છે
\item
  \textbf{ગેસ કલેક્શન}: બાયોગેસ ગેસ હોલ્ડર ડોમમાં એકત્રિત થાય છે
\end{enumerate}

\textbf{પ્રક્રિયાના તબક્કાઓ}:

\begin{itemize}
\tightlist
\item
  \textbf{હાયડ્રોલિસિસ}: જટિલ કાર્બનિક પદાર્થો સરળ સંયોજનોમાં તૂટે છે
\item
  \textbf{એસિડોજેનેસિસ}: કાર્બનિક એસિડ રચના
\item
  \textbf{મિથેનોજેનેસિસ}: મિથેનોજેનિક બેક્ટેરિયા દ્વારા મિથેન ઉત્પાદન
\end{itemize}

\textbf{ઓપ્ટિમલ કંડિશન્સ}: તાપમાન 35-40^\circC, pH 6.8-7.2, રિટેન્શન ટાઇમ 15-30
દિવસ.

\end{solutionbox}
\begin{mnemonicbox}
``FAGH-HAM'' (Feed-Anaerobic-Gas-Holder,
Hydrolysis-Acidogenesis-Methanogenesis)

\end{mnemonicbox}
\subsection*{પ્રશ્ન 4(c) [7
ગુણ]}\label{q4c}

\textbf{ગ્રીનહાઉસ અસર સમજાવો.}

\begin{solutionbox}

\begin{center}
\textbf{Mermaid Diagram (Code)}
\begin{verbatim}
{Shaded}
{Highlighting}[]
graph LR
    A[સૂર્ય કિરણોત્સર્ગ] {-{-}{} B[પૃથ્વીની સપાટી]}
    B {-{-}{} C[ગરમી શોષણ]}
    C {-{-}{} D[ઇન્ફ્રારેડ કિરણોત્સર્ગ]}
    D {-{-}{} E[ગ્રીનહાઉસ ગેસેસ]}
    E {-{-}{} F[હીટ ટ્રેપિંગ]}
    F {-{-}{} G[પૃથ્વી તરફ પુનઃકિરણોત્સર્ગ]}
    G {-{-}{} H[ગ્લોબલ વોર્મિંગ]}
{Highlighting}
{Shaded}
\end{verbatim}
\end{center}

\textbf{ગ્રીનહાઉસ અસર} એ પ્રક્રિયા છે જેમાં વાતાવરણીય ગેસેસ સૂર્યથી આવતી ગરમીને
પકડી રાખે છે, જેનાથી પૃથ્વીની સપાટીનું તાપમાન સામાન્ય કરતાં વધારે થાય છે.

\textbf{કુદરતી ગ્રીનહાઉસ અસર}:

\begin{itemize}
\tightlist
\item
  \textbf{સૂર્ય કિરણોત્સર્ગ}: સૂર્ય શોર્ટ-વેવ કિરણોત્સર્ગ (દૃશ્ય પ્રકાશ) ઉત્સર્જિત કરે છે
\item
  \textbf{સપાટી શોષણ}: પૃથ્વી સૂર્ય ઊર્જા શોષીને ગરમ થાય છે
\item
  \textbf{હીટ રી-ઇમિશન}: પૃથ્વી લોંગ-વેવ ઇન્ફ્રારેડ કિરણોત્સર્ગ ઉત્સર્જિત કરે છે
\item
  \textbf{ગેસ શોષણ}: ગ્રીનહાઉસ ગેસેસ ઇન્ફ્રારેડ કિરણોત્સર્ગ શોષે છે
\item
  \textbf{હીટ રિટેન્શન}: પકડાયેલી ગરમી નીચલા વાતાવરણને ગરમ કરે છે
\end{itemize}

\textbf{ગ્રીનહાઉસ ગેસેસ અને યોગદાન}:

\begin{itemize}
\tightlist
\item
  \textbf{કાર્બન ડાયોક્સાઇડ (CO2)}: 76\% - અશ્મિભૂત ઇંધણ દહન, વનનાશ
\item
  \textbf{મિથેન (CH4)}: 16\% - કૃષિ, લેન્ડફિલ, પશુધન
\item
  \textbf{નાઇટ્રસ ઓક્સાઇડ (N2O)}: 6\% - ફર્ટિલાઇઝર, અશ્મિભૂત ઇંધણ દહન
\item
  \textbf{ફ્લોરિનેટેડ ગેસેસ}: 2\% - ઔદ્યોગિક પ્રક્રિયાઓ, રેફ્રિજરેશન
\end{itemize}

\textbf{વધેલી ગ્રીનહાઉસ અસર}: માનવીય પ્રવૃત્તિઓ ગ્રીનહાઉસ ગેસની સાંદ્રતા વધારે છે,
હીટ ટ્રેપિંગ તીવ્ર બનાવે છે.

\textbf{પરિણામો}:

\begin{itemize}
\tightlist
\item
  \textbf{ગ્લોબલ ટેમ્પરેચર રાઇઝ}: પ્રિ-ઇન્ડસ્ટ્રિયલ કાળથી સરેરાશ 1.1^\circC વધારો
\item
  \textbf{આબોહવા પરિવર્તન}: બદલાયેલા વરસાદી પેટર્ન, આત્યંતિક હવામાન ઘટનાઓ
\item
  \textbf{સમુદ્રી સપાટીમાં વધારો}: થર્મલ વિસ્તરણ અને બરફની ચાદર પીગળવી
\item
  \textbf{ઇકોસિસ્ટમ વિક્ષેપ}: પ્રજાતિઓનું સ્થાનાંતરણ, કોરલ બ્લીચિંગ, જંગલની આગ
\end{itemize}

\textbf{શમન વ્યૂહરચનાઓ}:

\begin{itemize}
\tightlist
\item
  \textbf{નવીકરણીય ઊર્જા}: અશ્મિભૂત ઇંધણ અવલંબન ઘટાડવું
\item
  \textbf{ઊર્જા કાર્યક્ષમતા}: ટેકનોલોજી અને પ્રથાઓમાં સુધારો
\item
  \textbf{કાર્બન સિક્વેસ્ટ્રેશન}: વન પુનઃસ્થાપન, કાર્બન કેપ્ચર સ્ટોરેજ
\item
  \textbf{આંતરરાષ્ટ્રીય સહકાર}: પેરિસ એગ્રીમેન્ટ, ઉત્સર્જન ઘટાડાના લક્ષ્યો
\end{itemize}

\end{solutionbox}
\begin{mnemonicbox}
``SSAHR-CMNO-GTSE-RECC''
(Solar-Surface-Absorption-Heat-Radiation, CO2-Methane-Nitrous-Other,
Global-Temperature-Sea-Ecosystem,
Renewable-Efficiency-Carbon-Cooperation)

\end{mnemonicbox}
\subsection*{પ્રશ્ન 4(a) અથવા [3
ગુણ]}\label{q4a}

\textbf{આબોહવા પરિવર્તન શું છે?}

\begin{solutionbox}


\vspace{-5pt}
\captionof{table}{આબોહવા પરિવર્તનના સૂચકો}
\vspace{-10pt}
\begin{longtable}[]{@{}lll@{}}
\toprule\noalign{}
સૂચક & પરિવર્તન & પુરાવા \\
\midrule\noalign{}
\endhead
\bottomrule\noalign{}
\endlastfoot
તાપમાન & +1.1^\circC 1880 થી & વૈશ્વિક તાપમાન રેકોર્ડ્સ \\
સમુદ્રી સ્તર & +21 cm 1900 થી & સેટેલાઇટ માપન \\
આર્કટિક બરફ & -13\% પ્રતિ દાયકા & સેટેલાઇટ ઇમેજરી \\
\end{longtable}

\textbf{આબોહવા પરિવર્તન} એ વૈશ્વિક તાપમાન અને હવામાનની પેટર્નમાં લાંબાગાળાના
ફેરફારોનો સંદર્ભ છે, જે મુખ્યત્વે 20મી સદીના મધ્યથી માનવીય પ્રવૃત્તિઓને કારણે થયા છે.

\textbf{મુખ્ય લાક્ષણિકતાઓ}:

\begin{itemize}
\tightlist
\item
  \textbf{તાપમાન વૃદ્ધિ}: વૈશ્વિક સરેરાશ તાપમાનમાં વધારો
\item
  \textbf{હવામાનની આત્યંતિકતા}: વધુ વારંવાર વાવાઝોડા, દુષ્કાળ, પૂર
\item
  \textbf{ઇકોસિસ્ટમ ફેરફારો}: પ્રજાતિ સ્થાનાંતરણ, આવાસ નુકસાન
\end{itemize}

\textbf{પ્રાથમિક કારણ}: અશ્મિભૂત ઇંધણ દહન, વનનાશ, ઔદ્યોગિક પ્રક્રિયાઓથી વધેલા
ગ્રીનહાઉસ ગેસ ઉત્સર્જન.

\end{solutionbox}
\begin{mnemonicbox}
``TSE'' (Temperature-Sea level-Ecosystem)

\end{mnemonicbox}
\subsection*{પ્રશ્ન 4(b) અથવા [4
ગુણ]}\label{q4b}

\textbf{આબોહવા પરિવર્તનને નિયંત્રિત કરવા કયા કયા પગલાં ભરી શકાય?}

\begin{solutionbox}


\vspace{-5pt}
\captionof{table}{ગ્લોબલ વોર્મિંગ નિયંત્રણ પગલાં}
\vspace{-10pt}
\begin{longtable}[]{@{}lll@{}}
\toprule\noalign{}
શ્રેણી & પગલાં & અસર \\
\midrule\noalign{}
\endhead
\bottomrule\noalign{}
\endlastfoot
ઊર્જા & નવીકરણીય સ્રોતો, કાર્યક્ષમતા & CO2 ઉત્સર્જન ઘટાડવું \\
પરિવહન & ઇલેક્ટ્રિક વાહનો, સાર્વજનિક પરિવહન & ઇંધણ વપરાશ ઓછો \\
ઉદ્યોગ & સ્વચ્છ ટેકનોલોજી, કાર્બન કેપ્ચર & ઉત્સર્જન ઘટાડવું \\
વ્યક્તિગત & ઊર્જા બચત, જીવનશૈલીમાં ફેરફાર & સંચિત અસર \\
\end{longtable}

\textbf{નિયંત્રણ પગલાં}:

\textbf{સરકારી સ્તરે}:

\begin{itemize}
\tightlist
\item
  \textbf{નીતિ ફ્રેમવર્ક}: કાર્બન પ્રાઇસિંગ, ઉત્સર્જન ધોરણો
\item
  \textbf{નવીકરણીય ઊર્જા}: સોલર, વિન્ડ પાવર પ્રમોશન
\item
  \textbf{પબ્લિક ટ્રાન્સપોર્ટ}: માસ ટ્રાન્ઝિટ સિસ્ટમ ડેવલપમેન્ટ
\end{itemize}

\textbf{ઔદ્યોગિક સ્તરે}:

\begin{itemize}
\tightlist
\item
  \textbf{સ્વચ્છ ટેકનોલોજી}: કાર્યક્ષમ પ્રક્રિયાઓ, કચરો ઘટાડવો
\item
  \textbf{કાર્બન કેપ્ચર}: સ્ટોરેજ અને યુટિલાઇઝેશન ટેકનોલોજીઓ
\item
  \textbf{ટકાઉ પ્રથાઓ}: ગ્રીન મેન્યુફેક્ચરિંગ, સર્ક્યુલર ઇકોનોમી
\end{itemize}

\textbf{વ્યક્તિગત સ્તરે}:

\begin{itemize}
\tightlist
\item
  \textbf{ઊર્જા બચત}: LED લાઇટ્સ, કાર્યક્ષમ ઉપકરણો
\item
  \textbf{પરિવહન}: ચાલવું, સાયક્લિંગ, કારપૂલિંગ
\item
  \textbf{જીવનશૈલીમાં ફેરફાર}: ઓછો વપરાશ, રિસાયક્લિંગ
\end{itemize}

\end{solutionbox}
\begin{mnemonicbox}
``PRT-CCS-ECL'' (Policy-Renewable-Transport,
Carbon-Clean-Sustainable, Energy-Communication-Lifestyle)

\end{mnemonicbox}
\subsection*{પ્રશ્ન 4(c) અથવા [7
ગુણ]}\label{q4c}

\textbf{આબોહવા પરિવર્તનને હળવું કરવા વૈશ્વિક સ્તરે કયા અગત્યના કરારો થયા છે?}

\begin{solutionbox}


\vspace{-5pt}
\captionof{table}{મુખ્ય આબોહવા કરારો}
\vspace{-10pt}
\begin{longtable}[]{@{}lll@{}}
\toprule\noalign{}
કરાર & વર્ષ & મુખ્ય લક્ષણો \\
\midrule\noalign{}
\endhead
\bottomrule\noalign{}
\endlastfoot
UNFCCC & 1992 & ફ્રેમવર્ક કન્વેન્શન \\
ક્યોટો પ્રોટોકોલ & 1997 & બંધનકર્તા ઉત્સર્જન લક્ષ્યો \\
પેરિસ એગ્રીમેન્ટ & 2015 & વૈશ્વિક તાપમાન મર્યાદા \\
\end{longtable}

\textbf{મહત્વપૂર્ણ વૈશ્વિક આબોહવા કરારો}:

\textbf{1. યુનાઇટેડ નેશન્સ ફ્રેમવર્ક કન્વેન્શન ઓન ક્લાઇમેટ ચેન્જ (UNFCCC) - 1992}:

\begin{itemize}
\tightlist
\item
  \textbf{ઉદ્દેશ્ય}: ગ્રીનહાઉસ ગેસની સાંદ્રતા સ્થિર કરવી
\item
  \textbf{સિદ્ધાંતો}: સામાન્ય પરંતુ વિભેદિત જવાબદારીઓ
\item
  \textbf{ફ્રેમવર્ક}: ભાવિ આબોહવા વાટાઘાટોનો આધાર
\end{itemize}

\textbf{2. ક્યોટો પ્રોટોકોલ - 1997}:

\begin{itemize}
\tightlist
\item
  \textbf{બંધનકર્તા લક્ષ્યો}: વિકસિત દેશો 5.2\% ઉત્સર્જન ઘટાડો (1990 સ્તર)
\item
  \textbf{લવચીક મેકેનિઝમ}: ઉત્સર્જન ટ્રેડિંગ, ક્લીન ડેવલપમેન્ટ મેકેનિઝમ
\item
  \textbf{કમિટમેન્ટ પીરિયડ}: પ્રથમ (2008-2012), બીજો (2013-2020)
\end{itemize}

\textbf{3. પેરિસ એગ્રીમેન્ટ - 2015}:

\begin{itemize}
\tightlist
\item
  \textbf{તાપમાન લક્ષ્ય}: ગ્લોબલ વોર્મિંગને 2^\circC કરતાં નીચે, પ્રાધાન્ય 1.5^\circC
\item
  \textbf{રાષ્ટ્રીય નિર્ધારિત યોગદાન (NDCs)}: દેશો પોતાના લક્ષ્યો સેટ કરે છે
\item
  \textbf{પુનરાવલોકન મેકેનિઝમ}: પાંચ વર્ષીય મૂલ્યાંકન અને વિસ્તૃતિકરણ ચક્ર
\item
  \textbf{આબોહવા ફાઇનાન્સ}: વિકાસશીલ દેશો માટે વાર્ષિક \$100 બિલિયન
\end{itemize}

\textbf{4. અન્ય મહત્વપૂર્ણ કરારો}:

\begin{itemize}
\tightlist
\item
  \textbf{મોન્ટ્રીયલ પ્રોટોકોલ (1987)}: ઓઝોન સ્તર સંરક્ષણ, અપ્રત્યક્ષ આબોહવા લાભો
\item
  \textbf{કોપેનહેગન એકોર્ડ (2009)}: ઉત્સર્જન ઘટાડા પર રાજકીય કરાર
\item
  \textbf{દોહા એમેન્ડમેન્ટ (2012)}: ક્યોટો પ્રોટોકોલ કમિટમેન્ટ વિસ્તૃત
\end{itemize}

\textbf{અમલીકરણના પડકારો}:

\begin{itemize}
\tightlist
\item
  \textbf{અનુપાલન}: સ્વૈચ્છિક બનામ ફરજિયાત પ્રતિબદ્ધતાઓ
\item
  \textbf{ફાઇનાન્સિંગ}: શમન અને અનુકૂલન માટે પૂરતું ફંડિંગ
\item
  \textbf{ટેકનોલોજી ટ્રાન્સફર}: વિકાસશીલ દેશો માટે સ્વચ્છ ટેકનોલોજી પહોંચ
\item
  \textbf{મોનિટરિંગ}: પારદર્શક રિપોર્ટિંગ અને વેરિફિકેશન સિસ્ટમ્સ
\end{itemize}

\textbf{તાજેતરના વિકાસો}:

\begin{itemize}
\tightlist
\item
  \textbf{આર્ટિકલ 6 નિયમો}: પેરિસ એગ્રીમેન્ટ હેઠળ આંતરરાષ્ટ્રીય કાર્બન માર્કેટ્સ
\item
  \textbf{લોસ એન્ડ ડેમેજ}: આબોહવા-સંવેદનશીલ દેશો માટે સહાય
\item
  \textbf{નેટ-ઝીરો કમિટમેન્ટ્સ}: દેશો કાર્બન ન્યુટ્રાલિટીની પ્રતિજ્ઞા લે છે
\end{itemize}

\end{solutionbox}
\begin{mnemonicbox}
``UKPOM-CDOG-TFMC''
(UNFCCC-Kyoto-Paris-Other-Montreal, Copenhagen-Doha-Other-Goals,
Technology-Finance-Monitoring-Commitments)

\end{mnemonicbox}
\subsection*{પ્રશ્ન 5(a) [3
ગુણ]}\label{q5a}

\textbf{ઓઝોન સ્તરની ક્ષતિની અસરો સમજાવો.}

\begin{solutionbox}


\vspace{-5pt}
\captionof{table}{ઓઝોન ઘટાડાની અસરો}
\vspace{-10pt}
\begin{longtable}[]{@{}lll@{}}
\toprule\noalign{}
અસરનું ક્ષેત્ર & અસર & પરિણામ \\
\midrule\noalign{}
\endhead
\bottomrule\noalign{}
\endlastfoot
માનવ આરોગ્ય & વધેલું UV-B કિરણોત્સર્ગ & ચામડીનો કેન્સર, મોતિયાંબિંદુ \\
પર્યાવરણ & ઇકોસિસ્ટમ વિક્ષેપ & દરિયાઈ ખાદ્ય શૃંખલાને નુકસાન \\
કૃષિ & પાકને નુકસાન & ખાદ્ય ઉત્પાદનમાં ઘટાડો \\
\end{longtable}

**ઓઝોન સ્તર ઘટાડાના પરિણામે પૃથ્વીની સપાટી પર વધુ અલ્ટ્રાવાયોલેટ-B (UV-B)
કિરણોત્સર્ગ પહોંચે છે.

\textbf{અસરો}:

\begin{itemize}
\tightlist
\item
  \textbf{માનવ આરોગ્ય}: ચામડીના કેન્સરનો દર વધારે, આંખને નુકસાન, રોગપ્રતિકારક
  તંત્રનું દમન
\item
  \textbf{દરિયાઈ ઇકોસિસ્ટમ્સ}: ફાયટોપ્લાન્કટનમાં ઘટાડો સમુદ્રી ખાદ્ય શૃંખલાને અસર
  કરે છે
\item
  \textbf{કૃષિ અસર}: પાકની ઉપજમાં ઘટાડો, છોડની વૃદ્ધિમાં અવરોધ
\end{itemize}

\textbf{કારણ}: ક્લોરોફ્લોરોકાર્બન્સ (CFCs) સ્ટ્રેટોસ્ફિયરમાં ઓઝોન અણુઓનો નાશ કરે છે.

\end{solutionbox}
\begin{mnemonicbox}
``HMA'' (Human-Marine-Agricultural)

\end{mnemonicbox}
\subsection*{પ્રશ્ન 5(b) [4
ગુણ]}\label{q5b}

\textbf{ગ્રીનહાઉસ વાયુઓ પર ટૂંકી નોંધ લખો.}

\begin{solutionbox}


\vspace{-5pt}
\captionof{table}{મુખ્ય ગ્રીનહાઉસ ગેસેસ}
\vspace{-10pt}
\begin{longtable}[]{@{}lll@{}}
\toprule\noalign{}
ગેસ & સ્રોતો & ગ્લોબલ વોર્મિંગ પોટેન્શિયલ \\
\midrule\noalign{}
\endhead
\bottomrule\noalign{}
\endlastfoot
CO2 & અશ્મિભૂત ઇંધણ, વનનાશ & 1 (સંદર્ભ) \\
CH4 & કૃષિ, લેન્ડફિલ & CO2 કરતાં 25 ગણું \\
N2O & ફર્ટિલાઇઝર, દહન & CO2 કરતાં 298 ગણું \\
F-ગેસેસ & ઔદ્યોગિક પ્રક્રિયાઓ & CO2 કરતાં 1,000-20,000 ગણું \\
\end{longtable}

\textbf{ગ્રીનહાઉસ ગેસેસ} એ વાતાવરણીય સંયોજનો છે જે પૃથ્વીની સપાટીથી વિકરાળેલી
ગરમીને પકડી રાખે છે.

\textbf{મુખ્ય ગ્રીનહાઉસ ગેસેસ}:

\begin{itemize}
\tightlist
\item
  \textbf{કાર્બન ડાયોક્સાઇડ (CO2)}: સૌથી વધુ મુખ્ય, અશ્મિભૂત ઇંધણ દહનથી
\item
  \textbf{મિથેન (CH4)}: શક્તિશાળી પરંતુ ટૂંકી આયુષ્ય, કૃષિમાંથી
\item
  \textbf{નાઇટ્રસ ઓક્સાઇડ (N2O)}: લાંબી આયુષ્ય, ફર્ટિલાઇઝર અને ઉદ્યોગોથી
\item
  \textbf{ફ્લોરિનેટેડ ગેસેસ}: ખૂબ શક્તિશાળી, રેફ્રિજરેશન અને ઔદ્યોગિક ઉપયોગથી
\end{itemize}

\textbf{ગુણધર્મો}: ઇન્ફ્રારેડ કિરણોત્સર્ગ શોષે છે, દૃશ્ય પ્રકાશ માટે પારદર્શક, વિવિધ
વાતાવરણીય આયુષ્ય.

\textbf{ગ્લોબલ વોર્મિંગ પોટેન્શિયલ}: ચોક્કસ સમયગાળા દરમિયાન CO2 ની તુલનામાં ગરમી
પકડવાની ક્ષમતા માપે છે.

\end{solutionbox}
\begin{mnemonicbox}
``CMNF'' (Carbon dioxide-Methane-Nitrous
oxide-Fluorinated gases)

\end{mnemonicbox}
\subsection*{પ્રશ્ન 5(c) [7
ગુણ]}\label{q5c}

\textbf{5R નો ખ્યાલ સમજાવો.}

\begin{solutionbox}

\begin{center}
\textbf{Mermaid Diagram (Code)}
\begin{verbatim}
{Shaded}
{Highlighting}[]
graph TD
    A[5R ખ્યાલ] {-{-}{} B[Refuse {-} ઇનકાર]}
    A {-{-}{} C[Reduce {-} ઘટાડો]}
    A {-{-}{} D[Reuse {-} પુનઃઉપયોગ]}
    A {-{-}{} E[Repurpose {-} નવો હેતુ]}
    A {-{-}{} F[Recycle {-} પુનર્ચક્રણ]}
    B {-{-}{} G[બિનજરૂરી વસ્તુઓ ટાળો]}
    C {-{-}{} H[વપરાશ ઓછો કરો]}
    D {-{-}{} I[વસ્તુઓનો વારંવાર ઉપયોગ]}
    E {-{-}{} J[નવા ઉપયોગ શોધો]}
    F {-{-}{} K[નવા ઉત્પાદનોમાં પ્રક્રિયા]}
{Highlighting}
{Shaded}
\end{verbatim}
\end{center}

\textbf{5R ખ્યાલ} એ કચરા વ્યવસ્થાપનની હાયરાર્કી છે જે કચરા નિવારણ અને સંસાધન
સંરક્ષણને પ્રાથમિકતા આપે છે.

\textbf{પ્રાથમિકતાના ક્રમમાં પાંચ R's}:

\textbf{1. Refuse - ઇનકાર}:

\begin{itemize}
\tightlist
\item
  \textbf{વ્યાખ્યા}: બિનજરૂરી વસ્તુઓ સ્વીકારવાનો ઇનકાર
\item
  \textbf{ઉદાહરણો}: સિંગલ-યુઝ પ્લાસ્ટિક, પ્રમોશનલ ફ્રીબીઝ, વધુ પેકેજિંગ
\item
  \textbf{અસર}: સ્રોતે કચરાનું ઉત્પાદન અટકાવે છે
\end{itemize}

\textbf{2. Reduce - ઘટાડો}:

\begin{itemize}
\tightlist
\item
  \textbf{વ્યાખ્યા}: વપરાશ અને કચરા ઉત્પાદન ઓછું કરવું
\item
  \textbf{ઉદાહરણો}: ફક્ત જરૂરી વસ્તુઓ ખરીદવી, ટકાઉ ઉત્પાદનો પસંદ કરવા, ઊર્જા
  બચત
\item
  \textbf{અસર}: સંસાધન નિષ્કર્ષણ અને કચરાના પ્રમાણમાં ઘટાડો
\end{itemize}

\textbf{3. Reuse - પુનઃઉપયોગ}:

\begin{itemize}
\tightlist
\item
  \textbf{વ્યાખ્યા}: વસ્તુઓનો તેમના મૂળ સ્વરૂપમાં વારંવાર ઉપયોગ
\item
  \textbf{ઉદાહરણો}: સ્ટોરેજ માટે કાચની બરણીઓ, કપડાંનું દાન, ફર્નિચરનો પુનઃઉપયોગ
\item
  \textbf{અસર}: ઉત્પાદનની આયુષ્ય વધારે છે, બદલીની જરૂરિયાત ઘટાડે છે
\end{itemize}

\textbf{4. Repurpose - નવો હેતુ}:

\begin{itemize}
\tightlist
\item
  \textbf{વ્યાખ્યા}: ફેંકવાને બદલે વસ્તુઓ માટે નવા ઉપયોગો શોધવા
\item
  \textbf{ઉદાહરણો}: ટાયર પ્લાન્ટર, બોટલ વેઝ, કાર્ડબોર્ડ ઓર્ગેનાઇઝર
\item
  \textbf{અસર}: સર્જનાત્મક કચરા વાળવું, કલાત્મક મૂલ્ય ઉમેરો
\end{itemize}

\textbf{5. Recycle - પુનર્ચક્રણ}:

\begin{itemize}
\tightlist
\item
  \textbf{વ્યાખ્યા}: કચરા સામગ્રીને નવા ઉત્પાદનોમાં પ્રક્રિયા કરવી
\item
  \textbf{ઉદાહરણો}: કાગળનું રિસાયકલિંગ, ધાતુ પુનઃપ્રાપ્તિ, પ્લાસ્ટિક રિપ્રોસેસિંગ
\item
  \textbf{અસર}: સંસાધન પુનઃપ્રાપ્તિ, લેન્ડફિલ ભાર ઘટાડવો
\end{itemize}

\textbf{5R અભિગમના ફાયદાઓ}:

\begin{itemize}
\tightlist
\item
  \textbf{પર્યાવરણીય}: ઘટેલું પ્રદૂષણ, સંસાધન સંરક્ષણ, ઇકોસિસ્ટમ સંરક્ષણ
\item
  \textbf{આર્થિક}: ખર્ચ બચત, રિસાયકલિંગ ઉદ્યોગમાં રોજગાર સર્જન
\item
  \textbf{સામાજિક}: સમુદાયિક જાગરૂકતા, ટકાઉ જીવનશૈલી પ્રોત્સાહન
\end{itemize}

\textbf{અમલીકરણ હાયરાર્કી}: પહેલા ઇનકાર અને ઘટાડા પર ધ્યાન આપો (નિવારણ), પછી
પુનઃઉપયોગ અને નવો હેતુ (કચરો વાળવું), અંતે રિસાયકલ (કચરા પ્રક્રિયા).

\textbf{પડકારો}: વર્તન પરિવર્તનની જરૂરિયાતો, ઇન્ફ્રાસ્ટ્રક્ચર વિકાસ, આર્થિક
પ્રોત્સાહનોનું સંકલન.

\end{solutionbox}
\begin{mnemonicbox}
``Real Recycling Requires Refusing Rubbish''
(Refuse-Reduce-Reuse-Repurpose-Recycle)

\end{mnemonicbox}
\subsection*{પ્રશ્ન 5(a) અથવા [3
ગુણ]}\label{q5a}

\textbf{વન્યજીવ સંરક્ષણ કાયદો, 1972 ની નોંધપાત્ર વિશેષતાઓ લખો.}

\begin{solutionbox}


\vspace{-5pt}
\captionof{table}{વન્યજીવ સંરક્ષણ કાયદો 1972 ની વિશેષતાઓ}
\vspace{-10pt}
\begin{longtable}[]{@{}lll@{}}
\toprule\noalign{}
વિશેષતા & વર્ણન & દંડ \\
\midrule\noalign{}
\endhead
\bottomrule\noalign{}
\endlastfoot
સંરક્ષિત પ્રજાતિઓ & અનુસૂચિત પ્રાણીઓ/છોડ & દંડ + કેદ \\
શિકાર પ્રતિબંધ & શિકાર પર પ્રતિબંધ & 7 વર્ષ સુધી જેલ \\
વેપાર નિયંત્રણ & વન્યજીવ ઉત્પાદન વેપાર નિયંત્રણ & જપ્તી + દંડ \\
\end{longtable}

\textbf{વન્યજીવ સંરક્ષણ કાયદો, 1972} ભારતમાં વન્યજીવ સંરક્ષણ માટે કાનૂની માળખું
પ્રદાન કરે છે.

\textbf{નોંધપાત્ર વિશેષતાઓ}:

\begin{itemize}
\tightlist
\item
  \textbf{પ્રજાતિ સંરક્ષણ}: સંરક્ષણ સ્તર પ્રમાણે પ્રજાતિઓનું છ અનુસૂચીમાં વર્ગીકરણ
\item
  \textbf{શિકાર પ્રતિબંધ}: સંરક્ષિત પ્રજાતિઓના શિકાર પર સંપૂર્ણ પ્રતિબંધ
\item
  \textbf{આવાસ સંરક્ષણ}: સંરક્ષિત વિસ્તારોનું હોદ્દો અને વ્યવસ્થાપન
\item
  \textbf{વેપાર નિયંત્રણ}: વન્યજીવ ઉત્પાદન વાણિજ્યનું નિયંત્રણ
\end{itemize}

\textbf{અમલીકરણ}: વન્યજીવ અપરાધ નિયંત્રણ બ્યુરો, વન વિભાગો, વન્યજીવ અપરાધો માટે
વિશેષ અદાલતો.

\textbf{સુધારાઓ}: નવી પ્રજાતિઓ સામેલ કરવા અને જોગવાઈઓ મજબૂત બનાવવા માટે નિયમિત
અપડેટ્સ.

\end{solutionbox}
\begin{mnemonicbox}
``SHTE'' (Species-Hunting-Trade-Enforcement)

\end{mnemonicbox}
\subsection*{પ્રશ્ન 5(b) અથવા [4
ગુણ]}\label{q5b}

\textbf{ભારતમાં પર્યાવરણ નીતિઓ કઇ કઇ છે?}

\begin{solutionbox}


\vspace{-5pt}
\captionof{table}{ભારતની મુખ્ય પર્યાવરણ નીતિઓ}
\vspace{-10pt}
\begin{longtable}[]{@{}lll@{}}
\toprule\noalign{}
નીતિ & વર્ષ & ફોકસ એરિયા \\
\midrule\noalign{}
\endhead
\bottomrule\noalign{}
\endlastfoot
રાષ્ટ્રીય પર્યાવરણ નીતિ & 2006 & વ્યાપક માળખું \\
રાષ્ટ્રીય જળ નીતિ & 2012 & જળ સંસાધન વ્યવસ્થાપન \\
રાષ્ટ્રીય વન નીતિ & 1988 & વન સંરક્ષણ \\
આબોહવા પરિવર્તન પર રાષ્ટ્રીય કાર્ય યોજના & 2008 & આબોહવા પરિવર્તન શમન \\
\end{longtable}

\textbf{મુખ્ય પર્યાવરણ નીતિઓ}:

\textbf{રાષ્ટ્રીય પર્યાવરણ નીતિ (2006)}:

\begin{itemize}
\tightlist
\item
  \textbf{ઉદ્દેશ્ય}: પર્યાવરણ સંરક્ષણ સાથે ટકાઉ વિકાસ
\item
  \textbf{સિદ્ધાંતો}: પ્રદૂષક ચુકવે, સાવચેતીનો અભિગમ
\item
  \textbf{અમલીકરણ}: વિભાગો વચ્ચે એકીકરણ
\end{itemize}

\textbf{ક્ષેત્રીય નીતિઓ}:

\begin{itemize}
\tightlist
\item
  \textbf{રાષ્ટ્રીય જળ નીતિ}: એકીકૃત જળ સંસાધન વ્યવસ્થાપન
\item
  \textbf{રાષ્ટ્રીય વન નીતિ}: 33\% વન આવરણનું લક્ષ્ય
\item
  \textbf{રાષ્ટ્રીય સોલર મિશન}: નવીકરણીય ઊર્જા પ્રોત્સાહન
\item
  \textbf{કચરા વ્યવસ્થાપન નિયમો}: ઘન કચરો, ઈ-કચરો, પ્લાસ્ટિક કચરા વ્યવસ્થાપન
\end{itemize}

\textbf{નિયમનકારી માળખું}: પર્યાવરણ સંરક્ષણ કાયદો, જળ અધિનિયમ, વાયુ અધિનિયમ,
વન સંરક્ષણ અધિનિયમ.

\end{solutionbox}
\begin{mnemonicbox}
``NWFS'' (National-Water-Forest-Solar)

\end{mnemonicbox}
\subsection*{પ્રશ્ન 5(c) અથવા [7
ગુણ]}\label{q5c}

\textbf{વરસાદી પાણીનો સંચય વિગતે સમજાવો.}

\begin{solutionbox}

\begin{center}
\textbf{Mermaid Diagram (Code)}
\begin{verbatim}
{Shaded}
{Highlighting}[]
graph LR
    A[વરસાદ] {-{-}{} B[કેચમેન્ટ એરિયા]}
    B {-{-}{} C[કલેક્શન સિસ્ટમ]}
    C {-{-}{} D[ફર્સ્ટ ફ્લશ ડાયવર્ટર]}
    D {-{-}{} E[ફિલ્ટરેશન]}
    E {-{-}{} F[સ્ટોરેજ ટાંકી]}
    F {-{-}{} G[વિતરણ]}
    H[રીચાર્જ પિટ] {-{-}{} I[ભૂગર્ભ જળ]}
    C {-{-}{} H}
{Highlighting}
{Shaded}
\end{verbatim}
\end{center}

\textbf{વરસાદી પાણીનો સંચય} એ ફાયદાકારક હેતુઓ માટે વરસાદી પાણીનું સંગ્રહ, સંચય અને
ઉપયોગ છે.

\textbf{વરસાદી પાણી સંચય સિસ્ટમના ઘટકો}:

\textbf{1. કેચમેન્ટ એરિયા}:

\begin{itemize}
\tightlist
\item
  \textbf{કાર્ય}: વરસાદ સંગ્રહ માટેની સપાટી (છત, ખુલ્લા વિસ્તારો)
\item
  \textbf{સામગ્રી}: સ્વચ્છ, બિન-ઝેરી હોવી જોઈએ (એસ્બેસ્ટોસ, લેડ પેઇન્ટેડ સપાટીઓ
  ટાળો)
\item
  \textbf{ગણતરી}: સંગ્રહ = કેચમેન્ટ એરિયા \times વરસાદ \times રનઓફ કોએફિશિયન્ટ
\end{itemize}

\textbf{2. સંગ્રહ અને પરિવહન સિસ્ટમ}:

\begin{itemize}
\tightlist
\item
  \textbf{ગટર}: કેચમેન્ટ સપાટીથી પાણીને ચેનલ કરે છે
\item
  \textbf{ડાઉનસ્પાઉટ્સ}: ગટર્સથી પાણી લઈ જતા વર્ટિકલ પાઇપ્સ
\item
  \textbf{પરિવહન}: વિવિધ ઘટકોને જોડતા પાઇપ્સ
\end{itemize}

\textbf{3. ફર્સ્ટ ફ્લશ ડાયવર્ટર}:

\begin{itemize}
\tightlist
\item
  \textbf{હેતુ}: કાટમાળ સાથેનું પ્રારંભિક ગંદું પાણી દૂર કરે છે
\item
  \textbf{પ્રકારો}: મેન્યુઅલ વાલ્વ, ઑટોમેટિક ડાયવર્ટર, ફ્લોટિંગ બોલ સિસ્ટમ
\item
  \textbf{ક્ષમતા}: સામાન્ય રીતે 100 ચો.મી. છતના વિસ્તાર દીઠ 10-15 લિટર
\end{itemize}

\textbf{4. ફિલ્ટરેશન સિસ્ટમ}:

\begin{itemize}
\tightlist
\item
  \textbf{કોર્સ ફિલ્ટર}: પાંદડા, કાટમાળ દૂર કરે છે (મેશ સ્ક્રીન)
\item
  \textbf{ફાઇન ફિલ્ટર}: રેતી, કાંકરી, એક્ટિવેટેડ કાર્બન
\item
  \textbf{સ્લો સેન્ડ ફિલ્ટર}: પીવાના પાણી માટે જૈવિક ટ્રીટમેન્ટ
\end{itemize}

\textbf{5. સ્ટોરેજ સિસ્ટમ}:

\begin{itemize}
\tightlist
\item
  \textbf{સરફેસ સ્ટોરેજ}: જમીન ઉપર ટાંકીઓ, જળાશયો
\item
  \textbf{અન્ડરગ્રાઉન્ડ સ્ટોરેજ}: જમીન નીચે સમ્પ્સ, સિસ્ટર્ન્સ
\item
  \textbf{સામગ્રી}: ફેરોસિમેન્ટ, પ્લાસ્ટિક, કોંક્રિટ, ફાઇબરગ્લાસ
\end{itemize}

\textbf{વરસાદી પાણી સંચયના પ્રકારો}:

\textbf{A. છતની સંચય}:

\begin{itemize}
\tightlist
\item
  \textbf{ડાયરેક્ટ સ્ટોરેજ}: તાત્કાલિક ઉપયોગ માટે ટાંકીમાં વરસાદી પાણી સંગ્રહ
\item
  \textbf{ઇન્ડાયરેક્ટ રીચાર્જ}: ભૂગર્ભ જળ રીચાર્જ કરવા માટે પાણીને દિશા આપવી
\end{itemize}

\textbf{B. સરફેસ વોટર હાર્વેસ્ટિંગ}:

\begin{itemize}
\tightlist
\item
  \textbf{ચેક ડેમ્સ}: સ્ટ્રીમ્સ વચ્ચે નાના અવરોધો
\item
  \textbf{પર્કોલેશન ટાંકીઓ}: કૃત્રિમ રીચાર્જ સ્ટ્રક્ચર્સ
\item
  \textbf{કન્ટૂર બંડિંગ}: જળ સંચય સાથે માટી સંરક્ષણ
\end{itemize}

\textbf{ફાયદાઓ}:

\begin{itemize}
\tightlist
\item
  \textbf{જળ સુરક્ષા}: બાહ્ય જળ સ્રોતો પર નિર્ભરતા ઘટાડે છે
\item
  \textbf{ભૂગર્ભ જળ રીચાર્જ}: પાણીના સ્તરમાં ઘટાડો અટકાવે છે
\item
  \textbf{પૂર નિયંત્રણ}: સપાટીનો અપવાહ અને શહેરી પૂર ઘટાડે છે
\item
  \textbf{ગુણવત્તા સુધારણા}: પ્રદૂષિત વિસ્તારોમાં સામાન્ય રીતે ભૂગર્ભ જળ કરતાં વધુ
  સારું
\item
  \textbf{ખર્ચ-અસરકારક}: જળ પુરવઠા યોજનાઓ કરતાં ઓછું
\item
  \textbf{ઊર્જા બચત}: પમ્પિંગ જરૂરિયાતો ઘટાડે છે
\end{itemize}

\textbf{ડિઝાઇન વિચારણાઓ}:

\begin{itemize}
\tightlist
\item
  \textbf{વરસાદી પેટર્ન}: મોસમી વિતરણ, તીવ્રતા
\item
  \textbf{પાણીની માંગ}: ઘરેલું જરૂરિયાતો, ઉપયોગ પેટર્ન
\item
  \textbf{સ્ટોરેજ ક્ષમતા}: સૂકા સમયગાળાના આધારે
\item
  \textbf{ગુણવત્તા જરૂરિયાતો}: પીવાના બનામ બિન-પીવાના ઉપયોગ
\item
  \textbf{સાઇટ કંડિશન્સ}: જગ્યાની ઉપલબ્ધતા, માટીની પારગમ્યતા
\end{itemize}

\textbf{જાળવણી જરૂરિયાતો}:

\begin{itemize}
\tightlist
\item
  \textbf{નિયમિત સફાઈ}: ગટર, ફિલ્ટર, સ્ટોરેજ ટાંકીઓ
\item
  \textbf{છતની જાળવણી}: દૂષણ સ્રોતો અટકાવવા
\item
  \textbf{સિસ્ટમ નિરીક્ષણ}: લીકેજ, અવરોધો તપાસવા
\item
  \textbf{પાણીની ગુણવત્તા પરીક્ષણ}: પીવાના ઉપયોગ માટે સમયાંતરે વિશ્લેષણ
\end{itemize}

\textbf{સરકારી પહેલો}:

\begin{itemize}
\tightlist
\item
  \textbf{બિલ્ડિંગ કોડ્સ}: નવા બાંધકામોમાં વરસાદી પાણી સંચય ફરજિયાત
\item
  \textbf{સબસિડી}: ઇન્સ્ટોલેશન માટે નાણાકીય પ્રોત્સાહનો
\item
  \textbf{જાગૃતિ કાર્યક્રમો}: સમુદાયિક શિક્ષણ અને તાલીમ
\item
  \textbf{તકનીકી સહાય}: ડિઝાઇન ગાઇડલાઇન્સ, અમલીકરણ સહાય
\end{itemize}

\textbf{પડકારો}:

\begin{itemize}
\tightlist
\item
  \textbf{પ્રારંભિક ખર્ચ}: સંપૂર્ણ સિસ્ટમ માટે સેટઅપ ખર્ચ
\item
  \textbf{જાળવણી}: નિયમિત જાળવણીની જરૂરિયાતો
\item
  \textbf{જગ્યાની જરૂરિયાતો}: સ્ટોરેજ ટાંકી માટે જગ્યાની જરૂર
\item
  \textbf{મોસમી ઉપલબ્ધતા}: મોનસૂન પેટર્ન પર નિર્ભરતા
\item
  \textbf{ગુણવત્તાની ચિંતાઓ}: સંભવિત દૂષણ મુદ્દાઓ
\end{itemize}

\textbf{ગણતરીનું ઉદાહરણ}:

\begin{itemize}
\tightlist
\item
  છતનો વિસ્તાર: 100 ચો.મી.
\item
  વાર્ષિક વરસાદ: 1000 મી.મી.
\item
  રનઓફ કોએફિશિયન્ટ: 0.8
\item
  સંચયપાત્ર પાણી = 100 \times 1 \times 0.8 = 80,000 લિટર/વર્ષ
\end{itemize}

\end{solutionbox}
\begin{mnemonicbox}
``CCFFS-RSBD-WGFQC-RCSMQ''
(Catchment-Collection-Flush-Filter-Storage,
Rooftop-Surface-Benefits-Design, Water-Groundwater-Flood-Quality-Cost,
Regular-Check-System-Maintenance-Quality)

\end{mnemonicbox}

\end{document}
