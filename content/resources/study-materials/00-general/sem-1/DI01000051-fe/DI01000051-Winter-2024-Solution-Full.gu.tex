%% METADATA
%% subject-code: DI01000051
%% subject-name: Fundamentals of Electronics
%% semester: 1
%% examination: Winter-2024
%% date: 07-01-2025
%% description: Solution guide for Fundamentals of Electronics
%% tags: study-material, solutions, gtu, DI01000051
%% END METADATA

\documentclass{article}
% GTU Solutions - Gujarati Preamble
% Includes common preamble + Gujarati font setup

% Basic setup
\usepackage[margin=1in]{geometry}
\author{Milav Dabgar}

% Math and tables
\usepackage{amsmath,amssymb,amsthm}
\usepackage{booktabs}
\usepackage{tabularx}
\usepackage{graphicx}
\usepackage{float}  % Required for [H] float placement

% Code listings with syntax highlighting
\usepackage{xcolor}
\usepackage{listings}
\lstset{
  basicstyle=\small\ttfamily,
  breaklines=true,
  numbers=left,
  numberstyle=\tiny\color{gray},
  xleftmargin=2em,
  frame=single,
  showstringspaces=false,
  tabsize=2,
  keywordstyle=\color{blue},
  commentstyle=\color{green!60!black},
  stringstyle=\color{purple}
}

% Optional: TikZ for diagrams (remove if not needed)
\usepackage{tikz}
\usepackage{circuitikz}
\usetikzlibrary{shapes,arrows,positioning,calc}

% Header/footer with author and website
\usepackage{fancyhdr}
\usepackage{lastpage}

\pagestyle{fancy}
\fancyhf{}
\fancyhead[L]{\small\itshape\leftmark}
\fancyhead[R]{\small Milav Dabgar}
\fancyfoot[L]{\small\href{https://www.milav.in}{www.milav.in}}
\fancyfoot[R]{\small Page \thepage\ of \pageref{LastPage}}
\renewcommand{\headrulewidth}{0.4pt}
\renewcommand{\footrulewidth}{0.4pt}

% Hyperref (load before fontspec for Gujarati)
\usepackage[
  colorlinks=true,
  linkcolor=blue,
  urlcolor=blue,
  citecolor=blue,
  pdfauthor={Milav Dabgar},
  pdfsubject={GTU Exam Solutions},
  pdfkeywords={GTU, Java, Programming, Solutions, Gujarati},
  bookmarks=true
]{hyperref}

% Gujarati font setup
\usepackage{fontspec}
\usepackage{polyglossia}
\setdefaultlanguage{gujarati}
\setotherlanguage{english}
\newfontfamily\gujaratifont[Script=Gujarati,AutoFakeBold=2.5,AutoFakeSlant=0.3]{Noto Sans Gujarati}
\setmainfont[Script=Gujarati,AutoFakeBold=2.5,AutoFakeSlant=0.3]{Noto Sans Gujarati}
\setmonofont[Scale=0.9]{Noto Sans Gujarati}
\newfontfamily\englishfont[Script=Gujarati,AutoFakeBold=2.5,AutoFakeSlant=0.3]{Noto Sans Gujarati}
\gappto\captionsgujarati{
  \renewcommand{\tablename}{કોષ્ટક}
  \renewcommand{\figurename}{આકૃતિ}
}
\newcommand{\gu}[1]{{\gujaratifont #1}}


\title{Fundamentals of Electronics (DI01000051) - Winter 2024 Solution}
\date{January 07, 2025}

\hypersetup{
  pdftitle={Fundamentals of Electronics (DI01000051) - Winter 2024 Solution},
  pdfsubject={GTU Exam Solution - Winter-2024},
  pdfauthor={Milav Dabgar},
  pdfkeywords={study-material, solutions, gtu, DI01000051},
  pdfcreator={xelatex}
}

\begin{document}
\maketitle

\setcounter{tocdepth}{5}
\tableofcontents
\newpage

% ========================================
% QUESTION 1(a): Active and Passive Components (3 marks)
% Demonstrates: Definitions, Examples, Comparison
% ========================================

\section{Question 1}
\subsection{Question 1(a) [3 marks]}
\textbf{ઉદાહરણ સાથે એક્ટિવ અને પેસીવ કમ્પોનન્ટને વ્યાખ્યાયિત કરો.}

\subsubsection{Solution}
ઈલેક્ટ્રોનિક કમ્પોનન્ટ્સને મુખ્યત્વે બે શ્રેણીઓમાં વહેંચવામાં આવે છે: એક્ટિવ (Active) અને પેસીવ (Passive) કમ્પોનન્ટ્સ, જે તેમની એનર્જી જનરેટ અથવા કંટ્રોલ કરવાની ક્ષમતા પર આધારિત છે.

\paragraph{Active Components:}
આ એવા કમ્પોનન્ટ્સ છે જે એનર્જી (પાવર) જનરેટ કરી શકે છે અથવા કરંટના પ્રવાહને નિયંત્રિત કરી શકે છે. તેમને કાર્ય કરવા માટે બાહ્ય પાવર સોર્સની જરૂર પડે છે. તેઓ સિગ્નલને એમ્પ્લીફાય (amplify) કરવા માટે સક્ષમ છે.
\begin{itemize}
    \item \textbf{ઉદાહરણો:} Transistors (BJT, FET), Diodes (Zener, Tunnel), Integrated Circuits (ICs), Batteries, Generators.
\end{itemize}

\paragraph{Passive Components:}
આ એવા કમ્પોનન્ટ્સ છે જે માત્ર એનર્જી મેળવે છે, અને તેને કાં તો ડિસીપેટ (ગરમી તરીકે વ્યય), એબ્સોર્બ, અથવા ઇલેક્ટ્રિક/મેગ્નેટિક ફિલ્ડમાં સંગ્રહ કરે છે. તેઓ પાવર જનરેટ કરી શકતા નથી કે સિગ્નલને એમ્પ્લીફાય કરી શકતા નથી. તેમને કાર્ય કરવા માટે અલગ બાહ્ય પાવર સોર્સની જરૂર પડતી નથી.
\begin{itemize}
    \item \textbf{ઉદાહરણો:} Resistors (એનર્જી ડિસીપેટ કરે છે), Capacitors (ઇલેક્ટ્રિક એનર્જી સંગ્રહ કરે છે), Inductors (મેગ્નેટિક એનર્જી સંગ્રહ કરે છે), Transformers.
\end{itemize}

\paragraph{મેમરી ટ્રીક:}
\emph{Active Adds Action (Gain/Control); Passive Plays Part (Stores/Dissipates).}

% ========================================
% QUESTION 1(b): LDR Construction and Working (4 marks)
% Demonstrates: Construction Diagram, Working Principle
% ========================================

\subsection{Question 1(b) [4 marks]}
\textbf{LDR નુ બંધારણ અને કાર્ય સમજાવો.}

\subsubsection{Solution}
Light Dependent Resistor (LDR), જેને ફોટોરેઝિસ્ટર પણ કહેવાય છે, તે એક એવો કમ્પોનન્ટ છે જેનો રેઝિસ્ટન્સ પ્રકાશની તીવ્રતા વધવાની સાથે ઘટે છે.

\paragraph{બંધારણ (Construction):}
તેમાં ઇન્સ્યુલેટીંગ સિરામિક સબસ્ટ્રેટ પર જમા થયેલ લાઈટ-સેન્સિટિવ મટિરિયલ, સામાન્ય રીતે કેડમિયમ સલ્ફાઇડ (CdS), હોય છે.
\begin{enumerate}
    \item સેમિકન્ડક્ટર મટિરિયલને ઝિગ-ઝેગ ટ્રેકમાં ગોઠવવામાં આવે છે જેથી ખુલ્લી સપાટીનો વિસ્તાર મહત્તમ કરી શકાય અને ડિવાઇસને કોમ્પેક્ટ રાખી શકાય.
    \item ઝિગ-ઝેગ ટ્રેકના છેડે કનેક્શન પૂરું પાડવા માટે મેટલ ઇલેક્ટ્રોડ્સ મૂકવામાં આવે છે.
    \item ભેજથી બચાવવા માટે સમગ્ર એસેમ્બલીને પારદર્શક પ્લાસ્ટિક અથવા કાચના કેસિંગમાં બંધ કરવામાં આવે છે, જે પ્રકાશને પસાર થવા દે છે.
\end{enumerate}

\paragraph{આકૃતિ:}
\begin{figure}[H]
\centering
\begin{circuitikz}[scale=1.2]
    \draw (0,0) to[ldR, l=LDR, o-o] (3,0);
\end{circuitikz}
\caption{LDR નો સિમ્બોલ}
\end{figure}

\paragraph{કાર્યપદ્ધતિ (Working):}
LDR \textbf{Photoconductivity} ના સિદ્ધાંત પર કાર્ય કરે છે.
\begin{enumerate}
    \item \textbf{ડાર્ક કન્ડીશન:} પ્રકાશની ગેરહાજરીમાં, મટિરિયલમાં ખૂબ ઓછા મુક્ત ઇલેક્ટ્રોન હોય છે. તે લગભગ ઇન્સ્યુલેટર જેવું વર્તન કરે છે અને ખૂબ ઊંચો રેઝિસ્ટન્સ (મેગા ઓહ્મ રેન્જ) આપે છે.
    \item \textbf{લાઈટ કન્ડીશન:} જ્યારે પ્રકાશના ફોટોન CdS મટિરિયલ પર પડે છે, ત્યારે તેઓ વેલેન્સ ઇલેક્ટ્રોનને એનર્જી આપે છે.
    \item જો ફોટોન એનર્જી પૂરતી હોય, તો કોવેલેન્ટ બોન્ડ તૂટી જાય છે, અને ઇલેક્ટ્રોન-હોલ પેર જનરેટ થાય છે.
    \item આ મુક્ત ચાર્જ કેરિયર્સ મટિરિયલની વાહકતા (conductivity) વધારે છે, જેનાથી તેનો રેઝિસ્ટન્સ ખૂબ જ ઓછો થઈ જાય છે (થોડા સો ઓહ્મ સુધી).
\end{enumerate}

\paragraph{મેમરી ટ્રીક:}
\emph{LDR: Light Drops Resistance (વધુ પ્રકાશ = ઓછો અવરોધ).}

% ========================================
% QUESTION 1(c): Capacitance & Electrolytic Capacitor (7 marks)
% Demonstrates: Definition, Diagrams, Detailed Working
% ========================================

\subsection{Question 1(c) [7 marks]}
\textbf{કેપેસીટન્સની વ્યાખ્યા લખો અને એલ્યુમીનીયમ ઈલેક્ટ્રોલાઈટ વેટ પ્રકારનો કેપેસીટર સમજાવો.}

\subsubsection{Solution}
\paragraph{Capacitance ની વ્યાખ્યા:}
કેપેસીટન્સ (\(C\)) એ સિસ્ટમની ઇલેક્ટ્રિક ચાર્જ સંગ્રહ કરવાની ક્ષમતા છે. તેને સિસ્ટમમાં ઇલેક્ટ્રિક ચાર્જ (\(Q\)) માં થતા ફેરફાર અને તેના ઇલેક્ટ્રિક પોટેન્શિયલ (\(V\)) માં થતા અનુરૂપ ફેરફારના ગુણોત્તર તરીકે વ્યાખ્યાયિત કરવામાં આવે છે.
\[ C = \frac{Q}{V} \]
કેપેસીટન્સનો એકમ ફેરાડ (Farad - F) છે.

\paragraph{એલ્યુમીનીયમ ઈલેક્ટ્રોલાઈટ કેપેસીટર (વેટ પ્રકાર):}
આ પ્રકારના કેપેસીટરનો ઉપયોગ ત્યારે થાય છે જ્યારે મોટા કેપેસીટન્સ મૂલ્યોની જરૂર હોય (દા.ત., પાવર સપ્લાય ફિલ્ટર્સમાં). તે પોલરાઇઝડ (polarized) કમ્પોનન્ટ છે.

\paragraph{બંધારણ (Construction):}
\begin{enumerate}
    \item \textbf{એનોડ (Anode):} શુદ્ધ એલ્યુમિનિયમ ફોઇલ પોઝિટિવ પ્લેટ (એનોડ) તરીકે કામ કરે છે. સપાટીનું ક્ષેત્રફળ વધારવા માટે તેની સપાટીને કોતરવામાં (etched) આવે છે. તેના પર એલ્યુમિનિયમ ઓક્સાઇડ (\(Al_2O_3\)) નું પાતળું સ્તર રચાય છે, જે \textbf{ડાઇઇલેક્ટ્રિક (Dielectric)} તરીકે કામ કરે છે.
    \item \textbf{કેથોડ (Cathode):} બીજી એલ્યુમિનિયમ ફોઇલ નેગેટિવ કનેક્શન તરીકે કામ કરે છે.
    \item \textbf{ઈલેક્ટ્રોલાઈટ (Electrolyte):} પ્લેટોની વચ્ચે, પ્રવાહી ઈલેક્ટ્રોલાઈટ (જેમ કે બોરેક્સ અથવા ગ્લાયકોલ) માં પલાળેલું કાગળનું સેપરેટર મૂકવામાં આવે છે. ઈલેક્ટ્રોલાઈટ વાસ્તવિક કેથોડ તરીકે કામ કરે છે અને ઓક્સાઇડ સ્તર સાથે સંપર્ક બનાવે છે.
    \item આ એસેમ્બલીને નળાકાર આકારમાં વાળવામાં આવે છે અને એલ્યુમિનિયમ કેનમાં બંધ કરવામાં આવે છે.
\end{enumerate}

\paragraph{આકૃતિ:}
\begin{figure}[H]
\centering
\begin{tikzpicture}
    % Cylinder body
    \draw[thick] (0,0) ellipse (1cm and 0.3cm);
    \draw[thick] (-1,0) -- (-1,-3);
    \draw[thick] (1,0) -- (1,-3);
    \draw[thick] (-1,-3) arc (180:360:1cm and 0.3cm);
    
    % Internal layers representation
    \draw[fill=lightgray] (-0.8,-0.5) rectangle (0.8,-2.5);
    \node at (0,-1) {Calculation Layers};
    \node at (0,-1.5) {Al Foil + Paper};
    
    % Leads
    \draw[thick] (-0.3,0) -- (-0.3,1) node[above] {Positive (+)};
    \draw[thick] (0.3,0) -- (0.3,1) node[above] {Negative (-)};
\end{tikzpicture}
\caption{ઈલેક્ટ્રોલાઈટ કેપેસીટરનું બંધારણ}
\end{figure}

\paragraph{કાર્યપદ્ધતિ (Working):}
જ્યારે યોગ્ય પોલારિટી (એનોડ પર પોઝિટિવ) સાથે DC વોલ્ટેજ લાગુ કરવામાં આવે છે, ત્યારે વિદ્યુત વિચ્છેદન (electrolysis) પ્રક્રિયા પાતળા ઓક્સાઇડ સ્તરને જાળવી રાખે છે. આ ઓક્સાઇડ સ્તર અત્યંત પાતળું હોય છે, જે સૂત્ર \(C = \frac{\epsilon A}{d}\) મુજબ ખૂબ ઊંચું કેપેસીટન્સ મૂલ્ય આપે છે (નાનું \(d\)). જો રિવર્સ પોલારિટી લાગુ કરવામાં આવે, તો ઓક્સાઇડ સ્તર તૂટી જાય છે, જેના કારણે શોર્ટ સર્કિટ થાય છે અને સંભવિત વિસ્ફોટ થઈ શકે છે.

\paragraph{મેમરી ટ્રીક:}
\emph{Wet Paper Holds Charge; Polarity Matters (Oxide Dielectric).}

% ========================================
% QUESTION 1(c) OR: Resistor Color Code (7 marks)
% Demonstrates: Table, Calculation, Example
% ========================================

\subsection{Question 1(c) OR [7 marks]}
\textbf{રેઝિસ્ટરની કલર બેન્ડ કોડિંગ પધ્ધતી સમજાવો. ૩૨ \(\Omega\) \(\pm\) ૧૦ \% કિંમત નો કલર બેન્ડ લખો.}

\subsubsection{Solution}
રેઝિસ્ટર કલર કોડિંગ એ એક આંતરરાષ્ટ્રીય ધોરણ છે જેનો ઉપયોગ નાના રેઝિસ્ટરના રેઝિસ્ટન્સ મૂલ્ય અને ટોલરન્સ (tolerance) દર્શાવવા માટે થાય છે. રેઝિસ્ટર પર નંબર પ્રિન્ટ કરવા ખૂબ નાના હોવાથી, તેમના મૂલ્યો દર્શાવવા માટે બોડી પર રંગીન પટ્ટાઓ દોરવામાં આવે છે.

\paragraph{કલર કોડ કોષ્ટક:}
ઇલેક્ટ્રોનિક ઇન્ડસ્ટ્રીઝ એલાયન્સ (EIA) કલર કોડ સિસ્ટમને વ્યાખ્યાયિત કરે છે. દરેક રંગ ચોક્કસ અંક (0-9), ગુણક ફેક્ટર અને ટોલરન્સ ટકાવારીને અનુરૂપ છે.
\begin{table}[H]
\centering
\caption{રેઝિસ્ટર કલર કોડ્સ}
\begin{tabularx}{\textwidth}{|l|X|X|X|}
\hline
\textbf{Color} & \textbf{Digit} & \textbf{Multiplier} & \textbf{Tolerance} \\
\hline
Black (કાળો) & 0 & \(10^0\) & - \\
Brown (ભૂરો) & 1 & \(10^1\) & \(\pm 1\%\) \\
Red (લાલ) & 2 & \(10^2\) & \(\pm 2\%\) \\
Orange (નારંગી) & 3 & \(10^3\) & - \\
Yellow (પીળો) & 4 & \(10^4\) & - \\
Green (લીલો) & 5 & \(10^5\) & \(\pm 0.5\%\) \\
Blue (વાદળી) & 6 & \(10^6\) & \(\pm 0.25\%\) \\
Violet (જાંબલી) & 7 & \(10^7\) & \(\pm 0.1\%\) \\
Grey (ગ્રે) & 8 & \(10^8\) & - \\
White (સફેદ) & 9 & \(10^9\) & - \\
Gold (ગોલ્ડ) & - & \(10^{-1}\) & \(\pm 5\%\) \\
Silver (સિલ્વર) & - & \(10^{-2}\) & \(\pm 10\%\) \\
\hline
\end{tabularx}
\end{table}

\paragraph{4-બેન્ડ સિસ્ટમ સમજૂતી:}
સૌથી સામાન્ય રેઝિસ્ટર પ્રકારો 4-બેન્ડ કોડનો ઉપયોગ કરે છે. ડાબેથી જમણે વાંચતા:
\begin{itemize}
    \item \textbf{1st Band (પ્રથમ સાર્થક અંક):} આ બેન્ડ રેઝિસ્ટન્સ મૂલ્યનો પ્રથમ અંક રજૂ કરે છે.
    \item \textbf{2nd Band (બીજો સાર્થક અંક):} આ બેન્ડ રેઝિસ્ટન્સ મૂલ્યનો બીજો અંક રજૂ કરે છે.
    \item \textbf{3rd Band (ગુણક - Multiplier):} આ બેન્ડ પ્રથમ બે અંકો પછી ઉમેરવાના શૂન્યોની સંખ્યા, અથવા અસરકારક રીતે 10 ની ઘાત સૂચવે છે જેના વડે સાર્થક અંકો ગુણવામાં આવે છે.
    \item \textbf{4th Band (ટોલરન્સ):} આ બેન્ડ સામાન્ય રીતે ગેપ પછી આવે છે અને રેઝિસ્ટરની ચોકસાઈ સૂચવે છે. સિલ્વર \(\pm 10\%\) અને ગોલ્ડ \(\pm 5\%\) સૂચવે છે.
\end{itemize}

\paragraph{32 \(\Omega\) \(\pm\)10\% માટે ગણતરી:}
\subparagraph{મૂલ્ય વિભાજન (Value Breakdown):}
આપણને \(10\%\) ટોલરન્સ સાથે \(32 \Omega\) નું રેઝિસ્ટન્સ મૂલ્ય આપવામાં આવ્યું છે. અમે આને પ્રમાણભૂત ફોર્મેટમાં વિભાજીત કરીએ છીએ:
\[ Value = (1st Digit)(2nd Digit) \times Multiplier \]
\[ 32 \Omega = 32 \times 1 \]
\[ 32 \Omega = 32 \times 10^0 \]

\subparagraph{રંગો સાથે મેપિંગ (Mapping to Colors):}
હવે, અમે આ ઘટકોને તેમના સંબંધિત રંગો સાથે મેપ કરીએ છીએ:
\begin{enumerate}
    \item \textbf{1st Band (અંક 3):} કોષ્ટક જોતાં, અંક 3 \textbf{Orange (નારંગી)} રંગને અનુરૂપ છે.
    \item \textbf{2nd Band (અંક 2):} અંક 2 \textbf{Red (લાલ)} રંગને અનુરૂપ છે.
    \item \textbf{3rd Band (ગુણક \(10^0\)):} ગુણક \(1\) છે (કારણ કે આપણને બરાબર 32 જોઈએ છે). કોષ્ટક દર્શાવે છે કે \(10^0\) \textbf{Black (કાળો)} રંગને અનુરૂપ છે.
    \item \textbf{4th Band (ટોલરન્સ \(\pm 10\%\)):} ટોલરન્સ \(10\%\) તરીકે ઉલ્લેખિત છે, જે \textbf{Silver (સિલ્વર)} રંગને અનુરૂપ છે.
\end{enumerate}

\textbf{અંતિમ જવાબ:} રેઝિસ્ટર બોડી પર કલર બેન્ડનો ક્રમ \textbf{Orange - Red - Black - Silver} હશે.

\paragraph{મેમરી ટ્રીક:}
\emph{B B R O Y of Great Britain had a Very Good Wife (GS). Black-0, Brown-1, Red-2\dots}

% ========================================
% QUESTION 2(a): Definitions (3 marks)
% Demonstrates: Precise definitions
% ========================================

\section{Question 2}
\subsection{Question 2(a) [3 marks]}
\textbf{નીચેના શબ્દો વ્યાખ્યાયિત કરો : 1) રેક્ટિફાયર 2) રિપલ ફેક્ટર 3) ફિલ્ટર}

\subsubsection{Solution}
\begin{enumerate}
    \item \textbf{Rectifier:} રેક્ટિફાયર એ ઇલેક્ટ્રોનિક ઉપકરણ અથવા સર્કિટ છે જે અલ્ટરનેટિંગ કરંટ (AC) ને ડાયરેક્ટ કરંટ (DC) માં રૂપાંતરિત કરે છે. તે સામાન્ય રીતે કરંટને માત્ર એક જ દિશામાં વહેવા દેવા માટે એક અથવા વધુ ડાયોડ્સનો ઉપયોગ કરે છે.
    \item \textbf{Ripple Factor:} રેક્ટિફાયરનું આઉટપુટ શુદ્ધ DC હોતું નથી પરંતુ તેમાં AC ના કેટલાક ઘટકો હોય છે જેને રિપલ્સ કહેવાય છે. આઉટપુટ વોલ્ટેજના AC ઘટકના RMS મૂલ્ય અને આઉટપુટ વોલ્ટેજના DC મૂલ્યના ગુણોત્તરને રિપલ ફેક્ટર તરીકે વ્યાખ્યાયિત કરવામાં આવે છે.
    \[ \text{Ripple Factor} (\gamma) = \frac{V_{ac(rms)}}{V_{dc}} \]
    \item \textbf{Filter:} ફિલ્ટર એ એક સર્કિટ છે જેનો ઉપયોગ રેક્ટિફાયર પછી પલ્સેટિંગ DC આઉટપુટમાંથી અનિચ્છનીય AC ઘટકો (રિપલ્સ) દૂર કરવા અને લોડને સ્થિર/સ્મૂધ DC વોલ્ટેજ પ્રદાન કરવા માટે થાય છે. તેમાં સામાન્ય રીતે કેપેસિટર અને ઇન્ડક્ટર્સ હોય છે.
\end{enumerate}

\paragraph{મેમરી ટ્રીક:}
\emph{RRF: Rectify converts, Ripple measures AC junk, Filter cleans it up.}


% ========================================
% QUESTION 2(b): Positive Clipper (4 marks)
% Demonstrates: Circuit Diagram, Waveforms, Working
% ========================================

\subsection{Question 2(b) [4 marks]}
\textbf{પોઝિટીવ ક્લિપર સર્કિટ દોરી વેવફોર્મ સાથે સમજાવો.}

\subsubsection{Solution}
પોઝિટીવ ક્લિપર એ એક સર્કિટ છે જે ઇનપુટ AC સિગ્નલની પોઝિટિવ હાફ સાયકલને દૂર કરે છે (કાપી નાખે છે).

\paragraph{સર્કિટ ડાયાગ્રામ:}
ડાયોડ લોડ સાથે સીરિઝમાં (Series) અથવા પેરેલલમાં જોડાયેલ હોઈ શકે છે. અહીં સીરિઝ પોઝિટિવ ક્લિપર દર્શાવેલ છે.
\begin{figure}[H]
\centering
\begin{circuitikz}[scale=1]
    \draw (0,0) to[sV, l=Input] (0,2);
    \draw (0,2) to[diode, l=D, -*] (3,2);
    \draw (3,2) to[R, l=\(R_L\), -*] (3,0);
    \draw (3,0) -- (0,0);
    \draw (3,2) -- (4,2) node[right] {Output};
    \draw (3,0) -- (4,0);
\end{circuitikz}
\caption{Series Positive Clipper}
\end{figure}

\paragraph{કાર્યપદ્ધતિ:}
\begin{enumerate}
    \item \textbf{Positive Half Cycle:} ઇનપુટની પોઝિટિવ હાફ સાયકલ દરમિયાન, ડાયોડ રિવર્સ બાયસ થાય છે (કેથોડ એનોડની સાપેક્ષમાં પોઝિટિવ). તે ઓપન સ્વિચ તરીકે વર્તે છે. લોડ રેઝિસ્ટર \(R_L\) માંથી કોઈ કરંટ વહેતો નથી. તેથી આઉટપુટ વોલ્ટેજ \(V_o = 0\). 
    \item \textbf{Negative Half Cycle:} નેગેટિવ હાફ સાયકલ દરમિયાન, ડાયોડ ફોરવર્ડ બાયસ થાય છે (એનોડ કેથોડની સાપેક્ષમાં પોઝિટિવ). તે ક્લોઝ સ્વિચ તરીકે વર્તે છે. લોડમાંથી કરંટ વહે છે, અને આઉટપુટ નેગેટિવ ઇનપુટ સાયકલને અનુસરે છે.
\end{enumerate}

\paragraph{વેવફોર્મ્સ:}
ઇનપુટ સંપૂર્ણ સાઈન વેવ (sine wave) બતાવે છે, જ્યારે આઉટપુટ માત્ર નેગેટિવ હાફ સાયકલ બતાવે છે.

\paragraph{મેમરી ટ્રીક:}
\emph{Positive Clipper: Diode Blocks Plus (Reverse Biased in Positive).}


% ========================================
% QUESTION 2(c): Full Wave Rectifier (Center Tap) (7 marks)
% Demonstrates: Circuit, Operation, Waveforms
% ========================================

\subsection{Question 2(c) [7 marks]}
\textbf{બે ડાયોડ - ફુલ વેવ રેક્ટિફાયરની કાર્ય પધ્ધતી સમજાવો.}

\subsubsection{Solution}
ફુલ વેવ રેક્ટિફાયર AC સાયકલના બંને અર્ધભાગ (હાફ સાયકલ) ને પલ્સેટિંગ DC માં રૂપાંતરિત કરે છે. સેન્ટર-ટેપ્ડ ફુલ વેવ રેક્ટિફાયર બે ડાયોડ અને સેન્ટર-ટેપ્ડ ટ્રાન્સફોર્મરનો ઉપયોગ કરે છે.

\paragraph{સર્કિટ ડાયાગ્રામ:}
\begin{figure}[H]
\centering
\begin{circuitikz}[scale=1]
    \draw (0,0) node[transformer core] (T) {};
    \draw (T.A1) -- ++(-1,0) to[sV, l=AC Input] ++(0,-2) -- (T.A2);
    \draw (T.B1) to[diode, l=D1] ++(2,0) coordinate (top);
    \draw (T.B2) to[diode, l=D2] ++(2,0) coordinate (bot);
    \draw (top) -- (bot) coordinate[midway] (mid);
    \draw (T.base) -- ++(1,0) coordinate (center); 
    \draw (mid) to[R, l=\(R_L\)] (center);
\end{circuitikz}
\caption{સેન્ટર ટેપ્ડ ફુલ વેવ રેક્ટિફાયર}
\end{figure}

\paragraph{બંધારણ:}
તેમાં બે સેકન્ડરી વાઇન્ડિંગ્સ સાથે સેન્ટર-ટેપ્ડ ટ્રાન્સફોર્મરનો સમાવેશ થાય છે. બે ડાયોડ \(D_1\) અને \(D_2\) સેકન્ડરી વાઇન્ડિંગના છેડા સાથે જોડાયેલા છે. લોડ રેઝિસ્ટર \(R_L\) ડાયોડ્સના સામાન્ય કેથોડ પોઈન્ટ અને સેન્ટર ટેપ વચ્ચે જોડાયેલ છે.

\paragraph{કાર્યપદ્ધતિ:}
\subparagraph{Positive Half Cycle:}
જ્યારે પોઈન્ટ A (ઉપર) સેન્ટર ટેપ C ની સાપેક્ષમાં પોઝિટિવ હોય અને પોઈન્ટ B (નીચે) નેગેટિવ હોય:
\begin{itemize}
    \item ડાયોડ \(D_1\) ફોરવર્ડ બાયસ થાય છે અને કન્ડક્ટ (conduct) કરે છે.
    \item ડાયોડ \(D_2\) રિવર્સ બાયસ થાય છે અને OFF રહે છે.
    \item કરંટ \(D_1\), લોડ \(R_L\) અને સેન્ટર ટેપ દ્વારા વહે છે.
\end{itemize}

\subparagraph{Negative Half Cycle:}
જ્યારે પોઈન્ટ A નેગેટિવ હોય અને પોઈન્ટ B પોઝિટિવ હોય:
\begin{itemize}
    \item ડાયોડ \(D_2\) ફોરવર્ડ બાયસ થાય છે અને કન્ડક્ટ કરે છે.
    \item ડાયોડ \(D_1\) રિવર્સ બાયસ થાય છે અને OFF રહે છે.
    \item કરંટ \(D_2\), લોડ \(R_L\) અને સેન્ટર ટેપ દ્વારા વહે છે.
\end{itemize}
બંને કિસ્સાઓમાં, લોડ \(R_L\) માંથી કરંટ \textbf{એક જ દિશામાં} વહે છે, જે એકદિશીય (unidirectional) આઉટપુટ આપે છે.

\paragraph{મેમરી ટ્રીક:}
\emph{Center Tap: Two Diodes take turns; One pushes, One waits.}


% ========================================
% QUESTION 2(a) OR: Rectifier Applications (3 marks)
% Demonstrates: List of uses
% ========================================

\subsection{Question 2(a) OR [3 marks]}
\textbf{રેક્ટિફાયર વ્યાખ્યાયિત કરો અને તેની એપ્લિકેશન લખો.}

\subsubsection{Solution}
\paragraph{વ્યાખ્યા:}
રેક્ટિફાયર એક એવું ઉપકરણ છે જે અલ્ટરનેટિંગ કરંટ (AC) ને ડાયરેક્ટ કરંટ (DC) માં રૂપાંતરિત કરે છે.

\paragraph{ઉપયોગો (Applications):}
રેક્ટિફાયર ઇલેક્ટ્રોનિક્સમાં મહત્વપૂર્ણ ઘટકો છે અને તેનો ઉપયોગ નીચે મુજબ થાય છે:
\begin{enumerate}
    \item \textbf{DC Power Supplies:} ઇલેક્ટ્રોનિક સર્કિટ્સ (જેમ કે ટીવી, મોબાઈલ ચાર્જર્સ, કોમ્પ્યુટર્સ) માટે DC વોલ્ટેજ પ્રદાન કરવા.
    \item \textbf{Battery Charging:} વાહનો, UPS સિસ્ટમ્સ અને પોર્ટેબલ ઉપકરણોમાં બેટરી ચાર્જ કરવા.
    \item \textbf{Demodulation:} રેડિયો રીસીવરોમાં સિગ્નલને એક્સ્ટ્રેક્ટ કરવા (AM detection).
    \item \textbf{Electroplating:} ઔદ્યોગિક રાસાયણિક પ્રક્રિયાઓ માટે સ્થિર DC કરંટ પ્રદાન કરવા.
\end{enumerate}

\paragraph{મેમરી ટ્રીક:}
\emph{Rectifier: AC to DC. Used in Power, Charge, Radio, Plating.}


% ========================================
% QUESTION 2(b) OR: Pi Filter (4 marks)
% Demonstrates: Circuit, Operation
% ========================================

\subsection{Question 2(b) OR [4 marks]}
\textbf{Pi (\(\pi\)) પ્રકારના કેપેસીટર ફીલ્ટરનુ કાર્ય સમજાવો.}

\subsubsection{Solution}
પાઈ ફિલ્ટર (જેને CLC ફિલ્ટર પણ કહેવાય છે) બે કેપેસિટર અને એક ઇન્ડક્ટરથી બનેલું હોય છે જે ગ્રીક અક્ષર Pi (\(\pi\)) ના આકારમાં ગોઠવાયેલું હોય છે.

\paragraph{સર્કિટ ડાયાગ્રામ:}
\begin{figure}[H]
\centering
\begin{circuitikz}[scale=1]
    \draw (0,1) to[short, o-] (1,1) to[C, l=C1] (1,0) to[short, -o] (0,0);
    \draw (1,1) to[L, l=L] (3,1) to[C, l=C2] (3,0) -- (1,0); 
    \draw (3,1) -- (4,1) to[R, l=Load] (4,0) -- (3,0);
\end{circuitikz}
\caption{Pi ફિલ્ટર સર્કિટ}
\end{figure}

\paragraph{કાર્યપદ્ધતિ:}
\begin{enumerate}
    \item \textbf{Capacitor \(C_1\):} આ પ્રથમ કેપેસિટર રેક્ટિફાયર આઉટપુટ સાથે સમાંતરમાં જોડાયેલું છે. તે AC રિપલ્સને ઓછો રિએક્ટન્સ આપે છે અને મોટાભાગના રિપલ્સને ગ્રાઉન્ડમાં બાયપાસ કરે છે. તે પીક વોલ્ટેજ સુધી ચાર્જ થાય છે.
    \item \textbf{Inductor \(L\):} સીરિઝ ઇન્ડક્ટર બાકી રહેલા કોઈપણ AC ઘટકોને ઉચ્ચ રિએક્ટન્સ પ્રદાન કરે છે, તેમને બ્લોક કરે છે, જ્યારે DC કરંટને લગભગ શૂન્ય અવરોધ આપે છે.
    \item \textbf{Capacitor \(C_2\):} બીજું કેપેસિટર ઇન્ડક્ટરમાંથી પસાર થયેલા કોઈપણ શેષ AC રિપલ્સને વધુ બાયપાસ કરે છે, લોડ પર ખૂબ જ સ્મૂધ DC આઉટપુટ સુનિશ્ચિત કરે છે.
\end{enumerate}

\paragraph{મેમરી ટ્રીક:}
\emph{Pi Filter: C-L-C Sandwich. Bypasses AC, Blocks AC, Smooths DC.}


% ========================================
% QUESTION 2(c) OR: Half Wave vs Full Wave Bridge (7 marks)
% Demonstrates: Comparison Table
% ========================================

\subsection{Question 2(c) OR [7 marks]}
\textbf{હાફ વેવ અને ફુલવેવ બ્રીજ રેક્ટિફાયરને સરખાવો.}

\subsubsection{Solution}
હાફ વેવ અને ફુલ વેવ બ્રિજ રેક્ટિફાયર વચ્ચેનો તફાવત:

\begin{table}[H]
\centering
\caption{રેક્ટિફાયર્સની સરખામણી}
\begin{tabularx}{\textwidth}{|X|X|X|}
\hline
\textbf{Parameter} & \textbf{Half Wave Rectifier} & \textbf{Bridge Rectifier} \\
\hline
Number of Diodes & 1 ડાયોડનો ઉપયોગ કરે છે. & 4 ડાયોડનો ઉપયોગ કરે છે. \\
\hline
Transformer & સાદા ટ્રાન્સફોર્મરની જરૂર છે. & સાદા ટ્રાન્સફોર્મરની જરૂર છે (સેન્ટર ટેપની જરૂર નથી). \\
\hline
Efficiency & ઓછી કાર્યક્ષમતા (મહત્તમ 40.6\%). & ઉચ્ચ કાર્યક્ષમતા (મહત્તમ 81.2\%). \\
\hline
Ripple Factor & ઉચ્ચ રિપલ (1.21). આઉટપુટ પલ્સેટિંગ છે. & ઓછું રિપલ (0.48). આઉટપુટ વધુ સ્મૂધ છે. \\
\hline
Peak Inverse Voltage (PIV) & PIV રેટિંગ \(V_m\) છે. & PIV રેટિંગ \(V_m\) છે. \\
\hline
Output Frequency & ઇનપુટ ફ્રીક્વન્સી જેટલી જ (\(f_{in}\)). & ઇનપુટ ફ્રીક્વન્સીથી બમણી (\(2f_{in}\)). \\
\hline
Cost & ખૂબ ઓછી કિંમત. & મધ્યમ (4 ડાયોડની કિંમત). \\
\hline
\end{tabularx}
\end{table}

\subparagraph{વિગતવાર સરખામણીના મુદ્દા (Detailed Comparison Points):}
\begin{itemize}
    \item \textbf{રિપલ ફેક્ટર (Ripple Factor):} હાફ-વેવ રેક્ટિફાયરમાં ખૂબ જ ઉંચો રિપલ ફેક્ટર (\(\gamma = 1.21\)) હોય છે, જેનો અર્થ છે કે આઉટપુટમાં AC ઘટક DC ઘટક કરતાં મોટો છે. આને સ્મૂધ કરવા માટે ભારે અને ખર્ચાળ ફિલ્ટરિંગ સર્કિટની જરૂર પડે છે. તેનાથી વિપરીત, બ્રિજ રેક્ટિફાયરમાં રિપલ ફેક્ટર ઘણો ઓછો (\(\gamma = 0.48\)) હોય છે, જે સ્મૂધ આઉટપુટ આપે છે જેને ફિલ્ટર કરવું સરળ છે.
    \item \textbf{ટ્રાન્સફોર્મર ઉપયોગિતા (Transformer Utilization):} હાફ-વેવ રેક્ટિફાયરમાં, ટ્રાન્સફોર્મર સેકન્ડરી વાઇન્ડિંગમાં માત્ર એક હાફ-સાયકલ દરમિયાન કરંટ વહે છે. આ કોરના DC સંતૃપ્તિ (saturation) નું કારણ બને છે અને નબળા ટ્રાન્સફોર્મર યુટિલાઈઝેશન ફેક્ટર (TUF) માં પરિણમે છે. બ્રિજ રેક્ટિફાયર બંને હાફ-સાયકલનો અસરકારક રીતે ઉપયોગ કરે છે અને ઘણો ઊંચો TUF આપે છે.
    \item \textbf{આઉટપુટ પાવર:} સમાન AC ઇનપુટ વોલ્ટેજ માટે, બ્રિજ રેક્ટિફાયર હાફ-વેવ રેક્ટિફાયરની સરખામણીમાં નોંધપાત્ર રીતે વધારે DC આઉટપુટ પાવર પ્રદાન કરે છે કારણ કે તે સંપૂર્ણ AC વેવનો ઉપયોગ કરે છે.
\end{itemize}

\subparagraph{સારાંશ:}
બ્રિજ રેક્ટિફાયર સામાન્ય રીતે ઉચ્ચ ગુણવત્તાવાળા DC પાવર સપ્લાય માટે જરૂરી છે કારણ કે તે વધુ કાર્યક્ષમ છે અને ઓછું રિપલ ધરાવે છે. હાફ વેવ રેક્ટિફાયરનો ઉપયોગ ભાગ્યે જ થાય છે.

\paragraph{મેમરી ટ્રીક:}
\emph{Bridge is Better: Double Efficiency, Half Ripple, Needs 4 Diodes.}

% ========================================
% QUESTION 3(a): Symbols (3 marks)
% Demonstrates: Circuit Symbols
% ========================================

\section{Question 3}
\subsection{Question 3(a) [3 marks]}
\textbf{સિમ્બોલ દોરો: 1) ઝેનર ડાયોડ 2) LED 3) વેરેક્ટર ડાયોડ.}

\subsubsection{Solution}
\begin{figure}[H]
\centering
\begin{tabularx}{\textwidth}{X X X}
    \centering
    \begin{circuitikz}[scale=1]
        \draw (0,0) to[zD] (2,0);
        \node at (1,-0.5) {Zener Diode};
    \end{circuitikz} &
    \centering
    \begin{circuitikz}[scale=1]
        \draw (0,0) to[leD] (2,0);
        \node at (1,-0.5) {LED};
    \end{circuitikz} &
    \centering
    \begin{circuitikz}[scale=1]
        \draw (0,0) to[D] (2,0); % vD not supported, using D
        \node at (1,-0.5) {Varactor Diode};
    \end{circuitikz}
\end{tabularx}
\caption{વિશિષ્ટ ડાયોડના પ્રતીકો}
\end{figure}

\begin{enumerate}
    \item \textbf{Zener Diode:} તે રિવર્સ બ્રેકડાઉન વલણમાં કામ કરે છે. પ્રતીકમાં `Z' આકારનો કેથોડ છે.
    \item \textbf{LED (Light Emitting Diode):} તે પ્રકાશ ઉત્સર્જિત કરે છે. તીર બહારની તરફ નિર્દેશ કરે છે.
    \item \textbf{Varactor Diode:} વેરિયેબલ કેપેસિટર તરીકે કામ કરે છે. પ્રતીકમાં કેપેસિટર પ્લેટ શામેલ છે.
\end{enumerate}

\paragraph{મેમરી ટ્રીક:}
\emph{Zener bends like Z, LED shines out, Varactor is a Variable Cap.}


% ========================================
% QUESTION 3(b): Zener Voltage Regulator (4 marks)
% Demonstrates: Circuit, Operation
% ========================================

\subsection{Question 3(b) [4 marks]}
\textbf{વોલ્ટેજ રેગ્યુલેટર તરીકે ઝેનર ડાયોડનું કામ સમજાવો.}

\subsubsection{Solution}
ઝેનર ડાયોડ વોલ્ટેજ રેગ્યુલેટર ઇનપુટ વોલ્ટેજ અથવા લોડ કરંટમાં ફેરફાર હોવા છતાં આઉટપુટ વોલ્ટેજને અચળ જાળવી રાખે છે.

\paragraph{સર્કિટ ડાયાગ્રામ:}
\begin{figure}[H]
\centering
\begin{circuitikz}[scale=1]
    \draw (0,0) to[sV, l=\(V_{in}\)] (0,3) to[R, l=\(R_S\)] (3,3) coordinate (top);
    \draw (top) to[zD, l=\(D_Z\), *-*] (3,0); 
    \draw (top) -- (5,3) to[R, l=\(R_L\)] (5,0) -- (0,0);
    \draw (5,3) -- (6,3) node[right] {+ \(V_{out}\)};
    \draw (5,0) -- (6,0) node[right] {-};
\end{circuitikz}
\caption{ઝેનર વોલ્ટેજ રેગ્યુલેટર}
\end{figure}

\paragraph{કાર્ય સિદ્ધાંત:}
ઝેનર ડાયોડ લોડ \(R_L\) સમાંતરમાં રિવર્સ બાયસ સાથે જોડાયેલ છે. સીરિઝ રેઝિસ્ટર \(R_S\) પ્રવાહને મર્યાદિત કરે છે.
\begin{enumerate}
    \item \textbf{ઇનપુટ ભિન્નતા સામે રેગ્યુલેશન:} જો ઇનપુટ વોલ્ટેજ \(V_{in}\) વધે છે, તો કુલ પ્રવાહ વધે છે. ઝેનર ડાયોડ વધુ પ્રવાહનું વહન કરે છે (\(I_Z\) વધે છે), જેના કારણે \(R_S\) માં વોલ્ટેજ ડ્રોપ વધે છે. ઝેનર (અને લોડ) પરનો વોલ્ટેજ \(V_Z\) પર અચળ રહે છે.
    \item \textbf{લોડ ભિન્નતા સામે રેગ્યુલેશન:} જો લોડ કરંટ \(I_L\) બદલાય છે (દા.ત., ઘટે છે), તો કુલ પ્રવાહને અચળ રાખવા માટે ઝેનર કરંટ \(I_Z\) તેટલી જ માત્રામાં વધે છે. આથી, વોલ્ટેજ \(V_{out}\) \(V_Z\) પર ક્લેમ્પ્ડ રહે છે.
\end{enumerate}

\paragraph{મેમરી ટ્રીક:}
\emph{Zener acts like a Wall at \(V_Z\). Excess flows through Zener, Load stays safe.}


% ========================================
% QUESTION 3(c): NPN Transistor Working (7 marks)
% Demonstrates: Structure, Biasing, Electron flow
% ========================================

\subsection{Question 3(c) [7 marks]}
\textbf{NPN ટ્રાન્ઝિસ્ટર નું વર્કીંગ સમજાવો.}

\subsubsection{Solution}
NPN ટ્રાન્ઝિસ્ટરમાં બે N-ટાઈપ સ્તરો વચ્ચે સેન્ડવીચ કરેલ P-ટાઈપ સેમિકન્ડક્ટરનું એક પાતળું સ્તર હોય છે.

\paragraph{બાંધકામ અને બાયસિંગ:}
\begin{itemize}
    \item \textbf{Emitter (N):} ભારે ડોપિંગ કરેલું, ઇલેક્ટ્રોન પૂરા પાડે છે.
    \item \textbf{Base (P):} હળવા ડોપિંગ કરેલું અને ખૂબ પાતળું, પ્રવાહને નિયંત્રિત કરે છે.
    \item \textbf{Collector (N):} મધ્યમ ડોપિંગ કરેલું, ઇલેક્ટ્રોન એકત્રિત કરે છે.
\end{itemize}
એક્ટિવ મોડ ઓપરેશન માટે, એમિટર-બેઝ જંકશન ફોરવર્ડ બાયસ હોય છે, અને કલેક્ટર-બેઝ જંકશન રિવર્સ બાયસ હોય છે.

\paragraph{Working Diagram:}
\begin{figure}[H]
\centering
\begin{tikzpicture}[scale=1]
    % NPN Block
    \draw[thick] (0,0) rectangle (2,2) node[midway] {N (Emitter)};
    \draw[thick] (2,0) rectangle (3,2) node[midway] {P};
    \draw[thick] (3,0) rectangle (5,2) node[midway] {N (Collector)};
    
    % Biasing
    \draw (0,1) -- (-1,1) -- (-1,-1) to[battery1, l=\(V_{BE}\)] (2.5,-1) -- (2.5,0);
    \draw (5,1) -- (6,1) -- (6,-1) to[battery1, l=\(V_{CB}\)] (2.5,-1);
    
    % Electron Flow
    \draw[->, dashed] (0.5, 0.5) -- (4.5, 0.5) node[right] {Electrons};
\end{tikzpicture}
\caption{NPN ટ્રાન્ઝિસ્ટર બાયસિંગ અને પ્રવાહ}
\end{figure}

\paragraph{કાર્યપદ્ધતિ:}
\subparagraph{Electron Injection:}
એમિટર-બેઝ જંકશન ફોરવર્ડ બાયસ હોવાથી, N-ટાઈપ એમિટરમાંથી ઇલેક્ટ્રોન બેઝ તરફ ધકેલવામાં આવે છે.
\subparagraph{Base Transport:}
બેઝ ખૂબ જ પાતળો અને ઓછો ડોપ થયેલો છે. માત્ર થોડા ઇલેક્ટ્રોન (આશરે 5\%) બેઝ કરંટ (\(I_B\)) બનાવવા માટે P-પ્રદેશમાં હોલ્સ સાથે ફરી જોડાય છે (recombine).
\subparagraph{Collection:}
બાકીના 95\% ઇલેક્ટ્રોન બેઝને પાર કરે છે અને કલેક્ટર જંકશન સુધી પહોંચે છે. કલેક્ટર-બેઝ જંકશન રિવર્સ બાયસ હોવાથી (કલેક્ટર સાથે પોઝિટિવ જોડાયેલ), આ ઇલેક્ટ્રોન કલેક્ટરની પોઝિટિવ ક્ષમતા દ્વારા ભારપૂર્વક આકર્ષિત થાય છે અને કલેક્ટર કરંટ (\(I_C\)) બનાવવા માટે બહાર નીકળી જાય છે.

\subparagraph{Current Equation:}
કિર્ચોફના કરંટ કાયદા (KCL) ને લાગુ પાડતા:
\[ I_E = I_B + I_C \]
જ્યાં \(I_E\) એમિટર કરંટ છે, \(I_B\) બેઝ કરંટ છે, અને \(I_C\) કલેક્ટર કરંટ છે. કારણ કે \(I_B\) ખૂબ નાનો છે, \(I_C \approx I_E\).

\paragraph{મેમરી ટ્રીક:}
\emph{NPN: Not Pointing In (Arrow points out). Emitter Emits, Base Controls, Collector Collects.}


% ========================================
% QUESTION 3(a) OR: Transistor Definition (3 marks)
% Demonstrates: Definition
% ========================================

\subsection{Question 3(a) OR [3 marks]}
\textbf{ટ્રાન્ઝીસ્ટરની વ્યાખ્યા આપી તેના પ્રકારો વરુણવો.}

\subsubsection{Solution}
\paragraph{વ્યાખ્યા:}
ટ્રાન્ઝિસ્ટર એ ત્રણ ટર્મિનલ સેમિકન્ડક્ટર ઉપકરણ છે જેનો ઉપયોગ ઇલેક્ટ્રિકલ સિગ્નલો અને પાવરને એમ્પ્લીફાય અથવા સ્વિચ કરવા માટે થાય છે. તે સામાન્ય રીતે બાહ્ય સર્કિટના જોડાણ માટે ઓછામાં ઓછા ત્રણ ટર્મિનલ્સ સાથે સેમિકન્ડક્ટર સામગ્રીથી બનેલું હોય છે.

\paragraph{ટ્રાન્ઝિસ્ટરના પ્રકારો:}
ટ્રાન્ઝિસ્ટરને મુખ્યત્વે બે મુખ્ય પરિવારોમાં વર્ગીકૃત કરવામાં આવે છે, દરેક ચોક્કસ ઓપરેટિંગ સિદ્ધાંતો સાથે:
\begin{enumerate}
    \item \textbf{BJT (Bipolar Junction Transistor):} કરંટ દ્વારા નિયંત્રિત. તે વહન (conduction) માટે ઇલેક્ટ્રોન અને હોલ્સ બંનેનો ઉપયોગ કરે છે.
    \begin{itemize}
        \item NPN ટ્રાન્ઝિસ્ટર: ઇલેક્ટ્રોન બહુમતી વાહકો છે (ઝડપી, લોકપ્રિય).
        \item PNP ટ્રાન્ઝિસ્ટર: હોલ્સ બહુમતી વાહકો છે.
    \end{itemize}
    \item \textbf{FET (Field Effect Transistor):} વોલ્ટેજ દ્વારા નિયંત્રિત. તે માત્ર એક પ્રકારના ચાર્જ કેરિયરનો ઉપયોગ કરે છે (યુનિપોલર).
    \begin{itemize}
        \item JFET (Junction FET): સરળ બાંધકામ, ઉચ્ચ ઇનપુટ અવબાધ. (N-channel, P-channel).
        \item MOSFET (Metal Oxide Semiconductor FET): ખૂબ જ ઉચ્ચ ઇનપુટ અવબાધ, ડિજિટલ સર્કિટમાં વપરાય છે. (Depletion Type, Enhancement Type).
    \end{itemize}
\end{enumerate}

\paragraph{ઉપયોગો:}
ટ્રાન્ઝિસ્ટર એ આધુનિક ઈલેક્ટ્રોનિક્સના બિલ્ડીંગ બ્લોક્સ છે, જેનો ઉપયોગ એમ્પ્લીફાયર, સ્વીચો, વોલ્ટેજ રેગ્યુલેટર અને ઓસિલેટરમાં થાય છે.

\paragraph{મેમરી ટ્રીક:}
\emph{Transistor = Transfer + Resistor. BJT (Bi) and FET (Field) are main families.}


% ========================================
% QUESTION 3(b) OR: PNP Transistor Construction (4 marks)
% Demonstrates: Block diagram
% ========================================

\subsection{Question 3(b) OR [4 marks]}
\textbf{PNP ટ્રાન્ઝીસ્ટર નુ બંધારણ દોરો અને સમજાવો.}

\subsubsection{Solution}
P-ટાઈપ સેમિકન્ડક્ટરના બે સ્તરો વચ્ચે N-ટાઈપ સેમિકન્ડક્ટરના પાતળા સ્તરને સેન્ડવીચ કરીને PNP ટ્રાન્ઝિસ્ટર બનાવવામાં આવે છે.

\paragraph{માળખાકીય ડાયાગ્રામ:}
\begin{figure}[H]
\centering
\begin{tikzpicture}[scale=1]
    \draw[thick] (0,0) rectangle (1.5,2) node[midway] {P (Emitter)};
    \draw[thick] (1.5,0) rectangle (2.5,2) node[midway] {N (Base)};
    \draw[thick] (2.5,0) rectangle (4,2) node[midway] {P (Collector)};
    \draw (0,1) -- (-0.5,1) node[left] {E};
    \draw (2,1) -- (2,2.5) node[above] {B};
    \draw (4,1) -- (4.5,1) node[right] {C};
\end{tikzpicture}
\caption{PNP ટ્રાન્ઝિસ્ટરનું બંધારણ}
\end{figure}

\paragraph{ટર્મિનલ્સ:}
\begin{enumerate}
    \item \textbf{Emitter (E):} P-ટાઈપ વિસ્તાર. મોટાભાગના ચાર્જ કેરિયર્સ (હોલ્સ) પૂરા પાડવા માટે તે ભારે ડોપ થયેલ છે.
    \item \textbf{Base (B):} N-ટાઈપ વિસ્તાર. તે ખૂબ જ પાતળું અને હળવા ડોપ થયેલું છે. તે એમિટરથી કલેક્ટર સુધીના હોલ્સના પ્રવાહને નિયંત્રિત કરે છે.
    \item \textbf{Collector (C):} P-ટાઈપ વિસ્તાર. ગરમીને દૂર કરવા માટે તે મધ્યમ ડોપ થયેલ અને એમિટર કરતા કદમાં મોટું હોય છે. તે એમિટર દ્વારા પૂરા પાડવામાં આવેલ હોલ્સને એકત્રિત કરે છે.
\end{enumerate}

\paragraph{મેમરી ટ્રીક:}
\emph{PNP: Pointing In (Arrow points in). P-N-P Sandwich.}


% ========================================
% QUESTION 3(c) OR: PN Junction Diode V-I Characteristics (7 marks)
% Demonstrates: Graph, Explanation
% ========================================

\subsection{Question 3(c) OR [7 marks]}
\textbf{P-N જંકશન ડાયોડની V-I લાક્ષણીકતા સમજાવો.}

\subsubsection{Solution}
V-I લાક્ષણીકતા વળાંક ડાયોડ (\(V\)) તરફના વોલ્ટેજ અને તેમાંથી વહેતા કરંટ (\(I\)) વચ્ચેનો સંબંધ દર્શાવે છે.

\paragraph{લાક્ષણીકતા ગ્રાફ:}
\begin{figure}[H]
\centering
\begin{tikzpicture}[scale=0.8]
    \draw[->] (-3,0) -- (4,0) node[right] {\(V\) (Volts)};
    \draw[->] (0,-3) -- (0,4) node[above] {\(I\) (mA)};
    
    % Forward Bias curve
    \draw[thick, blue] (0,0) -- (0.7,0) .. controls (1,0.5) and (1.2,3) .. (1.5,4);
    \node at (2,3) {Forward Bias};
    \node at (0.7,-0.5) {Knee Voltage};

    % Reverse Bias curve
    \draw[thick, red] (0,0) -- (-2, -0.2) -- (-2, -3);
    \node at (-1.5,-1) {Reverse Bias};
    \node at (-2,-3.5) {Breakdown};
\end{tikzpicture}
\caption{PN ડાયોડની V-I લાક્ષણીકતાઓ}
\end{figure}

\paragraph{સમજૂતી:}
\subparagraph{Forward Bias Region:}
\begin{itemize}
    \item જ્યારે ડાયોડ ફોરવર્ડ બાયસ હોય છે (એનોડ પોઝિટિવ), શરૂઆતમાં બેરિયર પોટેન્શિયલ દૂર ન થાય ત્યાં સુધી નહિવત પ્રવાહ વહે છે.
    \item \textbf{Knee Voltage (Cut-in Voltage):} આ તે વોલ્ટેજ છે જેના પર વર્તમાન ઝડપથી વધવાનો શરૂ થાય છે. સિલિકોન માટે, તે આશરે \(0.7V\) છે, અને જર્મેનિયમ માટે, આશરે \(0.3V\).
    \item આ વોલ્ટેજથી આગળ, વોલ્ટેજ સાથે કરંટ ઝડપથી વધે છે.
\end{itemize}

\subparagraph{Reverse Bias Region:}
\begin{itemize}
    \item જ્યારે રિવર્સ બાયસ (એનોડ નેગેટિવ), લઘુમતી કેરિયર્સને કારણે \textbf{Reverse Saturation Current} (\(I_0\)) નામનો ખૂબ નાનો પ્રવાહ વહે છે. આ પ્રવાહ તાપમાન પર આધારિત છે પરંતુ બ્રેકડાઉન સુધી વોલ્ટેજથી લગભગ સ્વતંત્ર છે.
    \item રિવર્સ બાયસમાં ડેપ્લેશન રિજન (depletion region) ની પહોળાઈ વધે છે, જે બહુમતી વાહક પ્રવાહને અટકાવે છે.
    \item \textbf{Reverse Breakdown:} જો રિવર્સ વોલ્ટેજ મર્યાદા (બ્રેકડાઉન વોલ્ટેજ) થી આગળ વધે છે, તો સહસંયોજક બોન્ડ (covalent bonds) તૂટી જાય છે (ઝેનર અથવા એવેલેન્ચ અસર), અને મોટો પ્રવાહ વહે છે, જે ડાયોડને નુકસાન પહોંચાડી શકે છે.
\end{itemize}

\paragraph{લાક્ષણીકતાઓ પર આધારિત ઉપયોગો:}
\begin{itemize}
    \item \textbf{Rectifiers:} AC ને DC માં કન્વર્ટ કરવા માટે ફોરવર્ડ બાયસ ગુણધર્મનો ઉપયોગ કરે છે.
    \item \textbf{Switches:} ફોરવર્ડ બાયસમાં ક્લોઝ સ્વિચ તરીકે અને રિવર્સ બાયસમાં ઓપન સ્વિચ તરીકે કામ કરે છે.
    \item \textbf{Protection Circuits:} રિવર્સ પોલારિટી નુકસાન અટકાવે છે.
\end{itemize}

\paragraph{મેમરી ટ્રીક:}
\emph{Forward: Obstacle at 0.7V then FREE FLOW. Reverse: Wall holds until it BREAKS.}

% ========================================
% QUESTION 4(a): PNP/NPN Symbols & Construction (3 marks)
% Demonstrates: Symbols, Block Diagram
% ========================================

\section{Question 4}
\subsection{Question 4(a) [3 marks]}
\textbf{PNP અને NPN ટ્રાન્ઝિસ્ટરના સિમ્બોલ અને બંધારણ દોરો.}

\subsubsection{Solution}
\paragraph{પ્રતીકો (Symbols):}
\begin{figure}[H]
\centering
\begin{tabularx}{\textwidth}{X X}
    \centering
    \begin{circuitikz}[scale=1]
        \draw (0,0) node[npn] (T) {};
        \node at (T) [right=0.5cm] {NPN (Not Pointing In)};
    \end{circuitikz} &
    \centering
    \begin{circuitikz}[scale=1]
        \draw (0,0) node[pnp] (T) {};
        \node at (T) [right=0.5cm] {PNP (Pointing In)};
    \end{circuitikz}
\end{tabularx}
\caption{ટ્રાન્ઝિસ્ટર પ્રતીકો}
\end{figure}

\paragraph{બાંધકામ બ્લોક ડાયાગ્રામ:}
\begin{figure}[H]
\centering
\begin{tikzpicture}[scale=0.8]
    % NPN
    \draw (0,0) rectangle (1,2) node[midway] {N};
    \draw (1,0) rectangle (2,2) node[midway] {P};
    \draw (2,0) rectangle (3,2) node[midway] {N};
    \node at (1.5,-0.5) {NPN Construction};
    
    % PNP
    \draw (5,0) rectangle (6,2) node[midway] {P};
    \draw (6,0) rectangle (7,2) node[midway] {N};
    \draw (7,0) rectangle (8,2) node[midway] {P};
    \node at (6.5,-0.5) {PNP Construction};
\end{tikzpicture}
\caption{બાંધકામ માળખું}
\end{figure}

\paragraph{મેમરી ટ્રીક:}
\emph{NPN: Arrow OUT. PNP: Arrow IN.}


% ========================================
% QUESTION 4(b): CE Characteristics (4 marks)
% Demonstrates: Graph
% ========================================

\subsection{Question 4(b) [4 marks]}
\textbf{CE કન્ફિગરેશનની ઇનપુટ અને આઉટપુટ લાક્ષણીકતાઓ દોરો અને સમજાવો.}

\subsubsection{Solution}
ઉચ્ચ કરંટ અને વોલ્ટેજ ગેઇનને કારણે કોમન એમિટર (CE) કન્ફિગરેશન સૌથી વધુ ઉપયોગમાં લેવાતું કન્ફિગરેશન છે.

\paragraph{ઇનપુટ લાક્ષણીકતાઓ:}
તે અચળ આઉટપુટ વોલ્ટેજ (\(V_{CE}\)) પર ઇનપુટ કરંટ (\(I_B\)) વિરુદ્ધ ઇનપુટ વોલ્ટેજ (\(V_{BE}\)) નો આલેખ છે.
\begin{itemize}
    \item તે ફોરવર્ડ-બાયસ ડાયોડ લાક્ષણીકતા જેવું લાગે છે.
    \item જેમ જેમ \(V_{BE}\) કટ-ઇન વોલ્ટેજ (Si માટે 0.7V) થી આગળ વધે છે, તેમ \(I_B\) ઝડપથી વધે છે.
    \item ઉચ્ચ \(V_{CE}\) આલેખને સહેજ જમણી તરફ ખસેડે છે (Early Effect).
\end{itemize}

\paragraph{આઉટપુટ લાક્ષણીકતાઓ:}
તે અચળ ઇનપુટ કરંટ (\(I_B\)) પર આઉટપુટ કરંટ (\(I_C\)) વિરુદ્ધ આઉટપુટ વોલ્ટેજ (\(V_{CE}\)) નો આલેખ છે.
\begin{itemize}
    \item \textbf{Active Region:} \(I_C\) લગભગ સ્થિર અને \(I_B\) ના પ્રમાણસર છે. સામાન્ય એમ્પ્લીફાયર કામગીરી.
    \item \textbf{Saturation Region:} બંને જંકશન ફોરવર્ડ બાયસ હોય છે. \(V_{CE}\) ખૂબ ઓછું છે.
    \item \textbf{Cut-off Region:} બંને જંકશન રિવર્સ બાયસ હોય છે. \(I_C \approx 0\).
\end{itemize}

\paragraph{મેમરી ટ્રીક:}
\emph{Input like Diode. Output like Steps (Flat lines controlled by \(I_B\)).}


% ========================================
% QUESTION 4(c): Current Gain Relation (7 marks)
% Demonstrates: Derivation
% ========================================

\subsection{Question 4(c) [7 marks]}
\textbf{\(\alpha\) અને \(\beta\) વચ્ચેનો સંબંધ તારવો.}

\subsubsection{Solution}
\paragraph{વ્યાખ્યાઓ (Definitions):}
\begin{itemize}
    \item \(\alpha\) (Alpha): કોમન બેઝ (CB) કન્ફિગરેશનમાં કરંટ ગેઇન.
    \[ \alpha = \frac{I_C}{I_E} \]
    \item \(\beta\) (Beta): કોમન એમિટર (CE) કન્ફિગરેશનમાં કરંટ ગેઇન.
    \[ \beta = \frac{I_C}{I_B} \]
\end{itemize}

\paragraph{ડેરીવેશન (તારવણી):}
અમે મૂળભૂત ટ્રાન્ઝિસ્ટર વર્તમાન સંબંધ સાથે પ્રારંભ કરીએ છીએ:
\[ I_E = I_B + I_C \]
અમારો ધ્યેય \(\alpha\) ને \(\beta\) ના સંદર્ભમાં વ્યક્ત કરવાનો છે.
Step 1: આખા સમીકરણને \(I_C\) વડે ભાગો:
\[ \frac{I_E}{I_C} = \frac{I_B}{I_C} + \frac{I_C}{I_C} \]
Step 2: વ્યાખ્યાઓ \(\frac{I_E}{I_C} = \frac{1}{\alpha}\) અને \(\frac{I_B}{I_C} = \frac{1}{\beta}\) સમીકરણમાં મૂકો:
\[ \frac{1}{\alpha} = \frac{1}{\beta} + 1 \]
Step 3: \(\alpha\) માટે ઉકેલો. જમણી બાજુ પર લસાઅ (LCM) લો:
\[ \frac{1}{\alpha} = \frac{1 + \beta}{\beta} \]
Step 4: \(\alpha\) મેળવવા માટે બંને બાજુ ઉલટાવો:
\[ \alpha = \frac{\beta}{1 + \beta} \]

તેનાથી વિપરીત, \(\alpha\) ના સંદર્ભમાં \(\beta\) શોધવા માટે:
\[ \frac{1}{\beta} = \frac{1}{\alpha} - 1 = \frac{1 - \alpha}{\alpha} \]
ઉલટાવતા:
\[ \beta = \frac{\alpha}{1 - \alpha} \]
આ બતાવે છે કે \(\beta\) એ \(\alpha\) કરતા ઘણો મોટો છે કારણ કે છેદ \((1-\alpha)\) ખૂબ નાનો છે (કારણ કે \(\alpha \approx 0.99\)).
ઉદાહરણ તરીકે, જો \(\alpha = 0.99\), તો \(\beta = \frac{0.99}{0.01} = 99\).

\paragraph{ગામા (\(\gamma\)) સાથે સંબંધ:}
\(\gamma\) એ કોમન કલેક્ટર (CC) કન્ફિગરેશનનો કરંટ ગેઇન છે, જે \(\gamma = I_E / I_B\) તરીકે વ્યાખ્યાયિત છે.
સંબંધ \(I_E = I_B + I_C\) પરથી, \(I_B\) વડે ભાગતા:
\[ \frac{I_E}{I_B} = 1 + \frac{I_C}{I_B} \]
\[ \gamma = 1 + \beta \]
કેમ કે \(\beta\) સામાન્ય રીતે મોટો હોય છે (50-300), \(\gamma\) લગભગ \(\beta\) સમાન છે. આ સ્પષ્ટ કરે છે કે CC કન્ફિગરેશનમાં પણ ખૂબ જ ઉચ્ચ કરંટ ગેઇન હોય છે.

\paragraph{મેમરી ટ્રીક:}
\emph{\(\beta\) is Big (Greater than 1). \(\alpha\) is Always less than 1.}


% ========================================
% QUESTION 4(a) OR: Transistor Regions (3 marks)
% Demonstrates: List
% ========================================

\subsection{Question 4(a) OR [3 marks]}
\textbf{ટ્રાન્ઝિસ્ટર ના ઓપરેટિંગ રીજીયન લખો.}

\subsubsection{Solution}
ટ્રાન્ઝિસ્ટરમાં તેના બે જંકશન (એમિટર-બેઝ અને કલેક્ટર-બેઝ) ના બાયસિંગ પર આધારિત ત્રણ મુખ્ય ઓપરેટિંગ પ્રદેશો હોય છે.

\begin{enumerate}
    \item \textbf{Active Region:}
    \begin{itemize}
        \item \textbf{Biasing:} એમિટર-બેઝ જંકશન ફોરવર્ડ બાયસ (Forward Biased) હોય છે, જ્યારે કલેક્ટર-બેઝ જંકશન રિવર્સ બાયસ (Reverse Biased) હોય છે.
        \item \textbf{કરંટ:} આઉટપુટ કરંટ \(I_C\) ઇનપુટ કરંટ \(I_B\) (\(I_C = \beta I_B\)) દ્વારા નિયંત્રિત થાય છે.
        \item \textbf{ઉપયોગ:} સિગ્નલ એમ્પ્લીફિકેશન (લીનિયર એમ્પ્લીફાયર) માટે વપરાય છે.
    \end{itemize}
    \item \textbf{Saturation Region:}
    \begin{itemize}
        \item \textbf{Biasing:} એમિટર-બેઝ અને કલેક્ટર-બેઝ બંને જંકશન ફોરવર્ડ બાયસ (Forward Biased) હોય છે.
        \item \textbf{કરંટ:} બેઝ કરંટથી સ્વતંત્ર મહત્તમ કલેક્ટર પ્રવાહ વહે છે. \(V_{CE}\) લગભગ શૂન્ય (0.2V) છે.
        \item \textbf{ઉપયોગ:} ડિજિટલ લોજિકમાં ક્લોઝ સ્વિચ (ON state) તરીકે કાર્ય કરે છે.
    \end{itemize}
    \item \textbf{Cut-off Region:}
    \begin{itemize}
        \item \textbf{Biasing:} બંને એમિટર-ભેઝ અને કલેક્ટર-બેઝ જંકશન રિવર્સ બાયસ (Reverse Biased) હોય છે.
        \item \textbf{કરંટ:} આદર્શ રીતે શૂન્ય કલેક્ટર કરંટ વહે છે (માત્ર માઇનોરિટી લિકેજ કરંટ).
        \item \textbf{ઉપયોગ:} ઓપન સ્વિચ (OFF state) તરીકે કાર્ય કરે છે.
    \end{itemize}
\end{enumerate}

\paragraph{મેમરી ટ્રીક:}
\emph{Active (FR) = Amplifier. Saturation (FF) = Switch ON. Cut-off (RR) = Switch OFF.}


% ========================================
% QUESTION 4(b) OR: Transistor as Amplifier (4 marks)
% Demonstrates: Circuit
% ========================================

\subsection{Question 4(b) OR [4 marks]}
\textbf{એમ્પ્લીફાયર તરીકે ટ્રાન્ઝિસ્ટર સમજાવો.}

\subsubsection{Solution}
એમ્પ્લીફાયર નબળા સિગ્નલના કંપવિસ્તાર (amplitude) ને વધારે છે. કોમન એમિટર (CE) કન્ફિગરેશનમાં ટ્રાન્ઝિસ્ટર કાર્યક્ષમ એમ્પ્લીફાયર તરીકે કામ કરે છે.

\paragraph{સર્કિટ ડાયાગ્રામ:}
\begin{figure}[H]
\centering
\begin{circuitikz}[scale=1]
    \draw (0,0) node[npn] (T) {};
    \draw (T.E) -- (0,-2) node[ground]{};
    \draw (T.B) to[C, l=\(C_{in}\)] (-2,0) to[sV, l=\(V_{in}\)] (-2,-2) node[ground]{};
    \draw (-2,0) to[R, l=\(R_B\)] (-2, 2) -- (0,2) -- (T.C);
    \draw (0,2) to[R, l=\(R_C\)] (0,4) node[vcc]{\(V_{CC}\)}; 
    \draw (T.C) to[C, l=\(C_{out}\), *-o] (2,0) node[right]{\(V_{out}\)};
\end{circuitikz}
\caption{સિંગલ સ્ટેજ CE એમ્પ્લીફાયર}
\end{figure}

\paragraph{કાર્યપદ્ધતિ:}
\begin{itemize}
    \item ટ્રાન્ઝિસ્ટર એક્ટિવ રિજનમાં બાયસ થયેલું છે.
    \item બેઝ પર નાનું AC ઇનપુટ સિગ્નલ \(V_{in}\) લાગુ કરવામાં આવે છે. આ બેઝ કરંટ \(I_B\) માં નાના ફેરફારોનું કારણ બને છે.
    \item \(I_C = \beta I_B\) હોવાથી, આ નાના ફેરફારો કલેક્ટર કરંટ \(I_C\) માં મોટા ફેરફારોમાં પરિણમે છે.
    \item આ મોટો બદલાતો પ્રવાહ લોડ રેઝિસ્ટર \(R_C\) માંથી વહે છે, જે મોટો એમ્પ્લીફાઈડ વોલ્ટેજ આઉટપુટ \(V_{out}\) ઉત્પન્ન કરે છે.
    \item આઉટપુટ ઇનપુટની સાપેક્ષમાં 180 ડિગ્રી ફેઝ-શિફ્ટ થયેલ હોય છે.
\end{itemize}

\paragraph{મેમરી ટ્રીક:}
\emph{Small Base tickle makes Big Collector laugh. 180 degree flip.}


% ========================================
% QUESTION 4(c) OR: Compare CB, CC, CE (7 marks)
% Demonstrates: Comparison Table
% ========================================

\subsection{Question 4(c) OR [7 marks]}
\textbf{CB, CC અને CE કન્ફિગરેશનની સરખામણી કરો.}

\subsubsection{Solution}
\begin{table}[H]
\centering
\caption{BJT કન્ફિગરેશનની સરખામણી}
\begin{tabularx}{\textwidth}{|X|X|X|X|}
\hline
\textbf{Parameter} & \textbf{Common Base (CB)} & \textbf{Common Emitter (CE)} & \textbf{Common Collector (CC)} \\ \hline
\textbf{Common Terminal} & Base & Emitter & Collector \\ \hline
\textbf{Input Terminal} & Emitter & Base & Base \\ \hline
\textbf{Output Terminal} & Collector & Collector & Emitter \\ \hline
\textbf{Current Gain} & Low (\(\alpha < 1\)) & High (\(\beta\)) & Very High (\(\gamma = 1+\beta\)) \\ \hline
\textbf{Voltage Gain} & High & High & Low (Less than 1) \\ \hline
\textbf{Input Resistance} & Very Low & Moderate & Very High \\ \hline
\textbf{Output Resistance} & Very High & Moderate & Very Low \\ \hline
\textbf{Phase Shift} & 0 degrees & 180 degrees & 0 degrees \\ \hline
\end{tabularx}
\end{table}

\paragraph{વિગતવાર સરખામણી (Detailed Comparison):}
BJT એમ્પ્લીફાયરનું પ્રદર્શન તેના કન્ફિગરેશન પર આધાર રાખે છે:
\begin{itemize}
    \item \textbf{CE Configuration:} તે કરંટ અને વોલ્ટેજ ગેઇન બંને પ્રદાન કરે છે, જેના પરિણામે ખૂબ ઊંચો પાવર ગેઇન (\(> 10,000\)) મળે છે. આ તેને મોટાભાગના એમ્પ્લીફિકેશન હેતુઓ માટે પ્રમાણભૂત પસંદગી બનાવે છે. તે એકમાત્ર કન્ફિગરેશન છે જે \(180^{\circ}\) ફેઝ શિફ્ટ ઉમેરે છે.
    \item \textbf{CC Configuration:} એમિટર ફોલોઅર (Emitter Follower) તરીકે ઓળખાય છે કારણ કે આઉટપુટ વોલ્ટેજ ઇનપુટ વોલ્ટેજને અનુસરે છે. તેનો ઉચ્ચ ઇનપુટ અવબાધ અને ઓછો આઉટપુટ અવબાધ તેને ઉચ્ચ-અવબાધ સ્ત્રોતોમાંથી લો-ઇમ્પિડન્સ લોડ્સ (જેમ કે સ્પીકર્સ) ચલાવવા માટે આદર્શ બનાવે છે.
    \item \textbf{CB Configuration:} ખૂબ જ ઓછો ઇનપુટ રેઝિસ્ટન્સ ધરાવે છે. તે સામાન્ય ઓડિયો માટે ભાગ્યે જ ઉપયોગમાં લેવાય છે પરંતુ ઓછા ઇમ્પિડન્સ સ્ત્રોતો સાથે મેચ કરવા માટે ઉચ્ચ-આવર્તન રેડિયો ફ્રિકવન્સી (RF) સર્કિટમાં ઉપયોગી છે.
\end{itemize}

\paragraph{પસંદગી માર્ગદર્શિકા (Selection Guide):}
\begin{itemize}
    \item \textbf{CE પસંદ કરો} જ્યારે તમને મહત્તમ પાવર ગેઇનની જરૂર હોય (દા.ત., ઓડિયો એમ્પ્લીફાયર, રેડિયો સિગ્નલ). તે વોલ્ટેજ અને કરંટ બંનેને નોંધપાત્ર રીતે વધારે છે.
    \item \textbf{CC પસંદ કરો} જ્યારે તમારે ઉચ્ચ ઇમ્પિડન્સ સ્ત્રોતને લો ઇમ્પિડન્સ લોડ સાથે જોડવાની જરૂર હોય (દા.ત., માઇક્રોફોન આઉટપુટને સ્પીકર એમ્પ્લીફાયર ઇનપુટ સાથે જોડવું). તે ઇમ્પિડન્સ મેચર તરીકે કામ કરે છે.
    \item \textbf{CB પસંદ કરો} જ્યારે તમને સ્થિર ઉચ્ચ-આવર્તન કામગીરીની જરૂર હોય (દા.ત., એન્ટેના સર્કિટ).
\end{itemize}

\paragraph{મેમરી ટ્રીક:}
\emph{CE is King (Gain). CC is Buffer (Impedance). CB is Fast (High Freq).}

% ========================================
% QUESTION 5(a): IC 555 Pin Diagram (3 marks)
% Demonstrates: Pinout Diagram
% ========================================

\section{Question 5}
\subsection{Question 5(a) [3 marks]}
\textbf{IC 555 નો પીન ડાયાગ્રામ દોરો.}

\subsubsection{Solution}
IC 555 એ 8-પિન DIP (Dual Inline Package) ટાઈમર IC છે.

\begin{figure}[H]
\centering
\begin{circuitikz}[scale=1]
    \draw (0,0) rectangle (4,4);
    \node at (2,2) [align=center] {\textbf{IC 555} \\ Timer};
    
    % Left Pins
    \draw (0,3.5) -- (-1,3.5) node[left] {1 (GND)};
    \draw (0,2.5) -- (-1,2.5) node[left] {2 (TRIG)};
    \draw (0,1.5) -- (-1,1.5) node[left] {3 (OUT)};
    \draw (0,0.5) -- (-1,0.5) node[left] {4 (RESET)};
    
    % Right Pins
    \draw (4,0.5) -- (5,0.5) node[right] {5 (CV)};
    \draw (4,1.5) -- (5,1.5) node[right] {6 (THRES)};
    \draw (4,2.5) -- (5,2.5) node[right] {7 (DISCH)};
    \draw (4,3.5) -- (5,3.5) node[right] {8 (VCC)};
    
    % Notch
    \draw (1.5,4) arc (180:360:0.5);
\end{circuitikz}
\caption{IC 555 પીન રચના}
\end{figure}

\paragraph{મેમરી ટ્રીક:}
\emph{G-T-O-R (Ground, Trigger, Output, Reset) on Left. V-D-T-C (Vcc, Discharge, Thres, Control) on Right.}


% ========================================
% QUESTION 5(b): IC 555 Features (4 marks)
% Demonstrates: List
% ========================================

\subsection{Question 5(b) [4 marks]}
\textbf{IC 555 ના ફીચર્સ (લાક્ષણિકતાઓ) લખો.}

\subsubsection{Solution}
NE555 ટાઈમર IC ની મુખ્ય લાક્ષણિકતાઓ નીચે મુજબ છે:

\begin{enumerate}
    \item \textbf{Supply Voltage:} આ ડિવાઇસ +5V થી +18V સુધીના DC પાવર સપ્લાય વોલ્ટેજની વિશાળ શ્રેણી પર કામ કરે છે, જે તેને TTL અને CMOS જેવી વિવિધ લોજિક ફેમિલી સાથે સુસંગત બનાવે છે.
    \item \textbf{Current Capability:} આઉટપુટ પિન (પિન 3) 200mA સુધીનો કરંટ સિંક અથવા સોર્સ કરી શકે છે. આ ઉચ્ચ ડ્રાઇવ ક્ષમતાનો અર્થ એ છે કે તે વધારાના ટ્રાન્ઝિસ્ટર વિના સીધા રિલે, નાના લેમ્પ અને LED જેવા લોડ ચલાવી શકે છે.
    \item \textbf{Timing Range:} તે બાહ્ય રેઝિસ્ટર અને કેપેસિટરના મૂલ્યો દ્વારા થોડા માઈક્રોસેકન્ડ્સથી લઈને અનેક કલાકો સુધીનો ચોક્કસ સમય વિલંબ (Time Delay) ઉત્પન્ન કરી શકે છે.
    \item \textbf{Modes of Operation:} તે ત્રણ મુખ્ય મોડ્સમાં કાર્ય કરે છે:
    \begin{itemize}
        \item Monostable (One-shot): ટ્રિગર થાય ત્યારે એક જ પલ્સ જનરેટ કરે છે.
        \item Astable (Oscillator): સતત સ્ક્વેર વેવ જનરેટ કરે છે (ફ્રી-રનિંગ).
        \item Bistable (Flip-flop): સ્ટેટ સ્ટોરેજ માટે સરળ ફ્લિપ-ફ્લોપ તરીકે કાર્ય કરે છે.
    \end{itemize}
    \item \textbf{Duty Cycle:} અસ્ટેબલ મોડમાં આઉટપુટ વેવફોર્મની ડ્યુટી સાયકલ ટાઈમિંગ રેઝિસ્ટર્સના ગુણોત્તરને બદલીને એડજસ્ટ કરી શકાય છે.
    \item \textbf{Compatibility:} તે અન્ય 555 વર્ઝન સાથે પિન-સુસંગત છે અને TTL અને CMOS સુસંગત લોજિક લેવલ પ્રદાન કરે છે.
    \item \textbf{Stability:} તે શ્રેષ્ઠ તાપમાન સ્થિરતા પ્રદાન કરે છે, તાપમાનમાં ફેરફાર સાથે ટાઈમિંગ ચોકસાઈ માત્ર 0.005\% પ્રતિ ડિગ્રી સેલ્સિયસ બદલાય છે.
\end{enumerate}

\paragraph{મેમરી ટ્રીક:}
\emph{Wide Voltage, High Current, Micro-to-Hours, 3 Modes (M-A-B).}


% ========================================
% QUESTION 5(c): Monostable Multivibrator (7 marks)
% Demonstrates: Circuit, Operation, Waveform
% ========================================

\subsection{Question 5(c) [7 marks]}
\textbf{IC 555 નો ઉપયોગ કરીને મોનોસ્ટેબલ મલ્ટીવાઈબ્રેટર દોરો અને સમજાવો.}

\subsubsection{Solution}
મોનોસ્ટેબલ મલ્ટીવાઈબ્રેટરમાં એક સ્થિર અવસ્થા (Low) અને એક અર્ધ-સ્થિર અવસ્થા (High) હોય છે. ટ્રિગર કરવામાં આવે ત્યારે તે ચોક્કસ સમયગાળાનો એક પલ્સ ઉત્પન્ન કરે છે.

\paragraph{સર્કિટ ડાયાગ્રામ:}
\begin{figure}[H]
\centering
\begin{circuitikz}[scale=1]
    \draw (0,0) node[dipchip, num pins=8, hide numbers, no topmark, external pins width=0](C){IC 555};
    
    % Pin Connections
    \draw (C.pin 1) -- ++(-0.5,0) node[ground]{} node[left, pos=1]{1 (GND)};
    \draw (C.pin 8) -- ++(0,1) node[vcc]{Vcc} node[right, pos=0.1]{8};
    \draw (C.pin 4) -- ++(0,0.5) -- (C.pin 8); % Reset to Vcc
    
    % Timing RC
    \draw (C.pin 8) ++ (2,1.5) node[vcc]{Vcc} to[R, l=\(R_A\)] ++(0,-3) coordinate (node7) -- (C.pin 7);
    \draw (node7) -- (C.pin 6);
    \draw (node7) to[C, l=\(C\)] ++(0,-2) node[ground]{};
    
    % Trigger
    \draw (C.pin 2) -- ++(-1,0) to[sV, l=Trig] ++(0,-2) node[ground]{};
    
    % Control Voltage
    \draw (C.pin 5) to[C, l=0.01\(\mu\)F] ++(0,-1) node[ground]{};
    
    % Output
    \draw (C.pin 3) -- ++(1,0) node[right]{Output};
    
    \node at (C.bpin 1) [left, font=\tiny] {GND};
    \node at (C.bpin 2) [left, font=\tiny] {TRIG};
    \node at (C.bpin 3) [right, font=\tiny] {OUT};
    \node at (C.bpin 4) [left, font=\tiny] {RST};
    \node at (C.bpin 5) [right, font=\tiny] {CV};
    \node at (C.bpin 6) [right, font=\tiny] {THR};
    \node at (C.bpin 7) [right, font=\tiny] {DIS};
    \node at (C.bpin 8) [left, font=\tiny] {VCC};
\end{circuitikz}
\caption{મોનોસ્ટેબલ મલ્ટીવાઈબ્રેટર સર્કિટ}
\end{figure}

\paragraph{કાર્યપદ્ધતિ:}
\begin{enumerate}
    \item \textbf{સ્થિર સ્થિતિ (Low):} શરૂઆતમાં, આઉટપુટ લો (Low) હોય છે. ડિસ્ચાર્જ પિન (7) આંતરિક રીતે ગ્રાઉન્ડ સાથે જોડાય છે, કેપેસિટર C ને ડિસ્ચાર્જ રાખે છે.
    \item \textbf{ટ્રિગરિંગ:} જ્યારે પિન 2 પર નેગેટિવ ટ્રિગર પલ્સ (\(1/3 V_{cc}\) કરતા ઓછું) લાગુ કરવામાં આવે છે, ત્યારે આંતરિક ફ્લિપ-ફ્લોપ સેટ થાય છે.
    \item \textbf{અર્ધ-સ્થિર સ્થિતિ (High):} આઉટપુટ હાઈ (High) થાય છે. પિન 7 ખુલે છે (ઓપન સર્કિટ), જેનાથી કેપેસિટર C રેઝિસ્ટર \(R_A\) દ્વારા ચાર્જ થવાનું શરૂ કરે છે.
    \item \textbf{રીસેટ:} જ્યારે કેપેસિટર વોલ્ટેજ \(2/3 V_{cc}\) સુધી પહોંચે છે, ત્યારે થ્રેશોલ્ડ પિન (6) આંતરિક ફ્લિપ-ફ્લોપને રીસેટ કરે છે. આઉટપુટ લો (Low) થાય છે, અને પિન 7 કેપેસિટરને તરત જ ડિસ્ચાર્જ કરે છે.
\end{enumerate}

\paragraph{પલ્સ પહોળાઈ સૂત્ર:}
આઉટપુટ પલ્સનો સમયગાળો (\(T\)) ટાઈમ કોન્સ્ટન્ટ \(R_A C\) દ્વારા નક્કી થાય છે:
\[ T = 1.1 \times R_A \times C \]

\paragraph{મેમરી ટ્રીક:}
\emph{One Shot. Trigger Low -> High -> Charge C -> 2/3 Vcc -> Reset.}

% ========================================
% QUESTION 5(a) OR: IC 555 Applications (3 marks)
% Demonstrates: List
% ========================================

\subsection{Question 5(a) OR [3 marks]}
\textbf{IC 555 ના ઉપયોગો (Applications) લખો.}

\subsubsection{Solution}
IC 555 ની વૈવિધ્યતા તેને ઘણી એપ્લિકેશનો માટે યોગ્ય બનાવે છે:

\begin{enumerate}
    \item \textbf{Timing Applications (Monostable Mode):}
    \begin{itemize}
        \item Delay Timers: ચોક્કસ વિલંબ પછી ઉપકરણોને ચાલુ/બંધ કરવા માટે વપરાય છે.
        \item Pulse Generation: લોજિક સર્કિટ માટે ચોક્કસ પલ્સ બનાવવા.
        \item Missing Pulse Detector: આવર્તક ઇનપુટ પલ્સ ખૂટે છે કે કેમ તે શોધે છે.
    \end{itemize}
    \item \textbf{Waveform Generation (Astable Mode):}
    \begin{itemize}
        \item Square Wave Generator: ડિજિટલ સર્કિટ માટે ક્લોક સ્ત્રોત તરીકે વપરાય છે.
        \item Ramp Generator: કોન્સ્ટન્ટ કરંટ સોર્સ ચાર્જિંગ કરંટનો ઉપયોગ કરીને.
        \item Tone Generator: ઓડિયો ટોન ઉત્પન્ન કરવા માટે એલાર્મ સર્કિટમાં વપરાય છે.
    \end{itemize}
    \item \textbf{Power Electronics:}
    \begin{itemize}
        \item PWM (Pulse Width Modulation) Controller: DC મોટર્સની ઝડપ અથવા LED ની તેજસ્વીતાને નિયંત્રિત કરવા માટે વપરાય છે.
        \item DC-DC Converters: વોલ્ટેજ વધારવા માટે ચાર્જ પંપ સર્કિટમાં ઉપયોગ કરી શકાય છે.
    \end{itemize}
    \item \textbf{Others:}
    \begin{itemize}
        \item Burglar Alarms: સુરક્ષા માટેના સાદા સર્કિટ્સ.
        \item Traffic Light Control: વિવિધ વિલંબ સાથે ટ્રાફિક લાઇટને અનુક્રમમાં કરવા.
        \item Frequency Divider: ઇનપુટ ફ્રીક્વન્સીને વિભાજીત કરવા માટે વપરાય છે.
    \end{itemize}
\end{enumerate}

\paragraph{મેમરી ટ્રીક:}
\emph{Time, Wave, PWM, Alarm.}


% ========================================
% QUESTION 5(b) OR: IC 555 Block Diagram (4 marks)
% Demonstrates: Block Diagram
% ========================================

\subsection{Question 5(b) OR [4 marks]}
\textbf{IC 555 નો બ્લોક ડાયાગ્રામ દોરો.}

\subsubsection{Solution}
આંતરિક બંધારણમાં બે કમ્પેરેટર, એક ફ્લિપ-ફ્લોપ, ડિસ્ચાર્જ ટ્રાન્ઝિસ્ટર અને વોલ્ટેજ ડિવાઈડરનો સમાવેશ થાય છે.

\begin{figure}[H]
\centering
\begin{tikzpicture}[scale=0.8]
    % Voltage Divider
    \draw (0,0) node[ground] {} to[R, l=5k\(\Omega\)] (0,2) to[R, l=5k\(\Omega\)] (0,4) to[R, l=5k\(\Omega\)] (0,6) node[vcc] {8 (Vcc)};
    
    % Comparators
    \draw (3,1.5) node[op amp] (C2) {C2}; % Trigger
    \draw (3,4.5) node[op amp] (C1) {C1}; % Threshold
    
    % Connections to Comparators
    \draw (C2.+) -- ++(-0.5,0) -- ++(0,-1.5) node[left] {2 (Trig)};
    \draw (0,2) -- (C2.-);
    \draw (C1.-) -- ++(-0.5,0) -- ++(0,1.5) node[left] {6 (Thres)};
    \draw (0,4) -- (C1.+);
    
    % Flip Flop (SR Latch)
    \draw (6,2) rectangle (8,4);
    \node at (7,3) {SR Latch};
    \node at (6.2, 3.5) {R};
    \node at (6.2, 2.5) {S};
    \node at (7.8, 3.5) {\(\bar{Q}\)};
    
    % Connections to FF
    \draw (C1.out) -- (6, 3.5); % R
    \draw (C2.out) -- (6, 2.5); % S
    \draw (6, 4.5) -- (7, 4.5) -- (7, 4) node[above left] {Reset}; 
    
    % Output Stage
    \draw (8, 3.5) -- (9, 3.5) node[buffer] (buf) {}; % Inverting buffer usually
    \draw (buf.out) -- (10.5, 3.5) node[right] {3 (Out)};
    
    % Discharge Stage
    \draw (8, 3.5) -- (8.5, 3.5) -- (8.5, 1) node[npn] (T) {};
    \draw (T.E) node[ground] {};
    \draw (T.C) -- (8.5, 2) node[right] {7 (Disch)};
    
\end{tikzpicture}
\caption{IC 555 ફંક્શનલ બ્લોક ડાયાગ્રામ}
\end{figure}

\paragraph{મેમરી ટ્રીક:}
\emph{3 Resistors (5k), 2 Comparators, 1 Flip-Flop, 1 Transistor (Discharge).}


% ========================================
% QUESTION 5(c) OR: Astable Multivibrator (7 marks)
% Demonstrates: Circuit, Operation
% ========================================

\subsection{Question 5(c) OR [7 marks]}
\textbf{IC 555 નો ઉપયોગ કરીને અસ્ટેબલ મલ્ટીવાઈબ્રેટર દોરો અને સમજાવો.}

\subsubsection{Solution}
અસ્ટેબલ મલ્ટીવાઈબ્રેટરમાં કોઈ સ્થિર અવસ્થા હોતી નથી. તે સતત High અને Low સ્થિતિઓ વચ્ચે સ્વિચ કરે છે, સ્ક્વેર વેવ આઉટપુટ ઉત્પન્ન કરે છે (ફ્રી-રનિંગ ઓસિલેટર).

\paragraph{સર્કિટ ડાયાગ્રામ:}
\begin{figure}[H]
\centering
\begin{circuitikz}[scale=1]
    \draw (0,0) node[dipchip, num pins=8, hide numbers, no topmark, external pins width=0](C){IC 555};
    
    % Pin Connections
    \draw (C.pin 1) -- ++(-0.5,0) node[ground]{} node[left, pos=1]{1 (GND)};
    \draw (C.pin 8) -- ++(0,1) node[vcc]{Vcc} node[right, pos=0.1]{8};
    \draw (C.pin 4) -- ++(0,0.5) -- (C.pin 8); % Reset to Vcc
    
    % Timing RA, RB, C
    \draw (C.pin 8) ++ (2,1.5) node[vcc]{Vcc} to[R, l=\(R_A\)] ++(0,-1.5) coordinate (node7) -- (C.pin 7);
    \draw (node7) to[R, l=\(R_B\)] ++(0,-1.5) coordinate (node62);
    \draw (node62) to[C, l=\(C\)] ++(0,-1.5) node[ground]{};
    
    % Connect 6 and 2 to capacitor
    \draw (node62) -- (C.pin 6);
    \draw (node62) -- ++(0.5,0) -- ++(0, -2.5) -- ++(-4,0) |- (C.pin 2);
    
    % Control Voltage
    \draw (C.pin 5) to[C, l=0.01\(\mu\)F] ++(0,-1) node[ground]{};
    
    % Output
    \draw (C.pin 3) -- ++(1,0) node[right]{Output (Square Wave)};
    
    \node at (C.bpin 1) [left, font=\tiny] {GND};
    \node at (C.bpin 2) [left, font=\tiny] {TRIG};
    \node at (C.bpin 3) [right, font=\tiny] {OUT};
    \node at (C.bpin 4) [left, font=\tiny] {RST};
    \node at (C.bpin 5) [right, font=\tiny] {CV};
    \node at (C.bpin 6) [right, font=\tiny] {THR};
    \node at (C.bpin 7) [right, font=\tiny] {DIS};
    \node at (C.bpin 8) [left, font=\tiny] {VCC};
\end{circuitikz}
\caption{અસ્ટેબલ મલ્ટીવાઈબ્રેટર સર્કિટ}
\end{figure}

\paragraph{કાર્યપદ્ધતિ:}
\begin{enumerate}
    \item \textbf{ચાર્જિંગ:} કેપેસિટર C \(R_A\) અને \(R_B\) દ્વારા \(V_{cc}\) તરફ ચાર્જ થાય છે. આઉટપુટ High રહે છે.
    \item \textbf{થ્રેશોલ્ડ:} જ્યારે કેપેસિટર વોલ્ટેજ \(2/3 V_{cc}\) સુધી પહોંચે છે, ત્યારે પિન 6 ફ્લિપ-ફ્લોપને રીસેટ કરે છે. આઉટપુટ Low થાય છે.
    \item \textbf{ડિસ્ચાર્જિંગ:} પિન 7 ગ્રાઉન્ડ સાથે જોડાય છે (ખુલે છે). કેપેસિટર \(R_B\) દ્વારા પિન 7 માં ડિસ્ચાર્જ થાય છે.
    \item \textbf{ટ્રિગર:} જ્યારે કેપેસિટર વોલ્ટેજ \(1/3 V_{cc}\) સુધી ઘટી જાય છે, ત્યારે પિન 2 ફ્લિપ-ફ્લોપને સેટ કરે છે. આઉટપુટ High થાય છે, અને ચક્ર પુનરાવર્તિત થાય છે.
\end{enumerate}

\paragraph{ફ્રીક્વન્સી સૂત્ર:}
\[ f = \frac{1.44}{(R_A + 2R_B)C} \]
\[ \text{High Time } T_{on} = 0.693 (R_A + R_B) C \]
\[ \text{Low Time } T_{off} = 0.693 (R_B) C \]

\paragraph{મેમરી ટ્રીક:}
\emph{\(R_A\) and \(R_B\) charge. Only \(R_B\) discharges. Toggle between 1/3 and 2/3 Vcc.}

\end{document}
