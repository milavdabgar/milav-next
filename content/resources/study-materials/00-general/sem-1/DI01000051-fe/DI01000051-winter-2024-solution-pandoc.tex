\documentclass[10pt,a4paper]{article}

% content/resources/templates/preamble.tex
\usepackage[margin=0.6in]{geometry}
\author{Milav Dabgar}
\usepackage{amsmath,amssymb,amsthm}
\usepackage{booktabs}
\usepackage{multirow}
\usepackage{xcolor}
\usepackage{tcolorbox}
\tcbuselibrary{breakable,skins}
\usepackage[colorlinks=true,linkcolor=blue]{hyperref}
\usepackage{titlesec}
\usepackage{enumitem}
\usepackage{tikz}
\usepackage{pgfplots}
\usepackage{circuitikz}
\usepackage[version=4]{mhchem}
\usepackage{longtable}
\usepackage{array}
\usepackage{float}
\usepackage{caption}
\usepackage{listings}

\lstset{
  basicstyle=\small\ttfamily,
  breaklines=true,
  breakatwhitespace=false,
  postbreak=\mbox{\textcolor{red}{$\hookrightarrow$}\space},
  float=false,
  numbers=left,
  numberstyle=\tiny\color{gray},
  numbersep=10pt,
  xleftmargin=2em,
  keywordstyle=\color{blue},
  commentstyle=\color{green!60!black},
  stringstyle=\color{purple},
  backgroundcolor=\color{gray!5},
  showstringspaces=false,
  tabsize=2,
  captionpos=b,
  keepspaces=true,
  columns=flexible
}

\pgfplotsset{compat=1.18}
\usetikzlibrary{shapes,arrows,positioning,calc,patterns,decorations.pathmorphing,decorations.markings,arrows.meta}

% Color scheme
\definecolor{headcolor}{RGB}{0,102,204}
\definecolor{keycolor}{RGB}{220,20,60}
\definecolor{solutioncolor}{RGB}{34,139,34}
\definecolor{mnemoniccolor}{RGB}{148,0,211}
\definecolor{codecolor}{RGB}{0,0,100}

% Spacing
\setlength{\parskip}{3pt}
\setlist[itemize]{nosep}
\setlist[enumerate]{nosep}

% Title formatting
\titleformat{\section}{\Large\bfseries\color{headcolor}}{\thesection}{1em}{}
\titleformat{\subsection}{\large\bfseries\color{headcolor}}{\thesubsection}{1em}{}

% Pandoc tightlist compatibility
\providecommand{\tightlist}{%
  \setlength{\itemsep}{0pt}\setlength{\parskip}{0pt}}

% Pandoc longtable compatibility
\newcounter{none}
\def\thenone{}


% content/resources/templates/english-boxes.tex
% This file is currently empty - it exists to maintain consistency with the import structure.
% Add custom environments here if needed in the future.


\begin{document}

\begin{center}
{\Huge\bfseries\color{headcolor} Basic Electronics Solutions}\\[5pt]
{\LARGE DI01000051 -- Winter 2024}\\[3pt]
{\large Semester 1 Study Material}\\[3pt]
{\normalsize\textit{Detailed Solutions and Explanations}}
\end{center}

\vspace{10pt}

\subsection*{Question 1(a) [3 marks]}\label{q1a}

\textbf{Define Active and Passive Components with example.}

\begin{solutionbox}


\vspace{-5pt}
\captionof{table}{Active vs Passive Components}
\vspace{-10pt}
\begin{longtable}[]{@{}
  >{\raggedright\arraybackslash}p{(\linewidth - 6\tabcolsep) * \real{0.2500}}
  >{\raggedright\arraybackslash}p{(\linewidth - 6\tabcolsep) * \real{0.2500}}
  >{\raggedright\arraybackslash}p{(\linewidth - 6\tabcolsep) * \real{0.2500}}
  >{\raggedright\arraybackslash}p{(\linewidth - 6\tabcolsep) * \real{0.2500}}@{}}
\toprule\noalign{}
\begin{minipage}[b]{\linewidth}\raggedright
Component Type
\end{minipage} & \begin{minipage}[b]{\linewidth}\raggedright
Definition
\end{minipage} & \begin{minipage}[b]{\linewidth}\raggedright
Power
\end{minipage} & \begin{minipage}[b]{\linewidth}\raggedright
Examples
\end{minipage} \\
\midrule\noalign{}
\endhead
\bottomrule\noalign{}
\endlastfoot
\textbf{Active Components} & Components that can amplify signals and
control current flow & Can provide power gain & Transistor, Diode, IC \\
\textbf{Passive Components} & Components that cannot amplify signals &
Cannot provide power gain & Resistor, Capacitor, Inductor \\
\end{longtable}

\begin{itemize}
\tightlist
\item
  \textbf{Active components}: Control and amplify electrical signals
  using external power
\item
  \textbf{Passive components}: Store or dissipate energy without
  amplification
\end{itemize}

\end{solutionbox}
\begin{mnemonicbox}
``Active Amplifies, Passive Preserves''

\end{mnemonicbox}
\subsection*{Question 1(b) [4 marks]}\label{q1b}

\textbf{Explain construction and working of LDR.}

\begin{solutionbox}

\textbf{Construction:}

\begin{itemize}
\tightlist
\item
  \textbf{Serpentine track} of cadmium sulfide on ceramic substrate
\item
  \textbf{Metal electrodes} at both ends for connections
\item
  \textbf{Protective coating} prevents moisture damage
\end{itemize}

\textbf{Working Principle:}

\begin{verbatim}
    Light ↓
    ┌──────────────┐
    │  CdS Track   │  Resistance decreases
    │ { │}
    │     LDR      │
    └──────────────┘
         │    │
       Terminal Terminal
\end{verbatim}

\begin{itemize}
\tightlist
\item
  \textbf{Light intensity ↑}: Resistance ↓ (conducts more)
\item
  \textbf{Darkness}: Resistance ↑ (conducts less)
\item
  \textbf{Applications}: Street lights, automatic cameras
\end{itemize}

\end{solutionbox}
\begin{mnemonicbox}
``Light Low Resistance''

\end{mnemonicbox}
\subsection*{Question 1(c) [7 marks]}\label{q1c}

\textbf{Define Capacitance and explain Aluminum Electrolytic wet type
capacitor.}

\begin{solutionbox}

\textbf{Capacitance Definition:} Ability to store electrical charge. C =
Q/V (Farads)

\textbf{Aluminum Electrolytic Capacitor:}

\begin{verbatim}
    Positive Terminal
         │
    ┌────┴────┐
    │ Al Foil │  Anode
    │ Oxide   │  Dielectric
    │ Electro │  Cathode
    │ Al Foil │  Negative
    └────┬────┘
         │
    Negative Terminal
\end{verbatim}

\textbf{Construction:}

\begin{itemize}
\tightlist
\item
  \textbf{Anode}: Aluminum foil with oxide layer
\item
  \textbf{Dielectric}: Thin aluminum oxide film
\item
  \textbf{Cathode}: Liquid electrolyte with aluminum foil
\item
  \textbf{Polarity}: Must be connected correctly
\end{itemize}

\textbf{Features:}

\begin{itemize}
\tightlist
\item
  \textbf{High capacitance} values (1µF to 10,000µF)
\item
  \textbf{Polarized} - has positive and negative terminals
\item
  \textbf{Applications}: Power supply filtering, coupling
\end{itemize}

\end{solutionbox}
\begin{mnemonicbox}
``Aluminum Always Amplifies''

\end{mnemonicbox}
\subsection*{Question 1(c OR) [7
marks]}\label{question-1c-or-7-marks}

\textbf{Explain the color band coding method of Resistor. Write color
band of 32 Ω \pm10\% resistance.}

\begin{solutionbox}

\textbf{Color Code Table:}

\begin{longtable}[]{@{}llll@{}}
\toprule\noalign{}
Color & Digit & Multiplier & Tolerance \\
\midrule\noalign{}
\endhead
\bottomrule\noalign{}
\endlastfoot
Black & 0 & 1 & - \\
Brown & 1 & 10 & \pm1\% \\
Red & 2 & 100 & \pm2\% \\
Orange & 3 & 1K & - \\
Yellow & 4 & 10K & - \\
Green & 5 & 100K & \pm0.5\% \\
Blue & 6 & 1M & \pm0.25\% \\
Violet & 7 & 10M & \pm0.1\% \\
Gray & 8 & 100M & \pm0.05\% \\
White & 9 & 1G & - \\
Silver & - & 0.01 & \pm10\% \\
Gold & - & 0.1 & \pm5\% \\
\end{longtable}

\textbf{For 32 Ω \pm10\%:}

\begin{verbatim}
    ┌─────────────────────────────────┐
    │ Orange│Red│Gold│Silver│         │
    │   3   │ 2 │ 0.1│ 10\% │         │
    │   ↓   │ ↓ │  ↓ │  ↓   │         │
    │  1st  │2nd│Mult│Tol   │         │
    └─────────────────────────────────┘
\end{verbatim}

\textbf{Calculation:} 3 \times 2 \times 0.1 = 3.2 \times 10 = 32 Ω

\end{solutionbox}
\begin{mnemonicbox}
``Big Boys Race Our Young Girls But Violet Generally
Wins''

\end{mnemonicbox}
\subsection*{Question 2(a) [3 marks]}\label{q2a}

\textbf{Define following terms: 1) Rectifier 2) Ripple factor 3) Filter}

\begin{solutionbox}

\begin{longtable}[]{@{}ll@{}}
\toprule\noalign{}
Term & Definition \\
\midrule\noalign{}
\endhead
\bottomrule\noalign{}
\endlastfoot
\textbf{Rectifier} & Circuit that converts AC to pulsating DC \\
\textbf{Ripple Factor} & Ratio of AC component to DC component in
output \\
\textbf{Filter} & Circuit that smooths pulsating DC to pure DC \\
\end{longtable}

\begin{itemize}
\tightlist
\item
  \textbf{Rectifier}: Uses diodes to allow current in one direction
\item
  \textbf{Ripple factor}: Lower value means better filtering
\item
  \textbf{Filter}: Uses capacitors/inductors to reduce ripples
\end{itemize}

\end{solutionbox}
\begin{mnemonicbox}
``Rectify Ripples, Filter Fixes''

\end{mnemonicbox}
\subsection*{Question 2(b) [4 marks]}\label{q2b}

\textbf{Draw and explain positive clipper circuit with waveform.}

\begin{solutionbox}

\textbf{Circuit Diagram:}

\begin{verbatim}
    Input ○────┬────○ Output
               │
               D1 ↓ (Diode)
               │
               ○ +V (Clipping Level)
\end{verbatim}

\textbf{Working:}

\begin{itemize}
\tightlist
\item
  \textbf{When Vin \textgreater{} +V}: Diode conducts, output = +V
\item
  \textbf{When Vin \textless{} +V}: Diode off, output follows input
\item
  \textbf{Result}: Clips positive peaks above +V level
\end{itemize}

\textbf{Waveform:}

\begin{verbatim}
    Input     │    Output
              │
        ┌─┐   │      ──V
       ┌┘ └┐  │     ┌──┐
    ───┘   └──┼──   │  │
              │  ───┘  └───
              │
\end{verbatim}

\textbf{Applications}: Signal limiting, protection circuits

\end{solutionbox}
\begin{mnemonicbox}
``Positive Peaks Prevented''

\end{mnemonicbox}
\subsection*{Question 2(c) [7 marks]}\label{q2c}

\textbf{Explain working of full wave rectifier with two diodes.}

\begin{solutionbox}

\textbf{Circuit Diagram:}

\begin{verbatim}
    AC Input    D1 ↗     RL
         ┌─────────────○───┐
         │              │
    {  │ Center{-}tap   │  ○ Output}
         │ Transformer  │
         │              │
         └─────────────○───┘
              D2 ↘     
\end{verbatim}

\textbf{Working:}

\begin{itemize}
\tightlist
\item
  \textbf{Positive half-cycle}: D1 conducts, D2 off
\item
  \textbf{Negative half-cycle}: D2 conducts, D1 off
\item
  \textbf{Both diodes} work alternately
\item
  \textbf{Output frequency} = 2 \times input frequency
\end{itemize}

\textbf{Key Parameters:}

\begin{longtable}[]{@{}ll@{}}
\toprule\noalign{}
Parameter & Value \\
\midrule\noalign{}
\endhead
\bottomrule\noalign{}
\endlastfoot
\textbf{Peak Inverse Voltage} & 2Vm \\
\textbf{Efficiency} & 81.2\% \\
\textbf{Ripple Factor} & 0.48 \\
\textbf{Form Factor} & 1.11 \\
\end{longtable}

\textbf{Advantages:}

\begin{itemize}
\tightlist
\item
  \textbf{Better efficiency} than half-wave
\item
  \textbf{Lower ripple} content
\item
  \textbf{Higher transformer utilization}
\end{itemize}

\end{solutionbox}
\begin{mnemonicbox}
``Two Diodes, Two Halves''

\end{mnemonicbox}
\subsection*{Question 2(a OR) [3
marks]}\label{question-2a-or-3-marks}

\textbf{Define rectifier and write its applications.}

\begin{solutionbox}

\textbf{Definition:} Electronic circuit that converts alternating
current (AC) into direct current (DC) using diodes.

\textbf{Applications:}

\begin{longtable}[]{@{}ll@{}}
\toprule\noalign{}
Application & Use \\
\midrule\noalign{}
\endhead
\bottomrule\noalign{}
\endlastfoot
\textbf{Power Supplies} & DC voltage for electronic circuits \\
\textbf{Battery Chargers} & Converting AC mains to DC \\
\textbf{DC Motors} & Providing DC for motor drives \\
\textbf{Electronic Devices} & Laptops, phones, LED drivers \\
\end{longtable}

\begin{itemize}
\tightlist
\item
  \textbf{Primary function}: AC to DC conversion
\item
  \textbf{Essential component}: In all electronic devices
\end{itemize}

\end{solutionbox}
\begin{mnemonicbox}
``Rectify AC, Deliver DC''

\end{mnemonicbox}
\subsection*{Question 2(b OR) [4
marks]}\label{question-2b-or-4-marks}

\textbf{Explain working of Pi(π) type capacitor filter.}

\begin{solutionbox}

\textbf{Circuit Diagram:}

\begin{verbatim}
    Input   C1    L    C2   Output
    ○──────||────UUU────||───○
           │             │
           │             │
           ○─────────────○
              Ground
\end{verbatim}

\textbf{Working:}

\begin{itemize}
\tightlist
\item
  \textbf{C1}: Filters initial ripples from rectifier
\item
  \textbf{Inductor L}: Opposes current changes, smooths further
\item
  \textbf{C2}: Final filtering for smooth DC output
\item
  \textbf{Combined effect}: Excellent ripple reduction
\end{itemize}

\textbf{Characteristics:}

\begin{longtable}[]{@{}ll@{}}
\toprule\noalign{}
Parameter & Value \\
\midrule\noalign{}
\endhead
\bottomrule\noalign{}
\endlastfoot
\textbf{Ripple Factor} & Very low (\textless{} 0.01) \\
\textbf{Regulation} & Good \\
\textbf{Cost} & Higher due to inductor \\
\textbf{Applications} & High-quality power supplies \\
\end{longtable}

\textbf{Advantages:}

\begin{itemize}
\tightlist
\item
  \textbf{Excellent filtering} performance
\item
  \textbf{Low ripple} content
\item
  \textbf{Good voltage regulation}
\end{itemize}

\end{solutionbox}
\begin{mnemonicbox}
``Pi Provides Perfect''

\end{mnemonicbox}
\subsection*{Question 2(c OR) [7
marks]}\label{question-2c-or-7-marks}

\textbf{Compare half wave and full wave bridge rectifier.}

\begin{solutionbox}

\textbf{Comparison Table:}

\begin{longtable}[]{@{}lll@{}}
\toprule\noalign{}
Parameter & Half Wave & Full Wave Bridge \\
\midrule\noalign{}
\endhead
\bottomrule\noalign{}
\endlastfoot
\textbf{Diodes Required} & 1 & 4 \\
\textbf{Transformer} & Simple & No center-tap needed \\
\textbf{Efficiency} & 40.6\% & 81.2\% \\
\textbf{Ripple Factor} & 1.21 & 0.48 \\
\textbf{PIV} & Vm & Vm \\
\textbf{Output Frequency} & f & 2f \\
\textbf{Transformer Utilization} & 28.7\% & 81.2\% \\
\textbf{Cost} & Low & Moderate \\
\end{longtable}

\textbf{Circuit Diagrams:}

\textbf{Half Wave:}

\begin{verbatim}
    AC ○────D1────○ Output
        │          │
        │    RL    │
        │          │
        ○──────────○
\end{verbatim}

\textbf{Full Wave Bridge:}

\begin{verbatim}
         D1 ↗
    AC ○─────────○ Output
       │    RL   │
       │         │
    AC ○─────────○
         D2 ↘
\end{verbatim}

\textbf{Key Differences:}

\begin{itemize}
\tightlist
\item
  \textbf{Full wave}: Better efficiency and lower ripple
\item
  \textbf{Half wave}: Simpler but poor performance
\item
  \textbf{Bridge}: No center-tap transformer required
\end{itemize}

\end{solutionbox}
\begin{mnemonicbox}
``Half Wastes, Full Works''

\end{mnemonicbox}
\subsection*{Question 3(a) [3 marks]}\label{q3a}

\textbf{Draw the symbols of following: 1) Zener diode 2) LED 3) Varactor
diode}

\begin{solutionbox}

\textbf{Electronic Symbols:}

\begin{verbatim}
    Zener Diode:        LED:            Varactor Diode:
         │                  │                   │
       ──┤►├──           ──┤►├──            ──┤ ├──
         │ Z                │ ↗               │ │ │
                                              │ │ │
                                              ─────
\end{verbatim}

\textbf{Symbol Details:}

\begin{longtable}[]{@{}ll@{}}
\toprule\noalign{}
Component & Symbol Feature \\
\midrule\noalign{}
\endhead
\bottomrule\noalign{}
\endlastfoot
\textbf{Zener Diode} & Normal diode with Z-shaped cathode \\
\textbf{LED} & Diode with arrows showing light emission \\
\textbf{Varactor Diode} & Diode with parallel lines (variable
capacitor) \\
\end{longtable}

\begin{itemize}
\tightlist
\item
  \textbf{Zener}: Z indicates zener characteristics
\item
  \textbf{LED}: Arrows show light output direction
\item
  \textbf{Varactor}: Lines represent variable capacitance
\end{itemize}

\end{solutionbox}
\begin{mnemonicbox}
``Zener Zigs, LED Lights, Varactor Varies''

\end{mnemonicbox}
\subsection*{Question 3(b) [4 marks]}\label{q3b}

\textbf{Explain construction and working of LED.}

\begin{solutionbox}

\textbf{Construction:}

\begin{verbatim}
         Light Output ↑
      ┌──────────────────┐
      │   Wire Bond      │
      │ ┌──────────────┐ │
      │ │  P{-N Junction│ │}
      │ │      │       │ │
      │ └──────┼───────┘ │
      │   Cathode  Anode │
      └──────────────────┘
            LED Chip
\end{verbatim}

\textbf{Materials:}

\begin{itemize}
\tightlist
\item
  \textbf{P-type}: Boron-doped semiconductor
\item
  \textbf{N-type}: Phosphorus-doped semiconductor
\item
  \textbf{Common materials}: GaAs, GaP, GaN
\end{itemize}

\textbf{Working Principle:}

\begin{itemize}
\tightlist
\item
  \textbf{Forward bias}: Electrons recombine with holes
\item
  \textbf{Energy release}: In form of photons (light)
\item
  \textbf{Color}: Depends on semiconductor material and bandgap
\item
  \textbf{Efficiency}: High light output with low power
\end{itemize}

\textbf{Applications:}

\begin{itemize}
\tightlist
\item
  \textbf{Indicators}: Status lights, displays
\item
  \textbf{Lighting}: LED bulbs, strips
\item
  \textbf{Electronics}: Seven-segment displays
\end{itemize}

\end{solutionbox}
\begin{mnemonicbox}
``Light Emitting, Energy Efficient''

\end{mnemonicbox}
\subsection*{Question 3(c) [7 marks]}\label{q3c}

\textbf{Explain working characteristics of Zener diode.}

\begin{solutionbox}

\textbf{V-I Characteristics:}

\begin{verbatim}
                │ Forward
                │   ↗
                │ ↗ If
    ────────────┼───────── V
     Vz  │  │   │
         │  │   │
    ─────┼──┼───┼────
         │  │   │ Reverse
         │  Iz  │
         │      │
       Zener    │
      Region    │
\end{verbatim}

\textbf{Key Regions:}

\begin{longtable}[]{@{}ll@{}}
\toprule\noalign{}
Region & Characteristics \\
\midrule\noalign{}
\endhead
\bottomrule\noalign{}
\endlastfoot
\textbf{Forward Bias} & Normal diode operation (0.7V) \\
\textbf{Reverse Bias} & Small leakage current \\
\textbf{Zener Region} & Constant voltage (Vz) \\
\textbf{Breakdown} & Sharp voltage breakdown \\
\end{longtable}

\textbf{Important Parameters:}

\begin{itemize}
\tightlist
\item
  \textbf{Zener Voltage (Vz)}: Breakdown voltage
\item
  \textbf{Zener Current (Iz)}: Current in breakdown region
\item
  \textbf{Maximum Power}: Vz \times Iz(max)
\item
  \textbf{Temperature coefficient}: Voltage variation with temperature
\end{itemize}

\textbf{Applications:}

\begin{itemize}
\tightlist
\item
  \textbf{Voltage regulation}: Maintains constant output
\item
  \textbf{Reference voltage}: Precise voltage source
\item
  \textbf{Overvoltage protection}: Protects circuits
\end{itemize}

\textbf{Advantages:}

\begin{itemize}
\tightlist
\item
  \textbf{Sharp breakdown}: Well-defined voltage
\item
  \textbf{Low dynamic resistance}: Good regulation
\item
  \textbf{Wide range}: Available in many voltages
\end{itemize}

\end{solutionbox}
\begin{mnemonicbox}
``Zener Zones Zero variation''

\end{mnemonicbox}
\subsection*{Question 3(a OR) [3
marks]}\label{question-3a-or-3-marks}

\textbf{Enlist the applications of varactor diode.}

\begin{solutionbox}

\textbf{Applications Table:}

\begin{longtable}[]{@{}ll@{}}
\toprule\noalign{}
Application & Function \\
\midrule\noalign{}
\endhead
\bottomrule\noalign{}
\endlastfoot
\textbf{Voltage Controlled Oscillators} & Frequency tuning with
voltage \\
\textbf{Automatic Frequency Control} & Maintains oscillator frequency \\
\textbf{Electronic Tuning} & Radio/TV channel selection \\
\textbf{Phase Locked Loops} & Frequency synchronization \\
\textbf{Frequency Multipliers} & Harmonic generation \\
\textbf{Parametric Amplifiers} & Low-noise amplification \\
\end{longtable}

\textbf{Key Features:}

\begin{itemize}
\tightlist
\item
  \textbf{Voltage variable}: Capacitance changes with reverse voltage
\item
  \textbf{No mechanical parts}: Electronic tuning only
\item
  \textbf{Fast response}: Quick frequency changes
\end{itemize}

\end{solutionbox}
\begin{mnemonicbox}
``Voltage Varies Capacitance''

\end{mnemonicbox}
\subsection*{Question 3(b OR) [4
marks]}\label{question-3b-or-4-marks}

\textbf{Explain working of photo diode.}

\begin{solutionbox}

\textbf{Construction \& Symbol:}

\begin{verbatim}
      Light ↓ ↓ ↓
    ┌─────────────┐
    │    P{-type   │  Anode}
    │─────────────│  P{-N Junction}
    │    N{-type   │  Cathode}
    └─────────────┘
         │     │
      Cathode Anode
\end{verbatim}

\textbf{Working Principle:}

\begin{itemize}
\tightlist
\item
  \textbf{Light absorption}: Creates electron-hole pairs
\item
  \textbf{Reverse bias}: Widens depletion region
\item
  \textbf{Photocurrent}: Proportional to light intensity
\item
  \textbf{Fast response}: Quick detection capability
\end{itemize}

\textbf{Characteristics:}

\begin{longtable}[]{@{}ll@{}}
\toprule\noalign{}
Parameter & Description \\
\midrule\noalign{}
\endhead
\bottomrule\noalign{}
\endlastfoot
\textbf{Dark Current} & Current without light \\
\textbf{Photocurrent} & Current proportional to light \\
\textbf{Responsivity} & Current per unit light power \\
\textbf{Response Time} & Speed of detection \\
\end{longtable}

\textbf{Applications:}

\begin{itemize}
\tightlist
\item
  \textbf{Light sensors}: Automatic lighting systems
\item
  \textbf{Optical communication}: Fiber optic receivers
\item
  \textbf{Safety systems}: Smoke detectors
\item
  \textbf{Solar panels}: Light to electrical energy
\end{itemize}

\end{solutionbox}
\begin{mnemonicbox}
``Photo Produces Proportional current''

\end{mnemonicbox}
\subsection*{Question 3(c OR) [7
marks]}\label{question-3c-or-7-marks}

\textbf{Explain Zener diode as a voltage regulator.}

\begin{solutionbox}

\textbf{Voltage Regulator Circuit:}

\begin{verbatim}
    Vin ○──Rs──┬────○ Vout = Vz
               │
               Z ↓ (Zener)
               │
               ○ Ground
\end{verbatim}

\textbf{Working Principle:}

\begin{itemize}
\tightlist
\item
  \textbf{Zener operates} in breakdown region
\item
  \textbf{Output voltage} remains constant at Vz
\item
  \textbf{Series resistor Rs} limits current
\item
  \textbf{Load changes} don't affect output voltage
\end{itemize}

\textbf{Design Equations:}

\begin{longtable}[]{@{}ll@{}}
\toprule\noalign{}
Parameter & Formula \\
\midrule\noalign{}
\endhead
\bottomrule\noalign{}
\endlastfoot
\textbf{Series Resistance} & Rs = (Vin - Vz) / Iz \\
\textbf{Load Current} & IL = Vz / RL \\
\textbf{Zener Current} & Iz = Is - IL \\
\textbf{Power Dissipation} & Pz = Vz \times Iz \\
\end{longtable}

\textbf{Regulation Characteristics:}

\begin{itemize}
\tightlist
\item
  \textbf{Line regulation}: Output change with input variation
\item
  \textbf{Load regulation}: Output change with load variation
\item
  \textbf{Efficiency}: Generally low due to Zener power loss
\end{itemize}

\textbf{Advantages:}

\begin{itemize}
\tightlist
\item
  \textbf{Simple circuit}: Few components required
\item
  \textbf{Good regulation}: Stable output voltage
\item
  \textbf{Fast response}: Quick voltage correction
\end{itemize}

\textbf{Limitations:}

\begin{itemize}
\tightlist
\item
  \textbf{Poor efficiency}: Power wasted in Zener
\item
  \textbf{Limited current}: Cannot supply high currents
\item
  \textbf{Temperature sensitivity}: Voltage varies with temperature
\end{itemize}

\textbf{Applications:}

\begin{itemize}
\tightlist
\item
  \textbf{Reference voltage}: Precise voltage source
\item
  \textbf{Simple regulators}: Low current applications
\item
  \textbf{Protection circuits}: Overvoltage protection
\end{itemize}

\end{solutionbox}
\begin{mnemonicbox}
``Zener Zones provide Zero variation''

\end{mnemonicbox}
\subsection*{Question 4(a) [3 marks]}\label{q4a}

\textbf{Draw the symbol and construction of PNP and NPN transistor with
proper notation.}

\begin{solutionbox}

\textbf{Transistor Symbols:}

\begin{verbatim}
    NPN Transistor:        PNP Transistor:
    
    Collector (C)          Collector (C)
         │                      │
         ○                      ○
         │                      │
    Base ○─┤                Base ○─┤
         │ ↘                   │  ↙
         ○ Emitter              ○ Emitter
         │ (E)                  │ (E)
\end{verbatim}

\textbf{Construction Diagrams:}

\begin{verbatim}
    NPN Structure:         PNP Structure:
    
    ○ Collector            ○ Collector  
    │ N{-type               │ P{-}type}
    ├─────────             ├─────────
    │ P{-type    Base ○     │ N{-}type    Base ○}
    ├─────────             ├─────────
    │ N{-type               │ P{-}type}
    ○ Emitter              ○ Emitter
\end{verbatim}

\textbf{Terminal Identification:}

\begin{itemize}
\tightlist
\item
  \textbf{Emitter}: Heavily doped, arrow shows current direction
\item
  \textbf{Base}: Thin, lightly doped middle region
\item
  \textbf{Collector}: Moderately doped, collects charge carriers
\end{itemize}

\textbf{Current Direction:}

\begin{itemize}
\tightlist
\item
  \textbf{NPN}: Arrow points outward (emitter to base)
\item
  \textbf{PNP}: Arrow points inward (base to emitter)
\end{itemize}

\end{solutionbox}
\begin{mnemonicbox}
``NPN: Not Pointing iN, PNP: Pointing iN Please''

\end{mnemonicbox}
\subsection*{Question 4(b) [4 marks]}\label{q4b}

\textbf{Draw and Explain characteristics of CE amplifier.}

\begin{solutionbox}

\textbf{CE Amplifier Circuit:}

\begin{verbatim}
    Vcc
     │
     Rc
     │
    ○ Vout
     │
    ─C─  Collector
     │
    ─B─  Base ○ Vin
     │
    ─E─  Emitter
     │
     Re
     │
    ○ Ground
\end{verbatim}

\textbf{Input Characteristics (IB vs VBE):}

\begin{verbatim}
    IB │
    (mA)│     ┌──
        │   ┌─┘
        │ ┌─┘
        │┌┘
        └──────── VBE (V)
         0  0.7
\end{verbatim}

\textbf{Output Characteristics (IC vs VCE):}

\begin{verbatim}
    IC │  IB = 40µA
    (mA)│  ┌──────────
        │  │ IB = 30µA
        │ ┌┴──────────
        │ │ IB = 20µA  
        │┌┴───────────
        ││ IB = 10µA
        └┴─────────── VCE (V)
         0    5   10
\end{verbatim}

\textbf{Key Features:}

\begin{longtable}[]{@{}ll@{}}
\toprule\noalign{}
Parameter & CE Configuration \\
\midrule\noalign{}
\endhead
\bottomrule\noalign{}
\endlastfoot
\textbf{Current Gain} & β = IC/IB (high) \\
\textbf{Voltage Gain} & High \\
\textbf{Power Gain} & Very high \\
\textbf{Input Impedance} & Medium \\
\textbf{Output Impedance} & High \\
\textbf{Phase Shift} & 180^\circ \\
\end{longtable}

\textbf{Regions of Operation:}

\begin{itemize}
\tightlist
\item
  \textbf{Cut-off}: Both junctions reverse biased
\item
  \textbf{Active}: BE forward, BC reverse biased
\item
  \textbf{Saturation}: Both junctions forward biased
\end{itemize}

\end{solutionbox}
\begin{mnemonicbox}
``Common Emitter, Current Enlarged''

\end{mnemonicbox}
\subsection*{Question 4(c) [7 marks]}\label{q4c}

\textbf{Derive relation between current gains α, β and γ.}

\begin{solutionbox}

\textbf{Current Gain Definitions:}

\begin{longtable}[]{@{}lll@{}}
\toprule\noalign{}
Gain & Configuration & Formula \\
\midrule\noalign{}
\endhead
\bottomrule\noalign{}
\endlastfoot
\textbf{α (Alpha)} & Common Base & α = IC/IE \\
\textbf{β (Beta)} & Common Emitter & β = IC/IB \\
\textbf{γ (Gamma)} & Common Collector & γ = IE/IB \\
\end{longtable}

\textbf{Derivation:}

\textbf{Step 1: Basic Current Relation} IE = IB + IC \ldots{}
(Kirchhoff's Current Law)

\textbf{Step 2: Express IC in terms of IE} α = IC/IE Therefore: IC = α \times
IE \ldots{} (1)

\textbf{Step 3: Substitute in current equation} IE = IB + α \times IE IE - α
\times IE = IB IE(1 - α) = IB IE = IB/(1 - α) \ldots{} (2)

\textbf{Step 4: Find β} β = IC/IB From (1): IC = α \times IE From (2): IE =
IB/(1 - α) Therefore: IC = α \times IB/(1 - α)

\textbf{Step 5: Final relation for β} β = IC/IB = α/(1 - α) \ldots{} (3)

\textbf{Step 6: Express α in terms of β} From equation (3): β(1 - α) = α
β - βα = α

β = α + βα = α(1 + β) Therefore:

α = β/(1 + β) \ldots{} (4)


\textbf{Step 7: Find γ} γ = IE/IB From (2): γ = 1/(1 - α) Substituting α
from (4):

γ = 1/(1 - β/(1 + β))

γ = (1 + β)/(1 + β - β)

γ = 1 + β

\ldots{} (5)

\textbf{Final Relations:}

\begin{longtable}[]{@{}ll@{}}
\toprule\noalign{}
Relation & Formula \\
\midrule\noalign{}
\endhead
\bottomrule\noalign{}
\endlastfoot
\textbf{β in terms of α} & β = α/(1 - α) \\
\textbf{α in terms of β} & α = β/(1 + β) \\
\textbf{γ in terms of β} & γ = 1 + β \\
\textbf{Verification} & α + β \times α = β \\
\end{longtable}

\textbf{Typical Values:}

\begin{itemize}
\tightlist
\item
  α \approx 0.98 to 0.995
\item
  β \approx 50 to 200\\
\item
  γ \approx 51 to 201
\end{itemize}

\end{solutionbox}
\begin{mnemonicbox}
``Alpha Beta Gamma, Always Better Gains''

\end{mnemonicbox}
\subsection*{Question 4(a OR) [3
marks]}\label{question-4a-or-3-marks}

\textbf{Define Active, Saturation and Cut-off region for transistor
amplifier.}

\begin{solutionbox}

\textbf{Operating Regions:}

\begin{longtable}[]{@{}llll@{}}
\toprule\noalign{}
Region & Base-Emitter & Base-Collector & Characteristics \\
\midrule\noalign{}
\endhead
\bottomrule\noalign{}
\endlastfoot
\textbf{Active} & Forward Biased & Reverse Biased & Amplification
region \\
\textbf{Saturation} & Forward Biased & Forward Biased & Switch ON
state \\
\textbf{Cut-off} & Reverse Biased & Reverse Biased & Switch OFF state \\
\end{longtable}

\textbf{Detailed Description:}

\textbf{Active Region:}

\begin{itemize}
\tightlist
\item
  \textbf{Normal amplification} mode
\item
  \textbf{IC = β \times IB} relationship holds
\item
  \textbf{Linear operation} for small signals
\end{itemize}

\textbf{Saturation Region:}

\begin{itemize}
\tightlist
\item
  \textbf{Both junctions} forward biased
\item
  \textbf{Maximum collector current} flows
\item
  \textbf{VCE \approx 0.2V} (very low)
\item
  \textbf{Used in switching} applications
\end{itemize}

\textbf{Cut-off Region:}

\begin{itemize}
\tightlist
\item
  \textbf{No base current} (IB = 0)
\item
  \textbf{No collector current} (IC = 0)\\
\item
  \textbf{Transistor acts} like open switch
\end{itemize}

\end{solutionbox}
\begin{mnemonicbox}
``Active Amplifies, Saturated Switches, Cut-off
Cuts''

\end{mnemonicbox}
\subsection*{Question 4(b OR) [4
marks]}\label{question-4b-or-4-marks}

\textbf{Explain working of Transistor as an amplifier.}

\begin{solutionbox}

\textbf{Amplifier Circuit:}

\begin{verbatim}
    Vcc
     │
     Rc
     │  
    ○ Vout (amplified)
     │
    ─C─  NPN Transistor
     │
    ○ Vin ─B─
     │
    ─E─
     │
     Re
     │
    ○ Ground
\end{verbatim}

\textbf{Working Principle:}

\begin{itemize}
\tightlist
\item
  \textbf{Small input signal} applied to base-emitter
\item
  \textbf{Input resistance} is low (few kΩ)
\item
  \textbf{Small base current} controls large collector current
\item
  \textbf{Output taken} from collector-emitter
\item
  \textbf{Current amplification}: IC = β \times IB
\end{itemize}

\textbf{Amplification Process:}

\begin{longtable}[]{@{}lll@{}}
\toprule\noalign{}
Parameter & Input & Output \\
\midrule\noalign{}
\endhead
\bottomrule\noalign{}
\endlastfoot
\textbf{Signal Level} & Small & Large \\
\textbf{Current} & µA range & mA range \\
\textbf{Voltage} & mV range & V range \\
\textbf{Power} & µW range & mW range \\
\end{longtable}

\textbf{Key Features:}

\begin{itemize}
\tightlist
\item
  \textbf{Current gain}: β (50-200 typical)
\item
  \textbf{Voltage gain}: Depends on load resistance
\item
  \textbf{Power gain}: Product of current and voltage gains
\item
  \textbf{Phase inversion}: 180^\circ in CE configuration
\end{itemize}

\textbf{Applications:}

\begin{itemize}
\tightlist
\item
  \textbf{Audio amplifiers}: Music systems
\item
  \textbf{RF amplifiers}: Radio transmitters
\item
  \textbf{Op-amp stages}: Integrated circuits
\end{itemize}

\end{solutionbox}
\begin{mnemonicbox}
``Tiny signal Triggers Tremendous output''

\end{mnemonicbox}
\subsection*{Question 4(c OR) [7
marks]}\label{question-4c-or-7-marks}

\textbf{Compare CB, CC, and CE amplifier configuration.}

\begin{solutionbox}

\textbf{Comprehensive Comparison:}

\begin{longtable}[]{@{}
  >{\raggedright\arraybackslash}p{(\linewidth - 6\tabcolsep) * \real{0.2500}}
  >{\raggedright\arraybackslash}p{(\linewidth - 6\tabcolsep) * \real{0.2500}}
  >{\raggedright\arraybackslash}p{(\linewidth - 6\tabcolsep) * \real{0.2500}}
  >{\raggedright\arraybackslash}p{(\linewidth - 6\tabcolsep) * \real{0.2500}}@{}}
\toprule\noalign{}
\begin{minipage}[b]{\linewidth}\raggedright
Parameter
\end{minipage} & \begin{minipage}[b]{\linewidth}\raggedright
Common Base (CB)
\end{minipage} & \begin{minipage}[b]{\linewidth}\raggedright
Common Emitter (CE)
\end{minipage} & \begin{minipage}[b]{\linewidth}\raggedright
Common Collector (CC)
\end{minipage} \\
\midrule\noalign{}
\endhead
\bottomrule\noalign{}
\endlastfoot
\textbf{Input Terminal} & Emitter & Base & Base \\
\textbf{Output Terminal} & Collector & Collector & Emitter \\
\textbf{Common Terminal} & Base & Emitter & Collector \\
\textbf{Current Gain} & α \textless{} 1 & β \textgreater\textgreater{} 1
& γ = (1 + β) \\
\textbf{Voltage Gain} & High & High & \textless{} 1 (\approx1) \\
\textbf{Power Gain} & Medium & Very High & Medium \\
\textbf{Input Resistance} & Very Low (20-50Ω) & Medium (1-5kΩ) & Very
High (100kΩ) \\
\textbf{Output Resistance} & Very High (1MΩ) & High (50kΩ) & Low
(25Ω) \\
\textbf{Phase Shift} & 0^\circ & 180^\circ & 0^\circ \\
\textbf{Frequency Response} & Excellent & Good & Good \\
\textbf{Applications} & RF Amplifiers & Audio Amplifiers & Buffer,
Impedance Matching \\
\end{longtable}

\textbf{Circuit Diagrams:}

\textbf{Common Base:}

\begin{verbatim}
    Vcc          Vcc          Vcc
     │            │            │
     Rc           Rc           Re
     │            │            │
    ○Vout       ○Vout        ○Vin
     │            │            │
    ─C─          ─C─          ─C─
     │            │            │
    ─B─○Ground   ○Vin─B─      ─B─
     │            │            │
    ○Vin─E─      ─E─○Ground   ─E─○Vout
\end{verbatim}

\textbf{Key Characteristics:}

\textbf{Common Base (CB):}

\begin{itemize}
\tightlist
\item
  \textbf{High frequency} performance
\item
  \textbf{No current gain} but high voltage gain
\item
  \textbf{Input-output isolation} excellent
\item
  \textbf{Used in}: RF amplifiers, high-frequency circuits
\end{itemize}

\textbf{Common Emitter (CE):}

\begin{itemize}
\tightlist
\item
  \textbf{Most popular} configuration
\item
  \textbf{High current and voltage} gain
\item
  \textbf{Good compromise} of all parameters
\item
  \textbf{Used in}: Audio amplifiers, general amplification
\end{itemize}

\textbf{Common Collector (CC):}

\begin{itemize}
\tightlist
\item
  \textbf{Unity voltage gain} (voltage follower)
\item
  \textbf{High current gain}
\item
  \textbf{Impedance transformation} (high to low)
\item
  \textbf{Used in}: Buffer amplifiers, impedance matching
\end{itemize}

\textbf{Selection Criteria:}

\begin{longtable}[]{@{}lll@{}}
\toprule\noalign{}
Application & Best Configuration & Reason \\
\midrule\noalign{}
\endhead
\bottomrule\noalign{}
\endlastfoot
\textbf{High Frequency} & CB & Excellent frequency response \\
\textbf{General Amplification} & CE & High power gain \\
\textbf{Buffer/Isolation} & CC & High input, low output impedance \\
\textbf{Power Amplifiers} & CE & Maximum power gain \\
\end{longtable}

\end{solutionbox}
\begin{mnemonicbox}
``CB for Communication, CE for Common use, CC for
Coupling''

\end{mnemonicbox}
\subsection*{Question 5(a) [3 marks]}\label{q5a}

\textbf{Draw the pin diagram of IC 555.}

\begin{solutionbox}

\textbf{IC 555 Pin Diagram:}

\begin{verbatim}
    ┌─────────────────┐
    │    IC 555       │
  1 │○ Ground         │ 8 ○ Vcc
  2 │○ Trigger        │ 7 ○ Discharge  
  3 │○ Output         │ 6 ○ Threshold
  4 │○ Reset          │ 5 ○ Control Voltage
    └─────────────────┘
          DIP{-8 Package}
\end{verbatim}

\textbf{Pin Functions:}

\begin{longtable}[]{@{}lll@{}}
\toprule\noalign{}
Pin & Name & Function \\
\midrule\noalign{}
\endhead
\bottomrule\noalign{}
\endlastfoot
\textbf{1} & Ground & 0V reference \\
\textbf{2} & Trigger & Start timing cycle \\
\textbf{3} & Output & Timer output \\
\textbf{4} & Reset & Master reset (active low) \\
\textbf{5} & Control & Voltage reference control \\
\textbf{6} & Threshold & Stop timing cycle \\
\textbf{7} & Discharge & Timing capacitor discharge \\
\textbf{8} & Vcc & Power supply (+5V to +18V) \\
\end{longtable}

\textbf{Key Points:}

\begin{itemize}
\tightlist
\item
  \textbf{Dual-in-line} 8-pin package
\item
  \textbf{Power supply}: 5V to 18V DC
\item
  \textbf{Output current}: Up to 200mA
\item
  \textbf{Reset pin}: Normally connected to Vcc
\end{itemize}

\end{solutionbox}
\begin{mnemonicbox}
``Great Timer, Great Pins''

\end{mnemonicbox}
\subsection*{Question 5(b) [4 marks]}\label{q5b}

\textbf{List out Features of 555 Timer IC.}

\begin{solutionbox}

\textbf{Key Features:}

\begin{longtable}[]{@{}ll@{}}
\toprule\noalign{}
Feature & Specification \\
\midrule\noalign{}
\endhead
\bottomrule\noalign{}
\endlastfoot
\textbf{Supply Voltage} & 5V to 18V \\
\textbf{Output Current} & 200mA source/sink \\
\textbf{Temperature Range} & 0^\circC to 70^\circC \\
\textbf{Timing Range} & µs to hours \\
\textbf{Accuracy} & \pm1\% typical \\
\textbf{Modes} & Monostable, Astable, Bistable \\
\end{longtable}

\textbf{Technical Features:}

\begin{itemize}
\tightlist
\item
  \textbf{CMOS/TTL compatible} output levels
\item
  \textbf{High current} output capability
\item
  \textbf{Wide supply voltage} range
\item
  \textbf{Temperature stable} operation
\end{itemize}

\textbf{Functional Features:}

\begin{itemize}
\tightlist
\item
  \textbf{Three operating modes} available
\item
  \textbf{External timing} components
\item
  \textbf{Reset capability} for control
\item
  \textbf{Low power consumption} design
\end{itemize}

\textbf{Advantages:}

\begin{itemize}
\tightlist
\item
  \textbf{Versatile timer} for multiple applications
\item
  \textbf{Easy to use} with minimal external components
\item
  \textbf{Reliable operation} in various conditions
\end{itemize}

\end{solutionbox}
\begin{mnemonicbox}
``Fantastic Features, Flexible Functions''

\end{mnemonicbox}
\subsection*{Question 5(c) [7 marks]}\label{q5c}

\textbf{Explain Mono stable multivibrator using 555 timer IC.}

\begin{solutionbox}

\textbf{Monostable Circuit:}

\begin{verbatim}
    Vcc
     │
     ├─────○ 8 (Vcc)
     │
     R ────○ 7 (Discharge)
     │     │
     ├─────○ 6 (Threshold)
     │     │
    ○2○────┤ 4 (Reset)
     │  5○─┴─ (Control)
    ○3○     │
     │     ○ 1 (Ground)
     │     │
     C ────┘
     │
    ○ Ground
\end{verbatim}

\textbf{Working Principle:}

\textbf{Stable State:}

\begin{itemize}
\tightlist
\item
  \textbf{Output LOW} (approximately 0V)
\item
  \textbf{Capacitor discharged} through pin 7
\item
  \textbf{Threshold voltage} below Vcc/3
\end{itemize}

\textbf{Triggered State:}

\begin{itemize}
\tightlist
\item
  \textbf{Negative pulse} applied to trigger (pin 2)
\item
  \textbf{Output goes HIGH} immediately
\item
  \textbf{Discharge transistor} turns OFF
\item
  \textbf{Capacitor starts} charging through R
\end{itemize}

\textbf{Timing Period:}

\begin{itemize}
\tightlist
\item
  \textbf{Duration}: T = 1.1 \times R \times C
\item
  \textbf{Output remains HIGH} for calculated time
\item
  \textbf{Automatic return} to stable state
\end{itemize}

\textbf{Return to Stable:}

\begin{itemize}
\tightlist
\item
  \textbf{Capacitor voltage} reaches 2Vcc/3
\item
  \textbf{Threshold triggered} (pin 6)
\item
  \textbf{Output returns} to LOW
\item
  \textbf{Discharge begins} again
\end{itemize}

\textbf{Key Characteristics:}

\begin{longtable}[]{@{}ll@{}}
\toprule\noalign{}
Parameter & Description \\
\midrule\noalign{}
\endhead
\bottomrule\noalign{}
\endlastfoot
\textbf{Pulse Width} & T = 1.1 RC \\
\textbf{Trigger Level} & Vcc/3 \\
\textbf{Threshold Level} & 2Vcc/3 \\
\textbf{Output High} & \textasciitilde Vcc - 1.5V \\
\textbf{Output Low} & \textasciitilde0.1V \\
\end{longtable}

\textbf{Applications:}

\begin{itemize}
\tightlist
\item
  \textbf{Pulse generation}: Fixed width pulses
\item
  \textbf{Time delays}: Switch-on delays
\item
  \textbf{Missing pulse detection}: Watchdog timers
\item
  \textbf{Debouncing circuits}: Switch contact cleaning
\end{itemize}

\textbf{Design Example:} For T = 1ms: If C = 0.1µF, then R = 9.1kΩ

\end{solutionbox}
\begin{mnemonicbox}
``Mono means One pulse Only''

\end{mnemonicbox}
\subsection*{Question 5(a OR) [3
marks]}\label{question-5a-or-3-marks}

\textbf{List out applications of IC 555.}

\begin{solutionbox}

\textbf{Timer Applications:}

\begin{longtable}[]{@{}ll@{}}
\toprule\noalign{}
Category & Applications \\
\midrule\noalign{}
\endhead
\bottomrule\noalign{}
\endlastfoot
\textbf{Timing Circuits} & Delay timers, Pulse generators \\
\textbf{Oscillators} & Clock generators, Frequency dividers \\
\textbf{Control Circuits} & PWM controllers, Motor speed control \\
\textbf{Detection} & Missing pulse detectors, Burglar alarms \\
\textbf{Communication} & Tone generators, Frequency shift keying \\
\textbf{Automotive} & Turn signal flashers, Windshield wipers \\
\end{longtable}

\textbf{Mode-wise Applications:}

\textbf{Monostable Mode:}

\begin{itemize}
\tightlist
\item
  \textbf{Time delays} in circuits
\item
  \textbf{Pulse width} generation
\item
  \textbf{Debouncing} switches
\end{itemize}

\textbf{Astable Mode:}

\begin{itemize}
\tightlist
\item
  \textbf{LED flashers} and blinkers
\item
  \textbf{Clock signals} generation
\item
  \textbf{Tone generation} for buzzers
\end{itemize}

\textbf{Bistable Mode:}

\begin{itemize}
\tightlist
\item
  \textbf{Flip-flop} circuits
\item
  \textbf{Memory elements}
\item
  \textbf{Latch circuits}
\end{itemize}

\textbf{Common Projects:}

\begin{itemize}
\tightlist
\item
  \textbf{Electronic dice} using LEDs
\item
  \textbf{Traffic light} controllers
\item
  \textbf{Digital clocks} and timers
\end{itemize}

\end{solutionbox}
\begin{mnemonicbox}
``Timer for Tremendous Tasks''

\end{mnemonicbox}
\subsection*{Question 5(b OR) [4
marks]}\label{question-5b-or-4-marks}

\textbf{Draw and explain the internal block diagram of IC 555.}

\begin{solutionbox}

\textbf{Internal Block Diagram:}

\begin{verbatim}
         Vcc (8)
          │
    ┌─────┴─────┐
    │  Voltage  │
    │ Divider   │  5V
    │    5kΩ    ├─────○ Control (5)
    │    │      │
    │   10V     │
    │    5kΩ    │
    │    │      │
    │   10V     │
    │    5kΩ    │
    └─────┬─────┘
          │
    ┌─────┴─────┐
    │     +     │
Threshold(6)  ○─┤Compartor├─┐
    │     {-     │   A     │}
    └───────────┘         │
                          │ ┌─── SR ────┐
    ┌───────────┐         ├─┤   Flip    ├─○ Output (3)
    │     +     │         │ │   Flop    │
Trigger(2)○─────┤Compartor├─┘ └─────────┘
    │     {-     │   B         │}
    └───────────┘           Reset(4)○─┘
                              │
    ┌─────────────────────────┴─────┐
    │      Discharge Transistor     ├─○ Discharge (7)
    └───────────────────────────────┘
                              │
                           Ground (1)
\end{verbatim}

\textbf{Block Functions:}

\begin{longtable}[]{@{}ll@{}}
\toprule\noalign{}
Block & Function \\
\midrule\noalign{}
\endhead
\bottomrule\noalign{}
\endlastfoot
\textbf{Voltage Divider} & Creates Vcc/3 and 2Vcc/3 references \\
\textbf{Comparator A} & Compares threshold with 2Vcc/3 \\
\textbf{Comparator B} & Compares trigger with Vcc/3 \\
\textbf{SR Flip-Flop} & Controls output state \\
\textbf{Discharge Transistor} & Discharges timing capacitor \\
\textbf{Output Buffer} & Provides high current output \\
\end{longtable}

\textbf{Working:}

\begin{itemize}
\tightlist
\item
  \textbf{Comparators} set and reset flip-flop
\item
  \textbf{Output buffer} amplifies flip-flop output
\item
  \textbf{Discharge transistor} controlled by flip-flop
\item
  \textbf{Reference voltages} set trigger levels
\end{itemize}

\end{solutionbox}
\begin{mnemonicbox}
``Internal Intelligence, Integrated Implementation''

\end{mnemonicbox}
\subsection*{Question 5(c OR) [7
marks]}\label{question-5c-or-7-marks}

\textbf{Explain astable multivibrator using 555 timer IC.}

\begin{solutionbox}

\textbf{Astable Circuit:}

\begin{verbatim}
    Vcc
     │
     ├─────○ 8 (Vcc)
     │      │ 4 (Reset)
     R1────○ 7 (Discharge)
     │     │ 6 (Threshold)
     R2────┤
     │     │ 5 (Control)
    ○2○────┤
     │     │ 3 (Output)
    ○ │    │ 1 (Ground)
     │     │
     C ────┘
     │
    ○ Ground
\end{verbatim}

\textbf{Working Principle:}

\textbf{Charging Phase:}

\begin{itemize}
\tightlist
\item
  \textbf{Capacitor charges} through R1 + R2
\item
  \textbf{Output HIGH} during charging
\item
  \textbf{Charging time}: T1 = 0.693(R1 + R2)C
\item
  \textbf{Voltage rises} from Vcc/3 to 2Vcc/3
\end{itemize}

\textbf{Discharging Phase:}

\begin{itemize}
\tightlist
\item
  \textbf{Capacitor discharges} through R2 only
\item
  \textbf{Output LOW} during discharging\\
\item
  \textbf{Discharging time}: T2 = 0.693 \times R2 \times C
\item
  \textbf{Voltage falls} from 2Vcc/3 to Vcc/3
\end{itemize}

\textbf{Frequency Calculations:}

\begin{longtable}[]{@{}ll@{}}
\toprule\noalign{}
Parameter & Formula \\
\midrule\noalign{}
\endhead
\bottomrule\noalign{}
\endlastfoot
\textbf{Time HIGH} & T1 = 0.693(R1 + R2)C \\
\textbf{Time LOW} & T2 = 0.693 \times R2 \times C \\
\textbf{Total Period} & T = T1 + T2 = 0.693(R1 + 2R2)C \\
\textbf{Frequency} & f = 1.44/[(R1 + 2R2)C] \\
\textbf{Duty Cycle} & D = (R1 + R2)/(R1 + 2R2) \times 100\% \\
\end{longtable}

\textbf{Waveforms:}

\begin{verbatim}
    Vout  │  ┌───┐     ┌───┐
          │  │   │     │   │
          │  │   │     │   │
         ─┼──┘   └─────┘   └──
          │  T1   T2
          └───────────────── Time
               Period T
\end{verbatim}

\textbf{Design Example:} For f = 1kHz, D = 60\%:

\begin{itemize}
\tightlist
\item
  Choose C = 0.1µF
\item
  Calculate R1 = 7.2kΩ, R2 = 3.6kΩ
\end{itemize}

\textbf{Key Features:}

\begin{itemize}
\tightlist
\item
  \textbf{Continuous oscillation} without external trigger
\item
  \textbf{Frequency adjustable} by R and C values\\
\item
  \textbf{Duty cycle} always \textgreater{} 50\% in basic circuit
\item
  \textbf{Stable operation} over wide temperature range
\end{itemize}

\textbf{Applications:}

\begin{itemize}
\tightlist
\item
  \textbf{LED flashers} and blinkers
\item
  \textbf{Clock generators} for digital circuits
\item
  \textbf{Tone generators} for alarms
\item
  \textbf{PWM signal} generation
\end{itemize}

\textbf{Modifications for 50\% Duty Cycle:}

\begin{itemize}
\tightlist
\item
  \textbf{Add diode} in parallel with R2
\item
  \textbf{Separate paths} for charge and discharge
\item
  \textbf{Equal charge/discharge} times possible
\end{itemize}

\end{solutionbox}
\begin{mnemonicbox}
``Astable Always Alternates Automatically''

\end{mnemonicbox}

\end{document}
