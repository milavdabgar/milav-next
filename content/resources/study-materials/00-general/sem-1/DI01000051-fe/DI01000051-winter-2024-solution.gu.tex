\documentclass[10pt,a4paper]{article}
\usepackage[margin=0.6in]{geometry}
\usepackage{amsmath,amssymb,amsthm}
\usepackage{booktabs}
\usepackage{multirow}
\usepackage{xcolor}
\usepackage{tcolorbox}
\tcbuselibrary{breakable}
\usepackage[colorlinks=true,linkcolor=blue]{hyperref}
\usepackage{titlesec}
\usepackage{enumitem}
\usepackage{tikz}
\usepackage{circuitikz}
\usetikzlibrary{shapes,arrows,positioning,calc,decorations.pathmorphing,patterns}

% XeLaTeX for Gujarati support
\usepackage{fontspec}
\usepackage{polyglossia}
\setmainlanguage{gujarati}
\setotherlanguage{english}

% Gujarati font with proper script features
\setmainfont{Noto Sans Gujarati}[
  Script=Gujarati,
  Renderer=Harfbuzz,
  Language=Gujarati
]
\newfontfamily\englishfont{Times New Roman}

% Color scheme
\definecolor{headcolor}{RGB}{0,102,204}
\definecolor{keycolor}{RGB}{220,20,60}
\definecolor{solutioncolor}{RGB}{34,139,34}
\definecolor{mnemoniccolor}{RGB}{148,0,211}

% Custom environments
\newtcolorbox{solutionbox}{
 breakable,
 colback=solutioncolor!5!white,
 colframe=solutioncolor!75!black,
 fonttitle=\bfseries,
 title=ઉકેલ
}

\newtcolorbox{keyformula}{
 colback=keycolor!5!white,
 colframe=keycolor!75!black,
 fonttitle=\bfseries,
 title=મુખ્ય સૂત્ર
}

\newtcolorbox{mnemonicbox}{
 colback=mnemoniccolor!5!white,
 colframe=mnemoniccolor!75!black,
 fonttitle=\bfseries,
 title=મેમરી ટ્રીક
}

% Spacing
\setlength{\parskip}{3pt}
\setlist[itemize]{nosep}
\setlist[enumerate]{nosep}

% Title formatting
\titleformat{\section}{\Large\bfseries\color{headcolor}}{\thesection}{1em}{}
\titleformat{\subsection}{\large\bfseries\color{headcolor}}{\thesubsection}{1em}{}

\begin{document}

\begin{center}
{\Huge\bfseries\color{headcolor} ઈલેક્ટ્રોનિક્સના મૂળભૂત સિદ્ધાંતો}\\[5pt]
{\LARGE DI01000051 -- શિયાળુ 2024}\\[3pt]
{\large સેમેસ્ટર 1 અભ્યાસ સામગ્રી}\\[3pt]
{\normalsize\textit{વિગતવાર ઉકેલો અને સમજૂતીઓ}}
\end{center}

\vspace{10pt}

%----------------------------------------
\section*{પ્રશ્ન 1(અ) [3 ગુણ]}
\textbf{એક્ટિવ અને પેસીવ કમ્પોનન્ટ્સની ઉદાહરણ સાથે વ્યાખ્યા કરો.}

\begin{solutionbox}
\textbf{કોષ્ટક: એક્ટિવ વિ પેસીવ કમ્પોનન્ટ્સ}
\begin{center}
\begin{tabular}{|l|p{7cm}|l|}
\hline
\textbf{પ્રકાર} & \textbf{વ્યાખ્યા} & \textbf{ઉદાહરણો} \\
\hline
\textbf{એક્ટિવ} & કમ્પોનન્ટ્સ જે સિગ્નલોને વિસ્તૃત કરી શકે અને કરંટ પ્રવાહ નિયંત્રિત કરે. પાવર ગેઇન આપી શકે. & ટ્રાન્ઝિસ્ટર, ડાયોડ, IC \\
\hline
\textbf{પેસીવ} & કમ્પોનન્ટ્સ જે સિગ્નલોને વિસ્તૃત કરી શકતા નથી. ઊર્જાનો સંગ્રહ અથવા વિસર્જન કરે. & રેઝિસ્ટર, કેપેસિટર, ઇન્ડક્ટર \\
\hline
\end{tabular}
\end{center}

\textbf{તફાવત}: એક્ટિવ કમ્પોનન્ટ્સને કાર્ય કરવા માટે બાહ્ય પાવર સપ્લાયની જરૂર પડે છે.
\end{solutionbox}

\begin{mnemonicbox}
``એક્ટિવ વિસ્તારે, પેસીવ સાચવે''
\end{mnemonicbox}

%----------------------------------------
\section*{પ્રશ્ન 1(બ) [4 ગુણ]}
\textbf{LDR નું બંધારણ અને કાર્ય સમજાવો.}

\begin{solutionbox}
\textbf{બંધારણ:}
\begin{itemize}
\item કેડમિયમ સલ્ફાઇડ (CdS) જેવા ઉચ્ચ અવરોધક સેમિકન્ડક્ટરથી બનેલું.
\item સિરામિક સબસ્ટ્રેટ પર ઝિગ-ઝેગ (સર્પેન્ટાઇન) ટ્રેક તરીકે જમા કરવામાં આવે છે.
\end{itemize}

\textbf{ડાયાગ્રામ:}
\begin{center}
\begin{tikzpicture}[scale=0.8]
\draw[fill=yellow!20] (0,0) circle (1.5);
\draw[thick, red] (-0.8,-0.5) -- (-0.8,0.5) -- (-0.4,0.5) -- (-0.4,-0.5) -- (0,-0.5) -- (0,0.5) -- (0.4,0.5) -- (0.4,-0.5) -- (0.8,-0.5) -- (0.8,0.5);
\draw[thick] (-0.8,0) -- (-1.8,0);
\draw[thick] (0.8,0) -- (1.8,0);
\node at (0,-1) {CdS Track};
\foreach \x in {-1,0,1} \draw[->, orange, thick] (\x, 2) -- (\x, 1.2);
\node[orange] at (0, 2.2) {Light};
\end{tikzpicture}
\end{center}

\textbf{કાર્યસિદ્ધાંત:}
\begin{enumerate}
\item \textbf{અંધકાર}: ઉચ્ચ અવરોધ (M$\Omega$).
\item \textbf{પ્રકાશ}: પ્રકાશ ઊર્જા બોન્ડ્સ તોડે છે, ઇલેક્ટ્રોન-હોલ પેર્સ બનાવે છે. અવરોધ ઘટે છે (k$\Omega$).
\end{enumerate}
\end{solutionbox}

\begin{mnemonicbox}
``લાઇટ લો રેઝિસ્ટન્સ''
\end{mnemonicbox}

%----------------------------------------
\section*{પ્રશ્ન 1(ક) [7 ગુણ]}
\textbf{કેપેસિટન્સની વ્યાખ્યા લખો અને એલ્યુમિનિયમ ઇલેક્ટ્રોલાઇટ વેટ પ્રકારનો કેપેસિટર સમજાવો.}

\begin{solutionbox}
\textbf{કેપેસિટન્સ}: ઇલેક્ટ્રિકલ ચાર્જ સંગ્રહિત કરવાની ક્ષમતા. $C = Q/V$ (એકમ: ફેરાડ).

\textbf{એલ્યુમિનિયમ ઇલેક્ટ્રોલાઇટિક કેપેસિટર:}
\begin{itemize}
\item \textbf{એનોડ (+)}: ઓક્સાઇડ લેયર ($Al_2O_3$) સાથે એલ્યુમિનિયમ ફોઇલ. આ ડાઇઇલેક્ટ્રિક તરીકે વર્તે છે.
\item \textbf{કેથોડ (-)}: ઇલેક્ટ્રોલાઇટ સાથે સંપર્કમાં બીજી એલ્યુમિનિયમ ફોઇલ.
\item \textbf{ઇલેક્ટ્રોલાઇટ}: વાહક પ્રવાહી/જેલ ભરેલું પેપર.
\end{itemize}

\textbf{લક્ષણો}: ઊંચી કેપેસિટન્સ, પોલરાઇઝ્ડ (ધન/ઋણ ધ્યાન રાખવું).
\end{solutionbox}

%----------------------------------------
\section*{પ્રશ્ન 1(ક OR) [7 ગુણ]}
\textbf{રેઝિસ્ટરની કલર બેન્ડ કોડિંગ પદ્ધતિ સમજાવો. 32 $\Omega$ $\pm$ 10\% કિંમતનો કલર બેન્ડ લખો.}

\begin{solutionbox}
\textbf{કલર કોડ ટેબલ:} B B R O Y G B V G W (0-9).

\textbf{32 $\Omega$ $\pm$ 10\% માટે ગણતરી:}
\begin{itemize}
\item 1લો અંક: 3 $\rightarrow$ \textbf{કેસરી (Orange)}
\item 2જો અંક: 2 $\rightarrow$ \textbf{લાલ (Red)}
\item મલ્ટિપ્લાયર: $10^0 = 1$ $\rightarrow$ \textbf{કાળો (Black)}
\item ટોલરન્સ: $\pm 10\%$ $\rightarrow$ \textbf{ચાંદી (Silver)}
\item \textbf{રંગ બેન્ડ}: કેસરી - લાલ - કાળો - ચાંદી
\end{itemize}
\end{solutionbox}

%----------------------------------------
\section*{પ્રશ્ન 2(અ) [3 ગુણ]}
\textbf{નીચેના શબ્દો વ્યાખ્યાયિત કરો: 1) રેક્ટિફાયર 2) રિપલ ફેક્ટર 3) ફિલ્ટર}

\begin{solutionbox}
\begin{enumerate}
\item \textbf{રેક્ટિફાયર}: AC ને પલ્સેટિંગ DC માં બદલનાર સર્કિટ.
\item \textbf{રિપલ ફેક્ટર}: આઉટપુટમાં AC ઘટક અને DC ઘટકનો ગુણોત્તર.
\item \textbf{ફિલ્ટર}: પલ્સેટિંગ DC માંથી રિપલ્સ દૂર કરી સ્મૂથ DC બનાવનાર સર્કિટ.
\end{enumerate}
\end{solutionbox}

%----------------------------------------
\section*{પ્રશ્ન 2(બ) [4 ગુણ]}
\textbf{પોઝિટિવ ક્લિપર સર્કિટ દોરી વેવફોર્મ સાથે સમજાવો.}

\begin{solutionbox}
\textbf{સર્કિટ ડાયાગ્રામ:}
\begin{center}
\begin{circuitikz}[scale=0.8]
\draw (0,0) to[sinusoidal voltage source, l=$V_{in}$] (0,2) -- (2,2) to[R, l=$R$] (4,2) -- (6,2) node[right] {$V_{out}$};
\draw (4,2) -- (4,1.5) to[D, l=$D$] (4,0.5) to[battery1, l=$V_{ref}$] (4,-0.5) -- (4,-1);
\draw (0,0) -- (6,0);
\draw (4,-1) -- (4,0);
\end{circuitikz}
\end{center}

\textbf{કાર્ય}:
\begin{itemize}
\item જ્યારે $V_{in} > V_{ref}$, ડાયોડ શોર્ટ થાય છે અને આઉટપુટ $V_{ref}$ પર ક્લિપ થાય છે.
\item જ્યારે $V_{in} < V_{ref}$, ડાયોડ ઓપન હોય છે અને આઉટપુટ ઇનપુટ જેવું જ મળે છે.
\end{itemize}
\end{solutionbox}

%----------------------------------------
\section*{પ્રશ્ન 2(ક) [7 ગુણ]}
\textbf{બે ડાયોડથી ફુલ વેવ રેક્ટિફાયરની કાર્યપદ્ધતિ સમજાવો.}

\begin{solutionbox}
\textbf{સેન્ટર-ટેપ ફુલ વેવ રેક્ટિફાયર:}
\begin{itemize}
\item \textbf{પોઝિટિવ હાફ}: $D_1$ ચાલુ, $D_2$ બંધ. કરંટ $R_L$ માંથી વહે છે.
\item \textbf{નેગેટિવ હાફ}: $D_2$ ચાલુ, $D_1$ બંધ. કરંટ $R_L$ માંથી \textbf{એક જ દિશામાં} વહે છે.
\end{itemize}
\textbf{પરિણામ}: આઉટપુટ ફ્રીક્વન્સી $2f$. કાર્યક્ષમતા 81.2\%.
\end{solutionbox}

%----------------------------------------
\section*{પ્રશ્ન 2(અ OR) [3 ગુણ]}
\textbf{રેક્ટિફાયર વ્યાખ્યાયિત કરો અને તેની એપ્લિકેશન લખો.}

\begin{solutionbox}
\textbf{ઉપયોગો}:
\begin{itemize}
\item પાવર સપ્લાય (મોબાઈલ ચાર્જર, એડેપ્ટર).
\item બેટરી ચાર્જિંગ.
\item રેડિયો ડિટેક્શન.
\end{itemize}
\end{solutionbox}

%----------------------------------------
\section*{પ્રશ્ન 2(બ OR) [4 ગુણ]}
\textbf{Pi ($\pi$) પ્રકારના કેપેસિટર ફિલ્ટરનું કાર્ય સમજાવો.}

\begin{solutionbox}
\begin{center}
\begin{circuitikz}[scale=0.8]
\draw (0,0) to[short, o-] (1,0) to[C, l=$C_1$] (1,-2) to[short, -o] (0,-2);
\draw (1,0) to[L, l=$L$] (3,0) to[C, l=$C_2$] (3,-2) -- (1,-2);
\draw (3,0) to[short, -o] (4,0) node[right] {Out};
\draw (3,-2) to[short, -o] (4,-2);
\end{circuitikz}
\end{center}

\textbf{કાર્ય}: $C_1$ અને $C_2$ રિપલ્સને ગ્રાઉન્ડ કરે છે, અને $L$ રિપલ્સને બ્લોક કરે છે. આથી ખૂબ સ્મૂથ DC મળે છે.
\end{solutionbox}

%----------------------------------------
\section*{પ્રશ્ન 2(ક OR) [7 ગુણ]}
\textbf{હાફ વેવ અને ફુલ વેવ બ્રિજ રેક્ટિફાયરને સરખાવો.}

\begin{solutionbox}
\begin{center}
\begin{tabular}{|l|c|c|}
\hline
\textbf{પેરામીટર} & \textbf{હાફ વેવ} & \textbf{બ્રિજ રેક્ટિફાયર} \\
\hline
ડાયોડ સંખ્યા & 1 & 4 \\
\hline
ટ્રાન્સફોર્મર & સાદું & સાદું (સેન્ટર ટેપ નથી જોઈતું) \\
\hline
કાર્યક્ષમતા & 40.6\% & 81.2\% \\
\hline
રિપલ ફેક્ટર & 1.21 & 0.48 \\
\hline
આઉટપુટ ફ્રીક્વન્સી & $f$ & $2f$ \\
\hline
\end{tabular}
\end{center}
\end{solutionbox}

%----------------------------------------
\section*{પ્રશ્ન 3(અ) [3 ગુણ]}
\textbf{નીચેના પ્રતીકો દોરો: 1) ઝેનર ડાયોડ 2) LED 3) વેરેક્ટર ડાયોડ}

\begin{solutionbox}
\begin{center}
\begin{circuitikz}
\draw (0,0) to[zD, l=Zener] (0,2);
\draw (2,0) to[leD, l=LED] (2,2);
\draw (4,0) to[vC, l=Varactor] (4,2);
\end{circuitikz}
\end{center}
\end{solutionbox}

%----------------------------------------
\section*{પ્રશ્ન 3(બ) [4 ગુણ]}
\textbf{LED ની રચના અને કાર્ય સમજાવો.}

\begin{solutionbox}
\textbf{કાર્ય સિદ્ધાંત}: ફોરવર્ડ બાયાસમાં જ્યારે ઇલેક્ટ્રોન અને હોલ્સ રિકોમ્બાઇન થાય છે ત્યારે ઊર્જા પ્રકાશ સ્વરૂપે મુક્ત થાય છે.
\end{solutionbox}

%----------------------------------------
\section*{પ્રશ્ન 3(ક) [7 ગુણ]}
\textbf{ઝેનર ડાયોડની કાર્યકારી લાક્ષણિકતાઓ સમજાવો.}

\begin{solutionbox}
\textbf{વિસ્તારો}:
\begin{itemize}
\item \textbf{ફોરવર્ડ}: સામાન્ય ડાયોડ જેવું વર્તન.
\item \textbf{રિવર્સ બ્રેકડાઉન}: ચોક્કસ વોલ્ટેજ ($V_Z$) પછી કરંટ ઝડપથી વધે છે પણ વોલ્ટેજ અચળ રહે છે. આ ગુણધર્મ વોલ્ટેજ રેગ્યુલેશનમાં વપરાય છે.
\end{itemize}
\end{solutionbox}

%----------------------------------------
\section*{પ્રશ્ન 4(અ) [3 ગુણ]}
\textbf{NPN અને PNP ટ્રાન્ઝિસ્ટરના પ્રતીક અને રચના યોગ્ય નોટેશન સાથે દોરો.}

\begin{solutionbox}
\begin{center}
\begin{circuitikz}
\draw (0,0) node[npn, label={NPN}] (npn) {};
\draw (4,0) node[pnp, label={PNP}] (pnp) {};
\end{circuitikz}
\end{center}

\textbf{રચના}:
\begin{itemize}
\item \textbf{NPN}: P-ટાઈપ બેઝ N-ટાઈપ કલેક્ટર અને ઈમીટર વચ્ચે સેન્ડવીચ કરેલ હોય છે.
\item \textbf{PNP}: N-ટાઈપ બેઝ P-ટાઈપ કલેક્ટર અને ઈમીટર વચ્ચે સેન્ડવીચ કરેલ હોય છે.
\end{itemize}
\end{solutionbox}

%----------------------------------------
\section*{પ્રશ્ન 4(બ) [4 ગુણ]}
\textbf{CE એમ્પ્લિફાયરની લાક્ષણિકતાઓ દોરો અને સમજાવો.}

\begin{solutionbox}
\textbf{લાક્ષણિકતાઓ}:
\begin{enumerate}
\item \textbf{ઇનપુટ}: $I_B$ વિ $V_{BE}$ (અચળ $V_{CE}$). ફોરવર્ડ ડાયોડ કર્વ જેવું.
\item \textbf{આઉટપુટ}: $I_C$ વિ $V_{CE}$ (અચળ $I_B$).
    \begin{itemize}
    \item \textbf{એક્ટિવ}: આપેલ $I_B$ માટે $I_C$ અચળ.
    \item \textbf{સેચ્યુરેશન}: $V_{CE}$ ખૂબ ઓછું, $I_C$ ઝડપથી વધે છે.
    \item \textbf{કટ-ઓફ}: $I_B=0, I_C=0$.
    \end{itemize}
\end{enumerate}
\end{solutionbox}

%----------------------------------------
\section*{પ્રશ્ન 4(ક) [7 ગુણ]}
\textbf{કરંટ ગેઇન $\alpha$, $\beta$ અને $\gamma$ વચ્ચેનો સંબંધ તારવો.}

\begin{solutionbox}
વ્યાખ્યા: $\alpha = I_C/I_E$, $\beta = I_C/I_B$, $\gamma = I_E/I_B$.
આપણને ખબર છે $I_E = I_B + I_C$.

\textbf{1. $\beta$ વિ $\alpha$}:
$I_C$ વડે ભાગો: $I_E/I_C = I_B/I_C + 1$
$1/\alpha = 1/\beta + 1 \Rightarrow \beta = \alpha / (1-\alpha)$.

\textbf{2. $\gamma$ વિ $\alpha$}:
$I_E = I_B + I_C \Rightarrow I_E = I_B + \alpha I_E$
$\gamma = 1/(1-\alpha)$.

\textbf{3. $\gamma$ વિ $\beta$}:
$\gamma = 1 + \beta$.
\end{solutionbox}

%----------------------------------------
\section*{પ્રશ્ન 5(અ) [3 ગુણ]}
\textbf{IC 555 નો પિન ડાયાગ્રામ દોરો.}

\begin{solutionbox}
\begin{center}
\begin{tikzpicture}
\draw[thick] (0,0) rectangle (3,4);
\node at (1.5,3.5) {IC 555};
\node at (1.5,4) [above] {Notch};
\fill (1.5,4) circle (0.1);
% Pins
\foreach \i/\n in {1/GND, 2/Trig, 3/Out, 4/Reset} {
    \draw (0, 4-\i*0.8) -- (-0.5, 4-\i*0.8) node[left] {\i\ \n};
}
\foreach \i/\txt in {8/Vcc, 7/Disch, 6/Thresh, 5/Ctrl} {
    \draw (3, {4-(\i-4)*0.8+0.8-0.8}) -- (3.5, {4-(\i-4)*0.8}) node[right] {\txt\ \i};
}
\end{tikzpicture}
\end{center}
\end{solutionbox}

%----------------------------------------
\section*{પ્રશ્ન 5(ક) [7 ગુણ]}
\textbf{555 ટાઈમર IC નો ઉપયોગ કરીને મોનોસ્ટેબલ મલ્ટિવાઈબ્રેટર સમજાવો.}

\begin{solutionbox}
\textbf{સર્કિટ}: રેઝિસ્ટર $R$ અને કેપેસિટર $C$. પિન 6 અને 7 શોર્ટ કરેલ છે. ટ્રિગર પિન 2 પર અપાય છે.

\textbf{કાર્ય}:
\begin{itemize}
\item સ્થિર સ્થિતિ: આઉટપુટ Low.
\item ટ્રિગર (નેગેટિવ પલ્સ): આઉટપુટ High થાય છે. કેપેસિટર ચાર્જ થાય છે.
\item જ્યારે $V_c$ $2/3 V_{cc}$ પર પહોંચે, આઉટપુટ Low થાય છે.
\item \textbf{પલ્સ પહોળાઈ}: $T = 1.1 R C$.
\end{itemize}
\end{solutionbox}

%----------------------------------------
\section*{પ્રશ્ન 5(બ OR) [4 ગુણ]}
\textbf{IC 555 નો આંતરિક બ્લોક ડાયાગ્રામ દોરો અને સમજાવો.}

\begin{solutionbox}
\textbf{બ્લોક્સ}:
\begin{itemize}
\item વોલ્ટેજ ડિવાઈડર (5k-5k-5k): 1/3 અને 2/3 Vcc સેટ કરે છે.
\item કમ્પેરેટર (2): ટ્રિગર અને થ્રેશોલ્ડ ચેક કરે છે.
\item SR ફ્લિપ-ફ્લોપ: સ્થિતિ સાચવે છે.
\item ડિસ્ચાર્જ ટ્રાન્ઝિસ્ટર: કેપેસિટર ડિસ્ચાર્જ કરે છે.
\end{itemize}
\end{solutionbox}

%----------------------------------------
\section*{પ્રશ્ન 5(ક OR) [7 ગુણ]}
\textbf{555 ટાઈમર IC નો ઉપયોગ કરીને એસ્ટેબલ મલ્ટિવાઈબ્રેટર સમજાવો.}

\begin{solutionbox}
\textbf{સર્કિટ}: પિન 2 અને 6 શોર્ટ કરેલ. રેઝિસ્ટર $R_A, R_B$ અને કેપેસિટર $C$.
\textbf{કાર્ય}:
\begin{itemize}
\item ચાર્જ: $R_A + R_B$ મારફતે. સમય $t_{high} = 0.693 (R_A+R_B) C$.
\item ડિસ્ચાર્જ: $R_B$ મારફતે. સમય $t_{low} = 0.693 R_B C$.
\item આઉટપુટ High અને Low વચ્ચે બદલાય છે (સ્ક્વેર વેવ).
\item \textbf{ફ્રીક્વન્સી}: $f = 1.44 / ((R_A+2R_B)C)$.
\end{itemize}
\end{solutionbox}

\end{document}
