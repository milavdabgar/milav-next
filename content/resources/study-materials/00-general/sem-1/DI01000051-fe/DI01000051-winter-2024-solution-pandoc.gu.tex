\documentclass[10pt,a4paper]{article}

% content/resources/templates/preamble.tex
\usepackage[margin=0.6in]{geometry}
\author{Milav Dabgar}
\usepackage{amsmath,amssymb,amsthm}
\usepackage{booktabs}
\usepackage{multirow}
\usepackage{xcolor}
\usepackage{tcolorbox}
\tcbuselibrary{breakable,skins}
\usepackage[colorlinks=true,linkcolor=blue]{hyperref}
\usepackage{titlesec}
\usepackage{enumitem}
\usepackage{tikz}
\usepackage{pgfplots}
\usepackage{circuitikz}
\usepackage[version=4]{mhchem}
\usepackage{longtable}
\usepackage{array}
\usepackage{float}
\usepackage{caption}
\usepackage{listings}

\lstset{
  basicstyle=\small\ttfamily,
  breaklines=true,
  breakatwhitespace=false,
  postbreak=\mbox{\textcolor{red}{$\hookrightarrow$}\space},
  float=false,
  numbers=left,
  numberstyle=\tiny\color{gray},
  numbersep=10pt,
  xleftmargin=2em,
  keywordstyle=\color{blue},
  commentstyle=\color{green!60!black},
  stringstyle=\color{purple},
  backgroundcolor=\color{gray!5},
  showstringspaces=false,
  tabsize=2,
  captionpos=b,
  keepspaces=true,
  columns=flexible
}

\pgfplotsset{compat=1.18}
\usetikzlibrary{shapes,arrows,positioning,calc,patterns,decorations.pathmorphing,decorations.markings,arrows.meta}

% Color scheme
\definecolor{headcolor}{RGB}{0,102,204}
\definecolor{keycolor}{RGB}{220,20,60}
\definecolor{solutioncolor}{RGB}{34,139,34}
\definecolor{mnemoniccolor}{RGB}{148,0,211}
\definecolor{codecolor}{RGB}{0,0,100}

% Spacing
\setlength{\parskip}{3pt}
\setlist[itemize]{nosep}
\setlist[enumerate]{nosep}

% Title formatting
\titleformat{\section}{\Large\bfseries\color{headcolor}}{\thesection}{1em}{}
\titleformat{\subsection}{\large\bfseries\color{headcolor}}{\thesubsection}{1em}{}

% Pandoc tightlist compatibility
\providecommand{\tightlist}{%
  \setlength{\itemsep}{0pt}\setlength{\parskip}{0pt}}

% Pandoc longtable compatibility
\newcounter{none}
\def\thenone{}


% content/resources/templates/gujarati-boxes.tex
\usepackage{fontspec}
\usepackage{polyglossia}

% Set Gujarati as main language (document is primarily in Gujarati)
% Note: gloss-gujarati.ldf doesn't exist in polyglossia, but it will use hyphenation patterns
\setdefaultlanguage{gujarati}
\setotherlanguage{english}

% Configure Gujarati font properly
% Use Language=Default to prevent polyglossia from trying to add language-specific features
% that don't exist for Gujarati, which causes "empty feature" warnings
\newfontfamily\gujaratifont[Script=Gujarati,AutoFakeBold=2.5,AutoFakeSlant=0.3]{Noto Sans Gujarati}
\setmainfont[Script=Gujarati,AutoFakeBold=2.5,AutoFakeSlant=0.3]{Noto Sans Gujarati}
% Use Noto Sans Gujarati for monospace to support Gujarati in text
\setmonofont[Scale=0.9]{Noto Sans Gujarati}

% Configure English to use the same font
\newfontfamily\englishfont[Script=Gujarati,AutoFakeBold=2.5,AutoFakeSlant=0.3]{Noto Sans Gujarati}

% Translations for polyglossia
\gappto\captionsgujarati{
  \renewcommand{\tablename}{કોષ્ટક}
  \renewcommand{\figurename}{આકૃતિ}
}

% Helper for TikZ nodes to ensure Gujarati font
\newcommand{\gu}[1]{{\gujaratifont #1}}

% Custom environments
\newtcolorbox{solutionbox}{
    breakable,
    enhanced,
    colback=solutioncolor!5!white,
    colframe=solutioncolor!75!black,
    fonttitle=\bfseries,
    title=જવાબ
}

\newtcolorbox{solutionboxnobreak}{
 colback=solutioncolor!5!white,
 colframe=solutioncolor!75!black,
 fonttitle=\bfseries,
 title=જવાબ
}

\newtcolorbox{keyformula}{
 breakable,
 enhanced,
 colback=keycolor!5!white,
 colframe=keycolor!75!black,
 fonttitle=\bfseries,
 title=રાસાયણિક સમીકરણ/સૂત્ર
}

\newtcolorbox{mnemonicbox}{
 breakable,
 enhanced,
 colback=mnemoniccolor!5!white,
 colframe=mnemoniccolor!75!black,
 fonttitle=\bfseries,
 title=મેમરી ટ્રીક
}


\begin{document}

\begin{center}
{\Huge\bfseries\color{headcolor} Basic Electronics (Gujarati)}\\[5pt]
{\LARGE DI01000051 -- Winter 2024}\\[3pt]
{\large Semester 1 Study Material}\\[3pt]
{\normalsize\textit{Detailed Solutions and Explanations}}
\end{center}

\vspace{10pt}

\subsection*{પ્રશ્ન 1(અ) [3
ગુણ]}\label{uxaaauxab0uxab6uxaa8-1uxa85-3-uxa97uxaa3}

\textbf{એક્ટિવ અને પેસીવ કમ્પોનન્ટ્સની ઉદાહરણ સાથે વ્યાખ્યા કરો.}

\begin{solutionbox}


\vspace{-5pt}
\captionof{table}{એક્ટિવ વિ પેસીવ કમ્પોનન્ટ્સ}
\vspace{-10pt}
\begin{longtable}[]{@{}
  >{\raggedright\arraybackslash}p{(\linewidth - 6\tabcolsep) * \real{0.2500}}
  >{\raggedright\arraybackslash}p{(\linewidth - 6\tabcolsep) * \real{0.2500}}
  >{\raggedright\arraybackslash}p{(\linewidth - 6\tabcolsep) * \real{0.2500}}
  >{\raggedright\arraybackslash}p{(\linewidth - 6\tabcolsep) * \real{0.2500}}@{}}
\toprule\noalign{}
\begin{minipage}[b]{\linewidth}\raggedright
કમ્પોનન્ટ પ્રકાર
\end{minipage} & \begin{minipage}[b]{\linewidth}\raggedright
વ્યાખ્યા
\end{minipage} & \begin{minipage}[b]{\linewidth}\raggedright
પાવર
\end{minipage} & \begin{minipage}[b]{\linewidth}\raggedright
ઉદાહરણો
\end{minipage} \\
\midrule\noalign{}
\endhead
\bottomrule\noalign{}
\endlastfoot
\textbf{એક્ટિવ કમ્પોનન્ટ્સ} & સિગ્નલોને વિસ્તૃત કરી શકે અને કરંટ પ્રવાહ નિયંત્રિત કરે &
પાવર ગેઇન આપી શકે & ટ્રાન્ઝિસ્ટર, ડાયોડ, IC \\
\textbf{પેસીવ કમ્પોનન્ટ્સ} & સિગ્નલોને વિસ્તૃત કરી શકતા નથી & પાવર ગેઇન આપી શકતા
નથી & રેઝિસ્ટર, કેપેસિટર, ઇન્ડક્ટર \\
\end{longtable}

\begin{itemize}
\tightlist
\item
  \textbf{એક્ટિવ કમ્પોનન્ટ્સ}: બાહ્ય પાવરનો ઉપયોગ કરીને ઇલેક્ટ્રિકલ સિગ્નલોને
  નિયંત્રિત અને વિસ્તૃત કરે
\item
  \textbf{પેસીવ કમ્પોનન્ટ્સ}: વિસ્તારણ વિના ઊર્જાનો સંગ્રહ અથવા વિસર્જન કરે
\end{itemize}

\end{solutionbox}
\begin{mnemonicbox}
``એક્ટિવ વિસ્તારે, પેસીવ સાચવે''

\end{mnemonicbox}
\subsection*{પ્રશ્ન 1(બ) [4
ગુણ]}\label{uxaaauxab0uxab6uxaa8-1uxaac-4-uxa97uxaa3}

\textbf{LDR નું બંધારણ અને કાર્ય સમજાવો.}

\begin{solutionbox}

\textbf{બંધારણ:}

\begin{itemize}
\tightlist
\item
  \textbf{સર્પેન્ટાઇન ટ્રેક} સિરામિક સબસ્ટ્રેટ પર કેડમિયમ સલ્ફાઇડનો
\item
  \textbf{મેટલ ઇલેક્ટ્રોડ્સ} બંને છેડે કનેક્શન માટે
\item
  \textbf{પ્રોટેક્ટિવ કોટિંગ} ભેજથી બચાવવા માટે
\end{itemize}

\textbf{કાર્યસિદ્ધાંત:}

\begin{verbatim}
    Light ↓
    ┌─────────────┐
    │  CdS Track  │  રેઝિસ્ટન્સ ઘટે
    │ { │}
    │     LDR     │
    └─────────────┘
         │    │
       ટર્મિનલ ટર્મિનલ
\end{verbatim}

\begin{itemize}
\tightlist
\item
  \textbf{પ્રકાશ તીવ્રતા ↑}: રેઝિસ્ટન્સ ↓ (વધુ કંડક્ટ કરે)
\item
  \textbf{અંધકાર}: રેઝિસ્ટન્સ ↑ (ઓછું કંડક્ટ કરે)
\item
  \textbf{ઉપયોગો}: સ્ટ્રીટ લાઇટ્સ, ઓટોમેટિક કેમેરા
\end{itemize}

\end{solutionbox}
\begin{mnemonicbox}
``લાઇટ લો રેઝિસ્ટન્સ''

\end{mnemonicbox}
\subsection*{પ્રશ્ન 1(ક) [7
ગુણ]}\label{uxaaauxab0uxab6uxaa8-1uxa95-7-uxa97uxaa3}

\textbf{કેપેસિટન્સની વ્યાખ્યા લખો અને એલ્યુમિનિયમ ઇલેક્ટ્રોલાઇટ વેટ પ્રકારનો કેપેસિટર
સમજાવો.}

\begin{solutionbox}

\textbf{કેપેસિટન્સ વ્યાખ્યા:} ઇલેક્ટ્રિકલ ચાર્જ સંગ્રહિત કરવાની ક્ષમતા. C = Q/V
(ફેરાડ્સ)

\textbf{એલ્યુમિનિયમ ઇલેક્ટ્રોલાઇટિક કેપેસિટર:}

\begin{verbatim}
    પોઝિટિવ ટર્મિનલ
         │
    ┌────┴────┐
    │ Al Foil │  એનોડ
    │ Oxide   │  ડાઇઇલેક્ટ્રિક
    │ Electro │  કેથોડ
    │ Al Foil │  નેગેટિવ
    └────┬────┘
         │
    નેગેટિવ ટર્મિનલ
\end{verbatim}

\textbf{બંધારણ:}

\begin{itemize}
\tightlist
\item
  \textbf{એનોડ}: ઓક્સાઇડ લેયર સાથે એલ્યુમિનિયમ ફોઇલ
\item
  \textbf{ડાઇઇલેક્ટ્રિક}: પાતળી એલ્યુમિનિયમ ઓક્સાઇડ ફિલ્મ
\item
  \textbf{કેથોડ}: એલ્યુમિનિયમ ફોઇલ સાથે લિક્વિડ ઇલેક્ટ્રોલાઇટ
\item
  \textbf{પોલેરિટી}: યોગ્ય રીતે જોડવું જરૂરી
\end{itemize}

\textbf{લક્ષણો:}

\begin{itemize}
\tightlist
\item
  \textbf{ઉચ્ચ કેપેસિટન્સ} મૂલ્યો (1µF થી 10,000µF)
\item
  \textbf{પોલરાઇઝ્ડ} - પોઝિટિવ અને નેગેટિવ ટર્મિનલ છે
\item
  \textbf{ઉપયોગો}: પાવર સપ્લાય ફિલ્ટરિંગ, કપલિંગ
\end{itemize}

\end{solutionbox}
\begin{mnemonicbox}
``એલ્યુમિનિયમ હંમેશાં વિસ્તારે''

\end{mnemonicbox}
\subsection*{પ્રશ્ન 1(ક OR) [7
ગુણ]}\label{uxaaauxab0uxab6uxaa8-1uxa95-or-7-uxa97uxaa3}

\textbf{રેઝિસ્ટરની કલર બેન્ડ કોડિંગ પદ્ધતિ સમજાવો. 32 Ω \pm 10\% કિંમતનો કલર બેન્ડ
લખો.}

\begin{solutionbox}

\textbf{કલર કોડ ટેબલ:}

\begin{longtable}[]{@{}llll@{}}
\toprule\noalign{}
રંગ & અંક & ગુણાકાર & ટોલરન્સ \\
\midrule\noalign{}
\endhead
\bottomrule\noalign{}
\endlastfoot
કાળો & 0 & 1 & - \\
ભૂરો & 1 & 10 & \pm1\% \\
લાલ & 2 & 100 & \pm2\% \\
કેસરી & 3 & 1K & - \\
પીળો & 4 & 10K & - \\
લીલો & 5 & 100K & \pm0.5\% \\
વાદળી & 6 & 1M & \pm0.25\% \\
વાયોલેટ & 7 & 10M & \pm0.1\% \\
ધૂસર & 8 & 100M & \pm0.05\% \\
સફેદ & 9 & 1G & - \\
ચાંદી & - & 0.01 & \pm10\% \\
સોનું & - & 0.1 & \pm5\% \\
\end{longtable}

\textbf{32 Ω \pm 10\% માટે:}

\begin{verbatim}
    ┌─────────────────────────────────┐
    │ કેસરી│લાલ│સોનું│ચાંદી│         │
    │   3   │ 2 │ 0.1 │ 10\%│         │
    │   ↓   │ ↓ │  ↓  │  ↓  │         │
    │  1મો  │2જો│ગુણા│ટોલ  │         │
    └─────────────────────────────────┘
\end{verbatim}

\textbf{ગણતરી:} 3 \times 2 \times 0.1 = 3.2 \times 10 = 32 Ω

\end{solutionbox}
\begin{mnemonicbox}
``મોટા છોકરા દોડે અમારા યુવા છોકરીઓ પણ વાયોલેટ સામાન્યે
જીતે''

\end{mnemonicbox}
\subsection*{પ્રશ્ન 2(અ) [3
ગુણ]}\label{uxaaauxab0uxab6uxaa8-2uxa85-3-uxa97uxaa3}

\textbf{નીચેના શબ્દો વ્યાખ્યાયિત કરો: 1) રેક્ટિફાયર 2) રિપલ ફેક્ટર 3) ફિલ્ટર}

\begin{solutionbox}

\begin{longtable}[]{@{}ll@{}}
\toprule\noalign{}
શબ્દ & વ્યાખ્યા \\
\midrule\noalign{}
\endhead
\bottomrule\noalign{}
\endlastfoot
\textbf{રેક્ટિફાયર} & AC ને પલ્સેટિંગ DC માં બદલનાર સર્કિટ \\
\textbf{રિપલ ફેક્ટર} & આઉટપુટમાં AC ઘટક અને DC ઘટકનો ગુણોત્તર \\
\textbf{ફિલ્ટર} & પલ્સેટિંગ DC ને સ્મૂથ DC માં બદલનાર સર્કિટ \\
\end{longtable}

\begin{itemize}
\tightlist
\item
  \textbf{રેક્ટિફાયર}: એક જ દિશામાં કરંટ પસાર કરવા ડાયોડનો ઉપયોગ કરે
\item
  \textbf{રિપલ ફેક્ટર}: નીચું મૂલ્ય મતલબ સારું ફિલ્ટરિંગ
\item
  \textbf{ફિલ્ટર}: રિપલ્સ ઘટાડવા કેપેસિટર/ઇન્ડક્ટરનો ઉપયોગ કરે
\end{itemize}

\end{solutionbox}
\begin{mnemonicbox}
``રેક્ટિફાય રિપલ્સ, ફિલ્ટર ફિક્સ કરે''

\end{mnemonicbox}
\subsection*{પ્રશ્ન 2(બ) [4
ગુણ]}\label{uxaaauxab0uxab6uxaa8-2uxaac-4-uxa97uxaa3}

\textbf{પોઝિટિવ ક્લિપર સર્કિટ દોરી વેવફોર્મ સાથે સમજાવો.}

\begin{solutionbox}

\textbf{સર્કિટ ડાયાગ્રામ:}

\begin{verbatim}
    Input ○────┬────○ Output
               │
               D1 ↓ (ડાયોડ)
               │
               ○ +V (ક્લિપિંગ લેવલ)
\end{verbatim}

\textbf{કાર્યપદ્ધતિ:}

\begin{itemize}
\tightlist
\item
  \textbf{જ્યારે Vin \textgreater{} +V}: ડાયોડ કંડક્ટ કરે, આઉટપુટ = +V
\item
  \textbf{જ્યારે Vin \textless{} +V}: ડાયોડ બંધ, આઉટપુટ ઇનપુટને અનુસરે
\item
  \textbf{પરિણામ}: +V લેવલથી ઉપરના પોઝિટિવ પીક્સ ક્લિપ થાય
\end{itemize}

\textbf{વેવફોર્મ:}

\begin{verbatim}
    Input     │    Output
             │
        ┌─┐   │      ──V
       ┌┘ └┐  │     ┌──┐
    ───┘   └──┼──   │  │
              │  ───┘  └───
              │
\end{verbatim}

\textbf{ઉપયોગો}: સિગ્નલ લિમિટિંગ, પ્રોટેક્શન સર્કિટ્સ

\end{solutionbox}
\begin{mnemonicbox}
``પોઝિટિવ પીક્સ પ્રિવેન્ટેડ''

\end{mnemonicbox}
\subsection*{પ્રશ્ન 2(ક) [7
ગુણ]}\label{uxaaauxab0uxab6uxaa8-2uxa95-7-uxa97uxaa3}

\textbf{બે ડાયોડથી ફુલ વેવ રેક્ટિફાયરની કાર્યપદ્ધતિ સમજાવો.}

\begin{solutionbox}

\textbf{સર્કિટ ડાયાગ્રામ:}

\begin{verbatim}
    AC Input    D1 ↗     RL
         ┌─────────────○───┐
         │              │
    {  │ સેન્ટર{-}ટેપ       │  ○ Output}
         │ ટ્રાન્સફોર્મર       │
         │              │
         └─────────────○───┘
              D2 ↘     
\end{verbatim}

\textbf{કાર્યપદ્ધતિ:}

\begin{itemize}
\tightlist
\item
  \textbf{પોઝિટિવ હાફ-સાયકલ}: D1 કંડક્ટ કરે, D2 બંધ
\item
  \textbf{નેગેટિવ હાફ-સાયકલ}: D2 કંડક્ટ કરે, D1 બંધ
\item
  \textbf{બંને ડાયોડ} વારાફરતી કામ કરે
\item
  \textbf{આઉટપુટ ફ્રીક્વન્સી} = 2 \times ઇનપુટ ફ્રીક્વન્સી
\end{itemize}

\textbf{મુખ્ય પેરામીટર્સ:}

\begin{longtable}[]{@{}ll@{}}
\toprule\noalign{}
પેરામીટર & મૂલ્ય \\
\midrule\noalign{}
\endhead
\bottomrule\noalign{}
\endlastfoot
\textbf{પીક ઇન્વર્સ વોલ્ટેજ} & 2Vm \\
\textbf{કાર્યક્ષમતા} & 81.2\% \\
\textbf{રિપલ ફેક્ટર} & 0.48 \\
\textbf{ફોર્મ ફેક્ટર} & 1.11 \\
\end{longtable}

\textbf{ફાયદા:}

\begin{itemize}
\tightlist
\item
  \textbf{હાફ વેવ કરતાં સારી કાર્યક્ષમતા}
\item
  \textbf{ઓછું રિપલ} કન્ટેન્ટ
\item
  \textbf{વધુ ટ્રાન્સફોર્મર ઉપયોગ}
\end{itemize}

\end{solutionbox}
\begin{mnemonicbox}
``બે ડાયોડ, બે હાફ''

\end{mnemonicbox}
\subsection*{પ્રશ્ન 2(અ OR) [3
ગુણ]}\label{uxaaauxab0uxab6uxaa8-2uxa85-or-3-uxa97uxaa3}

\textbf{રેક્ટિફાયર વ્યાખ્યાયિત કરો અને તેની એપ્લિકેશન લખો.}

\begin{solutionbox}

\textbf{વ્યાખ્યા:} ઇલેક્ટ્રોનિક સર્કિટ જે ડાયોડનો ઉપયોગ કરીને AC કરંટને DC કરંટમાં
બદલે છે.

\textbf{એપ્લિકેશન્સ:}

\begin{longtable}[]{@{}ll@{}}
\toprule\noalign{}
એપ્લિકેશન & ઉપયોગ \\
\midrule\noalign{}
\endhead
\bottomrule\noalign{}
\endlastfoot
\textbf{પાવર સપ્લાય} & ઇલેક્ટ્રોનિક સર્કિટ્સ માટે DC વોલ્ટેજ \\
\textbf{બેટરી ચાર્જર} & AC મેઇન્સને DC માં કન્વર્ટ કરવા \\
\textbf{DC મોટર્સ} & મોટર ડ્રાઇવ્સ માટે DC પૂરું પાડવા \\
\textbf{ઇલેક્ટ્રોનિક ડિવાઇસ} & લેપટોપ, ફોન, LED ડ્રાઇવર્સ \\
\end{longtable}

\begin{itemize}
\tightlist
\item
  \textbf{પ્રમુખ્ય કાર્ય}: AC થી DC કન્વર્ઝન
\item
  \textbf{અનિવાર્ય ઘટક}: બધા ઇલેક્ટ્રોનિક ડિવાઇસમાં
\end{itemize}

\end{solutionbox}
\begin{mnemonicbox}
``AC રેક્ટિફાય કરે, DC ડિલિવર કરે''

\end{mnemonicbox}
\subsection*{પ્રશ્ન 2(બ OR) [4
ગુણ]}\label{uxaaauxab0uxab6uxaa8-2uxaac-or-4-uxa97uxaa3}

\textbf{Pi (π) પ્રકારના કેપેસિટર ફિલ્ટરનું કાર્ય સમજાવો.}

\begin{solutionbox}

\textbf{સર્કિટ ડાયાગ્રામ:}

\begin{verbatim}
    Input   C1    L    C2   Output
    ○──────||────UUU────||───○
           │             │
           │             │
           ○─────────────○
              Ground
\end{verbatim}

\textbf{કાર્યપદ્ધતિ:}

\begin{itemize}
\tightlist
\item
  \textbf{C1}: રેક્ટિફાયરથી આવતા પ્રારંભિક રિપલ્સ ફિલ્ટર કરે
\item
  \textbf{ઇન્ડક્ટર L}: કરંટ ચેન્જનો વિરોધ કરે, વધુ સ્મૂથ કરે
\item
  \textbf{C2}: સ્મૂથ DC આઉટપુટ માટે અંતિમ ફિલ્ટરિંગ
\item
  \textbf{સંયુક્ત અસર}: ઉત્તમ રિપલ ઘટાડો
\end{itemize}

\textbf{લક્ષણો:}

\begin{longtable}[]{@{}ll@{}}
\toprule\noalign{}
પેરામીટર & મૂલ્ય \\
\midrule\noalign{}
\endhead
\bottomrule\noalign{}
\endlastfoot
\textbf{રિપલ ફેક્ટર} & ખૂબ ઓછું (\textless{} 0.01) \\
\textbf{રેગ્યુલેશન} & સારું \\
\textbf{કિંમત} & ઇન્ડક્ટરને કારણે વધારે \\
\textbf{એપ્લિકેશન્સ} & ઉચ્ચ ગુણવત્તાની પાવર સપ્લાય \\
\end{longtable}

\textbf{ફાયદા:}

\begin{itemize}
\tightlist
\item
  \textbf{ઉત્તમ ફિલ્ટરિંગ} પર્ફોર્મન્સ
\item
  \textbf{ઓછું રિપલ} કન્ટેન્ટ
\item
  \textbf{સારું વોલ્ટેજ રેગ્યુલેશન}
\end{itemize}

\end{solutionbox}
\begin{mnemonicbox}
``Pi પરફેક્ટ પૂરું પાડે''

\end{mnemonicbox}
\subsection*{પ્રશ્ન 2(ક OR) [7
ગુણ]}\label{uxaaauxab0uxab6uxaa8-2uxa95-or-7-uxa97uxaa3}

\textbf{હાફ વેવ અને ફુલ વેવ બ્રિજ રેક્ટિફાયરને સરખાવો.}

\begin{solutionbox}

\textbf{તુલના કોષ્ટક:}

\begin{longtable}[]{@{}lll@{}}
\toprule\noalign{}
પેરામીટર & હાફ વેવ & ફુલ વેવ બ્રિજ \\
\midrule\noalign{}
\endhead
\bottomrule\noalign{}
\endlastfoot
\textbf{જરૂરી ડાયોડ} & 1 & 4 \\
\textbf{ટ્રાન્સફોર્મર} & સિમ્પલ & સેન્ટર-ટેપની જરૂર નથી \\
\textbf{કાર્યક્ષમતા} & 40.6\% & 81.2\% \\
\textbf{રિપલ ફેક્ટર} & 1.21 & 0.48 \\
\textbf{PIV} & Vm & Vm \\
\textbf{આઉટપુટ ફ્રીક્વન્સી} & f & 2f \\
\textbf{ટ્રાન્સફોર્મર ઉપયોગ} & 28.7\% & 81.2\% \\
\textbf{કિંમત} & ઓછી & મધ્યમ \\
\end{longtable}

\textbf{સર્કિટ ડાયાગ્રામ:}

\textbf{હાફ વેવ:}

\begin{verbatim}
    AC ○────D1────○ Output
        │          │
        │    RL    │
        │          │
        ○──────────○
\end{verbatim}

\textbf{ફુલ વેવ બ્રિજ:}

\begin{verbatim}
         D1 ↗
    AC ○─────────○ Output
       │    RL   │
       │         │
    AC ○─────────○
         D2 ↘
\end{verbatim}

\textbf{મુખ્ય તફાવતો:}

\begin{itemize}
\tightlist
\item
  \textbf{ફુલ વેવ}: સારી કાર્યક્ષમતા અને ઓછું રિપલ
\item
  \textbf{હાફ વેવ}: સરળ પણ નબળી કામગીરી
\item
  \textbf{બ્રિજ}: સેન્ટર-ટેપ ટ્રાન્સફોર્મરની જરૂર નથી
\end{itemize}

\end{solutionbox}
\begin{mnemonicbox}
``હાફ વેસ્ટ કરે, ફુલ કામ કરે''

\end{mnemonicbox}
\subsection*{પ્રશ્ન 3(અ) [3
ગુણ]}\label{uxaaauxab0uxab6uxaa8-3uxa85-3-uxa97uxaa3}

\textbf{નીચેના પ્રતીકો દોરો: 1) ઝેનર ડાયોડ 2) LED 3) વેરેક્ટર ડાયોડ}

\begin{solutionbox}

\textbf{ઇલેક્ટ્રોનિક પ્રતીકો:}

\begin{verbatim}
    ઝેનર ડાયોડ:        LED:            વેરેક્ટર ડાયોડ:
         │                  │                   │
       ──┤►├──           ──┤►├──            ──┤ ├──
         │ Z                │ ↗               │ │ │
                                              │ │ │
                                              ─────
\end{verbatim}

\textbf{પ્રતીક વિગતો:}

\begin{longtable}[]{@{}ll@{}}
\toprule\noalign{}
કમ્પોનન્ટ & પ્રતીક લક્ષણ \\
\midrule\noalign{}
\endhead
\bottomrule\noalign{}
\endlastfoot
\textbf{ઝેનર ડાયોડ} & Z આકારના કેથોડ સાથે સામાન્ય ડાયોડ \\
\textbf{LED} & પ્રકાશ ઉત્સર્જન દર્શાવતા તીર સાથે ડાયોડ \\
\textbf{વેરેક્ટર ડાયોડ} & સમાંતર લીટીઓ સાથે ડાયોડ (વેરિએબલ કેપેસિટર) \\
\end{longtable}

\begin{itemize}
\tightlist
\item
  \textbf{ઝેનર}: Z ઝેનર લક્ષણો દર્શાવે
\item
  \textbf{LED}: તીર પ્રકાશ આઉટપુટ દિશા દર્શાવે
\item
  \textbf{વેરેક્ટર}: લીટીઓ વેરિએબલ કેપેસિટન્સ દર્શાવે
\end{itemize}

\end{solutionbox}
\begin{mnemonicbox}
``ઝેનર ઝિગઝેગ, LED લાઇટ, વેરેક્ટર વેરી''

\end{mnemonicbox}
\subsection*{પ્રશ્ન 3(બ) [4
ગુણ]}\label{uxaaauxab0uxab6uxaa8-3uxaac-4-uxa97uxaa3}

\textbf{LED ની રચના અને કાર્ય સમજાવો.}

\begin{solutionbox}

\textbf{બંધારણ:}

\begin{verbatim}
         Light Output ↑
      ┌──────────────────┐
      │   Wire Bond      │
      │ ┌──────────────┐ │
      │ │  P{-N Junction│ │}
      │ │      │       │ │
      │ └──────┼───────┘ │
      │   Cathode  Anode │
      └──────────────────┘
            LED Chip
\end{verbatim}

\textbf{સામગ્રી:}

\begin{itemize}
\tightlist
\item
  \textbf{P-type}: બોરોન-ડોપ્ડ સેમિકન્ડક્ટર
\item
  \textbf{N-type}: ફોસ્ફોરસ-ડોપ્ડ સેમિકન્ડક્ટર
\item
  \textbf{સામાન્ય સામગ્રી}: GaAs, GaP, GaN
\end{itemize}

\textbf{કાર્યસિદ્ધાંત:}

\begin{itemize}
\tightlist
\item
  \textbf{ફોરવર્ડ બાયાસ}: ઇલેક્ટ્રોન હોલ્સ સાથે રિકોમ્બાઇન થાય
\item
  \textbf{ઊર્જા રિલીઝ}: ફોટોન (પ્રકાશ) રૂપમાં
\item
  \textbf{રંગ}: સેમિકન્ડક્ટર સામગ્રી અને બેન્ડગેપ પર આધાર રાખે
\item
  \textbf{કાર્યક્ષમતા}: ઓછી પાવર સાથે ઉચ્ચ લાઇટ આઉટપુટ
\end{itemize}

\textbf{ઉપયોગો:}

\begin{itemize}
\tightlist
\item
  \textbf{ઇન્ડિકેટર્સ}: સ્ટેટસ લાઇટ્સ, ડિસ્પ્લે
\item
  \textbf{લાઇટિંગ}: LED બલ્બ્સ, સ્ટ્રિપ્સ
\item
  \textbf{ઇલેક્ટ્રોનિક્સ}: સેવન-સેગમેન્ટ ડિસ્પ્લે
\end{itemize}

\end{solutionbox}
\begin{mnemonicbox}
``લાઇટ ઇમિટિંગ, એનર્જી એફિશિયન્ટ''

\end{mnemonicbox}
\subsection*{પ્રશ્ન 3(ક) [7
ગુણ]}\label{uxaaauxab0uxab6uxaa8-3uxa95-7-uxa97uxaa3}

\textbf{ઝેનર ડાયોડની કાર્યકારી લાક્ષણિકતાઓ સમજાવો.}

\begin{solutionbox}

\textbf{V-I લાક્ષણિકતાઓ:}

\begin{verbatim}
                │ Forward
                │   ↗
                │ ↗ If
    ────────────┼───────── V
     Vz  │  │   │
         │  │   │
    ─────┼──┼───┼────
         │  │   │ Reverse
         │  Iz  │
         │      │
       Zener    │
      Region    │
\end{verbatim}

\textbf{મુખ્ય વિસ્તારો:}

\begin{longtable}[]{@{}ll@{}}
\toprule\noalign{}
વિસ્તાર & લાક્ષણિકતાઓ \\
\midrule\noalign{}
\endhead
\bottomrule\noalign{}
\endlastfoot
\textbf{ફોરવર્ડ બાયાસ} & સામાન્ય ડાયોડ ઓપરેશન (0.7V) \\
\textbf{રિવર્સ બાયાસ} & નાનું લીકેજ કરંટ \\
\textbf{ઝેનર રીજીયન} & કોન્સ્ટન્ટ વોલ્ટેજ (Vz) \\
\textbf{બ્રેકડાઉન} & શાર્પ વોલ્ટેજ બ્રેકડાઉન \\
\end{longtable}

\textbf{મહત્વના પેરામીટર્સ:}

\begin{itemize}
\tightlist
\item
  \textbf{ઝેનર વોલ્ટેજ (Vz)}: બ્રેકડાઉન વોલ્ટેજ
\item
  \textbf{ઝેનર કરંટ (Iz)}: બ્રેકડાઉન વિસ્તારમાં કરંટ
\item
  \textbf{મેક્સિમમ પાવર}: Vz \times Iz(max)
\item
  \textbf{તાપમાન ગુણાંક}: તાપમાન સાથે વોલ્ટેજ વેરિએશન
\end{itemize}

\textbf{ઉપયોગો:}

\begin{itemize}
\tightlist
\item
  \textbf{વોલ્ટેજ રેગ્યુલેશન}: કોન્સ્ટન્ટ આઉટપુટ જાળવે
\item
  \textbf{રેફરન્સ વોલ્ટેજ}: ચોક્કસ વોલ્ટેજ સોર્સ
\item
  \textbf{ઓવરવોલ્ટેજ પ્રોટેક્શન}: સર્કિટ્સનું રક્ષણ કરે
\end{itemize}

\textbf{ફાયદા:}

\begin{itemize}
\tightlist
\item
  \textbf{શાર્પ બ્રેકડાઉન}: સારી રીતે વ્યાખ્યાયિત વોલ્ટેજ
\item
  \textbf{ઓછું ડાયનામિક રેઝિસ્ટન્સ}: સારું રેગ્યુલેશન
\item
  \textbf{વાઇડ રેન્જ}: ઘણા વોલ્ટેજમાં ઉપલબ્ધ
\end{itemize}

\end{solutionbox}
\begin{mnemonicbox}
``ઝેનર ઝોન ઝીરો વેરિએશન''

\end{mnemonicbox}
\subsection*{પ્રશ્ન 3(અ OR) [3
ગુણ]}\label{uxaaauxab0uxab6uxaa8-3uxa85-or-3-uxa97uxaa3}

\textbf{વેરેક્ટર ડાયોડની એપ્લિકેશનની યાદી બનાવો.}

\begin{solutionbox}

\textbf{એપ્લિકેશન ટેબલ:}

\begin{longtable}[]{@{}ll@{}}
\toprule\noalign{}
એપ્લિકેશન & કાર્ય \\
\midrule\noalign{}
\endhead
\bottomrule\noalign{}
\endlastfoot
\textbf{વોલ્ટેજ કંટ્રોલ્ડ ઓસિલેટર્સ} & વોલ્ટેજ સાથે ફ્રીક્વન્સી ટ્યુનિંગ \\
\textbf{ઓટોમેટિક ફ્રીક્વન્સી કંટ્રોલ} & ઓસિલેટર ફ્રીક્વન્સી જાળવે \\
\textbf{ઇલેક્ટ્રોનિક ટ્યુનિંગ} & રેડિયો/TV ચેનલ સિલેક્શન \\
\textbf{ફેઝ લૉક્ડ લૂપ્સ} & ફ્રીક્વન્સી સિંક્રોનાઇઝેશન \\
\textbf{ફ્રીક્વન્સી મલ્ટિપ્લાયર્સ} & હાર્મોનિક જનરેશન \\
\textbf{પેરામેટ્રિક એમ્પ્લિફાયર્સ} & લો-નોઇઝ એમ્પ્લિફિકેશન \\
\end{longtable}

\textbf{મુખ્ય લક્ષણો:}

\begin{itemize}
\tightlist
\item
  \textbf{વોલ્ટેજ વેરિએબલ}: રિવર્સ વોલ્ટેજ સાથે કેપેસિટન્સ બદલાય
\item
  \textbf{યાંત્રિક ભાગો નથી}: માત્ર ઇલેક્ટ્રોનિક ટ્યુનિંગ
\item
  \textbf{ઝડપી પ્રતિસાદ}: ઝડપી ફ્રીક્વન્સી ચેન્જ
\end{itemize}

\end{solutionbox}
\begin{mnemonicbox}
``વોલ્ટેજ વેરીઝ કેપેસિટન્સ''

\end{mnemonicbox}
\subsection*{પ્રશ્ન 3(બ OR) [4
ગુણ]}\label{uxaaauxab0uxab6uxaa8-3uxaac-or-4-uxa97uxaa3}

\textbf{ફોટો ડાયોડનું કાર્ય સમજાવો.}

\begin{solutionbox}

\textbf{બંધારણ અને પ્રતીક:}

\begin{verbatim}
      Light ↓ ↓ ↓
    ┌─────────────┐
    │    P{-type   │  એનોડ}
    │─────────────│  P{-N જંક્શન}
    │    N{-type   │  કેથોડ}
    └─────────────┘
         │     │
      કેથોડ એનોડ
\end{verbatim}

\textbf{કાર્યસિદ્ધાંત:}

\begin{itemize}
\tightlist
\item
  \textbf{પ્રકાશ અવશોષણ}: ઇલેક્ટ્રોન-હોલ પેર્સ બનાવે
\item
  \textbf{રિવર્સ બાયાસ}: ડિપ્લીશન રીજીયન વિસ્તૃત કરે
\item
  \textbf{ફોટોકરંટ}: પ્રકાશ તીવ્રતાના પ્રમાણમાં
\item
  \textbf{ઝડપી પ્રતિસાદ}: ઝડપી ડિટેક્શન ક્ષમતા
\end{itemize}

\textbf{લક્ષણો:}

\begin{longtable}[]{@{}ll@{}}
\toprule\noalign{}
પેરામીટર & વર્ણન \\
\midrule\noalign{}
\endhead
\bottomrule\noalign{}
\endlastfoot
\textbf{ડાર્ક કરંટ} & પ્રકાશ વિના કરંટ \\
\textbf{ફોટોકરંટ} & પ્રકાશના પ્રમાણમાં કરંટ \\
\textbf{રેસ્પોન્સિવિટી} & યુનિટ લાઇટ પાવર પર કરંટ \\
\textbf{રેસ્પોન્સ ટાઇમ} & ડિટેક્શનની ગતિ \\
\end{longtable}

\textbf{ઉપયોગો:}

\begin{itemize}
\tightlist
\item
  \textbf{લાઇટ સેન્સર્સ}: ઓટોમેટિક લાઇટિંગ સિસ્ટમ
\item
  \textbf{ઓપ્ટિકલ કમ્યુનિકેશન}: ફાઇબર ઓપ્ટિક રિસીવર્સ
\item
  \textbf{સેફટી સિસ્ટમ}: સ્મોક ડિટેક્ટર્સ
\item
  \textbf{સોલાર પેનલ્સ}: પ્રકાશથી ઇલેક્ટ્રિકલ એનર્જી
\end{itemize}

\end{solutionbox}
\begin{mnemonicbox}
``ફોટો પ્રોડ્યુસેસ પ્રોપોર્શનલ કરંટ''

\end{mnemonicbox}
\subsection*{પ્રશ્ન 3(ક OR) [7
ગુણ]}\label{uxaaauxab0uxab6uxaa8-3uxa95-or-7-uxa97uxaa3}

\textbf{ઝેનર ડાયોડને વોલ્ટેજ રેગ્યુલેટરના સ્વરૂપે સમજાવો.}

\begin{solutionbox}

\textbf{વોલ્ટેજ રેગ્યુલેટર સર્કિટ:}

\begin{verbatim}
    Vin ○──Rs──┬────○ Vout = Vz
               │
               Z ↓ (ઝેનર)
               │
               ○ Ground
\end{verbatim}

\textbf{કાર્યસિદ્ધાંત:}

\begin{itemize}
\tightlist
\item
  \textbf{ઝેનર ઓપરેટ} બ્રેકડાઉન રીજીયનમાં
\item
  \textbf{આઉટપુટ વોલ્ટેજ} Vz પર કોન્સ્ટન્ટ રહે
\item
  \textbf{સીરીઝ રેઝિસ્ટર Rs} કરંટ લિમિટ કરે
\item
  \textbf{લોડ ચેન્જ} આઉટપુટ વોલ્ટેજને અસર કરતા નથી
\end{itemize}

\textbf{ડિઝાઇન સમીકરણો:}

\begin{longtable}[]{@{}ll@{}}
\toprule\noalign{}
પેરામીટર & ફોર્મ્યુલા \\
\midrule\noalign{}
\endhead
\bottomrule\noalign{}
\endlastfoot
\textbf{સીરીઝ રેઝિસ્ટન્સ} & Rs = (Vin - Vz) / Iz \\
\textbf{લોડ કરંટ} & IL = Vz / RL \\
\textbf{ઝેનર કરંટ} & Iz = Is - IL \\
\textbf{પાવર ડિસિપેશન} & Pz = Vz \times Iz \\
\end{longtable}

\textbf{રેગ્યુલેશન લક્ષણો:}

\begin{itemize}
\tightlist
\item
  \textbf{લાઇન રેગ્યુલેશન}: ઇનપુટ વેરિએશન સાથે આઉટપુટ ચેન્જ
\item
  \textbf{લોડ રેગ્યુલેશન}: લોડ વેરિએશન સાથે આઉટપુટ ચેન્જ
\item
  \textbf{કાર્યક્ષમતા}: ઝેનર પાવર લોસને કારણે સામાન્યે ઓછી
\end{itemize}

\textbf{ફાયદા:}

\begin{itemize}
\tightlist
\item
  \textbf{સિમ્પલ સર્કિટ}: ઓછા કમ્પોનન્ટ્સ જરૂરી
\item
  \textbf{સારું રેગ્યુલેશન}: સ્ટેબલ આઉટપુટ વોલ્ટેજ
\item
  \textbf{ઝડપી પ્રતિસાદ}: ઝડપી વોલ્ટેજ કરેક્શન
\end{itemize}

\textbf{મર્યાદાઓ:}

\begin{itemize}
\tightlist
\item
  \textbf{નબળી કાર્યક્ષમતા}: ઝેનરમાં પાવર વેસ્ટ
\item
  \textbf{મર્યાદિત કરંટ}: ઉચ્ચ કરંટ સપ્લાય કરી શકતું નથી
\item
  \textbf{તાપમાન સેન્સિટિવિટી}: તાપમાન સાથે વોલ્ટેજ બદલાય
\end{itemize}

\textbf{ઉપયોગો:}

\begin{itemize}
\tightlist
\item
  \textbf{રેફરન્સ વોલ્ટેજ}: ચોક્કસ વોલ્ટેજ સોર્સ
\item
  \textbf{સિમ્પલ રેગ્યુલેટર્સ}: ઓછા કરંટ એપ્લિકેશન
\item
  \textbf{પ્રોટેક્શન સર્કિટ્સ}: ઓવરવોલ્ટેજ પ્રોટેક્શન
\end{itemize}

\end{solutionbox}
\begin{mnemonicbox}
``ઝેનર ઝોન્સ ઝીરો વેરિએશન પૂરા પાડે''

\end{mnemonicbox}
\subsection*{પ્રશ્ન 4(અ) [3
ગુણ]}\label{uxaaauxab0uxab6uxaa8-4uxa85-3-uxa97uxaa3}

\textbf{PNP અને NPN ટ્રાન્ઝિસ્ટરની સંજ્ઞા અને બંધારણ યોગ્ય નામ નિર્દેશ સાથે દોરો.}

\begin{solutionbox}

\textbf{ટ્રાન્ઝિસ્ટર પ્રતીકો:}

\begin{verbatim}
    NPN ટ્રાન્ઝિસ્ટર:        PNP ટ્રાન્ઝિસ્ટર:
    
    કલેક્ટર (C)          કલેક્ટર (C)
         │                      │
         ○                      ○
         │                      │
    બેસ ○─┤                બેસ ○─┤
         │ ↘                   │  ↙
         ○ એમિટર              ○ એમિટર
         │ (E)                  │ (E)
\end{verbatim}

\textbf{બંધારણ ડાયાગ્રામ:}

\begin{verbatim}
    NPN સ્ટ્રક્ચર:         PNP સ્ટ્રક્ચર:
    
    ○ કલેક્ટર            ○ કલેક્ટર  
    │ N{-type               │ P{-}type}
    ├─────────             ├─────────
    │ P{-type    બેસ ○     │ N{-}type    બેસ ○}
    ├─────────             ├─────────
    │ N{-type               │ P{-}type}
    ○ એમિટર              ○ એમિટર
\end{verbatim}

\textbf{ટર્મિનલ ઓળખ:}

\begin{itemize}
\tightlist
\item
  \textbf{એમિટર}: હેવી ડોપ્ડ, તીર કરંટ દિશા દર્શાવે
\item
  \textbf{બેસ}: પાતળું, લાઇટ ડોપ્ડ મધ્ય વિસ્તાર
\item
  \textbf{કલેક્ટર}: મોડરેટ ડોપ્ડ, ચાર્જ કેરિયર્સ એકત્રિત કરે
\end{itemize}

\textbf{કરંટ દિશા:}

\begin{itemize}
\tightlist
\item
  \textbf{NPN}: તીર બહારની તરફ પોઇન્ટ કરે (એમિટર થી બેસ)
\item
  \textbf{PNP}: તીર અંદરની તરફ પોઇન્ટ કરે (બેસ થી એમિટર)
\end{itemize}

\end{solutionbox}
\begin{mnemonicbox}
``NPN: અંદર પોઇન્ટ નથી, PNP: અંદર પોઇન્ટ કરે''

\end{mnemonicbox}
\subsection*{પ્રશ્ન 4(બ) [4
ગુણ]}\label{uxaaauxab0uxab6uxaa8-4uxaac-4-uxa97uxaa3}

\textbf{CE એમ્પ્લિફાયરની લાક્ષણિકતાઓ દોરો અને સમજાવો.}

\begin{solutionbox}

\textbf{CE એમ્પ્લિફાયર સર્કિટ:}

\begin{verbatim}
    Vcc
     │
     Rc
     │
    ○ Vout
     │
    ─C─  કલેક્ટર
     │
    ─B─  બેસ ○ Vin
     │
    ─E─  એમિટર
     │
     Re
     │
    ○ Ground
\end{verbatim}

\textbf{ઇનપુટ લાક્ષણિકતાઓ (IB vs VBE):}

\begin{verbatim}
    IB │
    (mA)│     ┌──
        │   ┌─┘
        │ ┌─┘
        │┌┘
        └──────── VBE (V)
         0  0.7
\end{verbatim}

\textbf{આઉટપુટ લાક્ષણિકતાઓ (IC vs VCE):}

\begin{verbatim}
    IC │  IB = 40µA
    (mA)│  ┌──────────
        │  │ IB = 30µA
        │ ┌┴──────────
        │ │ IB = 20µA  
        │┌┴───────────
        ││ IB = 10µA
        └┴─────────── VCE (V)
         0    5   10
\end{verbatim}

\textbf{મુખ્ય લક્ષણો:}

\begin{longtable}[]{@{}ll@{}}
\toprule\noalign{}
પેરામીટર & CE કન્ફિગરેશન \\
\midrule\noalign{}
\endhead
\bottomrule\noalign{}
\endlastfoot
\textbf{કરંટ ગેઇન} & β = IC/IB (ઉચ્ચ) \\
\textbf{વોલ્ટેજ ગેઇન} & ઉચ્ચ \\
\textbf{પાવર ગેઇન} & ખૂબ ઉચ્ચ \\
\textbf{ઇનપુટ ઇમ્પીડન્સ} & મધ્યમ \\
\textbf{આઉટપુટ ઇમ્પીડન્સ} & ઉચ્ચ \\
\textbf{ફેઝ શિફ્ટ} & 180^\circ \\
\end{longtable}

\textbf{ઓપરેશનના વિસ્તારો:}

\begin{itemize}
\tightlist
\item
  \textbf{કટ-ઓફ}: બંને જંક્શન રિવર્સ બાયાસ્ડ
\item
  \textbf{એક્ટિવ}: BE ફોરવર્ડ, BC રિવર્સ બાયાસ્ડ
\item
  \textbf{સેચ્યુરેશન}: બંને જંક્શન ફોરવર્ડ બાયાસ્ડ
\end{itemize}

\end{solutionbox}
\begin{mnemonicbox}
``કોમન એમિટર, કરંટ એન્લાર્જ્ડ''

\end{mnemonicbox}
\subsection*{પ્રશ્ન 4(ક) [7
ગુણ]}\label{uxaaauxab0uxab6uxaa8-4uxa95-7-uxa97uxaa3}

\textbf{કરંટ ગેઇન α, β અને γ વચ્ચેનો સંબંધ મેળવો.}

\begin{solutionbox}

\textbf{કરંટ ગેઇન વ્યાખ્યાઓ:}

\begin{longtable}[]{@{}lll@{}}
\toprule\noalign{}
ગેઇન & કન્ફિગરેશન & ફોર્મ્યુલા \\
\midrule\noalign{}
\endhead
\bottomrule\noalign{}
\endlastfoot
\textbf{α (આલ્ફા)} & કોમન બેસ & α = IC/IE \\
\textbf{β (બીટા)} & કોમન એમિટર & β = IC/IB \\
\textbf{γ (ગામા)} & કોમન કલેક્ટર & γ = IE/IB \\
\end{longtable}

\textbf{વ્યુત્પત્તિ:}

\textbf{પગલું 1: મૂળભૂત કરંટ સંબંધ} IE = IB + IC \ldots{} (કિર્ચહોફનો કરંટ કાયદો)

\textbf{પગલું 2: IE ના સંદર્ભમાં IC વ્યક્ત કરો} α = IC/IE તેથી: IC = α \times IE
\ldots{} (1)

\textbf{પગલું 3: કરંટ સમીકરણમાં બદલો} IE = IB + α \times IE IE - α \times IE = IB IE(1
- α) = IB IE = IB/(1 - α) \ldots{} (2)

\textbf{પગલું 4: β શોધો} β = IC/IB (1) થી: IC = α \times IE (2) થી: IE = IB/(1
- α) તેથી: IC = α \times IB/(1 - α)

\textbf{પગલું 5: β માટે અંતિમ સંબંધ} β = IC/IB = α/(1 - α) \ldots{} (3)

\textbf{પગલું 6: β ના સંદર્ભમાં α વ્યક્ત કરો} સમીકરણ (3) થી: β(1 - α) = α β -
βα = α

β = α + βα = α(1 + β) તેથી:

α = β/(1 + β) \ldots{} (4)


\textbf{પગલું 7: γ શોધો} γ = IE/IB (2) થી: γ = 1/(1 - α) (4) થી α બદલતાં: γ
= 1/(1 - β/(1 + β))

γ = (1 + β)/(1 + β - β)

γ = 1 + β \ldots{} (5)


\textbf{અંતિમ સંબંધો:}

\begin{longtable}[]{@{}ll@{}}
\toprule\noalign{}
સંબંધ & ફોર્મ્યુલા \\
\midrule\noalign{}
\endhead
\bottomrule\noalign{}
\endlastfoot
\textbf{α ના સંદર્ભમાં β} & β = α/(1 - α) \\
\textbf{β ના સંદર્ભમાં α} & α = β/(1 + β) \\
\textbf{β ના સંદર્ભમાં γ} & γ = 1 + β \\
\textbf{ચકાસણી} & α + β \times α = β \\
\end{longtable}

\textbf{સામાન્ય મૂલ્યો:}

\begin{itemize}
\tightlist
\item
  α \approx 0.98 થી 0.995
\item
  β \approx 50 થી 200\\
\item
  γ \approx 51 થી 201
\end{itemize}

\end{solutionbox}
\begin{mnemonicbox}
``આલ્ફા બીટા ગામા, હંમેશાં સારા ગેઇન્સ''

\end{mnemonicbox}
\subsection*{પ્રશ્ન 4(અ OR) [3
ગુણ]}\label{uxaaauxab0uxab6uxaa8-4uxa85-or-3-uxa97uxaa3}

\textbf{ટ્રાન્ઝિસ્ટર એમ્પ્લિફાયર માટે એક્ટિવ, સેચ્યુરેશન અને કટ-ઓફ રીજીયનની વ્યાખ્યા
આપો.}

\begin{solutionbox}

\textbf{ઓપરેટિંગ રીજીયન્સ:}

\begin{longtable}[]{@{}llll@{}}
\toprule\noalign{}
રીજીયન & બેસ-એમિટર & બેસ-કલેક્ટર & લાક્ષણિકતાઓ \\
\midrule\noalign{}
\endhead
\bottomrule\noalign{}
\endlastfoot
\textbf{એક્ટિવ} & ફોરવર્ડ બાયાસ્ડ & રિવર્સ બાયાસ્ડ & એમ્પ્લિફિકેશન રીજીયન \\
\textbf{સેચ્યુરેશન} & ફોરવર્ડ બાયાસ્ડ & ફોરવર્ડ બાયાસ્ડ & સ્વિચ ON સ્ટેટ \\
\textbf{કટ-ઓફ} & રિવર્સ બાયાસ્ડ & રિવર્સ બાયાસ્ડ & સ્વિચ OFF સ્ટેટ \\
\end{longtable}

\textbf{વિગતવાર વર્ણન:}

\textbf{એક્ટિવ રીજીયન:}

\begin{itemize}
\tightlist
\item
  \textbf{સામાન્ય એમ્પ્લિફિકેશન} મોડ
\item
  \textbf{IC = β \times IB} સંબંધ લાગુ
\item
  \textbf{નાના સિગ્નલ્સ માટે લીનિયર} ઓપરેશન
\end{itemize}

\textbf{સેચ્યુરેશન રીજીયન:}

\begin{itemize}
\tightlist
\item
  \textbf{બંને જંક્શન} ફોરવર્ડ બાયાસ્ડ
\item
  \textbf{મેક્સિમમ કલેક્ટર કરંટ} વહે
\item
  \textbf{VCE \approx 0.2V} (ખૂબ ઓછું)
\item
  \textbf{સ્વિચિંગ એપ્લિકેશન્સમાં} ઉપયોગ
\end{itemize}

\textbf{કટ-ઓફ રીજીયન:}

\begin{itemize}
\tightlist
\item
  \textbf{કોઈ બેસ કરંટ નથી} (IB = 0)
\item
  \textbf{કોઈ કલેક્ટર કરંટ નથી} (IC = 0)\\
\item
  \textbf{ટ્રાન્ઝિસ્ટર ઓપન સ્વિચ જેવું} કામ કરે
\end{itemize}

\end{solutionbox}
\begin{mnemonicbox}
``એક્ટિવ એમ્પ્લિફાય, સેચ્યુરેટેડ સ્વિચ, કટ-ઓફ કટ્સ''

\end{mnemonicbox}
\subsection*{પ્રશ્ન 4(બ OR) [4
ગુણ]}\label{uxaaauxab0uxab6uxaa8-4uxaac-or-4-uxa97uxaa3}

\textbf{એમ્પ્લિફાયર તરીકે ટ્રાન્ઝિસ્ટરનું કાર્ય સમજાવો.}

\begin{solutionbox}

\textbf{એમ્પ્લિફાયર સર્કિટ:}

\begin{verbatim}
    Vcc
     │
     Rc
     │  
    ○ Vout (એમ્પ્લિફાઇડ)
     │
    ─C─  NPN ટ્રાન્ઝિસ્ટર
     │
    ○ Vin ─B─
     │
    ─E─
     │
     Re
     │
    ○ Ground
\end{verbatim}

\textbf{કાર્યસિદ્ધાંત:}

\begin{itemize}
\tightlist
\item
  \textbf{નાનું ઇનપુટ સિગ્નલ} બેસ-એમિટર પર લાગુ
\item
  \textbf{ઇનપુટ રેઝિસ્ટન્સ} ઓછું (કેટલાક kΩ)
\item
  \textbf{નાનું બેસ કરંટ} મોટા કલેક્ટર કરંટને નિયંત્રિત કરે
\item
  \textbf{આઉટપુટ} કલેક્ટર-એમિટરથી લેવાય
\item
  \textbf{કરંટ એમ્પ્લિફિકેશન}: IC = β \times IB
\end{itemize}

\textbf{એમ્પ્લિફિકેશન પ્રક્રિયા:}

\begin{longtable}[]{@{}lll@{}}
\toprule\noalign{}
પેરામીટર & ઇનપુટ & આઉટપુટ \\
\midrule\noalign{}
\endhead
\bottomrule\noalign{}
\endlastfoot
\textbf{સિગ્નલ લેવલ} & નાનું & મોટું \\
\textbf{કરંટ} & µA રેન્જ & mA રેન્જ \\
\textbf{વોલ્ટેજ} & mV રેન્જ & V રેન્જ \\
\textbf{પાવર} & µW રેન્જ & mW રેન્જ \\
\end{longtable}

\textbf{મુખ્ય લક્ષણો:}

\begin{itemize}
\tightlist
\item
  \textbf{કરંટ ગેઇન}: β (50-200 સામાન્ય)
\item
  \textbf{વોલ્ટેજ ગેઇન}: લોડ રેઝિસ્ટન્સ પર આધાર રાખે
\item
  \textbf{પાવર ગેઇન}: કરંટ અને વોલ્ટેજ ગેઇનનું ગુણાકાર
\item
  \textbf{ફેઝ ઇન્વર્ઝન}: CE કન્ફિગરેશનમાં 180^\circ
\end{itemize}

\textbf{ઉપયોગો:}

\begin{itemize}
\tightlist
\item
  \textbf{ઓડિયો એમ્પ્લિફાયર્સ}: મ્યુઝિક સિસ્ટમ
\item
  \textbf{RF એમ્પ્લિફાયર્સ}: રેડિયો ટ્રાન્સમિટર્સ
\item
  \textbf{Op-amp સ્ટેજિસ}: ઇન્ટિગ્રેટેડ સર્કિટ્સ
\end{itemize}

\end{solutionbox}
\begin{mnemonicbox}
``નાનું સિગ્નલ મોટું આઉટપુટ ટ્રિગર કરે''

\end{mnemonicbox}
\subsection*{પ્રશ્ન 4(ક OR) [7
ગુણ]}\label{uxaaauxab0uxab6uxaa8-4uxa95-or-7-uxa97uxaa3}

\textbf{CB, CC તેમજ CE એમ્પ્લિફાયરને સરખાવો.}

\begin{solutionbox}

\textbf{વ્યાપક તુલના:}

\begin{longtable}[]{@{}llll@{}}
\toprule\noalign{}
પેરામીટર & કોમન બેસ (CB) & કોમન એમિટર (CE) & કોમન કલેક્ટર (CC) \\
\midrule\noalign{}
\endhead
\bottomrule\noalign{}
\endlastfoot
\textbf{ઇનપુટ ટર્મિનલ} & એમિટર & બેસ & બેસ \\
\textbf{આઉટપુટ ટર્મિનલ} & કલેક્ટર & કલેક્ટર & એમિટર \\
\textbf{કોમન ટર્મિનલ} & બેસ & એમિટર & કલેક્ટર \\
\textbf{કરંટ ગેઇન} & α \textless{} 1 & β \textgreater\textgreater{} 1 & γ
= (1 + β) \\
\textbf{વોલ્ટેજ ગેઇન} & ઉચ્ચ & ઉચ્ચ & \textless{} 1 (\approx1) \\
\textbf{પાવર ગેઇન} & મધ્યમ & ખૂબ ઉચ્ચ & મધ્યમ \\
\textbf{ઇનપુટ રેઝિસ્ટન્સ} & ખૂબ ઓછું (20-50Ω) & મધ્યમ (1-5kΩ) & ખૂબ ઉચ્ચ
(100kΩ) \\
\textbf{આઉટપુટ રેઝિસ્ટન્સ} & ખૂબ ઉચ્ચ (1MΩ) & ઉચ્ચ (50kΩ) & ઓછું (25Ω) \\
\textbf{ફેઝ શિફ્ટ} & 0^\circ & 180^\circ & 0^\circ \\
\textbf{ફ્રીક્વન્સી રેસ્પોન્સ} & ઉત્તમ & સારું & સારું \\
\textbf{એપ્લિકેશન્સ} & RF એમ્પ્લિફાયર્સ & ઓડિયો એમ્પ્લિફાયર્સ & બફર, ઇમ્પીડન્સ
મેચિંગ \\
\end{longtable}

\textbf{સર્કિટ ડાયાગ્રામ:}

\textbf{કોમન બેસ:}

\begin{verbatim}
    Vcc          Vcc          Vcc
     │            │            │
     Rc           Rc           Re
     │            │            │
    ○Vout       ○Vout        ○Vin
     │            │            │
    ─C─          ─C─          ─C─
     │            │            │
    ─B─○Ground   ○Vin─B─      ─B─
     │            │            │
    ○Vin─E─      ─E─○Ground   ─E─○Vout
\end{verbatim}

\textbf{મુખ્ય લાક્ષણિકતાઓ:}

\textbf{કોમન બેસ (CB):}

\begin{itemize}
\tightlist
\item
  \textbf{ઉચ્ચ ફ્રીક્વન્સી} પર્ફોર્મન્સ
\item
  \textbf{કરંટ ગેઇન નથી} પણ ઉચ્ચ વોલ્ટેજ ગેઇન
\item
  \textbf{ઇનપુટ-આઉટપુટ આઇસોલેશન} ઉત્તમ
\item
  \textbf{ઉપયોગ}: RF એમ્પ્લિફાયર્સ, ઉચ્ચ ફ્રીક્વન્સી સર્કિટ્સ
\end{itemize}

\textbf{કોમન એમિટર (CE):}

\begin{itemize}
\tightlist
\item
  \textbf{સૌથી વધુ લોકપ્રિય} કન્ફિગરેશન
\item
  \textbf{ઉચ્ચ કરંટ અને વોલ્ટેજ} ગેઇન
\item
  \textbf{બધા પેરામીટર્સનો સારો} સમજૂતો
\item
  \textbf{ઉપયોગ}: ઓડિયો એમ્પ્લિફાયર્સ, સામાન્ય એમ્પ્લિફિકેશન
\end{itemize}

\textbf{કોમન કલેક્ટર (CC):}

\begin{itemize}
\tightlist
\item
  \textbf{યુનિટી વોલ્ટેજ ગેઇન} (વોલ્ટેજ ફોલોઅર)
\item
  \textbf{ઉચ્ચ કરંટ ગેઇન}
\item
  \textbf{ઇમ્પીડન્સ ટ્રાન્સફોર્મેશન} (ઉચ્ચથી ઓછું)
\item
  \textbf{ઉપયોગ}: બફર એમ્પ્લિફાયર્સ, ઇમ્પીડન્સ મેચિંગ
\end{itemize}

\textbf{પસંદગીના માપદંડો:}

\begin{longtable}[]{@{}lll@{}}
\toprule\noalign{}
એપ્લિકેશન & શ્રેષ્ઠ કન્ફિગરેશન & કારણ \\
\midrule\noalign{}
\endhead
\bottomrule\noalign{}
\endlastfoot
\textbf{ઉચ્ચ ફ્રીક્વન્સી} & CB & ઉત્તમ ફ્રીક્વન્સી રેસ્પોન્સ \\
\textbf{સામાન્ય એમ્પ્લિફિકેશન} & CE & ઉચ્ચ પાવર ગેઇન \\
\textbf{બફર/આઇસોલેશન} & CC & ઉચ્ચ ઇનપુટ, ઓછું આઉટપુટ ઇમ્પીડન્સ \\
\textbf{પાવર એમ્પ્લિફાયર્સ} & CE & મેક્સિમમ પાવર ગેઇન \\
\end{longtable}

\end{solutionbox}
\begin{mnemonicbox}
``CB કમ્યુનિકેશન માટે, CE કોમન યુઝ માટે, CC કપલિંગ માટે''

\end{mnemonicbox}
\subsection*{પ્રશ્ન 5(અ) [3
ગુણ]}\label{uxaaauxab0uxab6uxaa8-5uxa85-3-uxa97uxaa3}

\textbf{IC 555 નો પિન ડાયાગ્રામ દોરો.}

\begin{solutionbox}

\textbf{IC 555 પિન ડાયાગ્રામ:}

\begin{verbatim}
    ┌─────────────────┐
    │    IC 555       │
  1 │○ Ground         │ 8 ○ Vcc
  2 │○ Trigger        │ 7 ○ Discharge  
  3 │○ Output         │ 6 ○ Threshold
  4 │○ Reset          │ 5 ○ Control Voltage
    └─────────────────┘
          DIP{-8 Package}
\end{verbatim}

\textbf{પિન કાર્યો:}

\begin{longtable}[]{@{}lll@{}}
\toprule\noalign{}
પિન & નામ & કાર્ય \\
\midrule\noalign{}
\endhead
\bottomrule\noalign{}
\endlastfoot
\textbf{1} & Ground & 0V રેફરન્સ \\
\textbf{2} & Trigger & ટાઇમિંગ સાયકલ શરૂ કરે \\
\textbf{3} & Output & ટાઇમર આઉટપુટ \\
\textbf{4} & Reset & માસ્ટર રીસેટ (એક્ટિવ લો) \\
\textbf{5} & Control & વોલ્ટેજ રેફરન્સ કંટ્રોલ \\
\textbf{6} & Threshold & ટાઇમિંગ સાયકલ બંધ કરે \\
\textbf{7} & Discharge & ટાઇમિંગ કેપેસિટર ડિસ્ચાર્જ \\
\textbf{8} & Vcc & પાવર સપ્લાય (+5V થી +18V) \\
\end{longtable}

\textbf{મુખ્ય મુદ્દાઓ:}

\begin{itemize}
\tightlist
\item
  \textbf{ડ્યુઅલ-ઇન-લાઇન} 8-પિન પેકેજ
\item
  \textbf{પાવર સપ્લાય}: 5V થી 18V DC
\item
  \textbf{આઉટપુટ કરંટ}: 200mA સુધી
\item
  \textbf{રીસેટ પિન}: સામાન્યે Vcc સાથે જોડાયેલ
\end{itemize}

\end{solutionbox}
\begin{mnemonicbox}
``ગ્રેટ ટાઇમર, ગ્રેટ પિન્સ''

\end{mnemonicbox}
\subsection*{પ્રશ્ન 5(બ) [4
ગુણ]}\label{uxaaauxab0uxab6uxaa8-5uxaac-4-uxa97uxaa3}

\textbf{555 ટાઇમર IC ની વિશેષતાઓની યાદી બનાવો.}

\begin{solutionbox}

\textbf{મુખ્ય લક્ષણો:}

\begin{longtable}[]{@{}ll@{}}
\toprule\noalign{}
લક્ષણ & વિશિષ્ટતા \\
\midrule\noalign{}
\endhead
\bottomrule\noalign{}
\endlastfoot
\textbf{સપ્લાય વોલ્ટેજ} & 5V થી 18V \\
\textbf{આઉટપુટ કરંટ} & 200mA સોર્સ/સિંક \\
\textbf{તાપમાન રેન્જ} & 0^\circC થી 70^\circC \\
\textbf{ટાઇમિંગ રેન્જ} & µs થી કલાકો \\
\textbf{ચોકસાઇ} & \pm1\% સામાન્ય \\
\textbf{મોડ્સ} & મોનોસ્ટેબલ, એસ્ટેબલ, બિસ્ટેબલ \\
\end{longtable}

\textbf{ટેકનિકલ લક્ષણો:}

\begin{itemize}
\tightlist
\item
  \textbf{CMOS/TTL કોમ્પેટિબલ} આઉટપુટ લેવલ્સ
\item
  \textbf{ઉચ્ચ કરંટ} આઉટપુટ ક્ષમતા
\item
  \textbf{વાઇડ સપ્લાય વોલ્ટેજ} રેન્જ
\item
  \textbf{તાપમાન સ્ટેબલ} ઓપરેશન
\end{itemize}

\textbf{કાર્યાત્મક લક્ષણો:}

\begin{itemize}
\tightlist
\item
  \textbf{ત્રણ ઓપરેટિંગ મોડ્સ} ઉપલબ્ધ
\item
  \textbf{બાહ્ય ટાઇમિંગ} કમ્પોનન્ટ્સ
\item
  \textbf{રીસેટ ક્ષમતા} કંટ્રોલ માટે
\item
  \textbf{ઓછા પાવર કન્ઝમ્પશન} ડિઝાઇન
\end{itemize}

\textbf{ફાયદા:}

\begin{itemize}
\tightlist
\item
  \textbf{વર્સેટાઇલ ટાઇમર} અનેક એપ્લિકેશન્સ માટે
\item
  \textbf{વાપરવામાં સરળ} ન્યૂનતમ બાહ્ય કમ્પોનન્ટ્સ સાથે
\item
  \textbf{વિશ્વસનીય ઓપરેશન} વિવિધ પરિસ્થિતિઓમાં
\end{itemize}

\end{solutionbox}
\begin{mnemonicbox}
``શાનદાર લક્ષણો, લવચીક કાર્યો''

\end{mnemonicbox}
\subsection*{પ્રશ્ન 5(ક) [7
ગુણ]}\label{uxaaauxab0uxab6uxaa8-5uxa95-7-uxa97uxaa3}

\textbf{555 ટાઇમર IC નો ઉપયોગ કરીને મોનો સ્ટેબલ મલ્ટીવાઇબ્રેટર સમજાવો.}

\begin{solutionbox}

\textbf{મોનોસ્ટેબલ સર્કિટ:}

\begin{verbatim}
    Vcc
     │
     ├─────○ 8 (Vcc)
     │
     R ────○ 7 (Discharge)
     │     │
     ├─────○ 6 (Threshold)
     │     │
    ○2○────┤ 4 (Reset)
     │  5○─┴─ (Control)
    ○3○     │
     │     ○ 1 (Ground)
     │     │
     C ────┘
     │
    ○ Ground
\end{verbatim}

\textbf{કાર્યસિદ્ધાંત:}

\textbf{સ્ટેબલ સ્ટેટ:}

\begin{itemize}
\tightlist
\item
  \textbf{આઉટપુટ LOW} (લગભગ 0V)
\item
  \textbf{કેપેસિટર ડિસ્ચાર્જ્ડ} પિન 7 મારફત
\item
  \textbf{થ્રેશહોલ્ડ વોલ્ટેજ} Vcc/3 થી નીચે
\end{itemize}

\textbf{ટ્રિગર્ડ સ્ટેટ:}

\begin{itemize}
\tightlist
\item
  \textbf{નેગેટિવ પલ્સ} ટ્રિગર (પિન 2) પર લાગુ
\item
  \textbf{આઉટપુટ HIGH તરત} જાય
\item
  \textbf{ડિસ્ચાર્જ ટ્રાન્ઝિસ્ટર} બંધ થાય
\item
  \textbf{કેપેસિટર R મારફત} ચાર્જ શરૂ કરે
\end{itemize}

\textbf{ટાઇમિંગ પીરિયડ:}

\begin{itemize}
\tightlist
\item
  \textbf{અવધિ}: T = 1.1 \times R \times C
\item
  \textbf{આઉટપુટ HIGH રહે} ગણતરી કરેલા સમય માટે
\item
  \textbf{ઓટોમેટિક રિટર્ન} સ્ટેબલ સ્ટેટમાં
\end{itemize}

\textbf{સ્ટેબલમાં પાછા ફરવું:}

\begin{itemize}
\tightlist
\item
  \textbf{કેપેસિટર વોલ્ટેજ} 2Vcc/3 સુધી પહોંચે
\item
  \textbf{થ્રેશહોલ્ડ ટ્રિગર} (પિન 6)
\item
  \textbf{આઉટપુટ LOW પર} પાછું
\item
  \textbf{ડિસ્ચાર્જ ફરીથી} શરૂ
\end{itemize}

\textbf{મુખ્ય લાક્ષણિકતાઓ:}

\begin{longtable}[]{@{}ll@{}}
\toprule\noalign{}
પેરામીટર & વર્ણન \\
\midrule\noalign{}
\endhead
\bottomrule\noalign{}
\endlastfoot
\textbf{પલ્સ વિડ્થ} & T = 1.1 RC \\
\textbf{ટ્રિગર લેવલ} & Vcc/3 \\
\textbf{થ્રેશહોલ્ડ લેવલ} & 2Vcc/3 \\
\textbf{આઉટપુટ HIGH} & \textasciitilde Vcc - 1.5V \\
\textbf{આઉટપુટ LOW} & \textasciitilde0.1V \\
\end{longtable}

\textbf{એપ્લિકેશન્સ:}

\begin{itemize}
\tightlist
\item
  \textbf{પલ્સ જનરેશન}: ફિક્સ્ડ વિડ્થ પલ્સિસ
\item
  \textbf{ટાઇમ ડિલે}: સ્વિચ-ઓન ડિલે
\item
  \textbf{મિસિંગ પલ્સ ડિટેક્શન}: વોચડોગ ટાઇમર્સ
\item
  \textbf{ડિબાઉન્સિંગ સર્કિટ્સ}: સ્વિચ કોન્ટેક્ટ ક્લીનિંગ
\end{itemize}

\textbf{ડિઝાઇન ઉદાહરણ:} T = 1ms માટે: જો C = 0.1µF, તો R = 9.1kΩ

\end{solutionbox}
\begin{mnemonicbox}
``મોનો મતલબ એક પલ્સ માત્ર''

\end{mnemonicbox}
\subsection*{પ્રશ્ન 5(અ OR) [3
ગુણ]}\label{uxaaauxab0uxab6uxaa8-5uxa85-or-3-uxa97uxaa3}

\textbf{IC 555 ની એપ્લિકેશનની યાદી બનાવો.}

\begin{solutionbox}

\textbf{ટાઇમર એપ્લિકેશન્સ:}

\begin{longtable}[]{@{}ll@{}}
\toprule\noalign{}
કેટેગરી & એપ્લિકેશન્સ \\
\midrule\noalign{}
\endhead
\bottomrule\noalign{}
\endlastfoot
\textbf{ટાઇમિંગ સર્કિટ્સ} & ડિલે ટાઇમર્સ, પલ્સ જનરેટર્સ \\
\textbf{ઓસિલેટર્સ} & ક્લોક જનરેટર્સ, ફ્રીક્વન્સી ડિવાઇડર્સ \\
\textbf{કંટ્રોલ સર્કિટ્સ} & PWM કંટ્રોલર્સ, મોટર સ્પીડ કંટ્રોલ \\
\textbf{ડિટેક્શન} & મિસિંગ પલ્સ ડિટેક્ટર્સ, બર્ગલર એલાર્મ \\
\textbf{કમ્યુનિકેશન} & ટોન જનરેટર્સ, ફ્રીક્વન્સી શિફ્ટ કીઇંગ \\
\textbf{ઓટોમોટિવ} & ટર્ન સિગ્નલ ફ્લેશર્સ, વિન્ડશીલ્ડ વાઇપર્સ \\
\end{longtable}

\textbf{મોડ-વાઇઝ એપ્લિકેશન્સ:}

\textbf{મોનોસ્ટેબલ મોડ:}

\begin{itemize}
\tightlist
\item
  \textbf{સર્કિટ્સમાં ટાઇમ ડિલે}
\item
  \textbf{પલ્સ વિડ્થ} જનરેશન
\item
  \textbf{સ્વિચ ડિબાઉન્સિંગ}
\end{itemize}

\textbf{એસ્ટેબલ મોડ:}

\begin{itemize}
\tightlist
\item
  \textbf{LED ફ્લેશર્સ} અને બ્લિન્કર્સ
\item
  \textbf{ક્લોક સિગ્નલ્સ} જનરેશન
\item
  \textbf{બઝર માટે ટોન} જનરેશન
\end{itemize}

\textbf{બિસ્ટેબલ મોડ:}

\begin{itemize}
\tightlist
\item
  \textbf{ફ્લિપ-ફ્લોપ} સર્કિટ્સ
\item
  \textbf{મેમરી એલિમેન્ટ્સ}
\item
  \textbf{લેચ સર્કિટ્સ}
\end{itemize}

\textbf{સામાન્ય પ્રોજેક્ટ્સ:}

\begin{itemize}
\tightlist
\item
  \textbf{LED સાથે ઇલેક્ટ્રોનિક ડાઇસ}
\item
  \textbf{ટ્રાફિક લાઇટ} કંટ્રોલર્સ
\item
  \textbf{ડિજિટલ ક્લોક્સ} અને ટાઇમર્સ
\end{itemize}

\end{solutionbox}
\begin{mnemonicbox}
``મહાન કાર્યો માટે ટાઇમર''

\end{mnemonicbox}
\subsection*{પ્રશ્ન 5(બ OR) [4
ગુણ]}\label{uxaaauxab0uxab6uxaa8-5uxaac-or-4-uxa97uxaa3}

\textbf{IC 555 નો આંતરિક બ્લોક ડાયાગ્રામ દોરો અને સમજાવો.}

\begin{solutionbox}

\textbf{આંતરિક બ્લોક ડાયાગ્રામ:}

\begin{verbatim}
         Vcc (8)
          │
    ┌─────┴─────┐
    │  Voltage  │
    │ Divider   │  5V
    │    5kΩ    ├─────○ Control (5)
    │    │      │
    │   10V     │
    │    5kΩ    │
    │    │      │
    │   10V     │
    │    5kΩ    │
    └─────┬─────┘
          │
    ┌─────┴─────┐
    │     +     │
Threshold(6)○─┤Comparator├─┐
    │     {-     │   A     │}
    └───────────┘         │
                          │ ┌─── SR ───┐
    ┌───────────┐         ├─┤   Flip   ├─○ Output (3)
    │     +     │         │ │   Flop   │
Trigger(2)○─────┤Comparator├─┘ └─────────┘
    │     {-     │   B         │}
    └───────────┘           Reset(4)○─┘
                              │
    ┌─────────────────────────┴─────┐
    │      Discharge Transistor     ├─○ Discharge (7)
    └───────────────────────────────┘
                              │
                           Ground (1)
\end{verbatim}

\textbf{બ્લોક કાર્યો:}

\begin{longtable}[]{@{}ll@{}}
\toprule\noalign{}
બ્લોક & કાર્ય \\
\midrule\noalign{}
\endhead
\bottomrule\noalign{}
\endlastfoot
\textbf{વોલ્ટેજ ડિવાઇડર} & Vcc/3 અને 2Vcc/3 રેફરન્સ બનાવે \\
\textbf{કોમ્પેરેટર A} & થ્રેશહોલ્ડને 2Vcc/3 સાથે તુલના કરે \\
\textbf{કોમ્પેરેટર B} & ટ્રિગરને Vcc/3 સાથે તુલના કરે \\
\textbf{SR ફ્લિપ-ફ્લોપ} & આઉટપુટ સ્ટેટ નિયંત્રિત કરે \\
\textbf{ડિસ્ચાર્જ ટ્રાન્ઝિસ્ટર} & ટાઇમિંગ કેપેસિટર ડિસ્ચાર્જ કરે \\
\textbf{આઉટપુટ બફર} & ઉચ્ચ કરંટ આઉટપુટ પૂરું પાડે \\
\end{longtable}

\textbf{કાર્યપદ્ધતિ:}

\begin{itemize}
\tightlist
\item
  \textbf{કોમ્પેરેટર્સ} ફ્લિપ-ફ્લોપને સેટ અને રીસેટ કરે
\item
  \textbf{આઉટપુટ બફર} ફ્લિપ-ફ્લોપ આઉટપુટ એમ્પ્લિફાય કરે
\item
  \textbf{ડિસ્ચાર્જ ટ્રાન્ઝિસ્ટર} ફ્લિપ-ફ્લોપ દ્વારા નિયંત્રિત
\item
  \textbf{રેફરન્સ વોલ્ટેજિસ} ટ્રિગર લેવલ્સ સેટ કરે
\end{itemize}

\end{solutionbox}
\begin{mnemonicbox}
``આંતરિક બુદ્ધિ, ઇન્ટિગ્રેટેડ અમલીકરણ''

\end{mnemonicbox}
\subsection*{પ્રશ્ન 5(ક OR) [7
ગુણ]}\label{uxaaauxab0uxab6uxaa8-5uxa95-or-7-uxa97uxaa3}

\textbf{555 ટાઇમર IC નો ઉપયોગ કરીને એસ્ટેબલ મલ્ટીવાઇબ્રેટર સમજાવો.}

\begin{solutionbox}

\textbf{એસ્ટેબલ સર્કિટ:}

\begin{verbatim}
    Vcc
     │
     ├─────○ 8 (Vcc)
     │      │ 4 (Reset)
     R1────○ 7 (Discharge)
     │     │ 6 (Threshold)
     R2────┤
     │     │ 5 (Control)
    ○2○────┤
     │     │ 3 (Output)
    ○ │    │ 1 (Ground)
     │     │
     C ────┘
     │
    ○ Ground
\end{verbatim}

\textbf{કાર્યસિદ્ધાંત:}

\textbf{ચાર્જિંગ ફેઝ:}

\begin{itemize}
\tightlist
\item
  \textbf{કેપેસિટર R1 + R2 મારફત} ચાર્જ થાય
\item
  \textbf{ચાર્જિંગ દરમ્યાન આઉટપુટ HIGH}
\item
  \textbf{ચાર્જિંગ ટાઇમ}: T1 = 0.693(R1 + R2)C
\item
  \textbf{વોલ્ટેજ Vcc/3 થી 2Vcc/3 સુધી} વધે
\end{itemize}

\textbf{ડિસ્ચાર્જિંગ ફેઝ:}

\begin{itemize}
\tightlist
\item
  \textbf{કેપેસિટર માત્ર R2 મારફત} ડિસ્ચાર્જ થાય
\item
  \textbf{ડિસ્ચાર્જિંગ દરમ્યાન આઉટપુટ LOW}\\
\item
  \textbf{ડિસ્ચાર્જિંગ ટાઇમ}: T2 = 0.693 \times R2 \times C
\item
  \textbf{વોલ્ટેજ 2Vcc/3 થી Vcc/3 સુધી} ઘટે
\end{itemize}

\textbf{ફ્રીક્વન્સી ગણતરીઓ:}

\begin{longtable}[]{@{}ll@{}}
\toprule\noalign{}
પેરામીટર & ફોર્મ્યુલા \\
\midrule\noalign{}
\endhead
\bottomrule\noalign{}
\endlastfoot
\textbf{ટાઇમ HIGH} & T1 = 0.693(R1 + R2)C \\
\textbf{ટાઇમ LOW} & T2 = 0.693 \times R2 \times C \\
\textbf{કુલ પીરિયડ} & T = T1 + T2 = 0.693(R1 + 2R2)C \\
\textbf{ફ્રીક્વન્સી} & f = 1.44/[(R1 + 2R2)C] \\
\textbf{ડ્યુટી સાયકલ} & D = (R1 + R2)/(R1 + 2R2) \times 100\% \\
\end{longtable}

\textbf{વેવફોર્મ્સ:}

\begin{verbatim}
    Vout  │  ┌───┐     ┌───┐
          │  │   │     │   │
          │  │   │     │   │
         ─┼──┘   └─────┘   └──
          │  T1   T2
          └───────────────── Time
               Period T
\end{verbatim}

\textbf{ડિઝાઇન ઉદાહરણ:} f = 1kHz, D = 60\% માટે:

\begin{itemize}
\tightlist
\item
  C = 0.1µF પસંદ કરો
\item
  R1 = 7.2kΩ, R2 = 3.6kΩ ગણતરી કરો
\end{itemize}

\textbf{મુખ્ય લક્ષણો:}

\begin{itemize}
\tightlist
\item
  \textbf{બાહ્ય ટ્રિગર વિના સતત ઓસિલેશન}
\item
  \textbf{R અને C મૂલ્યો દ્વારા ફ્રીક્વન્સી એડજસ્ટેબલ}\\
\item
  \textbf{બેસિક સર્કિટમાં ડ્યુટી સાયકલ} હંમેશાં \textgreater{} 50\%
\item
  \textbf{વાઇડ ટેમ્પરેચર રેન્જમાં સ્ટેબલ} ઓપરેશન
\end{itemize}

\textbf{એપ્લિકેશન્સ:}

\begin{itemize}
\tightlist
\item
  \textbf{LED ફ્લેશર્સ} અને બ્લિન્કર્સ
\item
  \textbf{ડિજિટલ સર્કિટ્સ માટે ક્લોક જનરેટર્સ}
\item
  \textbf{એલાર્મ માટે ટોન જનરેટર્સ}
\item
  \textbf{PWM સિગ્નલ} જનરેશન
\end{itemize}

\textbf{50\% ડ્યુટી સાયકલ માટે મોડિફિકેશન્સ:}

\begin{itemize}
\tightlist
\item
  \textbf{R2 ની સમાંતર ડાયોડ} ઉમેરો
\item
  \textbf{ચાર્જ અને ડિસ્ચાર્જ માટે અલગ પાથ}
\item
  \textbf{સમાન ચાર્જ/ડિસ્ચાર્જ ટાઇમ} શક્ય
\end{itemize}

\end{solutionbox}
\begin{mnemonicbox}
``એસ્ટેબલ હંમેશાં ઓટોમેટિક ઓલ્ટરનેટ્સ''

\end{mnemonicbox}

\end{document}
