\documentclass{article}

% content/resources/templates/preamble.tex
\usepackage[margin=0.6in]{geometry}
\author{Milav Dabgar}
\usepackage{amsmath,amssymb,amsthm}
\usepackage{booktabs}
\usepackage{multirow}
\usepackage{xcolor}
\usepackage{tcolorbox}
\tcbuselibrary{breakable,skins}
\usepackage[colorlinks=true,linkcolor=blue]{hyperref}
\usepackage{titlesec}
\usepackage{enumitem}
\usepackage{tikz}
\usepackage{pgfplots}
\usepackage{circuitikz}
\usepackage[version=4]{mhchem}
\usepackage{longtable}
\usepackage{array}
\usepackage{float}
\usepackage{caption}
\usepackage{listings}

\lstset{
  basicstyle=\small\ttfamily,
  breaklines=true,
  breakatwhitespace=false,
  postbreak=\mbox{\textcolor{red}{$\hookrightarrow$}\space},
  float=false,
  numbers=left,
  numberstyle=\tiny\color{gray},
  numbersep=10pt,
  xleftmargin=2em,
  keywordstyle=\color{blue},
  commentstyle=\color{green!60!black},
  stringstyle=\color{purple},
  backgroundcolor=\color{gray!5},
  showstringspaces=false,
  tabsize=2,
  captionpos=b,
  keepspaces=true,
  columns=flexible
}

\pgfplotsset{compat=1.18}
\usetikzlibrary{shapes,arrows,positioning,calc,patterns,decorations.pathmorphing,decorations.markings,arrows.meta}

% Color scheme
\definecolor{headcolor}{RGB}{0,102,204}
\definecolor{keycolor}{RGB}{220,20,60}
\definecolor{solutioncolor}{RGB}{34,139,34}
\definecolor{mnemoniccolor}{RGB}{148,0,211}
\definecolor{codecolor}{RGB}{0,0,100}

% Spacing
\setlength{\parskip}{3pt}
\setlist[itemize]{nosep}
\setlist[enumerate]{nosep}

% Title formatting
\titleformat{\section}{\Large\bfseries\color{headcolor}}{\thesection}{1em}{}
\titleformat{\subsection}{\large\bfseries\color{headcolor}}{\thesubsection}{1em}{}

% Pandoc tightlist compatibility
\providecommand{\tightlist}{%
  \setlength{\itemsep}{0pt}\setlength{\parskip}{0pt}}

% Pandoc longtable compatibility
\newcounter{none}
\def\thenone{}


% content/resources/templates/gujarati-boxes.tex
\usepackage{fontspec}
\usepackage{polyglossia}

% Set Gujarati as main language (document is primarily in Gujarati)
% Note: gloss-gujarati.ldf doesn't exist in polyglossia, but it will use hyphenation patterns
\setdefaultlanguage{gujarati}
\setotherlanguage{english}

% Configure Gujarati font properly
% Use Language=Default to prevent polyglossia from trying to add language-specific features
% that don't exist for Gujarati, which causes "empty feature" warnings
\newfontfamily\gujaratifont[Script=Gujarati,AutoFakeBold=2.5,AutoFakeSlant=0.3]{Noto Sans Gujarati}
\setmainfont[Script=Gujarati,AutoFakeBold=2.5,AutoFakeSlant=0.3]{Noto Sans Gujarati}
% Use Noto Sans Gujarati for monospace to support Gujarati in text
\setmonofont[Scale=0.9]{Noto Sans Gujarati}

% Configure English to use the same font
\newfontfamily\englishfont[Script=Gujarati,AutoFakeBold=2.5,AutoFakeSlant=0.3]{Noto Sans Gujarati}

% Translations for polyglossia
\gappto\captionsgujarati{
  \renewcommand{\tablename}{કોષ્ટક}
  \renewcommand{\figurename}{આકૃતિ}
}

% Helper for TikZ nodes to ensure Gujarati font
\newcommand{\gu}[1]{{\gujaratifont #1}}

% Custom environments
\newtcolorbox{solutionbox}{
    breakable,
    enhanced,
    colback=solutioncolor!5!white,
    colframe=solutioncolor!75!black,
    fonttitle=\bfseries,
    title=જવાબ
}

\newtcolorbox{solutionboxnobreak}{
 colback=solutioncolor!5!white,
 colframe=solutioncolor!75!black,
 fonttitle=\bfseries,
 title=જવાબ
}

\newtcolorbox{keyformula}{
 breakable,
 enhanced,
 colback=keycolor!5!white,
 colframe=keycolor!75!black,
 fonttitle=\bfseries,
 title=રાસાયણિક સમીકરણ/સૂત્ર
}

\newtcolorbox{mnemonicbox}{
 breakable,
 enhanced,
 colback=mnemoniccolor!5!white,
 colframe=mnemoniccolor!75!black,
 fonttitle=\bfseries,
 title=મેમરી ટ્રીક
}


% Custom commands for GTU solutions
% This file defines semantic commands for consistent formatting

% Question command with automatic formatting
\newcommand{\question}[2]{%
  \section*{Question #1}%
  \textbf{#2}%
}

% OR question variant
\newcommand{\questionor}[2]{%
  \section*{Question #1 OR}%
  \textbf{#2}%
}

% Proper table environment with caption
\newenvironment{answertable}[1]{%
  \begin{table}[htbp]
  \centering
  \caption{#1}
}{%
  \end{table}
}

% Proper figure environment for diagrams
\newenvironment{answerdiagram}[1]{%
  \begin{figure}[htbp]
  \centering
  \caption{#1}
}{%
  \end{figure}
}

% Semantic markup for key terms
\newcommand{\keyword}[1]{\textbf{#1}}
\newcommand{\code}[1]{\texttt{#1}}
\newcommand{\classname}[1]{\texttt{#1}}
\newcommand{\methodname}[1]{\texttt{#1}}

% Proper quotation marks
\newcommand{\mnemonic}[1]{``#1''}

\usetikzlibrary{mindmap,trees,shadows,backgrounds}

\title{Fundamentals of Electronics (DI01000051) - Winter 2024 Solution}
\date{January 7, 2025}

\begin{document}
\maketitle

\questionmarks{1(a)}{3}{એક્ટિવ અને પેસીવ કમ્પોનન્ટ્સની ઉદાહરણ સાથે વ્યાખ્યા કરો.}

\begin{solutionbox}
\textbf{જવાબ}:

\begin{center}
\captionof{table}{એક્ટિવ વિ પેસીવ કમ્પોનન્ટ્સ}
\begin{tabulary}{\linewidth}{L L L L}
    \toprule
    \textbf{કમ્પોનન્ટ પ્રકાર} & \textbf{વ્યાખ્યા} & \textbf{પાવર} & \textbf{ઉદાહરણો} \\
    \midrule
    \textbf{એક્ટિવ કમ્પોનન્ટ્સ} & સિગ્નલોને વિસ્તૃત કરી શકે અને કરંટ પ્રવાહ નિયંત્રિત કરે & પાવર ગેઇન આપી શકે & ટ્રાન્ઝિસ્ટર, ડાયોડ, IC \\
    \textbf{પેસીવ કમ્પોનન્ટ્સ} & સિગ્નલોને વિસ્તૃત કરી શકતા નથી & પાવર ગેઇન આપી શકતા નથી & રેઝિસ્ટર, કેપેસિટર, ઇન્ડક્ટર \\
    \bottomrule
\end{tabulary}
\end{center}

\begin{itemize}
    \item \keyword{એક્ટિવ કમ્પોનન્ટ્સ}: બાહ્ય પાવરનો ઉપયોગ કરીને ઇલેક્ટ્રિકલ સિગ્નલોને નિયંત્રિત અને વિસ્તૃત કરે
    \item \keyword{પેસીવ કમ્પોનન્ટ્સ}: વિસ્તારણ વિના ઊર્જાનો સંગ્રહ અથવા વિસર્જન કરે
\end{itemize}
\end{solutionbox}

\begin{mnemonicbox}
\mnemonic{"એક્ટિવ વિસ્તારે, પેસીવ સાચવે"}
\end{mnemonicbox}

\questionmarks{1(b)}{4}{LDR નું બંધારણ અને કાર્ય સમજાવો.}

\begin{solutionbox}
\textbf{જવાબ}:

\textbf{બંધારણ:}
\begin{itemize}
    \item \keyword{સર્પેન્ટાઇન ટ્રેક} સિરામિક સબસ્ટ્રેટ પર કેડમિયમ સલ્ફાઇડનો
    \item \keyword{મેટલ ઇલેક્ટ્રોડ્સ} બંને છેડે કનેક્શન માટે
    \item \keyword{પ્રોટેક્ટિવ કોટિંગ} ભેજથી બચાવવા માટે
\end{itemize}

\textbf{કાર્યસિદ્ધાંત:}
\begin{center}
    \begin{tikzpicture}[node distance=1.5cm, auto]
        \node [gtu block] (LDR) {LDR (CdS Track)};
        \node [gtu state, above=of LDR] (Light) {Light $\downarrow$};
        \node [gtu state, below left=of LDR] (T1) {ટર્મિનલ 1};
        \node [gtu state, below right=of LDR] (T2) {ટર્મિનલ 2};
        
        \draw [gtu arrow] (Light) -- (LDR);
        \draw [thick] (LDR) -- (T1);
        \draw [thick] (LDR) -- (T2);
        
        \node [right=0.5cm of LDR, align=left] {રેઝિસ્ટન્સ $\downarrow$\\કંડક્ટિવિટી $\uparrow$};
    \end{tikzpicture}
    \captionof{figure}{LDR કાર્ય}
\end{center}

\begin{itemize}
    \item \keyword{પ્રકાશ તીવ્રતા $\uparrow$}: રેઝિસ્ટન્સ $\downarrow$ (વધુ કંડક્ટ કરે)
    \item \keyword{અંધકાર}: રેઝિસ્ટન્સ $\uparrow$ (ઓછું કંડક્ટ કરે)
    \item \keyword{ઉપયોગો}: સ્ટ્રીટ લાઇટ્સ, ઓટોમેટિક કેમેરા
\end{itemize}
\end{solutionbox}

\begin{mnemonicbox}
\mnemonic{"લાઇટ લો રેઝિસ્ટન્સ"}
\end{mnemonicbox}

\questionmarks{1(c)}{7}{કેપેસિટન્સની વ્યાખ્યા લખો અને એલ્યુમિનિયમ ઇલેક્ટ્રોલાઇટ વેટ પ્રકારનો કેપેસિટર સમજાવો.}

\begin{solutionbox}
\textbf{જવાબ}:

\textbf{કેપેસિટન્સ વ્યાખ્યા:}
ઇલેક્ટ્રિકલ ચાર્જ સંગ્રહિત કરવાની ક્ષમતા. $C = Q/V$ (ફેરાડ્સ)

\textbf{એલ્યુમિનિયમ ઇલેક્ટ્રોલાઇટિક કેપેસિટર:}

\begin{center}
    \begin{tikzpicture}
        \draw[thick] (0,0) rectangle (4,3);
        \draw[thick, fill=gray!20] (0.5,0.5) rectangle (3.5,2.5);
        \node at (2,1.5) {ઇલેક્ટ્રોલાઇટ (કેથોડ)};
        \node at (2,2.8) {Al ફોઇલ (એનોડ)};
        \node at (2,0.2) {Al ફોઇલ (નેગેટિવ)};
        \draw[->] (2,3.5) node[above]{પોઝિટિવ ટર્મિનલ} -- (2,3);
        \draw[->] (2,-0.5) node[below]{નેગેટિવ ટર્મિનલ} -- (2,0);
    \end{tikzpicture}
    \captionof{figure}{એલ્યુમિનિયમ કેપેસિટર}
\end{center}

\textbf{બંધારણ:}
\begin{itemize}
    \item \keyword{એનોડ}: ઓક્સાઇડ લેયર સાથે એલ્યુમિનિયમ ફોઇલ
    \item \keyword{ડાઇઇલેક્ટ્રિક}: પાતળી એલ્યુમિનિયમ ઓક્સાઇડ ફિલ્મ
    \item \keyword{કેથોડ}: એલ્યુમિનિયમ ફોઇલ સાથે લિક્વિડ ઇલેક્ટ્રોલાઇટ
    \item \keyword{પોલેરિટી}: યોગ્ય રીતે જોડવું જરૂરી
\end{itemize}

\textbf{લક્ષણો:}
\begin{itemize}
    \item \keyword{ઉચ્ચ કેપેસિટન્સ} મૂલ્યો (1$\mu$F થી 10,000$\mu$F)
    \item \keyword{પોલરાઇઝ્ડ} - પોઝિટિવ અને નેગેટિવ ટર્મિનલ છે
    \item \keyword{ઉપયોગો}: પાવર સપ્લાય ફિલ્ટરિંગ, કપલિંગ
\end{itemize}
\end{solutionbox}

\begin{mnemonicbox}
\mnemonic{"એલ્યુમિનિયમ હંમેશાં વિસ્તારે"}
\end{mnemonicbox}

\questionmarks{1(c OR)}{7}{રેઝિસ્ટરની કલર બેન્ડ કોડિંગ પદ્ધતિ સમજાવો. 32 $\Omega$ $\pm$ 10\% કિંમતનો કલર બેન્ડ લખો.}

\begin{solutionbox}
\textbf{જવાબ}:

\textbf{કલર કોડ ટેબલ:}
\begin{center}
\captionof{table}{કલર કોડ}
\begin{tabulary}{\linewidth}{L L L L}
    \toprule
    \textbf{રંગ} & \textbf{અંક} & \textbf{ગુણાકાર} & \textbf{ટોલરન્સ} \\
    \midrule
    કાળો & 0 & 1 & - \\
    ભૂરો & 1 & 10 & $\pm$1\% \\
    લાલ & 2 & 100 & $\pm$2\% \\
    કેસરી & 3 & 1K & - \\
    પીળો & 4 & 10K & - \\
    લીલો & 5 & 100K & $\pm$0.5\% \\
    વાદળી & 6 & 1M & $\pm$0.25\% \\
    વાયોલેટ & 7 & 10M & $\pm$0.1\% \\
    ધૂસર & 8 & 100M & $\pm$0.05\% \\
    સફેદ & 9 & 1G & - \\
    ચાંદી & - & 0.01 & $\pm$10\% \\
    સોનું & - & 0.1 & $\pm$5\% \\
    \bottomrule
\end{tabulary}
\end{center}

\textbf{32 $\Omega$ $\pm$ 10\% માટે:}

\begin{center}
    \begin{tikzpicture}
        \draw[thick, fill=orange!20] (0,0) rectangle (6,1.5);
        \foreach \x/\c/\n in {1/orange/કેસરી (3), 2.5/red/લાલ (2), 4/black/કાળો ($\times$1), 5/gray!40/ચાંદી ($\pm$10\%)} {
            \draw[fill=\c] (\x,0) rectangle (\x+0.5,1.5) node[midway, below=0.8cm] {\n};
        }
        \draw[thick] (-1,0.75) -- (0,0.75);
        \draw[thick] (6,0.75) -- (7,0.75);
    \end{tikzpicture}
    \captionof{figure}{રેઝિસ્ટર કલર કોડ}
\end{center}

\textbf{ગણતરી:} $3 \times 2 \times 1 = 32 \Omega$
\end{solutionbox}

\begin{mnemonicbox}
\mnemonic{"મોટા છોકરા દોડે અમારા યુવા છોકરીઓ પણ વાયોલેટ સામાન્યે જીતે"}
\end{mnemonicbox}

\questionmarks{2(a)}{3}{નીચેના શબ્દો વ્યાખ્યાયિત કરો: 1) રેક્ટિફાયર 2) રિપલ ફેક્ટર 3) ફિલ્ટર}

\begin{solutionbox}
\textbf{જવાબ}:

\begin{center}
\captionof{table}{શબ્દ વ્યાખ્યાઓ}
\begin{tabulary}{\linewidth}{L L}
    \toprule
    \textbf{શબ્દ} & \textbf{વ્યાખ્યા} \\
    \midrule
    \textbf{રેક્ટિફાયર} & AC ને પલ્સેટિંગ DC માં બદલનાર સર્કિટ \\
    \textbf{રિપલ ફેક્ટર} & આઉટપુટમાં AC ઘટક અને DC ઘટકનો ગુણોત્તર \\
    \textbf{ફિલ્ટર} & પલ્સેટિંગ DC ને સ્મૂથ DC માં બદલનાર સર્કિટ \\
    \bottomrule
\end{tabulary}
\end{center}

\begin{itemize}
    \item \keyword{રેક્ટિફાયર}: એક જ દિશામાં કરંટ પસાર કરવા ડાયોડનો ઉપયોગ કરે
    \item \keyword{રિપલ ફેક્ટર}: નીચું મૂલ્ય મતલબ સારું ફિલ્ટરિંગ
    \item \keyword{ફિલ્ટર}: રિપલ્સ ઘટાડવા કેપેસિટર/ઇન્ડક્ટરનો ઉપયોગ કરે
\end{itemize}
\end{solutionbox}

\begin{mnemonicbox}
\mnemonic{"રેક્ટિફાય રિપલ્સ, ફિલ્ટર ફિક્સ કરે"}
\end{mnemonicbox}

\questionmarks{2(b)}{4}{પોઝિટિવ ક્લિપર સર્કિટ દોરી વેવફોર્મ સાથે સમજાવો.}

\begin{solutionbox}
\textbf{જવાબ}:

\textbf{સર્કિટ ડાયાગ્રામ:}
\begin{center}
    \begin{tikzpicture}
        \draw (0,2) node[left]{Input} -- (2,2) -- (4,2) node[right]{Output};
        \draw (2,2) -- (2,1.2);
        \draw (1.7,1.2) -- (2.3,1.2) -- (2,0.6) -- (1.7,1.2); % Diode
        \draw (1.7,0.6) -- (2.3,0.6);
        \draw (2,0.6) -- (2,0);
        \draw (1.7,0) -- (2.3,0); % Battery
        \draw (1.8,-0.2) -- (2.2,-0.2);
        \draw (2,-0.2) -- (2,-1);
        \draw (0,-1) node[left]{GND} -- (4,-1) node[right]{GND};
        \node at (2.5,1) {$D_1$};
        \node at (2.5,0) {$+V$};
    \end{tikzpicture}
    \captionof{figure}{પોઝિટિવ ક્લિપર}
\end{center}

\textbf{કાર્યપદ્ધતિ:}
\begin{itemize}
    \item \keyword{જ્યારે Vin > +V}: ડાયોડ કંડક્ટ કરે, આઉટપુટ = +V
    \item \keyword{જ્યારે Vin < +V}: ડાયોડ બંધ, આઉટપુટ ઇનપુટને અનુસરે
    \item \keyword{પરિણામ}: +V લેવલથી ઉપરના પોઝિટિવ પીક્સ ક્લિપ થાય
\end{itemize}

\textbf{વેવફોર્મ:}
\begin{center}
    \begin{tikzpicture}
        \draw[->] (0,0) -- (4,0) node[right] {t};
        \draw[->] (0,-1.5) -- (0,1.5) node[above] {V};
        \draw[dashed] (0,0.8) -- (4,0.8) node[right] {$+V$};
        \draw[thick] (0,0) sin (1,1.2) cos (2,0) sin (3,-1.2) cos (4,0);
        \node at (2,-1.8) {Input};
        
        \begin{scope}[xshift=5cm]
            \draw[->] (0,0) -- (4,0) node[right] {t};
            \draw[->] (0,-1.5) -- (0,1.5) node[above] {V};
            \draw[dashed] (0,0.8) -- (4,0.8) node[right] {$+V$};
            \draw[thick] (0,0) sin (0.6,0.8) -- (1.4,0.8) cos (2,0) sin (3,-1.2) cos (4,0);
            \node at (2,-1.8) {Output};
        \end{scope}
    \end{tikzpicture}
    \captionof{figure}{ક્લિપર વેવફોર્મ્સ}
\end{center}

\keyword{ઉપયોગો}: સિગ્નલ લિમિટિંગ, પ્રોટેક્શન સર્કિટ્સ
\end{solutionbox}

\begin{mnemonicbox}
\mnemonic{"પોઝિટિવ પીક્સ પ્રિવેન્ટેડ"}
\end{mnemonicbox}

\questionmarks{2(c)}{7}{બે ડાયોડથી ફુલ વેવ રેક્ટિફાયરની કાર્યપદ્ધતિ સમજાવો.}

\begin{solutionbox}
\textbf{જવાબ}:

\textbf{સર્કિટ ડાયાગ્રામ:}
\begin{center}
    \begin{tikzpicture}
        % Transformer
        \draw (0,0) node[left]{AC} -- (0.5,0);
        \draw (0.5,0) to[L] (0.5,-2);
        \draw (0,-2) node[left]{AC} -- (0.5,-2);
        
        % Secondary
        \draw (1.5,0) to[L] (1.5,-2);
        \draw (1.5,0) -- (2.5,0);
        \draw (1.5,-2) -- (2.5,-2);
        \draw (1.5,-1) -- (2,-1) -- (2,-1.5) -- (4,-1.5) node[right]{GND}; 
        
        \draw (2.5,0) to[D] (4.5,0) -- (4.5,-0.8); % D1
        \draw (2.5,-2) to[D] (4.5,-2) -- (4.5,-1.2); % D2
        \draw (4.5,-0.8) -- (5.5,-0.8) node[right]{Output};
        \draw (4.5,-1.2) -- (5.5,-1.2) node[right]{GND};
        \draw (5,-0.8) to[R, l=$R_L$] (5,-1.2);
    \end{tikzpicture}
    \captionof{figure}{ફુલ વેવ રેક્ટિફાયર}
\end{center}

\textbf{કાર્યપદ્ધતિ:}
\begin{itemize}
    \item \keyword{પોઝિટિવ હાફ-સાયકલ}: D1 કંડક્ટ કરે, D2 બંધ
    \item \keyword{નેગેટિવ હાફ-સાયકલ}: D2 કંડક્ટ કરે, D1 બંધ
    \item \keyword{બંને ડાયોડ} વારાફરતી કામ કરે
    \item \keyword{આઉટપુટ ફ્રીક્વન્સી} = 2 $\times$ ઇનપુટ ફ્રીક્વન્સી
\end{itemize}

\textbf{મુખ્ય પેરામીટર્સ:}
\begin{center}
\captionof{table}{FWR પેરામીટર્સ}
\begin{tabulary}{\linewidth}{L L}
    \toprule
    \textbf{પેરામીટર} & \textbf{મૂલ્ય} \\
    \midrule
    \textbf{પીક ઇન્વર્સ વોલ્ટેજ} & 2Vm \\
    \textbf{કાર્યક્ષમતા} & 81.2\% \\
    \textbf{રિપલ ફેક્ટર} & 0.48 \\
    \textbf{ફોર્મ ફેક્ટર} & 1.11 \\
    \bottomrule
\end{tabulary}
\end{center}

\textbf{ફાયદા:}
\begin{itemize}
    \item \keyword{હાફ વેવ કરતાં સારી કાર્યક્ષમતા}
    \item \keyword{ઓછું રિપલ} કન્ટેન્ટ
    \item \keyword{વધુ ટ્રાન્સફોર્મર ઉપયોગ}
\end{itemize}
\end{solutionbox}

\begin{mnemonicbox}
\mnemonic{"બે ડાયોડ, બે હાફ"}
\end{mnemonicbox}

\questionmarks{2(a OR)}{3}{રેક્ટિફાયર વ્યાખ્યાયિત કરો અને તેની એપ્લિકેશન લખો.}

\begin{solutionbox}
\textbf{જવાબ}:

\textbf{વ્યાખ્યા:}
ઇલેક્ટ્રોનિક સર્કિટ જે ડાયોડનો ઉપયોગ કરીને AC કરંટને DC કરંટમાં બદલે છે.

\textbf{એપ્લિકેશન્સ:}
\begin{center}
\captionof{table}{રેક્ટિફાયર એપ્લિકેશન્સ}
\begin{tabulary}{\linewidth}{L L}
    \toprule
    \textbf{એપ્લિકેશન} & \textbf{ઉપયોગ} \\
    \midrule
    \textbf{પાવર સપ્લાય} & ઇલેક્ટ્રોનિક સર્કિટ્સ માટે DC વોલ્ટેજ \\
    \textbf{બેટરી ચાર્જર} & AC મેઇન્સને DC માં કન્વર્ટ કરવા \\
    \textbf{DC મોટર્સ} & મોટર ડ્રાઇવ્સ માટે DC પૂરું પાડવા \\
    \textbf{ઇલેક્ટ્રોનિક ડિવાઇસ} & લેપટોપ, ફોન, LED ડ્રાઇવર્સ \\
    \bottomrule
\end{tabulary}
\end{center}
\end{solutionbox}

\begin{mnemonicbox}
\mnemonic{"AC રેક્ટિફાય કરે, DC ડિલિવર કરે"}
\end{mnemonicbox}

\questionmarks{2(b OR)}{4}{Pi ($\pi$) પ્રકારના કેપેસિટર ફિલ્ટરનું કાર્ય સમજાવો.}

\begin{solutionbox}
\textbf{જવાબ}:

\textbf{સર્કિટ ડાયાગ્રામ:}
\begin{center}
    \begin{tikzpicture}
        \draw (0,2) node[left]{Input} -- (1,2);
        \draw (1,2) to[C, l=$C_1$] (1,0);
        \draw (1,2) to[L, l=$L$] (3,2);
        \draw (3,2) to[C, l=$C_2$] (3,0);
        \draw (3,2) -- (4,2) node[right]{Output};
        \draw (0,0) node[left]{GND} -- (4,0) node[right]{GND};
    \end{tikzpicture}
    \captionof{figure}{Pi ફિલ્ટર}
\end{center}

\textbf{કાર્યપદ્ધતિ:}
\begin{itemize}
    \item \keyword{C1}: રેક્ટિફાયરથી આવતા પ્રારંભિક રિપલ્સ ફિલ્ટર કરે
    \item \keyword{ઇન્ડક્ટર L}: કરંટ ચેન્જનો વિરોધ કરે, વધુ સ્મૂથ કરે
    \item \keyword{C2}: સ્મૂથ DC આઉટપુટ માટે અંતિમ ફિલ્ટરિંગ
    \item \keyword{સંયુક્ત અસર}: ઉત્તમ રિપલ ઘટાડો
\end{itemize}

\textbf{ફાયદા:}
\begin{itemize}
    \item \keyword{ઉત્તમ ફિલ્ટરિંગ} પર્ફોર્મન્સ
    \item \keyword{ઓછું રિપલ} કન્ટેન્ટ
    \item \keyword{સારું વોલ્ટેજ રેગ્યુલેશન}
\end{itemize}
\end{solutionbox}

\begin{mnemonicbox}
\mnemonic{"Pi પરફેક્ટ પૂરું પાડે"}
\end{mnemonicbox}

\questionmarks{2(c OR)}{7}{હાફ વેવ અને ફુલ વેવ બ્રિજ રેક્ટિફાયરને સરખાવો.}

\begin{solutionbox}
\textbf{જવાબ}:

\textbf{તુલના કોષ્ટક:}
\begin{center}
\captionof{table}{તુલના}
\begin{tabulary}{\linewidth}{L L L}
    \toprule
    \textbf{પેરામીટર} & \textbf{હાફ વેવ} & \textbf{ફુલ વેવ બ્રિજ} \\
    \midrule
    \textbf{જરૂરી ડાયોડ} & 1 & 4 \\
    \textbf{ટ્રાન્સફોર્મર} & સિમ્પલ & સેન્ટર-ટેપની જરૂર નથી \\
    \textbf{કાર્યક્ષમતા} & 40.6\% & 81.2\% \\
    \textbf{રિપલ ફેક્ટર} & 1.21 & 0.48 \\
    \textbf{PIV} & Vm & Vm \\
    \textbf{આઉટપુટ ફ્રીક્વન્સી} & f & 2f \\
    \textbf{ટ્રાન્સફોર્મર ઉપયોગ} & 28.7\% & 81.2\% \\
    \textbf{કિંમત} & ઓછી & મધ્યમ \\
    \bottomrule
\end{tabulary}
\end{center}

\textbf{સર્કિટ ડાયાગ્રામ:}
\begin{center}
    \begin{tikzpicture}
        % Half Wave
        \draw (0,0) to[D] (1,0) to[R] (1,-1) -- (0,-1);
        \node at (0.5,-1.5) {હાફ વેવ};
        
        % Bridge
        \begin{scope}[xshift=4cm]
        \draw (0,0) -- (0.5,0.5) -- (1,0) -- (0.5,-0.5) -- (0,0);
        \node at (0.5,0) {Bridge};
        \draw (0,0) -- (-0.2,0);
        \draw (1,0) -- (1.2,0);
        \node at (0.5,-1.5) {બ્રિજ};
        \end{scope}
    \end{tikzpicture}
\end{center}
\end{solutionbox}

\begin{mnemonicbox}
\mnemonic{"હાફ વેસ્ટ કરે, ફુલ કામ કરે"}
\end{mnemonicbox}

\questionmarks{3(a)}{3}{નીચેના પ્રતીકો દોરો: 1) ઝેનર ડાયોડ 2) LED 3) વેરેક્ટર ડાયોડ}

\begin{solutionbox}
\textbf{જવાબ}:

\begin{center}
    \begin{tikzpicture}
        \node at (0,3) {\textbf{ઝેનર ડાયોડ}};
        \draw (0,2) -- (0,1);
        \draw (-0.5,1) -- (0.5,1) -- (0,0.5) -- (-0.5,1); % Triangle
        \draw (-0.5,0.5) -- (0.5,0.5); % Bar
        \draw (-0.5,0.5) -- (-0.5,0.8); % Z
        \draw (0.5,0.5) -- (0.5,0.2); % Z
        \draw (0,0.5) -- (0,-0.5);
        
        \begin{scope}[xshift=4cm]
        \node at (0,3) {\textbf{LED}};
        \draw (0,2) -- (0,1);
        \draw (-0.5,1) -- (0.5,1) -- (0,0.5) -- (-0.5,1);
        \draw (-0.5,0.5) -- (0.5,0.5);
        \draw (0,0.5) -- (0,-0.5);
        \draw[->] (0.8,0.8) -- (1.5,1.5);
        \draw[->] (0.8,0.5) -- (1.5,1.2);
        \end{scope}
        
        \begin{scope}[xshift=8cm]
        \node at (0,3) {\textbf{વેરેક્ટર}};
        \draw (0,2) -- (0,1);
        \draw (-0.5,1) -- (0.5,1) -- (0,0.5) -- (-0.5,1);
        \draw (-0.5,0.5) -- (0.5,0.5);
        \draw (-0.5,0.3) -- (0.5,0.3); % Second plate
        \draw (0,0.3) -- (0,-0.5);
        \end{scope}
    \end{tikzpicture}
    \captionof{figure}{ડાયોડ પ્રતીકો}
\end{center}

\textbf{પ્રતીક વિગતો:}
\begin{itemize}
    \item \keyword{ઝેનર ડાયોડ}: Z આકારના કેથોડ સાથે સામાન્ય ડાયોડ
    \item \keyword{LED}: પ્રકાશ ઉત્સર્જન દર્શાવતા તીર સાથે ડાયોડ
    \item \keyword{વેરેક્ટર ડાયોડ}: સમાંતર લીટીઓ સાથે ડાયોડ (વેરિએબલ કેપેસિટર)
\end{itemize}
\end{solutionbox}

\begin{mnemonicbox}
\mnemonic{"ઝેનર ઝિગઝેગ, LED લાઇટ, વેરેક્ટર વેરી"}
\end{mnemonicbox}

\questionmarks{3(b)}{4}{LED ની રચના અને કાર્ય સમજાવો.}

\begin{solutionbox}
\textbf{જવાબ}:

\textbf{બંધારણ:}
\begin{center}
    \begin{tikzpicture}
        \fill[gray!20] (0,0) rectangle (4,1);
        \draw (0,0) rectangle (4,3);
        \draw[fill=blue!10] (1,1) rectangle (3,2);
        \node at (2,1.5) {P-N જંક્શન};
        \draw[->, orange, thick] (2,2) -- (1,3);
        \draw[->, orange, thick] (2.2,2) -- (1.2,3);
        \node[orange] at (2,3.2) {Light};
        \draw (1.5,1) -- (1.5,0.5) node[below]{Anode};
        \draw (2.5,1) -- (2.5,0.5) node[below]{Cathode};
    \end{tikzpicture}
    \captionof{figure}{LED રચના}
\end{center}

\textbf{સામગ્રી:}
\begin{itemize}
    \item \keyword{P-type}: બોરોન-ડોપ્ડ સેમિકન્ડક્ટર
    \item \keyword{N-type}: ફોસ્ફોરસ-ડોપ્ડ સેમિકન્ડક્ટર
    \item \keyword{સામાન્ય સામગ્રી}: GaAs, GaP, GaN
\end{itemize}

\textbf{કાર્યસિદ્ધાંત:}
\begin{itemize}
    \item \keyword{ફોરવર્ડ બાયાસ}: ઇલેક્ટ્રોન હોલ્સ સાથે રિકોમ્બાઇન થાય
    \item \keyword{ઊર્જા રિલીઝ}: ફોટોન (પ્રકાશ) રૂપમાં
    \item \keyword{રંગ}: સેમિકન્ડક્ટર સામગ્રી અને બેન્ડગેપ પર આધાર રાખે
\end{itemize}
\end{solutionbox}

\begin{mnemonicbox}
\mnemonic{"લાઇટ ઇમિટિંગ, એનર્જી એફિશિયન્ટ"}
\end{mnemonicbox}

\questionmarks{3(c)}{7}{ઝેનર ડાયોડની કાર્યકારી લાક્ષણિકતાઓ સમજાવો.}

\begin{solutionbox}
\textbf{જવાબ}:

\textbf{V-I લાક્ષણિકતાઓ:}
\begin{center}
    \begin{tikzpicture}
        \draw[->] (-4,0) -- (4,0) node[right]{V};
        \draw[->] (0,-3) -- (0,3) node[above]{I};
        % Forward
        \draw[thick] (0,0) -- (1,0) .. controls (1.5,0.1) .. (2,2);
        \node at (2.5,2) {Forward};
        % Reverse
        \draw[thick] (0,0) -- (-2,0) -- (-2.5,-3);
        \draw[dashed] (-2.5,0) node[above]{$V_z$} -- (-2.5,-3);
        \node at (-3.5,-1.5) {Breakdown};
    \end{tikzpicture}
    \captionof{figure}{ઝેનર V-I લાક્ષણિકતાઓ}
\end{center}

\textbf{મુખ્ય વિસ્તારો:}
\begin{center}
\captionof{table}{ઝેનર વિસ્તારો}
\begin{tabulary}{\linewidth}{L L}
    \toprule
    \textbf{વિસ્તાર} & \textbf{લાક્ષણિકતાઓ} \\
    \midrule
    \textbf{ફોરવર્ડ બાયાસ} & સામાન્ય ડાયોડ ઓપરેશન (0.7V) \\
    \textbf{રિવર્સ બાયાસ} & નાનું લીકેજ કરંટ \\
    \textbf{ઝેનર રીજીયન} & કોન્સ્ટન્ટ વોલ્ટેજ (Vz) \\
    \textbf{બ્રેકડાઉન} & શાર્પ વોલ્ટેજ બ્રેકડાઉન \\
    \bottomrule
\end{tabulary}
\end{center}

\textbf{મહત્વના પેરામીટર્સ:}
\begin{itemize}
    \item \keyword{ઝેનર વોલ્ટેજ (Vz)}: બ્રેકડાઉન વોલ્ટેજ
    \item \keyword{ઝેનર કરંટ (Iz)}: બ્રેકડાઉન વિસ્તારમાં કરંટ
    \item \keyword{મેક્સિમમ પાવર}: Vz $\times$ Iz(max)
\end{itemize}
\end{solutionbox}

\begin{mnemonicbox}
\mnemonic{"ઝેનર ઝોન ઝીરો વેરિએશન"}
\end{mnemonicbox}

\questionmarks{3(a OR)}{3}{વેરેક્ટર ડાયોડની એપ્લિકેશનની યાદી બનાવો.}

\begin{solutionbox}
\textbf{જવાબ}:

\textbf{એપ્લિકેશન ટેબલ:}
\begin{center}
\captionof{table}{વેરેક્ટર એપ્લિકેશન્સ}
\begin{tabulary}{\linewidth}{L L}
    \toprule
    \textbf{એપ્લિકેશન} & \textbf{કાર્ય} \\
    \midrule
    \textbf{વોલ્ટેજ કંટ્રોલ્ડ ઓસિલેટર્સ} & વોલ્ટેજ સાથે ફ્રીક્વન્સી ટ્યુનિંગ \\
    \textbf{ઓટોમેટિક ફ્રીક્વન્સી કંટ્રોલ} & ઓસિલેટર ફ્રીક્વન્સી જાળવે \\
    \textbf{ઇલેક્ટ્રોનિક ટ્યુનિંગ} & રેડિયો/TV ચેનલ સિલેક્શન \\
    \textbf{ફેઝ લૉક્ડ લૂપ્સ} & ફ્રીક્વન્સી સિંક્રોનાઇઝેશન \\
    \textbf{ફ્રીક્વન્સી મલ્ટિપ્લાયર્સ} & હાર્મોનિક જનરેશન \\
    \textbf{પેરામેટ્રિક એમ્પ્લિફાયર્સ} & લો-નોઇઝ એમ્પ્લિફિકેશન \\
    \bottomrule
\end{tabulary}
\end{center}

\textbf{મુખ્ય લક્ષણો:}
\begin{itemize}
    \item \keyword{વોલ્ટેજ વેરિએબલ}: રિવર્સ વોલ્ટેજ સાથે કેપેસિટન્સ બદલાય
    \item \keyword{યાંત્રિક ભાગો નથી}: માત્ર ઇલેક્ટ્રોનિક ટ્યુનિંગ
    \item \keyword{ઝડપી પ્રતિસાદ}: ઝડપી ફ્રીક્વન્સી ચેન્જ
\end{itemize}
\end{solutionbox}

\begin{mnemonicbox}
\mnemonic{"વોલ્ટેજ વેરીઝ કેપેસિટન્સ"}
\end{mnemonicbox}

\questionmarks{3(b OR)}{4}{ફોટો ડાયોડનું કાર્ય સમજાવો.}

\begin{solutionbox}
\textbf{જવાબ}:

\textbf{બંધારણ અને પ્રતીક:}
\begin{center}
    \begin{tikzpicture}
        \draw (0,0) rectangle (3,2);
        \draw (0,1) -- (3,1);
        \node at (1.5,1.5) {P-type (Anode)};
        \node at (1.5,0.5) {N-type (Cathode)};
        \draw[->, thick, orange] (1.5,2.5) -- (1.5,2);
        \draw[->, thick, orange] (1.8,2.5) -- (1.8,2);
        \draw[->, thick, orange] (1.2,2.5) -- (1.2,2);
        \node[orange] at (1.5,2.8) {Light};
        \draw (3,1.5) -- (3.5,1.5) node[right]{Anode};
        \draw (3,0.5) -- (3.5,0.5) node[right]{Cathode};
    \end{tikzpicture}
    \captionof{figure}{ફોટો ડાયોડ}
\end{center}

\textbf{કાર્યસિદ્ધાંત:}
\begin{itemize}
    \item \keyword{પ્રકાશ અવશોષણ}: ઇલેક્ટ્રોન-હોલ પેર્સ બનાવે
    \item \keyword{રિવર્સ બાયાસ}: ડિપ્લીશન રીજીયન વિસ્તૃત કરે
    \item \keyword{ફોટોકરંટ}: પ્રકાશ તીવ્રતાના પ્રમાણમાં
    \item \keyword{ઝડપી પ્રતિસાદ}: ઝડપી ડિટેક્શન ક્ષમતા
\end{itemize}

\textbf{લક્ષણો:}
\begin{center}
\captionof{table}{ફોટો ડાયોડ}
\begin{tabulary}{\linewidth}{L L}
    \toprule
    \textbf{પેરામીટર} & \textbf{વર્ણન} \\
    \midrule
    \textbf{ડાર્ક કરંટ} & પ્રકાશ વિના કરંટ \\
    \textbf{ફોટોકરંટ} & પ્રકાશના પ્રમાણમાં કરંટ \\
    \textbf{રેસ્પોન્સિવિટી} & યુનિટ લાઇટ પાવર પર કરંટ \\
    \textbf{રેસ્પોન્સ ટાઇમ} & ડિટેક્શનની ગતિ \\
    \bottomrule
\end{tabulary}
\end{center}

\textbf{ઉપયોગો:}
\begin{itemize}
    \item \keyword{લાઇટ સેન્સર્સ}: ઓટોમેટિક લાઇટિંગ સિસ્ટમ
    \item \keyword{ઓપ્ટિકલ કમ્યુનિકેશન}: ફાઇબર ઓપ્ટિક રિસીવર્સ
    \item \keyword{સેફટી સિસ્ટમ}: સ્મોક ડિટેક્ટર્સ
\end{itemize}
\end{solutionbox}

\begin{mnemonicbox}
\mnemonic{"ફોટો પ્રોડ્યુસેસ પ્રોપોર્શનલ કરંટ"}
\end{mnemonicbox}

\questionmarks{3(c OR)}{7}{ઝેનર ડાયોડને વોલ્ટેજ રેગ્યુલેટરના સ્વરૂપે સમજાવો.}

\begin{solutionbox}
\textbf{જવાબ}:

\textbf{વોલ્ટેજ રેગ્યુલેટર સર્કિટ:}
\begin{center}
    \begin{tikzpicture}
        \draw (0,2) node[left]{Vin} -- (1,2) to[R, l=$R_s$] (3,2) -- (4,2) -- (5,2) node[right]{Vout ($V_z$)};
        \draw (3,2) -- (3,1.5);
        \draw (3,0.5) to[zD] (3,1.5); % Zener
        \draw (3,0.5) -- (3,0);
        \draw (0,0) node[left]{GND} -- (5,0) node[right]{GND};
        \draw (4,2) to[R, l=$R_L$] (4,0);
        \node at (2.5,1) {$Z$};
    \end{tikzpicture}
    \captionof{figure}{ઝેનર રેગ્યુલેટર}
\end{center}

\textbf{કાર્યસિદ્ધાંત:}
\begin{itemize}
    \item \keyword{ઝેનર ઓપરેટ} બ્રેકડાઉન રીજીયનમાં
    \item \keyword{આઉટપુટ વોલ્ટેજ} Vz પર કોન્સ્ટન્ટ રહે
    \item \keyword{સીરીઝ રેઝિસ્ટર Rs} કરંટ લિમિટ કરે
    \item \keyword{લોડ ચેન્જ} આઉટપુટ વોલ્ટેજને અસર કરતા નથી
\end{itemize}

\textbf{ડિઝાઇન સમીકરણો:}
\begin{center}
\captionof{table}{સમીકરણો}
\begin{tabulary}{\linewidth}{L L}
    \toprule
    \textbf{પેરામીટર} & \textbf{ફોર્મ્યુલા} \\
    \midrule
    \textbf{સીરીઝ રેઝિસ્ટન્સ} & $R_s = (Vin - Vz) / Iz$ \\
    \textbf{લોડ કરંટ} & $IL = Vz / RL$ \\
    \textbf{ઝેનર કરંટ} & $Iz = Is - IL$ \\
    \textbf{પાવર ડિસિપેશન} & $Pz = Vz \times Iz$ \\
    \bottomrule
\end{tabulary}
\end{center}

\textbf{ફાયદા:}
\begin{itemize}
    \item \keyword{સિમ્પલ સર્કિટ}: ઓછા કમ્પોનન્ટ્સ જરૂરી
    \item \keyword{સારું રેગ્યુલેશન}: સ્ટેબલ આઉટપુટ વોલ્ટેજ
    \item \keyword{ઝડપી પ્રતિસાદ}: ઝડપી વોલ્ટેજ કરેક્શન
\end{itemize}

\textbf{ઉપયોગો:}
\begin{itemize}
    \item \keyword{રેફરન્સ વોલ્ટેજ}: ચોક્કસ વોલ્ટેજ સોર્સ
    \item \keyword{સિમ્પલ રેગ્યુલેટર્સ}: ઓછા કરંટ એપ્લિકેશન
    \item \keyword{પ્રોટેક્શન સર્કિટ્સ}: ઓવરવોલ્ટેજ પ્રોટેક્શન
\end{itemize}
\end{solutionbox}

\begin{mnemonicbox}
\mnemonic{"ઝેનર ઝોન્સ ઝીરો વેરિએશન પૂરા પાડે"}
\end{mnemonicbox}

\questionmarks{4(a)}{3}{PNP અને NPN ટ્રાન્ઝિસ્ટરની સંજ્ઞા અને બંધારણ યોગ્ય નામ નિર્દેશ સાથે દોરો.}

\begin{solutionbox}
\textbf{જવાબ}:

\textbf{ટ્રાન્ઝિસ્ટર પ્રતીકો:}
\begin{center}
    \begin{tikzpicture}
        \node at (0,3) {\textbf{NPN}};
        \draw (0,1.5) circle(1);
        \draw (0,1) -- (0,2); % Base line
        \draw (0,1.5) -- (-1,1.5) node[left]{B};
        \draw (0,1.8) -- (0.8,2.2) node[right]{C};
        \draw (0,1.2) -- (0.8,0.8);
        \draw[->, thick] (0.4,1.0) -- (0.8,0.8); % Arrow Out
        \node at (1,0.8) {E};
        
        \begin{scope}[xshift=4cm]
        \node at (0,3) {\textbf{PNP}};
        \draw (0,1.5) circle(1);
        \draw (0,1) -- (0,2);
        \draw (0,1.5) -- (-1,1.5) node[left]{B};
        \draw (0,1.8) -- (0.8,2.2) node[right]{C};
        \draw (0,1.2) -- (0.8,0.8);
        \draw[->, thick] (0.8,0.8) -- (0.4,1.0); % Arrow In
        \node at (1,0.8) {E};
        \end{scope}
    \end{tikzpicture}
    \captionof{figure}{ટ્રાન્ઝિસ્ટર પ્રતીકો}
\end{center}

\textbf{બંધારણ ડાયાગ્રામ:}
\begin{center}
    \begin{tikzpicture}
        \node at (1.5,3.5) {\textbf{NPN}};
        \draw (0,0) rectangle (3,3);
        \draw (0,1) -- (3,1);
        \draw (0,2) -- (3,2);
        \node at (1.5,2.5) {કલેક્ટર (N)};
        \node at (1.5,1.5) {બેસ (P)};
        \node at (1.5,0.5) {એમિટર (N)};
        
        \begin{scope}[xshift=5cm]
        \node at (1.5,3.5) {\textbf{PNP}};
        \draw (0,0) rectangle (3,3);
        \draw (0,1) -- (3,1);
        \draw (0,2) -- (3,2);
        \node at (1.5,2.5) {કલેક્ટર (P)};
        \node at (1.5,1.5) {બેસ (N)};
        \node at (1.5,0.5) {એમિટર (P)};
        \end{scope}
    \end{tikzpicture}
    \captionof{figure}{ટ્રાન્ઝિસ્ટર રચના}
\end{center}

\textbf{ટર્મિનલ ઓળખ:}
\begin{itemize}
    \item \keyword{એમિટર}: હેવી ડોપ્ડ, તીર કરંટ દિશા દર્શાવે
    \item \keyword{બેસ}: પાતળું, લાઇટ ડોપ્ડ મધ્ય વિસ્તાર
    \item \keyword{કલેક્ટર}: મોડરેટ ડોપ્ડ, ચાર્જ કેરિયર્સ એકત્રિત કરે
\end{itemize}
\end{solutionbox}

\begin{mnemonicbox}
\mnemonic{"NPN: અંદર પોઇન્ટ નથી, PNP: અંદર પોઇન્ટ કરે"}
\end{mnemonicbox}

\questionmarks{4(b)}{4}{CE એમ્પ્લિફાયરની લાક્ષણિકતાઓ દોરો અને સમજાવો.}

\begin{solutionbox}
\textbf{જવાબ}:

\textbf{CE એમ્પ્લિફાયર સર્કિટ:}
\begin{center}
    \begin{tikzpicture}[scale=0.8]
        \draw (0,0) node[ground]{} to[R, l=$R_e$] (0,1) -- (0,1.5) node[npn, anchor=E] (Q) {};
        \draw (Q.C) -- (0,3.5) to[R, l=$R_c$] (0,5) node[vcc]{$V_{CC}$};
        \draw (Q.B) -- (-1, 2.3) to[short, -o] (-2,2.3) node[left]{$V_{in}$};
        \draw (0,3.5) to[short, -o] (2,3.5) node[right]{$V_{out}$};
    \end{tikzpicture}
    \captionof{figure}{CE સર્કિટ}
\end{center}

\textbf{લાક્ષણિકતાઓ:}
\begin{center}
    \begin{tikzpicture}[scale=0.6]
        \draw[->] (0,0) -- (4,0) node[right]{$V_{BE}$};
        \draw[->] (0,0) -- (0,4) node[above]{$I_B$};
        \draw[thick] (1.5,0) .. controls (2,0.5) .. (3,3.5);
        \node at (2,0) [below] {0.7V};
    \end{tikzpicture}
    \\
    \begin{tikzpicture}[scale=0.6]
        \draw[->] (0,0) -- (5,0) node[right]{$V_{CE}$};
        \draw[->] (0,0) -- (0,4) node[above]{$I_C$};
        \foreach \i in {1,2,3} {
             \draw[thick] (0,0) -- (0.5,\i) -- (4.5,\i);
             \node at (4.7,\i) {$I_{B\i}$};
        }
    \end{tikzpicture}
    \captionof{figure}{Input/Output લાક્ષણિકતાઓ}
\end{center}

\textbf{મુખ્ય લક્ષણો:}
\begin{itemize}
    \item \keyword{કરંટ ગેઇન}: $\beta = IC/IB$ (ઉચ્ચ)
    \item \keyword{વોલ્ટેજ ગેઇન}: ઉચ્ચ
    \item \keyword{પાવર ગેઇન}: ખૂબ ઉચ્ચ
    \item \keyword{ફેઝ શિફ્ટ}: 180$^o$
\end{itemize}
\end{solutionbox}

\begin{mnemonicbox}
\mnemonic{"કોમન એમિટર, કરંટ એન્લાર્જ્ડ"}
\end{mnemonicbox}

\questionmarks{4(c)}{7}{કરંટ ગેઇન $\alpha$, $\beta$ અને $\gamma$ વચ્ચેનો સંબંધ મેળવો.}

\begin{solutionbox}
\textbf{જવાબ}:

\textbf{કરંટ ગેઇન વ્યાખ્યાઓ:}
\begin{itemize}
    \item \textbf{$\alpha$ (આલ્ફા)}: $\alpha = IC/IE$
    \item \textbf{$\beta$ (બીટા)}: $\beta = IC/IB$
    \item \textbf{$\gamma$ (ગામા)}: $\gamma = IE/IB$
\end{itemize}

\textbf{વ્યુત્પત્તિ:}

\textbf{પગલું 1: મૂળભૂત કરંટ સંબંધ}
$IE = IB + IC$ ... (1)

\textbf{પગલું 2: IE ના સંદર્ભમાં IC}
$IC = \alpha IE$

\textbf{પગલું 3: કરંટ સમીકરણમાં બદલો}
$IE = IB + \alpha IE$ \\
$IE(1 - \alpha) = IB$ \\
$IE = IB/(1 - \alpha)$ ... (2)

\textbf{પગલું 4: $\beta$ શોધો}
$\beta = IC/IB = \alpha IE / IB = \alpha/(1 - \alpha)$

\textbf{પગલું 5: $\alpha$ શોધો}
$\beta = \alpha/(1-\alpha) \implies \alpha = \beta/(1+\beta)$

\textbf{પગલું 6: $\gamma$ શોધો}
$\gamma = IE/IB = 1/(1-\alpha) = 1 + \beta$

\textbf{અંતિમ સંબંધો:}
\begin{itemize}
    \item $\beta = \alpha/(1 - \alpha)$
    \item $\alpha = \beta/(1 + \beta)$
    \item $\gamma = 1 + \beta$
\end{itemize}
\end{solutionbox}

\begin{mnemonicbox}
\mnemonic{"આલ્ફા બીટા ગામા, હંમેશાં સારા ગેઇન્સ"}
\end{mnemonicbox}

\questionmarks{4(a OR)}{3}{ટ્રાન્ઝિસ્ટર એમ્પ્લિફાયર માટે એક્ટિવ, સેચ્યુરેશન અને કટ-ઓફ રીજીયનની વ્યાખ્યા આપો.}

\begin{solutionbox}
\textbf{જવાબ}:

\textbf{ઓપરેટિંગ રીજીયન્સ:}
\begin{center}
\captionof{table}{રીજીયન્સ}
\begin{tabulary}{\linewidth}{L L L L}
    \toprule
    \textbf{રીજીયન} & \textbf{બેસ-એમિટર} & \textbf{બેસ-કલેક્ટર} & \textbf{લાક્ષણિકતાઓ} \\
    \midrule
    \textbf{એક્ટિવ} & ફોરવર્ડ બાયાસ્ડ & રિવર્સ બાયાસ્ડ & એમ્પ્લિફિકેશન રીજીયન \\
    \textbf{સેચ્યુરેશન} & ફોરવર્ડ બાયાસ્ડ & ફોરવર્ડ બાયાસ્ડ & સ્વિચ ON સ્ટેટ \\
    \textbf{કટ-ઓફ} & રિવર્સ બાયાસ્ડ & રિવર્સ બાયાસ્ડ & સ્વિચ OFF સ્ટેટ \\
    \bottomrule
\end{tabulary}
\end{center}
\end{solutionbox}

\begin{mnemonicbox}
\mnemonic{"એક્ટિવ એમ્પ્લિફાય, સેચ્યુરેટેડ સ્વિચ, કટ-ઓફ કટ્સ"}
\end{mnemonicbox}

\questionmarks{4(b OR)}{4}{એમ્પ્લિફાયર તરીકે ટ્રાન્ઝિસ્ટરનું કાર્ય સમજાવો.}

\begin{solutionbox}
\textbf{જવાબ}:

\textbf{એમ્પ્લિફાયર સર્કિટ:}
\begin{center}
    \begin{tikzpicture}[scale=0.8]
        \draw (0,0) node[ground]{} to[R, l=$R_E$] (0,1) -- (0,1.5) node[npn, anchor=E] (Q) {};
        \draw (Q.C) -- (0,3.5) to[R, l=$R_C$] (0,5) node[vcc]{$V_{CC}$};
        \draw (0,3.5) to[short, -o] (1.5,3.5) node[right]{$V_{out}$};
        \draw (Q.B) -- (-1, 2.3) to[short, -o] (-2,2.3) node[left]{$V_{in}$};
    \end{tikzpicture}
    \captionof{figure}{CE એમ્પ્લિફાયર}
\end{center}

\textbf{કાર્યસિદ્ધાંત:}
\begin{itemize}
    \item \keyword{નાનું ઇનપુટ સિગ્નલ} બેસ-એમિટર પર લાગુ
    \item \keyword{ઇનપુટ રેઝિસ્ટન્સ} ઓછું (કેટલાક kΩ)
    \item \keyword{નાનું બેસ કરંટ} મોટા કલેક્ટર કરંટને નિયંત્રિત કરે
    \item \keyword{આઉટપુટ} કલેક્ટર-એમિટરથી લેવાય
    \item \keyword{કરંટ એમ્પ્લિફિકેશન}: $IC = \beta IB$
\end{itemize}
\end{solutionbox}

\begin{mnemonicbox}
\mnemonic{"નાનું સિગ્નલ મોટું આઉટપુટ ટ્રિગર કરે"}
\end{mnemonicbox}

\questionmarks{4(c OR)}{7}{CB, CC તેમજ CE એમ્પ્લિફાયરને સરખાવો.}

\begin{solutionbox}
\textbf{જવાબ}:

\textbf{વ્યાપક તુલના:}
\begin{center}
\captionof{table}{સરખામણી}
\begin{tabulary}{\linewidth}{L L L L}
    \toprule
    \textbf{પેરામીટર} & \textbf{CB} & \textbf{CE} & \textbf{CC} \\
    \midrule
    \textbf{ઇનપુટ ટર્મિનલ} & એમિટર & બેસ & બેસ \\
    \textbf{આઉટપુટ ટર્મિનલ} & કલેક્ટર & કલેક્ટર & એમિટર \\
    \textbf{કોમન ટર્મિનલ} & બેસ & એમિટર & કલેક્ટર \\
    \textbf{કરંટ ગેઇન} & $\alpha < 1$ & $\beta \gg 1$ & $\gamma = (1 + \beta)$ \\
    \textbf{વોલ્ટેજ ગેઇન} & ઉચ્ચ & ઉચ્ચ & $< 1$ \\
    \textbf{પાવર ગેઇન} & મધ્યમ & ખૂબ ઉચ્ચ & મધ્યમ \\
    \textbf{ઇનપુટ રેઝિસ્ટન્સ} & ખૂબ ઓછું & મધ્યમ & ખૂબ ઉચ્ચ \\
    \textbf{આઉટપુટ રેઝિસ્ટન્સ} & ખૂબ ઉચ્ચ & ઉચ્ચ & ઓછું \\
    \textbf{ફેઝ શિફ્ટ} & 0$^o$ & 180$^o$ & 0$^o$ \\
    \bottomrule
\end{tabulary}
\end{center}
\end{solutionbox}

\begin{mnemonicbox}
\mnemonic{"CB કમ્યુનિકેશન માટે, CE કોમન યુઝ માટે, CC કપલિંગ માટે"}
\end{mnemonicbox}

\questionmarks{5(a)}{3}{IC 555 નો પિન ડાયાગ્રામ દોરો.}

\begin{solutionbox}
\textbf{જવાબ}:

\textbf{IC 555 પિન ડાયાગ્રામ:}
\begin{center}
    \begin{tikzpicture}
        \draw[thick] (0,0) rectangle (4,4);
        \draw (2,4) arc(180:0:0.2); 
        \node at (2,2) {\textbf{IC 555}};
        \foreach \y/\t/\n in {3.5/1/GND, 2.5/2/Trig, 1.5/3/Out, 0.5/4/Reset} {
            \draw (0,\y) -- (-0.5,\y) node[left] {\n~(\t)};
            \draw (-0.1,\y) circle(0.05);
        }
        \foreach \y/\t/\n in {3.5/8/Vcc, 2.5/7/Disch, 1.5/6/Thresh, 0.5/5/Ctrl} {
            \draw (4,\y) -- (4.5,\y) node[right] {(\t)~\n};
            \draw (4.1,\y) circle(0.05);
        }
    \end{tikzpicture}
    \captionof{figure}{IC 555 પિનઆઉટ}
\end{center}

\textbf{પિન કાર્યો:}
\begin{itemize}
    \item \keyword{1 GND}: Ground
    \item \keyword{2 Trigger}: ટાઇમિંગ શરૂ કરે
    \item \keyword{3 Output}: આઉટપુટ
    \item \keyword{4 Reset}: રીસેટ
    \item \keyword{5 Control}: વોલ્ટેજ કંટ્રોલ
    \item \keyword{6 Threshold}: ટાઇમિંગ સમાપ્ત
    \item \keyword{7 Discharge}: ડિસ્ચાર્જ
    \item \keyword{8 Vcc}: પાવર સપ્લાય
\end{itemize}
\end{solutionbox}

\begin{mnemonicbox}
\mnemonic{"ગ્રેટ ટાઇમર, ગ્રેટ પિન્સ"}
\end{mnemonicbox}

\questionmarks{5(b)}{4}{555 ટાઇમર IC ની વિશેષતાઓની યાદી બનાવો.}

\begin{solutionbox}
\textbf{જવાબ}:

\textbf{મુખ્ય લક્ષણો:}
\begin{center}
\captionof{table}{555 લક્ષણો}
\begin{tabulary}{\linewidth}{L L}
    \toprule
    \textbf{લક્ષણ} & \textbf{વિશિષ્ટતા} \\
    \midrule
    \textbf{સપ્લાય વોલ્ટેજ} & 5V થી 18V \\
    \textbf{આઉટપુટ કરંટ} & 200mA સોર્સ/સિંક \\
    \textbf{તાપમાન રેન્જ} & 0$^o$C થી 70$^o$C \\
    \textbf{ટાઇમિંગ રેન્જ} & $\mu$s થી કલાકો \\
    \bottomrule
\end{tabulary}
\end{center}

\textbf{ફાયદા:}
\begin{itemize}
    \item \keyword{વર્સેટાઇલ ટાઇમર} અનેક એપ્લિકેશન્સ માટે
    \item \keyword{વાપરવામાં સરળ} ન્યૂનતમ બાહ્ય કમ્પોનન્ટ્સ સાથે
    \item \keyword{વિશ્વસનીય ઓપરેશન} વિવિધ પરિસ્થિતિઓમાં
\end{itemize}
\end{solutionbox}

\begin{mnemonicbox}
\mnemonic{"શાનદાર લક્ષણો, લવચીક કાર્યો"}
\end{mnemonicbox}

\questionmarks{5(c)}{7}{555 ટાઇમર IC નો ઉપયોગ કરીને મોનો સ્ટેબલ મલ્ટીવાઇબ્રેટર સમજાવો.}

\begin{solutionbox}
\textbf{જવાબ}:

\textbf{મોનોસ્ટેબલ સર્કિટ:}
\begin{center}
    \begin{tikzpicture}[scale=0.8]
        \draw (0,0) node[left]{GND} -- (6,0) node[right]{GND};
        \draw (0,5) node[left]{Vcc} -- (6,5);
        \draw (3,2) rectangle (5,4); 
        \node at (4,3) {555};
        
        \draw (4,4) -- (4,5); % 8
        \draw (4,2) -- (4,0); % 1
        
        \draw (2,5) to[R, l=$R$] (2,3.5) -- (3,3.5); % 7
        \node at (3.2,3.5) {7};
        \draw (2,3.5) -- (2,3); 
        \draw (2,3) -- (3,3); % 6
        \node at (3.2,3) {6}; 
        \draw (2,3) to[C, l=$C$] (2,0);
        
        \draw (3,2.5) -- (1,2.5) node[left]{Trigger (2)};
        \draw (5,3) -- (6,3) node[right]{Output (3)};
    \end{tikzpicture}
    \captionof{figure}{મોનોસ્ટેબલ સર્કિટ}
\end{center}

\textbf{કાર્યસિદ્ધાંત:}
\begin{itemize}
    \item \keyword{સ્ટેબલ સ્ટેટ}: આઉટપુટ LOW, કેપેસિટર ડિસ્ચાર્જ્ડ
    \item \keyword{ટ્રિગર્ડ સ્ટેટ}: નેગેટિવ પલ્સ -> આઉટપુટ HIGH, C ચાર્જ થાય
    \item \keyword{ટાઇમિંગ પીરિયડ}: $T = 1.1 \times R \times C$
\end{itemize}
\end{solutionbox}

\begin{mnemonicbox}
\mnemonic{"મોનો મતલબ એક પલ્સ માત્ર"}
\end{mnemonicbox}

\questionmarks{5(a OR)}{3}{IC 555 ની એપ્લિકેશનની યાદી બનાવો.}

\begin{solutionbox}
\textbf{જવાબ}:

\textbf{ટાઇમર એપ્લિકેશન્સ:}
\begin{center}
\captionof{table}{555 એપ્લિકેશન્સ}
\begin{tabulary}{\linewidth}{L L}
    \toprule
    \textbf{કેટેગરી} & \textbf{એપ્લિકેશન્સ} \\
    \midrule
    \textbf{ટાઇમિંગ સર્કિટ્સ} & ડિલે ટાઇમર્સ, પલ્સ જનરેટર્સ \\
    \textbf{ઓસિલેટર્સ} & ક્લોક જનરેટર્સ, ફ્રીક્વન્સી ડિવાઇડર્સ \\
    \textbf{કંટ્રોલ સર્કિટ્સ} & PWM કંટ્રોલર્સ, મોટર સ્પીડ કંટ્રોલ \\
    \textbf{ઓટોમોટિવ} & ટર્ન સિગ્નલ ફ્લેશર્સ, વિન્ડશીલ્ડ વાઇપર્સ \\
    \bottomrule
\end{tabulary}
\end{center}
\end{solutionbox}

\begin{mnemonicbox}
\mnemonic{"મહાન કાર્યો માટે ટાઇમર"}
\end{mnemonicbox}

\questionmarks{5(b OR)}{4}{IC 555 નો આંતરિક બ્લોક ડાયાગ્રામ દોરો અને સમજાવો.}

\begin{solutionbox}
\textbf{જવાબ}:

\textbf{આંતરિક બ્લોક ડાયાગ્રામ:}
\begin{center}
    \begin{tikzpicture}[scale=0.6, font=\small]
        \draw (2,8) node[above]{Vcc (8)};
        \draw (2,8) to[R, l=5k] (2,6) to[R, l=5k] (2,4) to[R, l=5k] (2,2) node[below]{GND (1)};
        
        \draw (4,6) node[shape=isosceles triangle, draw, shape border rotate=180](C1){};
        \node at (4,6) {Comp A};
        \draw (4,3) node[shape=isosceles triangle, draw, shape border rotate=180](C2){};
        \node at (4,3) {Comp B};
        
        \draw (6,3.5) rectangle (8,5.5);
        \node at (7,4.5) {SR FF};
        
        \draw (8,4.5) -- (9,4.5) node[right]{Output (3)};
        
        \draw (2,6) -- (C1.south); 
        \draw (2,4) -- (C2.north); 
        \draw (0,6.5) node[left]{Thresh(6)} -- (C1.north);
        \draw (0,2.5) node[left]{Trig(2)} -- (C2.south);
        
        \draw (C1.apex) -- (6,5); 
        \draw (C2.apex) -- (6,4); 
        
        \draw (7,3.5) -- (7,2);
        \draw (7,2) node[right]{Disch(7)};
    \end{tikzpicture}
    \captionof{figure}{આંતરિક બ્લોક ડાયાગ્રામ}
\end{center}

\textbf{બ્લોક કાર્યો:}
\begin{itemize}
    \item \keyword{વોલ્ટેજ ડિવાઇડર}: Vcc/3 અને 2Vcc/3 રેફરન્સ બનાવે
    \item \keyword{કોમ્પેરેટર્સ}: થ્રેશહોલ્ડ અને ટ્રિગરની તુલના કરે
    \item \keyword{SR ફ્લિપ-ફ્લોપ}: આઉટપુટ સ્ટેટ નિયંત્રિત કરે
    \item \keyword{આઉટપુટ બફર}: ઉચ્ચ કરંટ આઉટપુટ પૂરું પાડે
    \item \keyword{ડિસ્ચાર્જ ટ્રાન્ઝિસ્ટર}: ટાઇમિંગ કેપેસિટર ડિસ્ચાર્જ કરે
\end{itemize}
\end{solutionbox}

\begin{mnemonicbox}
\mnemonic{"આંતરિક બુદ્ધિ, ઇન્ટિગ્રેટેડ અમલીકરણ"}
\end{mnemonicbox}

\questionmarks{5(c OR)}{7}{555 ટાઇમર IC નો ઉપયોગ કરીને એસ્ટેબલ મલ્ટીવાઇબ્રેટર સમજાવો.}

\begin{solutionbox}
\textbf{જવાબ}:

\textbf{એસ્ટેબલ સર્કિટ:}
\begin{center}
    \begin{tikzpicture}[scale=0.8]
        \draw (0,5) node[left]{Vcc} -- (6,5);
        \draw (0,0) node[left]{GND} -- (6,0);
        \draw (3,2) rectangle (5,4);
        \node at (4,3) {555};
        
        \draw (4,4) -- (4,5); % 8
        \draw (4,2) -- (4,0); % 1
        
        \draw (2,5) to[R, l=$R_1$] (2,3.5);
        \draw (2,3.5) -- (3,3.5); % 7
        \node at (3.2,3.5) {7};
        \draw (2,3.5) to[R, l=$R_2$] (2,2.5);
        \draw (2,2.5) -- (3,2.5); % 6
        \node at (3.2,2.5) {6};
        \draw (2,2.5) -- (2,2) -- (3,2); % 2
        \node at (3.2,2) {2};
        \draw (2,2) to[C, l=$C$] (2,0);
        
        \draw (5,3) -- (6,3) node[right]{Output};
    \end{tikzpicture}
    \captionof{figure}{એસ્ટેબલ સર્કિટ}
\end{center}

\textbf{કાર્યસિદ્ધાંત:}
\begin{itemize}
    \item \keyword{ચાર્જિંગ ફેઝ}: કેપેસિટર R1 + R2 મારફત ચાર્જ થાય
    \item \keyword{ડિસ્ચાર્જિંગ ફેઝ}: કેપેસિટર માત્ર R2 મારફત ડિસ્ચાર્જ થાય
    \item \keyword{આઉટપુટ}: સતત HIGH અને LOW વચ્ચે બદલાય છે
\end{itemize}

\textbf{ફ્રીક્વન્સી ગણતરીઓ:}
\begin{itemize}
    \item $T1 = 0.693(R1 + R2)C$
    \item $T2 = 0.693 \times R2 \times C$
    \item $f = 1.44/[(R1 + 2R2)C]$
\end{itemize}

\textbf{ઉપયોગો:}
\begin{itemize}
    \item LED ફ્લેશર્સ, ક્લોક જનરેટર્સ, ટોન જનરેટર્સ
\end{itemize}
\end{solutionbox}

\begin{mnemonicbox}
\mnemonic{"એસ્ટેબલ હંમેશાં ઓટોમેટિક ઓલ્ટરનેટ્સ"}
\end{mnemonicbox}

\end{document}
