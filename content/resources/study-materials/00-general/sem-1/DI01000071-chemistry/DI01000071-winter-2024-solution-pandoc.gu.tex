\documentclass[10pt,a4paper]{article}

% content/resources/templates/preamble.tex
\usepackage[margin=0.6in]{geometry}
\author{Milav Dabgar}
\usepackage{amsmath,amssymb,amsthm}
\usepackage{booktabs}
\usepackage{multirow}
\usepackage{xcolor}
\usepackage{tcolorbox}
\tcbuselibrary{breakable,skins}
\usepackage[colorlinks=true,linkcolor=blue]{hyperref}
\usepackage{titlesec}
\usepackage{enumitem}
\usepackage{tikz}
\usepackage{pgfplots}
\usepackage{circuitikz}
\usepackage[version=4]{mhchem}
\usepackage{longtable}
\usepackage{array}
\usepackage{float}
\usepackage{caption}
\usepackage{listings}

\lstset{
  basicstyle=\small\ttfamily,
  breaklines=true,
  breakatwhitespace=false,
  postbreak=\mbox{\textcolor{red}{$\hookrightarrow$}\space},
  float=false,
  numbers=left,
  numberstyle=\tiny\color{gray},
  numbersep=10pt,
  xleftmargin=2em,
  keywordstyle=\color{blue},
  commentstyle=\color{green!60!black},
  stringstyle=\color{purple},
  backgroundcolor=\color{gray!5},
  showstringspaces=false,
  tabsize=2,
  captionpos=b,
  keepspaces=true,
  columns=flexible
}

\pgfplotsset{compat=1.18}
\usetikzlibrary{shapes,arrows,positioning,calc,patterns,decorations.pathmorphing,decorations.markings,arrows.meta}

% Color scheme
\definecolor{headcolor}{RGB}{0,102,204}
\definecolor{keycolor}{RGB}{220,20,60}
\definecolor{solutioncolor}{RGB}{34,139,34}
\definecolor{mnemoniccolor}{RGB}{148,0,211}
\definecolor{codecolor}{RGB}{0,0,100}

% Spacing
\setlength{\parskip}{3pt}
\setlist[itemize]{nosep}
\setlist[enumerate]{nosep}

% Title formatting
\titleformat{\section}{\Large\bfseries\color{headcolor}}{\thesection}{1em}{}
\titleformat{\subsection}{\large\bfseries\color{headcolor}}{\thesubsection}{1em}{}

% Pandoc tightlist compatibility
\providecommand{\tightlist}{%
  \setlength{\itemsep}{0pt}\setlength{\parskip}{0pt}}

% Pandoc longtable compatibility
\newcounter{none}
\def\thenone{}


% content/resources/templates/gujarati-boxes.tex
\usepackage{fontspec}
\usepackage{polyglossia}

% Set Gujarati as main language (document is primarily in Gujarati)
% Note: gloss-gujarati.ldf doesn't exist in polyglossia, but it will use hyphenation patterns
\setdefaultlanguage{gujarati}
\setotherlanguage{english}

% Configure Gujarati font properly
% Use Language=Default to prevent polyglossia from trying to add language-specific features
% that don't exist for Gujarati, which causes "empty feature" warnings
\newfontfamily\gujaratifont[Script=Gujarati,AutoFakeBold=2.5,AutoFakeSlant=0.3]{Noto Sans Gujarati}
\setmainfont[Script=Gujarati,AutoFakeBold=2.5,AutoFakeSlant=0.3]{Noto Sans Gujarati}
% Use Noto Sans Gujarati for monospace to support Gujarati in text
\setmonofont[Scale=0.9]{Noto Sans Gujarati}

% Configure English to use the same font
\newfontfamily\englishfont[Script=Gujarati,AutoFakeBold=2.5,AutoFakeSlant=0.3]{Noto Sans Gujarati}

% Translations for polyglossia
\gappto\captionsgujarati{
  \renewcommand{\tablename}{કોષ્ટક}
  \renewcommand{\figurename}{આકૃતિ}
}

% Helper for TikZ nodes to ensure Gujarati font
\newcommand{\gu}[1]{{\gujaratifont #1}}

% Custom environments
\newtcolorbox{solutionbox}{
    breakable,
    enhanced,
    colback=solutioncolor!5!white,
    colframe=solutioncolor!75!black,
    fonttitle=\bfseries,
    title=જવાબ
}

\newtcolorbox{solutionboxnobreak}{
 colback=solutioncolor!5!white,
 colframe=solutioncolor!75!black,
 fonttitle=\bfseries,
 title=જવાબ
}

\newtcolorbox{keyformula}{
 breakable,
 enhanced,
 colback=keycolor!5!white,
 colframe=keycolor!75!black,
 fonttitle=\bfseries,
 title=રાસાયણિક સમીકરણ/સૂત્ર
}

\newtcolorbox{mnemonicbox}{
 breakable,
 enhanced,
 colback=mnemoniccolor!5!white,
 colframe=mnemoniccolor!75!black,
 fonttitle=\bfseries,
 title=મેમરી ટ્રીક
}


\begin{document}

\begin{center}
{\Huge\bfseries\color{headcolor} Chemistry (Gujarati)}\\[5pt]
{\LARGE DI01000071 -- Winter 2024}\\[3pt]
{\large Semester 1 Study Material}\\[3pt]
{\normalsize\textit{Detailed Solutions and Explanations}}
\end{center}

\vspace{10pt}

\subsection*{પ્રશ્ન 1 [14
ગુણ]}\label{uxaaauxab0uxab6uxaa8-1-14-uxa97uxaa3}

\textbf{આપેલ વિકલ્પોમાંથી યોગ્ય વિકલ્પ પસંદ કરી ખાલી જગ્યાઓ પૂરો:}

\begin{solutionbox}

\begin{longtable}[]{@{}lll@{}}
\toprule\noalign{}
પ્રશ્ન & જવાબ & સમજૂતી \\
\midrule\noalign{}
\endhead
\bottomrule\noalign{}
\endlastfoot
(1) & [Ar]4s^{1}3d^{1}^{0} & Cu માં 29 ઇલેક્ટ્રોન છે, Aufbau નિયમનો અપવાદ \\
(2) & 14 & pH + pOH = 14 (25^\circC પર) \\
(3) & કેથોડ & શુદ્ધ તાંબુ નેગેટિવ ઇલેક્ટ્રોડ પર જમા થાય \\
(4) & Cu & તાંબુ સુરક્ષિત ઓક્સાઇડ સ્તર બનાવે છે \\
(5) & અર્ધ-ઘન & પીટ અંશતઃ વિઘટિત કાર્બનિક પદાર્થ છે \\
(6) & ડ્યુલોંગ & ડ્યુલોંગના સૂત્રથી ઉષ્મીય મૂલ્ય ગણાય \\
(7) & લિગ્નાઇટ & લિગ્નાઇટમાં સૌથી વધુ ભેજ (35-75\%) \\
(8) & પોઇઝ & ડાયનેમિક વિસ્કોસિટીનો SI એકમ \\
(9) & ઊંચું & ઊંચું ફ્લેશ પોઇન્ટ ઇગ્નિશન અટકાવે છે \\
(10) & પાયસ & તેલ-પાણીનું મિશ્રણ પાયસ બનાવે છે \\
(11) & બેકેલાઇટ & ફિનોલ ફોર્મેલ્ડિહાઇડ = બેકેલાઇટ \\
(12) & S & વલ્કેનાઇઝેશન માટે સલ્ફર વપરાય છે \\
(13) & PHBV & PHBV જૈવવિઘટનીય પોલિમર છે \\
(14) & વોલ્ટ & EMF વોલ્ટમાં માપાય છે \\
\end{longtable}

\end{solutionbox}
\begin{mnemonicbox}
``રાસાયણિક તાંબુ સુંદર ગુણધર્મો બનાવે''

\end{mnemonicbox}
\begin{center}\rule{0.5\linewidth}{0.5pt}\end{center}

\subsection*{પ્રશ્ન 2(A) [6
ગુણ]}\label{q2a}

\subsubsection{પ્રશ્ન 2(A)(1) [3
ગુણ]}\label{uxaaauxab0uxab6uxaa8-2a1-3-uxa97uxaa3}

\textbf{જુદાં જુદાં ક્ષેત્રોમાં pHની ત્રણ અગત્યતાની સૂચિ બનાવો.}

\begin{solutionbox}

\begin{longtable}[]{@{}
  >{\raggedright\arraybackslash}p{(\linewidth - 4\tabcolsep) * \real{0.2692}}
  >{\raggedright\arraybackslash}p{(\linewidth - 4\tabcolsep) * \real{0.2692}}
  >{\raggedright\arraybackslash}p{(\linewidth - 4\tabcolsep) * \real{0.4615}}@{}}
\toprule\noalign{}
\begin{minipage}[b]{\linewidth}\raggedright
ક્ષેત્ર
\end{minipage} & \begin{minipage}[b]{\linewidth}\raggedright
મહત્વ
\end{minipage} & \begin{minipage}[b]{\linewidth}\raggedright
એપ્લિકેશન
\end{minipage} \\
\midrule\noalign{}
\endhead
\bottomrule\noalign{}
\endlastfoot
\textbf{દવાશાસ્ત્ર} & લોહીનું pH જાળવણું & સામાન્ય pH 7.35-7.45 યોગ્ય શરીરિક
કાર્ય માટે \\
\textbf{કૃષિ} & માટીનું pH ઓપ્ટિમાઇઝેશન & pH 6-7 પાકની વૃદ્ધિ અને પોષણ માટે
આદર્શ \\
\textbf{ઉદ્યોગ} & ગુણવત્તા નિયંત્રણ & pH ખોરાક, કાપડ, દવાઓની ગુણવત્તાને અસર
કરે \\
\end{longtable}

\end{solutionbox}
\begin{mnemonicbox}
``દવા કૃષિ ઉદ્યોગ''

\end{mnemonicbox}
\subsubsection{પ્રશ્ન 2(A)(2) [3
ગુણ]}\label{uxaaauxab0uxab6uxaa8-2a2-3-uxa97uxaa3}

\textbf{વ્યાખ્યા આપો: બફર દ્રાવણો, અર્ધ-કોષ, વિદ્યુતવિભાજનનો ફેરાડેનો પ્રથમ
નિયમ.}

\begin{solutionbox}

\begin{itemize}
\tightlist
\item
  \textbf{બફર દ્રાવણો}: એવા દ્રાવણો જે થોડું એસિડ કે બેઝ ઉમેરવાથી pH બદલાવમાં
  પ્રતિકાર કરે
\item
  \textbf{અર્ધ-કોષ}: એક ઇલેક્ટ્રોડ તેના આયનિક દ્રાવણમાં ડૂબેલો, ઓક્સિડેશન કે રિડક્શન
  દર્શાવે
\item
  \textbf{ફેરાડેનો પ્રથમ નિયમ}: ઇલેક્ટ્રોડ પર જમા/મુક્ત થતા પદાર્થની માત્રા
  વીજળીની માત્રાના સીધા પ્રમાણમાં હોય
\end{itemize}

\end{solutionbox}
\begin{mnemonicbox}
``બફર મદદ ફેરાડે''

\end{mnemonicbox}
\subsubsection{પ્રશ્ન 2(A)(3) [3
ગુણ]}\label{uxaaauxab0uxab6uxaa8-2a3-3-uxa97uxaa3}

\textbf{ક્ષારણ દર ઉપર અસર કરતાં પરિબળો જણાવો.}

\begin{solutionbox}

\begin{longtable}[]{@{}lll@{}}
\toprule\noalign{}
પરિબળ & અસર & વર્ણન \\
\midrule\noalign{}
\endhead
\bottomrule\noalign{}
\endlastfoot
\textbf{ધાતુની શુદ્ધતા} & વધુ શુદ્ધતા = ઓછું ક્ષારણ & અશુદ્ધિઓ ગેલ્વેનિક કોષ બનાવે \\
\textbf{તાપમાન} & વધુ તાપમાન = ઝડપી ક્ષારણ & પ્રતિક્રિયા દર વધારે \\
\textbf{ભેજ} & વધુ ભેજ = વધુ ક્ષારણ & ઇલેક્ટ્રોકેમિકલ પ્રતિક્રિયાઓ પ્રોત્સાહન \\
\end{longtable}

\end{solutionbox}
\begin{mnemonicbox}
``શુદ્ધ તાપમાન ભેજ''

\end{mnemonicbox}
\begin{center}\rule{0.5\linewidth}{0.5pt}\end{center}

\subsection*{પ્રશ્ન 2(B) [8
ગુણ]}\label{q2b}

\subsubsection{પ્રશ્ન 2(B)(1) [4
ગુણ]}\label{uxaaauxab0uxab6uxaa8-2b1-4-uxa97uxaa3}

\textbf{કક્ષાઓ અને કક્ષકો વચ્ચે સરખામણી કરો (દરેકના ચાર મુદ્દાઓ).}

\begin{solutionbox}

\begin{longtable}[]{@{}lll@{}}
\toprule\noalign{}
પાસું & કક્ષાઓ & કક્ષકો \\
\midrule\noalign{}
\endhead
\bottomrule\noalign{}
\endlastfoot
\textbf{વ્યાખ્યા} & નિશ્ચિત ગોળાકાર માર્ગ & 3D સંભાવના પ્રદેશો \\
\textbf{આકાર} & ગોળાકાર/અંડાકાર & s,p,d,f આકારો \\
\textbf{ઊર્જા} & નિશ્ચિત ઊર્જા સ્તરો & ઊર્જા શ્રેણીઓ \\
\textbf{ઇલેક્ટ્રોન સ્થાન} & ચોક્કસ સ્થિતિ & મળવાની સંભાવના \\
\end{longtable}

\textbf{આકૃતિ:}

\begin{verbatim}
    કક્ષાઓ (બોહર મોડેલ)          કક્ષકો (ક્વાન્ટમ મોડેલ)
    
         e{-                           ∴∴∴∴∴}
      ○ ──── ○                       ∴∴∴∴∴∴∴
    ○         ○                    ∴∴∴∴∴∴∴∴∴
      ○ ──── ○                       ∴∴∴∴∴
         +                           ∴∴∴∴∴
\end{verbatim}

\end{solutionbox}
\begin{mnemonicbox}
``નિશ્ચિત આકાર ઊર્જા સ્થાન''

\end{mnemonicbox}
\subsubsection{પ્રશ્ન 2(B)(2) [4
ગુણ]}\label{uxaaauxab0uxab6uxaa8-2b2-4-uxa97uxaa3}

\textbf{દરેકના એક ઉદાહરણ સાથે તેના સ્ત્રોતો અને ભૌતિક સ્થિતિઓના આધારે ઇંધણોનું
વર્ગીકરણ કરો.}

\begin{solutionbox}


\begin{longtable}[]{@{}llll@{}}
\toprule\noalign{}
વર્ગીકરણ & પ્રકાર & ઉદાહરણ & વર્ણન \\
\midrule\noalign{}
\endhead
\bottomrule\noalign{}
\endlastfoot
\textbf{સ્ત્રોત આધારિત} & કુદરતી & કોલસો & કુદરતી રીતે બન્યું \\
& કૃત્રિમ & પેટ્રોલ & માનવ નિર્મિત \\
\textbf{ભૌતિક સ્થિતિ} & ઘન & લાકડું & ઓરડાના તાપમાને ઘન \\
& પ્રવાહી & ડીઝલ & ઓરડાના તાપમાને પ્રવાહી \\
& ગેસીય & LPG & ઓરડાના તાપમાને ગેસ \\
\end{longtable}

\end{solutionbox}
\begin{mnemonicbox}
``કુદરતી કૃત્રિમ, ઘન પ્રવાહી ગેસ''

\end{mnemonicbox}
\subsubsection{પ્રશ્ન 2(B)(3) [4
ગુણ]}\label{uxaaauxab0uxab6uxaa8-2b3-4-uxa97uxaa3}

\textbf{બાયોડીઝલ વિશે ચાર અગત્યના મુદ્દાઓ સમજાવો.}

\begin{solutionbox}

\begin{itemize}
\tightlist
\item
  \textbf{સ્ત્રોત}: વનસ્પતિ તેલ, પ્રાણીઓની ચરબી અથવા વપરાયેલા રસોઈ તેલમાંથી બને
\item
  \textbf{પ્રક્રિયા}: મેથેનોલ/ઇથેનોલ સાથે ટ્રાન્સએસ્ટેરિફિકેશન પ્રતિક્રિયાથી બને
\item
  \textbf{ગુણધર્મો}: જૈવવિઘટનીય, બિન-ઝેરી, નવીકરણીય ઇંધણ સ્ત્રોત
\item
  \textbf{ઉપયોગો}: ડીઝલ એન્જિનમાં વપરાય, ઉત્સર્જન 75\% ઘટાડે
\end{itemize}

\textbf{રાસાયણિક પ્રતિક્રિયા:}

\begin{verbatim}
વનસ્પતિ તેલ + મેથેનોલ \rightarrow બાયો-ડીઝલ + ગ્લિસેરોલ
\end{verbatim}

\end{solutionbox}
\begin{mnemonicbox}
``સ્ત્રોત પ્રક્રિયા ગુણધર્મો ઉપયોગો''

\end{mnemonicbox}
\begin{center}\rule{0.5\linewidth}{0.5pt}\end{center}

\subsection*{પ્રશ્ન 3(A) [6
ગુણ]}\label{q3a}

\subsubsection{પ્રશ્ન 3(A)(1) [3
ગુણ]}\label{uxaaauxab0uxab6uxaa8-3a1-3-uxa97uxaa3}

\textbf{ઉદાહરણની મદદથી દ્રાવ્ય, દ્રાવક અને દ્રાવણ સમજાવો.}

\begin{solutionbox}

\begin{longtable}[]{@{}lll@{}}
\toprule\noalign{}
ઘટક & વ્યાખ્યા & ઉદાહરણ \\
\midrule\noalign{}
\endhead
\bottomrule\noalign{}
\endlastfoot
\textbf{દ્રાવ્ય} & જે પદાર્થ ઓગળે છે & મીઠું (NaCl) \\
\textbf{દ્રાવક} & જેમાં પદાર્થ ઓગળે છે & પાણી (H_{2}O) \\
\textbf{દ્રાવણ} & સમાંગી મિશ્રણ & મીઠાનું પાણી \\
\end{longtable}

\textbf{ઉદાહરણ}: ખાંડ + પાણી = ખાંડનું દ્રાવણ

\begin{itemize}
\tightlist
\item
  ખાંડ = દ્રાવ્ય, પાણી = દ્રાવક, ખાંડનું પાણી = દ્રાવણ
\end{itemize}

\end{solutionbox}
\begin{mnemonicbox}
``દ્રાવ્ય દ્રાવક દ્રાવણ''

\end{mnemonicbox}
\subsubsection{પ્રશ્ન 3(A)(2) [3
ગુણ]}\label{uxaaauxab0uxab6uxaa8-3a2-3-uxa97uxaa3}

\textbf{NaClમાં વિદ્યુતસંયોજક બંધનું નિર્માણ સમજાવો.}

\begin{solutionbox}

\textbf{પ્રક્રિયા}:

\begin{itemize}
\tightlist
\item
  \textbf{પગલું 1}: Na એક ઇલેક્ટ્રોન ગુમાવે \rightarrow Na^{+} (કેટાયન)
\item
  \textbf{પગલું 2}: Cl એક ઇલેક્ટ્રોન મેળવે \rightarrow Cl^{-} (આયન)
\item
  \textbf{પગલું 3}: Na^{+} અને Cl^{-} વચ્ચે વિદ્યુતસ્થિતિક આકર્ષણ
\end{itemize}

\textbf{આકૃતિ:}

\begin{verbatim}
Na        Na^{+  +  e^{-}}
Cl + e^{-      Cl^{-}}

Na^{+  +  Cl^{-}    Na^{+}Cl^{-} (NaCl)}
\end{verbatim}

\end{solutionbox}
\begin{mnemonicbox}
``સોડિયમ ગુમાવે, ક્લોરિન મેળવે, આકર્ષણ બને''

\end{mnemonicbox}
\subsubsection{પ્રશ્ન 3(A)(3) [3
ગુણ]}\label{uxaaauxab0uxab6uxaa8-3a3-3-uxa97uxaa3}

\textbf{ગેસોલીન માટે ઓક્ટેન આંક સમજાવો.}

\begin{solutionbox}

\begin{longtable}[]{@{}ll@{}}
\toprule\noalign{}
પાસું & વર્ણન \\
\midrule\noalign{}
\endhead
\bottomrule\noalign{}
\endlastfoot
\textbf{વ્યાખ્યા} & ઇંધણની નોકિંગ સામે પ્રતિકારશક્તિનું માપ \\
\textbf{સ્કેલ} & 0-100, વધુ = વધુ સારી એન્ટી-નોક ગુણવત્તા \\
\textbf{માનક} & n-હેપ્ટેન = 0, આઇસો-ઓક્ટેન = 100 \\
\end{longtable}

\textbf{ઉપયોગો}: ઊંચા ઓક્ટેન ઇંધણ એન્જિન નોકિંગ અટકાવે, કામગીરી સુધારે

\end{solutionbox}
\begin{mnemonicbox}
``ઓક્ટેન નોકિંગ વિરોધી''

\end{mnemonicbox}
\begin{center}\rule{0.5\linewidth}{0.5pt}\end{center}

\subsection*{પ્રશ્ન 3(B) [8
ગુણ]}\label{q3b}

\subsubsection{પ્રશ્ન 3(B)(1) [4
ગુણ]}\label{uxaaauxab0uxab6uxaa8-3b1-4-uxa97uxaa3}

\textbf{અશુદ્ધ Cuનું વિદ્યુતશુદ્ધિકરણ રાસાયણિક સમીકરણો અને નામ નિર્દેશનવાળી આકૃતિ
સાથે સમજાવો.}

\begin{solutionbox}

\textbf{પ્રક્રિયા}:

\begin{itemize}
\tightlist
\item
  \textbf{એનોડ}: અશુદ્ધ તાંબુ ઓગળે
\item
  \textbf{કેથોડ}: શુદ્ધ તાંબુ જમા થાય
\item
  \textbf{ઇલેક્ટ્રોલાઇટ}: CuSO_{4} દ્રાવણ
\end{itemize}

\textbf{રાસાયણિક સમીકરણો}:

\begin{itemize}
\tightlist
\item
  એનોડ પર: Cu \rightarrow Cu^{2}^{+} + 2e^{-}
\item
  કેથોડ પર: Cu^{2}^{+} + 2e^{-} \rightarrow Cu
\end{itemize}

\textbf{આકૃતિ:}

\begin{verbatim}
    બેટરી
     ╔═╗
   {- ║ ║ +}
     ╚═╝
     │ │
  કેથોડ │ એનોડ
  (શુદ્ધ Cu)│(અશુદ્ધ Cu)
     │ │
    ╔═══════╗
    ║CuSO_{4  ║}
    ║દ્રાવણ   ║
    ╚═══════╝
\end{verbatim}

\end{solutionbox}
\begin{mnemonicbox}
``એનોડ ઓગળે, કેથોડ જમાવે''

\end{mnemonicbox}
\subsubsection{પ્રશ્ન 3(B)(2) [4
ગુણ]}\label{uxaaauxab0uxab6uxaa8-3b2-4-uxa97uxaa3}

\textbf{રાસાયણિક સમીકરણ સાથે ઇથિનની બનાવટ સમજાવો. તેના બે ગુણધર્મો અને બે
ઉપયોગો પણ લખો.}

\begin{solutionbox}

\textbf{તૈયારી}: C_{2}H_{5}OH \rightarrow C_{2}H_{4} + H_{2}O (ઇથેનોલનું નિર્જલીકરણ)

\textbf{ગુણધર્મો}:

\begin{itemize}
\tightlist
\item
  \textbf{ભૌતિક}: રંગહીન ગેસ, મીઠી સુગંધ
\item
  \textbf{રાસાયણિક}: અસંતૃપ્ત, ઉમેરણ પ્રતિક્રિયાઓ કરે
\end{itemize}

\textbf{ઉપયોગો}:

\begin{itemize}
\tightlist
\item
  \textbf{ઔદ્યોગિક}: પોલિઇથિલીન ઉત્પાદન
\item
  \textbf{કૃષિ}: ફળ પકવવા માટે વનસ્પતિ હોર્મોન
\end{itemize}

\end{solutionbox}
\begin{mnemonicbox}
``તૈયારી ગુણધર્મો ઉપયોગો''

\end{mnemonicbox}
\subsubsection{પ્રશ્ન 3(B)(3) [4
ગુણ]}\label{uxaaauxab0uxab6uxaa8-3b3-4-uxa97uxaa3}

\textbf{રાસાયણિક સમીકરણ સાથે Buna-S રબરની બનાવટ સમજાવો. તેના બે ગુણધર્મો અને બે
ઉપયોગો પણ લખો.}

\begin{solutionbox}

\textbf{તૈયારી}: બ્યુટાડાયન + સ્ટાયરીન \rightarrow Buna-S રબર (કોપોલિમેરાઇઝેશન)

\textbf{રાસાયણિક સમીકરણ}:

\begin{verbatim}
nC_{4}H_{6} + nC_{8}H_{8} \rightarrow [-C_{4}H_{6}-C_{8}H_{8}-]_{n}
\end{verbatim}

\textbf{ગુણધર્મો}:

\begin{itemize}
\tightlist
\item
  \textbf{યાંત્રિક}: સારો ઘર્ષણ પ્રતિકાર
\item
  \textbf{રાસાયણિક}: તેલ અને ઇંધણ પ્રતિરોધી
\end{itemize}

\textbf{ઉપયોગો}:

\begin{itemize}
\tightlist
\item
  \textbf{વાહન}: ટાયર ઉત્પાદન
\item
  \textbf{ઔદ્યોગિક}: કન્વેયર બેલ્ટ, હોઝ
\end{itemize}

\end{solutionbox}
\begin{mnemonicbox}
``બ્યુટાડાયન સ્ટાયરીન મજબૂત રબર બનાવે''

\end{mnemonicbox}
\begin{center}\rule{0.5\linewidth}{0.5pt}\end{center}

\subsection*{પ્રશ્ન 4(A) [6
ગુણ]}\label{q4a}

\subsubsection{પ્રશ્ન 4(A)(1) [3
ગુણ]}\label{uxaaauxab0uxab6uxaa8-4a1-3-uxa97uxaa3}

\textbf{ધાતુઓનું ક્ષારણ નિવારવા ધાતુક્લેડિંગ સમજાવો.}

\begin{solutionbox}

\begin{longtable}[]{@{}ll@{}}
\toprule\noalign{}
પાસું & વર્ણન \\
\midrule\noalign{}
\endhead
\bottomrule\noalign{}
\endlastfoot
\textbf{પ્રક્રિયા} & મૂળ ધાતુ પર ક્ષારણ-પ્રતિરોધી ધાતુનું આવરણ \\
\textbf{પદ્ધતિઓ} & હોટ ડિપિંગ, ઇલેક્ટ્રોપ્લેટિંગ, રોલ બોન્ડિંગ \\
\textbf{ઉદાહરણો} & ગેલ્વેનાઇઝ્ડ આયર્ન (Fe પર Zn), ટીન પ્લેટિંગ \\
\end{longtable}

\textbf{મિકેનિઝમ}: સુરક્ષિત સ્તર મૂળ ધાતુને ઓક્સિજન/ભેજના સંપર્કમાં આવતું અટકાવે

\end{solutionbox}
\begin{mnemonicbox}
``આવરણ ધાતુ સુરક્ષિત કરે''

\end{mnemonicbox}
\subsubsection{પ્રશ્ન 4(A)(2) [3
ગુણ]}\label{uxaaauxab0uxab6uxaa8-4a2-3-uxa97uxaa3}

\textbf{પાણીની સપાટી નીચે થતું ક્ષારણ રાસાયણિક પ્રક્રિયાઓ અને નામનિર્દેશનવાળી
આકૃતિ સાથે સમજાવો.}

\begin{solutionbox}

\textbf{પ્રક્રિયા}: વિભેદક વાયુકરણ પાણી-હવા સંપર્ક સ્થળે ક્ષારણ કારણે

\textbf{રાસાયણિક સમીકરણો}:

\begin{itemize}
\tightlist
\item
  એનોડ: Fe \rightarrow Fe^{2}^{+} + 2e^{-}
\item
  કેથોડ: O_{2} + 4H^{+} + 4e^{-} \rightarrow 2H_{2}O
\end{itemize}

\textbf{આકૃતિ:}

\begin{verbatim}
    હવા
────────────────
    પાણી (O_{2 સમૃદ્ધ)}
    ↓ કેથોડ
────Fe──────────
    ↑ એનોડ
    પાણી (O_{2 ગરીબ)}
\end{verbatim}

\end{solutionbox}
\begin{mnemonicbox}
``પાણી હવા સંપર્ક ક્ષારણ કરે''

\end{mnemonicbox}
\subsubsection{પ્રશ્ન 4(A)(3) [3
ગુણ]}\label{uxaaauxab0uxab6uxaa8-4a3-3-uxa97uxaa3}

\textbf{સૌર કોષોના કાર્યકારી સિદ્ધાંતને સમજાવો.}

\begin{solutionbox}

\begin{longtable}[]{@{}ll@{}}
\toprule\noalign{}
ઘટક & કાર્ય \\
\midrule\noalign{}
\endhead
\bottomrule\noalign{}
\endlastfoot
\textbf{ફોટોવોલ્ટેઇક અસર} & પ્રકાશ ઊર્જા વિદ્યુત ઊર્જામાં ફેરવાય \\
\textbf{p-n જંકશન} & ચાર્જ વિભાજન માટે વિદ્યુત ક્ષેત્ર બનાવે \\
\textbf{ઇલેક્ટ્રોન-હોલ જોડી} & ફોટોન સેમિકન્ડક્ટર સાથે અથડાય ત્યારે બને \\
\end{longtable}

\textbf{પ્રક્રિયા}: પ્રકાશ \rightarrow ઇલેક્ટ્રોન ઉત્તેજના \rightarrow પ્રવાહ \rightarrow વિદ્યુત ઊર્જા

\end{solutionbox}
\begin{mnemonicbox}
``ફોટો વોલ્ટેઇક જંકશન પ્રવાહ બનાવે''

\end{mnemonicbox}
\begin{center}\rule{0.5\linewidth}{0.5pt}\end{center}

\subsection*{પ્રશ્ન 4(B) [8
ગુણ]}\label{q4b}

\subsubsection{પ્રશ્ન 4(B)(1) [4
ગુણ]}\label{uxaaauxab0uxab6uxaa8-4b1-4-uxa97uxaa3}

\textbf{આકૃતિ સાથે સીમાવર્તી સ્નેહનનું કાર્ય દર્શાવો.}

\begin{solutionbox}

\textbf{કાર્ય}: પાતળો આણવિક સ્તર ધાતુની સપાટી પર ચોંટે, સીધો સંપર્ક અટકાવે

\textbf{મિકેનિઝમ}:

\begin{itemize}
\tightlist
\item
  \textbf{રચના}: સ્નેહક અણુઓ ધાતુની સપાટી પર ગોઠવાય
\item
  \textbf{સુરક્ષા}: સપાટીઓ વચ્ચે ઘર્ષણ અને ઘસારો ઘટાડે
\item
  \textbf{લોડ બેરિંગ}: પ્રવાહી ફિલ્મ તૂટે ત્યારે લોડ સહન કરે
\end{itemize}

\textbf{આકૃતિ:}

\begin{verbatim}
ગતિશીલ સપાટી
╔═══════════════╗
║               ║
║ ∘∘∘∘∘∘∘∘∘∘∘∘∘ ║  સીમા સ્તર
║               ║
╚═══════════════╝
સ્થિર સપાટી
\end{verbatim}

\end{solutionbox}
\begin{mnemonicbox}
``સીમા અવરોધ ધાતુ સંપર્ક અટકાવે''

\end{mnemonicbox}
\subsubsection{પ્રશ્ન 4(B)(2) [4
ગુણ]}\label{uxaaauxab0uxab6uxaa8-4b2-4-uxa97uxaa3}

\textbf{રેડવુડ વિસ્કોમીટર દ્વારા સિનગ્ધતા કેવી રીતે માપવામાં આવે છે તે
નામનિર્દેશનવાળી આકૃતિ સાથે સમજાવો.}

\begin{solutionbox}

\textbf{સિદ્ધાંત}: નિશ્ચિત કદના છિદ્રમાંથી નિશ્ચિત પ્રમાણ તેલ વહેવામાં લાગતો સમય

\textbf{કાર્યવિધિ}:

\begin{itemize}
\tightlist
\item
  \textbf{સેટઅપ}: તેલ ચેમ્બર ભરો, જરૂરી તાપમાને ગરમ કરો
\item
  \textbf{માપ}: 50ml તેલ વહેવાનો સમય નોંધો
\item
  \textbf{ગણતરી}: વિસ્કોસિટી = સમય \times સ્થિરાંક
\end{itemize}

\textbf{આકૃતિ:}

\begin{verbatim}
    ┌─────────────┐
    │  તેલ બાથ     │
    │ ┌─────────┐ │
    │ │   તેલ    │ │
    │ │ચેમ્બર     │ │
    │ └────┬────┘ │
    │      │છિદ્ર    │
    │      ▼      │
    │  ┌─────┐    │
    │  │50ml │    │
    │  │ફ્લાસ્ક │    │
    │  └─────┘    │
    └─────────────┘
\end{verbatim}

\end{solutionbox}
\begin{mnemonicbox}
``રેડવુડ સમય નોંધે''

\end{mnemonicbox}
\subsubsection{પ્રશ્ન 4(B)(3) [4
ગુણ]}\label{uxaaauxab0uxab6uxaa8-4b3-4-uxa97uxaa3}

\textbf{વ્યાખ્યા આપો: અર્ધવાહક, અવાહક પદાર્થ, સ્થિતિસ્થાપક પદાર્થ, યોગશીલ
બહુલીભવન.}

\begin{solutionbox}

\begin{longtable}[]{@{}
  >{\raggedright\arraybackslash}p{(\linewidth - 2\tabcolsep) * \real{0.3571}}
  >{\raggedright\arraybackslash}p{(\linewidth - 2\tabcolsep) * \real{0.6429}}@{}}
\toprule\noalign{}
\begin{minipage}[b]{\linewidth}\raggedright
શબ્દ
\end{minipage} & \begin{minipage}[b]{\linewidth}\raggedright
વ્યાખ્યા
\end{minipage} \\
\midrule\noalign{}
\endhead
\bottomrule\noalign{}
\endlastfoot
\textbf{અર્ધવાહક} & વાહક અને અવાહક વચ્ચેની વિદ્યુત વાહકતા ધરાવતો પદાર્થ \\
\textbf{અવાહક પદાર્થ} & વિદ્યુત પ્રવાહના વહેણને પ્રતિકાર કરતો પદાર્થ \\
\textbf{સ્થિતિસ્થાપક પદાર્થ} & લવચીક ગુણધર્મો ધરાવતો પોલિમર, ખેંચાઈને મૂળ આકારે
પાછો આવે \\
\textbf{યોગશીલ બહુલીભવન} & મોનોમર્સ નાના અણુઓ દૂર કર્યા વિના જોડાય \\
\end{longtable}

\textbf{ઉદાહરણો}: Si (અર્ધવાહક), રબર (અવાહક), રબર (સ્થિતિસ્થાપક),
પોલિઇથિલીન (યોગશીલ)

\end{solutionbox}
\begin{mnemonicbox}
``અર્ધ અવાહક સ્થિતિ યોગશીલ''

\end{mnemonicbox}
\begin{center}\rule{0.5\linewidth}{0.5pt}\end{center}

\subsection*{પ્રશ્ન 5(A) [6
ગુણ]}\label{q5a}

\subsubsection{પ્રશ્ન 5(A)(1) [3
ગુણ]}\label{uxaaauxab0uxab6uxaa8-5a1-3-uxa97uxaa3}

\textbf{ઉકેલો: 0.004 M HClના જલીય દ્રાવણની pH અને pOH ગણો. (log 4 =
0.6021)}

\begin{solutionbox}

\textbf{આપેલ}: [HCl] = 0.004 M = 4 \times 10^{-}^{3} M

\textbf{ઉકેલ}:

\begin{itemize}
\tightlist
\item
  HCl મજબૂત એસિડ છે, સંપૂર્ણ આયનીકરણ થાય
\item
  [H^{+}] = [HCl] = 4 \times 10^{-}^{3} M
\item
  pH = -log[H^{+}] = -log(4 \times 10^{-}^{3})
\item
  pH = -log 4 - log 10^{-}^{3} = -0.6021 + 3 = 2.398
\item
  pOH = 14 - pH = 14 - 2.398 = 11.602
\end{itemize}

\end{solutionbox}
\begin{solutionbox}
pH = 2.40, pOH = 11.60

\end{solutionbox}
\begin{mnemonicbox}
``મજબૂત એસિડ, સરળ ગણતરી''

\end{mnemonicbox}
\subsubsection{પ્રશ્ન 5(A)(2) [3
ગુણ]}\label{uxaaauxab0uxab6uxaa8-5a2-3-uxa97uxaa3}

\textbf{ઉદાહરણ સાથે બાહ્ય અર્ધવાહકો અને તેના પ્રકારો વર્ણવો.}

\begin{solutionbox}

\begin{longtable}[]{@{}llll@{}}
\toprule\noalign{}
પ્રકાર & ડોપન્ટ & મુખ્ય વાહકો & ઉદાહરણ \\
\midrule\noalign{}
\endhead
\bottomrule\noalign{}
\endlastfoot
\textbf{n-પ્રકાર} & દાતા અણુઓ (ગ્રૂપ V) & ઇલેક્ટ્રોન & Si + P \\
\textbf{p-પ્રકાર} & સ્વીકર્તા અણુઓ (ગ્રૂપ III) & હોલ્સ & Si + B \\
\end{longtable}

\textbf{ગુણધર્મો}:

\begin{itemize}
\tightlist
\item
  \textbf{n-પ્રકાર}: વધારાના ઇલેક્ટ્રોન વાહકતા વધારે
\item
  \textbf{p-પ્રકાર}: ઇલેક્ટ્રોન અછત સકારાત્મક હોલ્સ બનાવે
\end{itemize}

\end{solutionbox}
\begin{mnemonicbox}
``n-નેગેટિવ ઇલેક્ટ્રોન, p-પોઝિટિવ હોલ્સ''

\end{mnemonicbox}
\subsubsection{પ્રશ્ન 5(A)(3) [3
ગુણ]}\label{uxaaauxab0uxab6uxaa8-5a3-3-uxa97uxaa3}

\textbf{ઉષ્માસહ બહુલક અને ઉષ્માસ્થાપિત બહુલક વચ્ચેનાં ફરક આપો. (દરેકનાં ચાર મુદ્દાઓ)}

\begin{solutionbox}

\begin{longtable}[]{@{}lll@{}}
\toprule\noalign{}
ગુણધર્મ & ઉષ્માસહ & ઉષ્માસ્થાપિત \\
\midrule\noalign{}
\endhead
\bottomrule\noalign{}
\endlastfoot
\textbf{રચના} & રેખીય/શાખાવાળી સાંકળો & ક્રોસ-લિંક્ડ નેટવર્ક \\
\textbf{ગરમીની અસર} & ગરમ કરવાથી નરમ પડે & નરમ નથી પડતું \\
\textbf{પુનઃઉપયોગ} & પુનઃઉપયોગ શક્ય & પુનઃઉપયોગ અશક્ય \\
\textbf{ઉદાહરણો} & PVC, PE, PS & બેકેલાઇટ, ઇપોક્સી \\
\end{longtable}

\end{solutionbox}
\begin{mnemonicbox}
``ઉષ્મા-સહ = પુનઃઉપયોગ, ઉષ્મા-સ્થાપિત = કાયમી''

\end{mnemonicbox}
\begin{center}\rule{0.5\linewidth}{0.5pt}\end{center}

\subsection*{પ્રશ્ન 5(B) [8
ગુણ]}\label{q5b}

\subsubsection{પ્રશ્ન 5(B)(1) [4
ગુણ]}\label{uxaaauxab0uxab6uxaa8-5b1-4-uxa97uxaa3}

\textbf{હાઇડ્રોજન બંધ અને તેના પ્રકારો ઉદાહરણો સાથે વર્ણવો.}

\begin{solutionbox}

\textbf{વ્યાખ્યા}: હાઇડ્રોજન અને વિદ્યુતનેગેટિવ અણુઓ વચ્ચે નબળું વિદ્યુતસ્થિતિક આકર્ષણ

\textbf{પ્રકારો}:

\begin{longtable}[]{@{}lll@{}}
\toprule\noalign{}
પ્રકાર & વર્ણન & ઉદાહરણ \\
\midrule\noalign{}
\endhead
\bottomrule\noalign{}
\endlastfoot
\textbf{અંતરઅણવિક} & વિવિધ અણુઓ વચ્ચે & H_{2}O···H_{2}O \\
\textbf{અંતઃઅણવિક} & સમાન અણુમાં & o-નાઇટ્રોફિનોલ \\
\end{longtable}

\textbf{લક્ષણો}:

\begin{itemize}
\tightlist
\item
  \textbf{તાકાત}: 5-40 kJ/mol
\item
  \textbf{જરૂરિયાતો}: H, F, O, N સાથે જોડાયેલ
\end{itemize}

\textbf{આકૃતિ:}

\begin{verbatim}
H_{2O···H_{2}O···H_{2}O}
 │     │     │
δ^{+    δ^{-}    δ^{+}}
\end{verbatim}

\end{solutionbox}
\begin{mnemonicbox}
``હાઇડ્રોજનને FON મિત્રોની જરૂર'' (ફ્લોરિન, ઓક્સિજન,
નાઇટ્રોજન)

\end{mnemonicbox}
\subsubsection{પ્રશ્ન 5(B)(2) [4
ગુણ]}\label{uxaaauxab0uxab6uxaa8-5b2-4-uxa97uxaa3}

\textbf{પ્રાથમિક કોષ અને દ્વિતીયક કોષ વચ્ચે તફાવત કરો. (ચાર મુદ્દાઓ)}

\begin{solutionbox}

\begin{longtable}[]{@{}lll@{}}
\toprule\noalign{}
પાસું & પ્રાથમિક કોષ & દ્વિતીયક કોષ \\
\midrule\noalign{}
\endhead
\bottomrule\noalign{}
\endlastfoot
\textbf{રિચાર્જેબિલિટી} & રિચાર્જ ન થાય & રિચાર્જ થાય \\
\textbf{પ્રતિક્રિયા} & અપરિવર્તનીય & પરિવર્તનીય \\
\textbf{કિંમત} & ઓછી શરૂઆતી કિંમત & વધુ શરૂઆતી કિંમત \\
\textbf{ઉદાહરણો} & ડ્રાય સેલ, અલ્કલાઇન & લેડ-એસિડ, Li-ion \\
\end{longtable}

\textbf{ઉપયોગો}:

\begin{itemize}
\tightlist
\item
  \textbf{પ્રાથમિક}: રિમોટ કંટ્રોલ, ફ્લેશલાઇટ
\item
  \textbf{દ્વિતીયક}: કાર, ફોન, લેપટોપ
\end{itemize}

\end{solutionbox}
\begin{mnemonicbox}
``પ્રાથમિક = કાયમી, દ્વિતીયક = પરિવર્તનીય''

\end{mnemonicbox}
\subsubsection{પ્રશ્ન 5(B)(3) [4
ગુણ]}\label{uxaaauxab0uxab6uxaa8-5b3-4-uxa97uxaa3}

\textbf{નામનિર્દેશવાળી આકૃતિ દોરી લેડ-એસિડ સંગ્રાહક કોષની રચના, કાર્ય અને
રાસાયણિક સમીકરણો વર્ણવો.}

\begin{solutionbox}

\textbf{રચના}:

\begin{itemize}
\tightlist
\item
  \textbf{એનોડ}: લેડ (Pb)
\item
  \textbf{કેથોડ}: લેડ ડાયઓક્સાઇડ (PbO_{2})
\item
  \textbf{ઇલેક્ટ્રોલાઇટ}: પાતળું H_{2}SO_{4}
\end{itemize}

\textbf{રાસાયણિક સમીકરણો}:

\begin{itemize}
\tightlist
\item
  \textbf{ડિસ્ચાર્જ}: Pb + PbO_{2} + 2H_{2}SO_{4} \rightarrow 2PbSO_{4} + 2H_{2}O
\item
  \textbf{ચાર્જ}: 2PbSO_{4} + 2H_{2}O \rightarrow Pb + PbO_{2} + 2H_{2}SO_{4}
\end{itemize}

\textbf{આકૃતિ}:

\begin{verbatim}
    ┌─────────────────┐
    │   Pb  │  PbO_{2   │}
    │ ({-)   │   (+)   │}
    │એનોડ   │ કેથોડ     │
    │   │   │   │     │
    │ ┌─────────────┐ │
    │ │   H_{2SO_{4}     │ │}
    │ │ ઇલેક્ટ્રોલાઇટ    │ │
    │ └─────────────┘ │
    └─────────────────┘
\end{verbatim}

\textbf{કાર્ય}: ડિસ્ચાર્જ દરમિયાન રાસાયણિક ઊર્જા વિદ્યુત ઊર્જામાં ફેરવાય

\end{solutionbox}
\begin{mnemonicbox}
``લેડ એસિડ સ્ટોરેજ = પરિવર્તનીય ઊર્જા''

\end{mnemonicbox}

\end{document}
