\documentclass[10pt,a4paper]{article}

% content/resources/templates/preamble.tex
\usepackage[margin=0.6in]{geometry}
\author{Milav Dabgar}
\usepackage{amsmath,amssymb,amsthm}
\usepackage{booktabs}
\usepackage{multirow}
\usepackage{xcolor}
\usepackage{tcolorbox}
\tcbuselibrary{breakable,skins}
\usepackage[colorlinks=true,linkcolor=blue]{hyperref}
\usepackage{titlesec}
\usepackage{enumitem}
\usepackage{tikz}
\usepackage{pgfplots}
\usepackage{circuitikz}
\usepackage[version=4]{mhchem}
\usepackage{longtable}
\usepackage{array}
\usepackage{float}
\usepackage{caption}
\usepackage{listings}

\lstset{
  basicstyle=\small\ttfamily,
  breaklines=true,
  breakatwhitespace=false,
  postbreak=\mbox{\textcolor{red}{$\hookrightarrow$}\space},
  float=false,
  numbers=left,
  numberstyle=\tiny\color{gray},
  numbersep=10pt,
  xleftmargin=2em,
  keywordstyle=\color{blue},
  commentstyle=\color{green!60!black},
  stringstyle=\color{purple},
  backgroundcolor=\color{gray!5},
  showstringspaces=false,
  tabsize=2,
  captionpos=b,
  keepspaces=true,
  columns=flexible
}

\pgfplotsset{compat=1.18}
\usetikzlibrary{shapes,arrows,positioning,calc,patterns,decorations.pathmorphing,decorations.markings,arrows.meta}

% Color scheme
\definecolor{headcolor}{RGB}{0,102,204}
\definecolor{keycolor}{RGB}{220,20,60}
\definecolor{solutioncolor}{RGB}{34,139,34}
\definecolor{mnemoniccolor}{RGB}{148,0,211}
\definecolor{codecolor}{RGB}{0,0,100}

% Spacing
\setlength{\parskip}{3pt}
\setlist[itemize]{nosep}
\setlist[enumerate]{nosep}

% Title formatting
\titleformat{\section}{\Large\bfseries\color{headcolor}}{\thesection}{1em}{}
\titleformat{\subsection}{\large\bfseries\color{headcolor}}{\thesubsection}{1em}{}

% Pandoc tightlist compatibility
\providecommand{\tightlist}{%
  \setlength{\itemsep}{0pt}\setlength{\parskip}{0pt}}

% Pandoc longtable compatibility
\newcounter{none}
\def\thenone{}


% content/resources/templates/english-boxes.tex
% This file is currently empty - it exists to maintain consistency with the import structure.
% Add custom environments here if needed in the future.


\begin{document}

\begin{center}
{\Huge\bfseries\color{headcolor} Chemistry Solutions}\\[5pt]
{\LARGE DI01000071 -- Winter 2024}\\[3pt]
{\large Semester 1 Study Material}\\[3pt]
{\normalsize\textit{Detailed Solutions and Explanations}}
\end{center}

\vspace{10pt}

\subsection*{Question 1 [14 marks]}\label{question-1-14-marks}

\textbf{Fill in the blanks using appropriate choice from the given
options:}

\begin{solutionbox}

\begin{longtable}[]{@{}lll@{}}
\toprule\noalign{}
Question & Answer & Explanation \\
\midrule\noalign{}
\endhead
\bottomrule\noalign{}
\endlastfoot
(1) & [Ar]4s^{1}3d^{1}^{0} & Cu has 29 electrons, exception to Aufbau rule \\
(2) & 14 & pH + pOH = 14 at 25^\circC \\
(3) & cathode & Pure copper deposits at negative electrode \\
(4) & Cu & Copper forms protective oxide layer \\
(5) & semi-solid & Peat is partially decomposed organic matter \\
(6) & Dulong & Dulong's formula calculates calorific value \\
(7) & Lignite & Lignite has highest moisture (35-75\%) \\
(8) & Poise & SI unit of dynamic viscosity \\
(9) & High & High flash point prevents ignition \\
(10) & Emulsion & Oil-water mixture forms emulsion \\
(11) & Bakelite & Phenol formaldehyde = Bakelite \\
(12) & S & Sulfur used for vulcanization \\
(13) & PHBV & PHBV is biodegradable polymer \\
(14) & volt & EMF measured in volts \\
\end{longtable}

\end{solutionbox}
\begin{mnemonicbox}
``Chemical Copper Creates Beautiful Properties'' (for
remembering key concepts)

\end{mnemonicbox}
\begin{center}\rule{0.5\linewidth}{0.5pt}\end{center}

\subsection*{Question 2(A) [6 marks]}\label{q2a}

\subsubsection{Question 2(A)(1) [3
marks]}\label{question-2a1-3-marks}

\textbf{List the three importance of pH in various fields.}

\begin{solutionbox}

\begin{longtable}[]{@{}
  >{\raggedright\arraybackslash}p{(\linewidth - 4\tabcolsep) * \real{0.2188}}
  >{\raggedright\arraybackslash}p{(\linewidth - 4\tabcolsep) * \real{0.3750}}
  >{\raggedright\arraybackslash}p{(\linewidth - 4\tabcolsep) * \real{0.4062}}@{}}
\toprule\noalign{}
\begin{minipage}[b]{\linewidth}\raggedright
Field
\end{minipage} & \begin{minipage}[b]{\linewidth}\raggedright
Importance
\end{minipage} & \begin{minipage}[b]{\linewidth}\raggedright
Application
\end{minipage} \\
\midrule\noalign{}
\endhead
\bottomrule\noalign{}
\endlastfoot
\textbf{Medicine} & Blood pH maintenance & Normal pH 7.35-7.45 for
proper body function \\
\textbf{Agriculture} & Soil pH optimization & pH 6-7 ideal for crop
growth and nutrient absorption \\
\textbf{Industry} & Quality control & pH affects product quality in
food, textiles, pharmaceuticals \\
\end{longtable}

\end{solutionbox}
\begin{mnemonicbox}
``Medical Agriculture Industry'' (MAI)

\end{mnemonicbox}
\subsubsection{Question 2(A)(2) [3
marks]}\label{question-2a2-3-marks}

\textbf{Define: Buffer solutions, Half-cell, Faraday's first law of
electrolysis.}

\begin{solutionbox}

\begin{itemize}
\tightlist
\item
  \textbf{Buffer solutions}: Solutions that resist changes in pH when
  small amounts of acid or base are added
\item
  \textbf{Half-cell}: Single electrode immersed in its ionic solution,
  represents oxidation or reduction reaction
\item
  \textbf{Faraday's first law}: Amount of substance deposited/liberated
  at electrode is directly proportional to quantity of electricity
  passed
\end{itemize}

\end{solutionbox}
\begin{mnemonicbox}
``Buffers Help Faraday'' (BHF)

\end{mnemonicbox}
\subsubsection{Question 2(A)(3) [3
marks]}\label{question-2a3-3-marks}

\textbf{State the factors affecting the rate of corrosion.}

\begin{solutionbox}

\begin{longtable}[]{@{}
  >{\raggedright\arraybackslash}p{(\linewidth - 4\tabcolsep) * \real{0.2759}}
  >{\raggedright\arraybackslash}p{(\linewidth - 4\tabcolsep) * \real{0.2759}}
  >{\raggedright\arraybackslash}p{(\linewidth - 4\tabcolsep) * \real{0.4483}}@{}}
\toprule\noalign{}
\begin{minipage}[b]{\linewidth}\raggedright
Factor
\end{minipage} & \begin{minipage}[b]{\linewidth}\raggedright
Effect
\end{minipage} & \begin{minipage}[b]{\linewidth}\raggedright
Description
\end{minipage} \\
\midrule\noalign{}
\endhead
\bottomrule\noalign{}
\endlastfoot
\textbf{Metal purity} & Higher purity = Less corrosion & Impurities
create galvanic cells \\
\textbf{Temperature} & Higher temp = Faster corrosion & Increases
reaction rate \\
\textbf{Humidity} & Higher humidity = More corrosion & Promotes
electrochemical reactions \\
\end{longtable}

\end{solutionbox}
\begin{mnemonicbox}
``Pure Temperature Humidity'' (PTH)

\end{mnemonicbox}
\begin{center}\rule{0.5\linewidth}{0.5pt}\end{center}

\subsection*{Question 2(B) [8 marks]}\label{q2b}

\subsubsection{Question 2(B)(1) [4
marks]}\label{question-2b1-4-marks}

\textbf{Compare between orbits and orbitals (four points each).}

\begin{solutionbox}

\begin{longtable}[]{@{}lll@{}}
\toprule\noalign{}
Aspect & Orbits & Orbitals \\
\midrule\noalign{}
\endhead
\bottomrule\noalign{}
\endlastfoot
\textbf{Definition} & Fixed circular paths & 3D probability regions \\
\textbf{Shape} & Circular/elliptical & s,p,d,f shapes \\
\textbf{Energy} & Definite energy levels & Energy ranges \\
\textbf{Electron location} & Exact position & Probability of finding \\
\end{longtable}

\textbf{Diagram:}

\begin{verbatim}
    Orbits (Bohr Model)          Orbitals (Quantum Model)
    
         e{-                           ∴∴∴∴∴}
      ○ ──── ○                       ∴∴∴∴∴∴∴
    ○         ○                    ∴∴∴∴∴∴∴∴∴
      ○ ──── ○                       ∴∴∴∴∴
         +                           ∴∴∴∴∴
\end{verbatim}

\end{solutionbox}
\begin{mnemonicbox}
``Definite Shape Energy Location'' (DSEL)

\end{mnemonicbox}
\subsubsection{Question 2(B)(2) [4
marks]}\label{question-2b2-4-marks}

\textbf{Classify fuels on the basis of its sources and physical states
with one example of each.}

\begin{solutionbox}


\begin{longtable}[]{@{}llll@{}}
\toprule\noalign{}
Classification & Type & Example & Description \\
\midrule\noalign{}
\endhead
\bottomrule\noalign{}
\endlastfoot
\textbf{Source-based} & Natural & Coal & Formed naturally \\
& Artificial & Petrol & Man-made \\
\textbf{Physical state} & Solid & Wood & Solid at room temp \\
& Liquid & Diesel & Liquid at room temp \\
& Gaseous & LPG & Gas at room temp \\
\end{longtable}

\end{solutionbox}
\begin{mnemonicbox}
``Natural Artificial, Solid Liquid Gas'' (NASLG)

\end{mnemonicbox}
\subsubsection{Question 2(B)(3) [4
marks]}\label{question-2b3-4-marks}

\textbf{Explain bio-diesel with four important points.}

\begin{solutionbox}

\begin{itemize}
\tightlist
\item
  \textbf{Source}: Made from vegetable oils, animal fats, or waste
  cooking oil
\item
  \textbf{Process}: Produced by transesterification reaction with
  methanol/ethanol
\item
  \textbf{Properties}: Biodegradable, non-toxic, renewable fuel source
\item
  \textbf{Applications}: Used in diesel engines, reduces emissions by
  75\%
\end{itemize}

\textbf{Chemical Reaction:}

\begin{verbatim}
Vegetable Oil + Methanol \rightarrow Bio-diesel + Glycerol
\end{verbatim}

\end{solutionbox}
\begin{mnemonicbox}
``Source Process Properties Applications'' (SPPA)

\end{mnemonicbox}
\begin{center}\rule{0.5\linewidth}{0.5pt}\end{center}

\subsection*{Question 3(A) [6 marks]}\label{q3a}

\subsubsection{Question 3(A)(1) [3
marks]}\label{question-3a1-3-marks}

\textbf{Explain solute, solvent and solution with the help of example.}

\begin{solutionbox}

\begin{longtable}[]{@{}lll@{}}
\toprule\noalign{}
Component & Definition & Example \\
\midrule\noalign{}
\endhead
\bottomrule\noalign{}
\endlastfoot
\textbf{Solute} & Substance being dissolved & Salt (NaCl) \\
\textbf{Solvent} & Substance doing the dissolving & Water (H_{2}O) \\
\textbf{Solution} & Homogeneous mixture & Salt water \\
\end{longtable}

\textbf{Example}: Sugar + Water = Sugar solution

\begin{itemize}
\tightlist
\item
  Sugar = Solute, Water = Solvent, Sugar water = Solution
\end{itemize}

\end{solutionbox}
\begin{mnemonicbox}
``Solute Solvent Solution'' (SSS)

\end{mnemonicbox}
\subsubsection{Question 3(A)(2) [3
marks]}\label{question-3a2-3-marks}

\textbf{Explain the formation of Electrovalent bond in NaCl.}

\begin{solutionbox}

\textbf{Process}:

\begin{itemize}
\tightlist
\item
  \textbf{Step 1}: Na loses 1 electron \rightarrow Na^{+} (cation)
\item
  \textbf{Step 2}: Cl gains 1 electron \rightarrow Cl^{-} (anion)
\item
  \textbf{Step 3}: Electrostatic attraction between Na^{+} and Cl^{-}
\end{itemize}

\textbf{Diagram:}

\begin{verbatim}
Na        Na^{+  +  e^{-}}
Cl + e^{-      Cl^{-}}

Na^{+  +  Cl^{-}    Na^{+}Cl^{-} (NaCl)}
\end{verbatim}

\end{solutionbox}
\begin{mnemonicbox}
``Sodium Loses, Chlorine Gains, Attraction Forms''
(SLCGAF)

\end{mnemonicbox}
\subsubsection{Question 3(A)(3) [3
marks]}\label{question-3a3-3-marks}

\textbf{Explain Octane number for gasoline.}

\begin{solutionbox}

\begin{longtable}[]{@{}ll@{}}
\toprule\noalign{}
Aspect & Description \\
\midrule\noalign{}
\endhead
\bottomrule\noalign{}
\endlastfoot
\textbf{Definition} & Measure of fuel's resistance to knocking \\
\textbf{Scale} & 0-100, higher = better anti-knock properties \\
\textbf{Standard} & n-heptane = 0, iso-octane = 100 \\
\end{longtable}

\textbf{Applications}: High octane fuel prevents engine knocking,
improves performance

\end{solutionbox}
\begin{mnemonicbox}
``Octane Opposes Knocking'' (OOK)

\end{mnemonicbox}
\begin{center}\rule{0.5\linewidth}{0.5pt}\end{center}

\subsection*{Question 3(B) [8 marks]}\label{q3b}

\subsubsection{Question 3(B)(1) [4
marks]}\label{question-3b1-4-marks}

\textbf{Explain electrorefining of impure Cu with chemical equations and
a labeled diagram.}

\begin{solutionbox}

\textbf{Process}:

\begin{itemize}
\tightlist
\item
  \textbf{Anode}: Impure copper dissolves
\item
  \textbf{Cathode}: Pure copper deposits
\item
  \textbf{Electrolyte}: CuSO_{4} solution
\end{itemize}

\textbf{Chemical Equations}:

\begin{itemize}
\tightlist
\item
  At Anode: Cu \rightarrow Cu^{2}^{+} + 2e^{-}
\item
  At Cathode: Cu^{2}^{+} + 2e^{-} \rightarrow Cu
\end{itemize}

\textbf{Diagram:}

\begin{verbatim}
    Battery
     ╔═╗
   {- ║ ║ +}
     ╚═╝
     │ │
  Cathode │ Anode
  (Pure Cu)│(Impure Cu)
     │ │
    ╔═══════╗
    ║CuSO_{4  ║}
    ║Solution║
    ╚═══════╝
\end{verbatim}

\end{solutionbox}
\begin{mnemonicbox}
``Anode Dissolves, Cathode Deposits'' (ADCD)

\end{mnemonicbox}
\subsubsection{Question 3(B)(2) [4
marks]}\label{question-3b2-4-marks}

\textbf{Explain preparation of ethene with chemical equation. Also write
its two properties and two uses.}

\begin{solutionbox}

\textbf{Preparation}: C_{2}H_{5}OH \rightarrow C_{2}H_{4} + H_{2}O (Dehydration of ethanol)

\textbf{Properties}:

\begin{itemize}
\tightlist
\item
  \textbf{Physical}: Colorless gas, sweet smell
\item
  \textbf{Chemical}: Unsaturated, undergoes addition reactions
\end{itemize}

\textbf{Uses}:

\begin{itemize}
\tightlist
\item
  \textbf{Industrial}: Manufacturing polyethylene
\item
  \textbf{Agricultural}: Plant hormone for fruit ripening
\end{itemize}

\end{solutionbox}
\begin{mnemonicbox}
``Preparation Properties Uses'' (PPU)

\end{mnemonicbox}
\subsubsection{Question 3(B)(3) [4
marks]}\label{question-3b3-4-marks}

\textbf{Explain preparation of Buna-S rubber with chemical equation.
Also write its two properties and two uses.}

\begin{solutionbox}

\textbf{Preparation}: Butadiene + Styrene \rightarrow Buna-S rubber
(Copolymerization)

\textbf{Chemical Equation}:

\begin{verbatim}
nC_{4}H_{6} + nC_{8}H_{8} \rightarrow [-C_{4}H_{6}-C_{8}H_{8}-]_{n}
\end{verbatim}

\textbf{Properties}:

\begin{itemize}
\tightlist
\item
  \textbf{Mechanical}: Good abrasion resistance
\item
  \textbf{Chemical}: Oil and fuel resistant
\end{itemize}

\textbf{Uses}:

\begin{itemize}
\tightlist
\item
  \textbf{Automotive}: Tire manufacturing
\item
  \textbf{Industrial}: Conveyor belts, hoses
\end{itemize}

\end{solutionbox}
\begin{mnemonicbox}
``Butadiene Styrene Makes Strong Rubber'' (BSMSR)

\end{mnemonicbox}
\begin{center}\rule{0.5\linewidth}{0.5pt}\end{center}

\subsection*{Question 4(A) [6 marks]}\label{q4a}

\subsubsection{Question 4(A)(1) [3
marks]}\label{question-4a1-3-marks}

\textbf{Explain metal clading for the prevention of corrosion of
metals.}

\begin{solutionbox}

\begin{longtable}[]{@{}ll@{}}
\toprule\noalign{}
Aspect & Description \\
\midrule\noalign{}
\endhead
\bottomrule\noalign{}
\endlastfoot
\textbf{Process} & Coating base metal with corrosion-resistant metal \\
\textbf{Methods} & Hot dipping, electroplating, roll bonding \\
\textbf{Examples} & Galvanized iron (Zn on Fe), Tin plating \\
\end{longtable}

\textbf{Mechanism}: Protective layer prevents oxygen/moisture contact
with base metal

\end{solutionbox}
\begin{mnemonicbox}
``Coating Protects Metal'' (CPM)

\end{mnemonicbox}
\subsubsection{Question 4(A)(2) [3
marks]}\label{question-4a2-3-marks}

\textbf{Explain waterline corrosion with chemical equations and labeled
diagram.}

\begin{solutionbox}

\textbf{Process}: Differential aeration causes corrosion at water-air
interface

\textbf{Chemical Equations}:

\begin{itemize}
\tightlist
\item
  Anode: Fe \rightarrow Fe^{2}^{+} + 2e^{-}
\item
  Cathode: O_{2} + 4H^{+} + 4e^{-} \rightarrow 2H_{2}O
\end{itemize}

\textbf{Diagram:}

\begin{verbatim}
    Air
────────────────
    Water (O_{2 rich)}
    ↓ Cathode
────Fe──────────
    ↑ Anode
    Water (O_{2 poor)}
\end{verbatim}

\end{solutionbox}
\begin{mnemonicbox}
``Water Air Interface Corrodes'' (WAIC)

\end{mnemonicbox}
\subsubsection{Question 4(A)(3) [3
marks]}\label{question-4a3-3-marks}

\textbf{Explain the working principle of solar cells.}

\begin{solutionbox}

\begin{longtable}[]{@{}ll@{}}
\toprule\noalign{}
Component & Function \\
\midrule\noalign{}
\endhead
\bottomrule\noalign{}
\endlastfoot
\textbf{Photovoltaic effect} & Light energy converts to electrical
energy \\
\textbf{p-n junction} & Creates electric field for charge separation \\
\textbf{Electron-hole pairs} & Generated when photons hit
semiconductor \\
\end{longtable}

\textbf{Process}: Light \rightarrow Electron excitation \rightarrow Current flow \rightarrow
Electrical energy

\end{solutionbox}
\begin{mnemonicbox}
``Photo Voltaic Junction Creates Current'' (PVJCC)

\end{mnemonicbox}
\begin{center}\rule{0.5\linewidth}{0.5pt}\end{center}

\subsection*{Question 4(B) [8 marks]}\label{q4b}

\subsubsection{Question 4(B)(1) [4
marks]}\label{question-4b1-4-marks}

\textbf{Demonstrate the function of boundary lubrication with diagram.}

\begin{solutionbox}

\textbf{Function}: Thin molecular layer adheres to metal surfaces,
prevents direct contact

\textbf{Mechanism}:

\begin{itemize}
\tightlist
\item
  \textbf{Formation}: Lubricant molecules orient on metal surface
\item
  \textbf{Protection}: Reduces friction and wear between surfaces
\item
  \textbf{Load bearing}: Supports load when fluid film breaks down
\end{itemize}

\textbf{Diagram:}

\begin{verbatim}
Moving Surface
╔═══════════════╗
║               ║
║ ∘∘∘∘∘∘∘∘∘∘∘∘∘ ║  Boundary layer
║               ║
╚═══════════════╝
Stationary Surface
\end{verbatim}

\end{solutionbox}
\begin{mnemonicbox}
``Boundary Barriers Prevent Metal Contact'' (BBPMC)

\end{mnemonicbox}
\subsubsection{Question 4(B)(2) [4
marks]}\label{question-4b2-4-marks}

\textbf{Explain how viscosity is measured through redwood viscometer
with labelled diagram.}

\begin{solutionbox}

\textbf{Principle}: Time taken for fixed volume of oil to flow through
standard orifice

\textbf{Procedure}:

\begin{itemize}
\tightlist
\item
  \textbf{Setup}: Fill oil chamber, heat to required temperature
\item
  \textbf{Measurement}: Record time for 50ml oil flow
\item
  \textbf{Calculation}: Viscosity = Time \times Constant
\end{itemize}

\textbf{Diagram:}

\begin{verbatim}
    ┌──────────────┐
    │  Oil Bath    │
    │ ┌─────────┐  │
    │ │   Oil   │  │
    │ │Chamber  │  │
    │ └────┬────┘  │
    │      │Orifice│
    │      ▼       │
    │  ┌─────┐     │
    │  │50ml │     │
    │  │Flask│     │
    │  └─────┘     │
    └──────────────┘
\end{verbatim}

\end{solutionbox}
\begin{mnemonicbox}
``Redwood Records Time'' (RRT)

\end{mnemonicbox}
\subsubsection{Question 4(B)(3) [4
marks]}\label{question-4b3-4-marks}

\textbf{Define: Semiconductor, Insulating material, Elastomer, Addition
polymerization.}

\begin{solutionbox}

\begin{longtable}[]{@{}
  >{\raggedright\arraybackslash}p{(\linewidth - 2\tabcolsep) * \real{0.3333}}
  >{\raggedright\arraybackslash}p{(\linewidth - 2\tabcolsep) * \real{0.6667}}@{}}
\toprule\noalign{}
\begin{minipage}[b]{\linewidth}\raggedright
Term
\end{minipage} & \begin{minipage}[b]{\linewidth}\raggedright
Definition
\end{minipage} \\
\midrule\noalign{}
\endhead
\bottomrule\noalign{}
\endlastfoot
\textbf{Semiconductor} & Material with electrical conductivity between
conductor and insulator \\
\textbf{Insulating material} & Material that resists flow of electric
current \\
\textbf{Elastomer} & Polymer with elastic properties, can stretch and
return to original shape \\
\textbf{Addition polymerization} & Monomers join without elimination of
small molecules \\
\end{longtable}

\textbf{Examples}: Si (semiconductor), Rubber (insulator), Rubber
(elastomer), Polyethylene (addition)

\end{solutionbox}
\begin{mnemonicbox}
``Semi Insulating Elastic Addition'' (SIEA)

\end{mnemonicbox}
\begin{center}\rule{0.5\linewidth}{0.5pt}\end{center}

\subsection*{Question 5(A) [6 marks]}\label{q5a}

\subsubsection{Question 5(A)(1) [3
marks]}\label{question-5a1-3-marks}

\textbf{Solve: Calculate the pH and pOH of 0.004 M HCl aqueous solution.
(log 4 = 0.6021)}

\begin{solutionbox}

\textbf{Given}: [HCl] = 0.004 M = 4 \times 10^{-}^{3} M

\textbf{Solution}:

\begin{itemize}
\tightlist
\item
  HCl is strong acid, completely ionizes
\item
  [H^{+}] = [HCl] = 4 \times 10^{-}^{3} M
\item
  pH = -log[H^{+}] = -log(4 \times 10^{-}^{3})
\item
  pH = -log 4 - log 10^{-}^{3} = -0.6021 + 3 = 2.398
\item
  pOH = 14 - pH = 14 - 2.398 = 11.602
\end{itemize}

\end{solutionbox}
\begin{solutionbox}
pH = 2.40, pOH = 11.60

\end{solutionbox}
\begin{mnemonicbox}
``Strong Acid, Simple Calculation'' (SASC)

\end{mnemonicbox}
\subsubsection{Question 5(A)(2) [3
marks]}\label{question-5a2-3-marks}

\textbf{Describe extrinsic semiconductors and it types with examples.}

\begin{solutionbox}

\begin{longtable}[]{@{}llll@{}}
\toprule\noalign{}
Type & Dopant & Majority Carriers & Example \\
\midrule\noalign{}
\endhead
\bottomrule\noalign{}
\endlastfoot
\textbf{n-type} & Donor atoms (Group V) & Electrons & Si + P \\
\textbf{p-type} & Acceptor atoms (Group III) & Holes & Si + B \\
\end{longtable}

\textbf{Properties}:

\begin{itemize}
\tightlist
\item
  \textbf{n-type}: Extra electrons increase conductivity
\item
  \textbf{p-type}: Electron deficiency creates positive holes
\end{itemize}

\end{solutionbox}
\begin{mnemonicbox}
``n-negative electrons, p-positive holes'' (nnep)

\end{mnemonicbox}
\subsubsection{Question 5(A)(3) [3
marks]}\label{question-5a3-3-marks}

\textbf{Distinguish between thermoplastic polymers and thermosetting
polymer (Four points of each)}

\begin{solutionbox}

\begin{longtable}[]{@{}lll@{}}
\toprule\noalign{}
Property & Thermoplastic & Thermosetting \\
\midrule\noalign{}
\endhead
\bottomrule\noalign{}
\endlastfoot
\textbf{Structure} & Linear/branched chains & Cross-linked network \\
\textbf{Heat effect} & Softens on heating & Does not soften \\
\textbf{Reversibility} & Reversible process & Irreversible process \\
\textbf{Examples} & PVC, PE, PS & Bakelite, Epoxy \\
\end{longtable}

\end{solutionbox}
\begin{mnemonicbox}
``Thermo-plastic = Reversible, Thermo-setting =
Permanent'' (TPRTSP)

\end{mnemonicbox}
\begin{center}\rule{0.5\linewidth}{0.5pt}\end{center}

\subsection*{Question 5(B) [8 marks]}\label{q5b}

\subsubsection{Question 5(B)(1) [4
marks]}\label{question-5b1-4-marks}

\textbf{Describe hydrogen bond and its types with examples.}

\begin{solutionbox}

\textbf{Definition}: Weak electrostatic attraction between hydrogen and
electronegative atoms

\textbf{Types}:

\begin{longtable}[]{@{}lll@{}}
\toprule\noalign{}
Type & Description & Example \\
\midrule\noalign{}
\endhead
\bottomrule\noalign{}
\endlastfoot
\textbf{Intermolecular} & Between different molecules & H_{2}O···H_{2}O \\
\textbf{Intramolecular} & Within same molecule & o-nitrophenol \\
\end{longtable}

\textbf{Characteristics}:

\begin{itemize}
\tightlist
\item
  \textbf{Strength}: 5-40 kJ/mol
\item
  \textbf{Requirements}: H bonded to F, O, N
\end{itemize}

\textbf{Diagram:}

\begin{verbatim}
H_{2O···H_{2}O···H_{2}O}
 │     │     │
δ^{+    δ^{-}    δ^{+}}
\end{verbatim}

\end{solutionbox}
\begin{mnemonicbox}
``Hydrogen Needs FON friends'' (Fluorine, Oxygen,
Nitrogen)

\end{mnemonicbox}
\subsubsection{Question 5(B)(2) [4
marks]}\label{question-5b2-4-marks}

\textbf{Differentiate between Primary cell and Secondary cell. (Four
points)}

\begin{solutionbox}

\begin{longtable}[]{@{}lll@{}}
\toprule\noalign{}
Aspect & Primary Cell & Secondary Cell \\
\midrule\noalign{}
\endhead
\bottomrule\noalign{}
\endlastfoot
\textbf{Rechargeability} & Non-rechargeable & Rechargeable \\
\textbf{Reaction} & Irreversible & Reversible \\
\textbf{Cost} & Low initial cost & High initial cost \\
\textbf{Examples} & Dry cell, alkaline & Lead-acid, Li-ion \\
\end{longtable}

\textbf{Applications}:

\begin{itemize}
\tightlist
\item
  \textbf{Primary}: Remote controls, flashlights
\item
  \textbf{Secondary}: Cars, phones, laptops
\end{itemize}

\end{solutionbox}
\begin{mnemonicbox}
``Primary = Permanent, Secondary = Reversible''
(PPSR)

\end{mnemonicbox}
\subsubsection{Question 5(B)(3) [4
marks]}\label{question-5b3-4-marks}

\textbf{Describe construction, working and chemical equations of
lead-acid storage cell with a labelled diagram.}

\begin{solutionbox}

\textbf{Construction}:

\begin{itemize}
\tightlist
\item
  \textbf{Anode}: Lead (Pb)
\item
  \textbf{Cathode}: Lead dioxide (PbO_{2})
\item
  \textbf{Electrolyte}: Dilute H_{2}SO_{4}
\end{itemize}

\textbf{Chemical Equations}:

\begin{itemize}
\tightlist
\item
  \textbf{Discharge}: Pb + PbO_{2} + 2H_{2}SO_{4} \rightarrow 2PbSO_{4} + 2H_{2}O
\item
  \textbf{Charge}: 2PbSO_{4} + 2H_{2}O \rightarrow Pb + PbO_{2} + 2H_{2}SO_{4}
\end{itemize}

\textbf{Diagram}:

\begin{verbatim}
    ┌─────────────────┐
    │   Pb  │  PbO_{2   │}
    │ ({-)   │   (+)   │}
    │Anode  │ Cathode │
    │   │   │   │     │
    │ ┌─────────────┐ │
    │ │   H_{2SO_{4}     │ │}
    │ │ Electrolyte │ │
    │ └─────────────┘ │
    └─────────────────┘
\end{verbatim}

\textbf{Working}: Chemical energy converts to electrical energy during
discharge

\end{solutionbox}
\begin{mnemonicbox}
``Lead Acid Storage = Reversible Energy'' (LASRE)

\end{mnemonicbox}

\end{document}
