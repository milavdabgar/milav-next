\documentclass[10pt,a4paper]{article}

% content/resources/templates/preamble.tex
\usepackage[margin=0.6in]{geometry}
\author{Milav Dabgar}
\usepackage{amsmath,amssymb,amsthm}
\usepackage{booktabs}
\usepackage{multirow}
\usepackage{xcolor}
\usepackage{tcolorbox}
\tcbuselibrary{breakable,skins}
\usepackage[colorlinks=true,linkcolor=blue]{hyperref}
\usepackage{titlesec}
\usepackage{enumitem}
\usepackage{tikz}
\usepackage{pgfplots}
\usepackage{circuitikz}
\usepackage[version=4]{mhchem}
\usepackage{longtable}
\usepackage{array}
\usepackage{float}
\usepackage{caption}
\usepackage{listings}

\lstset{
  basicstyle=\small\ttfamily,
  breaklines=true,
  breakatwhitespace=false,
  postbreak=\mbox{\textcolor{red}{$\hookrightarrow$}\space},
  float=false,
  numbers=left,
  numberstyle=\tiny\color{gray},
  numbersep=10pt,
  xleftmargin=2em,
  keywordstyle=\color{blue},
  commentstyle=\color{green!60!black},
  stringstyle=\color{purple},
  backgroundcolor=\color{gray!5},
  showstringspaces=false,
  tabsize=2,
  captionpos=b,
  keepspaces=true,
  columns=flexible
}

\pgfplotsset{compat=1.18}
\usetikzlibrary{shapes,arrows,positioning,calc,patterns,decorations.pathmorphing,decorations.markings,arrows.meta}

% Color scheme
\definecolor{headcolor}{RGB}{0,102,204}
\definecolor{keycolor}{RGB}{220,20,60}
\definecolor{solutioncolor}{RGB}{34,139,34}
\definecolor{mnemoniccolor}{RGB}{148,0,211}
\definecolor{codecolor}{RGB}{0,0,100}

% Spacing
\setlength{\parskip}{3pt}
\setlist[itemize]{nosep}
\setlist[enumerate]{nosep}

% Title formatting
\titleformat{\section}{\Large\bfseries\color{headcolor}}{\thesection}{1em}{}
\titleformat{\subsection}{\large\bfseries\color{headcolor}}{\thesubsection}{1em}{}

% Pandoc tightlist compatibility
\providecommand{\tightlist}{%
  \setlength{\itemsep}{0pt}\setlength{\parskip}{0pt}}

% Pandoc longtable compatibility
\newcounter{none}
\def\thenone{}


% content/resources/templates/gujarati-boxes.tex
\usepackage{fontspec}
\usepackage{polyglossia}

% Set Gujarati as main language (document is primarily in Gujarati)
% Note: gloss-gujarati.ldf doesn't exist in polyglossia, but it will use hyphenation patterns
\setdefaultlanguage{gujarati}
\setotherlanguage{english}

% Configure Gujarati font properly
% Use Language=Default to prevent polyglossia from trying to add language-specific features
% that don't exist for Gujarati, which causes "empty feature" warnings
\newfontfamily\gujaratifont[Script=Gujarati,AutoFakeBold=2.5,AutoFakeSlant=0.3]{Noto Sans Gujarati}
\setmainfont[Script=Gujarati,AutoFakeBold=2.5,AutoFakeSlant=0.3]{Noto Sans Gujarati}
% Use Noto Sans Gujarati for monospace to support Gujarati in text
\setmonofont[Scale=0.9]{Noto Sans Gujarati}

% Configure English to use the same font
\newfontfamily\englishfont[Script=Gujarati,AutoFakeBold=2.5,AutoFakeSlant=0.3]{Noto Sans Gujarati}

% Translations for polyglossia
\gappto\captionsgujarati{
  \renewcommand{\tablename}{કોષ્ટક}
  \renewcommand{\figurename}{આકૃતિ}
}

% Helper for TikZ nodes to ensure Gujarati font
\newcommand{\gu}[1]{{\gujaratifont #1}}

% Custom environments
\newtcolorbox{solutionbox}{
    breakable,
    enhanced,
    colback=solutioncolor!5!white,
    colframe=solutioncolor!75!black,
    fonttitle=\bfseries,
    title=જવાબ
}

\newtcolorbox{solutionboxnobreak}{
 colback=solutioncolor!5!white,
 colframe=solutioncolor!75!black,
 fonttitle=\bfseries,
 title=જવાબ
}

\newtcolorbox{keyformula}{
 breakable,
 enhanced,
 colback=keycolor!5!white,
 colframe=keycolor!75!black,
 fonttitle=\bfseries,
 title=રાસાયણિક સમીકરણ/સૂત્ર
}

\newtcolorbox{mnemonicbox}{
 breakable,
 enhanced,
 colback=mnemoniccolor!5!white,
 colframe=mnemoniccolor!75!black,
 fonttitle=\bfseries,
 title=મેમરી ટ્રીક
}


\begin{document}

\begin{center}
{\Huge\bfseries\color{headcolor} એન્જિનિયરિંગ કેમિસ્ટ્રી સોલ્યુશન્સ}\\[5pt]
{\LARGE DI01000071 -- શિયાળો 2024}\\[3pt]
{\large સેમેસ્ટર 1 અભ્યાસ સામગ્રી}\\[3pt]
{\normalsize\textit{વિગતવાર ઉકેલો અને સમજૂતી}}
\end{center}

\vspace{10pt}

\section*{પ્રશ્ન 1 [14 ગુણ]}
\textbf{આપેલ વિકલ્પોમાંથી યોગ્ય વિકલ્પ પસંદ કરી ખાલી જગ્યાઓ પૂરો:}

\begin{solutionbox}
\textbf{જવાબ}:

\begin{longtable}{|p{0.8cm}|p{2.5cm}|p{10cm}|}
\hline
(1) & \ce{[Ar]4s^1 3d^10} & \textenglish{Cu} માં 29 ઇલેક્ટ્રોન છે, \textenglish{Aufbau} નિયમનો અપવાદ \\ \hline
(2) & 14 & pH + pOH = 14 (25$^\circ$C પર) \\ \hline
(3) & કેથોડ & શુદ્ધ તાંબુ નેગેટિવ ઇલેક્ટ્રોડ પર જમા થાય \\ \hline
(4) & \textenglish{Cu} & તાંબુ સુરક્ષિત ઓક્સાઇડ સ્તર બનાવે છે \\ \hline
(5) & અર્ધ-ઘન & પીટ અંશતઃ વિઘટિત કાર્બનિક પદાર્થ છે \\ \hline
(6) & ડ્યુલોંગ & ડ્યુલોંગના સૂત્રથી ઉષ્મીય મૂલ્ય ગણાય \\ \hline
(7) & લિગ્નાઇટ & લિગ્નાઇટમાં સૌથી વધુ ભેજ (35-75\%) \\ \hline
(8) & પોઇઝ & ડાયનેમિક વિસ્કોસિટીનો SI એકમ \\ \hline
(9) & ઊંચું & ઊંચું ફ્લેશ પોઇન્ટ ઇગ્નિશન અટકાવે છે \\ \hline
(10) & પાયસ & તેલ-પાણીનું મિશ્રણ પાયસ બનાવે છે \\ \hline
(11) & બેકેલાઇટ & ફિનોલ ફોર્મેલ્ડિહાઇડ = બેકેલાઇટ \\ \hline
(12) & S & વલ્કેનાઇઝેશન માટે સલ્ફર વપરાય છે \\ \hline
(13) & PHBV & PHBV જૈવવિઘટનીય પોલિમર છે \\ \hline
(14) & વોલ્ટ & EMF વોલ્ટમાં માપાય છે \\ \hline
\end{longtable}
\end{solutionbox}

\begin{mnemonicbox}
"રાસાયણિક તાંબુ સુંદર ગુણધર્મો બનાવે"
\end{mnemonicbox}

\section*{પ્રશ્ન 2(A) [6 ગુણ]}

\subsection*{પ્રશ્ન 2(A)(1) [3 ગુણ]}
\textbf{જુદાં જુદાં ક્ષેત્રોમાં pHની ત્રણ અગત્યતાની સૂચિ બનાવો.}

\begin{solutionbox}
\textbf{જવાબ}:

\begin{tabular}{|l|l|l|}
\hline
\textbf{ક્ષેત્ર} & \textbf{મહત્વ} & \textbf{એપ્લિકેશન} \\ \hline
\textbf{દવાશાસ્ત્ર} & લોહીનું pH જાળવણું & સામાન્ય pH 7.35-7.45 યોગ્ય શરીરિક કાર્ય માટે \\ \hline
\textbf{કૃષિ} & માટીનું pH ઓપ્ટિમાઇઝેશન & pH 6-7 પાકની વૃદ્ધિ અને પોષણ માટે આદર્શ \\ \hline
\textbf{ઉદ્યોગ} & ગુણવત્તા નિયંત્રણ & pH ખોરાક, કાપડ, દવાઓની ગુણવત્તાને અસર કરે \\ \hline
\end{tabular}
\end{solutionbox}

\begin{mnemonicbox}
"દવા કૃષિ ઉદ્યોગ"
\end{mnemonicbox}

\subsection*{પ્રશ્ન 2(A)(2) [3 ગુણ]}
\textbf{વ્યાખ્યા આપો: બફર દ્રાવણો, અર્ધ-કોષ, વિદ્યુતવિભાજનનો ફેરાડેનો પ્રથમ નિયમ.}

\begin{solutionbox}
\textbf{જવાબ}:
\begin{itemize}
    \item \textbf{બફર દ્રાવણો}: એવા દ્રાવણો જે થોડું એસિડ કે બેઝ ઉમેરવાથી pH બદલાવમાં પ્રતિકાર કરે.
    \item \textbf{અર્ધ-કોષ}: એક ઇલેક્ટ્રોડ તેના આયનિક દ્રાવણમાં ડૂબેલો, ઓક્સિડેશન કે રિડક્શન દર્શાવે.
    \item \textbf{ફેરાડેનો પ્રથમ નિયમ}: ઇલેક્ટ્રોડ પર જમા/મુક્ત થતા પદાર્થની માત્રા વીજળીની માત્રાના સીધા પ્રમાણમાં હોય ($w \propto Q$).
\end{itemize}
\end{solutionbox}

\begin{mnemonicbox}
"બફર મદદ ફેરાડે"
\end{mnemonicbox}

\subsection*{પ્રશ્ન 2(A)(3) [3 ગુણ]}
\textbf{ક્ષારણ દર ઉપર અસર કરતાં પરિબળો જણાવો.}

\begin{solutionbox}
\textbf{જવાબ}:
\begin{tabular}{|l|l|l|}
\hline
\textbf{પરિબળ} & \textbf{અસર} & \textbf{વર્ણન} \\ \hline
\textbf{ધાતુની શુદ્ધતા} & વધુ શુદ્ધતા = ઓછું ક્ષારણ & અશુદ્ધિઓ ગેલ્વેનિક કોષ બનાવે \\ \hline
\textbf{તાપમાન} & વધુ તાપમાન = ઝડપી ક્ષારણ & પ્રતિક્રિયા દર વધારે \\ \hline
\textbf{ભેજ} & વધુ ભેજ = વધુ ક્ષારણ & ઇલેક્ટ્રોકેમિકલ પ્રતિક્રિયાઓ પ્રોત્સાહન \\ \hline
\end{tabular}
\end{solutionbox}

\begin{mnemonicbox}
"શુદ્ધ તાપમાન ભેજ"
\end{mnemonicbox}

\section*{પ્રશ્ન 2(B) [8 ગુણ]}

\subsection*{પ્રશ્ન 2(B)(1) [4 ગુણ]}
\textbf{કક્ષાઓ અને કક્ષકો વચ્ચે સરખામણી કરો (દરેકના ચાર મુદ્દાઓ).}

\begin{solutionbox}
\textbf{જવાબ}:
\begin{tabular}{|l|p{5cm}|p{5cm}|}
\hline
\textbf{પાસું} & \textbf{કક્ષાઓ} & \textbf{કક્ષકો} \\ \hline
\textbf{વ્યાખ્યા} & નિશ્ચિત ગોળાકાર માર્ગ & 3D સંભાવના પ્રદેશો \\ \hline
\textbf{આકાર} & ગોળાકાર/અંડાકાર & s, p, d, f આકારો \\ \hline
\textbf{ઊર્જા} & નિશ્ચિત ઊર્જા સ્તરો & ઊર્જા શ્રેણીઓ \\ \hline
\textbf{ઇલેક્ટ્રોન સ્થાન} & ચોક્કસ સ્થિતિ & મળવાની સંભાવના \\ \hline
\end{tabular}

\vspace{0.5cm}
\textbf{આકૃતિ:}
\begin{center}
\begin{tikzpicture}[scale=0.8]
    % Bohr Model
    \node at (0, -2.5) {\textbf{કક્ષાઓ (બોહર)}};
    \draw (0,0) circle (0.5); \node at (0,0) {+};
    \draw (0,0) circle (1.2); \fill (1.2,0) circle (2pt) node[right] {$e^-$};
    \draw (0,0) circle (1.8);
    
    % Quantum Model
    \begin{scope}[xshift=5cm]
        \node at (0, -2.5) {\textbf{કક્ષકો (ક્વાન્ટમ)}};
        \foreach \i in {1,...,100} \fill (rand*1.5, rand*1.5) circle (0.5pt);
        \node at (0,0) {\textbf{+}};
        \node at (0,-1.8) {ઇલેક્ટ્રોન વાદળ};
    \end{scope}
\end{tikzpicture}
\end{center}
\end{solutionbox}

\begin{mnemonicbox}
"નિશ્ચિત આકાર ઊર્જા સ્થાન"
\end{mnemonicbox}

\subsection*{પ્રશ્ન 2(B)(2) [4 ગુણ]}
\textbf{દરેકના એક ઉદાહરણ સાથે તેના સ્ત્રોતો અને ભૌતિક સ્થિતિઓના આધારે ઇંધણોનું વર્ગીકરણ કરો.}

\begin{solutionbox}
\textbf{જવાબ}:
\begin{tabular}{|l|l|l|l|}
\hline
\textbf{વર્ગીકરણ} & \textbf{પ્રકાર} & \textbf{ઉદાહરણ} & \textbf{વર્ણન} \\ \hline
\textbf{સ્ત્રોત આધારિત} & કુદરતી & કોલસો & કુદરતી રીતે બન્યું \\ \hline
 & કૃત્રિમ & પેટ્રોલ & માનવ નિર્મિત \\ \hline
\textbf{ભૌતિક સ્થિતિ} & ઘન & લાકડું & ઓરડાના તાપમાને ઘન \\ \hline
 & પ્રવાહી & ડીઝલ & ઓરડાના તાપમાને પ્રવાહી \\ \hline
 & ગેસીય & LPG & ઓરડાના તાપમાને ગેસ \\ \hline
\end{tabular}
\end{solutionbox}

\begin{mnemonicbox}
"કુદરતી કૃત્રિમ, ઘન પ્રવાહી ગેસ"
\end{mnemonicbox}

\subsection*{પ્રશ્ન 2(B)(3) [4 ગુણ]}
\textbf{બાયોડીઝલ વિશે ચાર અગત્યના મુદ્દાઓ સમજાવો.}

\begin{solutionbox}
\textbf{જવાબ}:
\begin{itemize}
    \item \textbf{સ્ત્રોત}: વનસ્પતિ તેલ, પ્રાણીઓની ચરબી અથવા વપરાયેલા રસોઈ તેલમાંથી બને.
    \item \textbf{પ્રક્રિયા}: મેથેનોલ/ઇથેનોલ સાથે ટ્રાન્સએસ્ટેરિફિકેશન પ્રતિક્રિયાથી બને.
    \item \textbf{ગુણધર્મો}: જૈવવિઘટનીય, બિન-ઝેરી, નવીકરણીય ઇંધણ સ્ત્રોત.
    \item \textbf{ઉપયોગો}: ડીઝલ એન્જિનમાં વપરાય, ઉત્સર્જન 75\% ઘટાડે.
\end{itemize}

\begin{keyformula}
\[ \text{વનસ્પતિ તેલ} + \text{મેથેનોલ} \xrightarrow{\text{ઉદ્દીપક}} \text{બાયો-ડીઝલ} + \text{ગ્લિસેરોલ} \]
\end{keyformula}
\end{solutionbox}

\begin{mnemonicbox}
"સ્ત્રોત પ્રક્રિયા ગુણધર્મો ઉપયોગો"
\end{mnemonicbox}

\section*{પ્રશ્ન 3(A) [6 ગુણ]}

\subsection*{પ્રશ્ન 3(A)(1) [3 ગુણ]}
\textbf{ઉદાહરણની મદદથી દ્રાવ્ય, દ્રાવક અને દ્રાવણ સમજાવો.}

\begin{solutionbox}
\textbf{જવાબ}:
\begin{tabular}{|l|l|l|}
\hline
\textbf{ઘટક} & \textbf{વ્યાખ્યા} & \textbf{ઉદાહરણ} \\ \hline
\textbf{દ્રાવ્ય} & જે પદાર્થ ઓગળે છે & મીઠું (\ce{NaCl}) \\ \hline
\textbf{દ્રાવક} & જેમાં પદાર્થ ઓગળે છે & પાણી (\ce{H2O}) \\ \hline
\textbf{દ્રાવણ} & સમાંગી મિશ્રણ & મીઠાનું પાણી \\ \hline
\end{tabular}

\textbf{ઉદાહરણ}: ખાંડ + પાણી = ખાંડનું દ્રાવણ
\begin{itemize}
    \item ખાંડ = દ્રાવ્ય, પાણી = દ્રાવક, ખાંડનું પાણી = દ્રાવણ
\end{itemize}
\end{solutionbox}

\begin{mnemonicbox}
"દ્રાવ્ય દ્રાવક દ્રાવણ"
\end{mnemonicbox}

\subsection*{પ્રશ્ન 3(A)(2) [3 ગુણ]}
\textbf{NaClમાં વિદ્યુતસંયોજક બંધનું નિર્માણ સમજાવો.}

\begin{solutionbox}
\textbf{જવાબ}:
\textbf{પ્રક્રિયા}:
\begin{enumerate}
    \item \textbf{પગલું 1}: Na એક ઇલેક્ટ્રોન ગુમાવે $\to$ \ce{Na^+} (કેટાયન)
    \item \textbf{પગલું 2}: Cl એક ઇલેક્ટ્રોન મેળવે $\to$ \ce{Cl^-} (આયન)
    \item \textbf{પગલું 3}: \ce{Na^+} અને \ce{Cl^-} વચ્ચે વિદ્યુતસ્થિતિક આકર્ષણ \ce{NaCl} બનાવે છે.
\end{enumerate}

\begin{keyformula}
\begin{center}
    \ce{Na -> Na+ + e-} \\
    \ce{Cl + e- -> Cl-} \\
    \ce{Na+ + Cl- -> NaCl}
\end{center}
\end{keyformula}
\end{solutionbox}

\begin{mnemonicbox}
"સોડિયમ ગુમાવે, ક્લોરિન મેળવે, આકર્ષણ બને"
\end{mnemonicbox}

\subsection*{પ્રશ્ન 3(A)(3) [3 ગુણ]}
\textbf{ગેસોલીન માટે ઓક્ટેન આંક સમજાવો.}

\begin{solutionbox}
\textbf{જવાબ}:
\begin{tabular}{|l|l|}
\hline
\textbf{પાસું} & \textbf{વર્ણન} \\ \hline
\textbf{વ્યાખ્યા} & ઇંધણની નોકિંગ સામે પ્રતિકારશક્તિનું માપ \\ \hline
\textbf{સ્કેલ} & 0-100, વધુ = વધુ સારી એન્ટી-નોક ગુણવત્તા \\ \hline
\textbf{માનક} & n-હેપ્ટેન = 0, આઇસો-ઓક્ટેન = 100 \\ \hline
\end{tabular}
\textbf{ઉપયોગો}: ઊંચા ઓક્ટેન ઇંધણ એન્જિન નોકિંગ અટકાવે, કામગીરી સુધારે.
\end{solutionbox}

\begin{mnemonicbox}
"ઓક્ટેન નોકિંગ વિરોધી"
\end{mnemonicbox}

\section*{પ્રશ્ન 3(B) [8 ગુણ]}

\subsection*{પ્રશ્ન 3(B)(1) [4 ગુણ]}
\textbf{અશુદ્ધ Cuનું વિદ્યુતશુદ્ધિકરણ રાસાયણિક સમીકરણો અને નામ નિર્દેશનવાળી આકૃતિ સાથે સમજાવો.}

\begin{solutionbox}
\textbf{જવાબ}:
\textbf{પ્રક્રિયા}:
\begin{itemize}
    \item \textbf{એનોડ}: અશુદ્ધ તાંબુ (જાડું) - ઓગળે.
    \item \textbf{કેથોડ}: શુદ્ધ તાંબુ (પાતળી પટ્ટી) - જમા થાય.
    \item \textbf{ઇલેક્ટ્રોલાઇટ}: એસિડિક \ce{CuSO4} દ્રાવણ.
\end{itemize}

\textbf{રાસાયણિક સમીકરણો}:
\begin{itemize}
    \item એનોડ પર: \ce{Cu -> Cu^2+ + 2e-} (ઓક્સિડેશન)
    \item કેથોડ પર: \ce{Cu^2+ + 2e- -> Cu} (રિડક્શન)
\end{itemize}

\textbf{આકૃતિ:}
\begin{center}
\begin{tikzpicture}
    % Container
    \draw[thick] (0,0) rectangle (6,4);
    \fill[cyan!20] (0.2,0.2) rectangle (5.8,3.5);
    \node at (3,1) {\ce{CuSO4} દ્રાવણ};
    
    % Electrodes
    \draw[fill=gray!40] (1,2) rectangle (1.5,4.5); \node at (1.25, 4.8) {કેથોડ (-) શુદ્ધ Cu};
    \draw[fill=orange!60] (4.5,2) rectangle (5.5,4.5); \node at (5, 4.8) {એનોડ (+) અશુદ્ધ Cu};
    
    % Circuit
    \draw (1.25, 4.5) -- (1.25, 5.5) -- (2.5, 5.5);
    \draw (5, 4.5) -- (5, 5.5) -- (3.5, 5.5);
    \draw (2.5, 5.3) rectangle (3.5, 5.7); \node at (3, 5.5) {બેટરી};
    
    % Ions
    \node at (3, 2.5) {\ce{Cu^2+}};
    \draw[->] (4.5, 2.5) -- (3.5, 2.5);
    \draw[->] (2.5, 2.5) -- (1.5, 2.5);
\end{tikzpicture}
\end{center}
\end{solutionbox}

\begin{mnemonicbox}
"એનોડ ઓગળે, કેથોડ જમાવે"
\end{mnemonicbox}

\subsection*{પ્રશ્ન 3(B)(2) [4 ગુણ]}
\textbf{રાસાયણિક સમીકરણ સાથે ઇથિનની બનાવટ સમજાવો. તેના બે ગુણધર્મો અને બે ઉપયોગો પણ લખો.}

\begin{solutionbox}
\textbf{જવાબ}:
\textbf{તૈયારી}: ઇથેનોલનું નિર્જલીકરણ 170$^\circ$C પર સાંદ્ર \ce{H2SO4} સાથે.
\begin{keyformula}
\[ \ce{C2H5OH ->[Conc. H2SO4][170^\circ C] C2H4 + H2O} \]
\end{keyformula}

\textbf{ગુણધર્મો}:
\begin{itemize}
    \item \textbf{ભૌતિક}: રંગહીન, મીઠી સુગંધવાળો ગેસ.
    \item \textbf{રાસાયણિક}: અસંતૃપ્ત હાઇડ્રોકાર્બન, ઉમેરણ પ્રતિક્રિયાઓ કરે.
\end{itemize}

\textbf{ઉપયોગો}:
\begin{itemize}
    \item પોલિઇથિલીન પ્લાસ્ટિકના ઉત્પાદનમાં.
    \item ફળોને કૃત્રિમ રીતે પકવવા.
\end{itemize}
\end{solutionbox}

\begin{mnemonicbox}
"તૈયારી ગુણધર્મો ઉપયોગો"
\end{mnemonicbox}

\subsection*{પ્રશ્ન 3(B)(3) [4 ગુણ]}
\textbf{રાસાયણિક સમીકરણ સાથે Buna-S રબરની બનાવટ સમજાવો. તેના બે ગુણધર્મો અને બે ઉપયોગો પણ લખો.}

\begin{solutionbox}
\textbf{જવાબ}:
\textbf{તૈયારી}: 3:1 ગુણોત્તરમાં 1,3-બ્યુટાડાયન અને સ્ટાયરીનનું કોપોલિમેરાઇઝેશન.
\begin{keyformula}
\[ \ce{n CH2=CH-CH=CH2 + n C6H5-CH=CH2 -> -[CH2-CH=CH-CH2-CH(C6H5)-CH2]_n-} \]
\end{keyformula}
(બ્યુટાડાયન + સ્ટાયરીન $\to$ Buna-S)

\textbf{ગુણધર્મો}:
\begin{itemize}
    \item ઉચ્ચ ઘર્ષણ પ્રતિકાર.
    \item ઉચ્ચ લોડ-બેરિંગ ક્ષમતા.
\end{itemize}

\textbf{ઉપયોગો}:
\begin{itemize}
    \item ઓટોમોબાઇલ ટાયરના ઉત્પાદનમાં.
    \item કન્વેયર બેલ્ટ અને હોઝ.
\end{itemize}
\end{solutionbox}

\begin{mnemonicbox}
"બ્યુટાડાયન સ્ટાયરીન મજબૂત રબર બનાવે"
\end{mnemonicbox}

\section*{પ્રશ્ન 4(A) [6 ગુણ]}

\subsection*{પ્રશ્ન 4(A)(1) [3 ગુણ]}
\textbf{ધાતુઓનું ક્ષારણ નિવારવા ધાતુક્લેડિંગ સમજાવો.}

\begin{solutionbox}
\textbf{જવાબ}:
\begin{itemize}
    \item \textbf{પ્રક્રિયા}: ક્ષારણ-પ્રતિરોધી ધાતુના બે સ્તરો વચ્ચે મૂળ ધાતુને સેન્ડવીચ કરવી (જેમ કે Al, Ni).
    \item \textbf{પદ્ધતિ}: રોલ બોન્ડિંગ દ્વારા.
    \item \textbf{ઉપયોગ}: એરક્રાફ્ટ ઉદ્યોગમાં વપરાય છે (Alclad - શુદ્ધ એલ્યુમિનિયમ વચ્ચે સેન્ડવીચ કરેલ ડ્યુરાલ્યુમિન).
    \item \textbf{મિકેનિઝમ}: સુરક્ષિત સ્તર ઓક્સિજન અને ભેજ સામે ભૌતિક અવરોધ તરીકે કાર્ય કરે છે.
\end{itemize}
\end{solutionbox}

\begin{mnemonicbox}
"આવરણ ધાતુ સુરક્ષિત કરે"
\end{mnemonicbox}

\subsection*{પ્રશ્ન 4(A)(2) [3 ગુણ]}
\textbf{પાણીની સપાટી નીચે થતું ક્ષારણ રાસાયણિક પ્રક્રિયાઓ અને નામનિર્દેશનવાળી આકૃતિ સાથે સમજાવો.}

\begin{solutionbox}
\textbf{જવાબ}:
\textbf{પ્રક્રિયા}: પાણી-હવા સંપર્ક સ્થળે વિભેદક વાયુકરણ (differential aeration) ને કારણે થાય છે.

\begin{keyformula}
\begin{itemize}
    \item એનોડ: \ce{Fe -> Fe^2+ + 2e-} (અહીં ક્ષારણ થાય)
    \item કેથોડ: \ce{O2 + 2H2O + 4e- -> 4OH-}
\end{itemize}
\end{keyformula}

\textbf{આકૃતિ:}
\begin{center}
\begin{tikzpicture}
    \draw[fill=cyan!30] (0,0) rectangle (4,2.5);
    \draw[thick] (2,0) -- (2,3.5); \node at (2.5, 3.2) {લોખંડની ટાંકી};
    \draw[dashed] (0, 2.5) -- (4, 2.5); \node at (1, 2.8) {કેથોડ (વધુ $O_2$)};
    \node at (1, 1.5) {પાણી};
    \node at (3, 1.5) {એનોડ (ઓછો $O_2$)};
    \node at (3.2, 1) {(ક્ષારણ)};
    \draw[->] (2.2, 1) -- (2, 1);
\end{tikzpicture}
\end{center}
\end{solutionbox}

\begin{mnemonicbox}
"પાણી હવા સંપર્ક ક્ષારણ કરે"
\end{mnemonicbox}

\subsection*{પ્રશ્ન 4(A)(3) [3 ગુણ]}
\textbf{સૌર કોષોના કાર્યકારી સિદ્ધાંતને સમજાવો.}

\begin{solutionbox}
\textbf{જવાબ}:
\begin{tabular}{|l|l|}
\hline
\textbf{ઘટક} & \textbf{કાર્ય} \\ \hline
\textbf{ફોટોવોલ્ટેઇક અસર} & પ્રકાશ ઊર્જા વિદ્યુત ઊર્જામાં ફેરવાય \\ \hline
\textbf{p-n જંકશન} & ચાર્જ વિભાજન માટે વિદ્યુત ક્ષેત્ર બનાવે \\ \hline
\textbf{ઇલેક્ટ્રોન-હોલ જોડી} & ફોટોન સેમિકન્ડક્ટર સાથે અથડાય ત્યારે બને \\ \hline
\end{tabular}

\textbf{પ્રક્રિયા}: પ્રકાશ સપાટી પર પડે $\to$ ઇલેક્ટ્રોન ઉત્તેજિત $\to$ p-n જંકશન પાર $\to$ પ્રવાહ.
\end{solutionbox}

\begin{mnemonicbox}
"ફોટો વોલ્ટેઇક જંકશન પ્રવાહ બનાવે"
\end{mnemonicbox}

\section*{પ્રશ્ન 4(B) [8 ગુણ]}

\subsection*{પ્રશ્ન 4(B)(1) [4 ગુણ]}
\textbf{આકૃતિ સાથે સીમાવર્તી સ્નેહનનું કાર્ય દર્શાવો.}

\begin{solutionbox}
\textbf{જવાબ}:
\textbf{કાર્ય}: ઉચ્ચ લોડ અને ઓછી ઝડપ હેઠળ વપરાય. સ્નેહકનો પાતળો સ્તર ધાતુની સપાટી પર શોષાય છે, સીધો સંપર્ક અટકાવે છે.

\textbf{મિકેનિઝમ}:
\begin{itemize}
    \item સ્નેહક અણુઓના ધ્રુવીય છેડા ધાતુ સાથે જોડાય.
    \item હાઇડ્રોકાર્બન ચેઇન ગાદી બનાવે.
\end{itemize}

\textbf{આકૃતિ:}
\begin{center}
\begin{tikzpicture}
    % Surfaces
    \draw[fill=gray!30] (0,2.5) rectangle (5,3); \node at (2.5, 2.75) {ગતિશીલ સપાટી};
    \draw[fill=gray!30] (0,0) rectangle (5,0.5); \node at (2.5, 0.25) {સ્થિર સપાટી};
    
    % Lubricant molecules
    \foreach \x in {0.5, 1.0, ..., 4.5} {
        \draw[thick, red] (\x, 0.5) -- (\x, 1.2); \fill[red] (\x, 0.5) circle (2pt);
        \draw[thick, red] (\x, 2.5) -- (\x, 1.8); \fill[red] (\x, 2.5) circle (2pt);
    }
    \node at (5.5, 1.5) {તેલ ફિલ્મ};
\end{tikzpicture}
\end{center}
\end{solutionbox}

\begin{mnemonicbox}
"સીમા અવરોધ ધાતુ સંપર્ક અટકાવે"
\end{mnemonicbox}

\subsection*{પ્રશ્ન 4(B)(2) [4 ગુણ]}
\textbf{રેડવુડ વિસ્કોમીટર દ્વારા સિનગ્ધતા કેવી રીતે માપવામાં આવે છે તે નામનિર્દેશનવાળી આકૃતિ સાથે સમજાવો.}

\begin{solutionbox}
\textbf{જવાબ}:
\textbf{સિદ્ધાંત}: "રેડવુડ સેકન્ડ્સ" માં વિસ્કોસિટી માપે - ગુરુત્વાકર્ષણ હેઠળ પ્રમાણભૂત છિદ્રમાંથી 50ml તેલ વહેવા માટે લાગતો સમય.

\textbf{કાર્યવિધિ}:
\begin{enumerate}
    \item સાધનને સાફ અને લેવલ કરો.
    \item પોઇન્ટર લેવલ સુધી કપમાં તેલ ભરો. વોટર બાથ ગરમ કરો.
    \item બોલ વાલ્વ દૂર કરો, સ્ટોપવોચ શરૂ કરો.
    \item ફ્લાસ્કમાં 50ml તેલ એકત્રિત કરો. સ્ટોપ વોચ બંધ કરો.
\end{enumerate}

\textbf{આકૃતિ:}
\begin{center}
\begin{tikzpicture}[scale=0.8]
    % Outer Bath
    \draw (0,0) rectangle (4,5); \node at (3.5, 4) {વોટર બાથ};
    % Inner Cup
    \draw (1,1) rectangle (3,5); \node at (2, 3) {ઓઇલ કપ};
    % Orifice
    \draw (1.8, 1) -- (2.2, 1) -- (2, 0.5) -- cycle; \node at (2, 0.5) [below] {જેટ};
    % Flask
    \draw (1.5, -2) rectangle (2.5, -0.5); \node at (2, -1.25) {50ml ફ્લાસ્ક};
    % Stirrer/Thermometer simplified
    \draw[thick] (0.5, 5.5) -- (0.5, 1.5); 
\end{tikzpicture}
\end{center}
\end{solutionbox}

\begin{mnemonicbox}
"રેડવુડ સમય નોંધે"
\end{mnemonicbox}

\subsection*{પ્રશ્ન 4(B)(3) [4 ગુણ]}
\textbf{વ્યાખ્યા આપો: અર્ધવાહક, અવાહક પદાર્થ, સ્થિતિસ્થાપક પદાર્થ, યોગશીલ બહુલીભવન.}

\begin{solutionbox}
\textbf{જવાબ}:
\begin{tabular}{|p{4cm}|p{8cm}|}
\hline
\textbf{શબ્દ} & \textbf{વ્યાખ્યા} \\ \hline
\textbf{અર્ધવાહક} & વાહક અને અવાહક વચ્ચેની વાહકતા ધરાવતો પદાર્થ (દા.ત., Si, Ge). \\ \hline
\textbf{અવાહક પદાર્થ} & વિદ્યુત પ્રવાહના પ્રતિકાર કરતો પદાર્થ (દા.ત., રબર, કાચ). \\ \hline
\textbf{સ્થિતિસ્થાપક પદાર્થ} & ઉચ્ચ સ્થિતિસ્થાપકતા ધરાવતો પોલિમર (દા.ત., કુદરતી રબર). \\ \hline
\textbf{યોગશીલ બહુલીભવન} & આડપેદાશો વિના મોનોમર્સ જોડાય (દા.ત., PE, PVC). \\ \hline
\end{tabular}
\end{solutionbox}

\begin{mnemonicbox}
"અર્ધ અવાહક સ્થિતિ યોગશીલ"
\end{mnemonicbox}

\section*{પ્રશ્ન 5(A) [6 ગુણ]}

\subsection*{પ્રશ્ન 5(A)(1) [3 ગુણ]}
\textbf{ઉકેલો: 0.004 M HClના જલીય દ્રાવણની pH અને pOH ગણો. (log 4 = 0.6021)}

\begin{solutionbox}
\textbf{ઉકેલ}:
\begin{itemize}
    \item HCl મજબૂત એસિડ છે, સંપૂર્ણ આયનીકરણ થાય: \ce{HCl -> H+ + Cl-}
    \item $[\ce{H+}] = [\ce{HCl}] = 0.004 \text{ M} = 4 \times 10^{-3} \text{ M}$
    \item $\text{pH} = -\log[\ce{H+}] = -\log(4 \times 10^{-3})$
    \item $\text{pH} = -(\log 4 + \log 10^{-3}) = -(0.6021 - 3) = 2.3979 \approx 2.40$
    \item $\text{pOH} = 14 - \text{pH} = 14 - 2.40 = 11.60$
\end{itemize}
\textbf{જવાબ}: pH = 2.40, pOH = 11.60
\end{solutionbox}

\subsection*{પ્રશ્ન 5(A)(2) [3 ગુણ]}
\textbf{ઉદાહરણ સાથે બાહ્ય અર્ધવાહકો અને તેના પ્રકારો વર્ણવો.}

\begin{solutionbox}
\textbf{જવાબ}: બાહ્ય અર્ધવાહકોમાં વાહકતા વધારવા માટે અશુદ્ધિઓ ઉમેરવામાં આવે છે.
\begin{tabular}{|l|l|l|l|}
\hline
\textbf{પ્રકાર} & \textbf{ડોપન્ટ} & \textbf{મુખ્ય વાહક} & \textbf{ઉદાહરણ} \\ \hline
\textbf{n-પ્રકાર} & પેન્ટાવેલેન્ટ (Gr V) (P, As) & ઇલેક્ટ્રોન & Si + P \\ \hline
\textbf{p-પ્રકાર} & ટ્રાયવેલેન્ટ (Gr III) (B, Al) & હોલ્સ & Si + B \\ \hline
\end{tabular}
\end{solutionbox}

\begin{mnemonicbox}
"n-નેગેટિવ ઇલેક્ટ્રોન, p-પોઝિટિવ હોલ્સ"
\end{mnemonicbox}

\subsection*{પ્રશ્ન 5(A)(3) [3 ગુણ]}
\textbf{ઉષ્માસહ બહુલક અને ઉષ્માસ્થાપિત બહુલક વચ્ચેનાં ફરક આપો. (દરેકનાં ચાર મુદ્દાઓ)}

\begin{solutionbox}
\textbf{જવાબ}:
\begin{tabular}{|l|p{5cm}|p{5cm}|}
\hline
\textbf{ગુણધર્મ} & \textbf{ઉષ્માસહ} & \textbf{ઉષ્માસ્થાપિત} \\ \hline
\textbf{રચના} & રેખીય/શાખાવાળી સાંકળો & ક્રોસ-લિંક્ડ નેટવર્ક \\ \hline
\textbf{ગરમીની અસર} & ગરમ કરવાથી નરમ, ઠંડુ કરવાથી સખત & નરમ નથી પડતું \\ \hline
\textbf{પુનઃઉપયોગ} & પુનઃઉપયોગ શક્ય & પુનઃઉપયોગ અશક્ય \\ \hline
\textbf{ઉદાહરણ} & PE, PVC, PS & બેકેલાઇટ, મેલામાઇન \\ \hline
\end{tabular}
\end{solutionbox}

\begin{mnemonicbox}
"ઉષ્મા-સહ = પુનઃઉપયોગ, ઉષ્મા-સ્થાપિત = કાયમી"
\end{mnemonicbox}

\section*{પ્રશ્ન 5(B) [8 ગુણ]}

\subsection*{પ્રશ્ન 5(B)(1) [4 ગુણ]}
\textbf{હાઇડ્રોજન બંધ અને તેના પ્રકારો ઉદાહરણો સાથે વર્ણવો.}

\begin{solutionbox}
\textbf{જવાબ}:
\textbf{વ્યાખ્યા}: હાઇડ્રોજન અણુ (જે F, O, N સાથે જોડાયેલ હોય) અને અન્ય વિદ્યુતનેગેટિવ અણુ વચ્ચેનું નબળું આકર્ષણ.

\textbf{પ્રકારો}:
\begin{enumerate}
    \item \textbf{અંતરઅણવિક}: વિવિધ અણુઓ વચ્ચે (દા.ત., \ce{H2O}). ઉત્કલન બિંદુ વધારે.
    \item \textbf{અંતઃઅણવિક}: સમાન અણુમાં (દા.ત., o-નાઇટ્રોફિનોલ).
\end{enumerate}

\textbf{આકૃતિ (પાણીમાં):}
\begin{center}
\begin{tikzpicture}
    \node at (0,0) {\ce{H-O-H}};
    \node at (2,0) {\ce{H-O-H}};
    \node at (4,0) {\ce{H-O-H}};
    \draw[dashed, blue, thick] (0.5, 0) -- (1.5, 0);
    \draw[dashed, blue, thick] (2.5, 0) -- (3.5, 0);
    \node at (2, -0.5) {H-બંધ};
\end{tikzpicture}
\end{center}
\end{solutionbox}

\begin{mnemonicbox}
"હાઇડ્રોજનને FON મિત્રોની જરૂર"
\end{mnemonicbox}

\subsection*{પ્રશ્ન 5(B)(2) [4 ગુણ]}
\textbf{પ્રાથમિક કોષ અને દ્વિતીયક કોષ વચ્ચે તફાવત કરો. (ચાર મુદ્દાઓ)}

\begin{solutionbox}
\textbf{જવાબ}:
\begin{tabular}{|l|p{5cm}|p{5cm}|}
\hline
\textbf{પાસું} & \textbf{પ્રાથમિક કોષ} & \textbf{દ્વિતીયક કોષ} \\ \hline
\textbf{રિચાર્જેબિલિટી} & રિચાર્જ ન થાય & રિચાર્જ થાય \\ \hline
\textbf{પ્રતિક્રિયા} & અપરિવર્તનીય & પરિવર્તનીય \\ \hline
\textbf{આયુષ્ય} & ટૂંકું આયુષ્ય & લાંબુ આયુષ્ય \\ \hline
\textbf{ઉદાહરણ} & ડ્રાય સેલ & લેડ-એસિડ, Li-ion \\ \hline
\end{tabular}
\end{solutionbox}

\begin{mnemonicbox}
"પ્રાથમિક = કાયમી, દ્વિતીયક = પરિવર્તનીય"
\end{mnemonicbox}

\subsection*{પ્રશ્ન 5(B)(3) [4 ગુણ]}
\textbf{નામનિર્દેશવાળી આકૃતિ દોરી લેડ-એસિડ સંગ્રાહક કોષની રચના, કાર્ય અને રાસાયણિક સમીકરણો વર્ણવો.}

\begin{solutionbox}
\textbf{જવાબ}:
\textbf{રચના}:
\begin{itemize}
    \item \textbf{એનોડ}: લેડ (Pb).
    \item \textbf{કેથોડ}: લેડ ડાયઓક્સાઇડ (\ce{PbO2}).
    \item \textbf{ઇલેક્ટ્રોલાઇટ}: પાતળું \ce{H2SO4} (ઘનતા 1.25-1.30 g/cc).
\end{itemize}

\textbf{કાર્ય (ડિસ્ચાર્જ)}:
\begin{keyformula}
\begin{itemize}
    \item એનોડ: \ce{Pb + SO4^2- -> PbSO4 + 2e-}
    \item કેથોડ: \ce{PbO2 + 4H+ + SO4^2- + 2e- -> PbSO4 + 2H2O}
    \item એકંદર: \ce{Pb + PbO2 + 2H2SO4 -> 2PbSO4 + 2H2O + \text{ઊર્જા}}
\end{itemize}
\end{keyformula}

\textbf{આકૃતિ:}
\begin{center}
\begin{tikzpicture}[scale=0.9]
    \draw[thick] (0,0) rectangle (4,4);
    \fill[gray!20] (0.2, 0.2) rectangle (3.8, 3.5); \node at (2, 1) {પાતળું \ce{H2SO4}};
    
    \draw[fill=darkgray] (0.5, 2) rectangle (1, 4.5); \node at (0.75, 4.8) {Pb (એનોડ)};
    \draw[fill=brown] (3, 2) rectangle (3.5, 4.5); \node at (3.25, 4.8) {\ce{PbO2} (કેથોડ)};
    
    \node at (0.75, 3.8) {-}; \node at (3.25, 3.8) {+};
\end{tikzpicture}
\end{center}
\end{solutionbox}

\begin{mnemonicbox}
"લેડ એસિડ સ્ટોરેજ = પરિવર્તનીય ઊર્જા"
\end{mnemonicbox}

\end{document}
