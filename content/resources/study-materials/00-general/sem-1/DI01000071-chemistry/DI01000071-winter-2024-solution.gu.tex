\documentclass{article}

% content/resources/templates/preamble.tex
\usepackage[margin=0.6in]{geometry}
\author{Milav Dabgar}
\usepackage{amsmath,amssymb,amsthm}
\usepackage{booktabs}
\usepackage{multirow}
\usepackage{xcolor}
\usepackage{tcolorbox}
\tcbuselibrary{breakable,skins}
\usepackage[colorlinks=true,linkcolor=blue]{hyperref}
\usepackage{titlesec}
\usepackage{enumitem}
\usepackage{tikz}
\usepackage{pgfplots}
\usepackage{circuitikz}
\usepackage[version=4]{mhchem}
\usepackage{longtable}
\usepackage{array}
\usepackage{float}
\usepackage{caption}
\usepackage{listings}

\lstset{
  basicstyle=\small\ttfamily,
  breaklines=true,
  breakatwhitespace=false,
  postbreak=\mbox{\textcolor{red}{$\hookrightarrow$}\space},
  float=false,
  numbers=left,
  numberstyle=\tiny\color{gray},
  numbersep=10pt,
  xleftmargin=2em,
  keywordstyle=\color{blue},
  commentstyle=\color{green!60!black},
  stringstyle=\color{purple},
  backgroundcolor=\color{gray!5},
  showstringspaces=false,
  tabsize=2,
  captionpos=b,
  keepspaces=true,
  columns=flexible
}

\pgfplotsset{compat=1.18}
\usetikzlibrary{shapes,arrows,positioning,calc,patterns,decorations.pathmorphing,decorations.markings,arrows.meta}

% Color scheme
\definecolor{headcolor}{RGB}{0,102,204}
\definecolor{keycolor}{RGB}{220,20,60}
\definecolor{solutioncolor}{RGB}{34,139,34}
\definecolor{mnemoniccolor}{RGB}{148,0,211}
\definecolor{codecolor}{RGB}{0,0,100}

% Spacing
\setlength{\parskip}{3pt}
\setlist[itemize]{nosep}
\setlist[enumerate]{nosep}

% Title formatting
\titleformat{\section}{\Large\bfseries\color{headcolor}}{\thesection}{1em}{}
\titleformat{\subsection}{\large\bfseries\color{headcolor}}{\thesubsection}{1em}{}

% Pandoc tightlist compatibility
\providecommand{\tightlist}{%
  \setlength{\itemsep}{0pt}\setlength{\parskip}{0pt}}

% Pandoc longtable compatibility
\newcounter{none}
\def\thenone{}


% content/resources/templates/gujarati-boxes.tex
\usepackage{fontspec}
\usepackage{polyglossia}

% Set Gujarati as main language (document is primarily in Gujarati)
% Note: gloss-gujarati.ldf doesn't exist in polyglossia, but it will use hyphenation patterns
\setdefaultlanguage{gujarati}
\setotherlanguage{english}

% Configure Gujarati font properly
% Use Language=Default to prevent polyglossia from trying to add language-specific features
% that don't exist for Gujarati, which causes "empty feature" warnings
\newfontfamily\gujaratifont[Script=Gujarati,AutoFakeBold=2.5,AutoFakeSlant=0.3]{Noto Sans Gujarati}
\setmainfont[Script=Gujarati,AutoFakeBold=2.5,AutoFakeSlant=0.3]{Noto Sans Gujarati}
% Use Noto Sans Gujarati for monospace to support Gujarati in text
\setmonofont[Scale=0.9]{Noto Sans Gujarati}

% Configure English to use the same font
\newfontfamily\englishfont[Script=Gujarati,AutoFakeBold=2.5,AutoFakeSlant=0.3]{Noto Sans Gujarati}

% Translations for polyglossia
\gappto\captionsgujarati{
  \renewcommand{\tablename}{કોષ્ટક}
  \renewcommand{\figurename}{આકૃતિ}
}

% Helper for TikZ nodes to ensure Gujarati font
\newcommand{\gu}[1]{{\gujaratifont #1}}

% Custom environments
\newtcolorbox{solutionbox}{
    breakable,
    enhanced,
    colback=solutioncolor!5!white,
    colframe=solutioncolor!75!black,
    fonttitle=\bfseries,
    title=જવાબ
}

\newtcolorbox{solutionboxnobreak}{
 colback=solutioncolor!5!white,
 colframe=solutioncolor!75!black,
 fonttitle=\bfseries,
 title=જવાબ
}

\newtcolorbox{keyformula}{
 breakable,
 enhanced,
 colback=keycolor!5!white,
 colframe=keycolor!75!black,
 fonttitle=\bfseries,
 title=રાસાયણિક સમીકરણ/સૂત્ર
}

\newtcolorbox{mnemonicbox}{
 breakable,
 enhanced,
 colback=mnemoniccolor!5!white,
 colframe=mnemoniccolor!75!black,
 fonttitle=\bfseries,
 title=મેમરી ટ્રીક
}


% Custom commands for GTU solutions
% This file defines semantic commands for consistent formatting

% Question command with automatic formatting
\newcommand{\question}[2]{%
  \section*{Question #1}%
  \textbf{#2}%
}

% OR question variant
\newcommand{\questionor}[2]{%
  \section*{Question #1 OR}%
  \textbf{#2}%
}

% Proper table environment with caption
\newenvironment{answertable}[1]{%
  \begin{table}[htbp]
  \centering
  \caption{#1}
}{%
  \end{table}
}

% Proper figure environment for diagrams
\newenvironment{answerdiagram}[1]{%
  \begin{figure}[htbp]
  \centering
  \caption{#1}
}{%
  \end{figure}
}

% Semantic markup for key terms
\newcommand{\keyword}[1]{\textbf{#1}}
\newcommand{\code}[1]{\texttt{#1}}
\newcommand{\classname}[1]{\texttt{#1}}
\newcommand{\methodname}[1]{\texttt{#1}}

% Proper quotation marks
\newcommand{\mnemonic}[1]{``#1''}


\title{Engineering Chemistry (DI01000071) - Winter 2024 Solution}
\date{January 9, 2025}

\begin{document}
\maketitle

\questionmarks{1}{14}{આપેલ વિકલ્પોમાંથી યોગ્ય વિકલ્પ પસંદ કરી ખાલી જગ્યાઓ પૂરો:}

\begin{solutionbox}
\textbf{જવાબ}:

\begin{center}
\captionof{table}{પ્રશ્ન 1 જવાબો}
\begin{tabulary}{\linewidth}{L L L}
    \toprule
    \textbf{પ્રશ્ન} & \textbf{જવાબ} & \textbf{સમજૂતી} \\
    \midrule
    (1) & [Ar]4s\textsuperscript{1}3d\textsuperscript{10} & Cu માં 29 ઇલેક્ટ્રોન છે, Aufbau નિયમનો અપવાદ \\
    (2) & 14 & pH + pOH = 14 (25°C પર) \\
    (3) & કેથોડ & શુદ્ધ તાંબુ નેગેટિવ ઇલેક્ટ્રોડ પર જમા થાય \\
    (4) & Cu & તાંબુ સુરક્ષિત ઓક્સાઇડ સ્તર બનાવે છે \\
    (5) & અર્ધ-ઘન & પીટ અંશતઃ વિઘટિત કાર્બનિક પદાર્થ છે \\
    (6) & ડ્યુલોંગ & ડ્યુલોંગના સૂત્રથી ઉષ્મીય મૂલ્ય ગણાય \\
    (7) & લિગ્નાઇટ & લિગ્નાઇટમાં સૌથી વધુ ભેજ (35-75\%) \\
    (8) & પોઇઝ & ડાયનેમિક વિસ્કોસિટીનો SI એકમ \\
    (9) & ઊંચું & ઊંચું ફ્લેશ પોઇન્ટ ઇગ્નિશન અટકાવે છે \\
    (10) & પાયસ & તેલ-પાણીનું મિશ્રણ પાયસ બનાવે છે \\
    (11) & બેકેલાઇટ & ફિનોલ ફોર્મેલ્ડિહાઇડ = બેકેલાઇટ \\
    (12) & S & વલ્કેનાઇઝેશન માટે સલ્ફર વપરાય છે \\
    (13) & PHBV & PHBV જૈવવિઘટનીય પોલિમર છે \\
    (14) & વોલ્ટ & EMF વોલ્ટમાં માપાય છે \\
    \bottomrule
\end{tabulary}
\end{center}

\begin{mnemonicbox}
\mnemonic{"રાસાયણિક તાંબુ સુંદર ગુણધર્મો બનાવે"}
\end{mnemonicbox}
\end{solutionbox}

\questionmarks{2(A)(1)}{3}{જુદાં જુદાં ક્ષેત્રોમાં pHની ત્રણ અગત્યતાની સૂચિ બનાવો.}

\begin{solutionbox}
\textbf{જવાબ}:

\begin{center}
\captionof{table}{pH ની અગત્યતા}
\begin{tabulary}{\linewidth}{L L L}
    \toprule
    \textbf{ક્ષેત્ર} & \textbf{મહત્વ} & \textbf{એપ્લિકેશન} \\
    \midrule
    \textbf{દવાશાસ્ત્ર} & લોહીનું pH જાળવણું & સામાન્ય pH 7.35-7.45 યોગ્ય શરીરિક કાર્ય માટે \\
    \textbf{કૃષિ} & માટીનું pH ઓપ્ટિમાઇઝેશન & pH 6-7 પાકની વૃદ્ધિ અને પોષણ માટે આદર્શ \\
    \textbf{ઉદ્યોગ} & ગુણવત્તા નિયંત્રણ & pH ખોરાક, કાપડ, દવાઓની ગુણવત્તાને અસર કરે \\
    \bottomrule
\end{tabulary}
\end{center}

\begin{mnemonicbox}
\mnemonic{"દવા કૃષિ ઉદ્યોગ"}
\end{mnemonicbox}
\end{solutionbox}

\questionmarks{2(A)(2)}{3}{વ્યાખ્યા આપો: બફર દ્રાવણો, અર્ધ-કોષ, વિદ્યુતવિભાજનનો ફેરાડેનો પ્રથમ નિયમ.}

\begin{solutionbox}
\textbf{જવાબ}:

\begin{itemize}
    \item \keyword{બફર દ્રાવણો}: એવા દ્રાવણો જે થોડું એસિડ કે બેઝ ઉમેરવાથી pH બદલાવમાં પ્રતિકાર કરે
    \item \keyword{અર્ધ-કોષ}: એક ઇલેક્ટ્રોડ તેના આયનિક દ્રાવણમાં ડૂબેલો, ઓક્સિડેશન કે રિડક્શન દર્શાવે
    \item \keyword{ફેરાડેનો પ્રથમ નિયમ}: ઇલેક્ટ્રોડ પર જમા/મુક્ત થતા પદાર્થની માત્રા વીજળીની માત્રાના સીધા પ્રમાણમાં હોય
\end{itemize}

\begin{mnemonicbox}
\mnemonic{"બફર મદદ ફેરાડે"}
\end{mnemonicbox}
\end{solutionbox}

\questionmarks{2(A)(3)}{3}{ક્ષારણ દર ઉપર અસર કરતાં પરિબળો જણાવો.}

\begin{solutionbox}
\textbf{જવાબ}:

\begin{center}
\captionof{table}{ક્ષારણ પરિબળો}
\begin{tabulary}{\linewidth}{L L L}
    \toprule
    \textbf{પરિબળ} & \textbf{અસર} & \textbf{વર્ણન} \\
    \midrule
    \textbf{ધાતુની શુદ્ધતા} & વધુ શુદ્ધતા = ઓછું ક્ષારણ & અશુદ્ધિઓ ગેલ્વેનિક કોષ બનાવે \\
    \textbf{તાપમાન} & વધુ તાપમાન = ઝડપી ક્ષારણ & પ્રતિક્રિયા દર વધારે \\
    \textbf{ભેજ} & વધુ ભેજ = વધુ ક્ષારણ & ઇલેક્ટ્રોકેમિકલ પ્રતિક્રિયાઓ પ્રોત્સાહન \\
    \bottomrule
\end{tabulary}
\end{center}

\begin{mnemonicbox}
\mnemonic{"શુદ્ધ તાપમાન ભેજ"}
\end{mnemonicbox}
\end{solutionbox}

\questionmarks{2(B)(1)}{4}{કક્ષાઓ અને કક્ષકો વચ્ચે સરખામણી કરો (દરેકના ચાર મુદ્દાઓ).}

\begin{solutionbox}
\textbf{જવાબ}:

\begin{center}
\captionof{table}{કક્ષાઓ વિ કક્ષકો}
\begin{tabulary}{\linewidth}{L L L}
    \toprule
    \textbf{પાસું} & \textbf{કક્ષાઓ} & \textbf{કક્ષકો} \\
    \midrule
    \textbf{વ્યાખ્યા} & નિશ્ચિત ગોળાકાર માર્ગ & 3D સંભાવના પ્રદેશો \\
    \textbf{આકાર} & ગોળાકાર/અંડાકાર & s,p,d,f આકારો \\
    \textbf{ઊર્જા} & નિશ્ચિત ઊર્જા સ્તરો & ઊર્જા શ્રેણીઓ \\
    \textbf{ઇલેક્ટ્રોન સ્થાન} & ચોક્કસ સ્થિતિ & મળવાની સંભાવના \\
    \bottomrule
\end{tabulary}
\end{center}

\textbf{આકૃતિ:}

\begin{center}
\begin{tikzpicture}
    % Orbits (Bohr Model)
    \node at (-3, 2.5) {\textbf{કક્ષાઓ (બોહર)}};
    \draw (-3,0) circle (1.5cm);
    \draw (-3,0) circle (0.8cm);
    \filldraw (-3,0) circle (2pt) node[below=2pt] {+};
    \filldraw (-3, 1.5) circle (3pt) node[above] {e-};
    \filldraw (-3, -0.8) circle (3pt) node[below] {e-};

    % Orbitals (Quantum)
    \node at (3, 2.5) {\textbf{કક્ષકો (ક્વાન્ટમ)}};
    % Draw some random dots for electron cloud
    \foreach \i in {1,...,200}
        \fill[black!30] (3+rand*1.2, rand*1.2) circle (0.5pt);
    \filldraw (3,0) circle (2pt) node[below=2pt] {+};
    \node at (3, -1.5) {સંભાવના વાદળ};
\end{tikzpicture}
\captionof{figure}{બોહર કક્ષાઓ વિ ક્વાન્ટમ કક્ષકો}
\end{center}

\begin{mnemonicbox}
\mnemonic{"નિશ્ચિત આકાર ઊર્જા સ્થાન"}
\end{mnemonicbox}
\end{solutionbox}

\questionmarks{2(B)(2)}{4}{દરેકના એક ઉદાહરણ સાથે તેના સ્ત્રોતો અને ભૌતિક સ્થિતિઓના આધારે ઇંધણોનું વર્ગીકરણ કરો.}

\begin{solutionbox}
\textbf{જવાબ}:

\begin{center}
\captionof{table}{ઇંધણ વર્ગીકરણ}
\begin{tabulary}{\linewidth}{L L L L}
    \toprule
    \textbf{વર્ગીકરણ} & \textbf{પ્રકાર} & \textbf{ઉદાહરણ} & \textbf{વર્ણન} \\
    \midrule
    \textbf{સ્ત્રોત આધારિત} & કુદરતી & કોલસો & કુદરતી રીતે બન્યું \\
     & કૃત્રિમ & પેટ્રોલ & માનવ નિર્મિત \\
    \midrule
    \textbf{ભૌતિક સ્થિતિ} & ઘન & લાકડું & ઓરડાના તાપમાને ઘન \\
     & પ્રવાહી & ડીઝલ & ઓરડાના તાપમાને પ્રવાહી \\
     & ગેસીય & LPG & ઓરડાના તાપમાને ગેસ \\
    \bottomrule
\end{tabulary}
\end{center}

\begin{mnemonicbox}
\mnemonic{"કુદરતી કૃત્રિમ, ઘન પ્રવાહી ગેસ"}
\end{mnemonicbox}
\end{solutionbox}

\questionmarks{2(B)(3)}{4}{બાયોડીઝલ વિશે ચાર અગત્યના મુદ્દાઓ સમજાવો.}

\begin{solutionbox}
\textbf{જવાબ}:

\begin{itemize}
    \item \keyword{સ્ત્રોત}: વનસ્પતિ તેલ, પ્રાણીઓની ચરબી અથવા વપરાયેલા રસોઈ તેલમાંથી બને
    \item \keyword{પ્રક્રિયા}: મેથેનોલ/ઇથેનોલ સાથે ટ્રાન્સએસ્ટેરિફિકેશન પ્રતિક્રિયાથી બને
    \item \keyword{ગુણધર્મો}: જૈવવિઘટનીય, બિન-ઝેરી, નવીકરણીય ઇંધણ સ્ત્રોત
    \item \keyword{ઉપયોગો}: ડીઝલ એન્જિનમાં વપરાય, ઉત્સર્જન 75\% ઘટાડે
\end{itemize}

\textbf{રાસાયણિક પ્રતિક્રિયા:}
\begin{center}
    વનસ્પતિ તેલ + મેથેનોલ $\rightarrow$ બાયો-ડીઝલ + ગ્લિસેરોલ
\end{center}

\begin{mnemonicbox}
\mnemonic{"સ્ત્રોત પ્રક્રિયા ગુણધર્મો ઉપયોગો"}
\end{mnemonicbox}
\end{solutionbox}

\questionmarks{3(A)(1)}{3}{ઉદાહરણની મદદથી દ્રાવ્ય, દ્રાવક અને દ્રાવણ સમજાવો.}

\begin{solutionbox}
\textbf{જવાબ}:

\begin{center}
\captionof{table}{દ્રાવ્ય, દ્રાવક, દ્રાવણ}
\begin{tabulary}{\linewidth}{L L L}
    \toprule
    \textbf{ઘટક} & \textbf{વ્યાખ્યા} & \textbf{ઉદાહરણ} \\
    \midrule
    \textbf{દ્રાવ્ય} & જે પદાર્થ ઓગળે છે & મીઠું (NaCl) \\
    \textbf{દ્રાવક} & જેમાં પદાર્થ ઓગળે છે & પાણી (H\textsubscript{2}O) \\
    \textbf{દ્રાવણ} & સમાંગી મિશ્રણ & મીઠાનું પાણી \\
    \bottomrule
\end{tabulary}
\end{center}

\textbf{ઉદાહરણ}: ખાંડ + પાણી = ખાંડનું દ્રાવણ
\begin{itemize}
    \item ખાંડ = દ્રાવ્ય, પાણી = દ્રાવક, ખાંડનું પાણી = દ્રાવણ
\end{itemize}

\begin{mnemonicbox}
\mnemonic{"દ્રાવ્ય દ્રાવક દ્રાવણ"}
\end{mnemonicbox}
\end{solutionbox}

\questionmarks{3(A)(2)}{3}{NaClમાં વિદ્યુતસંયોજક બંધનું નિર્માણ સમજાવો.}

\begin{solutionbox}
\textbf{જવાબ}:

\textbf{પ્રક્રિયા}:
\begin{itemize}
    \item \keyword{પગલું 1}: Na એક ઇલેક્ટ્રોન ગુમાવે $\rightarrow$ Na\textsuperscript{+} (કેટાયન)
    \item \keyword{પગલું 2}: Cl એક ઇલેક્ટ્રોન મેળવે $\rightarrow$ Cl\textsuperscript{-} (આયન)
    \item \keyword{પગલું 3}: Na\textsuperscript{+} અને Cl\textsuperscript{-} વચ્ચે વિદ્યુતસ્થિતિક આકર્ષણ
\end{itemize}

\textbf{આકૃતિ:}

\begin{center}
\begin{tikzpicture}
    \node (Na) at (0,0) {Na};
    \node (Cl) at (3,0) {Cl};
    \draw[->, bend left] (Na) to node[midway, above] {$e^-$} (Cl);
    
    \node (NaIon) at (5,0) {Na$^+$};
    \node (ClIon) at (6,0) {Cl$^-$};
    \node at (4,0) {$\rightarrow$};
    
    \draw[dashed] (NaIon) -- (ClIon);
    \node at (5.5, -0.5) {વિદ્યુતસ્થિતિક આકર્ષણ};
\end{tikzpicture}
\captionof{figure}{NaCl બંધ રચના}
\end{center}

\begin{mnemonicbox}
\mnemonic{"સોડિયમ ગુમાવે, ક્લોરિન મેળવે, આકર્ષણ બને"}
\end{mnemonicbox}
\end{solutionbox}

\questionmarks{3(A)(3)}{3}{ગેસોલીન માટે ઓક્ટેન આંક સમજાવો.}

\begin{solutionbox}
\textbf{જવાબ}:

\begin{center}
\captionof{table}{ઓક્ટેન આંક}
\begin{tabulary}{\linewidth}{L L}
    \toprule
    \textbf{પાસું} & \textbf{વર્ણન} \\
    \midrule
    \textbf{વ્યાખ્યા} & ઇંધણની નોકિંગ સામે પ્રતિકારશક્તિનું માપ \\
    \textbf{સ્કેલ} & 0-100, વધુ = વધુ સારી એન્ટી-નોક ગુણવત્તા \\
    \textbf{માનક} & n-હેપ્ટેન = 0, આઇસો-ઓક્ટેન = 100 \\
    \bottomrule
\end{tabulary}
\end{center}

\textbf{ઉપયોગો}: ઊંચા ઓક્ટેન ઇંધણ એન્જિન નોકિંગ અટકાવે, કામગીરી સુધારે

\begin{mnemonicbox}
\mnemonic{"ઓક્ટેન નોકિંગ વિરોધી"}
\end{mnemonicbox}
\end{solutionbox}

\questionmarks{3(B)(1)}{4}{અશુદ્ધ Cuનું વિદ્યુતશુદ્ધિકરણ રાસાયણિક સમીકરણો અને નામ નિર્દેશનવાળી આકૃતિ સાથે સમજાવો.}

\begin{solutionbox}
\textbf{જવાબ}:

\textbf{પ્રક્રિયા}:
\begin{itemize}
    \item \keyword{એનોડ}: અશુદ્ધ તાંબુ ઓગળે
    \item \keyword{કેથોડ}: શુદ્ધ તાંબુ જમા થાય
    \item \keyword{ઇલેક્ટ્રોલાઇટ}: CuSO\textsubscript{4} દ્રાવણ
\end{itemize}

\textbf{રાસાયણિક સમીકરણો}:
\begin{itemize}
    \item એનોડ પર: Cu $\rightarrow$ Cu\textsuperscript{2+} + 2e\textsuperscript{-}
    \item કેથોડ પર: Cu\textsuperscript{2+} + 2e\textsuperscript{-} $\rightarrow$ Cu
\end{itemize}

\textbf{આકૃતિ:}

\begin{center}
\begin{tikzpicture}
    % Container
    \draw[thick] (0,0) rectangle (6,4);
    \fill[blue!10] (0.2,0.2) rectangle (5.8,3.5);
    \node at (3,1) {CuSO$_4$ દ્રાવણ};
    
    % Electrodes
    \draw[fill=gray!30] (1,2) rectangle (1.5,4.5); % Cathode
    \node[above] at (1.25,4.5) {કેથોડ (શુદ્ધ)};
    \node at (1.25, 3.8) {-};
    
    \draw[fill=orange!80] (4.5,2) rectangle (5.5,4.5); % Anode (Thick)
    \node[above] at (5,4.5) {એનોડ (અશુદ્ધ)};
    \node at (5, 3.8) {+};
    
    % Circuit
    \draw (1.25,4.5) -- (1.25,5) -- (5,5) -- (5,4.5);
    \node[draw, rectangle, fill=white] at (3.125, 5) {બેટરી};
    
    % Ions
    \draw[->] (4.5, 2.5) -- (1.5, 2.5);
    \node at (3, 2.7) {Cu$^{2+}$};
\end{tikzpicture}
\captionof{figure}{તાંબાનું શુદ્ધિકરણ}
\end{center}

\begin{mnemonicbox}
\mnemonic{"એનોડ ઓગળે, કેથોડ જમાવે"}
\end{mnemonicbox}
\end{solutionbox}

\questionmarks{3(B)(2)}{4}{રાસાયણિક સમીકરણ સાથે ઇથિનની બનાવટ સમજાવો. તેના બે ગુણધર્મો અને બે ઉપયોગો પણ લખો.}

\begin{solutionbox}
\textbf{જવાબ}:

\textbf{તૈયારી}:
\begin{center}
    C\textsubscript{2}H\textsubscript{5}OH $\xrightarrow{\Delta}$ C\textsubscript{2}H\textsubscript{4} + H\textsubscript{2}O (ઇથેનોલનું નિર્જલીકરણ)
\end{center}

\textbf{ગુણધર્મો}:
\begin{itemize}
    \item \keyword{ભૌતિક}: રંગહીન ગેસ, મીઠી સુગંધ
    \item \keyword{રાસાયણિક}: અસંતૃપ્ત, ઉમેરણ પ્રતિક્રિયાઓ કરે
\end{itemize}

\textbf{ઉપયોગો}:
\begin{itemize}
    \item \keyword{ઔદ્યોગિક}: પોલિઇથિલીન ઉત્પાદન
    \item \keyword{કૃષિ}: ફળ પકવવા માટે વનસ્પતિ હોર્મોન
\end{itemize}

\begin{mnemonicbox}
\mnemonic{"તૈયારી ગુણધર્મો ઉપયોગો"}
\end{mnemonicbox}
\end{solutionbox}

\questionmarks{3(B)(3)}{4}{રાસાયણિક સમીકરણ સાથે Buna-S રબરની બનાવટ સમજાવો. તેના બે ગુણધર્મો અને બે ઉપયોગો પણ લખો.}

\begin{solutionbox}
\textbf{જવાબ}:

\textbf{તૈયારી}:
બ્યુટાડાયન + સ્ટાયરીન $\rightarrow$ Buna-S રબર (કોપોલિમેરાઇઝેશન)

\textbf{રાસાયણિક સમીકરણ}:
\begin{center}
    nC\textsubscript{4}H\textsubscript{6} + nC\textsubscript{8}H\textsubscript{8} $\rightarrow$ [-C\textsubscript{4}H\textsubscript{6}-C\textsubscript{8}H\textsubscript{8}-]\textsubscript{n}
\end{center}

\textbf{ગુણધર્મો}:
\begin{itemize}
    \item \keyword{યાંત્રિક}: સારો ઘર્ષણ પ્રતિકાર
    \item \keyword{રાસાયણિક}: તેલ અને ઇંધણ પ્રતિરોધી
\end{itemize}

\textbf{ઉપયોગો}:
\begin{itemize}
    \item \keyword{વાહન}: ટાયર ઉત્પાદન
    \item \keyword{ઔદ્યોગિક}: કન્વેયર બેલ્ટ, હોઝ
\end{itemize}

\begin{mnemonicbox}
\mnemonic{"બ્યુટાડાયન સ્ટાયરીન મજબૂત રબર બનાવે"}
\end{mnemonicbox}
\end{solutionbox}

\questionmarks{4(A)(1)}{3}{ધાતુઓનું ક્ષારણ નિવારવા ધાતુક્લેડિંગ સમજાવો.}

\begin{solutionbox}
\textbf{જવાબ}:

\begin{center}
\captionof{table}{ધાતુક્લેડિંગ}
\begin{tabulary}{\linewidth}{L L}
    \toprule
    \textbf{પાસું} & \textbf{વર્ણન} \\
    \midrule
    \textbf{પ્રક્રિયા} & મૂળ ધાતુ પર ક્ષારણ-પ્રતિરોધી ધાતુનું આવરણ \\
    \textbf{પદ્ધતિઓ} & હોટ ડિપિંગ, ઇલેક્ટ્રોપ્લેટિંગ, રોલ બોન્ડિંગ \\
    \textbf{ઉદાહરણો} & ગેલ્વેનાઇઝ્ડ આયર્ન (Fe પર Zn), ટીન પ્લેટિંગ \\
    \bottomrule
\end{tabulary}
\end{center}

\textbf{મિકેનિઝમ}: સુરક્ષિત સ્તર મૂળ ધાતુને ઓક્સિજન/ભેજના સંપર્કમાં આવતું અટકાવે

\begin{mnemonicbox}
\mnemonic{"આવરણ ધાતુ સુરક્ષિત કરે"}
\end{mnemonicbox}
\end{solutionbox}

\questionmarks{4(A)(2)}{3}{પાણીની સપાટી નીચે થતું ક્ષારણ રાસાયણિક પ્રક્રિયાઓ અને નામનિર્દેશનવાળી આકૃતિ સાથે સમજાવો.}

\begin{solutionbox}
\textbf{જવાબ}:

\textbf{પ્રક્રિયા}: વિભેદક વાયુકરણ પાણી-હવા સંપર્ક સ્થળે ક્ષારણ કારણે

\textbf{રાસાયણિક સમીકરણો}:
\begin{itemize}
    \item એનોડ: Fe $\rightarrow$ Fe\textsuperscript{2+} + 2e\textsuperscript{-}
    \item કેથોડ: O\textsubscript{2} + 4H\textsuperscript{+} + 4e\textsuperscript{-} $\rightarrow$ 2H\textsubscript{2}O
\end{itemize}

\textbf{આકૃતિ:}

\begin{center}
\begin{tikzpicture}
    % Water and Air regions
    \fill[cyan!10] (-2, -2) rectangle (2, 0); % Water
    \node at (1.5, -1.8) {પાણી};
    \node at (1.5, 0.5) {હવા};
    
    % Metal rod
    \fill[gray!40] (-0.5, -1.8) rectangle (0.5, 0.2);
    \draw (-0.5, -1.8) rectangle (0.5, 0.2);
    \node at (0, -1) {Fe};

    % Waterline
    \draw[blue, thick] (-2, 0) -- (2, 0);
    
    % Corrosion spots
    \node[right] at (0.5, 0) {કેથોડ (વધુ O$_2$)};
    \node[right] at (0.5, -0.5) {એનોડ (ઓછું O$_2$)};
    
    \draw[->] (0.6, -0.5) -- (0.1, -0.5);
    \fill[red] (0.5, -0.5) circle (2pt); % Corrosion
\end{tikzpicture}
\captionof{figure}{પાણીની સપાટીનું ક્ષારણ}
\end{center}

\begin{mnemonicbox}
\mnemonic{"પાણી હવા સંપર્ક ક્ષારણ કરે"}
\end{mnemonicbox}
\end{solutionbox}

\questionmarks{4(A)(3)}{3}{સૌર કોષોના કાર્યકારી સિદ્ધાંતને સમજાવો.}

\begin{solutionbox}
\textbf{જવાબ}:

\begin{center}
\captionof{table}{સૌર કોષ સિદ્ધાંત}
\begin{tabulary}{\linewidth}{L L}
    \toprule
    \textbf{ઘટક} & \textbf{કાર્ય} \\
    \midrule
    \textbf{ફોટોવોલ્ટેઇક અસર} & પ્રકાશ ઊર્જા વિદ્યુત ઊર્જામાં ફેરવાય \\
    \textbf{p-n જંકશન} & ચાર્જ વિભાજન માટે વિદ્યુત ક્ષેત્ર બનાવે \\
    \textbf{ઇલેક્ટ્રોન-હોલ જોડી} & ફોટોન સેમિકન્ડક્ટર સાથે અથડાય ત્યારે બને \\
    \bottomrule
\end{tabulary}
\end{center}

\textbf{પ્રક્રિયા}: પ્રકાશ $\rightarrow$ ઇલેક્ટ્રોન ઉત્તેજના $\rightarrow$ પ્રવાહ $\rightarrow$ વિદ્યુત ઊર્જા

\begin{mnemonicbox}
\mnemonic{"ફોટો વોલ્ટેઇક જંકશન પ્રવાહ બનાવે"}
\end{mnemonicbox}
\end{solutionbox}

\questionmarks{4(B)(1)}{4}{આકૃતિ સાથે સીમાવર્તી સ્નેહનનું કાર્ય દર્શાવો.}

\begin{solutionbox}
\textbf{જવાબ}:

\textbf{કાર્ય}: પાતળો આણવિક સ્તર ધાતુની સપાટી પર ચોંટે, સીધો સંપર્ક અટકાવે

\textbf{મિકેનિઝમ}:
\begin{itemize}
    \item \keyword{રચના}: સ્નેહક અણુઓ ધાતુની સપાટી પર ગોઠવાય
    \item \keyword{સુરક્ષા}: સપાટીઓ વચ્ચે ઘર્ષણ અને ઘસારો ઘટાડે
    \item \keyword{લોડ બેરિંગ}: પ્રવાહી ફિલ્મ તૂટે ત્યારે લોડ સહન કરે
\end{itemize}

\textbf{આકૃતિ:}

\begin{center}
\begin{tikzpicture}
    % Surfaces
    \draw[thick] (0,3) -- (6,3); % Top surface
    \draw[thick] (0,0) -- (6,0); % Bottom surface
    
    % Aspirities
    \draw[fill=gray!20] (0,3) -- (1,2.5) -- (2,3) -- (3,2.5) -- (4,3) -- (5,2.5) -- (6,3);
    \draw[fill=gray!20] (0,0) -- (1,0.5) -- (2,0) -- (3,0.5) -- (4,0) -- (5,0.5) -- (6,0);
    
    % Lubricant Molecules (Circles)
    \foreach \x in {0.5, 1, ..., 5.5} {
        \draw[fill=orange] (\x, 1) circle (0.1);
        \draw[fill=orange] (\x, 1.3) circle (0.1);
        \draw[fill=orange] (\x, 1.6) circle (0.1);
        \draw[fill=orange] (\x, 1.9) circle (0.1);
    }
    \node at (3, 1.5) {સીમા સ્તર};
    
    % Motion arrows
    \draw[->, thick] (7, 3) -- (8, 3) node[right] {ગતિ};
    \draw[thick] (7, 0) -- (8, 0) node[right] {સ્થિર};
\end{tikzpicture}
\captionof{figure}{સીમાવર્તી સ્નેહન}
\end{center}

\begin{mnemonicbox}
\mnemonic{"સીમા અવરોધ ધાતુ સંપર્ક અટકાવે"}
\end{mnemonicbox}
\end{solutionbox}

\questionmarks{4(B)(2)}{4}{રેડવુડ વિસ્કોમીટર દ્વારા સિનગ્ધતા કેવી રીતે માપવામાં આવે છે તે નામનિર્દેશનવાળી આકૃતિ સાથે સમજાવો.}

\begin{solutionbox}
\textbf{જવાબ}:

\textbf{સિદ્ધાંત}: નિશ્ચિત કદના છિદ્રમાંથી નિશ્ચિત પ્રમાણ તેલ વહેવામાં લાગતો સમય

\textbf{કાર્યવિધિ}:
\begin{itemize}
    \item \keyword{સેટઅપ}: તેલ ચેમ્બર ભરો, જરૂરી તાપમાને ગરમ કરો
    \item \keyword{માપ}: 50ml તેલ વહેવાનો સમય નોંધો
    \item \keyword{ગણતરી}: વિસ્કોસિટી = સમય $\times$ સ્થિરાંક
\end{itemize}

\textbf{આકૃતિ:}

\begin{center}
\begin{tikzpicture}
    % Water Bath
    \draw[thick] (0,0) rectangle (4,4);
    \node at (3.5, 3.5) {વોટર બાથ};
    
    % Oil Cup
    \fill[orange!20] (1,1) rectangle (3,3.5);
    \draw[thick] (1,3.5) -- (1,1) -- (1.8,1) -- (1.8,0.5) -- (2.2,0.5) -- (2.2,1) -- (3,1) -- (3,3.5);
    \node at (2,2) {તેલ};
    
    % Orifice
    \draw[thick] (1.9, 0.5) -- (2.1, 0.5);
    \node[right] at (2.2, 0.7) {છિદ્ર};
    
    % Flask
    \draw (1.5, -2) -- (1.5, -0.5) -- (2.5, -0.5) -- (2.5, -2) -- cycle;
    \node at (2, -1.25) {50ml ફ્લાસ્ક};
    
    % Stream
    \draw[dashed, orange] (2, 0.5) -- (2, -0.5);
\end{tikzpicture}
\captionof{figure}{રેડવુડ વિસ્કોમીટર}
\end{center}

\begin{mnemonicbox}
\mnemonic{"રેડવુડ સમય નોંધે"}
\end{mnemonicbox}
\end{solutionbox}

\questionmarks{4(B)(3)}{4}{વ્યાખ્યા આપો: અર્ધવાહક, અવાહક પદાર્થ, સ્થિતિસ્થાપક પદાર્થ, યોગશીલ બહુલીભવન.}

\begin{solutionbox}
\textbf{જવાબ}:

\begin{center}
\captionof{table}{વ્યાખ્યાઓ}
\begin{tabulary}{\linewidth}{L L}
    \toprule
    \textbf{શબ્દ} & \textbf{વ્યાખ્યા} \\
    \midrule
    \textbf{અર્ધવાહક} & વાહક અને અવાહક વચ્ચેની વિદ્યુત વાહકતા ધરાવતો પદાર્થ \\
    \textbf{અવાહક પદાર્થ} & વિદ્યુત પ્રવાહના વહેણને પ્રતિકાર કરતો પદાર્થ \\
    \textbf{સ્થિતિસ્થાપક પદાર્થ} & લવચીક ગુણધર્મો ધરાવતો પોલિમર, ખેંચાઈને મૂળ આકારે પાછો આવે \\
    \textbf{યોગશીલ બહુલીભવન} & મોનોમર્સ નાના અણુઓ દૂર કર્યા વિના જોડાય \\
    \bottomrule
\end{tabulary}
\end{center}

\textbf{ઉદાહરણો}: Si (અર્ધવાહક), રબર (અવાહક), રબર (સ્થિતિસ્થાપક), પોલિઇથિલીન (યોગશીલ)

\begin{mnemonicbox}
\mnemonic{"અર્ધ અવાહક સ્થિતિ યોગશીલ"}
\end{mnemonicbox}
\end{solutionbox}

\questionmarks{5(A)(1)}{3}{ઉકેલો: 0.004 M HClના જલીય દ્રાવણની pH અને pOH ગણો. (log 4 = 0.6021)}

\begin{solutionbox}
\textbf{જવાબ}:

\textbf{આપેલ}: [HCl] = 0.004 M = 4 $\times$ 10\textsuperscript{-3} M

\textbf{ઉકેલ}:
\begin{itemize}
    \item HCl મજબૂત એસિડ છે, સંપૂર્ણ આયનીકરણ થાય
    \item [H\textsuperscript{+}] = [HCl] = 4 $\times$ 10\textsuperscript{-3} M
    \item pH = -log[H\textsuperscript{+}] = -log(4 $\times$ 10\textsuperscript{-3})
    \item pH = -log 4 - log 10\textsuperscript{-3} = -0.6021 + 3 = 2.398
    \item pOH = 14 - pH = 14 - 2.398 = 11.602
\end{itemize}

\textbf{જવાબ}: pH = 2.40, pOH = 11.60

\begin{mnemonicbox}
\mnemonic{"મજબૂત એસિડ, સરળ ગણતરી"}
\end{mnemonicbox}
\end{solutionbox}

\questionmarks{5(A)(2)}{3}{ઉદાહરણ સાથે બાહ્ય અર્ધવાહકો અને તેના પ્રકારો વર્ણવો.}

\begin{solutionbox}
\textbf{જવાબ}:

\begin{center}
\captionof{table}{બાહ્ય અર્ધવાહકો}
\begin{tabulary}{\linewidth}{L L L L}
    \toprule
    \textbf{પ્રકાર} & \textbf{ડોપન્ટ} & \textbf{મુખ્ય વાહકો} & \textbf{ઉદાહરણ} \\
    \midrule
    \textbf{n-પ્રકાર} & દાતા અણુઓ (ગ્રૂપ V) & ઇલેક્ટ્રોન & Si + P \\
    \textbf{p-પ્રકાર} & સ્વીકર્તા અણુઓ (ગ્રૂપ III) & હોલ્સ & Si + B \\
    \bottomrule
\end{tabulary}
\end{center}

\textbf{ગુણધર્મો}:
\begin{itemize}
    \item \keyword{n-પ્રકાર}: વધારાના ઇલેક્ટ્રોન વાહકતા વધારે
    \item \keyword{p-પ્રકાર}: ઇલેક્ટ્રોન અછત સકારાત્મક હોલ્સ બનાવે
\end{itemize}

\begin{mnemonicbox}
\mnemonic{"n-નેગેટિવ ઇલેક્ટ્રોન, p-પોઝિટિવ હોલ્સ"}
\end{mnemonicbox}
\end{solutionbox}

\questionmarks{5(A)(3)}{3}{ઉષ્માસહ બહુલક અને ઉષ્માસ્થાપિત બહુલક વચ્ચેનાં ફરક આપો. (દરેકનાં ચાર મુદ્દાઓ)}

\begin{solutionbox}
\textbf{જવાબ}:

\begin{center}
\captionof{table}{ઉષ્માસહ વિ ઉષ્માસ્થાપિત}
\begin{tabulary}{\linewidth}{L L L}
    \toprule
    \textbf{ગુણધર્મ} & \textbf{ઉષ્માસહ} & \textbf{ઉષ્માસ્થાપિત} \\
    \midrule
    \textbf{રચના} & રેખીય/શાખાવાળી સાંકળો & ક્રોસ-લિંક્ડ નેટવર્ક \\
    \textbf{ગરમીની અસર} & ગરમ કરવાથી નરમ પડે & નરમ નથી પડતું \\
    \textbf{પુનઃઉપયોગ} & પુનઃઉપયોગ શક્ય & પુનઃઉપયોગ અશક્ય \\
    \textbf{ઉદાહરણો} & PVC, PE, PS & બેકેલાઇટ, ઇપોક્સી \\
    \bottomrule
\end{tabulary}
\end{center}

\begin{mnemonicbox}
\mnemonic{"ઉષ્મા-સહ = પુનઃઉપયોગ, ઉષ્મા-સ્થાપિત = કાયમી"}
\end{mnemonicbox}
\end{solutionbox}

\questionmarks{5(B)(1)}{4}{હાઇડ્રોજન બંધ અને તેના પ્રકારો ઉદાહરણો સાથે વર્ણવો.}

\begin{solutionbox}
\textbf{જવાબ}:

\textbf{વ્યાખ્યા}: હાઇડ્રોજન અને વિદ્યુતનેગેટિવ અણુઓ વચ્ચે નબળું વિદ્યુતસ્થિતિક આકર્ષણ

\textbf{પ્રકારો}:
\begin{center}
\captionof{table}{હાઇડ્રોજન બંધ પ્રકારો}
\begin{tabulary}{\linewidth}{L L L}
    \toprule
    \textbf{પ્રકાર} & \textbf{વર્ણન} & \textbf{ઉદાહરણ} \\
    \midrule
    \textbf{અંતરઅણવિક} & વિવિધ અણુઓ વચ્ચે & H\textsubscript{2}O$\cdot\cdot\cdot$H\textsubscript{2}O \\
    \textbf{અંતઃઅણવિક} & સમાન અણુમાં & o-નાઇટ્રોફિનોલ \\
    \bottomrule
\end{tabulary}
\end{center}

\textbf{લક્ષણો}:
\begin{itemize}
    \item \keyword{તાકાત}: 5-40 kJ/mol
    \item \keyword{જરૂરિયાતો}: H, F, O, N સાથે જોડાયેલ
\end{itemize}

\textbf{આકૃતિ:}

\begin{center}
\begin{tikzpicture}
    \node (O1) at (0,0) {O};
    \node (H1a) at (-0.5,-0.5) {H};
    \node (H1b) at (0.5,-0.5) {H};
    \draw (O1) -- (H1a);
    \draw (O1) -- (H1b);
    \node [above=1pt] at (H1b) {$\delta^+$};
    \node [above=1pt] at (O1) {$\delta^-$};
    
    \node (O2) at (2,0) {O};
    \node (H2a) at (1.5,-0.5) {H};
    \node (H2b) at (2.5,-0.5) {H};
    \draw (O2) -- (H2a);
    \draw (O2) -- (H2b);
    \node [above=1pt] at (O2) {$\delta^-$};
    
    \draw[dashed, red, thick] (H1b) -- (O2);
    \node[red, above] at (1.25, -0.25) {H-બંધ};
\end{tikzpicture}
\captionof{figure}{પાણીમાં હાઇડ્રોજન બંધ}
\end{center}

\begin{mnemonicbox}
\mnemonic{"હાઇડ્રોજનને FON મિત્રોની જરૂર" (ફ્લોરિન, ઓક્સિજન, નાઇટ્રોજન)}
\end{mnemonicbox}
\end{solutionbox}

\questionmarks{5(B)(2)}{4}{પ્રાથમિક કોષ અને દ્વિતીયક કોષ વચ્ચે તફાવત કરો. (ચાર મુદ્દાઓ)}

\begin{solutionbox}
\textbf{જવાબ}:

\begin{center}
\captionof{table}{પ્રાથમિક વિ દ્વિતીયક કોષ}
\begin{tabulary}{\linewidth}{L L L}
    \toprule
    \textbf{પાસું} & \textbf{પ્રાથમિક કોષ} & \textbf{દ્વિતીયક કોષ} \\
    \midrule
    \textbf{રિચાર્જેબિલિટી} & રિચાર્જ ન થાય & રિચાર્જ થાય \\
    \textbf{પ્રતિક્રિયા} & અપરિવર્તનીય & પરિવર્તનીય \\
    \textbf{કિંમત} & ઓછી શરૂઆતી કિંમત & વધુ શરૂઆતી કિંમત \\
    \textbf{ઉદાહરણો} & ડ્રાય સેલ, અલ્કલાઇન & લેડ-એસિડ, Li-ion \\
    \bottomrule
\end{tabulary}
\end{center}

\textbf{ઉપયોગો}:
\begin{itemize}
    \item \keyword{પ્રાથમિક}: રિમોટ કંટ્રોલ, ફ્લેશલાઇટ
    \item \keyword{દ્વિતીયક}: કાર, ફોન, લેપટોપ
\end{itemize}

\begin{mnemonicbox}
\mnemonic{"પ્રાથમિક = કાયમી, દ્વિતીયક = પરિવર્તનીય"}
\end{mnemonicbox}
\end{solutionbox}

\questionmarks{5(B)(3)}{4}{નામનિર્દેશવાળી આકૃતિ દોરી લેડ-એસિડ સંગ્રાહક કોષની રચના, કાર્ય અને રાસાયણિક સમીકરણો વર્ણવો.}

\begin{solutionbox}
\textbf{જવાબ}:

\textbf{રચના}:
\begin{itemize}
    \item \keyword{એનોડ}: લેડ (Pb)
    \item \keyword{કેથોડ}: લેડ ડાયઓક્સાઇડ (PbO\textsubscript{2})
    \item \keyword{ઇલેક્ટ્રોલાઇટ}: પાતળું H\textsubscript{2}SO\textsubscript{4}
\end{itemize}

\textbf{રાસાયણિક સમીકરણો}:
\begin{itemize}
    \item \keyword{ડિસ્ચાર્જ}: Pb + PbO\textsubscript{2} + 2H\textsubscript{2}SO\textsubscript{4} $\rightarrow$ 2PbSO\textsubscript{4} + 2H\textsubscript{2}O
    \item \keyword{ચાર્જ}: 2PbSO\textsubscript{4} + 2H\textsubscript{2}O $\rightarrow$ Pb + PbO\textsubscript{2} + 2H\textsubscript{2}SO\textsubscript{4}
\end{itemize}

\textbf{આકૃતિ}:

\begin{center}
\begin{tikzpicture}
    % Container
    \draw[thick] (0,0) rectangle (5,4);
    \fill[gray!10] (0.2,0.2) rectangle (4.8,3.5);
    \node at (2.5, 1) {H$_2$SO$_4$ ઇલેક્ટ્રોલાઇટ};
    
    % Plates
    \foreach \x in {1, 2, 3, 4} {
        \draw[fill=gray!50] (\x-0.2, 1.5) rectangle (\x+0.2, 4.5);
    }
    
    % Terminals
    \node[above] at (1, 4.5) {Pb};
    \node[above] at (2, 4.5) {PbO$_2$};
    \node[above] at (3, 4.5) {Pb};
    \node[above] at (4, 4.5) {PbO$_2$};
    
    % Bus bars
    \draw (1,4.2) -- (3,4.2) -- (3,4.8) node[above] {- એનોડ};
    \draw (2,4.3) -- (4,4.3) -- (4,4.8) node[above] {+ કેથોડ};

\end{tikzpicture}
\captionof{figure}{લેડ-એસિડ બેટરી}
\end{center}

\textbf{કાર્ય}: ડિસ્ચાર્જ દરમિયાન રાસાયણિક ઊર્જા વિદ્યુત ઊર્જામાં ફેરવાય

\begin{mnemonicbox}
\mnemonic{"લેડ એસિડ સ્ટોરેજ = પરિવર્તનીય ઊર્જા"}
\end{mnemonicbox}
\end{solutionbox}

\end{document}
