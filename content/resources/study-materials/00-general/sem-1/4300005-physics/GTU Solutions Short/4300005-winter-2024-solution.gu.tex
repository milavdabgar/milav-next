\documentclass{article}

% content/resources/templates/preamble.tex
\usepackage[margin=0.6in]{geometry}
\author{Milav Dabgar}
\usepackage{amsmath,amssymb,amsthm}
\usepackage{booktabs}
\usepackage{multirow}
\usepackage{xcolor}
\usepackage{tcolorbox}
\tcbuselibrary{breakable,skins}
\usepackage[colorlinks=true,linkcolor=blue]{hyperref}
\usepackage{titlesec}
\usepackage{enumitem}
\usepackage{tikz}
\usepackage{pgfplots}
\usepackage{circuitikz}
\usepackage[version=4]{mhchem}
\usepackage{longtable}
\usepackage{array}
\usepackage{float}
\usepackage{caption}
\usepackage{listings}

\lstset{
  basicstyle=\small\ttfamily,
  breaklines=true,
  breakatwhitespace=false,
  postbreak=\mbox{\textcolor{red}{$\hookrightarrow$}\space},
  float=false,
  numbers=left,
  numberstyle=\tiny\color{gray},
  numbersep=10pt,
  xleftmargin=2em,
  keywordstyle=\color{blue},
  commentstyle=\color{green!60!black},
  stringstyle=\color{purple},
  backgroundcolor=\color{gray!5},
  showstringspaces=false,
  tabsize=2,
  captionpos=b,
  keepspaces=true,
  columns=flexible
}

\pgfplotsset{compat=1.18}
\usetikzlibrary{shapes,arrows,positioning,calc,patterns,decorations.pathmorphing,decorations.markings,arrows.meta}

% Color scheme
\definecolor{headcolor}{RGB}{0,102,204}
\definecolor{keycolor}{RGB}{220,20,60}
\definecolor{solutioncolor}{RGB}{34,139,34}
\definecolor{mnemoniccolor}{RGB}{148,0,211}
\definecolor{codecolor}{RGB}{0,0,100}

% Spacing
\setlength{\parskip}{3pt}
\setlist[itemize]{nosep}
\setlist[enumerate]{nosep}

% Title formatting
\titleformat{\section}{\Large\bfseries\color{headcolor}}{\thesection}{1em}{}
\titleformat{\subsection}{\large\bfseries\color{headcolor}}{\thesubsection}{1em}{}

% Pandoc tightlist compatibility
\providecommand{\tightlist}{%
  \setlength{\itemsep}{0pt}\setlength{\parskip}{0pt}}

% Pandoc longtable compatibility
\newcounter{none}
\def\thenone{}


% content/resources/templates/gujarati-boxes.tex
\usepackage{fontspec}
\usepackage{polyglossia}

% Set Gujarati as main language (document is primarily in Gujarati)
% Note: gloss-gujarati.ldf doesn't exist in polyglossia, but it will use hyphenation patterns
\setdefaultlanguage{gujarati}
\setotherlanguage{english}

% Configure Gujarati font properly
% Use Language=Default to prevent polyglossia from trying to add language-specific features
% that don't exist for Gujarati, which causes "empty feature" warnings
\newfontfamily\gujaratifont[Script=Gujarati,AutoFakeBold=2.5,AutoFakeSlant=0.3]{Noto Sans Gujarati}
\setmainfont[Script=Gujarati,AutoFakeBold=2.5,AutoFakeSlant=0.3]{Noto Sans Gujarati}
% Use Noto Sans Gujarati for monospace to support Gujarati in text
\setmonofont[Scale=0.9]{Noto Sans Gujarati}

% Configure English to use the same font
\newfontfamily\englishfont[Script=Gujarati,AutoFakeBold=2.5,AutoFakeSlant=0.3]{Noto Sans Gujarati}

% Translations for polyglossia
\gappto\captionsgujarati{
  \renewcommand{\tablename}{કોષ્ટક}
  \renewcommand{\figurename}{આકૃતિ}
}

% Helper for TikZ nodes to ensure Gujarati font
\newcommand{\gu}[1]{{\gujaratifont #1}}

% Custom environments
\newtcolorbox{solutionbox}{
    breakable,
    enhanced,
    colback=solutioncolor!5!white,
    colframe=solutioncolor!75!black,
    fonttitle=\bfseries,
    title=જવાબ
}

\newtcolorbox{solutionboxnobreak}{
 colback=solutioncolor!5!white,
 colframe=solutioncolor!75!black,
 fonttitle=\bfseries,
 title=જવાબ
}

\newtcolorbox{keyformula}{
 breakable,
 enhanced,
 colback=keycolor!5!white,
 colframe=keycolor!75!black,
 fonttitle=\bfseries,
 title=રાસાયણિક સમીકરણ/સૂત્ર
}

\newtcolorbox{mnemonicbox}{
 breakable,
 enhanced,
 colback=mnemoniccolor!5!white,
 colframe=mnemoniccolor!75!black,
 fonttitle=\bfseries,
 title=મેમરી ટ્રીક
}


% Custom commands for GTU solutions
% This file defines semantic commands for consistent formatting

% Question command with automatic formatting
\newcommand{\question}[2]{%
  \section*{Question #1}%
  \textbf{#2}%
}

% OR question variant
\newcommand{\questionor}[2]{%
  \section*{Question #1 OR}%
  \textbf{#2}%
}

% Proper table environment with caption
\newenvironment{answertable}[1]{%
  \begin{table}[htbp]
  \centering
  \caption{#1}
}{%
  \end{table}
}

% Proper figure environment for diagrams
\newenvironment{answerdiagram}[1]{%
  \begin{figure}[htbp]
  \centering
  \caption{#1}
}{%
  \end{figure}
}

% Semantic markup for key terms
\newcommand{\keyword}[1]{\textbf{#1}}
\newcommand{\code}[1]{\texttt{#1}}
\newcommand{\classname}[1]{\texttt{#1}}
\newcommand{\methodname}[1]{\texttt{#1}}

% Proper quotation marks
\newcommand{\mnemonic}[1]{``#1''}


\title{ભૌતિકશાસ્ત્ર (4300005) - શિયાળુ 2024 સોલ્યુશન}
\date{જાન્યુઆરી 7, 2025}

\begin{document}
\maketitle

\questionmarks{1(a)}{3}{ચોકસાઈ અને સચોટતા વ્યાખ્યાયિત કરો.}

\begin{solutionbox}
\begin{itemize}
    \item \textbf{ચોકસાઈ}: માપેલી કિંમતનો સાચી કિંમતની નજીકતાનો માપ
    \item \textbf{સચોટતા}: માપન કિંમતોની સુસંગતતા અથવા પુનરાવર્તિતા
\end{itemize}
\end{solutionbox}

\begin{mnemonicbox}
\mnemonic{ચોકસાઈ સત્યની નજીક, સચોટતા પુનરાવર્તનશીલ}
\end{mnemonicbox}

\questionmarks{1(b)}{4}{મૂળભૂત ભૌતિક એકમોનો ઉપયોગ કરીને કાર્ય અને વેગનું SI એકમ મેળવો.}

\begin{solutionbox}
\textbf{કાર્ય અને વેગના એકમોની ફોર્મ્યુલેશન:}

\begin{center}
\captionof{table}{કાર્ય અને વેગના એકમોની ફોર્મ્યુલેશન}
\begin{tabulary}{\linewidth}{|L|L|L|L|}
\hline
\textbf{ભૌતિક રાશિ} & \textbf{સૂત્ર} & \textbf{SI એકમ ફોર્મ્યુલેશન} & \textbf{SI એકમ} \\ \hline
કાર્ય (W) & $W = F \times d$ & $W = \text{[બળ]} \times \text{[અંતર]} = \text{[kg}\cdot\text{m/s}^2\text{]} \times \text{[m]} = \text{[kg}\cdot\text{m}^2/\text{s}^2\text{]}$ & Joule (J) \\ \hline
વેગ (v) & $v = d/t$ & $v = \text{[અંતર]}/\text{[સમય]} = \text{[m]}/\text{[s]}$ & m/s \\ \hline
\end{tabulary}
\end{center}

\begin{itemize}
    \item \textbf{કાર્ય}: જ્યારે બળ ($\text{kg}\cdot\text{m/s}^2$) અંતર (m) પર કાર્ય કરે છે, ત્યારે $\text{kg}\cdot\text{m}^2/\text{s}^2 = \text{Joule}$ મળે છે
    \item \textbf{વેગ}: જ્યારે કોઈ વસ્તુ સમય (s) માં અંતર (m) કાપે છે, ત્યારે m/s મળે છે
\end{itemize}
\end{solutionbox}

\begin{mnemonicbox}
\mnemonic{કાર્યમાં બળ અંતર, વેગમાં અંતર સમય}
\end{mnemonicbox}

\questionmarks{1(c)}{7}{સાધનની લઘુત્તમ માપ શક્તિ શું હોય? વર્નિયર કેલિપર્સની લઘુત્તમ માપ શક્તિનું સમીકરણ લખો. સુઘડ અને સ્વચ્છ આકૃતિ સાથે વર્નિયર કેલિપર્સ દ્વારા માપન સમજાવો.}

\begin{solutionbox}
\textbf{લઘુત્તમ માપ શક્તિ}: માપન સાધનથી સીધી રીતે માપી શકાય તેવી સૌથી નાની માપ.

\textbf{વર્નિયર કેલિપર્સની લઘુત્તમ માપ શક્તિનું સમીકરણ}:
$$
\text{લઘુત્તમ માપ શક્તિ} = 1 \text{ મુખ્ય સ્કેલ વિભાગ} - 1 \text{ વર્નિયર સ્કેલ વિભાગ}
$$
અથવા
$$
\text{લઘુત્તમ માપ શક્તિ} = \frac{\text{1 MSD ની કિંમત}}{\text{VSD ની સંખ્યા}}
$$

\textbf{આકૃતિ:}
\begin{center}
\begin{tikzpicture}
    % Main Scale
    \draw[thick] (0,0) rectangle (8,1);
    \foreach \x in {0,1,...,8} \draw (\x,1) -- (\x,0.7) node[above=3mm] {\x};
    \foreach \x in {0.5,1.5,...,7.5} \draw (\x,1) -- (\x,0.8);
    \node at (4,1.4) {મુખ્ય સ્કેલ (cm)};
    
    % Vernier Scale
    \draw[thick, fill=gray!10] (1.5,-0.5) rectangle (4.5,0);
    \foreach \x in {1.5,1.8,...,4.5} \draw (\x,0) -- (\x,-0.2);
    \node at (3,-0.8) {વર્નિયર સ્કેલ};
    
    % Jaws
    \draw[thick] (0,0) -- (0,-2) -- (0.5,-2) -- (0.5,-1) -- (1.5,-1) -- (1.5,0);
    \draw[thick] (1.5,-0.5) -- (1.5,-2) -- (2,-2) -- (2,-1) -- (4.5,-1);
    
    \node at (0.25,-2.2) {સ્થિર જડબું};
    \node at (1.75,-2.2) {ફરતું જડબું};
\end{tikzpicture}
\captionof{figure}{વર્નિયર કેલિપર}
\end{center}

\textbf{માપન પ્રક્રિયા}:
\begin{itemize}
    \item \textbf{પગલું 1}: વસ્તુની આસપાસ કેલિપરની બાજુઓ બંધ કરો
    \item \textbf{પગલું 2}: વર્નિયર સ્કેલના શૂન્ય પહેલાં આવતા મુખ્ય સ્કેલના વાંચનની નોંધ કરો
    \item \textbf{પગલું 3}: કયો વર્નિયર વિભાગ મુખ્ય સ્કેલના વિભાગ સાથે બરાબર સુમેળ કરે છે તે શોધો
    \item \textbf{પગલું 4}: વર્નિયર વાંચનને મુખ્ય સ્કેલ વાંચન સાથે ઉમેરો: $\text{કુલ} = \text{MSR} + (\text{VC} \times \text{LC})$
\end{itemize}

જ્યાં:
\begin{itemize}
    \item \textbf{MSR}: વર્નિયર શૂન્ય પહેલાં મુખ્ય સ્કેલ પર કિંમત
    \item \textbf{VC}: વર્નિયર સુમેળ
    \item \textbf{LC}: લઘુત્તમ માપ શક્તિ
\end{itemize}
\end{solutionbox}

\begin{mnemonicbox}
\mnemonic{મુખ્ય વત્તા મેળ બનાવે માપ}
\end{mnemonicbox}

\questionmarks{1(c) OR}{7}{સાધનની લઘુત્તમ માપ શક્તિ શું હોય? માઇક્રોમીટર સ્ક્રૂની લઘુત્તમ માપ શક્તિનું સમીકરણ લખો. સુઘડ અને સ્વચ્છ આકૃતિ સાથે માઇક્રોમીટર સ્ક્રૂમાં હકારાત્મક અને નકારાત્મક ભૂલ સમજાવો.}

\begin{solutionbox}
\textbf{લઘુત્તમ માપ શક્તિ}: માપન સાધનથી સીધી રીતે માપી શકાય તેવી સૌથી નાની માપ.

\textbf{માઇક્રોમીટર સ્ક્રૂની લઘુત્તમ માપ શક્તિનું સમીકરણ}:
$$
\text{લઘુત્તમ માપ શક્તિ} = \frac{\text{સ્ક્રૂનો પિચ}}{\text{વર્તુળાકાર સ્કેલ પરના વિભાગોની સંખ્યા}}
$$

\textbf{આકૃતિ:}
\begin{center}
\begin{tikzpicture}
    % Sleeve/Main Scale
    \draw[thick] (0,0) rectangle (3,1);
    \draw (0,0.5) -- (3,0.5); % Datum line
    \foreach \x in {0,0.5,...,2.5} \draw (\x,0.5) -- (\x,0.3);
    \foreach \x in {0,1,2} \node[below] at (\x,0.3) {\x};
    
    % Thimble/Circular Scale
    \draw[thick, fill=gray!20] (3,-0.2) rectangle (5,1.2);
    \foreach \y in {0,0.2,...,1} \draw (3,\y) -- (3.3,\y);
    \node[right] at (3.3,0.5) {0};
    \node[right] at (3.3,0.9) {5};
    \node[right] at (3.3,0.1) {45};
    
    \node at (1.5,1.3) {સ્લીવ (મુખ્ય સ્કેલ)};
    \node at (4,1.5) {થિમ્બલ (વર્તુળાકાર સ્કેલ)};
\end{tikzpicture}
\captionof{figure}{માઇક્રોમીટર સ્ક્રૂ ગેજ}
\end{center}

\textbf{ભૂલના પ્રકારો:}
\begin{itemize}
    \item \textbf{હકારાત્મક ભૂલ}: જ્યારે વર્તુળાકાર સ્કેલનો શૂન્ય સંદર્ભ રેખાની નીચે હોય. માપેલું વાંચન વાસ્તવિક કિંમત કરતાં વધારે થશે.
    \item \textbf{નકારાત્મક ભૂલ}: જ્યારે વર્તુળાકાર સ્કેલનો શૂન્ય સંદર્ભ રેખાની ઉપર હોય. માપેલું વાંચન વાસ્તવિક કિંમત કરતાં ઓછું થશે.
\end{itemize}

\textbf{ભૂલ સુધારણા}:
\begin{itemize}
    \item હકારાત્મક ભૂલ માટે: $\text{વાસ્તવિક વાંચન} = \text{નોંધાયેલું વાંચન} - \text{શૂન્ય ભૂલ}$
    \item નકારાત્મક ભૂલ માટે: $\text{વાસ્તવિક વાંચન} = \text{નોંધાયેલું વાંચન} + \text{શૂન્ય ભૂલ}$
\end{itemize}
\end{solutionbox}

\begin{mnemonicbox}
\mnemonic{હકારાત્મક હોય બાદ, નકારાત્મક જોઈએ ઉમેરવું}
\end{mnemonicbox}

\questionmarks{2(a)}{3}{વિદ્યુતક્ષેત્ર રેખાઓની લાક્ષણિકતાઓ લખો.}

\begin{solutionbox}
\textbf{વિદ્યુતક્ષેત્ર રેખાઓની લાક્ષણિકતાઓ:}

\begin{center}
\captionof{table}{વિદ્યુતક્ષેત્ર રેખાઓની લાક્ષણિકતાઓ}
\begin{tabulary}{\linewidth}{|L|L|}
\hline
\textbf{લાક્ષણિકતા} & \textbf{વર્ણન} \\ \hline
દિશા & હંમેશા ધન થી ઋણ ચાર્જ તરફ \\ \hline
આકાર & સમાન ક્ષેત્રો માટે સીધી રેખાઓ, અસમાન ક્ષેત્રો માટે વક્ર \\ \hline
ઘનતા & ક્ષેત્ર શક્તિના પ્રમાણમાં \\ \hline
માર્ગ & ક્યારેય એકબીજાને છેદતી નથી \\ \hline
પ્રકૃતિ & ધન ચાર્જથી શરૂ થાય છે અને ઋણ ચાર્જ પર સમાપ્ત થાય છે \\ \hline
\end{tabulary}
\end{center}
\end{solutionbox}

\begin{mnemonicbox}
\mnemonic{દિશા, ઘનતા, છેદતી નથી, શરૂ-અંત}
\end{mnemonicbox}

\questionmarks{2(b)}{4}{9 $\mu$F, 12 $\mu$F અને 15 $\mu$F કેપેસીટન્સ કિમત ધરાવતા કેપેસિટરના શ્રેણી અને સમાંતર બંને જોડાણ માટે પરિણામી કેપેસીટન્સની ગણતરી કરો}

\begin{solutionbox}
\textbf{આપેલ:}
$C_1 = 9 \mu\text{F}, C_2 = 12 \mu\text{F}, C_3 = 15 \mu\text{F}$

\textbf{શ્રેણી જોડાણ માટે}:
$$
\frac{1}{C_{eq}} = \frac{1}{C_1} + \frac{1}{C_2} + \frac{1}{C_3}
$$
$$
\frac{1}{C_{eq}} = \frac{1}{9} + \frac{1}{12} + \frac{1}{15} = \frac{20+15+12}{180} = \frac{47}{180}
$$
$$
C_{eq} = \frac{180}{47} \approx 3.83 \mu\text{F}
$$

\textbf{સમાંતર જોડાણ માટે}:
$$
C_{eq} = C_1 + C_2 + C_3
$$
$$
C_{eq} = 9 + 12 + 15 = 36 \mu\text{F}
$$
\end{solutionbox}

\begin{mnemonicbox}
\mnemonic{શ્રેણીમાં વ્યસ્ત સરવાળો, સમાંતરમાં સીધો સરવાળો}
\end{mnemonicbox}

\questionmarks{2(c)}{7}{કુલંબનો વ્યસ્ત વર્ગનો નિયમ સમજાવો અને તેનું સમીકરણ મેળવો. જો બે ઈલેક્ટ્રોન વચ્ચેનું અંતર 10 મીટર હોય તો તેમની વચ્ચે લાગતો કુલંબ બળ શોધો.($e=1.66 \times 10^{-19}$ C, $K= 9 \times 10^9 \text{ Nm}^2\text{ C}^{-2}$)}

\begin{solutionbox}
\textbf{કુલંબનો નિયમ}: બે બિંદુ ચાર્જ વચ્ચેનું સ્થિરવિદ્યુત બળ તે ચાર્જના ગુણાકારના સમપ્રમાણમાં અને તેમની વચ્ચેના અંતરના વર્ગના વ્યસ્ત પ્રમાણમાં હોય છે.

\textbf{સમીકરણ ફોર્મ્યુલેશન}:
$$
F \propto q_1q_2
$$
$$
F \propto \frac{1}{r^2}
$$
એકત્રિત કરતાં: $F \propto \frac{q_1q_2}{r^2}$
અચળાંક સાથે: $F = k\frac{q_1q_2}{r^2}$

જ્યાં $k = \frac{1}{4\pi\epsilon_0} = 9 \times 10^9 \text{ Nm}^2/\text{C}^2$

\textbf{આકૃતિ:}
\begin{center}
\begin{tikzpicture}
    \draw[dashed] (0,0) -- (4,0) node[midway, below] {$r$};
    \filldraw (0,0) circle (2pt) node[above] {$q_1$};
    \filldraw (4,0) circle (2pt) node[above] {$q_2$};
    
    \draw[->, thick] (0,0) -- (-1,0) node[left] {$F_{12}$};
    \draw[->, thick] (4,0) -- (5,0) node[right] {$F_{21}$};
\end{tikzpicture}
\captionof{figure}{કુલંબનો નિયમ}
\end{center}

\textbf{ગણતરી}:
$$
F = 9 \times 10^9 \times \frac{(1.66 \times 10^{-19}) \times (1.66 \times 10^{-19})}{(10)^2}
$$
$$
F = \frac{24.84 \times 10^{9-19-19}}{100} = 24.84 \times 10^{-31} \text{ N}
$$
$$
F \approx 2.48 \times 10^{-30} \text{ N}
$$
\end{solutionbox}

\begin{mnemonicbox}
\mnemonic{ચાર્જ ગુણાકાર, અંતર વર્ગ, બળ ઘટે}
\end{mnemonicbox}

\questionmarks{2(a) OR}{3}{વિદ્યુતક્ષેત્રને સમજાવો અને તેનો એકમ મેળવો.}

\begin{solutionbox}
\textbf{વિદ્યુતક્ષેત્ર}: ચાર્જની આસપાસનો વિસ્તાર જ્યાં અન્ય ચાર્જ બળ અનુભવે છે.

\textbf{વ્યાખ્યા}: કોઈ બિંદુ પર વિદ્યુતક્ષેત્ર એ બળ છે જે તે બિંદુ પર મૂકેલા એકમ ધન ચાર્જને અનુભવાય છે.
$$
E = \frac{F}{q}
$$

\textbf{એકમ ફોર્મ્યુલેશન}:
$$
E = \frac{F}{q} = \frac{[\text{N}]}{[\text{C}]} = \frac{[\text{kg}\cdot\text{m/s}^2]}{[\text{A}\cdot\text{s}]} = [\text{kg}\cdot\text{m}/(\text{A}\cdot\text{s}^3)]
$$
SI એકમ: N/C અથવા V/m
\end{solutionbox}

\begin{mnemonicbox}
\mnemonic{વિદ્યુતક્ષેત્ર એટલે ચાર્જ દીઠ બળ}
\end{mnemonicbox}

\questionmarks{2(b) OR}{4}{સ્વચ્છ આકૃતિ દોરી વિદ્યુત ફ્લક્સ સમજવો અને તેનો એકમ મેળવો.}

\begin{solutionbox}
\textbf{વિદ્યુત ફ્લક્સ}: આપેલા ક્ષેત્રફળમાંથી પસાર થતા વિદ્યુતક્ષેત્રનું માપ.

\textbf{સમીકરણ}:
$$
\phi_e = E \cdot A \cdot \cos\theta
$$

જ્યાં:
\begin{itemize}
    \item $E$ એ વિદ્યુતક્ષેત્ર છે
    \item $A$ એ ક્ષેત્રફળ છે
    \item $\theta$ એ $E$ અને ક્ષેત્રફળના લંબ વચ્ચેનો ખૂણો છે
\end{itemize}

\textbf{આકૃતિ:}
\begin{center}
\begin{tikzpicture}
    % Surface
    \draw[thick, fill=blue!5] (0,0) -- (3,0) -- (4,2) -- (1,2) -- cycle;
    \coordinate (C) at (1.5,1);
    
    % Normal
    \draw[->, thick] (C) -- (1.5, 3) node[above] {$\hat{n}$ (લંબ)};
    
    % Field vector
    \draw[->, thick, red] (C) -- (3, 2.5) node[right] {$\vec{E}$};
    
    % Angle
    \draw (1.5, 1.5) arc (90:45:0.5);
    \node at (1.8, 1.8) {$\theta$};
    
    % Field lines passing through
    \foreach \x in {0.5, 1, 2, 2.5}
        \draw[red, ->, opacity=0.5] (\x, 0.5) -- (\x+1.5, 2);
        
    \node[below] at (1.5,0) {સપાટી ક્ષેત્રફળ $A$};
\end{tikzpicture}
\captionof{figure}{વિદ્યુત ફ્લક્સ}
\end{center}

\textbf{એકમ ફોર્મ્યુલેશન}:
$$
\phi_e = E \cdot A \cdot \cos\theta = [\text{N/C}] \cdot [\text{m}^2] = [\text{N}\cdot\text{m}^2/\text{C}]
$$
SI એકમ: N$\cdot$m$^2$/C અથવા V$\cdot$m
\end{solutionbox}

\begin{mnemonicbox}
\mnemonic{ફ્લક્સ વહે ક્ષેત્ર અને ક્ષેત્રફળ દ્વારા}
\end{mnemonicbox}

\questionmarks{2(c) OR}{7}{કેપેસીટરની વ્યાખ્યા આપો અને તેનો યુનિટ મેળવો. સમાંતર પ્લેટ કેપેસિટરનું સૂત્ર આપો અને દરેક પદ સમજાવો. 20 cm x 20 cm ચોરસ પ્લેટો ધરાવતા અને 1.0 mm ના અંતરથી અલગ પડેલા સમાંતર પ્લેટ કેપેસિટરની કેપેસિટેન્સની ગણતરી કરો.}

\begin{solutionbox}
\textbf{કેપેસિટર}: વિદ્યુત ચાર્જ સંગ્રહિત કરતું ઉપકરણ.

\textbf{વ્યાખ્યા}: કેપેસિટન્સ એ સંગ્રહિત ચાર્જનો લાગુ કરેલા પોટેન્શિયલ તફાવત સાથેનો ગુણોત્તર છે.
$$
C = \frac{Q}{V}
$$

\textbf{એકમ ફોર્મ્યુલેશન}:
$$
C = \frac{Q}{V} = \frac{[\text{C}]}{[\text{V}]} = \text{Farad (F)}
$$

\textbf{સમાંતર પ્લેટ કેપેસિટર સૂત્ર}:
$$
C = \frac{\epsilon_0\epsilon_r A}{d}
$$

જ્યાં:
\begin{itemize}
    \item $C$ એ કેપેસિટન્સ છે
    \item $\epsilon_0$ એ મુક્ત અવકાશની પરાવૈદ્યુત્તા ($8.85 \times 10^{-12}$ F/m)
    \item $\epsilon_r$ એ ડાયલેક્ટ્રિકની સાપેક્ષ પરાવૈદ્યુત્તા છે
    \item $A$ એ પ્લેટોનો ઓવરલેપ ક્ષેત્રફળ છે
    \item $d$ એ પ્લેટો વચ્ચેનું અંતર છે
\end{itemize}

\textbf{આકૃતિ:}
\begin{center}
\begin{tikzpicture}
    \draw[thick] (0,1.5) -- (4,1.5) node[right] {પ્લેટ 1 ($+Q$)};
    \draw[thick] (0,0) -- (4,0) node[right] {પ્લેટ 2 ($-Q$)};
    
    \foreach \x in {0.5,1,...,3.5} \node at (\x, 1.2) {$+$};
    \foreach \x in {0.5,1,...,3.5} \node at (\x, 0.3) {$-$};
    
    \draw[<->] (-0.5,0) -- (-0.5,1.5) node[midway, left] {$d$};
    \node at (2,0.75) {ડાયલેક્ટ્રિક ($\epsilon_r$)};
    \node at (2,-0.5) {ક્ષેત્રફળ $A$};
\end{tikzpicture}
\captionof{figure}{સમાંતર પ્લેટ કેપેસિટર}
\end{center}

\textbf{ગણતરી}:
$$
C = \frac{8.85 \times 10^{-12} \times 1 \times 0.04}{0.001} = 354 \times 10^{-12} \text{ F} = 354 \text{ pF}
$$
\end{solutionbox}

\begin{mnemonicbox}
\mnemonic{કેપેસિટન્સ સંગ્રહે ચાર્જ નજીકના પ્લેટ વચ્ચે}
\end{mnemonicbox}

\questionmarks{3(a)}{3}{ઘન પદાર્થમાં ઉષ્માના વહનને ઉદાહરણ સાથે સમજાવો.}

\begin{solutionbox}
\textbf{ઉષ્મા વહન}: ઘન પદાર્થમાં પદાર્થની હલનચલન વિના ઉષ્મા ઊર્જાનું સ્થાનાંતરણ.

\textbf{પ્રક્રિયા}: ઉષ્મા ઊર્જા અણુઓના કંપન દ્વારા ઉચ્ચ તાપમાન ક્ષેત્રથી નિમ્ન તાપમાન ક્ષેત્ર તરફ સ્થાનાંતરિત થાય છે.

\textbf{આકૃતિ:}
\begin{center}
\begin{tikzpicture}
    % Rod
    \draw[thick, fill=gray!20] (0,0) rectangle (6,1);
    
    % Heat Source
    \node at (-1, 0.5) {ઉષ્મા સ્ત્રોત};
    \draw[->, thick, red] (-0.5, 0.5) -- (0, 0.5);
    
    % Temperature gradient illustration
    \shade[left color=red, right color=blue, opacity=0.5] (0,0) rectangle (6,1);
    
    \node[above] at (0,1) {ગરમ ($T_{high}$)};
    \node[above] at (6,1) {ઠંડુ ($T_{low}$)};
    
    \draw[->, thick] (1, -0.5) -- (5, -0.5) node[midway, below] {ઉષ્મા પ્રવાહ દિશા};
    
    % Molecules
    \foreach \x in {0.5, 1.5, ..., 5.5}
        \draw[circle, fill=black, inner sep=1pt] (\x, 0.5) circle (1pt);
    
    \node at (3, 1.5) {કંપન કરતા અણુઓ};
\end{tikzpicture}
\captionof{figure}{ઘન પદાર્થમાં ઉષ્મા વહન}
\end{center}

\textbf{ઉદાહરણ}: ગરમ ચામાં રાખેલો ધાતુનો ચમચો હેન્ડલ સુધી ગરમ થઈ જાય છે, જે વહન દ્વારા થાય છે.
\end{solutionbox}

\begin{mnemonicbox}
\mnemonic{ગરમ ઊર્જા આપે, અણુઓ સ્થાનાંતરિત કરે, બહાર વહે}
\end{mnemonicbox}

\questionmarks{3(b)}{4}{એક વ્યક્તિને 102 જેટલો તાવ આવે છે. અહીં તાપમાનનું એકમ કયો છે? આ તાપમાનને બાકીના બે એકમમાં રૂપાંતરિત કરો.}

\begin{solutionbox}
\textbf{તાપમાન એકમ}: 102$^\circ$F (ફેરનહાઈટ)

\textbf{રૂપાંતર સૂત્રો}:
\begin{itemize}
    \item $^\circ\text{C} = (^\circ\text{F} - 32) \times \frac{5}{9}$
    \item $\text{K} = ^\circ\text{C} + 273.15$
\end{itemize}

\textbf{ગણતરી}:
$$
^\circ\text{C} = (102 - 32) \times \frac{5}{9} = 38.89^\circ\text{C}
$$
$$
\text{K} = 38.89 + 273.15 = 312.04 \text{ K}
$$

\textbf{કોષ્ટક:}
\begin{center}
\captionof{table}{તાપમાન રૂપાંતર}
\begin{tabulary}{\linewidth}{|L|L|L|}
\hline
\textbf{ફેરનહાઈટ} & \textbf{સેલ્સિયસ} & \textbf{કેલ્વિન} \\ \hline
102$^\circ$F & 38.89$^\circ$C & 312.04 K \\ \hline
\end{tabulary}
\end{center}
\end{solutionbox}

\begin{mnemonicbox}
\mnemonic{ફેરનહાઈટ પહેલા, સેલ્સિયસ બદલો, કેલ્વિન છેલ્લે આવે}
\end{mnemonicbox}

\questionmarks{3(c)}{7}{પ્લેટિનમ રેઝિસ્ટન્સ થર્મોમીટરનો સિદ્ધાંત સમજાવો અને તેના ઉપયોગની યાદી બનાવો.}

\begin{solutionbox}
\textbf{સિદ્ધાંત}: પ્લેટિનમનો વિદ્યુત અવરોધ તાપમાન સાથે નિશ્ચિત અને સુસંગત રીતે બદલાય છે, જે ચોક્કસ તાપમાન માપન માટે અવકાશ આપે છે.

\textbf{કાર્યપ્રણાલી}: $R = R_0[1 + \alpha(T - T_0)]$ સંબંધ પર આધારિત.

\textbf{આકૃતિ:}
\begin{center}
\begin{tikzpicture}
    % Wheatstone Bridge Style
    \draw (0,2) -- (2,4) -- (4,2) -- (2,0) -- (0,2);
    \node[left] at (0,2) {A};
    \node[above] at (2,4) {B};
    \node[right] at (4,2) {C};
    \node[below] at (2,0) {D};
    
    % Resistors
    \draw[fill=white] (0.5,2.5) rectangle (1.5,3.5); \node at (1,3) {$P$};
    \draw[fill=white] (2.5,2.5) rectangle (3.5,3.5); \node at (3,3) {$Q$};
    \draw[fill=white] (0.5,0.5) rectangle (1.5,1.5); \node at (1,1) {$R$};
    
    % Platinum coil at D-C arm? Typically S is the unknown. 
    % Let's draw the probe connected to arm CD
    \draw[fill=white] (2.5, 0.5) rectangle (3.5, 1.5); 
    \node[right] at (3.5,1) {Pt કોઈલ ($S$)};
    
    % Galvanometer
    \draw (2,2) circle (0.3); \node at (2,2) {G};
    \draw (2,4) -- (2,2.3);
    \draw (2,1.7) -- (2,0);
    
    % Battery
    \draw (-1,2) -- (-1,-1) -- (5,-1) -- (5,2);
    \draw (1.8,-1) -- (2.2,-1); \node[below] at (2,-1) {બેટરી};
    
\end{tikzpicture}
\captionof{figure}{વ્હીટસ્ટોન બ્રિજ}
\end{center}

\textbf{ઉપયોગો}:
\begin{itemize}
    \item \textbf{ઔદ્યોગિક પ્રક્રિયા}: ઉત્પાદનમાં તાપમાન નિરીક્ષણ
    \item \textbf{વૈજ્ઞાનિક સંશોધન}: ઉચ્ચ ચોકસાઈની જરૂરિયાત વાળા પ્રયોગશાળા માપન
    \item \textbf{કેલિબ્રેશન}: અન્ય થર્મોમીટર્સના કેલિબ્રેશન માટે માનક
    \item \textbf{તબીબી ઉપયોગો}: તબીબી ઉપકરણોમાં તાપમાન નિરીક્ષણ
\end{itemize}
\end{solutionbox}

\begin{mnemonicbox}
\mnemonic{પ્લેટિનમ આપે ચોક્કસ અવરોધ-તાપમાન સંબંધ}
\end{mnemonicbox}

\questionmarks{3(a) OR}{3}{વિશિષ્ટ ઉષ્મા અને ઉષ્માધારિતા ની વ્યાખ્યાયિત લખો અને તેના એકમો લખો.}

\begin{solutionbox}
\textbf{વિશિષ્ટ ઉષ્મા}: 1 કિગ્રા પદાર્થનું તાપમાન 1 K વધારવા માટે જરૂરી ઉષ્મા ઊર્જાનું પ્રમાણ.

\textbf{ઉષ્માધારિતા}: સંપૂર્ણ વસ્તુનું તાપમાન 1 K વધારવા માટે જરૂરી ઉષ્મા ઊર્જાનું પ્રમાણ.

\begin{center}
\captionof{table}{ઉષ્મા ક્ષમતા શબ્દો}
\begin{tabulary}{\linewidth}{|L|L|L|}
\hline
\textbf{શબ્દ} & \textbf{સૂત્ર} & \textbf{SI એકમ} \\ \hline
વિશિષ્ટ ઉષ્મા (c) & $Q = mc\Delta T$ & J/(kg$\cdot$K) \\ \hline
ઉષ્માધારિતા (C) & $Q = C\Delta T$ & J/K \\ \hline
\end{tabulary}
\end{center}
\end{solutionbox}

\begin{mnemonicbox}
\mnemonic{વિશિષ્ટ પદાર્થ માટે, ધારિતા સંપૂર્ણ વસ્તુ માટે}
\end{mnemonicbox}

\questionmarks{3(b) OR}{4}{તરલ પદાર્થમાં ઉષ્માનયન ઉદાહરણ સાથે સમજાવો.}

\begin{solutionbox}
\textbf{ઉષ્મા અભિવહન}: તરલ (પ્રવાહી અથવા વાયુ) ની હલનચલન દ્વારા ઉષ્માનું સ્થાનાંતરણ.

\textbf{પ્રક્રિયા}: ગરમ તરલ પ્રસરણ પામે છે, ઓછી ઘનતા ધરાવે છે, ઉપર ઉઠે છે; ઠંડુ તરલ નીચે ઉતરે છે, જે અભિવહન વહેણ તરીકે ઓળખાતી સતત પરિભ્રમણ પદ્ધતિ બનાવે છે.

\textbf{આકૃતિ:}
\begin{center}
\begin{tikzpicture}
    % Container
    \draw[thick] (0,0) -- (0,3) -- (4,3) -- (4,0) -- cycle;
    
    % Heat Source
    \node[below] at (2,0) {ઉષ્મા સ્ત્રોત};
    \foreach \x in {1.5, 2, 2.5}
        \draw[->, thick, red] (\x, -0.5) -- (\x, 0);
        
    % Fluid currents
    \draw[thick, red, ->] (2,0.2) -- (2,2.5);
    \draw[thick, blue, ->] (0.5,2.5) -- (0.5,0.5) -- (1.5,0.2);
    \draw[thick, blue, ->] (3.5,2.5) -- (3.5,0.5) -- (2.5,0.2);
    \draw[thick, red, ->] (2,2.5) .. controls (1,2.8) .. (0.5,2.5);
    \draw[thick, red, ->] (2,2.5) .. controls (3,2.8) .. (3.5,2.5);
    
    \node[red] at (2, 1.5) {ગરમ ઉપર જાય};
    \node[blue] at (0.5, 1.5) {ઠંડુ નીચે આવે};
    \node[blue] at (3.5, 1.5) {ઠંડુ નીચે આવે};
    
\end{tikzpicture}
\captionof{figure}{અભિવહન વહેણ}
\end{center}

\textbf{ઉદાહરણ}: વાસણમાં ઉકળતું પાણી - ગરમ પાણી ઉપર ચઢે છે જ્યારે ઠંડુ પાણી નીચે ઉતરે છે.
\end{solutionbox}

\begin{mnemonicbox}
\mnemonic{ગરમ ઉપર જાય, ઠંડુ નીચે આવે, વહેણ ફરતું રહે}
\end{mnemonicbox}

\questionmarks{3(c) OR}{7}{ઉષ્મા વાહકતાના અચળાંકને વ્યાખ્યાયિત કરો. ઘન પદાર્થોમાં ઉષ્માના વહન માટે ઉષ્મા વાહકતાના અચળાંકનું સમીકરણ મેળવો.}

\begin{solutionbox}
\textbf{ઉષ્મા વાહકતાનો અચળાંક}: એકમ સમય દીઠ, એકમ ક્ષેત્રફળ દીઠ, એકમ તાપમાન પ્રવણતા દીઠ સ્થાનાંતરિત થતી ઉષ્માનું પ્રમાણ.

\textbf{વ્યાખ્યા}: જ્યારે તાપમાન પ્રવણતા એકમ હોય ત્યારે દર સેકન્ડે એકમ ક્ષેત્રફળ દ્વારા વહેતી ઉષ્માનું પ્રમાણ.

\textbf{ફોર્મ્યુલેશન}:
\begin{itemize}
    \item છેદફળ $A$ અને લંબાઈ $L$ ધરાવતા સળિયાને ધ્યાનમાં લો
    \item છેડા વચ્ચેનો તાપમાન તફાવત $\Delta T$ છે
    \item સમય $t$ માં ઉષ્મા પ્રવાહ $Q$ છે
\end{itemize}

ઉષ્મા પ્રવાહ = $Q/t$ \\
તાપમાન પ્રવણતા = $\Delta T/L$ \\
ક્ષેત્રફળ = $A$

ફોરિયરના નિયમ અનુસાર:
$$
\frac{Q}{t} \propto A \frac{\Delta T}{L}
$$
$$
\frac{Q}{t} = k \cdot A \cdot \frac{\Delta T}{L}
$$

પુનર્ગોઠવણી કરતાં:
$$
k = \frac{Q \cdot L}{t \cdot A \cdot \Delta T}
$$

જ્યાં $k$ એ ઉષ્મા વાહકતાનો અચળાંક છે.

\textbf{આકૃતિ:}
\begin{center}
\begin{tikzpicture}
    \draw[thick, fill=gray!20] (0,0) rectangle (5,1);
    \node[left] at (0,0.5) {$T_1$ (ગરમ)};
    \node[right] at (5,0.5) {$T_2$ (ઠંડુ)};
    \draw[<->] (0,-0.3) -- (5,-0.3) node[midway, below] {$L$};
    \node at (2.5, 0.5) {ક્ષેત્રફળ $A$};
    \draw[->, thick, red] (1, 1.3) -- (4, 1.3) node[midway, above] {ઉષ્મા પ્રવાહ $Q$};
\end{tikzpicture}
\captionof{figure}{ઉષ્મા વાહકતા}
\end{center}

\textbf{એકમ}: W/(m$\cdot$K)
\end{solutionbox}

\begin{mnemonicbox}
\mnemonic{ઉષ્મા જથ્થો સ્થાનાંતરિત થાય લંબાઈ દ્વારા, ક્ષેત્રફળ અને તાપમાન ભાગીને}
\end{mnemonicbox}

\questionmarks{4(a)}{3}{લંબગત તરંગો અને સંગત તરંગો વચ્ચેનો તફાવત આપો.}

\begin{solutionbox}
\textbf{લંબગત બનામ સંગત તરંગો:}

\begin{center}
\captionof{table}{લંબગત બનામ સંગત તરંગો}
\begin{tabulary}{\linewidth}{|L|L|L|}
\hline
\textbf{ગુણધર્મ} & \textbf{લંબગત તરંગો} & \textbf{સંગત તરંગો} \\ \hline
કણની ગતિ & તરંગ દિશાને લંબ & તરંગ દિશાને સમાંતર \\ \hline
માધ્યમ વિસ્થાપન & શિખર અને ગર્ત & સંકોચન અને વિરલન \\ \hline
ઉદાહરણો & પ્રકાશ તરંગો, પાણીના તરંગો & ધ્વનિ તરંગો, સિસ્મિક P-તરંગો \\ \hline
માધ્યમ જરૂરિયાતો & ઘન પદાર્થોમાં પ્રવાસ કરી શકે & ઘન, પ્રવાહી, વાયુમાં પ્રવાસ કરી શકે \\ \hline
ધ્રુવીકરણ & ધ્રુવીકૃત થઈ શકે & ધ્રુવીકૃત થઈ શકતા નથી \\ \hline
\end{tabulary}
\end{center}
\end{solutionbox}

\begin{mnemonicbox}
\mnemonic{લંબગત લે લંબ માર્ગ, સંગત સહાય સમાંતર સરકવામાં}
\end{mnemonicbox}

\questionmarks{4(b)}{4}{જો એક તરંગનો વેગ 350 m/s અને આવૃત્તિ 10 Hz છે તો તેની તરંગલંબાઇની ગણતરી કરો.}

\begin{solutionbox}
\textbf{તરંગ સમીકરણ}: $v = f\lambda$

જ્યાં:
\begin{itemize}
    \item $v$ એ તરંગ વેગ છે (350 m/s)
    \item $f$ એ આવૃત્તિ છે (10 Hz)
    \item $\lambda$ એ તરંગલંબાઈ છે (શોધવાની છે)
\end{itemize}

\textbf{ગણતરી}:
$$
\lambda = \frac{v}{f} = \frac{350}{10} = 35 \text{ m}
$$
\end{solutionbox}

\begin{mnemonicbox}
\mnemonic{વેગ બરાબર આવૃત્તિ ગુણાકાર તરંગલંબાઈ}
\end{mnemonicbox}

\questionmarks{4(c)}{7}{અલ્ટ્રાસોનિક તરંગોને વ્યાખ્યાયિત કરો અને તેની લાક્ષણિકતાઓ લખો. અલ્ટ્રાસોનિક તરંગની તેની ચાર મુખ્ય ઉપયોગો લખો.}

\begin{solutionbox}
\textbf{અલ્ટ્રાસોનિક તરંગો}: માનવ શ્રવણની ઉપલી મર્યાદા (20 kHz થી વધુ) કરતાં ઊંચી આવૃત્તિ ધરાવતા ધ્વનિ તરંગો.

\textbf{લાક્ષણિકતાઓ}:
\begin{itemize}
    \item \textbf{ઉચ્ચ આવૃત્તિ}: 20 kHz થી વધુ
    \item \textbf{ટૂંકી તરંગલંબાઈ}: નાની વસ્તુઓને શોધવાની ક્ષમતા આપે છે
    \item \textbf{દિશાસૂચક}: ચોક્કસ દિશામાં કેન્દ્રિત કરી શકાય છે
    \item \textbf{બિન-આયનીકરણ}: જૈવિક પેશીઓ માટે સલામત
    \item \textbf{પ્રવેશ}: વિવિધ માધ્યમોમાંથી પસાર થઈ શકે છે
\end{itemize}

\textbf{આકૃતિ:}
\begin{center}
\begin{tikzpicture}
    \draw[->] (0,0) -- (6,0) node[right] {સમય};
    \draw[->] (0,-1) -- (0,1) node[above] {કંપવિસ્તાર};
    \draw[domain=0:6, samples=200, smooth] plot (\x, {0.8*sin(\x r * 10)});
    \node at (3,1.2) {ઉચ્ચ આવર્તન ($f > 20$ kHz)};
    \draw[<->] (1.57/10, -0.8) -- (1.57/10 + 6.28/10, -0.8) node[midway, below] {$T < 50 \mu$s};
\end{tikzpicture}
\captionof{figure}{ઉચ્ચ આવર્તન અલ્ટ્રાસોનિક તરંગ}
\end{center}

\textbf{ઉપયોગો}:
\begin{itemize}
    \item \textbf{તબીબી}: નિદાનાત્મક ઇમેજિંગ, ઉપચારાત્મક પ્રક્રિયાઓ
    \item \textbf{ઔદ્યોગિક}: બિન-વિનાશક પરીક્ષણ, ખામી શોધ
    \item \textbf{સફાઈ}: સચોટ ભાગો માટે અલ્ટ્રાસોનિક ક્લીનિંગ બાથ
    \item \textbf{અંતર માપન}: સોનાર, પાર્કિંગ સેન્સર, લેવલ ઇન્ડિકેટર્સ
\end{itemize}
\end{solutionbox}

\begin{mnemonicbox}
\mnemonic{અલ્ટ્રાસોનિક ઉપયોગ ધ્વનિ શોધવા, સ્કેન કરવા, સાફ કરવા}
\end{mnemonicbox}

\questionmarks{4(a) OR}{3}{પ્રકાશના ધ્રુવીકરણને સ્વચ્છ આકૃતિ દોરી સમજાવો.}

\begin{solutionbox}
\textbf{ધ્રુવીકરણ}: પ્રકાશ તરંગોના કંપનોને એક જ સમતલમાં મર્યાદિત કરવાની પ્રક્રિયા.

\textbf{પ્રકારો}:
\begin{itemize}
    \item રેખીય ધ્રુવીકરણ
    \item વર્તુળાકાર ધ્રુવીકરણ
    \item ઇલિપ્ટિકલ ધ્રુવીકરણ
\end{itemize}

\textbf{આકૃતિ:}
\begin{center}
\begin{tikzpicture}
    % Axis
    \draw[->] (0,0) -- (8,0);
    
    % Unpolarized
    \foreach \x in {0.5, 1, 1.5} {
        \draw[<->] (\x, -0.5) -- (\x, 0.5);
        \draw (\x, 0) -- (\x+0.2, 0.2); 
        \draw (\x, 0) -- (\x-0.2, -0.2);
    }
    \node[below] at (1, -0.8) {અધ્રુવીભૂત};
    
    % Polarizer
    \draw[thick, fill=blue!10] (2.5, -1) rectangle (3.5, 1);
    \foreach \y in {-0.8, -0.6, ..., 0.8} \draw (2.5, \y) -- (3.5, \y); 
    \foreach \y in {-0.9, -0.7, ..., 0.9} \draw[dashed] (3, -1) -- (3, 1); 
    \node[above] at (3, 1) {પોલરાઇઝર};
    
    % Polarized
    \foreach \x in {4.5, 5.0, ..., 7} {
        \draw[<->] (\x, -0.5) -- (\x, 0.5);
    }
    \node[below] at (6, -0.8) {ધ્રુવીભૂત};
    
    \node at (4, 1.5) {પસાર અક્ષ (ઊભી)};
\end{tikzpicture}
\captionof{figure}{પ્રકાશનું ધ્રુવીકરણ}
\end{center}
\end{solutionbox}

\begin{mnemonicbox}
\mnemonic{ધ્રુવક પસંદ કરે વિશિષ્ટ સમતલો}
\end{mnemonicbox}

\questionmarks{4(b) OR}{4}{જો પ્રકાશ નો હવા માં વેગ $3 \times 10^8$ m/s અને પ્રકાશનો પાણી માં વેગ $2.25 \times 10^8$ m/s તો પ્રકાશનો વક્રીવનાંક શોધો.}

\begin{solutionbox}
\textbf{વક્રીભવનાંક સૂત્ર}: $n = c/v$

જ્યાં:
\begin{itemize}
    \item $n$ એ વક્રીભવનાંક છે
    \item $c$ એ શૂન્યાવકાશમાં (અથવા હવામાં) પ્રકાશનો વેગ છે ($3 \times 10^8$ m/s)
    \item $v$ એ માધ્યમમાં પ્રકાશનો વેગ છે ($2.25 \times 10^8$ m/s)
\end{itemize}

\textbf{ગણતરી}:
$$
n = \frac{3 \times 10^8}{2.25 \times 10^8} = \frac{3}{2.25} = \frac{300}{225} = \frac{4}{3} \approx 1.33
$$
\end{solutionbox}

\begin{mnemonicbox}
\mnemonic{ધીમો વેગ બતાવે ઊંચો સૂચક}
\end{mnemonicbox}

\questionmarks{4(c)(i) OR}{4}{વ્યાખ્યાયિત કરો: તરંગ નો વેગ, તરંગલંબાઈ અને આવૃતિ. અને તરંગ વેગ, તરંગલંબાઈ અને આવૃતિ વચ્ચેનો સંબંધ મેળવો.}

\begin{solutionbox}
\textbf{વ્યાખ્યાઓ}:
\begin{itemize}
    \item \textbf{તરંગ વેગ ($v$)}: તરંગ માધ્યમમાં જે ગતિથી પ્રવાસ કરે છે તે.
    \item \textbf{તરંગલંબાઈ ($\lambda$)}: તરંગ પર બે ક્રમિક સમાન બિંદુઓ વચ્ચેનું અંતર (જેમ કે, શિખર થી શિખર).
    \item \textbf{આવૃત્તિ ($f$)}: દર એકમ સમયે કોઈ બિંદુમાંથી પસાર થતા સંપૂર્ણ તરંગ ચક્રોની સંખ્યા.
\end{itemize}

\textbf{આકૃતિ:}
\begin{center}
\begin{tikzpicture}
    \draw[->] (0,0) -- (6,0) node[right] {અંતર};
    \draw[->] (0,-1) -- (0,1) node[above] {કંપવિસ્તાર};
    \draw[thick, domain=0:6, samples=100] plot (\x, {sin(\x r * 2)});
    
    \draw[<->] (1.57, 1.2) -- (4.71, 1.2) node[midway, above] {તરંગલંબાઈ $\lambda$};
    \draw[dashed] (1.57, 1) -- (1.57, 1.2);
    \draw[dashed] (4.71, 1) -- (4.71, 1.2);
\end{tikzpicture}
\captionof{figure}{તરંગ પરિમાણો}
\end{center}

\textbf{સંબંધ}:
\begin{itemize}
    \item સમય $T$ માં, તરંગ એક તરંગલંબાઈ $\lambda$ જેટલું અંતર પ્રવાસ કરે છે.
    \item વેગ = અંતર / સમય
    \item $v = \lambda / T$
    \item આવૃત્તિ $f = 1/T$ હોવાથી
    \item તેથી, $v = \lambda \cdot f$
\end{itemize}
\end{solutionbox}

\begin{mnemonicbox}
\mnemonic{વેગ બરાબર આવૃત્તિ ગુણાકાર તરંગલંબાઈ}
\end{mnemonicbox}

\questionmarks{4(c)(ii) OR}{3}{પ્રકાશના ગુણધર્મો લખો.}

\begin{solutionbox}
\textbf{પ્રકાશના ગુણધર્મો:}

\begin{center}
\captionof{table}{પ્રકાશના ગુણધર્મો}
\begin{tabulary}{\linewidth}{|L|L|}
\hline
\textbf{ગુણધર્મ} & \textbf{વર્ણન} \\ \hline
પ્રચાર & સમાંગી માધ્યમમાં સીધી રેખામાં ચાલે છે \\ \hline
વેગ & શૂન્યાવકાશમાં $3 \times 10^8$ m/s \\ \hline
પરાવર્તન & સપાટીઓ પરથી પરાવર્તન નિયમ અનુસરીને પરાવર્તિત થાય છે \\ \hline
વક્રીભવન & માધ્યમો વચ્ચે પસાર થતાં દિશા બદલે છે \\ \hline
વિભાજન & શ્વેત પ્રકાશ તેના ઘટક રંગોમાં વિભાજિત થાય છે \\ \hline
વ્યતિકરણ & તરંગો ભેગા થઈને પેટર્ન બનાવી શકે છે \\ \hline
વિવર્તન & અવરોધો અને નાના છિદ્રોમાંથી વળે છે \\ \hline
ધ્રુવીકરણ & એક સમતલમાં કંપન કરવા માટે મર્યાદિત કરી શકાય છે \\ \hline
દ્વૈત પ્રકૃતિ & તરંગ અને કણ બંને ગુણધર્મો દર્શાવે છે \\ \hline
\end{tabulary}
\end{center}
\end{solutionbox}

\begin{mnemonicbox}
\mnemonic{પ્રકાશ પરાવર્તે, વક્રીભવે, વિભાજિત થાય, વ્યતિકરણ કરે, ધ્રુવીકૃત થાય}
\end{mnemonicbox}

\questionmarks{5(a)}{3}{સમતલ સપાટી માટે પ્રકાશના વક્રીભવનના નિયમો સમજાવો. અને સ્નેલનો નિયમ સમજાવો.}

\begin{solutionbox}
\textbf{વક્રીભવનનો નિયમ}: જ્યારે પ્રકાશ એક માધ્યમથી બીજા માધ્યમમાં પસાર થાય છે, ત્યારે તે સીમા પર દિશા બદલે છે. આપતન કિરણ, વક્રીભૂત કિરણ અને લંબ એક જ સમતલમાં હોય છે.

\textbf{સ્નેલનો નિયમ}: આપતન કોણના સાઇનનો વક્રીભવન કોણના સાઇન સાથેનો ગુણોત્તર આપેલા માધ્યમોની જોડી માટે અચળ રહે છે.

$$
n_1 \sin(\theta_1) = n_2 \sin(\theta_2)
$$

જ્યાં:
\begin{itemize}
    \item $n_1, n_2$: માધ્યમ 1 અને 2 ના વક્રીભવનાંક
    \item $\theta_1$: આપતન કોણ
    \item $\theta_2$: વક્રીભવન કોણ
\end{itemize}

\textbf{આકૃતિ:}
\begin{center}
\begin{tikzpicture}
    % Boundary
    \draw[thick] (-3,0) -- (3,0);
    \node[above left] at (-3,0) {માધ્યમ 1 ($n_1$)};
    \node[below left] at (-3,0) {માધ્યમ 2 ($n_2$)};
    
    % Normal
    \draw[dashed] (0,-2) -- (0,2) node[above] {લંબ};
    
    % Rays
    \draw[thick, ->, red] (-2, 2) -- (0,0);
    \draw[thick, ->, red] (0,0) -- (1, -2);
    
    % Angles
    \draw (0,0.5) arc (90:135:0.5);
    \node at (-0.4, 0.8) {$\theta_1$};
    
    \draw (0,-0.5) arc (270:297:0.5);
    \node at (0.3, -0.8) {$\theta_2$};
    
\end{tikzpicture}
\captionof{figure}{પ્રકાશનું વક્રીભવન}
\end{center}
\end{solutionbox}

\begin{mnemonicbox}
\mnemonic{સાઇન બતાવે વેગ અલગ માધ્યમોમાં}
\end{mnemonicbox}

\questionmarks{5(b)}{4}{સ્ટેપ ઈન્ડેક્ષ ફાઈબર માં કોર વક્રીભવનાંક 1.30 હોય અને સંબંધિત વક્રીભવનાંક તફાવત $\Delta=0.02$ છે. ન્યુમેરિકલ એપેચર શોધો.}

\begin{solutionbox}
\textbf{આપેલ}:
કોર વક્રીભવનાંક $n_1 = 1.30$
સંબંધિત વક્રીભવનાંક તફાવત $\Delta = 0.02$

\textbf{સૂત્ર}:
સ્ટેપ ઈન્ડેક્સ ફાઈબર માટે:
$$
\text{NA} = n_1 \sqrt{2\Delta}
$$

\textbf{ગણતરી}:
$$
\text{NA} = 1.30 \times \sqrt{2 \times 0.02}
$$
$$
\text{NA} = 1.30 \times \sqrt{0.04}
$$
$$
\text{NA} = 1.30 \times 0.2
$$
$$
\text{NA} = 0.26
$$
\end{solutionbox}

\begin{mnemonicbox}
\mnemonic{ન્યુમેરિકલ એપેચર જોઈએ કોર અને ડેલ્ટા}
\end{mnemonicbox}

\questionmarks{5(c)}{7}{પ્રકાશનું પૂર્ણ આંતરિક પરાવર્તન સમજાવો. અને ક્રિટિકલ ખૂણાનું સમીકરણ મેળવો.}

\begin{solutionbox}
\textbf{પૂર્ણ આંતરિક પરાવર્તન (TIR)}: જ્યારે પ્રકાશ સઘન માધ્યમથી વિરલ માધ્યમમાં ક્રિટિકલ કોણથી વધુ કોણે જતો હોય ત્યારે માધ્યમોની સીમા પર પ્રકાશનું સંપૂર્ણ પરાવર્તન.

\textbf{TIR માટેની શરતો}:
\begin{itemize}
    \item પ્રકાશ સઘન માધ્યમથી વિરલ માધ્યમ તરફ જવો જોઈએ
    \item આપતન કોણ ક્રિટિકલ કોણથી વધુ હોવો જોઈએ
\end{itemize}

\textbf{ક્રિટિકલ કોણ ($\theta_c$)}: સઘન માધ્યમમાં આપતન કોણ જેના માટે વિરલ માધ્યમમાં વક્રીભવન કોણ 90$^\circ$ હોય.

\textbf{સમીકરણ}:
સ્નેલના નિયમનો ઉપયોગ કરીને: $n_1 \sin(\theta_1) = n_2 \sin(\theta_2)$
અહીં $n_1 > n_2$.
$\theta_1 = \theta_c$ પર, $\theta_2 = 90^\circ$.
$$
n_1 \sin(\theta_c) = n_2 \sin(90^\circ)
$$
$$
n_1 \sin(\theta_c) = n_2
$$
$$
\sin(\theta_c) = \frac{n_2}{n_1}
$$
$$
\theta_c = \sin^{-1}\left(\frac{n_2}{n_1}\right)
$$

\textbf{આકૃતિ:}
\begin{center}
\begin{tikzpicture}
    % Boundary
    \draw[thick] (-3,0) -- (3,0);
    \node[above] at (0,0.2) {વિરલ માધ્યમ ($n_2$)};
    \node[below] at (0,-0.2) {સઘન માધ્યમ ($n_1$)};
    
    % Critical Angle Case
    \draw[dashed] (0,-2) -- (0,1); 
    \draw[thick, ->, red] (-1.5, -2) -- (0,0);
    \draw[thick, ->, red] (0,0) -- (2, 0); 
    \draw (0,-0.5) arc (270:233:0.5);
    \node at (-0.3, -0.8) {$\theta_c$};
    
    % TIR Case
    \draw[dashed] (2,-2) -- (2,1);
    \draw[thick, ->, blue] (1, -2) -- (2,0);
    \draw[thick, ->, blue] (2,0) -- (3, -2);
    \draw (2,-0.5) arc (270:243:0.5);
    \node at (1.7, -0.8) {$>\theta_c$};

\end{tikzpicture}
\captionof{figure}{પૂર્ણ આંતરિક પરાવર્તન}
\end{center}
\end{solutionbox}

\begin{mnemonicbox}
\mnemonic{ક્રિટિકલ આવે સઘનથી વિરલ, સાઈન બરાબર ભાગાકાર}
\end{mnemonicbox}

\questionmarks{5(a) OR}{3}{ફાઈબર ઓપ્ટીકલ કેબલ માટે ન્યુમેરિકલ એપેચર અને એક્સેપ્ટન્સ ખૂણો સમજાવો.}

\begin{solutionbox}
\textbf{ન્યુમેરિકલ એપેચર (NA)}: ઓપ્ટિકલ ફાઈબરની પ્રકાશ-એકત્રિત કરવાની ક્ષમતાનું માપ.
$$
\text{NA} = \sin(\theta_a) = \sqrt{n_1^2 - n_2^2}
$$

\textbf{એક્સેપ્ટન્સ ખૂણો ($\theta_a$)}: મહત્તમ કોણ જેના પર પ્રકાશ ફાઈબરમાં પ્રવેશી શકે છે અને હજુ પણ પૂર્ણ આંતરિક પરાવર્તન અનુભવી શકે છે.

\textbf{આકૃતિ:}
\begin{center}
\begin{tikzpicture}
    % Fiber
    \draw[thick, fill=gray!10] (0,0.5) rectangle (4,1.5); 
    \draw[thick, fill=gray!30] (0,0) rectangle (4,0.5); 
    \draw[thick, fill=gray!30] (0,1.5) rectangle (4,2); 
    
    \node at (2,1) {કોર ($n_1$)};
    \node at (2,0.25) {ક્લેડિંગ ($n_2$)};
    \node at (2,1.75) {ક્લેડિંગ ($n_2$)};
    
    % Axis
    \draw[dashed] (-1,1) -- (5,1);
    
    % Acceptance Angle
    \draw[->, red, thick] (-1, 1) -- (0, 1); 
    \draw[->, red, thick] (-1, 0.5) -- (0, 1); 
    \draw[->, red, thick] (-1, 1.5) -- (0, 1); 
    
    % Cone arc
    \draw (-0.5,1) arc (180:206:0.5);
    \node at (-0.7, 0.8) {$\theta_a$};
    \draw (-0.5,1) arc (180:154:0.5);
    
    \node at (-1.5, 1) {સ્વીકૃતિ શંકુ};
    
\end{tikzpicture}
\captionof{figure}{ન્યુમેરિકલ એપેચર અને સ્વીકૃતિ શંકુ}
\end{center}
\end{solutionbox}

\begin{mnemonicbox}
\mnemonic{એક્સેપ્ટન્સ ખૂણો પ્રકાશ પ્રવેશાવે, ન્યુમેરિકલ એપેચર તેનો સાઈન કહેવાય}
\end{mnemonicbox}

\questionmarks{5(b) OR}{4}{લેસર નું આખું નામ લખો. તેની લાક્ષણિકતાઓ લખો.}

\begin{solutionbox}
\textbf{LASER}: Light Amplification by Stimulated Emission of Radiation (ઉત્તેજિત વિકિરણ ઉત્સર્જન દ્વારા પ્રકાશ વર્ધન)

\textbf{લેસરની લાક્ષણિકતાઓ}:

\begin{center}
\captionof{table}{લેસરની લાક્ષણિકતાઓ}
\begin{tabulary}{\linewidth}{|L|L|}
\hline
\textbf{લાક્ષણિકતા} & \textbf{વર્ણન} \\ \hline
એકવર્ણીય & એક જ તરંગલંબાઈ અથવા રંગ \\ \hline
સુસંગત & બધા તરંગો એક જ તબક્કામાં \\ \hline
અત્યંત દિશાત્મક & લઘુત્તમ વિચલન સાથે સીધી રેખામાં ચાલે છે \\ \hline
ઉચ્ચ તીવ્રતા & સાંકડી બીમમાં કેન્દ્રિત ઊર્જા \\ \hline
સમાંતરિત & ન્યૂનતમ ફેલાવા સાથે સમાંતર કિરણો \\ \hline
\end{tabulary}
\end{center}
\end{solutionbox}

\begin{mnemonicbox}
\mnemonic{લેસર પ્રકાશ: એકવર્ણીય, સુસંગત, દિશાત્મક, તીવ્ર}
\end{mnemonicbox}

\questionmarks{5(c) OR}{7}{ઓપ્ટિકલ ફાઈબર કેબલનું બંધારણને વિસ્તારમાં સમજાવો. અને સ્ટેપ ઇન્ડેક્સ અને ગ્રેડેડ ઇન્ડેક્સ ઓપ્ટિકલ ફાઈબર સમજાવો.}

\begin{solutionbox}
\textbf{ઓપ્ટિકલ ફાઈબર બંધારણ}:
\begin{enumerate}
    \item \textbf{કોર}: કેન્દ્રીય પ્રકાશ-પ્રસારિત કરનાર ભાગ (કાચ અથવા પ્લાસ્ટિક)
    \item \textbf{ક્લેડિંગ}: કોરને ઘેરે છે, કોર કરતાં ઓછા વક્રીભવનાંક સાથે
    \item \textbf{બફર કોટિંગ}: સુરક્ષાત્મક પ્લાસ્ટિક કોટિંગ
    \item \textbf{જેકેટ}: બાહ્ય સુરક્ષાત્મક આવરણ
\end{enumerate}

\textbf{આકૃતિ:}
\begin{center}
\begin{tikzpicture}
    % Concentric circles view
    \draw[fill=gray!10] (0,0) circle (2); % Jacket
    \draw[fill=gray!30] (0,0) circle (1.5); % Buffer
    \draw[fill=white] (0,0) circle (1); % Cladding
    \draw[fill=gray!50] (0,0) circle (0.5); % Core
    
    \node at (0,0) {કોર};
    \draw[->] (0, 0.8) -- (1.5, 2.5) node[above] {ક્લેડિંગ};
    \draw[->] (0, 1.3) -- (0, 2.5) node[above] {બફર કોટિંગ};
    \draw[->] (0, 1.8) -- (-1.5, 2.5) node[above] {જેકેટ};
    
    % Side view structure
    \draw[thick] (4, -0.5) rectangle (10, 0.5); % Core
    \draw[thick] (4, 0.5) rectangle (10, 1); % Cladding
    \draw[thick] (4, -0.5) rectangle (10, -1); % Cladding
    \draw[thick] (4, 1) rectangle (10, 1.2); % Buffer
    \draw[thick] (4, -1) rectangle (10, -1.2); % Buffer
    
    \node at (7, 0) {કોર};
    \node at (7, 0.75) {ક્લેડિંગ};
    \node at (7, -0.75) {ક્લેડિંગ};
    \node[right] at (10, 1.1) {બફર/જેકેટ};
\end{tikzpicture}
\captionof{figure}{ઓપ્ટિકલ ફાઈબર બંધારણ}
\end{center}

\textbf{સ્ટેપ ઇન્ડેક્સ ફાઈબર}:
\begin{itemize}
    \item કોર અને ક્લેડિંગ વચ્ચે વક્રીભવનાંકમાં અચાનક પરિવર્તન
    \item પ્રકાશ પૂર્ણ આંતરિક પરાવર્તન દ્વારા આડા-અવળા માર્ગમાં પ્રવાસ કરે છે
    \item ઉચ્ચ મોડલ ડિસ્પર્શન (સિગ્નલ ફેલાવો)
    \item સરળ બંધારણ
\end{itemize}

\textbf{ગ્રેડેડ ઇન્ડેક્સ ફાઈબર}:
\begin{itemize}
    \item કોરના કેન્દ્રથી ક્લેડિંગ સુધી વક્રીભવનાંકમાં ક્રમિક પરિવર્તન
    \item સતત વક્રીભવનને કારણે પ્રકાશ સર્પિલ માર્ગમાં પ્રવાસ કરે છે
    \item નિમ્ન મોડલ ડિસ્પર્શન
    \item વધુ જટિલ બંધારણ
\end{itemize}

\textbf{આકૃતિ: સિગ્નલ પ્રચાર}
\begin{center}
\begin{tikzpicture}
    % Step Index
    \node[left] at (0, 1) {સ્ટેપ ઇન્ડેક્સ:};
    \draw[thick] (0,0) rectangle (6,2);
    \draw[dashed] (0,1) -- (6,1);
    \draw[thick, red] (0,1) -- (1,0) -- (2,2) -- (3,0) -- (4,2) -- (5,0) -- (6,1);
    \node[right] at (6,1) {ઝિગ-ઝેગ પાથ};
    
    % Graded Index
    \node[left] at (0, -2) {ગ્રેડેડ ઇન્ડેક્સ:};
    \draw[thick] (0,-3) rectangle (6,-1);
    \draw[dashed] (0,-2) -- (6,-2);
    \draw[thick, blue, domain=0:6, samples=100] plot (\x, {-2 + 0.9*sin(\x r * 2)});
    \node[right] at (6,-2) {સર્પિલ પાથ};
\end{tikzpicture}
\captionof{figure}{સ્ટેપ ઇન્ડેક્સ બનામ ગ્રેડેડ ઇન્ડેક્સ}
\end{center}

\textbf{વક્રીભવનાંક પ્રોફાઇલ:}
\begin{center}
\begin{tikzpicture}
    % Step Index Profile
    \draw[->] (0,0) -- (2,0) node[right] {$r$}; 
    \draw[->] (0,0) -- (0,2); 
    \node[below] at (1, -0.2) {સ્ટેપ ઇન્ડેક્સ};
    
    \draw[thick] (0,1.5) -- (0.8, 1.5) -- (0.8, 0.5) -- (1.5, 0.5);
    \node[left] at (0, 1.5) {$n_1$};
    \node[left] at (0.8, 0.5) {$n_2$};
    
    % Graded Index Profile
    \draw[->] (4,0) -- (6,0) node[right] {$r$};
    \draw[->] (4,0) -- (4,2);
    \node[below] at (5, -0.2) {ગ્રેડેડ ઇન્ડેક્સ};
    
    \draw[thick] (4, 1.5) .. controls (4.8, 1.5) and (4.8, 0.5) .. (5.5, 0.5);
    \node[left] at (4, 1.5) {$n_1$};
    \node[left] at (5.5, 0.5) {$n_2$};
\end{tikzpicture}
\captionof{figure}{વક્રીભવનાંક પ્રોફાઇલ}
\end{center}
\end{solutionbox}

\begin{mnemonicbox}
\mnemonic{સ્ટેપ બતાવે અચાનક ફેરફાર, ગ્રેડેડ ધીમે ધીમે ઘટાડે}
\end{mnemonicbox}

\end{document}
