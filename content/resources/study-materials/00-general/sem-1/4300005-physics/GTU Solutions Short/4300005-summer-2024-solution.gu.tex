\documentclass{article}

% content/resources/templates/preamble.tex
\usepackage[margin=0.6in]{geometry}
\author{Milav Dabgar}
\usepackage{amsmath,amssymb,amsthm}
\usepackage{booktabs}
\usepackage{multirow}
\usepackage{xcolor}
\usepackage{tcolorbox}
\tcbuselibrary{breakable,skins}
\usepackage[colorlinks=true,linkcolor=blue]{hyperref}
\usepackage{titlesec}
\usepackage{enumitem}
\usepackage{tikz}
\usepackage{pgfplots}
\usepackage{circuitikz}
\usepackage[version=4]{mhchem}
\usepackage{longtable}
\usepackage{array}
\usepackage{float}
\usepackage{caption}
\usepackage{listings}

\lstset{
  basicstyle=\small\ttfamily,
  breaklines=true,
  breakatwhitespace=false,
  postbreak=\mbox{\textcolor{red}{$\hookrightarrow$}\space},
  float=false,
  numbers=left,
  numberstyle=\tiny\color{gray},
  numbersep=10pt,
  xleftmargin=2em,
  keywordstyle=\color{blue},
  commentstyle=\color{green!60!black},
  stringstyle=\color{purple},
  backgroundcolor=\color{gray!5},
  showstringspaces=false,
  tabsize=2,
  captionpos=b,
  keepspaces=true,
  columns=flexible
}

\pgfplotsset{compat=1.18}
\usetikzlibrary{shapes,arrows,positioning,calc,patterns,decorations.pathmorphing,decorations.markings,arrows.meta}

% Color scheme
\definecolor{headcolor}{RGB}{0,102,204}
\definecolor{keycolor}{RGB}{220,20,60}
\definecolor{solutioncolor}{RGB}{34,139,34}
\definecolor{mnemoniccolor}{RGB}{148,0,211}
\definecolor{codecolor}{RGB}{0,0,100}

% Spacing
\setlength{\parskip}{3pt}
\setlist[itemize]{nosep}
\setlist[enumerate]{nosep}

% Title formatting
\titleformat{\section}{\Large\bfseries\color{headcolor}}{\thesection}{1em}{}
\titleformat{\subsection}{\large\bfseries\color{headcolor}}{\thesubsection}{1em}{}

% Pandoc tightlist compatibility
\providecommand{\tightlist}{%
  \setlength{\itemsep}{0pt}\setlength{\parskip}{0pt}}

% Pandoc longtable compatibility
\newcounter{none}
\def\thenone{}


% content/resources/templates/gujarati-boxes.tex
\usepackage{fontspec}
\usepackage{polyglossia}

% Set Gujarati as main language (document is primarily in Gujarati)
% Note: gloss-gujarati.ldf doesn't exist in polyglossia, but it will use hyphenation patterns
\setdefaultlanguage{gujarati}
\setotherlanguage{english}

% Configure Gujarati font properly
% Use Language=Default to prevent polyglossia from trying to add language-specific features
% that don't exist for Gujarati, which causes "empty feature" warnings
\newfontfamily\gujaratifont[Script=Gujarati,AutoFakeBold=2.5,AutoFakeSlant=0.3]{Noto Sans Gujarati}
\setmainfont[Script=Gujarati,AutoFakeBold=2.5,AutoFakeSlant=0.3]{Noto Sans Gujarati}
% Use Noto Sans Gujarati for monospace to support Gujarati in text
\setmonofont[Scale=0.9]{Noto Sans Gujarati}

% Configure English to use the same font
\newfontfamily\englishfont[Script=Gujarati,AutoFakeBold=2.5,AutoFakeSlant=0.3]{Noto Sans Gujarati}

% Translations for polyglossia
\gappto\captionsgujarati{
  \renewcommand{\tablename}{કોષ્ટક}
  \renewcommand{\figurename}{આકૃતિ}
}

% Helper for TikZ nodes to ensure Gujarati font
\newcommand{\gu}[1]{{\gujaratifont #1}}

% Custom environments
\newtcolorbox{solutionbox}{
    breakable,
    enhanced,
    colback=solutioncolor!5!white,
    colframe=solutioncolor!75!black,
    fonttitle=\bfseries,
    title=જવાબ
}

\newtcolorbox{solutionboxnobreak}{
 colback=solutioncolor!5!white,
 colframe=solutioncolor!75!black,
 fonttitle=\bfseries,
 title=જવાબ
}

\newtcolorbox{keyformula}{
 breakable,
 enhanced,
 colback=keycolor!5!white,
 colframe=keycolor!75!black,
 fonttitle=\bfseries,
 title=રાસાયણિક સમીકરણ/સૂત્ર
}

\newtcolorbox{mnemonicbox}{
 breakable,
 enhanced,
 colback=mnemoniccolor!5!white,
 colframe=mnemoniccolor!75!black,
 fonttitle=\bfseries,
 title=મેમરી ટ્રીક
}


% Custom commands for GTU solutions
% This file defines semantic commands for consistent formatting

% Question command with automatic formatting
\newcommand{\question}[2]{%
  \section*{Question #1}%
  \textbf{#2}%
}

% OR question variant
\newcommand{\questionor}[2]{%
  \section*{Question #1 OR}%
  \textbf{#2}%
}

% Proper table environment with caption
\newenvironment{answertable}[1]{%
  \begin{table}[htbp]
  \centering
  \caption{#1}
}{%
  \end{table}
}

% Proper figure environment for diagrams
\newenvironment{answerdiagram}[1]{%
  \begin{figure}[htbp]
  \centering
  \caption{#1}
}{%
  \end{figure}
}

% Semantic markup for key terms
\newcommand{\keyword}[1]{\textbf{#1}}
\newcommand{\code}[1]{\texttt{#1}}
\newcommand{\classname}[1]{\texttt{#1}}
\newcommand{\methodname}[1]{\texttt{#1}}

% Proper quotation marks
\newcommand{\mnemonic}[1]{``#1''}


\title{ભૌતિકશાસ્ત્ર (4300005) - સમર 2024 સોલ્યુશન}
\date{જૂન 12, 2024}

\begin{document}
\maketitle

\questionmarks{1(a)}{3}{સાધિત ભૌતિક રાશીની વ્યાખ્યા લખો અને તેના કોઈ પણ ત્રણ ઉદાહરણોને એકમ અને ચિન્હ સાથે લખો.}

\begin{solutionbox}
સાધિત ભૌતિક રાશીઓ એ છે જે મૂળભૂત ભૌતિક રાશીઓના ગુણાકાર અથવા ભાગાકાર દ્વારા મેળવવામાં આવે છે.

\begin{center}
\captionof{table}{સાધિત ભૌતિક રાશીઓના ઉદાહરણો}
\begin{tabulary}{\linewidth}{|L|L|L|}
\hline
\textbf{સાધિત રાશી} & \textbf{S.I. એકમ} & \textbf{ચિહ્ન} \\ \hline
બળ & ન્યૂટન (N) & F \\ \hline
ઊર્જા & જૂલ (J) & E \\ \hline
વિદ્યુત પ્રવાહ & એમ્પિયર (A) & I \\ \hline
\end{tabulary}
\end{center}
\end{solutionbox}

\begin{mnemonicbox}
\mnemonic{FEI: બળ-ઊર્જા-વિદ્યુત પ્રવાહ મૂળભૂતમાંથી નિકળે છે}
\end{mnemonicbox}

\questionmarks{1(b)}{4}{ધાતુના સળિયાની લંબાઈ 12°C તાપમાને 64.522 cm છે અને 90°C તાપમાને 64.576 cm છે. તો સળિયાના રેખીય વિસ્તરણ ગુણાંક શોધો.}

\begin{solutionbox}
\textbf{સૂત્ર:} $\alpha = \frac{L_2 - L_1}{L_1 \times (T_2 - T_1)}$

\textbf{ગણતરી:}
\begin{itemize}
    \item પ્રારંભિક લંબાઈ ($L_1$) = 64.522 cm
    \item અંતિમ લંબાઈ ($L_2$) = 64.576 cm
    \item પ્રારંભિક તાપમાન ($T_1$) = 12°C
    \item અંતિમ તાપમાન ($T_2$) = 90°C
\end{itemize}

$$
\alpha = \frac{64.576 - 64.522}{64.522 \times (90 - 12)}
$$
$$
\alpha = \frac{0.054}{64.522 \times 78}
$$
$$
\alpha = \frac{0.054}{5032.716}
$$
$$
\alpha = 1.073 \times 10^{-5} /^\circ C
$$
\end{solutionbox}

\begin{mnemonicbox}
\mnemonic{લંબાઈમાં ફેરફાર પર મૂળ લંબાઈ અને તાપમાન ફેરફારનો ભાગ}
\end{mnemonicbox}

\questionmarks{1(c)}{7}{વર્નિયર કેલિપર્સનો સિદ્ધાંત, રચના અને કાર્ય પદ્ધતિ તેની આકૃતિ સાથે સમજાવો.}

\begin{solutionbox}
\textbf{સિદ્ધાંત}: વર્નિયર કેલિપર વર્નિયર સ્કેલના સિદ્ધાંત પર કામ કરે છે, જે મુખ્ય સ્કેલ કરતાં વધુ ચોકસાઈથી માપન કરવા દે છે.

\textbf{રચના:}
\begin{center}
\begin{tikzpicture}[node distance=1.5cm, auto]
    \node [gtu block] (V) {વર્નિયર કેલિપર};
    \node [gtu block, below left=1cm and -1cm of V] (M) {મુખ્ય સ્કેલ};
    \node [gtu block, below right=1cm and -1cm of V] (S) {વર્નિયર સ્કેલ};
    \node [gtu block, below=2.5cm of V] (J) {જડબાં (સ્થિર \& ચલિત)};
    
    \path [gtu arrow] (V) -- (M);
    \path [gtu arrow] (V) -- (S);
    \path [gtu arrow] (V) -- (J);
\end{tikzpicture}
\captionof{figure}{વર્નિયર કેલિપરના ઘટકો}
\end{center}

\textbf{આકૃતિ:}
\begin{center}
\begin{tikzpicture}
    % Main Scale
    \draw[thick] (0,0) rectangle (8,1);
    \foreach \x in {0,1,...,8} \draw (\x,1) -- (\x,0.7) node[above=3mm] {\x};
    \foreach \x in {0.5,1.5,...,7.5} \draw (\x,1) -- (\x,0.8);
    \node at (4,1.4) {મુખ્ય સ્કેલ (cm)};
    
    % Vernier Scale
    \draw[thick, fill=gray!10] (1.5,-0.5) rectangle (4.5,0);
    \foreach \x in {1.5,1.8,...,4.5} \draw (\x,0) -- (\x,-0.2);
    \node at (3,-0.8) {વર્નિયર સ્કેલ};
    
    % Jaws
    \draw[thick] (0,0) -- (0,-2) -- (0.5,-2) -- (0.5,-1) -- (1.5,-1) -- (1.5,0);
    \node at (0.25,-2.3) {સ્થિર જડબું};
    
    \draw[thick] (1.5,-0.5) -- (1.5,-2) -- (2,-2) -- (2,-1) -- (4.5,-1);
    \node at (1.75,-2.3) {ચલિત જડબું};
    
    % Object
    \draw[fill=blue!20] (0.5,-1.5) circle (0.5);
    \node at (0.5,-1.5) {વસ્તુ};
\end{tikzpicture}
\captionof{figure}{વર્નિયર કેલિપર}
\end{center}

\textbf{કાર્યપદ્ધતિ:}
\begin{itemize}
    \item \textbf{શૂન્ય ત્રુટિની તપાસ}: જડબાંઓ બંધ કરી વર્નિયરનો શૂન્ય મુખ્ય સ્કેલના શૂન્ય સાથે મેળ ખાય છે કે કેમ તે જોવું
    \item \textbf{બહારનું માપન}: વસ્તુને સ્થિર અને ચલિત જડબાં વચ્ચે મૂકો
    \item \textbf{વાંચન પ્રક્રિયા}: મુખ્ય સ્કેલ વાંચન + (મેળ ખાતા વર્નિયર વિભાગ $\times$ લઘુત્તમ માપ)
    \item \textbf{લઘુત્તમ માપ} = (મુખ્ય સ્કેલનો સૌથી નાનો વિભાગ)/(વર્નિયર સ્કેલના વિભાગોની સંખ્યા)
\end{itemize}
\end{solutionbox}

\begin{mnemonicbox}
\mnemonic{મુખ્ય સ્કેલ વાંચન વત્તા વર્નિયર ભાગ ગુણિયે લઘુત્તમ માપ}
\end{mnemonicbox}

\questionmarks{1(c OR)}{7}{માઇક્રોમિટર સ્ક્રૂ ગેજનો સિદ્ધાંત, રચના અને કાર્ય પદ્ધતિ તેની આકૃતિ સાથે સમજાવો.}

\begin{solutionbox}
\textbf{સિદ્ધાંત}: માઇક્રોમિટર સ્ક્રૂ ગેજ સ્ક્રૂની ગતિના સિદ્ધાંત પર કામ કરે છે - ફરતી ગતિને સીધી રેખાની ગતિમાં પરિવર્તિત કરવામાં આવે છે.

\textbf{રચના:}
\begin{center}
\begin{tikzpicture}[node distance=1.5cm, auto]
    \node [gtu block] (M) {માઇક્રોમિટર};
    \node [gtu block, below left=1cm of M] (F) {ફ્રેમ};
    \node [gtu block, below=1cm of M] (S) {સ્પિન્ડલ \& એનવિલ};
    \node [gtu block, below right=1cm of M] (T) {થિમ્બલ \& સ્લીવ};
    
    \path [gtu arrow] (M) -- (F);
    \path [gtu arrow] (M) -- (S);
    \path [gtu arrow] (M) -- (T);
\end{tikzpicture}
\captionof{figure}{માઇક્રોમિટરના ઘટકો}
\end{center}

\textbf{આકૃતિ:}
\begin{center}
\begin{tikzpicture}
    % Frame
    \draw[thick] (0,0) arc (180:360:2cm) -- (4,0) -- (4,2) -- (5,2) -- (5,-1) -- (4,-1);
    \draw[thick] (0,0) -- (0,1);
    
    % Anvil
    \draw[thick, fill=gray] (0,1) rectangle (0.5,1.5);
    \node[above] at (0.25,1.5) {એનવિલ};
    
    % Spindle
    \draw[thick, fill=gray] (0.5,1.1) rectangle (3.5,1.4);
    \node[above] at (2,1.4) {સ્પિન્ડલ};
    
    % Sleeve
    \draw[thick] (4,1) rectangle (6,1.5);
    \foreach \x in {4,4.2,...,6} \draw (\x,1.25) -- (\x,1.5);
    \draw (4,1.25) -- (6,1.25);
    \node[below] at (5,1) {સ્લીવ};
    
    % Thimble
    \draw[thick] (6,0.8) rectangle (8,1.7);
    \foreach \y in {0.8,1.0,...,1.7} \draw (6,\y) -- (6.2,\y);
    \node[right] at (8,1.25) {થિમ્બલ};
    
    % Ratchet
    \draw[thick] (8,1.1) rectangle (8.5,1.4);
    \node[right] at (8.5,1.25) {રેચેટ};
    
    % Frame Label
    \node at (2,-2.2) {U-ફ્રેમ};
\end{tikzpicture}
\captionof{figure}{માઇક્રોમિટર સ્ક્રૂ ગેજ}
\end{center}

\textbf{કાર્યપદ્ધતિ:}
\begin{itemize}
    \item \textbf{શૂન્ય ત્રુટિની તપાસ}: એનવિલ અને સ્પિન્ડલ બંધ કરી, ગોળાકાર સ્કેલનો શૂન્ય સંદર્ભ રેખા સાથે ગોઠવાય છે કે કેમ તપાસો
    \item \textbf{માપન પ્રક્રિયા}: વસ્તુને એનવિલ અને સ્પિન્ડલ વચ્ચે મૂકો
    \item \textbf{વાંચન}: મુખ્ય સ્કેલ વાંચન + (ગોળાકાર સ્કેલ વાંચન $\times$ લઘુત્તમ માપ)
    \item \textbf{લઘુત્તમ માપ} = પીચ/ગોળાકાર સ્કેલના વિભાગોની સંખ્યા
\end{itemize}
\end{solutionbox}

\begin{mnemonicbox}
\mnemonic{PST: પીચને સ્કેલથી ભાગીએ તો થિમ્બલનો લઘુત્તમ માપ મળે}
\end{mnemonicbox}

\questionmarks{2(a)}{3}{જો માઇક્રોમિટર સ્ક્રૂ ગેજની પિચ 1 mm હોય અને ગોળાકાર સ્કેલના કુલ 100 વિભાગ હોય તો ગોળાનો વ્યાસ શોધો. ગોળાકાર સ્કેલની ધાર મુખ્ય સ્કેલના 7 અને 8 mm વચ્ચે આવે છે અને ગોળાકાર સ્કેલના 65મો વિભાગ મુખ્ય સ્કેલની આડી રેખા સાથે મળે છે.}

\begin{solutionbox}
\textbf{સૂત્ર:} વ્યાસ = મુખ્ય સ્કેલ વાંચન + (ગોળાકાર સ્કેલ વાંચન $\times$ લઘુત્તમ માપ)

\textbf{ગણતરી:}
\begin{itemize}
    \item મુખ્ય સ્કેલ વાંચન = 7 mm
    \item ગોળાકાર સ્કેલ વાંચન = 65 વિભાગ
    \item લઘુત્તમ માપ = પીચ/વિભાગોની સંખ્યા = 1/100 = 0.01 mm
\end{itemize}

$$
\text{વ્યાસ} = 7 + (65 \times 0.01) = 7 + 0.65 = 7.65 \text{ mm}
$$
\end{solutionbox}

\begin{mnemonicbox}
\mnemonic{MSR + (CSR times LC) આપે છે અંતિમ માપણી}
\end{mnemonicbox}

\questionmarks{2(b)}{4}{કળા તફાવત અને સુસબદ્ધતા ને સમજાવો.}

\begin{solutionbox}
\textbf{કળા તફાવત:}
સમાન આવૃત્તિના બે તરંગો વચ્ચે કળા કોણનો તફાવત.

\begin{center}
\captionof{table}{કળા તફાવતની લાક્ષણિકતાઓ}
\begin{tabulary}{\linewidth}{|L|L|L|}
\hline
\textbf{કળા તફાવત} & \textbf{વ્યતિકરણનો પ્રકાર} & \textbf{પરિણામ} \\ \hline
0° અથવા 360° & રચનાત્મક & મહત્તમ કંપવિસ્તાર \\ \hline
180° & વિનાશક & લઘુત્તમ કંપવિસ્તાર \\ \hline
\end{tabulary}
\end{center}

\textbf{સુસબદ્ધતા:}
તરંગોની એવી ગુણવત્તા જેમાં કળા સંબંધ સતત રહે છે.

\textbf{સુસબદ્ધતાના પ્રકારો:}
\begin{itemize}
    \item \textbf{સમયગત સુસબદ્ધતા}: આવૃત્તિ સ્થિરતા સાથે સંબંધિત
    \item \textbf{અવકાશી સુસબદ્ધતા}: તરંગાગ્ર એકરૂપતા સાથે સંબંધિત
\end{itemize}
\end{solutionbox}

\begin{mnemonicbox}
\mnemonic{સતત કળા સંબંધ બનાવે સુસબદ્ધ તરંગો}
\end{mnemonicbox}

\questionmarks{2(c)}{7}{કેપેસિટર, કેપેસીટન્સ તથા સમાંતર પ્લેટ કેપેસિટરના કેપેસીટન્સ પર ડાઇલેટ્રિક મધ્યમની અસર સમજાવો.}

\begin{solutionbox}
\textbf{કેપેસિટર}: એવું ઉપકરણ જે વિદ્યુત ક્ષેત્રમાં વિદ્યુત ચાર્જ અને વિદ્યુત ઊર્જાને સંગ્રહિત કરે છે.

\textbf{કેપેસીટન્સ}: સંગ્રહિત ચાર્જનો લાગુ પોટેન્શિયલ તફાવત સાથેનો ગુણોત્તર.

\textbf{સૂત્ર:} $C = Q/V$

\textbf{સમાંતર પ્લેટ કેપેસિટર:}
કેપેસીટન્સ સૂત્ર: $C = \frac{\epsilon_0 A}{d}$
\begin{itemize}
    \item $\epsilon_0$ = મુક્ત અવકાશની પરાવૈદ્યુતાંક
    \item $A$ = પ્લેટનું ક્ષેત્રફળ
    \item $d$ = પ્લેટ વચ્ચેનું અંતર
\end{itemize}

\textbf{ડાઇલેક્ટ્રિકની અસર:}
\begin{itemize}
    \item કેપેસીટન્સને $K$ ગણો વધારે છે ($K$ = ડાઇલેક્ટ્રિક અચળાંક)
    \item નવું સૂત્ર: $C = \frac{K \epsilon_0 A}{d}$
\end{itemize}

\textbf{આકૃતિ:}
\begin{center}
\begin{tikzpicture}
    % Plates
    \draw[thick] (0,2) -- (4,2);
    \draw[thick] (0,0) -- (4,0);
    
    % Charges
    \foreach \x in {0.5,1,...,3.5} \node at (\x,2.3) {$+$};
    \foreach \x in {0.5,1,...,3.5} \node at (\x,-0.3) {$-$};
    
    % Dielectric
    \draw[fill=blue!10] (0.2,0.2) rectangle (3.8,1.8);
    \node at (2,1) {ડાઇલેક્ટ્રિક (K)};
    
    % Labels
    \draw[<->] (4.2,0) -- (4.2,2) node[midway, right] {$d$};
    \node at (2,2.8) {ક્ષેત્રફળ $A$};
    
    % Terminals
    \draw (2,2) -- (2,3);
    \draw (2,0) -- (2,-1);
\end{tikzpicture}
\captionof{figure}{ડાઇલેક્ટ્રિક સાથે સમાંતર પ્લેટ કેપેસિટર}
\end{center}
\end{solutionbox}

\begin{mnemonicbox}
\mnemonic{KIDS: K વધારે ડાઇલેક્ટ્રિક સંગ્રહ}
\end{mnemonicbox}

\questionmarks{2(a OR)}{3}{જો કોઈ બે નળાકારની લંબાઈ (6.52$\pm$0.01) cm અને (4.48$\pm$0.02) cm છે. તો તેમની લંબાઈના તફાવત ની પ્રતિશત ત્રુટિ મેળવો.}

\begin{solutionbox}
\textbf{ગણતરી:}
\begin{itemize}
    \item પ્રથમ નળાકારની લંબાઈ ($L_1$) = $6.52 \pm 0.01$ cm
    \item બીજા નળાકારની લંબાઈ ($L_2$) = $4.48 \pm 0.02$ cm
    \item લંબાઈનો તફાવત ($\Delta L$) = $L_1 - L_2 = 6.52 - 4.48 = 2.04$ cm
\end{itemize}

\textbf{તફાવતમાં નિરપેક્ષ ત્રુટિ} = $\sqrt{(0.01)^2 + (0.02)^2} = \sqrt{0.0001 + 0.0004} = \sqrt{0.0005} = 0.022$ cm

\textbf{પ્રતિશત ત્રુટિ} = $\frac{\text{નિરપેક્ષ ત્રુટિ}}{\text{માપેલી કિંમત}} \times 100$
$$
= \frac{0.022}{2.04} \times 100 = 1.08\%
$$
\end{solutionbox}

\begin{mnemonicbox}
\mnemonic{તફાવતની ગણતરી માટે ત્રુટિઓને વર્ગમાં ઉમેરો}
\end{mnemonicbox}

\questionmarks{2(b OR)}{4}{જરૂરી આકૃતિ સાથે વ્યતિકરણના પ્રકાર સમજાવો.}

\begin{solutionbox}
\textbf{વ્યતિકરણના પ્રકારો:}

\begin{center}
\captionof{table}{વ્યતિકરણ પ્રકારો}
\begin{tabulary}{\linewidth}{|L|L|L|L|}
\hline
\textbf{પ્રકાર} & \textbf{કળા તફાવત} & \textbf{પરિણામ} & \textbf{કંપવિસ્તાર} \\ \hline
રચનાત્મક & 0°, 360°, 720°... & પ્રબલીકરણ & મહત્તમ \\ \hline
વિનાશક & 180°, 540°, 900°... & રદ્દીકરણ & ન્યૂનતમ \\ \hline
\end{tabulary}
\end{center}

\textbf{રચનાત્મક વ્યતિકરણ:}
જ્યારે શિખર શિખરને મળે અથવા ખીણ ખીણને મળે ત્યારે.

\textbf{વિનાશક વ્યતિકરણ:}
જ્યારે શિખર ખીણને મળે ત્યારે.

\textbf{આકૃતિ:}
\begin{center}
\begin{tikzpicture}
    % Constructive
    \begin{scope}[xshift=0cm]
        \draw[blue] plot[domain=0:4*pi, samples=50] (\x/2, {0.5*sin(\x r)});
        \draw[red] plot[domain=0:4*pi, samples=50] (\x/2, {0.5*sin(\x r)});
        \node at (3,1) {તરંગ 1 \& 2 (સમાન કળા)};
        
        \draw[->] (3.2,0) -- (4,0);
        
        \draw[purple, thick] plot[domain=0:4*pi, samples=50] (\x/2+4.5, {sin(\x r)});
        \node at (7.5,1.2) {રચનાત્મક};
    \end{scope}
    
    % Destructive
    \begin{scope}[yshift=-2.5cm]
        \draw[blue] plot[domain=0:4*pi, samples=50] (\x/2, {0.5*sin(\x r)});
        \draw[red, dashed] plot[domain=0:4*pi, samples=50] (\x/2, {-0.5*sin(\x r)});
        \node at (3,1) {તરંગ 1 \& 2 (વિરુદ્ધ કળા)};
        
        \draw[->] (3.2,0) -- (4,0);
        
        \draw[purple, thick] (4.5,0) -- (10.8,0);
        \node at (7.5,0.5) {વિનાશક};
    \end{scope}
\end{tikzpicture}
\captionof{figure}{વ્યતિકરણના પ્રકારો}
\end{center}
\end{solutionbox}

\begin{mnemonicbox}
\mnemonic{શિખર + શિખર = રચનાત્મક, શિખર + ખીણ = વિનાશક}
\end{mnemonicbox}

\questionmarks{2(c OR)}{7}{બિંદુવત્ વિદ્યુતભારને કારણે વિદ્યુતસ્થિતિમાન માટેનું સમીકરણ તેની આકૃતિ સાથે તારવો.}

\begin{solutionbox}
\textbf{બિંદુ ચાર્જને કારણે પોટેન્શિયલ:}

\textbf{સૂત્ર વિકાસ:}
\begin{itemize}
    \item \textbf{વ્યાખ્યા}: એક પરીક્ષણ ચાર્જને અનંતથી તે બિંદુ સુધી લાવવા માટે એકમ ચાર્જ દીઠ કરેલું કાર્ય
    \item \textbf{સમીકરણ}: $V = W/q_0 = \int (F \cdot dr)$
\end{itemize}

\textbf{પગલે પગલે તારણ:}
\begin{enumerate}
    \item ચાર્જો વચ્ચેનું બળ (કુલોમ્બનો નિયમ): $F = \frac{1}{4\pi\epsilon_0} \times \frac{Qq}{r^2}$
    \item પરીક્ષણ ચાર્જ ખસેડવામાં કરેલું કાર્ય: $W = \int (F \cdot dr)$
    \item ત્રિજ્યા ગતિ માટે: $W = \frac{Q}{4\pi\epsilon_0} \times \int_{\infty}^{r} \frac{1}{r^2} dr$
    \item સંકલન: $W = \frac{Q}{4\pi\epsilon_0} \times [-\frac{1}{r}]_{\infty}^{r}$
    \item અંતિમ પરિણામ: $V = \frac{W}{q_0} = \frac{1}{4\pi\epsilon_0} \frac{Q}{r}$
\end{enumerate}

\textbf{અંતિમ સૂત્ર:} $V = \frac{1}{4\pi\epsilon_0} \frac{Q}{r}$

\textbf{આકૃતિ:}
\begin{center}
\begin{tikzpicture}
    % Charge Q
    \node[circle, draw, fill=red!20] (Q) at (0,0) {+Q};
    
    % Point P
    \node[circle, fill, inner sep=1.5pt] (P) at (4,0) {};
    \node[above] at (4,0) {P};
    
    % Distance r
    \draw[->] (Q) -- (P) node[midway, below] {$r$};
    
    % Infinity
    \node (Inf) at (7,0) {$\infty$};
    \draw[dashed, ->] (Inf) -- (P) node[midway, above] {પરીક્ષણ ચાર્જ $q_0$};
\end{tikzpicture}
\captionof{figure}{બિંદુ ચાર્જને કારણે પોટેન્શિયલ}
\end{center}
\end{solutionbox}

\begin{mnemonicbox}
\mnemonic{POD: Potential Over Distance અંતર પર પોટેન્શિયલ}
\end{mnemonicbox}

\questionmarks{3(a)}{3}{ઘર્ષણ અને ઇન્ડક્શન દ્વારા થતાં ચાર્જિંગ ને ટૂંકમાં સમજાવો.}

\begin{solutionbox}
\textbf{ઘર્ષણ દ્વારા ચાર્જિંગ:}
બે અલગ પદાર્થોને એકબીજા સાથે ઘસવાની પ્રક્રિયા.

\textbf{ઘર્ષણ ચાર્જિંગના પગલાં:}
\begin{itemize}
    \item ઇલેક્ટ્રોન એક પદાર્થથી બીજા પદાર્થમાં સ્થાનાંતરિત થાય છે
    \item ઇલેક્ટ્રોન ગુમાવતો પદાર્થ ધન ચાર્જિત થાય છે
    \item ઇલેક્ટ્રોન મેળવતો પદાર્થ ઋણ ચાર્જિત થાય છે
\end{itemize}

\textbf{ઇન્ડક્શન દ્વારા ચાર્જિંગ:}
સીધા સંપર્ક વિના ચાર્જિંગની પ્રક્રિયા.

\textbf{ઇન્ડક્શન ચાર્જિંગના પગલાં:}
\begin{itemize}
    \item ચાર્જિત પદાર્થને તટસ્થ વાહક નજીક લાવો
    \item તટસ્થ વાહકમાં ચાર્જનું પુનઃવિતરણ
    \item વાહકને ગ્રાઉન્ડ કરી ગ્રાઉન્ડ દૂર કરો
    \item ચાર્જિત પદાર્થને દૂર કરો
\end{itemize}
\end{solutionbox}

\begin{mnemonicbox}
\mnemonic{FTEE: ઘર્ષણ થી ઇલેક્ટ્રોન સરળતાથી ફેરવાય}
\end{mnemonicbox}

\questionmarks{3(b)}{4}{એક ટ્યુનીંગ ફોર્ક જેની આવૃત્તિ 256 Hz છે અને ગતિ 340 m/s છે. તેની (a) તરંગલંબાઈ અને (b) 50 કંપનમાં કાપેલું અંતર શોધો.}

\begin{solutionbox}
\textbf{સૂત્રો:}
\begin{itemize}
    \item તરંગલંબાઈ ($\lambda$) = ગતિ ($v$) / આવૃત્તિ ($f$)
    \item અંતર ($d$) = કંપનોની સંખ્યા ($n$) $\times$ તરંગલંબાઈ ($\lambda$)
\end{itemize}

\textbf{ગણતરી:}
(a) તરંગલંબાઈ ($\lambda$) = $v/f = 340/256 = 1.328$ m

(b) અંતર ($d$) = $n \times \lambda = 50 \times 1.328 = 66.4$ m
\end{solutionbox}

\begin{mnemonicbox}
\mnemonic{VFD: ગતિ, આવૃત્તિ અને અંતર એકબીજા સાથે જોડાયેલા છે}
\end{mnemonicbox}

\questionmarks{3(c)}{7}{બાયમેટાલીક થર્મોમિટરનો સિદ્ધાંત અને રચના ને આકૃતિ સાથે સમજાવો. તેના ફયદા તથા ગેરફયદા લખો.}

\begin{solutionbox}
\textbf{સિદ્ધાંત}: જુદી જુદી ધાતુઓ ગરમ થવા પર અલગ અલગ પ્રમાણમાં પ્રસરે છે, જેના કારણે પટ્ટી વળે છે.

\textbf{રચના:}
\begin{center}
\begin{tikzpicture}[node distance=1.5cm, auto]
    \node [gtu block] (B) {બાયમેટાલીક પટ્ટી};
    \node [gtu block, right=1cm of B] (P) {સૂચક};
    \node [gtu block, right=1cm of P] (S) {સ્કેલ};
    
    \path [gtu arrow] (B) -- (P);
    \path [gtu arrow] (P) -- (S);
\end{tikzpicture}
\captionof{figure}{રચના પ્રવાહ}
\end{center}

\textbf{કાર્યપદ્ધતિ:}
\begin{itemize}
    \item તાપમાન બદલાવાથી અલગ-અલગ પ્રસરણ દર થાય છે
    \item બાયમેટાલિક પટ્ટી ઓછા પ્રસરણ ગુણાંક વાળી ધાતુ તરફ વળે છે
    \item સૂચકની ગતિ તાપમાન દર્શાવે છે
\end{itemize}

\textbf{આકૃતિ:}
\begin{center}
\begin{tikzpicture}
    % Cold State
    \draw[thick, fill=gray!30] (0,0) rectangle (4,0.3);
    \draw[thick, fill=gray!60] (0,0.3) rectangle (4,0.6);
    \node[left] at (0,0.3) {સ્થિર છેડો};
    \node at (2,-0.5) {ઠંડુ (સીધું)};
    
    % Hot State
    \begin{scope}[xshift=6cm]
        \draw[thick, fill=gray!30] (0,0) arc (180:90:3cm) -- (3.3,3) arc (90:180:3.3cm) -- cycle;
        \draw[thick, fill=gray!60] (0,0.3) arc (180:90:2.7cm) -- (3,3) arc (90:180:3cm) -- cycle;
        \node at (1.5,-0.5) {ગરમ (વળેલું)};
        
        \draw[->] (2,2) -- (3,3.5);
        \node at (3.5,3.5) {આવર્તન};
    \end{scope}
\end{tikzpicture}
\captionof{figure}{બાયમેટાલિક પટ્ટીની કાર્યપદ્ધતિ}
\end{center}

\textbf{ફાયદા:}
\begin{itemize}
    \item સરળ, મજબૂત રચના
    \item વીજળી પુરવઠાની જરૂર નથી
    \item વિશાળ તાપમાન શ્રેણી
\end{itemize}

\textbf{ગેરફાયદા:}
\begin{itemize}
    \item અન્ય પ્રકારો કરતાં ઓછી ચોકસાઈ
    \item ધીમી પ્રતિક્રિયા સમય
    \item યાંત્રિક ઘસારાને આધીન
\end{itemize}
\end{solutionbox}

\begin{mnemonicbox}
\mnemonic{BEDS: બાયમેટાલિક તત્વો વિરૂપિત થાય તાણથી}
\end{mnemonicbox}

\questionmarks{3(a OR)}{3}{બિંદુવત વિદ્યુતભારથી ઉદ્ભવતા વિદ્યુતક્ષેત્ર ને સમજાવો.}

\begin{solutionbox}
\textbf{બિંદુ ચાર્જ પર કરેલું કાર્ય:}
વિદ્યુત ક્ષેત્ર $E$ માં બિંદુ ચાર્જ $q$ ને હલાવવામાં કરેલું કાર્ય.

\textbf{સૂત્ર:} $W = q(V_a - V_{\beta}) = q\Delta V$

જ્યાં:
\begin{itemize}
    \item $q$ = ખસેડાતો ચાર્જ
    \item $V_a$ = પ્રારંભિક સ્થિતિનું પોટેન્શિયલ
    \item $V_{\beta}$ = અંતિમ સ્થિતિનું પોટેન્શિયલ
    \item $\Delta V$ = પોટેન્શિયલ તફાવત
\end{itemize}

\textbf{મુખ્ય લક્ષણો:}
\begin{itemize}
    \item કાર્ય માર્ગથી સ્વતંત્ર છે
    \item વિદ્યુત ક્ષેત્રની વિરુદ્ધ ખસેડવામાં કાર્ય ધનાત્મક છે
    \item વિદ્યુત ક્ષેત્રની દિશામાં ખસેડવામાં કાર્ય ઋણાત્મક છે
\end{itemize}
\end{solutionbox}

\begin{mnemonicbox}
\mnemonic{PEW: પોટેન્શિયલ તફાવત $\times$ વિદ્યુત ચાર્જ = કાર્ય}
\end{mnemonicbox}

\questionmarks{3(b OR)}{4}{એક ધ્વનિનું તરંગ જેની ગતિ 0.33 km/s છે અને આવૃત્તિ 660 Hz છે. તે તરંગ 75 કંપન માં કેટલું અંતર કાપશે?}

\begin{solutionbox}
\textbf{સૂત્રો:}
\begin{itemize}
    \item તરંગલંબાઈ ($\lambda$) = ગતિ ($v$) / આવૃત્તિ ($f$)
    \item અંતર ($d$) = કંપનોની સંખ્યા ($n$) $\times$ તરંગલંબાઈ ($\lambda$)
\end{itemize}

\textbf{ગણતરી:}
\begin{itemize}
    \item ગતિનું રૂપાંતર: $v = 0.33 \text{ km/s} = 330 \text{ m/s}$
    \item તરંગલંબાઈ: $\lambda = v/f = 330/660 = 0.5 \text{ m}$
    \item અંતર: $d = n \times \lambda = 75 \times 0.5 = 37.5 \text{ m}$
\end{itemize}
\end{solutionbox}

\begin{mnemonicbox}
\mnemonic{FVW: આવૃત્તિમાં ગતિ ગુણતાં તરંગલંબાઈ મળે}
\end{mnemonicbox}

\questionmarks{3(c OR)}{7}{પારાવાળા થર્મોમિટરનો સિદ્ધાંત અને રચના આકૃતિ સાથે સમજાવો. તેના ફાયદા અને ગેર ફાયદા લખો.}

\begin{solutionbox}
\textbf{સિદ્ધાંત}: પારા થર્મોમિટર પારાના તાપીય પ્રસરણના સિદ્ધાંત પર કામ કરે છે.

\textbf{રચના:}
\begin{center}
\begin{tikzpicture}[node distance=1.5cm, auto]
    \node [gtu block] (B) {બલ્બ};
    \node [gtu block, right=1cm of B] (C) {કેશનળી};
    \node [gtu block, right=1cm of C] (V) {શૂન્યાવકાશ};
    
    \path [gtu arrow] (B) -- (C);
    \path [gtu arrow] (C) -- (V);
\end{tikzpicture}
\captionof{figure}{રચના ઘટકો}
\end{center}

\textbf{આકૃતિ:}
\begin{center}
\begin{tikzpicture}
    % Glass Stem
    \draw[thick] (0,0) rectangle (1,6);
    \node at (0.5,6.3) {કાચની નળી};
    
    % Capillary
    \draw[thick] (0.45,0.5) rectangle (0.55,5.5);
    \draw[fill=gray] (0.45,0.5) rectangle (0.55,3.5); % Mercury column
    
    % Bulb
    \draw[thick, fill=gray] (0.2,-0.5) arc (180:360:0.3cm) -- (0.8,0.5) -- (0.2,0.5) -- cycle;
    \node[right] at (0.8,-0.2) {પારાનો બલ્બ};
    
    % Scale
    \foreach \y in {1,1.5,...,5} \draw (0.6,\y) -- (0.8,\y);
    \node[right] at (0.8,5) {સ્કેલ};
    
    % Safety Bulb
    \draw[thick] (0.4,5.5) arc (180:0:0.1cm);
    \node[left] at (0.4,5.8) {સુરક્ષા};
\end{tikzpicture}
\captionof{figure}{પારા થર્મોમિટર}
\end{center}

\textbf{કાર્યપદ્ધતિ:}
\begin{itemize}
    \item પારો ગરમ થવાથી પ્રસરે છે
    \item પ્રસરણથી પારો કેશનળીમાં ઉપર ચઢે છે
    \item પારાના સ્તંભની ઊંચાઈ તાપમાન દર્શાવે છે
\end{itemize}

\textbf{ફાયદા:}
\begin{itemize}
    \item ઉચ્ચ ચોકસાઈ
    \item વિશાળ તાપમાન શ્રેણી (-38°C થી 357°C)
    \item પારાનું રૈખિક પ્રસરણ
    \item પારાના દોરાની સારી દૃશ્યતા
\end{itemize}

\textbf{ગેરફાયદા:}
\begin{itemize}
    \item પારો ઝેરી છે
    \item નાજુક કાચની રચના
    \item -38°C નીચે વાપરી શકાતું નથી
    \item તાપમાન ફેરફારોમાં ધીમી પ્રતિક્રિયા
\end{itemize}
\end{solutionbox}

\begin{mnemonicbox}
\mnemonic{MELT: પારો પ્રસરે રૈખિક તાપમાન સાથે}
\end{mnemonicbox}

\questionmarks{4(a)}{3}{સરખા માપના બે ધનઆયનને $5 \times 10^{-10}$ m અંતરથી અલગ રાખવામા આવ્યા છે. તેમના વચ્ચે લાગતું વિદ્યુત બળ $3.7 \times 10^{-9}$ N જેટલું છે. તો દરેક એટમ માથી કેટલા ઇલેક્ટ્રોન નીકળશે.}

\begin{solutionbox}
\textbf{સૂત્ર:} $F = \frac{1}{4\pi\epsilon_0} \times \frac{q_1q_2}{r^2}$

\textbf{ગણતરી:}
\begin{itemize}
    \item $F = 3.7 \times 10^{-9}$ N
    \item $r = 5 \times 10^{-10}$ m
    \item $q_1 = q_2 = ne$ ($n$ = ઇલેક્ટ્રોનની સંખ્યા, $e$ = ઇલેક્ટ્રોન ચાર્જ)
    \item $1/4\pi\epsilon_0 = 9 \times 10^9 \text{ Nm}^2/\text{C}^2$
    \item $e = 1.6 \times 10^{-19}$ C
\end{itemize}

$$
3.7 \times 10^{-9} = (9 \times 10^9) \times \frac{n^2 e^2}{(5 \times 10^{-10})^2}
$$
$$
3.7 \times 10^{-9} = (9 \times 10^9) \times \frac{n^2 \times (1.6 \times 10^{-19})^2}{25 \times 10^{-20}}
$$
ઉકેલ: $n = 1$ (દરેક પરમાણુમાંથી 1 ઇલેક્ટ્રોન નીકળ્યો)
\end{solutionbox}

\begin{mnemonicbox}
\mnemonic{FACE: બળ અસર કરે ચાર્જ સમાન રીતે}
\end{mnemonicbox}

\questionmarks{4(b)}{4}{સ્નેલનો નિયમ લખો અને તેનું સૂત્ર મેળવો.}

\begin{solutionbox}
\textbf{સ્નેલનો નિયમ}: આપાત કોણના સાઇનનો વક્રીભવન કોણના સાઇન સાથેનો ગુણોત્તર આપેલા માધ્યમના જોડા માટે અચળાંક છે.

\textbf{સૂત્ર:}
$$
\frac{\sin i}{\sin r} = \frac{n_2}{n_1} = \text{અચળાંક}
$$

\textbf{તારણના પગલાં:}
\begin{enumerate}
    \item પ્રકાશ વિવિધ માધ્યમોમાં વિવિધ ઝડપે પ્રવાસ કરે છે
    \item જ્યારે પ્રકાશ એક માધ્યમથી બીજા માધ્યમમાં પસાર થાય, ત્યારે તે દિશા બદલે છે
    \item ફર્મેટના ન્યૂનતમ સમયના સિદ્ધાંતનો ઉપયોગ કરીને
    \item ગતિઓનો ગુણોત્તર વક્રીભવન સૂચકાંકોના ગુણોત્તર સમાન છે
    \item અંતિમ સૂત્ર: $n_1 \sin i = n_2 \sin r$
\end{enumerate}

\textbf{આકૃતિ:}
\begin{center}
\begin{tikzpicture}
    % Interface
    \draw[thick] (-3,0) -- (3,0);
    \node[above] at (2.5,0) {માધ્યમ 1 ($n_1$)};
    \node[below] at (2.5,0) {માધ્યમ 2 ($n_2$)};
    
    % Normal
    \draw[dashed] (0,-2) -- (0,2);
    
    % Rays
    \draw[red, thick, ->] (-2,2) -- (0,0);
    \draw[red, thick, ->] (0,0) -- (1.5,-2);
    
    % Angles
    \draw (0,0.5) arc (90:135:0.5);
    \node at (-0.3,0.7) {$i$};
    
    \draw (0,-0.5) arc (270:306:0.5);
    \node at (0.3,-0.7) {$r$};
\end{tikzpicture}
\captionof{figure}{વક્રીભવન અને સ્નેલનો નિયમ}
\end{center}
\end{solutionbox}

\begin{mnemonicbox}
\mnemonic{SINIS: SIN I પર SIN R બરાબર વક્રીભવનાંક ગુણોત્તર}
\end{mnemonicbox}

\questionmarks{4(c)}{7}{અલ્ટ્રાસોનિક તરંગોના કોઈ પણ ત્રણ ઉપયોગો સમજાવો.}

\begin{solutionbox}
\textbf{અલ્ટ્રાસોનિક તરંગોના ઉપયોગો:}

\begin{center}
\captionof{table}{અલ્ટ્રાસોનિક ઉપયોગો}
\begin{tabulary}{\linewidth}{|L|L|L|}
\hline
\textbf{ઉપયોગ} & \textbf{સિદ્ધાંત} & \textbf{ઉપયોગિતા} \\ \hline
મેડિકલ ઇમેજિંગ & પેશીઓથી પરાવર્તન & આંતરિક અંગોનું વિઝ્યુઅલાઇઝેશન \\ \hline
NDT (બિન-વિનાશક પરીક્ષણ) & ખામીઓથી પરાવર્તન & સામગ્રીમાં ખામીઓ શોધવી \\ \hline
સફાઈ & કેવિટેશન અસર & ઘરેણાં, સર્જિકલ સાધનો સાફ કરવા \\ \hline
\end{tabulary}
\end{center}

\textbf{1. મેડિકલ ઇમેજિંગ (સોનોગ્રાફી):}
\begin{itemize}
    \item આવૃત્તિઓ: 1-10 MHz
    \item સિદ્ધાંત: પલ્સ-ઇકો તકનીક
    \item ઉપયોગો: ગર્ભસ્થ શિશુનું ઇમેજિંગ, અંગોનું સ્કેનિંગ, રક્ત પ્રવાહનું માપન
\end{itemize}

\textbf{2. ઔદ્યોગિક NDT:}
\begin{itemize}
    \item સામગ્રીમાં તિરાડો, છિદ્રો અને ખામીઓ શોધે છે
    \item ઉત્પાદનમાં ગુણવત્તા નિયંત્રણ
    \item સામગ્રીની જાડાઈનું માપન
\end{itemize}

\textbf{3. અલ્ટ્રાસોનિક સફાઈ:}
\begin{itemize}
    \item સૂક્ષ્મ બુદબુદો (કેવિટેશન) બનાવે છે
    \item સપાટીઓ પરથી દૂષિત પદાર્થોને દૂર કરે છે
    \item ઘરેણાં, ઑપ્ટિકલ ઘટકો, સર્જિકલ સાધનો માટે વપરાય છે
\end{itemize}
\end{solutionbox}

\begin{mnemonicbox}
\mnemonic{MIC: મેડિકલ, ઔદ્યોગિક, સફાઈ ઉપયોગો}
\end{mnemonicbox}

\questionmarks{4(a OR)}{3}{ત્રણ કેપેસિટર જેમના મૂલ્ય 5 $\mu$F, 10 $\mu$F અને 15 $\mu$F છે, તેમના શ્રેણી તથા સમાંતર જોડાણ માટેનો સમતુલ્ય કેપેસીટન્સ મેળવો.}

\begin{solutionbox}
\textbf{સમાંતર જોડાણ:}
$$
C_p = C_1 + C_2 + C_3 = 5 + 10 + 15 = 30 \mu F
$$

\textbf{શ્રેણી જોડાણ:}
$$
\frac{1}{C_s} = \frac{1}{C_1} + \frac{1}{C_2} + \frac{1}{C_3}
$$
$$
\frac{1}{C_s} = \frac{1}{5} + \frac{1}{10} + \frac{1}{15}
$$
$$
\frac{1}{C_s} = 0.2 + 0.1 + 0.067 = 0.367
$$
$$
C_s = \frac{1}{0.367} = 2.72 \mu F
$$
\end{solutionbox}

\begin{mnemonicbox}
\mnemonic{ASAP: શ્રેણીમાં ઉમેરો, સમાંતરમાં વ્યસ્ત ઉમેરો}
\end{mnemonicbox}

\questionmarks{4(b OR)}{4}{ઓપ્ટિકલ ફાઇબરની બનાવટને તેની આકૃતિ સાથે સમજાવો.}

\begin{solutionbox}
\textbf{ઓપ્ટિકલ ફાઇબરની રચના:}

\textbf{ઘટકો:}
\begin{itemize}
    \item કોર: પ્રકાશ સંચરણ માધ્યમ
    \item ક્લેડિંગ: ઓછા વક્રીભવનાંક સાથેનું બાહ્ય સ્તર
    \item બફર કોટિંગ: રક્ષણાત્મક પ્લાસ્ટિક આવરણ
\end{itemize}

\textbf{પરિમાણો:}
\begin{itemize}
    \item કોર વ્યાસ: 8-50 $\mu$m (સિંગલ મોડ), 50-100 $\mu$m (મલ્ટિમોડ)
    \item ક્લેડિંગ વ્યાસ: 125-140 $\mu$m
    \item કોર વક્રીભવનાંક $>$ ક્લેડિંગ વક્રીભવનાંક
\end{itemize}

\textbf{આકૃતિ:}
\begin{center}
\begin{tikzpicture}
    % Longitudinal view
    \draw[thick, fill=cyan!10] (0,0) rectangle (6,2); % Buffer
    \draw[thick, fill=gray!20] (0,0.3) rectangle (6,1.7); % Cladding
    \draw[thick, fill=white] (0,0.8) rectangle (6,1.2); % Core
    
    \node at (3,1) {કોર};
    \node at (3,1.5) {ક્લેડિંગ};
    \node at (3,1.85) {બફર};
    
    % Light Ray
    \draw[red, thick] (-1,0.5) -- (0,1) -- (1,0.8) -- (2,1.2) -- (3,0.8) -- (4,1.2) -- (5,0.8) -- (6,1.2);
    \node[red, left] at (-1,0.5) {પ્રકાશ};
    
    % Cross Section
    \begin{scope}[xshift=8cm, yshift=1cm]
        \draw[thick, fill=cyan!10] (0,0) circle (1.5);
        \draw[thick, fill=gray!20] (0,0) circle (1);
        \draw[thick, fill=white] (0,0) circle (0.5);
        
        \node at (0,0) {કોર};
        \node at (0,1.2) {ક્લેડિંગ};
        \node at (0,1.7) {બફર};
    \end{scope}
\end{tikzpicture}
\captionof{figure}{ઓપ્ટિકલ ફાઇબરની રચના}
\end{center}
\end{solutionbox}

\begin{mnemonicbox}
\mnemonic{CBC: કોર-બફર-ક્લેડિંગ અંદરથી બહાર}
\end{mnemonicbox}

\questionmarks{4(c OR)}{7}{મગ્નેટોસ્ટ્રીકશન પદ્ધતિ દ્વારા અલ્ટ્રાસોનિક તરંગનું ઉત્પાદન સમજાવો.}

\begin{solutionbox}
\textbf{મેગ્નેટોસ્ટ્રિક્શન પદ્ધતિ:}
ફેરોમેગ્નેટિક પદાર્થોના ચુંબકીય ક્ષેત્રમાં મૂકવાથી તેના પરિમાણમાં ફેરફાર થવાના ગુણધર્મનો ઉપયોગ કરીને અલ્ટ્રાસોનિક તરંગો પેદા કરવાની પ્રક્રિયા.

\textbf{સિદ્ધાંત:}
ફેરોમેગ્નેટિક પદાર્થો ચુંબકીત થવા પર લંબાઈમાં ફેરફાર કરે છે, જે યાંત્રિક કંપનો પેદા કરે છે અને અલ્ટ્રાસોનિક તરંગો ઉત્પન્ન કરે છે.

\textbf{રચના:}
\begin{center}
\begin{tikzpicture}[node distance=1.5cm, auto]
    \node [gtu block] (P) {AC પાવર};
    \node [gtu block, below=1cm of P] (C) {કોઇલ/સોલેનોઇડ};
    \node [gtu block, below=1cm of C] (R) {ફેરો. સળિયો};
    \node [gtu block, right=1cm of R] (U) {અલ્ટ્રાસોનિક તરંગો};
    
    \path [gtu arrow] (P) -- (C);
    \path [gtu arrow] (C) -- (R);
    \path [gtu arrow] (R) -- (U);
\end{tikzpicture}
\captionof{figure}{મેગ્નેટોસ્ટ્રિક્શન પ્રક્રિયા}
\end{center}

\textbf{કાર્યપ્રક્રિયા:}
\begin{enumerate}
    \item AC કરંટ સોલેનોઇડમાંથી પસાર થાય છે
    \item પરિવર્તનશીલ ચુંબકીય ક્ષેત્ર ઉત્પન્ન થાય છે
    \item ફેરોમેગ્નેટિક સળિયો ફૂલે છે અને સંકોચાય છે
    \item કંપનો માધ્યમમાં પ્રસારિત થાય છે
    \item અલ્ટ્રાસોનિક તરંગો ઉત્પન્ન થાય છે
\end{enumerate}

\textbf{આકૃતિ:}
\begin{center}
\begin{tikzpicture}
    % Rod
    \draw[thick, fill=gray!30] (2,0) rectangle (6,1);
    \node at (4,0.5) {ફેરોમેગ્નેટિક સળિયો};
    
    % Coil
    \foreach \x in {2.2,2.6,...,5.8} \draw[thick] (\x,1) arc (180:0:0.1cm and 0.2cm);
    \foreach \x in {2.2,2.6,...,5.8} \draw[thick, dashed] (\x,0) arc (180:360:0.1cm and 0.2cm);
    \draw[thick] (2.2,1.2) -- (2.2,2) -- (5.8,2) -- (5.8,1.2);
    \node[above] at (4,2) {AC સપ્લાય};
    
    % Waves
    \foreach \x in {6.2,6.4,...,7.0} \draw[blue] (\x,0) arc (-90:90:0.5);
    \node[right] at (7,0.5) {તરંગો};
\end{tikzpicture}
\captionof{figure}{મેગ્નેટોસ્ટ્રિક્શન કંપન}
\end{center}

\textbf{ફાયદા:}
\begin{itemize}
    \item સરળ બંધારણ
    \item ઉચ્ચ શક્તિ આઉટપુટ
    \item પ્રવાહીઓ માટે યોગ્ય
\end{itemize}

\textbf{ગેરફાયદા:}
\begin{itemize}
    \item 100 kHz નીચેની આવૃત્તિઓ સુધી મર્યાદિત
    \item ગરમી અસરો
    \item ઓછી કાર્યક્ષમતા
\end{itemize}
\end{solutionbox}

\begin{mnemonicbox}
\mnemonic{FAME: ફેરોમેગ્નેટિક પરિવર્તિત ચુંબકીય અસર}
\end{mnemonicbox}

\questionmarks{5(a)}{3}{ઉષ્મા પ્રસરણના ત્રણ પ્રકારને ટૂંકમાં સમજાવો.}

\begin{solutionbox}
\textbf{ઉષ્મા પ્રસરણના ત્રણ પ્રકારો:}

\begin{center}
\captionof{table}{ઉષ્મા પ્રસરણ મોડ્સ}
\begin{tabulary}{\linewidth}{|L|L|L|}
\hline
\textbf{પ્રકાર} & \textbf{માધ્યમની આવશ્યકતા} & \textbf{ઉદાહરણ} \\ \hline
વહન & ભૌતિક સંપર્ક & ધાતુના સળિયા દ્વારા ઉષ્મા \\ \hline
સંવહન & પ્રવાહી માધ્યમ & ગરમ હવા ઊપર ચઢવી \\ \hline
વિકિરણ & કોઈ માધ્યમની જરૂર નથી & સૂર્યથી ઉષ્મા \\ \hline
\end{tabulary}
\end{center}

\begin{enumerate}
    \item \textbf{વહન:}
    \begin{itemize}
        \item સીધા અણુઓના અથડામણ દ્વારા પ્રસરણ
        \item પદાર્થની જથ્થાબંધ ગતિવિધિ નથી
        \item ઘન પદાર્થોમાં સારું, ખાસ કરીને ધાતુઓમાં
    \end{itemize}
    
    \item \textbf{સંવહન:}
    \begin{itemize}
        \item પ્રવાહી ગતિ દ્વારા પ્રસરણ
        \item ઘનતામાં તફાવતની જરૂર પડે છે
        \item કુદરતી અથવા દબાણપૂર્વક સંવહન
    \end{itemize}
    
    \item \textbf{વિકિરણ:}
    \begin{itemize}
        \item વિદ્યુત ચુંબકીય તરંગો દ્વારા પ્રસરણ
        \item નિર્વાતમાં કામ કરે છે
        \item તાપમાન અને સપાટી ગુણધર્મો પર આધાર રાખે છે
    \end{itemize}
\end{enumerate}
\end{solutionbox}

\begin{mnemonicbox}
\mnemonic{CCR: વહન સંપર્ક, સંવહન પ્રવાહ, વિકિરણ કિરણો}
\end{mnemonicbox}

\questionmarks{5(b)}{4}{એક ઓપ્ટિકલ ફાઇબરના કોર અને ક્લેડિંગના વક્રીભવાંક અનુક્રમે 1.55 અને 1.5 છે. તો તેનો ન્યુમેરિકલ એપર્ચર અને એકપ્ટન્સ એંગલ શોધો.}

\begin{solutionbox}
\textbf{સૂત્રો:}
\begin{itemize}
    \item ન્યુમેરિકલ એપર્ચર (NA) = $\sqrt{n_1^2 - n_2^2}$
    \item સ્વીકૃતિ કોણ ($\theta_a$) = $\sin^{-1}(\text{NA})$
\end{itemize}

\textbf{ગણતરી:}
\begin{itemize}
    \item કોર વક્રીભવનાંક ($n_1$) = 1.55
    \item ક્લેડિંગ વક્રીભવનાંક ($n_2$) = 1.5
\end{itemize}

$$
\text{NA} = \sqrt{1.55^2 - 1.5^2} = \sqrt{2.4025 - 2.25} = \sqrt{0.1525} = 0.391
$$

$$
\theta_a = \sin^{-1}(0.391) = 23.03^\circ
$$
\end{solutionbox}

\begin{mnemonicbox}
\mnemonic{CORE: કોર ઓપ્ટિકલ રેફ્રેક્ટિવ-ઇન્ડેક્સ ચોક્કસપણે ગણો}
\end{mnemonicbox}

\questionmarks{5(c)}{7}{ઓપ્ટિકલ ફાઈબરના કોઈ પણ ત્રણ ઉપયોગો સમજાવો.}

\begin{solutionbox}
\textbf{ઓપ્ટિકલ ફાઇબરના ઉપયોગો:}

\begin{center}
\captionof{table}{મુખ્ય ઓપ્ટિકલ ફાઇબર ઉપયોગો}
\begin{tabulary}{\linewidth}{|L|L|L|}
\hline
\textbf{ઉપયોગ} & \textbf{ફાયદો} & \textbf{ઉદાહરણ} \\ \hline
સંચાર & ઉચ્ચ બેન્ડવિડ્થ & ઇન્ટરનેટ, ફોન નેટવર્ક \\ \hline
મેડિકલ & લવચીકતા, ઇમેજિંગ & એન્ડોસ્કોપી \\ \hline
સેન્સર & ઇએમઆઈથી રક્ષણ & તાપમાન સેન્સિંગ \\ \hline
\end{tabulary}
\end{center}

\textbf{1. સંચાર નેટવર્ક:}
\begin{itemize}
    \item ટેલિકોમ્યુનિકેશન અને ઇન્ટરનેટ
    \item કોપર કેબલ્સ કરતાં વધુ બેન્ડવિડ્થ
    \item લાંબા અંતર પર ઓછું સિગ્નલ ઘટાડો
\end{itemize}

\textbf{2. મેડિકલ એપ્લિકેશન:}
\begin{itemize}
    \item મિનિમલ ઇન્વેસિવ પ્રક્રિયાઓ માટે એન્ડોસ્કોપી
    \item ફોટોડાયનેમિક થેરાપી માટે પ્રકાશ ડિલિવરી
    \item સર્જિકલ પ્રકાશ
\end{itemize}

\textbf{3. સેન્સિંગ એપ્લિકેશન:}
\begin{itemize}
    \item તાપમાન અને દબાણ સેન્સર
    \item માળખાકીય મોનિટરિંગ માટે સ્ટ્રેન ગેજ
    \item નેવિગેશન માટે જાયરોસ્કોપ
\end{itemize}
\end{solutionbox}

\begin{mnemonicbox}
\mnemonic{CMS: સંચાર, મેડિકલ, સેન્સિંગ ઉપયોગો}
\end{mnemonicbox}

\questionmarks{5(a OR)}{3}{વિશિષ્ટ ઉષ્માને વિસ્તારથી સમજાવો.}

\begin{solutionbox}
\textbf{વિશિષ્ટ ઉષ્મા:}
1 કિલોગ્રામ પદાર્થનું તાપમાન 1 કેલ્વિન (અથવા 1°C) વધારવા માટે જરૂરી ઉષ્મા.

\textbf{સૂત્ર:} $Q = mc\Delta T$

જ્યાં:
\begin{itemize}
    \item $Q$ = ઉષ્મા ઊર્જા (J)
    \item $m$ = દ્રવ્યમાન (kg)
    \item $c$ = વિશિષ્ટ ઉષ્મા ક્ષમતા (J/kg·K)
    \item $\Delta T$ = તાપમાન ફેરફાર (K)
\end{itemize}

\textbf{એકમો:} J/kg·K અથવા J/kg·°C

\textbf{મહત્વ:}
\begin{itemize}
    \item પદાર્થોની થર્મલ જડતા માપે છે
    \item ઉચ્ચ વિશિષ્ટ ઉષ્માનો અર્થ પદાર્થને ગરમ કરવા માટે વધુ ઊર્જાની જરૂર પડે છે
    \item પાણીની અસામાન્ય રીતે ઉચ્ચ વિશિષ્ટ ઉષ્મા છે (4,186 J/kg·K)
\end{itemize}
\end{solutionbox}

\begin{mnemonicbox}
\mnemonic{STEM: વિશિષ્ટ ઉષ્મા માપે તાપમાન ફેરફાર ઊર્જા અને દ્રવ્યમાન દીઠ}
\end{mnemonicbox}

\questionmarks{5(b OR)}{4}{એક ઓપ્ટિકલ ફાઇબરના કોર અને ક્લેડિંગના વક્રીભવાંક અનુક્રમે 1.48 અને 1.45 છે. તો તેનો એકપ્ટન્સ એંગલ અને ક્રાંતિકોણ શોધો.}

\begin{solutionbox}
\textbf{સૂત્રો:}
\begin{itemize}
    \item ન્યુમેરિકલ એપર્ચર (NA) = $\sqrt{n_1^2 - n_2^2}$
    \item સ્વીકૃતિ કોણ ($\theta_a$) = $\sin^{-1}(\text{NA})$
    \item ક્રાંતિક કોણ ($\theta_c$) = $\sin^{-1}(n_2/n_1)$
\end{itemize}

\textbf{ગણતરી:}
\begin{itemize}
    \item કોર વક્રીભવનાંક ($n_1$) = 1.48
    \item ક્લેડિંગ વક્રીભવનાંક ($n_2$) = 1.45
\end{itemize}

$$
\text{NA} = \sqrt{1.48^2 - 1.45^2} = \sqrt{2.1904 - 2.1025} = \sqrt{0.0879} = 0.296
$$
$$
\theta_a = \sin^{-1}(0.296) = 17.2^\circ
$$
$$
\theta_c = \sin^{-1}(1.45/1.48) = \sin^{-1}(0.9797) = 78.4^\circ
$$
\end{solutionbox}

\begin{mnemonicbox}
\mnemonic{NA થી AA મળે, ગુણોત્તર થી ક્રાંતિક કોણ મળે}
\end{mnemonicbox}

\questionmarks{5(c OR)}{7}{ઈજનેરી અને મેડીકલ ક્ષેત્રમાં LASER ના ઉપયોગો સમજાવો.}

\begin{solutionbox}
\textbf{LASER ના ઉપયોગો:}

\begin{center}
\captionof{table}{LASER ઉપયોગો}
\begin{tabulary}{\linewidth}{|L|L|L|}
\hline
\textbf{ક્ષેત્ર} & \textbf{ઉપયોગ} & \textbf{ઉદાહરણ} \\ \hline
ઇજનેરી & કટિંગ/વેલ્ડિંગ & ધાતુ ફેબ્રિકેશન \\ \hline
ઇજનેરી & માપન & અંતર માપન \\ \hline
મેડિકલ & સર્જરી & આંખની સર્જરી (LASIK) \\ \hline
મેડિકલ & થેરાપી & કેન્સર સારવાર \\ \hline
\end{tabulary}
\end{center}

\textbf{ઇજનેરી ઉપયોગો:}
\begin{enumerate}
    \item \textbf{મટિરિયલ પ્રોસેસિંગ:}
    \begin{itemize}
        \item ધાતુ, પ્લાસ્ટિક, સિરામિક્સનું ચોક્કસ કટિંગ
        \item અસમાન સામગ્રીની વેલ્ડિંગ
        \item 3D પ્રિન્ટિંગ અને રેપિડ પ્રોટોટાઇપિંગ
    \end{itemize}
    
    \item \textbf{મેટ્રોલોજી અને માપન:}
    \begin{itemize}
        \item ઉચ્ચ ચોકસાઈ સાથે અંતર માપન
        \item બાંધકામ અને ઉત્પાદનમાં એલાઇનમેન્ટ
        \item 3D ઇમેજિંગ માટે હોલોગ્રાફી
    \end{itemize}
\end{enumerate}

\textbf{મેડિકલ ઉપયોગો:}
\begin{enumerate}
    \item \textbf{સર્જિકલ પ્રક્રિયાઓ:}
    \begin{itemize}
        \item આંખની સર્જરી (LASIK, મોતિયા નિકાલ)
        \item મિનિમલી ઇન્વેસિવ પ્રક્રિયાઓ
        \item દંત પ્રક્રિયાઓ
    \end{itemize}
    
    \item \textbf{થેરાપ્યુટિક ઉપયોગો:}
    \begin{itemize}
        \item કેન્સર માટે ફોટોડાયનેમિક થેરાપી
        \item દર્દ માટે લો-લેવલ લેસર થેરાપી
        \item કોસ્મેટિક પ્રક્રિયાઓ
    \end{itemize}
\end{enumerate}

\textbf{આકૃતિ:}
\begin{center}
\begin{tikzpicture}[node distance=1.5cm, auto]
    \node [gtu block] (L) {LASER};
    \node [gtu block, below left=1.5cm of L] (E) {ઇજનેરી};
    \node [gtu block, below right=1.5cm of L] (M) {મેડિકલ};
    
    \path [gtu arrow] (L) -- (E);
    \path [gtu arrow] (L) -- (M);
    
    \node [gtu block, below=0.5cm of E] (E1) {કટિંગ/વેલ્ડિંગ};
    \node [gtu block, below=0.5cm of M] (M1) {સર્જરી/થેરાપી};
    
    \path [gtu arrow] (E) -- (E1);
    \path [gtu arrow] (M) -- (M1);
\end{tikzpicture}
\captionof{figure}{LASER ના ઉપયોગોના ક્ષેત્રો}
\end{center}
\end{solutionbox}

\begin{mnemonicbox}
\mnemonic{SMART: સર્જરી, માપન, વિશ્લેષણ, રિપેર, અને ટ્રીટમેન્ટ}
\end{mnemonicbox}

\end{document}
