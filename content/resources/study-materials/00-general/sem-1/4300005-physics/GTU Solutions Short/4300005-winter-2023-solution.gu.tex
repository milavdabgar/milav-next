\documentclass{article}

% content/resources/templates/preamble.tex
\usepackage[margin=0.6in]{geometry}
\author{Milav Dabgar}
\usepackage{amsmath,amssymb,amsthm}
\usepackage{booktabs}
\usepackage{multirow}
\usepackage{xcolor}
\usepackage{tcolorbox}
\tcbuselibrary{breakable,skins}
\usepackage[colorlinks=true,linkcolor=blue]{hyperref}
\usepackage{titlesec}
\usepackage{enumitem}
\usepackage{tikz}
\usepackage{pgfplots}
\usepackage{circuitikz}
\usepackage[version=4]{mhchem}
\usepackage{longtable}
\usepackage{array}
\usepackage{float}
\usepackage{caption}
\usepackage{listings}

\lstset{
  basicstyle=\small\ttfamily,
  breaklines=true,
  breakatwhitespace=false,
  postbreak=\mbox{\textcolor{red}{$\hookrightarrow$}\space},
  float=false,
  numbers=left,
  numberstyle=\tiny\color{gray},
  numbersep=10pt,
  xleftmargin=2em,
  keywordstyle=\color{blue},
  commentstyle=\color{green!60!black},
  stringstyle=\color{purple},
  backgroundcolor=\color{gray!5},
  showstringspaces=false,
  tabsize=2,
  captionpos=b,
  keepspaces=true,
  columns=flexible
}

\pgfplotsset{compat=1.18}
\usetikzlibrary{shapes,arrows,positioning,calc,patterns,decorations.pathmorphing,decorations.markings,arrows.meta}

% Color scheme
\definecolor{headcolor}{RGB}{0,102,204}
\definecolor{keycolor}{RGB}{220,20,60}
\definecolor{solutioncolor}{RGB}{34,139,34}
\definecolor{mnemoniccolor}{RGB}{148,0,211}
\definecolor{codecolor}{RGB}{0,0,100}

% Spacing
\setlength{\parskip}{3pt}
\setlist[itemize]{nosep}
\setlist[enumerate]{nosep}

% Title formatting
\titleformat{\section}{\Large\bfseries\color{headcolor}}{\thesection}{1em}{}
\titleformat{\subsection}{\large\bfseries\color{headcolor}}{\thesubsection}{1em}{}

% Pandoc tightlist compatibility
\providecommand{\tightlist}{%
  \setlength{\itemsep}{0pt}\setlength{\parskip}{0pt}}

% Pandoc longtable compatibility
\newcounter{none}
\def\thenone{}


% content/resources/templates/gujarati-boxes.tex
\usepackage{fontspec}
\usepackage{polyglossia}

% Set Gujarati as main language (document is primarily in Gujarati)
% Note: gloss-gujarati.ldf doesn't exist in polyglossia, but it will use hyphenation patterns
\setdefaultlanguage{gujarati}
\setotherlanguage{english}

% Configure Gujarati font properly
% Use Language=Default to prevent polyglossia from trying to add language-specific features
% that don't exist for Gujarati, which causes "empty feature" warnings
\newfontfamily\gujaratifont[Script=Gujarati,AutoFakeBold=2.5,AutoFakeSlant=0.3]{Noto Sans Gujarati}
\setmainfont[Script=Gujarati,AutoFakeBold=2.5,AutoFakeSlant=0.3]{Noto Sans Gujarati}
% Use Noto Sans Gujarati for monospace to support Gujarati in text
\setmonofont[Scale=0.9]{Noto Sans Gujarati}

% Configure English to use the same font
\newfontfamily\englishfont[Script=Gujarati,AutoFakeBold=2.5,AutoFakeSlant=0.3]{Noto Sans Gujarati}

% Translations for polyglossia
\gappto\captionsgujarati{
  \renewcommand{\tablename}{કોષ્ટક}
  \renewcommand{\figurename}{આકૃતિ}
}

% Helper for TikZ nodes to ensure Gujarati font
\newcommand{\gu}[1]{{\gujaratifont #1}}

% Custom environments
\newtcolorbox{solutionbox}{
    breakable,
    enhanced,
    colback=solutioncolor!5!white,
    colframe=solutioncolor!75!black,
    fonttitle=\bfseries,
    title=જવાબ
}

\newtcolorbox{solutionboxnobreak}{
 colback=solutioncolor!5!white,
 colframe=solutioncolor!75!black,
 fonttitle=\bfseries,
 title=જવાબ
}

\newtcolorbox{keyformula}{
 breakable,
 enhanced,
 colback=keycolor!5!white,
 colframe=keycolor!75!black,
 fonttitle=\bfseries,
 title=રાસાયણિક સમીકરણ/સૂત્ર
}

\newtcolorbox{mnemonicbox}{
 breakable,
 enhanced,
 colback=mnemoniccolor!5!white,
 colframe=mnemoniccolor!75!black,
 fonttitle=\bfseries,
 title=મેમરી ટ્રીક
}


% Custom commands for GTU solutions
% This file defines semantic commands for consistent formatting

% Question command with automatic formatting
\newcommand{\question}[2]{%
  \section*{Question #1}%
  \textbf{#2}%
}

% OR question variant
\newcommand{\questionor}[2]{%
  \section*{Question #1 OR}%
  \textbf{#2}%
}

% Proper table environment with caption
\newenvironment{answertable}[1]{%
  \begin{table}[htbp]
  \centering
  \caption{#1}
}{%
  \end{table}
}

% Proper figure environment for diagrams
\newenvironment{answerdiagram}[1]{%
  \begin{figure}[htbp]
  \centering
  \caption{#1}
}{%
  \end{figure}
}

% Semantic markup for key terms
\newcommand{\keyword}[1]{\textbf{#1}}
\newcommand{\code}[1]{\texttt{#1}}
\newcommand{\classname}[1]{\texttt{#1}}
\newcommand{\methodname}[1]{\texttt{#1}}

% Proper quotation marks
\newcommand{\mnemonic}[1]{``#1''}


\title{ભૌતિકશાસ્ત્ર (4300005) - શિયાળુ 2023 સોલ્યુશન}
\date{જાન્યુઆરી 16, 2024}

\begin{document}
\maketitle

\questionmarks{1(અ)}{3}{વ્યાખ્યા આપો: (અ) મીટર (બ) કેલ્વિન (ક) ચોકસાઇ.}

\begin{solutionbox}
\begin{itemize}
    \item \textbf{મીટર}: મીટર એ લંબાઈનો SI એકમ છે, જેને 1/299,792,458 સેકન્ડના સમયગાળા દરમિયાન પ્રકાશ દ્વારા શૂન્યાવકાશમાં કાપવામાં આવતા અંતર તરીકે વ્યાખ્યાયિત કરવામાં આવે છે.
    \item \textbf{કેલ્વિન}: કેલ્વિન એ થર્મોડાયનામિક તાપમાનનો SI એકમ છે, જે બોલ્ટ્ઝમાન અચળાંક k ની સ્થિર સંખ્યાત્મક કિંમત $1.380649 \times 10^{-23}$ J/K સેટ કરીને વ્યાખ્યાયિત કરવામાં આવે છે.
    \item \textbf{ચોકસાઇ}: ચોકસાઇ એ માપવામાં આવતી જથ્થાની સાચી અથવા માનક કિંમતથી માપેલી કિંમતની નજીકતાની ડિગ્રી છે.
\end{itemize}
\end{solutionbox}

\begin{mnemonicbox}
\mnemonic{MKA - Meter measures Kilometers Accurately}
\end{mnemonicbox}

\questionmarks{1(બ)}{4}{વર્નિયર કેલિપર્સની રચના સ્વચ્છ આકૃતિ દોરી સમજાવો.}

\begin{solutionbox}
\textbf{આકૃતિ:}
\begin{center}
\begin{tikzpicture}
    % Main Scale
    \draw[thick] (0,0) rectangle (8,1);
    \foreach \x in {0,1,...,8} \draw (\x,1) -- (\x,0.7) node[above=3mm] {\x};
    \foreach \x in {0.5,1.5,...,7.5} \draw (\x,1) -- (\x,0.8);
    \node at (4,1.4) {Main Scale (મુખ્ય સ્કેલ) cm};
    
    % Vernier Scale
    \draw[thick, fill=gray!10] (1.5,-0.5) rectangle (4.5,0);
    \foreach \x in {1.5,1.8,...,4.5} \draw (\x,0) -- (\x,-0.2);
    \node at (3,-0.8) {Vernier Scale (વર્નિયર સ્કેલ)};
    
    % Jaws
    \draw[thick] (0,0) -- (0,-2) -- (0.5,-2) -- (0.5,-1) -- (1.5,-1) -- (1.5,0);
    \node at (0.25,-2.3) {Fixed Jaw (સ્થિર જડબું)};
    
    \draw[thick] (1.5,-0.5) -- (1.5,-2) -- (2,-2) -- (2,-1) -- (4.5,-1);
    \node at (1.75,-2.3) {Movable Jaw (હલનચલન જડબું)};
    
    % Object
    \draw[fill=blue!20] (0.5,-1.5) circle (0.5);
    \node at (0.5,-1.5) {Object (વસ્તુ)};
\end{tikzpicture}
\captionof{figure}{વર્નિયર કેલિપર્સની રચના}
\end{center}

વર્નિયર કેલિપર્સમાં શામેલ છે:
\begin{itemize}
    \item \textbf{મુખ્ય સ્કેલ}: માનક એકમોમાં ચિહ્નિત કરેલ સ્થિર સ્કેલ (mm અથવા ઇંચ)
    \item \textbf{વર્નિયર સ્કેલ}: મુખ્ય સ્કેલ પર સરકી શકે તેવો હલનચલન સ્કેલ
    \item \textbf{સ્થિર જડબું}: મુખ્ય સ્કેલ સાથે જોડાયેલ
    \item \textbf{હલનચલન જડબું}: વર્નિયર સ્કેલ સાથે જોડાયેલ
    \item \textbf{ઊંડાઈ પ્રોબ}: ખાડાની ઊંડાઈ માપવા માટે
    \item \textbf{બાહ્ય જડબાં}: બાહ્ય પરિમાણો માપવા માટે
    \item \textbf{આંતરિક જડબાં}: આંતરિક પરિમાણો માપવા માટે
\end{itemize}
\end{solutionbox}

\begin{mnemonicbox}
\mnemonic{FMMVJ - Fixed Main scale Makes Vernier Jaw move}
\end{mnemonicbox}

\questionmarks{1(ક)(1)}{4}{ભૌતિક રાશિ એટલે શું છે? દિશાની દૃષ્ટિએ તેના પ્રકારો સમજાવો.}

\begin{solutionbox}
ભૌતિક રાશિ એ ભૌતિક સિસ્ટમની એક માપી શકાય તેવી સંપત્તિ છે જેને માપન દ્વારા માત્રાત્મક કરી શકાય છે.

\textbf{દિશાના આધારે ભૌતિક રાશિઓના પ્રકારો:}

\begin{center}
\captionof{table}{અદિશ વિરુદ્ધ સદિશ રાશિઓ}
\begin{tabulary}{\linewidth}{|L|L|}
\hline
\textbf{અદિશ રાશિઓ} & \textbf{સદિશ રાશિઓ} \\ \hline
માત્ર પરિમાણ ધરાવે છે & પરિમાણ અને દિશા બંને ધરાવે છે \\ \hline
ઉદાહરણો: દળ, સમય, તાપમાન, ઊર્જા & ઉદાહરણો: વિસ્થાપન, વેગ, બળ, પ્રવેગ \\ \hline
સરળ સંખ્યાઓ દ્વારા રજૂ થાય છે & તીર અથવા નિર્દેશિત રેખા ખંડો દ્વારા રજૂ થાય છે \\ \hline
સરવાળો સરળ અંકગણિતને અનુસરે છે & સરવાળો સદિશ બીજગણિતને અનુસરે છે (સમાંતર ચતુષ્કોણનો નિયમ) \\ \hline
કોઈ દિશાત્મક ગુણધર્મો નથી & દિશા અને પરિમાણ દ્વારા સંપૂર્ણપણે નિર્દિષ્ટ છે \\ \hline
\end{tabulary}
\end{center}
\end{solutionbox}

\begin{mnemonicbox}
\mnemonic{SMAVD - Scalars have Magnitude Alone, Vectors have Direction}
\end{mnemonicbox}

\questionmarks{1(ક)(2)}{3}{એક માઇક્રોમીટરની પેચ 0.5 mm છે. જો તેના વતુળાકાર ભાગ પર 100 વિભાગ છે, તો તેની લઘુતમ માપવત્તા શોધો.}

\begin{solutionbox}
\textbf{ગણતરી:}
$$
\text{લઘુતમ માપવત્તા (L.C.)} = \frac{\text{પેચ}}{\text{વતુળાકાર સ્કેલ પરના વિભાગોની સંખ્યા}}
$$
$$
\text{L.C.} = \frac{0.5 \text{ mm}}{100} = 0.005 \text{ mm}
$$

તેથી, માઇક્રોમીટર સ્ક્રૂ ગેજની લઘુતમ માપવત્તા 0.005 mm છે.
\end{solutionbox}

\begin{mnemonicbox}
\mnemonic{PDL - Pitch Divided gives Least count}
\end{mnemonicbox}

\questionmarks{1(ક) OR}{7}{માઇક્રોમીટર સ્ક્રૂ ગેજની ત્રુટીઓ આકૃતિ દોરી સમજાવો.}

\begin{solutionbox}
\textbf{આકૃતિ:}
\begin{center}
\begin{tikzpicture}
    % Frame
    \draw[thick] (0,0) arc (180:360:2cm) -- (4,0) -- (4,2) -- (5,2) -- (5,-1) -- (4,-1);
    \draw[thick] (0,0) -- (0,1);
    
    % Anvil
    \draw[thick, fill=gray] (0,1) rectangle (0.5,1.5);
    \node[above] at (0.25,1.5) {Anvil (એનવિલ)};
    
    % Spindle
    \draw[thick, fill=gray] (0.5,1.1) rectangle (3.5,1.4);
    \node[above] at (2,1.4) {Spindle (સ્પિંડલ)};
    
    % Sleeve
    \draw[thick] (4,1) rectangle (6,1.5);
    \foreach \x in {4,4.2,...,6} \draw (\x,1.25) -- (\x,1.5);
    \draw (4,1.25) -- (6,1.25);
    \node[below] at (5,1) {Barrel/Sleeve (બેરલ/સ્લીવ)};
    
    % Thimble
    \draw[thick] (6,0.8) rectangle (8,1.7);
    \foreach \y in {0.8,1.0,...,1.7} \draw (6,\y) -- (6.2,\y);
    \node[right] at (8,1.25) {Thimble (થિમ્બલ)};
    
    % Ratchet
    \draw[thick] (8,1.1) rectangle (8.5,1.4);
    \node[right] at (8.5,1.25) {Ratchet (રેચેટ)};
\end{tikzpicture}
\captionof{figure}{માઇક્રોમીટર સ્ક્રૂ ગેજના ઘટકો}
\end{center}

માઇક્રોમીટર સ્ક્રૂ ગેજની સામાન્ય ત્રુટીઓ:
\begin{itemize}
    \item \textbf{શૂન્ય ત્રુટિ}: જ્યારે માપન ફલકો સંપર્કમાં હોય, ત્યારે થિમ્બલનો શૂન્ય ડેટમ લાઇન સાથે મેળ ખાતો નથી
    \begin{itemize}
        \item \textbf{ધન શૂન્ય ત્રુટિ}: જ્યારે થિમ્બલ પરનું શૂન્યનું ચિહ્ન ડેટમ લાઇનની નીચે હોય
        \item \textbf{ઋણ શૂન્ય ત્રુટિ}: જ્યારે થિમ્બલ પરનું શૂન્યનું ચિહ્ન ડેટમ લાઇનની ઉપર હોય
    \end{itemize}
    \item \textbf{બેકલેશ ત્રુટિ}: સ્ક્રૂ અને નટ વચ્ચેનો ખેલ, આગળ અને પાછળના હલનચલનમાં અલગ રીડિંગ્સ થાય છે
    \item \textbf{યંત્ર ત્રુટિ}: ઉત્પાદન ખામીઓ અથવા ઘસારાને કારણે
    \item \textbf{પેરેલેક્સ ત્રુટિ}: જ્યારે દૃષ્ટિની લાઇન સ્કેલ રીડિંગને લંબરૂપ ન હોય
\end{itemize}

\textbf{સુધારા સૂત્ર:} સાચું રીડિંગ = અવલોકિત રીડિંગ - શૂન્ય ત્રુટિ
\end{solutionbox}

\begin{mnemonicbox}
\mnemonic{ZBIP - Zero, Backlash, Instrument and Parallax errors make measurements trip}
\end{mnemonicbox}

\questionmarks{2(અ)}{3}{કુલંબનો વ્યસ્ત વર્ગનો નિયમ સમજાવો.}

\begin{solutionbox}
કુલંબનો વ્યસ્ત વર્ગનો નિયમ કહે છે કે બે બિંદુ ચાર્જ વચ્ચેનું ઇલેક્ટ્રોસ્ટેટિક બળ:
\begin{itemize}
    \item ચાર્જના પરિમાણના ગુણનફળના સીધા પ્રમાણમાં
    \item તેમની વચ્ચેના અંતરના વર્ગના વ્યસ્ત પ્રમાણમાં
    \item બે ચાર્જને જોડતી રેખા પર કાર્ય કરે છે
\end{itemize}

\textbf{ગણિતીય અભિવ્યક્તિ:}
$$
F = k\frac{q_1q_2}{r^2}
$$

જ્યાં:
\begin{itemize}
    \item $F$ = ચાર્જ વચ્ચેનું ઇલેક્ટ્રોસ્ટેટિક બળ
    \item $k$ = કુલંબનો અચળાંક ($9 \times 10^9 \text{ N}\cdot\text{m}^2/\text{C}^2$)
    \item $q_1, q_2$ = બે ચાર્જના પરિમાણ
    \item $r$ = ચાર્જ વચ્ચેનું અંતર
\end{itemize}
\end{solutionbox}

\begin{mnemonicbox}
\mnemonic{PDSA - Product of charges Directly, Square of distance inversely, Along the line}
\end{mnemonicbox}

\questionmarks{2(બ)}{4}{વિદ્યુત સ્થિતિમાનનો તફાવત સમજાવો.}

\begin{solutionbox}
વિદ્યુત સ્થિતિમાનનો તફાવત (વોલ્ટેજ) એ વિદ્યુત ક્ષેત્રમાં બે બિંદુઓની વચ્ચે ધન ટેસ્ટ ચાર્જને ખસેડવામાં એકમ ચાર્જ દીઠ થતું કાર્ય છે.

\textbf{ગણિતીય અભિવ્યક્તિ:}
$$
V = \frac{W}{q}
$$

જ્યાં:
\begin{itemize}
    \item $V$ = સ્થિતિમાનનો તફાવત (વોલ્ટ)
    \item $W$ = કરવામાં આવેલું કાર્ય (જૂલ)
    \item $q$ = ચાર્જ (કૂલંબ)
\end{itemize}

\textbf{મુખ્ય લક્ષણો:}
\begin{itemize}
    \item વોલ્ટમાં માપવામાં આવે છે (V)
    \item અદિશ રાશિ (માત્ર પરિમાણ ધરાવે છે)
    \item પથ-સ્વતંત્ર (માત્ર પ્રારંભિક અને અંતિમ સ્થિતિ પર આધારિત)
    \item એકમ ચાર્જ દીઠ ઊર્જાનું પ્રતિનિધિત્વ કરે છે
\end{itemize}
\end{solutionbox}

\begin{mnemonicbox}
\mnemonic{WPCS - Work Per Charge is what potential difference Says}
\end{mnemonicbox}

\questionmarks{2(ક)}{7}{કેપેસીટરનું શ્રેણીમાં તથા સમાંતર જોડાણમાટે સમતુલ્ય કેપેસિટન્સ વર્ણવો.}

\begin{solutionbox}
\textbf{શ્રેણી જોડાણ:}

\begin{center}
\begin{tikzpicture}
    \draw (0,0) to[C, l=$C_1$] (2,0) to[C, l=$C_2$] (4,0) to[C, l=$C_3$] (6,0);
    \draw (0,0) to[short, -o] (-0.5,0);
    \draw (6,0) to[short, -o] (6.5,0);
\end{tikzpicture}
\captionof{figure}{શ્રેણીમાં કેપેસિટર્સ}
\end{center}

\begin{itemize}
    \item જ્યારે કેપેસિટરો એકબીજાના છેડાથી જોડાયેલા હોય
    \item દરેક કેપેસિટર પર સમાન ચાર્જ: $Q = Q_1 = Q_2 = Q_3$
    \item કુલ પોટેન્શિયલ તફાવત: $V = V_1 + V_2 + V_3$
    \item સમતુલ્ય કેપેસિટન્સ સૂત્ર: $\frac{1}{C_{eq}} = \frac{1}{C_1} + \frac{1}{C_2} + \frac{1}{C_3} + ...$
    \item સમતુલ્ય કેપેસિટન્સ સૌથી નાના વ્યક્તિગત કેપેસિટન્સ કરતાં ઓછી હોય છે
\end{itemize}

\textbf{સમાંતર જોડાણ:}

\begin{center}
\begin{tikzpicture}
    \draw (0,0) -- (1,0) -- (1,1.5) to[C, l=$C_1$] (4,1.5) -- (4,0) -- (5,0);
    \draw (1,0) to[C, l=$C_2$] (4,0);
    \draw (1,0) -- (1,-1.5) to[C, l=$C_3$] (4,-1.5) -- (4,0);
    \draw (0,0) to[short, -o] (-0.5,0);
    \draw (5,0) to[short, -o] (5.5,0);
\end{tikzpicture}
\captionof{figure}{સમાંતરમાં કેપેસિટર્સ}
\end{center}

\begin{itemize}
    \item જ્યારે કેપેસિટરો એક જ બે બિંદુઓ વચ્ચે જોડાયેલા હોય
    \item દરેક કેપેસિટર પર સમાન પોટેન્શિયલ તફાવત: $V = V_1 = V_2 = V_3$
    \item કુલ ચાર્જ: $Q = Q_1 + Q_2 + Q_3$
    \item સમતુલ્ય કેપેસિટન્સ સૂત્ર: $C_{eq} = C_1 + C_2 + C_3 + ...$
    \item સમતુલ્ય કેપેસિટન્સ સૌથી મોટા વ્યક્તિગત કેપેસિટન્સ કરતાં વધુ હોય છે
\end{itemize}

\textbf{તુલનાત્મક કોષ્ટક:}

\begin{center}
\captionof{table}{શ્રેણી વિરુદ્ધ સમાંતર કેપેસિટર્સ}
\begin{tabulary}{\linewidth}{|L|L|L|}
\hline
\textbf{પરિમાણ} & \textbf{શ્રેણી} & \textbf{સમાંતર} \\ \hline
ચાર્જ & બધા કેપેસિટર પર સમાન & કેપેસિટન્સ અનુસાર વિતરિત \\ \hline
વોલ્ટેજ & કેપેસિટરો વચ્ચે વિભાજિત & બધા કેપેસિટર પર સમાન \\ \hline
સમતુલ્ય કેપેસિટન્સ & $1/C_{eq} = 1/C_1 + 1/C_2 + ...$ & $C_{eq} = C_1 + C_2 + ...$ \\ \hline
પરિણામી કેપેસિટન્સ & કોઈપણ વ્યક્તિગત C કરતાં નાની & કોઈપણ વ્યક્તિગત C કરતાં મોટી \\ \hline
\end{tabulary}
\end{center}
\end{solutionbox}

\begin{mnemonicbox}
\mnemonic{RAPS - Reciprocals Add in Parallel Sum}
\end{mnemonicbox}

\questionmarks{2(અ) OR}{3}{વિદ્યુતક્ષેત્ર રેખાઓની લાક્ષણિકતાઓ લખો.}

\begin{solutionbox}
\textbf{વિદ્યુત ક્ષેત્ર રેખાઓની લાક્ષણિકતાઓ:}
\begin{itemize}
    \item \textbf{દિશા}: હંમેશા ધન ચાર્જથી ઋણ ચાર્જ તરફ બતાવે છે
    \item \textbf{પ્રકૃતિ}: ધન ચાર્જથી શરૂ થાય છે અને ઋણ ચાર્જ પર પૂરી થાય છે
    \item \textbf{સાતત્ય}: ક્યારેય એકબીજાને છેદતી નથી
    \item \textbf{ઘનતા}: નજીકની રેખાઓ વધુ મજબૂત વિદ્યુત ક્ષેત્ર સૂચવે છે
    \item \textbf{લંબતા}: હંમેશા સમસ્થિતિમાન સપાટીઓને લંબ હોય છે
    \item \textbf{આકાર}: સમાન ક્ષેત્રો માટે સીધી રેખાઓ, અસમાન ક્ષેત્રો માટે વક્ર
    \item \textbf{ખુલ્લા/બંધ}: હંમેશા ખુલ્લા વક્રો, ચુંબકીય ક્ષેત્ર રેખાઓથી વિપરીત
\end{itemize}
\end{solutionbox}

\begin{mnemonicbox}
\mnemonic{DNCPS - Direction, Never cross, Closeness shows strength, Perpendicular, Straight/curved}
\end{mnemonicbox}

\questionmarks{2(બ) OR}{4}{વિદ્યુત ફ્લક્સ વિશે નોંધ લખો.}

\begin{solutionbox}
વિદ્યુત ફ્લક્સ એ આપેલા ક્ષેત્રફળમાંથી પસાર થતા વિદ્યુત ક્ષેત્રનું માપ છે.

\textbf{ગણિતીય અભિવ્યક્તિ:}
$$
\Phi_E = E \cdot A \cdot \cos\theta
$$

જ્યાં:
\begin{itemize}
    \item $\Phi_E$ = વિદ્યુત ફ્લક્સ (N·m$^2$/C અથવા V·m)
    \item $E$ = વિદ્યુત ક્ષેત્ર તીવ્રતા (N/C અથવા V/m)
    \item $A$ = સપાટીનું ક્ષેત્રફળ (m$^2$)
    \item $\theta$ = વિદ્યુત ક્ષેત્ર અને સપાટીના લંબ વચ્ચેનો ખૂણો
\end{itemize}

\textbf{મુખ્ય લક્ષણો:}
\begin{itemize}
    \item સદિશ રાશિ
    \item SI એકમ ન્યૂટન-મીટર-વર્ગ પ્રતિ કૂલંબ (N·m$^2$/C) અથવા વોલ્ટ-મીટર (V·m)
    \item સપાટીમાંથી પસાર થતી ક્ષેત્ર રેખાઓની સંખ્યાનું પ્રતિનિધિત્વ કરે છે
    \item ક્ષેત્ર સપાટીને લંબ હોય ત્યારે મહત્તમ ($\theta = 0^\circ$)
    \item ક્ષેત્ર સપાટીને સમાંતર હોય ત્યારે શૂન્ય ($\theta = 90^\circ$)
\end{itemize}
\end{solutionbox}

\begin{mnemonicbox}
\mnemonic{FACT - Flux = Area x Cos-theta x Field strength}
\end{mnemonicbox}

\questionmarks{2(ક) OR}{7}{કેપેસિટર અને કેપેસિટન્સ પર નોંધ લખો.}

\begin{solutionbox}
\textbf{કેપેસિટર:}
કેપેસિટર એ એક વિદ્યુત ઘટક છે જે વિદ્યુત ચાર્જ અને વિદ્યુત ક્ષેત્રમાં ઊર્જા સંગ્રહિત કરવા માટે રચાયેલ છે.

\textbf{મૂળભૂત રચના:}
\begin{center}
\begin{tikzpicture}
    % Plates
    \draw[thick] (0,2) -- (4,2);
    \draw[thick] (0,0) -- (4,0);
    
    % Dielectric
    \draw[fill=blue!10] (0.2,0.2) rectangle (3.8,1.8);
    \node at (2,1) {Dielectric (ડાયઇલેક્ટ્રિક)};
    
    % Terminals
    \draw (2,2) -- (2,2.5);
    \draw (2,0) -- (2,-0.5);
    
    \node at (2,2.7) {Plate 1};
    \node at (2,-0.7) {Plate 2};
\end{tikzpicture}
\captionof{figure}{સમાંતર પ્લેટ કેપેસિટર}
\end{center}

\textbf{કેપેસિટન્સ:}
આપેલા પોટેન્શિયલ તફાવત પર વિદ્યુત ચાર્જ સંગ્રહિત કરવાની કેપેસિટરની ક્ષમતા.

\textbf{ગણિતીય અભિવ્યક્તિ:}
$$
C = \frac{Q}{V}
$$

જ્યાં:
\begin{itemize}
    \item $C$ = કેપેસિટન્સ (ફેરાડ)
    \item $Q$ = વિદ્યુત ચાર્જ (કૂલંબ)
    \item $V$ = પોટેન્શિયલ તફાવત (વોલ્ટ)
\end{itemize}

\textbf{સમાંતર પ્લેટ કેપેસિટર માટે:}
$$
C = \frac{\epsilon_0 \epsilon_r A}{d}
$$

જ્યાં:
\begin{itemize}
    \item $\epsilon_0$ = મુક્ત અવકાશની પરમિટિવિટી ($8.85 \times 10^{-12} \text{ F/m}$)
    \item $\epsilon_r$ = ડાયઇલેક્ટ્રિકની સાપેક્ષ પરમિટિવિટી
    \item $A$ = પ્લેટ્સ વચ્ચેના ઓવરલેપનું ક્ષેત્રફળ
    \item $d$ = પ્લેટ્સ વચ્ચેનું અંતર
\end{itemize}

\textbf{કેપેસિટન્સને અસર કરતા પરિબળો:}
\begin{itemize}
    \item પ્લેટ ક્ષેત્રફળ સાથે વધે છે
    \item પ્લેટ અલગતા સાથે ઘટે છે
    \item ડાયઇલેક્ટ્રિક અચળાંક સાથે વધે છે
\end{itemize}

\textbf{કેપેસિટરના ઉપયોગો:}
\begin{itemize}
    \item ઊર્જા સંગ્રહ
    \item પાવર સપ્લાયમાં ફિલ્ટરિંગ
    \item સમય ગણતરી સર્કિટ્સ
    \item કપલિંગ અને ડિકપલિંગ
    \item પાવર ફેક્ટર સુધારણા
\end{itemize}
\end{solutionbox}

\begin{mnemonicbox}
\mnemonic{QVAD - Quotient of charge and Voltage, affected by Area and Distance}
\end{mnemonicbox}

\questionmarks{3(અ)}{3}{વ્યાખ્યા આપો: (અ) ઉષ્માગમન (બ) કિલોકેલરી (ક) થર્મોમીટર.}

\begin{solutionbox}
\begin{itemize}
    \item \textbf{ઉષ્માગમન}: માધ્યમની જરૂર વિના વિદ્યુતચુંબકીય તરંગોના રૂપમાં થર્મલ ઊર્જાનું સ્થાનાંતરણ, જે નિર્વાત અથવા પારદર્શક માધ્યમોમાં થાય છે.
    \item \textbf{કિલોકેલરી}: 1000 કૅલરીના બરાબર ગરમીની ઊર્જાનો એકમ, જ્યાં એક કૅલરી એ પ્રમાણભૂત પરિસ્થિતિઓમાં 1 ગ્રામ પાણીનું તાપમાન 1°C વધારવા માટે જરૂરી ગરમીની માત્રા છે.
    \item \textbf{થર્મોમીટર}: તાપમાન માપવા માટે વપરાતું સાધન જે ભૌતિક ગુણધર્મ (જેમ કે પારાનો વિસ્તાર) જે તાપમાન સાથે બદલાય છે તેના આધારે કાર્ય કરે છે.
\end{itemize}
\end{solutionbox}

\begin{mnemonicbox}
\mnemonic{RKT - Radiation needs no medium, Kilocalorie measures energy, Thermometer shows temperature}
\end{mnemonicbox}

\questionmarks{3(બ)}{4}{ઉષ્માવહનાંકનો નિયમ સમજાવો.}

\begin{solutionbox}
ઉષ્માવહનાંકનો નિયમ (ફોરિયરનો નિયમ) કહે છે કે પદાર્થ દ્વારા ઉષ્મા પ્રવાહનો દર:
\begin{itemize}
    \item વિભાગના ક્ષેત્રફળના સીધા પ્રમાણમાં
    \item તાપમાન ઢાળના સીધા પ્રમાણમાં
    \item પદાર્થના થર્મલ વાહકતા પર આધારિત
\end{itemize}

\textbf{ગણિતીય અભિવ્યક્તિ:}
$$
\frac{Q}{t} = -kA\frac{dT}{dx}
$$

જ્યાં:
\begin{itemize}
    \item $Q/t$ = ઉષ્મા પ્રવાહનો દર (J/s અથવા W)
    \item $k$ = પદાર્થની થર્મલ વાહકતા (W/m·K)
    \item $A$ = આડછેદનું ક્ષેત્રફળ (m$^2$)
    \item $dT/dx$ = તાપમાન ઢાળ (K/m)
    \item નકારાત્મક ચિહ્ન સૂચવે છે કે ઉષ્મા ઉચ્ચ તાપમાનથી નીચા તાપમાન તરફ વહે છે
\end{itemize}
\end{solutionbox}

\begin{mnemonicbox}
\mnemonic{GAKT - Gradient And area with K gives heat Transfer}
\end{mnemonicbox}

\questionmarks{3(ક)(1)}{3}{1 વ્યક્તિને 102°F તાવ છે. તો તે સેલ્સિયસ અને કેલ્વિનમાં કેટલો હશે?}

\begin{solutionbox}
\textbf{ફેરનહીટથી સેલ્સિયસમાં રૂપાંતર:}
$$
C = (F - 32) \times \frac{5}{9}
$$
$$
C = (102 - 32) \times \frac{5}{9}
$$
$$
C = 70 \times 0.555
$$
$$
C = 38.89^\circ C
$$

\textbf{સેલ્સિયસથી કેલ્વિનમાં રૂપાંતર:}
$$
K = C + 273.15
$$
$$
K = 38.89 + 273.15
$$
$$
K = 312.04 \text{ K}
$$

તેથી, $102^\circ F = 38.89^\circ C = 312.04 \text{ K}$
\end{solutionbox}

\begin{mnemonicbox}
\mnemonic{FSK - From Fahrenheit Subtract 32, multiply by 5/9, then add 273.15 for Kelvin}
\end{mnemonicbox}

\questionmarks{3(ક)(2)}{4}{સેલ્સિયસ અને ફેરનહીટ માપક્રમ સમજાવો.}

\begin{solutionbox}
\textbf{સેલ્સિયસ અને ફેરનહીટ તાપમાન માપક્રમોની તુલના:}

\begin{center}
\captionof{table}{સેલ્સિયસ વિરુદ્ધ ફેરનહીટ}
\begin{tabulary}{\linewidth}{|L|L|L|}
\hline
\textbf{પરિમાણ} & \textbf{સેલ્સિયસ માપક્રમ} & \textbf{ફેરનહીટ માપક્રમ} \\ \hline
પાણીનું હિમબિંદુ & 0°C & 32°F \\ \hline
પાણીનું ઉત્કલનબિંદુ & 100°C & 212°F \\ \hline
વિભાગોની સંખ્યા & 100 વિભાગો & 180 વિભાગો \\ \hline
વિકસાવનાર & એન્ડર્સ સેલ્સિયસ (1742) & ગેબ્રિયલ ફેરનહીટ (1724) \\ \hline
ઉપયોગ & વિશ્વભરના મોટાભાગના દેશોમાં & મુખ્યત્વે USA અને તેના પ્રદેશોમાં \\ \hline
સંબંધ & $C = (F - 32) \times 5/9$ & $F = (C \times 9/5) + 32$ \\ \hline
\end{tabulary}
\end{center}

\textbf{આકૃતિ:}
\begin{center}
\begin{tikzpicture}
    % Celsius
    \draw[thick] (0,0) -- (0,5);
    \foreach \y/\t in {0/-17.8, 1.5/0, 5/100} \draw (-0.2,\y) -- (0,\y) node[left] {\t$^\circ$C};
    \node at (0, 5.5) {Celsius};
    
    % Fahrenheit
    \draw[thick] (4,0) -- (4,5);
    \foreach \y/\t in {0/0, 1.5/32, 5/212} \draw (4,\y) -- (4.2,\y) node[right] {\t$^\circ$F};
    \node at (4, 5.5) {Fahrenheit};
    
    % Connection lines
    \draw[dashed] (0,5) -- (4,5) node[midway, above] {Water Boils};
    \draw[dashed] (0,1.5) -- (4,1.5) node[midway, above] {Water Freezes};
    \draw[dashed] (0,0) -- (4,0);
\end{tikzpicture}
\captionof{figure}{તાપમાન માપક્રમની તુલના}
\end{center}
\end{solutionbox}

\begin{mnemonicbox}
\mnemonic{FBIC - Fahrenheit has Bigger numbers, Interval of 180, Conversion needs 5/9 or 9/5}
\end{mnemonicbox}

\questionmarks{3(અ) OR}{3}{ઉષ્માધારીતા ની વ્યાખ્યા, એકમ અને સૂત્ર લખો.}

\begin{solutionbox}
\textbf{વ્યાખ્યા:} ઉષ્માધારીતા એ કોઈ પદાર્થના તાપમાનમાં એક ડિગ્રી (સેલ્સિયસ અથવા કેલ્વિન) વધારવા માટે જરૂરી ઉષ્મા ઊર્જાની માત્રા છે.

\textbf{સૂત્ર:}
$$
C = \frac{Q}{\Delta T}
$$

જ્યાં:
\begin{itemize}
    \item $C$ = ઉષ્માધારીતા (J/K અથવા J/$^\circ$C)
    \item $Q$ = આપવામાં આવેલી ઉષ્મા ઊર્જા (જૂલ)
    \item $\Delta T$ = તાપમાનમાં ફેરફાર (K અથવા $^\circ$C)
\end{itemize}

\textbf{એકમ:} જૂલ પ્રતિ કેલ્વિન (J/K) અથવા જૂલ પ્રતિ ડિગ્રી સેલ્સિયસ (J/$^\circ$C)
\end{solutionbox}

\begin{mnemonicbox}
\mnemonic{QTC - Quotient of heat and Temperature Change gives heat capacity}
\end{mnemonicbox}

\questionmarks{3(બ) OR}{4}{ઉષ્મા પ્રવાહની પદ્ધતિઓ સમજાવો}

\begin{solutionbox}
\textbf{ઉષ્મા પ્રવાહની ત્રણ પદ્ધતિઓ:}

\begin{center}
\captionof{table}{ઉષ્મા પ્રવાહની પદ્ધતિઓ}
\begin{tabulary}{\linewidth}{|L|L|L|L|}
\hline
\textbf{પદ્ધતિ} & \textbf{વ્યાખ્યા} & \textbf{ઉદાહરણો} & \textbf{માધ્યમની જરૂરિયાત} \\ \hline
\textbf{વહન} & પદાર્થના મોટા ભાગના હલનચલન વિના સીધા અણુઓના અથડામણ દ્વારા ઉષ્માનું સ્થાનાંતરણ & ધાતુના સળિયા દ્વારા ઉષ્મા, રસોઈના વાસણ & હા (ઘન પદાર્થ પસંદગીયુક્ત) \\ \hline
\textbf{સંવહન} & ગરમ થયેલા કણોના એક વિસ્તારથી બીજા વિસ્તારમાં હલનચલન દ્વારા ઉષ્માનું સ્થાનાંતરણ & ઉકળતું પાણી, રૂમ હીટર, સમુદ્રી પવન & હા (પ્રવાહી - તરલ અથવા વાયુ) \\ \hline
\textbf{વિકિરણ} & માધ્યમની જરૂરિયાત વિના વિદ્યુતચુંબકીય તરંગો દ્વારા ઉષ્માનું સ્થાનાંતરણ & સૌર વિકિરણ, માઇક્રોવેવ હીટિંગ, ઇન્ફ્રારેડ હીટર & ના (નિર્વાતમાં કાર્ય કરે છે) \\ \hline
\end{tabulary}
\end{center}
\end{solutionbox}

\begin{mnemonicbox}
\mnemonic{CoCRa - Conduction needs Contact, Convection needs Currents, Radiation needs no medium}
\end{mnemonicbox}

\questionmarks{3(ક) OR}{7}{બાયમેટાલિક થર્મોમીટર સમજાવો.}

\begin{solutionbox}
\textbf{આકૃતિ:}
\begin{center}
\begin{tikzpicture}
    % Strip
    \draw[thick, fill=gray!30] (0,0) arc (160:20:2) -- ++(0,-0.4) arc (20:160:2.4) -- cycle;
    \draw[thick] (0.2,-0.2) arc (160:20:2.1); % Dividing line
    \node at (2.5,1.5) {Metal 1 (ધાતુ 1)};
    \node at (2.5,1.0) {Metal 2 (ધાતુ 2)};
    
    % Pointer
    \draw[thick, ->] (3.8,0.2) -- (5,1.5);
    \node[above] at (5,1.5) {Pointer (પોઈન્ટર)};
    
    % Scale
    \draw (5.5,0) arc (-30:30:2);
    \foreach \a in {-30,-20,...,30} \draw (5.5,0) ++(\a:2) -- ++(\a:0.2);
    \node at (6.5,-1) {Scale (સ્કેલ)};
    
    % Fixed end
    \fill[pattern=north east lines] (-0.5,-1) rectangle (0.5,1);
    \draw (0,-1) -- (0,1);
    \node[below] at (0,-1) {Fixed end (સ્થિર છેડો)};
\end{tikzpicture}
\captionof{figure}{બાયમેટાલિક પટ્ટી થર્મોમીટર}
\end{center}

\textbf{કાર્ય સિદ્ધાંત:}
\begin{itemize}
    \item બે અલગ-અલગ ધાતુઓના અસમાન થર્મલ વિસ્તરણ પર આધારિત
    \item બે ધાતુની પટ્ટીઓ, જેમાં થર્મલ વિસ્તરણના અલગ-અલગ ગુણાંકો હોય છે, તેને એકસાથે જોડવામાં આવે છે
    \item ગરમ થતાં, એક ધાતુ બીજી કરતાં વધુ ફેલાય છે
    \item આ અસમાન વિસ્તરણને કારણે પટ્ટી ઓછા વિસ્તરણવાળી ધાતુ તરફ વળે છે
    \item વળવાની માત્રા તાપમાન ફેરફારના પ્રમાણમાં હોય છે
    \item પટ્ટી સાથે જોડાયેલ એક પોઇન્ટર અંશાંકિત સ્કેલ પર તાપમાન દર્શાવે છે
\end{itemize}

\textbf{ફાયદા:}
\begin{itemize}
    \item સરળ, મજબૂત બાંધકામ
    \item કોઈ પ્રવાહી કે વાયુની જરૂર નથી
    \item વિશાળ તાપમાન શ્રેણી
    \item યાંત્રિક આઘાતોનો પ્રતિકાર કરે છે
    \item થર્મોસ્ટેટ બનાવવા માટે વાપરી શકાય છે
\end{itemize}

\textbf{ઉપયોગો:} ઘરના હીટિંગ/કૂલિંગ સિસ્ટમમાં થર્મોસ્ટેટ, ઓટોમોબાઇલ કૂલિંગ સિસ્ટમ, ઓવન તાપમાન નિયંત્રણો, સર્કિટ બ્રેકર.
\end{solutionbox}

\begin{mnemonicbox}
\mnemonic{BENDS - Bimetallic strips Expand, Not equally, Different metals, Show temperature}
\end{mnemonicbox}

\questionmarks{4(અ)}{3}{વ્યાખ્યા આપો: (અ) આવૃત્તિ (બ) ઇન્ફ્રાસોનિક તરંગો (ક) પડઘો.}

\begin{solutionbox}
\begin{itemize}
    \item \textbf{આવૃત્તિ}: એકમ સમયમાં પૂર્ણ થતા આંદોલનો અથવા ચક્રોની સંખ્યા, હર્ટ્ઝ (Hz)માં માપવામાં આવે છે.
    \item \textbf{ઇન્ફ્રાસોનિક તરંગો}: માનવ સાંભળવાની નીચલી મર્યાદા (20 Hz નીચે)ની આવૃત્તિઓવાળા ધ્વનિ તરંગો જે માણસો દ્વારા સાંભળી શકાતા નથી પરંતુ અન્ય પ્રાણીઓ દ્વારા શોધી શકાય છે.
    \item \textbf{પડઘો}: એક અવાજ જે શ્રોતા તરફ પાછો પરાવર્તિત થાય છે અને મૂળ ધ્વનિના અલગ પુનરાવર્તન તરીકે સાંભળવા માટે પૂરતા સમયના વિલંબ સાથે આવે છે.
\end{itemize}
\end{solutionbox}

\begin{mnemonicbox}
\mnemonic{FIE - Frequency counts cycles, Infrasonic is below hearing, Echo comes back after reflection}
\end{mnemonicbox}

\questionmarks{4(બ)}{4}{લંબગત તરંગ અને સંગત તરંગ વચ્ચેનો તફાવત આપો.}

\begin{solutionbox}
\textbf{લંબગત અને સંગત તરંગો વચ્ચે તુલના:}

\begin{center}
\captionof{table}{લંબગત વિરુદ્ધ સંગત તરંગો}
\begin{tabulary}{\linewidth}{|L|L|L|}
\hline
\textbf{પરિમાણ} & \textbf{લંબગત તરંગો} & \textbf{સંગત તરંગો} \\ \hline
કણના હલનચલનની દિશા & તરંગ પ્રસરણને સમાંતર & તરંગ પ્રસરણને લંબરૂપ \\ \hline
ઉદાહરણ & ધ્વનિ તરંગો, P-તરંગો & પ્રકાશ તરંગો, પાણી પરના તરંગો \\ \hline
માધ્યમની જરૂરિયાત & ઘન, પ્રવાહી અને વાયુઓ & ઘન અને પ્રવાહીની સપાટી (વાયુ નહીં) \\ \hline
ઘટકો & સંકોચન અને વિરલીકરણ & શિખર અને ખીણ \\ \hline
ધ્રુવીકરણ & ધ્રુવીકૃત થઈ શકતા નથી & ધ્રુવીકૃત થઈ શકે છે \\ \hline
દૃશ્યમાનતા & સ્પ્રિંગ/સ્લિંકી & દોરડી ઉપર-નીચે \\ \hline
\end{tabulary}
\end{center}

\textbf{આકૃતિ:}
\begin{center}
\begin{tikzpicture}
    % Longitudinal
    \node[left] at (-1,1.5) {Sangat (સંગત)};
    \foreach \x in {0,0.2,...,5} {
        \draw[thick] (\x,1.2) -- (\x,1.8);
    }
    % Compressions
    \foreach \x in {1,1.1,1.2, 3,3.1,3.2} {
        \draw[thick] (\x,1.2) -- (\x,1.8);
    }
    \draw[->] (0,2) -- (5,2) node[midway, above] {Propagation (પ્રસરણ)};
    \draw[<->] (2.5,1) -- (3.5,1) node[midway, below] {Particles (કણો)};
    
    % Transverse
    \node[left] at (-1,-0.5) {Lambgat (લંબગત)};
    \draw[thick, domain=0:5, samples=100] plot (\x, {sin(\x r * 2.5) * 0.5 - 0.5});
    \draw[->] (0,0.5) -- (5,0.5) node[midway, above] {Propagation (પ્રસરણ)};
    \draw[<->] (2.5,-1.2) -- (2.5,-0.2) node[midway, left] {Particles (કણો)};
\end{tikzpicture}
\captionof{figure}{તરંગોના પ્રકાર}
\end{center}
\end{solutionbox}

\begin{mnemonicbox}
\mnemonic{PPCP - Particles move Parallel in Longitudinal, Perpendicular in Transverse, Compressions vs Crests, Polarization only in Transverse}
\end{mnemonicbox}

\questionmarks{4(ક)(1)}{4}{અલ્ટ્રાસોનિક તરંગોના ત્રણ ગુણધર્મો અને ઉપયોગો આપો.}

\begin{solutionbox}
\textbf{અલ્ટ્રાસોનિક તરંગોના ગુણધર્મો:}
\begin{itemize}
    \item 20,000 Hz ઉપરની આવૃત્તિ શ્રેણી (માનવ શ્રવણની બહાર)
    \item ટૂંકી તરંગલંબાઈઓ નાના પદાર્થોના શોધવા માટે મદદ કરે છે
    \item સાંભળી શકાય તેવા ધ્વનિની તુલનામાં ઉચ્ચ દિશાનિર્દેશતા
    \item ચોક્કસ માધ્યમોમાં ઉચ્ચ પ્રવેશ
    \item અવરોધોની આસપાસ ઓછું વિવર્તન
    \item પ્રવાહીઓમાં ગુહાકરણ થાય છે
\end{itemize}

\textbf{અલ્ટ્રાસોનિક તરંગોના ઉપયોગો:}
\begin{center}
\captionof{table}{અલ્ટ્રાસોનિક તરંગોના ઉપયોગો}
\begin{tabulary}{\linewidth}{|L|L|}
\hline
\textbf{ક્ષેત્ર} & \textbf{ઉપયોગો} \\ \hline
\textbf{તબીબી} & સોનોગ્રાફી, કિડની સ્ટોન વિનાશ, ફિઝિયોથેરાપી \\ \hline
\textbf{ઔદ્યોગિક} & બિન-વિનાશક પરીક્ષણ, સફાઈ, વેલ્ડિંગ, ડ્રિલિંગ \\ \hline
\textbf{નેવિગેશન} & SONAR, અંતર માપન, અવરોધ શોધ \\ \hline
\textbf{અન્ય} & કૂતરા સીટી, જીવજંતુ નિયંત્રણ, ધ્વનિ સ્થાનનિર્ધારણ \\ \hline
\end{tabulary}
\end{center}
\end{solutionbox}

\begin{mnemonicbox}
\mnemonic{FWD-MNO - Frequency high, Wavelength short, Direction focused; Medical imaging, NDT testing, Ocean mapping}
\end{mnemonicbox}

\questionmarks{4(ક)(2)}{3}{ધ્વનિ તરંગના વેગ, તરંગલંબાઈ અને આવૃત્તિ વચ્ચેનો સંબંધ તારવો.}

\begin{solutionbox}
\textbf{સિદ્ધાંત:}
એક તરંગને ધ્યાનમાં લો જેમાં:
\begin{itemize}
    \item તરંગલંબાઈ ($\lambda$): સમાન બિંદુઓ વચ્ચેનું અંતર
    \item આવૃત્તિ ($f$): એક સેકન્ડમાં કોઈ બિંદુમાંથી પસાર થતા તરંગોની સંખ્યા
    \item આવર્તકાળ ($T$): એક ચક્ર પૂર્ણ કરવા માટેનો સમય
\end{itemize}

એક આવર્તકાળ ($T$) દરમિયાન, તરંગ એક તરંગલંબાઈ ($\lambda$)ના અંતરને કાપે છે.
$$
\text{વેગ} = \text{અંતર}/\text{સમય} = \lambda/T
$$
આવૃત્તિ $f = 1/T$ હોવાથી, આપણે લખી શકીએ:
$$
v = \lambda \times f
$$
જ્યાં $v$ = તરંગનો વેગ (m/s), $\lambda$ = તરંગલંબાઈ (m), $f$ = આવૃત્તિ (Hz).

\textbf{આકૃતિ:}
\begin{center}
\begin{tikzpicture}
    \draw[thick, ->] (0,0) -- (6,0) node[right] {Distance (અંતર)};
    \draw[thick, ->] (0,-1) -- (0,1) node[above] {Amplitude (કંપવિસ્તાર)};
    \draw[thick, blue, domain=0:6, samples=100] plot (\x, {sin(\x r * 2)});
    
    \draw[<->] (1.57,1.2) -- (4.71,1.2) node[midway, above] {Wavelength $\lambda$ (તરંગલંબાઈ)};
    \draw[dashed] (1.57,1) -- (1.57,1.2);
    \draw[dashed] (4.71,1) -- (4.71,1.2);
\end{tikzpicture}
\captionof{figure}{તરંગલંબાઈ દ્રશ્યીકરણ}
\end{center}
\end{solutionbox}

\begin{mnemonicbox}
\mnemonic{VLF - Velocity equals Lambda times Frequency}
\end{mnemonicbox}

\questionmarks{4(અ) OR}{3}{પ્રતિઘોષ સમય માટેનું સેબાઇનનું સૂત્ર સમજાવો.}

\begin{solutionbox}
સેબાઇનનું સૂત્ર બંધ જગ્યામાં પ્રતિઘોષ સમયની ગણતરી કરે છે:

\textbf{સૂત્ર:}
$$
RT_{60} = \frac{0.161 \times V}{A}
$$

જ્યાં:
\begin{itemize}
    \item $RT_{60}$ = પ્રતિઘોષ સમય (સેકન્ડ) ધ્વનિને 60 dB ઘટાડવા માટે
    \item $V$ = રૂમનું કદ (m$^3$)
    \item $A$ = કુલ ધ્વનિ શોષણ (m$^2$ sabins)
\end{itemize}

\textbf{કુલ શોષણ ($A$)} ની ગણતરી આ રીતે થાય છે:
$$
A = \sum \alpha_i S_i = \alpha_1 S_1 + \alpha_2 S_2 + ...
$$
જ્યાં $\alpha_i$ = શોષણ ગુણાંક અને $S_i$ = સપાટી ક્ષેત્રફળ.
\end{solutionbox}

\begin{mnemonicbox}
\mnemonic{VAS - Volume And Surface absorption determine reverberation time}
\end{mnemonicbox}

\questionmarks{4(બ) OR}{4}{પ્રકાશનું વિવર્તન એટલે શું? તેના પ્રકાર આકૃતિ સાથે સમજાવો.}

\begin{solutionbox}
\textbf{વ્યાખ્યા:} વિવર્તન એ અવરોધોની આસપાસ અથવા ખુલ્લી જગ્યાઓમાંથી પ્રકાશ તરંગોનું વળવું છે, જે પ્રકાશના તરંગ સ્વભાવને દર્શાવે છે.

\textbf{વિવર્તનના પ્રકારો:}

1. \textbf{ફ્રેસનેલ વિવર્તન}: સ્ત્રોત અથવા સ્ક્રીન અવરોધથી મર્યાદિત અંતરે. ગોળાકાર તરંગાગ્રો. જટિલ પેટર્ન.

\begin{center}
\begin{tikzpicture}
    \node[circle, fill, inner sep=1.5pt] (S) at (0,0) {}; \node[left] at (S) {Source (સ્ત્રોત)};
    \draw[thick] (2,-1) -- (2,1); \draw[white, thick] (2,-0.2) -- (2,0.2); % Opening (છિદ્ર)
    \draw[thick] (4,-1.5) -- (4,1.5); \node[right] at (4,0) {Screen (પડદો)};
    
    \draw[->] (S) -- (2,0.2);
    \draw[->] (S) -- (2,-0.2);
\end{tikzpicture}
\captionof{figure}{ફ્રેસનેલ વિવર્તન}
\end{center}

2. \textbf{ફ્રૌનહોફર વિવર્તન}: સ્ત્રોત અને સ્ક્રીન અનંત અંતરે. સમતલ તરંગાગ્રો. સરળ પેટર્ન.

\begin{center}
\begin{tikzpicture}
    \foreach \y in {-0.4,-0.2,0,0.2,0.4} \draw[->] (0,\y) -- (2,\y);
    \node[left] at (0,0) {Plane Waves (સમતલ તરંગો)};
    
    \draw[thick] (2,-1) -- (2,1); \draw[white, thick] (2,-0.2) -- (2,0.2); % Opening (છિદ્ર)
    
    \foreach \y in {-0.4,-0.2,0,0.2,0.4} \draw[->] (2,\y*0.5) -- (4,\y*0.5); 
    
    \draw[thick] (4.5,-1.5) -- (4.5,1.5); \node[right] at (4.5,0) {Screen (પડદો)};
\end{tikzpicture}
\captionof{figure}{ફ્રૌનહોફર વિવર્તન}
\end{center}
\end{solutionbox}

\begin{mnemonicbox}
\mnemonic{FPSS - Fresnel has Finite distances, Spherical waves; Fraunhofer has Source at infinity, Straight (plane) waves}
\end{mnemonicbox}

\questionmarks{4(ક)(1) OR}{3}{એક રેડિયોતરંગની આવૃત્તિ 480 Hz અને ધ્વનિનો વેગ 330 m/s હોય તો તરંગલંબાઈ શોધો.}

\begin{solutionbox}
\textbf{આપેલ છે:}
\begin{itemize}
    \item આવૃત્તિ ($f$) = 480 Hz
    \item વેગ ($v$) = 330 m/s
\end{itemize}

\textbf{શોધવાનું છે:} તરંગલંબાઈ ($\lambda$)

\textbf{સૂત્ર:} $v = \lambda \times f \Rightarrow \lambda = v/f$

\textbf{ગણતરી:}
$$
\lambda = \frac{330}{480} = 0.6875 \text{ m}
$$

તેથી, રેડિયો તરંગની તરંગલંબાઈ 0.6875 m અથવા 68.75 cm છે.
\end{solutionbox}

\begin{mnemonicbox}
\mnemonic{WFV - Wavelength equals Velocity divided by Frequency}
\end{mnemonicbox}

\questionmarks{4(ક)(2) OR}{4}{ધ્વનિ તરંગોના ગુણધર્મો આપો}

\begin{solutionbox}
\textbf{ધ્વનિ તરંગોના ગુણધર્મો:}

\begin{center}
\captionof{table}{ધ્વનિ તરંગ ગુણધર્મો}
\begin{tabulary}{\linewidth}{|L|L|}
\hline
\textbf{ગુણધર્મ} & \textbf{વર્ણન} \\ \hline
તરંગ સ્વભાવ & યાંત્રિક, લંબગત તરંગ છે જેને માધ્યમની જરૂર પડે છે \\ \hline
આવૃત્તિ શ્રેણી & માનવો માટે: 20 Hz થી 20,000 Hz \\ \hline
વેગ & ~343 m/s હવામાં; ઘનમાં સૌથી ઝડપી \\ \hline
પરાવર્તન & સપાટીઓ પરથી પરાવર્તિત થાય છે (પડઘા) \\ \hline
વક્રીભવન & અલગ-અલગ માધ્યમો વચ્ચે દિશા બદલે છે \\ \hline
વિવર્તન & અવરોધોની આસપાસ વળે છે \\ \hline
વ્યતિકરણ & રચનાત્મક અથવા વિનાશક વ્યતિકરણ \\ \hline
અનુનાદ & કુદરતી આવૃત્તિઓએ વર્ધન \\ \hline
\end{tabulary}
\end{center}
\end{solutionbox}

\begin{mnemonicbox}
\mnemonic{WARDS-FIR - Wave needs medium, Audible range limited, Reflected, Diffracted, Speed varies, Frequency determines pitch, Intensity determines loudness, Resonates at natural frequencies}
\end{mnemonicbox}

\questionmarks{5(અ)}{3}{લેસરનો અર્થ અને ગુણધર્મો જણાવો.}

\begin{solutionbox}
\textbf{LASER}: Light Amplification by Stimulated Emission of Radiation

\textbf{લેસર પ્રકાશના ગુણધર્મો:}
\begin{itemize}
    \item \textbf{એકવર્ણીય}: એક તરંગલંબાઈ
    \item \textbf{સુસંબદ્ધ}: બધા તરંગો એકબીજા સાથે કળામાં હોય છે
    \item \textbf{દિશાત્મક}: સીધી રેખામાં પ્રવાસ કરે છે, નીચું વિચલન
    \item \textbf{તીવ્ર}: ઉચ્ચ ઊર્જા કેન્દ્રિકરણ
    \item \textbf{સમાંતર}: પ્રકાશ કિરણો સમાંતર હોય છે
\end{itemize}
\end{solutionbox}

\begin{mnemonicbox}
\mnemonic{MCCDI - Monochromatic and Coherent, Collimated, Directional, Intense}
\end{mnemonicbox}

\questionmarks{5(બ)}{4}{ઓપ્ટિકલ ફાઈબર વિષે માહિતી આપો.}

\begin{solutionbox}
\textbf{ઓપ્ટિકલ ફાઈબર}: લવચીક, પારદર્શક ફાઈબર જે સંપૂર્ણ આંતરિક પરાવર્તન દ્વારા પ્રકાશ સિગ્નલો પ્રસારિત કરે છે.

\textbf{રચના:}
\begin{center}
\begin{tikzpicture}
    \draw[fill=white] (0,0) ellipse (0.2 and 0.5); % Core end
    \draw (0,0.5) -- (4,0.5);
    \draw (0,-0.5) -- (4,-0.5);
    \draw (4,0) ellipse (0.2 and 0.5);
    \node at (2,0) {Core (કોર) ($n_1$)};
    
    \draw (0,0.8) -- (4,0.8);
    \draw (0,-0.8) -- (4,-0.8);
    \node at (2,0.65) {Cladding (ક્લેડિંગ) ($n_2$)};
    \node at (2,-0.65) {Cladding (ક્લેડિંગ) ($n_2$)};
    
    \node[right] at (4.5,0) {$n_1 > n_2$};
\end{tikzpicture}
\captionof{figure}{ઓપ્ટિકલ ફાઈબર રચના}
\end{center}

\textbf{ઘટકો:}
\begin{itemize}
    \item \textbf{કોર}: કેન્દ્રીય વિસ્તાર (ઉચ્ચ વક્રીભવનાંક)
    \item \textbf{ક્લેડિંગ}: બાહ્ય ઓપ્ટિકલ પદાર્થ (નીચો વક્રીભવનાંક)
    \item \textbf{બફર કોટિંગ}: રક્ષણાત્મક આવરણ
\end{itemize}

\textbf{પ્રકારો:} સિંગલ-મોડ (નાનો કોર), મલ્ટી-મોડ (મોટો કોર).
\end{solutionbox}

\begin{mnemonicbox}
\mnemonic{CCTLT - Core Carries light, Cladding keeps it in, Total internal reflection, Low loss transmission}
\end{mnemonicbox}

\questionmarks{5(ક)(1)}{7}{સ્નેલનો નિયમ સમજાવો.}

\begin{solutionbox}
\textbf{વ્યાખ્યા:} સ્નેલનો નિયમ કહે છે કે આપતિના ખૂણાના સાઇનનો વક્રીભવનના ખૂણાના સાઇન સાથેનો ગુણોત્તર અચળ રહે છે.

\textbf{સૂત્ર:} $n_1 \sin(\theta_1) = n_2 \sin(\theta_2)$

\textbf{આકૃતિ:}
\begin{center}
\begin{tikzpicture}
    % Boundary
    \draw[thick] (-2,0) -- (2,0);
    \node[above left] at (-2,0) {Medium 1 (માધ્યમ 1) ($n_1$)};
    \node[below left] at (-2,0) {Medium 2 (માધ્યમ 2) ($n_2$)};
    
    % Normal
    \draw[dashed] (0,-2) -- (0,2);
    
    % Rays
    \draw[thick, red, ->] (-1.5,1.5) -- (0,0);
    \draw[thick, red, ->] (0,0) -- (1,-1.5);
    
    % Angles
    \draw (0,0.5) arc (90:135:0.5);
    \node at (-0.3,0.7) {$\theta_1$};
    
    \draw (0,-0.5) arc (270:303:0.5);
    \node at (0.3,-0.7) {$\theta_2$};
\end{tikzpicture}
\captionof{figure}{વક્રીભવન (સ્નેલનો નિયમ)}
\end{center}
\end{solutionbox}

\begin{mnemonicbox}
\mnemonic{SINS - Sine of incidence over sine of refraction equals N1 over N2}
\end{mnemonicbox}

\questionmarks{5(ક)(2)}{0}{એસેપ્ટન્સ એંગલ સમજાવો.}

\begin{solutionbox}
\textbf{એસેપ્ટન્સ એંગલ} એ મહત્તમ ખૂણો છે જેના પર પ્રકાશ ઓપ્ટિકલ ફાઈબરમાં પ્રવેશી શકે છે અને હજુ પણ સંપૂર્ણ આંતરિક પરાવર્તન અનુભવી શકે છે.

\textbf{સૂત્ર:} $\theta_a = \sin^{-1}(NA)$ જ્યાં $NA = \sqrt{n_1^2 - n_2^2}$

\textbf{આકૃતિ:}
\begin{center}
\begin{tikzpicture}
    % Fiber face
    \draw[thick] (2,-1) -- (2,1);
    \draw[thick] (2,0.5) -- (6,0.5);
    \draw[thick] (2,-0.5) -- (6,-0.5);
    
    % Cone
    \draw[dashed] (0,0) -- (2,0.5);
    \draw[dashed] (0,0) -- (2,-0.5);
    \draw[->] (1,0) arc (0:14:1);
    \node at (1.5,0.1) {$\theta_a$};
    
    % Axis
    \draw[dotted] (-1,0) -- (6,0);
\end{tikzpicture}
\captionof{figure}{એસેપ્ટન્સ કોન}
\end{center}
\end{solutionbox}

\begin{mnemonicbox}
\mnemonic{CAP - Core and cladding indices Affect the acceptance angle}
\end{mnemonicbox}

\questionmarks{5(અ) OR}{3}{લેસરના ઉપયોગો લખો.}

\begin{solutionbox}
\textbf{લેસરના ઉપયોગો:}
\begin{center}
\captionof{table}{લેસરના ઉપયોગો}
\begin{tabulary}{\linewidth}{|L|L|}
\hline
\textbf{ક્ષેત્ર} & \textbf{ઉપયોગો} \\ \hline
તબીબી & સર્જરી, આંખની સારવાર, કેન્સર થેરાપી \\ \hline
ઔદ્યોગિક & કટિંગ, વેલ્ડિંગ, 3D પ્રિન્ટિંગ \\ \hline
સંચાર & ફાઇબર ઓપ્ટિક્સ \\ \hline
વૈજ્ઞાનિક & સ્પેક્ટ્રોસ્કોપી, હોલોગ્રાફી \\ \hline
ગ્રાહક & બારકોડ સ્કેનર, પ્રિન્ટર \\ \hline
લશ્કરી & રેન્જ શોધ, શસ્ત્રો \\ \hline
\end{tabulary}
\end{center}
\end{solutionbox}

\begin{mnemonicbox}
\mnemonic{MICSM - Medical, Industrial, Communication, Scientific, Military}
\end{mnemonicbox}

\questionmarks{5(બ) OR}{4}{પ્રકાશનું પૂર્ણ આંતરિક પરાવર્તન પર ટૂંક નોંધ લખો.}

\begin{solutionbox}
\textbf{પૂર્ણ આંતરિક પરાવર્તન (TIR)} ત્યારે થાય છે જ્યારે ઘન માધ્યમમાં પ્રવાસ કરતો પ્રકાશ ક્રાંતિક ખૂણા કરતાં મોટા ખૂણે ઓછા ઘન માધ્યમ સાથેની સીમાને અથડાય છે.

\textbf{શરતો:}
\begin{itemize}
    \item પ્રકાશ ઘન માધ્યમથી ઓછા ઘન માધ્યમમાં પ્રવાસ કરવો જોઈએ ($n_1 > n_2$)
    \item આપતિનો ખૂણો > ક્રાંતિક ખૂણો ($\theta_i > \theta_c$)
\end{itemize}

\textbf{ક્રાંતિક ખૂણાનું સૂત્ર:} $\theta_c = \sin^{-1}(n_2/n_1)$

\textbf{આકૃતિ:}
\begin{center}
\begin{tikzpicture}
    \draw (-3,0) -- (3,0);
    \node[below] at (0,0) {Air (હવા) ($n_2$)};
    \node[above] at (0,0) {Water (પાણી) ($n_1$)};
    
    % Case 1: Refraction
    \draw[->] (-2,1) -- (-2,0);
    \draw[->] (-2,0) -- (-1.5,-1);
    
    % Case 2: Critical
    \draw[->] (0,1) -- (0,0);
    \draw[->] (0,0) -- (1,0); 
    
    % Case 3: TIR
    \draw[->] (2,1.5) -- (2,0);
    \draw[->] (2,0) -- (1,1.5);
\end{tikzpicture}
\captionof{figure}{પૂર્ણ આંતરિક પરાવર્તન}
\end{center}
\end{solutionbox}

\begin{mnemonicbox}
\mnemonic{CANDO - Critical Angle, N1 Denser, Only when angle > Critical}
\end{mnemonicbox}

\questionmarks{5(ક)(1) OR}{3}{પાણીમાં પ્રકાશનો વેગ $2.25 \times 10^8$ m/s અને હવામાં પ્રકાશનો વેગ $3 \times 10^8$ m/s હોય તો પાણીનો વક્રીભવનાંક શોધો.}

\begin{solutionbox}
\textbf{આપેલ છે:}
\begin{itemize}
    \item $v_w = 2.25 \times 10^8$ m/s
    \item $v_a = 3 \times 10^8$ m/s
\end{itemize}

\textbf{સૂત્ર:} $n = c/v \Rightarrow n_w = v_a/v_w$

\textbf{ગણતરી:}
$$
n_w = \frac{3 \times 10^8}{2.25 \times 10^8} = \frac{3}{2.25} = 1.33
$$

તેથી, પાણીનો વક્રીભવનાંક 1.33 છે.
\end{solutionbox}

\begin{mnemonicbox}
\mnemonic{SVN - Speed in Vacuum divided by Speed in medium gives refractive iNdex}
\end{mnemonicbox}

\questionmarks{5(ક)(2) OR}{4}{સ્ટેપ ઈન્ડેક્ષ ફાઈબર વિષે નોંધ લખો.}

\begin{solutionbox}
\textbf{સ્ટેપ ઈન્ડેક્ષ ફાઈબર:}
એક પ્રકારનો ઓપ્ટિકલ ફાઈબર જ્યાં વક્રીભવનાંક કોર અને ક્લેડિંગ વચ્ચે અચાનક બદલાય છે.

\textbf{લક્ષણો:}
\begin{itemize}
    \item કોર-ક્લેડિંગ સીમા પર અચાનક ફેરફાર
    \item સિંગલ-મોડ અને મલ્ટી-મોડ
    \item સરળ બાંધકામ
    \item મલ્ટી-મોડમાં વધુ ફેલાવો
\end{itemize}

\textbf{આકૃતિ:}
\begin{center}
\begin{tikzpicture}
    % Fiber profile
    \draw[thick] (0,0) -- (3,0);
    \draw[thick] (0,1) -- (3,1);
    \draw[thick] (0,2) -- (3,2);
    \draw[thick] (0,3) -- (3,3);
    
    \node at (1.5,0.5) {Cladding (ક્લેડિંગ)};
    \node at (1.5,1.5) {Core (કોર)};
    \node at (1.5,2.5) {Cladding (ક્લેડિંગ)};
    
    % RI Profile
    \draw[->] (4,0) -- (6,0) node[right] {n};
    \draw[->] (4,0) -- (4,3) node[above] {Position (સ્થાન)};
    
    \draw[thick, blue] (4,0.5) -- (5,0.5) -- (5,1) -- (5.5,1) -- (5.5,2) -- (5,2) -- (5,2.5);
    \node[below] at (5,0) {$n_2$};
    \node[below] at (5.5,0) {$n_1$};
\end{tikzpicture}
\captionof{figure}{સ્ટેપ ઈન્ડેક્ષ ફાઈબર પ્રોફાઇલ}
\end{center}
\end{solutionbox}

\begin{mnemonicbox}
\mnemonic{SACS - Step change, Abrupt profile, Core guides, Simple}
\end{mnemonicbox}

\end{document}
