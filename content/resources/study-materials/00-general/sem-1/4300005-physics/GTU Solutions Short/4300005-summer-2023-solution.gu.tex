\documentclass{article}

% content/resources/templates/preamble.tex
\usepackage[margin=0.6in]{geometry}
\author{Milav Dabgar}
\usepackage{amsmath,amssymb,amsthm}
\usepackage{booktabs}
\usepackage{multirow}
\usepackage{xcolor}
\usepackage{tcolorbox}
\tcbuselibrary{breakable,skins}
\usepackage[colorlinks=true,linkcolor=blue]{hyperref}
\usepackage{titlesec}
\usepackage{enumitem}
\usepackage{tikz}
\usepackage{pgfplots}
\usepackage{circuitikz}
\usepackage[version=4]{mhchem}
\usepackage{longtable}
\usepackage{array}
\usepackage{float}
\usepackage{caption}
\usepackage{listings}

\lstset{
  basicstyle=\small\ttfamily,
  breaklines=true,
  breakatwhitespace=false,
  postbreak=\mbox{\textcolor{red}{$\hookrightarrow$}\space},
  float=false,
  numbers=left,
  numberstyle=\tiny\color{gray},
  numbersep=10pt,
  xleftmargin=2em,
  keywordstyle=\color{blue},
  commentstyle=\color{green!60!black},
  stringstyle=\color{purple},
  backgroundcolor=\color{gray!5},
  showstringspaces=false,
  tabsize=2,
  captionpos=b,
  keepspaces=true,
  columns=flexible
}

\pgfplotsset{compat=1.18}
\usetikzlibrary{shapes,arrows,positioning,calc,patterns,decorations.pathmorphing,decorations.markings,arrows.meta}

% Color scheme
\definecolor{headcolor}{RGB}{0,102,204}
\definecolor{keycolor}{RGB}{220,20,60}
\definecolor{solutioncolor}{RGB}{34,139,34}
\definecolor{mnemoniccolor}{RGB}{148,0,211}
\definecolor{codecolor}{RGB}{0,0,100}

% Spacing
\setlength{\parskip}{3pt}
\setlist[itemize]{nosep}
\setlist[enumerate]{nosep}

% Title formatting
\titleformat{\section}{\Large\bfseries\color{headcolor}}{\thesection}{1em}{}
\titleformat{\subsection}{\large\bfseries\color{headcolor}}{\thesubsection}{1em}{}

% Pandoc tightlist compatibility
\providecommand{\tightlist}{%
  \setlength{\itemsep}{0pt}\setlength{\parskip}{0pt}}

% Pandoc longtable compatibility
\newcounter{none}
\def\thenone{}


% content/resources/templates/gujarati-boxes.tex
\usepackage{fontspec}
\usepackage{polyglossia}

% Set Gujarati as main language (document is primarily in Gujarati)
% Note: gloss-gujarati.ldf doesn't exist in polyglossia, but it will use hyphenation patterns
\setdefaultlanguage{gujarati}
\setotherlanguage{english}

% Configure Gujarati font properly
% Use Language=Default to prevent polyglossia from trying to add language-specific features
% that don't exist for Gujarati, which causes "empty feature" warnings
\newfontfamily\gujaratifont[Script=Gujarati,AutoFakeBold=2.5,AutoFakeSlant=0.3]{Noto Sans Gujarati}
\setmainfont[Script=Gujarati,AutoFakeBold=2.5,AutoFakeSlant=0.3]{Noto Sans Gujarati}
% Use Noto Sans Gujarati for monospace to support Gujarati in text
\setmonofont[Scale=0.9]{Noto Sans Gujarati}

% Configure English to use the same font
\newfontfamily\englishfont[Script=Gujarati,AutoFakeBold=2.5,AutoFakeSlant=0.3]{Noto Sans Gujarati}

% Translations for polyglossia
\gappto\captionsgujarati{
  \renewcommand{\tablename}{કોષ્ટક}
  \renewcommand{\figurename}{આકૃતિ}
}

% Helper for TikZ nodes to ensure Gujarati font
\newcommand{\gu}[1]{{\gujaratifont #1}}

% Custom environments
\newtcolorbox{solutionbox}{
    breakable,
    enhanced,
    colback=solutioncolor!5!white,
    colframe=solutioncolor!75!black,
    fonttitle=\bfseries,
    title=જવાબ
}

\newtcolorbox{solutionboxnobreak}{
 colback=solutioncolor!5!white,
 colframe=solutioncolor!75!black,
 fonttitle=\bfseries,
 title=જવાબ
}

\newtcolorbox{keyformula}{
 breakable,
 enhanced,
 colback=keycolor!5!white,
 colframe=keycolor!75!black,
 fonttitle=\bfseries,
 title=રાસાયણિક સમીકરણ/સૂત્ર
}

\newtcolorbox{mnemonicbox}{
 breakable,
 enhanced,
 colback=mnemoniccolor!5!white,
 colframe=mnemoniccolor!75!black,
 fonttitle=\bfseries,
 title=મેમરી ટ્રીક
}


% Custom commands for GTU solutions
% This file defines semantic commands for consistent formatting

% Question command with automatic formatting
\newcommand{\question}[2]{%
  \section*{Question #1}%
  \textbf{#2}%
}

% OR question variant
\newcommand{\questionor}[2]{%
  \section*{Question #1 OR}%
  \textbf{#2}%
}

% Proper table environment with caption
\newenvironment{answertable}[1]{%
  \begin{table}[htbp]
  \centering
  \caption{#1}
}{%
  \end{table}
}

% Proper figure environment for diagrams
\newenvironment{answerdiagram}[1]{%
  \begin{figure}[htbp]
  \centering
  \caption{#1}
}{%
  \end{figure}
}

% Semantic markup for key terms
\newcommand{\keyword}[1]{\textbf{#1}}
\newcommand{\code}[1]{\texttt{#1}}
\newcommand{\classname}[1]{\texttt{#1}}
\newcommand{\methodname}[1]{\texttt{#1}}

% Proper quotation marks
\newcommand{\mnemonic}[1]{``#1''}

\usetikzlibrary{decorations.pathmorphing}

\title{Physics (4300005) - Summer 2023 Solution}
\date{August 04, 2023}

\begin{document}
\maketitle

\questionmarks{1(a)}{3}{SI માં બેઝ યુનિટ તેમના સિમ્બોલ સાથે લખો.}

\begin{solutionbox}
\begin{answertable}{SI બેઝ યુનિટ્સ}
\begin{tabulary}{\linewidth}{|L|L|L|}
\hline
\textbf{ભૌતિક રાશિ} & \textbf{એકમ} & \textbf{સિમ્બોલ} \\ \hline
લંબાઈ & મીટર & m \\ \hline
દ્રવ્યમાન & કિલોગ્રામ & kg \\ \hline
સમય & સેકન્ડ & s \\ \hline
વિદ્યુત પ્રવાહ & એમ્પિયર & A \\ \hline
તાપમાન & કેલ્વિન & K \\ \hline
પદાર્થનું પ્રમાણ & મોલ & mol \\ \hline
પ્રકાશ તીવ્રતા & કેન્ડેલા & cd \\ \hline
\end{tabulary}
\end{answertable}
\end{solutionbox}

\begin{mnemonicbox}
\mnemonic{"લાંબુ માપ તાપમાન અશક્તિ પ્રકાશે કેવી માનવતા"}
\end{mnemonicbox}

\questionmarks{1(b)}{4}{વર્નિયર કેલિપરની રચના અને કાર્ય સમજાવો. તેની લઘુત્તમ માપ શક્તિ અને શૂન્ય ત્રુટી સમજાવો.}

\begin{solutionbox}
\textbf{વર્નિયર કેલિપરની રચના:}

\begin{center}
\begin{tikzpicture}
    % Main Scale
    \draw[thick, fill=gray!10] (0,0) rectangle (8, 1);
    \foreach \x in {0.5,1,...,7.5} \draw (\x, 1) -- (\x, 0.7);
    \foreach \x in {0,1,...,8} \draw[thick] (\x, 1) -- (\x, 0.5) node[below] {\small \x};
    \node at (4, -0.3) {મુખ્ય સ્કેલ};
    
    % Vernier Scale
    \draw[thick, fill=blue!10] (1.5, 0) rectangle (4.5, -0.8);
    \foreach \x in {1.7,2.0,...,4.3} \draw (\x, 0) -- (\x, -0.3);
    \node at (3, -1.1) {વર્નિયર સ્કેલ};
    
    % Jaws
    \draw[thick, fill=gray!20] (0,0) -- (0,-2) -- (0.5,-2) -- (0.5,-1) -- cycle; % Fixed Jaw
    \node at (0.25, -2.3) {ફિક્સ્ડ જૉ};
    
    \draw[thick, fill=blue!20] (1.5,0) -- (1.5,-2) -- (2,-2) -- (2,-1) -- cycle; % Movable Jaw
    \node at (1.75, -2.3) {મૂવેબલ જૉ};
    
    % Object
    \draw[fill=red!30] (0.5, -1.5) circle (0.5);
    \node at (0.5, -1.5) {વસ્તુ};
\end{tikzpicture}
\captionof{figure}{વર્નિયર કેલિપરની રચના}
\end{center}

\begin{itemize}
    \item \textbf{મુખ્ય સ્કેલ}: મિલિમીટરમાં અંકિત ફિક્સ થયેલો સ્કેલ
    \item \textbf{વર્નિયર સ્કેલ}: મુખ્ય સ્કેલ કરતાં થોડા નાના વિભાગો ધરાવતો સરકી શકે તેવો સ્કેલ
    \item \textbf{ફિક્સ્ડ જૉ}: મુખ્ય સ્કેલ સાથે જોડાયેલો
    \item \textbf{મૂવેબલ જૉ}: વર્નિયર સ્કેલ સાથે જોડાયેલો
    \item \textbf{ઊંડાઈ માપક રોડ}: ઊંડાઈ માપવા માટે
    \item \textbf{લોકિંગ સ્ક્રૂ}: માપન વખતે સ્થિતિ ફિક્સ કરવા માટે
\end{itemize}

\textbf{કાર્ય}: વસ્તુને બે જૉ વચ્ચે મૂકવામાં આવે છે, મૂવેબલ જૉને વસ્તુને સારી રીતે પકડવા માટે એડજસ્ટ કરવામાં આવે છે. મુખ્ય સ્કેલ વાંચન અને વર્નિયર સ્કેલના સંપાતી મૂલ્યને ઉમેરીને માપ નોંધવામાં આવે છે.

\textbf{લઘુત્તમ માપ શક્તિ}: વર્નિયર કેલિપર દ્વારા માપી શકાતું સૌથી નાનું માપ.
\[ \text{LC} = \frac{\text{મુખ્ય સ્કેલ પર 1 વિભાગ}}{\text{વર્નિયર સ્કેલ પર વિભાગોની સંખ્યા}} \]

\textbf{શૂન્ય ત્રુટી}: જ્યારે જૉ બંધ હોય ત્યારે કેલિપર શૂન્ય સિવાયનું વાંચન બતાવે તે ત્રુટી.
\begin{itemize}
    \item \textbf{ધન ત્રુટી}: વાંચનમાંથી બાદ કરવી
    \item \textbf{ઋણ ત્રુટી}: વાંચનમાં ઉમેરવી
\end{itemize}
\end{solutionbox}

\begin{mnemonicbox}
\mnemonic{"વર્નિયર ચોક્કસ માપ લેતા સમયે ત્રુટીઓ ટાળે"}
\end{mnemonicbox}

\questionmarks{1(c)(i)}{4}{ચોકસાઈ અને સચોટતા વચ્ચેનો તફાવત લખો.}

\begin{solutionbox}
\begin{answertable}{ચોકસાઈ વિ સચોટતા}
\begin{tabulary}{\linewidth}{|L|L|}
\hline
\textbf{ચોકસાઈ} & \textbf{સચોટતા} \\ \hline
માપનું સાચા મૂલ્યની નજીકતા & માપની પુનરાવર્તનીયતા \\ \hline
પદ્ધતિગત ત્રુટીઓથી પ્રભાવિત & અનિયમિત ત્રુટીઓથી પ્રભાવિત \\ \hline
માપનના સરેરાશ દ્વારા દર્શાવાય છે & માપના પ્રમાણિત વિચલન દ્વારા દર્શાવાય છે \\ \hline
કેલિબ્રેશન દ્વારા સુધારી શકાય & વધુ સારા ઉપકરણો વાપરીને સુધારી શકાય \\ \hline
ઉદાહરણ: જો સાચું મૂલ્ય 10 cm હોય, તો 9.9, 10.1, અને 10.0 cm ના માપ ચોક્કસ છે & ઉદાહરણ: 9.8, 9.8, 9.8 cm ના માપ સચોટ છે પણ સાચું મૂલ્ય 10 cm હોય તો ચોક્કસ નથી \\ \hline
\end{tabulary}
\end{answertable}
\end{solutionbox}

\begin{mnemonicbox}
\mnemonic{"ચોક્સાઈ ચોક્કસ સાચા મૂલ્યે, સચોટતા સરખાં સમાન વાંચને"}
\end{mnemonicbox}

\questionmarks{1(c)(ii)}{2}{માઇક્રોમીટર સ્ક્રૂ ગેજની પિચ 0.5 mm છે અને તેના વર્તુળાકાર સ્કેલ પર 50 વિભાગો છે. તેની લઘુત્તમ માપ શક્તિ શોધો.}

\begin{solutionbox}
\textbf{સૂત્ર}: 
\[ \text{લઘુત્તમ માપ શક્તિ} = \frac{\text{પિચ}}{\text{વર્તુળાકાર સ્કેલ પર વિભાગોની સંખ્યા}} \]

\textbf{ગણતરી}:
\[ \text{LC} = \frac{0.5 \text{ mm}}{50} = 0.01 \text{ mm} \]

\textbf{માઇક્રોમીટર સ્ક્રૂ ગેજની લઘુત્તમ માપ શક્તિ = 0.01 mm}
\end{solutionbox}

\questionmarks{1(c)(iii)}{1}{ઉષ્માનું SI એકમ શું છે?}

\begin{solutionbox}
ઉષ્માનું SI એકમ \textbf{જૂલ (J)} છે
\end{solutionbox}

\orquestionmarks{1(c)(i)}{4}{નિરપેક્ષ અને સાપેક્ષ ત્રુટીઓની ગણતરી કેવી રીતે કરવામાં આવે છે?}

\begin{solutionbox}
\textbf{નિરપેક્ષ ત્રુટિ ($\Delta a$)}: માપેલા મૂલ્ય અને સાચા મૂલ્ય વચ્ચેનો તફાવત.
ઘણા માપો માટે, તે માપેલા મૂલ્ય અને સરેરાશ મૂલ્ય વચ્ચેનો તફાવત છે.

\textbf{નિરપેક્ષ ત્રુટિની ગણતરી}:
\begin{itemize}
    \item \textbf{એક માપ માટે}: $\Delta a = |\text{માપેલું મૂલ્ય} - \text{સાચું મૂલ્ય}|$
    \item \textbf{ઘણા માપો માટે}: 
    \begin{enumerate}
        \item સરેરાશ ગણો ($a_m$)
        \item દરેક માપ માટે: $\Delta a_i = |a_i - a_m|$
        \item સરેરાશ નિરપેક્ષ ત્રુટિ: $\Delta a = (\Delta a_1 + \Delta a_2 + ... + \Delta a_n) \div n$
    \end{enumerate}
\end{itemize}

\textbf{સાપેક્ષ ત્રુટિ ($\epsilon_r$)}: નિરપેક્ષ ત્રુટિનો સાચા મૂલ્ય સાથેનો ગુણોત્તર.
\[ \epsilon_r = \frac{\text{નિરપેક્ષ ત્રુટિ}}{\text{સાચું મૂલ્ય}} = \frac{\Delta a}{\text{સાચું મૂલ્ય}} \]

\textbf{ટકાવારી ત્રુટિ ($\epsilon_p$)}: ટકાવારીમાં વ્યક્ત થયેલી સાપેક્ષ ત્રુટિ.
\[ \epsilon_p = \text{સાપેક્ષ ત્રુટિ} \times 100 = \left(\frac{\Delta a}{\text{સાચું મૂલ્ય}}\right) \times 100\% \]
\end{solutionbox}

\begin{mnemonicbox}
\mnemonic{"નિરપેક્ષ નિશ્ચિત મૂલ્યની ગણતરી, સાપેક્ષ સાચા સંદર્ભે સંબંધિત"}
\end{mnemonicbox}

\orquestionmarks{1(c)(ii)}{2}{વર્નિયર કેલિપરનો મુખ્ય સ્કેલ mm માં અંકિત કરવામાં આવેલ છે અને તેના વર્નિયર સ્કેલ પર 50 વિભાગો છે. તેની લઘુત્તમ માપ શક્તિ શોધો.}

\begin{solutionbox}
\textbf{સૂત્ર}: 
\[ \text{લઘુત્તમ માપ શક્તિ} = \frac{\text{મુખ્ય સ્કેલ પર 1 વિભાગ}}{\text{વર્નિયર સ્કેલ પર વિભાગોની સંખ્યા}} \]

\textbf{ગણતરી}:
મુખ્ય સ્કેલ પર 1 વિભાગ = 1 mm
\[ \text{LC} = \frac{1 \text{ mm}}{50} = 0.02 \text{ mm} \]

\textbf{વર્નિયર કેલિપરની લઘુત્તમ માપ શક્તિ = 0.02 mm}
\end{solutionbox}

\orquestionmarks{1(c)(iii)}{1}{ઉષ્મા પ્રસરણના કયા પ્રકારમાં માધ્યમની જરૂર નથી?}

\begin{solutionbox}
\textbf{વિકિરણ (Radiation)} ઉષ્મા પ્રસરણ માટે માધ્યમની જરૂર નથી.
\end{solutionbox}

\questionmarks{2(a)}{3}{વિદ્યુત ક્ષેત્ર રેખાઓની લાક્ષણિકતાઓ લખો.}

\begin{solutionbox}
\textbf{વિદ્યુત ક્ષેત્ર રેખાઓની લાક્ષણિકતાઓ}:
\begin{enumerate}
    \item વિદ્યુત ક્ષેત્ર રેખાઓ ધન ચાર્જથી શરૂ થાય છે અને ઋણ ચાર્જ પર સમાપ્ત થાય છે
    \item ક્ષેત્ર રેખાઓ ક્યારેય એકબીજાને છેદતી નથી
    \item ક્ષેત્ર રેખાઓ હંમેશા વાહકની સપાટી પર લંબરૂપ હોય છે
    \item ક્ષેત્ર રેખાઓની સંખ્યા ચાર્જના જથ્થા સાથે પ્રમાણસર હોય છે
    \item નજીકની ક્ષેત્ર રેખાઓ મજબૂત વિદ્યુત ક્ષેત્ર સૂચવે છે
    \item ક્ષેત્ર રેખાઓ સતત વક્ર હોય છે
    \item ક્ષેત્ર રેખાઓ લંબાઈમાં સંકોચાય છે અને પહોળાઈમાં વિસ્તરે છે
\end{enumerate}

\begin{answerdiagram}{વિદ્યુત ક્ષેત્ર રેખાઓની ભૂમિતિ}
\begin{tikzpicture}
    % Positive charge
    \node[circle, draw, fill=blue!10, minimum size=0.8cm] (P) at (-2,0) {+};
    % Negative charge
    \node[circle, draw, fill=red!10, minimum size=0.8cm] (N) at (2,0) {-};
    
    % Field lines
    \foreach \angle in {0, 45, ..., 315} {
        \draw[->, blue] (P) -- +(\angle:0.8cm);
    }
    \foreach \angle in {0, 45, ..., 315} {
        \draw[<-, red] (N) -- +(\angle:0.8cm);
    }
    
    % Connecting lines
    \draw[->, thick, purple, bend left=10] (P) to[out=45,in=135] (N);
    \draw[->, thick, purple, bend left=10] (P) to[out=-45,in=-135] (N);
    \draw[->, thick, purple] (P) -- (N);
    
    % Non-crossing annotation
    \node at (0, -1.5) {\small રેખાઓ + થી શરૂ થઈ - પર અંત પામે છે};
\end{tikzpicture}
\end{answerdiagram}
\end{solutionbox}

\begin{mnemonicbox}
\mnemonic{"વિદ્યુત ક્ષેત્ર: ધનથી શરૂ, ઋણે સમાપ્ત, ક્યારેય છેદાતી નથી"}
\end{mnemonicbox}

\questionmarks{2(b)}{4}{ઇલેક્ટ્રોસ્ટેટિક બળ માટે કુલંબનો વ્યસ્ત વર્ગનો નિયમને સમજાવો.}

\begin{solutionbox}
\textbf{કુલંબનો વ્યસ્ત વર્ગનો નિયમ}: બે બિંદુ ચાર્જ વચ્ચેનું ઇલેક્ટ્રોસ્ટેટિક બળ ચાર્જના જથ્થાના ગુણાકાર સાથે સીધું પ્રમાણસર અને તેમની વચ્ચેના અંતરના વર્ગ સાથે વ્યસ્ત પ્રમાણસર હોય છે.

\textbf{ગણિતીય સ્વરૂપ}:
\[ F = k \frac{q_1 q_2}{r^2} \]

જ્યાં:
\begin{itemize}
    \item $F$ = ઇલેક્ટ્રોસ્ટેટિક બળ (ન્યૂટનમાં)
    \item $k$ = ઇલેક્ટ્રોસ્ટેટિક અચળાંક ($9\times 10^9$ N·m$^2$/C$^2$)
    \item $q_1, q_2$ = ચાર્જના જથ્થા (કુલંબમાં)
    \item $r$ = ચાર્જ વચ્ચેનું અંતર (મીટરમાં)
\end{itemize}

\textbf{ગુણધર્મો}:
\begin{itemize}
    \item \textbf{સદિશ રાશિ}: બળ બે ચાર્જને જોડતી રેખા પર કાર્ય કરે છે
    \item \textbf{આકર્ષક/અપાકર્ષક}: સમાન ચાર્જ એકબીજાને અપાકર્ષિત કરે છે, વિપરીત ચાર્જ આકર્ષિત કરે છે
    \item \textbf{કેન્દ્રીય બળ}: ન્યૂટનના ત્રીજા નિયમને અનુસરે છે
    \item \textbf{માધ્યમ પર આધાર}: ચાર્જ વચ્ચેના માધ્યમ પર આધાર રાખે છે ($k$ બદલાય છે)
\end{itemize}

\begin{answerdiagram}{કુલંબ નો નિયમ}
\begin{tikzpicture}
    \node[circle, draw, fill=green!10] (q1) at (0,0) {$q_1$};
    \node[circle, draw, fill=green!10] (q2) at (4,0) {$q_2$};
    
    \draw[dashed] (q1) -- (q2) node[midway, below] {$r$};
    
    \draw[->, thick, blue] (q1) -- (-1.5,0) node[left] {$F_{12}$};
    \draw[->, thick, blue] (q2) -- (5.5,0) node[right] {$F_{21}$};
    
    \node at (2, 1) {અપાકર્ષણ બળ (સમાન વિદ્યુતભાર)};
\end{tikzpicture}
\end{answerdiagram}
\end{solutionbox}

\begin{mnemonicbox}
\mnemonic{"ચાર્જ અંતરના વર્ગ સાથે વ્યસ્ત સંબંધ ધરાવે"}
\end{mnemonicbox}

\questionmarks{2(c)(i)}{4}{શ્રેણી અને સમાંતર સંયોજનમાં જોડાયેલા કેપેસિટર્સની સમતુલ્ય કેપેસીટન્સ માટે સૂત્ર મેળવો.}

\begin{solutionbox}
\textbf{શ્રેણી સંયોજન માટે}:

\begin{answerdiagram}{શ્રેણીમાં કેપેસિટર્સ}
\begin{tikzpicture}
    \draw (0,2) to[short, o-] (1,2)
          to[C, l=$C_1$, v=$V_1$] (3,2)
          to[C, l=$C_2$, v=$V_2$] (5,2)
          to[C, l=$C_3$, v=$V_3$] (7,2)
          to[short, -o] (8,2);
    \node at (0,2) [left] {+};
    \node at (8,2) [right] {-};
    \node at (4,1) {\small ચાર્જ $Q$ સમાન છે};
\end{tikzpicture}
\end{answerdiagram}

જ્યારે કેપેસિટર્સ શ્રેણી સંયોજનમાં જોડાય છે:
\begin{itemize}
    \item દરેક કેપેસિટર પર સમાન ચાર્જ $Q$ હોય છે
    \item વિભવાંતર દરેક કેપેસિટર વચ્ચે વહેંચાય છે
    \item $V = V_1 + V_2 + V_3$
\end{itemize}

દરેક કેપેસિટર માટે: $V_1 = Q/C_1$, $V_2 = Q/C_2$, $V_3 = Q/C_3$

કુલ વોલ્ટેજ: 
\[ V = \frac{Q}{C_1} + \frac{Q}{C_2} + \frac{Q}{C_3} = Q \left(\frac{1}{C_1} + \frac{1}{C_2} + \frac{1}{C_3}\right) \]

સમતુલ્ય કેપેસિટન્સ માટે: $V = Q/C_{eq}$

તેથી: 
\[ \frac{1}{C_{eq}} = \frac{1}{C_1} + \frac{1}{C_2} + \frac{1}{C_3} \]

\textbf{સમાંતર સંયોજન માટે}:

\begin{answerdiagram}{સમાંતરમાં કેપેસિટર્સ}
\begin{tikzpicture}
    \draw (0,2) to[short, o-] (2,2) -- (2,3) to[C, l=$C_1$] (6,3) -- (6,2) to[short, -o] (8,2);
    \draw (2,2) to[C, l=$C_2$] (6,2);
    \draw (2,2) -- (2,1) to[C, l=$C_3$] (6,1) -- (6,2);
    
    \node at (0,2) [left] {+};
    \node at (8,2) [right] {-};
    \node at (4,-0.5) {\small વોલ્ટેજ $V$ સમાન છે};
\end{tikzpicture}
\end{answerdiagram}

જ્યારે કેપેસિટર્સ સમાંતર સંયોજનમાં જોડાય છે:
\begin{itemize}
    \item દરેક કેપેસિટર પર સમાન વિભવાંતર $V$ હોય છે
    \item કુલ ચાર્જ દરેક કેપેસિટર વચ્ચે વહેંચાય છે
    \item $Q = Q_1 + Q_2 + Q_3$
\end{itemize}

દરેક કેપેસિટર માટે: $Q_1 = C_1V$, $Q_2 = C_2V$, $Q_3 = C_3V$

કુલ ચાર્જ: 
\[ Q = C_1V + C_2V + C_3V = (C_1 + C_2 + C_3)V \]

સમતુલ્ય કેપેસિટન્સ માટે: $Q = C_{eq}V$

તેથી: 
\[ C_{eq} = C_1 + C_2 + C_3 \]
\end{solutionbox}

\begin{mnemonicbox}
\mnemonic{"શ્રેણીમાં વ્યસ્ત કેપેસિટન્સની સરવાળો, સમાંતરમાં કેપેસિટન્સનો સરવાળો"}
\end{mnemonicbox}

\questionmarks{2(c)(ii)}{2}{8 $\mu$F અને 9 $\mu$F કેપેસિટન્સ ધરાવતા બે કેપેસિટર્સ સમાંતર સંયોજનમાં જોડાયેલા છે. સમતુલ્ય કેપેસિટન્સ શોધો.}

\begin{solutionbox}
\textbf{સમાંતર સંયોજન માટે સૂત્ર}: $C_{eq} = C_1 + C_2$

\textbf{આપેલ}:
\begin{itemize}
    \item $C_1 = 8 \mu\text{F}$
    \item $C_2 = 9 \mu\text{F}$
\end{itemize}

\textbf{ગણતરી}:
\[ C_{eq} = 8 \mu\text{F} + 9 \mu\text{F} = 17 \mu\text{F} \]

\textbf{આથી, સમતુલ્ય કેપેસિટન્સ = 17 $\mu$F}
\end{solutionbox}

\questionmarks{2(c)(iii)}{1}{LASER નું પૂરું નામ લખો.}

\begin{solutionbox}
\textbf{LASER}: Light Amplification by Stimulated Emission of Radiation
(પ્રકાશનું ઉત્તેજિત ઉત્સર્જન દ્વારા પ્રવર્ધન)
\end{solutionbox}

\orquestionmarks{2(a)}{3}{કેપેસિટર શું છે? કેપેસિટન્સને વ્યાખ્યાયિત કરો અને તેનું એકમ લખો.}

\begin{solutionbox}
\textbf{કેપેસિટર}: એક ઉપકરણ જે વિદ્યુત ક્ષેત્રના સ્વરૂપમાં વિદ્યુત ચાર્જ અને વિદ્યુત ઊર્જા સંગ્રહિત કરે છે.

\textbf{કેપેસિટન્સ}: કેપેસિટરની વિદ્યુત ચાર્જ સંગ્રહિત કરવાની ક્ષમતા. તે લાગુ કરેલ વિભવાંતર સાથે સંગ્રહિત ચાર્જના ગુણોત્તર તરીકે વ્યાખ્યાયિત થાય છે.

\textbf{ગણિતીય સ્વરૂપ}:
\[ C = \frac{Q}{V} \]

જ્યાં:
\begin{itemize}
    \item $C$ = કેપેસિટન્સ
    \item $Q$ = કેપેસિટર પર સંગ્રહિત ચાર્જ
    \item $V$ = કેપેસિટર પરનો વિભવાંતર
\end{itemize}

\textbf{કેપેસિટન્સનું એકમ}: ફેરડ (F)

\begin{answerdiagram}{સમાંતર પ્લેટ કેપેસિટર}
\begin{tikzpicture}
    \draw[thick] (0, 0) -- (0, 3) node[midway, left] {+Q}; 
    \draw[thick] (1, 0) -- (1, 3) node[midway, right] {-Q};
    \draw (0, 1.5) -- (-1, 1.5);
    \draw (1, 1.5) -- (2, 1.5);
    \foreach \y in {0.2, 0.5, ..., 2.8} \node at (-0.2, \y) {+};
    \foreach \y in {0.2, 0.5, ..., 2.8} \node at (1.2, \y) {-};
    \node at (0.5, 3.2) {ડાઇલેક્ટ્રિક};
\end{tikzpicture}
\end{answerdiagram}
\end{solutionbox}

\begin{mnemonicbox}
\mnemonic{"કેપેસિટર ચાર્જ સંગ્રહે, વોલ્ટેજ વિભાજિત કરે"}
\end{mnemonicbox}

\orquestionmarks{2(b)}{4}{વિદ્યુત ક્ષેત્રની તીવ્રતા અને વિદ્યુત સ્થિતિમાન સમજાવો.}

\begin{solutionbox}
\textbf{વિદ્યુત ક્ષેત્રની તીવ્રતા}:
\begin{itemize}
    \item \textbf{વ્યાખ્યા}: તે બિંદુ પર મૂકાયેલા એકમ ધન ચાર્જને લાગતું બળ
    \item \textbf{સૂત્ર}: $E = F/q$
    \item \textbf{એકમ}: ન્યૂટન/કુલંબ (N/C) અથવા વોલ્ટ/મીટર (V/m)
    \item \textbf{સદિશ રાશિ}: જેમાં તીવ્રતા અને દિશા બંને હોય છે
    \item \textbf{દિશા}: ધન ચાર્જ પર લાગતા બળની દિશા જેવી જ
\end{itemize}

\textbf{વિદ્યુત સ્થિતિમાન}:
\begin{itemize}
    \item \textbf{વ્યાખ્યા}: અનંતથી તે બિંદુ સુધી એકમ ધન ચાર્જને લાવવા માટે કરેલું કાર્ય
    \item \textbf{સૂત્ર}: $V = W/q$
    \item \textbf{એકમ}: વોલ્ટ (V) અથવા જૂલ/કુલંબ (J/C)
    \item \textbf{અદિશ રાશિ}: ફક્ત તીવ્રતા ધરાવે છે
    \item \textbf{ક્ષેત્ર સાથે સંબંધ}: $E = -dV/dr$ (ક્ષેત્ર સ્થિતિમાનનો નકારાત્મક ગ્રેડિયન્ટ છે)
\end{itemize}

\begin{answertable}{ક્ષેત્ર વિ સ્થિતિમાન}
\begin{tabulary}{\linewidth}{|L|L|L|}
\hline
\textbf{ગુણધર્મ} & \textbf{વિદ્યુત ક્ષેત્ર} & \textbf{વિદ્યુત સ્થિતિમાન} \\ \hline
વ્યાખ્યા & એકમ ચાર્જ દીઠ બળ & એકમ ચાર્જ દીઠ કાર્ય \\ \hline
પ્રકૃતિ & સદિશ & અદિશ \\ \hline
એકમ & N/C અથવા V/m & V અથવા J/C \\ \hline
નિર્ભરતા & $1/r^2$ સાથે બદલાય & $1/r$ સાથે બદલાય \\ \hline
દિશા & ધન ચાર્જથી દૂર & કોઈ દિશા નથી \\ \hline
\end{tabulary}
\end{answertable}
\end{solutionbox}

\begin{mnemonicbox}
\mnemonic{"વિદ્યુત ક્ષેત્ર બળ આપે; સ્થિતિમાન ઊર્જા આપે"}
\end{mnemonicbox}

\orquestionmarks{2(c)(i)}{4}{સમાંતર પ્લેટ કેપેસિટરના કેપેસીટન્સના સૂત્રનો ઉપયોગ કરીને પ્લેટનો ક્ષેત્રફળ, પ્લેટો વચ્ચેનું અંતર અને પ્લેટો વચ્ચે ડાઇલેક્ટ્રિક સામગ્રીની ઉપસ્થિતિની તેની કેપેસિટન્સ પર અસરને સમજાવો.}

\begin{solutionbox}
\textbf{સમાંતર પ્લેટ કેપેસિટરના કેપેસિટન્સનું સૂત્ર}:
\[ C = \frac{\epsilon_0 \epsilon_r A}{d} \]

જ્યાં:
\begin{itemize}
    \item $C$ = કેપેસિટન્સ
    \item $\epsilon_0$ = નિર્વાત અવકાશની પરમિટિવિટી ($8.85\times 10^{-12}$ F/m)
    \item $\epsilon_r$ = ડાઇલેક્ટ્રિકની સાપેક્ષ પરમિટિવિટી
    \item $A$ = પ્લેટોના ઓવરલેપનો ક્ષેત્રફળ
    \item $d$ = પ્લેટો વચ્ચેનું અંતર
\end{itemize}

\textbf{પ્લેટના ક્ષેત્રફળની અસર ($A$)}:
\begin{itemize}
    \item કેપેસિટન્સ પ્લેટના ક્ષેત્રફળ સાથે સીધું પ્રમાણસર છે
    \item ક્ષેત્રફળ વધારતાં $\rightarrow$ કેપેસિટન્સ વધે છે
    \item ક્ષેત્રફળ બમણો કરતાં $\rightarrow$ કેપેસિટન્સ બમણું થાય છે
\end{itemize}

\textbf{અંતરની અસર ($d$)}:
\begin{itemize}
    \item કેપેસિટન્સ પ્લેટો વચ્ચેના અંતર સાથે વ્યસ્ત પ્રમાણસર છે
    \item અંતર વધારતાં $\rightarrow$ કેપેસિટન્સ ઘટે છે
    \item અંતર બમણું કરતાં $\rightarrow$ કેપેસિટન્સ અડધું થાય છે
\end{itemize}

\textbf{ડાઇલેક્ટ્રિક સામગ્રીની અસર ($\epsilon_r$)}:
\begin{itemize}
    \item કેપેસિટન્સ ડાઇલેક્ટ્રિકની સાપેક્ષ પરમિટિવિટી સાથે સીધું પ્રમાણસર છે
    \item ડાઇલેક્ટ્રિક દાખલ કરતાં $\rightarrow$ કેપેસિટન્સ વધે છે
    \item ડાઇલેક્ટ્રિક અચળાંક આ વધારાનું માપ કરે છે: $C_{\text{dielectric}} = \epsilon_r \times C_{\text{air}}$
\end{itemize}

\begin{answerdiagram}{કેપેસિટન્સને અસર કરતા પરિબળો}
\begin{tikzpicture}
    % Plates
    \draw[thick, fill=gray!20] (0,0) rectangle (4,0.2) node[midway] {પ્લેટ ક્ષેત્રફળ $A$};
    \draw[thick, fill=gray!20] (0,2) rectangle (4,2.2);
    
    % Distance
    \draw[<->] (4.2, 0.2) -- (4.2, 2) node[midway, right] {$d$};
    
    % Dielectric
    \draw[fill=blue!10, dashed] (0.5, 0.2) rectangle (3.5, 2) node[midway] {ડાઇલેક્ટ્રિક $\epsilon_r$};
    
    \node at (2, -0.5) {સમાંતર પ્લેટ રચના};
\end{tikzpicture}
\end{answerdiagram}
\end{solutionbox}

\begin{mnemonicbox}
\mnemonic{"ક્ષેત્રફળ વધારે, અંતર ઘટાડે, ડાઇલેક્ટ્રિક ગુણાકારે"}
\end{mnemonicbox}

\orquestionmarks{2(c)(ii)}{2}{0.5 $\mu$F ના કેપેસિટરની પ્લેટો વચ્ચેનો વોલ્ટેજ 150 V છે. પ્લેટો પર ઇલેક્ટ્રિક ચાર્જનું મૂલ્ય શોધો.}

\begin{solutionbox}
\textbf{સૂત્ર}: $Q = CV$

\textbf{આપેલ}:
\begin{itemize}
    \item કેપેસિટન્સ ($C$) = 0.5 $\mu$F = $0.5 \times 10^{-6}$ F
    \item વોલ્ટેજ ($V$) = 150 V
\end{itemize}

\textbf{ગણતરી}:
\[ Q = CV = 0.5 \times 10^{-6} \times 150 = 75 \times 10^{-6} \text{ C} = 75 \mu\text{C} \]

\textbf{આથી, પ્લેટો પરનો ચાર્જ = 75 $\mu$C}
\end{solutionbox}

\orquestionmarks{2(c)(iii)}{1}{ઓપ્ટિકલ ફાઇબરના બે ભાગ કોર અને ક્લેડિંગ માંથી, કયો ભાગ મોટો રીફ્રેક્ટિવ ઇન્ડેક્સ ધરાવે છે?}

\begin{solutionbox}
\textbf{કોર (core)} ક્લેડિંગ કરતાં વધારે રીફ્રેક્ટિવ ઇન્ડેક્સ ધરાવે છે.
\end{solutionbox}

\questionmarks{3(a)}{3}{ઉષ્માનું વહન અને ઉષ્મા નયન વ્યાખ્યાયિત કરો.}

\begin{solutionbox}
\textbf{ઉષ્મા વહન (Heat Conduction)}:
\begin{itemize}
    \item કણોની વાસ્તવિક હલનચલન વિના પદાર્થ દ્વારા ઉષ્માનું સ્થાનાંતર
    \item સીધા પરમાણુ અથડામણોને કારણે થાય છે
    \item ઉષ્મા ઉચ્ચ તાપમાનથી નીચા તાપમાન તરફ વહે છે
    \item ધાતુઓ ઉષ્માના સારા વાહક છે
    \item ઉદાહરણો: ધાતુના સળિયા દ્વારા ઉષ્માનું સ્થાનાંતર, રસોઈના વાસણ
\end{itemize}

\textbf{ઉષ્મા નયન (Heat Convection)}:
\begin{itemize}
    \item પદાર્થની વાસ્તવિક હલનચલન દ્વારા ઉષ્માનું સ્થાનાંતર
    \item પ્રવાહીઓમાં (પ્રવાહી અને વાયુઓ) થાય છે
    \item નયન પ્રવાહો (convection currents) ની રચના શામેલ છે
    \item ઉદાહરણો: રૂમ હીટર, દરિયાઈ લહેર, ઉકળતું પાણી
\end{itemize}

\begin{answerdiagram}{ફેરબદલીના પ્રકારો}
\begin{tikzpicture}
    % Conduction
    \node[gtu block] (hot) at (0,0) {Hot};
    \node[gtu block] (cold) at (4,0) {Cold};
    \draw[->, thick, red, decorate, decoration={snake}] (hot) -- (cold) node[midway, above] {ઉષ્મા વહન (ઘન)};
    
    % Convection
    \draw[fill=blue!10] (6,-1) rectangle (8,1);
    \node at (7, -1.3) {પ્રવાહી};
    \draw[->, thick, red, bend right] (6.5, -0.5) to (7.5, 0.5);
    \draw[->, thick, blue, bend right] (7.5, 0.5) to (6.5, -0.5);
    \node at (7, 1.2) {નયન પ્રવાહો};
\end{tikzpicture}
\end{answerdiagram}
\end{solutionbox}

\begin{mnemonicbox}
\mnemonic{"વહન અણુઓને જોડે છે; નયન સામગ્રી વહન કરે છે"}
\end{mnemonicbox}

\questionmarks{3(b)}{4}{મર્ક્યુરી થર્મોમીટરની રચના અને કાર્ય સમજાવો.}

\begin{solutionbox}
\textbf{મર્ક્યુરી થર્મોમીટરની રચના}:

\begin{answerdiagram}{મર્ક્યુરી થર્મોમીટર}
\begin{tikzpicture}
    \draw[thick] (0,0) -- (8,0);
    \draw[thick] (0,0.5) -- (8,0.5);
    \draw[thick, fill=gray!20] (-1, 0.25) circle (0.4); % Bulb
    \draw[thick, fill=gray!50] (-1, 0.25) circle (0.3); % Mercury
    \draw[thick, gray!50] (-0.7, 0.25) -- (4, 0.25); % Mercury thread
    
    % Scale
    \foreach \x in {0,1,...,7} \draw (\x, 0.5) -- (\x, 0.7) node[above] {\small \x0};
    \foreach \x in {0.5,1.5,...,7.5} \draw (\x, 0.5) -- (\x, 0.6);
    
    \node at (-1, -0.5) {કાચનો બલ્બ};
    \node at (4, -0.5) {કેશતંતુ નળી};
    \node at (7, 1.0) {સ્કેલ ($^{\circ}$C)};
\end{tikzpicture}
\end{answerdiagram}

\begin{itemize}
    \item \textbf{કાચનો બલ્બ}: પારો ધરાવે છે, જળાશય તરીકે કામ કરે છે
    \item \textbf{કેશતંતુ નળી}: બલ્બ સાથે જોડાયેલી પાતળી કાચની નળી
    \item \textbf{સ્કેલ}: તાપમાનના નિશાન સાથે અંકિત
    \item \textbf{સુરક્ષા કવર}: કેશતંતુ નળી અને સ્કેલનું રક્ષણ કરે છે
\end{itemize}

\textbf{કાર્ય સિદ્ધાંત}:
\begin{enumerate}
    \item પારાના થર્મલ પ્રસરણ પર આધારિત
    \item જ્યારે તાપમાન વધે છે, ત્યારે પારો વિસ્તરે છે અને નળીમાં ઉપર ચઢે છે
    \item જ્યારે તાપમાન ઘટે છે, ત્યારે પારો સંકોચાય છે અને સ્તર નીચે આવે છે
    \item પારાના સ્તર પર સ્કેલમાંથી તાપમાન વાંચવામાં આવે છે
\end{enumerate}

\textbf{તાપમાન શ્રેણી}: -38.83$^{\circ}$C થી 356.73$^{\circ}$C.

\textbf{ફાયદાઓ}: ઉચ્ચ ચોકસાઈ, રેખીય પ્રસરણ, દૃશ્યમાન.
\textbf{મર્યાદાઓ}: ઝેરી પારો, ખૂબ નીચા તાપમાન માપી શકાતા નથી.
\end{solutionbox}

\begin{mnemonicbox}
\mnemonic{"પારો કેશતંતુમાંથી પસાર થાય, તાપમાન બતાવે"}
\end{mnemonicbox}

\questionmarks{3(c)(i)}{4}{ઉષ્મા વાહકતાના નિયમો જણાવો અને ઉષ્મા વાહકતા અચળાંકનું સૂત્ર મેળવો.}

\begin{solutionbox}
\textbf{ઉષ્મા વાહકતાના નિયમો}:
\begin{enumerate}
    \item ઉષ્મા પ્રવાહ તાપમાન તફાવત ($\Delta T$) સાથે સીધો પ્રમાણસર છે
    \item ઉષ્મા પ્રવાહ આડછેદના ક્ષેત્રફળ ($A$) સાથે સીધો પ્રમાણસર છે
    \item ઉષ્મા પ્રવાહ લંબાઈ ($L$) સાથે વ્યસ્ત પ્રમાણસર છે
    \item ઉષ્મા પ્રવાહ સમય ($t$) સાથે સીધો પ્રમાણસર છે
\end{enumerate}

\textbf{ઉષ્મા વાહકતા અચળાંકનું તારણ}:

ફૂરિયરના નિયમ મુજબ:
\[ Q \propto A \times t \times \frac{\Delta T}{L} \]

પ્રમાણસરતા અચળાંક $K$ સાથે સમીકરણમાં રૂપાંતરિત કરતા:
\[ Q = K \times A \times t \times \frac{\Delta T}{L} \]

પુનઃ ગોઠવણી:
\[ K = \frac{Q \times L}{A \times t \times \Delta T} \]

જ્યાં:
\begin{itemize}
    \item $Q$ = વહન થયેલ ઉષ્મા (જૂલમાં)
    \item $L$ = વાહકની લંબાઈ (મીટરમાં)
    \item $A$ = આડછેદનું ક્ષેત્રફળ (m$^2$ માં)
    \item $t$ = સમય (સેકન્ડમાં)
    \item $\Delta T$ = તાપમાન તફાવત (કેલ્વિનમાં)
    \item $K$ = ઉષ્મા વાહકતા અચળાંક (W/m·K માં)
\end{itemize}

\begin{answerdiagram}{ફેરબદલી મોડેલ}
\begin{tikzpicture}
    % Bar
    \draw[thick, fill=orange!10] (0,0) rectangle (4,1);
    \node at (2, 0.5) {વાહક (લંબાઈ $L$, ક્ષેત્રફળ $A$)};
    
    % Temps
    \node at (-1, 0.5) {ગરમ $T_1$};
    \node at (5, 0.5) {ઠંડુ $T_2$};
    
    % Heat flow
    \draw[->, ultra thick, red] (-0.5, 0.5) -- (0, 0.5);
    \draw[->, ultra thick, red] (4, 0.5) -- (4.5, 0.5) node[right] {ઉષ્મા $Q$};
\end{tikzpicture}
\end{answerdiagram}
\end{solutionbox}

\begin{mnemonicbox}
\mnemonic{"ક્ષેત્રફળ વધુ, તાપમાન ઉચ્ચ, લંબાઈ ઓછી હોય ત્યારે ઉષ્મા ઝડપથી વહે છે"}
\end{mnemonicbox}

\questionmarks{3(c)(ii)}{2}{કાચની બારીના તકતીનું કુલ ક્ષેત્રફળ 0.5 m$^2$ છે. જો કાચની જાડાઈ 0.6 cm હોય, અંદરનું તાપમાન 30$^{\circ}$C અને બહારનું તાપમાન 20$^{\circ}$C હોય તો તકતી દ્વારા પ્રતિ કલાક વહન થતી ઉષ્માની ગણતરી કરો. કાચનો ઉષ્મા વાહકતા અચળાંક 1.0 Wm$^{-1}$K$^{-1}$ છે.}

\begin{solutionbox}
\textbf{સૂત્ર}: $Q = \frac{K \times A \times t \times \Delta T}{L}$

\textbf{આપેલ}:
\begin{itemize}
    \item ક્ષેત્રફળ ($A$) = 0.5 m$^2$
    \item જાડાઈ ($L$) = 0.6 cm = 0.006 m
    \item અંદરનું તાપમાન ($T_1$) = 30$^{\circ}$C
    \item બહારનું તાપમાન ($T_2$) = 20$^{\circ}$C
    \item તાપમાન તફાવત ($\Delta T$) = 10$^{\circ}$C = 10 K
    \item ઉષ્મા વાહકતા અચળાંક ($K$) = 1.0 W/m·K
    \item સમય ($t$) = 1 કલાક = 3600 સેકન્ડ
\end{itemize}

\textbf{ગણતરી}:
\[ Q = \frac{1.0 \times 0.5 \times 3600 \times 10}{0.006} \]
\[ Q = \frac{18000}{0.006} \]
\[ Q = 3,000,000 \text{ J} = 3000 \text{ kJ} \]

\textbf{આથી, વહન થયેલ ઉષ્મા = 3000 kJ પ્રતિ કલાક}
\end{solutionbox}

\questionmarks{3(c)(iii)}{1}{પ્રકાશનો કયો ગુણધર્મ ઓપ્ટિકલ ફાઇબર દ્વારા પ્રકાશના પ્રસારણ માટે જવાબદાર છે?}

\begin{solutionbox}
\textbf{પૂર્ણ આંતરિક પરાવર્તન (Total Internal Reflection - TIR)} ઓપ્ટિકલ ફાઇબર દ્વારા પ્રકાશના પ્રસારણ માટે જવાબદાર છે.
\end{solutionbox}

\orquestionmarks{3(a)}{3}{ઉષ્મા ક્ષમતા અને વિશિષ્ટ ઉષ્મા વ્યાખ્યાયિત કરો.}

\begin{solutionbox}
\textbf{ઉષ્મા ક્ષમતા (Heat Capacity)}:
\begin{itemize}
    \item પદાર્થનું તાપમાન 1$^{\circ}$C અથવા 1K વધારવા માટે જરૂરી ઉષ્મા ઊર્જાની માત્રા
    \item પદાર્થની દળ અને સામગ્રી પર આધારિત છે
    \item સૂત્ર: $C = Q/\Delta T$
    \item એકમ: જૂલ/કેલ્વિન (J/K)
\end{itemize}

\textbf{વિશિષ્ટ ઉષ્મા (Specific Heat)}:
\begin{itemize}
    \item 1 કિલોગ્રામ પદાર્થનું તાપમાન 1$^{\circ}$C અથવા 1K વધારવા માટે જરૂરી ઉષ્મા ઊર્જાની માત્રા
    \item સામગ્રીનો ગુણધર્મ છે, દળથી સ્વતંત્ર છે
    \item સૂત્ર: $c = Q/(m\times\Delta T)$
    \item એકમ: જૂલ/kg·K (J/kg·K)
\end{itemize}

\textbf{સંબંધ}: ઉષ્મા ક્ષમતા ($C$) = દળ ($m$) $\times$ વિશિષ્ટ ઉષ્મા ($c$)

\begin{answertable}{ઉષ્મા ક્ષમતા વિ વિશિષ્ટ ઉષ્મા}
\begin{tabulary}{\linewidth}{|L|L|L|}
\hline
\textbf{ગુણધર્મ} & \textbf{ઉષ્મા ક્ષમતા} & \textbf{વિશિષ્ટ ઉષ્મા} \\ \hline
વ્યાખ્યા & પદાર્થ માટે પ્રતિ ડિગ્રી ઉષ્મા & એકમ દળ દીઠ પ્રતિ ડિગ્રી ઉષ્મા \\ \hline
સંજ્ઞા & $C$ & $c$ \\ \hline
એકમ & J/K & J/kg·K \\ \hline
આધાર & દળ અને સામગ્રી & ફક્ત સામગ્રી \\ \hline
સૂત્ર & $Q/\Delta T$ & $Q/(m\times\Delta T)$ \\ \hline
\end{tabulary}
\end{answertable}
\end{solutionbox}

\begin{mnemonicbox}
\mnemonic{"ઉષ્મા ક્ષમતા સંપૂર્ણ પદાર્થ માટે, વિશિષ્ટ ઉષ્મા એક કિલોગ્રામ માટે"}
\end{mnemonicbox}

\orquestionmarks{3(b)}{4}{ઓપ્ટિકલ પાયરોમીટરની રચના અને કાર્ય સમજાવો.}

\begin{solutionbox}
\textbf{ઓપ્ટિકલ પાયરોમીટરની રચના}:

\begin{answerdiagram}{ઓપ્ટિકલ પાયરોમીટર બ્લોક ડાયાગ્રામ}
\begin{tikzpicture}[node distance=1.5cm]
    \node[gtu block] (obj) {ગરમ પદાર્થ};
    \node[gtu block, right=of obj] (lens) {ઓબ્જેક્ટિવ લેન્સ};
    \node[gtu block, right=of lens] (lamp) {ફિલામેન્ટ લેમ્પ};
    \node[gtu block, right=of lamp] (filter) {રેડ ફિલ્ટર};
    \node[gtu block, right=of filter] (eye) {આઈપીસ};
    
    \node[gtu block, below=of lamp] (ammeter) {એમીટર};
    \node[gtu block, left=of ammeter] (bat) {બેટરી};
    \node[gtu block, right=of ammeter] (rheo) {રિયોસ્ટેટ};
    
    \draw[gtu arrow, dashed] (obj) -- (lens);
    \draw[gtu arrow, dashed] (lens) -- (lamp);
    \draw[gtu arrow, dashed] (lamp) -- (filter);
    \draw[gtu arrow, dashed] (filter) -- (eye);
    
    \draw[thick] (bat) -- (ammeter);
    \draw[thick] (ammeter) -- (rheo);
    \draw[thick] (rheo) -| (lamp);
    \draw[thick] (lamp) -| (bat);
\end{tikzpicture}
\end{answerdiagram}

\begin{itemize}
    \item \textbf{ટેલિસ્કોપ}: ગરમ પદાર્થ જોવા માટે
    \item \textbf{ફિલામેન્ટ લેમ્પ}: કેલિબ્રેટ કરેલ ટંગસ્ટન ફિલામેન્ટ
    \item \textbf{રિયોસ્ટેટ}: ફિલામેન્ટમાંથી પસાર થતો પ્રવાહ નિયંત્રિત કરવા
    \item \textbf{એમીટર}: પ્રવાહ માપવા માટે
    \item \textbf{રેડ ફિલ્ટર}: તરંગલંબાઇ મેચ કરવા માટે
    \item \textbf{આઈપીસ}: જોવા માટે
\end{itemize}

\textbf{કાર્ય સિદ્ધાંત}:
\begin{enumerate}
    \item ગરમ પદાર્થની તેજસ્વીતા સાથે પ્રમાણભૂત લેમ્પ ફિલામેન્ટની સરખામણી પર આધારિત
    \item પદાર્થને ટેલિસ્કોપ દ્વારા જોવામાં આવે છે
    \item જ્યાં સુધી ફિલામેન્ટ તેજસ્વીતા પદાર્થની તેજસ્વીતા સાથે મેચ ન થાય ત્યાં સુધી પ્રવાહ એડજસ્ટ કરવામાં આવે છે
    \item મેચ પોઈન્ટ પર, ફિલામેન્ટ પદાર્થની પૃષ્ઠભૂમિ સામે "અદૃશ્ય" થઈ જાય છે
    \item કેલિબ્રેટ કરેલ સ્કેલ અથવા એમીટર રીડિંગ પરથી તાપમાન નક્કી કરવામાં આવે છે
\end{enumerate}

\textbf{તાપમાન શ્રેણી}: 700$^{\circ}$C થી 3000$^{\circ}$C
\end{solutionbox}

\begin{mnemonicbox}
\mnemonic{"પાયરોમીટર તેજસ્વીતા સરખામણી દ્વારા ચોક્કસ તાપમાન આપે છે"}
\end{mnemonicbox}

\orquestionmarks{3(c)(i)}{4}{ઘન પદાર્થોનું રેખીય ઉષ્મીય પ્રસરણ વ્યાખ્યાયિત કરો અને રેખીય ઉષ્મીય પ્રસરણ અચળાંકનું સૂત્ર મેળવો.}

\begin{solutionbox}
\textbf{રેખીય ઉષ્મીય પ્રસરણ}: ઘન પદાર્થનું તાપમાન વધારતાં તેની લંબાઈમાં થતો વધારો.

\textbf{રેખીય ઉષ્મીય પ્રસરણ અચળાંક ($\alpha$)}: તાપમાનના પ્રતિ એકમ ફેરફાર દીઠ લંબાઈમાં થતો આંશિક ફેરફાર.

\textbf{તારણ}:
નાના તાપમાન ફેરફારો માટે:
\begin{itemize}
    \item લંબાઈમાં ફેરફાર ($\Delta L$) મુળ લંબાઈ ($L_0$) સાથે સીધો પ્રમાણસર છે
    \item $\Delta L$ તાપમાન ફેરફાર ($\Delta T$) સાથે સીધો પ્રમાણસર છે
\end{itemize}

તેથી: $\Delta L \propto L_0 \times \Delta T$

પ્રમાણસરતા અચળાંક $\alpha$ સાથે સમીકરણમાં રૂપાંતરિત કરતા:
\[ \Delta L = \alpha \times L_0 \times \Delta T \]

પુનઃ ગોઠવણી:
\[ \alpha = \frac{\Delta L}{L_0 \times \Delta T} \]

જ્યાં:
\begin{itemize}
    \item $\Delta L$ = લંબાઈમાં ફેરફાર (મીટરમાં)
    \item $L_0$ = મુળ લંબાઈ (મીટરમાં)
    \item $\Delta T$ = તાપમાન ફેરફાર (કેલ્વિન અથવા સેલ્સિયસમાં)
    \item $\alpha$ = રેખીય ઉષ્મીય પ્રસરણ અચળાંક (પ્રતિ $^{\circ}$C અથવા પ્રતિ K)
\end{itemize}

\textbf{અંતિમ લંબાઈ}: $L = L_0(1 + \alpha\Delta T)$

\begin{answerdiagram}{રેખીય પ્રસરણ}
\begin{tikzpicture}
    % Before
    \draw[thick, fill=blue!20] (0,1) rectangle (4,1.5);
    \node at (2, 1.25) {$L_0$ ($T_1$ પર)};
    
    % After
    \draw[thick, fill=red!20] (0,0) rectangle (5,0.5);
    \draw[dashed] (4, -0.2) -- (4, 1.7);
    \draw[<->] (4, -0.2) -- (5, -0.2) node[midway, below] {$\Delta L$};
    \node at (2, 0.25) {$L$ ($T_2$ પર)};
\end{tikzpicture}
\end{answerdiagram}
\end{solutionbox}

\begin{mnemonicbox}
\mnemonic{"રેખીય પ્રસરણ કુલ લંબાઈ વધારો આપે છે"}
\end{mnemonicbox}

\orquestionmarks{3(c)(ii)}{2}{0$^{\circ}$C પર સ્ટીલના સળિયાની લંબાઈ 150 cm છે. જો તેનો રેખીય ઉષ્મીય પ્રસરણ અચળાંક 12 $\times$ 10$^{-6}$ પ્રતિ $^{\circ}$C હોય, તો 200$^{\circ}$C પર તેની લંબાઈ કેટલી હશે?}

\begin{solutionbox}
\textbf{સૂત્ર}: $L = L_0(1 + \alpha\Delta T)$

\textbf{આપેલ}:
\begin{itemize}
    \item મુળ લંબાઈ ($L_0$) = 150 cm
    \item તાપમાન ફેરફાર ($\Delta T$) = 200$^{\circ}$C
    \item રેખીય પ્રસરણ અચળાંક ($\alpha$) = $12 \times 10^{-6}$ પ્રતિ $^{\circ}$C
\end{itemize}

\textbf{ગણતરી}:
\[ L = 150(1 + 12 \times 10^{-6} \times 200) \]
\[ L = 150(1 + 24 \times 10^{-4}) \]
\[ L = 150(1 + 0.0024) = 150 \times 1.0024 = 150.36 \text{ cm} \]

\textbf{આથી, સ્ટીલના સળિયાની અંતિમ લંબાઈ = 150.36 cm}
\end{solutionbox}

\orquestionmarks{3(c)(iii)}{1}{સામાન્ય પ્રકાશના ઉત્સર્જન માટે કયા પ્રકારનું રેડિયેશન ઉત્સર્જન જવાબદાર છે?}

\begin{solutionbox}
સામાન્ય પ્રકાશના ઉત્સર્જન માટે \textbf{સ્વયંસ્ફુરિત ઉત્સર્જન (Spontaneous emission)} જવાબદાર છે.
\end{solutionbox}

\questionmarks{4(a)}{3}{તરંગના કંપવિસ્તાર, આવૃત્તિ અને આવર્તકાળ વ્યાખ્યાયિત કરો.}

\begin{solutionbox}
\textbf{કંપવિસ્તાર (Amplitude)}:
\begin{itemize}
    \item મધ્યમાન સ્થાનથી માધ્યમના કણોનું મહત્તમ સ્થાનાંતર
    \item તરંગની ઊર્જા દર્શાવે છે
    \item '$A$' વડે દર્શાવાય છે, મીટર (m) માં મપાય છે
\end{itemize}

\textbf{આવૃત્તિ (Frequency)}:
\begin{itemize}
    \item એકમ સમય દીઠ પૂર્ણ થતા દોલનોની સંખ્યા
    \item '$f$' અથવા '$\nu$' વડે દર્શાવાય છે, હર્ટ્ઝ (Hz) માં મપાય છે
    \item $f = v/\lambda$
\end{itemize}

\textbf{આવર્તકાળ (Time Period)}:
\begin{itemize}
    \item એક દોલન પૂર્ણ કરવા માટે લાગતો સમય
    \item '$T$' વડે દર્શાવાય છે, સેકન્ડ (s) માં મપાય છે
    \item $T = 1/f$
\end{itemize}

\begin{answerdiagram}{તરંગના પરિમાણો}
\begin{tikzpicture}
    \draw[->] (0,0) -- (6,0) node[right] {સમય};
    \draw[->] (0,-2) -- (0,2) node[above] {સ્થાનાંતર};
    
    \draw[thick, blue, domain=0:6, samples=100] plot (\x, {1.5*sin(100*\x)});
    
    \draw[<->] (0.9,0) -- (0.9,1.5) node[midway, left] {કંપવિસ્તાર $A$};
    \draw[<->] (0.9,1.5) -- (4.5,1.5);
    \draw[dashed] (4.5,0) -- (4.5,1.5);
    
    \draw[<->] (0.9,-1.7) -- (4.5,-1.7) node[midway, below] {આવર્તકાળ $T$};
    \draw[dashed] (0.9,0) -- (0.9,-1.7);
    \draw[dashed] (4.5,0) -- (4.5,-1.7);
\end{tikzpicture}
\end{answerdiagram}
\end{solutionbox}

\begin{mnemonicbox}
\mnemonic{"કંપવિસ્તાર ઊર્જા ગોઠવે, આવૃત્તિ ચક્રો શોધે, આવર્તકાળ એક ચક્ર ટ્રેક કરે"}
\end{mnemonicbox}

\questionmarks{4(b)}{4}{લંબગત અને સંગત તરંગો વચ્ચેનો તફાવત લખો.}

\begin{solutionbox}
\begin{answertable}{લંબગત વિ સંગત તરંગો}
\begin{tabulary}{\linewidth}{|L|L|L|}
\hline
\textbf{ગુણધર્મ} & \textbf{લંબગત તરંગો (Transverse)} & \textbf{સંગત તરંગો (Longitudinal)} \\ \hline
ગતિની દિશા & પ્રસરણની લંબ દિશામાં & પ્રસરણની દિશામાં સમાંતર \\ \hline
રચના & શૃંગ અને ગર્ત & સંઘનન અને વિઘનન \\ \hline
ઉદાહરણો & પ્રકાશ, પાણી, EM તરંગો & ધ્વનિ, ભૂકંપના P-તરંગો \\ \hline
માધ્યમ & શૂન્યાવકાશમાં પણ ગતિ કરી શકે & પદાર્થ માધ્યમ જરૂરી છે \\ \hline
ધ્રુવીભવન & ધ્રુવીભૂત થઈ શકે & ધ્રુવીભૂત થઈ શકતા નથી \\ \hline
સમીકરણ & $y = A \sin(kx - \omega t)$ & $s = A \sin(kx - \omega t)$ \\ \hline
\end{tabulary}
\end{answertable}

\begin{answerdiagram}{તરંગના પ્રકારો}
\begin{tikzpicture}
    % Transverse
    \node at (-1, 1) {લંબગત:};
    \draw[thick, blue, domain=0:4, samples=100] plot (\x, {0.5*sin(360*\x)});
    \draw[->] (0,0) -- (4.5,0) node[right] {પ્રસરણ};
    \draw[<->] (1, -0.6) -- (1, 0.6) node[above] {કણ};
    
    % Longitudinal
    \node at (-1, -1.5) {સંગત:};
    \foreach \x in {0, 0.4, 0.8, 1.2, 1.6, 2.0, 2.4, 2.8, 3.2, 3.6, 4.0} {
        \draw[thick] (\x, -1) -- (\x, -2);
    }
    \foreach \x in {0.9, 1.0, 1.1, 2.9, 3.0, 3.1} {
        \draw[thick] (\x, -1) -- (\x, -2); 
    }
    \draw[->] (0,-2.2) -- (4.5,-2.2) node[right] {પ્રસરણ};
    \draw[<->] (2, -1.5) -- (2.5, -1.5) node[right] {કણ};
\end{tikzpicture}
\end{answerdiagram}
\end{solutionbox}

\begin{mnemonicbox}
\mnemonic{"લંબગત લંબરૂપે ચાલે, સંગત લંબાઈ સાથે ચાલે"}
\end{mnemonicbox}

\questionmarks{4(c)(i)}{5}{પીઝોઇલેક્ટ્રિક પદ્ધતિનો ઉપયોગ કરીને અલ્ટ્રાસોનિક તરંગ કેવી રીતે ઉત્પન્ન થાય છે?}

\begin{solutionbox}
\begin{answerdiagram}{પીઝોઇલેક્ટ્રિક પદ્ધતિ}
\begin{tikzpicture}[node distance=1.5cm]
    \node[gtu block] (osc) {ઓસિલેટર};
    \node[gtu block, right=of osc] (amp) {એમ્પ્લીફાયર};
    \node[gtu block, right=of amp] (crys) {પીઝોઇલેક્ટ્રિક ક્રિસ્ટલ};
    \node[gtu output, right=of crys] (wave) {અલ્ટ્રાસોનિક તરંગો};
    
    \draw[gtu arrow] (osc) -- (amp);
    \draw[gtu arrow] (amp) -- (crys);
    \draw[gtu arrow] (crys) -- (wave);
\end{tikzpicture}
\end{answerdiagram}

\textbf{કાર્ય સિદ્ધાંત}:
\begin{enumerate}
    \item પીઝોઇલેક્ટ્રિક અસર પર આધારિત.
    \item પીઝોઇલેક્ટ્રિક ક્રિસ્ટલ (ક્વાર્ટઝ, ટ ટુર્મેલિન) પર ઉચ્ચ-આવૃત્તિ AC વોલ્ટેજ લાગુ કરવામાં આવે છે.
    \item ક્રિસ્ટલ લાગુ કરેલ વોલ્ટેજની આવૃત્તિએ કંપન કરે છે.
    \item અનુનાદ પર (લાગુ આવૃત્તિ = કુદરતી આવૃત્તિ), મહત્તમ કંપવિસ્તારના સ્પંદનો થાય છે.
    \item અલ્ટ્રાસોનિક તરંગો ઉત્પન્ન થાય છે.
\end{enumerate}

\textbf{આવૃત્તિ શ્રેણી}: 20 kHz થી અનેક MHz.
\textbf{ફાયદાઓ}: ઉચ્ચ કાર્યક્ષમતા, ચોક્કસ નિયંત્રણ, કોમ્પેક્ટ.
\end{solutionbox}

\begin{mnemonicbox}
\mnemonic{"પીઝો તરંગો બનાવે જ્યારે વીજળીથી યોગ્ય રીતે પલ્સ કરવામાં આવે"}
\end{mnemonicbox}

\questionmarks{4(c)(ii)}{2}{ધ્વનિ તરંગના કોઈપણ બે ગુણધર્મો સમજાવો.}

\begin{solutionbox}
\begin{enumerate}
    \item \textbf{ધ્વનિનું પરાવર્તન (Reflection)}:
    \begin{itemize}
        \item અવરોધો પરથી પાછા ફરે છે
        \item નિયમનું પાલન કરે છે: આપાત કોણ = પરાવર્તન કોણ
        \item પડઘા (Echoes) બનાવે છે
    \end{itemize}
    \item \textbf{ધ્વનિનું વક્રીભવન (Refraction)}:
    \begin{itemize}
        \item જુદી જુદી ઝડપ ધરાવતા માધ્યમોમાંથી પસાર થતી વખતે વાંકું વળવું
        \item ધ્વનિ ફોકસિંગ અને રાત્રિના સમયે શ્રવણક્ષમતા સમજાવે છે
    \end{itemize}
\end{enumerate}

\begin{answerdiagram}{ધ્વનિ ગુણધર્મો}
\begin{tikzpicture}
    % Reflection
    \draw[thick] (0,0) -- (3,0);
    \draw[->, purple] (0.5, 1.5) -- (1.5, 0);
    \draw[->, purple] (1.5, 0) -- (2.5, 1.5);
    \node at (1.5, -0.3) {પરાવર્તન};
    
    % Refraction
    \draw[thick] (5,0.75) -- (8,0.75);
    \draw[->, blue] (5.5, 1.5) -- (6.5, 0.75);
    \draw[->, blue] (6.5, 0.75) -- (7.0, 0);
    \node at (6.5, -0.3) {વક્રીભવન};
\end{tikzpicture}
\end{answerdiagram}
\end{solutionbox}

\begin{mnemonicbox}
\mnemonic{"ધ્વનિ મુસાફરી દરમિયાન નોંધપાત્ર વક્રીભવન બતાવે છે"}
\end{mnemonicbox}

\orquestionmarks{4(a)}{3}{તરંગની તરંગલંબાઇ, કળા અને વેગ વ્યાખ્યાયિત કરો.}

\begin{solutionbox}
\textbf{તરંગલંબાઇ ($\lambda$)}: સમાન કળામાં રહેલા બે ક્રમિક બિંદુઓ વચ્ચેનું અંતર. એક પૂર્ણ દોલન દરમિયાન કાપેલું અંતર. $v = \lambda f$.

\textbf{કળા (Phase)}: ચોક્કસ બિંદુ અને સમયે દોલનની સ્થિતિ. $2\pi$ નો તફાવત ધરાવતા બિંદુઓ સમાન કળામાં હોય છે; $\pi$ નો તફાવત ધરાવતા વિપરીત કળામાં હોય છે.

\textbf{વેગ ($v$)}: જે દરે તરંગ પ્રસરણ પામે છે. $v = \lambda f$. માધ્યમ પર આધાર રાખે છે.

\begin{answerdiagram}{તરંગના ગુણધર્મો}
\begin{tikzpicture}
    \draw[->] (0,0) -- (6,0) node[right] {અંતર};
    \draw[thick, blue, domain=0:6, samples=100] plot (\x, {sin(100*\x)});
    
    \draw[<->] (0.9, 1.2) -- (4.5, 1.2) node[midway, above] {તરંગલંબાઇ $\lambda$};
    \draw[dashed] (0.9, 1) -- (0.9, 1.2);
    \draw[dashed] (4.5, 1) -- (4.5, 1.2);
    
    \draw[<->, red] (4.5, -0.5) -- (5.5, -0.5) node[midway, below] {વેગ $v$};
\end{tikzpicture}
\end{answerdiagram}
\end{solutionbox}

\begin{mnemonicbox}
\mnemonic{"તરંગલંબાઇ એક ચક્ર લપેટે, કળા સ્થાન દર્શાવે, વેગ પ્રસરણ ઝડપ આપે"}
\end{mnemonicbox}

\orquestionmarks{4(b)}{4}{તરંગોના સહાયક અને વિનાશક વ્યતિકરણ સમજાવો.}

\begin{solutionbox}
\textbf{સહાયક વ્યતિકરણ (Constructive Interference)}:
\begin{itemize}
    \item તરંગો સમાન કળામાં મળે છે (શૃંગ શૃંગ સાથે મળે)
    \item કળા તફાવત = $2n\pi$, પથ તફાવત = $n\lambda$
    \item \textbf{પરિણામ}: મોટો કંપવિસ્તાર (વ્યક્તિગતનો સરવાળો)
\end{itemize}

\textbf{વિનાશક વ્યતિકરણ (Destructive Interference)}:
\begin{itemize}
    \item તરંગો વિપરીત કળામાં મળે છે (શૃંગ ગર્ત સાથે મળે)
    \item કળા તફાવત = $(2n+1)\pi$, પથ તફાવત = $(n+1/2)\lambda$
    \item \textbf{પરિણામ}: નાનો કંપવિસ્તાર (વ્યક્તિગતનો તફાવત)
\end{itemize}

\begin{answerdiagram}{વ્યતિકરણના પ્રકારો}
\begin{tikzpicture}
    % Constructive
    \node at (1.5, 2.5) {સહાયક};
    \draw[blue, thick] (0,1.5) sin (0.5,2) cos (1,1.5) sin (1.5,1) cos (2,1.5);
    \draw[blue, thick, dashed] (0,1.5) sin (0.5,2) cos (1,1.5) sin (1.5,1) cos (2,1.5);
    \draw[->, thick] (1,1.2) -- (1,0.8);
    \draw[red, very thick] (0,0.5) sin (0.5,1.5) cos (1,0.5) sin (1.5,-0.5) cos (2,0.5);
    
    % Destructive
    \node at (5.5, 2.5) {વિનાશક};
    \draw[blue, thick] (4,1.5) sin (4.5,2) cos (5,1.5) sin (5.5,1) cos (6,1.5);
    \draw[blue, thick, dashed] (4,1.5) sin (4.5,1) cos (5,1.5) sin (5.5,2) cos (6,1.5);
    \draw[->, thick] (5,1.2) -- (5,0.8);
    \draw[red, very thick] (4,0.5) -- (6,0.5);
\end{tikzpicture}
\end{answerdiagram}
\end{solutionbox}

\begin{mnemonicbox}
\mnemonic{"સહાયક મોટા તરંગો બનાવે; વિનાશક તરંગની ઊંચાઈ ઘટાડે"}
\end{mnemonicbox}

\orquestionmarks{4(c)(i)}{5}{મેગ્નેટોસ્ટ્રિક્શન પદ્ધતિનો ઉપયોગ કરીને અલ્ટ્રાસોનિક તરંગ કેવી રીતે ઉત્પન્ન થાય છે?}

\begin{solutionbox}
\begin{answerdiagram}{મેગ્નેટોસ્ટ્રિક્શન ઓસિલેટર}
\begin{tikzpicture}[node distance=1.5cm]
    \node[gtu block] (osc) {ઓસિલેટર};
    \node[gtu block, right=of osc] (amp) {એમ્પ્લીફાયર};
    \node[gtu block, right=of amp] (coil) {ફેરોમેગ્નેટિક રોડ પર કોઇલ};
    \node[gtu output, right=of coil] (wave) {અલ્ટ્રાસોનિક તરંગો};
    
    \draw[gtu arrow] (osc) -- (amp);
    \draw[gtu arrow] (amp) -- (coil);
    \draw[gtu arrow] (coil) -- (wave);
\end{tikzpicture}
\end{answerdiagram}

\textbf{કાર્ય સિદ્ધાંત}:
\begin{enumerate}
    \item મેગ્નેટોસ્ટ્રિક્શન અસર પર આધારિત (ચુંબકીય ક્ષેત્રમાં પરિમાણીય ફેરફાર).
    \item ફેરોમેગ્નેટિક રોડ (Ni, Fe) પર ઓલ્ટરનેટીંગ ચુંબકીય ક્ષેત્ર લાગુ કરવામાં આવે છે.
    \item રોડ લાગુ કરેલ ક્ષેત્રની આવૃત્તિએ વિસ્તરે/સંકોચાય છે.
    \item કંપનો અલ્ટ્રાસોનિક તરંગો ઉત્પન્ન કરે છે.
\end{enumerate}

\textbf{આવૃત્તિ શ્રેણી}: 20 kHz થી 100 kHz.
\textbf{ફાયદાઓ}: ઉચ્ચ શક્તિ, મજબૂત.
\textbf{મર્યાદાઓ}: માત્ર ઓછી આવૃત્તિ, ગરમીની સમસ્યાઓ.
\end{solutionbox}

\begin{mnemonicbox}
\mnemonic{"ચુંબકીય સામગ્રી મામૂલી રીતે હલીને અલ્ટ્રાસોનિક તરંગો બનાવે છે"}
\end{mnemonicbox}

\orquestionmarks{4(c)(ii)}{2}{પ્રકાશ તરંગના કોઈપણ બે ગુણધર્મો સમજાવો.}

\begin{solutionbox}
\begin{enumerate}
    \item \textbf{પરાવર્તન (Reflection)}: સપાટી પરથી પાછા ફરવું. ખૂણો i = ખૂણો r. અરીસામાં વપરાય છે.
    \item \textbf{વક્રીભવન (Refraction)}: માધ્યમ બદલતી વખતે વાંકું વળવું. સ્નેલનો નિયમ: $n_1 \sin\theta_1 = n_2 \sin\theta_2$. લેન્સમાં વપરાય છે.
\end{enumerate}

\begin{answerdiagram}{પ્રકાશ ગુણધર્મો}
\begin{tikzpicture}
    % Reflection
    \draw[thick] (0,0) -- (2,0);
    \draw[->, purple] (0.2, 1) -- (1, 0);
    \draw[->, purple] (1, 0) -- (1.8, 1);
    \node at (1, -0.3) {પરાવર્તન};
    
    % Refraction
    \draw[thick] (4,0.5) -- (6,0.5);
    \draw[->, blue] (4.5, 1.2) -- (5, 0.5);
    \draw[->, blue] (5, 0.5) -- (5.3, -0.2);
    \node at (5, -0.5) {વક્રીભવન};
\end{tikzpicture}
\end{answerdiagram}
\end{solutionbox}

\begin{mnemonicbox}
\mnemonic{"પ્રકાશ અરીસામાંથી પરાવર્તિત અને માધ્યમ દ્વારા વક્રીભવન થવાનું પસંદ કરે છે"}
\end{mnemonicbox}

\questionmarks{5(a)}{3}{LASER ની લાક્ષણિકતાઓ લખો.}

\begin{solutionbox}
\begin{answertable}{LASER લાક્ષણિકતાઓ}
\begin{tabulary}{\linewidth}{|L|L|}
\hline
\textbf{લાક્ષણિકતા} & \textbf{વર્ણન} \\ \hline
એકવર્ણી (Monochromatic) & એક તરંગલંબાઇ (શુદ્ધ રંગ) \\ \hline
સંસક્ત (Coherent) & સમાન કળામાં તરંગો (ઉચ્ચ વ્યતિકરણ) \\ \hline
દિશાસૂચક (Directional) & લાંબા અંતર પર ન્યૂનતમ વિચલન \\ \hline
ઉચ્ચ તીવ્રતા (High Intensity) & કેન્દ્રિત ઊર્જા \\ \hline
\end{tabulary}
\end{answertable}

\begin{answerdiagram}{લેસર વિ સામાન્ય પ્રકાશ}
\begin{tikzpicture}
    % Ordinary
    \node at (0, 1) {સામાન્ય પ્રકાશ};
    \draw[->] (0,0) -- (1,0.5);
    \draw[->] (0,0) -- (1,-0.5);
    \draw[->] (0,0) -- (1,0.2);
    \draw[->] (0,0) -- (1,-0.2);
    
    % Laser
    \node at (4, 1) {LASER};
    \draw[->, thick, red] (4,0.2) -- (6,0.2);
    \draw[->, thick, red] (4,0) -- (6,0);
    \draw[->, thick, red] (4,-0.2) -- (6,-0.2);
\end{tikzpicture}
\end{answerdiagram}
\end{solutionbox}

\begin{mnemonicbox}
\mnemonic{"LASER પ્રકાશ: એકવર્ણી, સંસક્ત, દિશાસૂચક, તીવ્ર"}
\end{mnemonicbox}

\questionmarks{5(b)}{4}{એન્જીનિયરિંગ અને તબીબી ક્ષેત્રમાં LASER નું મહત્વ ચર્ચો.}

\begin{solutionbox}
\textbf{એન્જીનિયરિંગ}:
\begin{enumerate}
    \item \textbf{ઉત્પાદન}: કટીંગ, વેલ્ડીંગ, 3D પ્રિન્ટીંગ.
    \item \textbf{માપન}: LIDAR, ગોઠવણી.
    \item \textbf{સંદેશાવ્યવહાર}: ફાઇબર ઓપ્ટિક્સ, ફ્રી-સ્પેસ.
\end{enumerate}

\textbf{તબીબી}:
\begin{enumerate}
    \item \textbf{શસ્ત્રક્રિયા}: લોહી વગરનું કટીંગ, LASIK.
    \item \textbf{નિદાન}: ઇમેજિંગ, સ્પેક્ટ્રોસ્કોપી.
    \item \textbf{ઉપચાર}: કેન્સર સારવાર, પીડા વ્યવસ્થાપન.
    \item \textbf{દંતચિકિત્સા}: દાંત સફેદ કરવા.
\end{enumerate}
\end{solutionbox}

\begin{mnemonicbox}
\mnemonic{"LASER ઉત્પાદન વધારે, ચોક્કસ માપે, ડેટા વાતચીત કરે, દર્દીઓ મટાડે"}
\end{mnemonicbox}

\questionmarks{5(c)(i)}{5}{LASER ના ઉત્પાદન માટે વસ્તી વ્યુત્ક્રમણ અને મેટાસ્ટેબલ સ્થિતિનું મહત્વ શું છે?}

\begin{solutionbox}
\textbf{વસ્તી વ્યુત્ક્રમણ (Population Inversion)}:
\begin{itemize}
    \item સ્થિતિ જ્યાં ભૂમિ સ્થિતિ કરતાં વધુ અણુઓ ઉત્તેજિત સ્થિતિમાં હોય છે.
    \item ઉત્તેજિત ઉત્સર્જનને શોષણ પર પ્રભુત્વ મેળવવા માટે આવશ્યક છે.
    \item પ્રકાશ પ્રવર્ધન સક્ષમ કરે છે.
\end{itemize}

\textbf{મેટાસ્ટેબલ સ્થિતિ}:
\begin{itemize}
    \item લાંબા જીવનકાળ ધરાવતી ઉત્તેજિત સ્થિતિ ($10^{-3}$s વિ $10^{-8}$s).
    \item ઉત્તેજિત અણુઓના સંચયને મંજૂરી આપે છે.
    \item વસ્તી વ્યુત્ક્રમણ સ્થાપિત કરવા માટે જરૂરી છે.
\end{itemize}

\begin{answerdiagram}{ઊર્જા સ્તરો}
\begin{tikzpicture}
    % Levels
    \draw[thick] (0,0) -- (5,0) node[right] {$E_1$ (ભૂમિ)};
    \draw[thick] (0,2) -- (5,2) node[right] {$E_2$ (મેટાસ્ટેબલ)};
    \draw[thick] (0,3.5) -- (5,3.5) node[right] {$E_3$ (અસ્થિર)};
    
    % Pumping
    \draw[->, thick, blue] (0.5,0) -- (0.5,3.5) node[midway, left] {પંપ};
    
    % Fast Transition
    \draw[->, dashed] (1.5,3.5) -- (1.5,2) node[midway, right] {ઝડપી ક્ષય};
    
    % Laser
    \draw[->, thick, red, decorate, decoration={snake}] (3,2) -- (3,0) node[midway, right] {લેસર ઉત્સર્જન};
    
    % Atoms
    \foreach \x in {2.5, 3.0, 3.5} \node[circle, fill=red, inner sep=1.5pt] at (\x, 2.2) {};
    \foreach \x in {4.0} \node[circle, fill=blue, inner sep=1.5pt] at (\x, 0.2) {};
    \node at (3, 2.5) {વસ્તી વ્યુત્ક્રમણ};
\end{tikzpicture}
\end{answerdiagram}
\end{solutionbox}

\begin{mnemonicbox}
\mnemonic{"વસ્તી વ્યુત્ક્રમણ ઇલેક્ટ્રોન ઉચ્ચ રાખે; મેટાસ્ટેબલ સ્થિતિ લાંબી જાળવે"}
\end{mnemonicbox}

\questionmarks{5(c)(ii)}{2}{ક્રમબદ્ધ સૂચક ઓપ્ટિકલ ફાઇબર સમજાવો.}

\begin{solutionbox}
\textbf{ક્રમબદ્ધ સૂચક (GRIN) ફાઇબર}:
\begin{itemize}
    \item કોરનો રીફ્રેક્ટિવ ઇન્ડેક્સ કેન્દ્રથી પરિઘ સુધી પેરાબોલિક રીતે ઘટે છે ($n(r) = n_1(1 - \alpha r^2)$).
    \item પ્રકાશ વક્ર પથમાં મુસાફરી કરે છે.
    \item મોડલ વિક્ષેપણ ઘટાડે છે અને બેન્ડવિડ્થ વધારે છે.
\end{itemize}

\begin{answerdiagram}{ક્રમબદ્ધ સૂચક પ્રોફાઇલ}
\begin{tikzpicture}
    % Profile
    \draw[thick, ->] (0,-1.5) -- (0,1.5) node[above] {$n$};
    \draw[thick, ->] (0,0) -- (3,0) node[right] {$r$};
    \draw[thick, blue, domain=0:2] plot({2*exp(-\x*\x)}, \x);
    \draw[thick, blue, domain=0:2] plot({2*exp(-\x*\x)}, -\x);
    
    % Ray path
    \draw[fill=gray!10] (4,-1) rectangle (8,1);
    \draw[thick, red] (4,0) sin (5, 0.8) cos (6,0) sin (7,-0.8) cos (8,0);
    \node at (6, 1.2) {વક્ર કિરણ પથ};
\end{tikzpicture}
\end{answerdiagram}
\end{solutionbox}

\begin{mnemonicbox}
\mnemonic{"ક્રમબદ્ધ સૂચક વિક્ષેપણ સરળ બનાવીને પ્રસારણ ધીમે ધીમે સુધારે"}
\end{mnemonicbox}

\orquestionmarks{5(a)}{3}{પ્રકાશનું વક્રીભવન વ્યાખ્યાયિત કરો અને સ્નેલનો નિયમ લખો.}

\begin{solutionbox}
\textbf{વક્રીભવન}: ઝડપ ફેરફારને કારણે માધ્યમો વચ્ચે પસાર થતા પ્રકાશનું વાંકું વળવું.
\textbf{સ્નેલનો નિયમ}: ખૂણાના સાઇનનો ગુણોત્તર રીફ્રેક્ટિવ ઇન્ડેક્સના ગુણોત્તર જેટલો હોય છે.
\[ n_1 \sin \theta_1 = n_2 \sin \theta_2 \]

\begin{answerdiagram}{વક્રીભવન}
\begin{tikzpicture}
    \fill[blue!5] (0,-2) rectangle (4,0) node[midway] {$n_2$};
    \fill[white] (0,0) rectangle (4,2) node[midway] {$n_1$};
    \draw[thick] (0,0) -- (4,0);
    \draw[dashed] (2,-2) -- (2,2);
    
    \draw[->, red, thick] (0.5, 1.5) -- (2,0);
    \draw[->, red, thick] (2,0) -- (3, -1.8);
    
    \node at (1.5, 0.5) {$\theta_1$};
    \node at (2.5, -0.5) {$\theta_2$};
\end{tikzpicture}
\end{answerdiagram}
\end{solutionbox}

\begin{mnemonicbox}
\mnemonic{"સાઇન ગુણોત્તર સૂચક ગુણોત્તર જેટલો"}
\end{mnemonicbox}

\orquestionmarks{5(b)}{4}{એન્જીનિયરિંગ અને તબીબી ક્ષેત્રમાં ઓપ્ટિકલ ફાઇબરનું મહત્વ ચર્ચો.}

\begin{solutionbox}
\textbf{એન્જીનિયરિંગ}:
\begin{itemize}
    \item સંદેશાવ્યવહાર: હાઇ સ્પીડ ઇન્ટરનેટ, સુરક્ષિત ડેટા.
    \item સેન્સર્સ: દબાણ, તાપમાન મોનિટરિંગ.
    \item ઔદ્યોગિક: દૂરસ્થ નિરીક્ષણ.
\end{itemize}

\textbf{તબીબી}:
\begin{itemize}
    \item નિદાન: એન્ડોસ્કોપી.
    \item શસ્ત્રક્રિયા: લેસર ડિલિવરી, માઇક્રોસર્જરી.
    \item ઇમેજિંગ: OCT.
\end{itemize}
\end{solutionbox}

\begin{mnemonicbox}
\mnemonic{"ઓપ્ટિકલ ફાઇબર જોડે, સમજે, જુએ અને સારવાર કરે"}
\end{mnemonicbox}

\orquestionmarks{5(c)(i)}{5}{ઓપ્ટિકલ ફાઇબરના સંખ્યાત્મક છિદ્ર અને સ્વીકૃતિ ખૂણા માટે સૂત્ર મેળવો.}

\begin{solutionbox}
\textbf{સંખ્યાત્મક છિદ્ર (NA)}:
\begin{enumerate}
    \item કોર-ક્લેડિંગ ઇન્ટરફેસ પર, નિર્ણાયક ખૂણો $\theta_c$: $\sin \theta_c = n_2/n_1$.
    \item કોરમાં મહત્તમ ખૂણો: $90^{\circ} - \theta_c$.
    \item પ્રવેશ પર સ્નેલનો નિયમ લાગુ કરો (હવા $n_0=1$): $\sin \theta_a = n_1 \sin(90^{\circ} - \theta_c) = n_1 \cos \theta_c$.
    \item બદલો $\cos \theta_c = \sqrt{1 - \sin^2 \theta_c} = \sqrt{1 - (n_2/n_1)^2}$.
    \item $\sin \theta_a = n_1 \sqrt{1 - (n_2/n_1)^2} = \sqrt{n_1^2 - n_2^2}$.
\end{enumerate}

\textbf{સૂત્ર}: $\text{NA} = \sin \theta_a = \sqrt{n_1^2 - n_2^2}$

\begin{answerdiagram}{સંખ્યાત્મક છિદ્ર}
\begin{tikzpicture}
    \draw[thick] (0,0) -- (6,0);
    \draw[thick] (0,2) -- (6,2);
    \draw[fill=blue!5] (0,0) rectangle (6,2);
    \node at (5, 1) {$n_1$ (કોર)};
    \node at (5, 2.3) {$n_2$ (ક્લેડિંગ)};
    
    \draw[dashed] (-1, 1) -- (2, 1);
    \draw[->, red, thick] (-1, 0.2) -- (0, 1) node[midway, below] {કિરણ};
    \draw[->, red, thick] (0, 1) -- (2, 2);
    
    \draw (0.5, 1) arc (0:-38:0.5);
    \node at (0.8, 0.8) {$\theta_a$};
\end{tikzpicture}
\end{answerdiagram}
\end{solutionbox}

\begin{mnemonicbox}
\mnemonic{"NA સ્વીકૃતિ ખૂણો નોંધે; n-વર્ગ તફાવતનો વર્ગમૂળ મહત્તમ સાઇન બતાવે"}
\end{mnemonicbox}

\orquestionmarks{5(c)(ii)}{2}{પગલું સૂચક ઓપ્ટિકલ ફાઇબર સમજાવો.}

\begin{solutionbox}
\textbf{પગલું સૂચક ફાઇબર}:
\begin{itemize}
    \item નીચા સમાન ક્લેડિંગ ઇન્ડેક્સ $n_2$ થી ઘેરાયેલ સમાન કોર ઇન્ડેક્સ $n_1$.
    \item તીક્ષ્ણ "પગલું" સંક્રમણ.
    \item પ્રકારો: સિંગલ-મોડ (નાનો કોર), મલ્ટિ-મોડ (મોટો કોર).
    \item મર્યાદા: મોડલ વિક્ષેપણ.
\end{itemize}

\begin{answerdiagram}{પગલું સૂચક પ્રોફાઇલ}
\begin{tikzpicture}
    % Profile
    \draw[thick, ->] (0,0) -- (3,0) node[right] {$r$};
    \draw[thick, ->] (0,-1.5) -- (0,1.5) node[above] {$n$};
    \draw[thick, blue] (0,0.8) -- (1.5,0.8) -- (1.5,0.4) -- (2.5,0.4);
    \node at (0.5, 1) {$n_1$};
    \node at (2, 0.6) {$n_2$};
    
    % Ray
    \draw[fill=blue!5] (4,-1) rectangle (8,1);
    \draw[thick, red] (4,0) -- (5, 1) -- (7, -1) -- (8, 0);
    \node at (6, 0) {ઝિગઝેગ પથ};
\end{tikzpicture}
\end{answerdiagram}
\end{solutionbox}

\begin{mnemonicbox}
\mnemonic{"પગલું સૂચક સંપૂર્ણ સીમા સાથે બે અલગ સૂચકો બતાવે"}
\end{mnemonicbox}

\end{document}

