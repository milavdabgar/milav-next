\documentclass[10pt,a4paper]{article}

% content/resources/templates/preamble.tex
\usepackage[margin=0.6in]{geometry}
\author{Milav Dabgar}
\usepackage{amsmath,amssymb,amsthm}
\usepackage{booktabs}
\usepackage{multirow}
\usepackage{xcolor}
\usepackage{tcolorbox}
\tcbuselibrary{breakable,skins}
\usepackage[colorlinks=true,linkcolor=blue]{hyperref}
\usepackage{titlesec}
\usepackage{enumitem}
\usepackage{tikz}
\usepackage{pgfplots}
\usepackage{circuitikz}
\usepackage[version=4]{mhchem}
\usepackage{longtable}
\usepackage{array}
\usepackage{float}
\usepackage{caption}
\usepackage{listings}

\lstset{
  basicstyle=\small\ttfamily,
  breaklines=true,
  breakatwhitespace=false,
  postbreak=\mbox{\textcolor{red}{$\hookrightarrow$}\space},
  float=false,
  numbers=left,
  numberstyle=\tiny\color{gray},
  numbersep=10pt,
  xleftmargin=2em,
  keywordstyle=\color{blue},
  commentstyle=\color{green!60!black},
  stringstyle=\color{purple},
  backgroundcolor=\color{gray!5},
  showstringspaces=false,
  tabsize=2,
  captionpos=b,
  keepspaces=true,
  columns=flexible
}

\pgfplotsset{compat=1.18}
\usetikzlibrary{shapes,arrows,positioning,calc,patterns,decorations.pathmorphing,decorations.markings,arrows.meta}

% Color scheme
\definecolor{headcolor}{RGB}{0,102,204}
\definecolor{keycolor}{RGB}{220,20,60}
\definecolor{solutioncolor}{RGB}{34,139,34}
\definecolor{mnemoniccolor}{RGB}{148,0,211}
\definecolor{codecolor}{RGB}{0,0,100}

% Spacing
\setlength{\parskip}{3pt}
\setlist[itemize]{nosep}
\setlist[enumerate]{nosep}

% Title formatting
\titleformat{\section}{\Large\bfseries\color{headcolor}}{\thesection}{1em}{}
\titleformat{\subsection}{\large\bfseries\color{headcolor}}{\thesubsection}{1em}{}

% Pandoc tightlist compatibility
\providecommand{\tightlist}{%
  \setlength{\itemsep}{0pt}\setlength{\parskip}{0pt}}

% Pandoc longtable compatibility
\newcounter{none}
\def\thenone{}


% content/resources/templates/english-boxes.tex
% This file is currently empty - it exists to maintain consistency with the import structure.
% Add custom environments here if needed in the future.


\begin{document}

\begin{center}
{\Huge\bfseries\color{headcolor} Subject Name Solutions}\\[5pt]
{\LARGE 4300005 -- Winter 2024}\\[3pt]
{\large Semester 1 Study Material}\\[3pt]
{\normalsize\textit{Detailed Solutions and Explanations}}
\end{center}

\vspace{10pt}

\subsection*{Question 1(a) [3 marks]}\label{q1a}

\textbf{Define accuracy and precision.}

\begin{solutionbox}

\begin{itemize}
\tightlist
\item
  \textbf{Accuracy}: Closeness of a measured value to the true value
\item
  \textbf{Precision}: Consistency or repeatability of measurement values
\end{itemize}

\end{solutionbox}
\begin{mnemonicbox}
``Accuracy Aims at Truth, Precision Produces
Repeatability''

\end{mnemonicbox}
\subsection*{Question 1(b) [4 marks]}\label{q1b}

\textbf{Derive SI unit of work and Velocity using fundamental physical
units.}

\begin{solutionbox}


\vspace{-5pt}
\captionof{table}{Derivation of Work and Velocity Units}
\vspace{-10pt}
\begin{longtable}[]{@{}
  >{\raggedright\arraybackslash}p{(\linewidth - 6\tabcolsep) * \real{0.3273}}
  >{\raggedright\arraybackslash}p{(\linewidth - 6\tabcolsep) * \real{0.1636}}
  >{\raggedright\arraybackslash}p{(\linewidth - 6\tabcolsep) * \real{0.3455}}
  >{\raggedright\arraybackslash}p{(\linewidth - 6\tabcolsep) * \real{0.1636}}@{}}
\toprule\noalign{}
\begin{minipage}[b]{\linewidth}\raggedright
Physical Quantity
\end{minipage} & \begin{minipage}[b]{\linewidth}\raggedright
Formula
\end{minipage} & \begin{minipage}[b]{\linewidth}\raggedright
SI Unit Derivation
\end{minipage} & \begin{minipage}[b]{\linewidth}\raggedright
SI Unit
\end{minipage} \\
\midrule\noalign{}
\endhead
\bottomrule\noalign{}
\endlastfoot
Work (W) &

W = F \times d &

W = [Force] \times [Distance] = [kg·m/s^{2}]

\times [m] = [kg·m^{2}/s^{2}] & Joule (J) \\
Velocity (v) &

v = d/t &

v = [Distance]/[Time] = [m]/[s]

& m/s \\
\end{longtable}

\begin{itemize}
\tightlist
\item
  \textbf{Work}: When a force (kg·m/s^{2}) acts through a distance (m), we
  get kg·m^{2}/s^{2} = Joule
\item
  \textbf{Velocity}: When an object covers distance (m) in time (s), we
  get m/s
\end{itemize}

\end{solutionbox}
\begin{mnemonicbox}
``Work Forces Distance, Velocity Distances Time''

\end{mnemonicbox}
\subsection*{Question 1(c) [7 marks]}\label{q1c}

\textbf{What is Least Count of instrument. State equation of Least count
of Vernier calipers. Explain measurement by vernier calipers with neat
and clean diagram.}

\begin{solutionbox}

\textbf{Least Count}: Smallest measurement that can be directly measured
using a measuring instrument.

\textbf{Equation for Least Count of Vernier Caliper}: Least Count = 1
Main Scale Division - 1 Vernier Scale Division or Least Count = Value of
1 MSD / Number of VSD

\textbf{Diagram: Vernier Caliper}

\begin{verbatim}
      ┌────────┐
      │        │
 ┌────┘    ┌───┘
 │         │
 │   ┌─────┘
 │   │
─┼───┼───┬───┬───┬───┬───┬───┬───┬───┬───┬───┬
 0   1   2   3   4   5   6   7   8   9   10
     │   │   │   │   │   │   │   │   │
     └───┴───┴───┴───┴───┴───┴───┴───┴───┘ 
      0   5  10  15  20  25  30  35  40  45
      Vernier Scale
\end{verbatim}

\textbf{Measurement Process}:

\begin{itemize}
\item
  \textbf{Step 1}: Close the jaws of caliper around the object
\item
  \textbf{Step 2}: Note the main scale reading just before the zero of
  vernier scale
\item
  \textbf{Step 3}: Find which vernier division exactly coincides with a
  main scale division
\item
  \textbf{Step 4}: Add the vernier reading to the main scale reading:
  Total = MSR + (VC \times LC)
\item
  \textbf{Main Scale Reading (MSR)}: Value on main scale just before
  vernier zero
\item
  \textbf{Vernier Coincidence (VC)}: Division number where vernier line
  aligns with main scale line
\item
  \textbf{Least Count (LC)}: Usually 0.02 mm or 0.001 inch
\end{itemize}

\end{solutionbox}
\begin{mnemonicbox}
``Main plus Matched makes Measurement''

\end{mnemonicbox}
\subsection*{Question 1(c) OR [7
marks]}\label{q1c}

\textbf{What is Least Count of instrument. State equation of Least count
of micrometer screw. Explain the positive and negative error in
micrometer screw with neat and clean diagram.}

\begin{solutionbox}

\textbf{Least Count}: Smallest measurement that can be directly measured
using a measuring instrument.

\textbf{Equation for Least Count of Micrometer Screw}: Least Count =
Pitch of screw / Number of divisions on circular scale

\textbf{Diagram: Micrometer Screw Gauge}

\begin{verbatim}
     ┌─────────────────┐
     │                 │
     │    ┌───────┐    │
     │    │       │    │
     └────┤       ├────┘
          │       │
          └───────┘
          
    0  5  10 15 20 25
    ────────────────────
       │
       V
   ┌───────┐
   │0 5    │   Circular Scale
   └───────┘
\end{verbatim}

\textbf{Positive Error}: When zero of circular scale is above the
reference line. The measured reading will be more than the actual value.

\textbf{Negative Error}: When zero of circular scale is below the
reference line. The measured reading will be less than the actual value.

\textbf{Error Correction}:

\begin{itemize}
\tightlist
\item
  For positive error: Actual Reading = Observed Reading - Zero Error
\item
  For negative error: Actual Reading = Observed Reading + Zero Error
\end{itemize}

\end{solutionbox}
\begin{mnemonicbox}
``Positive Produces Plus, Negative Needs Addition''

\end{mnemonicbox}
\subsection*{Question 2(a) [3 marks]}\label{q2a}

\textbf{Write characteristics of electric lines of force.}

\begin{solutionbox}


\vspace{-5pt}
\captionof{table}{Characteristics of Electric Field Lines}
\vspace{-10pt}
\begin{longtable}[]{@{}
  >{\raggedright\arraybackslash}p{(\linewidth - 2\tabcolsep) * \real{0.5517}}
  >{\raggedright\arraybackslash}p{(\linewidth - 2\tabcolsep) * \real{0.4483}}@{}}
\toprule\noalign{}
\begin{minipage}[b]{\linewidth}\raggedright
Characteristic
\end{minipage} & \begin{minipage}[b]{\linewidth}\raggedright
Description
\end{minipage} \\
\midrule\noalign{}
\endhead
\bottomrule\noalign{}
\endlastfoot
Direction & Always from positive to negative charge \\
Shape & Straight lines for uniform fields, curved for non-uniform
fields \\
Density & Proportional to field strength \\
Path & Never intersect each other \\
Nature & Start from positive and end at negative charges \\
\end{longtable}

\end{solutionbox}
\begin{mnemonicbox}
``Direction, Density, Never Cross, Start-End''

\end{mnemonicbox}
\subsection*{Question 2(b) [4 marks]}\label{q2b}

\textbf{Calculate the equivalent capacitance for both series and
parallel connection of capacitors having capacitance of values 9 μF, 12
μF \& 15 μF.}

\begin{solutionbox}

\textbf{For Series Connection}: 1/Ceq = 1/C_{1} + 1/C_{2} + 1/C_{3} 1/Ceq = 1/9 +
1/12 + 1/15 1/Ceq = 5/36 + 3/36 + 2.4/36 = 10.4/36 Ceq = 36/10.4 = 3.46
μF

\textbf{For Parallel Connection}: Ceq = C_{1} + C_{2} + C_{3} Ceq = 9 + 12 + 15 =
36 μF

\end{solutionbox}
\begin{mnemonicbox}
``Series Sums Reciprocals, Parallel Puts Together''

\end{mnemonicbox}
\subsection*{Question 2(c) [7 marks]}\label{q2c}

\textbf{Explain coulombs inverse square law and derive its equation.
Calculate coulomb force between two electrons separated by 10 meter.
(e=1.66 x 10^{-}^{1}^{9} C,

K= 9 x 10^{9} Nm^{2} C^{-}^{2})}


\begin{solutionbox}

\textbf{Coulomb's Law}: The electrostatic force between two point
charges is directly proportional to the product of the charges and
inversely proportional to the square of the distance between them.

\textbf{Equation Derivation}: F ∝ q_{1}q_{2} F ∝ 1/r^{2} Combining: F ∝ q_{1}q_{2}/r^{2}
With constant: F = k(q_{1}q_{2}/r^{2})

Where

k = 1/(4πε_{0}) = 9 \times 10^{9} Nm^{2}/C^{2}


\textbf{Diagram: Coulomb's Law}

\begin{verbatim}
     q_{1        q_{2}}
     ●─────────●
     ────r────
     F_{1^{2}   _{2}_{1}}
\end{verbatim}

\textbf{Calculation}: F = k(q_{1}q_{2}/r^{2}) F = 9 \times 10^{9} \times [(1.66 \times 10^{-}^{1}^{9}) \times
(1.66 \times 10^{-}^{1}^{9})] / (10)^{2}

F = 9 \times 10^{9} \times 2.76 \times 10^{-}^{3}^{8} / 100

F = 9 \times 2.76

\times 10^{-}^{3}^{8}^{-}^{2} \times 10^{9} F = 2.48 \times 10^{-}^{3}^{1} N

\end{solutionbox}
\begin{mnemonicbox}
``Charges Multiply, Distance Squares, Force
Declines''

\end{mnemonicbox}
\subsection*{Question 2(a) OR [3
marks]}\label{q2a}

\textbf{Explain electric field and and derive its unit.}

\begin{solutionbox}

\textbf{Electric Field}: The region around a charge where another charge
experiences a force.

\textbf{Definition}: Electric field at a point is the force experienced
by a unit positive charge placed at that point.

E = F/q

\textbf{Unit Derivation}: E = F/q = [N]/[C] =
[kg·m/s^{2}]/[A·s] = [kg·m/(A·s^{3})] SI unit: N/C or V/m

\end{solutionbox}
\begin{mnemonicbox}
``Electric field Equals Force per Charge''

\end{mnemonicbox}
\subsection*{Question 2(b) OR [4
marks]}\label{q2b}

\textbf{Explain electric flux with neat figure and derive its unit.}

\begin{solutionbox}

\textbf{Electric Flux}: Measure of the electric field passing through a
given area.

\textbf{Equation}: ϕ_{e} = E·A·cosθ

Where:

\begin{itemize}
\tightlist
\item
  E is the electric field
\item
  A is the area
\item
  θ is the angle between E and the normal to the area
\end{itemize}

\textbf{Diagram: Electric Flux}

\begin{verbatim}
       ↑ n (normal)
       │
       │  θ
       │/
───────┼───── E (electric field)
       │
       │
    Surface Area A
\end{verbatim}

\textbf{Unit Derivation}: ϕ_{e} = E·A·cosθ =
[N/C]·[m^{2}]·[dimensionless] = [N·m^{2}/C] Since 1 N/C = 1
V/m, flux unit = V·m = N·m^{2}/C

SI unit: N·m^{2}/C or V·m

\end{solutionbox}
\begin{mnemonicbox}
``Flux Flows through Fields and Areas''

\end{mnemonicbox}
\subsection*{Question 2(c) OR [7
marks]}\label{q2c}

\textbf{Define capacitor and derive its unit. Give the formula of
parallel plate capacitor and explain each term. Calculate the
capacitance of a parallel plate capacitor having 20 cm x 20 cm square
plates separated by a distance of 1.0 mm.}

\begin{solutionbox}

\textbf{Capacitor}: A device that stores electric charge.

\textbf{Definition}: Capacitance is the ratio of charge stored to the
potential difference applied. C = Q/V

\textbf{Unit Derivation}: C = Q/V = [C]/[V] =
[A·s]/[J/C] = [A·s]/[N·m/C] = [A^{2}·s^{4}/(kg·m^{2})] =
Farad (F)

\textbf{Parallel Plate Capacitor Formula}: C = ε_{0}εᵣA/d

Where:

\begin{itemize}
\tightlist
\item
  C is the capacitance
\item
  ε_{0} is the permittivity of free space (8.85 \times 10^{-}^{1}^{2} F/m)
\item
  εᵣ is the relative permittivity of dielectric
\item
  A is the area of overlap of plates
\item
  d is the distance between plates
\end{itemize}

\textbf{Diagram: Parallel Plate Capacitor}

\begin{verbatim}
    ┌───────────────┐ ┐
    │ + + + + + + + │ │
    └───────────────┘ │ d
    ┌───────────────┐ │
    │ {- {-} {-} {-} {-} {-} {-} │ │}
    └───────────────┘ ┘
          Area A
\end{verbatim}

\textbf{Calculation}: A = 20 cm \times 20 cm = 0.2 m \times 0.2 m = 0.04 m^{2} d =
1.0 mm = 0.001 m εᵣ = 1 (air) ε_{0} = 8.85 \times 10^{-}^{1}^{2} F/m

C = ε_{0}εᵣA/d = 8.85 \times 10^{-}^{1}^{2} \times 1 \times 0.04/0.001 = 354 \times 10^{-}^{1}^{2}

F = 354 pF


\end{solutionbox}
\begin{mnemonicbox}
``Capacitance Collects Charge between Closer Plates''

\end{mnemonicbox}
\subsection*{Question 3(a) [3 marks]}\label{q3a}

\textbf{Explain heat conduction in solid with example.}

\begin{solutionbox}

\textbf{Heat Conduction}: Transfer of heat through a solid material
without the movement of the material itself.

\textbf{Process}: Heat energy transfers from high temperature region to
low temperature region through molecular vibrations.

\textbf{Diagram: Heat Conduction}

\begin{verbatim}
   Hot                Cold
    ↓                  ↓
┌────────────────────────┐
│ { │}
└────────────────────────┘
     Heat flow 
\end{verbatim}

\textbf{Example}: Metal spoon in hot tea gets heated up at the handle
end through conduction.

\end{solutionbox}
\begin{mnemonicbox}
``Hot Energizes, Atoms Transfer, Conducts Outward''

\end{mnemonicbox}
\subsection*{Question 3(b) [4 marks]}\label{q3b}

\textbf{A person has fever 102. What is the temperature scale here?
Convert the temperature in remaining two scales.}

\begin{solutionbox}

\textbf{Temperature Scale}: 102^\circF (Fahrenheit)

\textbf{Conversion Formulas}:

\begin{itemize}
\tightlist
\item
  ^\circC = (^\circF - 32) \times 5/9
\item
  K = ^\circC + 273.15
\end{itemize}

\textbf{Calculation}: ^\circC = (102 - 32) \times 5/9 = 70 \times 5/9 = 38.89^\circC K =
38.89 + 273.15 = 312.04 K


\vspace{-5pt}
\captionof{table}{Temperature Conversion}
\vspace{-10pt}
\begin{longtable}[]{@{}lll@{}}
\toprule\noalign{}
Fahrenheit & Celsius & Kelvin \\
\midrule\noalign{}
\endhead
\bottomrule\noalign{}
\endlastfoot
102^\circF & 38.89^\circC & 312.04 K \\
\end{longtable}

\end{solutionbox}
\begin{mnemonicbox}
``Fahrenheit First, Convert Celsius, Kelvin Comes
last''

\end{mnemonicbox}
\subsection*{Question 3(c) [7 marks]}\label{q3c}

\textbf{Explain the principle of platinum resistance thermometer and
list out its uses.}

\begin{solutionbox}

\textbf{Principle}: The electrical resistance of platinum changes
predictably and consistently with temperature, allowing for precise
temperature measurement.

\textbf{Working}: Based on the relationship R = R_{0}[1 + α(T - T_{0})],
where R is resistance at temperature T, R_{0} is resistance at reference
temperature T_{0}, and α is temperature coefficient of resistance.

\textbf{Diagram: Platinum Resistance Thermometer}

\begin{verbatim}
    ┌───────────────┐
    │   Indicator   │
    └───┬───────┬───┘
        │       │
        │       │
    ┌───┴───────┴───┐
    │   Wheatstone   │
    │     Bridge     │
    └───┬───────┬───┘
        │       │
        │       │
    ┌───┴───────┴───┐
    │   Platinum    │
    │   Resistance  │
    │     Coil      │
    └───────────────┘
\end{verbatim}

\textbf{Uses}:

\begin{itemize}
\tightlist
\item
  \textbf{Industrial process}: Temperature monitoring in manufacturing
\item
  \textbf{Scientific research}: Laboratory measurements requiring high
  precision
\item
  \textbf{Calibration}: Standard for calibrating other thermometers
\item
  \textbf{Medical applications}: Temperature monitoring in medical
  equipment
\end{itemize}

\end{solutionbox}
\begin{mnemonicbox}
``Platinum Provides Precise Proportional Resistance''

\end{mnemonicbox}
\subsection*{Question 3(a) OR [3
marks]}\label{q3a}

\textbf{Define specific heat and heat capacity. And write its units.}

\begin{solutionbox}

\textbf{Specific Heat}: Amount of heat energy required to raise the
temperature of 1 kg of substance by 1 K.

\textbf{Heat Capacity}: Amount of heat energy required to raise the
temperature of an entire object by 1 K.


\vspace{-5pt}
\captionof{table}{Heat Capacity Terms}
\vspace{-10pt}
\begin{longtable}[]{@{}lll@{}}
\toprule\noalign{}
Term & Formula & SI Unit \\
\midrule\noalign{}
\endhead
\bottomrule\noalign{}
\endlastfoot
Specific Heat (c) & Q = mc∆T & J/(kg·K) \\
Heat Capacity (C) & Q = C∆T & J/K \\
\end{longtable}

\end{solutionbox}
\begin{mnemonicbox}
``Specific for Substance, Capacity for Complete
Object''

\end{mnemonicbox}
\subsection*{Question 3(b) OR [4
marks]}\label{q3b}

\textbf{Explain heat convection in fluid with example.}

\begin{solutionbox}

\textbf{Heat Convection}: Transfer of heat through a fluid (liquid or
gas) by the movement of the fluid itself.

\textbf{Process}: Hot fluid expands, becomes less dense, rises; cooler
fluid descends, creating a continuous circulation pattern called
convection current.

\textbf{Diagram: Convection Current}

\begin{verbatim}
      ↑      ↑      ↑
    warm    warm   warm
      \^{      \^{}      \^{}}
      |      |      |
   ┌──────────────────┐
   │  heat source     │
   └──────────────────┘
   
       Cool fluid
       ↓      ↓      ↓
\end{verbatim}

\textbf{Example}: Boiling water in a pot - heated water rises to the top
while cooler water sinks to the bottom.

\end{solutionbox}
\begin{mnemonicbox}
``Heat Rises, Cool Descends, Currents Circulate''

\end{mnemonicbox}
\subsection*{Question 3(c) OR [7
marks]}\label{q3c}

\textbf{Define coefficient of thermal conductivity. Derive its equation
of coefficient of thermal conductivity for heat transfer in solids.}

\begin{solutionbox}

\textbf{Coefficient of Thermal Conductivity}: The amount of heat
transferred per unit time per unit area per unit temperature gradient.

\textbf{Definition}: The quantity of heat flowing per second through
unit area when temperature gradient is unity.

\textbf{Derivation}:

\begin{itemize}
\tightlist
\item
  Consider a rod with cross-sectional area A and length L
\item
  Temperature difference between ends is ∆T
\item
  Heat flow Q in time t
\end{itemize}

Heat current = Q/t Temperature gradient = ∆T/L Area = A

According to Fourier's law: Q/t = k·A·(∆T/L)

Rearranging: k = (Q·L)/(t·A·∆T)

Where k is the coefficient of thermal conductivity.

\textbf{Diagram: Thermal Conductivity}

\begin{verbatim}
   T_{1                 T_{2}}
    ↓                  ↓
┌────────────────────────┐
│                        │ Area A
└────────────────────────┘
    ───── L ─────
        Heat flow 
\end{verbatim}

\textbf{Unit}: W/(m·K)

\end{solutionbox}
\begin{mnemonicbox}
``Heat Quantity Transfers Along Length Divided by
Area and Temperature''

\end{mnemonicbox}
\subsection*{Question 4(a) [3 marks]}\label{q4a}

\textbf{Write the difference between transverse waves and longitudinal
waves.}

\begin{solutionbox}


\vspace{-5pt}
\captionof{table}{Transverse vs Longitudinal Waves}
\vspace{-10pt}
\begin{longtable}[]{@{}
  >{\raggedright\arraybackslash}p{(\linewidth - 4\tabcolsep) * \real{0.2128}}
  >{\raggedright\arraybackslash}p{(\linewidth - 4\tabcolsep) * \real{0.3830}}
  >{\raggedright\arraybackslash}p{(\linewidth - 4\tabcolsep) * \real{0.4043}}@{}}
\toprule\noalign{}
\begin{minipage}[b]{\linewidth}\raggedright
Property
\end{minipage} & \begin{minipage}[b]{\linewidth}\raggedright
Transverse Waves
\end{minipage} & \begin{minipage}[b]{\linewidth}\raggedright
Longitudinal Waves
\end{minipage} \\
\midrule\noalign{}
\endhead
\bottomrule\noalign{}
\endlastfoot
Particle motion & Perpendicular to wave direction & Parallel to wave
direction \\
Medium displacement & Crests and troughs & Compressions and
rarefactions \\
Examples & Light waves, water waves & Sound waves, seismic P-waves \\
Medium requirements & Can travel through solids & Can travel through
solids, liquids, gases \\
Polarization & Can be polarized & Cannot be polarized \\
\end{longtable}

\end{solutionbox}
\begin{mnemonicbox}
``Transverse Takes Perpendicular Path, Longitudinal
Likes Linear Lanes''

\end{mnemonicbox}
\subsection*{Question 4(b) [4 marks]}\label{q4b}

\textbf{Calculate the wavelength of a wave having velocity 350 m/s and
frequency 10 Hz.}

\begin{solutionbox}

\textbf{Wave Equation}: v = fλ

Where:

\begin{itemize}
\tightlist
\item
  v is wave velocity (350 m/s)
\item
  f is frequency (10 Hz)
\item
  λ is wavelength (to be calculated)
\end{itemize}

\textbf{Calculation}: λ = v/f = 350/10 = 35 m

\end{solutionbox}
\begin{mnemonicbox}
``Velocity Values frequency times wavelength''

\end{mnemonicbox}
\subsection*{Question 4(c) [7 marks]}\label{q4c}

\textbf{Define Ultrasonic waves and write its characteristics. Write its
four major applications of Ultrasonic wave.}

\begin{solutionbox}

\textbf{Ultrasonic Waves}: Sound waves with frequencies higher than the
upper audible limit of human hearing (above 20 kHz).

\textbf{Characteristics}:

\begin{itemize}
\tightlist
\item
  \textbf{High frequency}: Above 20 kHz
\item
  \textbf{Short wavelength}: Enables detection of small objects
\item
  \textbf{Directional}: Can be focused in a specific direction
\item
  \textbf{Non-ionizing}: Safe for biological tissues
\item
  \textbf{Penetration}: Can travel through various media
\end{itemize}

\textbf{Diagram: Ultrasonic Wave}

\begin{verbatim}
      Amplitude
        ↑
        │   /{      /      /}
        │  /  {    /      /  }
 ───────┼─/────{──/──────/────────── Time}
        │/      {/      /      }
        │
      Period { 50 μs (f  20 kHz)}
\end{verbatim}

\textbf{Applications}:

\begin{itemize}
\tightlist
\item
  \textbf{Medical}: Diagnostic imaging, therapeutic procedures
\item
  \textbf{Industrial}: Non-destructive testing, flaw detection
\item
  \textbf{Cleaning}: Ultrasonic cleaning baths for precision parts
\item
  \textbf{Distance measurement}: Sonar, parking sensors, level
  indicators
\end{itemize}

\end{solutionbox}
\begin{mnemonicbox}
``Ultrasonic Uses Sound to Sense, Scan, Sanitize''

\end{mnemonicbox}
\subsection*{Question 4(a) OR [3
marks]}\label{q4a}

\textbf{Explain the polarization of light with neat diagram.}

\begin{solutionbox}

\textbf{Polarization}: The process of restricting the vibrations of
light waves to a single plane.

\textbf{Types}:

\begin{itemize}
\tightlist
\item
  Linear polarization
\item
  Circular polarization
\item
  Elliptical polarization
\end{itemize}

\textbf{Diagram: Light Polarization}

\begin{verbatim}
 Unpolarized Light  Polarizer   Polarized Light
       ↓              ↓             ↓
 ⊥↕⊢⊣|↖↗↘↙       ┌─────┐        
 ⊥↕⊢⊣|↖↗↘↙     │/////│       
 ⊥↕⊢⊣|↖↗↘↙       └─────┘        
  Multiple         Allows only    Single plane
  vibration        one plane      vibration
   planes
\end{verbatim}

\end{solutionbox}
\begin{mnemonicbox}
``Polarizers Pick Particular Planes''

\end{mnemonicbox}
\subsection*{Question 4(b) OR [4
marks]}\label{q4b}

\textbf{If velocity of light in air is 3 x 10^{8} m/s and velocity of light
in water is 2.25 x 10^{8} m/s. Calculate reflective index of water.}

\begin{solutionbox}

\textbf{Refractive Index Formula}: n = c/v

Where:

\begin{itemize}
\tightlist
\item
  n is the refractive index
\item
  c is the speed of light in vacuum (or air)
\item
  v is the speed of light in medium
\end{itemize}

\textbf{Calculation}: n = 3 \times 10^{8} / 2.25 \times 10^{8} = 3/2.25 = 4/3 = 1.33

\end{solutionbox}
\begin{mnemonicbox}
``Slower Speeds Show higher index''

\end{mnemonicbox}
\subsection*{Question 4(c)(i) OR [4
marks]}\label{q4c}

\textbf{Define: velocity, wavelength and frequency of wave. And derive
the relationship between wave velocity, wavelength and frequency.}

\begin{solutionbox}

\textbf{Wave Velocity (v)}: The speed at which a wave travels through a
medium.

\textbf{Wavelength (λ)}: The distance between two consecutive similar
points on a wave.

\textbf{Frequency (f)}: Number of complete wave cycles passing a point
per unit time.

\textbf{Diagram: Wave Parameters}

\begin{verbatim}
Amplitude
    ↑
    │   /{      /      /}
    │  /  {    /      /  }
────┼─/────{──/──────/───── Distance}
    │/      {/      /      }
    │
    ↑        ↑              ↑
  Wavelength (λ)    Period (T)
\end{verbatim}

\textbf{Derivation}:

\begin{itemize}
\tightlist
\item
  In time T (period), the wave travels a distance of one wavelength λ
\item
  So, v = λ/T
\item
  Since f = 1/T (frequency is inverse of period)
\item
  Therefore, v = λf
\end{itemize}

\end{solutionbox}
\begin{mnemonicbox}
``Velocity Values frequency times wavelength''

\end{mnemonicbox}
\subsection*{Question 4(c)(ii) OR [3
marks]}\label{q4c}

\textbf{Write properties of light.}

\begin{solutionbox}


\vspace{-5pt}
\captionof{table}{Properties of Light}
\vspace{-10pt}
\begin{longtable}[]{@{}ll@{}}
\toprule\noalign{}
Property & Description \\
\midrule\noalign{}
\endhead
\bottomrule\noalign{}
\endlastfoot
Propagation & Travels in straight lines in homogeneous medium \\
Speed & 3 \times 10^{8} m/s in vacuum \\
Reflection & Bounces off surfaces following law of reflection \\
Refraction & Changes direction when passing between media \\
Dispersion & White light splits into component colors \\
Interference & Waves can superimpose to create patterns \\
Diffraction & Bends around obstacles and through small openings \\
Polarization & Can be restricted to vibrate in one plane \\
Dual nature & Exhibits both wave and particle properties \\
\end{longtable}

\end{solutionbox}
\begin{mnemonicbox}
``Light Reflects, Refracts, Disperses, Interferes,
Polarizes''

\end{mnemonicbox}
\subsection*{Question 5(a) [3 marks]}\label{q5a}

\textbf{Explain law of refraction of light for plane surface. And
explain Snell's law.}

\begin{solutionbox}

\textbf{Law of Refraction}: When light passes from one medium to
another, it changes direction at the boundary.

\textbf{Snell's Law}: The ratio of the sine of the angle of incidence to
the sine of the angle of refraction is constant for a given pair of
media.

n_{1}sin(θ_{1}) = n_{2}sin(θ_{2})

Where:

\begin{itemize}
\tightlist
\item
  n_{1} is the refractive index of first medium
\item
  n_{2} is the refractive index of second medium
\item
  θ_{1} is the angle of incidence
\item
  θ_{2} is the angle of refraction
\end{itemize}

\textbf{Diagram: Refraction}

\begin{verbatim}
           ┌─────────────
   Normal  │
      ↑    │    Medium 1 (n_{1)}
      │    │
      │    │    Incident ray
      │   /│
      │  / │
      │ /  │
      │/θ_{1 │}
      ├────┼────────────────
      │{θ_{2} │}
      │ {  │}
      │  { │}
      │   {│    Medium 2 (n_{2})}
           │    Refracted ray
           │
           │
           └─────────────
\end{verbatim}

\end{solutionbox}
\begin{mnemonicbox}
``Sines Show Speeds in Separate Substances''

\end{mnemonicbox}
\subsection*{Question 5(b) [4 marks]}\label{q5b}

\textbf{A step index fiber has core refractive index of 1.30 and
relative refractive index difference is Δ=0.02. Find numerical
aperture.}

\begin{solutionbox}

\textbf{Numerical Aperture Formula}: NA = \sqrt(n_{1}^{2} - n_{2}^{2})

For step index fiber: NA = n_{1}\sqrt(2Δ)

Where:

\begin{itemize}
\tightlist
\item
  n_{1} is the core refractive index
\item
  Δ is the relative refractive index difference
\end{itemize}

\textbf{Calculation}: NA = 1.30 \times \sqrt(2 \times 0.02) NA = 1.30 \times \sqrt0.04 NA =
1.30 \times 0.2 NA = 0.26

\end{solutionbox}
\begin{mnemonicbox}
``Numerical Aperture Needs core And Delta''

\end{mnemonicbox}
\subsection*{Question 5(c) [7 marks]}\label{q5c}

\textbf{Explain Total internal reflection of light. And derive the
equation of critical angle.}

\begin{solutionbox}

\textbf{Total Internal Reflection (TIR)}: The complete reflection of
light at the boundary between two media when light travels from a denser
medium to a rarer medium at an angle greater than the critical angle.

\textbf{Conditions for TIR}:

\begin{enumerate}
\tightlist
\item
  Light must travel from denser to rarer medium
\item
  Angle of incidence must exceed critical angle
\end{enumerate}

\textbf{Critical Angle}: The angle of incidence in the denser medium for
which the angle of refraction in the rarer medium is 90^\circ.

\textbf{Derivation}: Using Snell's law: n_{1}sin(θ_{1}) = n_{2}sin(θ_{2})

At critical angle (θc):

\begin{itemize}
\tightlist
\item
  θ_{1} = θc
\item
  θ_{2} = 90^\circ
\item
  sin(90^\circ) = 1
\end{itemize}

Therefore: n_{1}sin(θc) = n_{2}sin(90^\circ) = n_{2} \times 1 = n_{2}

Rearranging: sin(θc) = n_{2}/n_{1}

\textbf{Diagram: Total Internal Reflection}

\begin{verbatim}
       Medium 1 (n_{1)}
       (Denser)
       ┌─────────────────
       │  {      /}
       │   {θc  /}
       │    {  /}
       │     {/}
       │     /{}
       │    /  {}
       │   /    {}
       │  /      {}
       └─────────────────
       Medium 2 (n_{2)}
       (Rarer)
\end{verbatim}

\end{solutionbox}
\begin{mnemonicbox}
``Critical Comes when Dense to Rare with Sine at
Ratio''

\end{mnemonicbox}
\subsection*{Question 5(a) OR [3
marks]}\label{q5a}

\textbf{Explain numerical aperture and acceptance angle for fiber optic
cable.}

\begin{solutionbox}

\textbf{Numerical Aperture (NA)}: Measure of the light-gathering ability
of an optical fiber.

\textbf{Acceptance Angle (θ_{a})}: Maximum angle at which light can enter
the fiber and still experience total internal reflection.

\textbf{Relationship}: NA = sin(θ_{a})

\textbf{Diagram: Numerical Aperture and Acceptance Angle}

\begin{verbatim}
                 θ_{a}
                /│{}
      Cladding /a│ {    Cladding}
      ────────┼──┼──┼────────
              │  │  │
      Core    │  │  │    Core
      ────────┼──┼──┼────────
              │  │  │
      Cladding│  │  │    Cladding
      ────────┴──┴──┴────────
\end{verbatim}

\end{solutionbox}
\begin{mnemonicbox}
``Acceptance Angle Allows light, Numerical Aperture
Names its Sine''

\end{mnemonicbox}
\subsection*{Question 5(b) OR [4
marks]}\label{q5b}

\textbf{Write full form LASER. Write its characteristics.}

\begin{solutionbox}

\textbf{LASER}: Light Amplification by Stimulated Emission of Radiation


\vspace{-5pt}
\captionof{table}{Characteristics of LASER}
\vspace{-10pt}
\begin{longtable}[]{@{}
  >{\raggedright\arraybackslash}p{(\linewidth - 2\tabcolsep) * \real{0.5517}}
  >{\raggedright\arraybackslash}p{(\linewidth - 2\tabcolsep) * \real{0.4483}}@{}}
\toprule\noalign{}
\begin{minipage}[b]{\linewidth}\raggedright
Characteristic
\end{minipage} & \begin{minipage}[b]{\linewidth}\raggedright
Description
\end{minipage} \\
\midrule\noalign{}
\endhead
\bottomrule\noalign{}
\endlastfoot
Monochromatic & Single wavelength or color \\
Coherent & All waves in same phase \\
Highly directional & Travels in straight line with minimal divergence \\
High intensity & Concentrated energy in narrow beam \\
Collimated & Parallel rays with minimal spreading \\
\end{longtable}

\end{solutionbox}
\begin{mnemonicbox}
``LASER Light: Mono, Coherent, Direct, Intense''

\end{mnemonicbox}
\subsection*{Question 5(c) OR [7
marks]}\label{q5c}

\textbf{Explain the construction of optical fiber cable in details. And
Explain Step index and Graded index optical fiber.}

\begin{solutionbox}

\textbf{Optical Fiber Construction}:

\begin{enumerate}
\tightlist
\item
  \textbf{Core}: Central light-transmitting portion (glass or plastic)
\item
  \textbf{Cladding}: Surrounds core, with lower refractive index than
  core
\item
  \textbf{Buffer Coating}: Protective plastic coating
\item
  \textbf{Jacket}: Outer protective covering
\end{enumerate}

\textbf{Diagram: Optical Fiber Structure}

\begin{verbatim}
      ┌───────────────┐
      │               │  Jacket
      │  ┌─────────┐  │
      │  │         │  │  Buffer Coating
      │  │  ┌───┐  │  │
      │  │  │   │  │  │
      │  │  │   │  │  │
      │  │  └───┘  │  │
      │  │    ↑    │  │
      │  └────┼────┘  │
      │       │       │
      └───────┼───────┘
              ↑
             Core
            Cladding
\end{verbatim}

\textbf{Step Index Fiber}:

\begin{itemize}
\tightlist
\item
  Abrupt change in refractive index between core and cladding
\item
  Light travels in zigzag path by total internal reflection
\item
  Higher modal dispersion (signal spreading)
\item
  Simpler construction
\end{itemize}

\textbf{Graded Index Fiber}:

\begin{itemize}
\tightlist
\item
  Gradual change in refractive index from center of core to cladding
\item
  Light travels in helical path due to continuous refraction
\item
  Lower modal dispersion
\item
  More complex construction
\end{itemize}

\textbf{Diagram: Step Index vs Graded Index Fiber}

\begin{verbatim}
Step Index:
       ─────────────────────
      /                      {}
     /    ┌────────────┐     {}
    /     │            │      {}
   |      │    Core    │      |
    {     │            │     /}
     {    └────────────┘    /}
      {      Cladding      /}
       ────────────────────
       
Graded Index:
       ─────────────────────
      /                      {}
     /     ┌──────────┐      {}
    /     /            {      }
   |     |     Core     |      |
    {                 /      /}
     {     └──────────┘      /}
      {      Cladding       /}
       ────────────────────
\end{verbatim}

\textbf{Refractive Index Profile}:

\begin{verbatim}
Step Index:           Graded Index:
    │                     │
n_{1 ─┤▄▄▄▄▄▄▄              ▄▄▄▄▄}
    │       │            ▄     ▄
    │       │           ▄       ▄
n_{2 ─┤       ▀▀▀▀▀▀▀    ▄         ▄}
    │                  ▀▀▀▀▀▀▀▀▀▀▀
    └─────── r        └─────── r
\end{verbatim}

\end{solutionbox}
\begin{mnemonicbox}
``Step Shows Sharp Shift, Graded Gradually Goes
down''

\end{mnemonicbox}

\end{document}
