\documentclass[10pt,a4paper]{article}

% content/resources/templates/preamble.tex
\usepackage[margin=0.6in]{geometry}
\author{Milav Dabgar}
\usepackage{amsmath,amssymb,amsthm}
\usepackage{booktabs}
\usepackage{multirow}
\usepackage{xcolor}
\usepackage{tcolorbox}
\tcbuselibrary{breakable,skins}
\usepackage[colorlinks=true,linkcolor=blue]{hyperref}
\usepackage{titlesec}
\usepackage{enumitem}
\usepackage{tikz}
\usepackage{pgfplots}
\usepackage{circuitikz}
\usepackage[version=4]{mhchem}
\usepackage{longtable}
\usepackage{array}
\usepackage{float}
\usepackage{caption}
\usepackage{listings}

\lstset{
  basicstyle=\small\ttfamily,
  breaklines=true,
  breakatwhitespace=false,
  postbreak=\mbox{\textcolor{red}{$\hookrightarrow$}\space},
  float=false,
  numbers=left,
  numberstyle=\tiny\color{gray},
  numbersep=10pt,
  xleftmargin=2em,
  keywordstyle=\color{blue},
  commentstyle=\color{green!60!black},
  stringstyle=\color{purple},
  backgroundcolor=\color{gray!5},
  showstringspaces=false,
  tabsize=2,
  captionpos=b,
  keepspaces=true,
  columns=flexible
}

\pgfplotsset{compat=1.18}
\usetikzlibrary{shapes,arrows,positioning,calc,patterns,decorations.pathmorphing,decorations.markings,arrows.meta}

% Color scheme
\definecolor{headcolor}{RGB}{0,102,204}
\definecolor{keycolor}{RGB}{220,20,60}
\definecolor{solutioncolor}{RGB}{34,139,34}
\definecolor{mnemoniccolor}{RGB}{148,0,211}
\definecolor{codecolor}{RGB}{0,0,100}

% Spacing
\setlength{\parskip}{3pt}
\setlist[itemize]{nosep}
\setlist[enumerate]{nosep}

% Title formatting
\titleformat{\section}{\Large\bfseries\color{headcolor}}{\thesection}{1em}{}
\titleformat{\subsection}{\large\bfseries\color{headcolor}}{\thesubsection}{1em}{}

% Pandoc tightlist compatibility
\providecommand{\tightlist}{%
  \setlength{\itemsep}{0pt}\setlength{\parskip}{0pt}}

% Pandoc longtable compatibility
\newcounter{none}
\def\thenone{}


% content/resources/templates/gujarati-boxes.tex
\usepackage{fontspec}
\usepackage{polyglossia}

% Set Gujarati as main language (document is primarily in Gujarati)
% Note: gloss-gujarati.ldf doesn't exist in polyglossia, but it will use hyphenation patterns
\setdefaultlanguage{gujarati}
\setotherlanguage{english}

% Configure Gujarati font properly
% Use Language=Default to prevent polyglossia from trying to add language-specific features
% that don't exist for Gujarati, which causes "empty feature" warnings
\newfontfamily\gujaratifont[Script=Gujarati,AutoFakeBold=2.5,AutoFakeSlant=0.3]{Noto Sans Gujarati}
\setmainfont[Script=Gujarati,AutoFakeBold=2.5,AutoFakeSlant=0.3]{Noto Sans Gujarati}
% Use Noto Sans Gujarati for monospace to support Gujarati in text
\setmonofont[Scale=0.9]{Noto Sans Gujarati}

% Configure English to use the same font
\newfontfamily\englishfont[Script=Gujarati,AutoFakeBold=2.5,AutoFakeSlant=0.3]{Noto Sans Gujarati}

% Translations for polyglossia
\gappto\captionsgujarati{
  \renewcommand{\tablename}{કોષ્ટક}
  \renewcommand{\figurename}{આકૃતિ}
}

% Helper for TikZ nodes to ensure Gujarati font
\newcommand{\gu}[1]{{\gujaratifont #1}}

% Custom environments
\newtcolorbox{solutionbox}{
    breakable,
    enhanced,
    colback=solutioncolor!5!white,
    colframe=solutioncolor!75!black,
    fonttitle=\bfseries,
    title=જવાબ
}

\newtcolorbox{solutionboxnobreak}{
 colback=solutioncolor!5!white,
 colframe=solutioncolor!75!black,
 fonttitle=\bfseries,
 title=જવાબ
}

\newtcolorbox{keyformula}{
 breakable,
 enhanced,
 colback=keycolor!5!white,
 colframe=keycolor!75!black,
 fonttitle=\bfseries,
 title=રાસાયણિક સમીકરણ/સૂત્ર
}

\newtcolorbox{mnemonicbox}{
 breakable,
 enhanced,
 colback=mnemoniccolor!5!white,
 colframe=mnemoniccolor!75!black,
 fonttitle=\bfseries,
 title=મેમરી ટ્રીક
}


\begin{document}

\begin{center}
{\Huge\bfseries\color{headcolor} Subject Name (Gujarati)}\\[5pt]
{\LARGE 4300005 -- Summer 2023}\\[3pt]
{\large Semester 1 Study Material}\\[3pt]
{\normalsize\textit{Detailed Solutions and Explanations}}
\end{center}

\vspace{10pt}

\subsection*{પ્રશ્ન 1(અ) {[}3
ગુણ{]}}\label{uxaaauxab0uxab6uxaa8-1uxa85-3-uxa97uxaa3}

\textbf{SI માં બેઝ યુનિટ તેમના સિમ્બોલ સાથે લખો.}

\begin{solutionbox}

{\def\LTcaptype{none} % do not increment counter
\begin{longtable}[]{@{}lll@{}}
\toprule\noalign{}
ભૌતિક રાશિ & એકમ & સિમ્બોલ \\
\midrule\noalign{}
\endhead
\bottomrule\noalign{}
\endlastfoot
લંબાઈ & મીટર & m \\
દ્રવ્યમાન & કિલોગ્રામ & kg \\
સમય & સેકન્ડ & s \\
વિદ્યુત પ્રવાહ & એમ્પિયર & A \\
તાપમાન & કેલ્વિન & K \\
પદાર્થનું પ્રમાણ & મોલ & mol \\
પ્રકાશ તીવ્રતા & કેન્ડેલા & cd \\
\end{longtable}
}

\end{solutionbox}
\begin{mnemonicbox}
``લાંબુ માપ તાપમાન અશક્તિ પ્રકાશે કેવી માનવતા''

\end{mnemonicbox}
\subsection*{પ્રશ્ન 1(બ) {[}4
ગુણ{]}}\label{uxaaauxab0uxab6uxaa8-1uxaac-4-uxa97uxaa3}

\textbf{વર્નિયર કેલિપરની રચના અને કાર્ય સમજાવો. તેની લઘુત્તમ માપ શક્તિ અને શૂન્ય
ત્રુટી સમજાવો.}

\begin{solutionbox}

\textbf{વર્નિયર કેલિપરની રચના:}

\begin{center}
\textbf{Mermaid Diagram (Code)}
\begin{verbatim}
{Shaded}
{Highlighting}[]
graph LR
    A[મુખ્ય સ્કેલ] {-{-}{-} B[ફિક્સ્ડ જૉ]}
    A {-{-}{-} C[વર્નિયર સ્કેલ]}
    C {-{-}{-} D[મૂવેબલ જૉ]}
    C {-{-}{-} E[ઊંડાઈ માપક રોડ]}
    A {-{-}{-} F[લોકિંગ સ્ક્રૂ]}
{Highlighting}
{Shaded}
\end{verbatim}
\end{center}

\begin{itemize}
\tightlist
\item
  \textbf{મુખ્ય સ્કેલ}: મિલિમીટરમાં અંકિત ફિક્સ થયેલો સ્કેલ
\item
  \textbf{વર્નિયર સ્કેલ}: મુખ્ય સ્કેલ કરતાં થોડા નાના વિભાગો ધરાવતો સરકી શકે તેવો
  સ્કેલ
\item
  \textbf{ફિક્સ્ડ જૉ}: મુખ્ય સ્કેલ સાથે જોડાયેલો
\item
  \textbf{મૂવેબલ જૉ}: વર્નિયર સ્કેલ સાથે જોડાયેલો
\item
  \textbf{ઊંડાઈ માપક રોડ}: ઊંડાઈ માપવા માટે
\item
  \textbf{લોકિંગ સ્ક્રૂ}: માપન વખતે સ્થિતિ ફિક્સ કરવા માટે
\end{itemize}

\textbf{કાર્ય}: વસ્તુને બે જૉ વચ્ચે મૂકવામાં આવે છે, મૂવેબલ જૉને વસ્તુને સારી રીતે પકડવા
માટે એડજસ્ટ કરવામાં આવે છે. મુખ્ય સ્કેલ વાંચન અને વર્નિયર સ્કેલના સંપાતી મૂલ્યને ઉમેરીને માપ
નોંધવામાં આવે છે.

\textbf{લઘુત્તમ માપ શક્તિ}: વર્નિયર કેલિપર દ્વારા માપી શકાતું સૌથી નાનું માપ. LC =
મુખ્ય સ્કેલ પર 1 વિભાગ ÷ વર્નિયર સ્કેલ પર વિભાગોની સંખ્યા

\textbf{શૂન્ય ત્રુટી}: જ્યારે જૉ બંધ હોય ત્યારે કેલિપર શૂન્ય સિવાયનું વાંચન બતાવે તે
ત્રુટી.

\begin{itemize}
\tightlist
\item
  \textbf{ધન ત્રુટી}: વાંચનમાંથી બાદ કરવી
\item
  \textbf{ઋણ ત્રુટી}: વાંચનમાં ઉમેરવી
\end{itemize}

\end{solutionbox}
\begin{mnemonicbox}
``વર્નિયર ચોક્કસ માપ લેતા સમયે ત્રુટીઓ ટાળે''

\end{mnemonicbox}
\subsection*{પ્રશ્ન 1(ક)(i) {[}4
ગુણ{]}}\label{uxaaauxab0uxab6uxaa8-1uxa95i-4-uxa97uxaa3}

\textbf{ચોકસાઈ અને સચોટતા વચ્ચેનો તફાવત લખો.}

\begin{solutionbox}

{\def\LTcaptype{none} % do not increment counter
\begin{longtable}[]{@{}
  >{\raggedright\arraybackslash}p{(\linewidth - 2\tabcolsep) * \real{0.4706}}
  >{\raggedright\arraybackslash}p{(\linewidth - 2\tabcolsep) * \real{0.5294}}@{}}
\toprule\noalign{}
\begin{minipage}[b]{\linewidth}\raggedright
ચોકસાઈ
\end{minipage} & \begin{minipage}[b]{\linewidth}\raggedright
સચોટતા
\end{minipage} \\
\midrule\noalign{}
\endhead
\bottomrule\noalign{}
\endlastfoot
માપનું સાચા મૂલ્યની નજીકતા & માપની પુનરાવર્તનીયતા \\
પદ્ધતિગત ત્રુટીઓથી પ્રભાવિત & અનિયમિત ત્રુટીઓથી પ્રભાવિત \\
માપનના સરેરાશ દ્વારા દર્શાવાય છે & માપના પ્રમાણિત વિચલન દ્વારા દર્શાવાય છે \\
કેલિબ્રેશન દ્વારા સુધારી શકાય & વધુ સારા ઉપકરણો વાપરીને સુધારી શકાય \\
ઉદાહરણ: જો સાચું મૂલ્ય 10 cm હોય, તો 9.9, 10.1, અને 10.0 cm ના માપ ચોક્કસ છે &
ઉદાહરણ: 9.8, 9.8, 9.8 cm ના માપ સચોટ છે પણ સાચું મૂલ્ય 10 cm હોય તો ચોક્કસ
નથી \\
\end{longtable}
}

\end{solutionbox}
\begin{mnemonicbox}
``ચોક્સાઈ ચોક્કસ સાચા મૂલ્યે, સચોટતા સરખાં સમાન વાંચને''

\end{mnemonicbox}
\subsection*{પ્રશ્ન 1(ક)(ii) {[}2
ગુણ{]}}\label{uxaaauxab0uxab6uxaa8-1uxa95ii-2-uxa97uxaa3}

\textbf{માઇક્રોમીટર સ્ક્રૂ ગેજની પિચ 0.5 mm છે અને તેના વર્તુળાકાર સ્કેલ પર 50
વિભાગો છે. તેની લઘુત્તમ માપ શક્તિ શોધો.}

\begin{solutionbox}

\textbf{સૂત્ર}: લઘુત્તમ માપ શક્તિ = પિચ ÷ વર્તુળાકાર સ્કેલ પર વિભાગોની સંખ્યા

\textbf{ગણતરી}: LC = 0.5 mm ÷ 50 = 0.01 mm

\textbf{માઇક્રોમીટર સ્ક્રૂ ગેજની લઘુત્તમ માપ શક્તિ = 0.01 mm}

\end{solutionbox}
\subsection*{પ્રશ્ન 1(ક)(iii) {[}1
ગુણ{]}}\label{uxaaauxab0uxab6uxaa8-1uxa95iii-1-uxa97uxaa3}

\textbf{ઉષ્માનું SI એકમ શું છે?}

\begin{solutionbox}

ઉષ્માનું SI એકમ \textbf{જૂલ (J)} છે

\end{solutionbox}
\subsection*{પ્રશ્ન 1(ક)(i) {[}4 ગુણ{]}
(OR)}\label{uxaaauxab0uxab6uxaa8-1uxa95i-4-uxa97uxaa3-or}

\textbf{નિરપેક્ષ અને સાપેક્ષ ત્રુટીઓની ગણતરી કેવી રીતે કરવામાં આવે છે?}

\begin{solutionbox}

\textbf{નિરપેક્ષ ત્રુટિ (Δa)}: માપેલા મૂલ્ય અને સાચા મૂલ્ય વચ્ચેનો તફાવત

\begin{itemize}
\tightlist
\item
  ઘણા માપો માટે, તે માપેલા મૂલ્ય અને સરેરાશ મૂલ્ય વચ્ચેનો તફાવત છે
\end{itemize}

\textbf{નિરપેક્ષ ત્રુટિની ગણતરી}:

\begin{itemize}
\tightlist
\item
  \textbf{એક માપ માટે}: Δa = \textbar માપેલું મૂલ્ય - સાચું મૂલ્ય\textbar{}
\item
  \textbf{ઘણા માપો માટે}:

  \begin{enumerate}
  \def\labelenumi{\arabic{enumi}.}
  \tightlist
  \item
    સરેરાશ ગણો (am)
  \item
    દરેક માપ માટે: Δai = \textbar ai - am\textbar{}
  \item
    સરેરાશ નિરપેક્ષ ત્રુટિ: Δa = (Δa1 + Δa2 + \ldots{} + Δan) ÷ n
  \end{enumerate}
\end{itemize}

\textbf{સાપેક્ષ ત્રુટિ (εr)}: નિરપેક્ષ ત્રુટિનો સાચા મૂલ્ય સાથેનો ગુણોત્તર

\begin{itemize}
\tightlist
\item
  εr = નિરપેક્ષ ત્રુટિ ÷ સાચું મૂલ્ય = Δa ÷ સાચું મૂલ્ય
\end{itemize}

\textbf{ટકાવારી ત્રુટિ (εp)}: ટકાવારીમાં વ્યક્ત થયેલી સાપેક્ષ ત્રુટિ

\begin{itemize}
\tightlist
\item
  εp = સાપેક્ષ ત્રુટિ × 100 = (Δa ÷ સાચું મૂલ્ય) × 100\%
\end{itemize}

\end{solutionbox}
\begin{mnemonicbox}
``નિરપેક્ષ નિશ્ચિત મૂલ્યની ગણતરી, સાપેક્ષ સાચા સંદર્ભે સંબંધિત''

\end{mnemonicbox}
\subsection*{પ્રશ્ન 1(ક)(ii) {[}2 ગુણ{]}
(OR)}\label{uxaaauxab0uxab6uxaa8-1uxa95ii-2-uxa97uxaa3-or}

\textbf{વર્નિયર કેલિપરનો મુખ્ય સ્કેલ mm માં અંકિત કરવામાં આવેલ છે અને તેના વર્નિયર સ્કેલ
પર 50 વિભાગો છે. તેની લઘુત્તમ માપ શક્તિ શોધો.}

\begin{solutionbox}

\textbf{સૂત્ર}: લઘુત્તમ માપ શક્તિ = મુખ્ય સ્કેલ પર 1 વિભાગ ÷ વર્નિયર સ્કેલ પર
વિભાગોની સંખ્યા

\textbf{ગણતરી}: મુખ્ય સ્કેલ પર 1 વિભાગ = 1 mm LC = 1 mm ÷ 50 = 0.02 mm

\textbf{વર્નિયર કેલિપરની લઘુત્તમ માપ શક્તિ = 0.02 mm}

\end{solutionbox}
\subsection*{પ્રશ્ન 1(ક)(iii) {[}1 ગુણ{]}
(OR)}\label{uxaaauxab0uxab6uxaa8-1uxa95iii-1-uxa97uxaa3-or}

\textbf{ઉષ્મા પ્રસરણના કયા પ્રકારમાં માધ્યમની જરૂર નથી?}

\begin{solutionbox}

\textbf{વિકિરણ (Radiation)} ઉષ્મા પ્રસરણ માટે માધ્યમની જરૂર નથી.

\end{solutionbox}
\subsection*{પ્રશ્ન 2(અ) {[}3
ગુણ{]}}\label{uxaaauxab0uxab6uxaa8-2uxa85-3-uxa97uxaa3}

\textbf{વિદ્યુત ક્ષેત્ર રેખાઓની લાક્ષણિકતાઓ લખો.}

\begin{solutionbox}

\textbf{વિદ્યુત ક્ષેત્ર રેખાઓની લાક્ષણિકતાઓ}:

\begin{enumerate}
\def\labelenumi{\arabic{enumi}.}
\tightlist
\item
  વિદ્યુત ક્ષેત્ર રેખાઓ ધન ચાર્જથી શરૂ થાય છે અને ઋણ ચાર્જ પર સમાપ્ત થાય છે
\item
  ક્ષેત્ર રેખાઓ ક્યારેય એકબીજાને છેદતી નથી
\item
  ક્ષેત્ર રેખાઓ હંમેશા વાહકની સપાટી પર લંબરૂપ હોય છે
\item
  ક્ષેત્ર રેખાઓની સંખ્યા ચાર્જના જથ્થા સાથે પ્રમાણસર હોય છે
\item
  નજીકની ક્ષેત્ર રેખાઓ મજબૂત વિદ્યુત ક્ષેત્ર સૂચવે છે
\item
  ક્ષેત્ર રેખાઓ સતત વક્ર હોય છે
\item
  ક્ષેત્ર રેખાઓ લંબાઈમાં સંકોચાય છે અને પહોળાઈમાં વિસ્તરે છે
\end{enumerate}

\textbf{આકૃતિ}:

\begin{verbatim}
     +           {-}
      {         /}
       {       /}
        {     /}
         {   /}
          { /}
           X
\end{verbatim}

\end{solutionbox}
\begin{mnemonicbox}
``વિદ્યુત ક્ષેત્ર: ધનથી શરૂ, ઋણે સમાપ્ત, ક્યારેય છેદાતી નથી''

\end{mnemonicbox}
\subsection*{પ્રશ્ન 2(બ) {[}4
ગુણ{]}}\label{uxaaauxab0uxab6uxaa8-2uxaac-4-uxa97uxaa3}

\textbf{ઇલેક્ટ્રોસ્ટેટિક બળ માટે કુલંબનો વ્યસ્ત વર્ગનો નિયમને સમજાવો.}

\begin{solutionbox}

\textbf{કુલંબનો વ્યસ્ત વર્ગનો નિયમ}: બે બિંદુ ચાર્જ વચ્ચેનું ઇલેક્ટ્રોસ્ટેટિક બળ ચાર્જના
જથ્થાના ગુણાકાર સાથે સીધું પ્રમાણસર અને તેમની વચ્ચેના અંતરના વર્ગ સાથે વ્યસ્ત પ્રમાણસર
હોય છે.

\textbf{ગણિતીય સ્વરૂપ}: F = k(q₁q₂)/r²

જ્યાં:

\begin{itemize}
\tightlist
\item
  F = ઇલેક્ટ્રોસ્ટેટિક બળ (ન્યૂટનમાં)
\item
  k = ઇલેક્ટ્રોસ્ટેટિક અચળાંક (9×10⁹ N·m²/C²)
\item
  q₁, q₂ = ચાર્જના જથ્થા (કુલંબમાં)
\item
  r = ચાર્જ વચ્ચેનું અંતર (મીટરમાં)
\end{itemize}

\textbf{ગુણધર્મો}:

\begin{itemize}
\tightlist
\item
  \textbf{સદિશ રાશિ}: બળ બે ચાર્જને જોડતી રેખા પર કાર્ય કરે છે
\item
  \textbf{આકર્ષક/અપાકર્ષક}: સમાન ચાર્જ એકબીજાને અપાકર્ષિત કરે છે, વિપરીત ચાર્જ
  આકર્ષિત કરે છે
\item
  \textbf{કેન્દ્રીય બળ}: ન્યૂટનના ત્રીજા નિયમને અનુસરે છે
\item
  \textbf{માધ્યમ પર આધાર}: ચાર્જ વચ્ચેના માધ્યમ પર આધાર રાખે છે (k બદલાય છે)
\end{itemize}

\textbf{આકૃતિ}:

\begin{verbatim}
     q₁           q₂
      O-----------O
      ←───F₁²───→ ←───F₂¹───
         r
\end{verbatim}

\end{solutionbox}
\begin{mnemonicbox}
``ચાર્જ અંતરના વર્ગ સાથે વ્યસ્ત સંબંધ ધરાવે''

\end{mnemonicbox}
\subsection*{પ્રશ્ન 2(ક)(i) {[}4
ગુણ{]}}\label{uxaaauxab0uxab6uxaa8-2uxa95i-4-uxa97uxaa3}

\textbf{શ્રેણી અને સમાંતર સંયોજનમાં જોડાયેલા કેપેસિટર્સની સમતુલ્ય કેપેસીટન્સ માટે સૂત્ર
મેળવો.}

\begin{solutionbox}

\textbf{શ્રેણી સંયોજન માટે}:

\begin{center}
\textbf{Mermaid Diagram (Code)}
\begin{verbatim}
{Shaded}
{Highlighting}[]
graph LR
    A["{+"] {-}{-}{-} B[C₁]}
    B {-{-}{-} C[C₂]}
    C {-{-}{-} D[C₃]}
    D {-{-}{-} E["{}{-}"]}
{Highlighting}
{Shaded}
\end{verbatim}
\end{center}

જ્યારે કેપેસિટર્સ શ્રેણી સંયોજનમાં જોડાય છે:

\begin{itemize}
\tightlist
\item
  દરેક કેપેસિટર પર સમાન ચાર્જ Q હોય છે
\item
  વિભવાંતર દરેક કેપેસિટર વચ્ચે વહેંચાય છે
\item
  V = V₁ + V₂ + V₃
\end{itemize}

દરેક કેપેસિટર માટે: V₁ = Q/C₁, V₂ = Q/C₂, V₃ = Q/C₃

કુલ વોલ્ટેજ: V = Q/C₁ + Q/C₂ + Q/C₃ = Q(1/C₁ + 1/C₂ + 1/C₃)

સમતુલ્ય કેપેસિટન્સ માટે: V = Q/Ceq

તેથી: 1/Ceq = 1/C₁ + 1/C₂ + 1/C₃

\textbf{સમાંતર સંયોજન માટે}:

\begin{center}
\textbf{Mermaid Diagram (Code)}
\begin{verbatim}
{Shaded}
{Highlighting}[]
graph LR
    A["{+"] {-}{-}{-} B[C₁]}
    A {-{-}{-} C[C₂]}
    A {-{-}{-} D[C₃]}
    B {-{-}{-} E["{}{-}"]}
    C {-{-}{-} E}
    D {-{-}{-} E}
{Highlighting}
{Shaded}
\end{verbatim}
\end{center}

જ્યારે કેપેસિટર્સ સમાંતર સંયોજનમાં જોડાય છે:

\begin{itemize}
\tightlist
\item
  દરેક કેપેસિટર પર સમાન વિભવાંતર V હોય છે
\item
  કુલ ચાર્જ દરેક કેપેસિટર વચ્ચે વહેંચાય છે
\item
  Q = Q₁ + Q₂ + Q₃
\end{itemize}

દરેક કેપેસિટર માટે: Q₁ = C₁V, Q₂ = C₂V, Q₃ = C₃V

કુલ ચાર્જ: Q = C₁V + C₂V + C₃V = (C₁ + C₂ + C₃)V

સમતુલ્ય કેપેસિટન્સ માટે: Q = CeqV

તેથી: Ceq = C₁ + C₂ + C₃

\end{solutionbox}
\begin{mnemonicbox}
``શ્રેણીમાં વ્યસ્ત કેપેસિટન્સની સરવાળો, સમાંતરમાં કેપેસિટન્સનો
સરવાળો''

\end{mnemonicbox}
\subsection*{પ્રશ્ન 2(ક)(ii) {[}2
ગુણ{]}}\label{uxaaauxab0uxab6uxaa8-2uxa95ii-2-uxa97uxaa3}

\textbf{8 μF અને 9 μF કેપેસિટન્સ ધરાવતા બે કેપેસિટર્સ સમાંતર સંયોજનમાં જોડાયેલા છે.
સમતુલ્ય કેપેસિટન્સ શોધો.}

\begin{solutionbox}

\textbf{સમાંતર સંયોજન માટે સૂત્ર}: Ceq = C₁ + C₂

\textbf{આપેલ}:

\begin{itemize}
\tightlist
\item
  C₁ = 8 μF
\item
  C₂ = 9 μF
\end{itemize}

\textbf{ગણતરી}: Ceq = 8 μF + 9 μF = 17 μF

\textbf{આથી, સમતુલ્ય કેપેસિટન્સ = 17 μF}

\end{solutionbox}
\subsection*{પ્રશ્ન 2(ક)(iii) {[}1
ગુણ{]}}\label{uxaaauxab0uxab6uxaa8-2uxa95iii-1-uxa97uxaa3}

\textbf{LASER નું પૂરું નામ લખો.}

\begin{solutionbox}

\textbf{LASER}: Light Amplification by Stimulated Emission of Radiation
(પ્રકાશનું ઉત્તેજિત ઉત્સર્જન દ્વારા પ્રવર્ધન)

\end{solutionbox}
\subsection*{પ્રશ્ન 2(અ) {[}3 ગુણ{]}
(OR)}\label{uxaaauxab0uxab6uxaa8-2uxa85-3-uxa97uxaa3-or}

\textbf{કેપેસિટર શું છે? કેપેસિટન્સને વ્યાખ્યાયિત કરો અને તેનું એકમ લખો.}

\begin{solutionbox}

\textbf{કેપેસિટર}: એક ઉપકરણ જે વિદ્યુત ક્ષેત્રના સ્વરૂપમાં વિદ્યુત ચાર્જ અને વિદ્યુત ઊર્જા
સંગ્રહિત કરે છે.

\textbf{કેપેસિટન્સ}: કેપેસિટરની વિદ્યુત ચાર્જ સંગ્રહિત કરવાની ક્ષમતા. તે લાગુ કરેલ
વિભવાંતર સાથે સંગ્રહિત ચાર્જના ગુણોત્તર તરીકે વ્યાખ્યાયિત થાય છે.

\textbf{ગણિતીય સ્વરૂપ}: C = Q/V

જ્યાં:

\begin{itemize}
\tightlist
\item
  C = કેપેસિટન્સ
\item
  Q = કેપેસિટર પર સંગ્રહિત ચાર્જ
\item
  V = કેપેસિટર પરનો વિભવાંતર
\end{itemize}

\textbf{કેપેસિટન્સનું એકમ}: ફેરડ (F)

\textbf{આકૃતિ}:

\begin{verbatim}
    +++++++  |  -------
           |   |
           |   |
        ---+---+---
           |   |
           |   |
    +++++++  |  -------
\end{verbatim}

\end{solutionbox}
\begin{mnemonicbox}
``કેપેસિટર ચાર્જ સંગ્રહે, વોલ્ટેજ વિભાજિત કરે''

\end{mnemonicbox}
\subsection*{પ્રશ્ન 2(બ) {[}4 ગુણ{]}
(OR)}\label{uxaaauxab0uxab6uxaa8-2uxaac-4-uxa97uxaa3-or}

\textbf{વિદ્યુત ક્ષેત્રની તીવ્રતા અને વિદ્યુત સ્થિતિમાન સમજાવો.}

\begin{solutionbox}

\textbf{વિદ્યુત ક્ષેત્રની તીવ્રતા}:

\begin{itemize}
\tightlist
\item
  \textbf{વ્યાખ્યા}: તે બિંદુ પર મૂકાયેલા એકમ ધન ચાર્જને લાગતું બળ
\item
  \textbf{સૂત્ર}: E = F/q
\item
  \textbf{એકમ}: ન્યૂટન/કુલંબ (N/C) અથવા વોલ્ટ/મીટર (V/m)
\item
  \textbf{સદિશ રાશિ}: જેમાં તીવ્રતા અને દિશા બંને હોય છે
\item
  \textbf{દિશા}: ધન ચાર્જ પર લાગતા બળની દિશા જેવી જ
\end{itemize}

\textbf{વિદ્યુત સ્થિતિમાન}:

\begin{itemize}
\tightlist
\item
  \textbf{વ્યાખ્યા}: અનંતથી તે બિંદુ સુધી એકમ ધન ચાર્જને લાવવા માટે કરેલું કાર્ય
\item
  \textbf{સૂત્ર}: V = W/q
\item
  \textbf{એકમ}: વોલ્ટ (V) અથવા જૂલ/કુલંબ (J/C)
\item
  \textbf{અદિશ રાશિ}: ફક્ત તીવ્રતા ધરાવે છે
\item
  \textbf{ક્ષેત્ર સાથે સંબંધ}: E = -dV/dr (ક્ષેત્ર સ્થિતિમાનનો નકારાત્મક ગ્રેડિયન્ટ છે)
\end{itemize}

\textbf{સરખામણીનું કોષ્ટક}:

{\def\LTcaptype{none} % do not increment counter
\begin{longtable}[]{@{}lll@{}}
\toprule\noalign{}
ગુણધર્મ & વિદ્યુત ક્ષેત્ર & વિદ્યુત સ્થિતિમાન \\
\midrule\noalign{}
\endhead
\bottomrule\noalign{}
\endlastfoot
વ્યાખ્યા & એકમ ચાર્જ દીઠ બળ & એકમ ચાર્જ દીઠ કાર્ય \\
પ્રકૃતિ & સદિશ & અદિશ \\
એકમ & N/C અથવા V/m & V અથવા J/C \\
નિર્ભરતા & 1/r² સાથે બદલાય & 1/r સાથે બદલાય \\
દિશા & ધન ચાર્જથી દૂર & કોઈ દિશા નથી \\
\end{longtable}
}

\end{solutionbox}
\begin{mnemonicbox}
``વિદ્યુત ક્ષેત્ર બળ આપે; સ્થિતિમાન ઊર્જા આપે''

\end{mnemonicbox}
\subsection*{પ્રશ્ન 2(ક)(i) {[}4 ગુણ{]}
(OR)}\label{uxaaauxab0uxab6uxaa8-2uxa95i-4-uxa97uxaa3-or}

\textbf{સમાંતર પ્લેટ કેપેસિટરના કેપેસીટન્સના સૂત્રનો ઉપયોગ કરીને પ્લેટનો ક્ષેત્રફળ, પ્લેટો
વચ્ચેનું અંતર અને પ્લેટો વચ્ચે ડાઇલેક્ટ્રિક સામગ્રીની ઉપસ્થિતિની તેની કેપેસિટન્સ પર અસરને
સમજાવો.}

\begin{solutionbox}

\textbf{સમાંતર પ્લેટ કેપેસિટરના કેપેસિટન્સનું સૂત્ર}: C = ε₀εᵣA/d

જ્યાં:

\begin{itemize}
\tightlist
\item
  C = કેપેસિટન્સ
\item
  ε₀ = નિર્વાત અવકાશની પરમિટિવિટી (8.85×10⁻¹² F/m)
\item
  εᵣ = ડાઇલેક્ટ્રિકની સાપેક્ષ પરમિટિવિટી
\item
  A = પ્લેટોના ઓવરલેપનો ક્ષેત્રફળ
\item
  d = પ્લેટો વચ્ચેનું અંતર
\end{itemize}

\textbf{પ્લેટના ક્ષેત્રફળની અસર (A)}:

\begin{itemize}
\tightlist
\item
  કેપેસિટન્સ પ્લેટના ક્ષેત્રફળ સાથે સીધું પ્રમાણસર છે
\item
  ક્ષેત્રફળ વધારતાં → કેપેસિટન્સ વધે છે
\item
  ક્ષેત્રફળ બમણો કરતાં → કેપેસિટન્સ બમણું થાય છે
\end{itemize}

\textbf{અંતરની અસર (d)}:

\begin{itemize}
\tightlist
\item
  કેપેસિટન્સ પ્લેટો વચ્ચેના અંતર સાથે વ્યસ્ત પ્રમાણસર છે
\item
  અંતર વધારતાં → કેપેસિટન્સ ઘટે છે
\item
  અંતર બમણું કરતાં → કેપેસિટન્સ અડધું થાય છે
\end{itemize}

\textbf{ડાઇલેક્ટ્રિક સામગ્રીની અસર (εᵣ)}:

\begin{itemize}
\tightlist
\item
  કેપેસિટન્સ ડાઇલેક્ટ્રિકની સાપેક્ષ પરમિટિવિટી સાથે સીધું પ્રમાણસર છે
\item
  ડાઇલેક્ટ્રિક દાખલ કરતાં → કેપેસિટન્સ વધે છે
\item
  ડાઇલેક્ટ્રિક અચળાંક આ વધારાનું માપ કરે છે: C(ડાઇલેક્ટ્રિક સાથે) = εᵣ × C(ડાઇલેક્ટ્રિક
  વગર)
\end{itemize}

\textbf{આકૃતિ}:

\begin{verbatim}
    +++++++  |  -------
           |   |
       A   | d |
        ---+---+---
           |εᵣ |
           |   |
    +++++++  |  -------
\end{verbatim}

\end{solutionbox}
\begin{mnemonicbox}
``ક્ષેત્રફળ વધારે, અંતર ઘટાડે, ડાઇલેક્ટ્રિક ગુણાકારે''

\end{mnemonicbox}
\subsection*{પ્રશ્ન 2(ક)(ii) {[}2 ગુણ{]}
(OR)}\label{uxaaauxab0uxab6uxaa8-2uxa95ii-2-uxa97uxaa3-or}

\textbf{0.5 μF ના કેપેસિટરની પ્લેટો વચ્ચેનો વોલ્ટેજ 150 V છે. પ્લેટો પર ઇલેક્ટ્રિક
ચાર્જનું મૂલ્ય શોધો.}

\begin{solutionbox}

\textbf{સૂત્ર}: Q = CV

\textbf{આપેલ}:

\begin{itemize}
\tightlist
\item
  કેપેસિટન્સ (C) = 0.5 μF = 0.5 × 10⁻⁶ F
\item
  વોલ્ટેજ (V) = 150 V
\end{itemize}

\textbf{ગણતરી}: Q = CV = 0.5 × 10⁻⁶ × 150 = 75 × 10⁻⁶ C = 75 μC

\textbf{આથી, પ્લેટો પરનો ચાર્જ = 75 μC}

\end{solutionbox}
\subsection*{પ્રશ્ન 2(ક)(iii) {[}1 ગુણ{]}
(OR)}\label{uxaaauxab0uxab6uxaa8-2uxa95iii-1-uxa97uxaa3-or}

\textbf{ઓપ્ટિકલ ફાઇબરના બે ભાગ કોર અને ક્લેડિંગ માંથી, કયો ભાગ મોટો રીફ્રેક્ટિવ
ઇન્ડેક્સ ધરાવે છે?}

\begin{solutionbox}

\textbf{કોર (core)} ક્લેડિંગ કરતાં વધારે રીફ્રેક્ટિવ ઇન્ડેક્સ ધરાવે છે.

\end{solutionbox}
\subsection*{પ્રશ્ન 3(અ) {[}3
ગુણ{]}}\label{uxaaauxab0uxab6uxaa8-3uxa85-3-uxa97uxaa3}

\textbf{ઉષ્માવહન અને ઉષ્માનયનને વ્યાખ્યાયિત કરો.}

\begin{solutionbox}

\textbf{ઉષ્માવહન}:

\begin{itemize}
\tightlist
\item
  કણોની વાસ્તવિક ગતિ વિના પદાર્થ મારફતે ઉષ્માનું સ્થાનાંતરણ
\item
  સીધા અણુઓના સંઘર્ષને કારણે થાય છે
\item
  ઉષ્મા ઉચ્ચ તાપમાનથી ઓછા તાપમાન તરફ વહે છે
\item
  ધાતુઓ ઉષ્માના સારા વાહક છે
\item
  ઉદાહરણ: ધાતુના સળિયા દ્વારા ઉષ્મા પ્રસરણ, રસોઈના વાસણ
\end{itemize}

\textbf{ઉષ્માનયન}:

\begin{itemize}
\tightlist
\item
  પદાર્થની વાસ્તવિક ગતિ દ્વારા ઉષ્માનું સ્થાનાંતરણ
\item
  પ્રવાહીઓ (દ્રવ્યો અને વાયુઓ)માં થાય છે
\item
  ઉષ્માનયન પ્રવાહોની રચના સમાવે છે
\item
  ઉદાહરણ: રૂમ હીટર, સમુદ્રનો પવન, ઉકળતું પાણી
\end{itemize}

\textbf{આકૃતિ}:

\begin{verbatim}
ઉષ્માવહન:
ગરમ     ઠંડુ
|->->->->|

ઉષ્માનયન:
      ↑
    ←   →
      ↓
    ઉષ્મા
\end{verbatim}

\end{solutionbox}
\begin{mnemonicbox}
``વહન વાહક જોડે; ઉષ્માનયન દ્રવ્યને ફેરવે''

\end{mnemonicbox}
\subsection*{પ્રશ્ન 3(બ) {[}4
ગુણ{]}}\label{uxaaauxab0uxab6uxaa8-3uxaac-4-uxa97uxaa3}

\textbf{પારાના થર્મોમીટરનું રચના અને કાર્ય સમજાવો.}

\begin{solutionbox}

\textbf{પારાના થર્મોમીટરની રચના}:

\begin{center}
\textbf{Mermaid Diagram (Code)}
\begin{verbatim}
{Shaded}
{Highlighting}[]
graph LR
    A[કાચનો બલ્બ] {-{-}{-} B[કેપિલરી ટ્યુબ]}
    B {-{-}{-} C[સ્કેલ]}
    C {-{-}{-} D[સુરક્ષાત્મક કાચનું આવરણ]}
{Highlighting}
{Shaded}
\end{verbatim}
\end{center}

\begin{itemize}
\tightlist
\item
  \textbf{કાચનો બલ્બ}: પારો ધરાવે છે, સંગ્રહ તરીકે કાર્ય કરે છે
\item
  \textbf{કેપિલરી ટ્યુબ}: બલ્બ સાથે જોડાયેલી પાતળી કાચની નળી
\item
  \textbf{સ્કેલ}: તાપમાન માપવા માટે અંશાંકિત
\item
  \textbf{સુરક્ષાત્મક કાચનું આવરણ}: કેપિલરી ટ્યુબ અને સ્કેલને સુરક્ષિત રાખે છે
\end{itemize}

\textbf{કાર્યસિદ્ધાંત}:

\begin{enumerate}
\def\labelenumi{\arabic{enumi}.}
\tightlist
\item
  પારાના થર્મલ વિસ્તરણ પર આધારિત
\item
  તાપમાન વધતાં, પારો વિસ્તરે છે અને કેપિલરીમાં ઉપર ચઢે છે
\item
  તાપમાન ઘટતાં, પારો સંકોચાય છે અને તેનું સ્તર નીચે જાય છે
\item
  પારાના સ્તર પરથી સ્કેલ પરથી તાપમાન વાંચવામાં આવે છે
\end{enumerate}

\textbf{તાપમાન શ્રેણી}: -38.83°C થી 356.73°C (પારાના ઠારણ બિંદુથી ઉત્કલન બિંદુ)

\textbf{ફાયદાઓ}:

\begin{itemize}
\tightlist
\item
  ઉચ્ચ ચોકસાઈ
\item
  રેખીય વિસ્તરણ
\item
  કેપિલરીમાં સ્પષ્ટ દેખાય છે
\end{itemize}

\textbf{મર્યાદાઓ}:

\begin{itemize}
\tightlist
\item
  ખૂબ ઓછા તાપમાનને માપી શકતું નથી
\item
  પારો ઝેરી છે
\item
  રિમોટ સેન્સિંગ માટે વાપરી શકાતું નથી
\end{itemize}

\end{solutionbox}
\begin{mnemonicbox}
``પારો કેપિલરીમાં ફરે છે, તાપમાન બતાવે છે''

\end{mnemonicbox}
\subsection*{પ્રશ્ન 3(ક)(i) {[}4
ગુણ{]}}\label{uxaaauxab0uxab6uxaa8-3uxa95i-4-uxa97uxaa3}

\textbf{ઉષ્માવાહકતાના નિયમો લખો અને ઉષ્માવાહકતા અંકનું સૂત્ર મેળવો.}

\begin{solutionbox}

\textbf{ઉષ્માવાહકતાના નિયમો}:

\begin{enumerate}
\def\labelenumi{\arabic{enumi}.}
\tightlist
\item
  ઉષ્મા પ્રવાહ તાપમાન તફાવત (ΔT) સાથે સીધો પ્રમાણસર છે
\item
  ઉષ્મા પ્રવાહ આડછેદના ક્ષેત્રફળ (A) સાથે સીધો પ્રમાણસર છે
\item
  ઉષ્મા પ્રવાહ લંબાઈ (L) સાથે વ્યસ્ત પ્રમાણસર છે
\item
  ઉષ્મા પ્રવાહ સમય (t) સાથે સીધો પ્રમાણસર છે
\end{enumerate}

\textbf{ઉષ્માવાહકતા અંકની તારણ}:

ફૂરિયરના નિયમ અનુસાર: Q ∝ A × t × ΔT/L

પ્રમાણસરતા અચળાંક K સાથે સમીકરણમાં રૂપાંતરિત કરતાં: Q = K × A × t × ΔT/L

ફરીથી ગોઠવતાં: K = (Q × L)/(A × t × ΔT)

જ્યાં:

\begin{itemize}
\tightlist
\item
  Q = વાહિત ઉષ્મા (જૂલમાં)
\item
  L = વાહકની લંબાઈ (મીટરમાં)
\item
  A = આડછેદનું ક્ષેત્રફળ (m² માં)
\item
  t = સમય (સેકન્ડમાં)
\item
  ΔT = તાપમાન તફાવત (કેલ્વિનમાં)
\item
  K = ઉષ્માવાહકતા અંક (W/m·K માં)
\end{itemize}

\textbf{આકૃતિ}:

\begin{verbatim}
ગરમ         ઠંડુ
T₁ ---------T₂
    લંબાઈ L
    ક્ષેત્રફળ A
    ઉષ્મા Q
\end{verbatim}

\end{solutionbox}
\begin{mnemonicbox}
``ઉષ્મા ઝડપથી વહે જ્યારે ક્ષેત્રફળ મોટું, તાપમાન વધુ, લંબાઈ
ઓછી''

\end{mnemonicbox}
\subsection*{પ્રશ્ન 3(ક)(ii) {[}2
ગુણ{]}}\label{uxaaauxab0uxab6uxaa8-3uxa95ii-2-uxa97uxaa3}

\textbf{એક કાચની વિંડોનું કુલ ક્ષેત્રફળ 0.5m² છે. જો કાચની જાડાઈ 0.6cm, અંદરનું
તાપમાન 30°C અને બહારનું તાપમાન 20°C છે તો વિંડો દ્વારા પ્રતિ કલાક થતી ઉષ્માનું
વહનનું ગણતરી કરો. કાચ માટે ઉષ્માવાહકતા અંક 1.0 Wm⁻¹K⁻¹ છે.}

\begin{solutionbox}

\textbf{સૂત્ર}: Q = (K × A × t × ΔT)/L

\textbf{આપેલ}:

\begin{itemize}
\tightlist
\item
  ક્ષેત્રફળ (A) = 0.5 m²
\item
  જાડાઈ (L) = 0.6 cm = 0.006 m
\item
  અંદરનું તાપમાન (T₁) = 30°C
\item
  બહારનું તાપમાન (T₂) = 20°C
\item
  તાપમાન તફાવત (ΔT) = 10°C = 10 K
\item
  ઉષ્માવાહકતા અંક (K) = 1.0 W/m·K
\item
  સમય (t) = 1 કલાક = 3600 સેકન્ડ
\end{itemize}

\textbf{ગણતરી}: Q = (1.0 × 0.5 × 3600 × 10)/0.006 Q = (18000)/0.006 Q =
3,000,000 J = 3000 kJ

\textbf{આથી, વાહિત ઉષ્મા = 3000 kJ પ્રતિ કલાક}

\end{solutionbox}
\subsection*{પ્રશ્ન 3(ક)(iii) {[}1
ગુણ{]}}\label{uxaaauxab0uxab6uxaa8-3uxa95iii-1-uxa97uxaa3}

\textbf{ઓપ્ટિકલ ફાઈબર દ્વારા પ્રકાશના પ્રસરણ માટે પ્રકાશના કયા ગુણધર્મ જવાબદાર
છે?}

\begin{solutionbox}

\textbf{સંપૂર્ણ આંતરિક પરાવર્તન (Total Internal Reflection - TIR)} ઓપ્ટિકલ
ફાઈબર દ્વારા પ્રકાશના પ્રસરણ માટે જવાબદાર છે.

\end{solutionbox}
\subsection*{પ્રશ્ન 3(અ) {[}3 ગુણ{]}
(OR)}\label{uxaaauxab0uxab6uxaa8-3uxa85-3-uxa97uxaa3-or}

\textbf{ઉષ્માધારિતા અને વિશિષ્ટ ઉષ્મા ને વ્યાખ્યાયિત કરો.}

\begin{solutionbox}

\textbf{ઉષ્માધારિતા}:

\begin{itemize}
\tightlist
\item
  કોઈ પદાર્થના તાપમાનમાં 1°C અથવા 1K વધારવા માટે જરૂરી ઉષ્મા ઊર્જાનો જથ્થો
\item
  પદાર્થના દ્રવ્યમાન અને સામગ્રી પર આધાર રાખે છે
\item
  સૂત્ર: C = Q/ΔT
\item
  એકમ: જૂલ/કેલ્વિન (J/K)
\end{itemize}

\textbf{વિશિષ્ટ ઉષ્મા}:

\begin{itemize}
\tightlist
\item
  કોઈ પદાર્થના 1 kg ના તાપમાનમાં 1°C અથવા 1K વધારવા માટે જરૂરી ઉષ્મા ઊર્જાનો
  જથ્થો
\item
  સામગ્રીનો ગુણધર્મ, દ્રવ્યમાન પર આધાર રાખતો નથી
\item
  સૂત્ર: c = Q/(m×ΔT)
\item
  એકમ: જૂલ/કિગ્રા·કેલ્વિન (J/kg·K)
\end{itemize}

\textbf{સંબંધ}: ઉષ્માધારિતા (C) = દ્રવ્યમાન (m) × વિશિષ્ટ ઉષ્મા (c)

\textbf{સરખામણીનું કોષ્ટક}:

{\def\LTcaptype{none} % do not increment counter
\begin{longtable}[]{@{}
  >{\raggedright\arraybackslash}p{(\linewidth - 4\tabcolsep) * \real{0.2571}}
  >{\raggedright\arraybackslash}p{(\linewidth - 4\tabcolsep) * \real{0.3714}}
  >{\raggedright\arraybackslash}p{(\linewidth - 4\tabcolsep) * \real{0.3714}}@{}}
\toprule\noalign{}
\begin{minipage}[b]{\linewidth}\raggedright
ગુણધર્મ
\end{minipage} & \begin{minipage}[b]{\linewidth}\raggedright
ઉષ્માધારિતા
\end{minipage} & \begin{minipage}[b]{\linewidth}\raggedright
વિશિષ્ટ ઉષ્મા
\end{minipage} \\
\midrule\noalign{}
\endhead
\bottomrule\noalign{}
\endlastfoot
વ્યાખ્યા & પદાર્થના 1 ડિગ્રી તાપમાન વધારવા માટે જરૂરી ઉષ્મા & એકમ દ્રવ્યમાન દીઠ 1
ડિગ્રી તાપમાન વધારવા માટે જરૂરી ઉષ્મા \\
સંકેત & C & c \\
એકમ & J/K & J/kg·K \\
આધાર & દ્રવ્યમાન અને સામગ્રી & ફક્ત સામગ્રી \\
સૂત્ર & Q/ΔT & Q/(m×ΔT) \\
\end{longtable}
}

\end{solutionbox}
\begin{mnemonicbox}
``ઉષ્માધારિતા પૂર્ણ પદાર્થ માટે, વિશિષ્ટ ઉષ્મા એક કિલોગ્રામ
માટે''

\end{mnemonicbox}
\subsection*{પ્રશ્ન 3(બ) {[}4 ગુણ{]}
(OR)}\label{uxaaauxab0uxab6uxaa8-3uxaac-4-uxa97uxaa3-or}

\textbf{ઓપ્ટિકલ પાયરોમીટરનું રચના અને કાર્ય સમજાવો.}

\begin{solutionbox}

\textbf{ઓપ્ટિકલ પાયરોમીટરની રચના}:

\begin{center}
\textbf{Mermaid Diagram (Code)}
\begin{verbatim}
{Shaded}
{Highlighting}[]
graph LR
    A[ટેલિસ્કોપ] {-{-}{-} B[ફિલામેન્ટ લેમ્પ]}
    B {-{-}{-} C[એમીટર]}
    C {-{-}{-} D[બેટરી]}
    D {-{-}{-} B}
    A {-{-}{-} E[રંગ ફિલ્ટર]}
    E {-{-}{-} F[આઈપીસ]}
{Highlighting}
{Shaded}
\end{verbatim}
\end{center}

\begin{itemize}
\tightlist
\item
  \textbf{ટેલિસ્કોપ}: ગરમ પદાર્થને જોવા માટે
\item
  \textbf{ફિલામેન્ટ લેમ્પ}: અંશાંકિત ટંગસ્ટન ફિલામેન્ટ
\item
  \textbf{રિઓસ્ટેટ}: ફિલામેન્ટ મારફતે પ્રવાહ એડજસ્ટ કરવા માટે
\item
  \textbf{એમીટર}: પ્રવાહ માપવા માટે
\item
  \textbf{રેડ ફિલ્ટર}: તરંગલંબાઈઓને મેળવવા માટે
\item
  \textbf{આઈપીસ}: જોવા માટે
\end{itemize}

\textbf{કાર્યસિદ્ધાંત}:

\begin{enumerate}
\def\labelenumi{\arabic{enumi}.}
\tightlist
\item
  ગરમ પદાર્થની ચળકાટને સ્ટાન્ડર્ડ લેમ્પ ફિલામેન્ટ સાથે સરખાવવા પર આધારિત
\item
  પદાર્થને ટેલિસ્કોપ દ્વારા જોવામાં આવે છે
\item
  ફિલામેન્ટની ચળકાટ પદાર્થની ચળકાટ સાથે મેળ ખાય ત્યાં સુધી પ્રવાહ એડજસ્ટ કરવામાં આવે
  છે
\item
  મેળ બિંદુ પર, ફિલામેન્ટ પદાર્થની પૃષ્ઠભૂમિ સામે ``અદ્રશ્ય'' થાય છે
\item
  અંશાંકિત સ્કેલ અથવા એમીટર વાંચન પરથી તાપમાન નક્કી કરવામાં આવે છે
\end{enumerate}

\textbf{તાપમાન શ્રેણી}: 700°C થી 3000°C

\textbf{ફાયદાઓ}:

\begin{itemize}
\tightlist
\item
  સંપર્ક વિનાનું માપન
\item
  ઉચ્ચ તાપમાન માપન
\item
  ચાલતા પદાર્થો માટે યોગ્ય
\end{itemize}

\end{solutionbox}
\begin{mnemonicbox}
``પાયરોમીટર ચળકાટની સરખામણી કરીને તાપમાન માપે છે''

\end{mnemonicbox}
\subsection*{પ્રશ્ન 3(ક)(i) {[}4 ગુણ{]}
(OR)}\label{uxaaauxab0uxab6uxaa8-3uxa95i-4-uxa97uxaa3-or}

\textbf{ઘન પદાર્થોના રેખીય ઉષ્મીય વિસ્તરણને વ્યાખ્યાયિત કરો અને રેખીય ઉષ્મીય
વિસ્તરણ ગુણાંકનું સૂત્ર મેળવો.}

\begin{solutionbox}

\textbf{રેખીય ઉષ્મીય વિસ્તરણ}: તાપમાનમાં વધારો થતાં ઘન પદાર્થની લંબાઈમાં થતો
વધારો

\textbf{રેખીય ઉષ્મીય વિસ્તરણ ગુણાંક (α)}: તાપમાનમાં એકમ ફેરફાર દીઠ લંબાઈમાં થતો
ભાગાત્મક ફેરફાર

\textbf{તારણ}:

નાના તાપમાન ફેરફાર માટે:

\begin{itemize}
\tightlist
\item
  લંબાઈમાં ફેરફાર (ΔL) મૂળ લંબાઈ (L₀) સાથે સીધો પ્રમાણસર છે
\item
  ΔL તાપમાન ફેરફાર (ΔT) સાથે સીધો પ્રમાણસર છે
\end{itemize}

તેથી: ΔL ∝ L₀ × ΔT

પ્રમાણસરતા અચળાંક α સાથે સમીકરણમાં રૂપાંતરિત કરતાં: ΔL = α × L₀ × ΔT

ફરીથી ગોઠવતાં: α = ΔL/(L₀ × ΔT)

જ્યાં:

\begin{itemize}
\tightlist
\item
  ΔL = લંબાઈમાં ફેરફાર (મીટરમાં)
\item
  L₀ = મૂળ લંબાઈ (મીટરમાં)
\item
  ΔT = તાપમાન ફેરફાર (કેલ્વિન અથવા સેલ્સિયસમાં)
\item
  α = રેખીય ઉષ્મીય વિસ્તરણ ગુણાંક (પ્રતિ °C અથવા પ્રતિ K)
\end{itemize}

\textbf{અંતિમ લંબાઈ}: L = L₀(1 + αΔT)

\textbf{આકૃતિ}:

\begin{verbatim}
ગરમ કરતા પહેલાં:
|----L₀----|

ગરમ કર્યા પછી:
|------L------|
\end{verbatim}

\end{solutionbox}
\begin{mnemonicbox}
``રેખીય વિસ્તરણ લંબાઈ વધારે ગરમી વધતાં''

\end{mnemonicbox}
\subsection*{પ્રશ્ન 3(ક)(ii) {[}2 ગુણ{]}
(OR)}\label{uxaaauxab0uxab6uxaa8-3uxa95ii-2-uxa97uxaa3-or}

\textbf{0°C પર સ્ટીલના સળિયાની લંબાઈ 150 cm છે. 200°C પર તેની લંબાઈ કેટલી હશે
જો તેનો રેખીય ઉષ્મીય વિસ્તરણનો ગુણાંક 12 × 10⁻⁶ પ્રતિ °C હો}

\begin{solutionbox}

\textbf{સૂત્ર}: L = L₀(1 + αΔT)

\textbf{આપેલ}:

\begin{itemize}
\tightlist
\item
  મૂળ લંબાઈ (L₀) = 150 cm
\item
  મૂળ તાપમાન = 0°C
\item
  અંતિમ તાપમાન = 200°C
\item
  તાપમાન ફેરફાર (ΔT) = 200°C
\item
  રેખીય ઉષ્મીય વિસ્તરણ ગુણાંક (α) = 12 × 10⁻⁶ પ્રતિ °C
\end{itemize}

\textbf{ગણતરી}: L = 150(1 + 12 × 10⁻⁶ × 200) L = 150(1 + 24 × 10⁻⁴) L =
150(1 + 0.0024) L = 150 × 1.0024 L = 150.36 cm

\textbf{આથી, સ્ટીલના સળિયાની અંતિમ લંબાઈ = 150.36 cm}

\end{solutionbox}
\subsection*{પ્રશ્ન 3(ક)(iii) {[}1 ગુણ{]}
(OR)}\label{uxaaauxab0uxab6uxaa8-3uxa95iii-1-uxa97uxaa3-or}

\textbf{સામાન્ય પ્રકાશના ઉત્સર્જન માટે કયા પ્રકારના ઉત્સર્જન જવાબદાર છે?}

\begin{solutionbox}

\textbf{સ્વયંસ્ફૂર્ત ઉત્સર્જન (Spontaneous emission)} સામાન્ય પ્રકાશના ઉત્સર્જન માટે
જવાબદાર છે.

\end{solutionbox}
\subsection*{પ્રશ્ન 4(અ) {[}3
ગુણ{]}}\label{uxaaauxab0uxab6uxaa8-4uxa85-3-uxa97uxaa3}

\textbf{તરંગની કંપવિસ્તાર, આવૃત્તિ અને આવર્તકાળને વ્યાખ્યાયિત કરો.}

\begin{solutionbox}

\textbf{કંપવિસ્તાર}:

\begin{itemize}
\tightlist
\item
  માધ્યમના કણોનું સમતોલન સ્થિતિથી મહત્તમ વિસ્થાપન
\item
  તરંગની ઊર્જા દર્શાવે છે
\item
  `A' દ્વારા દર્શાવાય છે
\item
  મીટરમાં (m) માપવામાં આવે છે
\end{itemize}

\textbf{આવૃત્તિ}:

\begin{itemize}
\tightlist
\item
  એકમ સમયમાં થતાં સંપૂર્ણ સ્પંદનોની સંખ્યા
\item
  `f' અથવા `ν' દ્વારા દર્શાવાય છે
\item
  હર્ટ્ઝ (Hz) અથવા સ્પંદન પ્રતિ સેકન્ડમાં માપવામાં આવે છે
\item
  તરંગલંબાઈ (λ) અને વેગ (v) સાથે સંબંધિત: f = v/λ
\end{itemize}

\textbf{આવર્તકાળ}:

\begin{itemize}
\tightlist
\item
  એક સ્પંદન પૂર્ણ કરવા માટે લાગતો સમય
\item
  `T' દ્વારા દર્શાવાય છે
\item
  સેકન્ડ (s)માં માપવામાં આવે છે
\item
  આવૃત્તિ સાથે સંબંધિત: T = 1/f
\end{itemize}

\textbf{આકૃતિ}:

\begin{verbatim}
    કંપવિસ્તાર
        ↕
        |    /\      /\
        |   /  \    /  \
--------+--/----\--/----\---> સમય
        |  \    /  \    /
        |   \  /    \  /
        |    \/      \/
        |<--T-->|
\end{verbatim}

\end{solutionbox}
\begin{mnemonicbox}
``કંપવિસ્તાર ઊર્જા સૂચવે, આવૃત્તિ ચક્ર સૂચવે, આવર્તકાળ એક ચક્રનો
સમય સૂચવે''

\end{mnemonicbox}
\subsection*{પ્રશ્ન 4(બ) {[}4
ગુણ{]}}\label{uxaaauxab0uxab6uxaa8-4uxaac-4-uxa97uxaa3}

\textbf{લંબગત અને સંગત તરંગો વચ્ચેનો તફાવત લખો.}

\begin{solutionbox}

{\def\LTcaptype{none} % do not increment counter
\begin{longtable}[]{@{}
  >{\raggedright\arraybackslash}p{(\linewidth - 4\tabcolsep) * \real{0.2647}}
  >{\raggedright\arraybackslash}p{(\linewidth - 4\tabcolsep) * \real{0.3824}}
  >{\raggedright\arraybackslash}p{(\linewidth - 4\tabcolsep) * \real{0.3529}}@{}}
\toprule\noalign{}
\begin{minipage}[b]{\linewidth}\raggedright
ગુણધર્મ
\end{minipage} & \begin{minipage}[b]{\linewidth}\raggedright
લંબગત તરંગો
\end{minipage} & \begin{minipage}[b]{\linewidth}\raggedright
સંગત તરંગો
\end{minipage} \\
\midrule\noalign{}
\endhead
\bottomrule\noalign{}
\endlastfoot
કણોની ગતિની દિશા & તરંગના પ્રસરણની દિશાને લંબરૂપ & તરંગના પ્રસરણની દિશાને
સમાંતર \\
રચના & શિખર અને ખીણો & સંકોચન અને વિરલીકરણ \\
ઉદાહરણો & પ્રકાશ તરંગો, પાણીના તરંગો, વિદ્યુતચુંબકીય તરંગો & ધ્વનિ તરંગો, ભૂકંપીય
પી-તરંગો \\
માધ્યમની જરૂરીયાત & નિર્વાતમાં પ્રસરી શકે છે (દા.ત., પ્રકાશ) & ભૌતિક માધ્યમની જરૂર
પડે છે \\
ધ્રુવીભવન & ધ્રુવીભૂત થઈ શકે છે & ધ્રુવીભૂત થઈ શકતા નથી \\
વેગ & ઘન માધ્યમમાં સામાન્ય રીતે વધારે ઝડપી & ઘન માધ્યમમાં સામાન્ય રીતે ધીમા \\
ગાણિતિક સૂત્ર & y = A sin(kx - ωt) & s = A sin(kx - ωt) \\
\end{longtable}
}

\textbf{આકૃતિ}:

\begin{verbatim}
લંબગત:
  ^  ^     ^
 / \/ \   / \
/     \ /    \
---------->
પ્રસરણની દિશા

સંગત:
|||||   |||||   |||||
    |||||   |||||
---------->
પ્રસરણની દિશા
\end{verbatim}

\end{solutionbox}
\begin{mnemonicbox}
``લંબગત લંબરૂપ ગતિમાં, સંગત સમાંતર ગતિમાં''

\end{mnemonicbox}
\subsection*{પ્રશ્ન 4(ક)(i) {[}5
ગુણ{]}}\label{uxaaauxab0uxab6uxaa8-4uxa95i-5-uxa97uxaa3}

\textbf{પીઝોઇલેક્ટ્રિક પદ્ધતિનો ઉપયોગ કરીને અલ્ટ્રાસોનિક તરંગ કેવી રીતે ઉત્પન્ન થાય
છે?}

\begin{solutionbox}

\textbf{પીઝોઇલેક્ટ્રિક પદ્ધતિ દ્વારા અલ્ટ્રાસોનિક તરંગ ઉત્પન્ન કરવી}:

\begin{center}
\textbf{Mermaid Diagram (Code)}
\begin{verbatim}
{Shaded}
{Highlighting}[]
graph LR
    A[ઓસિલેટર] {-{-}{} B[એમ્પ્લિફાયર]}
    B {-{-}{} C[પીઝોઇલેક્ટ્રિક ક્રિસ્ટલ]}
    C {-{-}{} D[અલ્ટ્રાસોનિક તરંગો]}
{Highlighting}
{Shaded}
\end{verbatim}
\end{center}

\textbf{કાર્યસિદ્ધાંત}:

\begin{enumerate}
\def\labelenumi{\arabic{enumi}.}
\tightlist
\item
  પીઝોઇલેક્ટ્રિક ઇફેક્ટ પર આધારિત - યાંત્રિક દબાણના પ્રતિભાવમાં વિદ્યુત ચાર્જ ઉત્પન્ન
  કરવો અને તેનાથી ઉલટું
\item
  પીઝોઇલેક્ટ્રિક ક્રિસ્ટલ (ક્વાર્ટ્ઝ, ટુરમેલાઇન, રોશેલ સોલ્ટ) પર ઉચ્ચ-આવૃત્તિ AC વોલ્ટેજ
  લાગુ કરવામાં આવે છે
\item
  ક્રિસ્ટલ લાગુ કરેલ વોલ્ટેજની સમાન આવૃત્તિએ કંપન કરે છે
\item
  જ્યારે આવૃત્તિ ક્રિસ્ટલની કુદરતી આવૃત્તિ સાથે મેળ ખાય, ત્યારે અનુનાદ થાય છે
\item
  મહત્તમ કંપવિસ્તારના કંપનો અલ્ટ્રાસોનિક તરંગો ઉત્પન્ન કરે છે
\end{enumerate}

\textbf{ઘટકો}:

\begin{itemize}
\tightlist
\item
  \textbf{ઓસિલેટર}: ઉચ્ચ-આવૃત્તિ વિદ્યુત આંદોલનો ઉત્પન્ન કરે છે
\item
  \textbf{એમ્પ્લિફાયર}: આંદોલનોના કંપવિસ્તાર વધારે છે
\item
  \textbf{પીઝોઇલેક્ટ્રિક ક્રિસ્ટલ}: વિદ્યુત ઊર્જાને યાંત્રિક કંપનોમાં રૂપાંતરિત કરે છે
\item
  \textbf{માઉન્ટિંગ}: ક્રિસ્ટલને યોગ્ય રીતે સપોર્ટ કરે છે
\end{itemize}

\textbf{આવૃત્તિ શ્રેણી}: 20 kHz થી અનેક MHz

\textbf{ફાયદાઓ}:

\begin{itemize}
\tightlist
\item
  ઉચ્ચ કાર્યક્ષમતા
\item
  ચોક્કસ આવૃત્તિ નિયંત્રણ
\item
  નાનો કદ
\item
  કોઈ હલનચલન કરતા ભાગો નથી
\end{itemize}

\end{solutionbox}
\begin{mnemonicbox}
``પીઝો પરિવર્તિત થાય છે જ્યારે વિદ્યુત પ્રવાહ પસાર થાય છે''

\end{mnemonicbox}
\subsection*{પ્રશ્ન 4(ક)(ii) {[}2
ગુણ{]}}\label{uxaaauxab0uxab6uxaa8-4uxa95ii-2-uxa97uxaa3}

\textbf{ધ્વનિ તરંગના કોઈપણ બે ગુણધર્મો સમજાવો.}

\begin{solutionbox}

\textbf{1. ધ્વનિનું પરાવર્તન}:

\begin{itemize}
\tightlist
\item
  ધ્વનિ તરંગો અવરોધકો પરથી પરાવર્તિત થાય છે
\item
  પરાવર્તનના નિયમને અનુસરે છે: આપાત કોણ = પરાવર્તિત કોણ
\item
  દૂરના પદાર્થો પરથી પરાવર્તિત થઈને પડઘો ઉત્પન્ન કરે છે
\item
  ઉપયોગો: સોનાર, પડઘા સ્થાનિકરણ, ધ્વનિક ડિઝાઈન
\end{itemize}

\textbf{2. ધ્વનિનું વક્રીભવન}:

\begin{itemize}
\tightlist
\item
  એક માધ્યમથી બીજા માધ્યમમાં જતાં ધ્વનિ તરંગોનું વળવું
\item
  વિવિધ માધ્યમોમાં ધ્વનિના વેગમાં ફેરફારને કારણે થાય છે
\item
  ઉદાહરણો: ગુંબજોમાં ધ્વનિ કેન્દ્રિત થવી, રાત્રે ધ્વનિ વધુ સારી રીતે સંભળાવી
\item
  ઉપયોગો: ધ્વનિક લેન્સ, મેડિકલ અલ્ટ્રાસાઉન્ડ
\end{itemize}

\textbf{આકૃતિ}:

\begin{verbatim}
પરાવર્તન:        વક્રીભવન:
  \  |              /|
   \ |             / |
    \|            /  |
----|----       ---|---
    |/           /|
    |           / |
    |          /  |
\end{verbatim}

\end{solutionbox}
\begin{mnemonicbox}
``ધ્વનિ પરાવર્તિત થાય, વળાંક લે, પ્રસરણ કરે''

\end{mnemonicbox}
\subsection*{પ્રશ્ન 4(અ) {[}3 ગુણ{]}
(OR)}\label{uxaaauxab0uxab6uxaa8-4uxa85-3-uxa97uxaa3-or}

\textbf{તરંગની તરંગલંબાઇ, કલા અને વેગને વ્યાખ્યાયિત કરો.}

\begin{solutionbox}

\textbf{તરંગલંબાઈ}:

\begin{itemize}
\tightlist
\item
  કલામાં રહેલા બે ક્રમિક બિંદુઓ વચ્ચેનું અંતર
\item
  એક સંપૂર્ણ દોલન દરમિયાન પ્રવાસ કરેલું અંતર
\item
  `λ' (લેમ્ડા) દ્વારા દર્શાવાય છે
\item
  મીટર (m)માં માપવામાં આવે છે
\item
  આવૃત્તિ (f) અને વેગ (v) સાથે સંબંધિત: λ = v/f
\end{itemize}

\textbf{કલા}:

\begin{itemize}
\tightlist
\item
  કોઈ ચોક્કસ બિંદુ અને સમયે દોલનની સ્થિતિ
\item
  રેડિયન અથવા ડિગ્રીમાં માપવામાં આવે છે
\item
  સંપૂર્ણ ચક્ર = 2π રેડિયન અથવા 360°
\item
  એક જ કલા ધરાવતા બિંદુઓ એકસરખા કલામાં છે
\item
  π રેડિયન (180°) થી અલગ હોય તે બિંદુઓ વિરુદ્ધ કલામાં છે
\end{itemize}

\textbf{વેગ}:

\begin{itemize}
\tightlist
\item
  માધ્યમમાં તરંગના પ્રસરણનો દર
\item
  `v' દ્વારા દર્શાવાય છે
\item
  મીટર પ્રતિ સેકંડ (m/s)માં માપવામાં આવે છે
\item
  તરંગલંબાઈ અને આવૃત્તિ સાથે સંબંધિત: v = λf
\item
  તરંગની લાક્ષણિકતાઓ પર નહીં પણ માધ્યમના ગુણધર્મો પર આધાર રાખે છે
\end{itemize}

\textbf{આકૃતિ}:

\begin{verbatim}
    |<---λ--->|
    |         |
    /\        /\        /\
   /  \      /  \      /  \
--/----\----/----\----/----\-->
  \    /    \    /    \    /
   \  /      \  /      \  /
    \/        \/        \/
    
    |---v·t---|
\end{verbatim}

\end{solutionbox}
\begin{mnemonicbox}
``તરંગલંબાઈ એક ચક્રની લંબાઈ, કલા સ્થિતિ બતાવે, વેગ પ્રસરણની
ઝડપ દર્શાવે''

\end{mnemonicbox}
\subsection*{પ્રશ્ન 4(બ) {[}4 ગુણ{]}
(OR)}\label{uxaaauxab0uxab6uxaa8-4uxaac-4-uxa97uxaa3-or}

\textbf{તરંગોની સહાયક અને વિનાશક વ્યતિકરણ સમજાવો.}

\begin{solutionbox}

\textbf{વ્યતિકરણ}: અવકાશમાં એક જ બિંદુ પર બે અથવા વધુ તરંગોના અધ્યારોપણને કારણે
નવી તરંગ પેટર્ન ઉત્પન્ન થવી

\textbf{સહાયક વ્યતિકરણ}:

\begin{itemize}
\tightlist
\item
  જ્યારે તરંગો એકસરખા કલામાં મળે (શિખર સાથે શિખર મળે) ત્યારે થાય છે
\item
  કલા તફાવત = 0, 2π, 4π, \ldots{} (0°, 360°, 720°, \ldots)
\item
  પથ તફાવત = nλ (n = 0, 1, 2, \ldots)
\item
  પરિણામે વ્યક્તિગત તરંગો કરતાં મોટી કંપવિસ્તાર થાય છે
\item
  પરિણામી કંપવિસ્તાર = વ્યક્તિગત કંપવિસ્તારનો સરવાળો
\end{itemize}

\textbf{વિનાશક વ્યતિકરણ}:

\begin{itemize}
\tightlist
\item
  જ્યારે તરંગો વિરુદ્ધ કલામાં મળે (શિખર સાથે ખીણ મળે) ત્યારે થાય છે
\item
  કલા તફાવત = π, 3π, 5π, \ldots{} (180°, 540°, 900°, \ldots)
\item
  પથ તફાવત = (n+1/2)λ (n = 0, 1, 2, \ldots)
\item
  પરિણામે વ્યક્તિગત તરંગો કરતાં નાની કંપવિસ્તાર થાય છે
\item
  જો કંપવિસ્તાર સમાન હોય તો સંપૂર્ણ રદ્દીકરણ થાય છે
\end{itemize}

\textbf{આકૃતિ}:

\begin{verbatim}
સહાયક:                 વિનાશક:
  /\    /\                  /\    \/
 /  \  /  \                /  \  /  \
/    \/    \              /    \/    \
--------------            --------------
      |                         |
      \/                        |
     /  \                       |
    /    \                 --------- 
   /      \
  /        \
\end{verbatim}

\end{solutionbox}
\begin{mnemonicbox}
``સહાયક શિખર-શિખર મેળવે; વિનાશક શિખર-ખીણ મેળવે''

\end{mnemonicbox}
\subsection*{પ્રશ્ન 4(ક)(i) {[}5 ગુણ{]}
(OR)}\label{uxaaauxab0uxab6uxaa8-4uxa95i-5-uxa97uxaa3-or}

\textbf{મેગ્નેટોસ્ટ્રિક્શન પદ્ધતિનો ઉપયોગ કરીને અલ્ટ્રાસોનિક તરંગ કેવી રીતે ઉત્પન્ન થાય
છે?}

\begin{solutionbox}

\textbf{મેગ્નેટોસ્ટ્રિક્શન પદ્ધતિ દ્વારા અલ્ટ્રાસોનિક તરંગ ઉત્પન્ન કરવી}:

\begin{center}
\textbf{Mermaid Diagram (Code)}
\begin{verbatim}
{Shaded}
{Highlighting}[]
graph LR
    A[ઓસિલેટર] {-{-}{} B[એમ્પ્લિફાયર]}
    B {-{-}{} C[ફેરોમેગ્નેટિક રોડ ફરતે કોઈલ]}
    C {-{-}{} D[મેગ્નેટોસ્ટ્રિક્ટિવ રોડના કંપન]}
    D {-{-}{} E[અલ્ટ્રાસોનિક તરંગો]}
{Highlighting}
{Shaded}
\end{verbatim}
\end{center}

\textbf{કાર્યસિદ્ધાંત}:

\begin{enumerate}
\def\labelenumi{\arabic{enumi}.}
\tightlist
\item
  મેગ્નેટોસ્ટ્રિક્શન ઇફેક્ટ પર આધારિત - ચુંબકીય ક્ષેત્રમાં મૂકતાં ફેરોમેગ્નેટિક પદાર્થોમાં
  પરિમાણમાં ફેરફાર
\item
  જ્યારે ચુંબકીય ક્ષેત્ર લાગુ કરવામાં આવે, ત્યારે રોડ સંકોચાય છે
\item
  જ્યારે ક્ષેત્ર દૂર કરવામાં આવે, ત્યારે રોડ તેના મૂળ કદ સુધી પાછો ફેલાય છે
\item
  પ્રત્યાવર્તી પ્રવાહ પ્રત્યાવર્તી ચુંબકીય ક્ષેત્ર ઉત્પન્ન કરે છે
\item
  રોડ લાગુ કરેલ પ્રવાહની આવૃત્તિએ કંપન કરે છે
\item
  આ કંપનો અલ્ટ્રાસોનિક તરંગો ઉત્પન્ન કરે છે
\end{enumerate}

\textbf{ઘટકો}:

\begin{itemize}
\tightlist
\item
  \textbf{ઓસિલેટર}: ઉચ્ચ-આવૃત્તિ વિદ્યુત આંદોલનો ઉત્પન્ન કરે છે
\item
  \textbf{એમ્પ્લિફાયર}: આંદોલનોના કંપવિસ્તાર વધારે છે
\item
  \textbf{કોઈલ}: પ્રવાહ પસાર થાય ત્યારે ચુંબકીય ક્ષેત્ર ઉત્પન્ન કરે છે
\item
  \textbf{ફેરોમેગ્નેટિક રોડ}: નિકલ, આયર્ન-નિકલ મિશ્રધાતુ, અથવા ફેરાઇટ્સ
\item
  \textbf{માઉન્ટિંગ}: રોડને યોગ્ય રીતે સપોર્ટ કરે છે
\end{itemize}

\textbf{આવૃત્તિ શ્રેણી}: 20 kHz થી 100 kHz (પીઝોઇલેક્ટ્રિક પદ્ધતિ કરતાં ઓછી)

\textbf{ફાયદાઓ}:

\begin{itemize}
\tightlist
\item
  ઉચ્ચ પાવર ને હેન્ડલ કરે છે
\item
  ઉચ્ચ-તીવ્રતાવાળા ઉપયોગો માટે યોગ્ય
\item
  મજબૂત બાંધણી
\item
  નીચી આવૃત્તિઓ પર સારી રીતે કામ કરે છે
\end{itemize}

\textbf{મર્યાદાઓ}:

\begin{itemize}
\tightlist
\item
  નીચી આવૃત્તિઓ સુધી મર્યાદિત
\item
  પીઝોઇલેક્ટ્રિક પદ્ધતિ કરતાં ઓછી કાર્યક્ષમતા
\item
  ઉચ્ચ આવૃત્તિઓએ રોડનું હીટિંગ
\end{itemize}

\end{solutionbox}
\begin{mnemonicbox}
``ચુંબકત્વથી સામગ્રી ફેરફાર પામે, અલ્ટ્રાસોનિક તરંગો બનાવે''

\end{mnemonicbox}
\subsection*{પ્રશ્ન 4(ક)(ii) {[}2 ગુણ{]}
(OR)}\label{uxaaauxab0uxab6uxaa8-4uxa95ii-2-uxa97uxaa3-or}

\textbf{પ્રકાશ તરંગના કોઈપણ બે ગુણધર્મો સમજાવો.}

\begin{solutionbox}

\textbf{1. પ્રકાશનું પરાવર્તન}:

\begin{itemize}
\tightlist
\item
  જ્યારે પ્રકાશ સપાટી પર પડે ત્યારે પાછો ફેંકાય છે
\item
  પરાવર્તનના નિયમને અનુસરે છે: આપાત કોણ = પરાવર્તિત કોણ
\item
  સ્મૂધ સપાટીઓ પરથી નિયમિત પરાવર્તન
\item
  ખરબચડી સપાટીઓ પરથી વિસરિત પરાવર્તન
\item
  ઉપયોગો: અરીસા, રિફ્લેક્ટર્સ, ઓપ્ટિકલ ઉપકરણો
\end{itemize}

\textbf{2. પ્રકાશનું વક્રીભવન}:

\begin{itemize}
\tightlist
\item
  એક માધ્યમથી બીજા માધ્યમમાં જતાં પ્રકાશનું વળવું
\item
  સ્નેલના નિયમને અનુસરે છે: n₁sin(θ₁) = n₂sin(θ₂)
\item
  વિવિધ માધ્યમોમાં પ્રકાશની ઝડપમાં ફેરફારને કારણે થાય છે
\item
  ઉદાહરણો: પાણીમાં લાકડીનો વાંકો દેખાવ
\item
  ઉપયોગો: લેન્સ, પ્રિઝમ, ઓપ્ટિકલ ફાઈબર
\end{itemize}

\textbf{આકૃતિ}:

\begin{verbatim}
પરાવર્તન:        વક્રીભવન:
  \  |              /|
   \ |             / |
    \|            /  |
----|----       ---|---
    |/           /|
    |           / |
    |          /  |
\end{verbatim}

\end{solutionbox}
\begin{mnemonicbox}
``પ્રકાશ અરીસામાં પરાવર્તિત થાય અને માધ્યમમાં વક્રીભવન
પામે''

\end{mnemonicbox}
\subsection*{પ્રશ્ન 5(અ) {[}3
ગુણ{]}}\label{uxaaauxab0uxab6uxaa8-5uxa85-3-uxa97uxaa3}

\textbf{લેસરની લાક્ષણિકતાઓ લખો.}

\begin{solutionbox}

\textbf{લેસરની લાક્ષણિકતાઓ}:

{\def\LTcaptype{none} % do not increment counter
\begin{longtable}[]{@{}ll@{}}
\toprule\noalign{}
લાક્ષણિકતા & વર્ણન \\
\midrule\noalign{}
\endhead
\bottomrule\noalign{}
\endlastfoot
એકવર્ણીય & એક તરંગલંબાઈ/રંગ (ખૂબ સાંકડી આવૃત્તિ શ્રેણી) \\
સંસક્ત & બધા તરંગો એક જ કલામાં, ઉચ્ચ વ્યતિકરણ ઉત્પન્ન કરે છે \\
દિશાત્મક & અત્યંત સમાંતર, લાંબા અંતર પર પણ નહીવત વિસરણ \\
ઉચ્ચ તીવ્રતા & સાંકડા કિરણમાં કેન્દ્રિત ઊર્જા \\
ઉચ્ચ શુદ્ધતા & સામાન્ય પ્રકાશની તુલનામાં અત્યંત શુદ્ધ રંગ \\
\end{longtable}
}

\textbf{આકૃતિ}:

\begin{verbatim}
સામાન્ય પ્રકાશ:            લેસર:
  ---                        -------
 /   \                      |       |
/     \                     |       |
-------                     |       |
વિવિધ તરંગલંબાઈઓ            -------
અને દિશાઓ                 એક તરંગલંબાઈ,
                          એક દિશા
\end{verbatim}

\end{solutionbox}
\begin{mnemonicbox}
``લેસર પ્રકાશ: એકવર્ણીય, સંસક્ત, દિશાત્મક, તીવ્ર''

\end{mnemonicbox}
\subsection*{પ્રશ્ન 5(બ) {[}4
ગુણ{]}}\label{uxaaauxab0uxab6uxaa8-5uxaac-4-uxa97uxaa3}

\textbf{એન્જિનિયરિંગ અને મેડિકલ ક્ષેત્રે લેસરના મહત્વની ચર્ચા કરો.}

\begin{solutionbox}

\textbf{એન્જિનિયરિંગ ક્ષેત્રે લેસરનું મહત્વ}:

\begin{enumerate}
\def\labelenumi{\arabic{enumi}.}
\tightlist
\item
  \textbf{ઉત્પાદન}:

  \begin{itemize}
  \tightlist
  \item
    ધાતુઓનું ચોક્કસ કટિંગ અને વેલ્ડિંગ
  \item
    3D પ્રિન્ટિંગ અને રેપિડ પ્રોટોટાઇપિંગ
  \item
    સામગ્રી પર એન્ગ્રેવિંગ અને માર્કિંગ
  \end{itemize}
\item
  \textbf{માપન અને પરીક્ષણ}:

  \begin{itemize}
  \tightlist
  \item
    અંતર માપન (LIDAR)
  \item
    એલાઇનમેન્ટ અને લેવલિંગ
  \item
    નોન-ડિસ્ટ્રક્ટિવ ટેસ્ટિંગ
  \item
    હોલોગ્રાફી દ્વારા સ્ટ્રેસ એનાલિસિસ
  \end{itemize}
\item
  \textbf{સંચાર}:

  \begin{itemize}
  \tightlist
  \item
    ફાઇબર ઓપ્ટિક સંચાર
  \item
    ફ્રી-સ્પેસ ઓપ્ટિકલ કમ્યુનિકેશન
  \item
    ડેટા સ્ટોરેજ (CD/DVD/Blu-ray)
  \end{itemize}
\item
  \textbf{સામગ્રી પ્રક્રિયા}:

  \begin{itemize}
  \tightlist
  \item
    હીટ ટ્રીટમેન્ટ
  \item
    સપાટી સખત કરવી
  \item
    માઇક્રોમશીનિંગ
  \end{itemize}
\end{enumerate}

\textbf{મેડિકલ ક્ષેત્રે લેસરનું મહત્વ}:

\begin{enumerate}
\def\labelenumi{\arabic{enumi}.}
\tightlist
\item
  \textbf{સર્જરી}:

  \begin{itemize}
  \tightlist
  \item
    રક્તવિહીન કટિંગ (લેસર સ્કાલ્પેલ)
  \item
    નેત્ર સર્જરી (LASIK)
  \item
    ત્વચાને લગતી પ્રક્રિયાઓ
  \item
    ટ્યુમર દૂર કરવા
  \end{itemize}
\item
  \textbf{નિદાન}:

  \begin{itemize}
  \tightlist
  \item
    લેસર ઇમેજિંગ
  \item
    સ્પેક્ટ્રોસ્કોપી
  \item
    ફ્લો સાયટોમેટ્રી
  \item
    ઓપ્ટિકલ કોહેરન્સ ટોમોગ્રાફી
  \end{itemize}
\item
  \textbf{થેરાપી}:

  \begin{itemize}
  \tightlist
  \item
    કેન્સર માટે ફોટોડાયનેમિક થેરાપી
  \item
    લો-લેવલ લેસર થેરાપી
  \item
    દર્દ વ્યવસ્થાપન
  \item
    કોસ્મેટિક પ્રક્રિયાઓ (વાળ દૂર કરવા, ત્વચા યુવાન બનાવવા)
  \end{itemize}
\item
  \textbf{દંત ચિકિત્સા}:

  \begin{itemize}
  \tightlist
  \item
    ખાડા શોધવા
  \item
    દાંત સફેદ કરવા
  \item
    પેઢાની સર્જરી
  \end{itemize}
\end{enumerate}

\end{solutionbox}
\begin{mnemonicbox}
``લેસર ઉત્પાદન સુધારે, માપન કરે, માહિતી મોકલે, દર્દીઓને સારા
કરે''

\end{mnemonicbox}
\subsection*{પ્રશ્ન 5(ક)(i) {[}5
ગુણ{]}}\label{uxaaauxab0uxab6uxaa8-5uxa95i-5-uxa97uxaa3}

\textbf{લેસરના ઉત્પાદન માટે પોપ્યુલેશન ઇન્વર્જન અને મેટાસ્ટેબલ સ્ટેટનું શું મહત્વ છે?}

\begin{solutionbox}

\textbf{પોપ્યુલેશન ઇન્વર્જન}:

\begin{itemize}
\tightlist
\item
  \textbf{વ્યાખ્યા}: એવી સ્થિતિ જ્યાં ગ્રાઉન્ડ સ્ટેટ કરતાં ઉત્તેજિત અવસ્થામાં વધારે અણુઓ
  હોય (સામાન્ય સમતુલનથી ઉલટું)
\item
  \textbf{મહત્વ}:

  \begin{enumerate}
  \def\labelenumi{\arabic{enumi}.}
  \tightlist
  \item
    લેસર ક્રિયા માટે આવશ્યક શરત
  \item
    ઉત્તેજિત ઉત્સર્જન શોષણ કરતાં પ્રભાવશાળી બને તેવું વાતાવરણ બનાવે છે
  \item
    પ્રકાશનું વર્ધન સક્ષમ બનાવે છે (નકારાત્મક શોષણ)
  \item
    તેના વગર, ઉત્સર્જિત ફોટોન શોષિત થશે, જે લેસર ક્રિયાને અટકાવશે
  \item
    ઉત્તેજિત ઉત્સર્જનની શ્રૃંખલા પ્રતિક્રિયા માટે જરૂરી છે
  \end{enumerate}
\end{itemize}

\textbf{આકૃતિ}:

\begin{verbatim}
સામાન્ય:              પોપ્યુલેશન ઇન્વર્જન:
  -----                  -----
  |   | થોડા અણુઓ       ||||||| ઘણા અણુઓ
  -----                  -----
     ↑                      ↓
     |                      |
  -----                  -----
  ||||||| ઘણા અણુઓ        |   | થોડા અણુઓ
  -----                  -----
ગ્રાઉન્ડ સ્ટેટ           ગ્રાઉન્ડ સ્ટેટ
\end{verbatim}

\textbf{મેટાસ્ટેબલ સ્ટેટ}:

\begin{itemize}
\tightlist
\item
  \textbf{વ્યાખ્યા}: પ્રમાણમાં લાંબા જીવનકાળ (10⁻³ થી 10⁻⁷ સેકન્ડ) વાળી ઉત્તેજિત
  ઊર્જા અવસ્થા
\item
  \textbf{મહત્વ}:

  \begin{enumerate}
  \def\labelenumi{\arabic{enumi}.}
  \tightlist
  \item
    ઉત્તેજિત અણુઓનું સંચય કરવાની મંજૂરી આપે છે (અસ્થાયી ઊર્જા સંગ્રહ)
  \item
    પોપ્યુલેશન ઇન્વર્જન સ્થાપિત કરવા માટે સમય આપે છે
  \item
    લાંબો જીવનકાળ ઝડપી સ્વયંસ્ફૂર્ત ઉત્સર્જનને રોકે છે
  \item
    સ્વયંસ્ફૂર્ત ઉત્સર્જન કરતાં ઉત્તેજિત ઉત્સર્જન પ્રભાવશાળી થાય તે સુનિશ્ચિત કરે છે
  \item
    સતત લેસર કાર્ય માટે આવશ્યક છે
  \end{enumerate}
\end{itemize}

\textbf{ઊર્જા સ્તર આકૃતિ}:

\begin{verbatim}
          |
  E₃ ----|---- ટૂંકા-જીવનકાળ વાળી ઉત્તેજિત અવસ્થા
          |
          v ઝડપી સંક્રમણ (નોન-રેડિયેટિવ)
          |
  E₂ ----|---- મેટાસ્ટેબલ સ્ટેટ (લાંબો જીવનકાળ)
          |
          v ઉત્તેજિત ઉત્સર્જન (લેસર)
          |
  E₁ ----|---- ગ્રાઉન્ડ સ્ટેટ
\end{verbatim}

\end{solutionbox}
\begin{mnemonicbox}
``પોપ્યુલેશન ઇન્વર્જન ઉત્તેજિત અવસ્થામાં અણુઓ રાખે; મેટાસ્ટેબલ સ્ટેટ
આ પરિસ્થિતિ લાંબો સમય ટકાવે''

\end{mnemonicbox}
\subsection*{પ્રશ્ન 5(ક)(ii) {[}2
ગુણ{]}}\label{uxaaauxab0uxab6uxaa8-5uxa95ii-2-uxa97uxaa3}

\textbf{ગ્રેડેડ ઈન્ડેક્સ ઓપ્ટિકલ ફાઈબર સમજાવો.}

\begin{solutionbox}

\textbf{ગ્રેડેડ ઈન્ડેક્સ ઓપ્ટિકલ ફાઈબર}:

\begin{itemize}
\tightlist
\item
  \textbf{બંધારણ}: કેન્દ્રથી પરિધિ તરફ ક્રમશઃ ઘટતા રીફ્રેક્ટિવ ઈન્ડેક્સ સાથેનો કોર
\item
  \textbf{રીફ્રેક્ટિવ ઈન્ડેક્સ પ્રોફાઈલ}: પરવલયિક પેટર્નને અનુસરે છે: n(r) = n₁(1 -
  αr²)
\item
  \textbf{પ્રકાશનું પ્રસરણ}: પ્રકાશ ઝિગઝાગ પેટર્નને બદલે વક્ર માર્ગોમાં પ્રવાસ કરે છે
\item
  \textbf{પદ્ધતિ}: પરિધિની નજીકનો પ્રકાશ કેન્દ્ર કરતાં વધુ ઝડપથી પ્રવાસ કરે છે, જે
  લાંબા માર્ગ માટે વળતર આપે છે
\item
  \textbf{ફાયદાઓ}:

  \begin{enumerate}
  \def\labelenumi{\arabic{enumi}.}
  \tightlist
  \item
    સ્ટેપ ઈન્ડેક્સ ફાઈબરની તુલનામાં ઓછું મોડલ ડિસ્પર્શન
  \item
    ઉચ્ચ બેન્ડવિડ્થ
  \item
    ઓછું સિગ્નલ વિકૃતિકરણ
  \item
    મધ્યમ અંતરના સંચાર માટે યોગ્ય
  \end{enumerate}
\end{itemize}

\textbf{આડછેદન આકૃતિ}:

\begin{verbatim}
       ક્લેડિંગ
    ┌───────────┐
    │ ╭───────╮ │
    │ │       │ │
    │ │ કોર   │ │
    │ │       │ │
    │ ╰───────╯ │
    └───────────┘

રીફ્રેક્ટિવ ઈન્ડેક્સ પ્રોફાઈલ:
    │     ╱╲
    │    /  \
n   │   /    \
    │  /      \
    │ /        \
    │/          \
    └────────────
      અંતર
\end{verbatim}

\end{solutionbox}
\begin{mnemonicbox}
``ગ્રેડેડ ઈન્ડેક્સ ક્રમશઃ કોરમાંથી ક્લેડિંગની તરફ બદલાય,
ડિસ્પર્શન ઘટાડે''

\end{mnemonicbox}
\subsection*{પ્રશ્ન 5(અ) {[}3 ગુણ{]}
(OR)}\label{uxaaauxab0uxab6uxaa8-5uxa85-3-uxa97uxaa3-or}

\textbf{પ્રકાશનું વક્રીભવનને વ્યાખ્યાયિત કરો અને સ્નેલનો નિયમ લખો.}

\begin{solutionbox}

\textbf{પ્રકાશનું વક્રીભવન}:

\begin{itemize}
\tightlist
\item
  જ્યારે પ્રકાશ એક પારદર્શક માધ્યમથી બીજા માધ્યમમાં પસાર થાય ત્યારે તેનું વળવું
\item
  વિવિધ માધ્યમોમાં પ્રકાશની ગતિમાં ફેરફારને કારણે થાય છે
\item
  દિશા બદલાય છે પરંતુ આવૃત્તિ એક સરખી રહે છે
\item
  ઝડપ સાથે તરંગલંબાઈ બદલાય છે
\end{itemize}

\textbf{સ્નેલનો નિયમ}:

\begin{itemize}
\tightlist
\item
  વક્રીભવનને નિયંત્રિત કરતો ગાણિતિક સંબંધ
\item
  આપાત અને વક્રીભવન કોણના સાઇન્સનો ગુણોત્તર રીફ્રેક્ટિવ ઈન્ડેક્સના ગુણોત્તર સાથે સમાન
  હોય છે
\item
  સૂત્ર: n₁sin(θ₁) = n₂sin(θ₂)
\item
  જ્યાં:

  \begin{itemize}
  \tightlist
  \item
    n₁ = પ્રથમ માધ્યમનો રીફ્રેક્ટિવ ઈન્ડેક્સ
  \item
    n₂ = બીજા માધ્યમનો રીફ્રેક્ટિવ ઈન્ડેક્સ
  \item
    θ₁ = આપાત કોણ
  \item
    θ₂ = વક્રીભવન કોણ
  \end{itemize}
\end{itemize}

\textbf{આકૃતિ}:

\begin{verbatim}
માધ્યમ 1 (n₁)
    \   |
     \  | θ₁
      \ |
-------|---------
       |\ 
       | \ θ₂
       |  \
માધ્યમ 2 (n₂)
\end{verbatim}

\end{solutionbox}
\begin{mnemonicbox}
``સાઇન ગુણોત્તર ઈન્ડેક્સ ગુણોત્તર સાથે સમાન'' અથવા ``n₁Sin₁
= n₂Sin₂''

\end{mnemonicbox}
\subsection*{પ્રશ્ન 5(બ) {[}4 ગુણ{]}
(OR)}\label{uxaaauxab0uxab6uxaa8-5uxaac-4-uxa97uxaa3-or}

\textbf{ઇજનેરી અને મેડિકલ ક્ષેત્રે ઓપ્ટિકલ ફાઈબરના મહત્વની ચર્ચા કરો.}

\begin{solutionbox}

\textbf{ઇજનેરી ક્ષેત્રે ઓપ્ટિકલ ફાઈબરનું મહત્વ}:

\begin{enumerate}
\def\labelenumi{\arabic{enumi}.}
\tightlist
\item
  \textbf{સંચાર}:

  \begin{itemize}
  \tightlist
  \item
    હાઈ-સ્પીડ ઇન્ટરનેટ ટ્રાન્સમિશન
  \item
    લાંબા અંતર દૂરસંચાર
  \item
    સુરક્ષિત ડેટા ટ્રાન્સમિશન (ટેપ કરવામાં મુશ્કેલ)
  \item
    કોપર કેબલ્સ કરતાં વધુ બેન્ડવિડ્થ
  \end{itemize}
\item
  \textbf{સેન્સર્સ અને ઇન્સ્ટ્રુમેન્ટેશન}:

  \begin{itemize}
  \tightlist
  \item
    તાપમાન, દબાણ, તાણ માપન
  \item
    સ્ટ્રક્ચરલ હેલ્થ મોનિટરિંગ
  \item
    રાસાયણિક અને જૈવિક સેન્સિંગ
  \item
    ભૂકંપીય શોધ
  \end{itemize}
\item
  \textbf{ઔદ્યોગિક ઉપયોગો}:

  \begin{itemize}
  \tightlist
  \item
    જોખમી વિસ્તારોનું દૂરસ્થ નિરીક્ષણ
  \item
    ઔદ્યોગિક પ્રક્રિયા નિયંત્રણ
  \item
    પાવર સિસ્ટમ મોનિટરિંગ
  \item
    ખનન અને પેટ્રોલિયમ શોધ
  \end{itemize}
\item
  \textbf{કમ્પ્યુટિંગ}:

  \begin{itemize}
  \tightlist
  \item
    ઘટકો વચ્ચે હાઈ-સ્પીડ ડેટા ટ્રાન્સફર
  \item
    ઓપ્ટિકલ ઇન્ટરકનેક્ટ્સ
  \item
    ક્વોન્ટમ કમ્પ્યુટિંગ જોડાણો
  \end{itemize}
\end{enumerate}

\textbf{મેડિકલ ક્ષેત્રે ઓપ્ટિકલ ફાઈબરનું મહત્વ}:

\begin{enumerate}
\def\labelenumi{\arabic{enumi}.}
\tightlist
\item
  \textbf{નિદાન}:

  \begin{itemize}
  \tightlist
  \item
    આંતરિક અંગોના પરીક્ષણ માટે એન્ડોસ્કોપી
  \item
    ન્યૂનતમ આક્રમક સર્જરી માટે લેપ્રોસ્કોપી
  \item
    રક્તવાહિનીઓના પરીક્ષણ માટે એન્જિયોસ્કોપી
  \item
    શ્વસન માર્ગના પરીક્ષણ માટે બ્રોન્કોસ્કોપી
  \end{itemize}
\item
  \textbf{સર્જરી}:

  \begin{itemize}
  \tightlist
  \item
    ચોક્કસ ઓપરેશન માટે લેસર પ્રકાશ વિતરણ
  \item
    ફોટોડાયનેમિક થેરાપી
  \item
    માઇક્રોસર્જરી માર્ગદર્શન
  \item
    દૂરસ્થ સર્જરી મોનિટરિંગ
  \end{itemize}
\item
  \textbf{ઇમેજિંગ}:

  \begin{itemize}
  \tightlist
  \item
    ઓપ્ટિકલ કોહેરન્સ ટોમોગ્રાફી
  \item
    કોન્ફોકલ માઇક્રોસ્કોપી
  \item
    ઓપ્ટોજેનેટિક્સ
  \item
    મેડિકલ સ્પેક્ટ્રોસ્કોપી
  \end{itemize}
\item
  \textbf{સારવાર}:

  \begin{itemize}
  \tightlist
  \item
    ત્વચાની સ્થિતિઓ માટે ફોટોથેરાપી
  \item
    લેસર સારવાર વિતરણ
  \item
    રિયલ-ટાઇમ મોનિટરિંગ માટે બાયોસેન્સિંગ
  \item
    લક્ષિત દવા વિતરણ
  \end{itemize}
\end{enumerate}

\end{solutionbox}
\begin{mnemonicbox}
``ઓપ્ટિકલ ફાઈબર જોડાણ કરે, માપે, દેખાડે, સારવાર કરે''

\end{mnemonicbox}
\subsection*{પ્રશ્ન 5(ક)(i) {[}5 ગુણ{]}
(OR)}\label{uxaaauxab0uxab6uxaa8-5uxa95i-5-uxa97uxaa3-or}

\textbf{ઓપ્ટિકલ ફાઈબરની ન્યુમરિકલ એપર્ચર અને એસેપટન્સ એન્ગલ માટે સૂત્ર મેળવો.}

\begin{solutionbox}

\textbf{ન્યુમરિકલ એપર્ચર (NA) તારણ}:

\begin{center}
\textbf{Mermaid Diagram (Code)}
\begin{verbatim}
{Shaded}
{Highlighting}[]
graph LR
    A[કોર{-ક્લેડિંગ ઇન્ટરફેસ પર ક્રિટિકલ એંગલ વિચારો] {-}{-}{} B[સ્નેલનો નિયમ લાગુ કરો]}
    B {-{-}{} C[એસેપટન્સ એંગલ સાથે સંબંધિત કરો]}
    C {-{-}{} D[NA સૂત્ર તારણ કરો]}
{Highlighting}
{Shaded}
\end{verbatim}
\end{center}

\textbf{સ્ટેપ 1}: કોર-ક્લેડિંગ ઇન્ટરફેસ પર ક્રિટિકલ એંગલ (θc) વિચારો

\begin{itemize}
\tightlist
\item
  ક્રિટિકલ એંગલ પર, વક્રીભવન પામેલા કિરણ ઇન્ટરફેસ સાથે ઘસાય છે
\item
  sin(θc) = n₂/n₁ (જ્યાં n₁ = કોર ઈન્ડેક્સ, n₂ = ક્લેડિંગ ઈન્ડેક્સ)
\end{itemize}

\textbf{સ્ટેપ 2}: કોરમાં પ્રવાસ કરતા કિરણ માટે, સંપૂર્ણ આંતરિક પરાવર્તન માટેની શરત
લાગુ કરો

\begin{itemize}
\tightlist
\item
  કિરણ ક્રિટિકલ એંગલ કરતાં મોટા ખૂણે પડવું જોઈએ
\item
  કોરમાં મહત્તમ ખૂણો: 90° - θc
\end{itemize}

\textbf{સ્ટેપ 3}: હવામાંથી (n₀ = 1) દાખલ થતા કિરણ માટે, સ્નેલનો નિયમ લાગુ કરો

\begin{itemize}
\tightlist
\item
  n₀sin(θₐ) = n₁sin(θ₁)
\item
  sin(θₐ) = n₁sin(θ₁)
\item
  જ્યાં θₐ એસેપટન્સ એંગલ છે
\end{itemize}

\textbf{સ્ટેપ 4}: θ₁ નું મહત્તમ મૂલ્ય (90° - θc) વાપરો

\begin{itemize}
\tightlist
\item
  sin(θₐ) = n₁sin(90° - θc) = n₁cos(θc)
\end{itemize}

\textbf{સ્ટેપ 5}: sin(θc) = n₂/n₁ પ્રતિસ્થાપિત કરો

\begin{itemize}
\tightlist
\item
  cos(θc) = √(1 - sin²(θc)) = √(1 - (n₂/n₁)²)
\end{itemize}

\textbf{સ્ટેપ 6}: તેથી:

\begin{itemize}
\tightlist
\item
  sin(θₐ) = n₁√(1 - (n₂/n₁)²) = √(n₁² - n₂²)
\end{itemize}

\textbf{અંતિમ સૂત્ર}:

\begin{itemize}
\tightlist
\item
  ન્યુમરિકલ એપર્ચર (NA) = sin(θₐ) = √(n₁² - n₂²)
\item
  જ્યાં θₐ એસેપટન્સ એંગલ છે
\end{itemize}

\textbf{આકૃતિ}:

\begin{verbatim}
              θₐ
               \
 હવા (n₀)       \
-----------------\-----------
                  \
 કોર (n₁)         \_____ θ₁
                    \
                     \
----------------------------
 ક્લેડિંગ (n₂)
\end{verbatim}

\end{solutionbox}
\begin{mnemonicbox}
``NA એસેપટન્સ એંગલ સૂચવે; √(n₁² - n₂²) મહત્તમ સાઇન દર્શાવે''

\end{mnemonicbox}
\subsection*{પ્રશ્ન 5(ક)(ii) {[}2 ગુણ{]}
(OR)}\label{uxaaauxab0uxab6uxaa8-5uxa95ii-2-uxa97uxaa3-or}

\textbf{સ્ટેપ ઈન્ડેક્સ ઓપ્ટિકલ ફાઈબર સમજાવો.}

\begin{solutionbox}

\textbf{સ્ટેપ ઈન્ડેક્સ ઓપ્ટિકલ ફાઈબર}:

\begin{itemize}
\tightlist
\item
  \textbf{બંધારણ}: એકસમાન રીફ્રેક્ટિવ ઈન્ડેક્સ ધરાવતો કોર જે ઓછા એકસમાન રીફ્રેક્ટિવ
  ઈન્ડેક્સ ધરાવતા ક્લેડિંગથી ઘેરાયેલો હોય છે
\item
  \textbf{રીફ્રેક્ટિવ ઈન્ડેક્સ પ્રોફાઈલ}: કોર અને ક્લેડિંગ વચ્ચે અણધારી સંક્રમણ (પગથિયું)
\item
  \textbf{પ્રકાશનું પ્રસરણ}: પ્રકાશ સંપૂર્ણ આંતરિક પરાવર્તન દ્વારા ઝિગઝાગ માર્ગમાં
  પ્રવાસ કરે છે
\item
  \textbf{પ્રકારો}:

  \begin{enumerate}
  \def\labelenumi{\arabic{enumi}.}
  \tightlist
  \item
    સિંગલ-મોડ: નાનો કોર (8-10 μm), પ્રકાશનો એક મોડ વહન કરે છે
  \item
    મલ્ટિ-મોડ: મોટો કોર (50-100 μm), પ્રકાશના ઘણા મોડ્સ વહન કરે છે
  \end{enumerate}
\end{itemize}

\textbf{લાક્ષણિકતાઓ}:

\begin{itemize}
\tightlist
\item
  સરળ બંધારણ
\item
  ગ્રેડેડ ઈન્ડેક્સ કરતાં ઓછી બેન્ડવિડ્થ
\item
  મલ્ટિ-મોડમાં મોડલ ડિસ્પર્શનથી પીડાય છે
\item
  કેટલાક કિરણો માટે લાંબો માર્ગ પલ્સ ફેલાવાનું કારણ બને છે
\end{itemize}

\textbf{આડછેદન આકૃતિ}:

\begin{verbatim}
       ક્લેડિંગ (n₂)
    ┌───────────┐
    │ ┌───────┐ │
    │ │       │ │
    │ │ કોર    │ │
    │ │ (n₁)  │ │
    │ └───────┘ │
    └───────────┘

રીફ્રેક્ટિવ ઈન્ડેક્સ પ્રોફાઈલ:
    │ ┌──┐
    │ │  │
n   │ │  │
    │ │  │
    │ │  │
    │ └──┘
    └─────────
      અંતર
\end{verbatim}

\end{solutionbox}
\begin{mnemonicbox}
``સ્ટેપ ઈન્ડેક્સ ફાઈબરમાં અચાનક પરિવર્તન થાય ઈન્ડેક્સમાં''

\end{mnemonicbox}
\end{document}