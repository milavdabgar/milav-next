\documentclass[10pt,a4paper]{article}

% content/resources/templates/preamble.tex
\usepackage[margin=0.6in]{geometry}
\author{Milav Dabgar}
\usepackage{amsmath,amssymb,amsthm}
\usepackage{booktabs}
\usepackage{multirow}
\usepackage{xcolor}
\usepackage{tcolorbox}
\tcbuselibrary{breakable,skins}
\usepackage[colorlinks=true,linkcolor=blue]{hyperref}
\usepackage{titlesec}
\usepackage{enumitem}
\usepackage{tikz}
\usepackage{pgfplots}
\usepackage{circuitikz}
\usepackage[version=4]{mhchem}
\usepackage{longtable}
\usepackage{array}
\usepackage{float}
\usepackage{caption}
\usepackage{listings}

\lstset{
  basicstyle=\small\ttfamily,
  breaklines=true,
  breakatwhitespace=false,
  postbreak=\mbox{\textcolor{red}{$\hookrightarrow$}\space},
  float=false,
  numbers=left,
  numberstyle=\tiny\color{gray},
  numbersep=10pt,
  xleftmargin=2em,
  keywordstyle=\color{blue},
  commentstyle=\color{green!60!black},
  stringstyle=\color{purple},
  backgroundcolor=\color{gray!5},
  showstringspaces=false,
  tabsize=2,
  captionpos=b,
  keepspaces=true,
  columns=flexible
}

\pgfplotsset{compat=1.18}
\usetikzlibrary{shapes,arrows,positioning,calc,patterns,decorations.pathmorphing,decorations.markings,arrows.meta}

% Color scheme
\definecolor{headcolor}{RGB}{0,102,204}
\definecolor{keycolor}{RGB}{220,20,60}
\definecolor{solutioncolor}{RGB}{34,139,34}
\definecolor{mnemoniccolor}{RGB}{148,0,211}
\definecolor{codecolor}{RGB}{0,0,100}

% Spacing
\setlength{\parskip}{3pt}
\setlist[itemize]{nosep}
\setlist[enumerate]{nosep}

% Title formatting
\titleformat{\section}{\Large\bfseries\color{headcolor}}{\thesection}{1em}{}
\titleformat{\subsection}{\large\bfseries\color{headcolor}}{\thesubsection}{1em}{}

% Pandoc tightlist compatibility
\providecommand{\tightlist}{%
  \setlength{\itemsep}{0pt}\setlength{\parskip}{0pt}}

% Pandoc longtable compatibility
\newcounter{none}
\def\thenone{}


% content/resources/templates/english-boxes.tex
% This file is currently empty - it exists to maintain consistency with the import structure.
% Add custom environments here if needed in the future.


\begin{document}

\begin{center}
{\Huge\bfseries\color{headcolor} Subject Name Solutions}\\[5pt]
{\LARGE 4300005 -- Summer 2024}\\[3pt]
{\large Semester 1 Study Material}\\[3pt]
{\normalsize\textit{Detailed Solutions and Explanations}}
\end{center}

\vspace{10pt}

\subsection*{Question 1(a) [3 marks]}\label{q1a}

\textbf{Define derived physical quantities and give three examples with
their S.I. unit and symbol.}

\begin{solutionbox}
Derived physical quantities are those which are
obtained by multiplication or division of fundamental physical
quantities.


\vspace{-5pt}
\captionof{table}{Examples of Derived Physical Quantities}
\vspace{-10pt}
\begin{longtable}[]{@{}lll@{}}
\toprule\noalign{}
Derived Quantity & S.I. Unit & Symbol \\
\midrule\noalign{}
\endhead
\bottomrule\noalign{}
\endlastfoot
Force & Newton (N) & F \\
Energy & Joule (J) & E \\
Electric Current & Ampere (A) & I \\
\end{longtable}

\end{solutionbox}
\begin{mnemonicbox}
``FEI: Force-Energy-Current derive from
fundamentals''

\end{mnemonicbox}
\subsection*{Question 1(b) [4 marks]}\label{q1b}

\textbf{The length of a metal rod is 64.522 cm at 12^\circC temperature and
64.576 cm at 90^\circC temperature. Find the coefficient of linear expansion
of the metal rod.}

\begin{solutionbox}
\textbf{Formula:} α = (L_{2} - L_{1})/[L_{1} \times (T_{2} - T_{1})]

\textbf{Calculation:}

\begin{itemize}
\tightlist
\item
  Initial length (L_{1}) = 64.522 cm
\item
  Final length (L_{2}) = 64.576 cm
\item
  Initial temperature (T_{1}) = 12^\circC
\item
  Final temperature (T_{2}) = 90^\circC
\end{itemize}

α = (64.576 - 64.522)/[64.522 \times (90 - 12)]

α = 0.054/(64.522 \times 78) α

= 0.054/5032.716

α = 1.073 \times 10^{-}^{5} /^\circC


\end{solutionbox}
\begin{mnemonicbox}
``Change in Length over Original Length times Change
in Temperature''

\end{mnemonicbox}
\subsection*{Question 1(c) [7 marks]}\label{q1c}

\textbf{Explain with figure: The principle, construction and working of
a vernier calliper.}

\begin{solutionbox}
\textbf{Principle}: Vernier caliper works on the
principle of vernier scale, which allows measurements with accuracy
greater than the main scale.

\textbf{Construction:}

\begin{center}
\textbf{Mermaid Diagram (Code)}
\begin{verbatim}
{Shaded}
{Highlighting}[]
graph TD
    A[Vernier Caliper] {-{-}{} B[Main Scale]}
    A {-{-}{} C[Vernier Scale]}
    A {-{-}{} D[Fixed Jaw]}
    A {-{-}{} E[Movable Jaw]}
    A {-{-}{} F[Depth Probe]}
    A {-{-}{} G[Locking Screw]}
{Highlighting}
{Shaded}
\end{verbatim}
\end{center}

\textbf{Working:}

\begin{itemize}
\tightlist
\item
  \textbf{Zero error check}: Close jaws and note if zero of vernier
  coincides with zero of main scale
\item
  \textbf{External measurement}: Place object between fixed and movable
  jaws
\item
  \textbf{Reading process}: Note main scale reading + (coinciding
  vernier division \times least count)
\item
  \textbf{Least count} = (Smallest division on main scale)/(Number of
  divisions on vernier scale)
\end{itemize}

\textbf{Diagram:}

\begin{verbatim}
                 ┌───────┐            
 Main Scale ───▶│       │◀── Vernier Scale
  (in mm)       │   ┌───┴───┐     
     0    5    10   15      20    25    30
     |....|....|....|....|....|....|....|
     |    0    5    |0    5    |    
     └────┬─────────┘           
          │                  
      Fixed Jaw        Movable Jaw
\end{verbatim}

\end{solutionbox}
\begin{mnemonicbox}
``Main Scale Reading Plus Vernier Division Times
Least Count''

\end{mnemonicbox}
\subsection*{Question 1(c) OR [7
marks]}\label{q1c}

\textbf{Explain with figure: The principle, construction and working of
a micrometre screw gauge.}

\begin{solutionbox}
\textbf{Principle}: Micrometer screw gauge works on the
principle of screw motion - rotational motion is converted into linear
motion.

\textbf{Construction:}

\begin{center}
\textbf{Mermaid Diagram (Code)}
\begin{verbatim}
{Shaded}
{Highlighting}[]
graph TD
    A[Micrometer Screw Gauge] {-{-}{} B[Frame]}
    A {-{-}{} C[Anvil]}
    A {-{-}{} D[Spindle]}
    A {-{-}{} E[Sleeve/Main Scale]}
    A {-{-}{} F[Thimble/Circular Scale]}
    A {-{-}{} G[Ratchet]}
    A {-{-}{} H[Lock Nut]}
{Highlighting}
{Shaded}
\end{verbatim}
\end{center}

\textbf{Working:}

\begin{itemize}
\tightlist
\item
  \textbf{Zero error check}: Close anvil and spindle, note if zero of
  circular scale aligns with reference line
\item
  \textbf{Measurement process}: Place object between anvil and spindle
\item
  \textbf{Reading}: Main scale reading + (Circular scale reading \times Least
  count)
\item
  \textbf{Least Count} = Pitch/Number of divisions on circular scale
\end{itemize}

\textbf{Diagram:}

\begin{verbatim}
                     Ratchet
                        ▲
                        │
        Frame           │        Thimble/Circular Scale
          ┌─────────────┴─────┐  ┌───┐
          │                   │  │   │
Anvil ──▶ O═══════════════════O══O   │
          │     │             │  │   │
          └─────┼─────────────┘  └───┘
                │                   │
                ▼                   ▼
            Spindle             Sleeve/Main Scale
\end{verbatim}

\end{solutionbox}
\begin{mnemonicbox}
``PST: Pitch divided by Scale gives Thimble's least
count''

\end{mnemonicbox}
\subsection*{Question 2(a) [3 marks]}\label{q2a}

\textbf{Find the diameter of a sphere if pitch of micrometer screw gauge
is 1 mm and there are 100 divisions on circular scale. The edge of
circular scale lies between 7 and 8 mm of the main scale and 65th
division of the circular scale coincides with the horizontal line of the
main scale.}

\begin{solutionbox}
\textbf{Formula:} Diameter = Main scale reading +
(Circular scale reading \times Least count)

\textbf{Calculation:}

\begin{itemize}
\tightlist
\item
  Main scale reading = 7 mm
\item
  Circular scale reading = 65 divisions
\item
  Least count = Pitch/Number of divisions = 1/100 = 0.01 mm
\end{itemize}

Diameter = 7 + (65 \times 0.01) = 7 + 0.65 = 7.65 mm

\end{solutionbox}
\begin{mnemonicbox}
``MSR + (CSR \times LC) gives the final measurement''

\end{mnemonicbox}
\subsection*{Question 2(b) [4 marks]}\label{q2b}

\textbf{Explain phase difference and coherence.}

\begin{solutionbox}
\textbf{Phase Difference:} The difference in phase
angle between two waves of the same frequency.


\vspace{-5pt}
\captionof{table}{Phase Difference Characteristics}
\vspace{-10pt}
\begin{longtable}[]{@{}lll@{}}
\toprule\noalign{}
Phase Difference & Interference Type & Result \\
\midrule\noalign{}
\endhead
\bottomrule\noalign{}
\endlastfoot
0^\circ or 360^\circ & Constructive & Maximum amplitude \\
180^\circ & Destructive & Minimum amplitude \\
\end{longtable}

\textbf{Coherence:} Property of waves that have a constant phase
relationship.

\textbf{Types of Coherence:}

\begin{itemize}
\tightlist
\item
  \textbf{Temporal coherence}: Related to frequency stability
\item
  \textbf{Spatial coherence}: Related to wavefront uniformity
\end{itemize}

\end{solutionbox}
\begin{mnemonicbox}
``Constant Phase Relationship Creates Coherent
waves''

\end{mnemonicbox}
\subsection*{Question 2(c) [7 marks]}\label{q2c}

\textbf{Explain capacitor, its capacitance and the effect of dielectric
material on the capacitance of parallel plate capacitor.}

\begin{solutionbox}
\textbf{Capacitor}: Device that stores electric charge
and electrical energy in an electric field.

\textbf{Capacitance}: Ratio of charge stored to potential difference
applied.

\textbf{Formula:} C = Q/V

\textbf{Parallel Plate Capacitor:} Capacitance formula: C = ε_{0}A/d

\begin{itemize}
\tightlist
\item
  ε_{0} = Permittivity of free space
\item
  A = Area of plates
\item
  d = Distance between plates
\end{itemize}

\textbf{Effect of Dielectric:}

\begin{itemize}
\tightlist
\item
  Increases capacitance by K times (K = dielectric constant)
\item
  New formula: C = Kε_{0}A/d
\end{itemize}

\textbf{Diagram:}

\begin{verbatim}
    ┌───────────────┐  │
    │      ++++     │  │
    │      ++++     │  │ d
    │      ++++     │  │
    └───────────────┘  │
           │           
           │          
           V          
    ┌───────────────┐
    │      {-{-}{-}{-}     │}
    │      {-{-}{-}{-}     │ ◄── Dielectric}
    │      {-{-}{-}{-}     │}
    └───────────────┘
           │
           │
    Area = A
\end{verbatim}

\end{solutionbox}
\begin{mnemonicbox}
``KIDS: K Increases Dielectric Storage''

\end{mnemonicbox}
\subsection*{Question 2(a) OR [3
marks]}\label{q2a}

\textbf{If the lengths of two cylinders are (6.52\pm0.01) cm and
(4.48\pm0.02) cm respectively. Find the difference in their length with
percentage error.}

\begin{solutionbox}
\textbf{Calculation:}

\begin{itemize}
\tightlist
\item
  Length of first cylinder (L_{1}) = 6.52 \pm 0.01 cm
\item
  Length of second cylinder (L_{2}) = 4.48 \pm 0.02 cm
\item
  Difference in length (ΔL) = L_{1} - L_{2} = 6.52 - 4.48 = 2.04 cm
\end{itemize}

\textbf{Absolute error in difference} = \sqrt[(0.01)^{2} + (0.02)^{2}] =
\sqrt(0.0001 + 0.0004) = \sqrt0.0005 = 0.022 cm

\textbf{Percentage error} = (Absolute error/Measured value) \times 100 =
(0.022/2.04) \times 100 = 1.08\%

\end{solutionbox}
\begin{mnemonicbox}
``Add errors in quadrature for difference
calculations''

\end{mnemonicbox}
\subsection*{Question 2(b) OR [4
marks]}\label{q2b}

\textbf{Explain the types of interference with relevant figures.}

\begin{solutionbox}
\textbf{Types of Interference:}


\vspace{-5pt}
\captionof{table}{Interference Types}
\vspace{-10pt}
\begin{longtable}[]{@{}llll@{}}
\toprule\noalign{}
Type & Phase Difference & Result & Wave Amplitude \\
\midrule\noalign{}
\endhead
\bottomrule\noalign{}
\endlastfoot
Constructive & 0^\circ, 360^\circ, 720^\circ\ldots{} & Reinforcement & Maximum \\
Destructive & 180^\circ, 540^\circ, 900^\circ\ldots{} & Cancellation & Minimum \\
\end{longtable}

\textbf{Constructive Interference:} When crest meets crest or trough
meets trough.

\textbf{Destructive Interference:} When crest meets trough.

\textbf{Diagram:}

\begin{verbatim}
Constructive Interference:
    ⟋⟍⟋⟍⟋⟍     Wave 1
    ⟋⟍⟋⟍⟋⟍     Wave 2
    ⟋⟍⟋⟍⟋⟍
     ⬍⬍⬍⬍      Result: Larger amplitude
     
Destructive Interference:
    ⟋⟍⟋⟍⟋⟍     Wave 1
    ⟍⟋⟍⟋⟍⟋     Wave 2
    {-{-}{-}{-}{-}{-}{-}{-}{-} Result: Flat line (cancellation)}
\end{verbatim}

\end{solutionbox}
\begin{mnemonicbox}
``Crest + Crest = Constructive, Crest + Trough =
Destructive''

\end{mnemonicbox}
\subsection*{Question 2(c) OR [7
marks]}\label{q2c}

\textbf{Derive the expression for potential due to point charge with
necessary figure.}

\begin{solutionbox}
\textbf{Potential at a point due to point charge:}

\textbf{Formula development:}

\begin{itemize}
\tightlist
\item
  \textbf{Definition}: Work done per unit charge to bring a test charge
  from infinity to that point
\item
  \textbf{Expression}: V = W/q_{0} = \int(F·dr)
\end{itemize}

\textbf{Step-by-step derivation:}

\begin{enumerate}
\tightlist
\item
  Force between charges (Coulomb's law): F = (1/4πε_{0}) \times (Qq/r^{2})
\item
  Work done moving test charge: W = \int(F·dr)
\item
  For radial motion: W = (Q/4πε_{0}) \times \int(1/r^{2})dr from \infty to r
\item
  Integrating: W = (Q/4πε_{0}) \times [-1/r]ᵣ\infty
\item
Final result:

V = W/q_{0} = (1/4πε_{0}) \times (Q/r)

\end{enumerate}

\textbf{Final formula:} V = (1/4πε_{0}) \times (Q/r)

\textbf{Diagram:}

\begin{verbatim}
              P (Point where
                potential is calculated)
              *
              │
              │
              │r
              │
              │
        Q     │
        ●─────┘
   Point Charge at origin
\end{verbatim}

\end{solutionbox}
\begin{mnemonicbox}
``POD: Potential Over Distance equals charge over r''

\end{mnemonicbox}
\subsection*{Question 3(a) [3 marks]}\label{q3a}

\textbf{Explain in brief charging by friction and induction methods.}

\begin{solutionbox}
\textbf{Charging by Friction:} Process of charging by
rubbing two different materials together.

\textbf{Steps in friction charging:}

\begin{itemize}
\tightlist
\item
  Electrons transfer from one material to another
\item
  Material losing electrons becomes positively charged
\item
  Material gaining electrons becomes negatively charged
\end{itemize}

\textbf{Charging by Induction:} Process of charging without direct
contact.

\textbf{Steps in induction charging:}

\begin{itemize}
\tightlist
\item
  Bring charged body near a neutral conductor
\item
  Redistribution of charges in neutral body
\item
  Ground the conductor and remove ground
\item
  Remove the charged body
\end{itemize}

\end{solutionbox}
\begin{mnemonicbox}
``FTEE: Friction Transfers Electrons Easily''

\end{mnemonicbox}
\subsection*{Question 3(b) [4 marks]}\label{q3b}

\textbf{A tuning fork vibrates at frequency of 256 Hz. If its velocity
is 340 m/s, find (a) wavelength and (b) distance travelled by it in 50
oscillations.}

\begin{solutionbox}
\textbf{Formulas:}

\begin{itemize}
\tightlist
\item
  Wavelength (λ) = Velocity (v) / Frequency (f)
\item
  Distance (d) = Number of oscillations (n) \times Wavelength (λ)
\end{itemize}

\textbf{Calculation:} (a) Wavelength (λ) = v/f = 340/256 = 1.328 m

\begin{enumerate}
\tightlist
\item
Distance (d) = n \times

λ = 50 \times 1.328 = 66.4 m

\end{enumerate}

\end{solutionbox}
\begin{mnemonicbox}
``VFD: Velocity, Frequency and Distance are
connected''

\end{mnemonicbox}
\subsection*{Question 3(c) [7 marks]}\label{q3c}

\textbf{Write the principle and construction of a bimetallic thermometer
with a labelled diagram. Also mention its advantages and disadvantages.}

\begin{solutionbox}
\textbf{Principle}: Different metals expand differently
when heated, causing the strip to bend.

\textbf{Construction:}

\begin{center}
\textbf{Mermaid Diagram (Code)}
\begin{verbatim}
{Shaded}
{Highlighting}[]
graph TD
    A[Bimetallic Thermometer] {-{-}{} B[Fixed End]}
    A {-{-}{} C[Bimetallic Strip]}
    A {-{-}{} D[Pointer]}
    A {-{-}{} E[Scale]}
    A {-{-}{} F[Protective Case]}
    C {-{-}{} G[Metal with Higher Expansion]}
    C {-{-}{} H[Metal with Lower Expansion]}
{Highlighting}
{Shaded}
\end{verbatim}
\end{center}

\textbf{Working:}

\begin{itemize}
\tightlist
\item
  Temperature change causes different expansion rates
\item
  Bimetallic strip bends toward metal with lower expansion coefficient
\item
  Pointer movement indicates temperature
\end{itemize}

\textbf{Diagram:}

\begin{verbatim}
          Pointer
             │
             ▼
        ┌────┴─────┐
Scale ──┤          │
        │          │
        │ ┌────────┘
        │ │
        └─┘
          ▲
          │
    Bimetallic Strip
 (Two different metals laminated)
 
 At higher temperature:
          
          ┌────────┐
          │        │
          │ ┌──────┘
          │ │
          └─┘
            {}
             { (bends due to }
              { different expansion)}
\end{verbatim}

\textbf{Advantages:}

\begin{itemize}
\tightlist
\item
  Simple, robust design
\item
  No power supply needed
\item
  Wide temperature range
\end{itemize}

\textbf{Disadvantages:}

\begin{itemize}
\tightlist
\item
  Less accurate than other types
\item
  Slow response time
\item
  Subject to mechanical wear
\end{itemize}

\end{solutionbox}
\begin{mnemonicbox}
``BEDS: Bimetallic Elements Deform with Stress''

\end{mnemonicbox}
\subsection*{Question 3(a) OR [3
marks]}\label{q3a}

\textbf{Explain work done on a point charge in an electric field.}

\begin{solutionbox}
\textbf{Work Done on Point Charge:} The work done to
move a point charge q in an electric field E.

\textbf{Formula:} W = q(V_{a} - Vᵦ) = qΔV

Where:

\begin{itemize}
\tightlist
\item
  q = charge being moved
\item
  V_{a} = potential at initial position
\item
  Vᵦ = potential at final position
\item
  ΔV = potential difference
\end{itemize}

\textbf{Key properties:}

\begin{itemize}
\tightlist
\item
  Work is independent of path taken
\item
  Work is positive when moving against electric field
\item
  Work is negative when moving along electric field
\end{itemize}

\end{solutionbox}
\begin{mnemonicbox}
``PEW: Potential difference \times Electric charge =
Work''

\end{mnemonicbox}
\subsection*{Question 3(b) OR [4
marks]}\label{q3b}

\textbf{What will be the distance travelled by a sound wave in 75
vibrations if its speed is 0.33 km/s and frequency is 660 Hz.}

\begin{solutionbox}
\textbf{Formulas:}

\begin{itemize}
\tightlist
\item
  Wavelength (λ) = Velocity (v) / Frequency (f)
\item
  Distance (d) = Number of vibrations (n) \times Wavelength (λ)
\end{itemize}

\textbf{Calculation:}

\begin{itemize}
\tightlist
\item
Convert velocity:

v = 0.33 km/s = 330 m/s

\item
Wavelength:

λ = v/f = 330/660 = 0.5 m

\item
Distance:

d = n \times

λ = 75 \times 0.5 = 37.5 m

\end{itemize}

\end{solutionbox}
\begin{mnemonicbox}
``FVW: Frequency into Velocity gives Wavelength''

\end{mnemonicbox}
\subsection*{Question 3(c) OR [7
marks]}\label{q3c}

\textbf{Write the principle and construction of a Mercury thermometer
with a labelled diagram. Also mention its advantages and disadvantages.}

\begin{solutionbox}
\textbf{Principle}: Mercury thermometer works on the
principle of thermal expansion of mercury when heated.

\textbf{Construction:}

\begin{center}
\textbf{Mermaid Diagram (Code)}
\begin{verbatim}
{Shaded}
{Highlighting}[]
graph TD
    A[Mercury Thermometer] {-{-}{} B[Glass Bulb]}
    A {-{-}{} C[Capillary Tube]}
    A {-{-}{} D[Scale]}
    A {-{-}{} E[Mercury]}
    A {-{-}{} F[Vacuum/Nitrogen Space]}
    A {-{-}{} G[Safety Bulb]}
{Highlighting}
{Shaded}
\end{verbatim}
\end{center}

\textbf{Working:}

\begin{itemize}
\tightlist
\item
  Mercury expands when heated
\item
  Expansion causes mercury to rise in capillary
\item
  Height of mercury column indicates temperature
\end{itemize}

\textbf{Diagram:}

\begin{verbatim}
      ┌─────┐
      │     │ ◄── Scale
      │     │
      │  │  │ ◄── Capillary Tube
      │  │  │
      │  │  │
      │  │  │
      │ ┌┴┐ │ ◄── Mercury Bulb
      │ └─┘ │
      └─────┘
\end{verbatim}

\textbf{Advantages:}

\begin{itemize}
\tightlist
\item
  High accuracy
\item
  Wide temperature range (-38^\circC to 357^\circC)
\item
  Linear expansion of mercury
\item
  Good visibility of mercury thread
\end{itemize}

\textbf{Disadvantages:}

\begin{itemize}
\tightlist
\item
  Mercury is toxic
\item
  Fragile glass construction
\item
  Cannot be used below -38^\circC
\item
  Slow response to temperature changes
\end{itemize}

\end{solutionbox}
\begin{mnemonicbox}
``MELT: Mercury Expands Linearly with Temperature''

\end{mnemonicbox}
\subsection*{Question 4(a) [3 marks]}\label{q4a}

\textbf{The electric force between two positive ions of equal magnitude
separated by distance 5\times10^{-}^{1}^{0} m from eachother is 3.7 \times 10^{-}^{9} N. How many
electrons would have been removed from each atom.}

\begin{solutionbox}
\textbf{Formula:} F = (1/4πε_{0}) \times (q_{1}q_{2}/r^{2})

\textbf{Calculation:}

\begin{itemize}
\tightlist
\item
  F = 3.7 \times 10^{-}^{9} N
\item
  r = 5 \times 10^{-}^{1}^{0} m
\item
q_{1} = q_{2} = ne (n = number of electrons,

e = electron charge)

\item
  1/4πε_{0} = 9 \times 10^{9} Nm^{2}/C^{2}
\item
  e = 1.6 \times 10^{-}^{1}^{9} C
\end{itemize}

3.7 \times 10^{-}^{9} = (9 \times 10^{9}) \times (n^{2}e^{2}/(5 \times 10^{-}^{1}^{0})^{2}) 3.7 \times 10^{-}^{9} = (9 \times 10^{9}) \times
(n^{2} \times (1.6 \times 10^{-}^{1}^{9})^{2}/25 \times 10^{-}^{2}^{0}) Solving: n = 1 (1 electron removed from
each atom)

\end{solutionbox}
\begin{mnemonicbox}
``FACE: Force Affects Charge Equally''

\end{mnemonicbox}
\subsection*{Question 4(b) [4 marks]}\label{q4b}

\textbf{State Snell's law and derive its formula.}

\begin{solutionbox}
\textbf{Snell's Law}: The ratio of sine of angle of
incidence to sine of angle of refraction is constant for a given pair of
media.

\textbf{Formula:} (sin i)/(sin r) = n_{2}/n_{1} = constant

\textbf{Derivation steps:}

\begin{enumerate}
\tightlist
\item
  Light travels at different speeds in different media
\item
  When light passes from one medium to another, it changes direction
\item
  Using Fermat's principle of least time
\item
  Ratio of speeds equals ratio of refractive indices
\item
  Final formula: n_{1}sin i = n_{2}sin r
\end{enumerate}

\textbf{Diagram:}

\begin{verbatim}
        Medium 1 (n_{1)}
        │
        │         i
Normal  │        /
        │       /
────────┼──────/───────
        │     /
        │    /
        │   /  r
        │  /
        Medium 2 (n_{2)}
\end{verbatim}

\end{solutionbox}
\begin{mnemonicbox}
``SINIS: SIN I over SIN R equals refractive index
ratio''

\end{mnemonicbox}
\subsection*{Question 4(c) [7 marks]}\label{q4c}

\textbf{Explain any three applications of Ultrasonic waves.}

\begin{solutionbox}
\textbf{Applications of Ultrasonic Waves:}


\vspace{-5pt}
\captionof{table}{Ultrasonic Applications}
\vspace{-10pt}
\begin{longtable}[]{@{}
  >{\raggedright\arraybackslash}p{(\linewidth - 4\tabcolsep) * \real{0.4483}}
  >{\raggedright\arraybackslash}p{(\linewidth - 4\tabcolsep) * \real{0.3793}}
  >{\raggedright\arraybackslash}p{(\linewidth - 4\tabcolsep) * \real{0.1724}}@{}}
\toprule\noalign{}
\begin{minipage}[b]{\linewidth}\raggedright
Application
\end{minipage} & \begin{minipage}[b]{\linewidth}\raggedright
Principle
\end{minipage} & \begin{minipage}[b]{\linewidth}\raggedright
Use
\end{minipage} \\
\midrule\noalign{}
\endhead
\bottomrule\noalign{}
\endlastfoot
Medical Imaging & Reflection from tissues & Visualize internal organs \\
NDT (Non-Destructive Testing) & Reflection from defects & Find flaws in
materials \\
Cleaning & Cavitation effect & Clean jewelry, surgical instruments \\
\end{longtable}

\textbf{1. Medical Imaging (Sonography):}

\begin{itemize}
\tightlist
\item
  Frequencies: 1-10 MHz
\item
  Principle: Pulse-echo technique
\item
  Uses: Fetal imaging, organ scanning, blood flow measurement
\end{itemize}

\textbf{2. Industrial NDT:}

\begin{itemize}
\tightlist
\item
  Detects cracks, voids, and flaws in materials
\item
  Quality control in manufacturing
\item
  Thickness measurement of materials
\end{itemize}

\textbf{3. Ultrasonic Cleaning:}

\begin{itemize}
\tightlist
\item
  Creates microscopic bubbles (cavitation)
\item
  Removes contaminants from surfaces
\item
  Used for jewelry, optical components, surgical instruments
\end{itemize}

\end{solutionbox}
\begin{mnemonicbox}
``MIC: Medical, Industrial, Cleaning applications''

\end{mnemonicbox}
\subsection*{Question 4(a) OR [3
marks]}\label{q4a}

\textbf{Obtain the equivalent capacitance for series and parallel
combinations of 3 capacitors having capacitances 5 µF, 10 µF and 15 µF
respectively.}

\begin{solutionbox}
\textbf{Parallel Combination:} C_{p} = C_{1} + C_{2} + C_{3} = 5 +
10 + 15 = 30 µF

\textbf{Series Combination:} 1/C_{s} = 1/C_{1} + 1/C_{2} + 1/C_{3} 1/C_{s} = 1/5 + 1/10
+ 1/15 1/C_{s} = 0.2 + 0.1 + 0.067 = 0.367 C_{s} = 1/0.367 = 2.72 µF

\end{solutionbox}
\begin{mnemonicbox}
``ASAP: Add for Series, Add inverse for Parallel''

\end{mnemonicbox}
\subsection*{Question 4(b) OR [4
marks]}\label{q4b}

\textbf{Explain the construction of an optical fibre with a neat
diagram.}

\begin{solutionbox}
\textbf{Construction of Optical Fiber:}

\textbf{Components:}

\begin{itemize}
\tightlist
\item
  Core: Light transmission medium
\item
  Cladding: Outer layer with lower refractive index
\item
  Buffer coating: Protective plastic covering
\end{itemize}

\textbf{Parameters:}

\begin{itemize}
\tightlist
\item
  Core diameter: 8-50 μm (single mode), 50-100 μm (multimode)
\item
  Cladding diameter: 125-140 μm
\item
  Core refractive index \textgreater{} Cladding refractive index
\end{itemize}

\textbf{Diagram:}

\begin{verbatim}
Cross{-section:}
         ┌───────────┐
         │     ┌─────┴─────┐
         │     │     │     │
         │     │     │     │
         │     │     │     │
         │     └─────┬─────┘
         └───────────┘
          │     │     │
   Buffer │ Cladding │ Core
          │     │     │

Longitudinal view:
  ┌──────────────────────────────┐
  │ ┌────────────────────────┐   │
  │ │                        │   │
  │ │           Core         │   │  Light
  │ │                        │   │  ray
  │ └────────────────────────┘   │  ↘
  └──────────────────────────────┘   ⟍⟋⟍⟋⟍⟋
              Cladding               ⟍⟋⟍⟋⟍⟋
\end{verbatim}

\end{solutionbox}
\begin{mnemonicbox}
``CBC: Core-Buffer-Cladding from inside out''

\end{mnemonicbox}
\subsection*{Question 4(c) OR [7
marks]}\label{q4c}

\textbf{Explain production of ultrasonic waves by magnetostriction
method.}

\begin{solutionbox}
\textbf{Magnetostriction Method:} The process of
generating ultrasonic waves using the property of ferromagnetic
materials to change dimensions when placed in a magnetic field.

\textbf{Principle:} Ferromagnetic materials change length when
magnetized, producing mechanical vibrations that create ultrasonic
waves.

\textbf{Construction:}

\begin{center}
\textbf{Mermaid Diagram (Code)}
\begin{verbatim}
{Shaded}
{Highlighting}[]
graph TD
    A[Magnetostriction Generator] {-{-}{} B[AC Power Supply]}
    A {-{-}{} C[Coil/Solenoid]}
    A {-{-}{} D[Ferromagnetic Rod]}
    A {-{-}{} E[Acoustic Medium]}
    A {-{-}{} F[Cooling System]}
{Highlighting}
{Shaded}
\end{verbatim}
\end{center}

\textbf{Working Process:}

\begin{enumerate}
\tightlist
\item
  AC current passes through solenoid
\item
  Alternating magnetic field produced
\item
  Ferromagnetic rod expands and contracts
\item
  Vibrations transmitted to medium
\item
  Ultrasonic waves generated
\end{enumerate}

\textbf{Diagram:}

\begin{verbatim}
    ┌───────────────────┐
    │                   │
    │  AC Power Supply  │
    │                   │
    └─────────┬─────────┘
              │
              ▼
    ┌─────────────────────┐
    │                     │
    │  ┌───┐      ┌───┐   │
    │  │   │      │   │   │
    │  │   │      │   │   │
    │  │   │      │   │   │
    │  └───┘      └───┘   │
    │   Coil       Coil   │
    │                     │
    └──────────┬──────────┘
               │
      ┌────────┴────────┐
      │ Ferromagnetic   │ ─ Ultrasonic
      │ Rod (Nickel)    │    Waves
      └─────────────────┘
\end{verbatim}

\textbf{Advantages:}

\begin{itemize}
\tightlist
\item
  Simple construction
\item
  High power output
\item
  Suitable for liquids
\end{itemize}

\textbf{Disadvantages:}

\begin{itemize}
\tightlist
\item
  Limited to frequencies below 100 kHz
\item
  Heating effects
\item
  Lower efficiency
\end{itemize}

\end{solutionbox}
\begin{mnemonicbox}
``FAME: Ferromagnetic Alternating Magnetic Effect''

\end{mnemonicbox}
\subsection*{Question 5(a) [3 marks]}\label{q5a}

\textbf{Explain in brief the three modes of heat transfer.}

\begin{solutionbox}
\textbf{Three Modes of Heat Transfer:}


\vspace{-5pt}
\captionof{table}{Heat Transfer Modes}
\vspace{-10pt}
\begin{longtable}[]{@{}lll@{}}
\toprule\noalign{}
Mode & Medium Requirement & Example \\
\midrule\noalign{}
\endhead
\bottomrule\noalign{}
\endlastfoot
Conduction & Physical contact & Heat through metal rod \\
Convection & Fluid medium & Hot air rising \\
Radiation & No medium needed & Heat from sun \\
\end{longtable}

\textbf{1. Conduction:}

\begin{itemize}
\tightlist
\item
  Transfer through direct molecular collision
\item
  No bulk movement of matter
\item
  Good in solids, especially metals
\end{itemize}

\textbf{2. Convection:}

\begin{itemize}
\tightlist
\item
  Transfer through fluid movement
\item
  Requires density differences
\item
  Natural or forced convection
\end{itemize}

\textbf{3. Radiation:}

\begin{itemize}
\tightlist
\item
  Transfer through electromagnetic waves
\item
  Works in vacuum
\item
  Depends on temperature and surface properties
\end{itemize}

\end{solutionbox}
\begin{mnemonicbox}
``CCR: Conduction Contact, Convection Current,
Radiation Rays''

\end{mnemonicbox}
\subsection*{Question 5(b) [4 marks]}\label{q5b}

\textbf{Calculate the numerical aperture and acceptance angle of an
optical fibre if the refractive indices of core and cladding of an
optical fibre are 1.55 and 1.5 respectively.}

\begin{solutionbox}
\textbf{Formulas:}

\begin{itemize}
\tightlist
\item
  Numerical Aperture (NA) = \sqrt(n_{1}^{2} - n_{2}^{2})
\item
  Acceptance angle (θ_{a}) = sin^{-}^{1}(NA)
\end{itemize}

\textbf{Calculation:}

\begin{itemize}
\tightlist
\item
  Core refractive index (n_{1}) = 1.55
\item
  Cladding refractive index (n_{2}) = 1.5
\end{itemize}

NA = \sqrt(1.55^{2} - 1.5^{2}) NA = \sqrt(2.4025 - 2.25) NA = \sqrt0.1525 NA = 0.391

Acceptance angle (θ_{a}) = sin^{-}^{1}(0.391) θ_{a} = 23.03^\circ

\end{solutionbox}
\begin{mnemonicbox}
``CORE: Calculate Optical Refractive-index Exactly''

\end{mnemonicbox}
\subsection*{Question 5(c) [7 marks]}\label{q5c}

\textbf{Explain any three applications of optical fibres.}

\begin{solutionbox}
\textbf{Applications of Optical Fibers:}


\vspace{-5pt}
\captionof{table}{Major Optical Fiber Applications}
\vspace{-10pt}
\begin{longtable}[]{@{}lll@{}}
\toprule\noalign{}
Application & Advantage & Example \\
\midrule\noalign{}
\endhead
\bottomrule\noalign{}
\endlastfoot
Communications & High bandwidth & Internet, phone networks \\
Medical & Flexibility, imaging & Endoscopy \\
Sensors & Immunity to EMI & Temperature sensing \\
\end{longtable}

\textbf{1. Communication Networks:}

\begin{itemize}
\tightlist
\item
  Telecommunications and internet
\item
  Higher bandwidth than copper cables
\item
  Less signal attenuation over long distances
\item
  More secure against tapping
\end{itemize}

\textbf{2. Medical Applications:}

\begin{itemize}
\tightlist
\item
  Endoscopy for minimally invasive procedures
\item
  Light delivery for photodynamic therapy
\item
  Dental procedures
\item
  Surgical illumination
\end{itemize}

\textbf{3. Sensing Applications:}

\begin{itemize}
\tightlist
\item
  Temperature and pressure sensors
\item
  Strain gauges for structural monitoring
\item
  Chemical sensors
\item
  Gyroscopes for navigation
\end{itemize}

\end{solutionbox}
\begin{mnemonicbox}
``CMS: Communication, Medical, Sensing applications''

\end{mnemonicbox}
\subsection*{Question 5(a) OR [3
marks]}\label{q5a}

\textbf{Give a detailed explanation of specific heat.}

\begin{solutionbox}
\textbf{Specific Heat:} Amount of heat required to
raise the temperature of 1 kg of a substance by 1 Kelvin (or 1^\circC).

\textbf{Formula:} Q = mc∆T

Where:

\begin{itemize}
\tightlist
\item
  Q = Heat energy (J)
\item
  m = Mass (kg)
\item
  c = Specific heat capacity (J/kg·K)
\item
  ∆T = Temperature change (K)
\end{itemize}

\textbf{Units:} J/kg·K or J/kg·^\circC

\textbf{Significance:}

\begin{itemize}
\tightlist
\item
  Measures thermal inertia of materials
\item
  Higher specific heat means material requires more energy to heat up
\item
  Water has unusually high specific heat (4,186 J/kg·K)
\end{itemize}

\end{solutionbox}
\begin{mnemonicbox}
``STEM: Specific heat measures Temperature change per
Energy and Mass''

\end{mnemonicbox}
\subsection*{Question 5(b) OR [4
marks]}\label{q5b}

\textbf{If the refractive indices of core and cladding of an optical
fibre are 1.48 and 1.45 respectively. Calculate its acceptance angle and
critical angle.}

\begin{solutionbox}
\textbf{Formulas:}

\begin{itemize}
\tightlist
\item
  Numerical Aperture (NA) = \sqrt(n_{1}^{2} - n_{2}^{2})
\item
  Acceptance angle (θ_{a}) = sin^{-}^{1}(NA)
\item
  Critical angle (θc) = sin^{-}^{1}(n_{2}/n_{1})
\end{itemize}

\textbf{Calculation:}

\begin{itemize}
\tightlist
\item
  Core refractive index (n_{1}) = 1.48
\item
  Cladding refractive index (n_{2}) = 1.45
\end{itemize}

NA = \sqrt(1.48^{2} - 1.45^{2}) NA = \sqrt(2.1904 - 2.1025) NA = \sqrt0.0879 NA = 0.296

Acceptance angle (θ_{a}) = sin^{-}^{1}(0.296) θ_{a} = 17.2^\circ

Critical angle (θc) = sin^{-}^{1}(n_{2}/n_{1}) θc = sin^{-}^{1}(1.45/1.48) θc =
sin^{-}^{1}(0.9797) θc = 78.4^\circ

\end{solutionbox}
\begin{mnemonicbox}
``NA leads to AA, ratio leads to Critical Angle''

\end{mnemonicbox}
\subsection*{Question 5(c) OR [7
marks]}\label{q5c}

\textbf{Explain the applications of LASER in engineering and medical
field.}

\begin{solutionbox}
\textbf{Applications of LASER:}


\vspace{-5pt}
\captionof{table}{LASER Applications}
\vspace{-10pt}
\begin{longtable}[]{@{}lll@{}}
\toprule\noalign{}
Field & Application & Example \\
\midrule\noalign{}
\endhead
\bottomrule\noalign{}
\endlastfoot
Engineering & Cutting/Welding & Metal fabrication \\
Engineering & Measurements & Distance measurement \\
Medical & Surgery & Eye surgery (LASIK) \\
Medical & Therapy & Cancer treatment \\
\end{longtable}

\textbf{Engineering Applications:}

\textbf{1. Material Processing:}

\begin{itemize}
\tightlist
\item
  Precision cutting of metals, plastics, ceramics
\item
  Welding of dissimilar materials
\item
  Surface treatment and hardening
\item
  3D printing and rapid prototyping
\end{itemize}

\textbf{2. Metrology and Measurement:}

\begin{itemize}
\tightlist
\item
  Distance measurement with high precision
\item
  Alignment in construction and manufacturing
\item
  Interferometry for surface analysis
\item
  Holography for 3D imaging
\end{itemize}

\textbf{Medical Applications:}

\textbf{1. Surgical Procedures:}

\begin{itemize}
\tightlist
\item
  Eye surgery (LASIK, cataract removal)
\item
  Minimally invasive procedures
\item
  Dermatological treatments
\item
  Dental procedures
\end{itemize}

\textbf{2. Therapeutic Uses:}

\begin{itemize}
\tightlist
\item
  Photodynamic therapy for cancer
\item
  Low-level laser therapy for pain
\item
  Treatment of vascular lesions
\item
  Cosmetic procedures
\end{itemize}

\textbf{Diagram:}

\begin{verbatim}
Engineering Applications:
   LASER ──► Material Processing
     │
     ├───► Measurements
     │
     └───► Communications

Medical Applications:
   LASER ──► Surgical Procedures
     │
     ├───► Diagnostic Tools
     │
     └───► Therapeutic Treatments
\end{verbatim}

\end{solutionbox}
\begin{mnemonicbox}
``SMART: Surgery, Measurement, Analysis, Repair, and
Treatment''

\end{mnemonicbox}

\end{document}
