\documentclass[10pt,a4paper]{article}

% content/resources/templates/preamble.tex
\usepackage[margin=0.6in]{geometry}
\author{Milav Dabgar}
\usepackage{amsmath,amssymb,amsthm}
\usepackage{booktabs}
\usepackage{multirow}
\usepackage{xcolor}
\usepackage{tcolorbox}
\tcbuselibrary{breakable,skins}
\usepackage[colorlinks=true,linkcolor=blue]{hyperref}
\usepackage{titlesec}
\usepackage{enumitem}
\usepackage{tikz}
\usepackage{pgfplots}
\usepackage{circuitikz}
\usepackage[version=4]{mhchem}
\usepackage{longtable}
\usepackage{array}
\usepackage{float}
\usepackage{caption}
\usepackage{listings}

\lstset{
  basicstyle=\small\ttfamily,
  breaklines=true,
  breakatwhitespace=false,
  postbreak=\mbox{\textcolor{red}{$\hookrightarrow$}\space},
  float=false,
  numbers=left,
  numberstyle=\tiny\color{gray},
  numbersep=10pt,
  xleftmargin=2em,
  keywordstyle=\color{blue},
  commentstyle=\color{green!60!black},
  stringstyle=\color{purple},
  backgroundcolor=\color{gray!5},
  showstringspaces=false,
  tabsize=2,
  captionpos=b,
  keepspaces=true,
  columns=flexible
}

\pgfplotsset{compat=1.18}
\usetikzlibrary{shapes,arrows,positioning,calc,patterns,decorations.pathmorphing,decorations.markings,arrows.meta}

% Color scheme
\definecolor{headcolor}{RGB}{0,102,204}
\definecolor{keycolor}{RGB}{220,20,60}
\definecolor{solutioncolor}{RGB}{34,139,34}
\definecolor{mnemoniccolor}{RGB}{148,0,211}
\definecolor{codecolor}{RGB}{0,0,100}

% Spacing
\setlength{\parskip}{3pt}
\setlist[itemize]{nosep}
\setlist[enumerate]{nosep}

% Title formatting
\titleformat{\section}{\Large\bfseries\color{headcolor}}{\thesection}{1em}{}
\titleformat{\subsection}{\large\bfseries\color{headcolor}}{\thesubsection}{1em}{}

% Pandoc tightlist compatibility
\providecommand{\tightlist}{%
  \setlength{\itemsep}{0pt}\setlength{\parskip}{0pt}}

% Pandoc longtable compatibility
\newcounter{none}
\def\thenone{}


% content/resources/templates/gujarati-boxes.tex
\usepackage{fontspec}
\usepackage{polyglossia}

% Set Gujarati as main language (document is primarily in Gujarati)
% Note: gloss-gujarati.ldf doesn't exist in polyglossia, but it will use hyphenation patterns
\setdefaultlanguage{gujarati}
\setotherlanguage{english}

% Configure Gujarati font properly
% Use Language=Default to prevent polyglossia from trying to add language-specific features
% that don't exist for Gujarati, which causes "empty feature" warnings
\newfontfamily\gujaratifont[Script=Gujarati,AutoFakeBold=2.5,AutoFakeSlant=0.3]{Noto Sans Gujarati}
\setmainfont[Script=Gujarati,AutoFakeBold=2.5,AutoFakeSlant=0.3]{Noto Sans Gujarati}
% Use Noto Sans Gujarati for monospace to support Gujarati in text
\setmonofont[Scale=0.9]{Noto Sans Gujarati}

% Configure English to use the same font
\newfontfamily\englishfont[Script=Gujarati,AutoFakeBold=2.5,AutoFakeSlant=0.3]{Noto Sans Gujarati}

% Translations for polyglossia
\gappto\captionsgujarati{
  \renewcommand{\tablename}{કોષ્ટક}
  \renewcommand{\figurename}{આકૃતિ}
}

% Helper for TikZ nodes to ensure Gujarati font
\newcommand{\gu}[1]{{\gujaratifont #1}}

% Custom environments
\newtcolorbox{solutionbox}{
    breakable,
    enhanced,
    colback=solutioncolor!5!white,
    colframe=solutioncolor!75!black,
    fonttitle=\bfseries,
    title=જવાબ
}

\newtcolorbox{solutionboxnobreak}{
 colback=solutioncolor!5!white,
 colframe=solutioncolor!75!black,
 fonttitle=\bfseries,
 title=જવાબ
}

\newtcolorbox{keyformula}{
 breakable,
 enhanced,
 colback=keycolor!5!white,
 colframe=keycolor!75!black,
 fonttitle=\bfseries,
 title=રાસાયણિક સમીકરણ/સૂત્ર
}

\newtcolorbox{mnemonicbox}{
 breakable,
 enhanced,
 colback=mnemoniccolor!5!white,
 colframe=mnemoniccolor!75!black,
 fonttitle=\bfseries,
 title=મેમરી ટ્રીક
}


\begin{document}

\begin{center}
{\Huge\bfseries\color{headcolor} Subject Name (Gujarati)}\\[5pt]
{\LARGE 4300005 -- Winter 2023}\\[3pt]
{\large Semester 1 Study Material}\\[3pt]
{\normalsize\textit{Detailed Solutions and Explanations}}
\end{center}

\vspace{10pt}

\subsection*{પ્રશ્ન 1(અ) [3
ગુણ]}\label{uxaaauxab0uxab6uxaa8-1uxa85-3-uxa97uxaa3}

\textbf{વ્યાખ્યા આપો: (અ) મીટર (બ) કેલ્વિન (ક) ચોકસાઇ.}

\begin{solutionbox}

\begin{itemize}
\tightlist
\item
  \textbf{મીટર}: મીટર એ લંબાઈનો SI એકમ છે, જેને 1/299,792,458 સેકન્ડના સમયગાળા
  દરમિયાન પ્રકાશ દ્વારા શૂન્યાવકાશમાં કાપવામાં આવતા અંતર તરીકે વ્યાખ્યાયિત કરવામાં
  આવે છે.
\item
  \textbf{કેલ્વિન}: કેલ્વિન એ થર્મોડાયનામિક તાપમાનનો SI એકમ છે, જે બોલ્ટ્ઝમાન
  અચળાંક k ની સ્થિર સંખ્યાત્મક કિંમત 1.380649 \times 10\^{}-23 J/K સેટ કરીને
  વ્યાખ્યાયિત કરવામાં આવે છે.
\item
  \textbf{ચોકસાઇ}: ચોકસાઇ એ માપવામાં આવતી જથ્થાની સાચી અથવા માનક કિંમતથી
  માપેલી કિંમતની નજીકતાની ડિગ્રી છે.
\end{itemize}

\end{solutionbox}
\begin{mnemonicbox}
``MKA - Meter measures Kilometers Accurately''

\end{mnemonicbox}
\subsection*{પ્રશ્ન 1(બ) [4
ગુણ]}\label{uxaaauxab0uxab6uxaa8-1uxaac-4-uxa97uxaa3}

\textbf{વર્નિયર કેલિપર્સની રચના સ્વચ્છ આકૃતિ દોરી સમજાવો.}

\begin{solutionbox}

\textbf{આકૃતિ:}

\begin{verbatim}
     |--|--|--|--|--|--|--|--|--|--|
     |--|--|--|     Main Scale    |--|
     |  |  |  |  |  |  |  |  |  |
     0  1  2  3  4  5  6  7  8  9  10 cm
        |--|--|--|--|--|--|--|--|--|--|
        |    Vernier Scale       |
        0  1  2  3  4  5  6  7  8  9  
\end{verbatim}

વર્નિયર કેલિપર્સમાં શામેલ છે:

\begin{itemize}
\tightlist
\item
  \textbf{મુખ્ય સ્કેલ}: માનક એકમોમાં ચિહ્નિત કરેલ સ્થિર સ્કેલ (mm અથવા ઇંચ)
\item
  \textbf{વર્નિયર સ્કેલ}: મુખ્ય સ્કેલ પર સરકી શકે તેવો હલનચલન સ્કેલ
\item
  \textbf{સ્થિર જડબું}: મુખ્ય સ્કેલ સાથે જોડાયેલ
\item
  \textbf{હલનચલન જડબું}: વર્નિયર સ્કેલ સાથે જોડાયેલ
\item
  \textbf{ઊંડાઈ પ્રોબ}: ખાડાની ઊંડાઈ માપવા માટે
\item
  \textbf{બાહ્ય જડબાં}: બાહ્ય પરિમાણો માપવા માટે
\item
  \textbf{આંતરિક જડબાં}: આંતરિક પરિમાણો માપવા માટે
\end{itemize}

\end{solutionbox}
\begin{mnemonicbox}
``FMMVJ - Fixed Main scale Makes Vernier Jaw move''

\end{mnemonicbox}
\subsection*{પ્રશ્ન 1(ક)(1) [4
ગુણ]}\label{uxaaauxab0uxab6uxaa8-1uxa951-4-uxa97uxaa3}

\textbf{ભૌતિક રાશિ એટલે શું છે? દિશાની દૃષ્ટિએ તેના પ્રકારો સમજાવો.}

\begin{solutionbox}

ભૌતિક રાશિ એ ભૌતિક સિસ્ટમની એક માપી શકાય તેવી સંપત્તિ છે જેને માપન દ્વારા
માત્રાત્મક કરી શકાય છે.

\textbf{દિશાના આધારે ભૌતિક રાશિઓના પ્રકારો:}

\begin{longtable}[]{@{}
  >{\raggedright\arraybackslash}p{(\linewidth - 2\tabcolsep) * \real{0.5000}}
  >{\raggedright\arraybackslash}p{(\linewidth - 2\tabcolsep) * \real{0.5000}}@{}}
\toprule\noalign{}
\begin{minipage}[b]{\linewidth}\raggedright
અદિશ રાશિઓ
\end{minipage} & \begin{minipage}[b]{\linewidth}\raggedright
સદિશ રાશિઓ
\end{minipage} \\
\midrule\noalign{}
\endhead
\bottomrule\noalign{}
\endlastfoot
માત્ર પરિમાણ ધરાવે છે & પરિમાણ અને દિશા બંને ધરાવે છે \\
ઉદાહરણો: દળ, સમય, તાપમાન, ઊર્જા & ઉદાહરણો: વિસ્થાપન, વેગ, બળ, પ્રવેગ \\
સરળ સંખ્યાઓ દ્વારા રજૂ થાય છે & તીર અથવા નિર્દેશિત રેખા ખંડો દ્વારા રજૂ થાય છે \\
સરવાળો સરળ અંકગણિતને અનુસરે છે & સરવાળો સદિશ બીજગણિતને અનુસરે છે (સમાંતર ચતુષ્કોણનો
નિયમ) \\
કોઈ દિશાત્મક ગુણધર્મો નથી & દિશા અને પરિમાણ દ્વારા સંપૂર્ણપણે નિર્દિષ્ટ છે \\
\end{longtable}

\end{solutionbox}
\begin{mnemonicbox}
``SMAVD - Scalars have Magnitude Alone, Vectors have
Direction''

\end{mnemonicbox}
\subsection*{પ્રશ્ન 1(ક)(2) [3
ગુણ]}\label{uxaaauxab0uxab6uxaa8-1uxa952-3-uxa97uxaa3}

\textbf{એક માઇક્રોમીટરની પેચ 0.5 mm છે. જો તેના વતુળાકાર ભાગ પર 100 વિભાગ છે,
તો તેની લઘુતમ માપવત્તા શોધો.}

\begin{solutionbox}

\textbf{ગણતરી:} લઘુતમ માપવત્તા (L.C.) = પેચ / વતુળાકાર સ્કેલ પરના વિભાગોની
સંખ્યા L.C. = 0.5 mm / 100 = 0.005 mm

તેથી, માઇક્રોમીટર સ્ક્રૂ ગેજની લઘુતમ માપવત્તા 0.005 mm છે.

\end{solutionbox}
\begin{mnemonicbox}
``PDL - Pitch Divided gives Least count''

\end{mnemonicbox}
\subsection*{પ્રશ્ન 1(ક) OR [7
ગુણ]}\label{uxaaauxab0uxab6uxaa8-1uxa95-or-7-uxa97uxaa3}

\textbf{માઇક્રોમીટર સ્ક્રૂ ગેજની ત્રુટીઓ આકૃતિ દોરી સમજાવો.}

\begin{solutionbox}

\textbf{આકૃતિ:}

\begin{verbatim}
    Ratchet   Barrel   Thimble
      |         |        |
      V         V        V
    [===]======|======[=====]
         \              /
          \            /
           \          /
            \        /
             \      /
              \    /
               Anvil
\end{verbatim}

માઇક્રોમીટર સ્ક્રૂ ગેજની સામાન્ય ત્રુટીઓ:

\begin{itemize}
\tightlist
\item
  \textbf{શૂન્ય ત્રુટિ}: જ્યારે માપન ફલકો સંપર્કમાં હોય, ત્યારે થિમ્બલનો શૂન્ય ડેટમ
  લાઇન સાથે મેળ ખાતો નથી

  \begin{itemize}
  \tightlist
  \item
    \textbf{ધન શૂન્ય ત્રુટિ}: જ્યારે થિમ્બલ પરનું શૂન્યનું ચિહ્ન ડેટમ લાઇનની નીચે હોય
  \item
    \textbf{ઋણ શૂન્ય ત્રુટિ}: જ્યારે થિમ્બલ પરનું શૂન્યનું ચિહ્ન ડેટમ લાઇનની ઉપર હોય
  \end{itemize}
\item
  \textbf{બેકલેશ ત્રુટિ}: સ્ક્રૂ અને નટ વચ્ચેનો ખેલ, આગળ અને પાછળના હલનચલનમાં અલગ
  રીડિંગ્સ થાય છે
\item
  \textbf{યંત્ર ત્રુટિ}: ઉત્પાદન ખામીઓ અથવા ઘસારાને કારણે
\item
  \textbf{પેરેલેક્સ ત્રુટિ}: જ્યારે દૃષ્ટિની લાઇન સ્કેલ રીડિંગને લંબરૂપ ન હોય
\end{itemize}

\textbf{સુધારા સૂત્ર:} સાચું રીડિંગ = અવલોકિત રીડિંગ - શૂન્ય ત્રુટિ

\end{solutionbox}
\begin{mnemonicbox}
``ZBIP - Zero, Backlash, Instrument and Parallax
errors make measurements trip''

\end{mnemonicbox}
\subsection*{પ્રશ્ન 2(અ) [3
ગુણ]}\label{uxaaauxab0uxab6uxaa8-2uxa85-3-uxa97uxaa3}

\textbf{કુલંબનો વ્યસ્ત વર્ગનો નિયમ સમજાવો.}

\begin{solutionbox}

કુલંબનો વ્યસ્ત વર્ગનો નિયમ કહે છે કે બે બિંદુ ચાર્જ વચ્ચેનું ઇલેક્ટ્રોસ્ટેટિક બળ:

\begin{itemize}
\tightlist
\item
  ચાર્જના પરિમાણના ગુણનફળના સીધા પ્રમાણમાં
\item
  તેમની વચ્ચેના અંતરના વર્ગના વ્યસ્ત પ્રમાણમાં
\item
  બે ચાર્જને જોડતી રેખા પર કાર્ય કરે છે
\end{itemize}

\textbf{ગણિતીય અભિવ્યક્તિ:} F = k(q_{1}q_{2})/r^{2}

જ્યાં:

\begin{itemize}
\tightlist
\item
  F = ચાર્જ વચ્ચેનું ઇલેક્ટ્રોસ્ટેટિક બળ
\item
  k = કુલંબનો અચળાંક (9 \times 10^{9} N·m^{2}/C^{2})
\item
  q_{1}, q_{2} = બે ચાર્જના પરિમાણ
\item
  r = ચાર્જ વચ્ચેનું અંતર
\end{itemize}

\end{solutionbox}
\begin{mnemonicbox}
``PDSA - Product of charges Directly, Square of
distance inversely, Along the line''

\end{mnemonicbox}
\subsection*{પ્રશ્ન 2(બ) [4
ગુણ]}\label{uxaaauxab0uxab6uxaa8-2uxaac-4-uxa97uxaa3}

\textbf{વિદ્યુત સ્થિતિમાનનો તફાવત સમજાવો.}

\begin{solutionbox}

વિદ્યુત સ્થિતિમાનનો તફાવત (વોલ્ટેજ) એ વિદ્યુત ક્ષેત્રમાં બે બિંદુઓની વચ્ચે ધન ટેસ્ટ ચાર્જને
ખસેડવામાં એકમ ચાર્જ દીઠ થતું કાર્ય છે.

\textbf{ગણિતીય અભિવ્યક્તિ:} V = W/q

જ્યાં:

\begin{itemize}
\tightlist
\item
  V = સ્થિતિમાનનો તફાવત (વોલ્ટ)
\item
  W = કરવામાં આવેલું કાર્ય (જૂલ)
\item
  q = ચાર્જ (કૂલંબ)
\end{itemize}

\textbf{મુખ્ય લક્ષણો:}

\begin{itemize}
\tightlist
\item
  વોલ્ટમાં માપવામાં આવે છે (V)
\item
  અદિશ રાશિ (માત્ર પરિમાણ ધરાવે છે)
\item
  પથ-સ્વતંત્ર (માત્ર પ્રારંભિક અને અંતિમ સ્થિતિ પર આધારિત)
\item
  એકમ ચાર્જ દીઠ ઊર્જાનું પ્રતિનિધિત્વ કરે છે
\end{itemize}

\end{solutionbox}
\begin{mnemonicbox}
``WPCS - Work Per Charge is what potential
difference Says''

\end{mnemonicbox}
\subsection*{પ્રશ્ન 2(ક) [7
ગુણ]}\label{uxaaauxab0uxab6uxaa8-2uxa95-7-uxa97uxaa3}

\textbf{કેપેસીટરનું શ્રેણીમાં તથા સમાંતર જોડાણમાટે સમતુલ્ય કેપેસિટન્સ વર્ણવો.}

\begin{solutionbox}

\textbf{શ્રેણી જોડાણ:}

\textbf{આકૃતિ:}

\begin{verbatim}
    -----||----||----||----- 
        C_{1}    C_{2}    C_{3}
\end{verbatim}

\begin{itemize}
\tightlist
\item
  જ્યારે કેપેસિટરો એકબીજાના છેડાથી જોડાયેલા હોય
\item
દરેક કેપેસિટર પર સમાન ચાર્જ:

Q = Q_{1} = Q_{2} = Q_{3}

\item
  કુલ પોટેન્શિયલ તફાવત: V = V_{1} + V_{2} + V_{3}
\item
  સમતુલ્ય કેપેસિટન્સ સૂત્ર: 1/C_{e}q = 1/C_{1} + 1/C_{2} + 1/C_{3} + \ldots{}
\item
  સમતુલ્ય કેપેસિટન્સ સૌથી નાના વ્યક્તિગત કેપેસિટન્સ કરતાં ઓછી હોય છે
\end{itemize}

\textbf{સમાંતર જોડાણ:}

\textbf{આકૃતિ:}

\begin{verbatim}
    -----||-----
         C_{1}     
    -----||-----
         C_{2}     
    -----||-----
         C_{3}     
\end{verbatim}

\begin{itemize}
\tightlist
\item
  જ્યારે કેપેસિટરો એક જ બે બિંદુઓ વચ્ચે જોડાયેલા હોય
\item
દરેક કેપેસિટર પર સમાન પોટેન્શિયલ તફાવત:

V = V_{1} = V_{2} = V_{3}

\item
  કુલ ચાર્જ: Q = Q_{1} + Q_{2} + Q_{3}
\item
  સમતુલ્ય કેપેસિટન્સ સૂત્ર: C_{e}q = C_{1} + C_{2} + C_{3} + \ldots{}
\item
  સમતુલ્ય કેપેસિટન્સ સૌથી મોટા વ્યક્તિગત કેપેસિટન્સ કરતાં વધુ હોય છે
\end{itemize}

\textbf{તુલનાત્મક કોષ્ટક:}

\begin{longtable}[]{@{}
  >{\raggedright\arraybackslash}p{(\linewidth - 4\tabcolsep) * \real{0.3793}}
  >{\raggedright\arraybackslash}p{(\linewidth - 4\tabcolsep) * \real{0.2759}}
  >{\raggedright\arraybackslash}p{(\linewidth - 4\tabcolsep) * \real{0.3448}}@{}}
\toprule\noalign{}
\begin{minipage}[b]{\linewidth}\raggedright
પરિમાણ
\end{minipage} & \begin{minipage}[b]{\linewidth}\raggedright
શ્રેણી
\end{minipage} & \begin{minipage}[b]{\linewidth}\raggedright
સમાંતર
\end{minipage} \\
\midrule\noalign{}
\endhead
\bottomrule\noalign{}
\endlastfoot
ચાર્જ & બધા કેપેસિટર પર સમાન & કેપેસિટન્સ અનુસાર વિતરિત \\
વોલ્ટેજ & કેપેસિટરો વચ્ચે વિભાજિત & બધા કેપેસિટર પર સમાન \\
સમતુલ્ય કેપેસિટન્સ & 1/C_{e}q = 1/C_{1} + 1/C_{2} + \ldots{} & C_{e}q = C_{1} + C_{2} +
\ldots{} \\
પરિણામી કેપેસિટન્સ & કોઈપણ વ્યક્તિગત C કરતાં નાની & કોઈપણ વ્યક્તિગત C કરતાં
મોટી \\
\end{longtable}

\end{solutionbox}
\begin{mnemonicbox}
``RAPS - Reciprocals Add in Parallel Sum''

\end{mnemonicbox}
\subsection*{પ્રશ્ન 2(અ) OR [3
ગુણ]}\label{uxaaauxab0uxab6uxaa8-2uxa85-or-3-uxa97uxaa3}

\textbf{વિદ્યુતક્ષેત્ર રેખાઓની લાક્ષણિકતાઓ લખો.}

\begin{solutionbox}

\textbf{વિદ્યુત ક્ષેત્ર રેખાઓની લાક્ષણિકતાઓ:}

\begin{itemize}
\tightlist
\item
  \textbf{દિશા}: હંમેશા ધન ચાર્જથી ઋણ ચાર્જ તરફ બતાવે છે
\item
  \textbf{પ્રકૃતિ}: ધન ચાર્જથી શરૂ થાય છે અને ઋણ ચાર્જ પર પૂરી થાય છે
\item
  \textbf{સાતત્ય}: ક્યારેય એકબીજાને છેદતી નથી
\item
  \textbf{ઘનતા}: નજીકની રેખાઓ વધુ મજબૂત વિદ્યુત ક્ષેત્ર સૂચવે છે
\item
  \textbf{લંબતા}: હંમેશા સમસ્થિતિમાન સપાટીઓને લંબ હોય છે
\item
  \textbf{આકાર}: સમાન ક્ષેત્રો માટે સીધી રેખાઓ, અસમાન ક્ષેત્રો માટે વક્ર
\item
  \textbf{ખુલ્લા/બંધ}: હંમેશા ખુલ્લા વક્રો, ચુંબકીય ક્ષેત્ર રેખાઓથી વિપરીત
\end{itemize}

\end{solutionbox}
\begin{mnemonicbox}
``DNCPS - Direction, Never cross, Closeness shows
strength, Perpendicular, Straight/curved''

\end{mnemonicbox}
\subsection*{પ્રશ્ન 2(બ) OR [4
ગુણ]}\label{uxaaauxab0uxab6uxaa8-2uxaac-or-4-uxa97uxaa3}

\textbf{વિદ્યુત ફ્લક્સ વિશે નોંધ લખો.}

\begin{solutionbox}

વિદ્યુત ફ્લક્સ એ આપેલા ક્ષેત્રફળમાંથી પસાર થતા વિદ્યુત ક્ષેત્રનું માપ છે.

\textbf{ગણિતીય અભિવ્યક્તિ:} Φ_{e} = E·A·cosθ

જ્યાં:

\begin{itemize}
\tightlist
\item
  Φ_{e} = વિદ્યુત ફ્લક્સ (N·m^{2}/C અથવા V·m)
\item
  E = વિદ્યુત ક્ષેત્ર તીવ્રતા (N/C અથવા V/m)
\item
  A = સપાટીનું ક્ષેત્રફળ (m^{2})
\item
  θ = વિદ્યુત ક્ષેત્ર અને સપાટીના લંબ વચ્ચેનો ખૂણો
\end{itemize}

\textbf{મુખ્ય લક્ષણો:}

\begin{itemize}
\tightlist
\item
  સદિશ રાશિ
\item
  SI એકમ ન્યૂટન-મીટર-વર્ગ પ્રતિ કૂલંબ (N·m^{2}/C) અથવા વોલ્ટ-મીટર (V·m)
\item
  સપાટીમાંથી પસાર થતી ક્ષેત્ર રેખાઓની સંખ્યાનું પ્રતિનિધિત્વ કરે છે
\item
  ક્ષેત્ર સપાટીને લંબ હોય ત્યારે મહત્તમ (θ = 0^\circ)
\item
  ક્ષેત્ર સપાટીને સમાંતર હોય ત્યારે શૂન્ય (θ = 90^\circ)
\end{itemize}

\end{solutionbox}
\begin{mnemonicbox}
``FACT - Flux = Area \times Cosθ \times Field sTreength''

\end{mnemonicbox}
\subsection*{પ્રશ્ન 2(ક) OR [7
ગુણ]}\label{uxaaauxab0uxab6uxaa8-2uxa95-or-7-uxa97uxaa3}

\textbf{કેપેસિટર અને કેપેસિટન્સ પર નોંધ લખો.}

\begin{solutionbox}

\textbf{કેપેસિટર:} કેપેસિટર એ એક વિદ્યુત ઘટક છે જે વિદ્યુત ચાર્જ અને વિદ્યુત ક્ષેત્રમાં
ઊર્જા સંગ્રહિત કરવા માટે રચાયેલ છે.

\textbf{મૂળભૂત રચના:}

\begin{verbatim}
    Plate 1      Plate 2
    ////////    ////////
    ////////    //////// — Dielectric
    ////////    ////////
    ////////    ////////
\end{verbatim}

\textbf{કેપેસિટન્સ:} આપેલા પોટેન્શિયલ તફાવત પર વિદ્યુત ચાર્જ સંગ્રહિત કરવાની
કેપેસિટરની ક્ષમતા.

\textbf{ગણિતીય અભિવ્યક્તિ:} C = Q/V

જ્યાં:

\begin{itemize}
\tightlist
\item
  C = કેપેસિટન્સ (ફેરાડ)
\item
  Q = વિદ્યુત ચાર્જ (કૂલંબ)
\item
  V = પોટેન્શિયલ તફાવત (વોલ્ટ)
\end{itemize}

\textbf{સમાંતર પ્લેટ કેપેસિટર માટે:} C = ε_{0}εᵣA/d

જ્યાં:

\begin{itemize}
\tightlist
\item
  ε_{0} = મુક્ત અવકાશની પરમિટિવિટી (8.85 \times 10^{-}^{1}^{2} F/m)
\item
  εᵣ = ડાયઇલેક્ટ્રિકની સાપેક્ષ પરમિટિવિટી
\item
  A = પ્લેટ્સ વચ્ચેના ઓવરલેપનું ક્ષેત્રફળ
\item
  d = પ્લેટ્સ વચ્ચેનું અંતર
\end{itemize}

\textbf{કેપેસિટન્સને અસર કરતા પરિબળો:}

\begin{itemize}
\tightlist
\item
  પ્લેટ ક્ષેત્રફળ સાથે વધે છે
\item
  પ્લેટ અલગતા સાથે ઘટે છે
\item
  ડાયઇલેક્ટ્રિક અચળાંક સાથે વધે છે
\end{itemize}

\textbf{કેપેસિટરના ઉપયોગો:}

\begin{itemize}
\tightlist
\item
  ઊર્જા સંગ્રહ
\item
  પાવર સપ્લાયમાં ફિલ્ટરિંગ
\item
  સમય ગણતરી સર્કિટ્સ
\item
  કપલિંગ અને ડિકપલિંગ
\item
  પાવર ફેક્ટર સુધારણા
\end{itemize}

\end{solutionbox}
\begin{mnemonicbox}
``QVAD - Quotient of charge and Voltage, affected by
Area and Distance''

\end{mnemonicbox}
\subsection*{પ્રશ્ન 3(અ) [3
ગુણ]}\label{uxaaauxab0uxab6uxaa8-3uxa85-3-uxa97uxaa3}

\textbf{વ્યાખ્યા આપો: (અ) ઉષ્માગમન (બ) કિલોકેલરી (ક) થર્મોમીટર.}

\begin{solutionbox}

\begin{itemize}
\tightlist
\item
  \textbf{ઉષ્માગમન}: માધ્યમની જરૂર વિના વિદ્યુતચુંબકીય તરંગોના રૂપમાં થર્મલ ઊર્જાનું
  સ્થાનાંતરણ, જે નિર્વાત અથવા પારદર્શક માધ્યમોમાં થાય છે.
\item
  \textbf{કિલોકેલરી}: 1000 કૅલરીના બરાબર ગરમીની ઊર્જાનો એકમ, જ્યાં એક કૅલરી એ
  પ્રમાણભૂત પરિસ્થિતિઓમાં 1 ગ્રામ પાણીનું તાપમાન 1^\circC વધારવા માટે જરૂરી ગરમીની
  માત્રા છે.
\item
  \textbf{થર્મોમીટર}: તાપમાન માપવા માટે વપરાતું સાધન જે ભૌતિક ગુણધર્મ (જેમ કે
  પારાનો વિસ્તાર) જે તાપમાન સાથે બદલાય છે તેના આધારે કાર્ય કરે છે.
\end{itemize}

\end{solutionbox}
\begin{mnemonicbox}
``RKT - Radiation needs no medium, Kilocalorie
measures energy, Thermometer shows temperature''

\end{mnemonicbox}
\subsection*{પ્રશ્ન 3(બ) [4
ગુણ]}\label{uxaaauxab0uxab6uxaa8-3uxaac-4-uxa97uxaa3}

\textbf{ઉષ્માવહનાંકનો નિયમ સમજાવો.}

\begin{solutionbox}

ઉષ્માવહનાંકનો નિયમ (ફોરિયરનો નિયમ) કહે છે કે પદાર્થ દ્વારા ઉષ્મા પ્રવાહનો દર:

\begin{itemize}
\tightlist
\item
  વિભાગના ક્ષેત્રફળના સીધા પ્રમાણમાં
\item
  તાપમાન ઢાળના સીધા પ્રમાણમાં
\item
  પદાર્થના થર્મલ વાહકતા પર આધારિત
\end{itemize}

\textbf{ગણિતીય અભિવ્યક્તિ:} Q/t = -kA(dT/dx)

જ્યાં:

\begin{itemize}
\tightlist
\item
  Q/t = ઉષ્મા પ્રવાહનો દર (J/s અથવા W)
\item
  k = પદાર્થની થર્મલ વાહકતા (W/m·K)
\item
  A = આડછેદનું ક્ષેત્રફળ (m^{2})
\item
  dT/dx = તાપમાન ઢાળ (K/m)
\item
  નકારાત્મક ચિહ્ન સૂચવે છે કે ઉષ્મા ઉચ્ચ તાપમાનથી નીચા તાપમાન તરફ વહે છે
\end{itemize}

\end{solutionbox}
\begin{mnemonicbox}
``GAKT - Gradient And area with K gives heat
Transfer''

\end{mnemonicbox}
\subsection*{પ્રશ્ન 3(ક)(1) [3
ગુણ]}\label{uxaaauxab0uxab6uxaa8-3uxa951-3-uxa97uxaa3}

\textbf{1 વ્યક્તિને 102^\circF તાવ છે. તો તે સેલ્સિયસ અને કેલ્વિનમાં કેટલો હશે?}

\begin{solutionbox}

\textbf{ફેરનહીટથી સેલ્સિયસમાં રૂપાંતર:} C = (F - 32) \times 5/9 C = (102 - 32) \times
5/9

C = 70 \times 5/9

C = 38.89^\circC


\textbf{સેલ્સિયસથી કેલ્વિનમાં રૂપાંતર:} K = C + 273.15 K = 38.89 + 273.15 K =
312.04 K

તેથી, 102^\circF = 38.89^\circC = 312.04 K

\end{solutionbox}
\begin{mnemonicbox}
``FSK - From Fahrenheit Subtract 32, multiply by
5/9, then add 273.15 for Kelvin''

\end{mnemonicbox}
\subsection*{પ્રશ્ન 3(ક)(2) [4
ગુણ]}\label{uxaaauxab0uxab6uxaa8-3uxa952-4-uxa97uxaa3}

\textbf{સેલ્સિયસ અને ફેરનહીટ માપક્રમ સમજાવો.}

\begin{solutionbox}

\textbf{સેલ્સિયસ અને ફેરનહીટ તાપમાન માપક્રમોની તુલના:}

\begin{longtable}[]{@{}lll@{}}
\toprule\noalign{}
પરિમાણ & સેલ્સિયસ માપક્રમ & ફેરનહીટ માપક્રમ \\
\midrule\noalign{}
\endhead
\bottomrule\noalign{}
\endlastfoot
પાણીનું હિમબિંદુ & 0^\circC & 32^\circF \\
પાણીનું ઉત્કલનબિંદુ & 100^\circC & 212^\circF \\
વિભાગોની સંખ્યા & 100 વિભાગો & 180 વિભાગો \\
વિકસાવનાર & એન્ડર્સ સેલ્સિયસ (1742) & ગેબ્રિયલ ફેરનહીટ (1724) \\
ઉપયોગ & વિશ્વભરના મોટાભાગના દેશોમાં & મુખ્યત્વે USA અને તેના પ્રદેશોમાં \\
સંબંધ &

C = (F - 32) \times 5/9 &

F = (C \times 9/5) + 32 \\

\end{longtable}

\textbf{આકૃતિ:}

\begin{verbatim}
Celsius     Fahrenheit
  100^\circC  —— 212^\circF  (Water boils)
    |          |
    |          |
    |          |
   0^\circC   —— 32^\circF   (Water freezes)
    |          |
  -17.8^\circC —— 0^\circF
\end{verbatim}

\end{solutionbox}
\begin{mnemonicbox}
``FBIC - Fahrenheit has Bigger numbers, Interval of
180, Conversion needs 5/9 or 9/5''

\end{mnemonicbox}
\subsection*{પ્રશ્ન 3(અ) OR [3
ગુણ]}\label{uxaaauxab0uxab6uxaa8-3uxa85-or-3-uxa97uxaa3}

\textbf{ઉષ્માધારીતા ની વ્યાખ્યા, એકમ અને સૂત્ર લખો.}

\begin{solutionbox}

\textbf{વ્યાખ્યા:} ઉષ્માધારીતા એ કોઈ પદાર્થના તાપમાનમાં એક ડિગ્રી (સેલ્સિયસ અથવા
કેલ્વિન) વધારવા માટે જરૂરી ઉષ્મા ઊર્જાની માત્રા છે.

\textbf{સૂત્ર:} C = Q/ΔT

જ્યાં:

\begin{itemize}
\tightlist
\item
  C = ઉષ્માધારીતા (J/K અથવા J/^\circC)
\item
  Q = આપવામાં આવેલી ઉષ્મા ઊર્જા (જૂલ)
\item
  ΔT = તાપમાનમાં ફેરફાર (K અથવા ^\circC)
\end{itemize}

\textbf{એકમ:} જૂલ પ્રતિ કેલ્વિન (J/K) અથવા જૂલ પ્રતિ ડિગ્રી સેલ્સિયસ (J/^\circC)

\end{solutionbox}
\begin{mnemonicbox}
``QTC - Quotient of heat and Temperature Change
gives heat capacity''

\end{mnemonicbox}
\subsection*{પ્રશ્ન 3(બ) OR [4
ગુણ]}\label{uxaaauxab0uxab6uxaa8-3uxaac-or-4-uxa97uxaa3}

\textbf{ઉષ્મા પ્રવાહની પદ્ધતિઓ સમજાવો}

\begin{solutionbox}

\textbf{ઉષ્મા પ્રવાહની ત્રણ પદ્ધતિઓ:}

\begin{longtable}[]{@{}
  >{\raggedright\arraybackslash}p{(\linewidth - 6\tabcolsep) * \real{0.1333}}
  >{\raggedright\arraybackslash}p{(\linewidth - 6\tabcolsep) * \real{0.2667}}
  >{\raggedright\arraybackslash}p{(\linewidth - 6\tabcolsep) * \real{0.2222}}
  >{\raggedright\arraybackslash}p{(\linewidth - 6\tabcolsep) * \real{0.3778}}@{}}
\toprule\noalign{}
\begin{minipage}[b]{\linewidth}\raggedright
પદ્ધતિ
\end{minipage} & \begin{minipage}[b]{\linewidth}\raggedright
વ્યાખ્યા
\end{minipage} & \begin{minipage}[b]{\linewidth}\raggedright
ઉદાહરણો
\end{minipage} & \begin{minipage}[b]{\linewidth}\raggedright
માધ્યમની જરૂરિયાત
\end{minipage} \\
\midrule\noalign{}
\endhead
\bottomrule\noalign{}
\endlastfoot
\textbf{વહન} & પદાર્થના મોટા ભાગના હલનચલન વિના સીધા અણુઓના અથડામણ દ્વારા
ઉષ્માનું સ્થાનાંતરણ & ધાતુના સળિયા દ્વારા ઉષ્મા, રસોઈના વાસણ & હા (ઘન પદાર્થ
પસંદગીયુક્ત) \\
\textbf{સંવહન} & ગરમ થયેલા કણોના એક વિસ્તારથી બીજા વિસ્તારમાં હલનચલન દ્વારા
ઉષ્માનું સ્થાનાંતરણ & ઉકળતું પાણી, રૂમ હીટર, સમુદ્રી પવન & હા (પ્રવાહી - તરલ અથવા
વાયુ) \\
\textbf{વિકિરણ} & માધ્યમની જરૂરિયાત વિના વિદ્યુતચુંબકીય તરંગો દ્વારા ઉષ્માનું
સ્થાનાંતરણ & સૌર વિકિરણ, માઇક્રોવેવ હીટિંગ, ઇન્ફ્રારેડ હીટર & ના (નિર્વાતમાં કાર્ય
કરે છે) \\
\end{longtable}

\end{solutionbox}
\begin{mnemonicbox}
``CoCRa - Conduction needs Contact, Convection needs
Currents, Radiation needs no medium''

\end{mnemonicbox}
\subsection*{પ્રશ્ન 3(ક) OR [7
ગુણ]}\label{uxaaauxab0uxab6uxaa8-3uxa95-or-7-uxa97uxaa3}

\textbf{બાયમેટાલિક થર્મોમીટર સમજાવો.}

\begin{solutionbox}

\textbf{આકૃતિ:}

\begin{verbatim}
                  Pointer
                     |
                     V
                   /---\
                  /     \
    Fixed end    /       \    Movement
    |-----------|         |--------------|
    |///////////|         |//////////////|
    |^^^^^^^^^^^|         |^^^^^^^^^^^^^^| <- Metal 1 (higher expansion)
    |-----------|         |--------------|
                 \       /
                  \     /
                   \---/
                   Scale
\end{verbatim}

\textbf{કાર્ય સિદ્ધાંત:}

\begin{itemize}
\tightlist
\item
  બે અલગ-અલગ ધાતુઓના અસમાન થર્મલ વિસ્તરણ પર આધારિત
\item
  બે ધાતુની પટ્ટીઓ, જેમાં થર્મલ વિસ્તરણના અલગ-અલગ ગુણાંકો હોય છે, તેને એકસાથે જોડવામાં
  આવે છે
\item
  ગરમ થતાં, એક ધાતુ બીજી કરતાં વધુ ફેલાય છે
\item
  આ અસમાન વિસ્તરણને કારણે પટ્ટી ઓછા વિસ્તરણવાળી ધાતુ તરફ વળે છે
\item
  વળવાની માત્રા તાપમાન ફેરફારના પ્રમાણમાં હોય છે
\item
  પટ્ટી સાથે જોડાયેલ એક પોઇન્ટર અંશાંકિત સ્કેલ પર તાપમાન દર્શાવે છે
\end{itemize}

\textbf{ફાયદા:}

\begin{itemize}
\tightlist
\item
  સરળ, મજબૂત બાંધકામ
\item
  કોઈ પ્રવાહી કે વાયુની જરૂર નથી
\item
  વિશાળ તાપમાન શ્રેણી
\item
  યાંત્રિક આઘાતોનો પ્રતિકાર કરે છે
\item
  થર્મોસ્ટેટ બનાવવા માટે વાપરી શકાય છે
\end{itemize}

\textbf{મર્યાદાઓ:}

\begin{itemize}
\tightlist
\item
  પ્રવાહી-ઇન-ગ્લાસ થર્મોમીટર કરતાં ઓછું ચોક્કસ
\item
  તાપમાન ફેરફારો માટે ધીમી પ્રતિક્રિયા
\item
  સમય જતાં યાંત્રિક થાક વિષય
\end{itemize}

\textbf{ઉપયોગો:}

\begin{itemize}
\tightlist
\item
  ઘરના હીટિંગ/કૂલિંગ સિસ્ટમમાં થર્મોસ્ટેટ
\item
  ઓટોમોબાઇલ કૂલિંગ સિસ્ટમ
\item
  ઓવન તાપમાન નિયંત્રણો
\item
  સર્કિટ બ્રેકર
\end{itemize}

\end{solutionbox}
\begin{mnemonicbox}
``BENDS - Bimetallic strips Expand, Not equally,
Different metals, Show temperature''

\end{mnemonicbox}
\subsection*{પ્રશ્ન 4(અ) [3
ગુણ]}\label{uxaaauxab0uxab6uxaa8-4uxa85-3-uxa97uxaa3}

\textbf{વ્યાખ્યા આપો: (અ) આવૃત્તિ (બ) ઇન્ફ્રાસોનિક તરંગો (ક) પડઘો.}

\begin{solutionbox}

\begin{itemize}
\tightlist
\item
  \textbf{આવૃત્તિ}: એકમ સમયમાં પૂર્ણ થતા આંદોલનો અથવા ચક્રોની સંખ્યા, હર્ટ્ઝ (Hz)માં
  માપવામાં આવે છે.
\item
  \textbf{ઇન્ફ્રાસોનિક તરંગો}: માનવ સાંભળવાની નીચલી મર્યાદા (20 Hz નીચે)ની
  આવૃત્તિઓવાળા ધ્વનિ તરંગો જે માણસો દ્વારા સાંભળી શકાતા નથી પરંતુ અન્ય પ્રાણીઓ
  દ્વારા શોધી શકાય છે.
\item
  \textbf{પડઘો}: એક અવાજ જે શ્રોતા તરફ પાછો પરાવર્તિત થાય છે અને મૂળ ધ્વનિના
  અલગ પુનરાવર્તન તરીકે સાંભળવા માટે પૂરતા સમયના વિલંબ સાથે આવે છે.
\end{itemize}

\end{solutionbox}
\begin{mnemonicbox}
``FIE - Frequency counts cycles, Infrasonic is below
hearing, Echo comes back after reflection''

\end{mnemonicbox}
\subsection*{પ્રશ્ન 4(બ) [4
ગુણ]}\label{uxaaauxab0uxab6uxaa8-4uxaac-4-uxa97uxaa3}

\textbf{લંબગત તરંગ અને સંગત તરંગ વચ્ચેનો તફાવત આપો.}

\begin{solutionbox}

\textbf{લંબગત અને સંગત તરંગો વચ્ચે તુલના:}

\begin{longtable}[]{@{}
  >{\raggedright\arraybackslash}p{(\linewidth - 4\tabcolsep) * \real{0.2292}}
  >{\raggedright\arraybackslash}p{(\linewidth - 4\tabcolsep) * \real{0.3958}}
  >{\raggedright\arraybackslash}p{(\linewidth - 4\tabcolsep) * \real{0.3750}}@{}}
\toprule\noalign{}
\begin{minipage}[b]{\linewidth}\raggedright
પરિમાણ
\end{minipage} & \begin{minipage}[b]{\linewidth}\raggedright
લંબગત તરંગો
\end{minipage} & \begin{minipage}[b]{\linewidth}\raggedright
સંગત તરંગો
\end{minipage} \\
\midrule\noalign{}
\endhead
\bottomrule\noalign{}
\endlastfoot
\textbf{કણના હલનચલનની દિશા} & તરંગ પ્રસરણને સમાંતર & તરંગ પ્રસરણને લંબરૂપ \\
\textbf{ઉદાહરણ} & ધ્વનિ તરંગો, ભૂકંપમાં P-તરંગો & પ્રકાશ તરંગો, પાણીની સપાટી પર
તરંગો, ભૂકંપમાં S-તરંગો \\
\textbf{માધ્યમની જરૂરિયાત} & ઘન, પ્રવાહી અને વાયુઓ દ્વારા પ્રવાસ કરી શકે છે & ઘન
અને પ્રવાહીઓની સપાટી દ્વારા પ્રવાસ કરી શકે છે પરંતુ વાયુઓ દ્વારા નહીં \\
\textbf{ઘટકો} & સંકોચન અને વિરલીકરણ & શિખર અને ખીણ \\
\textbf{ધ્રુવીકરણ} & ધ્રુવીકૃત થઈ શકતા નથી & ધ્રુવીકૃત થઈ શકે છે \\
\textbf{દૃશ્યમાનતા} & સંકોચિત અને વિસ્તૃત સ્પ્રીંગ અથવા સ્લિંકી જેવા & ઉપર-નીચે હલતી
દોરડી જેવા \\
\end{longtable}

\textbf{આકૃતિ:}

\begin{verbatim}
Longitudinal: -->-->-->-->-->--> (Direction of propagation)
              <--><--><--><-->   (Particle movement)
              
Transverse:   -->-->-->-->-->--> (Direction of propagation)
                ↑   ↓   ↑   ↓    (Particle movement)
\end{verbatim}

\end{solutionbox}
\begin{mnemonicbox}
``PPCP - Particles move Parallel in Longitudinal,
Perpendicular in Transverse, Compressions vs Crests, Polarization only
in Transverse''

\end{mnemonicbox}
\subsection*{પ્રશ્ન 4(ક)(1) [4
ગુણ]}\label{uxaaauxab0uxab6uxaa8-4uxa951-4-uxa97uxaa3}

\textbf{અલ્ટ્રાસોનિક તરંગોના ત્રણ ગુણધર્મો અને ઉપયોગો આપો.}

\begin{solutionbox}

\textbf{અલ્ટ્રાસોનિક તરંગોના ગુણધર્મો:}

\begin{itemize}
\tightlist
\item
  20,000 Hz ઉપરની આવૃત્તિ શ્રેણી (માનવ શ્રવણની બહાર)
\item
  ટૂંકી તરંગલંબાઈઓ નાના પદાર્થોના શોધવા માટે મદદ કરે છે
\item
  સાંભળી શકાય તેવા ધ્વનિની તુલનામાં ઉચ્ચ દિશાનિર્દેશતા
\item
  ચોક્કસ માધ્યમોમાં ઉચ્ચ પ્રવેશ
\item
  અવરોધોની આસપાસ ઓછું વિવર્તન
\item
  પ્રવાહીઓમાં ગુહાકરણ થાય છે
\end{itemize}

\textbf{અલ્ટ્રાસોનિક તરંગોના ઉપયોગો:}

\begin{longtable}[]{@{}ll@{}}
\toprule\noalign{}
ક્ષેત્ર & ઉપયોગો \\
\midrule\noalign{}
\endhead
\bottomrule\noalign{}
\endlastfoot
\textbf{તબીબી} & સોનોગ્રાફી, કિડની સ્ટોન વિનાશ, ફિઝિયોથેરાપી \\
\textbf{ઔદ્યોગિક} & બિન-વિનાશક પરીક્ષણ, સફાઈ, વેલ્ડિંગ, ડ્રિલિંગ \\
\textbf{નેવિગેશન} & SONAR, અંતર માપન, અવરોધ શોધ \\
\textbf{અન્ય} & કૂતરા સીટી, જીવજંતુ નિયંત્રણ, ધ્વનિ સ્થાનનિર્ધારણ \\
\end{longtable}

\end{solutionbox}
\begin{mnemonicbox}
``FWD-MNO - Frequency high, Wavelength short,
Direction focused; Medical imaging, NDT testing, Ocean mapping''

\end{mnemonicbox}
\subsection*{પ્રશ્ન 4(ક)(2) [3
ગુણ]}\label{uxaaauxab0uxab6uxaa8-4uxa952-3-uxa97uxaa3}

\textbf{ધ્વનિ તરંગના વેગ, તરંગલંબાઈ અને આવૃત્તિ વચ્ચેનો સંબંધ તારવો.}

\begin{solutionbox}

\textbf{સિદ્ધાંત:}

એક તરંગને ધ્યાનમાં લો જેમાં:

\begin{itemize}
\tightlist
\item
  તરંગલંબાઈ (λ): સમાન બિંદુઓ વચ્ચેનું અંતર
\item
  આવૃત્તિ (f): એક સેકન્ડમાં કોઈ બિંદુમાંથી પસાર થતા તરંગોની સંખ્યા
\item
  આવર્તકાળ (T): એક ચક્ર પૂર્ણ કરવા માટેનો સમય
\end{itemize}

એક આવર્તકાળ (T) દરમિયાન, તરંગ એક તરંગલંબાઈ (λ)ના અંતરને કાપે છે.

તેથી, વેગ = અંતર/સમય = λ/T

આવૃત્તિ f = 1/T હોવાથી, આપણે લખી શકીએ:

v = λ \times f

જ્યાં:

\begin{itemize}
\tightlist
\item
  v = તરંગનો વેગ (m/s)
\item
  λ = તરંગલંબાઈ (m)
\item
  f = આવૃત્તિ (Hz)
\end{itemize}

\textbf{આકૃતિ:}

\begin{verbatim}
    λ
<--------->
 ___       ___       ___
/   \     /   \     /   \
     \___/     \___/     
     
v = λ \times f
\end{verbatim}

\end{solutionbox}
\begin{mnemonicbox}
``VLF - Velocity equals Lambda times Frequency''

\end{mnemonicbox}
\subsection*{પ્રશ્ન 4(અ) OR [3
ગુણ]}\label{uxaaauxab0uxab6uxaa8-4uxa85-or-3-uxa97uxaa3}

\textbf{પ્રતિઘોષ સમય માટેનું સેબાઇનનું સૂત્ર સમજાવો.}

\begin{solutionbox}

સેબાઇનનું સૂત્ર બંધ જગ્યામાં પ્રતિઘોષ સમયની ગણતરી કરે છે:

\textbf{સૂત્ર:} RT_{6}_{0} = 0.161 \times V/A

જ્યાં:

\begin{itemize}
\tightlist
\item
  RT_{6}_{0} = પ્રતિઘોષ સમય (સેકન્ડ) ધ્વનિને 60 dB ઘટાડવા માટે
\item
  V = રૂમનું કદ (m^{3})
\item
  A = કુલ ધ્વનિ શોષણ (m^{2} sabins)
\item
  0.161 = અચળાંક (મેટ્રિક એકમોમાં ગણતરી માટે)
\end{itemize}

\textbf{કુલ શોષણ (A)} ની ગણતરી આ રીતે થાય છે: A = α_{1}S_{1} + α_{2}S_{2} + α_{3}S_{3} +
\ldots{} + α_{n}S_{n}

જ્યાં:

\begin{itemize}
\tightlist
\item
  αᵢ = પદાર્થ i નો શોષણ ગુણાંક
\item
  Sᵢ = પદાર્થ i નું સપાટી ક્ષેત્રફળ (m^{2})
\end{itemize}

\textbf{ઉપયોગો:}

\begin{itemize}
\tightlist
\item
  કોન્સર્ટ હોલ, ઓડિટોરિયમ, રેકોર્ડિંગ સ્ટુડિયોની ધ્વનિક ડિઝાઇન
\item
  જરૂરી ધ્વનિક ઉપચારની નિર્ધારણ
\item
  મૌજૂદા જગ્યાઓની ધ્વનિક ગુણવત્તાનું મૂલ્યાંકન
\end{itemize}

\end{solutionbox}
\begin{mnemonicbox}
``VAS - Volume And Surface absorption determine
reverberation time''

\end{mnemonicbox}
\subsection*{પ્રશ્ન 4(બ) OR [4
ગુણ]}\label{uxaaauxab0uxab6uxaa8-4uxaac-or-4-uxa97uxaa3}

\textbf{પ્રકાશનું વિવર્તન એટલે શું? તેના પ્રકાર આકૃતિ સાથે સમજાવો.}

\begin{solutionbox}

\textbf{વ્યાખ્યા:} વિવર્તન એ અવરોધોની આસપાસ અથવા ખુલ્લી જગ્યાઓમાંથી પ્રકાશ તરંગોનું
વળવું છે, જે પ્રકાશના તરંગ સ્વભાવને દર્શાવે છે.

\textbf{વિવર્તનના પ્રકારો:}

\textbf{1. ફ્રેસનેલ વિવર્તન:}

\begin{itemize}
\tightlist
\item
  સ્ત્રોત અથવા સ્ક્રીન (અથવા બંને) અવરોધથી મર્યાદિત અંતરે
\item
  ગોળાકાર તરંગાગ્રો
\item
  વધુ જટિલ હસ્તક્ષેપ પેટર્ન
\end{itemize}

\textbf{આકૃતિ:}

\begin{verbatim}
Source                 Screen
  •                      ┃
   \     __________      ┃
    \   |          |     ┃
     \  |  Opening |     ┃
      \ |__________|     ┃
       \                 ┃
        \                ┃
         \               ┃
\end{verbatim}

\textbf{2. ફ્રૌનહોફર વિવર્તન:}

\begin{itemize}
\tightlist
\item
  સ્ત્રોત અને સ્ક્રીન અનંત અંતરે (અથવા અસરકારક રીતે લેન્સનો ઉપયોગ કરીને)
\item
  સમતલ તરંગાગ્રો
\item
  સરળ હસ્તક્ષેપ પેટર્ન
\item
  પ્રાથમિક ભૌતિકશાસ્ત્રમાં વધુ સામાન્યપણે અભ્યાસ કરવામાં આવે છે
\end{itemize}

\textbf{આકૃતિ:}

\begin{verbatim}
Plane                      Screen
waves   __________          ┃
\rightarrow\rightarrow\rightarrow\rightarrow\rightarrow\rightarrow\rightarrow|          |         ┃
\rightarrow\rightarrow\rightarrow\rightarrow\rightarrow\rightarrow\rightarrow|  Opening |\rightarrow\rightarrow\rightarrow\rightarrow\rightarrow\rightarrow\rightarrow\rightarrow\rightarrow┃
\rightarrow\rightarrow\rightarrow\rightarrow\rightarrow\rightarrow\rightarrow|__________|         ┃
                            ┃
\end{verbatim}

\end{solutionbox}
\begin{mnemonicbox}
``FPSS - Fresnel has Finite distances, Spherical
waves; Fraunhofer has Source at infinity, Straight (plane) waves''

\end{mnemonicbox}
\subsection*{પ્રશ્ન 4(ક)(1) OR [3
ગુણ]}\label{uxaaauxab0uxab6uxaa8-4uxa951-or-3-uxa97uxaa3}

\textbf{એક રેડિયોતરંગની આવૃત્તિ 480 Hz અને ધ્વનિનો વેગ 330 m/s હોય તો તરંગલંબાઈ
શોધો.}

\begin{solutionbox}

\textbf{આપેલ છે:}

\begin{itemize}
\tightlist
\item
  આવૃત્તિ (f) = 480 Hz
\item
  ધ્વનિનો વેગ (v) = 330 m/s
\end{itemize}

\textbf{શોધવાનું છે:} તરંગલંબાઈ (λ)

\textbf{સૂત્ર:} v = λ \times f

\textbf{ગણતરી:} λ = v/f λ = 330 m/s \div 480 Hz λ = 0.6875 m λ = 68.75 cm

તેથી, રેડિયો તરંગની તરંગલંબાઈ 0.6875 m અથવા 68.75 cm છે.

\end{solutionbox}
\begin{mnemonicbox}
``WFV - Wavelength equals Velocity divided by
Frequency''

\end{mnemonicbox}
\subsection*{પ્રશ્ન 4(ક)(2) OR [4
ગુણ]}\label{uxaaauxab0uxab6uxaa8-4uxa952-or-4-uxa97uxaa3}

\textbf{ધ્વનિ તરંગોના ગુણધર્મો આપો}

\begin{solutionbox}

\textbf{ધ્વનિ તરંગોના ગુણધર્મો:}

\begin{longtable}[]{@{}
  >{\raggedright\arraybackslash}p{(\linewidth - 2\tabcolsep) * \real{0.4348}}
  >{\raggedright\arraybackslash}p{(\linewidth - 2\tabcolsep) * \real{0.5652}}@{}}
\toprule\noalign{}
\begin{minipage}[b]{\linewidth}\raggedright
ગુણધર્મ
\end{minipage} & \begin{minipage}[b]{\linewidth}\raggedright
વર્ણન
\end{minipage} \\
\midrule\noalign{}
\endhead
\bottomrule\noalign{}
\endlastfoot
\textbf{તરંગ સ્વભાવ} & ધ્વનિ એક યાંત્રિક, લંબગત તરંગ છે જેને માધ્યમની જરૂર પડે છે \\
\textbf{આવૃત્તિ શ્રેણી} & માનવો માટે સાંભળી શકાય તેવી શ્રેણી: 20 Hz થી 20,000
Hz \\
\textbf{વેગ} & રૂમ તાપમાને હવામાં \textasciitilde343 m/s; માધ્યમ સાથે બદલાય
છે \\
\textbf{પરાવર્તન} & સપાટીઓ પરથી પરાવર્તિત થાય છે, પડઘા અને પ્રતિધ્વનિ બનાવે
છે \\
\textbf{વક્રીભવન} & અલગ-અલગ ઘનતાના માધ્યમોની વચ્ચે પસાર થતી વખતે દિશા બદલે
છે \\
\textbf{વિવર્તન} & અવરોધોની આસપાસ અને ખુલ્લી જગ્યાઓમાંથી વળે છે \\
\textbf{વ્યતિકરણ} & તરંગો એકબીજા પર ઉપરાઇ રચનાત્મક અથવા વિનાશક વ્યતિકરણ
બનાવી શકે છે \\
\textbf{અનુનાદ} & પદાર્થોની કુદરતી આવૃત્તિઓએ વર્ધન \\
\end{longtable}

\textbf{ધ્વનિના વેગને અસર કરતા પરિબળો:}

\begin{itemize}
\tightlist
\item
  વાયુઓમાં તાપમાન સાથે વધે છે
\item
  વાયુઓ કરતાં પ્રવાહીઓમાં ઝડપી
\item
  ઘન પદાર્થોમાં સૌથી ઝડપી
\item
  આપેલા માધ્યમમાં આવૃત્તિ અને આયામથી સ્વતંત્ર
\end{itemize}

\end{solutionbox}
\begin{mnemonicbox}
``WARDS-FIR - Wave needs medium, Audible range
limited, Reflected, Diffracted, Speed varies, Frequency determines
pitch, Intensity determines loudness, Resonates at natural frequencies''

\end{mnemonicbox}
\subsection*{પ્રશ્ન 5(અ) [3
ગુણ]}\label{uxaaauxab0uxab6uxaa8-5uxa85-3-uxa97uxaa3}

\textbf{લેસરનો અર્થ અને ગુણધર્મો જણાવો.}

\begin{solutionbox}

\textbf{LASER}: Light Amplification by Stimulated Emission of Radiation
(પ્રેરિત ઉત્સર્જન દ્વારા પ્રકાશનું વર્ધન)

\textbf{લેસર પ્રકાશના ગુણધર્મો:}

\begin{itemize}
\tightlist
\item
  \textbf{એકવર્ણીય}: એક તરંગલંબાઈ અથવા તરંગલંબાઈઓની ખૂબ સાંકડી પટ્ટી
\item
  \textbf{સુસંબદ્ધ}: બધા તરંગો એકબીજા સાથે કળામાં હોય છે
\item
  \textbf{દિશાત્મક}: નીચું વિચલન, ન્યૂનતમ ફેલાવા સાથે સીધી રેખામાં પ્રવાસ કરે છે
\item
  \textbf{તીવ્ર}: નાના વિસ્તારમાં ઉચ્ચ ઊર્જા કેન્દ્રિકરણ
\item
  \textbf{સમાંતર}: પ્રકાશ કિરણો ન્યૂનતમ વિચલન સાથે સમાંતર હોય છે
\end{itemize}

\end{solutionbox}
\begin{mnemonicbox}
``MCCDI - Monochromatic and Coherent, Collimated,
Directional, Intense''

\end{mnemonicbox}
\subsection*{પ્રશ્ન 5(બ) [4
ગુણ]}\label{uxaaauxab0uxab6uxaa8-5uxaac-4-uxa97uxaa3}

\textbf{ઓપ્ટિકલ ફાઈબર વિષે માહિતી આપો.}

\begin{solutionbox}

\textbf{ઓપ્ટિકલ ફાઈબર}: એક લવચીક, પારદર્શક ફાઈબર જે કાચ અથવા પ્લાસ્ટિકથી બનેલી
હોય છે જે સંપૂર્ણ આંતરિક પરાવર્તન દ્વારા પ્રકાશ સિગ્નલો પ્રસારિત કરે છે.

\textbf{રચના:}

\begin{verbatim}
       ┌───────────┐
       │           │
       │  Core     │  n_{1} (Higher refractive index)
       │           │
┌──────┴───────────┴──────┐
│                         │
│      Cladding           │  n_{2} (Lower refractive index)
│                         │
└─────────────────────────┘
       Protective coating
\end{verbatim}

\textbf{ઘટકો:}

\begin{itemize}
\tightlist
\item
  \textbf{કોર}: કેન્દ્રીય વિસ્તાર જ્યાં પ્રકાશ પ્રવાસ કરે છે (ઉચ્ચ વક્રીભવનાંક)
\item
  \textbf{ક્લેડિંગ}: કોરની આજુબાજુનું બાહ્ય ઓપ્ટિકલ પદાર્થ (નીચો વક્રીભવનાંક)
\item
  \textbf{બફર કોટિંગ}: રક્ષણાત્મક બાહ્ય આવરણ
\end{itemize}

\textbf{પ્રકારો:}

\begin{itemize}
\tightlist
\item
  \textbf{સિંગલ-મોડ}: નાનો કોર (8-10 μm), ફક્ત એક મોડ વહન કરે છે
\item
  \textbf{મલ્ટી-મોડ}: મોટો કોર (50-100 μm), બહુવિધ મોડ વહન કરે છે

  \begin{itemize}
  \tightlist
  \item
    \textbf{સ્ટેપ-ઇન્ડેક્સ}: વક્રીભવનાંકમાં અચાનક ફેરફાર
  \item
    \textbf{ગ્રેડેડ-ઇન્ડેક્સ}: વક્રીભવનાંકમાં ક્રમિક ફેરફાર
  \end{itemize}
\end{itemize}

\textbf{ફાયદા:}

\begin{itemize}
\tightlist
\item
  ઊંચી બેન્ડવિડ્થ અને ડેટા ટ્રાન્સમિશન દર
\item
  ઇલેક્ટ્રોમેગ્નેટિક હસ્તક્ષેપથી મુક્ત
\item
  લાંબા અંતર પર ઓછું સિગ્નલ ક્ષીણન
\item
  નાનું કદ અને હલકું વજન
\item
  વધારેલી સુરક્ષા (ટેપ કરવામાં મુશ્કેલ)
\end{itemize}

\end{solutionbox}
\begin{mnemonicbox}
``CCTLT - Core Carries light, Cladding keeps it in,
Total internal reflection, Low loss transmission''

\end{mnemonicbox}
\subsection*{પ્રશ્ન 5(ક)(1) [7
ગુણ]}\label{uxaaauxab0uxab6uxaa8-5uxa951-7-uxa97uxaa3}

\textbf{સ્નેલનો નિયમ સમજાવો.}

\begin{solutionbox}

\textbf{વ્યાખ્યા:} સ્નેલનો નિયમ (વક્રીભવનનો નિયમ) કહે છે કે આપતિના ખૂણાના સાઇનનો
વક્રીભવનના ખૂણાના સાઇન સાથેનો ગુણોત્તર કોઈપણ બે ચોક્કસ માધ્યમો માટે અચળ રહે છે.

\textbf{સૂત્ર:} n_{1}sin(θ_{1}) = n_{2}sin(θ_{2})

જ્યાં:

\begin{itemize}
\tightlist
\item
  n_{1} = માધ્યમ 1 નો વક્રીભવનાંક
\item
  θ_{1} = આપતિનો ખૂણો
\item
  n_{2} = માધ્યમ 2 નો વક્રીભવનાંક
\item
  θ_{2} = વક્રીભવનનો ખૂણો
\end{itemize}

\textbf{આકૃતિ:}

\begin{verbatim}
              Normal
                │
                │
                │
Medium 1 (n_{1})   │       θ_{1}
                │      /
                │     /
                │    /
                │   /
                │  /
----------------│-/---------------- Boundary
                │/ θ_{2}
                /│
               / │
              /  │
             /   │
Medium 2 (n_{2})    │
                 │
\end{verbatim}

\textbf{ઉદાહરણો:}

\begin{itemize}
\tightlist
\item
  હવામાંથી પાણીમાં પ્રવેશ કરતી વખતે પ્રકાશનું વળવું
\item
  પાણીની અંદરની વસ્તુઓનું દેખીતું વિસ્થાપન
\item
  મેઘધનુષની રચના
\item
  લેન્સ અને પ્રિઝમની ડિઝાઇન
\end{itemize}

\textbf{વિશેષ કિસ્સાઓ:}

\begin{itemize}
\tightlist
\item
  જ્યારે પ્રકાશ ઓછા ઘન માધ્યમથી વધુ ઘન માધ્યમમાં પ્રવાસ કરે છે (n_{1} \textless{}
  n_{2}), તે લંબ તરફ વળે છે (θ_{1} \textgreater{} θ_{2})
\item
  જ્યારે પ્રકાશ વધુ ઘન માધ્યમથી ઓછા ઘન માધ્યમમાં પ્રવાસ કરે છે (n_{1} \textgreater{}
  n_{2}), તે લંબથી દૂર વળે છે (θ_{1} \textless{} θ_{2})
\item
  જ્યારે આપતિનો ખૂણો 0^\circ (લંબ આપતિ) હોય, ત્યારે કોઈ વક્રીભવન થતું નથી
\end{itemize}

\end{solutionbox}
\begin{mnemonicbox}
``SINS - Sine of incidence over sine of refraction
equals N_{1} over N_{2}''

\end{mnemonicbox}
\subsection*{પ્રશ્ન 5(ક)(2) [0
ગુણ]}\label{uxaaauxab0uxab6uxaa8-5uxa952-0-uxa97uxaa3}

\textbf{એસેપ્ટન્સ એંગલ સમજાવો.}

\begin{solutionbox}

\textbf{એસેપ્ટન્સ એંગલ} એ મહત્તમ ખૂણો છે જેના પર પ્રકાશ ઓપ્ટિકલ ફાઈબરમાં પ્રવેશી શકે છે
અને હજુ પણ સંપૂર્ણ આંતરિક પરાવર્તન અનુભવી શકે છે.

\textbf{સૂત્ર:} θ_{a} = sin^{-}^{1}(NA)

જ્યાં:

\begin{itemize}
\tightlist
\item
  θ_{a} = એસેપ્ટન્સ એંગલ
\item
  NA = ન્યુમેરિકલ એપર્ચર
\end{itemize}

\textbf{ન્યુમેરિકલ એપર્ચર (NA):} NA = \sqrt(n_{1}^{2} - n_{2}^{2})

જ્યાં:

\begin{itemize}
\tightlist
\item
  n_{1} = કોરનો વક્રીભવનાંક
\item
  n_{2} = ક્લેડિંગનો વક્રીભવનાંક
\end{itemize}

\textbf{આકૃતિ:}

\begin{verbatim}
              Acceptance cone
                   /\
                  /  \
                 /    \
                /      \
               /        \
              /    θ_{a}    \
             /____________\
            ┌──────────────┐
            │   Core       │
            │              │
            └──────────────┘
                Fiber
\end{verbatim}

\textbf{મહત્વ:}

\begin{itemize}
\tightlist
\item
  ફાઈબરની પ્રકાશ-એકત્રિત કરવાની ક્ષમતા નક્કી કરે છે
\item
  મોટો એસેપ્ટન્સ એંગલ એટલે વધુ પ્રકાશ ફાઈબરમાં પ્રવેશી શકે છે
\item
  ફાઈબરની માહિતી-વહન ક્ષમતા સાથે સંબંધિત
\item
  પ્રકાશ સ્ત્રોતો સાથે કપલિંગ કાર્યક્ષમતા માટે મહત્વપૂર્ણ
\end{itemize}

\end{solutionbox}
\begin{mnemonicbox}
``CAP - Core and cladding indices Affect the
acceptance angle which determines the Path light can take''

\end{mnemonicbox}
\subsection*{પ્રશ્ન 5(અ) OR [3
ગુણ]}\label{uxaaauxab0uxab6uxaa8-5uxa85-or-3-uxa97uxaa3}

\textbf{લેસરના ઉપયોગો લખો.}

\begin{solutionbox}

\textbf{લેસરના ઉપયોગો:}

\begin{longtable}[]{@{}ll@{}}
\toprule\noalign{}
ક્ષેત્ર & ઉપયોગો \\
\midrule\noalign{}
\endhead
\bottomrule\noalign{}
\endlastfoot
\textbf{તબીબી} & સર્જરી, આંખની સારવાર, કેન્સર થેરાપી, ત્વચાવિજ્ઞાન, દંત
પ્રક્રિયાઓ \\
\textbf{ઔદ્યોગિક} & કટિંગ, વેલ્ડિંગ, ડ્રિલિંગ, માર્કિંગ, પદાર્થ પ્રક્રિયા, 3D
પ્રિન્ટિંગ \\
\textbf{સંચાર} & ફાઇબર ઓપ્ટિક ડેટા ટ્રાન્સમિશન, મુક્ત અવકાશ ઓપ્ટિકલ સંચાર \\
\textbf{વૈજ્ઞાનિક} & સ્પેક્ટ્રોસ્કોપી, હોલોગ્રાફી, ન્યુક્લિયર ફ્યુઝન, કણ ત્વરણ \\
\textbf{ગ્રાહક} & બારકોડ સ્કેનર, DVD/બ્લુ-રે પ્લેયર, લેસર પોઇન્ટર, પ્રિન્ટર \\
\textbf{લશ્કરી} & રેન્જ શોધ, લક્ષ્ય નિર્ધારણ, માર્ગદર્શક સિસ્ટમ, શસ્ત્રો \\
\end{longtable}

\end{solutionbox}
\begin{mnemonicbox}
``MICSM - Medical procedures, Industrial cutting,
Communication systems, Scientific research, Military applications''

\end{mnemonicbox}
\subsection*{પ્રશ્ન 5(બ) OR [4
ગુણ]}\label{uxaaauxab0uxab6uxaa8-5uxaac-or-4-uxa97uxaa3}

\textbf{પ્રકાશનું પૂર્ણ આંતરિક પરાવર્તન પર ટૂંક નોંધ લખો.}

\begin{solutionbox}

\textbf{પૂર્ણ આંતરિક પરાવર્તન (TIR)} એ એક ઓપ્ટિકલ ઘટના છે જે ત્યારે થાય છે જ્યારે ઘન
માધ્યમમાં પ્રવાસ કરતો પ્રકાશ ક્રાંતિક ખૂણા કરતાં મોટા ખૂણે ઓછા ઘન માધ્યમ સાથેની
સીમાને અથડાય છે.

\textbf{TIR માટે જરૂરી શરતો:}

\begin{itemize}
\tightlist
\item
  પ્રકાશ ઘન માધ્યમથી ઓછા ઘન માધ્યમમાં પ્રવાસ કરવો જોઈએ (n_{1} \textgreater{} n_{2})
\item
  આપતિનો ખૂણો ક્રાંતિક ખૂણા કરતાં વધુ હોવો જોઈએ (θᵢ \textgreater{} θc)
\end{itemize}

\textbf{ક્રાંતિક ખૂણાનું સૂત્ર:} θc = sin^{-}^{1}(n_{2}/n_{1})

\textbf{આકૃતિ:}

\begin{verbatim}
                Normal
                  |
                  |
Denser medium     |      θᵢ < θc (Refraction)
(n_{1})              |     /
                  |    /
                  |   /    θᵣ
                  |  /      /
                  | /      /
------------------+/------/----------
                   \     /
Less dense medium   \   /
(n_{2})                 \ /
                      |
                      |
                      |
                      
                Normal
                  |
                  |
Denser medium     |    θᵢ = θc (Critical angle)
(n_{1})              |   /
                  |  /
                  | /
------------------+/---------------------
                   \
Less dense medium   \ 90^\circ
(n_{2})                 \
                      |
                      |
                      
                Normal
                  |
                  |
Denser medium     |    θᵢ > θc (Total Internal Reflection)
(n_{1})              |   /
                  |  /
                  | /       /
------------------+/-------/------------
                   \       /
Less dense medium   \     /
(n_{2})                 \   /
                      \ /
                       |
\end{verbatim}

\textbf{ઉપયોગો:}

\begin{itemize}
\tightlist
\item
  સંચાર માટે ઓપ્ટિકલ ફાઈબર
\item
  પ્રિઝમ અને બાયનોક્યુલર
\item
  હીરાની ચમક
\item
  મૃગજળની રચના
\item
  તબીબી ઇમેજિંગ માટે એન્ડોસ્કોપ
\end{itemize}

\end{solutionbox}
\begin{mnemonicbox}
``CANDO - Critical Angle needed, n_{1} must be Denser
than n_{2}, Only works when angle is greater than critical, Angle
determines reflection vs refraction''

\end{mnemonicbox}
\subsection*{પ્રશ્ન 5(ક)(1) OR [3
ગુણ]}\label{uxaaauxab0uxab6uxaa8-5uxa951-or-3-uxa97uxaa3}

\textbf{પાણીમાં પ્રકાશનો વેગ 2.25\times10^{8} m/s અને હવામાં પ્રકાશનો વેગ 3\times10^{8} m/s હોય
તો પાણીનો વક્રીભવનાંક શોધો.}

\begin{solutionbox}

\textbf{આપેલ છે:}

\begin{itemize}
\tightlist
\item
  પાણીમાં પ્રકાશનો વેગ (vw) = 2.25\times10^{8} m/s
\item
  હવામાં પ્રકાશનો વેગ (va) = 3\times10^{8} m/s
\end{itemize}

\textbf{શોધવાનું છે:} પાણીનો વક્રીભવનાંક (nw)

\textbf{સૂત્ર:} n = c/v

\textbf{હવાની સાપેક્ષે પાણીના વક્રીભવનાંકની ગણતરી માટે:} nw = va/vw

\textbf{ગણતરી:} nw = 3\times10^{8} m/s \div 2.25\times10^{8} m/s nw = 3 \div 2.25 nw = 1.33

તેથી, પાણીનો વક્રીભવનાંક 1.33 છે.

\end{solutionbox}
\begin{mnemonicbox}
``SVN - Speed of light in Vacuum divided by Speed in
medium gives refractive iNdex''

\end{mnemonicbox}
\subsection*{પ્રશ્ન 5(ક)(2) OR [4
ગુણ]}\label{uxaaauxab0uxab6uxaa8-5uxa952-or-4-uxa97uxaa3}

\textbf{સ્ટેપ ઈન્ડેક્ષ ફાઈબર વિષે નોંધ લખો.}

\begin{solutionbox}

\textbf{સ્ટેપ ઈન્ડેક્ષ ફાઈબર:} એક પ્રકારનો ઓપ્ટિકલ ફાઈબર જ્યાં વક્રીભવનાંક કોર અને
ક્લેડિંગ વચ્ચે અચાનક બદલાય છે.

\textbf{રચના:}

\textbf{આકૃતિ:}

\begin{verbatim}
    ┌───────────────────────┐
    │                       │ n_{1}
    │        Core           │
    │                       │
    └───────────────────────┘
    ┌───────────────────────┐
    │                       │ n_{2}
    │       Cladding        │
    │                       │
    └───────────────────────┘

Refractive Index Profile:
    n_{1} ────────┐
               │
               │
    n_{2}         └────────
        Core     Cladding
\end{verbatim}

\textbf{લક્ષણો:}

\begin{itemize}
\tightlist
\item
  કોર-ક્લેડિંગ સીમા પર વક્રીભવનાંકમાં અચાનક ફેરફાર
\item
  સિંગલ-મોડ અને મલ્ટી-મોડ બંને રૂપરેખાઓમાં ઉપલબ્ધ
\item
  ગ્રેડેડ-ઇન્ડેક્સ ફાઈબર કરતાં સરળ બાંધકામ
\item
  મલ્ટી-મોડ રૂપરેખામાં વધુ મોડલ ફેલાવો
\end{itemize}

\textbf{પ્રકારો:}

\begin{itemize}
\tightlist
\item
  \textbf{સિંગલ-મોડ સ્ટેપ ઇન્ડેક્સ ફાઈબર}:

  \begin{itemize}
  \tightlist
  \item
    ખૂબ નાનો કોર વ્યાસ (8-10 μm)
  \item
    ફક્ત પ્રકાશના એક મોડને પસાર થવાની મંજૂરી આપે છે
  \item
    ઓછું સિગ્નલ વિકૃતિ
  \item
    લાંબા અંતરના સંચાર માટે વપરાય છે
  \end{itemize}
\item
  \textbf{મલ્ટી-મોડ સ્ટેપ ઇન્ડેક્સ ફાઈબર}:

  \begin{itemize}
  \tightlist
  \item
    મોટો કોર વ્યાસ (50-100 μm)
  \item
    બહુવિધ પ્રકાશ પથની મંજૂરી આપે છે
  \item
    ઉચ્ચ મોડલ ફેલાવો
  \item
    ટૂંકા અંતર માટે યોગ્ય
  \end{itemize}
\end{itemize}

\textbf{ફાયદા:}

\begin{itemize}
\tightlist
\item
  સરળ અને સસ્તું ઉત્પાદન
\item
  ટૂંકા અંતરના અનુપ્રયોગો માટે સારું
\item
  મલ્ટી-મોડ સંસ્કરણોમાં પ્રકાશને કપલ કરવું સરળ
\item
  સિંગલ-મોડ ફાઈબર કરતાં વળવાના નુકસાન પ્રત્યે ઓછું સંવેદનશીલ
\end{itemize}

\textbf{મર્યાદાઓ:}

\begin{itemize}
\tightlist
\item
  મલ્ટી-મોડ રૂપરેખામાં ઉચ્ચ મોડલ ફેલાવો
\item
  અલગ-અલગ પથની લંબાઈને કારણે બેન્ડવિડ્થ મર્યાદાઓ
\item
  ઉચ્ચ-ગતિ, લાંબા અંતરના પ્રસારણ માટે આદર્શ નથી
\end{itemize}

\end{solutionbox}
\begin{mnemonicbox}
``SACS - Step change at boundary, Abrupt index
profile, Core guides light, Simple construction''

\end{mnemonicbox}

\end{document}
