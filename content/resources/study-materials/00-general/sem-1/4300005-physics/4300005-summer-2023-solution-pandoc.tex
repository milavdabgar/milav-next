\documentclass[10pt,a4paper]{article}

% content/resources/templates/preamble.tex
\usepackage[margin=0.6in]{geometry}
\author{Milav Dabgar}
\usepackage{amsmath,amssymb,amsthm}
\usepackage{booktabs}
\usepackage{multirow}
\usepackage{xcolor}
\usepackage{tcolorbox}
\tcbuselibrary{breakable,skins}
\usepackage[colorlinks=true,linkcolor=blue]{hyperref}
\usepackage{titlesec}
\usepackage{enumitem}
\usepackage{tikz}
\usepackage{pgfplots}
\usepackage{circuitikz}
\usepackage[version=4]{mhchem}
\usepackage{longtable}
\usepackage{array}
\usepackage{float}
\usepackage{caption}
\usepackage{listings}

\lstset{
  basicstyle=\small\ttfamily,
  breaklines=true,
  breakatwhitespace=false,
  postbreak=\mbox{\textcolor{red}{$\hookrightarrow$}\space},
  float=false,
  numbers=left,
  numberstyle=\tiny\color{gray},
  numbersep=10pt,
  xleftmargin=2em,
  keywordstyle=\color{blue},
  commentstyle=\color{green!60!black},
  stringstyle=\color{purple},
  backgroundcolor=\color{gray!5},
  showstringspaces=false,
  tabsize=2,
  captionpos=b,
  keepspaces=true,
  columns=flexible
}

\pgfplotsset{compat=1.18}
\usetikzlibrary{shapes,arrows,positioning,calc,patterns,decorations.pathmorphing,decorations.markings,arrows.meta}

% Color scheme
\definecolor{headcolor}{RGB}{0,102,204}
\definecolor{keycolor}{RGB}{220,20,60}
\definecolor{solutioncolor}{RGB}{34,139,34}
\definecolor{mnemoniccolor}{RGB}{148,0,211}
\definecolor{codecolor}{RGB}{0,0,100}

% Spacing
\setlength{\parskip}{3pt}
\setlist[itemize]{nosep}
\setlist[enumerate]{nosep}

% Title formatting
\titleformat{\section}{\Large\bfseries\color{headcolor}}{\thesection}{1em}{}
\titleformat{\subsection}{\large\bfseries\color{headcolor}}{\thesubsection}{1em}{}

% Pandoc tightlist compatibility
\providecommand{\tightlist}{%
  \setlength{\itemsep}{0pt}\setlength{\parskip}{0pt}}

% Pandoc longtable compatibility
\newcounter{none}
\def\thenone{}


% content/resources/templates/english-boxes.tex
% This file is currently empty - it exists to maintain consistency with the import structure.
% Add custom environments here if needed in the future.


\begin{document}

\begin{center}
{\Huge\bfseries\color{headcolor} Subject Name Solutions}\\[5pt]
{\LARGE 4300005 -- Summer 2023}\\[3pt]
{\large Semester 1 Study Material}\\[3pt]
{\normalsize\textit{Detailed Solutions and Explanations}}
\end{center}

\vspace{10pt}

\subsection*{Question 1(a) [3 marks]}\label{q1a}

\textbf{Write base units with their symbols in SI.}

\begin{solutionbox}

\begin{longtable}[]{@{}lll@{}}
\toprule\noalign{}
Physical Quantity & Base Unit & Symbol \\
\midrule\noalign{}
\endhead
\bottomrule\noalign{}
\endlastfoot
Length & meter & m \\
Mass & kilogram & kg \\
Time & second & s \\
Electric current & ampere & A \\
Temperature & kelvin & K \\
Amount of substance & mole & mol \\
Luminous intensity & candela & cd \\
\end{longtable}

\end{solutionbox}
\begin{mnemonicbox}
``Learn Measurements Through Accurate Techniques Like
Modern Scientists''

\end{mnemonicbox}
\subsection*{Question 1(b) [4 marks]}\label{q1b}

\textbf{Explain construction and working of a vernier caliper. Explain
its least count and zero error.}

\begin{solutionbox}

\textbf{Construction of Vernier Caliper:}

\begin{center}
\textbf{Mermaid Diagram (Code)}
\begin{verbatim}
{Shaded}
{Highlighting}[]
graph LR
    A[Main Scale] {-{-}{-} B[Fixed Jaw]}
    A {-{-}{-} C[Vernier Scale]}
    C {-{-}{-} D[Movable Jaw]}
    C {-{-}{-} E[Depth Rod]}
    A {-{-}{-} F[Locking Screw]}
{Highlighting}
{Shaded}
\end{verbatim}
\end{center}

\begin{itemize}
\tightlist
\item
  \textbf{Main scale}: Fixed scale with millimeter divisions
\item
  \textbf{Vernier scale}: Sliding scale with divisions slightly smaller
  than main scale
\item
  \textbf{Fixed jaw}: Connected to main scale
\item
  \textbf{Movable jaw}: Attached to vernier scale
\item
  \textbf{Depth rod}: For measuring depths
\item
  \textbf{Locking screw}: To fix position during measurement
\end{itemize}

\textbf{Working}: Object is placed between jaws, movable jaw is adjusted
to hold object firmly. Reading is taken by noting main scale reading and
adding vernier coincidence value.

\textbf{Least Count}: Smallest measurement possible with vernier
caliper. LC = 1 division on main scale \div Number of divisions on vernier
scale

\textbf{Zero Error}: Error when caliper shows non-zero reading with jaws
closed.

\begin{itemize}
\tightlist
\item
  \textbf{Positive error}: Subtract from reading
\item
  \textbf{Negative error}: Add to reading
\end{itemize}

\end{solutionbox}
\begin{mnemonicbox}
``Very Careful Measurements Leave Count Errors Zero''

\end{mnemonicbox}
\subsection*{Question 1(c)(i) [4 marks]}\label{q1c}

\textbf{Distinguish between accuracy and precision.}

\begin{solutionbox}

\begin{longtable}[]{@{}
  >{\raggedright\arraybackslash}p{(\linewidth - 2\tabcolsep) * \real{0.4762}}
  >{\raggedright\arraybackslash}p{(\linewidth - 2\tabcolsep) * \real{0.5238}}@{}}
\toprule\noalign{}
\begin{minipage}[b]{\linewidth}\raggedright
Accuracy
\end{minipage} & \begin{minipage}[b]{\linewidth}\raggedright
Precision
\end{minipage} \\
\midrule\noalign{}
\endhead
\bottomrule\noalign{}
\endlastfoot
Closeness of measurement to true value & Repeatability of measurement \\
Affected by systematic errors & Affected by random errors \\
Represented by mean of measurements & Represented by standard
deviation \\
Improved by calibration & Improved by using better instruments \\
Example: If true value is 10 cm, measurements of 9.9, 10.1, and 10.0 cm
are accurate & Example: Measurements of 9.8, 9.8, 9.8 cm are precise but
not accurate if true value is 10 cm \\
\end{longtable}

\end{solutionbox}
\begin{mnemonicbox}
``Accurate measurements Are Always At true value,
Precise measurements Produce Perfect repeatability''

\end{mnemonicbox}
\subsection*{Question 1(c)(ii) [2
marks]}\label{q1c}

\textbf{Pitch of a micrometer screw gauge is 0.5 mm and there are 50
divisions on its circular scale. Find its least count.}

\begin{solutionbox}

\textbf{Formula}: Least Count = Pitch \div Number of divisions on circular
scale

\textbf{Calculation}: LC = 0.5 mm \div 50 = 0.01 mm

\textbf{Least Count of micrometer screw gauge = 0.01 mm}

\end{solutionbox}
\subsection*{Question 1(c)(iii) [1
mark]}\label{q1c}

\textbf{What is SI unit of heat?}

\begin{solutionbox}

SI unit of heat is \textbf{Joule (J)}

\end{solutionbox}
\subsection*{Question 1(c)(i) [4 marks]
(OR)}\label{q1c}

\textbf{How are absolute and relative errors calculated?}

\begin{solutionbox}

\textbf{Absolute Error (Δa)}: Difference between measured value and true
value

\begin{itemize}
\tightlist
\item
  For multiple measurements, it's difference between measured value and
  mean value
\end{itemize}

\textbf{Calculation of Absolute Error}:

\begin{itemize}
\tightlist
\item
  \textbf{Single measurement}: Δa = \textbar Measured value - True
  value\textbar{}
\item
  \textbf{Multiple measurements}:

  \begin{enumerate}
  \tightlist
  \item
    Calculate mean (am)
  \item
    For each measurement: Δai = \textbar ai - am\textbar{}
  \item
    Mean absolute error: Δa = (Δa1 + Δa2 + \ldots{} + Δan) \div n
  \end{enumerate}
\end{itemize}

\textbf{Relative Error (εr)}: Ratio of absolute error to true value

\begin{itemize}
\tightlist
\item
  εr = Absolute error \div True value = Δa \div True value
\end{itemize}

\textbf{Percentage Error (εp)}: Relative error expressed as percentage

\begin{itemize}
\tightlist
\item
  εp = Relative error \times 100 = (Δa \div True value) \times 100\%
\end{itemize}

\end{solutionbox}
\begin{mnemonicbox}
``Absolute Always measures Actual deviation; Relative
References the total value''

\end{mnemonicbox}
\subsection*{Question 1(c)(ii) [2 marks]
(OR)}\label{q1c}

\textbf{Main scale of a vernier caliper is calibrated in mm and there
are 50 divisions on its vernier scale. Find its least count.}

\begin{solutionbox}

\textbf{Formula}: Least Count = 1 division on main scale \div Number of
divisions on vernier scale

\textbf{Calculation}: 1 division on main scale = 1 mm LC = 1 mm \div 50 =
0.02 mm

\textbf{Least Count of vernier caliper = 0.02 mm}

\end{solutionbox}
\subsection*{Question 1(c)(iii) [1 mark]
(OR)}\label{q1c}

\textbf{In which of the mode of heat transfer, medium is not required?}

\begin{solutionbox}

\textbf{Radiation} does not require a medium for heat transfer.

\end{solutionbox}
\subsection*{Question 2(a) [3 marks]}\label{q2a}

\textbf{Write characteristics of electric field lines.}

\begin{solutionbox}

\textbf{Characteristics of Electric Field Lines}:

\begin{enumerate}
\tightlist
\item
  Electric field lines start from positive charge and end on negative
  charge
\item
  Field lines never cross each other
\item
  Field lines are always perpendicular to the surface of conductor
\item
  Number of field lines is proportional to magnitude of charge
\item
  Closer field lines indicate stronger electric field
\item
  Field lines are continuous curves
\item
  Field lines contract longitudinally and expand laterally
\end{enumerate}

\textbf{Diagram}:

\begin{verbatim}
     +           {-}
      {         /}
       {       /}
        {     /}
         {   /}
          { /}
           X
\end{verbatim}

\end{solutionbox}
\begin{mnemonicbox}
``Electric Field Lines: Start Positive, End Negative,
Cross Never''

\end{mnemonicbox}
\subsection*{Question 2(b) [4 marks]}\label{q2b}

\textbf{Explain Coulomb's inverse square law for electrostatic forces.}

\begin{solutionbox}

\textbf{Coulomb's Inverse Square Law}: The electrostatic force between
two point charges is directly proportional to the product of magnitudes
of charges and inversely proportional to the square of distance between
them.

\textbf{Mathematical Form}: F = k(q_{1}q_{2})/r^{2}

Where:

\begin{itemize}
\tightlist
\item
  F = electrostatic force (in Newtons)
\item
  k = electrostatic constant (9\times10^{9} N·m^{2}/C^{2})
\item
  q_{1}, q_{2} = magnitudes of charges (in Coulombs)
\item
  r = distance between charges (in meters)
\end{itemize}

\textbf{Properties}:

\begin{itemize}
\tightlist
\item
  \textbf{Vector Quantity}: Force acts along the line joining the two
  charges
\item
  \textbf{Attractive/Repulsive}: Like charges repel, unlike charges
  attract
\item
  \textbf{Central Force}: Follows Newton's third law
\item
  \textbf{Medium Dependence}: Depends on the medium between charges (k
  changes)
\end{itemize}

\textbf{Diagram}:

\begin{verbatim}
     q_{1}           q_{2}
      O-----------O
      \leftarrow───F_{1}^{2}───\rightarrow \leftarrow───F_{2}^{1}───
         r
\end{verbatim}

\end{solutionbox}
\begin{mnemonicbox}
``Charges Attract/Repel Leveraging Distance Squared''

\end{mnemonicbox}
\subsection*{Question 2(c)(i) [4 marks]}\label{q2c}

\textbf{Derive formula for equivalent capacitance of capacitors
connected in series and parallel combination.}

\begin{solutionbox}

\textbf{For Series Combination}:

\begin{center}
\textbf{Mermaid Diagram (Code)}
\begin{verbatim}
{Shaded}
{Highlighting}[]
graph LR
    A["{+"] {-}{-}{-} B[C_{1}]}
    B {-{-}{-} C[C_{2}]}
    C {-{-}{-} D[C_{3}]}
    D {-{-}{-} E["{}{-}"]}
{Highlighting}
{Shaded}
\end{verbatim}
\end{center}

When capacitors are connected in series:

\begin{itemize}
\tightlist
\item
  Same charge Q appears on each capacitor
\item
  Potential difference distributes across capacitors
\item
  V = V_{1} + V_{2} + V_{3}
\end{itemize}

For each capacitor: V_{1} = Q/C_{1}, V_{2} = Q/C_{2}, V_{3} = Q/C_{3}

Total voltage:

V = Q/C_{1} + Q/C_{2} + Q/C_{3} = Q(1/C_{1} + 1/C_{2} + 1/C_{3})


For equivalent capacitance: V = Q/Ceq

Therefore: 1/Ceq = 1/C_{1} + 1/C_{2} + 1/C_{3}

\textbf{For Parallel Combination}:

\begin{center}
\textbf{Mermaid Diagram (Code)}
\begin{verbatim}
{Shaded}
{Highlighting}[]
graph LR
    A["{+"] {-}{-}{-} B[C_{1}]}
    A {-{-}{-} C[C_{2}]}
    A {-{-}{-} D[C_{3}]}
    B {-{-}{-} E["{}{-}"]}
    C {-{-}{-} E}
    D {-{-}{-} E}
{Highlighting}
{Shaded}
\end{verbatim}
\end{center}

When capacitors are connected in parallel:

\begin{itemize}
\tightlist
\item
  Same potential difference V across each capacitor
\item
  Total charge distributes among capacitors
\item
  Q = Q_{1} + Q_{2} + Q_{3}
\end{itemize}

For each capacitor: Q_{1} = C_{1}V, Q_{2} = C_{2}V, Q_{3} = C_{3}V

Total charge:

Q = C_{1}V + C_{2}V + C_{3}V = (C_{1} + C_{2} + C_{3})V


For equivalent capacitance: Q = CeqV

Therefore: Ceq = C_{1} + C_{2} + C_{3}

\end{solutionbox}
\begin{mnemonicbox}
``Series Sums Reciprocals, Parallel Puts Capacitance
Together''

\end{mnemonicbox}
\subsection*{Question 2(c)(ii) [2
marks]}\label{q2c}

\textbf{Two capacitors of capacitances 8 μF and 9 μF are connected in
parallel combination. Find equivalent capacitance.}

\begin{solutionbox}

\textbf{Formula for parallel combination}: Ceq = C_{1} + C_{2}

\textbf{Given}:

\begin{itemize}
\tightlist
\item
  C_{1} = 8 μF
\item
  C_{2} = 9 μF
\end{itemize}

\textbf{Calculation}: Ceq = 8 μF + 9 μF = 17 μF

\textbf{Therefore, equivalent capacitance = 17 μF}

\end{solutionbox}
\subsection*{Question 2(c)(iii) [1
mark]}\label{q2c}

\textbf{Write full name of ``LASER''.}

\begin{solutionbox}

\textbf{LASER}: Light Amplification by Stimulated Emission of Radiation

\end{solutionbox}
\subsection*{Question 2(a) [3 marks]
(OR)}\label{q2a}

\textbf{What is a capacitor? Define capacitance and write its unit.}

\begin{solutionbox}

\textbf{Capacitor}: A device that stores electric charge and electrical
energy in the form of electric field.

\textbf{Capacitance}: The ability of a capacitor to store electric
charge. It is defined as the ratio of charge stored to the potential
difference applied.

\textbf{Mathematical Form}: C = Q/V

Where:

\begin{itemize}
\tightlist
\item
  C = capacitance
\item
  Q = charge stored on capacitor
\item
  V = potential difference across capacitor
\end{itemize}

\textbf{Unit of Capacitance}: Farad (F)

\textbf{Diagram}:

\begin{verbatim}
    +++++++  |  -------
           |   |
           |   |
        ---+---+---
           |   |
           |   |
    +++++++  |  -------
\end{verbatim}

\end{solutionbox}
\begin{mnemonicbox}
``Capacitors Collect Charge, Volts Vary
Fastidiously''

\end{mnemonicbox}
\subsection*{Question 2(b) [4 marks]
(OR)}\label{q2b}

\textbf{Explain intensity of electric field and electric potential.}

\begin{solutionbox}

\textbf{Electric Field Intensity}:

\begin{itemize}
\tightlist
\item
  \textbf{Definition}: Force experienced by unit positive charge placed
  at that point
\item
  \textbf{Formula}: E = F/q
\item
  \textbf{Unit}: Newton/Coulomb (N/C) or Volt/meter (V/m)
\item
  \textbf{Vector Quantity}: Has both magnitude and direction
\item
  \textbf{Direction}: Same as force on positive charge
\end{itemize}

\textbf{Electric Potential}:

\begin{itemize}
\tightlist
\item
  \textbf{Definition}: Work done to bring unit positive charge from
  infinity to that point
\item
  \textbf{Formula}: V = W/q
\item
  \textbf{Unit}: Volt (V) or Joule/Coulomb (J/C)
\item
  \textbf{Scalar Quantity}: Has only magnitude
\item
  \textbf{Relation with field}: E = -dV/dr (field is negative gradient
  of potential)
\end{itemize}

\textbf{Comparison Table}:

\begin{longtable}[]{@{}lll@{}}
\toprule\noalign{}
Property & Electric Field & Electric Potential \\
\midrule\noalign{}
\endhead
\bottomrule\noalign{}
\endlastfoot
Definition & Force per unit charge & Work done per unit charge \\
Nature & Vector & Scalar \\
Unit & N/C or V/m & V or J/C \\
Dependence & Varies as 1/r^{2} & Varies as 1/r \\
Direction & Away from +ve charge & No direction \\
\end{longtable}

\end{solutionbox}
\begin{mnemonicbox}
``Electric Field Forces charges; Potential Provides
energy''

\end{mnemonicbox}
\subsection*{Question 2(c)(i) [4 marks]
(OR)}\label{q2c}

\textbf{Using formula of capacitance of a parallel plate capacitor,
explain effect of plate area, separation between plates and presence of
dielectric material between the plates on its capacitance.}

\begin{solutionbox}

\textbf{Formula for capacitance of parallel plate capacitor}: C =
ε_{0}εᵣA/d

Where:

\begin{itemize}
\tightlist
\item
  C = capacitance
\item
  ε_{0} = permittivity of free space (8.85\times10^{-}^{1}^{2} F/m)
\item
  εᵣ = relative permittivity of dielectric
\item
  A = area of overlap between plates
\item
  d = distance between plates
\end{itemize}

\textbf{Effect of Plate Area (A)}:

\begin{itemize}
\tightlist
\item
  Capacitance is directly proportional to area of plates
\item
  Increasing area \rightarrow Increases capacitance
\item
  Doubling area \rightarrow Doubles capacitance
\end{itemize}

\textbf{Effect of Separation (d)}:

\begin{itemize}
\tightlist
\item
  Capacitance is inversely proportional to distance between plates
\item
  Increasing separation \rightarrow Decreases capacitance
\item
  Doubling separation \rightarrow Halves capacitance
\end{itemize}

\textbf{Effect of Dielectric Material (εᵣ)}:

\begin{itemize}
\tightlist
\item
  Capacitance is directly proportional to relative permittivity of
  dielectric
\item
  Inserting dielectric \rightarrow Increases capacitance
\item
  Dielectric constant measures this increase: C(with dielectric) = εᵣ \times
  C(without dielectric)
\end{itemize}

\textbf{Diagram}:

\begin{verbatim}
    +++++++  |  -------
           |   |
       A   | d |
        ---+---+---
           |εᵣ |
           |   |
    +++++++  |  -------
\end{verbatim}

\end{solutionbox}
\begin{mnemonicbox}
``Area Amplifies, Distance Diminishes, Dielectrics
Double''

\end{mnemonicbox}
\subsection*{Question 2(c)(ii) [2 marks]
(OR)}\label{q2c}

\textbf{Voltage between plates of a capacitor of capacitance 0.5 μF is
150 V. Find magnitude of electric charge on plates.}

\begin{solutionbox}

\textbf{Formula}: Q = CV

\textbf{Given}:

\begin{itemize}
\tightlist
\item
  Capacitance (C) = 0.5 μF = 0.5 \times 10^{-}^{6} F
\item
  Voltage (V) = 150 V
\end{itemize}

\textbf{Calculation}: Q = CV = 0.5 \times 10^{-}^{6} \times 150 = 75 \times 10^{-}^{6} C = 75 μC

\textbf{Therefore, charge on plates = 75 μC}

\end{solutionbox}
\subsection*{Question 2(c)(iii) [1 mark]
(OR)}\label{q2c}

\textbf{Of the two parts of an optical fiber, the core and the cladding,
which one has larger refractive index?}

\begin{solutionbox}

The \textbf{core} has a larger refractive index than the cladding.

\end{solutionbox}
\subsection*{Question 3(a) [3 marks]}\label{q3a}

\textbf{Define conduction and convection of heat.}

\begin{solutionbox}

\textbf{Heat Conduction}:

\begin{itemize}
\tightlist
\item
  Transfer of heat through matter without actual movement of particles
\item
  Occurs due to direct molecular collisions
\item
  Heat flows from higher to lower temperature region
\item
  Metals are good conductors of heat
\item
  Examples: Heat transfer through metal rod, cooking pot
\end{itemize}

\textbf{Heat Convection}:

\begin{itemize}
\tightlist
\item
  Transfer of heat through actual movement of matter
\item
  Occurs in fluids (liquids and gases)
\item
  Involves formation of convection currents
\item
  Examples: Room heater, sea breeze, boiling water
\end{itemize}

\textbf{Diagram}:

\begin{verbatim}
Conduction:
Hot     Cold
|->->->->|

Convection:
      ↑
    \leftarrow   \rightarrow
      ↓
    Heat
\end{verbatim}

\end{solutionbox}
\begin{mnemonicbox}
``Conduction Connects molecules; Convection Carries
material''

\end{mnemonicbox}
\subsection*{Question 3(b) [4 marks]}\label{q3b}

\textbf{Explain construction and working of mercury thermometer.}

\begin{solutionbox}

\textbf{Construction of Mercury Thermometer}:

\begin{center}
\textbf{Mermaid Diagram (Code)}
\begin{verbatim}
{Shaded}
{Highlighting}[]
graph LR
    A[Glass Bulb] {-{-}{-} B[Capillary Tube]}
    B {-{-}{-} C[Scale]}
    C {-{-}{-} D[Protective Glass Cover]}
{Highlighting}
{Shaded}
\end{verbatim}
\end{center}

\begin{itemize}
\tightlist
\item
  \textbf{Glass bulb}: Contains mercury, acts as reservoir
\item
  \textbf{Capillary tube}: Thin glass tube connected to bulb
\item
  \textbf{Scale}: Calibrated with temperature markings
\item
  \textbf{Protective glass cover}: Protects capillary tube and scale
\end{itemize}

\textbf{Working Principle}:

\begin{enumerate}
\tightlist
\item
  Based on thermal expansion of mercury
\item
  When temperature increases, mercury expands and rises in capillary
\item
  When temperature decreases, mercury contracts and level falls
\item
  Temperature is read from scale at mercury level
\end{enumerate}

\textbf{Temperature Range}: -38.83^\circC to 356.73^\circC (mercury's freezing to
boiling point)

\textbf{Advantages}:

\begin{itemize}
\tightlist
\item
  High accuracy
\item
  Linear expansion
\item
  Clearly visible in capillary
\end{itemize}

\textbf{Limitations}:

\begin{itemize}
\tightlist
\item
  Cannot measure very low temperatures
\item
  Mercury is toxic
\item
  Cannot be used for remote sensing
\end{itemize}

\end{solutionbox}
\begin{mnemonicbox}
``Mercury Moves Through Capillary, Showing
Temperature''

\end{mnemonicbox}
\subsection*{Question 3(c)(i) [4 marks]}\label{q3c}

\textbf{State laws of thermal conductivity and derive formula of
coefficient of thermal conductivity.}

\begin{solutionbox}

\textbf{Laws of Thermal Conductivity}:

\begin{enumerate}
\tightlist
\item
  Heat flow is directly proportional to temperature difference (ΔT)
\item
  Heat flow is directly proportional to cross-sectional area (A)
\item
  Heat flow is inversely proportional to length (L)
\item
  Heat flow is directly proportional to time (t)
\end{enumerate}

\textbf{Derivation of Coefficient of Thermal Conductivity}:

According to Fourier's law: Q ∝ A \times t \times ΔT/L

Converting to equation with proportionality constant K: Q = K \times A \times t \times
ΔT/L

Rearranging: K = (Q \times L)/(A \times t \times ΔT)

Where:

\begin{itemize}
\tightlist
\item
  Q = Heat conducted (in Joules)
\item
  L = Length of conductor (in meters)
\item
  A = Cross-sectional area (in m^{2})
\item
  t = Time (in seconds)
\item
  ΔT = Temperature difference (in Kelvin)
\item
  K = Coefficient of thermal conductivity (in W/m·K)
\end{itemize}

\textbf{Diagram}:

\begin{verbatim}
Hot         Cold
T_{1} ---------T_{2}
    Length L
    Area A
    Heat Q
\end{verbatim}

\end{solutionbox}
\begin{mnemonicbox}
``Heat Transfers Faster when Area Larger, Temperature
higher, Length shorter''

\end{mnemonicbox}
\subsection*{Question 3(c)(ii) [2
marks]}\label{q3c}

\textbf{The total area of glass window pane is 0.5m^{2}. Calculate amount
of heat conducted per hour through the pane if thickness of glass is
0.6cm, the inside temperature is 30^\circC and outside temperature is 20^\circC.
Coefficient of thermal conductivity of glass is 1.0 Wm^{-}^{1}K^{-}^{1}.}

\begin{solutionbox}

\textbf{Formula}: Q = (K \times A \times t \times ΔT)/L

\textbf{Given}:

\begin{itemize}
\tightlist
\item
  Area (A) = 0.5 m^{2}
\item
  Thickness (L) = 0.6 cm = 0.006 m
\item
  Inside temperature (T_{1}) = 30^\circC
\item
  Outside temperature (T_{2}) = 20^\circC
\item
  Temperature difference (ΔT) = 10^\circC = 10 K
\item
  Coefficient of thermal conductivity (K) = 1.0 W/m·K
\item
  Time (t) = 1 hour = 3600 seconds
\end{itemize}

\textbf{Calculation}: Q = (1.0 \times 0.5 \times 3600 \times 10)/0.006 Q =
(18000)/0.006

Q = 3,000,000

J = 3000 kJ


\textbf{Therefore, heat conducted = 3000 kJ per hour}

\end{solutionbox}
\subsection*{Question 3(c)(iii) [1
mark]}\label{q3c}

\textbf{Which property of light is responsible for transmission of light
through optical fibre?}

\begin{solutionbox}

\textbf{Total Internal Reflection (TIR)} is responsible for transmission
of light through optical fiber.

\end{solutionbox}
\subsection*{Question 3(a) [3 marks]
(OR)}\label{q3a}

\textbf{Define heat capacity and specific heat.}

\begin{solutionbox}

\textbf{Heat Capacity}:

\begin{itemize}
\tightlist
\item
  Amount of heat energy required to raise temperature of an object by
  1^\circC or 1K
\item
  Depends on mass and material of object
\item
  Formula: C = Q/ΔT
\item
  Unit: Joule/Kelvin (J/K)
\end{itemize}

\textbf{Specific Heat}:

\begin{itemize}
\tightlist
\item
  Amount of heat energy required to raise temperature of 1 kg of
  substance by 1^\circC or 1K
\item
  Property of material, independent of mass
\item
  Formula: c = Q/(m\timesΔT)
\item
  Unit: Joule/kg·K (J/kg·K)
\end{itemize}

\textbf{Relation}: Heat capacity (C) = mass (m) \times specific heat (c)

\textbf{Comparison Table}:

\begin{longtable}[]{@{}
  >{\raggedright\arraybackslash}p{(\linewidth - 4\tabcolsep) * \real{0.2500}}
  >{\raggedright\arraybackslash}p{(\linewidth - 4\tabcolsep) * \real{0.3750}}
  >{\raggedright\arraybackslash}p{(\linewidth - 4\tabcolsep) * \real{0.3750}}@{}}
\toprule\noalign{}
\begin{minipage}[b]{\linewidth}\raggedright
Property
\end{minipage} & \begin{minipage}[b]{\linewidth}\raggedright
Heat Capacity
\end{minipage} & \begin{minipage}[b]{\linewidth}\raggedright
Specific Heat
\end{minipage} \\
\midrule\noalign{}
\endhead
\bottomrule\noalign{}
\endlastfoot
Definition & Heat per degree for object & Heat per degree per unit
mass \\
Symbol & C & c \\
Unit & J/K & J/kg·K \\
Depends on & Mass and material & Only material \\
Formula & Q/ΔT & Q/(m\timesΔT) \\
\end{longtable}

\end{solutionbox}
\begin{mnemonicbox}
``Heat Capacity for Complete object, Specific heat
for Single kilogram''

\end{mnemonicbox}
\subsection*{Question 3(b) [4 marks]
(OR)}\label{q3b}

\textbf{Explain construction and working of optical pyrometer.}

\begin{solutionbox}

\textbf{Construction of Optical Pyrometer}:

\begin{center}
\textbf{Mermaid Diagram (Code)}
\begin{verbatim}
{Shaded}
{Highlighting}[]
graph LR
    A[Telescope] {-{-}{-} B[Filament Lamp]}
    B {-{-}{-} C[Ammeter]}
    C {-{-}{-} D[Battery]}
    D {-{-}{-} B}
    A {-{-}{-} E[Color Filter]}
    E {-{-}{-} F[Eyepiece]}
{Highlighting}
{Shaded}
\end{verbatim}
\end{center}

\begin{itemize}
\tightlist
\item
  \textbf{Telescope}: To view hot object
\item
  \textbf{Filament lamp}: Calibrated tungsten filament
\item
  \textbf{Rheostat}: To adjust current through filament
\item
  \textbf{Ammeter}: To measure current
\item
  \textbf{Red filter}: To match wavelengths
\item
  \textbf{Eyepiece}: For viewing
\end{itemize}

\textbf{Working Principle}:

\begin{enumerate}
\tightlist
\item
  Based on comparing brightness of hot object with standard lamp
  filament
\item
  Object is viewed through telescope
\item
  Current adjusted until filament brightness matches object brightness
\item
  At match point, filament ``disappears'' against object background
\item
  Temperature determined from calibrated scale or ammeter reading
\end{enumerate}

\textbf{Temperature Range}: 700^\circC to 3000^\circC

\textbf{Advantages}:

\begin{itemize}
\tightlist
\item
  Non-contact measurement
\item
  High temperature measurement
\item
  Suitable for moving objects
\end{itemize}

\end{solutionbox}
\begin{mnemonicbox}
``Pyrometer Produces Perfect Temperature by
Brightness Comparison''

\end{mnemonicbox}
\subsection*{Question 3(c)(i) [4 marks]
(OR)}\label{q3c}

\textbf{Define linear thermal expansion of solids and derive formula of
coefficient linear thermal expansion.}

\begin{solutionbox}

\textbf{Linear Thermal Expansion}: Increase in length of a solid
material when its temperature increases

\textbf{Coefficient of Linear Thermal Expansion (α)}: Fractional change
in length per unit change in temperature

\textbf{Derivation}:

For small temperature changes:

\begin{itemize}
\tightlist
\item
  Change in length (ΔL) is directly proportional to original length (L_{0})
\item
  ΔL is directly proportional to change in temperature (ΔT)
\end{itemize}

Therefore: ΔL ∝ L_{0} \times ΔT

Converting to equation with proportionality constant α: ΔL = α \times L_{0} \times ΔT

Rearranging: α = ΔL/(L_{0} \times ΔT)

Where:

\begin{itemize}
\tightlist
\item
  ΔL = Change in length (in meters)
\item
  L_{0} = Original length (in meters)
\item
  ΔT = Change in temperature (in Kelvin or Celsius)
\item
  α = Coefficient of linear thermal expansion (per ^\circC or per K)
\end{itemize}

\textbf{Final length}: L = L_{0}(1 + αΔT)

\textbf{Diagram}:

\begin{verbatim}
Before heating:
|----L_{0}----|

After heating:
|------L------|
\end{verbatim}

\end{solutionbox}
\begin{mnemonicbox}
``Linear Expansion Numerically Gives Total Length
Increase''

\end{mnemonicbox}
\subsection*{Question 3(c)(ii) [2 marks]
(OR)}\label{q3c}

\textbf{Length of a steel rod at 0^\circC is 150 cm. What will be its length
at 200^\circC, if its coefficient of linear thermal expansion is 12 \times 10^{-}^{6}
per ^\circC.}

\begin{solutionbox}

\textbf{Formula}: L = L_{0}(1 + αΔT)

\textbf{Given}:

\begin{itemize}
\tightlist
\item
  Original length (L_{0}) = 150 cm
\item
  Original temperature = 0^\circC
\item
  Final temperature = 200^\circC
\item
  Temperature change (ΔT) = 200^\circC
\item
  Coefficient of linear expansion (α) = 12 \times 10^{-}^{6} per ^\circC
\end{itemize}

\textbf{Calculation}: L = 150(1 + 12 \times 10^{-}^{6} \times 200) L = 150(1 + 24 \times
10^{-}^{4})

L = 150(1 + 0.0024)

L = 150 \times 1.0024

L = 150.36 cm


\textbf{Therefore, final length of steel rod = 150.36 cm}

\end{solutionbox}
\subsection*{Question 3(c)(iii) [1 mark]
(OR)}\label{q3c}

\textbf{Which type of emission of radiation is responsible for emission
of ordinary light?}

\begin{solutionbox}

\textbf{Spontaneous emission} is responsible for emission of ordinary
light.

\end{solutionbox}
\subsection*{Question 4(a) [3 marks]}\label{q4a}

\textbf{Define amplitude, frequency and time period of a wave.}

\begin{solutionbox}

\textbf{Amplitude}:

\begin{itemize}
\tightlist
\item
  Maximum displacement of medium particles from equilibrium position
\item
  Represents energy of wave
\item
  Denoted by `A'
\item
  Measured in meters (m)
\end{itemize}

\textbf{Frequency}:

\begin{itemize}
\tightlist
\item
  Number of complete oscillations per unit time
\item
  Denoted by `f' or `ν'
\item
  Measured in hertz (Hz) or cycles per second
\item
  Related to wavelength (λ) and velocity (v): f = v/λ
\end{itemize}

\textbf{Time Period}:

\begin{itemize}
\tightlist
\item
  Time taken to complete one oscillation
\item
  Denoted by `T'
\item
  Measured in seconds (s)
\item
  Related to frequency: T = 1/f
\end{itemize}

\textbf{Diagram}:

\begin{verbatim}
    Amplitude
        ↕
        |    /\      /\
        |   /  \    /  \
--------+--/----\--/----\---> Time
        |  \    /  \    /
        |   \  /    \  /
        |    \/      \/
        |<--T-->|
\end{verbatim}

\end{solutionbox}
\begin{mnemonicbox}
``Amplitude Adjusts energy, Frequency Finds cycles,
Time-period Tracks one cycle''

\end{mnemonicbox}
\subsection*{Question 4(b) [4 marks]}\label{q4b}

\textbf{Write difference between transverse and longitudinal waves.}

\begin{solutionbox}

\begin{longtable}[]{@{}
  >{\raggedright\arraybackslash}p{(\linewidth - 4\tabcolsep) * \real{0.2128}}
  >{\raggedright\arraybackslash}p{(\linewidth - 4\tabcolsep) * \real{0.3830}}
  >{\raggedright\arraybackslash}p{(\linewidth - 4\tabcolsep) * \real{0.4043}}@{}}
\toprule\noalign{}
\begin{minipage}[b]{\linewidth}\raggedright
Property
\end{minipage} & \begin{minipage}[b]{\linewidth}\raggedright
Transverse Waves
\end{minipage} & \begin{minipage}[b]{\linewidth}\raggedright
Longitudinal Waves
\end{minipage} \\
\midrule\noalign{}
\endhead
\bottomrule\noalign{}
\endlastfoot
Direction of particle motion & Perpendicular to wave propagation &
Parallel to wave propagation \\
Formation of & Crests and troughs & Compressions and rarefactions \\
Examples & Light waves, water waves, electromagnetic waves & Sound
waves, seismic P-waves \\
Medium requirement & Can travel through vacuum (e.g., light) & Requires
material medium \\
Polarization & Can be polarized & Cannot be polarized \\
Speed & Generally faster in solids & Generally slower in solids \\
Mathematical representation &

y = A sin(kx - ωt) &

s = A sin(kx - ωt) \\

\end{longtable}

\textbf{Diagram}:

\begin{verbatim}
Transverse:
  ^  ^     ^
 / \/ \   / \
/     \ /    \
---------->
Direction of propagation

Longitudinal:
|||||   |||||   |||||
    |||||   |||||
---------->
Direction of propagation
\end{verbatim}

\end{solutionbox}
\begin{mnemonicbox}
``Transverse Travels perpendicular, Longitudinal Lies
along length''

\end{mnemonicbox}
\subsection*{Question 4(c)(i) [5 marks]}\label{q4c}

\textbf{How is ultrasonic wave produced using piezoelectric method?}

\begin{solutionbox}

\textbf{Piezoelectric Method for Ultrasonic Wave Production}:

\begin{center}
\textbf{Mermaid Diagram (Code)}
\begin{verbatim}
{Shaded}
{Highlighting}[]
graph LR
    A[Oscillator] {-{-}{} B[Amplifier]}
    B {-{-}{} C[Piezoelectric Crystal]}
    C {-{-}{} D[Ultrasonic Waves]}
{Highlighting}
{Shaded}
\end{verbatim}
\end{center}

\textbf{Working Principle}:

\begin{enumerate}
\tightlist
\item
  Based on piezoelectric effect - generating electric charge in response
  to mechanical stress and vice versa
\item
  High-frequency AC voltage applied across piezoelectric crystal
  (quartz, tourmaline, Rochelle salt)
\item
  Crystal vibrates at same frequency as applied voltage
\item
  When frequency matches natural frequency of crystal, resonance occurs
\item
  Maximum amplitude vibrations generate ultrasonic waves
\end{enumerate}

\textbf{Components}:

\begin{itemize}
\tightlist
\item
  \textbf{Oscillator}: Generates high-frequency electrical oscillations
\item
  \textbf{Amplifier}: Increases amplitude of oscillations
\item
  \textbf{Piezoelectric crystal}: Converts electrical energy to
  mechanical vibrations
\item
  \textbf{Mounting}: Supports crystal properly
\end{itemize}

\textbf{Frequency Range}: 20 kHz to several MHz

\textbf{Advantages}:

\begin{itemize}
\tightlist
\item
  High efficiency
\item
  Precise frequency control
\item
  Compact size
\item
  No moving parts
\end{itemize}

\end{solutionbox}
\begin{mnemonicbox}
``Piezo Produces waves when Properly Pulsed with
electricity''

\end{mnemonicbox}
\subsection*{Question 4(c)(ii) [2
marks]}\label{q4c}

\textbf{Explain any two properties of sound wave.}

\begin{solutionbox}

\textbf{1. Reflection of Sound}:

\begin{itemize}
\tightlist
\item
  Sound waves bounce back from obstacles
\item
  Follows law of reflection: angle of incidence = angle of reflection
\item
  Creates echo when reflected from distant objects
\item
  Applications: Sonar, echo location, acoustic design
\end{itemize}

\textbf{2. Refraction of Sound}:

\begin{itemize}
\tightlist
\item
  Bending of sound waves when passing from one medium to another
\item
  Caused by change in speed of sound in different media
\item
  Examples: Sound focusing in domes, sound heard better at night
\item
  Applications: Acoustic lenses, medical ultrasound
\end{itemize}

\textbf{Diagram}:

\begin{verbatim}
Reflection:        Refraction:
  \  |              /|
   \ |             / |
    \|            /  |
----|----       ---|---
    |/           /|
    |           / |
    |          /  |
\end{verbatim}

\end{solutionbox}
\begin{mnemonicbox}
``Sound Shows Remarkable Refractions During travel''

\end{mnemonicbox}
\subsection*{Question 4(a) [3 marks]
(OR)}\label{q4a}

\textbf{Define wavelength, phase and velocity of a wave.}

\begin{solutionbox}

\textbf{Wavelength}:

\begin{itemize}
\tightlist
\item
  Distance between two consecutive points in phase
\item
  Distance traveled during one complete oscillation
\item
  Denoted by `λ' (lambda)
\item
  Measured in meters (m)
\item
  Related to frequency (f) and velocity (v): λ = v/f
\end{itemize}

\textbf{Phase}:

\begin{itemize}
\tightlist
\item
  State of oscillation at a specific point and time
\item
  Measured in radians or degrees
\item
  Full cycle = 2π radians or 360^\circ
\item
  Points having same phase are in phase
\item
  Points differing by π radians (180^\circ) are in opposite phase
\end{itemize}

\textbf{Velocity}:

\begin{itemize}
\tightlist
\item
  Rate at which wave propagates through medium
\item
  Denoted by `v'
\item
  Measured in meters per second (m/s)
\item
  Related to wavelength and frequency: v = λf
\item
  Depends on properties of medium, not on wave characteristics
\end{itemize}

\textbf{Diagram for wavelength, phase and velocity}:

\begin{verbatim}
    |<---λ--->|
    |         |
    /\        /\        /\
   /  \      /  \      /  \
--/----\----/----\----/----\-->
  \    /    \    /    \    /
   \  /      \  /      \  /
    \/        \/        \/
    
    |---v·t---|
\end{verbatim}

\end{solutionbox}
\begin{mnemonicbox}
``Wavelength Wraps one cycle, Phase Portrays
position, Velocity Values propagation speed''

\end{mnemonicbox}
\subsection*{Question 4(b) [4 marks]
(OR)}\label{q4b}

\textbf{Explain constructive and destructive interference of waves.}

\begin{solutionbox}

\textbf{Interference}: Superposition of two or more waves at same point
in space resulting in a new wave pattern

\textbf{Constructive Interference}:

\begin{itemize}
\tightlist
\item
  Occurs when waves meet in phase (crest meets crest)
\item
  Phase difference = 0, 2π, 4π, \ldots{} (0^\circ, 360^\circ, 720^\circ, \ldots)
\item
  Path difference = nλ (n = 0, 1, 2, \ldots)
\item
  Results in amplitude larger than individual waves
\item
  Resultant amplitude = sum of individual amplitudes
\end{itemize}

\textbf{Destructive Interference}:

\begin{itemize}
\tightlist
\item
  Occurs when waves meet in opposite phase (crest meets trough)
\item
  Phase difference = π, 3π, 5π, \ldots{} (180^\circ, 540^\circ, 900^\circ, \ldots)
\item
  Path difference = (n+1/2)λ (n = 0, 1, 2, \ldots)
\item
  Results in amplitude smaller than individual waves
\item
  Complete cancellation if amplitudes are equal
\end{itemize}

\textbf{Diagram}:

\begin{verbatim}
Constructive:             Destructive:
  /\    /\                  /\    \/
 /  \  /  \                /  \  /  \
/    \/    \              /    \/    \
--------------            --------------
      |                         |
      \/                        |
     /  \                       |
    /    \                 --------- 
   /      \
  /        \
\end{verbatim}

\end{solutionbox}
\begin{mnemonicbox}
``Constructive Creates Larger waves; Destructive
Diminishes wave height''

\end{mnemonicbox}
\subsection*{Question 4(c)(i) [5 marks]
(OR)}\label{q4c}

\textbf{How is ultrasonic wave produced using magnetostriction method?}

\begin{solutionbox}

\textbf{Magnetostriction Method for Ultrasonic Wave Production}:

\begin{center}
\textbf{Mermaid Diagram (Code)}
\begin{verbatim}
{Shaded}
{Highlighting}[]
graph LR
    A[Oscillator] {-{-}{} B[Amplifier]}
    B {-{-}{} C[Coil around Ferromagnetic Rod]}
    C {-{-}{} D[Magnetostrictive Rod Vibrations]}
    D {-{-}{} E[Ultrasonic Waves]}
{Highlighting}
{Shaded}
\end{verbatim}
\end{center}

\textbf{Working Principle}:

\begin{enumerate}
\tightlist
\item
  Based on magnetostriction effect - dimensional change in ferromagnetic
  materials when placed in magnetic field
\item
  When magnetic field is applied, rod contracts
\item
  When field is removed, rod expands back to original size
\item
  Alternating current creates alternating magnetic field
\item
  Rod vibrates at frequency of applied current
\item
  These vibrations generate ultrasonic waves
\end{enumerate}

\textbf{Components}:

\begin{itemize}
\tightlist
\item
  \textbf{Oscillator}: Generates high-frequency electrical oscillations
\item
  \textbf{Amplifier}: Increases amplitude of oscillations
\item
  \textbf{Coil}: Creates magnetic field when current passes
\item
  \textbf{Ferromagnetic rod}: Nickel, iron-nickel alloy, or ferrites
\item
  \textbf{Mounting}: Supports rod properly
\end{itemize}

\textbf{Frequency Range}: 20 kHz to 100 kHz (lower than piezoelectric
method)

\textbf{Advantages}:

\begin{itemize}
\tightlist
\item
  Handles high power
\item
  Suitable for high-intensity applications
\item
  Rugged construction
\item
  Works well at lower frequencies
\end{itemize}

\textbf{Limitations}:

\begin{itemize}
\tightlist
\item
  Limited to lower frequencies
\item
  Lower efficiency than piezoelectric method
\item
  Heating of rod at high frequencies
\end{itemize}

\end{solutionbox}
\begin{mnemonicbox}
``Magnetic Materials Move Minutely Making ultrasonic
waves''

\end{mnemonicbox}
\subsection*{Question 4(c)(ii) [2 marks]
(OR)}\label{q4c}

\textbf{Explain any two properties of light wave.}

\begin{solutionbox}

\textbf{1. Reflection of Light}:

\begin{itemize}
\tightlist
\item
  Light bounces back when it strikes a surface
\item
  Follows law of reflection: angle of incidence = angle of reflection
\item
  Specular reflection from smooth surfaces
\item
  Diffuse reflection from rough surfaces
\item
  Applications: Mirrors, reflectors, optical instruments
\end{itemize}

\textbf{2. Refraction of Light}:

\begin{itemize}
\tightlist
\item
  Bending of light when passing from one medium to another
\item
  Follows Snell's law: n_{1}sin(θ_{1}) = n_{2}sin(θ_{2})
\item
  Caused by change in speed of light in different media
\item
  Examples: Bent appearance of stick in water
\item
  Applications: Lenses, prisms, fiber optics
\end{itemize}

\textbf{Diagram}:

\begin{verbatim}
Reflection:        Refraction:
  \  |              /|
   \ |             / |
    \|            /  |
----|----       ---|---
    |/           /|
    |           / |
    |          /  |
\end{verbatim}

\end{solutionbox}
\begin{mnemonicbox}
``Light Likes to Reflect from mirrors and Refract
through media''

\end{mnemonicbox}
\subsection*{Question 5(a) [3 marks]}\label{q5a}

\textbf{Write characteristics of LASER.}

\begin{solutionbox}

\textbf{Characteristics of LASER}:

\begin{longtable}[]{@{}
  >{\raggedright\arraybackslash}p{(\linewidth - 2\tabcolsep) * \real{0.5517}}
  >{\raggedright\arraybackslash}p{(\linewidth - 2\tabcolsep) * \real{0.4483}}@{}}
\toprule\noalign{}
\begin{minipage}[b]{\linewidth}\raggedright
Characteristic
\end{minipage} & \begin{minipage}[b]{\linewidth}\raggedright
Description
\end{minipage} \\
\midrule\noalign{}
\endhead
\bottomrule\noalign{}
\endlastfoot
Monochromatic & Single wavelength/color (very narrow frequency range) \\
Coherent & All waves in same phase, creating high interference \\
Directional & Highly collimated, minimal divergence over long
distances \\
High intensity & Concentrated energy in narrow beam \\
High purity & Extremely pure color compared to ordinary light \\
\end{longtable}

\textbf{Diagram}:

\begin{verbatim}
Ordinary Light:            LASER:
  ---                        -------
 /   \                      |       |
/     \                     |       |
-------                     |       |
Different                   -------
wavelengths                 Single wavelength,
& directions                single direction
\end{verbatim}

\end{solutionbox}
\begin{mnemonicbox}
``LASER Light: Monochromatic, Coherent, Directional,
Intense''

\end{mnemonicbox}
\subsection*{Question 5(b) [4 marks]}\label{q5b}

\textbf{Discuss importance of LASER in engineering and medical field.}

\begin{solutionbox}

\textbf{Importance of LASER in Engineering}:

\begin{enumerate}
\tightlist
\item
  \textbf{Manufacturing}:

  \begin{itemize}
  \tightlist
  \item
    Precision cutting and welding of metals
  \item
    3D printing and rapid prototyping
  \item
    Engraving and marking materials
  \end{itemize}
\item
  \textbf{Measurement and Testing}:

  \begin{itemize}
  \tightlist
  \item
    Distance measurement (LIDAR)
  \item
    Alignment and leveling
  \item
    Non-destructive testing
  \item
    Holography for stress analysis
  \end{itemize}
\item
  \textbf{Communications}:

  \begin{itemize}
  \tightlist
  \item
    Fiber optic communications
  \item
    Free-space optical communication
  \item
    Data storage (CD/DVD/Blu-ray)
  \end{itemize}
\item
  \textbf{Material Processing}:

  \begin{itemize}
  \tightlist
  \item
    Heat treatment
  \item
    Surface hardening
  \item
    Micromachining
  \end{itemize}
\end{enumerate}

\textbf{Importance of LASER in Medical Field}:

\begin{enumerate}
\tightlist
\item
  \textbf{Surgery}:

  \begin{itemize}
  \tightlist
  \item
    Bloodless cutting (laser scalpel)
  \item
    Ophthalmic surgery (LASIK)
  \item
    Dermatological procedures
  \item
    Tumor removal
  \end{itemize}
\item
  \textbf{Diagnostics}:

  \begin{itemize}
  \tightlist
  \item
    Laser imaging
  \item
    Spectroscopy
  \item
    Flow cytometry
  \item
    Optical coherence tomography
  \end{itemize}
\item
  \textbf{Therapy}:

  \begin{itemize}
  \tightlist
  \item
    Photodynamic therapy for cancer
  \item
    Low-level laser therapy
  \item
    Pain management
  \item
    Cosmetic procedures (hair removal, skin rejuvenation)
  \end{itemize}
\item
  \textbf{Dentistry}:

  \begin{itemize}
  \tightlist
  \item
    Cavity detection
  \item
    Teeth whitening
  \item
    Gum surgery
  \end{itemize}
\end{enumerate}

\end{solutionbox}
\begin{mnemonicbox}
``LASER Enhances Manufacturing, Measures precisely,
Communicates data, Heals patients''

\end{mnemonicbox}
\subsection*{Question 5(c)(i) [5 marks]}\label{q5c}

\textbf{What is importance of population inversion and metastable state
for production of LASER?}

\begin{solutionbox}

\textbf{Population Inversion}:

\begin{itemize}
\tightlist
\item
  \textbf{Definition}: State where more atoms are in excited state than
  in ground state (reverse of normal equilibrium)
\item
  \textbf{Importance}:

  \begin{enumerate}
  \tightlist
  \item
    Essential condition for laser action to occur
  \item
    Creates environment for stimulated emission to dominate over
    absorption
  \item
    Enables amplification of light (negative absorption)
  \item
    Without it, emitted photons would be absorbed, preventing laser
    action
  \item
    Required for chain reaction of stimulated emission
  \end{enumerate}
\end{itemize}

\textbf{Diagram}:

\begin{verbatim}
Normal:              Population Inversion:
  -----                  -----
  |   | Few atoms        ||||||| Many atoms
  -----                  -----
     ↑                      ↓
     |                      |
  -----                  -----
  ||||||| Many atoms     |   | Few atoms
  -----                  -----
Ground state          Ground state
\end{verbatim}

\textbf{Metastable State}:

\begin{itemize}
\tightlist
\item
  \textbf{Definition}: Excited energy state with relatively long
  lifetime (10^{-}^{3} to 10^{-}^{7} seconds)
\item
  \textbf{Importance}:

  \begin{enumerate}
  \tightlist
  \item
    Allows accumulation of excited atoms (temporary energy reservoir)
  \item
    Provides time for population inversion to establish
  \item
    Long lifetime prevents rapid spontaneous emission
  \item
    Ensures stimulated emission dominates over spontaneous emission
  \item
    Essential for continuous laser operation
  \end{enumerate}
\end{itemize}

\textbf{Energy Level Diagram}:

\begin{verbatim}
          |
  E_{3} ----|---- Short-lived excited state
          |
          v Fast transition (non-radiative)
          |
  E_{2} ----|---- Metastable state (long lifetime)
          |
          v Stimulated emission (LASER)
          |
  E_{1} ----|---- Ground state
\end{verbatim}

\end{solutionbox}
\begin{mnemonicbox}
``Population Inversion Makes Electrons Stay high;
Metastable maintains this Situation Longer''

\end{mnemonicbox}
\subsection*{Question 5(c)(ii) [2
marks]}\label{q5c}

\textbf{Explain graded index optical fibre.}

\begin{solutionbox}

\textbf{Graded Index Optical Fiber}:

\begin{itemize}
\tightlist
\item
  \textbf{Structure}: Core with gradually decreasing refractive index
  from center to periphery
\item
  \textbf{Refractive Index Profile}: Follows parabolic pattern: n(r) =
  n_{1}(1 - αr^{2})
\item
  \textbf{Light Propagation}: Light travels in curved paths rather than
  zigzag pattern
\item
  \textbf{Mechanism}: Light near periphery travels faster than at
  center, compensating for longer path
\item
  \textbf{Advantages}:

  \begin{enumerate}
  \tightlist
  \item
    Reduced modal dispersion compared to step index fiber
  \item
    Higher bandwidth
  \item
    Less signal distortion
  \item
    Suitable for medium-distance communication
  \end{enumerate}
\end{itemize}

\textbf{Cross-sectional Diagram}:

\begin{verbatim}
       Cladding
    ┌───────────┐
    │ ╭───────╮ │
    │ │       │ │
    │ │ Core  │ │
    │ │       │ │
    │ ╰───────╯ │
    └───────────┘

Refractive Index Profile:
    │     ╱╲
    │    /  \
n   │   /    \
    │  /      \
    │ /        \
    │/          \
    └────────────
      Distance
\end{verbatim}

\end{solutionbox}
\begin{mnemonicbox}
``Graded Index Gradually Improves transmission by
Smoothing dispersion''

\end{mnemonicbox}
\subsection*{Question 5(a) [3 marks]
(OR)}\label{q5a}

\textbf{Define refraction of light and write Snell's law.}

\begin{solutionbox}

\textbf{Refraction of Light}:

\begin{itemize}
\tightlist
\item
  Bending of light when it passes from one transparent medium to another
\item
  Occurs due to change in speed of light in different media
\item
  Direction changes but frequency remains same
\item
  Wavelength changes with speed
\end{itemize}

\textbf{Snell's Law}:

\begin{itemize}
\tightlist
\item
  Mathematical relationship governing refraction
\item
  States that ratio of sines of angles of incidence and refraction
  equals ratio of refractive indices
\item
  Formula: n_{1}sin(θ_{1}) = n_{2}sin(θ_{2})
\item
  Where:

  \begin{itemize}
  \tightlist
  \item
    n_{1} = Refractive index of first medium
  \item
    n_{2} = Refractive index of second medium
  \item
    θ_{1} = Angle of incidence
  \item
    θ_{2} = Angle of refraction
  \end{itemize}
\end{itemize}

\textbf{Diagram}:

\begin{verbatim}
Medium 1 (n_{1})
    \   |
     \  | θ_{1}
      \ |
-------|---------
       |\ 
       | \ θ_{2}
       |  \
Medium 2 (n_{2})
\end{verbatim}

\end{solutionbox}
\begin{mnemonicbox}
``Sine ratios Equal Index ratios'' or ``n_{1}Sin_{1} =
n_{2}Sin_{2}''

\end{mnemonicbox}
\subsection*{Question 5(b) [4 marks]
(OR)}\label{q5b}

\textbf{Discuss importance of optical fibre in engineering and medical
field.}

\begin{solutionbox}

\textbf{Importance of Optical Fiber in Engineering}:

\begin{enumerate}
\tightlist
\item
  \textbf{Communications}:

  \begin{itemize}
  \tightlist
  \item
    High-speed internet transmission
  \item
    Long-distance telecommunications
  \item
    Secure data transmission (difficult to tap)
  \item
    Higher bandwidth than copper cables
  \end{itemize}
\item
  \textbf{Sensors and Instrumentation}:

  \begin{itemize}
  \tightlist
  \item
    Temperature, pressure, strain measurement
  \item
    Structural health monitoring
  \item
    Chemical and biological sensing
  \item
    Seismic detection
  \end{itemize}
\item
  \textbf{Industrial Applications}:

  \begin{itemize}
  \tightlist
  \item
    Remote inspection of hazardous areas
  \item
    Industrial process control
  \item
    Power system monitoring
  \item
    Mining and petroleum exploration
  \end{itemize}
\item
  \textbf{Computing}:

  \begin{itemize}
  \tightlist
  \item
    High-speed data transfer between components
  \item
    Optical interconnects
  \item
    Quantum computing connections
  \end{itemize}
\end{enumerate}

\textbf{Importance of Optical Fiber in Medical Field}:

\begin{enumerate}
\tightlist
\item
  \textbf{Diagnostics}:

  \begin{itemize}
  \tightlist
  \item
    Endoscopy for internal organ examination
  \item
    Laparoscopy for minimally invasive surgery
  \item
    Angioscopy for blood vessel examination
  \item
    Bronchoscopy for respiratory tract examination
  \end{itemize}
\item
  \textbf{Surgery}:

  \begin{itemize}
  \tightlist
  \item
    Laser light delivery for precise operations
  \item
    Photodynamic therapy
  \item
    Microsurgery guidance
  \item
    Remote surgery monitoring
  \end{itemize}
\item
  \textbf{Imaging}:

  \begin{itemize}
  \tightlist
  \item
    Optical coherence tomography
  \item
    Confocal microscopy
  \item
    Optogenetics
  \item
    Medical spectroscopy
  \end{itemize}
\item
  \textbf{Treatment}:

  \begin{itemize}
  \tightlist
  \item
    Phototherapy for skin conditions
  \item
    Laser treatment delivery
  \item
    Biosensing for real-time monitoring
  \item
    Targeted drug delivery
  \end{itemize}
\end{enumerate}

\end{solutionbox}
\begin{mnemonicbox}
``Optical Fibers Connect, Sense, Visualize, and
Treat''

\end{mnemonicbox}
\subsection*{Question 5(c)(i) [5 marks]
(OR)}\label{q5c}

\textbf{Derive formula for numerical aperture and angle of acceptance of
optical fibre.}

\begin{solutionbox}

\textbf{Numerical Aperture (NA) Derivation}:

\begin{center}
\textbf{Mermaid Diagram (Code)}
\begin{verbatim}
{Shaded}
{Highlighting}[]
graph LR
    A[Consider Critical Angle at Core{-Cladding Interface] {-}{-}{} B[Apply Snell{}s Law]}
    B {-{-}{} C[Relate to Angle of Acceptance]}
    C {-{-}{} D[Derive NA Formula]}
{Highlighting}
{Shaded}
\end{verbatim}
\end{center}

\textbf{Step 1}: Consider critical angle (θc) at core-cladding interface

\begin{itemize}
\tightlist
\item
  At critical angle, refracted ray grazes along interface
\item
  sin(θc) = n_{2}/n_{1} (where n_{1} = core index, n_{2} = cladding index)
\end{itemize}

\textbf{Step 2}: For a ray traveling in core, apply condition for total
internal reflection

\begin{itemize}
\tightlist
\item
  Ray must strike at angle greater than critical angle
\item
  Maximum angle in core: 90^\circ - θc
\end{itemize}

\textbf{Step 3}: For ray entering from air (n_{0} = 1), apply Snell's law

\begin{itemize}
\tightlist
\item
  n_{0}sin(θ_{a}) = n_{1}sin(θ_{1})
\item
  sin(θ_{a}) = n_{1}sin(θ_{1})
\item
  Where θ_{a} is acceptance angle
\end{itemize}

\textbf{Step 4}: Use maximum value of θ_{1} (90^\circ - θc)

\begin{itemize}
\tightlist
\item
  sin(θ_{a}) = n_{1}sin(90^\circ - θc) = n_{1}cos(θc)
\end{itemize}

\textbf{Step 5}: Substitute sin(θc) = n_{2}/n_{1}

\begin{itemize}
\tightlist
\item
  cos(θc) = \sqrt(1 - sin^{2}(θc)) = \sqrt(1 - (n_{2}/n_{1})^{2})
\end{itemize}

\textbf{Step 6}: Therefore:

\begin{itemize}
\tightlist
\item
  sin(θ_{a}) = n_{1}\sqrt(1 - (n_{2}/n_{1})^{2}) = \sqrt(n_{1}^{2} - n_{2}^{2})
\end{itemize}

\textbf{Final Formula}:

\begin{itemize}
\tightlist
\item
  Numerical Aperture (NA) = sin(θ_{a}) = \sqrt(n_{1}^{2} - n_{2}^{2})
\item
  Where θ_{a} is angle of acceptance
\end{itemize}

\textbf{Diagram}:

\begin{verbatim}
              θ_{a}
               \
 Air (n_{0})       \
-----------------\-----------
                  \
 Core (n_{1})        \_____ θ_{1}
                    \
                     \
----------------------------
 Cladding (n_{2})
\end{verbatim}

\end{solutionbox}
\begin{mnemonicbox}
``NA Notes Acceptance angle; \sqrt(n_{1}^{2} - n_{2}^{2}) Shows
maximum sine''

\end{mnemonicbox}
\subsection*{Question 5(c)(ii) [2 marks]
(OR)}\label{q5c}

\textbf{Explain step index optical fibre.}

\begin{solutionbox}

\textbf{Step Index Optical Fiber}:

\begin{itemize}
\tightlist
\item
  \textbf{Structure}: Core with uniform refractive index surrounded by
  cladding with lower uniform refractive index
\item
  \textbf{Refractive Index Profile}: Sharp transition (step) between
  core and cladding
\item
  \textbf{Light Propagation}: Light travels in zigzag path by total
  internal reflection
\item
  \textbf{Types}:

  \begin{enumerate}
  \tightlist
  \item
    Single-mode: Small core (8-10 μm), carries one mode of light
  \item
    Multi-mode: Large core (50-100 μm), carries multiple modes
  \end{enumerate}
\end{itemize}

\textbf{Characteristics}:

\begin{itemize}
\tightlist
\item
  Simple construction
\item
  Lower bandwidth than graded index
\item
  Suffers from modal dispersion in multi-mode
\item
  Longer path for some rays causes pulse spreading
\end{itemize}

\textbf{Cross-sectional Diagram}:

\begin{verbatim}
       Cladding (n_{2})
    ┌───────────┐
    │ ┌───────┐ │
    │ │       │ │
    │ │ Core  │ │
    │ │ (n_{1})  │ │
    │ └───────┘ │
    └───────────┘

Refractive Index Profile:
    │ ┌──┐
    │ │  │
n   │ │  │
    │ │  │
    │ │  │
    │ └──┘
    └─────────
      Distance
\end{verbatim}

\end{solutionbox}
\begin{mnemonicbox}
``Step Index Shows Two distinct Indices with Perfect
boundary''

\end{mnemonicbox}

\end{document}
