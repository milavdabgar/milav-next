\documentclass[10pt,a4paper]{article}

% content/resources/templates/preamble.tex
\usepackage[margin=0.6in]{geometry}
\author{Milav Dabgar}
\usepackage{amsmath,amssymb,amsthm}
\usepackage{booktabs}
\usepackage{multirow}
\usepackage{xcolor}
\usepackage{tcolorbox}
\tcbuselibrary{breakable,skins}
\usepackage[colorlinks=true,linkcolor=blue]{hyperref}
\usepackage{titlesec}
\usepackage{enumitem}
\usepackage{tikz}
\usepackage{pgfplots}
\usepackage{circuitikz}
\usepackage[version=4]{mhchem}
\usepackage{longtable}
\usepackage{array}
\usepackage{float}
\usepackage{caption}
\usepackage{listings}

\lstset{
  basicstyle=\small\ttfamily,
  breaklines=true,
  breakatwhitespace=false,
  postbreak=\mbox{\textcolor{red}{$\hookrightarrow$}\space},
  float=false,
  numbers=left,
  numberstyle=\tiny\color{gray},
  numbersep=10pt,
  xleftmargin=2em,
  keywordstyle=\color{blue},
  commentstyle=\color{green!60!black},
  stringstyle=\color{purple},
  backgroundcolor=\color{gray!5},
  showstringspaces=false,
  tabsize=2,
  captionpos=b,
  keepspaces=true,
  columns=flexible
}

\pgfplotsset{compat=1.18}
\usetikzlibrary{shapes,arrows,positioning,calc,patterns,decorations.pathmorphing,decorations.markings,arrows.meta}

% Color scheme
\definecolor{headcolor}{RGB}{0,102,204}
\definecolor{keycolor}{RGB}{220,20,60}
\definecolor{solutioncolor}{RGB}{34,139,34}
\definecolor{mnemoniccolor}{RGB}{148,0,211}
\definecolor{codecolor}{RGB}{0,0,100}

% Spacing
\setlength{\parskip}{3pt}
\setlist[itemize]{nosep}
\setlist[enumerate]{nosep}

% Title formatting
\titleformat{\section}{\Large\bfseries\color{headcolor}}{\thesection}{1em}{}
\titleformat{\subsection}{\large\bfseries\color{headcolor}}{\thesubsection}{1em}{}

% Pandoc tightlist compatibility
\providecommand{\tightlist}{%
  \setlength{\itemsep}{0pt}\setlength{\parskip}{0pt}}

% Pandoc longtable compatibility
\newcounter{none}
\def\thenone{}


% content/resources/templates/english-boxes.tex
% This file is currently empty - it exists to maintain consistency with the import structure.
% Add custom environments here if needed in the future.


\begin{document}

\begin{center}
{\Huge\bfseries\color{headcolor} Subject Name Solutions}\\[5pt]
{\LARGE 4300005 -- Winter 2023}\\[3pt]
{\large Semester 1 Study Material}\\[3pt]
{\normalsize\textit{Detailed Solutions and Explanations}}
\end{center}

\vspace{10pt}

\subsection*{Question 1(a) [3 marks]}\label{q1a}

\textbf{Define: (a) Meter (b) Kelvin (c) Accuracy.}

\begin{solutionbox}

\begin{itemize}
\tightlist
\item
  \textbf{Meter}: The meter is the SI unit of length, defined as the
  distance traveled by light in vacuum during a time interval of
  1/299,792,458 of a second.
\item
  \textbf{Kelvin}: The kelvin is the SI unit of thermodynamic
  temperature, defined by setting the fixed numerical value of the
  Boltzmann constant k to 1.380649 \times 10\^{}-23 J/K.
\item
  \textbf{Accuracy}: Accuracy is the degree of closeness of a measured
  value to the true or standard value of the quantity being measured.
\end{itemize}

\end{solutionbox}
\begin{mnemonicbox}
``MKA - Meter measures Kilometers Accurately''

\end{mnemonicbox}
\subsection*{Question 1(b) [4 marks]}\label{q1b}

\textbf{Explain construction of Vernier calipers with clean figure.}

\begin{solutionbox}

\textbf{Diagram:}

\begin{verbatim}
     |--|--|--|--|--|--|--|--|--|--|
     |--|--|--|     Main Scale    |--|
     |  |  |  |  |  |  |  |  |  |
     0  1  2  3  4  5  6  7  8  9  10 cm
        |--|--|--|--|--|--|--|--|--|--|
        |    Vernier Scale       |
        0  1  2  3  4  5  6  7  8  9  
\end{verbatim}

Vernier calipers consist of:

\begin{itemize}
\tightlist
\item
  \textbf{Main scale}: Fixed scale marked in standard units (mm or
  inches)
\item
  \textbf{Vernier scale}: Movable scale that slides along the main scale
\item
  \textbf{Fixed jaw}: Attached to the main scale
\item
  \textbf{Movable jaw}: Connected to the vernier scale
\item
  \textbf{Depth probe}: For measuring depth of cavities
\item
  \textbf{External jaws}: For measuring outer dimensions
\item
  \textbf{Internal jaws}: For measuring inner dimensions
\end{itemize}

\end{solutionbox}
\begin{mnemonicbox}
``FMMVJ - Fixed Main scale Makes Vernier Jaw move''

\end{mnemonicbox}
\subsection*{Question 1(c)(1) [4 marks]}\label{q1c}

\textbf{What is physical quantities? Explain its types depending on
direction.}

\begin{solutionbox}

A physical quantity is a measurable property of a physical system that
can be quantified by measurement.

\textbf{Types of physical quantities based on direction:}

\begin{longtable}[]{@{}
  >{\raggedright\arraybackslash}p{(\linewidth - 2\tabcolsep) * \real{0.5000}}
  >{\raggedright\arraybackslash}p{(\linewidth - 2\tabcolsep) * \real{0.5000}}@{}}
\toprule\noalign{}
\begin{minipage}[b]{\linewidth}\raggedright
Scalar Quantities
\end{minipage} & \begin{minipage}[b]{\linewidth}\raggedright
Vector Quantities
\end{minipage} \\
\midrule\noalign{}
\endhead
\bottomrule\noalign{}
\endlastfoot
Have only magnitude & Have both magnitude and direction \\
Examples: mass, time, temperature, energy & Examples: displacement,
velocity, force, acceleration \\
Represented by simple numbers & Represented by arrows or directed line
segments \\
Addition follows simple arithmetic & Addition follows vector algebra
(parallelogram law) \\
No directional properties & Completely specified by direction and
magnitude \\
\end{longtable}

\end{solutionbox}
\begin{mnemonicbox}
``SMAVD - Scalars have Magnitude Alone, Vectors have
Direction''

\end{mnemonicbox}
\subsection*{Question 1(c)(2) [3 marks]}\label{q1c}

\textbf{Pitch of micrometer screw is 0.5 mm. If its circular scale is
divided in equal 100 divisions, Calculate L.C.}

\begin{solutionbox}

\textbf{Calculation:} Least Count (L.C.) = Pitch / Number of divisions
on circular scale L.C. = 0.5 mm / 100 = 0.005 mm

Therefore, the least count of the micrometer screw gauge is 0.005 mm.

\end{solutionbox}
\begin{mnemonicbox}
``PDL - Pitch Divided gives Least count''

\end{mnemonicbox}
\subsection*{Question 1(c) OR [7
marks]}\label{q1c}

\textbf{Explain errors of Micrometer screw gauge with figure.}

\begin{solutionbox}

\textbf{Diagram:}

\begin{verbatim}
    Ratchet   Barrel   Thimble
      |         |        |
      V         V        V
    [===]======|======[=====]
         \              /
          \            /
           \          /
            \        /
             \      /
              \    /
               Anvil
\end{verbatim}

Common errors in micrometer screw gauge:

\begin{itemize}
\tightlist
\item
  \textbf{Zero error}: When the measuring faces are in contact, the zero
  of thimble doesn't coincide with the datum line

  \begin{itemize}
  \tightlist
  \item
    \textbf{Positive zero error}: When the zero mark on thimble is below
    the datum line
  \item
    \textbf{Negative zero error}: When the zero mark on thimble is above
    the datum line
  \end{itemize}
\item
  \textbf{Backlash error}: Play between the screw and nut, causes
  different readings in forward and backward movement
\item
  \textbf{Instrumental error}: Due to manufacturing defects or wear and
  tear
\item
  \textbf{Parallax error}: When line of sight isn't perpendicular to
  scale reading
\end{itemize}

\textbf{Correction formula:} True reading = Observed reading - Zero
error

\end{solutionbox}
\begin{mnemonicbox}
``ZBIP - Zero, Backlash, Instrument and Parallax
errors make measurements trip''

\end{mnemonicbox}
\subsection*{Question 2(a) [3 marks]}\label{q2a}

\textbf{Explain Coulomb's inverse square law.}

\begin{solutionbox}

Coulomb's inverse square law states that the electrostatic force between
two point charges is:

\begin{itemize}
\tightlist
\item
  Directly proportional to the product of the magnitudes of charges
\item
  Inversely proportional to the square of the distance between them
\item
  Acts along the line joining the two charges
\end{itemize}

\textbf{Mathematical expression:} F = k(q_{1}q_{2})/r^{2}

Where:

\begin{itemize}
\tightlist
\item
  F = Electrostatic force between charges
\item
  k = Coulomb's constant (9 \times 10^{9} N·m^{2}/C^{2})
\item
  q_{1}, q_{2} = Magnitudes of the two charges
\item
  r = Distance between the charges
\end{itemize}

\end{solutionbox}
\begin{mnemonicbox}
``PDSA - Product of charges Directly, Square of
distance inversely, Along the line''

\end{mnemonicbox}
\subsection*{Question 2(b) [4 marks]}\label{q2b}

\textbf{Explain electrical potential difference.}

\begin{solutionbox}

Electrical potential difference (voltage) is the work done per unit
charge in moving a positive test charge between two points in an
electric field.

\textbf{Mathematical expression:} V = W/q

Where:

\begin{itemize}
\tightlist
\item
  V = Potential difference (volts)
\item
  W = Work done (joules)
\item
  q = Charge (coulombs)
\end{itemize}

\textbf{Key characteristics:}

\begin{itemize}
\tightlist
\item
  Measured in volts (V)
\item
  Scalar quantity (has magnitude only)
\item
  Path independent (depends only on initial and final positions)
\item
  Represents energy per unit charge
\end{itemize}

\end{solutionbox}
\begin{mnemonicbox}
``WPCS - Work Per Charge is what potential difference
Says''

\end{mnemonicbox}
\subsection*{Question 2(c) [7 marks]}\label{q2c}

\textbf{Explain equivalent capacitance of capacitors in series and in
parallel combinations.}

\begin{solutionbox}

\textbf{Series Combination:}

\textbf{Diagram:}

\begin{verbatim}
    -----||----||----||----- 
        C_{1}    C_{2}    C_{3}
\end{verbatim}

\begin{itemize}
\tightlist
\item
  When capacitors are connected end-to-end
\item
Same charge on each capacitor:

Q = Q_{1} = Q_{2} = Q_{3}

\item
  Total potential difference: V = V_{1} + V_{2} + V_{3}
\item
  Equivalent capacitance formula: 1/C_{e}q = 1/C_{1} + 1/C_{2} + 1/C_{3} + \ldots{}
\item
  Equivalent capacitance is less than the smallest individual
  capacitance
\end{itemize}

\textbf{Parallel Combination:}

\textbf{Diagram:}

\begin{verbatim}
    -----||-----
         C_{1}     
    -----||-----
         C_{2}     
    -----||-----
         C_{3}     
\end{verbatim}

\begin{itemize}
\tightlist
\item
  When capacitors are connected between the same two points
\item
Same potential difference across each:

V = V_{1} = V_{2} = V_{3}

\item
  Total charge: Q = Q_{1} + Q_{2} + Q_{3}
\item
  Equivalent capacitance formula: C_{e}q = C_{1} + C_{2} + C_{3} + \ldots{}
\item
  Equivalent capacitance is greater than the largest individual
  capacitance
\end{itemize}

\textbf{Comparison Table:}

\begin{longtable}[]{@{}
  >{\raggedright\arraybackslash}p{(\linewidth - 4\tabcolsep) * \real{0.3793}}
  >{\raggedright\arraybackslash}p{(\linewidth - 4\tabcolsep) * \real{0.2759}}
  >{\raggedright\arraybackslash}p{(\linewidth - 4\tabcolsep) * \real{0.3448}}@{}}
\toprule\noalign{}
\begin{minipage}[b]{\linewidth}\raggedright
Parameter
\end{minipage} & \begin{minipage}[b]{\linewidth}\raggedright
Series
\end{minipage} & \begin{minipage}[b]{\linewidth}\raggedright
Parallel
\end{minipage} \\
\midrule\noalign{}
\endhead
\bottomrule\noalign{}
\endlastfoot
Charge & Same on all capacitors & Distributed as per capacitance \\
Voltage & Divided across capacitors & Same across all capacitors \\
Equivalent capacitance & 1/C_{e}q = 1/C_{1} + 1/C_{2} + \ldots{} & C_{e}q = C_{1} + C_{2}
+ \ldots{} \\
Resulting capacitance & Smaller than any individual C & Larger than any
individual C \\
\end{longtable}

\end{solutionbox}
\begin{mnemonicbox}
``RAPS - Reciprocals Add in Parallel Sum''

\end{mnemonicbox}
\subsection*{Question 2(a) OR [3
marks]}\label{q2a}

\textbf{Write characteristics of electrical lines.}

\begin{solutionbox}

\textbf{Characteristics of electric field lines:}

\begin{itemize}
\tightlist
\item
  \textbf{Direction}: Always point from positive to negative charge
\item
  \textbf{Nature}: Start from positive charge and end at negative charge
\item
  \textbf{Continuity}: Never intersect each other
\item
  \textbf{Density}: Closer lines indicate stronger electric field
\item
  \textbf{Perpendicularity}: Always perpendicular to equipotential
  surfaces
\item
  \textbf{Shape}: Straight lines for uniform fields, curved for
  non-uniform fields
\item
  \textbf{Open/Closed}: Always open curves, unlike magnetic field lines
\end{itemize}

\end{solutionbox}
\begin{mnemonicbox}
``DNCPS - Direction, Never cross, Closeness shows
strength, Perpendicular, Straight/curved''

\end{mnemonicbox}
\subsection*{Question 2(b) OR [4
marks]}\label{q2b}

\textbf{Explain electric flux.}

\begin{solutionbox}

Electric flux is a measure of the electric field passing through a given
area.

\textbf{Mathematical expression:} Φ_{e} = E·A·cosθ

Where:

\begin{itemize}
\tightlist
\item
  Φ_{e} = Electric flux (N·m^{2}/C or V·m)
\item
  E = Electric field strength (N/C or V/m)
\item
  A = Area of the surface (m^{2})
\item
  θ = Angle between electric field and normal to the surface
\end{itemize}

\textbf{Key characteristics:}

\begin{itemize}
\tightlist
\item
  Vector quantity
\item
  SI unit is newton-meter-squared per coulomb (N·m^{2}/C) or volt-meter
  (V·m)
\item
  Represents the number of field lines passing through a surface
\item
  Maximum when field is perpendicular to surface (θ = 0^\circ)
\item
  Zero when field is parallel to surface (θ = 90^\circ)
\end{itemize}

\end{solutionbox}
\begin{mnemonicbox}
``FACT - Flux = Area \times Cosθ \times Field sTreength''

\end{mnemonicbox}
\subsection*{Question 2(c) OR [7
marks]}\label{q2c}

\textbf{Explain capacitor and capacitance.}

\begin{solutionbox}

\textbf{Capacitor:} A capacitor is an electrical component designed to
store electric charge and energy in an electric field.

\textbf{Basic structure:}

\begin{verbatim}
    Plate 1      Plate 2
    ////////    ////////
    ////////    //////// — Dielectric
    ////////    ////////
    ////////    ////////
\end{verbatim}

\textbf{Capacitance:} The ability of a capacitor to store electric
charge at a given potential difference.

\textbf{Mathematical expression:} C = Q/V

Where:

\begin{itemize}
\tightlist
\item
  C = Capacitance (farads)
\item
  Q = Electric charge (coulombs)
\item
  V = Potential difference (volts)
\end{itemize}

\textbf{For a parallel plate capacitor:} C = ε_{0}εᵣA/d

Where:

\begin{itemize}
\tightlist
\item
  ε_{0} = Permittivity of free space (8.85 \times 10^{-}^{1}^{2} F/m)
\item
  εᵣ = Relative permittivity of dielectric
\item
  A = Area of overlap between plates
\item
  d = Distance between plates
\end{itemize}

\textbf{Factors affecting capacitance:}

\begin{itemize}
\tightlist
\item
  Increases with plate area
\item
  Decreases with plate separation
\item
  Increases with dielectric constant
\end{itemize}

\textbf{Applications of capacitors:}

\begin{itemize}
\tightlist
\item
  Energy storage
\item
  Filtering in power supplies
\item
  Timing circuits
\item
  Coupling and decoupling
\item
  Power factor correction
\end{itemize}

\end{solutionbox}
\begin{mnemonicbox}
``QVAD - Quotient of charge and Voltage, affected by
Area and Distance''

\end{mnemonicbox}
\subsection*{Question 3(a) [3 marks]}\label{q3a}

\textbf{Define: (a) Heat radiation (b) Kilocalorie (c) Thermometer.}

\begin{solutionbox}

\begin{itemize}
\tightlist
\item
  \textbf{Heat radiation}: The transfer of thermal energy in the form of
  electromagnetic waves without requiring a medium, occurring in vacuum
  or transparent media.
\item
  \textbf{Kilocalorie}: A unit of heat energy equal to 1000 calories,
  where one calorie is the amount of heat required to raise the
  temperature of 1 gram of water by 1^\circC at standard conditions.
\item
  \textbf{Thermometer}: An instrument used to measure temperature based
  on a physical property (like expansion of mercury) that changes with
  temperature.
\end{itemize}

\end{solutionbox}
\begin{mnemonicbox}
``RKT - Radiation needs no medium, Kilocalorie
measures energy, Thermometer shows temperature''

\end{mnemonicbox}
\subsection*{Question 3(b) [4 marks]}\label{q3b}

\textbf{Explain law of thermal conductivity.}

\begin{solutionbox}

The law of thermal conductivity (Fourier's law) states that the rate of
heat transfer through a material is:

\begin{itemize}
\tightlist
\item
  Directly proportional to the area of the section
\item
  Directly proportional to the temperature gradient
\item
  Dependent on the material's thermal conductivity
\end{itemize}

\textbf{Mathematical expression:} Q/t = -kA(dT/dx)

Where:

\begin{itemize}
\tightlist
\item
  Q/t = Rate of heat transfer (J/s or W)
\item
  k = Thermal conductivity of material (W/m·K)
\item
  A = Cross-sectional area (m^{2})
\item
  dT/dx = Temperature gradient (K/m)
\item
  Negative sign indicates heat flows from higher to lower temperature
\end{itemize}

\end{solutionbox}
\begin{mnemonicbox}
``GAKT - Gradient And area with K gives heat
Transfer''

\end{mnemonicbox}
\subsection*{Question 3(c)(1) [3 marks]}\label{q3c}

\textbf{A person has a fever of 102^\circF. So how much would it be in
Celsius and Kelvin?}

\begin{solutionbox}

\textbf{To convert from Fahrenheit to Celsius:} C = (F - 32) \times 5/9 C =
(102 - 32) \times 5/9

C = 70 \times 5/9

C = 38.89^\circC


\textbf{To convert from Celsius to Kelvin:} K = C + 273.15 K = 38.89 +
273.15 K = 312.04 K

Therefore, 102^\circF = 38.89^\circC = 312.04 K

\end{solutionbox}
\begin{mnemonicbox}
``FSK - From Fahrenheit Subtract 32, multiply by 5/9,
then add 273.15 for Kelvin''

\end{mnemonicbox}
\subsection*{Question 3(c)(2) [4 marks]}\label{q3c}

\textbf{Explain Celsius and Fahrenheit scale.}

\begin{solutionbox}

\textbf{Comparison of Celsius and Fahrenheit Temperature Scales:}

\begin{longtable}[]{@{}
  >{\raggedright\arraybackslash}p{(\linewidth - 4\tabcolsep) * \real{0.2500}}
  >{\raggedright\arraybackslash}p{(\linewidth - 4\tabcolsep) * \real{0.3409}}
  >{\raggedright\arraybackslash}p{(\linewidth - 4\tabcolsep) * \real{0.4091}}@{}}
\toprule\noalign{}
\begin{minipage}[b]{\linewidth}\raggedright
Parameter
\end{minipage} & \begin{minipage}[b]{\linewidth}\raggedright
Celsius Scale
\end{minipage} & \begin{minipage}[b]{\linewidth}\raggedright
Fahrenheit Scale
\end{minipage} \\
\midrule\noalign{}
\endhead
\bottomrule\noalign{}
\endlastfoot
Freezing point of water & 0^\circC & 32^\circF \\
Boiling point of water & 100^\circC & 212^\circF \\
Number of divisions & 100 divisions & 180 divisions \\
Developed by & Anders Celsius (1742) & Gabriel Fahrenheit (1724) \\
Used in & Most countries worldwide & Primarily USA and its
territories \\
Relation &

C = (F - 32) \times 5/9 &

F = (C \times 9/5) + 32 \\

\end{longtable}

\textbf{Diagram:}

\begin{verbatim}
Celsius     Fahrenheit
  100^\circC  —— 212^\circF  (Water boils)
    |          |
    |          |
    |          |
   0^\circC   —— 32^\circF   (Water freezes)
    |          |
  -17.8^\circC —— 0^\circF
\end{verbatim}

\end{solutionbox}
\begin{mnemonicbox}
``FBIC - Fahrenheit has Bigger numbers, Interval of
180, Conversion needs 5/9 or 9/5''

\end{mnemonicbox}
\subsection*{Question 3(a) OR [3
marks]}\label{q3a}

\textbf{Write definition, formula and unit of Heat capacity.}

\begin{solutionbox}

\textbf{Definition:} Heat capacity is the amount of heat energy required
to raise the temperature of an object by one degree (Celsius or Kelvin).

\textbf{Formula:} C = Q/ΔT

Where:

\begin{itemize}
\tightlist
\item
  C = Heat capacity (J/K or J/^\circC)
\item
  Q = Heat energy supplied (joules)
\item
  ΔT = Change in temperature (K or ^\circC)
\end{itemize}

\textbf{Units:} Joules per kelvin (J/K) or joules per degree Celsius
(J/^\circC)

\end{solutionbox}
\begin{mnemonicbox}
``QTC - Quotient of heat and Temperature Change gives
heat capacity''

\end{mnemonicbox}
\subsection*{Question 3(b) OR [4
marks]}\label{q3b}

\textbf{Explain Modes of Heat Transfer}

\begin{solutionbox}

\textbf{Three modes of heat transfer:}

\begin{longtable}[]{@{}
  >{\raggedright\arraybackslash}p{(\linewidth - 6\tabcolsep) * \real{0.1333}}
  >{\raggedright\arraybackslash}p{(\linewidth - 6\tabcolsep) * \real{0.2667}}
  >{\raggedright\arraybackslash}p{(\linewidth - 6\tabcolsep) * \real{0.2222}}
  >{\raggedright\arraybackslash}p{(\linewidth - 6\tabcolsep) * \real{0.3778}}@{}}
\toprule\noalign{}
\begin{minipage}[b]{\linewidth}\raggedright
Mode
\end{minipage} & \begin{minipage}[b]{\linewidth}\raggedright
Definition
\end{minipage} & \begin{minipage}[b]{\linewidth}\raggedright
Examples
\end{minipage} & \begin{minipage}[b]{\linewidth}\raggedright
Medium Required
\end{minipage} \\
\midrule\noalign{}
\endhead
\bottomrule\noalign{}
\endlastfoot
\textbf{Conduction} & Transfer of heat through direct molecular
collision without bulk motion of matter & Heat through metal rod,
cooking pan & Yes (solid preferred) \\
\textbf{Convection} & Transfer of heat by movement of heated particles
from one region to another & Boiling water, room heater, sea breeze &
Yes (fluid - liquid or gas) \\
\textbf{Radiation} & Transfer of heat via electromagnetic waves without
requiring medium & Solar radiation, microwave heating, infrared heaters
& No (works in vacuum) \\
\end{longtable}

\end{solutionbox}
\begin{mnemonicbox}
``CoCRa - Conduction needs Contact, Convection needs
Currents, Radiation needs no medium''

\end{mnemonicbox}
\subsection*{Question 3(c) OR [7
marks]}\label{q3c}

\textbf{Explain bimetallic thermometer.}

\begin{solutionbox}

\textbf{Diagram:}

\begin{verbatim}
                  Pointer
                     |
                     V
                   /---\
                  /     \
    Fixed end    /       \    Movement
    |-----------|         |--------------|
    |///////////|         |//////////////|
    |^^^^^^^^^^^|         |^^^^^^^^^^^^^^| <- Metal 1 (higher expansion)
    |-----------|         |--------------|
                 \       /
                  \     /
                   \---/
                   Scale
\end{verbatim}

\textbf{Working principle:}

\begin{itemize}
\tightlist
\item
  Based on differential thermal expansion of two different metals
\item
  Two metal strips with different coefficients of thermal expansion are
  bonded together
\item
  When heated, one metal expands more than the other
\item
  This uneven expansion causes the strip to bend toward the metal with
  lower expansion
\item
  The amount of bending is proportional to temperature change
\item
  A pointer attached to the strip indicates temperature on a calibrated
  scale
\end{itemize}

\textbf{Advantages:}

\begin{itemize}
\tightlist
\item
  Simple, robust construction
\item
  No liquid or gas required
\item
  Wide temperature range
\item
  Resistant to mechanical shocks
\item
  Can be used to make thermostats
\end{itemize}

\textbf{Limitations:}

\begin{itemize}
\tightlist
\item
  Less accurate than liquid-in-glass thermometers
\item
  Slower response to temperature changes
\item
  Subject to mechanical fatigue over time
\end{itemize}

\textbf{Applications:}

\begin{itemize}
\tightlist
\item
  Thermostats in home heating/cooling systems
\item
  Automobile cooling systems
\item
  Oven temperature controls
\item
  Circuit breakers
\end{itemize}

\end{solutionbox}
\begin{mnemonicbox}
``BENDS - Bimetallic strips Expand, Not equally,
Different metals, Show temperature''

\end{mnemonicbox}
\subsection*{Question 4(a) [3 marks]}\label{q4a}

\textbf{Define: (a) Frequency (b) Infrasonic waves (c) Echo.}

\begin{solutionbox}

\begin{itemize}
\tightlist
\item
  \textbf{Frequency}: The number of complete oscillations or cycles per
  unit time, measured in hertz (Hz).
\item
  \textbf{Infrasonic waves}: Sound waves with frequencies below the
  lower limit of human hearing (below 20 Hz) that cannot be heard by
  humans but may be detected by other animals.
\item
  \textbf{Echo}: A sound that is reflected back to the listener with
  sufficient time delay to be heard as a distinct repetition of the
  original sound.
\end{itemize}

\end{solutionbox}
\begin{mnemonicbox}
``FIE - Frequency counts cycles, Infrasonic is below
hearing, Echo comes back after reflection''

\end{mnemonicbox}
\subsection*{Question 4(b) [4 marks]}\label{q4b}

\textbf{Give distinction between Longitudinal and Transverse waves.}

\begin{solutionbox}

\textbf{Comparison between Longitudinal and Transverse Waves:}

\begin{longtable}[]{@{}
  >{\raggedright\arraybackslash}p{(\linewidth - 4\tabcolsep) * \real{0.2292}}
  >{\raggedright\arraybackslash}p{(\linewidth - 4\tabcolsep) * \real{0.3958}}
  >{\raggedright\arraybackslash}p{(\linewidth - 4\tabcolsep) * \real{0.3750}}@{}}
\toprule\noalign{}
\begin{minipage}[b]{\linewidth}\raggedright
Parameter
\end{minipage} & \begin{minipage}[b]{\linewidth}\raggedright
Longitudinal Waves
\end{minipage} & \begin{minipage}[b]{\linewidth}\raggedright
Transverse Waves
\end{minipage} \\
\midrule\noalign{}
\endhead
\bottomrule\noalign{}
\endlastfoot
\textbf{Direction of particle motion} & Parallel to wave propagation &
Perpendicular to wave propagation \\
\textbf{Example} & Sound waves, P-waves in earthquakes & Light waves,
ripples on water surface, S-waves in earthquakes \\
\textbf{Medium requirement} & Can travel through solids, liquids and
gases & Can travel through solids and surfaces of liquids but not
through gases \\
\textbf{Components} & Compressions and rarefactions & Crests and
troughs \\
\textbf{Polarization} & Cannot be polarized & Can be polarized \\
\textbf{Visualization} & Like a spring or slinky compressed and expanded
& Like a rope being moved up and down \\
\end{longtable}

\textbf{Diagram:}

\begin{verbatim}
Longitudinal: -->-->-->-->-->--> (Direction of propagation)
              <--><--><--><-->   (Particle movement)
              
Transverse:   -->-->-->-->-->--> (Direction of propagation)
                ↑   ↓   ↑   ↓    (Particle movement)
\end{verbatim}

\end{solutionbox}
\begin{mnemonicbox}
``PPCP - Particles move Parallel in Longitudinal,
Perpendicular in Transverse, Compressions vs Crests, Polarization only
in Transverse''

\end{mnemonicbox}
\subsection*{Question 4(c)(1) [4 marks]}\label{q4c}

\textbf{Give three properties and uses of ultrasonic waves.}

\begin{solutionbox}

\textbf{Properties of ultrasonic waves:}

\begin{itemize}
\tightlist
\item
  Frequency ranges above 20,000 Hz (beyond human hearing)
\item
  Short wavelengths allow detection of small objects
\item
  High directivity compared to audible sound
\item
  High penetration in certain media
\item
  Less diffraction around obstacles
\item
  Cause cavitation in liquids
\end{itemize}

\textbf{Uses of ultrasonic waves:}

\begin{longtable}[]{@{}
  >{\raggedright\arraybackslash}p{(\linewidth - 2\tabcolsep) * \real{0.3333}}
  >{\raggedright\arraybackslash}p{(\linewidth - 2\tabcolsep) * \real{0.6667}}@{}}
\toprule\noalign{}
\begin{minipage}[b]{\linewidth}\raggedright
Field
\end{minipage} & \begin{minipage}[b]{\linewidth}\raggedright
Applications
\end{minipage} \\
\midrule\noalign{}
\endhead
\bottomrule\noalign{}
\endlastfoot
\textbf{Medical} & Sonography, kidney stone destruction,
physiotherapy \\
\textbf{Industrial} & Non-destructive testing, cleaning, welding,
drilling \\
\textbf{Navigation} & SONAR, distance measurement, obstacle detection \\
\textbf{Other} & Dog whistles, pest control, echolocation \\
\end{longtable}

\end{solutionbox}
\begin{mnemonicbox}
``FWD-MNO - Frequency high, Wavelength short,
Direction focused; Medical imaging, NDT testing, Ocean mapping''

\end{mnemonicbox}
\subsection*{Question 4(c)(2) [3 marks]}\label{q4c}

\textbf{Derive relation between velocity, wavelength and frequency.}

\begin{solutionbox}

\textbf{Derivation:}

Consider a wave traveling with:

\begin{itemize}
\tightlist
\item
  Wavelength (λ): Distance between consecutive similar points
\item
  Frequency (f): Number of waves passing a point per second
\item
  Time period (T): Time to complete one cycle
\end{itemize}

During one time period (T), the wave travels a distance equal to one
wavelength (λ).

Therefore, velocity = distance/time = λ/T

Since frequency f = 1/T, we can write:

v = λ \times f

Where:

\begin{itemize}
\tightlist
\item
  v = velocity of the wave (m/s)
\item
  λ = wavelength (m)
\item
  f = frequency (Hz)
\end{itemize}

\textbf{Diagram:}

\begin{verbatim}
    λ
<--------->
 ___       ___       ___
/   \     /   \     /   \
     \___/     \___/     
     
v = λ \times f
\end{verbatim}

\end{solutionbox}
\begin{mnemonicbox}
``VLF - Velocity equals Lambda times Frequency''

\end{mnemonicbox}
\subsection*{Question 4(a) OR [3
marks]}\label{q4a}

\textbf{Explain Sabine's formula for reverberation time.}

\begin{solutionbox}

Sabine's formula calculates the reverberation time in an enclosed space:

\textbf{Formula:} RT_{6}_{0} = 0.161 \times V/A

Where:

\begin{itemize}
\tightlist
\item
  RT_{6}_{0} = Reverberation time (seconds) for sound to decay by 60 dB
\item
  V = Volume of the room (m^{3})
\item
  A = Total sound absorption (m^{2} sabins)
\item
  0.161 = Constant (for calculation in metric units)
\end{itemize}

\textbf{Total absorption (A)} is calculated as: A = α_{1}S_{1} + α_{2}S_{2} + α_{3}S_{3} +
\ldots{} + α_{n}S_{n}

Where:

\begin{itemize}
\tightlist
\item
  αᵢ = Absorption coefficient of material i
\item
  Sᵢ = Surface area of material i (m^{2})
\end{itemize}

\textbf{Applications:}

\begin{itemize}
\tightlist
\item
  Acoustic design of concert halls, auditoriums, recording studios
\item
  Determination of required acoustic treatment
\item
  Evaluation of acoustic quality of existing spaces
\end{itemize}

\end{solutionbox}
\begin{mnemonicbox}
``VAS - Volume And Surface absorption determine
reverberation time''

\end{mnemonicbox}
\subsection*{Question 4(b) OR [4
marks]}\label{q4b}

\textbf{What is diffraction of light? Explain its types with diagram.}

\begin{solutionbox}

\textbf{Definition:} Diffraction is the bending of light waves around
obstacles or through openings, showing the wave nature of light.

\textbf{Types of diffraction:}

\textbf{1. Fresnel Diffraction:}

\begin{itemize}
\tightlist
\item
  Source or screen (or both) at finite distance from the obstacle
\item
  Spherical wavefronts
\item
  More complex interference pattern
\end{itemize}

\textbf{Diagram:}

\begin{verbatim}
Source                 Screen
  •                      ┃
   \     __________      ┃
    \   |          |     ┃
     \  |  Opening |     ┃
      \ |__________|     ┃
       \                 ┃
        \                ┃
         \               ┃
\end{verbatim}

\textbf{2. Fraunhofer Diffraction:}

\begin{itemize}
\tightlist
\item
  Source and screen at infinite distance (or effectively using lenses)
\item
  Plane wavefronts
\item
  Simpler interference pattern
\item
  More commonly studied in elementary physics
\end{itemize}

\textbf{Diagram:}

\begin{verbatim}
Plane                      Screen
waves   __________          ┃
\rightarrow\rightarrow\rightarrow\rightarrow\rightarrow\rightarrow\rightarrow|          |         ┃
\rightarrow\rightarrow\rightarrow\rightarrow\rightarrow\rightarrow\rightarrow|  Opening |\rightarrow\rightarrow\rightarrow\rightarrow\rightarrow\rightarrow\rightarrow\rightarrow\rightarrow┃
\rightarrow\rightarrow\rightarrow\rightarrow\rightarrow\rightarrow\rightarrow|__________|         ┃
                            ┃
\end{verbatim}

\end{solutionbox}
\begin{mnemonicbox}
``FPSS - Fresnel has Finite distances, Spherical
waves; Fraunhofer has Source at infinity, Straight (plane) waves''

\end{mnemonicbox}
\subsection*{Question 4(c)(1) OR [3
marks]}\label{q4c}

\textbf{Find the wavelength of a radio wave if the frequency is 480 Hz
and the speed of sound is 330 m/s.}

\begin{solutionbox}

\textbf{Given:}

\begin{itemize}
\tightlist
\item
  Frequency (f) = 480 Hz
\item
  Speed of sound (v) = 330 m/s
\end{itemize}

\textbf{To find:} Wavelength (λ)

\textbf{Formula:} v = λ \times f

\textbf{Calculation:} λ = v/f λ = 330 m/s \div 480 Hz λ = 0.6875 m λ =
68.75 cm

Therefore, the wavelength of the radio wave is 0.6875 m or 68.75 cm.

\end{solutionbox}
\begin{mnemonicbox}
``WFV - Wavelength equals Velocity divided by
Frequency''

\end{mnemonicbox}
\subsection*{Question 4(c)(2) OR [4
marks]}\label{q4c}

\textbf{Give properties of sound waves}

\begin{solutionbox}

\textbf{Properties of sound waves:}

\begin{longtable}[]{@{}
  >{\raggedright\arraybackslash}p{(\linewidth - 2\tabcolsep) * \real{0.4348}}
  >{\raggedright\arraybackslash}p{(\linewidth - 2\tabcolsep) * \real{0.5652}}@{}}
\toprule\noalign{}
\begin{minipage}[b]{\linewidth}\raggedright
Property
\end{minipage} & \begin{minipage}[b]{\linewidth}\raggedright
Description
\end{minipage} \\
\midrule\noalign{}
\endhead
\bottomrule\noalign{}
\endlastfoot
\textbf{Wave nature} & Sound is a mechanical, longitudinal wave
requiring a medium \\
\textbf{Frequency range} & Audible range for humans: 20 Hz to 20,000
Hz \\
\textbf{Speed} & \textasciitilde343 m/s in air at room temperature;
varies with medium \\
\textbf{Reflection} & Bounces off surfaces, creating echoes and
reverberation \\
\textbf{Refraction} & Changes direction when passing between media of
different densities \\
\textbf{Diffraction} & Bends around obstacles and through openings \\
\textbf{Interference} & Waves can superimpose to create constructive or
destructive interference \\
\textbf{Resonance} & Amplification at natural frequencies of objects \\
\end{longtable}

\textbf{Factors affecting speed of sound:}

\begin{itemize}
\tightlist
\item
  Increases with temperature in gases
\item
  Faster in liquids than gases
\item
  Fastest in solids
\item
  Independent of frequency and amplitude in a given medium
\end{itemize}

\end{solutionbox}
\begin{mnemonicbox}
``WARDS-FIR - Wave needs medium, Audible range
limited, Reflected, Diffracted, Speed varies, Frequency determines
pitch, Intensity determines loudness, Resonates at natural frequencies''

\end{mnemonicbox}
\subsection*{Question 5(a) [3 marks]}\label{q5a}

\textbf{State the meaning and properties of Laser.}

\begin{solutionbox}

\textbf{LASER}: Light Amplification by Stimulated Emission of Radiation

\textbf{Properties of laser light:}

\begin{itemize}
\tightlist
\item
  \textbf{Monochromatic}: Single wavelength or very narrow band of
  wavelengths
\item
  \textbf{Coherent}: All waves are in phase with each other
\item
  \textbf{Directional}: Low divergence, travels in a straight line with
  minimal spreading
\item
  \textbf{Intense}: High energy concentration in a small area
\item
  \textbf{Collimated}: Light rays are parallel with minimal divergence
\end{itemize}

\end{solutionbox}
\begin{mnemonicbox}
``MCCDI - Monochromatic and Coherent, Collimated,
Directional, Intense''

\end{mnemonicbox}
\subsection*{Question 5(b) [4 marks]}\label{q5b}

\textbf{Give information about optical fiber.}

\begin{solutionbox}

\textbf{Optical Fiber}: A flexible, transparent fiber made of glass or
plastic that transmits light signals through total internal reflection.

\textbf{Structure:}

\begin{verbatim}
       ┌───────────┐
       │           │
       │  Core     │  n_{1} (Higher refractive index)
       │           │
┌──────┴───────────┴──────┐
│                         │
│      Cladding           │  n_{2} (Lower refractive index)
│                         │
└─────────────────────────┘
       Protective coating
\end{verbatim}

\textbf{Components:}

\begin{itemize}
\tightlist
\item
  \textbf{Core}: Central region where light travels (higher refractive
  index)
\item
  \textbf{Cladding}: Outer optical material surrounding the core (lower
  refractive index)
\item
  \textbf{Buffer coating}: Protective outer covering
\end{itemize}

\textbf{Types:}

\begin{itemize}
\tightlist
\item
  \textbf{Single-mode}: Small core (8-10 μm), carries only one mode
\item
  \textbf{Multi-mode}: Larger core (50-100 μm), carries multiple modes

  \begin{itemize}
  \tightlist
  \item
    \textbf{Step-index}: Abrupt change in refractive index
  \item
    \textbf{Graded-index}: Gradual change in refractive index
  \end{itemize}
\end{itemize}

\textbf{Advantages:}

\begin{itemize}
\tightlist
\item
  High bandwidth and data transmission rates
\item
  Immune to electromagnetic interference
\item
  Low signal attenuation over long distances
\item
  Small size and lightweight
\item
  Enhanced security (difficult to tap)
\end{itemize}

\end{solutionbox}
\begin{mnemonicbox}
``CCTLT - Core Carries light, Cladding keeps it in,
Total internal reflection, Low loss transmission''

\end{mnemonicbox}
\subsection*{Question 5(c)(1) [7 marks]}\label{q5c}

\textbf{Explain Snell's law.}

\begin{solutionbox}

\textbf{Definition:} Snell's law (law of refraction) states that the
ratio of the sine of the angle of incidence to the sine of the angle of
refraction is constant for any two specific media.

\textbf{Formula:} n_{1}sin(θ_{1}) = n_{2}sin(θ_{2})

Where:

\begin{itemize}
\tightlist
\item
  n_{1} = Refractive index of medium 1
\item
  θ_{1} = Angle of incidence
\item
  n_{2} = Refractive index of medium 2
\item
  θ_{2} = Angle of refraction
\end{itemize}

\textbf{Diagram:}

\begin{verbatim}
              Normal
                │
                │
                │
Medium 1 (n_{1})   │       θ_{1}
                │      /
                │     /
                │    /
                │   /
                │  /
----------------│-/---------------- Boundary
                │/ θ_{2}
                /│
               / │
              /  │
             /   │
Medium 2 (n_{2})    │
                 │
\end{verbatim}

\textbf{Examples:}

\begin{itemize}
\tightlist
\item
  Light bending when entering water from air
\item
  Apparent displacement of objects underwater
\item
  Formation of rainbow
\item
  Design of lenses and prisms
\end{itemize}

\textbf{Special cases:}

\begin{itemize}
\tightlist
\item
  When light travels from less dense to more dense medium (n_{1}
  \textless{} n_{2}), it bends toward the normal (θ_{1} \textgreater{} θ_{2})
\item
  When light travels from more dense to less dense medium (n_{1}
  \textgreater{} n_{2}), it bends away from the normal (θ_{1} \textless{} θ_{2})
\item
  When angle of incidence equals 0^\circ (normal incidence), no refraction
  occurs
\end{itemize}

\end{solutionbox}
\begin{mnemonicbox}
``SINS - Sine of incidence over sine of refraction
equals N_{1} over N_{2}''

\end{mnemonicbox}
\subsection*{Question 5(c)(2) [0 marks]}\label{q5c}

\textbf{Explain the Acceptance angle.}

\begin{solutionbox}

\textbf{Acceptance angle} is the maximum angle at which light can enter
an optical fiber and still experience total internal reflection.

\textbf{Formula:} θ_{a} = sin^{-}^{1}(NA)

Where:

\begin{itemize}
\tightlist
\item
  θ_{a} = Acceptance angle
\item
  NA = Numerical aperture
\end{itemize}

\textbf{Numerical Aperture (NA):} NA = \sqrt(n_{1}^{2} - n_{2}^{2})

Where:

\begin{itemize}
\tightlist
\item
  n_{1} = Refractive index of the core
\item
  n_{2} = Refractive index of the cladding
\end{itemize}

\textbf{Diagram:}

\begin{verbatim}
              Acceptance cone
                   /\
                  /  \
                 /    \
                /      \
               /        \
              /    θ_{a}    \
             /____________\
            ┌──────────────┐
            │   Core       │
            │              │
            └──────────────┘
                Fiber
\end{verbatim}

\textbf{Significance:}

\begin{itemize}
\tightlist
\item
  Determines the light-gathering capacity of the fiber
\item
  Larger acceptance angle means more light can enter the fiber
\item
  Related to the fiber's information-carrying capacity
\item
  Critical for coupling efficiency with light sources
\end{itemize}

\end{solutionbox}
\begin{mnemonicbox}
``CAP - Core and cladding indices Affect the
acceptance angle which determines the Path light can take''

\end{mnemonicbox}
\subsection*{Question 5(a) OR [3
marks]}\label{q5a}

\textbf{Write the uses of Laser.}

\begin{solutionbox}

\textbf{Uses of Laser:}

\begin{longtable}[]{@{}
  >{\raggedright\arraybackslash}p{(\linewidth - 2\tabcolsep) * \real{0.3333}}
  >{\raggedright\arraybackslash}p{(\linewidth - 2\tabcolsep) * \real{0.6667}}@{}}
\toprule\noalign{}
\begin{minipage}[b]{\linewidth}\raggedright
Field
\end{minipage} & \begin{minipage}[b]{\linewidth}\raggedright
Applications
\end{minipage} \\
\midrule\noalign{}
\endhead
\bottomrule\noalign{}
\endlastfoot
\textbf{Medical} & Surgery, eye treatment, cancer therapy, dermatology,
dental procedures \\
\textbf{Industrial} & Cutting, welding, drilling, marking, material
processing, 3D printing \\
\textbf{Communications} & Fiber optic data transmission, free-space
optical communication \\
\textbf{Scientific} & Spectroscopy, holography, nuclear fusion, particle
acceleration \\
\textbf{Consumer} & Barcode scanners, DVD/Blu-ray players, laser
pointers, printers \\
\textbf{Military} & Range finding, target designation, guidance systems,
weapons \\
\end{longtable}

\end{solutionbox}
\begin{mnemonicbox}
``MICSM - Medical procedures, Industrial cutting,
Communication systems, Scientific research, Military applications''

\end{mnemonicbox}
\subsection*{Question 5(b) OR [4
marks]}\label{q5b}

\textbf{Write a short note on total internal reflection of light.}

\begin{solutionbox}

\textbf{Total Internal Reflection (TIR)} is an optical phenomenon that
occurs when light traveling in a denser medium hits the boundary with a
less dense medium at an angle greater than the critical angle.

\textbf{Conditions required for TIR:}

\begin{itemize}
\tightlist
\item
  Light must travel from a denser medium to a less dense medium (n_{1}
  \textgreater{} n_{2})
\item
  The angle of incidence must exceed the critical angle (θᵢ
  \textgreater{} θc)
\end{itemize}

\textbf{Critical angle formula:} θc = sin^{-}^{1}(n_{2}/n_{1})

\textbf{Diagram:}

\begin{verbatim}
                Normal
                  |
                  |
Denser medium     |      θᵢ < θc (Refraction)
(n_{1})              |     /
                  |    /
                  |   /    θᵣ
                  |  /      /
                  | /      /
------------------+/------/----------
                   \     /
Less dense medium   \   /
(n_{2})                 \ /
                      |
                      |
                      |
                      
                Normal
                  |
                  |
Denser medium     |    θᵢ = θc (Critical angle)
(n_{1})              |   /
                  |  /
                  | /
------------------+/---------------------
                   \
Less dense medium   \ 90^\circ
(n_{2})                 \
                      |
                      |
                      
                Normal
                  |
                  |
Denser medium     |    θᵢ > θc (Total Internal Reflection)
(n_{1})              |   /
                  |  /
                  | /       /
------------------+/-------/------------
                   \       /
Less dense medium   \     /
(n_{2})                 \   /
                      \ /
                       |
\end{verbatim}

\textbf{Applications:}

\begin{itemize}
\tightlist
\item
  Optical fibers for communication
\item
  Prisms and binoculars
\item
  Diamond brilliance
\item
  Mirage formation
\item
  Endoscopes for medical imaging
\end{itemize}

\end{solutionbox}
\begin{mnemonicbox}
``CANDO - Critical Angle needed, n_{1} must be Denser
than n_{2}, Only works when angle is greater than critical, Angle
determines reflection vs refraction''

\end{mnemonicbox}
\subsection*{Question 5(c)(1) OR [3
marks]}\label{q5c}

\textbf{If the speed of light in water is 2.25\times10^{8} m/s and the speed of
light in air is 3\times10^{8} m/s, find the refractive index of water.}

\begin{solutionbox}

\textbf{Given:}

\begin{itemize}
\tightlist
\item
  Speed of light in water (vw) = 2.25\times10^{8} m/s
\item
  Speed of light in air (va) = 3\times10^{8} m/s
\end{itemize}

\textbf{To find:} Refractive index of water (nw)

\textbf{Formula:} n = c/v

\textbf{For calculation of refractive index of water relative to air:}
nw = va/vw

\textbf{Calculation:} nw = 3\times10^{8} m/s \div 2.25\times10^{8} m/s nw = 3 \div 2.25 nw =
1.33

Therefore, the refractive index of water is 1.33.

\end{solutionbox}
\begin{mnemonicbox}
``SVN - Speed of light in Vacuum divided by Speed in
medium gives refractive iNdex''

\end{mnemonicbox}
\subsection*{Question 5(c)(2) OR [4
marks]}\label{q5c}

\textbf{Write a note on step index fiber.}

\begin{solutionbox}

\textbf{Step Index Fiber:} A type of optical fiber where the refractive
index changes abruptly between the core and cladding.

\textbf{Structure:}

\textbf{Diagram:}

\begin{verbatim}
    ┌───────────────────────┐
    │                       │ n_{1}
    │        Core           │
    │                       │
    └───────────────────────┘
    ┌───────────────────────┐
    │                       │ n_{2}
    │       Cladding        │
    │                       │
    └───────────────────────┘

Refractive Index Profile:
    n_{1} ────────┐
               │
               │
    n_{2}         └────────
        Core     Cladding
\end{verbatim}

\textbf{Characteristics:}

\begin{itemize}
\tightlist
\item
  Abrupt change in refractive index at core-cladding boundary
\item
  Available in both single-mode and multi-mode configurations
\item
  Simpler construction than graded-index fiber
\item
  More modal dispersion in multi-mode configuration
\end{itemize}

\textbf{Types:}

\begin{itemize}
\tightlist
\item
  \textbf{Single-mode step index fiber}:

  \begin{itemize}
  \tightlist
  \item
    Very small core diameter (8-10 μm)
  \item
    Only allows one mode of light propagation
  \item
    Low signal distortion
  \item
    Used for long-distance communication
  \end{itemize}
\item
  \textbf{Multi-mode step index fiber}:

  \begin{itemize}
  \tightlist
  \item
    Larger core diameter (50-100 μm)
  \item
    Allows multiple light paths
  \item
    Higher modal dispersion
  \item
    Suitable for shorter distances
  \end{itemize}
\end{itemize}

\textbf{Advantages:}

\begin{itemize}
\tightlist
\item
  Simpler and cheaper manufacturing
\item
  Good for short-distance applications
\item
  Easier to couple light into multi-mode versions
\item
  Less sensitive to bending losses than single-mode fibers
\end{itemize}

\textbf{Limitations:}

\begin{itemize}
\tightlist
\item
  Higher modal dispersion in multi-mode configuration
\item
  Bandwidth limitations due to different path lengths
\item
  Not ideal for high-speed, long-distance transmission
\end{itemize}

\end{solutionbox}
\begin{mnemonicbox}
``SACS - Step change at boundary, Abrupt index
profile, Core guides light, Simple construction''

\end{mnemonicbox}

\end{document}
