\documentclass[10pt,a4paper]{article}

% content/resources/templates/preamble.tex
\usepackage[margin=0.6in]{geometry}
\author{Milav Dabgar}
\usepackage{amsmath,amssymb,amsthm}
\usepackage{booktabs}
\usepackage{multirow}
\usepackage{xcolor}
\usepackage{tcolorbox}
\tcbuselibrary{breakable,skins}
\usepackage[colorlinks=true,linkcolor=blue]{hyperref}
\usepackage{titlesec}
\usepackage{enumitem}
\usepackage{tikz}
\usepackage{pgfplots}
\usepackage{circuitikz}
\usepackage[version=4]{mhchem}
\usepackage{longtable}
\usepackage{array}
\usepackage{float}
\usepackage{caption}
\usepackage{listings}

\lstset{
  basicstyle=\small\ttfamily,
  breaklines=true,
  breakatwhitespace=false,
  postbreak=\mbox{\textcolor{red}{$\hookrightarrow$}\space},
  float=false,
  numbers=left,
  numberstyle=\tiny\color{gray},
  numbersep=10pt,
  xleftmargin=2em,
  keywordstyle=\color{blue},
  commentstyle=\color{green!60!black},
  stringstyle=\color{purple},
  backgroundcolor=\color{gray!5},
  showstringspaces=false,
  tabsize=2,
  captionpos=b,
  keepspaces=true,
  columns=flexible
}

\pgfplotsset{compat=1.18}
\usetikzlibrary{shapes,arrows,positioning,calc,patterns,decorations.pathmorphing,decorations.markings,arrows.meta}

% Color scheme
\definecolor{headcolor}{RGB}{0,102,204}
\definecolor{keycolor}{RGB}{220,20,60}
\definecolor{solutioncolor}{RGB}{34,139,34}
\definecolor{mnemoniccolor}{RGB}{148,0,211}
\definecolor{codecolor}{RGB}{0,0,100}

% Spacing
\setlength{\parskip}{3pt}
\setlist[itemize]{nosep}
\setlist[enumerate]{nosep}

% Title formatting
\titleformat{\section}{\Large\bfseries\color{headcolor}}{\thesection}{1em}{}
\titleformat{\subsection}{\large\bfseries\color{headcolor}}{\thesubsection}{1em}{}

% Pandoc tightlist compatibility
\providecommand{\tightlist}{%
  \setlength{\itemsep}{0pt}\setlength{\parskip}{0pt}}

% Pandoc longtable compatibility
\newcounter{none}
\def\thenone{}


% content/resources/templates/gujarati-boxes.tex
\usepackage{fontspec}
\usepackage{polyglossia}

% Set Gujarati as main language (document is primarily in Gujarati)
% Note: gloss-gujarati.ldf doesn't exist in polyglossia, but it will use hyphenation patterns
\setdefaultlanguage{gujarati}
\setotherlanguage{english}

% Configure Gujarati font properly
% Use Language=Default to prevent polyglossia from trying to add language-specific features
% that don't exist for Gujarati, which causes "empty feature" warnings
\newfontfamily\gujaratifont[Script=Gujarati,AutoFakeBold=2.5,AutoFakeSlant=0.3]{Noto Sans Gujarati}
\setmainfont[Script=Gujarati,AutoFakeBold=2.5,AutoFakeSlant=0.3]{Noto Sans Gujarati}
% Use Noto Sans Gujarati for monospace to support Gujarati in text
\setmonofont[Scale=0.9]{Noto Sans Gujarati}

% Configure English to use the same font
\newfontfamily\englishfont[Script=Gujarati,AutoFakeBold=2.5,AutoFakeSlant=0.3]{Noto Sans Gujarati}

% Translations for polyglossia
\gappto\captionsgujarati{
  \renewcommand{\tablename}{કોષ્ટક}
  \renewcommand{\figurename}{આકૃતિ}
}

% Helper for TikZ nodes to ensure Gujarati font
\newcommand{\gu}[1]{{\gujaratifont #1}}

% Custom environments
\newtcolorbox{solutionbox}{
    breakable,
    enhanced,
    colback=solutioncolor!5!white,
    colframe=solutioncolor!75!black,
    fonttitle=\bfseries,
    title=જવાબ
}

\newtcolorbox{solutionboxnobreak}{
 colback=solutioncolor!5!white,
 colframe=solutioncolor!75!black,
 fonttitle=\bfseries,
 title=જવાબ
}

\newtcolorbox{keyformula}{
 breakable,
 enhanced,
 colback=keycolor!5!white,
 colframe=keycolor!75!black,
 fonttitle=\bfseries,
 title=રાસાયણિક સમીકરણ/સૂત્ર
}

\newtcolorbox{mnemonicbox}{
 breakable,
 enhanced,
 colback=mnemoniccolor!5!white,
 colframe=mnemoniccolor!75!black,
 fonttitle=\bfseries,
 title=મેમરી ટ્રીક
}


\begin{document}

\begin{center}
{\Huge\bfseries\color{headcolor} Subject Name (Gujarati)}\\[5pt]
{\LARGE 4300005 -- Summer 2024}\\[3pt]
{\large Semester 1 Study Material}\\[3pt]
{\normalsize\textit{Detailed Solutions and Explanations}}
\end{center}

\vspace{10pt}

\subsection*{પ્રશ્ન 1(a) {[}3
ગુણ{]}}\label{uxaaauxab0uxab6uxaa8-1a-3-uxa97uxaa3}

\textbf{સાધિત ભૌતિક રાશીની વ્યાખ્યા લખો અને તેના કોઈ પણ ત્રણ ઉદાહરણોને એકમ અને
ચિન્હ સાથે લખો.}

\begin{solutionbox}
સાધિત ભૌતિક રાશીઓ એ છે જે મૂળભૂત ભૌતિક રાશીઓના ગુણાકાર અથવા
ભાગાકાર દ્વારા મેળવવામાં આવે છે.


{\def\LTcaptype{none} % do not increment counter
\vspace{-5pt}
\captionof{table}{સાધિત ભૌતિક રાશીઓના ઉદાહરણો}
\vspace{-10pt}
\begin{longtable}[]{@{}lll@{}}
\toprule\noalign{}
સાધિત રાશી & S.I. એકમ & ચિહ્ન \\
\midrule\noalign{}
\endhead
\bottomrule\noalign{}
\endlastfoot
બળ & ન્યૂટન (N) & F \\
ઊર્જા & જૂલ (J) & E \\
વિદ્યુત પ્રવાહ & એમ્પિયર (A) & I \\
\end{longtable}
}

\end{solutionbox}
\begin{mnemonicbox}
``FEI: બળ-ઊર્જા-વિદ્યુત પ્રવાહ મૂળભૂતમાંથી નિકળે છે''

\end{mnemonicbox}
\subsection*{પ્રશ્ન 1(b) {[}4
ગુણ{]}}\label{uxaaauxab0uxab6uxaa8-1b-4-uxa97uxaa3}

\textbf{ધાતુના સળિયાની લંબાઈ 12°C તાપમાને 64.522 cm છે અને 90°C તાપમાને 64.576
cm છે. તો સળિયાના રેખીય વિસ્તરણ ગુણાંક શોધો.}

\begin{solutionbox}
\textbf{સૂત્ર:} α = (L₂ - L₁)/{[}L₁ × (T₂ - T₁){]}

\textbf{ગણતરી:}

\begin{itemize}
\tightlist
\item
  પ્રારંભિક લંબાઈ (L₁) = 64.522 cm
\item
  અંતિમ લંબાઈ (L₂) = 64.576 cm
\item
  પ્રારંભિક તાપમાન (T₁) = 12°C
\item
  અંતિમ તાપમાન (T₂) = 90°C
\end{itemize}

α = (64.576 - 64.522)/{[}64.522 × (90 - 12){]} α = 0.054/(64.522 × 78) α
= 0.054/5032.716 α = 1.073 × 10⁻⁵ /°C

\end{solutionbox}
\begin{mnemonicbox}
``લંબાઈમાં ફેરફાર પર મૂળ લંબાઈ અને તાપમાન ફેરફારનો ભાગ''

\end{mnemonicbox}
\subsection*{પ્રશ્ન 1(c) {[}7
ગુણ{]}}\label{uxaaauxab0uxab6uxaa8-1c-7-uxa97uxaa3}

\textbf{વર્નિયર કેલિપર્સનો સિદ્ધાંત, રચના અને કાર્ય પદ્ધતિ તેની આકૃતિ સાથે સમજાવો.}

\begin{solutionbox}
\textbf{સિદ્ધાંત}: વર્નિયર કેલિપર વર્નિયર સ્કેલના સિદ્ધાંત પર કામ
કરે છે, જે મુખ્ય સ્કેલ કરતાં વધુ ચોકસાઈથી માપન કરવા દે છે.

\textbf{રચના:}

\begin{center}
\textbf{Mermaid Diagram (Code)}
\begin{verbatim}
{Shaded}
{Highlighting}[]
graph TD
    A[વર્નિયર કેલિપર] {-{-}{} B[મુખ્ય સ્કેલ]}
    A {-{-}{} C[વર્નિયર સ્કેલ]}
    A {-{-}{} D[સ્થિર જડબું]}
    A {-{-}{} E[ચલિત જડબું]}
    A {-{-}{} F[ઊંડાઈ માપક]}
    A {-{-}{} G[લોકિંગ સ્ક્રૂ]}
{Highlighting}
{Shaded}
\end{verbatim}
\end{center}

\textbf{કાર્યપદ્ધતિ:}

\begin{itemize}
\tightlist
\item
  \textbf{શૂન્ય ત્રુટિની તપાસ}: જડબાંઓ બંધ કરી વર્નિયરનો શૂન્ય મુખ્ય સ્કેલના શૂન્ય સાથે
  મેળ ખાય છે કે કેમ તે જોવું
\item
  \textbf{બહારનું માપન}: વસ્તુને સ્થિર અને ચલિત જડબાં વચ્ચે મૂકો
\item
  \textbf{વાંચન પ્રક્રિયા}: મુખ્ય સ્કેલ વાંચન + (મેળ ખાતા વર્નિયર વિભાગ × લઘુત્તમ
  માપ)
\item
  \textbf{લઘુત્તમ માપ} = (મુખ્ય સ્કેલનો સૌથી નાનો વિભાગ)/(વર્નિયર સ્કેલના
  વિભાગોની સંખ્યા)
\end{itemize}

\textbf{આકૃતિ:}

\begin{verbatim}
                 ┌───────┐            
 મુખ્ય સ્કેલ ───▶│       │◀── વર્નિયર સ્કેલ
  (મીમીમાં)      │   ┌───┴───┐     
     0    5    10   15      20    25    30
     |....|....|....|....|....|....|....|
     |    0    5    |0    5    |    
     └────┬─────────┘           
          │                  
      સ્થિર જડબું        ચલિત જડબું
\end{verbatim}

\end{solutionbox}
\begin{mnemonicbox}
``મુખ્ય સ્કેલ વાંચન વત્તા વર્નિયર ભાગ ગુણિયે લઘુત્તમ માપ''

\end{mnemonicbox}
\subsection*{પ્રશ્ન 1(c) OR {[}7
ગુણ{]}}\label{uxaaauxab0uxab6uxaa8-1c-or-7-uxa97uxaa3}

\textbf{માઇક્રોમિટર સ્ક્રૂ ગેજનો સિદ્ધાંત, રચના અને કાર્ય પદ્ધતિ તેની આકૃતિ સાથે
સમજાવો.}

\begin{solutionbox}
\textbf{સિદ્ધાંત}: માઇક્રોમિટર સ્ક્રૂ ગેજ સ્ક્રૂની ગતિના સિદ્ધાંત પર
કામ કરે છે - ફરતી ગતિને સીધી રેખાની ગતિમાં પરિવર્તિત કરવામાં આવે છે.

\textbf{રચના:}

\begin{center}
\textbf{Mermaid Diagram (Code)}
\begin{verbatim}
{Shaded}
{Highlighting}[]
graph TD
    A[માઇક્રોમિટર સ્ક્રૂ ગેજ] {-{-}{} B[ફ્રેમ]}
    A {-{-}{} C[એનવિલ]}
    A {-{-}{} D[સ્પિન્ડલ]}
    A {-{-}{} E[સ્લીવ/મુખ્ય સ્કેલ]}
    A {-{-}{} F[થિમ્બલ/ગોળાકાર સ્કેલ]}
    A {-{-}{} G[રેચેટ]}
    A {-{-}{} H[લોક નટ]}
{Highlighting}
{Shaded}
\end{verbatim}
\end{center}

\textbf{કાર્યપદ્ધતિ:}

\begin{itemize}
\tightlist
\item
  \textbf{શૂન્ય ત્રુટિની તપાસ}: એનવિલ અને સ્પિન્ડલ બંધ કરી, ગોળાકાર સ્કેલનો શૂન્ય
  સંદર્ભ રેખા સાથે ગોઠવાય છે કે કેમ તપાસો
\item
  \textbf{માપન પ્રક્રિયા}: વસ્તુને એનવિલ અને સ્પિન્ડલ વચ્ચે મૂકો
\item
  \textbf{વાંચન}: મુખ્ય સ્કેલ વાંચન + (ગોળાકાર સ્કેલ વાંચન × લઘુત્તમ માપ)
\item
  \textbf{લઘુત્તમ માપ} = પીચ/ગોળાકાર સ્કેલના વિભાગોની સંખ્યા
\end{itemize}

\textbf{આકૃતિ:}

\begin{verbatim}
                     રેચેટ
                        ▲
                        │
        ફ્રેમ            │        થિમ્બલ/ગોળાકાર સ્કેલ
          ┌─────────────┴─────┐  ┌───┐
          │                   │  │   │
એનવિલ ──▶ O═══════════════════O══O   │
          │     │             │  │   │
          └─────┼─────────────┘  └───┘
                │                   │
                ▼                   ▼
            સ્પિન્ડલ             સ્લીવ/મુખ્ય સ્કેલ
\end{verbatim}

\end{solutionbox}
\begin{mnemonicbox}
``PST: પીચને સ્કેલથી ભાગીએ તો થિમ્બલનો લઘુત્તમ માપ મળે''

\end{mnemonicbox}
\subsection*{પ્રશ્ન 2(a) {[}3
ગુણ{]}}\label{uxaaauxab0uxab6uxaa8-2a-3-uxa97uxaa3}

\textbf{જો માઇક્રોમિટર સ્ક્રૂ ગેજની પિચ 1 mm હોય અને ગોળાકાર સ્કેલના કુલ 100
વિભાગ હોય તો ગોળાનો વ્યાસ શોધો. ગોળાકાર સ્કેલની ધાર મુખ્ય સ્કેલના 7 અને 8 mm વચ્ચે
આવે છે અને ગોળાકાર સ્કેલના 65મો વિભાગ મુખ્ય સ્કેલની આડી રેખા સાથે મળે છે.}

\begin{solutionbox}
\textbf{સૂત્ર:} વ્યાસ = મુખ્ય સ્કેલ વાંચન + (ગોળાકાર સ્કેલ વાંચન ×
લઘુત્તમ માપ)

\textbf{ગણતરી:}

\begin{itemize}
\tightlist
\item
  મુખ્ય સ્કેલ વાંચન = 7 mm
\item
  ગોળાકાર સ્કેલ વાંચન = 65 વિભાગ
\item
  લઘુત્તમ માપ = પીચ/વિભાગોની સંખ્યા = 1/100 = 0.01 mm
\end{itemize}

વ્યાસ = 7 + (65 × 0.01) = 7 + 0.65 = 7.65 mm

\end{solutionbox}
\begin{mnemonicbox}
``MSR + (CSR × LC) આપે છે અંતિમ માપણી''

\end{mnemonicbox}
\subsection*{પ્રશ્ન 2(b) {[}4
ગુણ{]}}\label{uxaaauxab0uxab6uxaa8-2b-4-uxa97uxaa3}

\textbf{કળા તફાવત અને સુસબદ્ધતા ને સમજાવો.}

\begin{solutionbox}
\textbf{કળા તફાવત:} સમાન આવૃત્તિના બે તરંગો વચ્ચે કળા કોણનો
તફાવત.


{\def\LTcaptype{none} % do not increment counter
\vspace{-5pt}
\captionof{table}{કળા તફાવતની લાક્ષણિકતાઓ}
\vspace{-10pt}
\begin{longtable}[]{@{}lll@{}}
\toprule\noalign{}
કળા તફાવત & વ્યતિકરણનો પ્રકાર & પરિણામ \\
\midrule\noalign{}
\endhead
\bottomrule\noalign{}
\endlastfoot
0° અથવા 360° & રચનાત્મક & મહત્તમ કંપવિસ્તાર \\
180° & વિનાશક & લઘુત્તમ કંપવિસ્તાર \\
\end{longtable}
}

\textbf{સુસબદ્ધતા:} તરંગોની એવી ગુણવત્તા જેમાં કળા સંબંધ સતત રહે છે.

\textbf{સુસબદ્ધતાના પ્રકારો:}

\begin{itemize}
\tightlist
\item
  \textbf{સમયગત સુસબદ્ધતા}: આવૃત્તિ સ્થિરતા સાથે સંબંધિત
\item
  \textbf{અવકાશી સુસબદ્ધતા}: તરંગાગ્ર એકરૂપતા સાથે સંબંધિત
\end{itemize}

\end{solutionbox}
\begin{mnemonicbox}
``સતત કળા સંબંધ બનાવે સુસબદ્ધ તરંગો''

\end{mnemonicbox}
\subsection*{પ્રશ્ન 2(c) {[}7
ગુણ{]}}\label{uxaaauxab0uxab6uxaa8-2c-7-uxa97uxaa3}

\textbf{કેપેસિટર, કેપેસીટન્સ તથા સમાંતર પ્લેટ કેપેસિટરના કેપેસીટન્સ પર ડાઇલેટ્રિક
મધ્યમની અસર સમજાવો.}

\begin{solutionbox}
\textbf{કેપેસિટર}: એવું ઉપકરણ જે વિદ્યુત ક્ષેત્રમાં વિદ્યુત ચાર્જ અને
વિદ્યુત ઊર્જાને સંગ્રહિત કરે છે.

\textbf{કેપેસીટન્સ}: સંગ્રહિત ચાર્જનો લાગુ પોટેન્શિયલ તફાવત સાથેનો ગુણોત્તર.

\textbf{સૂત્ર:} C = Q/V

\textbf{સમાંતર પ્લેટ કેપેસિટર:} કેપેસીટન્સ સૂત્ર: C = ε₀A/d

\begin{itemize}
\tightlist
\item
  ε₀ = મુક્ત અવકાશની પરાવૈદ્યુતાંક
\item
  A = પ્લેટનું ક્ષેત્રફળ
\item
  d = પ્લેટ વચ્ચેનું અંતર
\end{itemize}

\textbf{ડાઇલેક્ટ્રિકની અસર:}

\begin{itemize}
\tightlist
\item
  કેપેસીટન્સને K ગણો વધારે છે (K = ડાઇલેક્ટ્રિક અચળાંક)
\item
  નવું સૂત્ર: C = Kε₀A/d
\end{itemize}

\textbf{આકૃતિ:}

\begin{verbatim}
    ┌───────────────┐  │
    │      ++++     │  │
    │      ++++     │  │ d
    │      ++++     │  │
    └───────────────┘  │
           │           
           │          
           V          
    ┌───────────────┐
    │      {-{-}{-}{-}     │}
    │      {-{-}{-}{-}     │ ◄── ડાઇલેક્ટ્રિક}
    │      {-{-}{-}{-}     │}
    └───────────────┘
           │
           │
    ક્ષેત્રફળ = A
\end{verbatim}

\end{solutionbox}
\begin{mnemonicbox}
``KIDS: K વધારે ડાઇલેક્ટ્રિક સંગ્રહ''

\end{mnemonicbox}
\subsection*{પ્રશ્ન 2(a) OR {[}3
ગુણ{]}}\label{uxaaauxab0uxab6uxaa8-2a-or-3-uxa97uxaa3}

\textbf{જો કોઈ બે નળાકારની લંબાઈ (6.52±0.01) cm અને (4.48±0.02) cm છે. તો
તેમની લંબાઈના તફાવત ની પ્રતિશત ત્રુટિ મેળવો.}

\begin{solutionbox}
\textbf{ગણતરી:}

\begin{itemize}
\tightlist
\item
  પ્રથમ નળાકારની લંબાઈ (L₁) = 6.52 ± 0.01 cm
\item
  બીજા નળાકારની લંબાઈ (L₂) = 4.48 ± 0.02 cm
\item
  લંબાઈનો તફાવત (ΔL) = L₁ - L₂ = 6.52 - 4.48 = 2.04 cm
\end{itemize}

\textbf{તફાવતમાં નિરપેક્ષ ત્રુટિ} = √{[}(0.01)² + (0.02)²{]} = √(0.0001 +
0.0004) = √0.0005 = 0.022 cm

\textbf{પ્રતિશત ત્રુટિ} = (નિરપેક્ષ ત્રુટિ/માપેલી કિંમત) × 100 = (0.022/2.04) ×
100 = 1.08\%

\end{solutionbox}
\begin{mnemonicbox}
``તફાવતની ગણતરી માટે ત્રુટિઓને વર્ગમાં ઉમેરો''

\end{mnemonicbox}
\subsection*{પ્રશ્ન 2(b) OR {[}4
ગુણ{]}}\label{uxaaauxab0uxab6uxaa8-2b-or-4-uxa97uxaa3}

\textbf{જરૂરી આકૃતિ સાથે વ્યતિકરણના પ્રકાર સમજાવો.}

\begin{solutionbox}
\textbf{વ્યતિકરણના પ્રકારો:}


{\def\LTcaptype{none} % do not increment counter
\vspace{-5pt}
\captionof{table}{વ્યતિકરણ પ્રકારો}
\vspace{-10pt}
\begin{longtable}[]{@{}llll@{}}
\toprule\noalign{}
પ્રકાર & કળા તફાવત & પરિણામ & તરંગ કંપવિસ્તાર \\
\midrule\noalign{}
\endhead
\bottomrule\noalign{}
\endlastfoot
રચનાત્મક & 0°, 360°, 720°\ldots{} & પ્રબલીકરણ & મહત્તમ \\
વિનાશક & 180°, 540°, 900°\ldots{} & રદ્દીકરણ & ન્યૂનતમ \\
\end{longtable}
}

\textbf{રચનાત્મક વ્યતિકરણ:} જ્યારે શિખર શિખરને મળે અથવા ખીણ ખીણને મળે ત્યારે.

\textbf{વિનાશક વ્યતિકરણ:} જ્યારે શિખર ખીણને મળે ત્યારે.

\textbf{આકૃતિ:}

\begin{verbatim}
રચનાત્મક વ્યતિકરણ:
    ⟋⟍⟋⟍⟋⟍     તરંગ 1
    ⟋⟍⟋⟍⟋⟍     તરંગ 2
    ⟋⟍⟋⟍⟋⟍
     ⬍⬍⬍⬍      પરિણામ: મોટો કંપવિસ્તાર
     
વિનાશક વ્યતિકરણ:
    ⟋⟍⟋⟍⟋⟍     તરંગ 1
    ⟍⟋⟍⟋⟍⟋     તરંગ 2
    {-{-}{-}{-}{-}{-}{-}{-}{-} પરિણામ: સીધી રેખા (રદ્દીકરણ)}
\end{verbatim}

\end{solutionbox}
\begin{mnemonicbox}
``શિખર + શિખર = રચનાત્મક, શિખર + ખીણ = વિનાશક''

\end{mnemonicbox}
\subsection*{પ્રશ્ન 2(c) OR {[}7
ગુણ{]}}\label{uxaaauxab0uxab6uxaa8-2c-or-7-uxa97uxaa3}

\textbf{બિંદુવત્ વિદ્યુતભારને કારણે વિદ્યુતસ્થિતિમાન માટેનું સમીકરણ તેની આકૃતિ સાથે
તારવો.}

\begin{solutionbox}
\textbf{બિંદુ ચાર્જને કારણે પોટેન્શિયલ:}

\textbf{સૂત્ર વિકાસ:}

\begin{itemize}
\tightlist
\item
  \textbf{વ્યાખ્યા}: એક પરીક્ષણ ચાર્જને અનંતથી તે બિંદુ સુધી લાવવા માટે એકમ ચાર્જ
  દીઠ કરેલું કાર્ય
\item
  \textbf{સમીકરણ}: V = W/q₀ = ∫(F·dr)
\end{itemize}

\textbf{પગલે પગલે તારણ:}

\begin{enumerate}
\def\labelenumi{\arabic{enumi}.}
\tightlist
\item
  ચાર્જો વચ્ચેનું બળ (કુલોમ્બનો નિયમ): F = (1/4πε₀) × (Qq/r²)
\item
  પરીક્ષણ ચાર્જ ખસેડવામાં કરેલું કાર્ય: W = ∫(F·dr)
\item
  ત્રિજ્યા ગતિ માટે: W = (Q/4πε₀) × ∫(1/r²)dr, ∞ થી r સુધી
\item
  સંકલન: W = (Q/4πε₀) × {[}-1/r{]}ᵣ∞
\item
  અંતિમ પરિણામ: V = W/q₀ = (1/4πε₀) × (Q/r)
\end{enumerate}

\textbf{અંતિમ સૂત્ર:} V = (1/4πε₀) × (Q/r)

\textbf{આકૃતિ:}

\begin{verbatim}
              P (પોટેન્શિયલ
                 જ્યાં ગણવાનું છે)
              *
              │
              │
              │r
              │
              │
        Q     │
        ●─────┘
   મૂળ બિંદુ પર ચાર્જ
\end{verbatim}

\end{solutionbox}
\begin{mnemonicbox}
``POD: Potential Over Distance અંતર પર પોટેન્શિયલ''

\end{mnemonicbox}
\subsection*{પ્રશ્ન 3(a) {[}3
ગુણ{]}}\label{uxaaauxab0uxab6uxaa8-3a-3-uxa97uxaa3}

\textbf{ઘર્ષણ અને ઇન્ડક્શન દ્વારા થતાં ચાર્જિંગ ને ટૂંકમાં સમજાવો.}

\begin{solutionbox}
\textbf{ઘર્ષણ દ્વારા ચાર્જિંગ:} બે અલગ પદાર્થોને એકબીજા સાથે
ઘસવાની પ્રક્રિયા.

\textbf{ઘર્ષણ ચાર્જિંગના પગલાં:}

\begin{itemize}
\tightlist
\item
  ઇલેક્ટ્રોન એક પદાર્થથી બીજા પદાર્થમાં સ્થાનાંતરિત થાય છે
\item
  ઇલેક્ટ્રોન ગુમાવતો પદાર્થ ધન ચાર્જિત થાય છે
\item
  ઇલેક્ટ્રોન મેળવતો પદાર્થ ઋણ ચાર્જિત થાય છે
\end{itemize}

\textbf{ઇન્ડક્શન દ્વારા ચાર્જિંગ:} સીધા સંપર્ક વિના ચાર્જિંગની પ્રક્રિયા.

\textbf{ઇન્ડક્શન ચાર્જિંગના પગલાં:}

\begin{itemize}
\tightlist
\item
  ચાર્જિત પદાર્થને તટસ્થ વાહક નજીક લાવો
\item
  તટસ્થ વાહકમાં ચાર્જનું પુનઃવિતરણ
\item
  વાહકને ગ્રાઉન્ડ કરી ગ્રાઉન્ડ દૂર કરો
\item
  ચાર્જિત પદાર્થને દૂર કરો
\end{itemize}

\end{solutionbox}
\begin{mnemonicbox}
``FTEE: ઘર્ષણ થી ઇલેક્ટ્રોન સરળતાથી ફેરવાય''

\end{mnemonicbox}
\subsection*{પ્રશ્ન 3(b) {[}4
ગુણ{]}}\label{uxaaauxab0uxab6uxaa8-3b-4-uxa97uxaa3}

\textbf{એક ટ્યુનીંગ ફોર્ક જેની આવૃત્તિ 256 Hz છે અને ગતિ 340 m/s છે. તેની (a)
તરંગલંબાઈ અને (b) 50 કંપનમાં કાપેલું અંતર શોધો.}

\begin{solutionbox}
\textbf{સૂત્રો:}

\begin{itemize}
\tightlist
\item
  તરંગલંબાઈ (λ) = ગતિ (v) / આવૃત્તિ (f)
\item
  અંતર (d) = કંપનોની સંખ્યા (n) × તરંગલંબાઈ (λ)
\end{itemize}

\textbf{ગણતરી:} (a) તરંગલંબાઈ (λ) = v/f = 340/256 = 1.328 m

\begin{enumerate}
\def\labelenumi{(\alph{enumi})}
\setcounter{enumi}{1}
\tightlist
\item
  અંતર (d) = n × λ = 50 × 1.328 = 66.4 m
\end{enumerate}

\end{solutionbox}
\begin{mnemonicbox}
``VFD: ગતિ, આવૃત્તિ અને અંતર એકબીજા સાથે જોડાયેલા છે''

\end{mnemonicbox}
\subsection*{પ્રશ્ન 3(c) {[}7
ગુણ{]}}\label{uxaaauxab0uxab6uxaa8-3c-7-uxa97uxaa3}

\textbf{બાયમેટાલીક થર્મોમિટરનો સિદ્ધાંત અને રચના ને આકૃતિ સાથે સમજાવો. તેના ફયદા
તથા ગેરફયદા લખો.}

\begin{solutionbox}
\textbf{સિદ્ધાંત}: જુદી જુદી ધાતુઓ ગરમ થવા પર અલગ અલગ પ્રમાણમાં
પ્રસરે છે, જેના કારણે પટ્ટી વળે છે.

\textbf{રચના:}

\begin{center}
\textbf{Mermaid Diagram (Code)}
\begin{verbatim}
{Shaded}
{Highlighting}[]
graph TD
    A[બાયમેટાલીક થર્મોમિટર] {-{-}{} B[સ્થિર છેડો]}
    A {-{-}{} C[બાયમેટાલીક પટ્ટી]}
    A {-{-}{} D[સૂચક]}
    A {-{-}{} E[સ્કેલ]}
    A {-{-}{} F[સુરક્ષાત્મક કેસ]}
    C {-{-}{} G[ઉચ્ચ પ્રસરણ ધરાવતી ધાતુ]}
    C {-{-}{} H[ઓછું પ્રસરણ ધરાવતી ધાતુ]}
{Highlighting}
{Shaded}
\end{verbatim}
\end{center}

\textbf{કાર્યપદ્ધતિ:}

\begin{itemize}
\tightlist
\item
  તાપમાન બદલાવાથી અલગ-અલગ પ્રસરણ દર થાય છે
\item
  બાયમેટાલિક પટ્ટી ઓછા પ્રસરણ ગુણાંક વાળી ધાતુ તરફ વળે છે
\item
  સૂચકની ગતિ તાપમાન દર્શાવે છે
\end{itemize}

\textbf{આકૃતિ:}

\begin{verbatim}
          સૂચક
             │
             ▼
        ┌────┴─────┐
સ્કેલ ───┤          │
        │          │
        │ ┌────────┘
        │ │
        └─┘
          ▲
          │
    બાયમેટાલિક પટ્ટી
 (બે અલગ ધાતુઓ જોડેલી)
 
 ઊંચા તાપમાને:
          
          ┌────────┐
          │        │
          │ ┌──────┘
          │ │
          └─┘
            {}
             { (અલગ{-}અલગ પ્રસરણને}
              {  કારણે વળી જાય છે)}
\end{verbatim}

\textbf{ફાયદા:}

\begin{itemize}
\tightlist
\item
  સરળ, મજબૂત રચના
\item
  વીજળી પુરવઠાની જરૂર નથી
\item
  વિશાળ તાપમાન શ્રેણી
\end{itemize}

\textbf{ગેરફાયદા:}

\begin{itemize}
\tightlist
\item
  અન્ય પ્રકારો કરતાં ઓછી ચોકસાઈ
\item
  ધીમી પ્રતિક્રિયા સમય
\item
  યાંત્રિક ઘસારાને આધીન
\end{itemize}

\end{solutionbox}
\begin{mnemonicbox}
``BEDS: બાયમેટાલિક તત્વો વિરૂપિત થાય તાણથી''

\end{mnemonicbox}
\subsection*{પ્રશ્ન 3(a) OR {[}3
ગુણ{]}}\label{uxaaauxab0uxab6uxaa8-3a-or-3-uxa97uxaa3}

\textbf{બિંદુવત વિદ્યુતભારથી ઉદ્ભવતા વિદ્યુતક્ષેત્ર ને સમજાવો.}

\begin{solutionbox}
\textbf{બિંદુ ચાર્જ પર કરેલું કાર્ય:} વિદ્યુત ક્ષેત્ર E માં બિંદુ ચાર્જ q
ને હલાવવામાં કરેલું કાર્ય.

\textbf{સૂત્ર:} W = q(Vₐ - Vᵦ) = qΔV

જ્યાં:

\begin{itemize}
\tightlist
\item
  q = ખસેડાતો ચાર્જ
\item
  Vₐ = પ્રારંભિક સ્થિતિનું પોટેન્શિયલ
\item
  Vᵦ = અંતિમ સ્થિતિનું પોટેન્શિયલ
\item
  ΔV = પોટેન્શિયલ તફાવત
\end{itemize}

\textbf{મુખ્ય લક્ષણો:}

\begin{itemize}
\tightlist
\item
  કાર્ય માર્ગથી સ્વતંત્ર છે
\item
  વિદ્યુત ક્ષેત્રની વિરુદ્ધ ખસેડવામાં કાર્ય ધનાત્મક છે
\item
  વિદ્યુત ક્ષેત્રની દિશામાં ખસેડવામાં કાર્ય ઋણાત્મક છે
\end{itemize}

\end{solutionbox}
\begin{mnemonicbox}
``PEW: પોટેન્શિયલ તફાવત × વિદ્યુત ચાર્જ = કાર્ય''

\end{mnemonicbox}
\subsection*{પ્રશ્ન 3(b) OR {[}4
ગુણ{]}}\label{uxaaauxab0uxab6uxaa8-3b-or-4-uxa97uxaa3}

\textbf{એક ધ્વનિનું તરંગ જેની ગતિ 0.33 km/s છે અને આવૃત્તિ 660 Hz છે. તે તરંગ 75 કંપન
માં કેટલું અંતર કાપશે?}

\begin{solutionbox}
\textbf{સૂત્રો:}

\begin{itemize}
\tightlist
\item
  તરંગલંબાઈ (λ) = ગતિ (v) / આવૃત્તિ (f)
\item
  અંતર (d) = કંપનોની સંખ્યા (n) × તરંગલંબાઈ (λ)
\end{itemize}

\textbf{ગણતરી:}

\begin{itemize}
\tightlist
\item
  ગતિનું રૂપાંતર: v = 0.33 km/s = 330 m/s
\item
  તરંગલંબાઈ: λ = v/f = 330/660 = 0.5 m
\item
  અંતર: d = n × λ = 75 × 0.5 = 37.5 m
\end{itemize}

\end{solutionbox}
\begin{mnemonicbox}
``FVW: આવૃત્તિમાં ગતિ ગુણતાં તરંગલંબાઈ મળે''

\end{mnemonicbox}
\subsection*{પ્રશ્ન 3(c) OR {[}7
ગુણ{]}}\label{uxaaauxab0uxab6uxaa8-3c-or-7-uxa97uxaa3}

\textbf{પારાવાળા થર્મોમિટરનો સિદ્ધાંત અને રચના આકૃતિ સાથે સમજાવો. તેના ફાયદા અને
ગેર ફાયદા લખો.}

\begin{solutionbox}
\textbf{સિદ્ધાંત}: પારા થર્મોમિટર પારાના તાપીય પ્રસરણના
સિદ્ધાંત પર કામ કરે છે.

\textbf{રચના:}

\begin{center}
\textbf{Mermaid Diagram (Code)}
\begin{verbatim}
{Shaded}
{Highlighting}[]
graph TD
    A[પારા થર્મોમિટર] {-{-}{} B[કાચનો બલ્બ]}
    A {-{-}{} C[કેશનળી]}
    A {-{-}{} D[સ્કેલ]}
    A {-{-}{} E[પારો]}
    A {-{-}{} F[વેક્યુમ/નાઇટ્રોજન જગ્યા]}
    A {-{-}{} G[સેફ્ટી બલ્બ]}
{Highlighting}
{Shaded}
\end{verbatim}
\end{center}

\textbf{કાર્યપદ્ધતિ:}

\begin{itemize}
\tightlist
\item
  પારો ગરમ થવાથી પ્રસરે છે
\item
  પ્રસરણથી પારો કેશનળીમાં ઉપર ચઢે છે
\item
  પારાના સ્તંભની ઊંચાઈ તાપમાન દર્શાવે છે
\end{itemize}

\textbf{આકૃતિ:}

\begin{verbatim}
      ┌─────┐
      │     │ ◄── સ્કેલ
      │     │
      │  │  │ ◄── કેશનળી
      │  │  │
      │  │  │
      │  │  │
      │ ┌┴┐ │ ◄── પારાનો બલ્બ
      │ └─┘ │
      └─────┘
\end{verbatim}

\textbf{ફાયદા:}

\begin{itemize}
\tightlist
\item
  ઉચ્ચ ચોકસાઈ
\item
  વિશાળ તાપમાન શ્રેણી (-38°C થી 357°C)
\item
  પારાનું રૈખિક પ્રસરણ
\item
  પારાના દોરાની સારી દૃશ્યતા
\end{itemize}

\textbf{ગેરફાયદા:}

\begin{itemize}
\tightlist
\item
  પારો ઝેરી છે
\item
  નાજુક કાચની રચના
\item
  -38°C નીચે વાપરી શકાતું નથી
\item
  તાપમાન ફેરફારોમાં ધીમી પ્રતિક્રિયા
\end{itemize}

\end{solutionbox}
\begin{mnemonicbox}
``MELT: પારો પ્રસરે રૈખિક તાપમાન સાથે''

\end{mnemonicbox}
\subsection*{પ્રશ્ન 4(a) {[}3
ગુણ{]}}\label{uxaaauxab0uxab6uxaa8-4a-3-uxa97uxaa3}

\textbf{સરખા માપના બે ધનઆયનને 5×10⁻¹⁰ m અંતરથી અલગ રાખવામા આવ્યા છે. તેમના વચ્ચે
લાગતું વિદ્યુત બળ 3.7 × 10⁻⁹ N જેટલું છે. તો દરેક એટમ માથી કેટલા ઇલેક્ટ્રોન નીકળશે.}

\begin{solutionbox}
\textbf{સૂત્ર:} F = (1/4πε₀) × (q₁q₂/r²)

\textbf{ગણતરી:}

\begin{itemize}
\tightlist
\item
  F = 3.7 × 10⁻⁹ N
\item
  r = 5 × 10⁻¹⁰ m
\item
  q₁ = q₂ = ne (n = ઇલેક્ટ્રોનની સંખ્યા, e = ઇલેક્ટ્રોન ચાર્જ)
\item
  1/4πε₀ = 9 × 10⁹ Nm²/C²
\item
  e = 1.6 × 10⁻¹⁹ C
\end{itemize}

3.7 × 10⁻⁹ = (9 × 10⁹) × (n²e²/(5 × 10⁻¹⁰)²) 3.7 × 10⁻⁹ = (9 × 10⁹) ×
(n² × (1.6 × 10⁻¹⁹)²/25 × 10⁻²⁰) ઉકેલ: n = 1 (દરેક પરમાણુમાંથી 1 ઇલેક્ટ્રોન
નીકળ્યો)

\end{solutionbox}
\begin{mnemonicbox}
``FACE: બળ અસર કરે ચાર્જ સમાન રીતે''

\end{mnemonicbox}
\subsection*{પ્રશ્ન 4(b) {[}4
ગુણ{]}}\label{uxaaauxab0uxab6uxaa8-4b-4-uxa97uxaa3}

\textbf{સ્નેલનો નિયમ લખો અને તેનું સૂત્ર મેળવો.}

\begin{solutionbox}
\textbf{સ્નેલનો નિયમ}: આપાત કોણના સાઇનનો વક્રીભવન કોણના સાઇન
સાથેનો ગુણોત્તર આપેલા માધ્યમના જોડા માટે અચળાંક છે.

\textbf{સૂત્ર:} (sin i)/(sin r) = n₂/n₁ = અચળાંક

\textbf{તારણના પગલાં:}

\begin{enumerate}
\def\labelenumi{\arabic{enumi}.}
\tightlist
\item
  પ્રકાશ વિવિધ માધ્યમોમાં વિવિધ ઝડપે પ્રવાસ કરે છે
\item
  જ્યારે પ્રકાશ એક માધ્યમથી બીજા માધ્યમમાં પસાર થાય, ત્યારે તે દિશા બદલે છે
\item
  ફર્મેટના ન્યૂનતમ સમયના સિદ્ધાંતનો ઉપયોગ કરીને
\item
  ગતિઓનો ગુણોત્તર વક્રીભવન સૂચકાંકોના ગુણોત્તર સમાન છે
\item
  અંતિમ સૂત્ર: n₁sin i = n₂sin r
\end{enumerate}

\textbf{આકૃતિ:}

\begin{verbatim}
        માધ્યમ 1 (n₁)
        │
        │         i
લંબરેખા  │        /
        │       /
────────┼──────/───────
        │     /
        │    /
        │   /  r
        │  /
        માધ્યમ 2 (n₂)
\end{verbatim}

\end{solutionbox}
\begin{mnemonicbox}
``SINIS: SIN I પર SIN R બરાબર વક્રીભવનાંક ગુણોત્તર''

\end{mnemonicbox}
\subsection*{પ્રશ્ન 4(c) {[}7
ગુણ{]}}\label{uxaaauxab0uxab6uxaa8-4c-7-uxa97uxaa3}

\textbf{અલ્ટ્રાસોનિક તરંગોના કોઈ પણ ત્રણ ઉપયોગો સમજાવો.}

\begin{solutionbox}
\textbf{અલ્ટ્રાસોનિક તરંગોના ઉપયોગો:}


{\def\LTcaptype{none} % do not increment counter
\vspace{-5pt}
\captionof{table}{અલ્ટ્રાસોનિક ઉપયોગો}
\vspace{-10pt}
\begin{longtable}[]{@{}lll@{}}
\toprule\noalign{}
ઉપયોગ & સિદ્ધાંત & ઉપયોગિતા \\
\midrule\noalign{}
\endhead
\bottomrule\noalign{}
\endlastfoot
મેડિકલ ઇમેજિંગ & પેશીઓથી પરાવર્તન & આંતરિક અંગોનું વિઝ્યુઅલાઇઝેશન \\
NDT (બિન-વિનાશક પરીક્ષણ) & ખામીઓથી પરાવર્તન & સામગ્રીમાં ખામીઓ શોધવી \\
સફાઈ & કેવિટેશન અસર & ઘરેણાં, સર્જિકલ સાધનો સાફ કરવા \\
\end{longtable}
}

\textbf{1. મેડિકલ ઇમેજિંગ (સોનોગ્રાફી):}

\begin{itemize}
\tightlist
\item
  આવૃત્તિઓ: 1-10 MHz
\item
  સિદ્ધાંત: પલ્સ-ઇકો તકનીક
\item
  ઉપયોગો: ગર્ભસ્થ શિશુનું ઇમેજિંગ, અંગોનું સ્કેનિંગ, રક્ત પ્રવાહનું માપન
\end{itemize}

\textbf{2. ઔદ્યોગિક NDT:}

\begin{itemize}
\tightlist
\item
  સામગ્રીમાં તિરાડો, છિદ્રો અને ખામીઓ શોધે છે
\item
  ઉત્પાદનમાં ગુણવત્તા નિયંત્રણ
\item
  સામગ્રીની જાડાઈનું માપન
\end{itemize}

\textbf{3. અલ્ટ્રાસોનિક સફાઈ:}

\begin{itemize}
\tightlist
\item
  સૂક્ષ્મ બુદબુદો (કેવિટેશન) બનાવે છે
\item
  સપાટીઓ પરથી દૂષિત પદાર્થોને દૂર કરે છે
\item
  ઘરેણાં, ઑપ્ટિકલ ઘટકો, સર્જિકલ સાધનો માટે વપરાય છે
\end{itemize}

\end{solutionbox}
\begin{mnemonicbox}
``MIC: મેડિકલ, ઔદ્યોગિક, સફાઈ ઉપયોગો''

\end{mnemonicbox}
\subsection*{પ્રશ્ન 4(a) OR {[}3
ગુણ{]}}\label{uxaaauxab0uxab6uxaa8-4a-or-3-uxa97uxaa3}

\textbf{ત્રણ કેપેસિટર જેમના મૂલ્ય 5 µF, 10 µF અને 15 µF છે, તેમના શ્રેણી તથા સમાંતર
જોડાણ માટેનો સમતુલ્ય કેપેસીટન્સ મેળવો.}

\begin{solutionbox}
\textbf{સમાંતર જોડાણ:} Cₚ = C₁ + C₂ + C₃ = 5 + 10 + 15 =
30 µF

\textbf{શ્રેણી જોડાણ:} 1/Cₛ = 1/C₁ + 1/C₂ + 1/C₃ 1/Cₛ = 1/5 + 1/10 + 1/15
1/Cₛ = 0.2 + 0.1 + 0.067 = 0.367 Cₛ = 1/0.367 = 2.72 µF

\end{solutionbox}
\begin{mnemonicbox}
``ASAP: શ્રેણીમાં ઉમેરો, સમાંતરમાં વ્યસ્ત ઉમેરો''

\end{mnemonicbox}
\subsection*{પ્રશ્ન 4(b) OR {[}4
ગુણ{]}}\label{uxaaauxab0uxab6uxaa8-4b-or-4-uxa97uxaa3}

\textbf{ઓપ્ટિકલ ફાઇબરની બનાવટને તેની આકૃતિ સાથે સમજાવો.}

\begin{solutionbox}
\textbf{ઓપ્ટિકલ ફાઇબરની રચના:}

\textbf{ઘટકો:}

\begin{itemize}
\tightlist
\item
  કોર: પ્રકાશ સંચરણ માધ્યમ
\item
  ક્લેડિંગ: ઓછા વક્રીભવનાંક સાથેનું બાહ્ય સ્તર
\item
  બફર કોટિંગ: રક્ષણાત્મક પ્લાસ્ટિક આવરણ
\end{itemize}

\textbf{પરિમાણો:}

\begin{itemize}
\tightlist
\item
  કોર વ્યાસ: 8-50 μm (સિંગલ મોડ), 50-100 μm (મલ્ટિમોડ)
\item
  ક્લેડિંગ વ્યાસ: 125-140 μm
\item
  કોર વક્રીભવનાંક \textgreater{} ક્લેડિંગ વક્રીભવનાંક
\end{itemize}

\textbf{આકૃતિ:}

\begin{verbatim}
આડછેદ:
         ┌───────────┐
         │     ┌─────┴─────┐
         │     │     │     │
         │     │     │     │
         │     │     │     │
         │     └─────┬─────┘
         └───────────┘
          │     │     │
   બફર   │ ક્લેડિંગ  │ કોર
          │     │     │

લંબછેદ:
  ┌──────────────────────────────┐
  │ ┌────────────────────────┐   │
  │ │                        │   │
  │ │           કોર          │   │
  │ │                        │   │  પ્રકાશ
  │ └────────────────────────┘   │  કિરણ
  └──────────────────────────────┘   ↘
              ક્લેડિંગ                ⟍⟋⟍⟋⟍⟋
                                    ⟍⟋⟍⟋⟍⟋
\end{verbatim}

\end{solutionbox}
\begin{mnemonicbox}
``CBC: કોર-બફર-ક્લેડિંગ અંદરથી બહાર''

\end{mnemonicbox}
\subsection*{પ્રશ્ન 4(c) OR {[}7
ગુણ{]}}\label{uxaaauxab0uxab6uxaa8-4c-or-7-uxa97uxaa3}

\textbf{મગ્નેટોસ્ટ્રીકશન પદ્ધતિ દ્વારા અલ્ટ્રાસોનિક તરંગનું ઉત્પાદન સમજાવો.}

\begin{solutionbox}
\textbf{મેગ્નેટોસ્ટ્રિક્શન પદ્ધતિ:} ફેરોમેગ્નેટિક પદાર્થોના ચુંબકીય
ક્ષેત્રમાં મૂકવાથી તેના પરિમાણમાં ફેરફાર થવાના ગુણધર્મનો ઉપયોગ કરીને અલ્ટ્રાસોનિક
તરંગો પેદા કરવાની પ્રક્રિયા.

\textbf{સિદ્ધાંત:} ફેરોમેગ્નેટિક પદાર્થો ચુંબકીત થવા પર લંબાઈમાં ફેરફાર કરે છે, જે
યાંત્રિક કંપનો પેદા કરે છે અને અલ્ટ્રાસોનિક તરંગો ઉત્પન્ન કરે છે.

\textbf{રચના:}

\begin{center}
\textbf{Mermaid Diagram (Code)}
\begin{verbatim}
{Shaded}
{Highlighting}[]
graph TD
    A[મેગ્નેટોસ્ટ્રિક્શન જનરેટર] {-{-}{} B[AC પાવર સપ્લાય]}
    A {-{-}{} C[કોઇલ/સોલેનોઇડ]}
    A {-{-}{} D[ફેરોમેગ્નેટિક સળિયો]}
    A {-{-}{} E[અવાજીય માધ્યમ]}
    A {-{-}{} F[શીતલન પ્રણાલી]}
{Highlighting}
{Shaded}
\end{verbatim}
\end{center}

\textbf{કાર્યપ્રક્રિયા:}

\begin{enumerate}
\def\labelenumi{\arabic{enumi}.}
\tightlist
\item
  AC કરંટ સોલેનોઇડમાંથી પસાર થાય છે
\item
  પરિવર્તનશીલ ચુંબકીય ક્ષેત્ર ઉત્પન્ન થાય છે
\item
  ફેરોમેગ્નેટિક સળિયો ફૂલે છે અને સંકોચાય છે
\item
  કંપનો માધ્યમમાં પ્રસારિત થાય છે
\item
  અલ્ટ્રાસોનિક તરંગો ઉત્પન્ન થાય છે
\end{enumerate}

\textbf{આકૃતિ:}

\begin{verbatim}
    ┌───────────────────┐
    │                   │
    │  AC પાવર સપ્લાય   │
    │                   │
    └─────────┬─────────┘
              │
              ▼
    ┌─────────────────────┐
    │                     │
    │  ┌───┐      ┌───┐   │
    │  │   │      │   │   │
    │  │   │      │   │   │
    │  │   │      │   │   │
    │  └───┘      └───┘   │
    │   કોઇલ      કોઇલ    │
    │                     │
    └──────────┬──────────┘
               │
      ┌────────┴────────┐
      │ ફેરોમેગ્નેટિક    │ ─→ અલ્ટ્રાસોનિક
      │ સળિયો (નિકલ)    │    તરંગો
      └─────────────────┘
\end{verbatim}

\textbf{ફાયદા:}

\begin{itemize}
\tightlist
\item
  સરળ બંધારણ
\item
  ઉચ્ચ શક્તિ આઉટપુટ
\item
  પ્રવાહીઓ માટે યોગ્ય
\end{itemize}

\textbf{ગેરફાયદા:}

\begin{itemize}
\tightlist
\item
  100 kHz નીચેની આવૃત્તિઓ સુધી મર્યાદિત
\item
  ગરમી અસરો
\item
  ઓછી કાર્યક્ષમતા
\end{itemize}

\end{solutionbox}
\begin{mnemonicbox}
``FAME: ફેરોમેગ્નેટિક પરિવર્તિત ચુંબકીય અસર''

\end{mnemonicbox}
\subsection*{પ્રશ્ન 5(a) {[}3
ગુણ{]}}\label{uxaaauxab0uxab6uxaa8-5a-3-uxa97uxaa3}

\textbf{ઉષ્મા પ્રસરણના ત્રણ પ્રકારને ટૂંકમાં સમજાવો.}

\begin{solutionbox}
\textbf{ઉષ્મા પ્રસરણના ત્રણ પ્રકારો:}


{\def\LTcaptype{none} % do not increment counter
\vspace{-5pt}
\captionof{table}{ઉષ્મા પ્રસરણ મોડ્સ}
\vspace{-10pt}
\begin{longtable}[]{@{}lll@{}}
\toprule\noalign{}
પ્રકાર & માધ્યમની આવશ્યકતા & ઉદાહરણ \\
\midrule\noalign{}
\endhead
\bottomrule\noalign{}
\endlastfoot
વહન & ભૌતિક સંપર્ક & ધાતુના સળિયા દ્વારા ઉષ્મા \\
સંવહન & પ્રવાહી માધ્યમ & ગરમ હવા ઊપર ચઢવી \\
વિકિરણ & કોઈ માધ્યમની જરૂર નથી & સૂર્યથી ઉષ્મા \\
\end{longtable}
}

\textbf{1. વહન:}

\begin{itemize}
\tightlist
\item
  સીધા અણુઓના અથડામણ દ્વારા પ્રસરણ
\item
  પદાર્થની જથ્થાબંધ ગતિવિધિ નથી
\item
  ઘન પદાર્થોમાં સારું, ખાસ કરીને ધાતુઓમાં
\end{itemize}

\textbf{2. સંવહન:}

\begin{itemize}
\tightlist
\item
  પ્રવાહી ગતિ દ્વારા પ્રસરણ
\item
  ઘનતામાં તફાવતની જરૂર પડે છે
\item
  કુદરતી અથવા દબાણપૂર્વક સંવહન
\end{itemize}

\textbf{3. વિકિરણ:}

\begin{itemize}
\tightlist
\item
  વિદ્યુત ચુંબકીય તરંગો દ્વારા પ્રસરણ
\item
  નિર્વાતમાં કામ કરે છે
\item
  તાપમાન અને સપાટી ગુણધર્મો પર આધાર રાખે છે
\end{itemize}

\end{solutionbox}
\begin{mnemonicbox}
``CCR: વહન સંપર્ક, સંવહન પ્રવાહ, વિકિરણ કિરણો''

\end{mnemonicbox}
\subsection*{પ્રશ્ન 5(b) {[}4
ગુણ{]}}\label{uxaaauxab0uxab6uxaa8-5b-4-uxa97uxaa3}

\textbf{એક ઓપ્ટિકલ ફાઇબરના કોર અને ક્લેડિંગના વક્રીભવાંક અનુક્રમે 1.55 અને 1.5 છે.
તો તેનો ન્યુમેરિકલ એપર્ચર અને એકપ્ટન્સ એંગલ શોધો.}

\begin{solutionbox}
\textbf{સૂત્રો:}

\begin{itemize}
\tightlist
\item
  ન્યુમેરિકલ એપર્ચર (NA) = √(n₁² - n₂²)
\item
  સ્વીકૃતિ કોણ (θₐ) = sin⁻¹(NA)
\end{itemize}

\textbf{ગણતરી:}

\begin{itemize}
\tightlist
\item
  કોર વક્રીભવનાંક (n₁) = 1.55
\item
  ક્લેડિંગ વક્રીભવનાંક (n₂) = 1.5
\end{itemize}

NA = √(1.55² - 1.5²) NA = √(2.4025 - 2.25) NA = √0.1525 NA = 0.391

સ્વીકૃતિ કોણ (θₐ) = sin⁻¹(0.391) θₐ = 23.03°

\end{solutionbox}
\begin{mnemonicbox}
``CORE: કોર ઓપ્ટિકલ રેફ્રેક્ટિવ-ઇન્ડેક્સ ચોક્કસપણે ગણો''

\end{mnemonicbox}
\subsection*{પ્રશ્ન 5(c) {[}7
ગુણ{]}}\label{uxaaauxab0uxab6uxaa8-5c-7-uxa97uxaa3}

\textbf{ઓપ્ટિકલ ફાઈબરના કોઈ પણ ત્રણ ઉપયોગો સમજાવો.}

\begin{solutionbox}
\textbf{ઓપ્ટિકલ ફાઇબરના ઉપયોગો:}


{\def\LTcaptype{none} % do not increment counter
\vspace{-5pt}
\captionof{table}{મુખ્ય ઓપ્ટિકલ ફાઇબર ઉપયોગો}
\vspace{-10pt}
\begin{longtable}[]{@{}lll@{}}
\toprule\noalign{}
ઉપયોગ & ફાયદો & ઉદાહરણ \\
\midrule\noalign{}
\endhead
\bottomrule\noalign{}
\endlastfoot
સંચાર & ઉચ્ચ બેન્ડવિડ્થ & ઇન્ટરનેટ, ફોન નેટવર્ક \\
મેડિકલ & લવચીકતા, ઇમેજિંગ & એન્ડોસ્કોપી \\
સેન્સર & ઇએમઆઈથી રક્ષણ & તાપમાન સેન્સિંગ \\
\end{longtable}
}

\textbf{1. સંચાર નેટવર્ક:}

\begin{itemize}
\tightlist
\item
  ટેલિકોમ્યુનિકેશન અને ઇન્ટરનેટ
\item
  કોપર કેબલ્સ કરતાં વધુ બેન્ડવિડ્થ
\item
  લાંબા અંતર પર ઓછું સિગ્નલ ઘટાડો
\item
  ટેપિંગ સામે વધુ સુરક્ષિત
\end{itemize}

\textbf{2. મેડિકલ એપ્લિકેશન:}

\begin{itemize}
\tightlist
\item
  મિનિમલ ઇન્વેસિવ પ્રક્રિયાઓ માટે એન્ડોસ્કોપી
\item
  ફોટોડાયનેમિક થેરાપી માટે પ્રકાશ ડિલિવરી
\item
  દંત પ્રક્રિયાઓ
\item
  સર્જિકલ પ્રકાશ
\end{itemize}

\textbf{3. સેન્સિંગ એપ્લિકેશન:}

\begin{itemize}
\tightlist
\item
  તાપમાન અને દબાણ સેન્સર
\item
  માળખાકીય મોનિટરિંગ માટે સ્ટ્રેન ગેજ
\item
  રાસાયણિક સેન્સર
\item
  નેવિગેશન માટે જાયરોસ્કોપ
\end{itemize}

\end{solutionbox}
\begin{mnemonicbox}
``CMS: સંચાર, મેડિકલ, સેન્સિંગ ઉપયોગો''

\end{mnemonicbox}
\subsection*{પ્રશ્ન 5(a) OR {[}3
ગુણ{]}}\label{uxaaauxab0uxab6uxaa8-5a-or-3-uxa97uxaa3}

\textbf{વિશિષ્ટ ઉષ્માને વિસ્તારથી સમજાવો.}

\begin{solutionbox}
\textbf{વિશિષ્ટ ઉષ્મા:} 1 કિલોગ્રામ પદાર્થનું તાપમાન 1 કેલ્વિન
(અથવા 1°C) વધારવા માટે જરૂરી ઉષ્મા.

\textbf{સૂત્ર:} Q = mcΔT

જ્યાં:

\begin{itemize}
\tightlist
\item
  Q = ઉષ્મા ઊર્જા (J)
\item
  m = દ્રવ્યમાન (kg)
\item
  c = વિશિષ્ટ ઉષ્મા ક્ષમતા (J/kg·K)
\item
  ΔT = તાપમાન ફેરફાર (K)
\end{itemize}

\textbf{એકમો:} J/kg·K અથવા J/kg·°C

\textbf{મહત્વ:}

\begin{itemize}
\tightlist
\item
  પદાર્થોની થર્મલ જડતા માપે છે
\item
  ઉચ્ચ વિશિષ્ટ ઉષ્માનો અર્થ પદાર્થને ગરમ કરવા માટે વધુ ઊર્જાની જરૂર પડે છે
\item
  પાણીની અસામાન્ય રીતે ઉચ્ચ વિશિષ્ટ ઉષ્મા છે (4,186 J/kg·K)
\end{itemize}

\end{solutionbox}
\begin{mnemonicbox}
``STEM: વિશિષ્ટ ઉષ્મા માપે તાપમાન ફેરફાર ઊર્જા અને દ્રવ્યમાન
દીઠ''

\end{mnemonicbox}
\subsection*{પ્રશ્ન 5(b) OR {[}4
ગુણ{]}}\label{uxaaauxab0uxab6uxaa8-5b-or-4-uxa97uxaa3}

\textbf{એક ઓપ્ટિકલ ફાઇબરના કોર અને ક્લેડિંગના વક્રીભવાંક અનુક્રમે 1.48 અને 1.45 છે.
તો તેનો એકપ્ટન્સ એંગલ અને ક્રાંતિકોણ શોધો.}

\begin{solutionbox}
\textbf{સૂત્રો:}

\begin{itemize}
\tightlist
\item
  ન્યુમેરિકલ એપર્ચર (NA) = √(n₁² - n₂²)
\item
  સ્વીકૃતિ કોણ (θₐ) = sin⁻¹(NA)
\item
  ક્રાંતિક કોણ (θc) = sin⁻¹(n₂/n₁)
\end{itemize}

\textbf{ગણતરી:}

\begin{itemize}
\tightlist
\item
  કોર વક્રીભવનાંક (n₁) = 1.48
\item
  ક્લેડિંગ વક્રીભવનાંક (n₂) = 1.45
\end{itemize}

NA = √(1.48² - 1.45²) NA = √(2.1904 - 2.1025) NA = √0.0879 NA = 0.296

સ્વીકૃતિ કોણ (θₐ) = sin⁻¹(0.296) θₐ = 17.2°

ક્રાંતિક કોણ (θc) = sin⁻¹(n₂/n₁) θc = sin⁻¹(1.45/1.48) θc = sin⁻¹(0.9797)
θc = 78.4°

\end{solutionbox}
\begin{mnemonicbox}
``NA થી AA મળે, ગુણોત્તર થી ક્રાંતિક કોણ મળે''

\end{mnemonicbox}
\subsection*{પ્રશ્ન 5(c) OR {[}7
ગુણ{]}}\label{uxaaauxab0uxab6uxaa8-5c-or-7-uxa97uxaa3}

\textbf{ઈજનેરી અને મેડીકલ ક્ષેત્રમાં LASER ના ઉપયોગો સમજાવો.}

\begin{solutionbox}
\textbf{LASER ના ઉપયોગો:}


{\def\LTcaptype{none} % do not increment counter
\vspace{-5pt}
\captionof{table}{LASER ઉપયોગો}
\vspace{-10pt}
\begin{longtable}[]{@{}lll@{}}
\toprule\noalign{}
ક્ષેત્ર & ઉપયોગ & ઉદાહરણ \\
\midrule\noalign{}
\endhead
\bottomrule\noalign{}
\endlastfoot
ઇજનેરી & કટિંગ/વેલ્ડિંગ & ધાતુ ફેબ્રિકેશન \\
ઇજનેરી & માપન & અંતર માપન \\
મેડિકલ & સર્જરી & આંખની સર્જરી (LASIK) \\
મેડિકલ & થેરાપી & કેન્સર સારવાર \\
\end{longtable}
}

\textbf{ઇજનેરી ઉપયોગો:}

\textbf{1. મટિરિયલ પ્રોસેસિંગ:}

\begin{itemize}
\tightlist
\item
  ધાતુ, પ્લાસ્ટિક, સિરામિક્સનું ચોક્કસ કટિંગ
\item
  અસમાન સામગ્રીની વેલ્ડિંગ
\item
  સપાટી ટ્રીટમેન્ટ અને હાર્ડનિંગ
\item
  3D પ્રિન્ટિંગ અને રેપિડ પ્રોટોટાઇપિંગ
\end{itemize}

\textbf{2. મેટ્રોલોજી અને માપન:}

\begin{itemize}
\tightlist
\item
  ઉચ્ચ ચોકસાઈ સાથે અંતર માપન
\item
  બાંધકામ અને ઉત્પાદનમાં એલાઇનમેન્ટ
\item
  સપાટી વિશ્લેષણ માટે ઇન્ટરફેરોમેટ્રી
\item
  3D ઇમેજિંગ માટે હોલોગ્રાફી
\end{itemize}

\textbf{મેડિકલ ઉપયોગો:}

\textbf{1. સર્જિકલ પ્રક્રિયાઓ:}

\begin{itemize}
\tightlist
\item
  આંખની સર્જરી (LASIK, મોતિયા નિકાલ)
\item
  મિનિમલી ઇન્વેસિવ પ્રક્રિયાઓ
\item
  ત્વચાની સારવાર
\item
  દંત પ્રક્રિયાઓ
\end{itemize}

\textbf{2. થેરાપ્યુટિક ઉપયોગો:}

\begin{itemize}
\tightlist
\item
  કેન્સર માટે ફોટોડાયનેમિક થેરાપી
\item
  દર્દ માટે લો-લેવલ લેસર થેરાપી
\item
  વાસ્ક્યુલર જખમોની સારવાર
\item
  કોસ્મેટિક પ્રક્રિયાઓ
\end{itemize}

\textbf{આકૃતિ:}

\begin{verbatim}
ઇજનેરી ઉપયોગો:
   LASER ──► મટિરિયલ પ્રોસેસિંગ
     │
     ├───► માપન
     │
     └───► સંચાર

મેડિકલ ઉપયોગો:
   LASER ──► સર્જિકલ પ્રક્રિયાઓ
     │
     ├───► નિદાન સાધનો
     │
     └───► થેરાપ્યુટિક સારવાર
\end{verbatim}

\end{solutionbox}
\begin{mnemonicbox}
``SMART: સર્જરી, માપન, વિશ્લેષણ, રિપેર, અને ટ્રીટમેન્ટ''

\end{mnemonicbox}
\end{document}