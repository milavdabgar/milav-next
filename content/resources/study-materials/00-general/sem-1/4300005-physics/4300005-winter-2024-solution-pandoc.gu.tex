\documentclass[10pt,a4paper]{article}

% content/resources/templates/preamble.tex
\usepackage[margin=0.6in]{geometry}
\author{Milav Dabgar}
\usepackage{amsmath,amssymb,amsthm}
\usepackage{booktabs}
\usepackage{multirow}
\usepackage{xcolor}
\usepackage{tcolorbox}
\tcbuselibrary{breakable,skins}
\usepackage[colorlinks=true,linkcolor=blue]{hyperref}
\usepackage{titlesec}
\usepackage{enumitem}
\usepackage{tikz}
\usepackage{pgfplots}
\usepackage{circuitikz}
\usepackage[version=4]{mhchem}
\usepackage{longtable}
\usepackage{array}
\usepackage{float}
\usepackage{caption}
\usepackage{listings}

\lstset{
  basicstyle=\small\ttfamily,
  breaklines=true,
  breakatwhitespace=false,
  postbreak=\mbox{\textcolor{red}{$\hookrightarrow$}\space},
  float=false,
  numbers=left,
  numberstyle=\tiny\color{gray},
  numbersep=10pt,
  xleftmargin=2em,
  keywordstyle=\color{blue},
  commentstyle=\color{green!60!black},
  stringstyle=\color{purple},
  backgroundcolor=\color{gray!5},
  showstringspaces=false,
  tabsize=2,
  captionpos=b,
  keepspaces=true,
  columns=flexible
}

\pgfplotsset{compat=1.18}
\usetikzlibrary{shapes,arrows,positioning,calc,patterns,decorations.pathmorphing,decorations.markings,arrows.meta}

% Color scheme
\definecolor{headcolor}{RGB}{0,102,204}
\definecolor{keycolor}{RGB}{220,20,60}
\definecolor{solutioncolor}{RGB}{34,139,34}
\definecolor{mnemoniccolor}{RGB}{148,0,211}
\definecolor{codecolor}{RGB}{0,0,100}

% Spacing
\setlength{\parskip}{3pt}
\setlist[itemize]{nosep}
\setlist[enumerate]{nosep}

% Title formatting
\titleformat{\section}{\Large\bfseries\color{headcolor}}{\thesection}{1em}{}
\titleformat{\subsection}{\large\bfseries\color{headcolor}}{\thesubsection}{1em}{}

% Pandoc tightlist compatibility
\providecommand{\tightlist}{%
  \setlength{\itemsep}{0pt}\setlength{\parskip}{0pt}}

% Pandoc longtable compatibility
\newcounter{none}
\def\thenone{}


% content/resources/templates/gujarati-boxes.tex
\usepackage{fontspec}
\usepackage{polyglossia}

% Set Gujarati as main language (document is primarily in Gujarati)
% Note: gloss-gujarati.ldf doesn't exist in polyglossia, but it will use hyphenation patterns
\setdefaultlanguage{gujarati}
\setotherlanguage{english}

% Configure Gujarati font properly
% Use Language=Default to prevent polyglossia from trying to add language-specific features
% that don't exist for Gujarati, which causes "empty feature" warnings
\newfontfamily\gujaratifont[Script=Gujarati,AutoFakeBold=2.5,AutoFakeSlant=0.3]{Noto Sans Gujarati}
\setmainfont[Script=Gujarati,AutoFakeBold=2.5,AutoFakeSlant=0.3]{Noto Sans Gujarati}
% Use Noto Sans Gujarati for monospace to support Gujarati in text
\setmonofont[Scale=0.9]{Noto Sans Gujarati}

% Configure English to use the same font
\newfontfamily\englishfont[Script=Gujarati,AutoFakeBold=2.5,AutoFakeSlant=0.3]{Noto Sans Gujarati}

% Translations for polyglossia
\gappto\captionsgujarati{
  \renewcommand{\tablename}{કોષ્ટક}
  \renewcommand{\figurename}{આકૃતિ}
}

% Helper for TikZ nodes to ensure Gujarati font
\newcommand{\gu}[1]{{\gujaratifont #1}}

% Custom environments
\newtcolorbox{solutionbox}{
    breakable,
    enhanced,
    colback=solutioncolor!5!white,
    colframe=solutioncolor!75!black,
    fonttitle=\bfseries,
    title=જવાબ
}

\newtcolorbox{solutionboxnobreak}{
 colback=solutioncolor!5!white,
 colframe=solutioncolor!75!black,
 fonttitle=\bfseries,
 title=જવાબ
}

\newtcolorbox{keyformula}{
 breakable,
 enhanced,
 colback=keycolor!5!white,
 colframe=keycolor!75!black,
 fonttitle=\bfseries,
 title=રાસાયણિક સમીકરણ/સૂત્ર
}

\newtcolorbox{mnemonicbox}{
 breakable,
 enhanced,
 colback=mnemoniccolor!5!white,
 colframe=mnemoniccolor!75!black,
 fonttitle=\bfseries,
 title=મેમરી ટ્રીક
}


\begin{document}

\begin{center}
{\Huge\bfseries\color{headcolor} Subject Name (Gujarati)}\\[5pt]
{\LARGE 4300005 -- Winter 2024}\\[3pt]
{\large Semester 1 Study Material}\\[3pt]
{\normalsize\textit{Detailed Solutions and Explanations}}
\end{center}

\vspace{10pt}

\subsection*{પ્રશ્ન 1(a) [3
ગુણ]}\label{q1a}

\textbf{ચોકસાઈ અને સચોટતા વ્યાખ્યાયિત કરો.}

\begin{solutionbox}

\begin{itemize}
\tightlist
\item
  \textbf{ચોકસાઈ}: માપેલી કિંમતનો સાચી કિંમતની નજીકતાનો માપ
\item
  \textbf{સચોટતા}: માપન કિંમતોની સુસંગતતા અથવા પુનરાવર્તિતા
\end{itemize}

\end{solutionbox}
\begin{mnemonicbox}
``ચોકસાઈ સત્યની નજીક, સચોટતા પુનરાવર્તનશીલ''

\end{mnemonicbox}
\subsection*{પ્રશ્ન 1(b) [4
ગુણ]}\label{q1b}

\textbf{મૂળભૂત ભૌતિક એકમોનો ઉપયોગ કરીને કાર્ય અને વેગનું SI એકમ મેળવો.}

\begin{solutionbox}


\vspace{-5pt}
\captionof{table}{કાર્ય અને વેગના એકમોની ફોર્મ્યુલેશન}
\vspace{-10pt}
\begin{longtable}[]{@{}
  >{\raggedright\arraybackslash}p{(\linewidth - 6\tabcolsep) * \real{0.3273}}
  >{\raggedright\arraybackslash}p{(\linewidth - 6\tabcolsep) * \real{0.1636}}
  >{\raggedright\arraybackslash}p{(\linewidth - 6\tabcolsep) * \real{0.3455}}
  >{\raggedright\arraybackslash}p{(\linewidth - 6\tabcolsep) * \real{0.1636}}@{}}
\toprule\noalign{}
\begin{minipage}[b]{\linewidth}\raggedright
ભૌતિક રાશિ
\end{minipage} & \begin{minipage}[b]{\linewidth}\raggedright
સૂત્ર
\end{minipage} & \begin{minipage}[b]{\linewidth}\raggedright
SI એકમ ફોર્મ્યુલેશન
\end{minipage} & \begin{minipage}[b]{\linewidth}\raggedright
SI એકમ
\end{minipage} \\
\midrule\noalign{}
\endhead
\bottomrule\noalign{}
\endlastfoot
કાર્ય (W) &

W = F \times d &

W = [બળ] \times [અંતર] = [kg·m/s^{2}] \times

[m] = [kg·m^{2}/s^{2}] & Joule (J) \\
વેગ (v) &

v = d/t &

v = [અંતર]/[સમય] = [m]/[s] & m/s \\

\end{longtable}

\begin{itemize}
\tightlist
\item
  \textbf{કાર્ય}: જ્યારે બળ (kg·m/s^{2}) અંતર (m) પર કાર્ય કરે છે, ત્યારે kg·m^{2}/s^{2} =
  Joule મળે છે
\item
  \textbf{વેગ}: જ્યારે કોઈ વસ્તુ સમય (s) માં અંતર (m) કાપે છે, ત્યારે m/s મળે છે
\end{itemize}

\end{solutionbox}
\begin{mnemonicbox}
``કાર્યમાં બળ અંતર, વેગમાં અંતર સમય''

\end{mnemonicbox}
\subsection*{પ્રશ્ન 1(c) [7
ગુણ]}\label{q1c}

\textbf{સાધનની લઘુત્તમ માપ શક્તિ શું હોય? વર્નિયર કેલિપર્સની લઘુત્તમ માપ શક્તિનું
સમીકરણ લખો. સુઘડ અને સ્વચ્છ આકૃતિ સાથે વર્નિયર કેલિપર્સ દ્વારા માપન સમજાવો.}

\begin{solutionbox}

\textbf{લઘુત્તમ માપ શક્તિ}: માપન સાધનથી સીધી રીતે માપી શકાય તેવી સૌથી નાની
માપ.

\textbf{વર્નિયર કેલિપર્સની લઘુત્તમ માપ શક્તિનું સમીકરણ}: લઘુત્તમ માપ શક્તિ = 1 મુખ્ય
સ્કેલ વિભાગ - 1 વર્નિયર સ્કેલ વિભાગ અથવા લઘુત્તમ માપ શક્તિ = 1 MSD ની કિંમત / VSD
ની સંખ્યા

\textbf{આકૃતિ: વર્નિયર કેલિપર}

\begin{verbatim}
      ┌────────┐
      │        │
 ┌────┘    ┌───┘
 │         │
 │   ┌─────┘
 │   │
─┼───┼───┬───┬───┬───┬───┬───┬───┬───┬───┬───┬
 0   1   2   3   4   5   6   7   8   9   10
     │   │   │   │   │   │   │   │   │
     └───┴───┴───┴───┴───┴───┴───┴───┴───┘ 
      0   5  10  15  20  25  30  35  40  45
      વર્નિયર સ્કેલ
\end{verbatim}

\textbf{માપન પ્રક્રિયા}:

\begin{itemize}
\item
  \textbf{પગલું 1}: વસ્તુની આસપાસ કેલિપરની બાજુઓ બંધ કરો
\item
  \textbf{પગલું 2}: વર્નિયર સ્કેલના શૂન્ય પહેલાં આવતા મુખ્ય સ્કેલના વાંચનની નોંધ કરો
\item
  \textbf{પગલું 3}: કયો વર્નિયર વિભાગ મુખ્ય સ્કેલના વિભાગ સાથે બરાબર સુમેળ કરે છે તે
  શોધો
\item
  \textbf{પગલું 4}: વર્નિયર વાંચનને મુખ્ય સ્કેલ વાંચન સાથે ઉમેરો: કુલ = MSR + (VC \times
  LC)
\item
  \textbf{મુખ્ય સ્કેલ વાંચન (MSR)}: વર્નિયર શૂન્ય પહેલાં મુખ્ય સ્કેલ પર કિંમત
\item
  \textbf{વર્નિયર સુમેળ (VC)}: જ્યાં વર્નિયર લાઇન મુખ્ય સ્કેલ લાઇન સાથે ગોઠવાય છે તે
  વિભાગ નંબર
\item
  \textbf{લઘુત્તમ માપ શક્તિ (LC)}: સામાન્ય રીતે 0.02 mm અથવા 0.001 ઈંચ
\end{itemize}

\end{solutionbox}
\begin{mnemonicbox}
``મુખ્ય વત્તા મેળ બનાવે માપ''

\end{mnemonicbox}
\subsection*{પ્રશ્ન 1(c) OR [7
ગુણ]}\label{q1c}

\textbf{સાધનની લઘુત્તમ માપ શક્તિ શું હોય? માઇક્રોમીટર સ્ક્રૂની લઘુત્તમ માપ શક્તિનું
સમીકરણ લખો. સુઘડ અને સ્વચ્છ આકૃતિ સાથે માઇક્રોમીટર સ્ક્રૂમાં હકારાત્મક અને નકારાત્મક
ભૂલ સમજાવો.}

\begin{solutionbox}

\textbf{લઘુત્તમ માપ શક્તિ}: માપન સાધનથી સીધી રીતે માપી શકાય તેવી સૌથી નાની
માપ.

\textbf{માઇક્રોમીટર સ્ક્રૂની લઘુત્તમ માપ શક્તિનું સમીકરણ}: લઘુત્તમ માપ શક્તિ = સ્ક્રૂનો
પિચ / વર્તુળાકાર સ્કેલ પરના વિભાગોની સંખ્યા

\textbf{આકૃતિ: માઇક્રોમીટર સ્ક્રૂ ગેજ}

\begin{verbatim}
     ┌─────────────────┐
     │                 │
     │    ┌───────┐    │
     │    │       │    │
     └────┤       ├────┘
          │       │
          └───────┘
          
    0  5  10 15 20 25
    ────────────────────
       │
       V
   ┌───────┐
   │0 5    │   વર્તુળાકાર સ્કેલ
   └───────┘
\end{verbatim}

\textbf{હકારાત્મક ભૂલ}: જ્યારે વર્તુળાકાર સ્કેલનો શૂન્ય સંદર્ભ રેખાની ઉપર હોય. માપેલું
વાંચન વાસ્તવિક કિંમત કરતાં વધારે થશે.

\textbf{નકારાત્મક ભૂલ}: જ્યારે વર્તુળાકાર સ્કેલનો શૂન્ય સંદર્ભ રેખાની નીચે હોય. માપેલું
વાંચન વાસ્તવિક કિંમત કરતાં ઓછું થશે.

\textbf{ભૂલ સુધારણા}:

\begin{itemize}
\tightlist
\item
  હકારાત્મક ભૂલ માટે: વાસ્તવિક વાંચન = નોંધાયેલું વાંચન - શૂન્ય ભૂલ
\item
  નકારાત્મક ભૂલ માટે: વાસ્તવિક વાંચન = નોંધાયેલું વાંચન + શૂન્ય ભૂલ
\end{itemize}

\end{solutionbox}
\begin{mnemonicbox}
``હકારાત્મક હોય બાદ, નકારાત્મક જોઈએ ઉમેરવું''

\end{mnemonicbox}
\subsection*{પ્રશ્ન 2(a) [3
ગુણ]}\label{q2a}

\textbf{વિદ્યુતક્ષેત્ર રેખાઓની લાક્ષણિકતાઓ લખો.}

\begin{solutionbox}


\vspace{-5pt}
\captionof{table}{વિદ્યુતક્ષેત્ર રેખાઓની લાક્ષણિકતાઓ}
\vspace{-10pt}
\begin{longtable}[]{@{}ll@{}}
\toprule\noalign{}
લાક્ષણિકતા & વર્ણન \\
\midrule\noalign{}
\endhead
\bottomrule\noalign{}
\endlastfoot
દિશા & હંમેશા ધન થી ઋણ ચાર્જ તરફ \\
આકાર & સમાન ક્ષેત્રો માટે સીધી રેખાઓ, અસમાન ક્ષેત્રો માટે વક્ર \\
ઘનતા & ક્ષેત્ર શક્તિના પ્રમાણમાં \\
માર્ગ & ક્યારેય એકબીજાને છેદતી નથી \\
પ્રકૃતિ & ધન ચાર્જથી શરૂ થાય છે અને ઋણ ચાર્જ પર સમાપ્ત થાય છે \\
\end{longtable}

\end{solutionbox}
\begin{mnemonicbox}
``દિશા, ઘનતા, છેદતી નથી, શરૂ-અંત''

\end{mnemonicbox}
\subsection*{પ્રશ્ન 2(b) [4
ગુણ]}\label{q2b}

\textbf{9 μF, 12 μF અને 15 μF કેપેસીટન્સ કિમત ધરાવતા કેપેસિટરના શ્રેણી અને સમાંતર
બંને જોડાણ માટે પરિણામી કેપેસીટન્સની ગણતરી કરો}

\begin{solutionbox}

\textbf{શ્રેણી જોડાણ માટે}: 1/Ceq = 1/C_{1} + 1/C_{2} + 1/C_{3} 1/Ceq = 1/9 + 1/12 +
1/15 1/Ceq = 5/36 + 3/36 + 2.4/36 = 10.4/36 Ceq = 36/10.4 = 3.46 μF

\textbf{સમાંતર જોડાણ માટે}: Ceq = C_{1} + C_{2} + C_{3} Ceq = 9 + 12 + 15 = 36 μF

\end{solutionbox}
\begin{mnemonicbox}
``શ્રેણીમાં વ્યસ્ત સરવાળો, સમાંતરમાં સીધો સરવાળો''

\end{mnemonicbox}
\subsection*{પ્રશ્ન 2(c) [7
ગુણ]}\label{q2c}

\textbf{કુલંબનો વ્યસ્ત વર્ગનો નિયમ સમજાવો અને તેનું સમીકરણ મેળવો. જો બે ઈલેક્ટ્રોન
વચ્ચેનું અંતર 10 મીટર હોય તો તેમની વચ્ચે લાગતો કુલંબ બળ શોધો.(e=1.66 x 10^{-}^{1}^{9} C,

K=

9 x 10^{9} Nm^{2} C^{-}^{2})}

\begin{solutionbox}

\textbf{કુલંબનો નિયમ}: બે બિંદુ ચાર્જ વચ્ચેનું સ્થિરવિદ્યુત બળ તે ચાર્જના ગુણાકારના
સમપ્રમાણમાં અને તેમની વચ્ચેના અંતરના વર્ગના વ્યસ્ત પ્રમાણમાં હોય છે.

\textbf{સમીકરણ ફોર્મ્યુલેશન}: F ∝ q_{1}q_{2} F ∝ 1/r^{2} એકત્રિત કરતાં: F ∝ q_{1}q_{2}/r^{2}
અચળાંક સાથે: F = k(q_{1}q_{2}/r^{2})

જ્યાં

k = 1/(4πε_{0}) = 9 \times 10^{9} Nm^{2}/C^{2}


\textbf{આકૃતિ: કુલંબનો નિયમ}

\begin{verbatim}
     q_{1        q_{2}}
     ●─────────●
     ────r────
     F_{1^{2}   _{2}_{1}}
\end{verbatim}

\textbf{ગણતરી}: F = k(q_{1}q_{2}/r^{2}) F = 9 \times 10^{9} \times [(1.66 \times 10^{-}^{1}^{9}) \times (1.66 \times
10^{-}^{1}^{9})] / (10)^{2}

F = 9 \times 10^{9} \times 2.76 \times 10^{-}^{3}^{8} / 100

F = 9 \times 2.76 \times

10^{-}^{3}^{8}^{-}^{2} \times 10^{9} F = 2.48 \times 10^{-}^{3}^{1} N

\end{solutionbox}
\begin{mnemonicbox}
``ચાર્જ ગુણાકાર, અંતર વર્ગ, બળ ઘટે''

\end{mnemonicbox}
\subsection*{પ્રશ્ન 2(a) OR [3
ગુણ]}\label{q2a}

\textbf{વિદ્યુતક્ષેત્રને સમજાવો અને તેનો એકમ મેળવો.}

\begin{solutionbox}

\textbf{વિદ્યુતક્ષેત્ર}: ચાર્જની આસપાસનો વિસ્તાર જ્યાં અન્ય ચાર્જ બળ અનુભવે છે.

\textbf{વ્યાખ્યા}: કોઈ બિંદુ પર વિદ્યુતક્ષેત્ર એ બળ છે જે તે બિંદુ પર મૂકેલા એકમ ધન
ચાર્જને અનુભવાય છે.

E = F/q

\textbf{એકમ ફોર્મ્યુલેશન}: E = F/q = [N]/[C] =
[kg·m/s^{2}]/[A·s] = [kg·m/(A·s^{3})] SI એકમ: N/C અથવા V/m

\end{solutionbox}
\begin{mnemonicbox}
``વિદ્યુતક્ષેત્ર એટલે ચાર્જ દીઠ બળ''

\end{mnemonicbox}
\subsection*{પ્રશ્ન 2(b) OR [4
ગુણ]}\label{q2b}

\textbf{સ્વચ્છ આકૃતિ દોરી વિદ્યુત ફ્લક્સ સમજવો અને તેનો એકમ મેળવો.}

\begin{solutionbox}

\textbf{વિદ્યુત ફ્લક્સ}: આપેલા ક્ષેત્રફળમાંથી પસાર થતા વિદ્યુતક્ષેત્રનું માપ.

\textbf{સમીકરણ}: ϕ_{e} = E·A·cosθ

જ્યાં:

\begin{itemize}
\tightlist
\item
  E એ વિદ્યુતક્ષેત્ર છે
\item
  A એ ક્ષેત્રફળ છે
\item
  θ એ E અને ક્ષેત્રફળના લંબ વચ્ચેનો ખૂણો છે
\end{itemize}

\textbf{આકૃતિ: વિદ્યુત ફ્લક્સ}

\begin{verbatim}
       ↑ n (લંબ)
       │
       │  θ
       │/
───────┼───── E (વિદ્યુતક્ષેત્ર)
       │
       │
    સપાટી ક્ષેત્રફળ A
\end{verbatim}

\textbf{એકમ ફોર્મ્યુલેશન}: ϕ_{e} = E·A·cosθ = [N/C]·[m^{2}]·[પરિમાણ
વિનાની] = [N·m^{2}/C] 1 N/C = 1 V/m હોવાથી, ફ્લક્સ એકમ = V·m = N·m^{2}/C

SI એકમ: N·m^{2}/C અથવા V·m

\end{solutionbox}
\begin{mnemonicbox}
``ફ્લક્સ વહે ક્ષેત્ર અને ક્ષેત્રફળ દ્વારા''

\end{mnemonicbox}
\subsection*{પ્રશ્ન 2(c) OR [7
ગુણ]}\label{q2c}

\textbf{કેપેસીટરની વ્યાખ્યા આપો અને તેનો યુનિટ મેળવો. સમાંતર પ્લેટ કેપેસિટરનું સૂત્ર આપો
અને દરેક પદ સમજાવો. 20 cm x 20 cm ચોરસ પ્લેટો ધરાવતા અને 1.0 mm ના અંતરથી અલગ
પડેલા સમાંતર પ્લેટ કેપેસિટરની કેપેસિટેન્સની ગણતરી કરો.}

\begin{solutionbox}

\textbf{કેપેસિટર}: વિદ્યુત ચાર્જ સંગ્રહિત કરતું ઉપકરણ.

\textbf{વ્યાખ્યા}: કેપેસિટન્સ એ સંગ્રહિત ચાર્જનો લાગુ કરેલા પોટેન્શિયલ તફાવત સાથેનો
ગુણોત્તર છે. C = Q/V

\textbf{એકમ ફોર્મ્યુલેશન}: C = Q/V = [C]/[V] = [A·s]/[J/C] =
[A·s]/[N·m/C] = [A^{2}·s^{4}/(kg·m^{2})] = Farad (F)

\textbf{સમાંતર પ્લેટ કેપેસિટર સૂત્ર}: C = ε_{0}εᵣA/d

જ્યાં:

\begin{itemize}
\tightlist
\item
  C એ કેપેસિટન્સ છે
\item
  ε_{0} એ મુક્ત અવકાશની પરાવૈદ્યુત્તા (8.85 \times 10^{-}^{1}^{2} F/m)
\item
  εᵣ એ ડાયલેક્ટ્રિકની સાપેક્ષ પરાવૈદ્યુત્તા છે
\item
  A એ પ્લેટોનો ઓવરલેપ ક્ષેત્રફળ છે
\item
  d એ પ્લેટો વચ્ચેનું અંતર છે
\end{itemize}

\textbf{આકૃતિ: સમાંતર પ્લેટ કેપેસિટર}

\begin{verbatim}
    ┌───────────────┐ ┐
    │ + + + + + + + │ │
    └───────────────┘ │ d
    ┌───────────────┐ │
    │ {- {-} {-} {-} {-} {-} {-} │ │}
    └───────────────┘ ┘
          ક્ષેત્રફળ A
\end{verbatim}

\textbf{ગણતરી}: A = 20 cm \times 20 cm = 0.2 m \times 0.2 m = 0.04 m^{2} d = 1.0 mm =
0.001 m εᵣ = 1 (હવા) ε_{0} = 8.85 \times 10^{-}^{1}^{2} F/m

C = ε_{0}εᵣA/d = 8.85 \times 10^{-}^{1}^{2} \times 1 \times 0.04/0.001 = 354 \times 10^{-}^{1}^{2}

F = 354 pF


\end{solutionbox}
\begin{mnemonicbox}
``કેપેસિટન્સ સંગ્રહે ચાર્જ નજીકના પ્લેટ વચ્ચે''

\end{mnemonicbox}
\subsection*{પ્રશ્ન 3(a) [3
ગુણ]}\label{q3a}

\textbf{ઘન પદાર્થમાં ઉષ્માના વહનને ઉદાહરણ સાથે સમજાવો.}

\begin{solutionbox}

\textbf{ઉષ્મા વહન}: ઘન પદાર્થમાં પદાર્થની હલનચલન વિના ઉષ્મા ઊર્જાનું સ્થાનાંતરણ.

\textbf{પ્રક્રિયા}: ઉષ્મા ઊર્જા અણુઓના કંપન દ્વારા ઉચ્ચ તાપમાન ક્ષેત્રથી નિમ્ન
તાપમાન ક્ષેત્ર તરફ સ્થાનાંતરિત થાય છે.

\textbf{આકૃતિ: ઉષ્મા વહન}

\begin{verbatim}
   ગરમ                ઠંડુ
    ↓                  ↓
┌────────────────────────┐
│ { │}
└────────────────────────┘
     ઉષ્મા પ્રવાહ 
\end{verbatim}

\textbf{ઉદાહરણ}: ગરમ ચામાં રાખેલો ધાતુનો ચમચો હેન્ડલ સુધી ગરમ થઈ જાય છે, જે વહન
દ્વારા થાય છે.

\end{solutionbox}
\begin{mnemonicbox}
``ગરમ ઊર્જા આપે, અણુઓ સ્થાનાંતરિત કરે, બહાર વહે''

\end{mnemonicbox}
\subsection*{પ્રશ્ન 3(b) [4
ગુણ]}\label{q3b}

\textbf{એક વ્યક્તિને 102 જેટલો તાવ આવે છે. અહીં તાપમાનનું એકમ કયો છે? આ તાપમાનને
બાકીના બે એકમમાં રૂપાંતરિત કરો.}

\begin{solutionbox}

\textbf{તાપમાન એકમ}: 102^\circF (ફેરનહાઈટ)

\textbf{રૂપાંતર સૂત્રો}:

\begin{itemize}
\tightlist
\item
  ^\circC = (^\circF - 32) \times 5/9
\item
  K = ^\circC + 273.15
\end{itemize}

\textbf{ગણતરી}: ^\circC = (102 - 32) \times 5/9 = 70 \times 5/9 = 38.89^\circC K = 38.89 +
273.15 = 312.04 K


\vspace{-5pt}
\captionof{table}{તાપમાન રૂપાંતર}
\vspace{-10pt}
\begin{longtable}[]{@{}lll@{}}
\toprule\noalign{}
ફેરનહાઈટ & સેલ્સિયસ & કેલ્વિન \\
\midrule\noalign{}
\endhead
\bottomrule\noalign{}
\endlastfoot
102^\circF & 38.89^\circC & 312.04 K \\
\end{longtable}

\end{solutionbox}
\begin{mnemonicbox}
``ફેરનહાઈટ પહેલા, સેલ્સિયસ બદલો, કેલ્વિન છેલ્લે આવે''

\end{mnemonicbox}
\subsection*{પ્રશ્ન 3(c) [7
ગુણ]}\label{q3c}

\textbf{પ્લેટિનમ રેઝિસ્ટન્સ થર્મોમીટરનો સિદ્ધાંત સમજાવો અને તેના ઉપયોગની યાદી
બનાવો.}

\begin{solutionbox}

\textbf{સિદ્ધાંત}: પ્લેટિનમનો વિદ્યુત અવરોધ તાપમાન સાથે નિશ્ચિત અને સુસંગત રીતે
બદલાય છે, જે ચોક્કસ તાપમાન માપન માટે અવકાશ આપે છે.

\textbf{કાર્યપ્રણાલી}: R = R_{0}[1 + α(T - T_{0})] સંબંધ પર આધારિત, જ્યાં R એ T
તાપમાને અવરોધ છે, R_{0} એ સંદર્ભ તાપમાન T_{0} પર અવરોધ છે, અને α એ અવરોધનો તાપમાન
ગુણાંક છે.

\textbf{આકૃતિ: પ્લેટિનમ રેઝિસ્ટન્સ થર્મોમીટર}

\begin{verbatim}
    ┌───────────────┐
    │   ઈન્ડિકેટર       │
    └───┬───────┬───┘
        │       │
        │       │
    ┌───┴───────┴───┐
    │   વ્હીટસ્ટોન      │
    │     બ્રિજ       │
    └───┬───────┬───┘
        │       │
        │       │
    ┌───┴───────┴───┐
    │   પ્લેટિનમ       │
    │   રેઝિસ્ટન્સ      │
    │    કોઈલ       │
    └───────────────┘
\end{verbatim}

\textbf{ઉપયોગો}:

\begin{itemize}
\tightlist
\item
  \textbf{ઔદ્યોગિક પ્રક્રિયા}: ઉત્પાદનમાં તાપમાન નિરીક્ષણ
\item
  \textbf{વૈજ્ઞાનિક સંશોધન}: ઉચ્ચ ચોકસાઈની જરૂરિયાત વાળા પ્રયોગશાળા માપન
\item
  \textbf{કેલિબ્રેશન}: અન્ય થર્મોમીટર્સના કેલિબ્રેશન માટે માનક
\item
  \textbf{તબીબી ઉપયોગો}: તબીબી ઉપકરણોમાં તાપમાન નિરીક્ષણ
\end{itemize}

\end{solutionbox}
\begin{mnemonicbox}
``પ્લેટિનમ આપે ચોક્કસ અવરોધ-તાપમાન સંબંધ''

\end{mnemonicbox}
\subsection*{પ્રશ્ન 3(a) OR [3
ગુણ]}\label{q3a}

\textbf{વિશિષ્ટ ઉષ્મા અને ઉષ્માધારિતા ની વ્યાખ્યાયિત લખો અને તેના એકમો લખો.}

\begin{solutionbox}

\textbf{વિશિષ્ટ ઉષ્મા}: 1 કિગ્રા પદાર્થનું તાપમાન 1 K વધારવા માટે જરૂરી ઉષ્મા
ઊર્જાનું પ્રમાણ.

\textbf{ઉષ્માધારિતા}: સંપૂર્ણ વસ્તુનું તાપમાન 1 K વધારવા માટે જરૂરી ઉષ્મા ઊર્જાનું
પ્રમાણ.


\vspace{-5pt}
\captionof{table}{ઉષ્મા ક્ષમતા શબ્દો}
\vspace{-10pt}
\begin{longtable}[]{@{}lll@{}}
\toprule\noalign{}
શબ્દ & સૂત્ર & SI એકમ \\
\midrule\noalign{}
\endhead
\bottomrule\noalign{}
\endlastfoot
વિશિષ્ટ ઉષ્મા (c) & Q = mc∆T & J/(kg·K) \\
ઉષ્માધારિતા (C) & Q = C∆T & J/K \\
\end{longtable}

\end{solutionbox}
\begin{mnemonicbox}
``વિશિષ્ટ પદાર્થ માટે, ધારિતા સંપૂર્ણ વસ્તુ માટે''

\end{mnemonicbox}
\subsection*{પ્રશ્ન 3(b) OR [4
ગુણ]}\label{q3b}

\textbf{તરલ પદાર્થમાં ઉષ્માનયન ઉદાહરણ સાથે સમજાવો.}

\begin{solutionbox}

\textbf{ઉષ્મા અભિવહન}: તરલ (પ્રવાહી અથવા વાયુ) ની હલનચલન દ્વારા ઉષ્માનું
સ્થાનાંતરણ.

\textbf{પ્રક્રિયા}: ગરમ તરલ પ્રસરણ પામે છે, ઓછી ઘનતા ધરાવે છે, ઉપર ઉઠે છે; ઠંડુ તરલ
નીચે ઉતરે છે, જે અભિવહન વહેણ તરીકે ઓળખાતી સતત પરિભ્રમણ પદ્ધતિ બનાવે છે.

\textbf{આકૃતિ: અભિવહન વહેણ}

\begin{verbatim}
      ↑      ↑      ↑
    ગરમ    ગરમ   ગરમ
      \^{      \^{}      \^{}}
      |      |      |
   ┌──────────────────┐
   │  ઉષ્મા સ્ત્રોત         │
   └──────────────────┘
   
       ઠંડુ તરલ
       ↓      ↓      ↓
\end{verbatim}

\textbf{ઉદાહરણ}: વાસણમાં ઉકળતું પાણી - ગરમ પાણી ઉપર ચઢે છે જ્યારે ઠંડુ પાણી નીચે
ઉતરે છે.

\end{solutionbox}
\begin{mnemonicbox}
``ગરમ ઉપર જાય, ઠંડુ નીચે આવે, વહેણ ફરતું રહે''

\end{mnemonicbox}
\subsection*{પ્રશ્ન 3(c) OR [7
ગુણ]}\label{q3c}

\textbf{ઉષ્મા વાહકતાના અચળાંકને વ્યાખ્યાયિત કરો. ઘન પદાર્થોમાં ઉષ્માના વહન માટે
ઉષ્મા વાહકતાના અચળાંકનું સમીકરણ મેળવો.}

\begin{solutionbox}

\textbf{ઉષ્મા વાહકતાનો અચળાંક}: એકમ સમય દીઠ, એકમ ક્ષેત્રફળ દીઠ, એકમ તાપમાન
પ્રવણતા દીઠ સ્થાનાંતરિત થતી ઉષ્માનું પ્રમાણ.

\textbf{વ્યાખ્યા}: જ્યારે તાપમાન પ્રવણતા એકમ હોય ત્યારે દર સેકન્ડે એકમ ક્ષેત્રફળ
દ્વારા વહેતી ઉષ્માનું પ્રમાણ.

\textbf{ફોર્મ્યુલેશન}:

\begin{itemize}
\tightlist
\item
  છેદફળ A અને લંબાઈ L ધરાવતા સળિયાને ધ્યાનમાં લો
\item
  છેડા વચ્ચેનો તાપમાન તફાવત ∆T છે
\item
  સમય t માં ઉષ્મા પ્રવાહ Q છે
\end{itemize}

ઉષ્મા પ્રવાહ = Q/t તાપમાન પ્રવણતા = ∆T/L ક્ષેત્રફળ = A

ફોરિયરના નિયમ અનુસાર: Q/t = k·A·(∆T/L)

પુનર્ગોઠવણી કરતાં: k = (Q·L)/(t·A·∆T)

જ્યાં k એ ઉષ્મા વાહકતાનો અચળાંક છે.

\textbf{આકૃતિ: ઉષ્મા વાહકતા}

\begin{verbatim}
   T_{1                 T_{2}}
    ↓                  ↓
┌────────────────────────┐
│                        │ ક્ષેત્રફળ A
└────────────────────────┘
    ───── L ─────
        ઉષ્મા પ્રવાહ 
\end{verbatim}

\textbf{એકમ}: W/(m·K)

\end{solutionbox}
\begin{mnemonicbox}
``ઉષ્મા જથ્થો સ્થાનાંતરિત થાય લંબાઈ દ્વારા, ક્ષેત્રફળ અને
તાપમાન ભાગીને''

\end{mnemonicbox}
\subsection*{પ્રશ્ન 4(a) [3
ગુણ]}\label{q4a}

\textbf{લંબગત તરંગો અને સંગત તરંગો વચ્ચેનો તફાવત આપો.}

\begin{solutionbox}


\vspace{-5pt}
\captionof{table}{લંબગત બનામ સંગત તરંગો}
\vspace{-10pt}
\begin{longtable}[]{@{}
  >{\raggedright\arraybackslash}p{(\linewidth - 4\tabcolsep) * \real{0.2128}}
  >{\raggedright\arraybackslash}p{(\linewidth - 4\tabcolsep) * \real{0.3830}}
  >{\raggedright\arraybackslash}p{(\linewidth - 4\tabcolsep) * \real{0.4043}}@{}}
\toprule\noalign{}
\begin{minipage}[b]{\linewidth}\raggedright
ગુણધર્મ
\end{minipage} & \begin{minipage}[b]{\linewidth}\raggedright
લંબગત તરંગો
\end{minipage} & \begin{minipage}[b]{\linewidth}\raggedright
સંગત તરંગો
\end{minipage} \\
\midrule\noalign{}
\endhead
\bottomrule\noalign{}
\endlastfoot
કણની ગતિ & તરંગ દિશાને લંબ & તરંગ દિશાને સમાંતર \\
માધ્યમ વિસ્થાપન & શિખર અને ગર્ત & સંકોચન અને વિરલન \\
ઉદાહરણો & પ્રકાશ તરંગો, પાણીના તરંગો & ધ્વનિ તરંગો, સિસ્મિક P-તરંગો \\
માધ્યમ જરૂરિયાતો & ઘન પદાર્થોમાં પ્રવાસ કરી શકે & ઘન, પ્રવાહી, વાયુમાં પ્રવાસ કરી
શકે \\
ધ્રુવીકરણ & ધ્રુવીકૃત થઈ શકે & ધ્રુવીકૃત થઈ શકતા નથી \\
\end{longtable}

\end{solutionbox}
\begin{mnemonicbox}
``લંબગત લે લંબ માર્ગ, સંગત સહાય સમાંતર સરકવામાં''

\end{mnemonicbox}
\subsection*{પ્રશ્ન 4(b) [4
ગુણ]}\label{q4b}

\textbf{જો એક તરંગનો વેગ 350 m/s અને આવૃત્તિ 10 Hz છે તો તેની તરંગલંબાઇની ગણતરી
કરો.}

\begin{solutionbox}

\textbf{તરંગ સમીકરણ}: v = fλ

જ્યાં:

\begin{itemize}
\tightlist
\item
  v એ તરંગ વેગ છે (350 m/s)
\item
  f એ આવૃત્તિ છે (10 Hz)
\item
  λ એ તરંગલંબાઈ છે (શોધવાની છે)
\end{itemize}

\textbf{ગણતરી}: λ = v/f = 350/10 = 35 m

\end{solutionbox}
\begin{mnemonicbox}
``વેગ બરાબર આવૃત્તિ ગુણાકાર તરંગલંબાઈ''

\end{mnemonicbox}
\subsection*{પ્રશ્ન 4(c) [7
ગુણ]}\label{q4c}

\textbf{અલ્ટ્રાસોનિક તરંગોને વ્યાખ્યાયિત કરો અને તેની લાક્ષણિકતાઓ લખો. અલ્ટ્રાસોનિક
તરંગની તેની ચાર મુખ્ય ઉપયોગો લખો.}

\begin{solutionbox}

\textbf{અલ્ટ્રાસોનિક તરંગો}: માનવ શ્રવણની ઉપલી મર્યાદા (20 kHz થી વધુ) કરતાં
ઊંચી આવૃત્તિ ધરાવતા ધ્વનિ તરંગો.

\textbf{લાક્ષણિકતાઓ}:

\begin{itemize}
\tightlist
\item
  \textbf{ઉચ્ચ આવૃત્તિ}: 20 kHz થી વધુ
\item
  \textbf{ટૂંકી તરંગલંબાઈ}: નાની વસ્તુઓને શોધવાની ક્ષમતા આપે છે
\item
  \textbf{દિશાસૂચક}: ચોક્કસ દિશામાં કેન્દ્રિત કરી શકાય છે
\item
  \textbf{બિન-આયનીકરણ}: જૈવિક પેશીઓ માટે સલામત
\item
  \textbf{પ્રવેશ}: વિવિધ માધ્યમોમાંથી પસાર થઈ શકે છે
\end{itemize}

\textbf{આકૃતિ: અલ્ટ્રાસોનિક તરંગ}

\begin{verbatim}
      આયામ
        ↑
        │   /{      /      /}
        │  /  {    /      /  }
 ───────┼─/────{──/──────/────────── સમય}
        │/      {/      /      }
        │
      અવધિ { 50 μs (f  20 kHz)}
\end{verbatim}

\textbf{ઉપયોગો}:

\begin{itemize}
\tightlist
\item
  \textbf{તબીબી}: નિદાનાત્મક ઇમેજિંગ, ઉપચારાત્મક પ્રક્રિયાઓ
\item
  \textbf{ઔદ્યોગિક}: બિન-વિનાશક પરીક્ષણ, ખામી શોધ
\item
  \textbf{સફાઈ}: સચોટ ભાગો માટે અલ્ટ્રાસોનિક ક્લીનિંગ બાથ
\item
  \textbf{અંતર માપન}: સોનાર, પાર્કિંગ સેન્સર, લેવલ ઇન્ડિકેટર્સ
\end{itemize}

\end{solutionbox}
\begin{mnemonicbox}
``અલ્ટ્રાસોનિક ઉપયોગ ધ્વનિ શોધવા, સ્કેન કરવા, સાફ કરવા''

\end{mnemonicbox}
\subsection*{પ્રશ્ન 4(a) OR [3
ગુણ]}\label{q4a}

\textbf{પ્રકાશના ધ્રુવીકરણને સ્વચ્છ આકૃતિ દોરી સમજાવો.}

\begin{solutionbox}

\textbf{ધ્રુવીકરણ}: પ્રકાશ તરંગોના કંપનોને એક જ સમતલમાં મર્યાદિત કરવાની પ્રક્રિયા.

\textbf{પ્રકારો}:

\begin{itemize}
\tightlist
\item
  રેખીય ધ્રુવીકરણ
\item
  વર્તુળાકાર ધ્રુવીકરણ
\item
  ઇલિપ્ટિકલ ધ્રુવીકરણ
\end{itemize}

\textbf{આકૃતિ: પ્રકાશ ધ્રુવીકરણ}

\begin{verbatim}
 અધ્રુવીય પ્રકાશ  ધ્રુવક   ધ્રુવીય પ્રકાશ
       ↓              ↓             ↓
 ⊥↕⊢⊣|↖↗↘↙       ┌─────┐        
 ⊥↕⊢⊣|↖↗↘↙     │/////│       
 ⊥↕⊢⊣|↖↗↘↙       └─────┘        
  અનેક            માત્ર એક        એક જ
  કંપન             સમતલ          સમતલ
 સમતલો            પસાર          કંપન
\end{verbatim}

\end{solutionbox}
\begin{mnemonicbox}
``ધ્રુવક પસંદ કરે વિશિષ્ટ સમતલો''

\end{mnemonicbox}
\subsection*{પ્રશ્ન 4(b) OR [4
ગુણ]}\label{q4b}

\textbf{જો પ્રકાશ નો હવા માં વેગ 3 x 10^{8} m/s અને પ્રકાશનો પાણી માં વેગ 2.25 x
10^{8} m/s તો પ્રકાશનો વક્રીવનાંક શોધો.}

\begin{solutionbox}

\textbf{વક્રીભવનાંક સૂત્ર}: n = c/v

જ્યાં:

\begin{itemize}
\tightlist
\item
  n એ વક્રીભવનાંક છે
\item
  c એ શૂન્યાવકાશમાં (અથવા હવામાં) પ્રકાશનો વેગ છે
\item
  v એ માધ્યમમાં પ્રકાશનો વેગ છે
\end{itemize}

\textbf{ગણતરી}: n = 3 \times 10^{8} / 2.25 \times 10^{8} = 3/2.25 = 4/3 = 1.33

\end{solutionbox}
\begin{mnemonicbox}
``ધીમો વેગ બતાવે ઊંચો સૂચક''

\end{mnemonicbox}
\subsection*{પ્રશ્ન 4(c)(i) OR [4
ગુણ]}\label{q4c}

\textbf{વ્યાખ્યાયિત કરો: તરંગ નો વેગ, તરંગલંબાઈ અને આવૃતિ. અને તરંગ વેગ, તરંગલંબાઈ અને
આવૃતિ વચ્ચેનો સંબંધ મેળવો.}

\begin{solutionbox}

\textbf{તરંગ વેગ (v)}: તરંગ માધ્યમમાં જે ગતિથી પ્રવાસ કરે છે તે.

\textbf{તરંગલંબાઈ (λ)}: તરંગ પર બે ક્રમિક સમાન બિંદુઓ વચ્ચેનું અંતર.

\textbf{આવૃત્તિ (f)}: દર એકમ સમયે કોઈ બિંદુમાંથી પસાર થતા સંપૂર્ણ તરંગ ચક્રોની
સંખ્યા.

\textbf{આકૃતિ: તરંગ પરિમાણો}

\begin{verbatim}
આયામ
    ↑
    │   /{      /      /}
    │  /  {    /      /  }
────┼─/────{──/──────/───── અંતર}
    │/      {/      /      }
    │
    ↑        ↑              ↑
તરંગલંબાઈ (λ)    અવધિ (T)
\end{verbatim}

\textbf{ફોર્મ્યુલેશન}:

\begin{itemize}
\tightlist
\item
  સમય T (અવધિ) માં, તરંગ એક તરંગલંબાઈ λ જેટલું અંતર પ્રવાસ કરે છે
\item
  તેથી, v = λ/T
\item
  આવૃત્તિ f = 1/T (આવૃત્તિ એ અવધિનો વ્યસ્ત છે)
\item
  તેથી, v = λf
\end{itemize}

\end{solutionbox}
\begin{mnemonicbox}
``વેગ બરાબર આવૃત્તિ ગુણાકાર તરંગલંબાઈ''

\end{mnemonicbox}
\subsection*{પ્રશ્ન 4(c)(ii) OR [3
ગુણ]}\label{q4c}

\textbf{પ્રકાશના ગુણધર્મો લખો.}

\begin{solutionbox}


\vspace{-5pt}
\captionof{table}{પ્રકાશના ગુણધર્મો}
\vspace{-10pt}
\begin{longtable}[]{@{}ll@{}}
\toprule\noalign{}
ગુણધર્મ & વર્ણન \\
\midrule\noalign{}
\endhead
\bottomrule\noalign{}
\endlastfoot
પ્રચાર & સમાંગી માધ્યમમાં સીધી રેખામાં ચાલે છે \\
વેગ & શૂન્યાવકાશમાં 3 \times 10^{8} m/s \\
પરાવર્તન & સપાટીઓ પરથી પરાવર્તન નિયમ અનુસરીને પરાવર્તિત થાય છે \\
વક્રીભવન & માધ્યમો વચ્ચે પસાર થતાં દિશા બદલે છે \\
વિભાજન & શ્વેત પ્રકાશ તેના ઘટક રંગોમાં વિભાજિત થાય છે \\
વ્યતિકરણ & તરંગો ભેગા થઈને પેટર્ન બનાવી શકે છે \\
વિવર્તન & અવરોધો અને નાના છિદ્રોમાંથી વળે છે \\
ધ્રુવીકરણ & એક સમતલમાં કંપન કરવા માટે મર્યાદિત કરી શકાય છે \\
દ્વૈત પ્રકૃતિ & તરંગ અને કણ બંને ગુણધર્મો દર્શાવે છે \\
\end{longtable}

\end{solutionbox}
\begin{mnemonicbox}
``પ્રકાશ પરાવર્તે, વક્રીભવે, વિભાજિત થાય, વ્યતિકરણ કરે,
ધ્રુવીકૃત થાય''

\end{mnemonicbox}
\subsection*{પ્રશ્ન 5(a) [3
ગુણ]}\label{q5a}

\textbf{સમતલ સપાટી માટે પ્રકાશના વક્રીભવનના નિયમો સમજાવો. અને સ્નેલનો નિયમ
સમજાવો.}

\begin{solutionbox}

\textbf{વક્રીભવનનો નિયમ}: જ્યારે પ્રકાશ એક માધ્યમથી બીજા માધ્યમમાં પસાર થાય છે,
ત્યારે તે સીમા પર દિશા બદલે છે.

\textbf{સ્નેલનો નિયમ}: આપતન કોણના સાઇનનો વક્રીભવન કોણના સાઇન સાથેનો ગુણોત્તર
આપેલા માધ્યમોની જોડી માટે અચળ રહે છે.

n_{1}sin(θ_{1}) = n_{2}sin(θ_{2})

જ્યાં:

\begin{itemize}
\tightlist
\item
  n_{1} એ પ્રથમ માધ્યમનો વક્રીભવનાંક છે
\item
  n_{2} એ બીજા માધ્યમનો વક્રીભવનાંક છે
\item
  θ_{1} એ આપતન કોણ છે
\item
  θ_{2} એ વક્રીભવન કોણ છે
\end{itemize}

\textbf{આકૃતિ: વક્રીભવન}

\begin{verbatim}
           ┌─────────────
   લંબ     │
      ↑    │    માધ્યમ 1 (n_{1)}
      │    │
      │    │    આપતન કિરણ
      │   /│
      │  / │
      │ /  │
      │/θ_{1 │}
      ├────┼────────────────
      │{θ_{2} │}
      │ {  │}
      │  { │}
      │   {│    માધ્યમ 2 (n_{2})}
           │    વક્રીભવન કિરણ
           │
           │
           └─────────────
\end{verbatim}

\end{solutionbox}
\begin{mnemonicbox}
``સાઇન બતાવે વેગ અલગ માધ્યમોમાં''

\end{mnemonicbox}
\subsection*{પ્રશ્ન 5(b) [4
ગુણ]}\label{q5b}

\textbf{સ્ટેપ ઈન્ડેક્ષ ફાઈબર માં કોર વક્રીભવનાંક 1.30 હોય અને સંબંધિત વક્રીભવનાંક
તફાવત Δ=0.02 છે. ન્યુમેરિકલ એપેચર શોધો.}

\begin{solutionbox}

\textbf{ન્યુમેરિકલ એપેચર સૂત્ર}: NA = \sqrt(n_{1}^{2} - n_{2}^{2})

સ્ટેપ ઈન્ડેક્સ ફાઈબર માટે: NA = n_{1}\sqrt(2Δ)

જ્યાં:

\begin{itemize}
\tightlist
\item
  n_{1} એ કોર વક્રીભવનાંક છે
\item
  Δ એ સંબંધિત વક્રીભવનાંક તફાવત છે
\end{itemize}

\textbf{ગણતરી}: NA = 1.30 \times \sqrt(2 \times 0.02) NA = 1.30 \times \sqrt0.04 NA = 1.30 \times
0.2 NA = 0.26

\end{solutionbox}
\begin{mnemonicbox}
``ન્યુમેરિકલ એપેચર જોઈએ કોર અને ડેલ્ટા''

\end{mnemonicbox}
\subsection*{પ્રશ્ન 5(c) [7
ગુણ]}\label{q5c}

\textbf{પ્રકાશનું પૂર્ણ આંતરિક પરાવર્તન સમજાવો. અને ક્રિટિકલ ખૂણાનું સમીકરણ મેળવો.}

\begin{solutionbox}

\textbf{પૂર્ણ આંતરિક પરાવર્તન (TIR)}: જ્યારે પ્રકાશ સઘન માધ્યમથી વિરલ માધ્યમમાં
ક્રિટિકલ કોણથી વધુ કોણે જતો હોય ત્યારે માધ્યમોની સીમા પર પ્રકાશનું સંપૂર્ણ પરાવર્તન.

\textbf{TIR માટેની શરતો}:

\begin{enumerate}
\tightlist
\item
  પ્રકાશ સઘન માધ્યમથી વિરલ માધ્યમ તરફ જવો જોઈએ
\item
  આપતન કોણ ક્રિટિકલ કોણથી વધુ હોવો જોઈએ
\end{enumerate}

\textbf{ક્રિટિકલ કોણ}: સઘન માધ્યમમાં આપતન કોણ જેના માટે વિરલ માધ્યમમાં વક્રીભવન
કોણ 90^\circ હોય.

\textbf{ફોર્મ્યુલેશન}: સ્નેલના નિયમનો ઉપયોગ કરીને: n_{1}sin(θ_{1}) = n_{2}sin(θ_{2})

ક્રિટિકલ કોણ (θc) પર:

\begin{itemize}
\tightlist
\item
  θ_{1} = θc
\item
  θ_{2} = 90^\circ
\item
  sin(90^\circ) = 1
\end{itemize}

તેથી: n_{1}sin(θc) = n_{2}sin(90^\circ) = n_{2} \times 1 = n_{2}

પુનર્ગોઠવણી કરતાં: sin(θc) = n_{2}/n_{1}

\textbf{આકૃતિ: પૂર્ણ આંતરિક પરાવર્તન}

\begin{verbatim}
       માધ્યમ 1 (n_{1)}
       (સઘન)
       ┌─────────────────
       │  {      /}
       │   {θc  /}
       │    {  /}
       │     {/}
       │     /{}
       │    /  {}
       │   /    {}
       │  /      {}
       └─────────────────
       માધ્યમ 2 (n_{2)}
       (વિરલ)
\end{verbatim}

\end{solutionbox}
\begin{mnemonicbox}
``ક્રિટિકલ આવે સઘનથી વિરલ, સાઈન બરાબર ભાગાકાર''

\end{mnemonicbox}
\subsection*{પ્રશ્ન 5(a) OR [3
ગુણ]}\label{q5a}

\textbf{ફાઈબર ઓપ્ટીકલ કેબલ માટે ન્યુમેરિકલ એપેચર અને એક્સેપ્ટન્સ ખૂણો સમજાવો.}

\begin{solutionbox}

\textbf{ન્યુમેરિકલ એપેચર (NA)}: ઓપ્ટિકલ ફાઈબરની પ્રકાશ-એકત્રિત કરવાની ક્ષમતાનું
માપ.

\textbf{એક્સેપ્ટન્સ ખૂણો (θ_{a})}: મહત્તમ કોણ જેના પર પ્રકાશ ફાઈબરમાં પ્રવેશી શકે છે અને
હજુ પણ પૂર્ણ આંતરિક પરાવર્તન અનુભવી શકે છે.

\textbf{સંબંધ}: NA = sin(θ_{a})

\textbf{આકૃતિ: ન્યુમેરિકલ એપેચર અને એક્સેપ્ટન્સ ખૂણો}

\begin{verbatim}
                 θ_{a}
                /│{}
      ક્લેડિંગ /a│ {    ક્લેડિંગ}
      ────────┼──┼──┼────────
              │  │  │
      કોર     │  │  │    કોર
      ────────┼──┼──┼────────
              │  │  │
      ક્લેડિંગ│  │  │    ક્લેડિંગ
      ────────┴──┴──┴────────
\end{verbatim}

\end{solutionbox}
\begin{mnemonicbox}
``એક્સેપ્ટન્સ ખૂણો પ્રકાશ પ્રવેશાવે, ન્યુમેરિકલ એપેચર તેનો સાઈન
કહેવાય''

\end{mnemonicbox}
\subsection*{પ્રશ્ન 5(b) OR [4
ગુણ]}\label{q5b}

\textbf{લેસર નું આખું નામ લખો. તેની લાક્ષણિકતાઓ લખો.}

\begin{solutionbox}

\textbf{LASER}: Light Amplification by Stimulated Emission of Radiation
(ઉત્તેજિત વિકિરણ ઉત્સર્જન દ્વારા પ્રકાશ વર્ધન)


\vspace{-5pt}
\captionof{table}{લેસરની લાક્ષણિકતાઓ}
\vspace{-10pt}
\begin{longtable}[]{@{}ll@{}}
\toprule\noalign{}
લાક્ષણિકતા & વર્ણન \\
\midrule\noalign{}
\endhead
\bottomrule\noalign{}
\endlastfoot
એકવર્ણીય & એક જ તરંગલંબાઈ અથવા રંગ \\
સુસંગત & બધા તરંગો એક જ તબક્કામાં \\
અત્યંત દિશાત્મક & લઘુત્તમ વિચલન સાથે સીધી રેખામાં ચાલે છે \\
ઉચ્ચ તીવ્રતા & સાંકડી બીમમાં કેન્દ્રિત ઊર્જા \\
સમાંતરિત & ન્યૂનતમ ફેલાવા સાથે સમાંતર કિરણો \\
\end{longtable}

\end{solutionbox}
\begin{mnemonicbox}
``લેસર પ્રકાશ: એકવર્ણીય, સુસંગત, દિશાત્મક, તીવ્ર''

\end{mnemonicbox}
\subsection*{પ્રશ્ન 5(c) OR [7
ગુણ]}\label{q5c}

\textbf{ઓપ્ટિકલ ફાઈબર કેબલનું બંધારણને વિસ્તારમાં સમજાવો. અને સ્ટેપ ઇન્ડેક્સ અને ગ્રેડેડ
ઇન્ડેક્સ ઓપ્ટિકલ ફાઈબર સમજાવો.}

\begin{solutionbox}

\textbf{ઓપ્ટિકલ ફાઈબર બંધારણ}:

\begin{enumerate}
\tightlist
\item
  \textbf{કોર}: કેન્દ્રીય પ્રકાશ-પ્રસારિત કરનાર ભાગ (કાચ અથવા પ્લાસ્ટિક)
\item
  \textbf{ક્લેડિંગ}: કોરને ઘેરે છે, કોર કરતાં ઓછા વક્રીભવનાંક સાથે
\item
  \textbf{બફર કોટિંગ}: સુરક્ષાત્મક પ્લાસ્ટિક કોટિંગ
\item
  \textbf{જેકેટ}: બાહ્ય સુરક્ષાત્મક આવરણ
\end{enumerate}

\textbf{આકૃતિ: ઓપ્ટિકલ ફાઈબર સ્ટ્રક્ચર}

\begin{verbatim}
      ┌───────────────┐
      │               │  જેકેટ
      │  ┌─────────┐  │
      │  │         │  │  બફર કોટિંગ
      │  │  ┌───┐  │  │
      │  │  │   │  │  │
      │  │  │   │  │  │
      │  │  └───┘  │  │
      │  │    ↑    │  │
      │  └────┼────┘  │
      │       │       │
      └───────┼───────┘
              ↑
             કોર
            ક્લેડિંગ
\end{verbatim}

\textbf{સ્ટેપ ઇન્ડેક્સ ફાઈબર}:

\begin{itemize}
\tightlist
\item
  કોર અને ક્લેડિંગ વચ્ચે વક્રીભવનાંકમાં અચાનક પરિવર્તન
\item
  પ્રકાશ પૂર્ણ આંતરિક પરાવર્તન દ્વારા આડા-અવળા માર્ગમાં પ્રવાસ કરે છે
\item
  ઉચ્ચ મોડલ ડિસ્પર્શન (સિગ્નલ ફેલાવો)
\item
  સરળ બંધારણ
\end{itemize}

\textbf{ગ્રેડેડ ઇન્ડેક્સ ફાઈબર}:

\begin{itemize}
\tightlist
\item
  કોરના કેન્દ્રથી ક્લેડિંગ સુધી વક્રીભવનાંકમાં ક્રમિક પરિવર્તન
\item
  સતત વક્રીભવનને કારણે પ્રકાશ સર્પિલ માર્ગમાં પ્રવાસ કરે છે
\item
  નિમ્ન મોડલ ડિસ્પર્શન
\item
  વધુ જટિલ બંધારણ
\end{itemize}

\textbf{આકૃતિ: સ્ટેપ ઇન્ડેક્સ બનામ ગ્રેડેડ ઇન્ડેક્સ ફાઈબર}

\begin{verbatim}
સ્ટેપ ઇન્ડેક્સ:
       ────────────────────
      /                     {}
     /    ┌────────────┐     {}
    /     │            │      {}
   |      │    કોર      │       |
    {     │            │      /}
     {    └────────────┘     /}
      {      ક્લેડિંગ          /}
       ────────────────────
       
ગ્રેડેડ ઇન્ડેક્સ:
       ─────────────────────
      /                      {}
     /     ┌──────────┐      {}
    /     /            {      }
   |     |     કોર      |      |
    {                 /      /}
     {     └──────────┘      /}
      {      ક્લેડિંગ       /}
       ────────────────────
\end{verbatim}

\textbf{વક્રીભવનાંક પ્રોફાઇલ}:

\begin{verbatim}
સ્ટેપ ઇન્ડેક્સ:           ગ્રેડેડ ઇન્ડેક્સ:
    │                     │
n_{1 ─┤▄▄▄▄▄▄▄              ▄▄▄▄▄}
    │       │            ▄     ▄
    │       │           ▄       ▄
n_{2 ─┤       ▀▀▀▀▀▀▀    ▄         ▄}
    │                  ▀▀▀▀▀▀▀▀▀▀▀
    └─────── r        └─────── r
\end{verbatim}

\end{solutionbox}
\begin{mnemonicbox}
``સ્ટેપ બતાવે અચાનક ફેરફાર, ગ્રેડેડ ધીમે ધીમે ઘટાડે''

\end{mnemonicbox}

\end{document}
