\documentclass[10pt,a4paper]{article}

% content/resources/templates/preamble.tex
\usepackage[margin=0.6in]{geometry}
\author{Milav Dabgar}
\usepackage{amsmath,amssymb,amsthm}
\usepackage{booktabs}
\usepackage{multirow}
\usepackage{xcolor}
\usepackage{tcolorbox}
\tcbuselibrary{breakable,skins}
\usepackage[colorlinks=true,linkcolor=blue]{hyperref}
\usepackage{titlesec}
\usepackage{enumitem}
\usepackage{tikz}
\usepackage{pgfplots}
\usepackage{circuitikz}
\usepackage[version=4]{mhchem}
\usepackage{longtable}
\usepackage{array}
\usepackage{float}
\usepackage{caption}
\usepackage{listings}

\lstset{
  basicstyle=\small\ttfamily,
  breaklines=true,
  breakatwhitespace=false,
  postbreak=\mbox{\textcolor{red}{$\hookrightarrow$}\space},
  float=false,
  numbers=left,
  numberstyle=\tiny\color{gray},
  numbersep=10pt,
  xleftmargin=2em,
  keywordstyle=\color{blue},
  commentstyle=\color{green!60!black},
  stringstyle=\color{purple},
  backgroundcolor=\color{gray!5},
  showstringspaces=false,
  tabsize=2,
  captionpos=b,
  keepspaces=true,
  columns=flexible
}

\pgfplotsset{compat=1.18}
\usetikzlibrary{shapes,arrows,positioning,calc,patterns,decorations.pathmorphing,decorations.markings,arrows.meta}

% Color scheme
\definecolor{headcolor}{RGB}{0,102,204}
\definecolor{keycolor}{RGB}{220,20,60}
\definecolor{solutioncolor}{RGB}{34,139,34}
\definecolor{mnemoniccolor}{RGB}{148,0,211}
\definecolor{codecolor}{RGB}{0,0,100}

% Spacing
\setlength{\parskip}{3pt}
\setlist[itemize]{nosep}
\setlist[enumerate]{nosep}

% Title formatting
\titleformat{\section}{\Large\bfseries\color{headcolor}}{\thesection}{1em}{}
\titleformat{\subsection}{\large\bfseries\color{headcolor}}{\thesubsection}{1em}{}

% Pandoc tightlist compatibility
\providecommand{\tightlist}{%
  \setlength{\itemsep}{0pt}\setlength{\parskip}{0pt}}

% Pandoc longtable compatibility
\newcounter{none}
\def\thenone{}


% content/resources/templates/english-boxes.tex
% This file is currently empty - it exists to maintain consistency with the import structure.
% Add custom environments here if needed in the future.


\begin{document}

\begin{center}
{\Huge\bfseries\color{headcolor} Communication Skills Solutions}\\[5pt]
{\LARGE 4300002 -- Summer 2024}\\[3pt]
{\large Semester 1 Study Material}\\[3pt]
{\normalsize\textit{Detailed Solutions and Explanations}}
\end{center}

\vspace{10pt}

\subsection*{Question 1(a) [3 marks]}\label{q1a}

\textbf{Select (✓) the most appropriate option from the given options.
(Any 3)}

\begin{enumerate}
\item
  The sign of successful communication is when \_\_\_\_\_\_\_\_\_.
\begin{solutionbox}
d.~The Sender gets the desired response from the
  receiver.
\item
  The process of converting signal (coded message) into understanding
\end{solutionbox}
\begin{solutionbox}
is\ldots{}  b. Decoding
\item
  Use of Tone, Stress, and Intonation of one's voice in Communication
\end{solutionbox}
\begin{solutionbox}
is\ldots{}  c.~Paralanguage
\item
  David Berlo's \_\_\_\_\_\_\_\_ Model is an expansion of the
\end{solutionbox}
\begin{solutionbox}
Shannon-Weaver Model of Communication.  b. SMCR:
  Sender Message Channel Receiver
\end{enumerate}

\end{solutionbox}
\subsection*{Question 1(b) [4 marks]}\label{q1b}

\textbf{Justify Fate and Friendship vs.~Duty as the Central Themes of
the Story ``After Twenty Years''.}

\begin{solutionbox}
In ``After Twenty Years'' by O. Henry, fate brings two
friends to a fateful encounter where duty conflicts with friendship. Bob
waits for Jimmy after 20 years, unaware that his friend is now a
policeman who must arrest him.


\vspace{-5pt}
\captionof{table}{Themes in ``After Twenty Years''}
\vspace{-10pt}
\begin{longtable}[]{@{}ll@{}}
\toprule\noalign{}
Theme & Evidence \\
\midrule\noalign{}
\endhead
\bottomrule\noalign{}
\endlastfoot
Friendship & Bob travels miles to keep a 20-year promise \\
Fate & Ironically reunites friends as criminal and cop \\
Duty & Jimmy chooses legal obligation over friendship \\
Conflict & Personal loyalty versus professional responsibility \\
\end{longtable}

\end{solutionbox}
\begin{mnemonicbox}
``Friends Face Duty's Divide''

\end{mnemonicbox}
\subsection*{Question 1(c) [7 marks]}\label{q1c}

\textbf{Write a Brief Note in about 120 words on the following. (Any
Two)}

\subsubsection{1. Shannon-Weaver Model of Communication Process (Explain
with
Diagram)}\label{shannon-weaver-model-of-communication-process-explain-with-diagram}

\begin{solutionbox}
The Shannon-Weaver Model explains how information flows
from sender to receiver through a communication channel.

\textbf{Diagram:}

\begin{center}
\textbf{Mermaid Diagram (Code)}
\begin{verbatim}
{Shaded}
{Highlighting}[]
graph LR
    A[Information Source] {-{-}{} B[Encoder/Transmitter]}
    B {-{-}{} C[Channel]}
    C {-{-}{} D[Decoder/Receiver]}
    D {-{-}{} E[Destination]}
    F[Noise Source] {-{-}{} C}
{Highlighting}
{Shaded}
\end{verbatim}
\end{center}

\begin{itemize}
\tightlist
\item
  \textbf{Information Source}: Creates and decides what message to send
\item
  \textbf{Encoder}: Converts message into signals or code
\item
  \textbf{Channel}: Medium through which message travels
\item
  \textbf{Decoder}: Converts signals back into understandable message
\item
  \textbf{Destination}: Person receiving the message
\item
  \textbf{Noise}: Any interference disrupting message transmission
\end{itemize}

\end{solutionbox}
\begin{mnemonicbox}
``Send Encode Channel Decode Receive''

\end{mnemonicbox}
\subsubsection{2. Communication: Definition, Need, and Application at
Workplace}\label{communication-definition-need-and-application-at-workplace}

\begin{solutionbox}
Communication is the exchange of information, ideas,
and feelings between individuals.


\vspace{-5pt}
\captionof{table}{Communication Essentials}
\vspace{-10pt}
\begin{longtable}[]{@{}
  >{\raggedright\arraybackslash}p{(\linewidth - 2\tabcolsep) * \real{0.4706}}
  >{\raggedright\arraybackslash}p{(\linewidth - 2\tabcolsep) * \real{0.5294}}@{}}
\toprule\noalign{}
\begin{minipage}[b]{\linewidth}\raggedright
Aspect
\end{minipage} & \begin{minipage}[b]{\linewidth}\raggedright
Details
\end{minipage} \\
\midrule\noalign{}
\endhead
\bottomrule\noalign{}
\endlastfoot
Definition & Process of sharing information, ideas, and emotions \\
Need & Enables coordination, problem-solving, and
relationship-building \\
Workplace Applications & Team collaboration, customer service, conflict
resolution \\
\end{longtable}

\begin{itemize}
\tightlist
\item
  \textbf{Essential Need}: Required for instruction, feedback, and
  coordination
\item
  \textbf{Workplace Benefits}: Increases productivity, improves employee
  engagement
\item
  \textbf{Forms Used}: Verbal, written, digital, and non-verbal
  communications
\end{itemize}

\end{solutionbox}
\begin{mnemonicbox}
``Share, Connect, Achieve''

\end{mnemonicbox}
\subsubsection{3. Barriers to Communication with
Illustrations}\label{barriers-to-communication-with-illustrations}

\begin{solutionbox}
Communication barriers prevent effective exchange of
messages between sender and receiver.


\vspace{-5pt}
\captionof{table}{Types of Communication Barriers}
\vspace{-10pt}
\begin{longtable}[]{@{}ll@{}}
\toprule\noalign{}
Barrier Type & Examples \\
\midrule\noalign{}
\endhead
\bottomrule\noalign{}
\endlastfoot
Physical & Distance, noise, poor technology connection \\
Psychological & Bias, lack of attention, emotional state \\
Language & Jargon, ambiguity, poor translation \\
Cultural & Differing values, norms, and customs \\
\end{longtable}

\begin{itemize}
\tightlist
\item
  \textbf{Physical Barriers}: Noisy room preventing clear hearing
\item
  \textbf{Psychological Barriers}: Preconceived notions affecting
  understanding
\item
  \textbf{Language Barriers}: Technical terms unknown to the receiver
\item
  \textbf{Cultural Barriers}: Different gestures meaning opposite things
  in two cultures
\end{itemize}

\end{solutionbox}
\begin{mnemonicbox}
``PLCS: Physical, Language, Cultural, State-of-mind''

\end{mnemonicbox}
\subsection*{Alternative Question 1(c) [7
marks]}\label{q1c}

\begin{solutionbox}
\textbf{Answer the following questions in one or two sentences. (Any
Seven)}

\begin{enumerate}
\item
  \textbf{What is Encoding in the Process of Communication?} Converting
  thoughts/ideas into symbols, words, or gestures that the receiver can
  understand.
\item
  \textbf{Define Decoding in the Process of Communication.} The process
  where the receiver interprets the message and converts it into
  meaningful information.
\item
  \textbf{Why is Feedback essential for a successful Communication?}
  Feedback confirms whether the message was correctly understood and
  allows the sender to adjust communication if necessary.
\item
  \textbf{Which type of Communication is more effective? Verbal or
  Non-Verbal?} Non-verbal is often more effective as it conveys emotions
  and attitudes that may not be expressed verbally.
\item
  \textbf{How does Non-Verbal Communication supplement Verbal
  Communication?} It reinforces, contradicts, substitutes, complements,
  or accents the verbal message, adding layers of meaning.
\item
  \textbf{State the components of Paralanguage serving the purpose of
  communication.} Tone, pitch, volume, rate, quality of voice, and vocal
  fillers like ``um'' or ``ah.''
\item
  \textbf{In which form/s can Visual Communication be represented?}
  Charts, graphs, maps, photographs, videos, signs, symbols, and
  illustrations.
\item
  \textbf{Explain any two Barriers to Communication.} Physical barriers
  include noise and distance; psychological barriers include prejudice
  and emotional state.
\end{enumerate}

\end{solutionbox}
\subsection*{Question 2(a) [3 marks]}\label{q2a}

\textbf{Identify Noun/s from the following sentences.}

\begin{enumerate}
\tightlist
\item
  The old \textbf{man} is known for his \textbf{wisdom}.
\item
  \textbf{Kritika} bought a \textbf{handbag} for \textbf{herself}.
\item
  \textbf{Aryan} was scolded for his \textbf{forgetfulness}.
\end{enumerate}

\subsection*{Question 2(b) [4 marks]}\label{q2b}

\textbf{Do as directed.}

\begin{enumerate}
\item
  The Examination of Communication Skills in English was \textbf{quite}
  easy. (Apply a suitable Adverb from quiet, quite, quietly and Rewrite
  the Sentence.)
\item
  \textbf{Wow}! That was truly an exquisite performance! (Apply a
  suitable Interjection from Wow, Oh, Ouch and Rewrite the Sentence.)
\item
  The place is a Seven-Star Resort \textbf{where} celebrities are
  staying. (Join these two sentences using a suitable Conjunction from
  Which, Where, When and Rewrite the revised Sentence.)
\item
  The \textbf{wealthy} woman bought \textbf{diamond} jewellery. (Rewrite
  the Sentence and Underline Adjective/s.)
\end{enumerate}

\subsection*{Question 2(c) [7 marks]}\label{q2c}

\textbf{Fill in the blanks using the appropriate form of the verbs given
in brackets.}

\begin{enumerate}
\item
  \textbf{Did} she \textbf{have} dinner last night? (Do\ldots have,
  Does\ldots.have, Did\ldots.have)
\item
  I \textbf{have been} to the Statue of Unity many times. (have been,
  has been, had been)
\item
  Meera got three calls from her friend, while she \textbf{was having}
  dinner with her family last night. (is doing, was having, were doing)
\item
  Mrs.~Dhingra \textbf{is talking} on the phone at this moment. (has
  been talking, have been talking, is talking)
\item
  The roads are completely wet as it \textbf{has been raining} since
  morning. (had rained, has rained, has been raining)
\item
  Some students \textbf{have paid} already their Tuition fees in time.
  (has\ldots paid, have\ldots paid, have been paying)
\item
  When we \textbf{arrived} at the stadium, the match \textbf{had already
  begun}. (arrive, have begun; arrived, had already began; arrived, had
  already begun)
\end{enumerate}

\subsection*{Alternative Question 2(a) [3
marks]}\label{q2a}

\textbf{Fill in the blanks with suitable Pronoun/s.}

\begin{enumerate}
\item
  That purse is mine. I opened \textbf{it} to see if there was any money
  inside. (it's, it, its)
\item
  Vishala and Viral decided that \textbf{they} would go on a trip to
  Shimla. (she, he, they)
\item
  Mother baked the cookies \textbf{herself}. (itself, herself,
  themselves)
\end{enumerate}

\subsection*{Alternative Question 2(b) [4
marks]}\label{q2b}

\textbf{Do as directed.}

\begin{enumerate}
\item
  Rahul has been to Dubai just \textbf{once}. (Apply a suitable Adverb
  from ``one, once, or ones'' and Rewrite the Sentence.)
\item
  We will go for sightseeing tomorrow \textbf{unless} it rains. (Apply a
  suitable Conjunction from ``if, otherwise, unless'' and Rewrite the
  Sentence.)
\item
  The modest don't boast \textbf{of} their achievements. (Apply a
  suitable Preposition from ``at, of, for'' and Rewrite the Sentence.)
\item
  \textbf{Four} cats ran into the backyard. (Underline Adjective/s.)
\end{enumerate}

\subsection*{Alternative Question 2(c) [7
marks]}\label{q2c}

\textbf{Fill in the blanks using the appropriate form of the verbs given
in brackets.}

\begin{enumerate}
\item
  Ritu \textbf{has been suffering} from Insomnia since October 2023.
  (has suffered, have been suffering, has been suffering)
\item
  Look! The young ones of langurs \textbf{are wrestling} like boys. (is
  wrestling, has wrestled, are wrestling)
\item
  Meera \textbf{is looking for} a job nowadays. (is looking, was looking
  for, is looking for)
\item
  He usually \textbf{takes} tea, but today he \textbf{is drinking}
  coffee. (is taking\ldots is drinking, takes\ldots is drinking,
  took\ldots drank)
\item
  I \textbf{had never seen} such a beautiful beach before I went to
  Miami. (had\ldots saw, has\ldots been seen, had\ldots seen)
\item
  India \textbf{will become} a developed country by 2047. (was, will
  become, is)
\item
  The lights suddenly went off, while we \textbf{were playing} carrom
  yesterday. (have played, were playing, are playing)
\end{enumerate}

\subsection*{Question 3(a) [3 marks]}\label{q3a}

\textbf{Identify the sentence pattern of the sentences given below. (Any
Three)}

\begin{enumerate}
\item
\begin{solutionbox}
They / worked / hard.  Subject + Verb + Adverb
\item
\end{solutionbox}
\begin{solutionbox}
It / was / a very pleasant talk.  Subject + Verb +
  Complement
\item
\end{solutionbox}
\begin{solutionbox}
Many students / witnessed / a Play.  Subject + Verb +
  Object
\item
\end{solutionbox}
\begin{solutionbox}
Leopard / roars.  Subject + Verb
\end{enumerate}

\end{solutionbox}
\subsection*{Question 3(b) [4 marks]}\label{q3b}

\textbf{Fill in the blanks with a suitable Modal Auxiliary. (Any Four)}

\begin{enumerate}
\item
  There are black clouds. It \textbf{may} rain today. (can, may, should)
\item
  The children \textbf{should} obey their parents and teachers. (need,
  could, should)
\item
  \textbf{May} India win the 2027 Cricket World Cup! (can, may)
\item
  Kartik \textbf{must} have attended the meeting. (Use Certainty
  indicating Modal Auxiliary)
\item
  \textbf{Could} you lend me your bike for an hour, please? (Use
  Politeness indicating Modal Auxiliary)
\end{enumerate}

\subsection*{Question 3(c) [7 marks]}\label{q3c}

\textbf{Fill in the blanks using the appropriate form of the verbs. (Any
Seven)}

\begin{enumerate}
\item
  Time and tide \textbf{wait} for none. (wait/waits)
\item
  The director and producer of the movie \textbf{were} present
  yesterday. (was, were)
\item
  Rakesh as well as his friends \textbf{is} invited to the party. (is,
  are)
\item
  Neither of the Teams \textbf{has} performed their best in IPL. (has,
  have)
\item
  As the guests ate much of the Ice cream, a little \textbf{was} left
  for the kids. (was, were)
\item
  The problems of today's youth \textbf{are} many. (is, are)
\item
  Lots of food \textbf{is} wasted globally each year. (is, are)
\item
  Each of the parcels \textbf{weighs} 15 kgs. (weigh, weighs)
\end{enumerate}

\subsection*{Alternative Question 3(a) [3
marks]}\label{q3a}

\textbf{Identify the sentence pattern of the sentences given below. (Any
Three)}

\begin{enumerate}
\item
\begin{solutionbox}
She / sings / a song.  Subject + Verb + Object
\item
\end{solutionbox}
\begin{solutionbox}
They / came / suddenly.  Subject + Verb + Adverb
\item
\end{solutionbox}
\begin{solutionbox}
People / cried.  Subject + Verb
\item
\end{solutionbox}
\begin{solutionbox}
We / are / Indians.  Subject + Verb + Complement
\end{enumerate}

\end{solutionbox}
\subsection*{Alternative Question 3(b) [4
marks]}\label{q3b}

\textbf{Fill in the blanks with a suitable Modal Auxiliary. (Any Four)}

\begin{enumerate}
\item
  Rakhi \textbf{had to} keep quiet as the students were reading in the
  next room. (has to, have to, had to)
\item
  \textbf{Would} you lend me a pen, please? (should, will, must)
\item
  My father \textbf{could} climb a tall tree when he was young. (can,
  could)
\item
  One \textbf{must not} speak loudly in the hospital. (Use Prohibition
  indicating Modal Auxiliary)
\item
  You \textbf{need not} worry about her as she is completely recovered
  from illness now. (Use Absence of Necessity indicating Modal
  Auxiliary)
\end{enumerate}

\subsection*{Alternative Question 3(c) [7
marks]}\label{q3c}

\textbf{Fill in the blanks using the appropriate form of the verbs. (Any
Seven)}

\begin{enumerate}
\item
  Walnut Brownie with hot chocolate sauce \textbf{is} my favorite dish.
  (is/are)
\item
  The poet and the statesman \textbf{have} arrived. (has/have)
\item
  Each day and each hour \textbf{brings} us a fresh anxiety.
  (bring/brings)
\item
  Either Kartik or Kritika \textbf{has} eaten all the Wafers. (has/have)
\item
  Neither you all nor your friend \textbf{is} to be blamed. (is/are)
\item
  More than half of the time \textbf{is} over still he hasn't turned up.
  (is, are)
\item
  You as well as I \textbf{am} responsible for our losses. (am, are)
\item
  Plenty of shops \textbf{accept} payments by a credit card. (accept,
  accepts)
\end{enumerate}

\subsection*{Question 4(a) [3 marks]}\label{q4a}

\textbf{Choose the Correct Option: (Any Three)}

\begin{enumerate}
\item
  Pari Tibba/Hill of the Fairies was also known as
\begin{solutionbox}
\_\_\_\_\_\_\_\_\_\_\_.  (d) Burnt Hill
\item
  Bob and Jimmy were born and brought up in \_\_\_\_\_\_\_\_\_\_\_\_\_\_
\end{solutionbox}
\begin{solutionbox}
city of USA.  (c) New York
\item
  \_\_\_\_\_\_\_\_ gives his harness bells a shake to ask if there is
\end{solutionbox}
\begin{solutionbox}
some mistake.  (c) Horse
\item
  According to the poet, humans should work towards\_\_\_\_\_\_\_\_\_\_.
\end{solutionbox}
\begin{solutionbox}
(d) Perfection
\end{enumerate}

\end{solutionbox}
\subsection*{Question 4(b) [4 marks]}\label{q4b}

\begin{solutionbox}
\textbf{Answer the following questions in brief. (20 to 40 Words) (Any
Two)}

\begin{enumerate}
\tightlist
\item
  \textbf{Comment on the gradual change in the behavior of Birds and
  Animals towards the Author in the story ``Leopard''.}
\end{enumerate}

\end{solutionbox}
\begin{solutionbox}
Initially, birds and animals were wary of the author,
fleeing at his approach. Gradually, as he became a regular visitor who
posed no threat, they grew accustomed to his presence. Eventually, they
accepted him as part of their environment, carrying on their natural
activities even when he was nearby.

\begin{enumerate}
\tightlist
\item
  \textbf{Where did Jimmy and Bob have their last dinner? What did they
  promise to each other then?}
\end{enumerate}

\end{solutionbox}
\begin{solutionbox}
Jimmy and Bob had their last dinner at ``Big Joe''
Brady's restaurant. They promised to meet again at the same spot exactly
twenty years later, regardless of their circumstances or distances
traveled, demonstrating their commitment to their friendship.

\begin{enumerate}
\tightlist
\item
  \textbf{Why was Bob under arrest? Why didn't Jimmy himself arrest
  Bob?}
\end{enumerate}

\end{solutionbox}
\begin{solutionbox}
Bob was under arrest because he was a wanted criminal
in Chicago. Jimmy didn't arrest Bob himself because of their friendship;
instead, he sent another officer to make the arrest while leaving a note
explaining the situation, showing his internal conflict between duty and
friendship.

\end{solutionbox}
\subsection*{Question 4(c) [7 marks]}\label{q4c}

\textbf{Write a Brief Note in about 120 words on the following. (Any
Two)}

\subsubsection{1. Author's Two Encounters with the
Leopard}\label{authors-two-encounters-with-the-leopard}

\begin{solutionbox}
In Ruskin Bond's ``Leopard,'' the author has two
significant encounters with the magnificent wild cat.


\vspace{-5pt}
\captionof{table}{The Two Leopard Encounters}
\vspace{-10pt}
\begin{longtable}[]{@{}
  >{\raggedright\arraybackslash}p{(\linewidth - 4\tabcolsep) * \real{0.3333}}
  >{\raggedright\arraybackslash}p{(\linewidth - 4\tabcolsep) * \real{0.3939}}
  >{\raggedright\arraybackslash}p{(\linewidth - 4\tabcolsep) * \real{0.2727}}@{}}
\toprule\noalign{}
\begin{minipage}[b]{\linewidth}\raggedright
Encounter
\end{minipage} & \begin{minipage}[b]{\linewidth}\raggedright
Description
\end{minipage} & \begin{minipage}[b]{\linewidth}\raggedright
Outcome
\end{minipage} \\
\midrule\noalign{}
\endhead
\bottomrule\noalign{}
\endlastfoot
First Meeting & Author spots leopard drinking from stream & Mutual
watchfulness, respect \\
Second Meeting & Leopard appears unexpectedly on path & Gentle retreat,
acknowledgment \\
\end{longtable}

\begin{itemize}
\tightlist
\item
  \textbf{First Encounter}: The author observes the leopard drinking
  from the stream. The animal senses his presence but continues
  drinking, showing a level of comfort
\item
  \textbf{Second Encounter}: The leopard appears on the path ahead of
  the author. They maintain eye contact before the leopard calmly
  retreats into the forest
\end{itemize}

These encounters highlight the delicate relationship between humans and
wildlife, with mutual respect allowing for peaceful coexistence.

\end{solutionbox}
\begin{mnemonicbox}
``Watch, Wait, Withdraw''

\end{mnemonicbox}
\subsubsection{2. Central Idea of the Poem ``Stopping by Woods on a
Snowy
Evening''}\label{central-idea-of-the-poem-stopping-by-woods-on-a-snowy-evening}

\begin{solutionbox}
Robert Frost's poem conveys the tension between
appreciating natural beauty and fulfilling life's obligations.


\vspace{-5pt}
\captionof{table}{Central Ideas in the Poem}
\vspace{-10pt}
\begin{longtable}[]{@{}ll@{}}
\toprule\noalign{}
Theme & Evidence \\
\midrule\noalign{}
\endhead
\bottomrule\noalign{}
\endlastfoot
Natural Beauty & ``Woods fill up with snow'' \\
Momentary Pause & Stopping between woods and frozen lake \\
Duty vs.~Desire & Horse's impatience vs.~speaker's wish to stay \\
Life's Responsibilities & ``Miles to go before I sleep'' \\
\end{longtable}

\begin{itemize}
\tightlist
\item
  \textbf{Appreciation of Beauty}: The speaker pauses to admire the
  serene, snow-filled woods
\item
  \textbf{Life's Duties}: Despite the attraction of natural beauty,
  life's responsibilities call
\item
  \textbf{Deeper Meaning}: The repeated line ``miles to go before I
  sleep'' suggests both literal journey and life's remaining
  responsibilities
\end{itemize}

\end{solutionbox}
\begin{mnemonicbox}
``Pause, Ponder, Proceed''

\end{mnemonicbox}
\subsubsection{3. Freedom envisioned by Tagore in ``Where the Mind is
without
Fear''}\label{freedom-envisioned-by-tagore-in-where-the-mind-is-without-fear}

\begin{solutionbox}
In his poem, Tagore envisions an ideal free India
characterized by intellectual and spiritual freedom.


\vspace{-5pt}
\captionof{table}{Tagore's Vision of Freedom}
\vspace{-10pt}
\begin{longtable}[]{@{}ll@{}}
\toprule\noalign{}
Aspect & Vision \\
\midrule\noalign{}
\endhead
\bottomrule\noalign{}
\endlastfoot
Intellectual & Knowledge without borders, free from fear \\
Social & No divisions based on caste, religion, or region \\
Cultural & Truth-seeking without prejudice \\
National & Dignity, rational thinking, and progress \\
\end{longtable}

\begin{itemize}
\tightlist
\item
  \textbf{Fearless Mind}: Citizens think and express freely without
  intimidation
\item
  \textbf{Knowledge without Barriers}: Learning transcends narrow
  domestic walls
\item
  \textbf{Truth-Seeking}: Words come from depth of truth, not
  superficial sources
\item
  \textbf{Rational Thinking}: Reason guides actions, not superstition or
  blind customs
\end{itemize}

\end{solutionbox}
\begin{mnemonicbox}
``Freedom Through Knowledge, Truth, and Reason''

\end{mnemonicbox}
\subsection*{Alternative Question 4(a) [3
marks]}\label{q4a}

\textbf{Choose the Correct Option: (Any Three)}

\begin{enumerate}
\item
  ``Where the Mind is without Fear'' is written by the Poet
\begin{solutionbox}
\_\_\_\_\_\_\_\_\_\_\_\_\_\_\_\_\_\_.  (c)
  Rabindranath Tagore
\item
\end{solutionbox}
\begin{solutionbox}
What made Bob realize that the Cop wasn't Jimmy?  (d)
  His jaw
\item
  ``Stopping by Woods on a Snowy Evening'' is written by the Poet
\end{solutionbox}
\begin{solutionbox}
\_\_\_\_\_\_\_\_\_\_\_.  (d) Robert Frost
\item
  Apart from the author, \_\_\_\_\_\_\_\_\_\_ was the regular visitor of
\end{solutionbox}
\begin{solutionbox}
the stream.  (b) Forktail
\end{enumerate}

\end{solutionbox}
\subsection*{Alternative Question 4(b) [4
marks]}\label{q4b}

\begin{solutionbox}
\textbf{Answer the following questions in brief. (20 to 40 Words) (Any
Two)}

\begin{enumerate}
\tightlist
\item
  \textbf{Comment on the Author's approach to the Birds and Animals in
  ``Leopard''.}
\end{enumerate}

\end{solutionbox}
\begin{solutionbox}
The author approached wildlife with patience and
respect, never intruding or disturbing their natural behaviors. He
maintained appropriate distance, moved slowly, and avoided direct eye
contact with threatening gestures. This considerate approach earned him
eventual acceptance, allowing him to observe their natural behaviors.

\begin{enumerate}
\tightlist
\item
  \textbf{How does the little horse of the poet react to being stopped
  by the woods? Why?}
\end{enumerate}

\end{solutionbox}
\begin{solutionbox}
The horse shakes its harness bells questioningly,
thinking it unusual to stop without a farmhouse nearby. It's impatient
because it's accustomed to practical journeys with definite
destinations, not pausing to appreciate natural beauty, and senses they
have responsibilities waiting ahead.

\begin{enumerate}
\tightlist
\item
  \textbf{Explicate Tagore's Vision of India when he says ``Where the
  world has not been broken up into fragments by narrow domestic
  walls.''}
\end{enumerate}

\end{solutionbox}
\begin{solutionbox}
Tagore envisions an India free from divisions based on
caste, religion, region, or language. He dreams of a unified nation
where people don't separate themselves with prejudice or discrimination.
These ``narrow domestic walls'' represent artificial social barriers
that prevent national unity and human connection.

\end{solutionbox}
\subsection*{Alternative Question 4(c) [7
marks]}\label{q4c}

\textbf{Write a Short Note in about 120 words on the following. (Any
Two)}

\subsubsection{1. Lessons learnt from the story ``After Twenty
Years''}\label{lessons-learnt-from-the-story-after-twenty-years}

\begin{solutionbox}
O. Henry's ``After Twenty Years'' offers profound life
lessons through its ironic twist.


\vspace{-5pt}
\captionof{table}{Key Lessons from ``After Twenty Years''}
\vspace{-10pt}
\begin{longtable}[]{@{}ll@{}}
\toprule\noalign{}
Lesson & Explanation \\
\midrule\noalign{}
\endhead
\bottomrule\noalign{}
\endlastfoot
Life Changes People & Bob and Jimmy's drastically different paths \\
Duty vs.~Friendship & Jimmy's moral conflict as policeman and friend \\
Irony of Fate & Friends meeting as criminal and cop \\
Honor in Different Forms & Both men keep their promises differently \\
\end{longtable}

\begin{itemize}
\tightlist
\item
  \textbf{Character Evolution}: People change significantly over time,
  sometimes in unexpected directions
\item
  \textbf{Moral Dilemma}: Professional duty may conflict with personal
  relationships
\item
  \textbf{Honor and Integrity}: Jimmy fulfills both his promise (by
  sending a note) and his duty (by arranging the arrest)
\item
  \textbf{Consequences of Choices}: Our decisions shape our destiny,
  leading us to unforeseen circumstances
\end{itemize}

\end{solutionbox}
\begin{mnemonicbox}
``Time Changes Paths, Choices Matter''

\end{mnemonicbox}
\subsubsection{2. `India after Independence' envisaged by Rabindranath
Tagore}\label{india-after-independence-envisaged-by-rabindranath-tagore}

\begin{solutionbox}
Though written before independence, Tagore's poem
outlines his vision for a free India.


\vspace{-5pt}
\captionof{table}{Tagore's Vision for Independent India}
\vspace{-10pt}
\begin{longtable}[]{@{}ll@{}}
\toprule\noalign{}
Aspect & Vision \\
\midrule\noalign{}
\endhead
\bottomrule\noalign{}
\endlastfoot
Freedom & Beyond political - includes intellectual and spiritual \\
Society & Unified, without divisive barriers \\
Knowledge & Constantly expanding, driven by reason \\
Character & Truthful, dignified, and perfectionist \\
\end{longtable}

\begin{itemize}
\tightlist
\item
  \textbf{Intellectual Freedom}: Minds functioning without fear or
  oppression
\item
  \textbf{Social Harmony}: No artificial divisions based on caste,
  religion, or region
\item
  \textbf{Progressive Thinking}: Clear, logical reasoning guiding
  national progress
\item
  \textbf{Moral Character}: Words emerging from truthfulness, actions
  from dignity
\end{itemize}

\end{solutionbox}
\begin{mnemonicbox}
``Free Minds, United People, Progressive Nation''

\end{mnemonicbox}
\subsubsection{3. The Author's strong efforts to find out the Forktail's
nest in
``Leopard''}\label{the-authors-strong-efforts-to-find-out-the-forktails-nest-in-leopard}

\begin{solutionbox}
In ``Leopard,'' the author's search for the forktail's
nest demonstrates his passion for wildlife observation.


\vspace{-5pt}
\captionof{table}{The Forktail Nest Search}
\vspace{-10pt}
\begin{longtable}[]{@{}
  >{\raggedright\arraybackslash}p{(\linewidth - 2\tabcolsep) * \real{0.4706}}
  >{\raggedright\arraybackslash}p{(\linewidth - 2\tabcolsep) * \real{0.5294}}@{}}
\toprule\noalign{}
\begin{minipage}[b]{\linewidth}\raggedright
Aspect
\end{minipage} & \begin{minipage}[b]{\linewidth}\raggedright
Details
\end{minipage} \\
\midrule\noalign{}
\endhead
\bottomrule\noalign{}
\endlastfoot
Bird Characteristics & Small, black and white water bird frequenting
stream \\
Search Methodology & Patient observation, tracking movements \\
Challenges & Forktail's secretive nature, difficult terrain \\
Significance & Represents author's deeper connection with wildlife \\
\end{longtable}

\begin{itemize}
\tightlist
\item
  \textbf{Persistent Observation}: The author regularly visited the
  stream to study the forktail's habits
\item
  \textbf{Methodical Approach}: He followed the bird's flight patterns,
  noting where it disappeared
\item
  \textbf{Challenges Faced}: Dense vegetation and slippery rocks
  complicated the search
\item
  \textbf{Symbolic Meaning}: The quest represents mankind's desire to
  understand nature's secrets
\end{itemize}

\end{solutionbox}
\begin{mnemonicbox}
``Watch, Follow, Discover''

\end{mnemonicbox}
\subsection*{Question 5(a) [3 marks]}\label{q5a}

\textbf{Choose the Correct Option: (Any Three)}

\begin{enumerate}
\item
  The language used in business/formal emails should be\ldots{}
\begin{solutionbox}
(c) professional
\item
  \_\_\_\_\_\_\_\_\_ Email is written in response to the Complaints
\end{solutionbox}
\begin{solutionbox}
raised by the Clients.  (a) Adjusting
\item
  A written letter requesting information on the Product/Material is
\end{solutionbox}
\begin{solutionbox}
called..  (a) Inquiry Letter
\item
  \_\_\_\_\_\_\_\_\_\_\_ is used to send mass emails without disclosing
\end{solutionbox}
\begin{solutionbox}
the email IDs of the recipients.  (c) Bcc
\end{enumerate}

\end{solutionbox}
\subsection*{Question 5(b) [4 marks]}\label{q5b}

\textbf{Do as directed. (Attempt Any One)}

\subsubsection{1. Elucidate the 7 Cs of Business Communication in about
120
words}\label{elucidate-the-7-cs-of-business-communication-in-about-120-words}

\begin{solutionbox}
The 7 Cs framework ensures effective business
communication through essential principles.


\vspace{-5pt}
\captionof{table}{The 7 Cs of Business Communication}
\vspace{-10pt}
\begin{longtable}[]{@{}ll@{}}
\toprule\noalign{}
Principle & Meaning \\
\midrule\noalign{}
\endhead
\bottomrule\noalign{}
\endlastfoot
Clarity & Using simple language with clear purpose \\
Conciseness & Being brief without sacrificing completeness \\
Completeness & Including all necessary information \\
Concreteness & Using specific facts and figures \\
Correctness & Ensuring accuracy in grammar and facts \\
Consideration & Considering audience's perspective \\
Courtesy & Being respectful and thoughtful \\
\end{longtable}

These principles provide a systematic approach for creating messages
that achieve their purpose while building positive relationships.
Following the 7 Cs helps avoid misunderstandings, saves time, and
improves communication effectiveness in professional settings.

\end{solutionbox}
\begin{mnemonicbox}
``Clear, Concise, Complete Communication Creates
Correct Connection''

\end{mnemonicbox}
\subsubsection{2. Write a request letter to the Head of your respective
Department in your College to sanction your leave for a
week}\label{write-a-request-letter-to-the-head-of-your-respective-department-in-your-college-to-sanction-your-leave-for-a-week}

\begin{solutionbox}

\begin{verbatim}
[Your Name]
[Your Class/Roll Number]
[College Name]
[Address]
[Date]

The Head of Department
[Department Name]
[College Name]
[Address]

Subject: Request for One Week Leave

Respected Sir/Madam,

I am writing to request a leave of absence for one week from [start date] to [end date] due to [brief reason - family function/medical treatment/personal emergency].

During my absence, I will ensure that I complete all pending assignments upon my return. I have also arranged with my classmates to share their notes with me so that I don't miss any important lessons.

I would be grateful if you could kindly grant me leave for the mentioned period. I shall report back to college on [date of return].

Thank you for your consideration.

Yours sincerely,

[Your Signature]
[Your Name]
[Roll Number]
\end{verbatim}

\end{solutionbox}
\subsection*{Question 5(c) [7 marks]}\label{q5c}

\textbf{Draft the following Business Email: (Any One)}

\subsubsection{1. HYUNDAI MOTORS LTD Email Inquiry for
Batteries}\label{hyundai-motors-ltd-email-inquiry-for-batteries}

\begin{solutionbox}

\begin{verbatim}
From: purchase@hyundaimotors.com
To: sales@envision-energy.com
Subject: Inquiry for Lithium-ion Batteries (Model No. ID89-Z) for Hyundai Karrier EV

Dear Mr. Bruce Craig,

I am writing on behalf of HYUNDAI MOTORS LTD, Mumbai, India, to inquire about your Lithium-ion Batteries (Model No. ID89-Z).

We are planning to launch our new SUV car 'Hyundai Karrier EV' in the Asian Market on May 01, 2024, and require 20,000 units of your futureproof Lithium-ion Batteries (Model No. ID89-Z).

We would appreciate if you could provide us with the following information:

1. Detailed specifications and features of the batteries
2. Unit price and applicable discounts for bulk orders
3. Delivery timeframe and shipping terms
4. Warranty periods and after-sales service options
5. Payment terms and conditions

Please send us your latest catalog and a quotation for the required quantity at your earliest convenience.

Thank you for your prompt attention to this inquiry. We look forward to your response.

Yours sincerely,

Manoj Nalawade
Purchase Manager
HYUNDAI MOTORS LTD
Mumbai, India
Email: purchase@hyundaimotors.com
Tel: [Phone Number]
\end{verbatim}

\end{solutionbox}
\subsubsection{2. HYUNDAI MOTORS LTD Order Email for
Batteries}\label{hyundai-motors-ltd-order-email-for-batteries}

\begin{solutionbox}

\begin{verbatim}
From: purchase@hyundaimotors.com
To: sales@envision-energy.com
Subject: Purchase Order: 20,000 Lithium-ion Batteries (Model No. ID89-Z)

Dear Mr. Bruce Craig,

Following our earlier communications and your quotation dated [reference date], we would like to place a firm order for:

Item: Lithium-ion Batteries (Model No. ID89-Z)
Quantity: 20,000 units
Unit Price: [Price as per quotation]
Total Value: [Total amount]

These batteries are required for our new SUV car 'Hyundai Karrier EV' which is scheduled to be launched in the Asian Market on May 01, 2024.

Delivery Requirements:
- Required delivery date: On or before February 29, 2024
- Delivery Address: HYUNDAI MOTORS LTD, [Complete Address], Mumbai, India
- Shipping Method: [Preferred shipping method]

Payment Terms:
- As agreed in your quotation [reference details]
- [Any additional payment details]

Please confirm receipt of this order and provide an estimated shipping schedule at your earliest convenience. Also, kindly send us the invoice with complete bank details for payment processing.

We look forward to a successful business relationship.

Yours sincerely,

Manoj Nalawade
Purchase Manager
HYUNDAI MOTORS LTD
Mumbai, India
Email: purchase@hyundaimotors.com
Tel: [Phone Number]

Purchase Order No.: [Order reference number]
\end{verbatim}

\end{solutionbox}
\subsection*{Alternative Question 5(a) [3
marks]}\label{q5a}

\textbf{Choose the Correct Option: (Any Three)}

\begin{enumerate}
\item
  `Dear Sir/Madam' or `Respected Sir/Madam' is called
\begin{solutionbox}
\_\_\_\_\_\_\_\_\_\_\_\_.  (a) Salutation
\item
  Signature is placed \_\_\_\_\_\_\_\_\_\_\_\_\_\_\_\_\_\_\_.
\end{solutionbox}
\begin{solutionbox}
(a) Below the complimentary close
\item
  A written communication used to raise your concerns with a product,
  service or to address other types of grievances is called
\end{solutionbox}
\begin{solutionbox}
\_\_\_\_\_\_\_\_\_\_.  (d) Complaint Letter
\item
  \_\_\_\_\_\_\_\_ refers to any additional documents that you've
\end{solutionbox}
\begin{solutionbox}
attached to your letter.  (c) Enclosure
\end{enumerate}

\end{solutionbox}
\subsection*{Alternative Question 5(b) [4
marks]}\label{q5b}

\textbf{Do as directed. (Attempt Any One)}

\subsubsection{1. Explain the Parts/Format of a Business Letter in about
120
words}\label{explain-the-partsformat-of-a-business-letter-in-about-120-words}

\begin{solutionbox}
A business letter follows a structured format with
specific components arranged in a standard order.


\vspace{-5pt}
\captionof{table}{Parts of a Business Letter}
\vspace{-10pt}
\begin{longtable}[]{@{}ll@{}}
\toprule\noalign{}
Component & Description \\
\midrule\noalign{}
\endhead
\bottomrule\noalign{}
\endlastfoot
Letterhead/Sender's Address & Company information at the top \\
Date & Written below the letterhead \\
Reference (if any) & Letter identification number \\
Inside Address & Recipient's name, designation, and address \\
Salutation & Formal greeting (Dear Sir/Madam) \\
Subject Line & Brief description of letter's purpose \\
Body & Main content divided into paragraphs \\
Complimentary Close & Formal closing phrase (Yours sincerely) \\
Signature & Handwritten signature followed by typed name \\
Designation & Sender's position in the organization \\
Enclosure (if any) & Indication of attached documents \\
\end{longtable}

Each part serves a specific purpose in creating a professional and
complete business communication that adheres to formal standards.

\end{solutionbox}
\begin{mnemonicbox}
``Header, Address, Salutation, Body, Close, Sign''

\end{mnemonicbox}
\subsubsection{2. Place an order through Email to GLOBAL FURNITURE,
Nagpur for Office Furniture for your newly constructed office in SEZ-2,
GIDC,
Ahmedabad}\label{place-an-order-through-email-to-global-furniture-nagpur-for-office-furniture-for-your-newly-constructed-office-in-sez-2-gidc-ahmedabad}

\begin{solutionbox}

\begin{verbatim}
From: [Your Email]
To: sales@globalfurniture.com
Subject: Purchase Order for Office Furniture - PIONEER CONSULTANCY

Dear Sir/Madam,

We are pleased to place an order for office furniture for our newly constructed office in SEZ-2, GIDC, Ahmedabad. The details are as follows:

Order Details:

1. Executive Desks (Model: GF-ED1001) - 5 units
2. Executive Chairs (Model: GF-EC2002) - 5 units
3. Staff Workstations (Model: GF-SWS3003) - 20 units
4. Staff Chairs (Model: GF-SC4004) - 20 units
5. Conference Table (Model: GF-CT5005) - 1 unit
6. Conference Chairs (Model: GF-CC6006) - 12 units
7. File Cabinets (Model: GF-FC7007) - 10 units
8. Reception Desk (Model: GF-RD8008) - 1 unit

Delivery Address:
PIONEER CONSULTANCY
SEZ-2, GIDC, Ahmedabad - [Pincode]
Contact Person: [Name]
Contact Number: [Phone number]

Delivery Timeline: We request delivery within 3 weeks from the date of this order.

Payment Terms: As discussed, we will make 50% advance payment upon order confirmation and the remaining 50% after successful delivery and installation.

Please confirm receipt of this order and provide an estimated delivery date at your earliest convenience.

Thank you for your attention to this matter.

Yours sincerely,

[Your Name]
[Your Designation]
PIONEER CONSULTANCY
Email: [Your Email]
Tel: [Your Phone Number]
\end{verbatim}

\end{solutionbox}
\subsection*{Alternative Question 5(c) [7
marks]}\label{q5c}

\textbf{Draft the following Business Letter: (Any One)}

\subsubsection{1. Draft a complaint letter to GLOBAL FURNITURE, Nagpur
asking for compensation or replacement as you found some of the pieces
of furniture delivered in the damaged
condition}\label{draft-a-complaint-letter-to-global-furniture-nagpur-asking-for-compensation-or-replacement-as-you-found-some-of-the-pieces-of-furniture-delivered-in-the-damaged-condition}

\begin{solutionbox}

\begin{verbatim}
PIONEER CONSULTANCY
SEZ-2, GIDC, Ahmedabad - [Pincode]
Tel: [Phone Number] | Email: [Email Address]

[Date]

The Customer Service Manager
GLOBAL FURNITURE
[Full Address]
Nagpur - [Pincode]

Subject: Complaint Regarding Damaged Furniture Delivery - Order No. [Order Number]

Dear Sir/Madam,

We regret to inform you that we have received damaged furniture items in our recent order (Order No. [Order Number]) delivered on [Delivery Date].

Upon inspection, we found the following items damaged:

1. Executive Desk (Model: GF-ED1001, Serial No. [Number]) - Scratches on the surface and a broken drawer
2. Staff Chairs (Model: GF-SC4004) - 3 units with damaged armrests
3. File Cabinet (Model: GF-FC7007) - Dent on the right side panel

This is particularly disappointing as we needed these items urgently for our newly established office. The damage has caused inconvenience and delayed our office setup.

We request you to:
- Replace the damaged items at the earliest possible date, or
- Offer appropriate compensation/discount for the damaged goods

We have preserved the original packaging and have photographic evidence of the damage, which can be provided if required.

We have been a loyal customer of GLOBAL FURNITURE and expect a prompt resolution to this matter.

Yours sincerely,

[Your Name]
[Your Designation]
PIONEER CONSULTANCY

Enclosure: Photographs of damaged furniture
\end{verbatim}

\end{solutionbox}
\subsubsection{2. GLOBAL FURNITURE, Nagpur has received a complaint from
PIONEER CONSULTANCY, SEZ-2, Ahmedabad regarding some of the pieces of
furniture delivered in the damaged condition. On behalf of GLOBAL
FURNITURE, draft a suitable Adjustment
Letter}\label{global-furniture-nagpur-has-received-a-complaint-from-pioneer-consultancy-sez-2-ahmedabad-regarding-some-of-the-pieces-of-furniture-delivered-in-the-damaged-condition.-on-behalf-of-global-furniture-draft-a-suitable-adjustment-letter}

\begin{solutionbox}

\begin{verbatim}
GLOBAL FURNITURE
[Full Address]
Nagpur - [Pincode]
Tel: [Phone Number] | Email: [Email Address]

[Date]

[Contact Person's Name]
[Designation]
PIONEER CONSULTANCY
SEZ-2, GIDC, Ahmedabad - [Pincode]

Subject: Response to Your Complaint - Order No. [Order Number]

Dear [Contact Person's Name],

Thank you for your letter dated [Complaint Date] regarding the damaged furniture items received in your recent order. We sincerely apologize for the inconvenience caused to you.

At GLOBAL FURNITURE, we take pride in delivering high-quality products to our valued customers, and we regret that we have fallen short of our standards in this instance.

After investigating the matter, we understand that the damage occurred during transportation. Based on your complaint and the evidence provided, we are pleased to offer the following resolution:

1. We will replace all the damaged items (Executive Desk, three Staff Chairs, and File Cabinet) with brand new pieces at no additional cost.
2. The replacement furniture will be delivered to your office within 5 working days.
3. As a goodwill gesture for the inconvenience caused, we are offering a 10% discount on your next order.
4. Our installation team will visit your premises on [Date] to set up the replacement furniture.

Please note that our delivery team will contact you 24 hours before delivery to confirm a convenient time. They will also collect the damaged items during the same visit.

We value your business and are committed to maintaining a long-term relationship with PIONEER CONSULTANCY. Please feel free to contact me directly at [Phone Number] if you have any questions or need further assistance.

Thank you for your understanding and patience.

Yours sincerely,

[Your Name]
Customer Service Manager
GLOBAL FURNITURE

CC: Logistics Department
\end{verbatim}

\end{solutionbox}
\subsection*{Review Questions \& Answers
Summary}\label{review-questions-answers-summary}

Below is a summarized table of key concepts covered in this exam:

\begin{longtable}[]{@{}
  >{\raggedright\arraybackslash}p{(\linewidth - 4\tabcolsep) * \real{0.2250}}
  >{\raggedright\arraybackslash}p{(\linewidth - 4\tabcolsep) * \real{0.3250}}
  >{\raggedright\arraybackslash}p{(\linewidth - 4\tabcolsep) * \real{0.4500}}@{}}
\toprule\noalign{}
\begin{minipage}[b]{\linewidth}\raggedright
Section
\end{minipage} & \begin{minipage}[b]{\linewidth}\raggedright
Key Concepts
\end{minipage} & \begin{minipage}[b]{\linewidth}\raggedright
Important Points
\end{minipage} \\
\midrule\noalign{}
\endhead
\bottomrule\noalign{}
\endlastfoot
Communication Basics & Shannon-Weaver Model, Encoding/Decoding &
Communication requires source, encoder, channel, decoder, receiver \\
Communication Barriers & Physical, Psychological, Language, Cultural &
Awareness helps overcome barriers effectively \\
Grammar & Nouns, Pronouns, Adjectives, Adverbs & Proper use maintains
clarity in communication \\
Tenses & Present, Past, Future forms & Consistency in tense ensures
message clarity \\
Sentence Patterns & S+V, S+V+O, S+V+C, S+V+A & Understanding patterns
improves sentence construction \\
Modal Auxiliaries & Can, Could, May, Might, Should, Must & Express
ability, possibility, permission, obligation \\
Subject-Verb Agreement & Singular/Plural subjects with matching verbs &
Ensures grammatical correctness \\
Literature & ``After Twenty Years,'' ``Leopard,'' poetry themes &
Illustrates communication through literary analysis \\
Business Communication & 7 Cs, Letters, Emails & Formats follow specific
structural guidelines \\
\end{longtable}

\textbf{Study Tips for Weak Students:}

\begin{itemize}
\tightlist
\item
  Focus on identifying parts of speech (nouns, verbs, adjectives) in
  simple sentences
\item
  Practice subject-verb agreement with basic examples
\item
  Memorize the 7 Cs of communication using the mnemonic
\item
  Learn letter/email formats as templates to follow
\item
  Use mnemonics provided to remember key concepts
\item
  Study the tables in this solution for quick revision
\item
  Review sentence patterns using the simple examples given
\end{itemize}

Remember: Communication is about clarity and connection. Focus on
understanding basic principles rather than complex theories.


\end{document}
