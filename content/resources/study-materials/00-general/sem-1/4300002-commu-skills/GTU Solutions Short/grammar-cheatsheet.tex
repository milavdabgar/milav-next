\documentclass[10pt,landscape,a4paper]{article}
\usepackage[utf8]{inputenc}
\usepackage[english]{babel}
\usepackage{multicol}
\usepackage{calc}
\usepackage{ifthen}
\usepackage[landscape]{geometry}
\usepackage{amsmath,amsthm,amsfonts,amssymb}
\usepackage{color,graphicx,overpic}
\usepackage{hyperref}
\usepackage{enumitem}
\usepackage{upquote}
\usepackage{xcolor}
\usepackage{array}
\usepackage{booktabs}
\usepackage{colortbl}

\geometry{top=1cm,left=1cm,right=1cm,bottom=1cm}

% Turn off header and footer
\pagestyle{empty}

% Redefine section commands to use less space
\makeatletter
\renewcommand{\section}{\@startsection{section}{1}{0mm}%
                                {-1ex plus -.5ex minus -.2ex}%
                                {0.5ex plus .2ex}%x
                                {\normalfont\large\bfseries}}
\renewcommand{\subsection}{\@startsection{subsection}{2}{0mm}%
                                {-1explus -.5ex minus -.2ex}%
                                {0.5ex plus .2ex}%
                                {\normalfont\normalsize\bfseries}}
\renewcommand{\subsubsection}{\@startsection{subsubsection}{3}{0mm}%
                                {-1ex plus -.5ex minus -.2ex}%
                                {1ex plus .2ex}%
                                {\normalfont\small\bfseries}}
\makeatother

% Don't print section numbers
\setcounter{secnumdepth}{0}

\setlength{\parindent}{0pt}
\setlength{\parskip}{0pt plus 0.5ex}

% Define colors
\definecolor{myblue}{RGB}{0,102,204}
\definecolor{mygreen}{RGB}{34,139,34}
\definecolor{myred}{RGB}{204,0,0}
\definecolor{mygray}{RGB}{128,128,128}
\definecolor{lightgray}{RGB}{240,240,240}

% Custom box for important information
\newcommand{\importantbox}[1]{%
    \noindent\fcolorbox{myblue}{lightgray}{%
        \parbox{\dimexpr\linewidth-2\fboxsep-2\fboxrule}{#1}%
    }%
}

\newcommand{\examplebox}[1]{%
    \noindent\fcolorbox{mygreen}{white}{%
        \parbox{\dimexpr\linewidth-2\fboxsep-2\fboxrule}{\textcolor{mygreen}{\textbf{Example:}} #1}%
    }%
}

% -----------------------------------------------------------------------

\begin{document}
\raggedright
\footnotesize
\begin{multicols}{3}

% multicol parameters
\setlength{\columnseprule}{0.25pt}
\setlength{\premulticols}{1pt}
\setlength{\postmulticols}{1pt}
\setlength{\multicolsep}{1pt}
\setlength{\columnsep}{2pt}

\begin{center}
    \Large{\textbf{Communication Skills in English}} \\
    \large{\textbf{Grammar Cheatsheet}} \\
    \small{Course Code: 4300002 / DI01000031}
\end{center}

\section{Parts of Speech}

\subsection{1. Noun}
\textbf{Definition:} Name of person, place, thing, or idea.

\textbf{Types:}
\begin{itemize}[leftmargin=*, noitemsep, topsep=0pt]
    \item \textbf{Proper:} Specific names (Ram, India, Monday)
    \item \textbf{Common:} General names (boy, city, day)
    \item \textbf{Collective:} Group (team, family, crowd)
    \item \textbf{Abstract:} Ideas/qualities (honesty, love, wisdom)
    \item \textbf{Material:} Substances (gold, water, wood)
\end{itemize}

\examplebox{The \textbf{teacher} (common) visited \textbf{Delhi} (proper) with her \textbf{team} (collective).}

\subsection{2. Pronoun}
\textbf{Definition:} Word used instead of noun.

\textbf{Types:}
\begin{itemize}[leftmargin=*, noitemsep, topsep=0pt]
    \item \textbf{Personal:} I, we, you, he, she, it, they
    \item \textbf{Possessive:} mine, yours, his, hers, ours, theirs
    \item \textbf{Reflexive:} myself, yourself, himself, themselves
    \item \textbf{Demonstrative:} this, that, these, those
    \item \textbf{Interrogative:} who, whom, whose, which, what
    \item \textbf{Relative:} who, whom, whose, which, that
    \item \textbf{Indefinite:} someone, anyone, everybody, none
\end{itemize}

\examplebox{\textbf{They} themselves admitted the misconduct. (Personal + Reflexive)}

\subsection{3. Verb}
\textbf{Definition:} Action or state of being word.

\textbf{Types:}
\begin{itemize}[leftmargin=*, noitemsep, topsep=0pt]
    \item \textbf{Main Verb:} express action (run, eat, sleep)
    \item \textbf{Auxiliary Verb:} help main verb (is, am, are, was, were, have, has, had, do, does, did)
    \item \textbf{Modal Auxiliary:} can, could, may, might, must, shall, should, will, would, ought to
\end{itemize}

\textbf{Forms:} V1 (base), V2 (past), V3 (past participle), V4 (present participle), V5 (s/es form)

\examplebox{She \textbf{has been studying} (auxiliary + main verb) for three hours.}

\subsection{4. Adjective}
\textbf{Definition:} Describes or modifies noun/pronoun.

\textbf{Types:}
\begin{itemize}[leftmargin=*, noitemsep, topsep=0pt]
    \item \textbf{Quality:} good, bad, beautiful, ugly
    \item \textbf{Quantity:} some, any, much, little, enough
    \item \textbf{Number:} one, two, first, second, many, few
    \item \textbf{Demonstrative:} this, that, these, those
    \item \textbf{Possessive:} my, your, his, her, its, our, their
    \item \textbf{Interrogative:} which, what, whose
\end{itemize}

\textbf{Comparison:}
\begin{itemize}[leftmargin=*, noitemsep, topsep=0pt]
    \item Positive: tall, beautiful, good
    \item Comparative: taller, more beautiful, better
    \item Superlative: tallest, most beautiful, best
\end{itemize}

\examplebox{This is the \textbf{most beautiful} garden in the city.}

\subsection{5. Adverb}
\textbf{Definition:} Modifies verb, adjective, or another adverb.

\textbf{Types:}
\begin{itemize}[leftmargin=*, noitemsep, topsep=0pt]
    \item \textbf{Manner:} quickly, slowly, carefully, well
    \item \textbf{Time:} now, then, today, yesterday, always
    \item \textbf{Place:} here, there, everywhere, outside
    \item \textbf{Frequency:} always, often, sometimes, never
    \item \textbf{Degree:} very, quite, too, enough, almost
\end{itemize}

\examplebox{Usha runs \textbf{fast}. (manner) \\ She \textbf{always} arrives \textbf{early}. (frequency + time)}

\subsection{6. Preposition}
\textbf{Definition:} Shows relationship between noun/pronoun and other words.

\textbf{Common Prepositions:}
\begin{itemize}[leftmargin=*, noitemsep, topsep=0pt]
    \item \textbf{Time:} at, on, in, by, since, for, during
    \item \textbf{Place:} at, on, in, under, above, below, between
    \item \textbf{Direction:} to, from, into, towards, through
    \item \textbf{Others:} with, without, about, of, for, by
\end{itemize}

\importantbox{
\textbf{Time Rules:}
\begin{itemize}[leftmargin=*, noitemsep, topsep=0pt]
    \item \textbf{at} specific time: at 5 PM, at noon
    \item \textbf{on} days/dates: on Monday, on 15th Jan
    \item \textbf{in} months/years/long periods: in March, in 2024
\end{itemize}
}

\examplebox{We are meeting \textbf{at} the cafe. \\ He is good \textbf{at} math. \\ Wait here \textbf{until} I get back.}

\subsection{7. Conjunction}
\textbf{Definition:} Joins words, phrases, or clauses.

\textbf{Types:}
\begin{itemize}[leftmargin=*, noitemsep, topsep=0pt]
    \item \textbf{Coordinating:} and, but, or, nor, for, so, yet
    \item \textbf{Subordinating:} because, since, if, unless, although, while, when, before, after, until
    \item \textbf{Correlative:} either...or, neither...nor, both...and, not only...but also
\end{itemize}

\examplebox{It was raining, \textbf{so} we turned back. \\ \textbf{Although} I was tired, I finished the work. \\ \textbf{Because} he was late, he missed the bus.}

\subsection{8. Interjection}
\textbf{Definition:} Expresses sudden emotion or feeling.

\textbf{Examples:} Wow! Alas! Hurrah! Oh! Ouch! Bravo! Ah!

\examplebox{Wow! What a beautiful view! \\ Alas! He failed in the exam.}

\section{Tenses}

\subsection{Present Tenses}

\subsubsection{1. Simple Present}
\textbf{Form:} V1 / V1+s/es

\textbf{Usage:}
\begin{itemize}[leftmargin=*, noitemsep, topsep=0pt]
    \item Habitual actions: I wake up at 6 AM.
    \item Universal truths: The sun rises in the east.
    \item Scheduled future: The train leaves at 5 PM.
\end{itemize}

\textbf{Time words:} always, usually, often, sometimes, rarely, never, every day/week/month

\examplebox{
\textbf{(+)} Whenever we meet, we plan a trip. \\
\textbf{(-)} He does not play cricket. \\
\textbf{(?)} Do you speak English?
}

\subsubsection{2. Present Continuous}
\textbf{Form:} is/am/are + V4 (ing)

\textbf{Usage:}
\begin{itemize}[leftmargin=*, noitemsep, topsep=0pt]
    \item Action happening now: She is reading a book.
    \item Temporary action: I am staying with my friend.
    \item Future plan: We are meeting tomorrow.
\end{itemize}

\textbf{Time words:} now, at present, currently, at the moment, today

\examplebox{
\textbf{(+)} It is raining outside now. \\
\textbf{(-)} They are not working today. \\
\textbf{(?)} What are you doing?
}

\subsubsection{3. Present Perfect}
\textbf{Form:} has/have + V3

\textbf{Usage:}
\begin{itemize}[leftmargin=*, noitemsep, topsep=0pt]
    \item Completed action (time not important): I have finished my work.
    \item Past action with present result: She has lost her key.
    \item Experience: Have you ever visited Paris?
\end{itemize}

\textbf{Time words:} just, already, yet, ever, never, recently, lately, so far, until now

\examplebox{
\textbf{(+)} Hello Samay, I haven't seen you for ages. \\
\textbf{(-)} She hasn't arrived yet. \\
\textbf{(?)} Have you finished your homework?
}

\subsubsection{4. Present Perfect Continuous}
\textbf{Form:} has/have + been + V4 (ing)

\textbf{Usage:}
\begin{itemize}[leftmargin=*, noitemsep, topsep=0pt]
    \item Action started in past, still continuing
    \item Emphasize duration
\end{itemize}

\textbf{Time words:} for, since, all day/week/month, how long

\examplebox{
\textbf{(+)} They have been living in Switzerland for seven years. \\
\textbf{(-)} I haven't been feeling well recently. \\
\textbf{(?)} How long have you been waiting?
}

\subsection{Past Tenses}

\subsubsection{1. Simple Past}
\textbf{Form:} V2

\textbf{Usage:}
\begin{itemize}[leftmargin=*, noitemsep, topsep=0pt]
    \item Completed action in past: I went to school yesterday.
    \item Past habit: He always carried an umbrella.
    \item Historical fact: Gandhi fought for independence.
\end{itemize}

\textbf{Time words:} yesterday, last week/month/year, ago, in 2020, then

\examplebox{
\textbf{(+)} Shikhar Dhawan scored a century in the last match. \\
\textbf{(+)} Who invented the computer? \\
\textbf{(-)} I didn't see him yesterday.
}

\subsubsection{2. Past Continuous}
\textbf{Form:} was/were + V4 (ing)

\textbf{Usage:}
\begin{itemize}[leftmargin=*, noitemsep, topsep=0pt]
    \item Action in progress at specific time in past
    \item Two actions happening simultaneously
    \item Interrupted action
\end{itemize}

\textbf{Time words:} when, while, as, at that time, at 5 PM yesterday

\examplebox{
\textbf{(+)} Vijay was waiting for me when I arrived. \\
\textbf{(+)} Yesterday evening the phone rang three times while we were having dinner. \\
\textbf{(-)} She wasn't listening to music.
}

\subsubsection{3. Past Perfect}
\textbf{Form:} had + V3

\textbf{Usage:}
\begin{itemize}[leftmargin=*, noitemsep, topsep=0pt]
    \item Action completed before another past action
    \item Past of past
\end{itemize}

\textbf{Time words:} before, after, already, just, never, by the time

\examplebox{
\textbf{(+)} Had you ever visited China before your trip in 2006? \\
\textbf{(+)} When I reached the station, the train had already left. \\
\textbf{(-)} She hadn't finished her work before the deadline.
}

\subsection{Future Tenses}

\subsubsection{1. Simple Future}
\textbf{Form:} will/shall + V1

\textbf{Usage:}
\begin{itemize}[leftmargin=*, noitemsep, topsep=0pt]
    \item Future prediction: It will rain tomorrow.
    \item Spontaneous decision: I'll help you with that.
    \item Promise: I will be there on time.
\end{itemize}

\textbf{Time words:} tomorrow, next week/month/year, soon, later, in future

\examplebox{
\textbf{(+)} Tomorrow will be a holiday. \\
\textbf{(+)} Will you lend me a pen, please? \\
\textbf{(-)} I won't attend the party.
}

\importantbox{
\textbf{Tense Quick Reference:}
\begin{tabular}{@{}ll@{}}
\textbf{Present:} & V1/V1+s/es \\
\textbf{Present Cont.:} & is/am/are + V4 \\
\textbf{Present Perfect:} & has/have + V3 \\
\textbf{Present Perf. Cont.:} & has/have + been + V4 \\
\textbf{Past:} & V2 \\
\textbf{Past Cont.:} & was/were + V4 \\
\textbf{Past Perfect:} & had + V3 \\
\textbf{Future:} & will/shall + V1 \\
\end{tabular}
}

\section{Modal Auxiliaries}

\subsection{Can / Could}
\textbf{Usage:}
\begin{itemize}[leftmargin=*, noitemsep, topsep=0pt]
    \item \textbf{Can:} Ability, permission, possibility (present)
    \item \textbf{Could:} Ability (past), polite request, possibility
\end{itemize}

\examplebox{
I \textbf{can} swim. (ability) \\
\textbf{Can} I use your phone? (permission) \\
\textbf{Could} you lend me your scooter, please? (polite request) \\
She \textbf{could} run fast when she was young. (past ability)
}

\subsection{May / Might}
\textbf{Usage:}
\begin{itemize}[leftmargin=*, noitemsep, topsep=0pt]
    \item \textbf{May:} Permission, possibility (50\%)
    \item \textbf{Might:} Less possibility (30-40\%), polite suggestion
\end{itemize}

\examplebox{
\textbf{May} I come in? (permission) \\
She \textbf{may} come tomorrow. (possibility) \\
\textbf{May} God bless you! (wish) \\
It \textbf{might} rain today. (less possibility)
}

\subsection{Must}
\textbf{Usage:}
\begin{itemize}[leftmargin=*, noitemsep, topsep=0pt]
    \item Strong obligation/necessity
    \item Prohibition (must not)
    \item Logical conclusion/certainty
\end{itemize}

\examplebox{
You \textbf{must} wear a helmet. (obligation) \\
You \textbf{must not} speak loudly in the hospital. (prohibition) \\
He \textbf{must} be sick; he didn't come to school. (conclusion)
}

\subsection{Should / Ought to}
\textbf{Usage:}
\begin{itemize}[leftmargin=*, noitemsep, topsep=0pt]
    \item Advice, recommendation
    \item Moral obligation
    \item Expected situation
\end{itemize}

\examplebox{
We \textbf{should} keep promises. (moral obligation) \\
You \textbf{should} exercise daily. (advice) \\
We \textbf{should} honour our parents. (moral duty) \\
A patient \textbf{should} follow the doctor's advice. (recommendation)
}

\subsection{Will / Would}
\textbf{Usage:}
\begin{itemize}[leftmargin=*, noitemsep, topsep=0pt]
    \item \textbf{Will:} Future action, promise, request
    \item \textbf{Would:} Polite request, past habit, conditional
\end{itemize}

\examplebox{
I \textbf{will} help you. (promise) \\
\textbf{Will} you close the door? (request) \\
\textbf{Would} you like some tea? (polite offer) \\
He \textbf{would} often visit us. (past habit)
}

\subsection{Shall}
\textbf{Usage:}
\begin{itemize}[leftmargin=*, noitemsep, topsep=0pt]
    \item Future (with I/we)
    \item Suggestion/offer
    \item Formal obligation
\end{itemize}

\examplebox{
\textbf{Shall} I open the window? (offer) \\
\textbf{Shall} we go for a walk? (suggestion) \\
We \textbf{shall} overcome. (determination)
}

\subsection{Need}
\textbf{Usage:}
\begin{itemize}[leftmargin=*, noitemsep, topsep=0pt]
    \item Necessity/requirement
    \item Used in negative/interrogative
\end{itemize}

\examplebox{
You \textbf{need} to work hard. (necessity) \\
\textbf{Need} I come tomorrow? (question) \\
You \textbf{needn't} worry. (no necessity)
}

\importantbox{
\textbf{Modal Quick Reference:}
\begin{tabular}{@{}ll@{}}
\textbf{Can/Could} & Ability, Permission \\
\textbf{May/Might} & Possibility, Permission \\
\textbf{Must} & Obligation, Prohibition \\
\textbf{Should/Ought to} & Advice, Duty \\
\textbf{Will/Would} & Future, Request \\
\textbf{Shall} & Offer, Suggestion \\
\textbf{Need} & Necessity \\
\end{tabular}
}

\section{Subject-Verb Agreement}

\subsection{Basic Rules}

\importantbox{
\textbf{Golden Rule:} Singular subject takes singular verb; Plural subject takes plural verb.
}

\subsubsection{Rule 1: Singular Subjects}
\textbf{He/She/It/Singular noun + Singular Verb (V1+s/es)}

\examplebox{
She \textbf{sings} beautifully. \\
The book \textbf{is} on the table. \\
Each of the boxes \textbf{weighs} 10 kgs.
}

\subsubsection{Rule 2: Plural Subjects}
\textbf{They/We/You/Plural noun + Plural Verb (V1)}

\examplebox{
They \textbf{play} cricket every day. \\
The students \textbf{are} in the classroom. \\
None of them \textbf{attend} to their work these days.
}

\subsubsection{Rule 3: Compound Subjects with 'and'}
\textbf{Subject 1 + and + Subject 2 = Plural verb}

\examplebox{
Ram and Shyam \textbf{are} friends. \\
The secretary and the member \textbf{have} come to visit.
}

\textbf{Exception:} When two subjects form one unit or refer to same person:

\examplebox{
Bread and butter \textbf{is} the primary need. \\
My uncle and guide \textbf{is} my best friend. (same person) \\
Apple pie and custard \textbf{is} my favourite dish. (one dish)
}

\subsubsection{Rule 4: Subjects joined by 'or', 'nor', 'either...or', 'neither...nor'}
\textbf{Verb agrees with nearest subject}

\examplebox{
Either the teacher or the students \textbf{are} responsible. \\
Neither the students nor the teacher \textbf{is} present.
}

\subsubsection{Rule 5: Collective Nouns}
\textbf{Usually take singular verb}

\examplebox{
The team \textbf{is} playing well. \\
The committee \textbf{has} decided. \\
The family \textbf{is} very large.
}

\subsubsection{Rule 6: Each, Every, Everyone, Everybody}
\textbf{Always singular verb}

\examplebox{
Each student \textbf{has} a book. \\
Everyone \textbf{is} ready. \\
Every boy and girl \textbf{is} present.
}

\subsubsection{Rule 7: Some, All, Most, None}
\textbf{Depends on noun that follows}

\examplebox{
Some of the water \textbf{is} polluted. (uncountable) \\
Some of the students \textbf{are} absent. (countable) \\
All of the money \textbf{is} gone. \\
All of the books \textbf{are} new.
}

\subsubsection{Rule 8: Phrases between Subject and Verb}
\textbf{Ignore phrases; verb agrees with main subject}

\examplebox{
The deputy along with thirty miners \textbf{was} killed. \\
The book on the table \textbf{is} mine. \\
One of the students \textbf{is} absent.
}

\subsubsection{Rule 9: There is/are}
\textbf{Verb agrees with noun that follows}

\examplebox{
There \textbf{is} a book on the table. (singular) \\
There \textbf{are} books on the table. (plural)
}

\subsubsection{Rule 10: Time, Money, Distance}
\textbf{Singular verb when considered as single unit}

\examplebox{
Ten rupees \textbf{is} not enough. \\
Two hours \textbf{is} a long time. \\
Five kilometers \textbf{is} a short distance.
}

\importantbox{
\textbf{Tricky Words Always Singular:}
\begin{itemize}[leftmargin=*, noitemsep, topsep=0pt]
    \item Each, every, either, neither
    \item Everyone, everybody, someone, somebody
    \item Anyone, anybody, no one, nobody
    \item Nothing, everything, something
\end{itemize}
}

\section{Sentence Patterns}

\subsection{Pattern 1: SV (Subject + Verb)}
\textbf{Intransitive verbs (no object needed)}

\examplebox{
People \textbf{cried}. \\
Birds \textbf{fly}. \\
The sun \textbf{rises}.
}

\subsection{Pattern 2: SVO (Subject + Verb + Object)}
\textbf{Transitive verbs (require object)}

\examplebox{
She \textbf{sings} a song. \\
I \textbf{read} books. \\
He \textbf{plays} cricket.
}

\subsection{Pattern 3: SVA (Subject + Verb + Adverb)}
\textbf{Verb modified by adverb}

\examplebox{
Lata \textbf{sang} sweetly. \\
He \textbf{drives} carefully. \\
They \textbf{worked} hard.
}

\subsection{Pattern 4: SVC (Subject + Verb + Complement)}
\textbf{Linking verbs (be, become, seem, appear, look, feel)}

\examplebox{
You \textbf{are} intelligent. \\
She \textbf{became} a teacher. \\
The food \textbf{smells} good.
}

\subsection{Pattern 5: SVOO (Subject + Verb + Indirect Object + Direct Object)}

\examplebox{
I \textbf{gave} him a book. \\
She \textbf{told} me the truth. \\
He \textbf{bought} her a gift.
}

\subsection{Pattern 6: SVOC (Subject + Verb + Object + Complement)}

\examplebox{
We \textbf{elected} him president. \\
They \textbf{found} the movie boring. \\
I \textbf{consider} him honest.
}

\importantbox{
\textbf{Sentence Pattern Summary:}
\begin{tabular}{@{}ll@{}}
\textbf{SV} & Birds fly \\
\textbf{SVO} & I read books \\
\textbf{SVA} & She sings beautifully \\
\textbf{SVC} & He is happy \\
\textbf{SVOO} & I gave him a pen \\
\textbf{SVOC} & We made him captain \\
\end{tabular}
}

\section{Common Errors to Avoid}

\subsection{1. Double Negatives}
\textcolor{myred}{\textbf{Wrong:}} I don't have nothing. \\
\textcolor{mygreen}{\textbf{Right:}} I don't have anything. / I have nothing.

\subsection{2. Incorrect Tense Sequence}
\textcolor{myred}{\textbf{Wrong:}} When I will come, I call you. \\
\textcolor{mygreen}{\textbf{Right:}} When I come, I will call you.

\subsection{3. Misplaced Modifiers}
\textcolor{myred}{\textbf{Wrong:}} I only have five rupees. (suggests that's all you do) \\
\textcolor{mygreen}{\textbf{Right:}} I have only five rupees.

\subsection{4. Incorrect Prepositions}
\textcolor{myred}{\textbf{Wrong:}} Depends on/off the situation \\
\textcolor{mygreen}{\textbf{Right:}} Depends on the situation

\textcolor{myred}{\textbf{Wrong:}} Good in mathematics \\
\textcolor{mygreen}{\textbf{Right:}} Good at mathematics

\subsection{5. Pronoun-Antecedent Agreement}
\textcolor{myred}{\textbf{Wrong:}} Everyone should bring their books. \\
\textcolor{mygreen}{\textbf{Right:}} Everyone should bring his/her book. \\
\textcolor{mygreen}{\textbf{Better:}} All students should bring their books.

\subsection{6. Confusing Words}
\begin{tabular}{@{}ll@{}}
\textbf{accept} (receive) & \textbf{except} (exclude) \\
\textbf{affect} (influence) & \textbf{effect} (result) \\
\textbf{its} (possessive) & \textbf{it's} (it is) \\
\textbf{their} (possessive) & \textbf{there} (place) \\
\textbf{whose} (possessive) & \textbf{who's} (who is) \\
\textbf{then} (time) & \textbf{than} (comparison) \\
\end{tabular}

\section{Quick Grammar Tips}

\subsection{Articles (a, an, the)}
\begin{itemize}[leftmargin=*, noitemsep, topsep=0pt]
    \item \textbf{a} before consonant sound: a book, a university
    \item \textbf{an} before vowel sound: an apple, an hour
    \item \textbf{the} for specific/known things: the sun, the Taj Mahal
\end{itemize}

\subsection{Active vs Passive Voice}
\textbf{Active:} Subject performs action \\
\textit{She writes a letter.}

\textbf{Passive:} Object receives action \\
\textit{A letter is written by her.}

\textbf{Form:} Object + is/am/are/was/were + V3 + by + Subject

\subsection{Direct vs Indirect Speech}
\textbf{Direct:} He said, "I am busy." \\
\textbf{Indirect:} He said that he was busy.

\textbf{Changes:}
\begin{itemize}[leftmargin=*, noitemsep, topsep=0pt]
    \item Remove quotation marks
    \item Add 'that'
    \item Change tense (present → past)
    \item Change pronouns
    \item Change time/place words
\end{itemize}

\subsection{Question Formation}
\textbf{Yes/No Questions:} Auxiliary + Subject + Main Verb? \\
\textit{Do you speak English?}

\textbf{Wh-Questions:} Wh-word + Auxiliary + Subject + Main Verb? \\
\textit{What are you doing?}

\subsection{Negative Sentences}
\textbf{With Auxiliary:} Subject + Auxiliary + not + Main Verb \\
\textit{She is not coming.}

\textbf{Without Auxiliary:} Subject + do/does/did + not + V1 \\
\textit{He does not play cricket.}

\section{Important Verb Forms}

\begin{tabular}{@{}llll@{}}
\textbf{V1} & \textbf{V2} & \textbf{V3} & \textbf{V4} \\
\hline
go & went & gone & going \\
do & did & done & doing \\
have & had & had & having \\
see & saw & seen & seeing \\
come & came & come & coming \\
run & ran & run & running \\
write & wrote & written & writing \\
speak & spoke & spoken & speaking \\
take & took & taken & taking \\
make & made & made & making \\
give & gave & given & giving \\
know & knew & known & knowing \\
think & thought & thought & thinking \\
find & found & found & finding \\
tell & told & told & telling \\
get & got & got/gotten & getting \\
feel & felt & felt & feeling \\
become & became & become & becoming \\
leave & left & left & leaving \\
put & put & put & putting \\
bring & brought & brought & bringing \\
begin & began & begun & beginning \\
keep & kept & kept & keeping \\
hold & held & held & holding \\
hear & heard & heard & hearing \\
\end{tabular}

\section{Exam Strategy}

\subsection{For MCQs}
\begin{itemize}[leftmargin=*, noitemsep, topsep=0pt]
    \item Read question carefully
    \item Eliminate obviously wrong answers
    \item Check subject-verb agreement
    \item Verify tense consistency
    \item Look for preposition usage
\end{itemize}

\subsection{For Fill in the Blanks}
\begin{itemize}[leftmargin=*, noitemsep, topsep=0pt]
    \item Identify subject (singular/plural)
    \item Check time indicators (yesterday, now, etc.)
    \item Look for modal auxiliary requirements
    \item Ensure proper verb form
    \item Read complete sentence for context
\end{itemize}

\subsection{For Sentence Pattern Questions}
\begin{itemize}[leftmargin=*, noitemsep, topsep=0pt]
    \item Identify subject first
    \item Find main verb
    \item Check if verb needs object
    \item Look for complement or adverb
    \item Mark the pattern (SV, SVO, SVC, SVA)
\end{itemize}

\subsection{Last Minute Tips}
\begin{itemize}[leftmargin=*, noitemsep, topsep=0pt]
    \item Memorize irregular verb forms
    \item Practice modal auxiliary usage
    \item Review subject-verb agreement rules
    \item Remember tense formulas
    \item Know sentence patterns
    \item Practice identifying parts of speech
    \item Revise common prepositions
    \item Learn conjunction types
\end{itemize}

\section{Mnemonic Devices}

\subsection{FANBOYS}
\textbf{Coordinating Conjunctions:} \\
For, And, Nor, But, Or, Yet, So

\subsection{Modal Usage}
\textbf{CAN-COULD}: Capability \\
\textbf{MAY-MIGHT}: Possibility \\
\textbf{MUST}: Obligation \\
\textbf{SHOULD}: Advice \\
\textbf{WILL-WOULD}: Future/Request

\subsection{Tense Time Words}
\textbf{Present:} always, usually, often, now \\
\textbf{Past:} yesterday, ago, last, then \\
\textbf{Future:} tomorrow, next, soon, later \\
\textbf{Perfect:} already, just, yet, ever, never \\
\textbf{Continuous:} for, since, while, when

\subsection{Preposition of Time}
\textbf{AT-ON-IN Method:} \\
\textbf{AT} = Specific time (at 5 PM) \\
\textbf{ON} = Days/dates (on Monday) \\
\textbf{IN} = Months/years (in 2024)

\subsection{Subject-Verb Agreement}
\textbf{ESSE Rule:} \\
\textbf{E}ach, \textbf{E}very, \textbf{S}ome, \textbf{S}omebody/someone, \textbf{E}verybody/everyone = Singular verb

\section{Practice Exercise Examples}

\subsection{Identify Parts of Speech}
1. \textit{She sings beautifully.} \\
   She = Pronoun, sings = Verb, beautifully = Adverb

2. \textit{The quick brown fox jumps.} \\
   The = Article, quick = Adj., brown = Adj., fox = Noun, jumps = Verb

\subsection{Fill Correct Tense}
1. I \_\_\_\_ (study) for three hours. \\
   Answer: \textbf{have been studying} (Present Perfect Continuous)

2. She \_\_\_\_ (come) yesterday. \\
   Answer: \textbf{came} (Simple Past)

\subsection{Choose Correct Modal}
1. You \_\_\_\_ obey traffic rules. (should/would) \\
   Answer: \textbf{should}

2. \_\_\_\_ you help me? (May/Could) \\
   Answer: \textbf{Could}

\subsection{Subject-Verb Agreement}
1. Each of the students \_\_\_\_ present. (is/are) \\
   Answer: \textbf{is}

2. The team \_\_\_\_ playing well. (is/are) \\
   Answer: \textbf{is}

\subsection{Identify Sentence Pattern}
1. \textit{Birds fly.} - \textbf{SV} \\
2. \textit{I read books.} - \textbf{SVO} \\
3. \textit{She is beautiful.} - \textbf{SVC} \\
4. \textit{He runs fast.} - \textbf{SVA}

\vfill
\hrule
\vspace{2mm}
\begin{center}
\small
\textbf{All the Best for Your Exam!} \\
\textit{Practice makes perfect. Review regularly for better retention.} \\
\vspace{1mm}
\textit{Created for GTU Diploma Students | Communication Skills in English}
\end{center}

\end{multicols}
\end{document}