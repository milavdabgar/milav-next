\documentclass{article}

% content/resources/templates/preamble.tex
\usepackage[margin=0.6in]{geometry}
\author{Milav Dabgar}
\usepackage{amsmath,amssymb,amsthm}
\usepackage{booktabs}
\usepackage{multirow}
\usepackage{xcolor}
\usepackage{tcolorbox}
\tcbuselibrary{breakable,skins}
\usepackage[colorlinks=true,linkcolor=blue]{hyperref}
\usepackage{titlesec}
\usepackage{enumitem}
\usepackage{tikz}
\usepackage{pgfplots}
\usepackage{circuitikz}
\usepackage[version=4]{mhchem}
\usepackage{longtable}
\usepackage{array}
\usepackage{float}
\usepackage{caption}
\usepackage{listings}

\lstset{
  basicstyle=\small\ttfamily,
  breaklines=true,
  breakatwhitespace=false,
  postbreak=\mbox{\textcolor{red}{$\hookrightarrow$}\space},
  float=false,
  numbers=left,
  numberstyle=\tiny\color{gray},
  numbersep=10pt,
  xleftmargin=2em,
  keywordstyle=\color{blue},
  commentstyle=\color{green!60!black},
  stringstyle=\color{purple},
  backgroundcolor=\color{gray!5},
  showstringspaces=false,
  tabsize=2,
  captionpos=b,
  keepspaces=true,
  columns=flexible
}

\pgfplotsset{compat=1.18}
\usetikzlibrary{shapes,arrows,positioning,calc,patterns,decorations.pathmorphing,decorations.markings,arrows.meta}

% Color scheme
\definecolor{headcolor}{RGB}{0,102,204}
\definecolor{keycolor}{RGB}{220,20,60}
\definecolor{solutioncolor}{RGB}{34,139,34}
\definecolor{mnemoniccolor}{RGB}{148,0,211}
\definecolor{codecolor}{RGB}{0,0,100}

% Spacing
\setlength{\parskip}{3pt}
\setlist[itemize]{nosep}
\setlist[enumerate]{nosep}

% Title formatting
\titleformat{\section}{\Large\bfseries\color{headcolor}}{\thesection}{1em}{}
\titleformat{\subsection}{\large\bfseries\color{headcolor}}{\thesubsection}{1em}{}

% Pandoc tightlist compatibility
\providecommand{\tightlist}{%
  \setlength{\itemsep}{0pt}\setlength{\parskip}{0pt}}

% Pandoc longtable compatibility
\newcounter{none}
\def\thenone{}


% content/resources/templates/english-boxes.tex

% Custom environments
\newtcolorbox{solutionbox}{
 breakable,
 enhanced,
 colback=solutioncolor!5!white,
 colframe=solutioncolor!75!black,
 fonttitle=\bfseries,
 title=Solution
}

\newtcolorbox{solutionboxnobreak}{
 colback=solutioncolor!5!white,
 colframe=solutioncolor!75!black,
 fonttitle=\bfseries,
 title=Solution
}

\newtcolorbox{keyformula}{
 breakable,
 enhanced,
 colback=keycolor!5!white,
 colframe=keycolor!75!black,
 fonttitle=\bfseries,
 title=Key Formula
}

\newtcolorbox{mnemonicboxenv}{
 breakable,
 enhanced,
 colback=mnemoniccolor!5!white,
 colframe=mnemoniccolor!75!black,
 fonttitle=\bfseries,
 title=Mnemonic
}

\newcommand{\mnemonicbox}[1]{%
  \begin{mnemonicboxenv}
    #1
  \end{mnemonicboxenv}
}


% Custom commands for GTU solutions
% This file defines semantic commands for consistent formatting

% Question command with automatic formatting
\newcommand{\question}[2]{%
  \section*{Question #1}%
  \textbf{#2}%
}

% OR question variant
\newcommand{\questionor}[2]{%
  \section*{Question #1 OR}%
  \textbf{#2}%
}

% Proper table environment with caption
\newenvironment{answertable}[1]{%
  \begin{table}[htbp]
  \centering
  \caption{#1}
}{%
  \end{table}
}

% Proper figure environment for diagrams
\newenvironment{answerdiagram}[1]{%
  \begin{figure}[htbp]
  \centering
  \caption{#1}
}{%
  \end{figure}
}

% Semantic markup for key terms
\newcommand{\keyword}[1]{\textbf{#1}}
\newcommand{\code}[1]{\texttt{#1}}
\newcommand{\classname}[1]{\texttt{#1}}
\newcommand{\methodname}[1]{\texttt{#1}}

% Proper quotation marks
\newcommand{\mnemonic}[1]{``#1''}


\title{Communication Skills in English (4300002) - Winter 2023 Solution}
\date{February 01, 2024}

\begin{document}
\maketitle

\questionmarks{Question 1(a)}{3}{marks}

\textbf{Answer the following questions. (Any Three)}

\begin{enumerate}
    \item \textbf{What was the author's attitude towards men?}
    
    \textbf{Answer}: 
    In ``Leopard,'' the author was initially wary of men who hunted and disturbed wildlife. He preferred solitude in nature away from human interference, showing a critical attitude toward those who didn't respect the wilderness.
    
    \item \textbf{Whose note was handed over to Bob in the end?}
    
    \textbf{Answer}: 
    Jimmy Wells' note was handed over to Bob at the end of ``After Twenty Years.'' The note explained that Jimmy was actually the uniformed policeman who had met Bob earlier but couldn't arrest his old friend himself.
    
    \item \textbf{What is the role played by the horse in this poem?}
    
    \textbf{Answer}: 
    In ``Stopping by Woods on a Snowy Evening,'' the horse serves as a connection to civilization and responsibilities. It reminds the speaker of his duties and obligations by shaking its harness bells, questioning the unusual stop.
    
    \item \textbf{What kind of freedom does the poet desire for his country?}
    
    \textbf{Answer}: 
    In ``Where the Mind is Without Fear,'' Tagore desires intellectual, social, and spiritual freedom where people think fearlessly, knowledge flows freely, and the nation isn't divided by narrow domestic walls.
    
    \item \textbf{What promises do you think the poet has to keep?}
    
    \textbf{Answer}: 
    In ``Stopping by Woods on a Snowy Evening,'' the poet has promises of daily responsibilities and social obligations to fulfill. The repeated line ``miles to go before I sleep'' suggests both literal journey and life's commitments.
\end{enumerate}

\questionmarks{Question 1(b)}{4}{marks}

\textbf{Choose the correct options. (Any Four)}

\begin{enumerate}
    \item \dots.. is called Hill of Fairies
    
    \textbf{Answer}: (d) Pari Tibba
    
    \item The location of the story The Leopard is near \dots.
    
    \textbf{Answer}: (b) Mussoorie
    
    \item The horse finds it queer because there is no\dots\dots.
    
    \textbf{Answer}: (a) farmhouse
    
    \item What is meant by ``mind is without fear and head is held high''.
    
    \textbf{Answer}: (a) to be fearless and self-respecting
    
    \item Who wrote the note?
    
    \textbf{Answer}: (b) Jimmy
\end{enumerate}

\questionmarks{Question 1(c)}{7}{marks}

\textbf{You are Rahul Roy. You have placed an order for 500 compass boxes. But the consignment had only 475 boxes. Draft a complaint letter to the supplier.}

\textbf{Answer}:

\begin{lstlisting}
Rahul Roy
123, Student Quarters
Gujarat Technological University
Ahmedabad - 380015

February 1, 2024

The Sales Manager
ABC School Supplies
45, Industrial Area
Vadodara - 390010

Subject: Complaint Regarding Incomplete Delivery of Compass Boxes

Dear Sir/Madam,

Reference: Purchase Order No. SR/CB/2024/78 dated January 15, 2024

I am writing to express my dissatisfaction regarding the recent delivery of compass boxes ordered by me on January 15, 2024.

As per my purchase order, I had ordered 500 compass boxes, but the consignment delivered on January 30, 2024, contained only 475 boxes, resulting in a shortage of 25 boxes.

This shortage has caused significant inconvenience as these items were urgently required for distribution to students for their upcoming technical drawing examinations.

I request you to expedite the delivery of the remaining 25 compass boxes immediately. Alternatively, please adjust the billing amount accordingly.

I look forward to your prompt action in resolving this matter.

Yours sincerely,

Rahul Roy
Mobile: 9876543210
Email: rahul.roy@email.com
\end{lstlisting}

\orquestionmarks{Question 1(c)}{7}{marks}

\textbf{Draft an email asking for the illustrated catalogue and quotation of certain electronic goods required by your firm.}

\textbf{Answer}:

\begin{lstlisting}
From: purchase@techinnovations.com
To: sales@electronicstore.com
Subject: Request for Catalog and Quotation of Electronic Goods

Dear Sir/Madam,

I am writing on behalf of Tech Innovations Ltd., a leading IT services company based in Ahmedabad.

We are looking to purchase the following electronic items for our newly established office:

1. Laptop computers - 20 units (i5 processor or higher)
2. Desktop computers - 15 units (with minimum 8GB RAM)
3. LED projectors - 5 units
4. Multi-function printers - 8 units
5. Wireless routers - 10 units

We request you to kindly send us your latest illustrated catalog featuring these items along with a detailed quotation including:

- Technical specifications of each product
- Unit prices and applicable discounts for bulk purchase
- Warranty information
- Delivery timeframe
- Payment terms and conditions

As we need these items urgently for our new office, we would appreciate receiving your response by February 10, 2024.

Thank you for your attention to this request. We look forward to doing business with you.

Sincerely,

Purchase Manager
Tech Innovations Ltd.
Ahmedabad, Gujarat
Tel: 079-12345678
Email: purchase@techinnovations.com
\end{lstlisting}

\questionmarks{Question 2(a)}{3}{marks}

\textbf{Friendship of Bob and Jimmy}

\textbf{Answer}:
In ``After Twenty Years,'' O. Henry portrays an extraordinary friendship between Bob and Jimmy that transcends time and circumstances.

\begin{table}[H]
\centering
\caption{Bob and Jimmy's Friendship}
\begin{tabulary}{\linewidth}{L L}
\toprule
\textbf{Aspect} & \textbf{Description} \\
\midrule
Loyalty & Bob travels miles to keep 20-year-old promise \\
Memory & Bob describes Jimmy with fondness and specific details \\
Change & While apart, lives diverged drastically - one a criminal, one a policeman \\
Conflict & Friendship tested against duty and moral obligation \\
\bottomrule
\end{tabulary}
\end{table}

\begin{mnemonicbox}Promises Kept Despite Paths Divided\end{mnemonicbox}

\questionmarks{Question 2(b)}{4}{marks}

\textbf{Description of nature in Stopping by Woods on a Snowy Evening}

\textbf{Answer}:
Frost creates a serene winter landscape that becomes a moment of contemplation for the speaker.

\begin{table}[H]
\centering
\caption{Nature Elements in the Poem}
\begin{tabulary}{\linewidth}{L L}
\toprule
\textbf{Element} & \textbf{Description} \\
\midrule
Woods & Dark, deep, filling with snow - mysterious and alluring \\
Snow & Falling softly, creating stillness and silence \\
Evening & Darkest evening of the year - symbolizing depth of winter \\
Lake & Frozen - adding to the stillness of the scene \\
\bottomrule
\end{tabulary}
\end{table}

\begin{itemize}
    \item \textbf{Visual Imagery}: Snow-filled woods creating a peaceful natural setting
    \item \textbf{Sound Imagery}: ``The sweep of easy wind and downy flake'' contrasted with the sound of harness bells
    \item \textbf{Contrast}: Beautiful natural scene vs. human obligations waiting
\end{itemize}

\begin{mnemonicbox}Silent Snow Surrounds Still Scene\end{mnemonicbox}

\questionmarks{Question 2(c)}{7}{marks}

\textbf{Answer the following questions.}

\begin{enumerate}
    \item \textbf{How does the poet describe the old habit?}
    
    \textbf{Answer}: 
    In ``Where the Mind is without Fear,'' Tagore describes old habits as ``dead habit'' which ``straitens'' (narrows/constrains) freedom of thought and action. He sees outdated customs and traditions as restricting progress and preventing clear, logical thinking needed for a free nation.
    
    \item \textbf{Where was the speaker going? What stopped him on the way?}
    
    \textbf{Answer}: 
    In ``Stopping by Woods on a Snowy Evening,'' the speaker was traveling to fulfill obligations (``promises to keep''). The beauty of ``woods filling up with snow'' stopped him momentarily on his journey, creating a pause between responsibilities.
    
    \item \textbf{What did the stranger say to the policeman?}
    
    \textbf{Answer}: 
    In ``After Twenty Years,'' the stranger (Bob) told the policeman (actually Jimmy) that he was waiting for his friend Jimmy Wells with whom he had promised to meet at that spot exactly twenty years ago, and described how successful he had become out West.
    
    \item \textbf{Why did the author return to mountains?}
    
    \textbf{Answer}: 
    In ``Leopard,'' the author returned to the mountains to escape the chaos of city life, seek solitude in nature, and observe wildlife in their natural habitat. He was particularly drawn to the peaceful harmony of the mountain environment.
    
    \item \textbf{What was there in place of the store at that spot twenty years ago?}
    
    \textbf{Answer}: 
    In ``After Twenty Years,'' there was ``Big Joe'' Brady's restaurant at that spot twenty years ago, where Bob and Jimmy had their last dinner together before parting ways.
    
    \item \textbf{What does the speaker wish to convey through the phrase ``fill up with the snow''?}
    
    \textbf{Answer}: 
    The phrase ``fill up with snow'' conveys the continuous, gentle snowfall creating a sense of completeness, stillness, and transformation of the landscape. It suggests nature's quiet persistence and the gradual covering of all human traces.
    
    \item \textbf{How does the poet describe `heaven of freedom'?}
    
    \textbf{Answer}: 
    In ``Where the Mind is Without Fear,'' Tagore describes his ``heaven of freedom'' as a place where people hold their heads high with dignity, knowledge flows freely, the world isn't broken by narrow divisions, words come from truth, reason guides action, and thoughts lead to perfection.
\end{enumerate}

\orquestionmarks{Question 2(a)}{3}{marks}

\textbf{What did the author do to find forktails's home?}

\textbf{Answer}:
In ``Leopard,'' the author made persistent efforts to locate the forktail's nest.

\begin{table}[H]
\centering
\caption{Author's Search for Forktail's Nest}
\begin{tabulary}{\linewidth}{L L}
\toprule
\textbf{Actions} & \textbf{Methods} \\
\midrule
Observation & Patiently watched forktail's movements daily \\
Tracking & Followed the bird's flight patterns upstream \\
Exploration & Searched among rocks and crevices near water \\
Patience & Returned repeatedly despite unsuccessful attempts \\
\bottomrule
\end{tabulary}
\end{table}

\begin{mnemonicbox}Watch, Follow, Search, Return\end{mnemonicbox}

\orquestionmarks{Question 2(b)}{4}{marks}

\textbf{What did Bob tell the man in the overcoat?}

\textbf{Answer}:
Bob shared his life story and friendship with Jimmy to the man in the overcoat, unaware he was speaking to a plainclothes officer.

\begin{itemize}
    \item \textbf{Told about his friendship} with Jimmy Wells from twenty years ago
    \item \textbf{Described their promise} to meet at that exact spot after twenty years
    \item \textbf{Shared his success story} of becoming wealthy in the West, implying his criminal activities
    \item \textbf{Expressed confidence} that Jimmy would honor their appointment despite the time passed
    \item \textbf{Revealed identifying details} about himself that helped confirm his identity as a wanted criminal
\end{itemize}

\begin{mnemonicbox}Past Promises, Present Success, Pending Reunion\end{mnemonicbox}

\orquestionmarks{Question 2(c)}{7}{marks}

\textbf{Answer the following questions.}

\begin{enumerate}
    \item \textbf{Who does the poet address as `thee' and my father?}
    
    \textbf{Answer}: 
    In ``Where the Mind is Without Fear,'' Tagore addresses God as `thee' and `my father,' appealing to the divine to guide his country to true freedom. This spiritual invocation shows his belief that achieving his vision requires divine guidance.
    
    \item \textbf{What are the sights and sounds that the poet experiences in the woods?}
    
    \textbf{Answer}: 
    In ``Stopping by Woods,'' the poet experiences the visual sight of ``woods filling up with snow'' and ``frozen lake.'' The sounds include his horse's ``harness bells'' shaking, ``easy wind,'' and the soft landing of ``downy flake'' (snowflakes) - creating a contrast between humanity's sounds and nature's quiet.
    
    \item \textbf{What kind of man was his friend Jimmy?}
    
    \textbf{Answer}: 
    Jimmy Wells was portrayed as honest, principled, and loyal. He kept his appointment after twenty years, yet chose duty over friendship when faced with Bob's criminal status. His note to Bob showed compassion and personal conflict, revealing his integrity and emotional depth.
    
    \item \textbf{What happened when the leopard sensed the author's presence?}
    
    \textbf{Answer}: 
    When the leopard sensed the author's presence in ``Leopard,'' it paused its drinking, looked directly at the author without aggression, and continued its activity, showing mutual acknowledgment and respect. This moment demonstrated the growing trust between human and wild animal.
    
    \item \textbf{What kind of characters does the story after twenty years have?}
    
    \textbf{Answer}: 
    ``After Twenty Years'' features characters who represent opposing life paths: Bob (the criminal who achieved material success through questionable means) and Jimmy (the honest policeman who chose duty and integrity). Their contrast creates the story's moral dilemma and ironic ending.
    
    \item \textbf{What according to the speaker will surprise the horse?}
    
    \textbf{Answer}: 
    In ``Stopping by Woods,'' the speaker suggests the horse would be surprised by stopping ``without a farmhouse near'' in the dark woods. This unusual pause without practical purpose highlights the contrast between animal practicality and human appreciation for beauty.
    
    \item \textbf{What is the poet's opinion about the work, idleness, reason and old customs?}
    
    \textbf{Answer}: 
    In ``Where the Mind is Without Fear,'' Tagore advocates for purposeful work over idleness, valuing reason over blind following of outdated customs. He believes clear reason should guide actions, not ``dead habit,'' and that meaningful work leads to progress toward perfection.
\end{enumerate}

\questionmarks{Question 3(a)}{3}{marks}

\textbf{Join the sentence using appropriate connector.}

\begin{enumerate}
    \item He \textbf{who} had a good teacher, passed the exam. (who/which/whom)
    
    \item He has been living in Ahmedabad \textbf{for} several months. (since/for/in)
    
    \item The treasure is \textbf{between} the palm tree and the hut. (between/among)
\end{enumerate}

\questionmarks{Question 3(b)}{4}{marks}

\textbf{Fill in the blanks using proper preposition.}

\begin{enumerate}
    \item She got \textbf{into} her car and drove away. (in/into)
    
    \item My interview is \textbf{at} 3:00 pm. (at/on)
    
    \item I prefer milk \textbf{to} tea. (too/to/than)
    
    \item Vinod is fond \textbf{of} flying kites. (from/of/to)
\end{enumerate}

\questionmarks{Question 3(c)}{7}{marks}

\textbf{Write about the activities happening around you in the classroom using present continuous tense.}

\textbf{Answer}:

In our active classroom, various educational activities are taking place simultaneously. The professor is explaining grammar rules while writing examples on the whiteboard. Some students are taking notes diligently, while others are asking questions for clarity. Two students at the back are discussing a difficult concept in whispers. The class representative is distributing assignment sheets to everyone. A few students near the windows are looking outside as it is raining heavily. The ceiling fan is rotating at full speed because it's quite warm today. Our English language assistant is moving around the classroom, helping individuals who are struggling with exercises. Everyone is participating enthusiastically in the learning process.

\begin{mnemonicbox}See-Hear-Do: Everyone is learning something new\end{mnemonicbox}

\orquestionmarks{Question 3(a)}{3}{marks}

\textbf{Fill in the blanks using appropriate adjectives or adverbs.}

\begin{enumerate}
    \item Virat played \textbf{well} in the final match. (well/good)
    
    \item Kishan plays on the flute \textbf{nicely}. (nice/nicely)
    
    \item Tenali Raman was a \textbf{wise} pandit. (wise/wisely)
\end{enumerate}

\orquestionmarks{Question 3(b)}{4}{marks}

\textbf{Fill in the blanks with proper auxiliary verbs)}

\begin{enumerate}
    \item \textbf{May} we play football? (Permission)
    
    \item The astrologer \textbf{had to} leave the village as he lied to people. (compulsion)
    
    \item I \textbf{wish to} go for the science Stream. (wish)
    
    \item \textbf{Shall} we go to their help? (suggestion)
\end{enumerate}

\orquestionmarks{Question 3(c)}{7}{marks}

\textbf{Write about what you did last Sunday using Simple Past Tense.}

\textbf{Answer}:

Last Sunday, I woke up early and completed my pending assignments. After breakfast, I helped my mother with household chores. Then, I met my friends at the local park where we played cricket for two hours. I scored thirty runs and took two wickets in the match. At noon, I returned home and had lunch with my family. In the afternoon, I visited my grandparents who live nearby. My grandmother prepared special sweets for me. I spent some quality time with them and listened to my grandfather's interesting stories. In the evening, I watched a documentary on wildlife conservation. Later, I revised my lessons for the upcoming test. Before going to bed, I planned my schedule for the next week. Overall, I enjoyed my Sunday with a perfect balance of work and relaxation.

\begin{mnemonicbox}Morning Work, Noon Play, Evening Family, Night Study\end{mnemonicbox}


\questionmarks{Question 4(a)}{3}{marks}

\textbf{Identify the interjection the following sentences.}

\begin{enumerate}
    \item \textbf{Wow!} I passed my English test.
    
    \item \textbf{Hurrah!} we have won the match.
    
    \item \textbf{Ouch!} That hurts.
\end{enumerate}

\questionmarks{Question 4(b)}{4}{marks}

\textbf{Rearrange the jumbled words to make meaningful sentences and punctuate them.}

\begin{enumerate}
    \item Ramesh/all/prepared/questions/the
    
    \textbf{Answer}: Ramesh prepared all the questions.
    
    \item The files/the peon/in the shelf/was arranging
    
    \textbf{Answer}: The peon was arranging the files in the shelf.
    
    \item Whose/the boy/pen is missing/complained/to Sir
    
    \textbf{Answer}: The boy complained to Sir whose pen is missing.
    
    \item Are/what/you/now/doing?
    
    \textbf{Answer}: What are you doing now?
\end{enumerate}

\questionmarks{Question 4(c)}{7}{marks}

\textbf{Fill in the blanks using appropriate form of the verb.}

\begin{enumerate}
    \item Whenever we meet, we \textbf{plan} a trip.
    
    \item Who \textbf{invented} the bicycle?
    
    \item Hello Nitya, I \textbf{haven't seen} you for ages. How are you?
    
    \item He \textbf{goes} to that place every year.
    
    \item I \textbf{am reading} a novel nowadays.
    
    \item The sun \textbf{is shining} brightly.
    
    \item Vijay \textbf{was waiting} for me when I arrived.
\end{enumerate}

\orquestionmarks{Question 4(a)}{3}{marks}

\textbf{Use appropriate noun or pronoun.}

\begin{enumerate}
    \item I know Mr. James. \textbf{He} is a very good doctor.
    
    \item They are very good friends. Their \textbf{friendship} is example for many of us.
    
    \item Mary is tired. \textbf{She} wants to sleep.
\end{enumerate}

\orquestionmarks{Question 4(b)}{4}{marks}

\textbf{Identify the pattern of the following.}

\begin{enumerate}
    \item The police / arrested / the thief
    
    \textbf{Answer}: SVO
    
    \item People/ cried.
    
    \textbf{Answer}: SV
    
    \item The winner was rewarded.
    
    \textbf{Answer}: SV
    
    \item They / came / suddenly.
    
    \textbf{Answer}: SVA
\end{enumerate}

\orquestionmarks{Question 4(c)}{7}{marks}

\textbf{Fill in the blanks using right verb.}

\begin{enumerate}
    \item A lot of good we take \textbf{is} wasted.
    
    \item The trouble with these guys \textbf{is} their rustic approach.
    
    \item A little dust \textbf{is} visible.
    
    \item All the oil \textbf{has} been stolen.
    
    \item None of them \textbf{attend} to their work these days.
    
    \item Some of the pipes \textbf{run} for several miles.
    
    \item Either Ram or his brother \textbf{works} as a manager here.
\end{enumerate}

\questionmarks{Question 5(a)}{3}{marks}

\textbf{What is non-verbal communication?}

\textbf{Answer}:
Non-verbal communication is the transmission of messages without using words.

\begin{table}[H]
\centering
\caption{Types of Non-verbal Communication}
\begin{tabulary}{\linewidth}{L L}
\toprule
\textbf{Type} & \textbf{Examples} \\
\midrule
Kinesics & Body movements, gestures, facial expressions \\
Proxemics & Use of physical space, distance between people \\
Paralanguage & Voice tone, pitch, volume, pace, pauses \\
Appearance & Clothing, grooming, physical characteristics \\
\bottomrule
\end{tabulary}
\end{table}

\begin{mnemonicbox}Body Shows What Words Don't Tell\end{mnemonicbox}

\questionmarks{Question 5(b)}{4}{marks}

\textbf{Discuss the barriers of effective communication.}

\textbf{Answer}:
Communication barriers prevent messages from being properly understood, creating misunderstandings between sender and receiver.

\begin{table}[H]
\centering
\caption{Common Communication Barriers}
\begin{tabulary}{\linewidth}{L L}
\toprule
\textbf{Barrier Type} & \textbf{Examples} \\
\midrule
Physical & Noise, distance, technical problems \\
Psychological & Bias, preconceptions, emotional state \\
Semantic & Jargon, language differences, ambiguity \\
Organizational & Hierarchical levels, information overload \\
\bottomrule
\end{tabulary}
\end{table}

\begin{itemize}
    \item \textbf{Physical Barriers}: Environmental factors interfering with transmission
    \item \textbf{Psychological Barriers}: Mental states affecting reception of message
    \item \textbf{Semantic Barriers}: Problems with meaning and interpretation
    \item \textbf{Organizational Barriers}: Structural issues hindering information flow
\end{itemize}

\begin{mnemonicbox}PESO: Physical, Emotional, Semantic, Organizational\end{mnemonicbox}

\questionmarks{Question 5(c)}{7}{marks}

\textbf{You have an email inquiring about prices of plastic toys manufactured by you. Draft a reply.}

\textbf{Answer}:

\begin{lstlisting}
From: sales@brighttoysltd.com
To: customer@email.com
Subject: RE: Inquiry About Prices of Plastic Toys

Dear [Customer's Name],

Thank you for your email dated [Date] inquiring about the prices of our plastic toys. We appreciate your interest in Bright Toys Ltd. products.

We are pleased to provide you with the following price information:

1. Educational Building Blocks (Set of 50) - 450 per set
2. Animal Figurines (Set of 12) - 350 per set
3. Toy Vehicles (Assorted) - 200-300 per piece depending on size
4. Doll Houses with Furniture - 1,200 per set
5. Action Figures - 250-400 per piece depending on complexity

For bulk orders (more than 50 pieces), we offer a 15% discount on the listed prices. All our toys are made from non-toxic, durable plastic and comply with international safety standards.

We have attached our complete product catalog with detailed specifications and additional product images for your reference.

If you have any other questions or would like to place an order, please feel free to contact us. We look forward to doing business with you.

Best regards,

Sales Manager
Bright Toys Ltd.
Tel: 079-98765432
Email: sales@brighttoysltd.com
Website: www.brighttoysltd.com
\end{lstlisting}

\orquestionmarks{Question 5(a)}{3}{marks}

\textbf{How will you know if communication was successful?}

\textbf{Answer}:
Successful communication occurs when the intended message is properly understood and acted upon by the receiver.

\begin{table}[H]
\centering
\caption{Indicators of Successful Communication}
\begin{tabulary}{\linewidth}{L L}
\toprule
\textbf{Indicator} & \textbf{Description} \\
\midrule
Feedback & Receiver responds appropriately to the message \\
Action & Desired response or action is taken by the receiver \\
Understanding & Clear comprehension without confusion \\
Engagement & Active participation in the communication process \\
\bottomrule
\end{tabulary}
\end{table}

\begin{mnemonicbox}FAUE: Feedback, Action, Understanding, Engagement\end{mnemonicbox}

\orquestionmarks{Question 5(b)}{4}{marks}

\textbf{What is paralanguage?}

\textbf{Answer}:
Paralanguage refers to how we say things rather than what we say - the vocal but non-verbal components of communication.

\begin{table}[H]
\centering
\caption{Elements of Paralanguage}
\begin{tabulary}{\linewidth}{L L}
\toprule
\textbf{Element} & \textbf{Description} \\
\midrule
Voice Quality & Pitch, rhythm, tempo, resonance \\
Voice Characteristics & Volume, tone, inflection \\
Vocal Interferences & Fillers like ``um,'' ``ah,'' ``well'' \\
Silence & Strategic pauses that communicate meaning \\
\bottomrule
\end{tabulary}
\end{table}

\begin{itemize}
    \item \textbf{Vocal Tone}: Conveys emotions (excitement, boredom, sarcasm)
    \item \textbf{Volume}: Emphasizes importance or indicates emotional state
    \item \textbf{Speed}: Fast speaking might indicate excitement or nervousness
    \item \textbf{Pitch}: High pitch may show surprise; low pitch can suggest seriousness
\end{itemize}

\begin{mnemonicbox}TVPS: Tone, Volume, Pace, Silence\end{mnemonicbox}

\orquestionmarks{Question 5(c)}{7}{marks}

\textbf{One of your customers has complained the curtains supplied by you are of inferior quality and not in accordance with the samples shown to him. Draft a reply expressing your regrets and showing willingness to replace the goods.}

\textbf{Answer}:

\begin{lstlisting}
ELEGANT INTERIORS
123, Commercial Complex
Ahmedabad - 380001
Tel: 079-12345678 | Email: support@elegantinteriors.com

February 1, 2024

Mr. Rajesh Mehta
45, Green Park Society
Ahmedabad - 380015

Subject: Response to Complaint Regarding Curtain Quality

Dear Mr. Mehta,

Reference: Your complaint letter dated January 25, 2024

I sincerely apologize for the inconvenience caused to you regarding the curtains supplied by our company that did not match the quality of the samples shown to you.

At Elegant Interiors, we pride ourselves on maintaining high quality standards, and I regret that we have fallen short of these standards in your case. After investigating the matter, we found that there was an unfortunate mix-up in our inventory, resulting in the dispatch of a different quality material than what was agreed upon.

We understand your disappointment and are committed to resolving this issue promptly. We would like to offer the following solution:

1. We will collect the current curtains from your residence at a time convenient to you.
2. We will replace them with new curtains that match exactly with the samples shown to you, at no additional cost.
3. As a goodwill gesture, we would like to offer a 15% discount on your next purchase from our store.

Our representative will contact you within 48 hours to arrange the collection and replacement. Please feel free to contact me directly at 9876543210 if you have any questions or need further assistance.

Once again, please accept our sincere apologies for this unfortunate situation. We value your business and look forward to serving you better in the future.

Yours sincerely,

[Signature]
Manager
Customer Relations Department
Elegant Interiors
\end{lstlisting}

\end{document}
