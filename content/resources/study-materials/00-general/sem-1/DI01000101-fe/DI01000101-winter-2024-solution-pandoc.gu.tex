\documentclass[10pt,a4paper]{article}

% content/resources/templates/preamble.tex
\usepackage[margin=0.6in]{geometry}
\author{Milav Dabgar}
\usepackage{amsmath,amssymb,amsthm}
\usepackage{booktabs}
\usepackage{multirow}
\usepackage{xcolor}
\usepackage{tcolorbox}
\tcbuselibrary{breakable,skins}
\usepackage[colorlinks=true,linkcolor=blue]{hyperref}
\usepackage{titlesec}
\usepackage{enumitem}
\usepackage{tikz}
\usepackage{pgfplots}
\usepackage{circuitikz}
\usepackage[version=4]{mhchem}
\usepackage{longtable}
\usepackage{array}
\usepackage{float}
\usepackage{caption}
\usepackage{listings}

\lstset{
  basicstyle=\small\ttfamily,
  breaklines=true,
  breakatwhitespace=false,
  postbreak=\mbox{\textcolor{red}{$\hookrightarrow$}\space},
  float=false,
  numbers=left,
  numberstyle=\tiny\color{gray},
  numbersep=10pt,
  xleftmargin=2em,
  keywordstyle=\color{blue},
  commentstyle=\color{green!60!black},
  stringstyle=\color{purple},
  backgroundcolor=\color{gray!5},
  showstringspaces=false,
  tabsize=2,
  captionpos=b,
  keepspaces=true,
  columns=flexible
}

\pgfplotsset{compat=1.18}
\usetikzlibrary{shapes,arrows,positioning,calc,patterns,decorations.pathmorphing,decorations.markings,arrows.meta}

% Color scheme
\definecolor{headcolor}{RGB}{0,102,204}
\definecolor{keycolor}{RGB}{220,20,60}
\definecolor{solutioncolor}{RGB}{34,139,34}
\definecolor{mnemoniccolor}{RGB}{148,0,211}
\definecolor{codecolor}{RGB}{0,0,100}

% Spacing
\setlength{\parskip}{3pt}
\setlist[itemize]{nosep}
\setlist[enumerate]{nosep}

% Title formatting
\titleformat{\section}{\Large\bfseries\color{headcolor}}{\thesection}{1em}{}
\titleformat{\subsection}{\large\bfseries\color{headcolor}}{\thesubsection}{1em}{}

% Pandoc tightlist compatibility
\providecommand{\tightlist}{%
  \setlength{\itemsep}{0pt}\setlength{\parskip}{0pt}}

% Pandoc longtable compatibility
\newcounter{none}
\def\thenone{}


% content/resources/templates/gujarati-boxes.tex
\usepackage{fontspec}
\usepackage{polyglossia}

% Set Gujarati as main language (document is primarily in Gujarati)
% Note: gloss-gujarati.ldf doesn't exist in polyglossia, but it will use hyphenation patterns
\setdefaultlanguage{gujarati}
\setotherlanguage{english}

% Configure Gujarati font properly
% Use Language=Default to prevent polyglossia from trying to add language-specific features
% that don't exist for Gujarati, which causes "empty feature" warnings
\newfontfamily\gujaratifont[Script=Gujarati,AutoFakeBold=2.5,AutoFakeSlant=0.3]{Noto Sans Gujarati}
\setmainfont[Script=Gujarati,AutoFakeBold=2.5,AutoFakeSlant=0.3]{Noto Sans Gujarati}
% Use Noto Sans Gujarati for monospace to support Gujarati in text
\setmonofont[Scale=0.9]{Noto Sans Gujarati}

% Configure English to use the same font
\newfontfamily\englishfont[Script=Gujarati,AutoFakeBold=2.5,AutoFakeSlant=0.3]{Noto Sans Gujarati}

% Translations for polyglossia
\gappto\captionsgujarati{
  \renewcommand{\tablename}{કોષ્ટક}
  \renewcommand{\figurename}{આકૃતિ}
}

% Helper for TikZ nodes to ensure Gujarati font
\newcommand{\gu}[1]{{\gujaratifont #1}}

% Custom environments
\newtcolorbox{solutionbox}{
    breakable,
    enhanced,
    colback=solutioncolor!5!white,
    colframe=solutioncolor!75!black,
    fonttitle=\bfseries,
    title=જવાબ
}

\newtcolorbox{solutionboxnobreak}{
 colback=solutioncolor!5!white,
 colframe=solutioncolor!75!black,
 fonttitle=\bfseries,
 title=જવાબ
}

\newtcolorbox{keyformula}{
 breakable,
 enhanced,
 colback=keycolor!5!white,
 colframe=keycolor!75!black,
 fonttitle=\bfseries,
 title=રાસાયણિક સમીકરણ/સૂત્ર
}

\newtcolorbox{mnemonicbox}{
 breakable,
 enhanced,
 colback=mnemoniccolor!5!white,
 colframe=mnemoniccolor!75!black,
 fonttitle=\bfseries,
 title=મેમરી ટ્રીક
}


\begin{document}

\begin{center}
{\Huge\bfseries\color{headcolor} Basic Electronics (Gujarati)}\\[5pt]
{\LARGE DI01000101 -- Winter 2024}\\[3pt]
{\large Semester 1 Study Material}\\[3pt]
{\normalsize\textit{Detailed Solutions and Explanations}}
\end{center}

\vspace{10pt}

\subsection*{પ્રશ્ન 1(અ) [3
ગુણ]}\label{uxaaauxab0uxab6uxaa8-1uxa85-3-uxa97uxaa3}

\textbf{ઓહમના નિયમને તેની મર્યાદા અને ઉપયોગિતા સાથે સમજાવો.}

\begin{solutionbox}

\textbf{ટેબલ: ઓહમના નિયમનો સારાંશ}

\begin{longtable}[]{@{}ll@{}}
\toprule\noalign{}
પાસું & વર્ણન \\
\midrule\noalign{}
\endhead
\bottomrule\noalign{}
\endlastfoot
\textbf{વિધાન} & વાહક દ્વારા પસાર થતો કરંટ વોલ્ટેજના સીધા પ્રમાણમાં હોય છે \\
\textbf{સૂત્ર} & V = I \times R \\
\textbf{એકમો} & V (વોલ્ટ), I (એમ્પિયર), R (ઓહ્મ) \\
\end{longtable}

\textbf{મર્યાદાઓ:}

\begin{itemize}
\tightlist
\item
  \textbf{તાપમાન આધારિત}: તાપમાન સાથે અવરોધ બદલાય છે
\item
  \textbf{બિન-રેખીય પદાર્થો}: સેમિકન્ડક્ટર, ડાયોડ પર લાગુ નહીં
\item
  \textbf{AC સર્કિટ}: રિએક્ટિવ કોમ્પોનન્ટ્સ માટે બદલેલા સ્વરૂપની જરૂર
\end{itemize}

\textbf{ઉપયોગિતા:}

\begin{itemize}
\tightlist
\item
  \textbf{સર્કિટ વિશ્લેષણ}: અજાણા વોલ્ટેજ, કરંટ અથવા અવરોધની ગણતરી
\item
  \textbf{પાવર ગણતરી}: P = V^{2}/R, P = I^{2}R
\end{itemize}

\end{solutionbox}
\begin{mnemonicbox}
``વોલ્ટેજ ઇઝ રિયલી ઇમ્પોર્ટન્ટ'' (V = I \times R)

\end{mnemonicbox}
\subsection*{પ્રશ્ન 1(બ) [4
ગુણ]}\label{uxaaauxab0uxab6uxaa8-1uxaac-4-uxa97uxaa3}

\textbf{ફેરાડેના ઇલેક્ટ્રોમેગ્નેટિક ઇન્ડક્શનના નિયમને જરૂરી આકૃતિ સાથે સમજાવો.}

\begin{solutionbox}

\textbf{ફેરાડેના નિયમો:}

\begin{itemize}
\tightlist
\item
  \textbf{પ્રથમ નિયમ}: જ્યારે વાહક દ્વારા મેગ્નેટિક ફ્લક્સ બદલાય ત્યારે EMF પેદા થાય
  છે
\item
  \textbf{બીજો નિયમ}: EMF નું મેગ્નિટ્યૂડ ફ્લક્સ ચેન્જના દર સમાન હોય છે
\end{itemize}

\textbf{ગાણિતિક અભિવ્યક્તિ:}

\begin{verbatim}
e = -N \times (dΦ/dt)
\end{verbatim}

\textbf{આકૃતિ:}

\begin{verbatim}
    +{-{-}{-}{-}{-}{-}{-}+}
    |   N   |  N વળાંકો સાથે કોઇલ
    |       |
    +{-{-}{-}+{-}{-}{-}+}
        |
        |  ફરતું ચુંબક
    +{-{-}{-}v{-}{-}{-}+}
    | S | N |  
    +{-{-}{-}{-}{-}{-}{-}+}
       ↕
   ગતિની દિશા
\end{verbatim}

\textbf{ઉપયોગિતા:}

\begin{itemize}
\tightlist
\item
  \textbf{ટ્રાન્સફોર્મર}: મ્યુચ્યુઅલ ઇન્ડક્શન સિદ્ધાંત
\item
  \textbf{જનરેટર}: મિકેનિકલથી ઇલેક્ટ્રિકલ એનર્જી કન્વર્ઝન
\item
  \textbf{ઇન્ડક્ટર}: સેલ્ફ-ઇન્ડ્યૂસ્ડ EMF કરંટ ચેન્જનો વિરોધ કરે છે
\end{itemize}

\end{solutionbox}
\begin{mnemonicbox}
``ફ્લક્સ ચેન્જ જનરેટ્સ EMF'' (dΦ/dt = EMF)

\end{mnemonicbox}
\subsection*{પ્રશ્ન 1(ક) [7
ગુણ]}\label{uxaaauxab0uxab6uxaa8-1uxa95-7-uxa97uxaa3}

\textbf{કિર્ચહોફના વોલ્ટેજના નિયમ અને કિર્ચહોફના કરંટના નિયમને જરૂરી આકૃતિ સાથે
સમજાવો.}

\begin{solutionbox}

\textbf{ટેબલ: કિર્ચહોફના નિયમોની તુલના}

\begin{longtable}[]{@{}llll@{}}
\toprule\noalign{}
નિયમ & વિધાન & ગાણિતિક સ્વરૂપ & ઉપયોગ \\
\midrule\noalign{}
\endhead
\bottomrule\noalign{}
\endlastfoot
\textbf{KVL} & બંધ લૂપમાં વોલ્ટેજનો સરવાળો = 0 & ΣV = 0 & સિરીઝ સર્કિટ \\
\textbf{KCL} & નોડ પર કરંટનો સરવાળો = 0 & ΣI = 0 & પેરેલલ સર્કિટ \\
\end{longtable}

\textbf{KVL આકૃતિ:}

\begin{center}
\textbf{Mermaid Diagram (Code)}
\begin{verbatim}
{Shaded}
{Highlighting}[]
graph LR
    A["{+"] {-}{-}{} B[V1]}
    B {-{-}{} C[R1]}
    C {-{-}{} D[V2]}
    D {-{-}{} E[R2]}
    E {-{-}{} A}
    
    style A fill:\#f9f,stroke:\#333,stroke{-width:2px}
    style C fill:\#bbf,stroke:\#333,stroke{-width:2px}
    style E fill:\#bbf,stroke:\#333,stroke{-width:2px}
{Highlighting}
{Shaded}
\end{verbatim}
\end{center}

\textbf{KCL આકૃતિ:}

\begin{center}
\textbf{Mermaid Diagram (Code)}
\begin{verbatim}
{Shaded}
{Highlighting}[]
graph TD
    A[I1] {-{-}{} B((નોડ))}
    C[I2] {-{-}{} B}
    B {-{-}{} D[I3]}
    B {-{-}{} E[I4]}
    
    style B fill:\#f96,stroke:\#333,stroke{-width:4px}
{Highlighting}
{Shaded}
\end{verbatim}
\end{center}

\textbf{મુખ્ય મુદ્દાઓ:}

\begin{itemize}
\tightlist
\item
  \textbf{KVL}: બીજગણિતીય સરવાળો વોલ્ટેજ પોલેરિટી ધ્યાનમાં રાખે છે
\item
  \textbf{KCL}: કરંટની દિશાઓ ધ્યાનમાં રાખે છે (આવતો વિ જતો)
\item
  \textbf{ઉપયોગિતા}: સર્કિટ વિશ્લેષણ, અજાણા મૂલ્યો શોધવા
\end{itemize}

\end{solutionbox}
\begin{mnemonicbox}
``વોલ્ટેજ લૂપ્સ, કરંટ નોડ્સ'' (KVL લૂપ માટે, KCL નોડ માટે)

\end{mnemonicbox}
\subsection*{પ્રશ્ન 1(ક અથવા) [7
ગુણ]}\label{uxaaauxab0uxab6uxaa8-1uxa95-uxa85uxaa5uxab5-7-uxa97uxaa3}

\textbf{સ્ટેટિકલી ઇન્ડ્યૂસ્ડ EMF અને ડાયનેમિકલી ઇન્ડ્યૂસ્ડ EMF વચ્ચેનો તફાવત સમજાવો.}

\begin{solutionbox}

\textbf{ટેબલ: સ્ટેટિક વિ ડાયનેમિક EMF}

\begin{longtable}[]{@{}lll@{}}
\toprule\noalign{}
પેરામીટર & સ્ટેટિકલી ઇન્ડ્યૂસ્ડ EMF & ડાયનેમિકલી ઇન્ડ્યૂસ્ડ EMF \\
\midrule\noalign{}
\endhead
\bottomrule\noalign{}
\endlastfoot
\textbf{કારણ} & બદલાતું મેગ્નેટિક ફીલ્ડ & વાહક અને ફીલ્ડ વચ્ચે સંબંધિત ગતિ \\
\textbf{ફીલ્ડ} & સમય-બદલાતું, વાહક સ્થિર & સ્થિર ફીલ્ડ, વાહક ગતિશીલ \\
\textbf{ઉદાહરણો} & ટ્રાન્સફોર્મર, ઇન્ડક્ટર & જનરેટર, મોટર \\
\textbf{સૂત્ર} & e = -N(dΦ/dt) & e = BLv \\
\textbf{ઉપયોગિતા} & AC સર્કિટ, પાવર સપ્લાય & પાવર જનરેશન, મોટર્સ \\
\end{longtable}

\textbf{સ્ટેટિક EMF ના પ્રકારો:}

\begin{itemize}
\tightlist
\item
  \textbf{સેલ્ફ-ઇન્ડ્યૂસ્ડ}: એક જ કોઇલ ફ્લક્સ ચેન્જ બનાવે અને અનુભવે છે
\item
  \textbf{મ્યુચ્યુઅલી ઇન્ડ્યૂસ્ડ}: એક કોઇલ બીજી કોઇલને અસર કરે છે
\end{itemize}

\textbf{ડાયનેમિક EMF ના પરિબળો:}

\begin{itemize}
\tightlist
\item
  \textbf{મેગ્નેટિક ફીલ્ડ સ્ટ્રેન્થ (B)}: ટેસ્લા
\item
  \textbf{કન્ડક્ટર લેન્થ (L)}: મીટર
\item
  \textbf{વેલોસિટી (v)}: m/s
\end{itemize}

\end{solutionbox}
\begin{mnemonicbox}
``સ્ટેટિક સ્ટેઝ, ડાયનેમિક ડાન્સ'' (સ્ટેટિક = સ્થિર, ડાયનેમિક =
ગતિ)

\end{mnemonicbox}
\subsection*{પ્રશ્ન 2(અ) [3
ગુણ]}\label{uxaaauxab0uxab6uxaa8-2uxa85-3-uxa97uxaa3}

\textbf{ટ્રાન્સફોર્મરમાં થતાં વિવિધ પ્રકારના લોસ સમજાવો.}

\begin{solutionbox}

\textbf{ટેબલ: ટ્રાન્સફોર્મર લોસ}

\begin{longtable}[]{@{}llll@{}}
\toprule\noalign{}
લોસનો પ્રકાર & કારણ & સ્થાન & લક્ષણો \\
\midrule\noalign{}
\endhead
\bottomrule\noalign{}
\endlastfoot
\textbf{આયર્ન લોસ} & હિસ્ટેરેસિસ + એડી કરંટ & કોર & સ્થિર, ફ્રિક્વન્સી આધારિત \\
\textbf{કોપર લોસ} & I^{2}R હીટિંગ & વાઇન્ડિંગ & લોડ સાથે બદલાતું \\
\textbf{સ્ટ્રે લોસ} & લીકેજ ફ્લક્સ & એકંદર & ન્યૂનતમ \\
\end{longtable}

\textbf{આયર્ન લોસ:}

\begin{itemize}
\tightlist
\item
  \textbf{હિસ્ટેરેસિસ લોસ}: મેગ્નેટિક ડોમેઇન રિવર્સલ એનર્જી
\item
  \textbf{એડી કરંટ લોસ}: કોરમાં ફરતા કરંટ
\end{itemize}

\textbf{કોપર લોસ:}

\begin{itemize}
\tightlist
\item
  \textbf{પ્રાઇમરી વાઇન્ડિંગ}: I_{1}^{2}R_{1}
\item
  \textbf{સેકન્ડરી વાઇન્ડિંગ}: I_{2}^{2}R_{2}
\end{itemize}

\end{solutionbox}
\begin{mnemonicbox}
``આયર્ન કોર, કોપર કોઇલ'' (મુખ્ય લોસનું સ્થાન)

\end{mnemonicbox}
\subsection*{પ્રશ્ન 2(બ) [4
ગુણ]}\label{uxaaauxab0uxab6uxaa8-2uxaac-4-uxa97uxaa3}

\textbf{ટ્રાન્સફોર્મરનો કાર્ય સિદ્ધાંત સમજાવો.}

\begin{solutionbox}

\textbf{કાર્ય સિદ્ધાંત:} સામાન્ય મેગ્નેટિક કોર દ્વારા પ્રાઇમરી અને સેકન્ડરી વાઇન્ડિંગ
વચ્ચે \textbf{મ્યુચ્યુઅલ ઇલેક્ટ્રોમેગ્નેટિક ઇન્ડક્શન}.

\textbf{આકૃતિ:}

\begin{center}
\textbf{Mermaid Diagram (Code)}
\begin{verbatim}
{Shaded}
{Highlighting}[]
graph LR
    AC[AC સપ્લાય] {-{-}{} P[પ્રાઇમરી કોઇલ N1]}
    P {-{-}{} C[આયર્ન કોર]}
    C {-{-}{} S[સેકન્ડરી કોઇલ N2]}
    S {-{-}{} L[લોડ]}
    
    style C fill:\#f96,stroke:\#333,stroke{-width:3px}
    style P fill:\#bbf,stroke:\#333,stroke{-width:2px}
    style S fill:\#bfb,stroke:\#333,stroke{-width:2px}
{Highlighting}
{Shaded}
\end{verbatim}
\end{center}

\textbf{ઓપરેશન સ્ટેપ્સ:}

\begin{itemize}
\tightlist
\item
  \textbf{સ્ટેપ 1}: પ્રાઇમરીમાં AC કરંટ બદલાતું ફ્લક્સ બનાવે છે
\item
  \textbf{સ્ટેપ 2}: ફ્લક્સ કોર દ્વારા સેકન્ડરી સાથે લિંક થાય છે
\item
  \textbf{સ્ટેપ 3}: બદલાતું ફ્લક્સ સેકન્ડરીમાં EMF ઇન્ડ્યૂસ કરે છે
\item
  \textbf{સ્ટેપ 4}: સેકન્ડરી EMF લોડ દ્વારા કરંટ ચલાવે છે
\end{itemize}

\textbf{મુખ્ય સંબંધો:}

\begin{itemize}
\tightlist
\item
  \textbf{વોલ્ટેજ રેશિયો}: V_{2}/V_{1} = N_{2}/N_{1}
\item
  \textbf{કરંટ રેશિયો}: I_{1}/I_{2} = N_{2}/N_{1}
\end{itemize}

\end{solutionbox}
\begin{mnemonicbox}
``પ્રાઇમરી પ્રોડ્યૂસ, સેકન્ડરી સપ્લાય'' (એનર્જી ટ્રાન્સફરની
દિશા)

\end{mnemonicbox}
\subsection*{પ્રશ્ન 2(ક) [7
ગુણ]}\label{uxaaauxab0uxab6uxaa8-2uxa95-7-uxa97uxaa3}

\textbf{ટ્રાન્સફોર્મરનું EMF સૂત્ર તારવો.}

\begin{solutionbox}

\textbf{આપેલા પેરામીટર:}

\begin{itemize}
\tightlist
\item
  \textbf{N_{1}}: પ્રાઇમરી ટર્ન્સ, \textbf{N_{2}}: સેકન્ડરી ટર્ન્સ
\item
  \textbf{Φ_{m}}: મેક્સિમમ ફ્લક્સ, \textbf{f}: ફ્રિક્વન્સી
\end{itemize}

\textbf{EMF ડેરિવેશન:}

\textbf{સ્ટેપ 1: ફ્લક્સ વેરિએશન}

\begin{verbatim}
Φ = Φ_{m} sin(2πft)
\end{verbatim}

\textbf{સ્ટેપ 2: ફ્લક્સ ચેન્જનો દર}

\begin{verbatim}
dΦ/dt = 2πfΦ_{m} cos(2πft)
\end{verbatim}

\textbf{સ્ટેપ 3: મેક્સિમમ રેટ}

\begin{verbatim}
(dΦ/dt)_{m}_{a}_{x} = 2πfΦ_{m}
\end{verbatim}

\textbf{સ્ટેપ 4: RMS EMF સૂત્ર}

\begin{verbatim}
E_{1} = 4.44 \times f \times N_{1} \times Φ_{m}
E_{2} = 4.44 \times f \times N_{2} \times Φ_{m}
\end{verbatim}

\textbf{ટેબલ: EMF સૂત્રના ભાગો}

\begin{longtable}[]{@{}lll@{}}
\toprule\noalign{}
પ્રતીક & પેરામીટર & એકમો \\
\midrule\noalign{}
\endhead
\bottomrule\noalign{}
\endlastfoot
\textbf{E} & RMS EMF & વોલ્ટ \\
\textbf{f} & ફ્રિક્વન્સી & Hz \\
\textbf{N} & ટર્ન્સની સંખ્યા & - \\
\textbf{Φ_{m}} & મેક્સિમમ ફ્લક્સ & વેબર \\
\textbf{4.44} & ફોર્મ ફેક્ટર કોન્સ્ટન્ટ & - \\
\end{longtable}

\textbf{ટ્રાન્સફોર્મેશન રેશિયો:}

\begin{verbatim}
K = E_{2}/E_{1} = N_{2}/N_{1}
\end{verbatim}

\end{solutionbox}
\begin{mnemonicbox}
``ફોર-ફોર્ટી-ફોર ફ્લક્સ ફોર્મ્યુલા'' (4.44 ફેક્ટર)

\end{mnemonicbox}
\subsection*{પ્રશ્ન 2(અ અથવા) [3
ગુણ]}\label{uxaaauxab0uxab6uxaa8-2uxa85-uxa85uxaa5uxab5-3-uxa97uxaa3}

\textbf{ટ્રાન્સફોર્મરની ઉપયોગિતા સમજાવો.}

\begin{solutionbox}

\textbf{ટેબલ: ટ્રાન્સફોર્મર એપ્લિકેશન્સ}

\begin{longtable}[]{@{}lll@{}}
\toprule\noalign{}
ઉપયોગિતા & હેતુ & વોલ્ટેજ લેવલ \\
\midrule\noalign{}
\endhead
\bottomrule\noalign{}
\endlastfoot
\textbf{પાવર ટ્રાન્સમિશન} & ટ્રાન્સમિશન લોસ ઘટાડવા & સ્ટેપ-અપ (400kV) \\
\textbf{ડિસ્ટ્રિબ્યુશન} & ગ્રાહકો માટે સુરક્ષિત વોલ્ટેજ & સ્ટેપ-ડાઉન (230V) \\
\textbf{આઇસોલેશન} & ઇલેક્ટ્રિકલ આઇસોલેશન & 1:1 રેશિયો \\
\textbf{ઇલેક્ટ્રોનિક સર્કિટ} & DC પાવર સપ્લાય & સ્ટેપ-ડાઉન \\
\end{longtable}

\textbf{ઇન્ડસ્ટ્રિયલ એપ્લિકેશન્સ:}

\begin{itemize}
\tightlist
\item
  \textbf{વેલ્ડિંગ ટ્રાન્સફોર્મર}: હાઇ કરંટ, લો વોલ્ટેજ
\item
  \textbf{ઇન્સ્ટ્રુમેન્ટ ટ્રાન્સફોર્મર}: મેઝરમેન્ટ અને પ્રોટેક્શન
\item
  \textbf{ઓડિયો ટ્રાન્સફોર્મર}: ઇમ્પીડન્સ મેચિંગ
\end{itemize}

\end{solutionbox}
\begin{mnemonicbox}
``પાવર ડિસ્ટ્રિબ્યુશન આઇસોલેશન ઇલેક્ટ્રોનિક્સ'' (મુખ્ય એપ્લિકેશન
વિસ્તારો)

\end{mnemonicbox}
\subsection*{પ્રશ્ન 2(બ અથવા) [4
ગુણ]}\label{uxaaauxab0uxab6uxaa8-2uxaac-uxa85uxaa5uxab5-4-uxa97uxaa3}

\textbf{DC મોટર માટે બેક EMF અને ટોર્કનું સૂત્ર લખો.}

\begin{solutionbox}

\textbf{બેક EMF સૂત્ર:}

\begin{verbatim}
Eb = (φ \times Z \times N \times P) / (60 \times A)
\end{verbatim}

\textbf{સરળ સ્વરૂપ:}

\begin{verbatim}
Eb = K \times φ \times N
\end{verbatim}

\textbf{ટોર્ક સૂત્ર:}

\begin{verbatim}
T = (φ \times Z \times Ia \times P) / (2π \times A)
\end{verbatim}

\textbf{સરળ સ્વરૂપ:}

\begin{verbatim}
T = K \times φ \times Ia
\end{verbatim}

\textbf{ટેબલ: પ્રતીકોની વ્યાખ્યા}

\begin{longtable}[]{@{}lll@{}}
\toprule\noalign{}
પ્રતીક & પેરામીટર & એકમો \\
\midrule\noalign{}
\endhead
\bottomrule\noalign{}
\endlastfoot
\textbf{Eb} & બેક EMF & વોલ્ટ \\
\textbf{T} & ટોર્ક & N-m \\
\textbf{φ} & ફ્લક્સ પર પોલ & વેબર \\
\textbf{N} & સ્પીડ & RPM \\
\textbf{Ia} & આર્મેચર કરંટ & એમ્પિયર \\
\textbf{K} & મોટર કોન્સ્ટન્ટ & - \\
\end{longtable}

\end{solutionbox}
\begin{mnemonicbox}
``બેક EMF વિરોધ કરે, ટોર્ક પ્રસ્તાવિત કરે'' (EMF સપ્લાયનો
વિરોધ, ટોર્ક રોટેશન ચલાવે)

\end{mnemonicbox}
\subsection*{પ્રશ્ન 2(ક અથવા) [7
ગુણ]}\label{uxaaauxab0uxab6uxaa8-2uxa95-uxa85uxaa5uxab5-7-uxa97uxaa3}

\textbf{DC મોટરની રચના અને કાર્ય પદ્ધતિ આકૃતિ સાથે સમજાવો.}

\begin{solutionbox}

\textbf{રચનાના ભાગો:}

\textbf{ટેબલ: DC મોટરના પાર્ટ્સ}

\begin{longtable}[]{@{}lll@{}}
\toprule\noalign{}
કોમ્પોનન્ટ & કાર્ય & મટીરિયલ \\
\midrule\noalign{}
\endhead
\bottomrule\noalign{}
\endlastfoot
\textbf{સ્ટેટર} & મેગ્નેટિક ફીલ્ડ પ્રદાન કરે છે & કાસ્ટ આયર્ન/સ્ટીલ \\
\textbf{રોટર/આર્મેચર} & ફરતો ભાગ & સિલિકોન સ્ટીલ લેમિનેશન્સ \\
\textbf{કોમ્યુટેટર} & કરંટ દિશા બદલવા & કોપર સેગમેન્ટ્સ \\
\textbf{બ્રશેસ} & કરંટ સંગ્રહ & કાર્બન \\
\textbf{ફીલ્ડ વાઇન્ડિંગ} & ઇલેક્ટ્રોમેગ્નેટ & કોપર વાયર \\
\end{longtable}

\textbf{રચના આકૃતિ:}

\begin{verbatim}
graph TB
    subgraph "DC મોટર"
        N[N પોલ] {-{-} A[આર્મેચર]}
        A {-{-} S[S પોલ]}
        C[કોમ્યુટેટર] {-{-} B[બ્રશેસ]}
        F[ફીલ્ડ વાઇન્ડિંગ] {-{-} N}
        F {-{-} S}
    end
    
    style A fill:\#f96,stroke:\#333,stroke{-width:3px}
    style C fill:\#bbf,stroke:\#333,stroke{-width:2px}
\end{verbatim}

\textbf{કાર્ય સિદ્ધાંત:}

\begin{itemize}
\tightlist
\item
  \textbf{સ્ટેપ 1}: આર્મેચર કન્ડક્ટર દ્વારા કરંટ પસાર થાય છે
\item
  \textbf{સ્ટેપ 2}: મેગ્નેટિક ફીલ્ડ કરંટ સાથે ઇન્ટરેક્ટ થાય છે
\item
  \textbf{સ્ટેપ 3}: ફ્લેમિંગના ડાબા હાથના નિયમ દ્વારા બળ પેદા થાય છે
\item
  \textbf{સ્ટેપ 4}: કોમ્યુટેટર કરંટની દિશા બદલે છે
\item
  \textbf{સ્ટેપ 5}: સતત રોટેશન જાળવાય છે
\end{itemize}

\textbf{બળનું સૂત્ર:}

\begin{verbatim}
F = B \times I \times L
\end{verbatim}

\end{solutionbox}
\begin{mnemonicbox}
``કરંટ ક્રિએટ્સ સર્ક્યુલર મોશન'' (કરંટ ઇન્ટરેક્શન રોટેશન પેદા કરે
છે)

\end{mnemonicbox}
\subsection*{પ્રશ્ન 3(અ) [3
ગુણ]}\label{uxaaauxab0uxab6uxaa8-3uxa85-3-uxa97uxaa3}

\textbf{ટ્રાન્સફોર્મરની રચના સમજાવો.}

\begin{solutionbox}

\textbf{ટેબલ: ટ્રાન્સફોર્મર કન્સ્ટ્રક્શન}

\begin{longtable}[]{@{}lll@{}}
\toprule\noalign{}
કોમ્પોનન્ટ & મટીરિયલ & કાર્ય \\
\midrule\noalign{}
\endhead
\bottomrule\noalign{}
\endlastfoot
\textbf{કોર} & સિલિકોન સ્ટીલ લેમિનેશન્સ & મેગ્નેટિક ફ્લક્સ પાથ \\
\textbf{પ્રાઇમરી વાઇન્ડિંગ} & કોપર/એલ્યુમિનિયમ & ઇનપુટ એનર્જી \\
\textbf{સેકન્ડરી વાઇન્ડિંગ} & કોપર/એલ્યુમિનિયમ & આઉટપુટ એનર્જી \\
\textbf{ઇન્સ્યુલેશન} & વાર્નિશ/પેપર & ઇલેક્ટ્રિકલ આઇસોલેશન \\
\textbf{ટાંકી} & સ્ટીલ & ઓઇલ કન્ટેઇનમેન્ટ અને કૂલિંગ \\
\end{longtable}

\textbf{કોરના પ્રકારો:}

\begin{itemize}
\tightlist
\item
  \textbf{શેલ ટાઇપ}: વાઇન્ડિંગ કોર દ્વારા ઘેરાયેલું
\item
  \textbf{કોર ટાઇપ}: કોર વાઇન્ડિંગ દ્વારા ઘેરાયેલો
\end{itemize}

\textbf{કૂલિંગ મેથડ્સ:}

\begin{itemize}
\tightlist
\item
  \textbf{એર કૂલિંગ}: નાના ટ્રાન્સફોર્મર
\item
  \textbf{ઓઇલ કૂલિંગ}: મોટા ટ્રાન્સફોર્મર રેડિએટર સાથે
\end{itemize}

\end{solutionbox}
\begin{mnemonicbox}
``કોર કેરીઝ કરંટ કેરફુલી'' (કોર ડિઝાઇનનું મહત્વ)

\end{mnemonicbox}
\subsection*{પ્રશ્ન 3(બ) [4
ગુણ]}\label{uxaaauxab0uxab6uxaa8-3uxaac-4-uxa97uxaa3}

\textbf{DC મોટરની ઉપયોગિતા સમજાવો.}

\begin{solutionbox}

\textbf{ટેબલ: DC મોટર એપ્લિકેશન્સ}

\begin{longtable}[]{@{}lll@{}}
\toprule\noalign{}
મોટરનો પ્રકાર & સ્પીડ લક્ષણ & ઉપયોગિતા \\
\midrule\noalign{}
\endhead
\bottomrule\noalign{}
\endlastfoot
\textbf{શન્ટ} & સ્થિર સ્પીડ & ફેન, પંપ, લેથ \\
\textbf{સિરીઝ} & બદલાતી સ્પીડ & ટ્રેક્શન, ક્રેન \\
\textbf{કમ્પાઉન્ડ} & મધ્યમ વેરિએશન & એલિવેટર, કોમ્પ્રેસર \\
\end{longtable}

\textbf{ઇન્ડસ્ટ્રિયલ એપ્લિકેશન્સ:}

\begin{itemize}
\tightlist
\item
  \textbf{શન્ટ મોટર}: મશીન ટૂલ્સ જેને સ્થિર સ્પીડ જોઇએ
\item
  \textbf{સિરીઝ મોટર}: ઇલેક્ટ્રિક વાહનો, ભારે લોડ સ્ટાર્ટિંગ
\item
  \textbf{કમ્પાઉન્ડ મોટર}: રોલિંગ મિલ્સ, પંચ પ્રેસ
\end{itemize}

\textbf{ફાયદાઓ:}

\begin{itemize}
\tightlist
\item
  \textbf{સરળ સ્પીડ કન્ટ્રોલ}: વોલ્ટેજ/ફીલ્ડ કન્ટ્રોલ
\item
  \textbf{ઉચ્ચ સ્ટાર્ટિંગ ટોર્ક}: સિરીઝ મોટર
\item
  \textbf{રિવર્સિબલ ઓપરેશન}: ફીલ્ડ/આર્મેચર પોલેરિટી બદલો
\end{itemize}

\end{solutionbox}
\begin{mnemonicbox}
``શન્ટ સ્ટેઝ, સિરીઝ સ્પીડ્સ'' (સ્પીડ લક્ષણો)

\end{mnemonicbox}
\subsection*{પ્રશ્ન 3(ક) [7
ગુણ]}\label{uxaaauxab0uxab6uxaa8-3uxa95-7-uxa97uxaa3}

\textbf{DC મોટરના વિવિધ પ્રકાર સમજાવો.}

\begin{solutionbox}

\textbf{ટેબલ: DC મોટર વર્ગીકરણ}

\begin{longtable}[]{@{}llll@{}}
\toprule\noalign{}
પ્રકાર & ફીલ્ડ કનેક્શન & સ્પીડ-ટોર્ક & ઉપયોગિતા \\
\midrule\noalign{}
\endhead
\bottomrule\noalign{}
\endlastfoot
\textbf{શન્ટ} & આર્મેચરને સમાંતર & સ્થિર સ્પીડ, નીચો સ્ટાર્ટિંગ ટોર્ક & ફેન, પંપ \\
\textbf{સિરીઝ} & આર્મેચર સાથે સિરીઝ & બદલાતી સ્પીડ, ઉચ્ચ સ્ટાર્ટિંગ ટોર્ક &
ટ્રેક્શન \\
\textbf{કમ્પાઉન્ડ} & સિરીઝ અને શન્ટ બંને & મધ્યમ લક્ષણો & સામાન્ય હેતુ \\
\end{longtable}

\textbf{શન્ટ મોટર આકૃતિ:}

\begin{center}
\textbf{Mermaid Diagram (Code)}
\begin{verbatim}
{Shaded}
{Highlighting}[]
graph LR
    V[DC સપ્લાય] {-{-}{} A[આર્મેચર]}
    V {-{-}{} F[ફીલ્ડ વાઇન્ડિંગ]}
    A {-{-}{} V}
    F {-{-}{} V}
    
    style A fill:\#f96,stroke:\#333,stroke{-width:2px}
    style F fill:\#bbf,stroke:\#333,stroke{-width:2px}
{Highlighting}
{Shaded}
\end{verbatim}
\end{center}

\textbf{લક્ષણો:}

\begin{itemize}
\tightlist
\item
  \textbf{શન્ટ}: સ્પીડ ∝ (V - IaRa)/φ
\item
  \textbf{સિરીઝ}: ઉચ્ચ સ્ટાર્ટિંગ ટોર્ક, સ્પીડ લોડ સાથે બદલાય છે
\item
  \textbf{કમ્પાઉન્ડ}: બંને પ્રકારના ફાયદાઓ સંયોજિત
\end{itemize}

\textbf{સ્પીડ કન્ટ્રોલ મેથડ્સ:}

\begin{itemize}
\tightlist
\item
  \textbf{આર્મેચર કન્ટ્રોલ}: આર્મેચર વોલ્ટેજ બદલો
\item
  \textbf{ફીલ્ડ કન્ટ્રોલ}: ફીલ્ડ કરંટ બદલો
\item
  \textbf{રેઝિસ્ટન્સ કન્ટ્રોલ}: બાહ્ય રેઝિસ્ટન્સ ઉમેરો
\end{itemize}

\end{solutionbox}
\begin{mnemonicbox}
``શન્ટ સ્ટેડી, સિરીઝ સ્ટ્રોંગ, કમ્પાઉન્ડ કમ્બાઇન્ડ'' (મુખ્ય
લક્ષણો)

\end{mnemonicbox}
\subsection*{પ્રશ્ન 3(અ અથવા) [3
ગુણ]}\label{uxaaauxab0uxab6uxaa8-3uxa85-uxa85uxaa5uxab5-3-uxa97uxaa3}

\textbf{ટ્રાન્સફોર્મરનો ટ્રાન્સફોર્મેશન રેશિયો સમજાવો.}

\begin{solutionbox}

\textbf{વ્યાખ્યા:} ટ્રાન્સફોર્મેશન રેશિયો (K) એ સેકન્ડરી અને પ્રાઇમરી વોલ્ટેજ અથવા
ટર્ન્સનો રેશિયો છે.

\textbf{ગાણિતિક અભિવ્યક્તિ:}

\begin{verbatim}
K = N_{2}/N_{1} = E_{2}/E_{1} = V_{2}/V_{1}
\end{verbatim}

\textbf{ટેબલ: ટ્રાન્સફોર્મેશન રેશિયોના પ્રકારો}

\begin{longtable}[]{@{}llll@{}}
\toprule\noalign{}
રેશિયો & પ્રકાર & વોલ્ટેજ ચેન્જ & ઉપયોગિતા \\
\midrule\noalign{}
\endhead
\bottomrule\noalign{}
\endlastfoot
\textbf{K \textgreater{} 1} & સ્ટેપ-અપ & વધારે છે & પાવર ટ્રાન્સમિશન \\
\textbf{K \textless{} 1} & સ્ટેપ-ડાઉન & ઘટાડે છે & ડિસ્ટ્રિબ્યુશન \\
\textbf{K = 1} & આઇસોલેશન & સમાન & સુરક્ષા આઇસોલેશન \\
\end{longtable}

\textbf{કરંટ સંબંધ:}

\begin{verbatim}
I_{1}/I_{2} = N_{2}/N_{1} = K
\end{verbatim}

\textbf{પાવર સંબંધ:}

\begin{verbatim}
P_{1} = P_{2} (આદર્શ ટ્રાન્સફોર્મર)
\end{verbatim}

\end{solutionbox}
\begin{mnemonicbox}
``ટર્ન્સ ટેલ ટ્રાન્સફોર્મેશન'' (ટર્ન્સ રેશિયો વોલ્ટેજ રેશિયો નક્કી
કરે છે)

\end{mnemonicbox}
\subsection*{પ્રશ્ન 3(બ અથવા) [4
ગુણ]}\label{uxaaauxab0uxab6uxaa8-3uxaac-uxa85uxaa5uxab5-4-uxa97uxaa3}

\textbf{ઓટો ટ્રાન્સફોર્મરની ઉપયોગિતા સમજાવો.}

\begin{solutionbox}

\textbf{ટેબલ: ઓટો ટ્રાન્સફોર્મર એપ્લિકેશન્સ}

\begin{longtable}[]{@{}lll@{}}
\toprule\noalign{}
ઉપયોગિતા & ફાયદો & વોલ્ટેજ રેન્જ \\
\midrule\noalign{}
\endhead
\bottomrule\noalign{}
\endlastfoot
\textbf{મોટર સ્ટાર્ટિંગ} & સ્ટાર્ટિંગ કરંટ ઘટાડે છે & રેટેડનો 50-80\% \\
\textbf{વોલ્ટેજ રેગ્યુલેશન} & બારીક વોલ્ટેજ એડજસ્ટમેન્ટ & \pm10\% વેરિએશન \\
\textbf{લેબોરેટરી} & વેરિએબલ વોલ્ટેજ સોર્સ & ઇનપુટનો 0-110\% \\
\textbf{પાવર સિસ્ટમ} & ઇકોનોમિક ટ્રાન્સમિશન & નજીકના વોલ્ટેજ રેશિયો \\
\end{longtable}

\textbf{ફાયદાઓ:}

\begin{itemize}
\tightlist
\item
  \textbf{ઇકોનોમી}: ઓછું કોપર અને આયર્ન જરૂરી
\item
  \textbf{એફિશિયન્સી}: બે-વાઇન્ડિંગ ટ્રાન્સફોર્મર કરતાં વધારે
\item
  \textbf{સાઇઝ}: કોમ્પેક્ટ ડિઝાઇન
\item
  \textbf{રેગ્યુલેશન}: બેહતર વોલ્ટેજ રેગ્યુલેશન
\end{itemize}

\textbf{મર્યાદાઓ:}

\begin{itemize}
\tightlist
\item
  \textbf{આઇસોલેશન નથી}: સામાન્ય ઇલેક્ટ્રિકલ કનેક્શન
\item
  \textbf{સુરક્ષા}: વધારે ફોલ્ટ કરંટ
\end{itemize}

\end{solutionbox}
\begin{mnemonicbox}
``ઓટો એડજસ્ટ્સ એડવાન્ટેજિયસલી'' (ઓટોમેટિક વોલ્ટેજ એડજસ્ટમેન્ટ
ફાયદો)

\end{mnemonicbox}
\subsection*{પ્રશ્ન 3(ક અથવા) [7
ગુણ]}\label{uxaaauxab0uxab6uxaa8-3uxa95-uxa85uxaa5uxab5-7-uxa97uxaa3}

\textbf{DC શન્ટ મોટર માટે સ્પીડ કન્ટ્રોલ કરવાની રીતો સમજાવો.}

\begin{solutionbox}

\textbf{ટેબલ: સ્પીડ કન્ટ્રોલ મેથડ્સ}

\begin{longtable}[]{@{}llll@{}}
\toprule\noalign{}
મેથડ & રેન્જ & એફિશિયન્સી & ઉપયોગિતા \\
\midrule\noalign{}
\endhead
\bottomrule\noalign{}
\endlastfoot
\textbf{આર્મેચર કન્ટ્રોલ} & રેટેડ સ્પીડથી નીચે & ઉચ્ચ & પ્રિસાઇઝ સ્પીડ કન્ટ્રોલ \\
\textbf{ફીલ્ડ કન્ટ્રોલ} & રેટેડ સ્પીડથી ઉપર & ઉચ્ચ & કોન્સ્ટન્ટ પાવર ડ્રાઇવ્સ \\
\textbf{રેઝિસ્ટન્સ કન્ટ્રોલ} & રેટેડ સ્પીડથી નીચે & નીચી & સરળ એપ્લિકેશન્સ \\
\end{longtable}

\textbf{આર્મેચર કન્ટ્રોલ આકૃતિ:}

\begin{center}
\textbf{Mermaid Diagram (Code)}
\begin{verbatim}
{Shaded}
{Highlighting}[]
graph LR
    V[વેરિએબલ DC] {-{-}{} R[રિઓસ્ટેટ]}
    R {-{-}{} A[આર્મેચર]}
    A {-{-}{} V}
    V {-{-}{} F[ફીલ્ડ વાઇન્ડિંગ]}
    F {-{-}{} V}
    
    style R fill:\#f96,stroke:\#333,stroke{-width:2px}
    style A fill:\#bbf,stroke:\#333,stroke{-width:2px}
{Highlighting}
{Shaded}
\end{verbatim}
\end{center}

\textbf{સ્પીડ સૂત્રો:}

\begin{itemize}
\tightlist
\item
  \textbf{આર્મેચર કન્ટ્રોલ}: N ∝ (V - IaRa)/φ
\item
  \textbf{ફીલ્ડ કન્ટ્રોલ}: N ∝ V/φ
\item
  \textbf{રેઝિસ્ટન્સ કન્ટ્રોલ}: N ∝ (V - Ia(Ra + Rext))/φ
\end{itemize}

\textbf{આધુનિક મેથડ્સ:}

\begin{itemize}
\tightlist
\item
  \textbf{ચોપર કન્ટ્રોલ}: PWM વોલ્ટેજ કન્ટ્રોલ
\item
  \textbf{વોર્ડ-લિયોનાર્ડ સિસ્ટમ}: મોટર-જનરેટર સેટ
\item
  \textbf{ઇલેક્ટ્રોનિક કન્ટ્રોલ}: થાઇરિસ્ટર/IGBT ડ્રાઇવ્સ
\end{itemize}

\textbf{લક્ષણો:}

\begin{itemize}
\tightlist
\item
  \textbf{સ્મૂથ કન્ટ્રોલ}: સ્ટેપલેસ સ્પીડ વેરિએશન
\item
  \textbf{એફિશિયન્સી}: આર્મેચર કન્ટ્રોલ સૌથી એફિશિયન્ટ
\item
  \textbf{કોસ્ટ}: ફીલ્ડ કન્ટ્રોલ ઇકોનોમિકલ
\end{itemize}

\end{solutionbox}
\begin{mnemonicbox}
``આર્મેચર એક્યુરેટ, ફીલ્ડ ફાસ્ટ, રેઝિસ્ટન્સ રફ'' (કન્ટ્રોલ
લક્ષણો)

\end{mnemonicbox}
\subsection*{પ્રશ્ન 4(અ) [3
ગુણ]}\label{uxaaauxab0uxab6uxaa8-4uxa85-3-uxa97uxaa3}

\textbf{અલ્ટરનેટિંગ EMF નું વેક્ટર નિરૂપણ સમજાવો.}

\begin{solutionbox}

\textbf{વેક્ટર રિપ્રેઝન્ટેશન:} અલ્ટરનેટિંગ EMF ને સ્થિર મેગ્નિટ્યૂડ અને એંગ્યુલર વેલોસિટી
સાથે ફરતા વેક્ટર (ફેઝર) તરીકે દર્શાવી શકાય છે.

\textbf{ગાણિતિક સ્વરૂપ:}

\begin{verbatim}
e = Em sin(ωt + φ)
\end{verbatim}

\textbf{આકૃતિ:}

\begin{center}
\textbf{Mermaid Diagram (Code)}
\begin{verbatim}
{Shaded}
{Highlighting}[]
graph LR
    subgraph "ફેઝર ડાયાગ્રામ"
        O((O)) {-{-}{} E[Em]}
        O {-{-}{} A[ωt]}
    end
    
    style E fill:\#f96,stroke:\#333,stroke{-width:3px}
    style A fill:\#bbf,stroke:\#333,stroke{-width:2px}
{Highlighting}
{Shaded}
\end{verbatim}
\end{center}

\textbf{ટેબલ: વેક્ટર પેરામીટર}

\begin{longtable}[]{@{}llll@{}}
\toprule\noalign{}
પેરામીટર & પ્રતીક & એકમો & વર્ણન \\
\midrule\noalign{}
\endhead
\bottomrule\noalign{}
\endlastfoot
\textbf{મેગ્નિટ્યૂડ} & Em & વોલ્ટ & મેક્સિમમ EMF \\
\textbf{એંગ્યુલર વેલોસિટી} & ω & rad/s & રોટેશન સ્પીડ \\
\textbf{ફેઝ એંગલ} & φ & ડિગ્રી & પ્રારંભિક ફેઝ \\
\textbf{ફ્રિક્વન્સી} & f = ω/2π & Hz & સાઇકલ પર સેકન્ડ \\
\end{longtable}

\textbf{ફાયદાઓ:}

\begin{itemize}
\tightlist
\item
  \textbf{વિઝ્યુઅલ રિપ્રેઝન્ટેશન}: ફેઝ સંબંધો સમજવા સરળ
\item
  \textbf{ગાણિતિક સરળીકરણ}: જટિલ ગણતરીઓ સરળ બનાવે છે
\end{itemize}

\end{solutionbox}
\begin{mnemonicbox}
``વેક્ટર્સ વિઝ્યુઅલાઇઝ વોલ્ટેજ વેરિએશન'' (ફેઝર રિપ્રેઝન્ટેશન
ફાયદાઓ)

\end{mnemonicbox}
\subsection*{પ્રશ્ન 4(બ) [4
ગુણ]}\label{uxaaauxab0uxab6uxaa8-4uxaac-4-uxa97uxaa3}

\textbf{અલ્ટરનેટિંગ કરંટના સંદર્ભમાં નીચેના પદોની વ્યાખ્યા આપો: RMS વેલ્યુ, એવરેજ વેલ્યુ,
ફ્રિક્વન્સી, ટાઇમ પિરિયડ}

\begin{solutionbox}

\textbf{ટેબલ: AC પેરામીટર વ્યાખ્યા}

\begin{longtable}[]{@{}llll@{}}
\toprule\noalign{}
પદ & વ્યાખ્યા & સૂત્ર & એકમો \\
\midrule\noalign{}
\endhead
\bottomrule\noalign{}
\endlastfoot
\textbf{RMS વેલ્યુ} & સમાન હીટિંગ પેદા કરતો અસરકારક મૂલ્ય & Im/\sqrt2 & એમ્પિયર \\
\textbf{એવરેજ વેલ્યુ} & અર્ધ સાઇકલ પર સરેરાશ મૂલ્ય & 2Im/π & એમ્પિયર \\
\textbf{ફ્રિક્વન્સી} & સેકન્ડ દીઠ સાઇકલની સંખ્યા & f = 1/T & Hz \\
\textbf{ટાઇમ પિરિયડ} & એક સંપૂર્ણ સાઇકલ માટેનો સમય & T = 1/f & સેકન્ડ \\
\end{longtable}

\textbf{ગાણિતિક સંબંધો:}

\begin{itemize}
\tightlist
\item
  \textbf{ફોર્મ ફેક્ટર}: RMS/Average = π/2\sqrt2 = 1.11
\item
  \textbf{પીક ફેક્ટર}: Peak/RMS = \sqrt2 = 1.414
\item
  \textbf{એંગ્યુલર ફ્રિક્વન્સી}: ω = 2πf
\end{itemize}

\textbf{પ્રેક્ટિકલ વેલ્યુઝ:}

\begin{itemize}
\tightlist
\item
  \textbf{RMS કરંટ}: પાવર ગણતરીઓ માટે વપરાય છે
\item
  \textbf{એવરેજ કરંટ}: DC સમકક્ષ માટે વપરાય છે
\item
  \textbf{ફ્રિક્વન્સી}: 50 Hz (ભારત), 60 Hz (યુએસએ)
\end{itemize}

\end{solutionbox}
\begin{mnemonicbox}
``રિયલી મીન સ્ક્વેર, એવરેજ ફ્રિક્વન્સી ટાઇમ'' (મુખ્ય AC
પેરામીટર)

\end{mnemonicbox}
\subsection*{પ્રશ્ન 4(ક) [7
ગુણ]}\label{uxaaauxab0uxab6uxaa8-4uxa95-7-uxa97uxaa3}

\textbf{સ્ટાર જોડાણમાં લાઇન વોલ્ટેજ અને ફેઇઝ વોલ્ટેજ તથા લાઇન કરંટ અને ફેઇઝ કરંટ
વચ્ચેનો સંબંધ દર્શાવતા સૂત્ર તારવો.}

\begin{solutionbox}

\textbf{સ્ટાર કનેક્શન આકૃતિ:}

\begin{center}
\textbf{Mermaid Diagram (Code)}
\begin{verbatim}
{Shaded}
{Highlighting}[]
graph TD
    R[R ફેઝ] {-{-}{} N[ન્યુટ્રલ N]}
    Y[Y ફેઝ] {-{-}{} N}
    B[B ફેઝ] {-{-}{} N}
    
    R {-{-}{} LR[લાઇન R]}
    Y {-{-}{} LY[લાઇન Y]  }
    B {-{-}{} LB[લાઇન B]}
    
    style N fill:\#f96,stroke:\#333,stroke{-width:3px}
{Highlighting}
{Shaded}
\end{verbatim}
\end{center}

\textbf{વોલ્ટેજ સંબંધો:}

\textbf{ફેઝ વોલ્ટેજ:} VR, VY, VB (ન્યુટ્રલ સંદર્ભે) \textbf{લાઇન વોલ્ટેજ:} VRY,
VYB, VBR (લાઇન વચ્ચે)

\textbf{ફેઝર વિશ્લેષણ:}

\begin{verbatim}
VRY = VR - VY
\end{verbatim}

\textbf{બેલેન્સ્ડ સિસ્ટમ માટે:}

\begin{itemize}
\tightlist
\item
  ફેઝ વોલ્ટેજ મેગ્નિટ્યૂડમાં સમાન: VR = VY = VB = Vph
\item
  ફેઝ ડિફરન્સ = 120^\circ
\end{itemize}

\textbf{વેક્ટર એડિશન:} ફેઝર ડાયાગ્રામ અને કોસાઇન નિયમનો ઉપયોગ કરીને:

\begin{verbatim}
VL = \sqrt(Vph^{2} + Vph^{2} - 2Vph·Vph·cos(120^\circ))
VL = \sqrt(2Vph^{2} + Vph^{2}) = \sqrt3 \times Vph
\end{verbatim}

\textbf{અંતિમ સંબંધો:}

\textbf{ટેબલ: સ્ટાર કનેક્શન સંબંધો}

\begin{longtable}[]{@{}ll@{}}
\toprule\noalign{}
પેરામીટર & સંબંધ \\
\midrule\noalign{}
\endhead
\bottomrule\noalign{}
\endlastfoot
\textbf{લાઇન વોલ્ટેજ} & VL = \sqrt3 \times Vph \\
\textbf{લાઇન કરંટ} & IL = Iph \\
\textbf{પાવર} & P = \sqrt3 \times VL \times IL \times cosφ \\
\end{longtable}

\textbf{કરંટ સંબંધો:} સ્ટાર કનેક્શનમાં, લાઇન કરંટ ફેઝ કરંટ સમાન હોય છે:

\begin{verbatim}
IL = Iph
\end{verbatim}

\end{solutionbox}
\begin{mnemonicbox}
``સ્ટાર સ્કેલ્સ વોલ્ટેજ, સેમ કરંટ'' (વોલ્ટેજ માટે \sqrt3 ફેક્ટર, કરંટ
અપરિવર્તિત)

\end{mnemonicbox}
\subsection*{પ્રશ્ન 4(અ અથવા) [3
ગુણ]}\label{uxaaauxab0uxab6uxaa8-4uxa85-uxa85uxaa5uxab5-3-uxa97uxaa3}

\textbf{અલ્ટરનેટિંગ કરંટનું વેક્ટર નિરૂપણ સમજાવો.}

\begin{solutionbox}

\textbf{વેક્ટર રિપ્રેઝન્ટેશન:} AC કરંટને મેગ્નિટ્યૂડ અને ફેઝ એંગલ સાથે ફરતા ફેઝર તરીકે
દર્શાવાય છે.

\textbf{ગાણિતિક અભિવ્યક્તિ:}

\begin{verbatim}
i = Im sin(ωt + φ)
\end{verbatim}

\textbf{ફેઝર ડાયાગ્રામ:}

\begin{verbatim}
     Im
      ↗
     /|
    / |
   /  |φ  
  O{-{-}{-}+{-}{-}{-} રેફરન્સ}
      ωt
\end{verbatim}

\textbf{ટેબલ: કરંટ વેક્ટર એલિમેન્ટ્સ}

\begin{longtable}[]{@{}lll@{}}
\toprule\noalign{}
એલિમેન્ટ & પ્રતીક & વર્ણન \\
\midrule\noalign{}
\endhead
\bottomrule\noalign{}
\endlastfoot
\textbf{મેગ્નિટ્યૂડ} & Im & પીક કરંટ વેલ્યુ \\
\textbf{ફેઝ} & φ & લીડિંગ/લેગિંગ એંગલ \\
\textbf{એંગ્યુલર વેલોસિટી} & ω & રોટેશન સ્પીડ \\
\textbf{RMS વેલ્યુ} & I = Im/\sqrt2 & અસરકારક કરંટ \\
\end{longtable}

\textbf{ઉપયોગિતા:}

\begin{itemize}
\tightlist
\item
  \textbf{સર્કિટ વિશ્લેષણ}: વોલ્ટેજ અને કરંટ વચ્ચે ફેઝ સંબંધો
\item
  \textbf{પાવર ગણતરીઓ}: રિયલ અને રિએક્ટિવ પાવર કોમ્પોનન્ટ્સ
\end{itemize}

\end{solutionbox}
\begin{mnemonicbox}
``કરંટ સર્કલ્સ કન્ટિન્યુઅસલી'' (ફરતા ફેઝર કન્સેપ્ટ)

\end{mnemonicbox}
\subsection*{પ્રશ્ન 4(બ અથવા) [4
ગુણ]}\label{uxaaauxab0uxab6uxaa8-4uxaac-uxa85uxaa5uxab5-4-uxa97uxaa3}

\textbf{અલ્ટરનેટિંગ કરંટના સંદર્ભમાં નીચેના પદોની વ્યાખ્યા આપો: ફોર્મ ફેક્ટર, પીક
ફેક્ટર, કોણીય વેગ, એમ્પ્લિટ્યૂડ}

\begin{solutionbox}

\textbf{ટેબલ: AC કરંટ પેરામીટર}

\begin{longtable}[]{@{}llll@{}}
\toprule\noalign{}
પદ & વ્યાખ્યા & સૂત્ર & સામાન્ય મૂલ્ય \\
\midrule\noalign{}
\endhead
\bottomrule\noalign{}
\endlastfoot
\textbf{ફોર્મ ફેક્ટર} & RMS/Average વેલ્યુ રેશિયો & Irms/Iavg & 1.11 (સાઇન
વેવ) \\
\textbf{પીક ફેક્ટર} & Peak/RMS વેલ્યુ રેશિયો & Im/Irms & 1.414 (સાઇન વેવ) \\
\textbf{એંગ્યુલર વેલોસિટી} & ફેઝ ચેન્જનો દર & ω = 2πf & 314 rad/s (50Hz) \\
\textbf{એમ્પ્લિટ્યૂડ} & મેક્સિમમ ઇન્સ્ટન્ટેનિયસ વેલ્યુ & Im & પીક કરંટ \\
\end{longtable}

\textbf{ગાણિતિક સંબંધો:}

\begin{itemize}
\tightlist
\item
  \textbf{ફોર્મ ફેક્ટર}: વેવફોર્મ શેપ દર્શાવે છે
\item
  \textbf{પીક ફેક્ટર}: ક્રેસ્ટ ફેક્ટર દર્શાવે છે
\item
  \textbf{એંગ્યુલર વેલોસિટી}: ફ્રિક્વન્સી અને ફેઝ લિંક કરે છે
\item
  \textbf{એમ્પ્લિટ્યૂડ}: RMS અને એવરેજ વેલ્યુઝ નક્કી કરે છે
\end{itemize}

\textbf{પ્રેક્ટિકલ મહત્વ:}

\begin{itemize}
\tightlist
\item
  \textbf{ડિઝાઇન વિચારણાઓ}: ઇન્સ્યુલેશન માટે પીક ફેક્ટર
\item
  \textbf{વેવફોર્મ વિશ્લેષણ}: ડિસ્ટોર્શન માટે ફોર્મ ફેક્ટર
\item
  \textbf{સિંક્રોનાઇઝેશન}: ટાઇમિંગ માટે એંગ્યુલર વેલોસિટી
\end{itemize}

\end{solutionbox}
\begin{mnemonicbox}
``ફોર્મ પીક એંગ્યુલર એમ્પ્લિટ્યૂડ'' (ચાર મુખ્ય ફેક્ટર)

\end{mnemonicbox}
\subsection*{પ્રશ્ન 4(ક અથવા) [7
ગુણ]}\label{uxaaauxab0uxab6uxaa8-4uxa95-uxa85uxaa5uxab5-7-uxa97uxaa3}

\textbf{ડેલ્ટા જોડાણમાં લાઇન વોલ્ટેજ અને ફેઇઝ વોલ્ટેજ તથા લાઇન કરંટ અને ફેઇઝ કરંટ
વચ્ચેનો સંબંધ દર્શાવતા સૂત્ર તારવો.}

\begin{solutionbox}

\textbf{ડેલ્ટા કનેક્શન આકૃતિ:}

\begin{center}
\textbf{Mermaid Diagram (Code)}
\begin{verbatim}
{Shaded}
{Highlighting}[]
graph LR
    subgraph "ડેલ્ટા કનેક્શન"
        A[A] {-{-}{} B[B]}
        B {-{-}{} C[C]}
        C {-{-}{} A}
    end
    
    A {-{-}{} IA[IA]}
    B {-{-}{} IB[IB]}
    C {-{-}{} IC[IC]}
    
    style A fill:\#f96,stroke:\#333,stroke{-width:2px}
    style B fill:\#bbf,stroke:\#333,stroke{-width:2px}
    style C fill:\#bfb,stroke:\#333,stroke{-width:2px}
{Highlighting}
{Shaded}
\end{verbatim}
\end{center}

\textbf{વોલ્ટેજ સંબંધો:} ડેલ્ટા કનેક્શનમાં, લાઇન વોલ્ટેજ ફેઝ વોલ્ટેજ સમાન હોય છે:

\begin{verbatim}
VL = Vph
\end{verbatim}

\textbf{કરંટ વિશ્લેષણ:} દરેક લાઇન કરંટ બે ફેઝ કરંટનો વેક્ટર સમ છે.

\textbf{લાઇન કરંટ IA માટે:}

\begin{verbatim}
IA = IAB - ICA
\end{verbatim}

\textbf{ફેઝર વિશ્લેષણ:} બેલેન્સ્ડ સિસ્ટમ માટે ફેઝ કરંટ મેગ્નિટ્યૂડમાં સમાન:

\begin{itemize}
\tightlist
\item
  IAB = ICA = ICB = Iph
\item
  કરંટ વચ્ચે ફેઝ ડિફરન્સ = 120^\circ
\end{itemize}

\textbf{વેક્ટર સબટ્રેક્શન:}

\begin{verbatim}
IA = IAB - ICA = IAB - (-ICA)
\end{verbatim}

ફેઝર ડાયાગ્રામનો ઉપયોગ કરીને:

\begin{verbatim}
IL = \sqrt(Iph^{2} + Iph^{2} - 2Iph·Iph·cos(60^\circ))
IL = \sqrt(2Iph^{2} - Iph^{2}) = \sqrt3 \times Iph
\end{verbatim}

\textbf{અંતિમ સંબંધો:}

\textbf{ટેબલ: ડેલ્ટા કનેક્શન સંબંધો}

\begin{longtable}[]{@{}ll@{}}
\toprule\noalign{}
પેરામીટર & સંબંધ \\
\midrule\noalign{}
\endhead
\bottomrule\noalign{}
\endlastfoot
\textbf{લાઇન વોલ્ટેજ} & VL = Vph \\
\textbf{લાઇન કરંટ} & IL = \sqrt3 \times Iph \\
\textbf{પાવર} & P = \sqrt3 \times VL \times IL \times cosφ \\
\end{longtable}

\end{solutionbox}
\begin{mnemonicbox}
``ડેલ્ટા ડબલ્સ કરંટ, સેમ વોલ્ટેજ'' (કરંટ માટે \sqrt3 ફેક્ટર, વોલ્ટેજ
અપરિવર્તિત)

\end{mnemonicbox}
\subsection*{પ્રશ્ન 5(અ) [3
ગુણ]}\label{uxaaauxab0uxab6uxaa8-5uxa85-3-uxa97uxaa3}

\textbf{શુદ્ધ અવરોધ ધરાવતા પરિપથ માંથી અલ્ટરનેટિંગ કરંટની વર્તણૂક જરૂરી આકૃતિ અને
વેવફોર્મ સાથે સમજાવો.}

\begin{solutionbox}

\textbf{સર્કિટ આકૃતિ:}

\begin{center}
\textbf{Mermaid Diagram (Code)}
\begin{verbatim}
{Shaded}
{Highlighting}[]
graph LR
    AC[AC સોર્સ] {-{-}{} R[રેઝિસ્ટર R]}
    R {-{-}{} AC}
    
    style AC fill:\#f96,stroke:\#333,stroke{-width:2px}
    style R fill:\#bbf,stroke:\#333,stroke{-width:2px}
{Highlighting}
{Shaded}
\end{verbatim}
\end{center}

\textbf{વેવફોર્મ:}

\begin{verbatim}
    V,I
     ↑
     |    /{      /}
     |   /  {    /  }
     |  /    {  /    }
  {-{-}{-}+{-}{-}{-}{-}{-}{-}{-}{-}/{-}{-}{-}{-}{-}{-}/{-} t}
     |        /{      /}
     |       /  {    /}
     |      /    {  /}
     
  V અને I સમાન ફેઝમાં
\end{verbatim}

\textbf{ટેબલ: રેઝિસ્ટર દ્વારા AC}

\begin{longtable}[]{@{}lll@{}}
\toprule\noalign{}
પેરામીટર & સંબંધ & ફેઝ \\
\midrule\noalign{}
\endhead
\bottomrule\noalign{}
\endlastfoot
\textbf{ઓહમનો નિયમ} & V = IR & સમાન ફેઝ \\
\textbf{પાવર} & P = VI = I^{2}R & હંમેશા પોઝિટિવ \\
\textbf{ઇમ્પીડન્સ} & Z = R & શુદ્ધ રેઝિસ્ટિવ \\
\end{longtable}

\textbf{લક્ષણો:}

\begin{itemize}
\tightlist
\item
  \textbf{કરંટ અને વોલ્ટેજ સમાન ફેઝમાં}: કોઈ ફેઝ ડિફરન્સ નથી
\item
  \textbf{પાવર વપરાશ}: સતત પાવર ડિસિપેશન
\item
  \textbf{રેઝિસ્ટન્સ અપરિવર્તિત}: DC વેલ્યુ સમાન
\end{itemize}

\end{solutionbox}
\begin{mnemonicbox}
``રેઝિસ્ટર રિફ્યુઝ ફેઝ શિફ્ટ'' (કોઈ ફેઝ ડિફરન્સ નથી)

\end{mnemonicbox}
\subsection*{પ્રશ્ન 5(બ) [4
ગુણ]}\label{uxaaauxab0uxab6uxaa8-5uxaac-4-uxa97uxaa3}

\textbf{અલ્ટરનેટિંગ કરંટના સંદર્ભમાં નીચેના પદોની વ્યાખ્યા આપો: ઇમ્પીડન્સ, ફેઝ એંગલ,
પાવર ફેક્ટર, રિએક્ટિવ પાવર}

\begin{solutionbox}

\textbf{ટેબલ: AC સર્કિટ પેરામીટર}

\begin{longtable}[]{@{}llll@{}}
\toprule\noalign{}
પદ & વ્યાખ્યા & સૂત્ર & એકમો \\
\midrule\noalign{}
\endhead
\bottomrule\noalign{}
\endlastfoot
\textbf{ઇમ્પીડન્સ} & AC કરંટનો કુલ વિરોધ & Z = \sqrt(R^{2} + X^{2}) & ઓહ્મ \\
\textbf{ફેઝ એંગલ} & V અને I વચ્ચેનો કોણ & φ = tan^{-}^{1}(X/R) & ડિગ્રી \\
\textbf{પાવર ફેક્ટર} & ફેઝ એંગલનો કોસાઇન & PF = cosφ = R/Z & - \\
\textbf{રિએક્ટિવ પાવર} & રિએક્ટિવ કોમ્પોનન્ટમાં પાવર & Q = VI sinφ & VAR \\
\end{longtable}

\textbf{પાવર સંબંધો:}

\begin{itemize}
\tightlist
\item
  \textbf{એક્ટિવ પાવર}: P = VI cosφ (વોટ)
\item
  \textbf{રિએક્ટિવ પાવર}: Q = VI sinφ (VAR)
\item
  \textbf{એપેરન્ટ પાવર}: S = VI (VA)
\end{itemize}

\textbf{પાવર ત્રિકોણ:}

\begin{verbatim}
S^{2} = P^{2} + Q^{2}
\end{verbatim}

\textbf{પ્રેક્ટિકલ મહત્વ:}

\begin{itemize}
\tightlist
\item
  \textbf{ઉચ્ચ પાવર ફેક્ટર}: કાર્યક્ષમ પાવર ઉપયોગ
\item
  \textbf{નીચો પાવર ફેક્ટર}: સમાન પાવર માટે વધારે કરંટ
\item
  \textbf{રિએક્ટિવ પાવર}: કોઈ નેટ એનર્જી ટ્રાન્સફર નથી
\end{itemize}

\end{solutionbox}
\begin{mnemonicbox}
``ઇમ્પીડન્સ ફેઝ પાવર ક્વાડ્રેચર'' (ચાર મુખ્ય AC પેરામીટર)

\end{mnemonicbox}
\subsection*{પ્રશ્ન 5(ક) [7
ગુણ]}\label{uxaaauxab0uxab6uxaa8-5uxa95-7-uxa97uxaa3}

\textbf{જુદા જુદા પ્રકારના પ્રોટેક્ટિવ ડિવાઇસના નામ લખો અને કોઈ પણ એક પ્રોટેક્ટિવ
ડિવાઇસની રચના તથા કાર્ય વિસ્તારથી સમજાવો.}

\begin{solutionbox}

\textbf{ટેબલ: પ્રોટેક્ટિવ ડિવાઇસ}

\begin{longtable}[]{@{}lll@{}}
\toprule\noalign{}
ડિવાઇસ & પ્રોટેક્શન વિરુદ્ધ & ઉપયોગિતા \\
\midrule\noalign{}
\endhead
\bottomrule\noalign{}
\endlastfoot
\textbf{ફ્યુઝ} & ઓવરકરંટ & લો/મિડિયમ વોલ્ટેજ \\
\textbf{MCB} & ઓવરલોડ, શોર્ટ સર્કિટ & ઘરેલું/કોમર્શિયલ \\
\textbf{ELCB} & અર્થ લીકેજ & સુરક્ષા પ્રોટેક્શન \\
\textbf{રિલે} & વિવિધ ફોલ્ટ & ઇન્ડસ્ટ્રિયલ સિસ્ટમ \\
\textbf{સર્જ એરેસ્ટર} & ઓવરવોલ્ટેજ & ટ્રાન્સમિશન લાઇન \\
\end{longtable}

\textbf{MCB (મિનિએચર સર્કિટ બ્રેકર) - વિગતવાર સમજૂતી:}

\textbf{રચના:}

\begin{center}
\textbf{Mermaid Diagram (Code)}
\begin{verbatim}
{Shaded}
{Highlighting}[]
graph LR
    subgraph "MCB રચના"
      direction LR
        C[કોન્ટેક્ટ્સ] {-{-}{} A[આર્ક ચેમ્બર]}
        A {-{-}{} B[બાઇમેટાલિક સ્ટ્રિપ]}
        B {-{-}{} M[મેગ્નેટિક કોઇલ]}
        M {-{-}{} T[ટ્રિપ મેકેનિઝમ]}
    end
    
    style C fill:\#f96,stroke:\#333,stroke{-width:2px}
    style B fill:\#bbf,stroke:\#333,stroke{-width:2px}
    style M fill:\#bfb,stroke:\#333,stroke{-width:2px}
{Highlighting}
{Shaded}
\end{verbatim}
\end{center}

\textbf{કોમ્પોનન્ટ્સ:}

\begin{itemize}
\tightlist
\item
  \textbf{ફિક્સ્ડ અને મૂવિંગ કોન્ટેક્ટ્સ}: કરંટ વહન કરતા ભાગો
\item
  \textbf{બાઇમેટાલિક સ્ટ્રિપ}: થર્મલ પ્રોટેક્શન
\item
  \textbf{ઇલેક્ટ્રોમેગ્નેટિક કોઇલ}: મેગ્નેટિક પ્રોટેક્શન
\item
  \textbf{આર્ક ક્વેન્ચિંગ ચેમ્બર}: આર્ક એક્સ્ટિન્કશન
\item
  \textbf{ઓપરેટિંગ મેકેનિઝમ}: મેન્યુઅલ/ઓટોમેટિક ઓપરેશન
\end{itemize}

\textbf{કાર્ય સિદ્ધાંત:}

\textbf{ઓવરલોડ પ્રોટેક્શન:}

\begin{itemize}
\tightlist
\item
  કરંટ બાઇમેટાલિક સ્ટ્રિપ ગરમ કરે છે
\item
  સ્ટ્રિપ વળે છે અને ટ્રિપ મેકેનિઝમ ઓપરેટ કરે છે
\item
  ટેમ્પરરી ઓવરલોડ્સ વિરુદ્ધ પ્રોટેક્શન માટે ટાઇમ-ડિલે લક્ષણ
\end{itemize}

\textbf{શોર્ટ સર્કિટ પ્રોટેક્શન:}

\begin{itemize}
\tightlist
\item
  ઉચ્ચ ફોલ્ટ કરંટ મજબૂત મેગ્નેટિક ફીલ્ડ બનાવે છે
\item
  ઇલેક્ટ્રોમેગ્નેટિક ફોર્સ ટ્રિપ મેકેનિઝમ ઓપરેટ કરે છે
\item
  સુરક્ષા માટે ઇન્સ્ટન્ટેનિયસ ઓપરેશન
\end{itemize}

\textbf{ફાયદાઓ:}

\begin{itemize}
\tightlist
\item
  \textbf{પુનઃઉપયોગ}: ફોલ્ટ ક્લિયરન્સ પછી રીસેટ
\item
  \textbf{વિશ્વસનીય ઓપરેશન}: ડ્યુઅલ પ્રોટેક્શન મેકેનિઝમ
\item
  \textbf{સરળ મેન્ટેનન્સ}: સુલભ કોન્ટેક્ટ્સ
\end{itemize}

\end{solutionbox}
\begin{mnemonicbox}
``MCB મેગ્નેટિકલી કન્ટ્રોલ્સ બોથ'' (થર્મલ અને મેગ્નેટિક
પ્રોટેક્શન)

\end{mnemonicbox}
\subsection*{પ્રશ્ન 5(અ અથવા) [3
ગુણ]}\label{uxaaauxab0uxab6uxaa8-5uxa85-uxa85uxaa5uxab5-3-uxa97uxaa3}

\textbf{શુદ્ધ ઇન્ડક્ટર ધરાવતા પરિપથ માંથી અલ્ટરનેટિંગ કરંટની વર્તણૂક સમજાવો.}

\begin{solutionbox}

\textbf{આપેલ:} L ઇન્ડક્ટન્સ સાથે શુદ્ધ ઇન્ડક્ટર, લાગુ વોલ્ટેજ v = Vm sin(ωt)

\textbf{વોલ્ટેજ-કરંટ સંબંધ:}

\begin{verbatim}
v = L \times (di/dt)
\end{verbatim}

\textbf{લાગુ વોલ્ટેજ સબસ્ટિટ્યૂટ કરીને:}

\begin{verbatim}
Vm sin(ωt) = L \times (di/dt)
\end{verbatim}

\textbf{ઇન્ટીગ્રેશન:}

\begin{verbatim}
di = (Vm/L) sin(ωt) dt
i = -(Vm/ωL) cos(ωt) + C
\end{verbatim}

\textbf{સ્ટેડી સ્ટેટમાં, C = 0:}

\begin{verbatim}
i = -(Vm/ωL) cos(ωt)
i = (Vm/ωL) sin(ωt - 90^\circ)
\end{verbatim}

\textbf{ટેબલ: શુદ્ધ ઇન્ડક્ટર લક્ષણો}

\begin{longtable}[]{@{}lll@{}}
\toprule\noalign{}
પેરામીટર & મૂલ્ય & ફેઝ સંબંધ \\
\midrule\noalign{}
\endhead
\bottomrule\noalign{}
\endlastfoot
\textbf{કરંટ એમ્પ્લિટ્યૂડ} & Im = Vm/ωL & કરંટ વોલ્ટેજથી 90^\circ પાછળ \\
\textbf{ઇન્ડક્ટિવ રિએક્ટન્સ} & XL = ωL = 2πfL & ફ્રિક્વન્સી આધારિત \\
\textbf{પાવર} & P = 0 (એવરેજ) & કોઈ નેટ પાવર વપરાશ નથી \\
\end{longtable}

\end{solutionbox}
\begin{mnemonicbox}
``ઇન્ડક્ટર ઇમ્પીડ્સ, કરંટ લેગ્સ'' (XL કરંટનો વિરોધ, 90^\circ લેગ)

\end{mnemonicbox}
\subsection*{પ્રશ્ન 5(બ અથવા) [4
ગુણ]}\label{uxaaauxab0uxab6uxaa8-5uxaac-uxa85uxaa5uxab5-4-uxa97uxaa3}

\textbf{AC સર્કિટમાં પાવર અને પાવર ટ્રાયએંગલ સમજાવો.}

\begin{solutionbox}

\textbf{પાવરના પ્રકારો:}

\textbf{ટેબલ: AC પાવર કોમ્પોનન્ટ્સ}

\begin{longtable}[]{@{}lllll@{}}
\toprule\noalign{}
પાવરનો પ્રકાર & પ્રતીક & સૂત્ર & એકમો & વર્ણન \\
\midrule\noalign{}
\endhead
\bottomrule\noalign{}
\endlastfoot
\textbf{એક્ટિવ પાવર} & P & VI cosφ & વોટ & ઉપયોગી પાવર \\
\textbf{રિએક્ટિવ પાવર} & Q & VI sinφ & VAR & પરિભ્રમણ પાવર \\
\textbf{એપેરન્ટ પાવર} & S & VI & VA & કુલ પાવર \\
\end{longtable}

\textbf{પાવર ત્રિકોણ:}

\begin{center}
\textbf{Mermaid Diagram (Code)}
\begin{verbatim}
{Shaded}
{Highlighting}[]
graph LR
    O((O)) {-{-}{} P[P = VI cosφ]}
    O {-{-}{} Q[Q = VI sinφ]}
    P {-{-}{} S[S = VI]}
    
    style P fill:\#f96,stroke:\#333,stroke{-width:2px}
    style Q fill:\#bbf,stroke:\#333,stroke{-width:2px}
    style S fill:\#bfb,stroke:\#333,stroke{-width:2px}
{Highlighting}
{Shaded}
\end{verbatim}
\end{center}

\textbf{ગાણિતિક સંબંધો:}

\begin{verbatim}
S^{2} = P^{2} + Q^{2}
Power Factor = P/S = cosφ
\end{verbatim}

\textbf{મહત્વ:}

\begin{itemize}
\tightlist
\item
  \textbf{એક્ટિવ પાવર}: ઉપયોગી કાર્ય કરે છે (હીટિંગ, મિકેનિકલ)
\item
  \textbf{રિએક્ટિવ પાવર}: મેગ્નેટિક/ઇલેક્ટ્રિક ફીલ્ડ જાળવે છે
\item
  \textbf{પાવર ફેક્ટર}: કાર્યક્ષમતા સૂચક
\end{itemize}

\end{solutionbox}
\begin{mnemonicbox}
``પાવર ટ્રાયએંગલ: પ્લીઝ ક્વાલિફાય સ્ટુડન્ટ્સ'' (P, Q, S
કોમ્પોનન્ટ્સ)

\end{mnemonicbox}
\subsection*{પ્રશ્ન 5(ક અથવા) [7
ગુણ]}\label{uxaaauxab0uxab6uxaa8-5uxa95-uxa85uxaa5uxab5-7-uxa97uxaa3}

\textbf{એક લેમ્પને એક જગ્યાએથી કન્ટ્રોલ કરવો તેમજ દાદર માટેનું વાયરિંગ ડાયાગ્રામ સાથે
સમજાવો.}

\begin{solutionbox}

\textbf{1. એક જગ્યાએથી લેમ્પ કન્ટ્રોલ:}

\textbf{સર્કિટ આકૃતિ:}

\begin{verbatim}
Live {-{-}{-}{-}[S]{-}{-}{-}{-}[Lamp]{-}{-}{-}{-}+}
                          |
Neutral {-{-}{-}{-}{-}{-}{-}{-}{-}{-}{-}{-}{-}{-}{-}{-}{-}{-}+}

S = સિંગલ પોલ સિંગલ થ્રો સ્વિચ
\end{verbatim}

\textbf{કોમ્પોનન્ટ્સ:}

\begin{itemize}
\tightlist
\item
  \textbf{SPST સ્વિચ}: સિંગલ પોલ, સિંગલ થ્રો
\item
  \textbf{લાઇવ વાયર કન્ટ્રોલ}: સુરક્ષા માટે સ્વિચ લાઇવ વાયરમાં
\item
  \textbf{સરળ ઓન/ઓફ}: બેસિક કન્ટ્રોલ મેકેનિઝમ
\end{itemize}

\textbf{2. સીડીનું વાયરિંગ (ટુ-વે કન્ટ્રોલ):}

\textbf{સર્કિટ આકૃતિ:}

\begin{verbatim}
Live {-{-}{-}{-}[S1]{-}{-}{-}{-}+{-}{-}{-}{-}[S2]{-}{-}{-}{-}[Lamp]{-}{-}{-}{-}+}
            {    |    /                 |}
             {   |   /                  |}
              {  |  /                   |}
               { | /                    |}
                {|/                     |}
Neutral {-{-}{-}{-}{-}{-}{-}{-}{-}{-}{-}{-}{-}{-}{-}{-}{-}{-}{-}{-}{-}{-}{-}{-}{-}{-}{-}{-}{-}{-}{-}{-}+}

S1, S2 = બે{-દિશા સ્વિચ (SPDT)}
\end{verbatim}

\textbf{ટેબલ: સીડીના કન્ટ્રોલ માટે સ્વિચ પોઝિશન}

\begin{longtable}[]{@{}lll@{}}
\toprule\noalign{}
S1 પોઝિશન & S2 પોઝિશન & લેમ્પ સ્ટેટસ \\
\midrule\noalign{}
\endhead
\bottomrule\noalign{}
\endlastfoot
\textbf{ઉપર} & ઉપર & ચાલુ \\
\textbf{ઉપર} & નીચે & બંધ \\
\textbf{નીચે} & ઉપર & બંધ \\
\textbf{નીચે} & નીચે & ચાલુ \\
\end{longtable}

\textbf{કાર્ય સિદ્ધાંત:}

\begin{itemize}
\tightlist
\item
  \textbf{બે-દિશા સ્વિચ}: SPDT (સિંગલ પોલ ડબલ થ્રો)
\item
  \textbf{કોમન ટર્મિનલ}: લાઇવ અને લેમ્પ સાથે જોડાયેલું
\item
  \textbf{સ્ટ્રેપર્સ}: સ્વિચો વચ્ચે લિંક
\item
  \textbf{ટોગલ એક્શન}: કોઈ પણ સ્વિચ લેમ્પ કન્ટ્રોલ કરી શકે છે
\end{itemize}

\textbf{ઉપયોગિતા:}

\begin{itemize}
\tightlist
\item
  \textbf{સીડીની લાઇટિંગ}: ઉપર અને નીચેથી કન્ટ્રોલ
\item
  \textbf{લાંબા કોરિડોર}: બંને છેડેથી કન્ટ્રોલ
\item
  \textbf{બેડરૂમ લાઇટિંગ}: બેડ અને દરવાજાથી કન્ટ્રોલ
\end{itemize}

\textbf{ફાયદાઓ:}

\begin{itemize}
\tightlist
\item
  \textbf{સુવિધા}: અનેક સ્થળોએથી કન્ટ્રોલ
\item
  \textbf{એનર્જી સેવિંગ}: સરળ સ્વિચિંગ વેસ્ટેજ ઘટાડે છે
\item
  \textbf{સુરક્ષા}: અંધારામાં ચાલવાની જરૂર નથી
\end{itemize}

\textbf{ઇન્સ્ટોલેશન પોઇન્ટ્સ:}

\begin{itemize}
\tightlist
\item
  \textbf{યોગ્ય અર્થિંગ}: તમામ મેટલ પાર્ટ્સ અર્થ કરેલા
\item
  \textbf{કેબલ રેટિંગ}: પર્યાપ્ત કરંટ કેપેસિટી
\item
  \textbf{સ્વિચ ઊંચાઈ}: ફ્લોરથી સ્ટાન્ડર્ડ 4 ફૂટ
\end{itemize}

\end{solutionbox}
\begin{mnemonicbox}
``ટુ-વે ટોગલ્સ, ટુ પ્લેસિસ'' (બે સ્વિચ, બે સ્થળો)

\end{mnemonicbox}

\end{document}
