\documentclass[10pt,a4paper]{article}
\usepackage[margin=0.6in]{geometry}
\usepackage{amsmath,amssymb,amsthm}
\usepackage{booktabs}
\usepackage{multirow}
\usepackage{xcolor}
\usepackage{tcolorbox}
\tcbuselibrary{breakable}
\usepackage[colorlinks=true,linkcolor=blue]{hyperref}
\usepackage{titlesec}
\usepackage{enumitem}
\usepackage{tikz}
\usepackage{circuitikz}
\usetikzlibrary{shapes,arrows,positioning,calc,decorations.pathmorphing}

% XeLaTeX for Gujarati support
\usepackage{fontspec}
\usepackage{polyglossia}
\setmainlanguage{gujarati}
\setotherlanguage{english}

% Gujarati font with proper script features
\setmainfont{Noto Sans Gujarati}[
  Script=Gujarati,
  Renderer=Harfbuzz,
  Language=Gujarati
]
\newfontfamily\englishfont{Latin Modern Roman}

% Color scheme
\definecolor{headcolor}{RGB}{0,102,204}
\definecolor{keycolor}{RGB}{220,20,60}
\definecolor{solutioncolor}{RGB}{34,139,34}
\definecolor{mnemoniccolor}{RGB}{148,0,211}

% Custom environments
\newtcolorbox{solutionbox}{
 breakable,
 colback=solutioncolor!5!white,
 colframe=solutioncolor!75!black,
 fonttitle=\bfseries,
 title=ઉકેલ
}

\newtcolorbox{keyformula}{
 colback=keycolor!5!white,
 colframe=keycolor!75!black,
 fonttitle=\bfseries,
 title=મુખ્ય સૂત્ર
}

\newtcolorbox{mnemonicbox}{
 colback=mnemoniccolor!5!white,
 colframe=mnemoniccolor!75!black,
 fonttitle=\bfseries,
 title=મેમરી ટ્રીક
}

% Spacing
\setlength{\parskip}{3pt}
\setlist[itemize]{nosep}
\setlist[enumerate]{nosep}

% Title formatting
\titleformat{\section}{\Large\bfseries\color{headcolor}}{\thesection}{1em}{}
\titleformat{\subsection}{\large\bfseries\color{headcolor}}{\thesubsection}{1em}{}

\begin{document}

\begin{center}
{\Huge\bfseries\color{headcolor} ઇલેક્ટ્રિકલ એન્જિનિયરિંગના મૂળભૂત સિદ્ધાંતો}\\[5pt]
{\LARGE DI01000101 -- શિયાળુ 2024}\\[3pt]
{\large સેમેસ્ટર 1 અભ્યાસ સામગ્રી}\\[3pt]
{\normalsize\textit{વિગતવાર ઉકેલો અને સમજૂતીઓ}}
\end{center}

\vspace{10pt}

%----------------------------------------
\section*{પ્રશ્ન 1(અ) [3 ગુણ]}
\textbf{ઓહમના નિયમને તેની મર્યાદા અને ઉપયોગિતા સાથે સમજાવો.}

\begin{solutionbox}
\textbf{ઓહમના નિયમનો સારાંશ:}

\begin{center}
\begin{tabular}{|l|p{8cm}|}
\hline
\textbf{પાસું} & \textbf{વર્ણન} \\
\hline
વિધાન & અચળ ભૌતિક પરિસ્થિતિમાં વાહકમાંથી પસાર થતો વિદ્યુત પ્રવાહ તેના બે છેડા વચ્ચેના વિદ્યુત સ્થિતિમાનના તફાવત (વોલ્ટેજ) ના સમપ્રમાણમાં હોય છે. \\
\hline
સૂત્ર & $V = I \times R$ \\
\hline
એકમો & $V$ (વોલ્ટ), $I$ (એમ્પિયર), $R$ (ઓહ્મ) \\
\hline
\end{tabular}
\end{center}

\textbf{મર્યાદાઓ:}
\begin{itemize}
\item \textbf{તાપમાન આધારિત}: તાપમાન સાથે અવરોધ બદલાય છે.
\item \textbf{બિન-રેખીય પદાર્થો}: સેમિકન્ડક્ટર, ડાયોડ વગેરે પર લાગુ પડતું નથી.
\item \textbf{AC સર્કિટ}: રિએક્ટિવ કોમ્પોનન્ટ્સ માટે ઇમ્પીડન્સ ($Z$) ધ્યાનમાં લેવો પડે.
\end{itemize}

\textbf{ઉપયોગિતા:}
\begin{itemize}
\item \textbf{સર્કિટ વિશ્લેષણ}: અજાણા વોલ્ટેજ, કરંટ અથવા અવરોધની ગણતરી કરવા.
\item \textbf{પાવર ગણતરી}: $P = V^2/R$ અથવા $P = I^2R$.
\end{itemize}
\end{solutionbox}

\begin{mnemonicbox}
``વોલ્ટેજ ઇઝ રિયલી ઇમ્પોર્ટન્ટ'' ($V = I \times R$)
\end{mnemonicbox}

%----------------------------------------
\section*{પ્રશ્ન 1(બ) [4 ગુણ]}
\textbf{ફેરાડેના ઇલેક્ટ્રોમેગ્નેટિક ઇન્ડક્શનના નિયમને જરૂરી આકૃતિ સાથે સમજાવો.}

\begin{solutionbox}
\textbf{ફેરાડેના નિયમો:}
\begin{itemize}
\item \textbf{પ્રથમ નિયમ}: જ્યારે વાહક સાથે સંકળાયેલ મેગ્નેટિક ફ્લક્સ બદલાય છે ત્યારે તેમાં EMF પેદા થાય છે.
\item \textbf{બીજો નિયમ}: પેદા થતા EMF નું મૂલ્ય ફ્લક્સના ફેરફારના દરના સમપ્રમાણમાં હોય છે.
\end{itemize}

\begin{keyformula}
\[e = -N \frac{d\Phi}{dt}\]
\end{keyformula}

\textbf{આકૃતિ:}
\begin{center}
\begin{tikzpicture}[scale=1.0]
% Coil components
\draw[thick, decoration={aspect=0.3, segment length=3mm, amplitude=3mm,coil},decorate] (0,0) -- (3,0); 
\draw[thick] (0,0) -- (0,-1);
\draw[thick] (3,0) -- (3,-1);
\node at (1.5,0.6) {કોઇલ ($N$ આંટા)};
% Galvanometer
\draw[thick] (0,-1) circle (0.3);
\node at (0,-1) {G};
\draw[thick] (0,-1.3) -- (3,-1.3) -- (3,-1);

% Magnet
\draw[thick, fill=red!30] (-2, -0.2) rectangle (-0.5, 0.2);
\node at (-1.6, 0) {N};
\draw[thick, fill=blue!30] (-3.5, -0.2) rectangle (-2, 0.2);
\node at (-2.4, 0) {S};
\draw[->, thick, blue] (-1.8, 0.4) -- (-0.8, 0.4) node[midway, above] {ગતિ $v$};

\node at (-2, -0.8) {ફરતું ચુંબક};
\end{tikzpicture}
\end{center}

\textbf{ઉપયોગિતા:}
\begin{itemize}
\item \textbf{ટ્રાન્સફોર્મર}: મ્યુચ્યુઅલ ઇન્ડક્શન સિદ્ધાંત.
\item \textbf{જનરેટર}: યાંત્રિક ઉર્જાનું વિદ્યુત ઉર્જામાં રૂપાંતર.
\end{itemize}
\end{solutionbox}

\begin{mnemonicbox}
``ફ્લક્સ ચેન્જ જનરેટ્સ EMF'' ($d\Phi/dt = \text{EMF}$)
\end{mnemonicbox}

%----------------------------------------
\section*{પ્રશ્ન 1(ક) [7 ગુણ]}
\textbf{કિર્ચહોફના વોલ્ટેજના નિયમ (KVL) અને કિર્ચહોફના કરંટના નિયમને (KCL) જરૂરી આકૃતિ સાથે સમજાવો.}

\begin{solutionbox}
\textbf{તુલના:}
\begin{center}
\begin{tabular}{|l|p{5cm}|c|l|}
\hline
\textbf{નિયમ} & \textbf{વિધાન} & \textbf{ગાણિતિક રૂપ} & \textbf{ઉપયોગ} \\
\hline
KVL & બંધ લૂપમાં વોલ્ટેજનો સરવાળો શૂન્ય છે & $\sum V = 0$ & સિરીઝ સર્કિટ \\
\hline
KCL & નોડ પર કરંટનો સરવાળો શૂન્ય છે & $\sum I = 0$ & પેરેલલ સર્કિટ \\
\hline
\end{tabular}
\end{center}

\textbf{KVL આકૃતિ:}
\begin{center}
\begin{circuitikz}[scale=0.8]
\draw (0,0) to[battery1, l=$V_1$] (0,3)
      to[R, l=$R_1$] (3,3)
      to[battery1, l=$V_2$] (3,0)
      to[R, l=$R_2$] (0,0);
\draw[->] (1,1.5) arc (-60:240:0.5) node[midway, below] {લૂપ};
\end{circuitikz}
\end{center}

\textbf{KCL આકૃતિ:}
\begin{center}
\begin{circuitikz}[scale=0.8]
\draw (0,0) node[circ, label={નોડ $N$}] (N) {};
\draw[<-] (N) -- ++(-1.5,1) node[left] {$I_1$};
\draw[<-] (N) -- ++(0,1.5) node[above] {$I_2$};
\draw[->] (N) -- ++(1.5,1) node[right] {$I_3$};
\draw[->] (N) -- ++(0,-1.5) node[below] {$I_4$};
\end{circuitikz}
\end{center}

\textbf{મુખ્ય મુદ્દાઓ:}
\begin{itemize}
\item \textbf{KVL}: વોલ્ટેજ પોલેરિટી (+/-) ધ્યાનમાં રાખે છે.
\item \textbf{KCL}: કરંટની દિશા (આવતો = ધન, જતો = ઋણ) ધ્યાનમાં રાખે છે.
\end{itemize}
\end{solutionbox}

\begin{mnemonicbox}
``વોલ્ટેજ લૂપ્સ, કરંટ નોડ્સ'' (KVL લૂપ, KCL નોડ)
\end{mnemonicbox}

%----------------------------------------
\section*{પ્રશ્ન 1(ક અથવા) [7 ગુણ]}
\textbf{સ્ટેટિકલી ઇન્ડ્યૂસ્ડ EMF અને ડાયનેમિકલી ઇન્ડ્યૂસ્ડ EMF વચ્ચેનો તફાવત સમજાવો.}

\begin{solutionbox}
\begin{center}
\begin{tabular}{|l|p{5cm}|p{5cm}|}
\hline
\textbf{પેરામીટર} & \textbf{સ્ટેટિકલી ઇન્ડ્યૂસ્ડ EMF} & \textbf{ડાયનેમિકલી ઇન્ડ્યૂસ્ડ EMF} \\
\hline
કારણ & બદલાતું મેગ્નેટિક ફીલ્ડ & વાહક અને ફીલ્ડ વચ્ચે સાપેક્ષ ગતિ \\
\hline
ફીલ્ડ & સમય સાથે બદલાતું, વાહક સ્થિર & સ્થિર ફીલ્ડ, વાહક ગતિમાન \\
\hline
ઉદાહરણો & ટ્રાન્સફોર્મર, ઇન્ડક્ટર & જનરેટર, મોટર \\
\hline
સૂત્ર & $e = -N \frac{d\Phi}{dt}$ & $e = B l v \sin\theta$ \\
\hline
ઉપયોગિતા & AC સર્કિટ, પાવર સપ્લાય & પાવર જનરેશન, ડ્રાઈવ્સ \\
\hline
\end{tabular}
\end{center}

\textbf{સ્ટેટિક EMF ના પ્રકારો:}
\begin{itemize}
\item \textbf{સેલ્ફ-ઇન્ડ્યૂસ્ડ}: પોતાના જ કરંટ ફેરફારના કારણે કોઇલમાં EMF.
\item \textbf{મ્યુચ્યુઅલી ઇન્ડ્યૂસ્ડ}: નજીકની કોઇલના ફ્લક્સ ફેરફારના કારણે EMF.
\end{itemize}
\end{solutionbox}

\begin{mnemonicbox}
``સ્ટેટિક સ્ટેઝ, ડાયનેમિક ડાન્સ'' (સ્થિર vs ગતિ)
\end{mnemonicbox}

%----------------------------------------
\section*{પ્રશ્ન 2(અ) [3 ગુણ]}
\textbf{ટ્રાન્સફોર્મરમાં થતાં વિવિધ પ્રકારના લોસ સમજાવો.}

\begin{solutionbox}
\begin{center}
\begin{tabular}{|l|l|p{2.5cm}|p{4cm}|}
\hline
\textbf{લોસ પ્રકાર} & \textbf{કારણ} & \textbf{સ્થાન} & \textbf{લક્ષણો} \\
\hline
\textbf{આયર્ન લોસ} & હિસ્ટેરેસિસ + એડી કરંટ & કોર & અચળ, ફ્રિક્વન્સી આધારિત \\
\hline
\textbf{કોપર લોસ} & $I^2R$ હીટિંગ & વાઇન્ડિંગ & લોડ સાથે બદલાય ($I^2$) \\
\hline
\textbf{સ્ટ્રે લોસ} & લીકેજ ફ્લક્સ & અન્ય & નહિવત્ \\
\hline
\end{tabular}
\end{center}

\textbf{વિગતો:}
\begin{itemize}
\item \textbf{હિસ્ટેરેસિસ લોસ}: મેગ્નેટિક ડોમેઇન રિવર્સલના કારણે.
\item \textbf{એડી કરંટ લોસ}: કોરમાં ફરતા કરંટના કારણે (લેમિનેશનથી ઘટે છે).
\item \textbf{કોપર લોસ}: કરંટ પર આધારિત.
\end{itemize}
\end{solutionbox}

\begin{mnemonicbox}
``આયર્ન કોર, કોપર કોઇલ'' (મુખ્ય લોસ)
\end{mnemonicbox}

%----------------------------------------
\section*{પ્રશ્ન 2(બ) [4 ગુણ]}
\textbf{ટ્રાન્સફોર્મરનો કાર્ય સિદ્ધાંત સમજાવો.}

\begin{solutionbox}
\textbf{કાર્ય સિદ્ધાંત:} સામાન્ય મેગ્નેટિક કોર દ્વારા જોડાયેલ બે વાઇન્ડિંગ્સ વચ્ચે મ્યુચ્યુઅલ ઇલેક્ટ્રોમેગ્નેટિક ઇન્ડક્શન.

\textbf{આકૃતિ:}
\begin{center}
\begin{tikzpicture}[scale=0.8]
% Core
\draw[thick, fill=gray!20] (0,0) rectangle (4,3);
\draw[thick, fill=white] (1,1) rectangle (3,2);

% Primary
\foreach \y in {0.2,0.5,0.8,1.1,1.4,1.7,2.0,2.3}
    \draw[thick, blue] (-0.2,\y) to[out=0,in=180] (0.8,\y+0.2);
\node[left, blue] at (-0.2, 1.5) {પ્રાઇમરી ($N_1$)};
\node[left] at (-0.5, 2.5) {AC સપ્લાય};
\draw[->] (-1, 2.2) -- (-0.2, 2.2);

% Secondary
\foreach \y in {0.2,0.5,0.8,1.1,1.4,1.7,2.0,2.3}
    \draw[thick, red] (3.2,\y) to[out=180,in=0] (4.2,\y+0.2);
\node[right, red] at (4.2, 1.5) {સેકન્ડરી ($N_2$)};
\node[right] at (4.5, 2.5) {લોડ};
\draw[->] (4.2, 2.2) -- (5, 2.2);

\node at (2, 0.5) {ફ્લક્સ $\Phi$};
\draw[->] (1.5, 0.5) arc (180:90:0.5);
\end{tikzpicture}
\end{center}

\textbf{કાર્ય પદ્ધતિ:}
\begin{enumerate}
\item પ્રાઇમરીમાં AC કરંટ બદલાતું ફ્લક્સ ઉત્પન્ન કરે છે.
\item ફ્લક્સ કોર મારફતે સેકન્ડરી સાથે સંકળાય છે.
\item બદલાતા ફ્લક્સને કારણે સેકન્ડરીમાં EMF ઈન્ડ્યુસ થાય છે (ફેરાડેનો નિયમ).
\end{enumerate}

\begin{keyformula}
\[\frac{V_2}{V_1} = \frac{N_2}{N_1} = \frac{I_1}{I_2} = K\]
\end{keyformula}
\end{solutionbox}

\begin{mnemonicbox}
``પ્રાઇમરી પ્રોડુસ, સેકન્ડરી સપ્લાય''
\end{mnemonicbox}

%----------------------------------------
\section*{પ્રશ્ન 2(ક) [7 ગુણ]}
\textbf{ટ્રાન્સફોર્મરનું EMF સૂત્ર તારવો.}

\begin{solutionbox}
\textbf{આપેલ:}
\begin{itemize}
\item $N_1, N_2$: આંટાઓની સંખ્યા
\item $\Phi_m$: મહત્તમ ફ્લક્સ
\item $f$: ફ્રિક્વન્સી
\end{itemize}

\textbf{તારવણી:}
\begin{enumerate}
\item \textbf{ફ્લક્સ સમીકરણ}: $\Phi = \Phi_m \sin(2\pi f t)$
\item \textbf{ઇન્ડ્યૂસ્ડ EMF}: $e = -N \frac{d\Phi}{dt}$
\item \textbf{વિકલન}:
\[e = -N \frac{d}{dt}(\Phi_m \sin(2\pi f t)) = 2\pi f N \Phi_m \sin(2\pi f t - 90^\circ)\]
\item \textbf{મહત્તમ EMF}: $E_m = 2\pi f N \Phi_m$
\item \textbf{RMS EMF}: $E_{rms} = \frac{E_m}{\sqrt{2}} = \frac{2\pi f N \Phi_m}{\sqrt{2}} = 4.44 f N \Phi_m$
\end{enumerate}

\begin{keyformula}
\[E_1 = 4.44 f N_1 \Phi_m \quad \text{અને} \quad E_2 = 4.44 f N_2 \Phi_m\]
\end{keyformula}

\textbf{ટ્રાન્સફોર્મેશન રેશિયો:} $K = \frac{E_2}{E_1} = \frac{N_2}{N_1}$
\end{solutionbox}

\begin{mnemonicbox}
``4.44 ફ્લક્સ ફોર્મ્યુલા''
\end{mnemonicbox}

%----------------------------------------
\section*{પ્રશ્ન 2(અ અથવા) [3 ગુણ]}
\textbf{ટ્રાન્સફોર્મરની ઉપયોગિતા સમજાવો.}

\begin{solutionbox}
\begin{center}
\begin{tabular}{|l|p{4cm}|l|}
\hline
\textbf{ઉપયોગિતા} & \textbf{હેતુ} & \textbf{વોલ્ટેજ લેવલ} \\
\hline
પાવર ટ્રાન્સમિશન & ટ્રાન્સમિશન લોસ ઘટાડવા & સ્ટેપ-અપ (દા.ત. 400 kV) \\
\hline
ડિસ્ટ્રિબ્યુશન & ગ્રાહકો માટે સુરક્ષિત વોલ્ટેજ & સ્ટેપ-ડાઉન (દા.ત. 230 V) \\
\hline
આઇસોલેશન & ઇલેક્ટ્રિકલ સુરક્ષા & 1:1 રેશિયો \\
\hline
ઇલેક્ટ્રોનિક્સ & DC પાવર સપ્લાય & સ્ટેપ-ડાઉન \\
\hline
\end{tabular}
\end{center}

\textbf{ઔદ્યોગિક ઉપયોગ:} વેલ્ડિંગ ટ્રાન્સફોર્મર, ઇન્સ્ટ્રુમેન્ટ ટ્રાન્સફોર્મર (CT/PT).
\end{solutionbox}

\begin{mnemonicbox}
``પાવર ડિસ્ટ્રિબ્યુશન આઇસોલેશન ઇલેક્ટ્રોનિક્સ''
\end{mnemonicbox}

%----------------------------------------
\section*{પ્રશ્ન 2(બ અથવા) [4 ગુણ]}
\textbf{DC મોટર માટે બેક EMF અને ટોર્કનું સૂત્ર લખો.}

\begin{solutionbox}
\textbf{1. બેક EMF સૂત્ર:}
\[E_b = \frac{\phi Z N P}{60 A}\]
સરળ: $E_b = K \phi N$

\textbf{2. ટોર્ક સૂત્ર:}
\[T = \frac{\phi Z I_a P}{2\pi A}\]
સરળ: $T = K \phi I_a$

\textbf{જ્યાં:}
\begin{itemize}
\item $\phi$: ફ્લક્સ પર પોલ (વેબર)
\item $Z$: કુલ આર્મેચર વાહકો
\item $N$: સ્પિડ (RPM)
\item $P$: પોલની સંખ્યા
\item $I_a$: આર્મેચર કરંટ
\end{itemize}
\end{solutionbox}

\begin{mnemonicbox}
``બેક EMF વિરોધ કરે છે, ટોર્ક ચલાવે છે''
\end{mnemonicbox}

%----------------------------------------
\section*{પ્રશ્ન 2(ક અથવા) [7 ગુણ]}
\textbf{DC મોટરની રચના અને કાર્ય પદ્ધતિ આકૃતિ સાથે સમજાવો.}

\begin{solutionbox}
\textbf{રચના:}
\begin{itemize}
\item \textbf{સ્ટેટર}: યોક, પોલ, ફીલ્ડ વાઇન્ડિંગ (મેગ્નેટિક ફીલ્ડ આપે છે).
\item \textbf{રોટર (આર્મેચર)}: વાહક સ્લોટ્સ સાથે લેમિનેટેડ કોર.
\item \textbf{કોમ્યુટેટર}: કરંટની દિશા બદલવા માટે.
\item \textbf{બ્રશેસ}: કરંટ કલેક્શન માટે કાર્બન બ્રશ.
\end{itemize}

\textbf{આકૃતિ:}
\begin{center}
\begin{tikzpicture}[scale=0.8]
\draw (0,0) circle (1.5); % Armature
\node at (0,0) {A};
\draw[thick] (-2, -1) rectangle (-1.6, 1); % Pole N
\node at (-1.8, 0) {N};
\draw[thick] (1.6, -1) rectangle (2, 1); % Pole S
\node at (1.8, 0) {S};
% Brushes
\draw[fill=black] (-0.3, 1.5) rectangle (0.3, 1.7);
\draw[fill=black] (-0.3, -1.7) rectangle (0.3, -1.5);
\end{tikzpicture}
\end{center}

\textbf{કાર્ય સિદ્ધાંત:}
\begin{enumerate}
\item મેગ્નેટિક ફીલ્ડમાં મૂકેલા આર્મેચર કન્ડક્ટરમાંથી કરંટ પસાર થાય છે.
\item ફ્લેમિંગના ડાબા હાથના નિયમ મુજબ બળ લાગે છે ($F = BIl$).
\item આ બળ ટોર્ક પેદા કરે છે અને મોટર ફરે છે.
\item કોમ્યુટેટર એકધારી ગતિ જાળવવા કરંટની દિશા ઉલટાવે છે.
\end{enumerate}
\end{solutionbox}

\begin{mnemonicbox}
``કરંટ ક્રિએટ્સ સર્ક્યુલર મોશન''
\end{mnemonicbox}

%----------------------------------------
\section*{પ્રશ્ન 3(અ) [3 ગુણ]}
\textbf{ટ્રાન્સફોર્મરની રચના સમજાવો.}

\begin{solutionbox}
\textbf{મુખ્ય ઘટકો:}
\begin{center}
\begin{tabular}{|l|l|p{5cm}|}
\hline
\textbf{ઘટક} & \textbf{મટીરિયલ} & \textbf{કાર્ય} \\
\hline
કોર & સિલિકોન સ્ટીલ & મેગ્નેટિક ફ્લક્સ માટે રસ્તો. લેમિનેશનથી એડી કરંટ લોસ ઘટે છે. \\
\hline
વાઇન્ડિંગ્સ & કોપર/એલ્યુમિનિયમ & પ્રાઇમરી (ઇનપુટ), સેકન્ડરી (આઉટપુટ). \\
\hline
ઇન્સ્યુલેશન & પેપર/વાર્નિશ & શોર્ટ સર્કિટ અટકાવે છે. \\
\hline
ટાંકી & સ્ટીલ & સુરક્ષા અને કૂલિંગ (ઓઇલ ભરેલ). \\
\hline
\end{tabular}
\end{center}

\textbf{પ્રકારો:} શેલ ટાઇપ (વાઇન્ડિંગ કોરથી ઘેરાયેલ), કોર ટાઇપ (કોર વાઇન્ડિંગથી ઘેરાયેલ).
\end{solutionbox}

\begin{mnemonicbox}
``કોર કેરીઝ કરંટ કેરફુલી''
\end{mnemonicbox}

%----------------------------------------
\section*{પ્રશ્ન 3(બ) [4 ગુણ]}
\textbf{DC મોટરની ઉપયોગિતા સમજાવો.}

\begin{solutionbox}
\begin{center}
\begin{tabular}{|l|p{3cm}|p{5cm}|}
\hline
\textbf{મોટર પ્રકાર} & \textbf{લક્ષણો} & \textbf{ઉપયોગિતા} \\
\hline
શન્ટ મોટર & અચળ સ્પીડ & લેથ મશીન, પંખા, પંપ \\
\hline
સિરીઝ મોટર & ઉચ્ચ સ્ટાર્ટિંગ ટોર્ક & ટ્રેક્શન (ટ્રેન), ક્રેન, લિફ્ટ \\
\hline
કમ્પાઉન્ડ મોટર & સ્થિર સ્પીડ અને ટોર્ક & એલિવેટર, કોમ્પ્રેસર, રોલિંગ મિલ \\
\hline
\end{tabular}
\end{center}
\end{solutionbox}

\begin{mnemonicbox}
``શન્ટ સ્ટેઝ, સિરીઝ સ્પીડ્સ''
\end{mnemonicbox}

%----------------------------------------
\section*{પ્રશ્ન 3(ક) [7 ગુણ]}
\textbf{DC મોટરના વિવિધ પ્રકાર સમજાવો.}

\begin{solutionbox}
\textbf{ફીલ્ડ કનેક્શન આધારિત વર્ગીકરણ:}

\begin{enumerate}
\item \textbf{DC શન્ટ મોટર}:
\begin{itemize}
\item ફીલ્ડ વાઇન્ડિંગ આર્મેચર સાથે સમાંતરમાં જોડાયેલ.
\item અચળ ગતિ મોટર.
\end{itemize}
\begin{center}
\begin{circuitikz}[scale=0.7]
\draw (0,0) to[short] (2,0) to[short] (2,1) to[R, l=$R_{sh}$] (2,3) to[short] (2,4) to[short] (0,4);
\draw (2,0) to[short] (4,0) to[short] (4,1) to[rmeter, t=M, l=$R_a$] (4,3) to[short] (4,4) to[short] (2,4);
\draw (0,0) to[short] (-1,0) node[below] {$-$};
\draw (0,4) to[short] (-1,4) node[above] {$+$};
\node at (-1, 2) {$V_{dc}$};
\end{circuitikz}
\end{center}

\item \textbf{DC સિરીઝ મોટર}:
\begin{itemize}
\item ફીલ્ડ વાઇન્ડિંગ આર્મેચર સાથે શ્રેણીમાં જોડાયેલ.
\item ઉપયોગ: ભારે લોડ સ્ટાર્ટ કરવા.
\end{itemize}

\item \textbf{DC કમ્પાઉન્ડ મોટર}:
\begin{itemize}
\item સિરીઝ અને શન્ટ બંને ફીલ્ડ વાઇન્ડિંગ ધરાવે છે.
\end{itemize}
\end{enumerate}

\begin{keyformula}
શન્ટ: $N \propto \frac{V - I_a R_a}{\phi}$ \quad સિરીઝ: $N \propto \frac{V}{\sqrt{T}}$
\end{keyformula}
\end{solutionbox}

\begin{mnemonicbox}
``શન્ટ સ્ટેડી, સિરીઝ સ્ટ્રોંગ''
\end{mnemonicbox}

%----------------------------------------
\section*{પ્રશ્ન 3(અ અથવા) [3 ગુણ]}
\textbf{ટ્રાન્સફોર્મરનો ટ્રાન્સફોર્મેશન રેશિયો સમજાવો.}

\begin{solutionbox}
\textbf{વ્યાખ્યા}: સેકન્ડરી વોલ્ટેજ (અથવા આંટા) અને પ્રાઇમરી વોલ્ટેજ (અથવા આંટા) નો ગુણોત્તર.

\begin{keyformula}
\[K = \frac{N_2}{N_1} = \frac{E_2}{E_1} = \frac{V_2}{V_1} = \frac{I_1}{I_2}\]
\end{keyformula}

\begin{itemize}
\item જો $K > 1$: સ્ટેપ-અપ ટ્રાન્સફોર્મર.
\item જો $K < 1$: સ્ટેપ-ડાઉન ટ્રાન્સફોર્મર.
\item જો $K = 1$: આઇસોલેશન ટ્રાન્સફોર્મર.
\end{itemize}
\end{solutionbox}

\begin{mnemonicbox}
``ટર્ન્સ ટેલ ટ્રાન્સફોર્મેશન''
\end{mnemonicbox}

%----------------------------------------
\section*{પ્રશ્ન 3(બ અથવા) [4 ગુણ]}
\textbf{ઓટો ટ્રાન્સફોર્મરની ઉપયોગિતા સમજાવો.}

\begin{solutionbox}
\textbf{ઉપયોગિતા:}
\begin{enumerate}
\item \textbf{મોટર સ્ટાર્ટિંગ}: ઇન્ડક્શન મોટર માટે સ્ટાર્ટર તરીકે.
\item \textbf{વોલ્ટેજ રેગ્યુલેશન}: લેબમાં વેરિએક તરીકે (ચલ વોલ્ટેજ સોર્સ).
\item \textbf{પાવર સિસ્ટમ}: અલગ વોલ્ટેજ લેવલની સિસ્ટમને જોડવા.
\item \textbf{ટેસ્ટિંગ}: સાધનોને વિવિધ વોલ્ટેજ પર ચેક કરવા.
\end{enumerate}

\textbf{ફાયદા}: નાનું કદ, ઓછી કિંમત, વધારે કાર્યક્ષમતા.
\end{solutionbox}

\begin{mnemonicbox}
``ઓટો એડજસ્ટ્સ એડવાન્ટેજિયસલી''
\end{mnemonicbox}

%----------------------------------------
\section*{પ્રશ્ન 3(ક અથવા) [7 ગુણ]}
\textbf{DC શન્ટ મોટર માટે સ્પીડ કન્ટ્રોલ કરવાની રીતો સમજાવો.}

\begin{solutionbox}
\textbf{રીતો:}

\begin{enumerate}
\item \textbf{આર્મેચર કન્ટ્રોલ (રિઓસ્ટેટિક કન્ટ્રોલ)}:
\begin{itemize}
\item તર્ક: $N \propto V - I_a(R_a + R_{ext})$.
\item રેઝિસ્ટન્સ ઉમેરતા સ્પીડ ઘટે છે.
\item \textbf{અસર}: રેટેડ સ્પીડથી ઓછી સ્પીડ મળે.
\end{itemize}

\item \textbf{ફીલ્ડ કન્ટ્રોલ (ફ્લક્સ કન્ટ્રોલ)}:
\begin{itemize}
\item તર્ક: $N \propto 1/\phi$.
\item ફ્લક્સ ઘટાડતા સ્પીડ વધે છે.
\item \textbf{અસર}: રેટેડ સ્પીડથી વધારે સ્પીડ મળે.
\end{itemize}
\end{enumerate}

\textbf{આર્મેચર કન્ટ્રોલ આકૃતિ:}
\begin{center}
\begin{circuitikz}[scale=0.8]
\draw (0,0) to[battery1, l=$V$] (0,3) -- (2,3);
\draw (2,3) to[vR, l=$R_{ext}$] (4,3) -- (4,2) to[rmeter, t=A] (4,1) -- (2,1) -- (2,0) -- (0,0);
\draw (2,3) -- (2,2) to[L, l=$Field$] (2,1); 
\end{circuitikz}
\end{center}
\end{solutionbox}

\begin{mnemonicbox}
``આર્મેચર એક્યુરેટ, ફીલ્ડ ફાસ્ટ''
\end{mnemonicbox}

%----------------------------------------
\section*{પ્રશ્ન 4(અ) [3 ગુણ]}
\textbf{અલ્ટરનેટિંગ EMF નું વેક્ટર નિરૂપણ સમજાવો.}

\begin{solutionbox}
\textbf{કન્સેપ્ટ}: અલ્ટરનેટિંગ રાશિને $\omega$ rad/s વેગથી ફરતા વેક્ટર (ફેઝર) તરીકે દર્શાવી શકાય છે.

\textbf{સમીકરણ}: $e = E_m \sin(\omega t + \phi)$

\textbf{આકૃતિ:}
\begin{center}
\begin{tikzpicture}[scale=0.8]
\draw[->] (-1,0) -- (3,0) node[right] {રેફરન્સ};
\draw[->] (0,-1) -- (0,3);
\draw[->, thick, blue] (0,0) -- (2,2) node[right] {$E_m$};
\draw (0.5,0) arc (0:45:0.5) node[midway, right] {$\phi$};
\draw[dashed] (2,2) -- (0,2) node[left] {$e = E_m \sin\phi$};
\end{tikzpicture}
\end{center}
\end{solutionbox}

\begin{mnemonicbox}
``વેક્ટર્સ વિઝ્યુઅલાઇઝ વોલ્ટેજ''
\end{mnemonicbox}

%----------------------------------------
\section*{પ્રશ્ન 4(બ) [4 ગુણ]}
\textbf{વ્યાખ્યા આપો: RMS વેલ્યુ, એવરેજ વેલ્યુ, ફ્રિક્વન્સી, ટાઇમ પિરિયડ.}

\begin{solutionbox}
\begin{center}
\begin{tabular}{|l|p{8cm}|}
\hline
\textbf{પદ} & \textbf{વ્યાખ્યા} \\
\hline
RMS વેલ્યુ & અસરકારક DC મૂલ્ય જે સમાન ઉષ્મા ઉત્પન્ન કરે. $I_{rms} = I_m/\sqrt{2}$. \\
\hline
એવરેજ વેલ્યુ & અર્ધ સાઇકલ પરનો સરેરાશ. $I_{avg} = 2I_m/\pi$. \\
\hline
ફ્રિક્વન્સી & સેકન્ડ દીઠ સાઇકલની સંખ્યા. $f = 1/T$ (Hz). \\
\hline
ટાઇમ પિરિયડ & એક સાઇકલ પૂર્ણ કરવા માટે લાગતો સમય. $T = 1/f$ (સેકન્ડ). \\
\hline
\end{tabular}
\end{center}
\end{solutionbox}

\begin{mnemonicbox}
``રિયલી મીન સ્ક્વેર, એવરેજ ફ્રિક્વન્સી ટાઇમ''
\end{mnemonicbox}

%----------------------------------------
\section*{પ્રશ્ન 4(ક) [7 ગુણ]}
\textbf{સ્ટાર જોડાણમાં લાઇન વોલ્ટેજ અને ફેઇઝ વોલ્ટેજ તથા લાઇન કરંટ અને ફેઇઝ કરંટ વચ્ચેનો સંબંધ દર્શાવતા સૂત્ર તારવો.}

\begin{solutionbox}
\textbf{સ્ટાર કનેક્શન:}
\begin{itemize}
\item \textbf{લાઇન કરંટ}: $I_L = I_{ph}$ (શ્રેણી જોડાણ).
\item \textbf{લાઇન વોલ્ટેજ}: બે ફેઝ વોલ્ટેજનું વેક્ટર ડિફરન્સ. $V_L = \sqrt{3} V_{ph}$.
\end{itemize}

\textbf{આકૃતિ:}
\begin{center}
\begin{circuitikz}[scale=0.8]
\draw (0,0) node[below] {N} to[short] (0,0);
\draw (0,0) to[R, l=$Z_{ph}$] (0,2) node[above] {R};
\draw (0,0) to[R, l=$Z_{ph}$] (-1.7,-1) node[below] {Y};
\draw (0,0) to[R, l=$Z_{ph}$] (1.7,-1) node[below] {B};
\end{circuitikz}
\end{center}

\begin{keyformula}
\[V_L = \sqrt{3} V_{ph} \quad \text{અને} \quad I_L = I_{ph}\]
\end{keyformula}
\end{solutionbox}

\begin{mnemonicbox}
``સ્ટાર સ્કેલ્સ વોલ્ટેજ'' ($\sqrt{3}$ ગણું)
\end{mnemonicbox}

%----------------------------------------
\section*{પ્રશ્ન 4(અ અથવા) [3 ગુણ]}
\textbf{અલ્ટરનેટિંગ કરંટનું વેક્ટર નિરૂપણ સમજાવો.}

\begin{solutionbox}
\textbf{કન્સેપ્ટ}: વોલ્ટેજની જેમ, AC કરંટ પણ ફેઝર તરીકે દર્શાવી શકાય છે.
\[i = I_m \sin(\omega t \pm \phi)\]

\textbf{ટેબલ:}
\begin{center}
\begin{tabular}{|l|l|}
\hline
\textbf{રાશિ} & \textbf{પ્રતીક} \\
\hline
મૂલ્ય & $I_m$ (પિક) \\
\hline
RMS & $I = I_m/\sqrt{2}$ \\
\hline
ફેઝ એંગલ & $\phi$ (લીડ/લેગ) \\
\hline
\end{tabular}
\end{center}
\end{solutionbox}

\begin{mnemonicbox}
``કરંટ સર્કલ્સ કન્ટિન્યુઅસલી''
\end{mnemonicbox}

%----------------------------------------
\section*{પ્રશ્ન 4(બ અથવા) [4 ગુણ]}
\textbf{વ્યાખ્યા આપો: ફોર્મ ફેક્ટર, પીક ફેક્ટર, કોણીય વેગ, એમ્પ્લિટ્યૂડ.}

\begin{solutionbox}
\begin{center}
\begin{tabular}{|l|p{7cm}|l|}
\hline
\textbf{પદ} & \textbf{વ્યાખ્યા} & \textbf{મૂલ્ય (Sine)} \\
\hline
ફોર્મ ફેક્ટર & $K_f = I_{rms}/I_{avg}$ & $1.11$ \\
\hline
પીક ફેક્ટર & $K_p = I_{max}/I_{rms}$ & $1.414$ \\
\hline
કોણીય વેગ & ફેઝ બદલવાનો દર ($\omega = 2\pi f$). & $314$ rad/s \\
\hline
એમ્પ્લિટ્યૂડ & મહત્તમ કિંમત ($I_m$). & - \\
\hline
\end{tabular}
\end{center}
\end{solutionbox}

\begin{mnemonicbox}
``ફોર્મ પીક એંગ્યુલર એમ્પ્લિટ્યૂડ''
\end{mnemonicbox}

%----------------------------------------
\section*{પ્રશ્ન 4(ક અથવા) [7 ગુણ]}
\textbf{ડેલ્ટા જોડાણમાં લાઇન વોલ્ટેજ અને ફેઇઝ વોલ્ટેજ તથા લાઇન કરંટ અને ફેઇઝ કરંટ વચ્ચેનો સંબંધ દર્શાવતા સૂત્ર તારવો.}

\begin{solutionbox}
\textbf{ડેલ્ટા કનેક્શન:}
\begin{itemize}
\item \textbf{લાઇન વોલ્ટેજ}: $V_L = V_{ph}$ (સીધું જોડાણ).
\item \textbf{લાઇન કરંટ}: બે ફેઝ કરંટનું વેક્ટર ડિફરન્સ. $I_L = \sqrt{3} I_{ph}$.
\end{itemize}

\textbf{આકૃતિ:}
\begin{center}
\begin{circuitikz}[scale=0.8]
\draw (0,0) to[R, l=$Z_{ph}$] (2,0) to[R, l=$Z_{ph}$] (1,1.732) to[R, l=$Z_{ph}$] (0,0);
\node at (0,0) [below left] {B};
\node at (2,0) [below right] {Y};
\node at (1,1.732) [above] {R};
\end{circuitikz}
\end{center}

\begin{keyformula}
\[V_L = V_{ph} \quad \text{અને} \quad I_L = \sqrt{3} I_{ph}\]
\end{keyformula}
\end{solutionbox}

\begin{mnemonicbox}
``ડેલ્ટા ડબલ્સ કરંટ'' ($\sqrt{3}$ ગણું)
\end{mnemonicbox}

%----------------------------------------
\section*{પ્રશ્ન 5(અ) [3 ગુણ]}
\textbf{શુદ્ધ અવરોધ ધરાવતા પરિપથ માંથી અલ્ટરનેટિંગ કરંટની વર્તણૂક જરૂરી આકૃતિ અને વેવફોર્મ સાથે સમજાવો.}

\begin{solutionbox}
\textbf{વિશ્લેષણ:}
\begin{itemize}
\item વોલ્ટેજ અને કરંટ સમાન ફેઝમાં હોય છે ($\phi = 0^\circ$).
\item ઇમ્પીડન્સ = રેઝિસ્ટન્સ ($Z = R$).
\end{itemize}

\textbf{આકૃતિ:}
\begin{center}
\begin{minipage}{0.4\textwidth}
\centering
\begin{circuitikz}[scale=0.8]
\draw (0,0) to[sinusoidal voltage source, l=$V$] (0,2) -- (2,2) to[R, l=$R$] (2,0) -- (0,0);
\end{circuitikz}
\end{minipage}
\begin{minipage}{0.4\textwidth}
\centering
\begin{tikzpicture}[scale=0.8]
\draw[->] (0,-1.2) -- (0,1.5);
\draw[->] (0,0) -- (3,0) node[right] {$t$};
\draw[thick, blue] plot[domain=0:2.5*pi, samples=100] (\x/2, {sin(\x r)});
\draw[thick, red, dashed] plot[domain=0:2.5*pi, samples=100] (\x/2, {0.6*sin(\x r)});
\node[blue] at (1,1.2) {$V$};
\node[red] at (1,0.4) {$I$};
\end{tikzpicture}
\end{minipage}
\end{center}
\end{solutionbox}

\begin{mnemonicbox}
``રેઝિસ્ટર રિફ્યુઝ ફેઝ શિફ્ટ''
\end{mnemonicbox}

%----------------------------------------
\section*{પ્રશ્ન 5(બ) [4 ગુણ]}
\textbf{વ્યાખ્યા આપો: ઇમ્પીડન્સ, ફેઝ એંગલ, પાવર ફેક્ટર, રિએક્ટિવ પાવર.}

\begin{solutionbox}
\begin{center}
\begin{tabular}{|l|p{6cm}|l|}
\hline
\textbf{પદ} & \textbf{વ્યાખ્યા} & \textbf{સૂત્ર} \\
\hline
ઇમ્પીડન્સ & કરંટનો કુલ વિરોધ ($Z$). & $Z = \sqrt{R^2 + X^2}$ \\
\hline
ફેઝ એંગલ & $V$ અને $I$ વચ્ચેનો કોણ. & $\phi = \tan^{-1}(X/R)$ \\
\hline
પાવર ફેક્ટર & એક્ટિવ પાવર માટે જવાબદાર Cosine કોણ. & $PF = \cos\phi = R/Z$ \\
\hline
રિએક્ટિવ પાવર & સોર્સ અને લોડ વચ્ચે ફરતો પાવર. & $Q = VI \sin\phi$ \\
\hline
\end{tabular}
\end{center}
\end{solutionbox}

\begin{mnemonicbox}
``ઇમ્પીડન્સ ફેઝ પાવર ક્વાડ્રેચર''
\end{mnemonicbox}

%----------------------------------------
\section*{પ્રશ્ન 5(ક) [7 ગુણ]}
\textbf{જુદા જુદા પ્રકારના પ્રોટેક્ટિવ ડિવાઇસના નામ લખો અને MCB ની રચના અને કાર્ય સમજાવો.}

\begin{solutionbox}
\textbf{ડિવાઇસ}: ફ્યુઝ, MCB, MCCB, ELCB, રિલે.

\textbf{MCB (મિનિએચર સર્કિટ બ્રેકર):}
\begin{itemize}
\item \textbf{રચના}: કોન્ટેક્ટ્સ, આર્ક ચેમ્બર, બાઇમેટાલિક સ્ટ્રિપ (થર્મલ), મેગ્નેટિક કોઇલ (મેગ્નેટિક).
\end{itemize}

\textbf{કાર્ય સિદ્ધાંત:}
\begin{enumerate}
\item \textbf{ઓવરલોડ}: બાઇમેટાલિક સ્ટ્રિપ ગરમ થઈ વળે છે અને બ્રેકર ટ્રિપ કરે છે (ધીમું).
\item \textbf{શોર્ટ સર્કિટ}: મેગ્નેટિક કોઇલમાં હાઇ કરંટથી મજબૂત ફિલ્ડ બને છે જે ઇન્સ્ટન્ટ ટ્રિપ કરે છે (ઝડપી).
\end{enumerate}

\textbf{બ્લોક ડાયાગ્રામ:}
\begin{center}
\begin{tikzpicture}
\node[draw, rectangle] (Strip) {બાઇમેટાલિક સ્ટ્રિપ};
\node[draw, rectangle, right=of Strip] (Coil) {મેગ્નેટિક કોઇલ};
\node[draw, circle, below=of Strip] (Mech) {લેચ};
\draw[->] (Strip) -- (Mech) node[midway, left] {ઓવરલોડ};
\draw[->] (Coil) -- (Mech) node[midway, right] {શોર્ટ સર્કિટ};
\end{tikzpicture}
\end{center}
\end{solutionbox}

\begin{mnemonicbox}
``MCB મેગ્નેટિકલી કન્ટ્રોલ્સ બોથ''
\end{mnemonicbox}

%----------------------------------------
\section*{પ્રશ્ન 5(અ અથવા) [3 ગુણ]}
\textbf{શુદ્ધ ઇન્ડક્ટર ધરાવતા પરિપથ માંથી અલ્ટરનેટિંગ કરંટની સમીકરણ તારવો.}

\begin{solutionbox}
\textbf{આપેલ}: $v = V_m \sin(\omega t)$, $v = L \frac{di}{dt}$

\textbf{તારવણી}:
\[di = \frac{v}{L} dt = \frac{V_m}{L} \sin(\omega t) dt\]
\[i = \int \frac{V_m}{L} \sin(\omega t) dt = -\frac{V_m}{\omega L} \cos(\omega t)\]
\[i = \frac{V_m}{\omega L} \sin(\omega t - 90^\circ)\]

\textbf{તારણ}: કરંટ વોલ્ટેજ કરતાં $90^\circ$ પાછળ હોય છે. $X_L = \omega L$.
\end{solutionbox}

\begin{mnemonicbox}
``ઇન્ડક્ટર ઇમ્પીડ્સ, કરંટ લેગ્સ''
\end{mnemonicbox}

%----------------------------------------
\section*{પ્રશ્ન 5(બ અથવા) [4 ગુણ]}
\textbf{AC સર્કિટમાં પાવર અને પાવર ટ્રાયએંગલ સમજાવો.}

\begin{solutionbox}
\textbf{પાવરના પ્રકારો:}
\begin{itemize}
\item \textbf{એક્ટિવ ($P$)}: $VI \cos\phi$ (વોટ) - ઉપયોગી કાર્ય.
\item \textbf{રિએક્ટિવ ($Q$)}: $VI \sin\phi$ (VAR) - ફિલ્ડ મેન્ટેનન્સ.
\item \textbf{એપેરન્ટ ($S$)}: $VI$ (VA) - કુલ રેટિંગ.
\end{itemize}

\textbf{પાવર ત્રિકોણ:}
\begin{center}
\begin{tikzpicture}[scale=0.8]
\draw[->] (0,0) -- (4,0) node[midway, below] {$P$ (એક્ટિવ)};
\draw[->] (4,0) -- (4,3) node[midway, right] {$Q$ (રિએક્ટિવ)};
\draw[->] (0,0) -- (4,3) node[midway, above left] {$S$ (એપેરન્ટ)};
\draw (1,0) arc (0:36.8:1) node[midway, right] {$\phi$};
\end{tikzpicture}
\end{center}

$S^2 = P^2 + Q^2$. પાવર ફેક્ટર $\cos\phi = P/S$.
\end{solutionbox}

\begin{mnemonicbox}
``પાવર ટ્રાયએંગલ''
\end{mnemonicbox}

%----------------------------------------
\section*{પ્રશ્ન 5(ક અથવા) [7 ગુણ]}
\textbf{એક લેમ્પને એક જગ્યાએથી કન્ટ્રોલ કરવો તેમજ દાદર માટેનું વાયરિંગ ડાયાગ્રામ સાથે સમજાવો.}

\begin{solutionbox}
\textbf{1. એક જગ્યાએથી કન્ટ્રોલ:}
સ્વિચ ($S$) અને લેમ્પ ($L$) નું શ્રેણી જોડાણ.
\\
\textbf{ડાયાગ્રામ}: લાઇવ $\rightarrow$ સ્વિચ $\rightarrow$ લેમ્પ $\rightarrow$ ન્યુટ્રલ.

\textbf{2. સીડીનું વાયરિંગ (ટુ-વે કન્ટ્રોલ):}
બે SPDT (ટુ-વે) સ્વિચનો ઉપયોગ થાય છે.

\textbf{ડાયાગ્રામ:}
\begin{center}
\begin{circuitikz}[scale=1.0]
\draw (0,0) node[left]{લાઇવ} to[short] (1,0); 
% Switch 1
\draw (1,0) -- (1.5, 0.5) node[above]{A1};
\draw (1,0) -- (1.5, -0.5) node[below]{B1};
% Switch 2
\draw (4.5,0) -- (4, 0.5) node[above]{A2};
\draw (4.5,0) -- (4, -0.5) node[below]{B2};
\draw (4.5,0) to[lamp] (6,0) node[right]{ન્યુટ્રલ};
% Wires
\draw (1.5, 0.5) -- (4, 0.5);
\draw (1.5, -0.5) -- (4, -0.5);
\end{circuitikz}
\end{center}

\textbf{કાર્ય}: લેમ્પને બંનેમાંથી કોઈ પણ સ્વિચથી ચાલુ/બંધ કરી શકાય છે.
\end{solutionbox}

\begin{mnemonicbox}
``ટુ-વે ટોગલ્સ, ટુ પ્લેસિસ''
\end{mnemonicbox}

\end{document}
