\documentclass[10pt,a4paper]{article}

% content/resources/templates/preamble.tex
\usepackage[margin=0.6in]{geometry}
\author{Milav Dabgar}
\usepackage{amsmath,amssymb,amsthm}
\usepackage{booktabs}
\usepackage{multirow}
\usepackage{xcolor}
\usepackage{tcolorbox}
\tcbuselibrary{breakable,skins}
\usepackage[colorlinks=true,linkcolor=blue]{hyperref}
\usepackage{titlesec}
\usepackage{enumitem}
\usepackage{tikz}
\usepackage{pgfplots}
\usepackage{circuitikz}
\usepackage[version=4]{mhchem}
\usepackage{longtable}
\usepackage{array}
\usepackage{float}
\usepackage{caption}
\usepackage{listings}

\lstset{
  basicstyle=\small\ttfamily,
  breaklines=true,
  breakatwhitespace=false,
  postbreak=\mbox{\textcolor{red}{$\hookrightarrow$}\space},
  float=false,
  numbers=left,
  numberstyle=\tiny\color{gray},
  numbersep=10pt,
  xleftmargin=2em,
  keywordstyle=\color{blue},
  commentstyle=\color{green!60!black},
  stringstyle=\color{purple},
  backgroundcolor=\color{gray!5},
  showstringspaces=false,
  tabsize=2,
  captionpos=b,
  keepspaces=true,
  columns=flexible
}

\pgfplotsset{compat=1.18}
\usetikzlibrary{shapes,arrows,positioning,calc,patterns,decorations.pathmorphing,decorations.markings,arrows.meta}

% Color scheme
\definecolor{headcolor}{RGB}{0,102,204}
\definecolor{keycolor}{RGB}{220,20,60}
\definecolor{solutioncolor}{RGB}{34,139,34}
\definecolor{mnemoniccolor}{RGB}{148,0,211}
\definecolor{codecolor}{RGB}{0,0,100}

% Spacing
\setlength{\parskip}{3pt}
\setlist[itemize]{nosep}
\setlist[enumerate]{nosep}

% Title formatting
\titleformat{\section}{\Large\bfseries\color{headcolor}}{\thesection}{1em}{}
\titleformat{\subsection}{\large\bfseries\color{headcolor}}{\thesubsection}{1em}{}

% Pandoc tightlist compatibility
\providecommand{\tightlist}{%
  \setlength{\itemsep}{0pt}\setlength{\parskip}{0pt}}

% Pandoc longtable compatibility
\newcounter{none}
\def\thenone{}


% content/resources/templates/english-boxes.tex
% This file is currently empty - it exists to maintain consistency with the import structure.
% Add custom environments here if needed in the future.


\begin{document}

\begin{center}
{\Huge\bfseries\color{headcolor} Basic Electronics Solutions}\\[5pt]
{\LARGE DI01000101 -- Winter 2024}\\[3pt]
{\large Semester 1 Study Material}\\[3pt]
{\normalsize\textit{Detailed Solutions and Explanations}}
\end{center}

\vspace{10pt}

\subsection*{Question 1(a) [3 marks]}\label{q1a}

\textbf{Explain ohm's law with its limitation and application.}

\begin{solutionbox}


\vspace{-5pt}
\captionof{table}{Ohm's Law Summary}
\vspace{-10pt}
\begin{longtable}[]{@{}
  >{\raggedright\arraybackslash}p{(\linewidth - 2\tabcolsep) * \real{0.4000}}
  >{\raggedright\arraybackslash}p{(\linewidth - 2\tabcolsep) * \real{0.6000}}@{}}
\toprule\noalign{}
\begin{minipage}[b]{\linewidth}\raggedright
Aspect
\end{minipage} & \begin{minipage}[b]{\linewidth}\raggedright
Description
\end{minipage} \\
\midrule\noalign{}
\endhead
\bottomrule\noalign{}
\endlastfoot
\textbf{Statement} & Current through conductor is directly proportional
to voltage \\
\textbf{Formula} & V = I \times R \\
\textbf{Units} & V (Volts), I (Amperes), R (Ohms) \\
\end{longtable}

\textbf{Limitations:}

\begin{itemize}
\tightlist
\item
  \textbf{Temperature dependency}: Resistance changes with temperature
\item
  \textbf{Non-linear materials}: Does not apply to semiconductors,
  diodes
\item
  \textbf{AC circuits}: Modified form needed for reactive components
\end{itemize}

\textbf{Applications:}

\begin{itemize}
\tightlist
\item
  \textbf{Circuit analysis}: Calculate unknown voltage, current, or
  resistance
\item
  \textbf{Power calculations}: P = V^{2}/R, P = I^{2}R
\end{itemize}

\end{solutionbox}
\begin{mnemonicbox}
``Voltage Is Really Important'' (V = I \times R)

\end{mnemonicbox}
\subsection*{Question 1(b) [4 marks]}\label{q1b}

\textbf{Explain faraday's law of electromagnetic induction with
necessary figure.}

\begin{solutionbox}

\textbf{Faraday's Laws:}

\begin{itemize}
\tightlist
\item
  \textbf{First Law}: EMF is induced when magnetic flux changes through
  conductor
\item
  \textbf{Second Law}: Magnitude of EMF equals rate of flux change
\end{itemize}

\textbf{Mathematical Expression:}

\begin{verbatim}
e = -N \times (d\Phi/dt)
\end{verbatim}

\textbf{Diagram:}

\begin{verbatim}
    +{-{-}{-}{-}{-}{-}{-}+}
    |   N   |  Coil with N turns
    |       |
    +{-{-}{-}+{-}{-}{-}+}
        |
        |  Moving magnet
    +{-{-}{-}v{-}{-}{-}+}
    | S | N |  
    +{-{-}{-}{-}{-}{-}{-}+}
       ↕
   Motion direction
\end{verbatim}

\textbf{Applications:}

\begin{itemize}
\tightlist
\item
  \textbf{Transformers}: Mutual induction principle
\item
  \textbf{Generators}: Mechanical to electrical energy conversion
\item
  \textbf{Inductors}: Self-induced EMF opposes current changes
\end{itemize}

\end{solutionbox}
\begin{mnemonicbox}
``Flux Change Generates EMF'' (d\Phi/dt = EMF)

\end{mnemonicbox}
\subsection*{Question 1(c) [7 marks]}\label{q1c}

\textbf{Explain kirchhoff's voltage law and kirchhoff's current law with
necessary diagram.}

\begin{solutionbox}


\vspace{-5pt}
\captionof{table}{Kirchhoff's Laws Comparison}
\vspace{-10pt}
\begin{longtable}[]{@{}
  >{\raggedright\arraybackslash}p{(\linewidth - 6\tabcolsep) * \real{0.1042}}
  >{\raggedright\arraybackslash}p{(\linewidth - 6\tabcolsep) * \real{0.2292}}
  >{\raggedright\arraybackslash}p{(\linewidth - 6\tabcolsep) * \real{0.3958}}
  >{\raggedright\arraybackslash}p{(\linewidth - 6\tabcolsep) * \real{0.2708}}@{}}
\toprule\noalign{}
\begin{minipage}[b]{\linewidth}\raggedright
Law
\end{minipage} & \begin{minipage}[b]{\linewidth}\raggedright
Statement
\end{minipage} & \begin{minipage}[b]{\linewidth}\raggedright
Mathematical Form
\end{minipage} & \begin{minipage}[b]{\linewidth}\raggedright
Application
\end{minipage} \\
\midrule\noalign{}
\endhead
\bottomrule\noalign{}
\endlastfoot
\textbf{KVL} & Sum of voltages in closed loop = 0 & \SigmaV = 0 & Series
circuits \\
\textbf{KCL} & Sum of currents at node = 0 & \SigmaI = 0 & Parallel
circuits \\
\end{longtable}

\textbf{KVL Diagram:}

\begin{center}
\textbf{Mermaid Diagram (Code)}
\begin{verbatim}
{Shaded}
{Highlighting}[]
graph LR
    A["{+"] {-}{-}{} B[V1]}
    B {-{-}{} C[R1]}
    C {-{-}{} D[V2]}
    D {-{-}{} E[R2]}
    E {-{-}{} A}
    
    style A fill:\#f9f,stroke:\#333,stroke{-width:2px}
    style C fill:\#bbf,stroke:\#333,stroke{-width:2px}
    style E fill:\#bbf,stroke:\#333,stroke{-width:2px}
{Highlighting}
{Shaded}
\end{verbatim}
\end{center}

\textbf{KCL Diagram:}

\begin{center}
\textbf{Mermaid Diagram (Code)}
\begin{verbatim}
{Shaded}
{Highlighting}[]
graph TD
    A[I1] {-{-}{} B((Node))}
    C[I2] {-{-}{} B}
    B {-{-}{} D[I3]}
    B {-{-}{} E[I4]}
    
    style B fill:\#f96,stroke:\#333,stroke{-width:4px}
{Highlighting}
{Shaded}
\end{verbatim}
\end{center}

\textbf{Key Points:}

\begin{itemize}
\tightlist
\item
  \textbf{KVL}: Algebraic sum considers voltage polarities
\item
  \textbf{KCL}: Considers current directions (incoming vs outgoing)
\item
  \textbf{Applications}: Circuit analysis, finding unknown values
\end{itemize}

\end{solutionbox}
\begin{mnemonicbox}
``Voltage Loops, Current Nodes'' (KVL for loops, KCL
for nodes)

\end{mnemonicbox}
\subsection*{Question 1(c OR) [7
marks]}\label{question-1c-or-7-marks}

\textbf{Differentiate statically induced emf and dynamically induced
emf}

\begin{solutionbox}


\vspace{-5pt}
\captionof{table}{Static vs Dynamic EMF}
\vspace{-10pt}
\begin{longtable}[]{@{}
  >{\raggedright\arraybackslash}p{(\linewidth - 4\tabcolsep) * \real{0.1833}}
  >{\raggedright\arraybackslash}p{(\linewidth - 4\tabcolsep) * \real{0.4000}}
  >{\raggedright\arraybackslash}p{(\linewidth - 4\tabcolsep) * \real{0.4167}}@{}}
\toprule\noalign{}
\begin{minipage}[b]{\linewidth}\raggedright
Parameter
\end{minipage} & \begin{minipage}[b]{\linewidth}\raggedright
Statically Induced EMF
\end{minipage} & \begin{minipage}[b]{\linewidth}\raggedright
Dynamically Induced EMF
\end{minipage} \\
\midrule\noalign{}
\endhead
\bottomrule\noalign{}
\endlastfoot
\textbf{Cause} & Changing magnetic field & Relative motion between
conductor and field \\
\textbf{Field} & Time-varying, conductor stationary & Steady field,
conductor moving \\
\textbf{Examples} & Transformer, inductor & Generator, motor \\
\textbf{Formula} & e = -N(d\Phi/dt) & e = BLv \\
\textbf{Applications} & AC circuits, power supplies & Power generation,
motors \\
\end{longtable}

\textbf{Static EMF Types:}

\begin{itemize}
\tightlist
\item
  \textbf{Self-induced}: Same coil creates and experiences flux change
\item
  \textbf{Mutually induced}: One coil affects another coil
\end{itemize}

\textbf{Dynamic EMF Factors:}

\begin{itemize}
\tightlist
\item
  \textbf{Magnetic field strength (B)}: Tesla
\item
  \textbf{Conductor length (L)}: Meters\\
\item
  \textbf{Velocity (v)}: m/s
\end{itemize}

\end{solutionbox}
\begin{mnemonicbox}
``Static Stays, Dynamic Dances'' (Static =
stationary, Dynamic = motion)

\end{mnemonicbox}
\subsection*{Question 2(a) [3 marks]}\label{q2a}

\textbf{Explain various types of losses in transformer.}

\begin{solutionbox}


\vspace{-5pt}
\captionof{table}{Transformer Losses}
\vspace{-10pt}
\begin{longtable}[]{@{}
  >{\raggedright\arraybackslash}p{(\linewidth - 6\tabcolsep) * \real{0.2391}}
  >{\raggedright\arraybackslash}p{(\linewidth - 6\tabcolsep) * \real{0.1739}}
  >{\raggedright\arraybackslash}p{(\linewidth - 6\tabcolsep) * \real{0.2174}}
  >{\raggedright\arraybackslash}p{(\linewidth - 6\tabcolsep) * \real{0.3696}}@{}}
\toprule\noalign{}
\begin{minipage}[b]{\linewidth}\raggedright
Loss Type
\end{minipage} & \begin{minipage}[b]{\linewidth}\raggedright
Cause
\end{minipage} & \begin{minipage}[b]{\linewidth}\raggedright
Location
\end{minipage} & \begin{minipage}[b]{\linewidth}\raggedright
Characteristics
\end{minipage} \\
\midrule\noalign{}
\endhead
\bottomrule\noalign{}
\endlastfoot
\textbf{Iron Loss} & Hysteresis + Eddy currents & Core & Constant,
frequency dependent \\
\textbf{Copper Loss} & I^{2}R heating & Windings & Variable with load \\
\textbf{Stray Loss} & Leakage flux & Overall & Minimal \\
\end{longtable}

\textbf{Iron Losses:}

\begin{itemize}
\tightlist
\item
  \textbf{Hysteresis loss}: Magnetic domain reversal energy
\item
  \textbf{Eddy current loss}: Circulating currents in core
\end{itemize}

\textbf{Copper Losses:}

\begin{itemize}
\tightlist
\item
  \textbf{Primary winding}: I_{1}^{2}R_{1}
\item
  \textbf{Secondary winding}: I_{2}^{2}R_{2}
\end{itemize}

\end{solutionbox}
\begin{mnemonicbox}
``Iron Core, Copper Coil'' (Location of main losses)

\end{mnemonicbox}
\subsection*{Question 2(b) [4 marks]}\label{q2b}

\textbf{Explain working principle of transformer.}

\begin{solutionbox}

\textbf{Working Principle:} \textbf{Mutual electromagnetic induction}
between primary and secondary windings through common magnetic core.

\textbf{Diagram:}

\begin{center}
\textbf{Mermaid Diagram (Code)}
\begin{verbatim}
{Shaded}
{Highlighting}[]
graph LR
    AC[AC Supply] {-{-}{} P[Primary Coil N1]}
    P {-{-}{} C[Iron Core]}
    C {-{-}{} S[Secondary Coil N2]}
    S {-{-}{} L[Load]}
    
    style C fill:\#f96,stroke:\#333,stroke{-width:3px}
    style P fill:\#bbf,stroke:\#333,stroke{-width:2px}
    style S fill:\#bfb,stroke:\#333,stroke{-width:2px}
{Highlighting}
{Shaded}
\end{verbatim}
\end{center}

\textbf{Operation Steps:}

\begin{itemize}
\tightlist
\item
  \textbf{Step 1}: AC current in primary creates alternating flux
\item
  \textbf{Step 2}: Flux links secondary through core
\item
  \textbf{Step 3}: Changing flux induces EMF in secondary
\item
  \textbf{Step 4}: Secondary EMF drives current through load
\end{itemize}

\textbf{Key Relations:}

\begin{itemize}
\tightlist
\item
  \textbf{Voltage ratio}: V_{2}/V_{1} = N_{2}/N_{1}
\item
  \textbf{Current ratio}: I_{1}/I_{2} = N_{2}/N_{1}
\end{itemize}

\end{solutionbox}
\begin{mnemonicbox}
``Primary Produces, Secondary Supplies'' (Energy
transfer direction)

\end{mnemonicbox}
\subsection*{Question 2(c) [7 marks]}\label{q2c}

\textbf{Derive emf equation of transformer.}

\begin{solutionbox}

\textbf{Given Parameters:}

\begin{itemize}
\tightlist
\item
  \textbf{N_{1}}: Primary turns, \textbf{N_{2}}: Secondary turns
\item
  \textbf{\Phi_{m}}: Maximum flux, \textbf{f}: Frequency
\end{itemize}

\textbf{EMF Derivation:}

\textbf{Step 1: Flux Variation}

\begin{verbatim}
\Phi = \Phi_{m} sin(2\pift)
\end{verbatim}

\textbf{Step 2: Rate of Flux Change}

\begin{verbatim}
d\Phi/dt = 2\pif\Phi_{m} cos(2\pift)
\end{verbatim}

\textbf{Step 3: Maximum Rate}

\begin{verbatim}
(d\Phi/dt)_{m}_{a}_{x} = 2\pif\Phi_{m}
\end{verbatim}

\textbf{Step 4: RMS EMF Formula}

\begin{verbatim}
E_{1} = 4.44 \times f \times N_{1} \times \Phi_{m}
E_{2} = 4.44 \times f \times N_{2} \times \Phi_{m}
\end{verbatim}


\vspace{-5pt}
\captionof{table}{EMF Equation Components}
\vspace{-10pt}
\begin{longtable}[]{@{}lll@{}}
\toprule\noalign{}
Symbol & Parameter & Units \\
\midrule\noalign{}
\endhead
\bottomrule\noalign{}
\endlastfoot
\textbf{E} & RMS EMF & Volts \\
\textbf{f} & Frequency & Hz \\
\textbf{N} & Number of turns & - \\
\textbf{\Phi_{m}} & Maximum flux & Weber \\
\textbf{4.44} & Form factor constant & - \\
\end{longtable}

\textbf{Transformation Ratio:}

\begin{verbatim}
K = E_{2}/E_{1} = N_{2}/N_{1}
\end{verbatim}

\end{solutionbox}
\begin{mnemonicbox}
``Four-Forty-Four Flux Formula'' (4.44 factor)

\end{mnemonicbox}
\subsection*{Question 2(a OR) [3
marks]}\label{question-2a-or-3-marks}

\textbf{Write application of transformer.}

\begin{solutionbox}


\vspace{-5pt}
\captionof{table}{Transformer Applications}
\vspace{-10pt}
\begin{longtable}[]{@{}
  >{\raggedright\arraybackslash}p{(\linewidth - 4\tabcolsep) * \real{0.3514}}
  >{\raggedright\arraybackslash}p{(\linewidth - 4\tabcolsep) * \real{0.2432}}
  >{\raggedright\arraybackslash}p{(\linewidth - 4\tabcolsep) * \real{0.4054}}@{}}
\toprule\noalign{}
\begin{minipage}[b]{\linewidth}\raggedright
Application
\end{minipage} & \begin{minipage}[b]{\linewidth}\raggedright
Purpose
\end{minipage} & \begin{minipage}[b]{\linewidth}\raggedright
Voltage Level
\end{minipage} \\
\midrule\noalign{}
\endhead
\bottomrule\noalign{}
\endlastfoot
\textbf{Power transmission} & Reduce transmission losses & Step-up
(400kV) \\
\textbf{Distribution} & Safe voltage for consumers & Step-down (230V) \\
\textbf{Isolation} & Electrical isolation & 1:1 ratio \\
\textbf{Electronic circuits} & DC power supplies & Step-down \\
\end{longtable}

\textbf{Industrial Applications:}

\begin{itemize}
\tightlist
\item
  \textbf{Welding transformers}: High current, low voltage
\item
  \textbf{Instrument transformers}: Measurement and protection
\item
  \textbf{Audio transformers}: Impedance matching
\end{itemize}

\end{solutionbox}
\begin{mnemonicbox}
``Power Distribution Isolation Electronics'' (Main
application areas)

\end{mnemonicbox}
\subsection*{Question 2(b OR) [4
marks]}\label{question-2b-or-4-marks}

\textbf{Write equation for back emf and torque of D.C motor.}

\begin{solutionbox}

\textbf{Back EMF Equation:}

\begin{verbatim}
Eb = (\phi \times Z \times N \times P) / (60 \times A)
\end{verbatim}

\textbf{Simplified Form:}

\begin{verbatim}
Eb = K \times \phi \times N
\end{verbatim}

\textbf{Torque Equation:}

\begin{verbatim}
T = (\phi \times Z \times Ia \times P) / (2\pi \times A)
\end{verbatim}

\textbf{Simplified Form:}

\begin{verbatim}
T = K \times \phi \times Ia
\end{verbatim}


\vspace{-5pt}
\captionof{table}{Symbol Definitions}
\vspace{-10pt}
\begin{longtable}[]{@{}lll@{}}
\toprule\noalign{}
Symbol & Parameter & Units \\
\midrule\noalign{}
\endhead
\bottomrule\noalign{}
\endlastfoot
\textbf{Eb} & Back EMF & Volts \\
\textbf{T} & Torque & N-m \\
\textbf{\phi} & Flux per pole & Weber \\
\textbf{N} & Speed & RPM \\
\textbf{Ia} & Armature current & Amperes \\
\textbf{K} & Motor constant & - \\
\end{longtable}

\end{solutionbox}
\begin{mnemonicbox}
``Back EMF opposes, Torque proposes'' (EMF opposes
supply, torque drives rotation)

\end{mnemonicbox}
\subsection*{Question 2(c OR) [7
marks]}\label{question-2c-or-7-marks}

\textbf{Explain construction and working of D.C. motor with necessary
figure}

\begin{solutionbox}

\textbf{Construction Components:}


\vspace{-5pt}
\captionof{table}{DC Motor Parts}
\vspace{-10pt}
\begin{longtable}[]{@{}lll@{}}
\toprule\noalign{}
Component & Function & Material \\
\midrule\noalign{}
\endhead
\bottomrule\noalign{}
\endlastfoot
\textbf{Stator} & Provides magnetic field & Cast iron/steel \\
\textbf{Rotor/Armature} & Rotating part & Silicon steel laminations \\
\textbf{Commutator} & Current direction reversal & Copper segments \\
\textbf{Brushes} & Current collection & Carbon \\
\textbf{Field windings} & Electromagnets & Copper wire \\
\end{longtable}

\textbf{Construction Diagram:}

\begin{verbatim}
graph TB
    subgraph "DC Motor"
        N[N Pole] {-{-} A[Armature]}
        A {-{-} S[S Pole]}
        C[Commutator] {-{-} B[Brushes]}
        F[Field Winding] {-{-} N}
        F {-{-} S}
    end
    
    style A fill:\#f96,stroke:\#333,stroke{-width:3px}
    style C fill:\#bbf,stroke:\#333,stroke{-width:2px}
\end{verbatim}

\textbf{Working Principle:}

\begin{itemize}
\tightlist
\item
  \textbf{Step 1}: Current flows through armature conductors
\item
  \textbf{Step 2}: Magnetic field interacts with current
\item
  \textbf{Step 3}: Force generated by Fleming's left-hand rule
\item
  \textbf{Step 4}: Commutator reverses current direction
\item
  \textbf{Step 5}: Continuous rotation maintained
\end{itemize}

\textbf{Force Equation:}

\begin{verbatim}
F = B \times I \times L
\end{verbatim}

\end{solutionbox}
\begin{mnemonicbox}
``Current Creates Circular motion'' (Current
interaction produces rotation)

\end{mnemonicbox}
\subsection*{Question 3(a) [3 marks]}\label{q3a}

\textbf{Explain construction of transformer.}

\begin{solutionbox}


\vspace{-5pt}
\captionof{table}{Transformer Construction}
\vspace{-10pt}
\begin{longtable}[]{@{}lll@{}}
\toprule\noalign{}
Component & Material & Function \\
\midrule\noalign{}
\endhead
\bottomrule\noalign{}
\endlastfoot
\textbf{Core} & Silicon steel laminations & Magnetic flux path \\
\textbf{Primary winding} & Copper/Aluminum & Input energy \\
\textbf{Secondary winding} & Copper/Aluminum & Output energy \\
\textbf{Insulation} & Varnish/Paper & Electrical isolation \\
\textbf{Tank} & Steel & Oil containment \& cooling \\
\end{longtable}

\textbf{Core Types:}

\begin{itemize}
\tightlist
\item
  \textbf{Shell type}: Windings surrounded by core
\item
  \textbf{Core type}: Core surrounded by windings
\end{itemize}

\textbf{Cooling Methods:}

\begin{itemize}
\tightlist
\item
  \textbf{Air cooling}: Small transformers
\item
  \textbf{Oil cooling}: Large transformers with radiators
\end{itemize}

\end{solutionbox}
\begin{mnemonicbox}
``Core Carries Current Carefully'' (Core design
importance)

\end{mnemonicbox}
\subsection*{Question 3(b) [4 marks]}\label{q3b}

\textbf{Explain application of DC motor}

\begin{solutionbox}


\vspace{-5pt}
\captionof{table}{DC Motor Applications}
\vspace{-10pt}
\begin{longtable}[]{@{}lll@{}}
\toprule\noalign{}
Motor Type & Speed Characteristic & Applications \\
\midrule\noalign{}
\endhead
\bottomrule\noalign{}
\endlastfoot
\textbf{Shunt} & Constant speed & Fans, pumps, lathes \\
\textbf{Series} & Variable speed & Traction, cranes \\
\textbf{Compound} & Moderate variation & Elevators, compressors \\
\end{longtable}

\textbf{Industrial Applications:}

\begin{itemize}
\tightlist
\item
  \textbf{Shunt motors}: Machine tools requiring constant speed
\item
  \textbf{Series motors}: Electric vehicles, starting heavy loads
\item
  \textbf{Compound motors}: Rolling mills, punch presses
\end{itemize}

\textbf{Advantages:}

\begin{itemize}
\tightlist
\item
  \textbf{Easy speed control}: Voltage/field control
\item
  \textbf{High starting torque}: Series motors
\item
  \textbf{Reversible operation}: Change field/armature polarity
\end{itemize}

\end{solutionbox}
\begin{mnemonicbox}
``Shunt Stays, Series Speeds'' (Speed
characteristics)

\end{mnemonicbox}
\subsection*{Question 3(c) [7 marks]}\label{q3c}

\textbf{Explain different types of DC motor.}

\begin{solutionbox}


\vspace{-5pt}
\captionof{table}{DC Motor Classification}
\vspace{-10pt}
\begin{longtable}[]{@{}
  >{\raggedright\arraybackslash}p{(\linewidth - 6\tabcolsep) * \real{0.1176}}
  >{\raggedright\arraybackslash}p{(\linewidth - 6\tabcolsep) * \real{0.3529}}
  >{\raggedright\arraybackslash}p{(\linewidth - 6\tabcolsep) * \real{0.2745}}
  >{\raggedright\arraybackslash}p{(\linewidth - 6\tabcolsep) * \real{0.2549}}@{}}
\toprule\noalign{}
\begin{minipage}[b]{\linewidth}\raggedright
Type
\end{minipage} & \begin{minipage}[b]{\linewidth}\raggedright
Field Connection
\end{minipage} & \begin{minipage}[b]{\linewidth}\raggedright
Speed-Torque
\end{minipage} & \begin{minipage}[b]{\linewidth}\raggedright
Applications
\end{minipage} \\
\midrule\noalign{}
\endhead
\bottomrule\noalign{}
\endlastfoot
\textbf{Shunt} & Parallel to armature & Constant speed, low starting
torque & Fans, pumps \\
\textbf{Series} & Series with armature & Variable speed, high starting
torque & Traction \\
\textbf{Compound} & Both series \& shunt & Moderate characteristics &
General purpose \\
\end{longtable}

\textbf{Shunt Motor Diagram:}

\begin{center}
\textbf{Mermaid Diagram (Code)}
\begin{verbatim}
{Shaded}
{Highlighting}[]
graph LR
    V[DC Supply] {-{-}{} A[Armature]}
    V {-{-}{} F[Field Winding]}
    A {-{-}{} V}
    F {-{-}{} V}
    
    style A fill:\#f96,stroke:\#333,stroke{-width:2px}
    style F fill:\#bbf,stroke:\#333,stroke{-width:2px}
{Highlighting}
{Shaded}
\end{verbatim}
\end{center}

\textbf{Characteristics:}

\begin{itemize}
\tightlist
\item
  \textbf{Shunt}: Speed ∝ (V - IaRa)/\phi
\item
  \textbf{Series}: High starting torque, speed varies with load
\item
  \textbf{Compound}: Combines advantages of both types
\end{itemize}

\textbf{Speed Control Methods:}

\begin{itemize}
\tightlist
\item
  \textbf{Armature control}: Vary armature voltage
\item
  \textbf{Field control}: Vary field current
\item
  \textbf{Resistance control}: Add external resistance
\end{itemize}

\end{solutionbox}
\begin{mnemonicbox}
``Shunt Steady, Series Strong, Compound Combined''
(Key characteristics)

\end{mnemonicbox}
\subsection*{Question 3(a OR) [3
marks]}\label{question-3a-or-3-marks}

\textbf{Explain transformation ratio of transformer.}

\begin{solutionbox}

\textbf{Definition:} Transformation ratio (K) is the ratio of secondary
to primary voltage or turns.

\textbf{Mathematical Expression:}

\begin{verbatim}
K = N_{2}/N_{1} = E_{2}/E_{1} = V_{2}/V_{1}
\end{verbatim}


\vspace{-5pt}
\captionof{table}{Transformation Ratio Types}
\vspace{-10pt}
\begin{longtable}[]{@{}llll@{}}
\toprule\noalign{}
Ratio & Type & Voltage Change & Applications \\
\midrule\noalign{}
\endhead
\bottomrule\noalign{}
\endlastfoot
\textbf{K \textgreater{} 1} & Step-up & Increases & Power
transmission \\
\textbf{K \textless{} 1} & Step-down & Decreases & Distribution \\
\textbf{K = 1} & Isolation & Same & Safety isolation \\
\end{longtable}

\textbf{Current Relationship:}

\begin{verbatim}
I_{1}/I_{2} = N_{2}/N_{1} = K
\end{verbatim}

\textbf{Power Relationship:}

\begin{verbatim}
P_{1} = P_{2} (Ideal transformer)
\end{verbatim}

\end{solutionbox}
\begin{mnemonicbox}
``Turns Tell Transformation'' (Turns ratio determines
voltage ratio)

\end{mnemonicbox}
\subsection*{Question 3(b OR) [4
marks]}\label{question-3b-or-4-marks}

\textbf{Write application of autotransformer.}

\begin{solutionbox}


\vspace{-5pt}
\captionof{table}{Autotransformer Applications}
\vspace{-10pt}
\begin{longtable}[]{@{}lll@{}}
\toprule\noalign{}
Application & Advantage & Voltage Range \\
\midrule\noalign{}
\endhead
\bottomrule\noalign{}
\endlastfoot
\textbf{Motor starting} & Reduced starting current & 50-80\% of rated \\
\textbf{Voltage regulation} & Fine voltage adjustment & \pm10\%
variation \\
\textbf{Laboratory} & Variable voltage source & 0-110\% of input \\
\textbf{Power systems} & Economic transmission & Close voltage ratios \\
\end{longtable}

\textbf{Advantages:}

\begin{itemize}
\tightlist
\item
  \textbf{Economy}: Less copper and iron required
\item
  \textbf{Efficiency}: Higher than two-winding transformer
\item
  \textbf{Size}: Compact design
\item
  \textbf{Regulation}: Better voltage regulation
\end{itemize}

\textbf{Limitations:}

\begin{itemize}
\tightlist
\item
  \textbf{No isolation}: Common electrical connection
\item
  \textbf{Safety}: Higher fault current
\end{itemize}

\end{solutionbox}
\begin{mnemonicbox}
``Auto Adjusts Advantageously'' (Automatic voltage
adjustment benefit)

\end{mnemonicbox}
\subsection*{Question 3(c OR) [7
marks]}\label{question-3c-or-7-marks}

\textbf{Explain speed control of DC shunt motor}

\begin{solutionbox}


\vspace{-5pt}
\captionof{table}{Speed Control Methods}
\vspace{-10pt}
\begin{longtable}[]{@{}
  >{\raggedright\arraybackslash}p{(\linewidth - 6\tabcolsep) * \real{0.1951}}
  >{\raggedright\arraybackslash}p{(\linewidth - 6\tabcolsep) * \real{0.1951}}
  >{\raggedright\arraybackslash}p{(\linewidth - 6\tabcolsep) * \real{0.2927}}
  >{\raggedright\arraybackslash}p{(\linewidth - 6\tabcolsep) * \real{0.3171}}@{}}
\toprule\noalign{}
\begin{minipage}[b]{\linewidth}\raggedright
Method
\end{minipage} & \begin{minipage}[b]{\linewidth}\raggedright
Range
\end{minipage} & \begin{minipage}[b]{\linewidth}\raggedright
Efficiency
\end{minipage} & \begin{minipage}[b]{\linewidth}\raggedright
Applications
\end{minipage} \\
\midrule\noalign{}
\endhead
\bottomrule\noalign{}
\endlastfoot
\textbf{Armature control} & Below rated speed & High & Precise speed
control \\
\textbf{Field control} & Above rated speed & High & Constant power
drives \\
\textbf{Resistance control} & Below rated speed & Low & Simple
applications \\
\end{longtable}

\textbf{Armature Control Diagram:}

\begin{center}
\textbf{Mermaid Diagram (Code)}
\begin{verbatim}
{Shaded}
{Highlighting}[]
graph LR
    V[Variable DC] {-{-}{} R[Rheostat]}
    R {-{-}{} A[Armature]}
    A {-{-}{} V}
    V {-{-}{} F[Field Winding]}
    F {-{-}{} V}
    
    style R fill:\#f96,stroke:\#333,stroke{-width:2px}
    style A fill:\#bbf,stroke:\#333,stroke{-width:2px}
{Highlighting}
{Shaded}
\end{verbatim}
\end{center}

\textbf{Speed Equations:}

\begin{itemize}
\tightlist
\item
  \textbf{Armature control}: N ∝ (V - IaRa)/\phi
\item
  \textbf{Field control}: N ∝ V/\phi
\item
  \textbf{Resistance control}: N ∝ (V - Ia(Ra + Rext))/\phi
\end{itemize}

\textbf{Modern Methods:}

\begin{itemize}
\tightlist
\item
  \textbf{Chopper control}: PWM voltage control
\item
  \textbf{Ward-Leonard system}: Motor-generator set
\item
  \textbf{Electronic control}: Thyristor/IGBT drives
\end{itemize}

\textbf{Characteristics:}

\begin{itemize}
\tightlist
\item
  \textbf{Smooth control}: Stepless speed variation
\item
  \textbf{Efficiency}: Armature control most efficient
\item
  \textbf{Cost}: Field control economical
\end{itemize}

\end{solutionbox}
\begin{mnemonicbox}
``Armature Accurate, Field Fast, Resistance Rough''
(Control characteristics)

\end{mnemonicbox}
\subsection*{Question 4(a) [3 marks]}\label{q4a}

\textbf{Explain vector representation of alternating EMF.}

\begin{solutionbox}

\textbf{Vector Representation:} Alternating EMF can be represented as a
rotating vector (phasor) with constant magnitude and angular velocity.

\textbf{Mathematical Form:}

\begin{verbatim}
e = Em sin(\omegat + \phi)
\end{verbatim}

\textbf{Diagram:}

\begin{center}
\textbf{Mermaid Diagram (Code)}
\begin{verbatim}
{Shaded}
{Highlighting}[]
graph LR
    subgraph "Phasor Diagram"
        O((O)) {-{-}{} E[Em]}
        O {-{-}{} A[]}
    end
    
    style E fill:\#f96,stroke:\#333,stroke{-width:3px}
    style A fill:\#bbf,stroke:\#333,stroke{-width:2px}
{Highlighting}
{Shaded}
\end{verbatim}
\end{center}


\vspace{-5pt}
\captionof{table}{Vector Parameters}
\vspace{-10pt}
\begin{longtable}[]{@{}llll@{}}
\toprule\noalign{}
Parameter & Symbol & Units & Description \\
\midrule\noalign{}
\endhead
\bottomrule\noalign{}
\endlastfoot
\textbf{Magnitude} & Em & Volts & Maximum EMF \\
\textbf{Angular velocity} & \omega & rad/s & Rotation speed \\
\textbf{Phase angle} & \phi & Degrees & Initial phase \\
\textbf{Frequency} & f = \omega/2\pi & Hz & Cycles per second \\
\end{longtable}

\textbf{Advantages:}

\begin{itemize}
\tightlist
\item
  \textbf{Visual representation}: Easy to understand phase relationships
\item
  \textbf{Mathematical simplification}: Complex calculations made easier
\end{itemize}

\end{solutionbox}
\begin{mnemonicbox}
``Vectors Visualize Voltage Variation'' (Phasor
representation benefits)

\end{mnemonicbox}
\subsection*{Question 4(b) [4 marks]}\label{q4b}

\textbf{Define following terms w.r.t Alternating current: RMS value,
Average value, Frequency, Time period}

\begin{solutionbox}


\vspace{-5pt}
\captionof{table}{AC Parameters Definition}
\vspace{-10pt}
\begin{longtable}[]{@{}
  >{\raggedright\arraybackslash}p{(\linewidth - 6\tabcolsep) * \real{0.1765}}
  >{\raggedright\arraybackslash}p{(\linewidth - 6\tabcolsep) * \real{0.3529}}
  >{\raggedright\arraybackslash}p{(\linewidth - 6\tabcolsep) * \real{0.2647}}
  >{\raggedright\arraybackslash}p{(\linewidth - 6\tabcolsep) * \real{0.2059}}@{}}
\toprule\noalign{}
\begin{minipage}[b]{\linewidth}\raggedright
Term
\end{minipage} & \begin{minipage}[b]{\linewidth}\raggedright
Definition
\end{minipage} & \begin{minipage}[b]{\linewidth}\raggedright
Formula
\end{minipage} & \begin{minipage}[b]{\linewidth}\raggedright
Units
\end{minipage} \\
\midrule\noalign{}
\endhead
\bottomrule\noalign{}
\endlastfoot
\textbf{RMS Value} & Effective value producing same heating & Im/\sqrt2 &
Amperes \\
\textbf{Average Value} & Mean value over half cycle & 2Im/\pi & Amperes \\
\textbf{Frequency} & Number of cycles per second & f = 1/T & Hz \\
\textbf{Time Period} & Time for one complete cycle & T = 1/f &
Seconds \\
\end{longtable}

\textbf{Mathematical Relations:}

\begin{itemize}
\tightlist
\item
  \textbf{Form Factor}: RMS/Average = \pi/2\sqrt2 = 1.11
\item
  \textbf{Peak Factor}: Peak/RMS = \sqrt2 = 1.414
\item
  \textbf{Angular frequency}: \omega = 2\pif
\end{itemize}

\textbf{Practical Values:}

\begin{itemize}
\tightlist
\item
  \textbf{RMS current}: Used for power calculations
\item
  \textbf{Average current}: Used for DC equivalent
\item
  \textbf{Frequency}: 50 Hz (India), 60 Hz (USA)
\end{itemize}

\end{solutionbox}
\begin{mnemonicbox}
``Really Mean Square, Average Frequency Time'' (Key
AC parameters)

\end{mnemonicbox}
\subsection*{Question 4(c) [7 marks]}\label{q4c}

\textbf{Derive equation for relation between line and phase voltage and
current in star connection}

\begin{solutionbox}

\textbf{Star Connection Diagram:}

\begin{center}
\textbf{Mermaid Diagram (Code)}
\begin{verbatim}
{Shaded}
{Highlighting}[]
graph TD
    R[R Phase] {-{-}{} N[Neutral N]}
    Y[Y Phase] {-{-}{} N}
    B[B Phase] {-{-}{} N}
    
    R {-{-}{} LR[Line R]}
    Y {-{-}{} LY[Line Y]  }
    B {-{-}{} LB[Line B]}
    
    style N fill:\#f96,stroke:\#333,stroke{-width:3px}
{Highlighting}
{Shaded}
\end{verbatim}
\end{center}

\textbf{Voltage Relations:}

\textbf{Phase Voltages:} VR, VY, VB (with respect to neutral)
\textbf{Line Voltages:} VRY, VYB, VBR (between lines)

\textbf{Phasor Analysis:}

\begin{verbatim}
VRY = VR - VY
\end{verbatim}

\textbf{For balanced system:}

\begin{itemize}
\tightlist
\item
  Phase voltages are equal in magnitude: VR = VY = VB = Vph
\item
  Phase difference = 120^\circ
\end{itemize}

\textbf{Vector Addition:} Using phasor diagram and cosine rule:

\begin{verbatim}
VL = \sqrt(Vph^{2} + Vph^{2} - 2Vph·Vph·cos(120^\circ))
VL = \sqrt(2Vph^{2} + Vph^{2}) = \sqrt3 \times Vph
\end{verbatim}

\textbf{Final Relations:}


\vspace{-5pt}
\captionof{table}{Star Connection Relations}
\vspace{-10pt}
\begin{longtable}[]{@{}ll@{}}
\toprule\noalign{}
Parameter & Relationship \\
\midrule\noalign{}
\endhead
\bottomrule\noalign{}
\endlastfoot
\textbf{Line Voltage} & VL = \sqrt3 \times Vph \\
\textbf{Line Current} & IL = Iph \\
\textbf{Power} & P = \sqrt3 \times VL \times IL \times cos\phi \\
\end{longtable}

\textbf{Current Relations:} In star connection, line current equals
phase current:

\begin{verbatim}
IL = Iph
\end{verbatim}

\end{solutionbox}
\begin{mnemonicbox}
``Star Scales Voltage, Same current'' (\sqrt3 factor for
voltage, current unchanged)

\end{mnemonicbox}
\subsection*{Question 4(a OR) [3
marks]}\label{question-4a-or-3-marks}

\textbf{Explain vector representation of alternating current.}

\begin{solutionbox}

\textbf{Vector Representation:} AC current represented as rotating
phasor with magnitude and phase angle.

\textbf{Mathematical Expression:}

\begin{verbatim}
i = Im sin(\omegat + \phi)
\end{verbatim}

\textbf{Phasor Diagram:}

\begin{verbatim}
     Im
      ↗
     /|
    / |
   /  |  
  O{-{-}{-}+{-}{-}{-} Reference}
      
\end{verbatim}


\vspace{-5pt}
\captionof{table}{Current Vector Elements}
\vspace{-10pt}
\begin{longtable}[]{@{}lll@{}}
\toprule\noalign{}
Element & Symbol & Description \\
\midrule\noalign{}
\endhead
\bottomrule\noalign{}
\endlastfoot
\textbf{Magnitude} & Im & Peak current value \\
\textbf{Phase} & \phi & Leading/lagging angle \\
\textbf{Angular velocity} & \omega & Rotation speed \\
\textbf{RMS value} & I = Im/\sqrt2 & Effective current \\
\end{longtable}

\textbf{Applications:}

\begin{itemize}
\tightlist
\item
  \textbf{Circuit analysis}: Phase relationships between voltage and
  current
\item
  \textbf{Power calculations}: Real and reactive power components
\end{itemize}

\end{solutionbox}
\begin{mnemonicbox}
``Current Circles Continuously'' (Rotating phasor
concept)

\end{mnemonicbox}
\subsection*{Question 4(b OR) [4
marks]}\label{question-4b-or-4-marks}

\textbf{Define following terms w.r.t Alternating current: Form factor,
Peak factor, Angular velocity, Amplitude}

\begin{solutionbox}


\vspace{-5pt}
\captionof{table}{AC Current Parameters}
\vspace{-10pt}
\begin{longtable}[]{@{}
  >{\raggedright\arraybackslash}p{(\linewidth - 6\tabcolsep) * \real{0.1429}}
  >{\raggedright\arraybackslash}p{(\linewidth - 6\tabcolsep) * \real{0.2857}}
  >{\raggedright\arraybackslash}p{(\linewidth - 6\tabcolsep) * \real{0.2143}}
  >{\raggedright\arraybackslash}p{(\linewidth - 6\tabcolsep) * \real{0.3571}}@{}}
\toprule\noalign{}
\begin{minipage}[b]{\linewidth}\raggedright
Term
\end{minipage} & \begin{minipage}[b]{\linewidth}\raggedright
Definition
\end{minipage} & \begin{minipage}[b]{\linewidth}\raggedright
Formula
\end{minipage} & \begin{minipage}[b]{\linewidth}\raggedright
Typical Value
\end{minipage} \\
\midrule\noalign{}
\endhead
\bottomrule\noalign{}
\endlastfoot
\textbf{Form Factor} & RMS/Average value ratio & Irms/Iavg & 1.11 (sine
wave) \\
\textbf{Peak Factor} & Peak/RMS value ratio & Im/Irms & 1.414 (sine
wave) \\
\textbf{Angular Velocity} & Rate of phase change & \omega = 2\pif & 314 rad/s
(50Hz) \\
\textbf{Amplitude} & Maximum instantaneous value & Im & Peak current \\
\end{longtable}

\textbf{Mathematical Relations:}

\begin{itemize}
\tightlist
\item
  \textbf{Form factor}: Indicates waveform shape
\item
  \textbf{Peak factor}: Shows crest factor
\item
  \textbf{Angular velocity}: Links frequency and phase
\item
  \textbf{Amplitude}: Determines RMS and average values
\end{itemize}

\textbf{Practical Significance:}

\begin{itemize}
\tightlist
\item
  \textbf{Design considerations}: Peak factors for insulation
\item
  \textbf{Waveform analysis}: Form factors for distortion
\item
  \textbf{Synchronization}: Angular velocity for timing
\end{itemize}

\end{solutionbox}
\begin{mnemonicbox}
``Form Peak Angular Amplitude'' (Four key factors)

\end{mnemonicbox}
\subsection*{Question 4(c OR) [7
marks]}\label{question-4c-or-7-marks}

\textbf{Derive equation for relation between line and phase voltage and
current in delta connection}

\begin{solutionbox}

\textbf{Delta Connection Diagram:}

\begin{center}
\textbf{Mermaid Diagram (Code)}
\begin{verbatim}
{Shaded}
{Highlighting}[]
graph LR
    subgraph "Delta Connection"
        A[A] {-{-}{} B[B]}
        B {-{-}{} C[C]}
        C {-{-}{} A}
    end
    
    A {-{-}{} IA[IA]}
    B {-{-}{} IB[IB]}
    C {-{-}{} IC[IC]}
    
    style A fill:\#f96,stroke:\#333,stroke{-width:2px}
    style B fill:\#bbf,stroke:\#333,stroke{-width:2px}
    style C fill:\#bfb,stroke:\#333,stroke{-width:2px}
{Highlighting}
{Shaded}
\end{verbatim}
\end{center}

\textbf{Voltage Relations:} In delta connection, line voltage equals
phase voltage:

\begin{verbatim}
VL = Vph
\end{verbatim}

\textbf{Current Analysis:} Each line current is vector sum of two phase
currents.

\textbf{For Line Current IA:}

\begin{verbatim}
IA = IAB - ICA
\end{verbatim}

\textbf{Phasor Analysis:} For balanced system with phase currents equal
in magnitude:

\begin{itemize}
\tightlist
\item
  IAB = ICA = ICB = Iph
\item
  Phase difference between currents = 120^\circ
\end{itemize}

\textbf{Vector Subtraction:}

\begin{verbatim}
IA = IAB - ICA = IAB - (-ICA)
\end{verbatim}

Using phasor diagram:

\begin{verbatim}
IL = \sqrt(Iph^{2} + Iph^{2} - 2Iph·Iph·cos(60^\circ))
IL = \sqrt(2Iph^{2} - Iph^{2}) = \sqrt3 \times Iph
\end{verbatim}

\textbf{Final Relations:}


\vspace{-5pt}
\captionof{table}{Delta Connection Relations}
\vspace{-10pt}
\begin{longtable}[]{@{}ll@{}}
\toprule\noalign{}
Parameter & Relationship \\
\midrule\noalign{}
\endhead
\bottomrule\noalign{}
\endlastfoot
\textbf{Line Voltage} & VL = Vph \\
\textbf{Line Current} & IL = \sqrt3 \times Iph \\
\textbf{Power} & P = \sqrt3 \times VL \times IL \times cos\phi \\
\end{longtable}

\end{solutionbox}
\begin{mnemonicbox}
``Delta Doubles current, Same voltage'' (\sqrt3 factor
for current, voltage unchanged)

\end{mnemonicbox}
\subsection*{Question 5(a) [3 marks]}\label{q5a}

\textbf{Explain AC through pure resistor with necessary circuit and
waveform.}

\begin{solutionbox}

\textbf{Circuit Diagram:}

\begin{center}
\textbf{Mermaid Diagram (Code)}
\begin{verbatim}
{Shaded}
{Highlighting}[]
graph LR
    AC[AC Source] {-{-}{} R[Resistor R]}
    R {-{-}{} AC}
    
    style AC fill:\#f96,stroke:\#333,stroke{-width:2px}
    style R fill:\#bbf,stroke:\#333,stroke{-width:2px}
{Highlighting}
{Shaded}
\end{verbatim}
\end{center}

\textbf{Waveform:}

\begin{verbatim}
    V,I
     ↑
     |    /{      /}
     |   /  {    /  }
     |  /    {  /    }
  {-{-}{-}+{-}{-}{-}{-}{-}{-}{-}{-}/{-}{-}{-}{-}{-}{-}/{-} t}
     |        /{      /}
     |       /  {    /}
     |      /    {  /}
     
  V and I in phase
\end{verbatim}


\vspace{-5pt}
\captionof{table}{AC through Resistor}
\vspace{-10pt}
\begin{longtable}[]{@{}lll@{}}
\toprule\noalign{}
Parameter & Relationship & Phase \\
\midrule\noalign{}
\endhead
\bottomrule\noalign{}
\endlastfoot
\textbf{Ohm's Law} & V = IR & Same phase \\
\textbf{Power} & P = VI = I^{2}R & Always positive \\
\textbf{Impedance} & Z = R & Purely resistive \\
\end{longtable}

\textbf{Characteristics:}

\begin{itemize}
\tightlist
\item
  \textbf{Current and voltage in phase}: No phase difference
\item
  \textbf{Power consumption}: Continuous power dissipation
\item
  \textbf{Resistance unchanged}: Same as DC value
\end{itemize}

\end{solutionbox}
\begin{mnemonicbox}
``Resistor Refuses phase Shift'' (No phase
difference)

\end{mnemonicbox}
\subsection*{Question 5(b) [4 marks]}\label{q5b}

\textbf{Define following terms w.r.t Alternating current: Impedance,
Phase angle, Power factor, Reactive power}

\begin{solutionbox}


\vspace{-5pt}
\captionof{table}{AC Circuit Parameters}
\vspace{-10pt}
\begin{longtable}[]{@{}
  >{\raggedright\arraybackslash}p{(\linewidth - 6\tabcolsep) * \real{0.1765}}
  >{\raggedright\arraybackslash}p{(\linewidth - 6\tabcolsep) * \real{0.3529}}
  >{\raggedright\arraybackslash}p{(\linewidth - 6\tabcolsep) * \real{0.2647}}
  >{\raggedright\arraybackslash}p{(\linewidth - 6\tabcolsep) * \real{0.2059}}@{}}
\toprule\noalign{}
\begin{minipage}[b]{\linewidth}\raggedright
Term
\end{minipage} & \begin{minipage}[b]{\linewidth}\raggedright
Definition
\end{minipage} & \begin{minipage}[b]{\linewidth}\raggedright
Formula
\end{minipage} & \begin{minipage}[b]{\linewidth}\raggedright
Units
\end{minipage} \\
\midrule\noalign{}
\endhead
\bottomrule\noalign{}
\endlastfoot
\textbf{Impedance} & Total opposition to AC current & Z = \sqrt(R^{2} + X^{2}) &
Ohms \\
\textbf{Phase Angle} & Angle between V and I & \phi = tan^{-}^{1}(X/R) &
Degrees \\
\textbf{Power Factor} & Cosine of phase angle & PF = cos\phi = R/Z & - \\
\textbf{Reactive Power} & Power in reactive components & Q = VI sin\phi &
VAR \\
\end{longtable}

\textbf{Power Relations:}

\begin{itemize}
\tightlist
\item
  \textbf{Active Power}: P = VI cos\phi (Watts)
\item
  \textbf{Reactive Power}: Q = VI sin\phi (VAR)
\item
  \textbf{Apparent Power}: S = VI (VA)
\end{itemize}

\textbf{Power Triangle:}

\begin{verbatim}
S^{2} = P^{2} + Q^{2}
\end{verbatim}

\textbf{Practical Significance:}

\begin{itemize}
\tightlist
\item
  \textbf{High power factor}: Efficient power utilization
\item
  \textbf{Low power factor}: Higher current for same power
\item
  \textbf{Reactive power}: No net energy transfer
\end{itemize}

\end{solutionbox}
\begin{mnemonicbox}
``Impedance Phase Power Quadrature'' (Four key AC
parameters)

\end{mnemonicbox}
\subsection*{Question 5(c) [7 marks]}\label{q5c}

\textbf{Enlist different protective device and explain construction and
working of any one protective device.}

\begin{solutionbox}


\vspace{-5pt}
\captionof{table}{Protective Devices}
\vspace{-10pt}
\begin{longtable}[]{@{}lll@{}}
\toprule\noalign{}
Device & Protection Against & Application \\
\midrule\noalign{}
\endhead
\bottomrule\noalign{}
\endlastfoot
\textbf{Fuse} & Overcurrent & Low/Medium voltage \\
\textbf{MCB} & Overload, Short circuit & Domestic/Commercial \\
\textbf{ELCB} & Earth leakage & Safety protection \\
\textbf{Relay} & Various faults & Industrial systems \\
\textbf{Surge arrester} & Overvoltage & Transmission lines \\
\end{longtable}

\textbf{MCB (Miniature Circuit Breaker) - Detailed Explanation:}

\textbf{Construction:}

\begin{center}
\textbf{Mermaid Diagram (Code)}
\begin{verbatim}
{Shaded}
{Highlighting}[]
graph TD
    subgraph "MCB Construction"
      direction LR
        C[Contacts] {-{-}{} A[Arc Chamber]}
        A {-{-}{} B[Bimetallic Strip]}
        B {-{-}{} M[Magnetic Coil]}
        M {-{-}{} T[Trip Mechanism]}
    end
    
    style C fill:\#f96,stroke:\#333,stroke{-width:2px}
    style B fill:\#bbf,stroke:\#333,stroke{-width:2px}
    style M fill:\#bfb,stroke:\#333,stroke{-width:2px}
{Highlighting}
{Shaded}
\end{verbatim}
\end{center}

\textbf{Components:}

\begin{itemize}
\tightlist
\item
  \textbf{Fixed and moving contacts}: Current carrying parts
\item
  \textbf{Bimetallic strip}: Thermal protection
\item
  \textbf{Electromagnetic coil}: Magnetic protection
\item
  \textbf{Arc quenching chamber}: Arc extinction
\item
  \textbf{Operating mechanism}: Manual/automatic operation
\end{itemize}

\textbf{Working Principle:}

\textbf{Overload Protection:}

\begin{itemize}
\tightlist
\item
  Current heats bimetallic strip
\item
  Strip bends and trips mechanism
\item
  Time-delay characteristic protects against temporary overloads
\end{itemize}

\textbf{Short Circuit Protection:}

\begin{itemize}
\tightlist
\item
  High fault current creates strong magnetic field
\item
  Electromagnetic force operates trip mechanism
\item
  Instantaneous operation for safety
\end{itemize}

\textbf{Advantages:}

\begin{itemize}
\tightlist
\item
  \textbf{Reusable}: Reset after fault clearance
\item
  \textbf{Reliable operation}: Dual protection mechanism
\item
  \textbf{Easy maintenance}: Accessible contacts
\end{itemize}

\end{solutionbox}
\begin{mnemonicbox}
``MCB Magnetically Controls Both'' (Thermal and
magnetic protection)

\end{mnemonicbox}
\subsection*{Question 5(a OR) [3
marks]}\label{question-5a-or-3-marks}

\textbf{Derive equation of AC current passing through pure inductor}

\begin{solutionbox}

\textbf{Given:} Pure inductor with inductance L, applied voltage v = Vm
sin(\omegat)

\textbf{Voltage-Current Relationship:}

\begin{verbatim}
v = L \times (di/dt)
\end{verbatim}

\textbf{Substituting applied voltage:}

\begin{verbatim}
Vm sin(\omegat) = L \times (di/dt)
\end{verbatim}

\textbf{Integration:}

\begin{verbatim}
di = (Vm/L) sin(\omegat) dt
i = -(Vm/\omegaL) cos(\omegat) + C
\end{verbatim}

\textbf{At steady state, C = 0:}

\begin{verbatim}
i = -(Vm/\omegaL) cos(\omegat)
i = (Vm/\omegaL) sin(\omegat - 90^\circ)
\end{verbatim}


\vspace{-5pt}
\captionof{table}{Pure Inductor Characteristics}
\vspace{-10pt}
\begin{longtable}[]{@{}lll@{}}
\toprule\noalign{}
Parameter & Value & Phase Relationship \\
\midrule\noalign{}
\endhead
\bottomrule\noalign{}
\endlastfoot
\textbf{Current amplitude} & Im = Vm/\omegaL & Current lags voltage by 90^\circ \\
\textbf{Inductive reactance} & XL = \omegaL = 2\pifL & Frequency dependent \\
\textbf{Power} & P = 0 (average) & No net power consumption \\
\end{longtable}

\end{solutionbox}
\begin{mnemonicbox}
``Inductor Impedes, Current lags'' (XL opposes
current, 90^\circ lag)

\end{mnemonicbox}
\subsection*{Question 5(b OR) [4
marks]}\label{question-5b-or-4-marks}

\textbf{Explain concept of power and power triangle in AC circuit.}

\begin{solutionbox}

\textbf{Types of Power:}


\vspace{-5pt}
\captionof{table}{AC Power Components}
\vspace{-10pt}
\begin{longtable}[]{@{}lllll@{}}
\toprule\noalign{}
Power Type & Symbol & Formula & Units & Description \\
\midrule\noalign{}
\endhead
\bottomrule\noalign{}
\endlastfoot
\textbf{Active Power} & P & VI cos\phi & Watts & Useful power \\
\textbf{Reactive Power} & Q & VI sin\phi & VAR & Circulating power \\
\textbf{Apparent Power} & S & VI & VA & Total power \\
\end{longtable}

\textbf{Power Triangle:}

\begin{center}
\textbf{Mermaid Diagram (Code)}
\begin{verbatim}
{Shaded}
{Highlighting}[]
graph LR
    O((O)) {-{-}{} P[P = VI cos]}
    O {-{-}{} Q[Q = VI sin]}
    P {-{-}{} S[S = VI]}
    
    style P fill:\#f96,stroke:\#333,stroke{-width:2px}
    style Q fill:\#bbf,stroke:\#333,stroke{-width:2px}
    style S fill:\#bfb,stroke:\#333,stroke{-width:2px}
{Highlighting}
{Shaded}
\end{verbatim}
\end{center}

\textbf{Mathematical Relations:}

\begin{verbatim}
S^{2} = P^{2} + Q^{2}
Power Factor = P/S = cos\phi
\end{verbatim}

\textbf{Significance:}

\begin{itemize}
\tightlist
\item
  \textbf{Active power}: Does useful work (heating, mechanical)
\item
  \textbf{Reactive power}: Maintains magnetic/electric fields
\item
  \textbf{Power factor}: Efficiency indicator
\end{itemize}

\end{solutionbox}
\begin{mnemonicbox}
``Power Triangle: Please Qualify Students'' (P, Q, S
components)

\end{mnemonicbox}
\subsection*{Question 5(c OR) [7
marks]}\label{question-5c-or-7-marks}

\textbf{Explain wiring of lamp control from one place and staircase
type.}

\begin{solutionbox}

\textbf{1. Lamp Control from One Place:}

\textbf{Circuit Diagram:}

\begin{verbatim}
Live {-{-}{-}{-}[S]{-}{-}{-}{-}[Lamp]{-}{-}{-}{-}+}
                          |
Neutral {-{-}{-}{-}{-}{-}{-}{-}{-}{-}{-}{-}{-}{-}{-}{-}{-}{-}+}

S = Single Pole Single Throw Switch
\end{verbatim}

\textbf{Components:}

\begin{itemize}
\tightlist
\item
  \textbf{SPST Switch}: Single pole, single throw
\item
  \textbf{Live wire control}: Switch in live wire for safety
\item
  \textbf{Simple on/off}: Basic control mechanism
\end{itemize}

\textbf{2. Staircase Wiring (Two-Way Control):}

\textbf{Circuit Diagram:}

\begin{verbatim}
Live {-{-}{-}{-}[S1]{-}{-}{-}{-}+{-}{-}{-}{-}[S2]{-}{-}{-}{-}[Lamp]{-}{-}{-}{-}+}
            {    |    /                 |}
             {   |   /                  |}
              {  |  /                   |}
               { | /                    |}
                {|/                     |}
Neutral {-{-}{-}{-}{-}{-}{-}{-}{-}{-}{-}{-}{-}{-}{-}{-}{-}{-}{-}{-}{-}{-}{-}{-}{-}{-}{-}{-}{-}{-}{-}{-}+}

S1, S2 = Two{-way switches (SPDT)}
\end{verbatim}


\vspace{-5pt}
\captionof{table}{Switch Positions for Staircase Control}
\vspace{-10pt}
\begin{longtable}[]{@{}lll@{}}
\toprule\noalign{}
S1 Position & S2 Position & Lamp Status \\
\midrule\noalign{}
\endhead
\bottomrule\noalign{}
\endlastfoot
\textbf{Up} & Up & ON \\
\textbf{Up} & Down & OFF \\
\textbf{Down} & Up & OFF \\
\textbf{Down} & Down & ON \\
\end{longtable}

\textbf{Working Principle:}

\begin{itemize}
\tightlist
\item
  \textbf{Two-way switches}: SPDT (Single Pole Double Throw)
\item
  \textbf{Common terminal}: Connected to live and lamp
\item
  \textbf{Strappers}: Link switches together
\item
  \textbf{Toggle action}: Either switch can control lamp
\end{itemize}

\textbf{Applications:}

\begin{itemize}
\tightlist
\item
  \textbf{Staircase lighting}: Control from top and bottom
\item
  \textbf{Long corridors}: Control from both ends
\item
  \textbf{Bedroom lighting}: Control from bed and door
\end{itemize}

\textbf{Advantages:}

\begin{itemize}
\tightlist
\item
  \textbf{Convenience}: Control from multiple locations
\item
  \textbf{Energy saving}: Easy switching reduces wastage
\item
  \textbf{Safety}: No need to walk in dark
\end{itemize}

\textbf{Installation Points:}

\begin{itemize}
\tightlist
\item
  \textbf{Proper earthing}: All metal parts earthed
\item
  \textbf{Cable rating}: Adequate current capacity
\item
  \textbf{Switch height}: Standard 4 feet from floor
\end{itemize}

\end{solutionbox}
\begin{mnemonicbox}
``Two-way Toggles, Two places'' (Two switches, two
locations)

\end{mnemonicbox}

\end{document}
