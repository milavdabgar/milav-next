\documentclass{article}

% content/resources/templates/preamble.tex
\usepackage[margin=0.6in]{geometry}
\author{Milav Dabgar}
\usepackage{amsmath,amssymb,amsthm}
\usepackage{booktabs}
\usepackage{multirow}
\usepackage{xcolor}
\usepackage{tcolorbox}
\tcbuselibrary{breakable,skins}
\usepackage[colorlinks=true,linkcolor=blue]{hyperref}
\usepackage{titlesec}
\usepackage{enumitem}
\usepackage{tikz}
\usepackage{pgfplots}
\usepackage{circuitikz}
\usepackage[version=4]{mhchem}
\usepackage{longtable}
\usepackage{array}
\usepackage{float}
\usepackage{caption}
\usepackage{listings}

\lstset{
  basicstyle=\small\ttfamily,
  breaklines=true,
  breakatwhitespace=false,
  postbreak=\mbox{\textcolor{red}{$\hookrightarrow$}\space},
  float=false,
  numbers=left,
  numberstyle=\tiny\color{gray},
  numbersep=10pt,
  xleftmargin=2em,
  keywordstyle=\color{blue},
  commentstyle=\color{green!60!black},
  stringstyle=\color{purple},
  backgroundcolor=\color{gray!5},
  showstringspaces=false,
  tabsize=2,
  captionpos=b,
  keepspaces=true,
  columns=flexible
}

\pgfplotsset{compat=1.18}
\usetikzlibrary{shapes,arrows,positioning,calc,patterns,decorations.pathmorphing,decorations.markings,arrows.meta}

% Color scheme
\definecolor{headcolor}{RGB}{0,102,204}
\definecolor{keycolor}{RGB}{220,20,60}
\definecolor{solutioncolor}{RGB}{34,139,34}
\definecolor{mnemoniccolor}{RGB}{148,0,211}
\definecolor{codecolor}{RGB}{0,0,100}

% Spacing
\setlength{\parskip}{3pt}
\setlist[itemize]{nosep}
\setlist[enumerate]{nosep}

% Title formatting
\titleformat{\section}{\Large\bfseries\color{headcolor}}{\thesection}{1em}{}
\titleformat{\subsection}{\large\bfseries\color{headcolor}}{\thesubsection}{1em}{}

% Pandoc tightlist compatibility
\providecommand{\tightlist}{%
  \setlength{\itemsep}{0pt}\setlength{\parskip}{0pt}}

% Pandoc longtable compatibility
\newcounter{none}
\def\thenone{}


% content/resources/templates/gujarati-boxes.tex
\usepackage{fontspec}
\usepackage{polyglossia}

% Set Gujarati as main language (document is primarily in Gujarati)
% Note: gloss-gujarati.ldf doesn't exist in polyglossia, but it will use hyphenation patterns
\setdefaultlanguage{gujarati}
\setotherlanguage{english}

% Configure Gujarati font properly
% Use Language=Default to prevent polyglossia from trying to add language-specific features
% that don't exist for Gujarati, which causes "empty feature" warnings
\newfontfamily\gujaratifont[Script=Gujarati,AutoFakeBold=2.5,AutoFakeSlant=0.3]{Noto Sans Gujarati}
\setmainfont[Script=Gujarati,AutoFakeBold=2.5,AutoFakeSlant=0.3]{Noto Sans Gujarati}
% Use Noto Sans Gujarati for monospace to support Gujarati in text
\setmonofont[Scale=0.9]{Noto Sans Gujarati}

% Configure English to use the same font
\newfontfamily\englishfont[Script=Gujarati,AutoFakeBold=2.5,AutoFakeSlant=0.3]{Noto Sans Gujarati}

% Translations for polyglossia
\gappto\captionsgujarati{
  \renewcommand{\tablename}{કોષ્ટક}
  \renewcommand{\figurename}{આકૃતિ}
}

% Helper for TikZ nodes to ensure Gujarati font
\newcommand{\gu}[1]{{\gujaratifont #1}}

% Custom environments
\newtcolorbox{solutionbox}{
    breakable,
    enhanced,
    colback=solutioncolor!5!white,
    colframe=solutioncolor!75!black,
    fonttitle=\bfseries,
    title=જવાબ
}

\newtcolorbox{solutionboxnobreak}{
 colback=solutioncolor!5!white,
 colframe=solutioncolor!75!black,
 fonttitle=\bfseries,
 title=જવાબ
}

\newtcolorbox{keyformula}{
 breakable,
 enhanced,
 colback=keycolor!5!white,
 colframe=keycolor!75!black,
 fonttitle=\bfseries,
 title=રાસાયણિક સમીકરણ/સૂત્ર
}

\newtcolorbox{mnemonicbox}{
 breakable,
 enhanced,
 colback=mnemoniccolor!5!white,
 colframe=mnemoniccolor!75!black,
 fonttitle=\bfseries,
 title=મેમરી ટ્રીક
}


% Custom commands for GTU solutions
% This file defines semantic commands for consistent formatting

% Question command with automatic formatting
\newcommand{\question}[2]{%
  \section*{Question #1}%
  \textbf{#2}%
}

% OR question variant
\newcommand{\questionor}[2]{%
  \section*{Question #1 OR}%
  \textbf{#2}%
}

% Proper table environment with caption
\newenvironment{answertable}[1]{%
  \begin{table}[htbp]
  \centering
  \caption{#1}
}{%
  \end{table}
}

% Proper figure environment for diagrams
\newenvironment{answerdiagram}[1]{%
  \begin{figure}[htbp]
  \centering
  \caption{#1}
}{%
  \end{figure}
}

% Semantic markup for key terms
\newcommand{\keyword}[1]{\textbf{#1}}
\newcommand{\code}[1]{\texttt{#1}}
\newcommand{\classname}[1]{\texttt{#1}}
\newcommand{\methodname}[1]{\texttt{#1}}

% Proper quotation marks
\newcommand{\mnemonic}[1]{``#1''}


\title{ઇલેક્ટ્રિકલ એન્જિનિયરિંગના મૂળભૂત સિદ્ધાંતો (DI01000101) - શિયાળુ 2024 હલ}
\date{January 13, 2025}

\begin{document}
\maketitle

\questionmarks{1(a)}{3}{ઓહમના નિયમને તેની મર્યાદા અને ઉપયોગિતા સાથે સમજાવો.}

\begin{solutionbox}
\textbf{જવાબ:}

\textbf{ઓહમના નિયમનો સારાંશ:}
\begin{center}
\captionof{table}{ઓહમના નિયમનો સારાંશ}
\begin{tabulary}{\linewidth}{|L|L|}
\hline
\textbf{પાસું} & \textbf{વર્ણન} \\ \hline
\textbf{વિધાન} & વાહક દ્વારા પસાર થતો કરંટ વોલ્ટેજના સીધા પ્રમાણમાં હોય છે \\ \hline
\textbf{સૂત્ર} & $V = I \times R$ \\ \hline
\textbf{એકમો} & V (વોલ્ટ), I (એમ્પિયર), R (ઓહ્મ) \\ \hline
\end{tabulary}
\end{center}

\textbf{મર્યાદાઓ:}
\begin{itemize}
    \item \keyword{તાપમાન આધારિત}: તાપમાન સાથે અવરોધ બદલાય છે
    \item \keyword{બિન-રેખીય પદાર્થો}: સેમિકન્ડક્ટર, ડાયોડ પર લાગુ નહીં
    \item \keyword{AC સર્કિટ}: રિએક્ટિવ કોમ્પોનન્ટ્સ માટે બદલેલા સ્વરૂપની જરૂર
\end{itemize}

\textbf{ઉપયોગિતા:}
\begin{itemize}
    \item \keyword{સર્કિટ વિશ્લેષણ}: અજાણા વોલ્ટેજ, કરંટ અથવા અવરોધની ગણતરી
    \item \keyword{પાવર ગણતરી}: $P = V^2/R$, $P = I^2R$
\end{itemize}
\end{solutionbox}

\begin{mnemonicbox}
\mnemonic{"વોલ્ટેજ ઇઝ રિયલી ઇમ્પોર્ટન્ટ" (V = I × R)}
\end{mnemonicbox}

\questionmarks{1(b)}{4}{ફેરાડેના ઇલેક્ટ્રોમેગ્નેટિક ઇન્ડક્શનના નિયમને જરૂરી આકૃતિ સાથે સમજાવો.}

\begin{solutionbox}
\textbf{જવાબ:}

\textbf{ફેરાડેના નિયમો:}
\begin{itemize}
    \item \keyword{પ્રથમ નિયમ}: જ્યારે વાહક દ્વારા મેગ્નેટિક ફ્લક્સ બદલાય ત્યારે EMF પેદા થાય છે
    \item \keyword{બીજો નિયમ}: EMF નું મેગ્નિટ્યૂડ ફ્લક્સ ચેન્જના દર સમાન હોય છે
\end{itemize}

\textbf{ગાણિતિક અભિવ્યક્તિ:}
\[ e = -N \times \frac{d\Phi}{dt} \]

\textbf{આકૃતિ:}
\begin{center}
\begin{tikzpicture}
    % Coil
    \draw[thick] (0,0) ellipse (0.5 and 1.5);
    \draw[thick] (0,1.5) -- (2,1.5);
    \draw[thick] (0,-1.5) -- (2,-1.5);
    \draw[thick] (2,1.5) to[R, l=$R$] (2,-1.5);
    \node at (-1, 0) {કોઇલ (N વળાંકો)};

    % Magnet
    \draw[thick, fill=blue!20] (-3, -0.5) rectangle (-1.5, 0.5);
    \node at (-2.6, 0) {S};
    \draw[thick, fill=red!20] (-1.5, -0.5) rectangle (0, 0.5);
    \node at (-0.4, 0) {N};
    
    % Motion arrow
    \draw[->, very thick] (-1.5, 0.8) -- (0.5, 0.8) node[midway, above] {$v$ (ગતિ)};
    
    % Flux lines
    \draw[dashed, ->] (0,0) -- (1,0);
    \node at (0.5, -0.2) {$\Phi$};
\end{tikzpicture}
\captionof{figure}{ફેરાડેના નિયમનું ચિત્રણ}
\end{center}

\textbf{ઉપયોગિતા:}
\begin{itemize}
    \item \keyword{ટ્રાન્સફોર્મર}: મ્યુચ્યુઅલ ઇન્ડક્શન સિદ્ધાંત
    \item \keyword{જનરેટર}: મિકેનિકલથી ઇલેક્ટ્રિકલ એનર્જી કન્વર્ઝન
    \item \keyword{ઇન્ડક્ટર}: સેલ્ફ-ઇન્ડ્યૂસ્ડ EMF કરંટ ચેન્જનો વિરોધ કરે છે
\end{itemize}
\end{solutionbox}

\begin{mnemonicbox}
\mnemonic{"ફ્લક્સ ચેન્જ જનરેટ્સ EMF" (dΦ/dt = EMF)}
\end{mnemonicbox}

\questionmarks{1(c)}{7}{કિર્ચહોફના વોલ્ટેજના નિયમ અને કિર્ચહોફના કરંટના નિયમને જરૂરી આકૃતિ સાથે સમજાવો.}

\begin{solutionbox}
\textbf{જવાબ:}

\textbf{કિર્ચહોફના નિયમોની તુલના:}
\begin{center}
\captionof{table}{કિર્ચહોફના નિયમોની તુલના}
\begin{tabulary}{\linewidth}{|L|L|L|L|}
\hline
\textbf{નિયમ} & \textbf{વિધાન} & \textbf{ગાણિતિક સ્વરૂપ} & \textbf{ઉપયોગ} \\ \hline
\textbf{KVL} & બંધ લૂપમાં વોલ્ટેજનો સરવાળો = 0 & $\Sigma V = 0$ & સિરીઝ સર્કિટ \\ \hline
\textbf{KCL} & નોડ પર કરંટનો સરવાળો = 0 & $\Sigma I = 0$ & પેરેલલ સર્કિટ \\ \hline
\end{tabulary}
\end{center}

\textbf{KVL આકૃતિ:}
\begin{center}
\begin{tikzpicture}
    \draw (0,0) to[battery1, l=$V_1$] (0,3)
    to[R, l=$R_1$] (3,3)
    to[battery1, l=$V_2$, invert] (3,0)
    to[R, l=$R_2$] (0,0);
    \node at (1.5, 1.5) {લૂપ};
    \draw[->] (1.2, 1.2) arc (-45:225:0.5);
\end{tikzpicture}
\captionof{figure}{KVL બંધ લૂપ}
\end{center}

\textbf{KCL આકૃતિ:}
\begin{center}
\begin{tikzpicture}
    \node[circle, fill, inner sep=2pt, label=above:નોડ] (N) at (0,0) {};
    \draw[<-] (N) -- (180:2) node[left] {$I_1$};
    \draw[<-] (N) -- (90:2) node[above] {$I_2$};
    \draw[->] (N) -- (0:2) node[right] {$I_3$};
    \draw[->] (N) -- (-90:2) node[below] {$I_4$};
\end{tikzpicture}
\captionof{figure}{KCL નોડ}
\end{center}

\textbf{મુખ્ય મુદ્દાઓ:}
\begin{itemize}
    \item \keyword{KVL}: બીજગણિતીય સરવાળો વોલ્ટેજ પોલેરિટી ધ્યાનમાં રાખે છે
    \item \keyword{KCL}: કરંટની દિશાઓ ધ્યાનમાં રાખે છે (આવતો વિ જતો)
    \item \keyword{ઉપયોગિતા}: સર્કિટ વિશ્લેષણ, અજાણા મૂલ્યો શોધવા
\end{itemize}
\end{solutionbox}

\begin{mnemonicbox}
\mnemonic{"વોલ્ટેજ લૂપ્સ, કરંટ નોડ્સ" (KVL લૂપ માટે, KCL નોડ માટે)}
\end{mnemonicbox}

\questionmarks{1(c અથવા)}{7}{સ્ટેટિકલી ઇન્ડ્યૂસ્ડ EMF અને ડાયનેમિકલી ઇન્ડ્યૂસ્ડ EMF વચ્ચેનો તફાવત સમજાવો.}

\begin{solutionbox}
\textbf{જવાબ:}

\textbf{સ્ટેટિક વિ ડાયનેમિક EMF:}
\begin{center}
\captionof{table}{સ્ટેટિક વિ ડાયનેમિક EMF}
\begin{tabulary}{\linewidth}{|L|L|L|}
\hline
\textbf{પેરામીટર} & \textbf{સ્ટેટિકલી ઇન્ડ્યૂસ્ડ EMF} & \textbf{ડાયનેમિકલી ઇન્ડ્યૂસ્ડ EMF} \\ \hline
\textbf{કારણ} & બદલાતું મેગ્નેટિક ફીલ્ડ & વાહક અને ફીલ્ડ વચ્ચે સંબંધિત ગતિ \\ \hline
\textbf{ફીલ્ડ} & સમય-બદલાતું, વાહક સ્થિર & સ્થિર ફીલ્ડ, વાહક ગતિશીલ \\ \hline
\textbf{ઉદાહરણો} & ટ્રાન્સફોર્મર, ઇન્ડક્ટર & જનરેટર, મોટર \\ \hline
\textbf{સૂત્ર} & $e = -N(d\Phi/dt)$ & $e = BLv$ \\ \hline
\textbf{ઉપયોગિતા} & AC સર્કિટ, પાવર સપ્લાય & પાવર જનરેશન, મોટર્સ \\ \hline
\end{tabulary}
\end{center}

\textbf{સ્ટેટિક EMF ના પ્રકારો:}
\begin{itemize}
    \item \keyword{સેલ્ફ-ઇન્ડ્યૂસ્ડ}: એક જ કોઇલ ફ્લક્સ ચેન્જ બનાવે અને અનુભવે છે
    \item \keyword{મ્યુચ્યુઅલી ઇન્ડ્યૂસ્ડ}: એક કોઇલ બીજી કોઇલને અસર કરે છે
\end{itemize}

\textbf{ડાયનેમિક EMF ના પરિબળો:}
\begin{itemize}
    \item \keyword{મેગ્નેટિક ફીલ્ડ સ્ટ્રેન્થ (B)}: ટેસ્લા
    \item \keyword{કન્ડક્ટર લેન્થ (L)}: મીટર
    \item \keyword{વેલોસિટી (v)}: m/s
\end{itemize}
\end{solutionbox}

\begin{mnemonicbox}
\mnemonic{"સ્ટેટિક સ્ટેઝ, ડાયનેમિક ડાન્સ" (સ્ટેટિક = સ્થિર, ડાયનેમિક = ગતિ)}
\end{mnemonicbox}

\questionmarks{2(a)}{3}{ટ્રાન્સફોર્મરમાં થતાં વિવિધ પ્રકારના લોસ સમજાવો.}

\begin{solutionbox}
\textbf{જવાબ:}

\textbf{ટ્રાન્સફોર્મર લોસ:}
\begin{center}
\captionof{table}{ટ્રાન્સફોર્મર લોસ}
\begin{tabulary}{\linewidth}{|L|L|L|L|}
\hline
\textbf{લોસનો પ્રકાર} & \textbf{કારણ} & \textbf{સ્થાન} & \textbf{લક્ષણો} \\ \hline
\textbf{આયર્ન લોસ} & હિસ્ટેરેસિસ + એડી કરંટ & કોર & સ્થિર, ફ્રિક્વન્સી આધારિત \\ \hline
\textbf{કોપર લોસ} & $I^2R$ હીટિંગ & વાઇન્ડિંગ & લોડ સાથે બદલાતું \\ \hline
\textbf{સ્ટ્રે લોસ} & લીકેજ ફ્લક્સ & એકંદર & ન્યૂનતમ \\ \hline
\end{tabulary}
\end{center}

\textbf{આયર્ન લોસ:}
\begin{itemize}
    \item \keyword{હિસ્ટેરેસિસ લોસ}: મેગ્નેટિક ડોમેઇન રિવર્સલ એનર્જી
    \item \keyword{એડી કરંટ લોસ}: કોરમાં ફરતા કરંટ
\end{itemize}

\textbf{કોપર લોસ:}
\begin{itemize}
    \item \keyword{પ્રાઇમરી વાઇન્ડિંગ}: $I_1^2R_1$
    \item \keyword{સેકન્ડરી વાઇન્ડિંગ}: $I_2^2R_2$
\end{itemize}
\end{solutionbox}

\begin{mnemonicbox}
\mnemonic{"આયર્ન કોર, કોપર કોઇલ" (મુખ્ય લોસનું સ્થાન)}
\end{mnemonicbox}

\questionmarks{2(b)}{4}{ટ્રાન્સફોર્મરનો કાર્ય સિદ્ધાંત સમજાવો.}

\begin{solutionbox}
\textbf{જવાબ:}

\textbf{કાર્ય સિદ્ધાંત:}
સામાન્ય મેગ્નેટિક કોર દ્વારા પ્રાઇમરી અને સેકન્ડરી વાઇન્ડિંગ વચ્ચે \keyword{મ્યુચ્યુઅલ ઇલેક્ટ્રોમેગ્નેટિક ઇન્ડક્શન}.

\textbf{આકૃતિ:}
\begin{center}
\begin{tikzpicture}
    % Core
    \draw[thick, fill=gray!20] (0,0) rectangle (4,3);
    \draw[thick, fill=white] (1,1) rectangle (3,2);
    \node at (2, 2.5) {આયર્ન કોર};
    
    % Primary
    \draw[green!60!black, thick] (-0.5, 0.5) -- (0, 0.5);
    \foreach \y in {0.5, 0.8, ..., 2.5}
        \draw[green!60!black, thick] (0, \y) to[bend right] (0, \y+0.2);
    \node[left] at (-0.5, 1.5) {AC સપ્લાય};
    \node at (-0.8, 2) {$N_1$};
    
    % Secondary
    \draw[blue!60!black, thick] (4, 0.5) -- (4.5, 0.5);
    \foreach \y in {0.5, 0.8, ..., 2.5}
        \draw[blue!60!black, thick] (4, \y) to[bend left] (4, \y+0.2);
    \node[right] at (4.5, 1.5) {લોડ};
    \node at (4.8, 2) {$N_2$};
    
    % Flux
    \draw[dashed, ->] (2, 1.5) ellipse (0.5 and 0.2);
    \node at (2, 1.5) {$\Phi$};
\end{tikzpicture}
\captionof{figure}{ટ્રાન્સફોર્મર સિદ્ધાંત}
\end{center}

\textbf{ઓપરેશન સ્ટેપ્સ:}
\begin{itemize}
    \item \textbf{સ્ટેપ 1}: પ્રાઇમરીમાં AC કરંટ બદલાતું ફ્લક્સ બનાવે છે
    \item \textbf{સ્ટેપ 2}: ફ્લક્સ કોર દ્વારા સેકન્ડરી સાથે લિંક થાય છે
    \item \textbf{સ્ટેપ 3}: બદલાતું ફ્લક્સ સેકન્ડરીમાં EMF ઇન્ડ્યૂસ કરે છે
    \item \textbf{સ્ટેપ 4}: સેકન્ડરી EMF લોડ દ્વારા કરંટ ચલાવે છે
\end{itemize}

\textbf{મુખ્ય સંબંધો:}
\begin{itemize}
    \item \keyword{વોલ્ટેજ રેશિયો}: $V_2/V_1 = N_2/N_1$
    \item \keyword{કરંટ રેશિયો}: $I_1/I_2 = N_2/N_1$
\end{itemize}
\end{solutionbox}

\begin{mnemonicbox}
\mnemonic{"પ્રાઇમરી પ્રોડ્યૂસ, સેકન્ડરી સપ્લાય" (એનર્જી ટ્રાન્સફરની દિશા)}
\end{mnemonicbox}

\questionmarks{2(c)}{7}{ટ્રાન્સફોર્મરનું EMF સૂત્ર તારવો.}

\begin{solutionbox}
\textbf{જવાબ:}

\textbf{આપેલા પેરામીટર:}
\begin{itemize}
    \item $N_1$: પ્રાઇમરી ટર્ન્સ, $N_2$: સેકન્ડરી ટર્ન્સ
    \item $\Phi_m$: મેક્સિમમ ફ્લક્સ, $f$: ફ્રિક્વન્સી
\end{itemize}

\textbf{EMF ડેરિવેશન:}

\textbf{સ્ટેપ 1: ફ્લક્સ વેરિએશન}
\[ \Phi = \Phi_m \sin(2\pi ft) \]

\textbf{સ્ટેપ 2: ફ્લક્સ ચેન્જનો દર}
\[ \frac{d\Phi}{dt} = 2\pi f \Phi_m \cos(2\pi ft) \]

\textbf{સ્ટેપ 3: મેક્સિમમ રેટ}
\[ \left(\frac{d\Phi}{dt}\right)_{max} = 2\pi f \Phi_m \]

\textbf{સ્ટેપ 4: RMS EMF સૂત્ર}
\begin{align*}
E_1 &= 4.44 \times f \times N_1 \times \Phi_m \\
E_2 &= 4.44 \times f \times N_2 \times \Phi_m
\end{align*}

\textbf{EMF સૂત્રના ભાગો:}
\begin{center}
\captionof{table}{EMF સૂત્રના ભાગો}
\begin{tabulary}{\linewidth}{|L|L|L|}
\hline
\textbf{પ્રતીક} & \textbf{પેરામીટર} & \textbf{એકમો} \\ \hline
$E$ & RMS EMF & વોલ્ટ \\ \hline
$f$ & ફ્રિક્વન્સી & Hz \\ \hline
$N$ & ટર્ન્સની સંખ્યા & - \\ \hline
$\Phi_m$ & મેક્સિમમ ફ્લક્સ & વેબર \\ \hline
$4.44$ & ફોર્મ ફેક્ટર કોન્સ્ટન્ટ & - \\ \hline
\end{tabulary}
\end{center}

\textbf{ટ્રાન્સફોર્મેશન રેશિયો:}
\[ K = \frac{E_2}{E_1} = \frac{N_2}{N_1} \]
\end{solutionbox}

\begin{mnemonicbox}
\mnemonic{"ફોર-ફોર્ટી-ફોર ફ્લક્સ ફોર્મ્યુલા" (4.44 ફેક્ટર)}
\end{mnemonicbox}

\questionmarks{2(a અથવા)}{3}{ટ્રાન્સફોર્મરની ઉપયોગિતા સમજાવો.}

\begin{solutionbox}
\textbf{જવાબ:}

\textbf{ટ્રાન્સફોર્મર એપ્લિકેશન્સ:}
\begin{center}
\captionof{table}{ટ્રાન્સફોર્મર એપ્લિકેશન્સ}
\begin{tabulary}{\linewidth}{|L|L|L|}
\hline
\textbf{ઉપયોગિતા} & \textbf{હેતુ} & \textbf{વોલ્ટેજ રેન્જ} \\ \hline
\textbf{પાવર ટ્રાન્સમિશન} & ટ્રાન્સમિશન લોસ ઘટાડવા & સ્ટેપ-અપ (400kV) \\ \hline
\textbf{ડિસ્ટ્રિબ્યુશન} & ગ્રાહકો માટે સુરક્ષિત વોલ્ટેજ & સ્ટેપ-ડાઉન (230V) \\ \hline
\textbf{આઇસોલેશન} & ઇલેક્ટ્રિકલ આઇસોલેશન & 1:1 રેશિયો \\ \hline
\textbf{ઇલેક્ટ્રોનિક સર્કિટ} & DC પાવર સપ્લાય & સ્ટેપ-ડાઉન \\ \hline
\end{tabulary}
\end{center}

\textbf{ઇન્ડસ્ટ્રિયલ એપ્લિકેશન્સ:}
\begin{itemize}
    \item \keyword{વેલ્ડિંગ ટ્રાન્સફોર્મર}: હાઇ કરંટ, લો વોલ્ટેજ
    \item \keyword{ઇન્સ્ટ્રુમેન્ટ ટ્રાન્સફોર્મર}: મેઝરમેન્ટ અને પ્રોટેક્શન
    \item \keyword{ઓડિયો ટ્રાન્સફોર્મર}: ઇમ્પીડન્સ મેચિંગ
\end{itemize}
\end{solutionbox}

\begin{mnemonicbox}
\mnemonic{"પાવર ડિસ્ટ્રિબ્યુશન આઇસોલેશન ઇલેક્ટ્રોનિક્સ" (મુખ્ય એપ્લિકેશન વિસ્તારો)}
\end{mnemonicbox}

\questionmarks{2(b અથવા)}{4}{DC મોટર માટે બેક EMF અને ટોર્કનું સૂત્ર લખો.}

\begin{solutionbox}
\textbf{જવાબ:}

\textbf{બેક EMF સૂત્ર:}
\[ E_b = \frac{\phi Z N P}{60 A} \]

\textbf{સરળ સ્વરૂપ:}
\[ E_b = K \phi N \]

\textbf{ટોર્ક સૂત્ર:}
\[ T = \frac{\phi Z I_a P}{2\pi A} \]

\textbf{સરળ સ્વરૂપ:}
\[ T = K \phi I_a \]

\textbf{પ્રતીકોની વ્યાખ્યા:}
\begin{center}
\captionof{table}{પ્રતીકોની વ્યાખ્યા}
\begin{tabulary}{\linewidth}{|L|L|L|}
\hline
\textbf{પ્રતીક} & \textbf{પેરામીટર} & \textbf{એકમો} \\ \hline
$E_b$ & બેક EMF & વોલ્ટ \\ \hline
$T$ & ટોર્ક & N-m \\ \hline
$\phi$ & ફ્લક્સ પર પોલ & વેબર \\ \hline
$N$ & સ્પીડ & RPM \\ \hline
$I_a$ & આર્મેચર કરંટ & એમ્પિયર \\ \hline
$K$ & મોટર કોન્સ્ટન્ટ & - \\ \hline
\end{tabulary}
\end{center}
\end{solutionbox}

\begin{mnemonicbox}
\mnemonic{"બેક EMF વિરોધ કરે, ટોર્ક પ્રસ્તાવિત કરે" (EMF સપ્લાયનો વિરોધ, ટોર્ક રોટેશન ચલાવે)}
\end{mnemonicbox}

\questionmarks{2(c અથવા)}{7}{DC મોટરની રચના અને કાર્ય પદ્ધતિ આકૃતિ સાથે સમજાવો.}

\begin{solutionbox}
\textbf{જવાબ:}

\textbf{રચનાના ભાગો:}
\begin{center}
\captionof{table}{DC મોટરના પાર્ટ્સ}
\begin{tabulary}{\linewidth}{|L|L|L|}
\hline
\textbf{કોમ્પોનન્ટ} & \textbf{કાર્ય} & \textbf{મટીરિયલ} \\ \hline
\textbf{સ્ટેટર} & મેગ્નેટિક ફીલ્ડ પ્રદાન કરે છે & કાસ્ટ આયર્ન/સ્ટીલ \\ \hline
\textbf{રોટર/આર્મેચર} & ફરતો ભાગ & સિલિકોન સ્ટીલ લેમિનેશન્સ \\ \hline
\textbf{કોમ્યુટેટર} & કરંટ દિશા બદલવા & કોપર સેગમેન્ટ્સ \\ \hline
\textbf{બ્રશેસ} & કરંટ સંગ્રહ & કાર્બન \\ \hline
\textbf{ફીલ્ડ વાઇન્ડિંગ} & ઇલેક્ટ્રોમેગ્નેટ & કોપર વાયર \\ \hline
\end{tabulary}
\end{center}

\textbf{રચના આકૃતિ:}
\begin{center}
\begin{tikzpicture}
    % Stator
    \draw[thick] (0,0) circle (2);
    \node at (0, 2.2) {સ્ટેટર (યોક)};
    
    % Poles
    \fill[gray!30] (-0.5, 1.5) rectangle (0.5, 2); % N Pole Top
    \node at (0, 1.75) {N};
    \fill[gray!30] (-0.5, -2) rectangle (0.5, -1.5); % S Pole Bottom
    \node at (0, -1.75) {S};
    
    % Armature
    \draw[thick, fill=white] (0,0) circle (1.2);
    \node at (0,0) {આર્મેચર};
    
    % Commutator/Brushes
    \draw[thick, fill=orange!40] (-0.3, -0.3) rectangle (0.3, 0.3);
    \node at (0.5, 0) {કોમ્યુટેટર};
    \draw[fill=black] (-0.4, -0.1) rectangle (-0.3, 0.1); % Brush
    \draw[fill=black] (0.3, -0.1) rectangle (0.4, 0.1); % Brush
\end{tikzpicture}
\captionof{figure}{DC મોટર રચના}
\end{center}

\textbf{કાર્ય સિદ્ધાંત:}
\begin{itemize}
    \item \textbf{સ્ટેપ 1}: આર્મેચર કન્ડક્ટર દ્વારા કરંટ પસાર થાય છે
    \item \textbf{સ્ટેપ 2}: મેગ્નેટિક ફીલ્ડ કરંટ સાથે ઇન્ટરેક્ટ થાય છે
    \item \textbf{સ્ટેપ 3}: ફ્લેમિંગના ડાબા હાથના નિયમ દ્વારા બળ પેદા થાય છે
    \item \textbf{સ્ટેપ 4}: કોમ્યુટેટર કરંટની દિશા બદલે છે
    \item \textbf{સ્ટેપ 5}: સતત રોટેશન જાળવાય છે
\end{itemize}

\textbf{બળનું સૂત્ર:}
\[ F = B \times I \times L \]
\end{solutionbox}

\begin{mnemonicbox}
\mnemonic{"કરંટ ક્રિએટ્સ સર્ક્યુલર મોશન" (કરંટ ઇન્ટરેક્શન રોટેશન પેદા કરે છે)}
\end{mnemonicbox}

\questionmarks{3(a)}{3}{ટ્રાન્સફોર્મરની રચના સમજાવો.}

\begin{solutionbox}
\textbf{જવાબ:}

\textbf{ટ્રાન્સફોર્મર કન્સ્ટ્રક્શન:}
\begin{center}
\captionof{table}{ટ્રાન્સફોર્મર કન્સ્ટ્રક્શન}
\begin{tabulary}{\linewidth}{|L|L|L|}
\hline
\textbf{કોમ્પોનન્ટ} & \textbf{મટીરિયલ} & \textbf{કાર્ય} \\ \hline
\textbf{કોર} & સિલિકોન સ્ટીલ લેમિનેશન્સ & મેગ્નેટિક ફ્લક્સ પાથ \\ \hline
\textbf{પ્રાઇમરી વાઇન્ડિંગ} & કોપર/એલ્યુમિનિયમ & ઇનપુટ એનર્જી \\ \hline
\textbf{સેકન્ડરી વાઇન્ડિંગ} & કોપર/એલ્યુમિનિયમ & આઉટપુટ એનર્જી \\ \hline
\textbf{ઇન્સ્યુલેશન} & વાર્નિશ/પેપર & ઇલેક્ટ્રિકલ આઇસોલેશન \\ \hline
\textbf{ટાંકી} & સ્ટીલ & ઓઇલ કન્ટેઇનમેન્ટ અને કૂલિંગ \\ \hline
\end{tabulary}
\end{center}

\textbf{કોરના પ્રકારો:}
\begin{itemize}
    \item \keyword{શેલ ટાઇપ}: વાઇન્ડિંગ કોર દ્વારા ઘેરાયેલું
    \item \keyword{કોર ટાઇપ}: કોર વાઇન્ડિંગ દ્વારા ઘેરાયેલો
\end{itemize}

\textbf{કૂલિંગ મેથડ્સ:}
\begin{itemize}
    \item \keyword{એર કૂલિંગ}: નાના ટ્રાન્સફોર્મર
    \item \keyword{ઓઇલ કૂલિંગ}: મોટા ટ્રાન્સફોર્મર રેડિએટર સાથે
\end{itemize}
\end{solutionbox}

\begin{mnemonicbox}
\mnemonic{"કોર કેરીઝ કરંટ કેરફુલી" (કોર ડિઝાઇનનું મહત્વ)}
\end{mnemonicbox}

\questionmarks{3(b)}{4}{DC મોટરની ઉપયોગિતા સમજાવો.}

\begin{solutionbox}
\textbf{જવાબ:}

\textbf{DC મોટર એપ્લિકેશન્સ:}
\begin{center}
\captionof{table}{DC મોટર એપ્લિકેશન્સ}
\begin{tabulary}{\linewidth}{|L|L|L|}
\hline
\textbf{મોટરનો પ્રકાર} & \textbf{સ્પીડ લક્ષણ} & \textbf{ઉપયોગિતા} \\ \hline
\textbf{શન્ટ} & સ્થિર સ્પીડ & ફેન, પંપ, લેથ \\ \hline
\textbf{સિરીઝ} & બદલાતી સ્પીડ & ટ્રેક્શન, ક્રેન \\ \hline
\textbf{કમ્પાઉન્ડ} & મધ્યમ વેરિએશન & એલિવેટર, કોમ્પ્રેસર \\ \hline
\end{tabulary}
\end{center}

\textbf{ઇન્ડસ્ટ્રિયલ એપ્લિકેશન્સ:}
\begin{itemize}
    \item \keyword{શન્ટ મોટર}: મશીન ટૂલ્સ જેને સ્થિર સ્પીડ જોઇએ
    \item \keyword{સિરીઝ મોટર}: ઇલેક્ટ્રિક વાહનો, ભારે લોડ સ્ટાર્ટિંગ
    \item \keyword{કમ્પાઉન્ડ મોટર}: રોલિંગ મિલ્સ, પંચ પ્રેસ
\end{itemize}

\textbf{ફાયદાઓ:}
\begin{itemize}
    \item \keyword{સરળ સ્પીડ કન્ટ્રોલ}: વોલ્ટેજ/ફીલ્ડ કન્ટ્રોલ
    \item \keyword{ઉચ્ચ સ્ટાર્ટિંગ ટોર્ક}: સિરીઝ મોટર
    \item \keyword{રિવર્સિબલ ઓપરેશન}: ફીલ્ડ/આર્મેચર પોલેરિટી બદલો
\end{itemize}
\end{solutionbox}

\begin{mnemonicbox}
\mnemonic{"શન્ટ સ્ટેઝ, સિરીઝ સ્પીડ્સ" (સ્પીડ લક્ષણો)}
\end{mnemonicbox}

\questionmarks{3(c)}{7}{DC મોટરના વિવિધ પ્રકાર સમજાવો.}

\begin{solutionbox}
\textbf{જવાબ:}

\textbf{DC મોટર વર્ગીકરણ:}
\begin{center}
\captionof{table}{DC મોટર વર્ગીકરણ}
\begin{tabulary}{\linewidth}{|L|L|L|L|}
\hline
\textbf{પ્રકાર} & \textbf{ફીલ્ડ કનેક્શન} & \textbf{સ્પીડ-ટોર્ક} & \textbf{ઉપયોગિતા} \\ \hline
\textbf{શન્ટ} & આર્મેચરને સમાંતર & સ્થિર સ્પીડ, નીચો સ્ટાર્ટિંગ ટોર્ક & ફેન, પંપ \\ \hline
\textbf{સિરીઝ} & આર્મેચર સાથે સિરીઝ & બદલાતી સ્પીડ, ઉચ્ચ સ્ટાર્ટિંગ ટોર્ક & ટ્રેક્શન \\ \hline
\textbf{કમ્પાઉન્ડ} & સિરીઝ અને શન્ટ બંને & મધ્યમ લક્ષણો & સામાન્ય હેતુ \\ \hline
\end{tabulary}
\end{center}

\textbf{શન્ટ મોટર આકૃતિ:}
\begin{center}
\begin{tikzpicture}
    \draw (0,0) to[battery1, l=$V_{DC}$] (0,3) -- (3,3);
    \draw (3,3) to[R, l=ફીલ્ડ] (3,0) -- (0,0);
    \draw (1.5,3) to[short] (1.5,2.5) node[circle, draw, fill=white] {M} -- (1.5,0);
\end{tikzpicture}
\captionof{figure}{DC શન્ટ મોટર}
\end{center}

\textbf{લક્ષણો:}
\begin{itemize}
    \item \textbf{શન્ટ}: સ્પીડ $\propto (V - I_aR_a)/\phi$
    \item \textbf{સિરીઝ}: ઉચ્ચ સ્ટાર્ટિંગ ટોર્ક, સ્પીડ લોડ સાથે બદલાય છે
    \item \textbf{કમ્પાઉન્ડ}: બંને પ્રકારના ફાયદાઓ સંયોજિત
\end{itemize}

\textbf{સ્પીડ કન્ટ્રોલ મેથડ્સ:}
\begin{itemize}
    \item \keyword{આર્મેચર કન્ટ્રોલ}: આર્મેચર વોલ્ટેજ બદલો
    \item \keyword{ફીલ્ડ કન્ટ્રોલ}: ફીલ્ડ કરંટ બદલો
    \item \keyword{રેઝિસ્ટન્સ કન્ટ્રોલ}: બાહ્ય રેઝિસ્ટન્સ ઉમેરો
\end{itemize}
\end{solutionbox}

\begin{mnemonicbox}
\mnemonic{"શન્ટ સ્ટેડી, સિરીઝ સ્ટ્રોંગ, કમ્પાઉન્ડ કમ્બાઇન્ડ" (મુખ્ય લક્ષણો)}
\end{mnemonicbox}

\questionmarks{3(a OR)}{3}{ટ્રાન્સફોર્મરનો ટ્રાન્સફોર્મેશન રેશિયો સમજાવો.}

\begin{solutionbox}
\textbf{જવાબ:}

\textbf{વ્યાખ્યા:}
ટ્રાન્સફોર્મેશન રેશિયો (K) એ સેકન્ડરી અને પ્રાઇમરી વોલ્ટેજ અથવા ટર્ન્સનો રેશિયો છે.

\textbf{ગાણિતિક અભિવ્યક્તિ:}
\[ K = \frac{N_2}{N_1} = \frac{E_2}{E_1} = \frac{V_2}{V_1} \]

\textbf{ટ્રાન્સફોર્મેશન રેશિયોના પ્રકારો:}
\begin{center}
\captionof{table}{ટ્રાન્સફોર્મેશન રેશિયોના પ્રકારો}
\begin{tabulary}{\linewidth}{|L|L|L|L|}
\hline
\textbf{રેશિયો} & \textbf{પ્રકાર} & \textbf{વોલ્ટેજ ચેન્જ} & \textbf{ઉપયોગિતા} \\ \hline
$K > 1$ & સ્ટેપ-અપ & વધારે છે & પાવર ટ્રાન્સમિશન \\ \hline
$K < 1$ & સ્ટેપ-ડાઉન & ઘટાડે છે & ડિસ્ટ્રિબ્યુશન \\ \hline
$K = 1$ & આઇસોલેશન & સમાન & સુરક્ષા આઇસોલેશન \\ \hline
\end{tabulary}
\end{center}

\textbf{કરંટ સંબંધ:}
\[ \frac{I_1}{I_2} = \frac{N_2}{N_1} = K \]

\textbf{પાવર સંબંધ:}
\[ P_1 = P_2 \text{ (આદર્શ ટ્રાન્સફોર્મર)} \]
\end{solutionbox}

\begin{mnemonicbox}
\mnemonic{"ટર્ન્સ ટેલ ટ્રાન્સફોર્મેશન" (ટર્ન્સ રેશિયો વોલ્ટેજ રેશિયો નક્કી કરે છે)}
\end{mnemonicbox}

\questionmarks{3(b OR)}{4}{ઓટો ટ્રાન્સફોર્મરની ઉપયોગિતા સમજાવો.}

\begin{solutionbox}
\textbf{જવાબ:}

\textbf{ઓટો ટ્રાન્સફોર્મર એપ્લિકેશન્સ:}
\begin{center}
\captionof{table}{ઓટો ટ્રાન્સફોર્મર એપ્લિકેશન્સ}
\begin{tabulary}{\linewidth}{|L|L|L|}
\hline
\textbf{ઉપયોગિતા} & \textbf{ફાયદો} & \textbf{વોલ્ટેજ રેન્જ} \\ \hline
\textbf{મોટર સ્ટાર્ટિંગ} & સ્ટાર્ટિંગ કરંટ ઘટાડે છે & રેટેડનો 50-80\% \\ \hline
\textbf{વોલ્ટેજ રેગ્યુલેશન} & બારીક વોલ્ટેજ એડજસ્ટમેન્ટ & $\pm$10\% વેરિએશન \\ \hline
\textbf{લેબોરેટરી} & વેરિએબલ વોલ્ટેજ સોર્સ & ઇનપુટનો 0-110\% \\ \hline
\textbf{પાવર સિસ્ટમ} & ઇકોનોમિક ટ્રાન્સમિશન & નજીકના વોલ્ટેજ રેશિયો \\ \hline
\end{tabulary}
\end{center}

\textbf{ફાયદાઓ:}
\begin{itemize}
    \item \keyword{ઇકોનોમી}: ઓછું કોપર અને આયર્ન જરૂરી
    \item \keyword{એફિશિયન્સી}: બે-વાઇન્ડિંગ ટ્રાન્સફોર્મર કરતાં વધારે
    \item \keyword{સાઇઝ}: કોમ્પેક્ટ ડિઝાઇન
    \item \keyword{રેગ્યુલેશન}: બેહતર વોલ્ટેજ રેગ્યુલેશન
\end{itemize}

\textbf{મર્યાદાઓ:}
\begin{itemize}
    \item \keyword{આઇસોલેશન નથી}: સામાન્ય ઇલેક્ટ્રિકલ કનેક્શન
    \item \keyword{સુરક્ષા}: વધારે ફોલ્ટ કરંટ
\end{itemize}
\end{solutionbox}

\begin{mnemonicbox}
\mnemonic{"ઓટો એડજસ્ટ્સ એડવાન્ટેજિયસલી" (ઓટોમેટિક વોલ્ટેજ એડજસ્ટમેન્ટ ફાયદો)}
\end{mnemonicbox}

\questionmarks{3(c OR)}{7}{DC શન્ટ મોટર માટે સ્પીડ કન્ટ્રોલ કરવાની રીતો સમજાવો.}

\begin{solutionbox}
\textbf{જવાબ:}

\textbf{સ્પીડ કન્ટ્રોલ મેથડ્સ:}
\begin{center}
\captionof{table}{સ્પીડ કન્ટ્રોલ મેથડ્સ}
\begin{tabulary}{\linewidth}{|L|L|L|L|}
\hline
\textbf{મેથડ} & \textbf{રેન્જ} & \textbf{એફિશિયન્સી} & \textbf{ઉપયોગિતા} \\ \hline
\textbf{આર્મેચર કન્ટ્રોલ} & રેટેડ સ્પીડથી નીચે & ઉચ્ચ & પ્રિસાઇઝ સ્પીડ કન્ટ્રોલ \\ \hline
\textbf{ફીલ્ડ કન્ટ્રોલ} & રેટેડ સ્પીડથી ઉપર & ઉચ્ચ & કોન્સ્ટન્ટ પાવર ડ્રાઇવ્સ \\ \hline
\textbf{રેઝિસ્ટન્સ કન્ટ્રોલ} & રેટેડ સ્પીડથી નીચે & નીચી & સરળ એપ્લિકેશન્સ \\ \hline
\end{tabulary}
\end{center}

\textbf{આર્મેચર કન્ટ્રોલ આકૃતિ:}
\begin{center}
\begin{tikzpicture}
    \draw (0,0) to[battery1, l=$V$] (0,3) -- (3,3);
    \draw (3,3) to[R, l=$R_{ext}$] (3,2) to[short] (3,1.5) node[circle, draw, fill=white] {M} -- (3,0) -- (0,0);
    \draw (1.5,3) to[R, l=$Field$] (1.5,0);
\end{tikzpicture}
\captionof{figure}{આર્મેચર કન્ટ્રોલ}
\end{center}

\textbf{સ્પીડ સૂત્રો:}
\begin{itemize}
    \item \keyword{આર્મેચર કન્ટ્રોલ}: $N \propto (V - I_aR_a)/\phi$
    \item \keyword{ફીલ્ડ કન્ટ્રોલ}: $N \propto V/\phi$
    \item \keyword{રેઝિસ્ટન્સ કન્ટ્રોલ}: $N \propto (V - I_a(R_a + R_{ext}))/\phi$
\end{itemize}

\textbf{આધુનિક મેથડ્સ:}
\begin{itemize}
    \item \keyword{ચોપર કન્ટ્રોલ}: PWM વોલ્ટેજ કન્ટ્રોલ
    \item \keyword{વોર્ડ-લિયોનાર્ડ સિસ્ટમ}: મોટર-જનરેટર સેટ
    \item \keyword{ઇલેક્ટ્રોનિક કન્ટ્રોલ}: થાઇરિસ્ટર/IGBT ડ્રાઇવ્સ
\end{itemize}
\end{solutionbox}

\begin{mnemonicbox}
\mnemonic{"આર્મેચર એક્યુરેટ, ફીલ્ડ ફાસ્ટ, રેઝિસ્ટન્સ રફ" (કન્ટ્રોલ લક્ષણો)}
\end{mnemonicbox}

\questionmarks{4(a)}{3}{અલ્ટરનેટિંગ EMF નું વેક્ટર નિરૂપણ સમજાવો.}

\begin{solutionbox}
\textbf{જવાબ:}

\textbf{વેક્ટર રિપ્રેઝન્ટેશન:}
અલ્ટરનેટિંગ EMF ને સ્થિર મેગ્નિટ્યૂડ અને એંગ્યુલર વેલોસિટી સાથે ફરતા વેક્ટર (ફેઝર) તરીકે દર્શાવી શકાય છે.

\textbf{ગાણિતિક સ્વરૂપ:}
\[ e = E_m \sin(\omega t + \phi) \]

\textbf{આકૃતિ:}
\begin{center}
\begin{tikzpicture}
    \draw[->] (-1,0) -- (3,0) node[right] {રેફરન્સ};
    \draw[->] (0,-1) -- (0,3);
    \draw[thick, ->] (0,0) -- (45:2.5) node[right] {$E_m$};
    \draw (0.5,0) arc (0:45:0.5);
    \node at (0.8, 0.3) {$\omega t$};
\end{tikzpicture}
\captionof{figure}{EMF ફેઝર ડાયાગ્રામ}
\end{center}

\textbf{વેક્ટર પેરામીટર:}
\begin{center}
\captionof{table}{વેક્ટર પેરામીટર}
\begin{tabulary}{\linewidth}{|L|L|L|L|}
\hline
\textbf{પેરામીટર} & \textbf{પ્રતીક} & \textbf{એકમો} & \textbf{વર્ણન} \\ \hline
\textbf{મેગ્નિટ્યૂડ} & $E_m$ & વોલ્ટ & મેક્સિમમ EMF \\ \hline
\textbf{એંગ્યુલર વેલોસિટી} & $\omega$ & rad/s & રોટેશન સ્પીડ \\ \hline
\textbf{ફેઝ એંગલ} & $\phi$ & ડિગ્રી & પ્રારંભિક ફેઝ \\ \hline
\textbf{ફ્રિક્વન્સી} & $f = \omega/2\pi$ & Hz & સાઇકલ પર સેકન્ડ \\ \hline
\end{tabulary}
\end{center}
\end{solutionbox}

\begin{mnemonicbox}
\mnemonic{"વેક્ટર્સ વિઝ્યુઅલાઇઝ વોલ્ટેજ વેરિએશન" (ફેઝર રિપ્રેઝન્ટેશન ફાયદાઓ)}
\end{mnemonicbox}

\questionmarks{4(b)}{4}{અલ્ટરનેટિંગ કરંટના સંદર્ભમાં નીચેના પદોની વ્યાખ્યા આપો: RMS વેલ્યુ, એવરેજ વેલ્યુ, ફ્રિક્વન્સી, ટાઇમ પિરિયડ}

\begin{solutionbox}
\textbf{જવાબ:}

\textbf{AC પેરામીટર વ્યાખ્યા:}
\begin{center}
\captionof{table}{AC પેરામીટર વ્યાખ્યા}
\begin{tabulary}{\linewidth}{|L|L|L|L|}
\hline
\textbf{પદ} & \textbf{વ્યાખ્યા} & \textbf{સૂત્ર} & \textbf{એકમો} \\ \hline
\textbf{RMS વેલ્યુ} & સમાન હીટિંગ પેદા કરતો અસરકારક મૂલ્ય & $I_m/\sqrt{2}$ & એમ્પિયર \\ \hline
\textbf{એવરેજ વેલ્યુ} & અર્ધ સાઇકલ પર સરેરાશ મૂલ્ય & $2I_m/\pi$ & એમ્પિયર \\ \hline
\textbf{ફ્રિક્વન્સી} & સેકન્ડ દીઠ સાઇકલની સંખ્યા & $f = 1/T$ & Hz \\ \hline
\textbf{ટાઇમ પિરિયડ} & એક સંપૂર્ણ સાઇકલ માટેનો સમય & $T = 1/f$ & સેકન્ડ \\ \hline
\end{tabulary}
\end{center}

\textbf{ગાણિતિક સંબંધો:}
\begin{itemize}
    \item \keyword{ફોર્મ ફેક્ટર}: RMS/Average = $\pi/2\sqrt{2} = 1.11$
    \item \keyword{પીક ફેક્ટર}: Peak/RMS = $\sqrt{2} = 1.414$
    \item \keyword{એંગ્યુલર ફ્રિક્વન્સી}: $\omega = 2\pi f$
\end{itemize}
\end{solutionbox}

\begin{mnemonicbox}
\mnemonic{"રિયલી મીન સ્ક્વેર, એવરેજ ફ્રિક્વન્સી ટાઇમ" (મુખ્ય AC પેરામીટર)}
\end{mnemonicbox}

\questionmarks{4(c)}{7}{સ્ટાર જોડાણમાં લાઇન વોલ્ટેજ અને ફેઇઝ વોલ્ટેજ તથા લાઇન કરંટ અને ફેઇઝ કરંટ વચ્ચેનો સંબંધ દર્શાવતા સૂત્ર તારવો.}

\begin{solutionbox}
\textbf{જવાબ:}

\textbf{સ્ટાર કનેક્શન આકૃતિ:}
\begin{center}
\begin{tikzpicture}
    \node[circle, fill, inner sep=1.5pt] (N) at (0,0) {};
    \node[right] at (N) {N};
    \draw (N) -- (90:2) node[above] {R};
    \draw (N) -- (210:2) node[left] {Y};
    \draw (N) -- (330:2) node[right] {B};
    % Lines
    \draw[->] (90:2) -- (90:3) node[above] {Line R};
    \draw[->] (210:2) -- (210:3) node[left] {Line Y};
    \draw[->] (330:2) -- (330:3) node[right] {Line B};
\end{tikzpicture}
\captionof{figure}{સ્ટાર કનેક્શન}
\end{center}

\textbf{વોલ્ટેજ સંબંધો:}
\begin{itemize}
    \item \textbf{ફેઝ વોલ્ટેજ}: $V_R, V_Y, V_B$ (ન્યુટ્રલ સંદર્ભે)
    \item \textbf{લાઇન વોલ્ટેજ}: $V_{RY}, V_{YB}, V_{BR}$ (લાઇન વચ્ચે)
\end{itemize}

\textbf{ફેઝર વિશ્લેષણ:}
\[ V_{RY} = V_R - V_Y \]

\textbf{વેક્ટર એડિશન:}
કોસાઇન નિયમનો ઉપયોગ કરીને:
\[ V_L = \sqrt{V_{ph}^2 + V_{ph}^2 - 2V_{ph}V_{ph}\cos(120^\circ)} \]
\[ V_L = \sqrt{2V_{ph}^2 + V_{ph}^2} = \sqrt{3} \times V_{ph} \]

\textbf{સ્ટાર કનેક્શન સંબંધો:}
\begin{center}
\captionof{table}{સ્ટાર કનેક્શન સંબંધો}
\begin{tabulary}{\linewidth}{|L|L|}
\hline
\textbf{પેરામીટર} & \textbf{સંબંધ} \\ \hline
\textbf{લાઇન વોલ્ટેજ} & $V_L = \sqrt{3} \times V_{ph}$ \\ \hline
\textbf{લાઇન કરંટ} & $I_L = I_{ph}$ \\ \hline
\textbf{પાવર} & $P = \sqrt{3} \times V_L \times I_L \times \cos\phi$ \\ \hline
\end{tabulary}
\end{center}
\end{solutionbox}

\begin{mnemonicbox}
\mnemonic{"Star Scales Voltage, Same current" ($\sqrt{3}$ factor for voltage, current unchanged)}
\end{mnemonicbox}

\questionmarks{4(a OR)}{3}{અલ્ટરનેટિંગ કરંટનું વેક્ટર નિરૂપણ સમજાવો.}

\begin{solutionbox}
\textbf{જવાબ:}

\textbf{ વેક્ટર રિપ્રેઝન્ટેશન:}
AC કરંટને મેગ્નિટ્યૂડ અને ફેઝ એંગલ સાથે ફરતા ફેઝર તરીકે દર્શાવાય છે.

\textbf{ગાણિતિક અભિવ્યક્તિ:}
\[ i = I_m \sin(\omega t + \phi) \]

\textbf{ફેઝર ડાયાગ્રામ:}
\begin{center}
\begin{tikzpicture}
    \draw[->] (-1,0) -- (3,0) node[right] {રેફરન્સ};
    \draw[->] (0,-1) -- (0,3);
    \draw[thick, ->] (0,0) -- (60:2.5) node[right] {$I_m$};
    \draw (0.5,0) arc (0:60:0.5);
    \node at (0.8, 0.4) {$\phi$};
    \node at (0,0) [below left] {O};
\end{tikzpicture}
\captionof{figure}{કરંટ ફેઝર}
\end{center}

\textbf{કરંટ વેક્ટર એલિમેન્ટ્સ:}
\begin{center}
\captionof{table}{કરંટ વેક્ટર એલિમેન્ટ્સ}
\begin{tabulary}{\linewidth}{|L|L|L|}
\hline
\textbf{એલિમેન્ટ} & \textbf{પ્રતીક} & \textbf{વર્ણન} \\ \hline
\textbf{મેગ્નિટ્યૂડ} & $I_m$ & પીક કરંટ વેલ્યુ \\ \hline
\textbf{ફેઝ} & $\phi$ & લીડિંગ/લેગિંગ એંગલ \\ \hline
\textbf{એંગ્યુલર વેલોસિટી} & $\omega$ & રોટેશન સ્પીડ \\ \hline
\textbf{RMS વેલ્યુ} & $I = I_m/\sqrt{2}$ & અસરકારક કરંટ \\ \hline
\end{tabulary}
\end{center}
\end{solutionbox}

\begin{mnemonicbox}
\mnemonic{"કરંટ સર્કલ્સ કન્ટિન્યુઅસલી" (ફરતા ફેઝર કન્સેપ્ટ)}
\end{mnemonicbox}

\questionmarks{4(b OR)}{4}{અલ્ટરનેટિંગ કરંટના સંદર્ભમાં નીચેના પદોની વ્યાખ્યા આપો: ફોર્મ ફેક્ટર, પીક ફેક્ટર, કોણીય વેગ, એમ્પ્લિટ્યૂડ}

\begin{solutionbox}
\textbf{જવાબ:}

\textbf{AC કરંટ પેરામીટર:}
\begin{center}
\captionof{table}{AC કરંટ પેરામીટર}
\begin{tabulary}{\linewidth}{|L|L|L|L|}
\hline
\textbf{પદ} & \textbf{વ્યાખ્યા} & \textbf{સૂત્ર} & \textbf{સામાન્ય મૂલ્ય} \\ \hline
\textbf{ફોર્મ ફેક્ટર} & RMS/Average વેલ્યુ રેશિયો & $I_{rms}/I_{avg}$ & 1.11 (સાઇન વેવ) \\ \hline
\textbf{પીક ફેક્ટર} & Peak/RMS વેલ્યુ રેશિયો & $I_m/I_{rms}$ & 1.414 (સાઇન વેવ) \\ \hline
\textbf{એંગ્યુલર વેલોસિટી} & ફેઝ ચેન્જનો દર & $\omega = 2\pi f$ & 314 rad/s (50Hz) \\ \hline
\textbf{એમ્પ્લિટ્યૂડ} & મેક્સિમમ ઇન્સ્ટન્ટેનિયસ વેલ્યુ & $I_m$ & પીક કરંટ \\ \hline
\end{tabulary}
\end{center}

\textbf{પ્રેક્ટિકલ મહત્વ:}
\begin{itemize}
    \item \textbf{ડિઝાઇન વિચારણાઓ}: ઇન્સ્યુલેશન માટે પીક ફેક્ટર
    \item \textbf{વેવફોર્મ વિશ્લેષણ}: ડિસ્ટોર્શન માટે ફોર્મ ફેક્ટર
\end{itemize}
\end{solutionbox}

\begin{mnemonicbox}
\mnemonic{"ફોર્મ પીક એંગ્યુલર એમ્પ્લિટ્યૂડ" (ચાર મુખ્ય ફેક્ટર)}
\end{mnemonicbox}

\questionmarks{4(c OR)}{7}{ડેલ્ટા જોડાણમાં લાઇન વોલ્ટેજ અને ફેઇઝ વોલ્ટેજ તથા લાઇન કરંટ અને ફેઇઝ કરંટ વચ્ચેનો સંબંધ દર્શાવતા સૂત્ર તારવો.}

\begin{solutionbox}
\textbf{જવાબ:}

\textbf{ડેલ્ટા કનેક્શન આકૃતિ:}
\begin{center}
\begin{tikzpicture}
    \draw (0,0) -- (4,0) -- (2,3.46) -- cycle;
    \node at (2, -0.4) {C};
    \node at (4.2, 0) {B};
    \node at (2, 3.66) {A};
    \draw (0,0) node[circle,fill,inner sep=1.5pt]{};
    \draw (4,0) node[circle,fill,inner sep=1.5pt]{};
    \draw (2,3.46) node[circle,fill,inner sep=1.5pt]{};
    \draw[->] (0,0) -- (-1, -0.5) node[left] {$I_C$};
    \draw[->] (4,0) -- (5, -0.5) node[right] {$I_B$};
    \draw[->] (2,3.46) -- (2, 4.5) node[above] {$I_A$};
\end{tikzpicture}
\captionof{figure}{ડેલ્ટા કનેક્શન}
\end{center}

\textbf{વોલ્ટેજ સંબંધો:}
ડેલ્ટા કનેક્શનમાં, લાઇન વોલ્ટેજ ફેઝ વોલ્ટેજ સમાન હોય છે:
\[ V_L = V_{ph} \]

\textbf{કરંટ વિશ્લેષણ:}
દરેક લાઇન કરંટ બે ફેઝ કરંટનો વેક્ટર સમ છે.
\[ I_A = I_{AB} - I_{CA} \]

\textbf{વેક્ટર સબટ્રેક્શન:}
ફેઝર ડાયાગ્રામનો ઉપયોગ કરીને:
\[ I_L = \sqrt{I_{ph}^2 + I_{ph}^2 - 2I_{ph}I_{ph}\cos(60^\circ)} \]
\[ I_L = \sqrt{2I_{ph}^2 - I_{ph}^2} = \sqrt{3} \times I_{ph} \]

\textbf{ડેલ્ટા કનેક્શન સંબંધો:}
\begin{center}
\captionof{table}{ડેલ્ટા કનેક્શન સંબંધો}
\begin{tabulary}{\linewidth}{|L|L|}
\hline
\textbf{પેરામીટર} & \textbf{સંબંધ} \\ \hline
\textbf{લાઇન વોલ્ટેજ} & $V_L = V_{ph}$ \\ \hline
\textbf{લાઇન કરંટ} & $I_L = \sqrt{3} \times I_{ph}$ \\ \hline
\textbf{પાવર} & $P = \sqrt{3} \times V_L \times I_L \times \cos\phi$ \\ \hline
\end{tabulary}
\end{center}
\end{solutionbox}

\begin{mnemonicbox}
\mnemonic{"Delta Doubles current, Same voltage" ($\sqrt{3}$ factor for current, voltage unchanged)}
\end{mnemonicbox}

\questionmarks{5(a)}{3}{શુદ્ધ અવરોધ ધરાવતા પરિપથ માંથી અલ્ટરનેટિંગ કરંટની વર્તણૂક જરૂરી આકૃતિ અને વેવફોર્મ સાથે સમજાવો.}

\begin{solutionbox}
\textbf{જવાબ:}

\textbf{સર્કિટ આકૃતિ:}
\begin{center}
\begin{tikzpicture}
    \draw (0,0) to[sinusoidal voltage source, l=AC] (0,2) -- (2,2) to[R, l=$R$] (2,0) -- (0,0);
\end{tikzpicture}
\captionof{figure}{AC રેઝિસ્ટિવ સર્કિટ}
\end{center}

\textbf{વેવફોર્મ:}
\begin{center}
\begin{tikzpicture}
    \draw[->] (0,0) -- (4,0) node[right] {$t$};
    \draw[->] (0,-1.5) -- (0,1.5) node[above] {$V, I$};
    \draw[thick, blue] plot[domain=0:3.5, samples=100] (\x, {sin(\x r * 3.14)});
    \draw[dashed, red] plot[domain=0:3.5, samples=100] (\x, {0.7 * sin(\x r * 3.14)});
    \node[blue] at (1, 1.2) {V};
    \node[red] at (1, 0.5) {I};
\end{tikzpicture}
\captionof{figure}{V અને I સમાન ફેઝમાં}
\end{center}

\textbf{રેઝિસ્ટર દ્વારા AC:}
\begin{center}
\captionof{table}{રેઝિસ્ટર દ્વારા AC}
\begin{tabulary}{\linewidth}{|L|L|L|}
\hline
\textbf{પેરામીટર} & \textbf{સંબંધ} & \textbf{ફેઝ} \\ \hline
\textbf{ઓહમનો નિયમ} & $V = IR$ & સમાન ફેઝ \\ \hline
\textbf{પાવર} & $P = VI = I^2R$ & હંમેશા પોઝિટિવ \\ \hline
\textbf{ઇમ્પીડન્સ} & $Z = R$ & શુદ્ધ રેઝિસ્ટિવ \\ \hline
\end{tabulary}
\end{center}
\end{solutionbox}

\begin{mnemonicbox}
\mnemonic{"રેઝિસ્ટર રિફ્યુઝ ફેઝ શિફ્ટ" (કોઈ ફેઝ ડિફરન્સ નથી)}
\end{mnemonicbox}

\questionmarks{5(b)}{4}{અલ્ટરનેટિંગ કરંટના સંદર્ભમાં નીચેના પદોની વ્યાખ્યા આપો: ઇમ્પીડન્સ, ફેઝ એંગલ, પાવર ફેક્ટર, રિએક્ટિવ પાવર}

\begin{solutionbox}
\textbf{જવાબ:}

\textbf{AC સર્કિટ પેરામીટર:}
\begin{center}
\captionof{table}{AC સર્કિટ પેરામીટર}
\begin{tabulary}{\linewidth}{|L|L|L|L|}
\hline
\textbf{પદ} & \textbf{વ્યાખ્યા} & \textbf{સૂત્ર} & \textbf{એકમો} \\ \hline
\textbf{ઇમ્પીડન્સ} & AC કરંટનો કુલ વિરોધ & $Z = \sqrt{R^2 + X^2}$ & ઓહ્મ \\ \hline
\textbf{ફેઝ એંગલ} & V અને I વચ્ચેનો કોણ & $\phi = \tan^{-1}(X/R)$ & ડિગ્રી \\ \hline
\textbf{પાવર ફેક્ટર} & ફેઝ એંગલનો કોસાઇન & $PF = \cos\phi = R/Z$ & - \\ \hline
\textbf{રિએક્ટિવ પાવર} & રિએક્ટિવ કોમ્પોનન્ટમાં પાવર & $Q = VI \sin\phi$ & VAR \\ \hline
\end{tabulary}
\end{center}

\textbf{પાવર ત્રિકોણ સંબંધ:}
\[ S^2 = P^2 + Q^2 \]
\end{solutionbox}

\begin{mnemonicbox}
\mnemonic{"ઇમ્પીડન્સ ફેઝ પાવર ક્વાડ્રેચર" (ચાર મુખ્ય AC પેરામીટર)}
\end{mnemonicbox}

\questionmarks{5(c)}{7}{જુદા જુદા પ્રકારના પ્રોટેક્ટિવ ડિવાઇસના નામ લખો અને કોઈ પણ એક પ્રોટેક્ટિવ ડિવાઇસની રચના તથા કાર્ય વિસ્તારથી સમજાવો.}

\begin{solutionbox}
\textbf{જવાબ:}

\textbf{પ્રોટેક્ટિવ ડિવાઇસ:}
\begin{center}
\captionof{table}{પ્રોટેક્ટિવ ડિવાઇસ}
\begin{tabulary}{\linewidth}{|L|L|L|}
\hline
\textbf{ડિવાઇસ} & \textbf{પ્રોટેક્શન વિરુદ્ધ} & \textbf{ઉપયોગિતા} \\ \hline
\textbf{ફ્યુઝ} & ઓવરકરંટ & લો/મિડિયમ વોલ્ટેજ \\ \hline
\textbf{MCB} & ઓવરલોડ, શોર્ટ સર્કિટ & ઘરેલું/કોમર્શિયલ \\ \hline
\textbf{ELCB} & અર્થ લીકેજ & સુરક્ષા પ્રોટેક્શન \\ \hline
\textbf{રિલે} & વિવિધ ફોલ્ટ & ઇન્ડસ્ટ્રિયલ સિસ્ટમ \\ \hline
\textbf{સર્જ એરેસ્ટર} & ઓવરવોલ્ટેજ & ટ્રાન્સમિશન લાઇન \\ \hline
\end{tabulary}
\end{center}

\textbf{MCB (મિનિએચર સર્કિટ બ્રેકર) - વિગતવાર સમજૂતી:}

\textbf{રચના:}
\begin{center}
\begin{tikzpicture}
    % MCB housing
    \draw[thick] (0,0) rectangle (3,4);
    \node at (1.5, 4.3) {MCB};
    
    % Mechanism
    \draw[thick] (0.5, 3.5) -- (1.5, 3.5) -- (1.5, 2.5); % Contacts
    \draw[thick, red] (1.5, 2.5) -- (2.5, 2.5); % Bimetal
    \draw[thick, blue, decoration={coil, aspect=0.5, segment length=2mm, amplitude=2mm}, decorate] (1.5, 1.5) -- (1.5, 0.5); % Coil
    \node[right] at (2.5, 2.5) {બાઇમેટાલિક સ્ટ્રિપ};
    \node[right] at (1.8, 1) {મેગ્નેટિક કોઇલ};
\end{tikzpicture}
\captionof{figure}{MCB આંતરિક રચના}
\end{center}

\textbf{કોમ્પોનન્ટ્સ:}
\begin{itemize}
    \item \keyword{ફિક્સ્ડ અને મૂવિંગ કોન્ટેક્ટ્સ}: કરંટ વહન કરતા ભાગો
    \item \keyword{બાઇમેટાલિક સ્ટ્રિપ}: થર્મલ પ્રોટેક્શન
    \item \keyword{ઇલેક્ટ્રોમેગ્નેટિક કોઇલ}: મેગ્નેટિક પ્રોટેક્શન
    \item \keyword{આર્ક ક્વેન્ચિંગ ચેમ્બર}: આર્ક એક્સ્ટિન્કશન
\end{itemize}

\textbf{કાર્ય સિદ્ધાંત:}
\begin{itemize}
    \item \textbf{ઓવરલોડ પ્રોટેક્શન}: કરંટ બાઇમેટાલિક સ્ટ્રિપ ગરમ કરે છે, જે વળીને ટ્રિપ કરે છે.
    \item \textbf{શોર્ટ સર્કિટ પ્રોટેક્શન}: ઉચ્ચ કરંટ મજબૂત મેગ્નેટિક ફીલ્ડ બનાવે છે જે ત્વરિત ટ્રિપ કરે છે.
\end{itemize}

\textbf{ફાયદાઓ:}
\begin{itemize}
    \item \keyword{પુનઃઉપયોગ}: ફોલ્ટ ક્લિયરન્સ પછી રીસેટ
    \item \keyword{વિશ્વસનીય ઓપરેશન}: ડ્યુઅલ પ્રોટેક્શન મેકેનિઝમ
\end{itemize}
\end{solutionbox}

\begin{mnemonicbox}
\mnemonic{"MCB મેગ્નેટિકલી કન્ટ્રોલ્સ બોથ" (થર્મલ અને મેગ્નેટિક પ્રોટેક્શન)}
\end{mnemonicbox}

\questionmarks{5(a OR)}{3}{શુદ્ધ ઇન્ડક્ટર ધરાવતા પરિપથ માંથી અલ્ટરનેટિંગ કરંટની વર્તણૂક સમજાવો.}

\begin{solutionbox}
\textbf{જવાબ:}

\textbf{આપેલ:} L ઇન્ડક્ટન્સ સાથે શુદ્ધ ઇન્ડક્ટર, લાગુ વોલ્ટેજ $v = V_m \sin(\omega t)$

\textbf{વોલ્ટેજ-કરંટ સંબંધ:}
\[ v = L \times \frac{di}{dt} \]

\textbf{ઇન્ટીગ્રેશન:}
\[ i = -\frac{V_m}{\omega L} \cos(\omega t) = \frac{V_m}{\omega L} \sin(\omega t - 90^\circ) \]

\textbf{શુદ્ધ ઇન્ડક્ટર લક્ષણો:}
\begin{center}
\captionof{table}{શુદ્ધ ઇન્ડક્ટર લક્ષણો}
\begin{tabulary}{\linewidth}{|L|L|L|}
\hline
\textbf{પેરામીટર} & \textbf{મૂલ્ય} & \textbf{ફેઝ સંબંધ} \\ \hline
\textbf{કરંટ એમ્પ્લિટ્યૂડ} & $I_m = V_m/\omega L$ & કરંટ વોલ્ટેજથી 90$^\circ$ પાછળ \\ \hline
\textbf{ઇન્ડક્ટિવ રિએક્ટન્સ} & $X_L = \omega L = 2\pi fL$ & ફ્રિક્વન્સી આધારિત \\ \hline
\textbf{પાવર} & $P = 0$ (એવરેજ) & કોઈ નેટ પાવર વપરાશ નથી \\ \hline
\end{tabulary}
\end{center}
\end{solutionbox}

\begin{mnemonicbox}
\mnemonic{"ઇન્ડક્ટર ઇમ્પીડ્સ, કરંટ લેગ્સ" (XL કરંટનો વિરોધ, 90° લેગ)}
\end{mnemonicbox}

\questionmarks{5(b OR)}{4}{AC સર્કિટમાં પાવર અને પાવર ટ્રાયએંગલ સમજાવો.}

\begin{solutionbox}
\textbf{જવાબ:}

\textbf{AC પાવર કોમ્પોનન્ટ્સ:}
\begin{center}
\captionof{table}{AC પાવર કોમ્પોનન્ટ્સ}
\begin{tabulary}{\linewidth}{|L|L|L|L|L|}
\hline
\textbf{પાવરનો પ્રકાર} & \textbf{પ્રતીક} & \textbf{સૂત્ર} & \textbf{એકમો} & \textbf{વર્ણન} \\ \hline
\textbf{એક્ટિવ પાવર} & P & $VI \cos\phi$ & વોટ & ઉપયોગી પાવર \\ \hline
\textbf{રિએક્ટિવ પાવર} & Q & $VI \sin\phi$ & VAR & પરિભ્રમણ પાવર \\ \hline
\textbf{એપેરન્ટ પાવર} & S & $VI$ & VA & કુલ પાવર \\ \hline
\end{tabulary}
\end{center}

\textbf{પાવર ત્રિકોણ:}
\begin{center}
\begin{tikzpicture}
    \draw (0,0) -- (3,0) node[midway, below] {P ($kW$)};
    \draw (3,0) -- (3,2) node[midway, right] {Q ($kVAR$)};
    \draw (0,0) -- (3,2) node[midway, above left] {S ($kVA$)};
    \node at (0.6, 0.3) {$\phi$};
\end{tikzpicture}
\captionof{figure}{પાવર ત્રિકોણ}
\end{center}

\textbf{ગાણિતિક સંબંધો:}
\[ S^2 = P^2 + Q^2 \]
\[ \text{Power Factor} = P/S = \cos\phi \]
\end{solutionbox}

\begin{mnemonicbox}
\mnemonic{"પાવર ટ્રાયએંગલ: પ્લીઝ ક્વાલિફાય સ્ટુડન્ટ્સ" (P, Q, S કોમ્પોનન્ટ્સ)}
\end{mnemonicbox}

\questionmarks{5(c OR)}{7}{એક લેમ્પને એક જગ્યાએથી કન્ટ્રોલ કરવો તેમજ દાદર માટેનું વાયરિંગ ડાયાગ્રામ સાથે સમજાવો.}

\begin{solutionbox}
\textbf{જવાબ:}

\textbf{1. એક જગ્યાએથી લેમ્પ કન્ટ્રોલ:}

\textbf{સર્કિટ આકૃતિ:}
\begin{center}
\begin{tikzpicture}
    \node[left] at (0,2) {Live};
    \draw (0,2) -- (1,2) to[switch] (3,2) to[lamp] (5,2) -- (5,0);
    \node[left] at (0,0) {Neutral};
    \draw (0,0) -- (5,0);
\end{tikzpicture}
\captionof{figure}{વન-વે કન્ટ્રોલ}
\end{center}

\textbf{કોમ્પોનન્ટ્સ:}
\begin{itemize}
    \item \keyword{SPST સ્વિચ}: સિંગલ પોલ, સિંગલ થ્રો
    \item \keyword{લાઇવ વાયર કન્ટ્રોલ}: સુરક્ષા માટે સ્વિચ લાઇવ વાયરમાં
\end{itemize}

\textbf{2. સીડીનું વાયરિંગ (ટુ-વે કન્ટ્રોલ):}

\textbf{સર્કિટ આકૃતિ:}
\begin{center}
\begin{tikzpicture}
    \node[left] at (0,2) {Live};
    \draw (0,2) -- (1,2);
    % Switch 1
    \draw (1,2) -- (2,2.5); 
    \draw (1,2) -- (2,1.5);
    \fill (1,2) circle (2pt);
    \draw[dotted] (1,2) -- (2,2.5); % Position 1
    
    % Strappers
    \draw (2,2.5) -- (4,2.5);
    \draw (2,1.5) -- (4,1.5);
    
    % Switch 2
    \draw (4,2.5) -- (5,2);
    \draw (4,1.5) -- (5,2);
    \fill (5,2) circle (2pt);
    
    % Lamp to Neutral
    \draw (5,2) to[lamp] (7,2) -- (7,0);
    \node[left] at (0,0) {Neutral};
    \draw (0,0) -- (7,0);
\end{tikzpicture}
\captionof{figure}{ટુ-વે કન્ટ્રોલ}
\end{center}

\textbf{સીડીના કન્ટ્રોલ માટે સ્વિચ પોઝિશન:}
\begin{center}
\captionof{table}{સીડીના કન્ટ્રોલ માટે સ્વિચ પોઝિશન}
\begin{tabulary}{\linewidth}{|L|L|L|}
\hline
\textbf{S1 પોઝિશન} & \textbf{S2 પોઝિશન} & \textbf{લેમ્પ સ્ટેટસ} \\ \hline
\textbf{ઉપર} & ઉપર & ચાલુ \\ \hline
\textbf{ઉપર} & નીચે & બંધ \\ \hline
\textbf{નીચે} & ઉપર & બંધ \\ \hline
\textbf{નીચે} & નીચે & ચાલુ \\ \hline
\end{tabulary}
\end{center}

\textbf{ફાયદાઓ:}
\begin{itemize}
    \item \keyword{સુવિધા}: અનેક સ્થળોએથી કન્ટ્રોલ
    \item \keyword{સુરક્ષા}: અંધારામાં ચાલવાની જરૂર નથી
\end{itemize}
\end{solutionbox}

\begin{mnemonicbox}
\mnemonic{"ટુ-વે ટોગલ્સ, ટુ પ્લેસિસ" (બે સ્વિચ, બે સ્થળો)}
\end{mnemonicbox}

\end{document}
