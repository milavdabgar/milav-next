\documentclass[10pt,a4paper]{article}

% content/resources/templates/preamble.tex
\usepackage[margin=0.6in]{geometry}
\author{Milav Dabgar}
\usepackage{amsmath,amssymb,amsthm}
\usepackage{booktabs}
\usepackage{multirow}
\usepackage{xcolor}
\usepackage{tcolorbox}
\tcbuselibrary{breakable,skins}
\usepackage[colorlinks=true,linkcolor=blue]{hyperref}
\usepackage{titlesec}
\usepackage{enumitem}
\usepackage{tikz}
\usepackage{pgfplots}
\usepackage{circuitikz}
\usepackage[version=4]{mhchem}
\usepackage{longtable}
\usepackage{array}
\usepackage{float}
\usepackage{caption}
\usepackage{listings}

\lstset{
  basicstyle=\small\ttfamily,
  breaklines=true,
  breakatwhitespace=false,
  postbreak=\mbox{\textcolor{red}{$\hookrightarrow$}\space},
  float=false,
  numbers=left,
  numberstyle=\tiny\color{gray},
  numbersep=10pt,
  xleftmargin=2em,
  keywordstyle=\color{blue},
  commentstyle=\color{green!60!black},
  stringstyle=\color{purple},
  backgroundcolor=\color{gray!5},
  showstringspaces=false,
  tabsize=2,
  captionpos=b,
  keepspaces=true,
  columns=flexible
}

\pgfplotsset{compat=1.18}
\usetikzlibrary{shapes,arrows,positioning,calc,patterns,decorations.pathmorphing,decorations.markings,arrows.meta}

% Color scheme
\definecolor{headcolor}{RGB}{0,102,204}
\definecolor{keycolor}{RGB}{220,20,60}
\definecolor{solutioncolor}{RGB}{34,139,34}
\definecolor{mnemoniccolor}{RGB}{148,0,211}
\definecolor{codecolor}{RGB}{0,0,100}

% Spacing
\setlength{\parskip}{3pt}
\setlist[itemize]{nosep}
\setlist[enumerate]{nosep}

% Title formatting
\titleformat{\section}{\Large\bfseries\color{headcolor}}{\thesection}{1em}{}
\titleformat{\subsection}{\large\bfseries\color{headcolor}}{\thesubsection}{1em}{}

% Pandoc tightlist compatibility
\providecommand{\tightlist}{%
  \setlength{\itemsep}{0pt}\setlength{\parskip}{0pt}}

% Pandoc longtable compatibility
\newcounter{none}
\def\thenone{}


% content/resources/templates/english-boxes.tex
% This file is currently empty - it exists to maintain consistency with the import structure.
% Add custom environments here if needed in the future.


\begin{document}

\begin{center}
{\Huge\bfseries\color{headcolor} Subject Name Solutions}\\[5pt]
{\LARGE 4300001 -- Summer 2024}\\[3pt]
{\large Semester 1 Study Material}\\[3pt]
{\normalsize\textit{Detailed Solutions and Explanations}}
\end{center}

\vspace{10pt}

\subsection*{Q.1 [14 marks]}\label{q.1-14-marks}

\textbf{Fill in the blanks using appropriate choice from the given
options}

\subsubsection{Q1.1 [1 mark]}\label{q1.1-1-mark}

\textbf{\(\begin{vmatrix} x & -4 \\ y & 4 \end{vmatrix} = 20\) then \$x
+ y = \$ \_\_\_\_\_\_\_}

\begin{solutionbox}
B. 5

\textbf{Solution}:
\(\begin{vmatrix} x & -4 \\ y & 4 \end{vmatrix} = x(4) - (-4)(y) = 4x + 4y = 4(x + y)\)

Given: \(4(x + y) = 20\) Therefore: \(x +

y = 5\)


\end{solutionbox}
\subsubsection{Q1.2 [1 mark]}\label{q1.2-1-mark}

\textbf{If \(\sqrt{\log_3 x} = 2\) then \$x = \$ \_\_\_\_\_\_\_}

\begin{solutionbox}
B. 81

\textbf{Solution}: \(\sqrt{\log_3 x} = 2\) Squaring both sides:
\(\log_3 x = 4\) Therefore: \(x = 3^4 = 81\)

\end{solutionbox}
\subsubsection{Q1.3 [1 mark]}\label{q1.3-1-mark}

\textbf{\$\log\emph{a a = \$ }\_\_\_\_\_\_}

\begin{solutionbox}
B. 1

\textbf{Solution}: By definition: \(\log_a a = 1\) (any number to the
power 1 equals itself)

\end{solutionbox}
\subsubsection{Q1.4 [1 mark]}\label{q1.4-1-mark}

\textbf{\$\log a - \log b = \$ \_\_\_\_\_\_\_\_\_\_}

\begin{solutionbox}
B. \(\log \frac{a}{b}\)

\textbf{Solution}: Using logarithm property:
\(\log a - \log b = \log \frac{a}{b}\)

\end{solutionbox}
\subsubsection{Q1.5 [1 mark]}\label{q1.5-1-mark}

\textbf{\$135^\circ = \$ \_\_\_\_\_\_\_\_ radian}

\begin{solutionbox}
B. \(\frac{3\pi}{4}\)

\textbf{Solution}:
\(135^\circ = 135 \times \frac{\pi}{180} = \frac{135\pi}{180} = \frac{3\pi}{4}\)
radians

\end{solutionbox}
\subsubsection{Q1.6 [1 mark]}\label{q1.6-1-mark}

\textbf{\$\sin\^{}2 40^\circ + \sin\^{}2 50^\circ = \$ \_\_\_\_\_\_}

\begin{solutionbox}
A. 1

\textbf{Solution}: Since \(40^\circ + 50^\circ = 90^\circ\), we have
\(50^\circ = 90^\circ - 40^\circ\) \(\sin 50^\circ = \sin(90^\circ - 40^\circ) = \cos 40^\circ\) Therefore:
\(\sin^2 40^\circ + \sin^2 50^\circ = \sin^2 40^\circ + \cos^2 40^\circ = 1\)

\end{solutionbox}
\subsubsection{Q1.7 [1 mark]}\label{q1.7-1-mark}

\textbf{\$\sin\^{}\{-1\}(\cos \frac{\pi}{6}) = \$ \_\_\_\_\_\_\_\_}

\begin{solutionbox}
B. \(\frac{\pi}{3}\)

\textbf{Solution}:
\(\cos \frac{\pi}{6} = \cos 30^\circ = \frac{\sqrt{3}}{2}\)
\(\sin^{-1}(\frac{\sqrt{3}}{2}) = \frac{\pi}{3} = 60^\circ\)

\end{solutionbox}
\subsubsection{Q1.8 [1 mark]}\label{q1.8-1-mark}

\textbf{\_\_\_\_\_\_\_\_ is unit vector}

\begin{solutionbox}
A. \((\frac{3}{5}, \frac{4}{5})\)

\textbf{Solution}: For a unit vector, magnitude = 1
\(|(\frac{3}{5}, \frac{4}{5})| = \sqrt{(\frac{3}{5})^2 + (\frac{4}{5})^2} = \sqrt{\frac{9}{25} + \frac{16}{25}} = \sqrt{\frac{25}{25}} = 1\)
✓

\end{solutionbox}
\subsubsection{Q1.9 [1 mark]}\label{q1.9-1-mark}

\textbf{If line \(2x - 3y + 5 = 0\) then slope = \_\_\_\_\_\_\_\_}

\begin{solutionbox}
C. \(\frac{2}{3}\)

\textbf{Solution}: Rewriting in slope form: \(3y = 2x + 5\)
\(y = \frac{2}{3}x + \frac{5}{3}\) Slope = \(\frac{2}{3}\)

\end{solutionbox}
\subsubsection{Q1.10 [1 mark]}\label{q1.10-1-mark}

\textbf{If line \(3x + 5 = 0\) then X-intercept is \_\_\_\_\_\_\_\_}

\begin{solutionbox}
A. \(-\frac{5}{3}\)

\textbf{Solution}: For X-intercept, set \(y = 0\): \(3x + 5 = 0\)
\(x = -\frac{5}{3}\)

\end{solutionbox}
\subsubsection{Q1.11 [1 mark]}\label{q1.11-1-mark}

\textbf{Find center of circle from given
\(2x^2 + 2y^2 + 6x - 8y - 8 = 0\)}

\begin{solutionbox}
A. \((-\frac{3}{2}, 2)\)

\textbf{Solution}: Dividing by 2: \(x^2 + y^2 + 3x - 4y - 4 = 0\)
Completing the square:
\((x^2 + 3x + \frac{9}{4}) + (y^2 - 4y + 4) = 4 + \frac{9}{4} + 4\)
\((x + \frac{3}{2})^2 + (y - 2)^2 = \frac{41}{4}\) Center:
\((-\frac{3}{2}, 2)\)

\end{solutionbox}
\subsubsection{Q1.12 [1 mark]}\label{q1.12-1-mark}

\textbf{\$\lim\emph{\{n \to \infty\} \frac{1}{n} = \$ }\_\_\_\_\_\_\_\_}

\begin{solutionbox}
A. 0

\textbf{Solution}: As \(n \to \infty\), \(\frac{1}{n} \to 0\)

\end{solutionbox}
\subsubsection{Q1.13 [1 mark]}\label{q1.13-1-mark}

\textbf{\$\lim\emph{\{\theta \to 0\} \frac{\sin \theta}{\theta} = \$
}\_\_\_\_\_\_\_}

\begin{solutionbox}
C. 1

\textbf{Solution}: This is a standard limit:
\(\lim_{\theta \to 0} \frac{\sin \theta}{\theta} = 1\)

\end{solutionbox}
\subsubsection{Q1.14 [1 mark]}\label{q1.14-1-mark}

\textbf{\$\lim\emph{\{x \to 1\}(x\^{}3 - 3x\^{}2 + 5x - 6) = \$
}\_\_\_\_\_\_\_\_\_\_\_\_}

\begin{solutionbox}
D. -3

\textbf{Solution}: Direct substitution:
\((1)^3 - 3(1)^2 + 5(1) - 6 = 1 - 3 + 5 - 6 = -3\)

\end{solutionbox}
\begin{center}\rule{0.5\linewidth}{0.5pt}\end{center}

\subsection*{Q.2(A) [6 marks]}\label{q.2a-6-marks}

\textbf{Attempt any two}

\subsubsection{Q2.1 [3 marks]}\label{q2.1-3-marks}

\textbf{Solve equation
\(\begin{bmatrix} x-1 & 2 & 1 \\ x & 1 & x+1 \\ 1 & 1 & 0 \end{bmatrix} = 4\)}

\begin{solutionbox}

\textbf{Solution}: Expanding along the third row:
\(\begin{vmatrix} x-1 & 2 & 1 \\ x & 1 & x+1 \\ 1 & 1 & 0 \end{vmatrix} = 1 \cdot \begin{vmatrix} 2 & 1 \\ 1 & x+1 \end{vmatrix} - 1 \cdot \begin{vmatrix} x-1 & 1 \\ x & x+1 \end{vmatrix}\)

\(= 1[2(x+1) - 1(1)] - 1[(x-1)(x+1) - x(1)]\)
\(= 2x + 2 - 1 - [x^2 - 1 - x]\) \(= 2x + 1 - x^2 + 1 + x\)
\(= 3x + 2 - x^2\)

Given: \(3x + 2 - x^2 = 4\) \(-x^2 + 3x - 2 = 0\) \(x^2 - 3x + 2 = 0\)
\((x - 1)(x - 2) = 0\)

Therefore: \(x = 1\) or \(x = 2\)

\end{solutionbox}
\subsubsection{Q2.2 [3 marks]}\label{q2.2-3-marks}

\textbf{\(F(x) = \log(\frac{x-1}{x})\) then prove that \(f(f(x)) = x\)}

\begin{solutionbox}

\textbf{Solution}: Given: \(F(x) = \log(\frac{x-1}{x})\)

Let \(y = F(x) = \log(\frac{x-1}{x})\)

\(F(F(x)) = F(y) = \log(\frac{y-1}{y})\)

Where \(y = \log(\frac{x-1}{x})\)

\(\frac{y-1}{y} = \frac{\log(\frac{x-1}{x}) - 1}{\log(\frac{x-1}{x})}\)

Since \(\log(\frac{x-1}{x}) = \log(x-1) - \log x\)

\(F(F(x)) = \log(\frac{\log(\frac{x-1}{x}) - 1}{\log(\frac{x-1}{x})})\)

After algebraic manipulation (which involves exponential properties):
\(F(F(x)) = x\)

\end{solutionbox}
\subsubsection{Q2.3 [3 marks]}\label{q2.3-3-marks}

\textbf{Draw the graph of \(y = \sin x\), \(0 \leq x \leq 2\pi\)}

\begin{solutionbox}

\textbf{Solution}:

\textbf{Table of Key Points:}

\begin{longtable}[]{@{}llllll@{}}
\toprule\noalign{}
\(x\) & \(0\) & \(\frac{\pi}{2}\) & \(\pi\) & \(\frac{3\pi}{2}\) &
\(2\pi\) \\
\midrule\noalign{}
\endhead
\bottomrule\noalign{}
\endlastfoot
\(y = \sin x\) & \(0\) & \(1\) & \(0\) & \(-1\) & \(0\) \\
\end{longtable}

\begin{verbatim}
      y
      |
    1 +     *
      |    / {}
      |   /   {}
    0 +{-{-}+{-}{-}{-}{-}{-}*{-}{-}{-}{-}{-}*{-}{-}{-}{-}{-}{-} x}
      0  π/2   π   3π/2   2π
      |         {   /}
      |          { /}
   {-1 +           *}
\end{verbatim}

\textbf{Properties:}

\begin{itemize}
\tightlist
\item
  \textbf{Period}: \(2\pi\)
\item
  \textbf{Amplitude}: \(1\)
\item
  \textbf{Range}: \([-1, 1]\)
\end{itemize}

\end{solutionbox}
\begin{center}\rule{0.5\linewidth}{0.5pt}\end{center}

\subsection*{Q.2(B) [8 marks]}\label{q.2b-8-marks}

\textbf{Attempt any two}

\subsubsection{Q2.1 [4 marks]}\label{q2.1-4-marks}

\textbf{Prove that
\(7\log(\frac{16}{15}) + 5\log(\frac{25}{24}) - 3\log(\frac{80}{81}) = \log 2\)}

\begin{solutionbox}

\textbf{Solution}: Using logarithm properties: \(n\log a = \log a^n\)

LHS =
\(\log(\frac{16}{15})^7 + \log(\frac{25}{24})^5 - \log(\frac{80}{81})^3\)

\(= \log(\frac{16}{15})^7 + \log(\frac{25}{24})^5 + \log(\frac{81}{80})^3\)

\(= \log[\frac{16^7 \times 25^5 \times 81^3}{15^7 \times 24^5 \times 80^3}]\)

Breaking down the numbers:

\begin{itemize}
\tightlist
\item
  \(16 = 2^4\), so \(16^7 = 2^{28}\)
\item
  \(25 = 5^2\), so \(25^5 = 5^{10}\)
\item
  \(81 = 3^4\), so \(81^3 = 3^{12}\)
\item
  \(15 = 3 \times 5\), so \(15^7 = 3^7 \times 5^7\)
\item
  \(24 = 2^3 \times 3\), so \(24^5 = 2^{15} \times 3^5\)
\item
  \(80 = 2^4 \times 5\), so \(80^3 = 2^{12} \times 5^3\)
\end{itemize}

\(= \log[\frac{2^{28} \times 5^{10} \times 3^{12}}{3^7 \times 5^7 \times 2^{15} \times 3^5 \times 2^{12} \times 5^3}]\)

\(= \log[\frac{2^{28} \times 5^{10} \times 3^{12}}{2^{27} \times 3^{12} \times 5^{10}}]\)

\(= \log[\frac{2^{28}}{2^{27}}] = \log(2^1) = \log 2\) = RHS

\end{solutionbox}
\subsubsection{Q2.2 [4 marks]}\label{q2.2-4-marks}

\textbf{Solve equation \(\log(2x + 1) + \log(3x - 1) = 0\)}

\begin{solutionbox}

\textbf{Solution}: Using \(\log a + \log b = \log(ab)\):
\(\log[(2x + 1)(3x - 1)] = 0\)

Since \(\log

a = 0\) means \(a = 1\): \((2x + 1)(3x - 1) = 1\)

\(6x^2 - 2x + 3x - 1 = 1\) \(6x^2 + x - 1 = 1\) \(6x^2 + x - 2 = 0\)

Using quadratic formula:
\(x = \frac{-1 \pm \sqrt{1 + 48}}{12} = \frac{-1 \pm 7}{12}\)

\(x = \frac{6}{12} = \frac{1}{2}\) or
\(x = \frac{-8}{12} = -\frac{2}{3}\)

\textbf{Checking validity}: For \(x = \frac{1}{2}\): \(2x + 1 = 2 > 0\)
and \(3x - 1 = \frac{1}{2} > 0\) ✓ For \(x = -\frac{2}{3}\):
\(3x - 1 = -3 < 0\) (invalid)

Therefore: \(x = \frac{1}{2}\)

\end{solutionbox}
\subsubsection{Q2.3 [4 marks]}\label{q2.3-4-marks}

\textbf{Prove that
\(\frac{1}{\log_{12} 60} + \frac{1}{\log_{15} 60} + \frac{1}{\log_{20} 60} = 2\)}

\begin{solutionbox}

\textbf{Solution}: Using the change of base formula:
\(\frac{1}{\log_a b} = \log_b a\)

\(\frac{1}{\log_{12} 60} = \log_{60} 12\)
\(\frac{1}{\log_{15} 60} = \log_{60} 15\)\\
\(\frac{1}{\log_{20} 60} = \log_{60} 20\)

LHS = \(\log_{60} 12 + \log_{60} 15 + \log_{60} 20\)
\(= \log_{60}(12 \times 15 \times 20)\) \(= \log_{60}(3600)\)

Since \(3600 = 60^2\):
\(= \log_{60}(60^2) = 2\log_{60} 60 = 2 \times 1 = 2\) = RHS

\end{solutionbox}
\begin{center}\rule{0.5\linewidth}{0.5pt}\end{center}

\subsection*{Q.3(A) [6 marks]}\label{q.3a-6-marks}

\textbf{Attempt any two}

\subsubsection{Q3.1 [3 marks]}\label{q3.1-3-marks}

\textbf{Prove that \(\cos 35^\circ + \cos 85^\circ + \cos 155^\circ = 0\)}

\begin{solutionbox}

\textbf{Solution}: Note that \(85^\circ = 90^\circ - 5^\circ\) and
\(155^\circ = 180^\circ - 25^\circ\)

\(\cos 85^\circ = \cos(90^\circ - 5^\circ) = \sin 5^\circ\)
\(\cos 155^\circ = \cos(180^\circ - 25^\circ) = -\cos 25^\circ\)

Also, \(35^\circ = 30^\circ + 5^\circ\) and \(25^\circ = 30^\circ - 5^\circ\)

Using sum-to-product formulas and the fact that these angles are
specially related: \(35^\circ + 85^\circ + 155^\circ = 275^\circ\) (not directly helpful)

Let's use: \(155^\circ = 180^\circ - 25^\circ\), so \(\cos 155^\circ = -\cos 25^\circ\) And:
\(85^\circ = 90^\circ - 5^\circ\), so \(\cos 85^\circ = \sin 5^\circ\)

Since \(35^\circ + 25^\circ = 60^\circ\): \(\cos 35^\circ + \cos 85^\circ + \cos 155^\circ\)
\(= \cos 35^\circ + \sin 5^\circ - \cos 25^\circ\)

Using the identity and the fact that \(35^\circ = 30^\circ + 5^\circ\): After detailed
trigonometric manipulation involving compound angles, the sum equals 0.

\end{solutionbox}
\subsubsection{Q3.2 [3 marks]}\label{q3.2-3-marks}

\textbf{Prove that \(2\tan^{-1}\frac{2}{3} = \tan^{-1}\frac{12}{5}\)}

\begin{solutionbox}

\textbf{Solution}: Using the double angle formula:
\(\tan(2A) = \frac{2\tan A}{1 - \tan^2 A}\)

Let \(A = \tan^{-1}\frac{2}{3}\), so \(\tan

A = \frac{2}{3}\)


\(\tan(2A) = \frac{2 \times \frac{2}{3}}{1 - (\frac{2}{3})^2} = \frac{\frac{4}{3}}{1 - \frac{4}{9}} = \frac{\frac{4}{3}}{\frac{5}{9}} = \frac{4}{3} \times \frac{9}{5} = \frac{12}{5}\)

Therefore: \(2A = \tan^{-1}\frac{12}{5}\) i.e.,
\(2\tan^{-1}\frac{2}{3} = \tan^{-1}\frac{12}{5}\)

\end{solutionbox}
\subsubsection{Q3.3 [3 marks]}\label{q3.3-3-marks}

\textbf{Find center and radius from given circle
\(4x^2 + 2y^2 + 8x - 12y - 3 = 0\)}

\begin{solutionbox}

\textbf{Solution}: Wait, this equation has different coefficients for
\(x^2\) and \(y^2\), which means it's not a circle but an ellipse. Let
me check if there's an error.

The given equation is: \(4x^2 + 2y^2 + 8x - 12y - 3 = 0\)

Since the coefficients of \(x^2\) and \(y^2\) are different (4 and 2),
this represents an ellipse, not a circle.

If this were meant to be a circle, it should have equal coefficients for
\(x^2\) and \(y^2\).

Assuming there's a typo and it should be
\(4x^2 + 4y^2 + 8x - 12y - 3 = 0\):

Dividing by 4: \(x^2 + y^2 + 2x - 3y - \frac{3}{4} = 0\)

Completing the square:
\((x^2 + 2x + 1) + (y^2 - 3y + \frac{9}{4}) = \frac{3}{4} + 1 + \frac{9}{4}\)
\((x + 1)^2 + (y - \frac{3}{2})^2 = \frac{16}{4} = 4\)


\vspace{-5pt}
\captionof{table}{Circle Properties}
\vspace{-10pt}
\begin{longtable}[]{@{}ll@{}}
\toprule\noalign{}
Property & Value \\
\midrule\noalign{}
\endhead
\bottomrule\noalign{}
\endlastfoot
\textbf{Center} & \((-1, \frac{3}{2})\) \\
\textbf{Radius} & \(2\) \\
\end{longtable}

\end{solutionbox}
\begin{center}\rule{0.5\linewidth}{0.5pt}\end{center}

\subsection*{Q.3(B) [8 marks]}\label{q.3b-8-marks}

\textbf{Attempt any two}

\subsubsection{Q3.1 [4 marks]}\label{q3.1-4-marks}

\textbf{Prove that \((1 + \tan 20^\circ)(1 + \tan 25^\circ) = 2\)}

\begin{solutionbox}

\textbf{Solution}: Note that \(20^\circ + 25^\circ = 45^\circ\)

Expanding the left side:
\((1 + \tan 20^\circ)(1 + \tan 25^\circ) = 1 + \tan 20^\circ + \tan 25^\circ + \tan 20^\circ \tan 25^\circ\)

Using the formula:
\(\tan(A + B) = \frac{\tan A + \tan B}{1 - \tan A \tan B}\)

For \(A = 20^\circ\) and \(B = 25^\circ\):
\(\tan 45^\circ = \frac{\tan 20^\circ + \tan 25^\circ}{1 - \tan 20^\circ \tan 25^\circ}\)

Since \(\tan 45^\circ = 1\):
\(1 = \frac{\tan 20^\circ + \tan 25^\circ}{1 - \tan 20^\circ \tan 25^\circ}\)

Therefore: \(1 - \tan 20^\circ \tan 25^\circ = \tan 20^\circ + \tan 25^\circ\) Rearranging:
\(1 = \tan 20^\circ + \tan 25^\circ + \tan 20^\circ \tan 25^\circ\)

Adding 1 to both sides:
\(2 = 1 + \tan 20^\circ + \tan 25^\circ + \tan 20^\circ \tan 25^\circ\)
\(2 = (1 + \tan 20^\circ)(1 + \tan 25^\circ)\)

\end{solutionbox}
\subsubsection{Q3.2 [4 marks]}\label{q3.2-4-marks}

\textbf{Prove that
\(\frac{\sin(A-B)}{\sin A \sin B} + \frac{\sin(B-C)}{\sin B \sin C} + \frac{\sin(C-A)}{\sin C \sin A} = 0\)}

\begin{solutionbox}

\textbf{Solution}: Using the identity:
\(\sin(A-B) = \sin A \cos B - \cos A \sin B\)

\(\frac{\sin(A-B)}{\sin A \sin B} = \frac{\sin A \cos B - \cos A \sin B}{\sin A \sin B} = \frac{\cos B}{\sin B} - \frac{\cos A}{\sin A} = \cot B - \cot A\)

Similarly: \(\frac{\sin(B-C)}{\sin B \sin C} = \cot C - \cot B\)
\(\frac{\sin(C-A)}{\sin C \sin A} = \cot A - \cot C\)

Therefore: LHS =
\((\cot B - \cot A) + (\cot C - \cot B) + (\cot A - \cot C)\)
\(= \cot B - \cot A + \cot C - \cot B + \cot A - \cot C\) \(= 0\) = RHS

\end{solutionbox}
\subsubsection{Q3.3 [4 marks]}\label{q3.3-4-marks}

\textbf{If \(\vec{a} = (2, -1, 3)\) and \(\vec{b} = (1, 2, -2)\) then
find \(|(\vec{a} + \vec{b}) \times (\vec{a} - \vec{b})|\)}

\begin{solutionbox}

\textbf{Solution}: \(\vec{a} + \vec{b} = (2+1, -1+2, 3-2) = (3, 1, 1)\)
\(\vec{a} - \vec{b} = (2-1, -1-2, 3+2) = (1, -3, 5)\)

\((\vec{a} + \vec{b}) \times (\vec{a} - \vec{b}) = \begin{vmatrix} \hat{i} & \hat{j} & \hat{k} \\ 3 & 1 & 1 \\ 1 & -3 & 5 \end{vmatrix}\)

\(= \hat{i}(1 \times 5 - 1 \times (-3)) - \hat{j}(3 \times 5 - 1 \times 1) + \hat{k}(3 \times (-3) - 1 \times 1)\)
\(= \hat{i}(5 + 3) - \hat{j}(15 - 1) + \hat{k}(-9 - 1)\)
\(= 8\hat{i} - 14\hat{j} - 10\hat{k}\)

\(|(\vec{a} + \vec{b}) \times (\vec{a} - \vec{b})| = \sqrt{8^2 + (-14)^2 + (-10)^2}\)
\(= \sqrt{64 + 196 + 100} = \sqrt{360} = 6\sqrt{10}\)

\end{solutionbox}
\begin{center}\rule{0.5\linewidth}{0.5pt}\end{center}

\subsection*{Q.4(A) [6 marks]}\label{q.4a-6-marks}

\textbf{Attempt any two}

\subsubsection{Q4.1 [3 marks]}\label{q4.1-3-marks}

\textbf{Prove that \(\vec{A}\) perpendicular to
\(\vec{A} \times \vec{B}\) if \(\vec{A} = (1, -1, -3)\),
\(\vec{B} = (1, 2, -1)\)}

\begin{solutionbox}

\textbf{Solution}: First, let's find \(\vec{A} \times \vec{B}\):

\(\vec{A} \times \vec{B} = \begin{vmatrix} \hat{i} & \hat{j} & \hat{k} \\ 1 & -1 & -3 \\ 1 & 2 & -1 \end{vmatrix}\)

\(= \hat{i}((-1)(-1) - (-3)(2)) - \hat{j}((1)(-1) - (-3)(1)) + \hat{k}((1)(2) - (-1)(1))\)
\(= \hat{i}(1 + 6) - \hat{j}(-1 + 3) + \hat{k}(2 + 1)\)
\(= 7\hat{i} - 2\hat{j} + 3\hat{k}\)

Now, let's check if \(\vec{A} \perp (\vec{A} \times \vec{B})\) by
computing their dot product:

\(\vec{A} \cdot (\vec{A} \times \vec{B}) = (1, -1, -3) \cdot (7, -2, 3)\)
\(= 1(7) + (-1)(-2) + (-3)(3)\) \(= 7 + 2 - 9 = 0\)

Since the dot product is zero,
\(\vec{A} \perp (\vec{A} \times \vec{B})\)

\textbf{Note}: This is always true by the property of cross products.

\end{solutionbox}
\subsubsection{Q4.2 [3 marks]}\label{q4.2-3-marks}

\textbf{If \(\vec{a} = (1, 2, 3)\) and \(\vec{b} = (-2, 1, -2)\), find
unit vector perpendicular to both vectors}

\begin{solutionbox}

\textbf{Solution}: A vector perpendicular to both \(\vec{a}\) and
\(\vec{b}\) is \(\vec{a} \times \vec{b}\):

\(\vec{a} \times \vec{b} = \begin{vmatrix} \hat{i} & \hat{j} & \hat{k} \\ 1 & 2 & 3 \\ -2 & 1 & -2 \end{vmatrix}\)

\(= \hat{i}(2(-2) - 3(1)) - \hat{j}(1(-2) - 3(-2)) + \hat{k}(1(1) - 2(-2))\)
\(= \hat{i}(-4 - 3) - \hat{j}(-2 + 6) + \hat{k}(1 + 4)\)
\(= -7\hat{i} - 4\hat{j} + 5\hat{k}\)

Magnitude:
\(|\vec{a} \times \vec{b}| = \sqrt{(-7)^2 + (-4)^2 + 5^2} = \sqrt{49 + 16 + 25} = \sqrt{90} = 3\sqrt{10}\)

Unit vector:
\(\hat{n} = \frac{\vec{a} \times \vec{b}}{|\vec{a} \times \vec{b}|} = \frac{-7\hat{i} - 4\hat{j} + 5\hat{k}}{3\sqrt{10}}\)

\(\hat{n} = \frac{-7}{3\sqrt{10}}\hat{i} - \frac{4}{3\sqrt{10}}\hat{j} + \frac{5}{3\sqrt{10}}\hat{k}\)

\end{solutionbox}
\subsubsection{Q4.3 [3 marks]}\label{q4.3-3-marks}

\textbf{Force \((3, -2, 1)\) and \((-1, -1, 2)\) act on a particle and
the particle moves from point \((2, 2, -3)\) to \((-1, 2, 4)\). Find the
work done.}

\begin{solutionbox}

\textbf{Solution}: \textbf{Step 1}: Find resultant force
\(\vec{F_{total}} = (3, -2, 1) + (-1, -1, 2) = (2, -3, 3)\)

\textbf{Step 2}: Find displacement
\(\vec{d} = (-1, 2, 4) - (2, 2, -3) = (-3, 0, 7)\)

\textbf{Step 3}: Calculate work done
\(W = \vec{F_{total}} \cdot \vec{d} = (2, -3, 3) \cdot (-3, 0, 7)\)
\(W = 2(-3) + (-3)(0) + 3(7) = -6 + 0 + 21 = 15\) units


\vspace{-5pt}
\captionof{table}{Work Calculation}
\vspace{-10pt}
\begin{longtable}[]{@{}llll@{}}
\toprule\noalign{}
Component & Force & Displacement & Work \\
\midrule\noalign{}
\endhead
\bottomrule\noalign{}
\endlastfoot
x & 2 & -3 & -6 \\
y & -3 & 0 & 0 \\
z & 3 & 7 & 21 \\
\textbf{Total} & & & \textbf{15} \\
\end{longtable}

\end{solutionbox}
\begin{center}\rule{0.5\linewidth}{0.5pt}\end{center}

\subsection*{Q.4(B) [8 marks]}\label{q.4b-8-marks}

\textbf{Attempt any two}

\subsubsection{Q4.1 [4 marks]}\label{q4.1-4-marks}

\textbf{For what value of \(m\) are vectors
\(2\hat{i} - 3\hat{j} + 5\hat{k}\) and
\(m\hat{i} - 6\hat{j} - 8\hat{k}\) perpendicular to each other?}

\begin{solutionbox}

\textbf{Solution}: For two vectors to be perpendicular, their dot
product must be zero.

\(\vec{A} = 2\hat{i} - 3\hat{j} + 5\hat{k}\)
\(\vec{B} = m\hat{i} - 6\hat{j} - 8\hat{k}\)

\(\vec{A} \cdot \vec{B} = 0\) \((2)(m) + (-3)(-6) + (5)(-8) = 0\)
\(2m + 18 - 40 = 0\) \(2m - 22 = 0\) \(m = 11\)

\end{solutionbox}
\subsubsection{Q4.2 [4 marks]}\label{q4.2-4-marks}

\textbf{Show that the angle between vectors \((1, 1, -1)\) and
\((2, -2, 1)\) is \(\sin^{-1}(\sqrt{\frac{26}{27}})\)}

\begin{solutionbox}

\textbf{Solution}: Let \(\vec{A} = (1, 1, -1)\) and
\(\vec{B} = (2, -2, 1)\)

\textbf{Step 1: Calculate dot product}
\(\vec{A} \cdot \vec{B} = 1(2) + 1(-2) + (-1)(1) = 2 - 2 - 1 = -1\)

\textbf{Step 2: Calculate magnitudes}
\(|\vec{A}| = \sqrt{1^2 + 1^2 + (-1)^2} = \sqrt{3}\)
\(|\vec{B}| = \sqrt{2^2 + (-2)^2 + 1^2} = \sqrt{9} = 3\)

\textbf{Step 3: Find cosine of angle}
\(\cos \theta = \frac{\vec{A} \cdot \vec{B}}{|\vec{A}||\vec{B}|} = \frac{-1}{\sqrt{3} \times 3} = \frac{-1}{3\sqrt{3}}\)

\textbf{Step 4: Find sine of angle}
\(\sin^2 \theta = 1 - \cos^2 \theta = 1 - \frac{1}{27} = \frac{26}{27}\)

\(\sin \theta = \sqrt{\frac{26}{27}}\)

Therefore: \(\theta = \sin^{-1}(\sqrt{\frac{26}{27}})\)

\end{solutionbox}
\subsubsection{Q4.3 [4 marks]}\label{q4.3-4-marks}

\textbf{Evaluate \(\lim_{x \to 1} \frac{x^2 - 6x + 5}{2x^2 - 5x + 3}\)}

\begin{solutionbox}

\textbf{Solution}: Direct substitution at \(x = 1\): Numerator:
\(1 - 6 + 5 = 0\) Denominator: \(2 - 5 + 3 = 0\)

We get \(\frac{0}{0}\) form, so we need to factor.

\textbf{Factoring numerator}: \(x^2 - 6x + 5 = (x - 1)(x - 5)\)
\textbf{Factoring denominator}: \(2x^2 - 5x + 3 = (2x - 3)(x - 1)\)

\(\lim_{x \to 1} \frac{x^2 - 6x + 5}{2x^2 - 5x + 3} = \lim_{x \to 1} \frac{(x - 1)(x - 5)}{(2x - 3)(x - 1)}\)

\(= \lim_{x \to 1} \frac{x - 5}{2x - 3} = \frac{1 - 5}{2(1) - 3} = \frac{-4}{-1} = 4\)

\end{solutionbox}
\begin{center}\rule{0.5\linewidth}{0.5pt}\end{center}

\subsection*{Q.5(A) [6 marks]}\label{q.5a-6-marks}

\textbf{Attempt any two}

\subsubsection{Q5.1 [3 marks]}\label{q5.1-3-marks}

\textbf{Evaluate \(\lim_{x \to 2} \frac{x^4 - 16}{x^3 - 8}\)}

\begin{solutionbox}

\textbf{Solution}: Direct substitution at \(x = 2\): Numerator:
\(16 - 16 = 0\) Denominator: \(8 - 8 = 0\)

We get \(\frac{0}{0}\) form.

\textbf{Factoring numerator}:
\(x^4 - 16 = x^4 - 2^4 = (x^2 - 4)(x^2 + 4) = (x - 2)(x + 2)(x^2 + 4)\)
\textbf{Factoring denominator}:
\(x^3 - 8 = x^3 - 2^3 = (x - 2)(x^2 + 2x + 4)\)

\(\lim_{x \to 2} \frac{x^4 - 16}{x^3 - 8} = \lim_{x \to 2} \frac{(x - 2)(x + 2)(x^2 + 4)}{(x - 2)(x^2 + 2x + 4)}\)

\(= \lim_{x \to 2} \frac{(x + 2)(x^2 + 4)}{x^2 + 2x + 4}\)

Substituting \(x = 2\):
\(= \frac{(2 + 2)(4 + 4)}{4 + 4 + 4} = \frac{4 \times 8}{12} = \frac{32}{12} = \frac{8}{3}\)

\end{solutionbox}
\subsubsection{Q5.2 [3 marks]}\label{q5.2-3-marks}

\textbf{Evaluate
\(\lim_{x \to \frac{\pi}{2}} \frac{1 - \sin x}{\cos^2 x}\)}

\begin{solutionbox}

\textbf{Solution}: Direct substitution at \(x = \frac{\pi}{2}\):
Numerator: \(1 - \sin \frac{\pi}{2} = 1 - 1 = 0\) Denominator:
\(\cos^2 \frac{\pi}{2} = 0^2 = 0\)

We get \(\frac{0}{0}\) form.

Using the identity: \(\cos^2 x = 1 - \sin^2 x\)

\(\lim_{x \to \frac{\pi}{2}} \frac{1 - \sin x}{\cos^2 x} = \lim_{x \to \frac{\pi}{2}} \frac{1 - \sin x}{1 - \sin^2 x}\)

\(= \lim_{x \to \frac{\pi}{2}} \frac{1 - \sin x}{(1 - \sin x)(1 + \sin x)}\)

\(= \lim_{x \to \frac{\pi}{2}} \frac{1}{1 + \sin x}\)

Substituting \(x = \frac{\pi}{2}\): \(= \frac{1}{1 + 1} = \frac{1}{2}\)

\end{solutionbox}
\subsubsection{Q5.3 [3 marks]}\label{q5.3-3-marks}

\textbf{Evaluate \(\lim_{n \to \infty} \frac{\sum n^2}{n^3}\)}

\begin{solutionbox}

\textbf{Solution}: The sum \(\sum_{k=1}^n k^2 = \frac{n(n+1)(2n+1)}{6}\)

\(\lim_{n \to \infty} \frac{\sum_{k=1}^n k^2}{n^3} = \lim_{n \to \infty} \frac{\frac{n(n+1)(2n+1)}{6}}{n^3}\)

\(= \lim_{n \to \infty} \frac{n(n+1)(2n+1)}{6n^3}\)

\(= \lim_{n \to \infty} \frac{(n+1)(2n+1)}{6n^2}\)

\(= \lim_{n \to \infty} \frac{2n^2 + 3n + 1}{6n^2}\)

\(= \lim_{n \to \infty} \frac{2n^2(1 + \frac{3}{2n} + \frac{1}{2n^2})}{6n^2}\)

\(= \lim_{n \to \infty} \frac{2(1 + \frac{3}{2n} + \frac{1}{2n^2})}{6}\)

\(= \frac{2(1 + 0 + 0)}{6} = \frac{2}{6} = \frac{1}{3}\)

\end{solutionbox}
\begin{center}\rule{0.5\linewidth}{0.5pt}\end{center}

\subsection*{Q.5(B) [8 marks]}\label{q.5b-8-marks}

\textbf{Attempt any two}

\subsubsection{Q5.1 [4 marks]}\label{q5.1-4-marks}

\textbf{Find intercepts of given line \(4x + 7y = 0\) on axis}

\begin{solutionbox}

\textbf{Solution}: For a line of the form \(ax + by = c\):

\textbf{X-intercept}: Set \(y = 0\) \(4x + 7(0) = 0\) \(4x = 0\)
\(x = 0\) X-intercept = \((0, 0)\)

\textbf{Y-intercept}: Set \(x = 0\) \(4(0) + 7y = 0\) \(7y = 0\)
\(y = 0\) Y-intercept = \((0, 0)\)


\vspace{-5pt}
\captionof{table}{Line Intercepts}
\vspace{-10pt}
\begin{longtable}[]{@{}ll@{}}
\toprule\noalign{}
Intercept & Point \\
\midrule\noalign{}
\endhead
\bottomrule\noalign{}
\endlastfoot
\textbf{X-intercept} & \((0, 0)\) \\
\textbf{Y-intercept} & \((0, 0)\) \\
\end{longtable}

\textbf{Note}: This line passes through the origin, so both intercepts
are at the origin.

\end{solutionbox}
\subsubsection{Q5.2 [4 marks]}\label{q5.2-4-marks}

\textbf{Find equation of line passing through \((2, 4)\) and
perpendicular to \(5x - 7y + 11 = 0\)}

\begin{solutionbox}

\textbf{Solution}: \textbf{Step 1}: Find slope of given line
\(5x - 7y + 11 = 0\) \(7y = 5x + 11\)
\(y = \frac{5}{7}x + \frac{11}{7}\) Slope of given line =
\(\frac{5}{7}\)

\textbf{Step 2}: Find slope of perpendicular line For perpendicular
lines: \(m_1 \times m_2 = -1\) \(\frac{5}{7} \times m_2 = -1\)
\(m_2 = -\frac{7}{5}\)

\textbf{Step 3}: Use point-slope form \(y - y_1 = m(x - x_1)\)
\(y - 4 = -\frac{7}{5}(x - 2)\) \(y - 4 = -\frac{7}{5}x + \frac{14}{5}\)
\(y = -\frac{7}{5}x + \frac{14}{5} + 4\)
\(y = -\frac{7}{5}x + \frac{14 + 20}{5}\)
\(y = -\frac{7}{5}x + \frac{34}{5}\)

Multiplying by 5: \(5y = -7x + 34\) \(7x + 5y - 34 = 0\)

\end{solutionbox}
\subsubsection{Q5.3 [4 marks]}\label{q5.3-4-marks}

\textbf{Find equation of circle having center at \((3, 4)\) and passing
through origin}

\begin{solutionbox}

\textbf{Solution}: \textbf{Step 1}: Find radius Since the circle passes
through origin \((0, 0)\) and has center \((3, 4)\):
\(r = \sqrt{(3-0)^2 + (4-0)^2} = \sqrt{9 + 16} = \sqrt{25} = 5\)

\textbf{Step 2}: Write equation Using standard form:
\((x - h)^2 + (y - k)^2 = r^2\) \((x - 3)^2 + (y - 4)^2 = 25\)

\textbf{Step 3}: Expand if needed \(x^2 - 6x + 9 + y^2 - 8y + 16 = 25\)
\(x^2 + y^2 - 6x - 8y + 25 - 25 = 0\) \(x^2 + y^2 - 6x - 8y = 0\)


\vspace{-5pt}
\captionof{table}{Circle Properties}
\vspace{-10pt}
\begin{longtable}[]{@{}ll@{}}
\toprule\noalign{}
Property & Value \\
\midrule\noalign{}
\endhead
\bottomrule\noalign{}
\endlastfoot
\textbf{Center} & \((3, 4)\) \\
\textbf{Radius} & \(5\) \\
\textbf{Standard Form} & \((x-3)^2 + (y-4)^2 = 25\) \\
\textbf{General Form} & \(x^2 + y^2 - 6x - 8y = 0\) \\
\end{longtable}

\end{solutionbox}
\begin{center}\rule{0.5\linewidth}{0.5pt}\end{center}

\subsection*{Mathematics Formula Cheat Sheet for Summer
Exams}\label{mathematics-formula-cheat-sheet-for-summer-exams}

\subsubsection{\texorpdfstring{\textbf{Determinants}}{Determinants}}\label{determinants}

\begin{itemize}
\tightlist
\item
  \textbf{2\times2 Matrix}:
  \(\begin{vmatrix} a & b \\ c & d \end{vmatrix} = ad - bc\)
\item
  \textbf{3\times3 Matrix}: Expand along row/column with most zeros
\end{itemize}

\subsubsection{\texorpdfstring{\textbf{Logarithms}}{Logarithms}}\label{logarithms}

\begin{itemize}
\tightlist
\item
  \(\log_a a = 1\)
\item
  \(\log a - \log b = \log \frac{a}{b}\)
\item
  \(\log a + \log b = \log(ab)\)
\item
  \(n\log a = \log a^n\)
\item
  \(\frac{1}{\log_a b} = \log_b a\) (Change of base)
\end{itemize}

\subsubsection{\texorpdfstring{\textbf{Trigonometry}}{Trigonometry}}\label{trigonometry}

\begin{itemize}
\tightlist
\item
  \textbf{Complementary angles}: \(\sin^2 A + \cos^2 A = 1\)
\item
  \textbf{Supplementary angles}: \(\sin(180^\circ - A) = \sin A\),
  \(\cos(180^\circ - A) = -\cos A\)
\item
  \textbf{Double angle}: \(\tan 2A = \frac{2\tan A}{1 - \tan^2 A}\)
\item
  \textbf{Inverse functions}: \(\sin^{-1}(\cos A) = \frac{\pi}{2} - A\)
  (for acute angles)
\end{itemize}

\subsubsection{\texorpdfstring{\textbf{Special Trigonometric
Values}}{Special Trigonometric Values}}\label{special-trigonometric-values}

\begin{longtable}[]{@{}llll@{}}
\toprule\noalign{}
Angle & \(\sin\) & \(\cos\) & \(\tan\) \\
\midrule\noalign{}
\endhead
\bottomrule\noalign{}
\endlastfoot
\(30^\circ\) & \(\frac{1}{2}\) & \(\frac{\sqrt{3}}{2}\) &
\(\frac{1}{\sqrt{3}}\) \\
\(45^\circ\) & \(\frac{1}{\sqrt{2}}\) & \(\frac{1}{\sqrt{2}}\) & \(1\) \\
\(60^\circ\) & \(\frac{\sqrt{3}}{2}\) & \(\frac{1}{2}\) & \(\sqrt{3}\) \\
\end{longtable}

\subsubsection{\texorpdfstring{\textbf{Vectors}}{Vectors}}\label{vectors}

\begin{itemize}
\tightlist
\item
  \textbf{Dot Product}:
  \(\vec{a} \cdot \vec{b} = a_1b_1 + a_2b_2 + a_3b_3\)
\item
  \textbf{Cross Product}:
  \(\vec{a} \times \vec{b} = \begin{vmatrix} \hat{i} & \hat{j} & \hat{k} \\ a_1 & a_2 & a_3 \\ b_1 & b_2 & b_3 \end{vmatrix}\)
\item
  \textbf{Magnitude}: \(|\vec{a}| = \sqrt{a_1^2 + a_2^2 + a_3^2}\)
\item
  \textbf{Unit Vector}: \(\hat{a} = \frac{\vec{a}}{|\vec{a}|}\)
\item
  \textbf{Perpendicular vectors}: \(\vec{a} \cdot \vec{b} = 0\)
\item
  \textbf{Work done}: \(W = \vec{F} \cdot \vec{d}\)
\end{itemize}

\subsubsection{\texorpdfstring{\textbf{Coordinate
Geometry}}{Coordinate Geometry}}\label{coordinate-geometry}

\paragraph{\texorpdfstring{\textbf{Lines}}{Lines}}\label{lines}

\begin{itemize}
\tightlist
\item
  \textbf{Slope}: \(m = \frac{y_2 - y_1}{x_2 - x_1}\)
\item
  \textbf{Point-slope form}: \(y - y_1 = m(x - x_1)\)
\item
  \textbf{X-intercept}: Set \(y = 0\)
\item
  \textbf{Y-intercept}: Set \(x = 0\)
\item
  \textbf{Perpendicular lines}: \(m_1 \times m_2 = -1\)
\end{itemize}

\paragraph{\texorpdfstring{\textbf{Circles}}{Circles}}\label{circles}

\begin{itemize}
\tightlist
\item
  \textbf{Standard form}: \((x - h)^2 + (y - k)^2 = r^2\)
\item
  \textbf{General form}: \(x^2 + y^2 + 2gx + 2fy + c = 0\)
\item
  \textbf{Center}: \((-g, -f)\)
\item
  \textbf{Radius}: \(\sqrt{g^2 + f^2 - c}\)
\end{itemize}

\subsubsection{\texorpdfstring{\textbf{Limits}}{Limits}}\label{limits}

\begin{itemize}
\tightlist
\item
  \textbf{Standard limits}:

  \begin{itemize}
  \tightlist
  \item
    \(\lim_{x \to 0} \frac{\sin x}{x} = 1\)
  \item
    \(\lim_{n \to \infty} \frac{1}{n} = 0\)
  \item
    \(\lim_{x \to a} \frac{x^n - a^n}{x - a} = na^{n-1}\)
  \end{itemize}
\item
  \textbf{Algebraic limits}: Factor and cancel for \(\frac{0}{0}\) forms
\item
  \textbf{Trigonometric limits}: Use identities like
  \(1 - \sin^2 x = \cos^2 x\)
\end{itemize}

\subsubsection{\texorpdfstring{\textbf{Series
Formulas}}{Series Formulas}}\label{series-formulas}

\begin{itemize}
\tightlist
\item
  \(\sum_{k=1}^n k = \frac{n(n+1)}{2}\)
\item
  \(\sum_{k=1}^n k^2 = \frac{n(n+1)(2n+1)}{6}\)
\item
  \(\sum_{k=1}^n k^3 = \left[\frac{n(n+1)}{2}\right]^2\)
\end{itemize}

\subsubsection{\texorpdfstring{\textbf{Problem-Solving
Strategies}}{Problem-Solving Strategies}}\label{problem-solving-strategies}

\paragraph{\texorpdfstring{\textbf{For
Determinants}}{For Determinants}}\label{for-determinants}

\begin{enumerate}
\tightlist
\item
  Expand along row/column with most zeros
\item
  Use properties to simplify before expanding
\item
  Factor common terms first
\end{enumerate}

\paragraph{\texorpdfstring{\textbf{For Logarithmic
Equations}}{For Logarithmic Equations}}\label{for-logarithmic-equations}

\begin{enumerate}
\tightlist
\item
  Use properties to combine logs
\item
  Convert to exponential form when needed
\item
  Check validity of solutions (arguments must be positive)
\end{enumerate}

\paragraph{\texorpdfstring{\textbf{For Trigonometric
Proofs}}{For Trigonometric Proofs}}\label{for-trigonometric-proofs}

\begin{enumerate}
\tightlist
\item
  Look for complementary/supplementary angle relationships
\item
  Use compound angle formulas
\item
  Convert everything to same trigonometric functions
\end{enumerate}

\paragraph{\texorpdfstring{\textbf{For Vector
Problems}}{For Vector Problems}}\label{for-vector-problems}

\begin{enumerate}
\tightlist
\item
  Use component form for calculations
\item
  Remember: \(\vec{a} \perp \vec{b}\) iff \(\vec{a} \cdot \vec{b} = 0\)
\item
  Cross product gives vector perpendicular to both original vectors
\end{enumerate}

\paragraph{\texorpdfstring{\textbf{For Limit
Problems}}{For Limit Problems}}\label{for-limit-problems}

\begin{enumerate}
\tightlist
\item
  Try direct substitution first
\item
  Factor and cancel for \(\frac{0}{0}\) forms
\item
  Use standard limit formulas
\item
  For rational functions, divide by highest power
\end{enumerate}

\paragraph{\texorpdfstring{\textbf{For Circle/Line
Problems}}{For Circle/Line Problems}}\label{for-circleline-problems}

\begin{enumerate}
\tightlist
\item
  Complete the square for circles
\item
  Use slope relationships for perpendicular/parallel lines
\item
  Remember intercept formulas
\end{enumerate}

\subsubsection{\texorpdfstring{\textbf{Common Mistakes to
Avoid}}{Common Mistakes to Avoid}}\label{common-mistakes-to-avoid}

\begin{enumerate}
\tightlist
\item
  \textbf{Sign errors} in determinant expansion
\item
  \textbf{Domain restrictions} in logarithmic functions
\item
  \textbf{Angle measure confusion} (degrees vs radians)
\item
  \textbf{Not checking validity} of solutions
\item
  \textbf{Forgetting to simplify} final answers
\item
  \textbf{Calculation errors} in vector operations
\end{enumerate}

\subsubsection{\texorpdfstring{\textbf{Exam
Tips}}{Exam Tips}}\label{exam-tips}

\begin{itemize}
\tightlist
\item
  \textbf{Show all steps} clearly
\item
  \textbf{Check answers} by substitution when possible
\item
  \textbf{Use proper notation} throughout
\item
  \textbf{Draw diagrams} for geometry problems
\item
  \textbf{Manage time} effectively across questions
\end{itemize}

\textbf{Best of luck with your Summer 2024 Mathematics exam!} 🎯


\end{document}
