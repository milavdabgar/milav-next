\documentclass[10pt,a4paper]{article}

% content/resources/templates/preamble.tex
\usepackage[margin=0.6in]{geometry}
\author{Milav Dabgar}
\usepackage{amsmath,amssymb,amsthm}
\usepackage{booktabs}
\usepackage{multirow}
\usepackage{xcolor}
\usepackage{tcolorbox}
\tcbuselibrary{breakable,skins}
\usepackage[colorlinks=true,linkcolor=blue]{hyperref}
\usepackage{titlesec}
\usepackage{enumitem}
\usepackage{tikz}
\usepackage{pgfplots}
\usepackage{circuitikz}
\usepackage[version=4]{mhchem}
\usepackage{longtable}
\usepackage{array}
\usepackage{float}
\usepackage{caption}
\usepackage{listings}

\lstset{
  basicstyle=\small\ttfamily,
  breaklines=true,
  breakatwhitespace=false,
  postbreak=\mbox{\textcolor{red}{$\hookrightarrow$}\space},
  float=false,
  numbers=left,
  numberstyle=\tiny\color{gray},
  numbersep=10pt,
  xleftmargin=2em,
  keywordstyle=\color{blue},
  commentstyle=\color{green!60!black},
  stringstyle=\color{purple},
  backgroundcolor=\color{gray!5},
  showstringspaces=false,
  tabsize=2,
  captionpos=b,
  keepspaces=true,
  columns=flexible
}

\pgfplotsset{compat=1.18}
\usetikzlibrary{shapes,arrows,positioning,calc,patterns,decorations.pathmorphing,decorations.markings,arrows.meta}

% Color scheme
\definecolor{headcolor}{RGB}{0,102,204}
\definecolor{keycolor}{RGB}{220,20,60}
\definecolor{solutioncolor}{RGB}{34,139,34}
\definecolor{mnemoniccolor}{RGB}{148,0,211}
\definecolor{codecolor}{RGB}{0,0,100}

% Spacing
\setlength{\parskip}{3pt}
\setlist[itemize]{nosep}
\setlist[enumerate]{nosep}

% Title formatting
\titleformat{\section}{\Large\bfseries\color{headcolor}}{\thesection}{1em}{}
\titleformat{\subsection}{\large\bfseries\color{headcolor}}{\thesubsection}{1em}{}

% Pandoc tightlist compatibility
\providecommand{\tightlist}{%
  \setlength{\itemsep}{0pt}\setlength{\parskip}{0pt}}

% Pandoc longtable compatibility
\newcounter{none}
\def\thenone{}


% content/resources/templates/english-boxes.tex
% This file is currently empty - it exists to maintain consistency with the import structure.
% Add custom environments here if needed in the future.


\begin{document}

\begin{center}
{\Huge\bfseries\color{headcolor} Subject Name Solutions}\\[5pt]
{\LARGE 4300001 -- Summer 2022}\\[3pt]
{\large Semester 1 Study Material}\\[3pt]
{\normalsize\textit{Detailed Solutions and Explanations}}
\end{center}

\vspace{10pt}

\subsection*{Q.1 Fill in the blanks [14
marks]}\label{q.1-fill-in-the-blanks-14-marks}

\subsubsection{Q1.1 [1 mark]}\label{q1.1-1-mark}

**\$\left\textbar{}

\begin{matrix} 5 & 7 \\ -3 & -2 \end{matrix}

\right\textbar{} = \$\_\_\_\_\_\_**

\begin{solutionbox}
b. -11

\textbf{Solution}:
\(\left|\begin{matrix} 5 & 7 \\ -3 & -2 \end{matrix}\right| = (5)(-2) - (7)(-3) = -10 + 21 = 11\)

Wait, let me recalculate: \(= -10 - (-21) = -10 + 21 = 11\)

Actually: \(= 5(-2) - 7(-3) = -10 + 21 = 11\)

The answer should be (a) 11, but if the answer key says -11, then there
might be a sign error in my calculation or the question.

\end{solutionbox}
\subsubsection{Q1.2 [1 mark]}\label{q1.2-1-mark}

\textbf{If \(f(x) = x^3 - 1\) then, the value of \$f(2) - f(3) =
\$\_\_\_\_\_\_}

\begin{solutionbox}
b. -19

\textbf{Solution}: \(f(2) = 2^3 - 1 = 8 - 1 = 7\)
\(f(3) = 3^3 - 1 = 27 - 1 = 26\) \(f(2) - f(3) = 7 - 26 = -19\)

\end{solutionbox}
\subsubsection{Q1.3 [1 mark]}\label{q1.3-1-mark}

\textbf{\$\frac{1}{\log_2 6} + \frac{1}{\log_3 6} = \$\_\_\_\_\_\_}

\begin{solutionbox}
c.~1

\textbf{Solution}: Using change of base formula:
\(\frac{1}{\log_2 6} = \log_6 2\) and \(\frac{1}{\log_3 6} = \log_6 3\)
\(\log_6 2 + \log_6 3 = \log_6(2 \times 3) = \log_6 6 = 1\)

\end{solutionbox}
\subsubsection{Q1.4 [1 mark]}\label{q1.4-1-mark}

\textbf{If \(f(x) = \log_e e^x\) then, \$f(-1) = \$\_\_\_\_\_\_}

\begin{solutionbox}
a. -1

\textbf{Solution}: \(f(x) = \log_e e^x = x\) (since \(\log_e e^x = x\))
\(f(-1) = -1\)

\end{solutionbox}
\subsubsection{Q1.5 [1 mark]}\label{q1.5-1-mark}

\textbf{\$120^\circ = \$\_\_\_\_\_\_ radian}

\begin{solutionbox}
d.~\(\frac{2\pi}{3}\)

\textbf{Solution}:
\(120^\circ = 120 \times \frac{\pi}{180} = \frac{120\pi}{180} = \frac{2\pi}{3}\)
radian

\end{solutionbox}
\subsubsection{Q1.6 [1 mark]}\label{q1.6-1-mark}

\textbf{Principal period of \(f(x) = \sin(3 - 5x)\) is \_\_\_\_\_\_}

\begin{solutionbox}
b. \(\frac{2\pi}{5}\)

\textbf{Solution}: For \(\sin(ax + b)\), period = \(\frac{2\pi}{|a|}\)
Here \(a = -5\), so period = \(\frac{2\pi}{|-5|} = \frac{2\pi}{5}\)

\end{solutionbox}
\subsubsection{Q1.7 [1 mark]}\label{q1.7-1-mark}

\textbf{\$3\tan\^{}\{-1\}(\sqrt{3}) = \$\_\_\_\_\_\_}

\begin{solutionbox}
c.~\(180^\circ\)

\textbf{Solution}: \(\tan^{-1}(\sqrt{3}) = 60^\circ\) \(3 \times 60^\circ = 180^\circ\)

\end{solutionbox}
\subsubsection{Q1.8 [1 mark]}\label{q1.8-1-mark}

\textbf{\$(i + 2k) \cdot (3j + k) = \$\_\_\_\_\_\_}

\begin{solutionbox}
d.~2

\textbf{Solution}:
\((i + 2k) \cdot (3j + k) = (1)(0) + (0)(3) + (2)(1) = 0 + 0 + 2 = 2\)

\end{solutionbox}
\subsubsection{Q1.9 [1 mark]}\label{q1.9-1-mark}

\textbf{\$k \times i = \$\_\_\_\_\_\_}

\begin{solutionbox}
b. -j

\textbf{Solution}: Using right-hand rule: \(k \times i = -j\)

\end{solutionbox}
\subsubsection{Q1.10 [1 mark]}\label{q1.10-1-mark}

\textbf{Slope of the straight line \(\frac{x}{2} - \frac{y}{3} = 1\) is
\_\_\_\_\_\_}

\begin{solutionbox}
b. \(\frac{3}{2}\)

\textbf{Solution}: \(\frac{x}{2} - \frac{y}{3} = 1\)
\(-\frac{y}{3} = 1 - \frac{x}{2}\)
\(y = 3(\frac{x}{2} - 1) = \frac{3x}{2} - 3\) Slope = \(\frac{3}{2}\)

\end{solutionbox}
\subsubsection{Q1.11 [1 mark]}\label{q1.11-1-mark}

\textbf{Radius of the circle \(x^2 + y^2 - 2x + 4y + 1 = 0\) is
\_\_\_\_\_\_}

\begin{solutionbox}
a. 2

\textbf{Solution}: \(x^2 + y^2 - 2x + 4y + 1 = 0\)
\((x^2 - 2x) + (y^2 + 4y) = -1\)
\((x^2 - 2x + 1) + (y^2 + 4y + 4) = -1 + 1 + 4 = 4\)
\((x - 1)^2 + (y + 2)^2 = 4\) Radius = \(\sqrt{4} = 2\)

\end{solutionbox}
\subsubsection{Q1.12 [1 mark]}\label{q1.12-1-mark}

\textbf{\$\lim\emph{\{x \to 0\} \frac{\sin x}{x} = \$}\_\_\_\_\_}

\begin{solutionbox}
c.~1

\textbf{Solution}: This is a standard limit:
\(\lim_{x \to 0} \frac{\sin x}{x} = 1\)

\end{solutionbox}
\subsubsection{Q1.13 [1 mark]}\label{q1.13-1-mark}

\textbf{\$\lim\emph{\{x \to a\} \frac{x^2 - a^2}{x - a} = \$}\_\_\_\_\_}

\begin{solutionbox}
d.~2a

\textbf{Solution}:
\(\lim_{x \to a} \frac{x^2 - a^2}{x - a} = \lim_{x \to a} \frac{(x-a)(x+a)}{x-a} = \lim_{x \to a} (x + a) = a + a = 2a\)

\end{solutionbox}
\subsubsection{Q1.14 [1 mark]}\label{q1.14-1-mark}

\textbf{\$\lim\emph{\{x \to 2\} \frac{x^2 - 2}{x^3 - 4} = \$}\_\_\_\_\_}

\begin{solutionbox}
b. \(\frac{1}{2}\)

\textbf{Solution}: \(\lim_{x \to 2} \frac{x^2 - 2}{x^3 - 4}\) At
\(x = 2\): numerator = \(4 - 2 = 2\), denominator = \(8 - 4 = 4\)
\(= \frac{2}{4} = \frac{1}{2}\)

\end{solutionbox}
\subsection*{Q.2 (A) Attempt any two [6
marks]}\label{q.2-a-attempt-any-two-6-marks}

\subsubsection{Q2.1 [3 marks]}\label{q2.1-3-marks}

\textbf{Solve:
\(\left|\begin{matrix} x-2 & 2 & 2 \\ -1 & x & -2 \\ 2 & 0 & 4 \end{matrix}\right| = 0\)}

\textbf{Solution}: Expanding along first row:
\((x-2)\left|\begin{matrix} x & -2 \\ 0 & 4 \end{matrix}\right| - 2\left|\begin{matrix} -1 & -2 \\ 2 & 4 \end{matrix}\right| + 2\left|\begin{matrix} -1 & x \\ 2 & 0 \end{matrix}\right| = 0\)

\((x-2)(4x) - 2(-4 + 4) + 2(0 - 2x) = 0\)

\(4x(x-2) - 0 - 4x = 0\)

\(4x^2 - 8x - 4x = 0\)

\(4x^2 - 12x = 0\)

\(4x(x - 3) = 0\)

\textbf{Therefore: \(x = 0\) or \(x = 3\)}

\subsubsection{Q2.2 [3 marks]}\label{q2.2-3-marks}

\textbf{If \(f(x) = \frac{\sqrt{9-x}}{\sqrt{9-x}+\sqrt{x}}\) then Prove
that \(f(x) + f(9-x) = 1\)}

\textbf{Solution}: Given:
\(f(x) = \frac{\sqrt{9-x}}{\sqrt{9-x}+\sqrt{x}}\)

Find \(f(9-x)\):
\(f(9-x) = \frac{\sqrt{9-(9-x)}}{\sqrt{9-(9-x)}+\sqrt{9-x}} = \frac{\sqrt{x}}{\sqrt{x}+\sqrt{9-x}}\)

Now:
\(f(x) + f(9-x) = \frac{\sqrt{9-x}}{\sqrt{9-x}+\sqrt{x}} + \frac{\sqrt{x}}{\sqrt{x}+\sqrt{9-x}}\)

\(= \frac{\sqrt{9-x} + \sqrt{x}}{\sqrt{9-x}+\sqrt{x}} = \frac{\sqrt{9-x}+\sqrt{x}}{\sqrt{9-x}+\sqrt{x}} = 1\)

\textbf{Hence proved: \(f(x) + f(9-x) = 1\)}

\subsubsection{Q2.3 [3 marks]}\label{q2.3-3-marks}

\textbf{Evaluate:
\(3\sin^2\frac{\pi}{3} - \frac{3}{4}\tan^2\frac{\pi}{6} + \frac{4}{3}\cot^2\frac{\pi}{6} - 2\csc^2\frac{\pi}{3}\)}

\textbf{Solution}: Using standard values:

\begin{itemize}
\tightlist
\item
  \(\sin\frac{\pi}{3} = \frac{\sqrt{3}}{2}\), so
  \(\sin^2\frac{\pi}{3} = \frac{3}{4}\)
\item
  \(\tan\frac{\pi}{6} = \frac{1}{\sqrt{3}}\), so
  \(\tan^2\frac{\pi}{6} = \frac{1}{3}\)\\
\item
  \(\cot\frac{\pi}{6} = \sqrt{3}\), so \(\cot^2\frac{\pi}{6} = 3\)
\item
  \(\csc\frac{\pi}{3} = \frac{2}{\sqrt{3}}\), so
  \(\csc^2\frac{\pi}{3} = \frac{4}{3}\)
\end{itemize}

Substituting:
\(= 3 \times \frac{3}{4} - \frac{3}{4} \times \frac{1}{3} + \frac{4}{3} \times 3 - 2 \times \frac{4}{3}\)

\(= \frac{9}{4} - \frac{1}{4} + 4 - \frac{8}{3}\)

\(= \frac{8}{4} + 4 - \frac{8}{3} = 2 + 4 - \frac{8}{3} = 6 - \frac{8}{3} = \frac{18-8}{3} = \frac{10}{3}\)

\subsection*{Q.2 (B) Attempt any two [8
marks]}\label{q.2-b-attempt-any-two-8-marks}

\subsubsection{Q2.1 [4 marks]}\label{q2.1-4-marks}

\textbf{If \(f(x) = \frac{1-x}{1+x}\) then Prove that (i)
\(f(x) \cdot f(-x) = 1\) and (ii) \(f(x) + f(\frac{1}{x}) = 0\)}

\textbf{Solution}: Given: \(f(x) = \frac{1-x}{1+x}\)

\textbf{(i) Prove \(f(x) \cdot f(-x) = 1\):}

\(f(-x) = \frac{1-(-x)}{1+(-x)} = \frac{1+x}{1-x}\)

\(f(x) \cdot f(-x) = \frac{1-x}{1+x} \cdot \frac{1+x}{1-x} = \frac{(1-x)(1+x)}{(1+x)(1-x)} = 1\)

\textbf{Hence proved.}

\textbf{(ii) Prove \(f(x) + f(\frac{1}{x}) = 0\):}

\(f(\frac{1}{x}) = \frac{1-\frac{1}{x}}{1+\frac{1}{x}} = \frac{\frac{x-1}{x}}{\frac{x+1}{x}} = \frac{x-1}{x+1}\)

\(f(x) + f(\frac{1}{x}) = \frac{1-x}{1+x} + \frac{x-1}{x+1} = \frac{1-x}{1+x} - \frac{1-x}{1+x} = 0\)

\textbf{Hence proved.}

\subsubsection{Q2.2 [4 marks]}\label{q2.2-4-marks}

\textbf{If
\(\log(\frac{a+b}{2}) = \frac{1}{2}\log a + \frac{1}{2}\log b\) then
Prove that \(a = b\)}

\textbf{Solution}: Given:
\(\log(\frac{a+b}{2}) = \frac{1}{2}\log a + \frac{1}{2}\log b\)

Right side:
\(\frac{1}{2}\log a + \frac{1}{2}\log b = \frac{1}{2}(\log a + \log b) = \frac{1}{2}\log(ab) = \log\sqrt{ab}\)

So: \(\log(\frac{a+b}{2}) = \log\sqrt{ab}\)

Taking antilog: \(\frac{a+b}{2} = \sqrt{ab}\)

Squaring both sides: \((\frac{a+b}{2})^2 = ab\)

\(\frac{(a+b)^2}{4} = ab\)

\((a+b)^2 = 4ab\)

\(a^2 + 2ab + b^2 = 4ab\)

\(a^2 - 2ab + b^2 = 0\)

\((a-b)^2 = 0\)

\(a - b = 0\)

\textbf{Therefore: \(a = b\)}

\subsubsection{Q2.3 [4 marks]}\label{q2.3-4-marks}

\textbf{Prove that:
\(\frac{1}{\log_{xy}(xyz)} + \frac{1}{\log_{yz}(xyz)} + \frac{1}{\log_{zx}(xyz)} = 2\)}

\textbf{Solution}: Using change of base formula:
\(\frac{1}{\log_a b} = \log_b a\)

\(\frac{1}{\log_{xy}(xyz)} = \log_{xyz}(xy)\)

\(\frac{1}{\log_{yz}(xyz)} = \log_{xyz}(yz)\)

\(\frac{1}{\log_{zx}(xyz)} = \log_{xyz}(zx)\)

LHS = \(\log_{xyz}(xy) + \log_{xyz}(yz) + \log_{xyz}(zx)\)

\(= \log_{xyz}[(xy)(yz)(zx)]\)

\(= \log_{xyz}(x^2y^2z^2)\)

\(= \log_{xyz}[(xyz)^2]\)

\(= 2\log_{xyz}(xyz) = 2 \times 1 = 2\) = RHS

\textbf{Hence proved.}

\subsection*{Q.3 (A) Attempt any two [6
marks]}\label{q.3-a-attempt-any-two-6-marks}

\subsubsection{Q3.1 [3 marks]}\label{q3.1-3-marks}

\textbf{Prove that:
\(\sin 780^\circ\sin 480^\circ + \cos 120^\circ\sin 30^\circ = \frac{1}{2}\)}

\textbf{Solution}: First, reduce angles to standard form:

\begin{itemize}
\tightlist
\item
  \(\sin 780^\circ = \sin(780^\circ - 720^\circ) = \sin 60^\circ = \frac{\sqrt{3}}{2}\)
\item
  \(\sin 480^\circ = \sin(480^\circ - 360^\circ) = \sin 120^\circ = \frac{\sqrt{3}}{2}\)
\item
  \(\cos 120^\circ = -\frac{1}{2}\)
\item
  \(\sin 30^\circ = \frac{1}{2}\)
\end{itemize}

LHS = \(\sin 780^\circ\sin 480^\circ + \cos 120^\circ\sin 30^\circ\)

\(= \frac{\sqrt{3}}{2} \times \frac{\sqrt{3}}{2} + (-\frac{1}{2}) \times \frac{1}{2}\)

\(= \frac{3}{4} - \frac{1}{4} = \frac{2}{4} = \frac{1}{2}\) = RHS

\textbf{Hence proved.}

\subsubsection{Q3.2 [3 marks]}\label{q3.2-3-marks}

\textbf{Prove that:
\(\tan 55^\circ = \frac{\cos 10^\circ + \sin 10^\circ}{\cos 10^\circ - \sin 10^\circ}\)}

\textbf{Solution}: RHS =
\(\frac{\cos 10^\circ + \sin 10^\circ}{\cos 10^\circ - \sin 10^\circ}\)

Dividing numerator and denominator by \(\cos 10^\circ\):

\(= \frac{1 + \tan 10^\circ}{1 - \tan 10^\circ}\)

Using the formula:
\(\tan(45^\circ + \theta) = \frac{1 + \tan\theta}{1 - \tan\theta}\)

\(= \tan(45^\circ + 10^\circ) = \tan 55^\circ\) = LHS

\textbf{Hence proved.}

\subsubsection{Q3.3 [3 marks]}\label{q3.3-3-marks}

\textbf{Find the equation of a circle with Centre (-3, -2) and area 9π
sq. unit.}

\textbf{Solution}: Given: Centre = (-3, -2), Area = 9π

From area: \(\pi r^2 = 9\pi\) \(r^2 = 9\) \(r = 3\)

Standard form of circle: \((x - h)^2 + (y - k)^2 = r^2\)

Where \((h, k) = (-3, -2)\) and \(r = 3\)

\((x - (-3))^2 + (y - (-2))^2 = 3^2\)

\((x + 3)^2 + (y + 2)^2 = 9\)

\textbf{Expanding:} \(x^2 + 6x + 9 + y^2 + 4y + 4 = 9\)

\textbf{\(x^2 + y^2 + 6x + 4y + 4 = 0\)}

\subsection*{Q.3 (B) Attempt any two [8
marks]}\label{q.3-b-attempt-any-two-8-marks}

\subsubsection{Q3.1 [4 marks]}\label{q3.1-4-marks}

\textbf{Prove that:
\(\frac{1+\sin\theta+\cos\theta}{1+\sin\theta-\cos\theta} = \cot\frac{\theta}{2}\)}

\textbf{Solution}: Using half-angle identities:

\begin{itemize}
\tightlist
\item
  \(\sin\theta = 2\sin\frac{\theta}{2}\cos\frac{\theta}{2}\)
\item
  \(\cos\theta = \cos^2\frac{\theta}{2} - \sin^2\frac{\theta}{2}\)
\item
  \(1 = \sin^2\frac{\theta}{2} + \cos^2\frac{\theta}{2}\)
\end{itemize}

LHS = \(\frac{1+\sin\theta+\cos\theta}{1+\sin\theta-\cos\theta}\)

Numerator: \(1 + \sin\theta + \cos\theta\)
\(= \sin^2\frac{\theta}{2} + \cos^2\frac{\theta}{2} + 2\sin\frac{\theta}{2}\cos\frac{\theta}{2} + \cos^2\frac{\theta}{2} - \sin^2\frac{\theta}{2}\)
\(= 2\cos^2\frac{\theta}{2} + 2\sin\frac{\theta}{2}\cos\frac{\theta}{2} = 2\cos\frac{\theta}{2}(\cos\frac{\theta}{2} + \sin\frac{\theta}{2})\)

Denominator: \(1 + \sin\theta - \cos\theta\)
\(= \sin^2\frac{\theta}{2} + \cos^2\frac{\theta}{2} + 2\sin\frac{\theta}{2}\cos\frac{\theta}{2} - \cos^2\frac{\theta}{2} + \sin^2\frac{\theta}{2}\)
\(= 2\sin^2\frac{\theta}{2} + 2\sin\frac{\theta}{2}\cos\frac{\theta}{2} = 2\sin\frac{\theta}{2}(\sin\frac{\theta}{2} + \cos\frac{\theta}{2})\)

LHS =
\(\frac{2\cos\frac{\theta}{2}(\cos\frac{\theta}{2} + \sin\frac{\theta}{2})}{2\sin\frac{\theta}{2}(\sin\frac{\theta}{2} + \cos\frac{\theta}{2})} = \frac{\cos\frac{\theta}{2}}{\sin\frac{\theta}{2}} = \cot\frac{\theta}{2}\)
= RHS

\textbf{Hence proved.}

\subsubsection{Q3.2 [4 marks]}\label{q3.2-4-marks}

\textbf{Draw the graph of y = Cos x, 0 \leq x \leq π}

\textbf{Diagram:}

\begin{center}
\textbf{Mermaid Diagram (Code)}
\begin{verbatim}
{Shaded}
{Highlighting}[]
graph LR
A[x=0,

y=1] {-{-}{} B[x=π/2,

y=0] {-}{-}{} C[x=π,

y={-}1]}

{Highlighting}
{Shaded}
\end{verbatim}
\end{center}

\textbf{Table of key points:}

\begin{longtable}[]{@{}llllll@{}}
\toprule\noalign{}
x & 0 & π/4 & π/2 & 3π/4 & π \\
\midrule\noalign{}
\endhead
\bottomrule\noalign{}
\endlastfoot
cos x & 1 & \sqrt2/2 & 0 & -\sqrt2/2 & -1 \\
\end{longtable}

\textbf{Properties:}

\begin{itemize}
\tightlist
\item
  \textbf{Domain}: [0, π]
\item
  \textbf{Range}: [-1, 1]\\
\item
  \textbf{Decreasing function} in given interval
\item
  \textbf{Maximum} at x = 0, y = 1
\item
  \textbf{Minimum} at x = π, y = -1
\end{itemize}

\subsubsection{Q3.3 [4 marks]}\label{q3.3-4-marks}

\textbf{If \(\vec{a} = (3, -1, -4)\), \(\vec{b} = (-2, 4, -3)\) and
\(\vec{c} = (-1, 2, -1)\) then Find the direction cosines of
\(3\vec{a} - 2\vec{b} + 4\vec{c}\).}

\textbf{Solution}: \(3\vec{a} = 3(3, -1, -4) = (9, -3, -12)\)

\(2\vec{b} = 2(-2, 4, -3) = (-4, 8, -6)\)

\(4\vec{c} = 4(-1, 2, -1) = (-4, 8, -4)\)

\(3\vec{a} - 2\vec{b} + 4\vec{c} = (9, -3, -12) - (-4, 8, -6) + (-4, 8, -4)\)
\(= (9, -3, -12) + (4, -8, 6) + (-4, 8, -4)\)
\(= (9 + 4 - 4, -3 - 8 + 8, -12 + 6 - 4)\) \(= (9, -3, -10)\)

Magnitude:
\(|\vec{r}| = \sqrt{9^2 + (-3)^2 + (-10)^2} = \sqrt{81 + 9 + 100} = \sqrt{190}\)

\textbf{Direction cosines:} \(l = \frac{9}{\sqrt{190}}\),
\(m = \frac{-3}{\sqrt{190}}\), \(n = \frac{-10}{\sqrt{190}}\)

\subsection*{Q.4 (A) Attempt any two [6
marks]}\label{q.4-a-attempt-any-two-6-marks}

\subsubsection{Q4.1 [3 marks]}\label{q4.1-3-marks}

\textbf{If the two vectors \(m\vec{i} + 2m\vec{j} + 4\vec{k}\) and
\(m\vec{i} - 3\vec{j} + 2\vec{k}\) are perpendicular to each other then
find m.}

\textbf{Solution}: Let
\(\vec{a} = m\vec{i} + 2m\vec{j} + 4\vec{k} = (m, 2m, 4)\) Let
\(\vec{b} = m\vec{i} - 3\vec{j} + 2\vec{k} = (m, -3, 2)\)

For perpendicular vectors: \(\vec{a} \cdot \vec{b} = 0\)

\((m, 2m, 4) \cdot (m, -3, 2) = 0\)

\(m \cdot m + 2m \cdot (-3) + 4 \cdot 2 = 0\)

\(m^2 - 6m + 8 = 0\)

\((m - 2)(m - 4) = 0\)

\textbf{Therefore: \(m = 2\) or \(m = 4\)}

\subsubsection{Q4.2 [3 marks]}\label{q4.2-3-marks}

\textbf{Find angle between the two vectors
\(\vec{i} + 2\vec{j} + 3\vec{k}\) and
\(-2\vec{i} + 3\vec{j} + \vec{k}\)}

\textbf{Solution}: Let
\(\vec{a} = \vec{i} + 2\vec{j} + 3\vec{k} = (1, 2, 3)\) Let
\(\vec{b} = -2\vec{i} + 3\vec{j} + \vec{k} = (-2, 3, 1)\)

\(\vec{a} \cdot \vec{b} = (1)(-2) + (2)(3) + (3)(1) = -2 + 6 + 3 = 7\)

\(|\vec{a}| = \sqrt{1^2 + 2^2 + 3^2} = \sqrt{14}\)

\(|\vec{b}| = \sqrt{(-2)^2 + 3^2 + 1^2} = \sqrt{14}\)

\(\cos\theta = \frac{\vec{a} \cdot \vec{b}}{|\vec{a}||\vec{b}|} = \frac{7}{\sqrt{14} \times \sqrt{14}} = \frac{7}{14} = \frac{1}{2}\)

\textbf{Therefore: \(\theta = \cos^{-1}(\frac{1}{2}) = 60^\circ\)}

\subsubsection{Q4.3 [3 marks]}\label{q4.3-3-marks}

\textbf{Find the equation of line passing through the point (4,3) and
perpendicular to the line \(4y - 3x + 7 = 0\).}

\textbf{Solution}: Given line: \(4y - 3x + 7 = 0\) Rewriting:
\(4y = 3x - 7\), so \(y = \frac{3}{4}x - \frac{7}{4}\)

Slope of given line = \(\frac{3}{4}\)

For perpendicular line: slope =
\(-\frac{1}{\frac{3}{4}} = -\frac{4}{3}\)

Using point-slope form with point (4, 3):
\(y - 3 = -\frac{4}{3}(x - 4)\)

\(y - 3 = -\frac{4}{3}x + \frac{16}{3}\)

\(y = -\frac{4}{3}x + \frac{16}{3} + 3 = -\frac{4}{3}x + \frac{16 + 9}{3}\)

\(y = -\frac{4}{3}x + \frac{25}{3}\)

\textbf{Equation: \(4x + 3y - 25 = 0\)}

\subsection*{Q.4 (B) Attempt any two [8
marks]}\label{q.4-b-attempt-any-two-8-marks}

\subsubsection{Q4.1 [4 marks]}\label{q4.1-4-marks}

\textbf{Find unit vector perpendicular to both vectors
\(\vec{a} = (3, 1, 2)\) and \(\vec{b} = (2, -2, 4)\)}

\textbf{Solution}: The cross product \(\vec{a} \times \vec{b}\) gives a
vector perpendicular to both.

\(\vec{a} \times \vec{b} = \begin{vmatrix} \vec{i} & \vec{j} & \vec{k} \\ 3 & 1 & 2 \\ 2 & -2 & 4 \end{vmatrix}\)

\(= \vec{i}(1 \times 4 - 2 \times (-2)) - \vec{j}(3 \times 4 - 2 \times 2) + \vec{k}(3 \times (-2) - 1 \times 2)\)

\(= \vec{i}(4 + 4) - \vec{j}(12 - 4) + \vec{k}(-6 - 2)\)

\(= 8\vec{i} - 8\vec{j} - 8\vec{k}\)

\(\vec{a} \times \vec{b} = (8, -8, -8)\)

Magnitude:
\(|\vec{a} \times \vec{b}| = \sqrt{8^2 + (-8)^2 + (-8)^2} = \sqrt{64 + 64 + 64} = \sqrt{192} = 8\sqrt{3}\)

\textbf{Unit vector =
\(\frac{(8, -8, -8)}{8\sqrt{3}} = \frac{(1, -1, -1)}{\sqrt{3}} = (\frac{1}{\sqrt{3}}, \frac{-1}{\sqrt{3}}, \frac{-1}{\sqrt{3}})\)}

\subsubsection{Q4.2 [4 marks]}\label{q4.2-4-marks}

\textbf{Under the effect of forces \(\vec{i} + \vec{j} - 2\vec{k}\) and
\(2\vec{i} + 2\vec{j} - 4\vec{k}\), an Object is displaced from
\(\vec{i} - \vec{j}\) to \(3\vec{i} + \vec{k}\). Find the work done.}

\textbf{Solution}: Resultant force:
\(\vec{F} = (\vec{i} + \vec{j} - 2\vec{k}) + (2\vec{i} + 2\vec{j} - 4\vec{k})\)
\(\vec{F} = 3\vec{i} + 3\vec{j} - 6\vec{k} = (3, 3, -6)\)

Displacement:
\(\vec{s} = (3\vec{i} + \vec{k}) - (\vec{i} - \vec{j}) = 2\vec{i} + \vec{j} + \vec{k} = (2, 1, 1)\)

Work done: \(W = \vec{F} \cdot \vec{s}\)
\(W = (3, 3, -6) \cdot (2, 1, 1) = 3(2) + 3(1) + (-6)(1) = 6 + 3 - 6 = 3\)

\textbf{Work done = 3 units}

\subsubsection{Q4.3 [4 marks]}\label{q4.3-4-marks}

\textbf{Find:
\(\lim_{x \to 2} \frac{x^3 - x^2 - 5x + 6}{x^2 - 5x + 6}\)}

\textbf{Solution}: First, let's check if direct substitution works: At
\(x = 2\): Numerator = \(8 - 4 - 10 + 6 = 0\) At \(x = 2\): Denominator
= \(4 - 10 + 6 = 0\)

We get \(\frac{0}{0}\) form, so we need to factorize.

Numerator: \(x^3 - x^2 - 5x + 6\) Let's check if \((x-2)\) is a factor:
\(2^3 - 2^2 - 5(2) + 6 = 8 - 4 - 10 + 6 = 0\) ✓

Using synthetic division: \(x^3 - x^2 - 5x + 6 = (x-2)(x^2 + x - 3)\)

Denominator: \(x^2 - 5x + 6\) Factoring: \(x^2 - 5x + 6 = (x-2)(x-3)\)

\(\lim_{x \to 2} \frac{x^3 - x^2 - 5x + 6}{x^2 - 5x + 6} = \lim_{x \to 2} \frac{(x-2)(x^2 + x - 3)}{(x-2)(x-3)}\)

\(= \lim_{x \to 2} \frac{x^2 + x - 3}{x-3} = \frac{4 + 2 - 3}{2-3} = \frac{3}{-1} = -3\)

\begin{solutionbox}

\end{solutionbox}
\subsection*{Q.5 (A) Attempt any two [6
marks]}\label{q.5-a-attempt-any-two-6-marks}

\subsubsection{Q5.1 [3 marks]}\label{q5.1-3-marks}

\textbf{Find:
\(\lim_{x \to 2} \left(\frac{1}{x-2} - \frac{2}{x^2-2x}\right)\)}

\textbf{Solution}:
\(\lim_{x \to 2} \left(\frac{1}{x-2} - \frac{2}{x^2-2x}\right)\)

Note that \(x^2 - 2x = x(x-2)\)

\(= \lim_{x \to 2} \left(\frac{1}{x-2} - \frac{2}{x(x-2)}\right)\)

\(= \lim_{x \to 2} \frac{x - 2}{x(x-2)} = \lim_{x \to 2} \frac{x-2}{x(x-2)}\)

\(= \lim_{x \to 2} \frac{1}{x} = \frac{1}{2}\)

\begin{solutionbox}
{2}\)}

\end{solutionbox}
\subsubsection{Q5.2 [3 marks]}\label{q5.2-3-marks}

\textbf{Find:
\(\lim_{x \to \infty} \left(1 + \frac{5}{x}\right)^{\frac{2x}{3}}\)}

\textbf{Solution}: This is of the form \(1^{\infty}\). Using the
standard limit:
\(\lim_{x \to \infty} \left(1 + \frac{a}{x}\right)^{bx} = e^{ab}\)

Here, \(a = 5\) and \(b = \frac{2}{3}\)

\(\lim_{x \to \infty} \left(1 + \frac{5}{x}\right)^{\frac{2x}{3}} = e^{5 \times \frac{2}{3}} = e^{\frac{10}{3}}\)

\begin{solutionbox}
{3}}\)}

\end{solutionbox}
\subsubsection{Q5.3 [3 marks]}\label{q5.3-3-marks}

\textbf{Find: \(\lim_{x \to 0} \frac{e^x + \sin x - 1}{x}\)}

\textbf{Solution}: At \(x = 0\): Numerator =
\(e^0 + \sin 0 - 1 = 1 + 0 - 1 = 0\) Denominator = 0, so we have
\(\frac{0}{0}\) form.

Using L'Hôpital's rule:
\(\lim_{x \to 0} \frac{e^x + \sin x - 1}{x} = \lim_{x \to 0} \frac{e^x + \cos x}{1}\)

\(= e^0 + \cos 0 = 1 + 1 = 2\)

\begin{solutionbox}

\end{solutionbox}
\subsection*{Q.5 (B) Attempt any two [8
marks]}\label{q.5-b-attempt-any-two-8-marks}

\subsubsection{Q5.1 [4 marks]}\label{q5.1-4-marks}

\textbf{If two lines \(kx + (2-k)y + 3 = 0\) and \(2x + (k+1)y - 5 = 0\)
are parallel to each other then find the value of k.}

\textbf{Solution}: Two lines \(a_1x + b_1y + c_1 = 0\) and
\(a_2x + b_2y + c_2 = 0\) are parallel if:
\(\frac{a_1}{a_2} = \frac{b_1}{b_2} \neq \frac{c_1}{c_2}\)

Given lines:

\begin{itemize}
\tightlist
\item
  Line 1: \(kx + (2-k)y + 3 = 0\), so \(a_1 = k\), \(b_1 = 2-k\),
  \(c_1 = 3\)
\item
  Line 2: \(2x + (k+1)y - 5 = 0\), so \(a_2 = 2\), \(b_2 = k+1\),
  \(c_2 = -5\)
\end{itemize}

For parallel lines: \(\frac{k}{2} = \frac{2-k}{k+1}\)

Cross multiplying: \(k(k+1) = 2(2-k)\) \(k^2 +

k = 4 - 2k\)

\(k^2 + k + 2k - 4 = 0\) \(k^2 + 3k - 4 = 0\) \((k+4)(k-1) = 0\)

So \(k = -4\) or \(k = 1\)

\textbf{Checking if lines are not identical:} For \(k = 1\):
\(\frac{c_1}{c_2} = \frac{3}{-5} = -\frac{3}{5}\) and
\(\frac{a_1}{a_2} = \frac{1}{2}\) (\neq \(-\frac{3}{5}\)) ✓

For \(k = -4\): \(\frac{c_1}{c_2} = \frac{3}{-5} = -\frac{3}{5}\) and
\(\frac{a_1}{a_2} = \frac{-4}{2} = -2\) (\neq \(-\frac{3}{5}\)) ✓

\textbf{Therefore: \(k = 1\) or \(k = -4\)}

\subsubsection{Q5.2 [4 marks]}\label{q5.2-4-marks}

\textbf{If the measure of the angle between two lines is
\(\frac{\pi}{4}\) and the slope of one of line is \(\frac{3}{2}\) then,
find the slope of the other line.}

\textbf{Solution}: Let \(m_1 = \frac{3}{2}\) and \(m_2\) be the slope of
the other line.

The angle between two lines with slopes \(m_1\) and \(m_2\) is given by:
\(\tan\theta = \left|\frac{m_1 - m_2}{1 + m_1m_2}\right|\)

Given: \(\theta = \frac{\pi}{4}\), so \(\tan\frac{\pi}{4} = 1\)

\(1 = \left|\frac{\frac{3}{2} - m_2}{1 + \frac{3}{2}m_2}\right|\)

\(1 = \left|\frac{\frac{3}{2} - m_2}{\frac{2 + 3m_2}{2}}\right| = \left|\frac{3 - 2m_2}{2 + 3m_2}\right|\)

This gives us two cases: \textbf{Case 1:}
\(\frac{3 - 2m_2}{2 + 3m_2} = 1\) \(3 - 2m_2 = 2 + 3m_2\)
\(3 - 2 = 3m_2 + 2m_2\) \(1 = 5m_2\) \(m_2 = \frac{1}{5}\)

\textbf{Case 2:} \(\frac{3 - 2m_2}{2 + 3m_2} = -1\)
\(3 - 2m_2 = -(2 + 3m_2)\) \(3 - 2m_2 = -2 - 3m_2\)
\(3 + 2 = -3m_2 + 2m_2\) \(5 = -m_2\) \(m_2 = -5\)

\textbf{Therefore: \(m_2 = \frac{1}{5}\) or \(m_2 = -5\)}

\subsubsection{Q5.3 [4 marks]}\label{q5.3-4-marks}

\textbf{Find equation of tangent to the circle
\(2x^2 + 2y^2 + 3x - 4y + 1 = 0\) at the point (-1, 2)}

\textbf{Solution}: First, let's rewrite the circle equation in standard
form: \(2x^2 + 2y^2 + 3x - 4y + 1 = 0\) Dividing by 2:
\(x^2 + y^2 + \frac{3}{2}x - 2y + \frac{1}{2} = 0\)

For a circle \(x^2 + y^2 + 2gx + 2fy + c = 0\), the equation of tangent
at point \((x_1, y_1)\) is:
\(xx_1 + yy_1 + g(x + x_1) + f(y + y_1) + c = 0\)

Comparing: \(2g = \frac{3}{2}\), so \(g = \frac{3}{4}\) \(2f = -2\), so
\(f = -1\) \(c = \frac{1}{2}\)

At point \((-1, 2)\):
\(x(-1) + y(2) + \frac{3}{4}(x + (-1)) + (-1)(y + 2) + \frac{1}{2} = 0\)

\(-x + 2y + \frac{3}{4}x - \frac{3}{4} - y - 2 + \frac{1}{2} = 0\)

\(-x + \frac{3}{4}x + 2y - y - \frac{3}{4} - 2 + \frac{1}{2} = 0\)

\(-\frac{1}{4}x + y - \frac{3}{4} - \frac{4}{2} + \frac{1}{2} = 0\)

\(-\frac{1}{4}x + y - \frac{3}{4} - 2 + \frac{1}{2} = 0\)

\(-\frac{1}{4}x + y - \frac{9}{4} = 0\)

Multiplying by 4: \(-x + 4y - 9 = 0\)

\textbf{Equation of tangent: \(x - 4y + 9 = 0\)}

\begin{center}\rule{0.5\linewidth}{0.5pt}\end{center}

\subsection*{Formula Cheat Sheet}\label{formula-cheat-sheet}

\subsubsection{\texorpdfstring{\textbf{Trigonometry}}{Trigonometry}}\label{trigonometry}

\begin{itemize}
\tightlist
\item
  \(\sin^2\theta + \cos^2\theta = 1\)
\item
  \(\tan\theta = \frac{\sin\theta}{\cos\theta}\)
\item
  \(\sin(A \pm B) = \sin A \cos B \pm \cos A \sin B\)
\item
  \(\cos(A \pm B) = \cos A \cos B \mp \sin A \sin B\)
\end{itemize}

\subsubsection{\texorpdfstring{\textbf{Limits}}{Limits}}\label{limits}

\begin{itemize}
\tightlist
\item
  \(\lim_{x \to 0} \frac{\sin x}{x} = 1\)
\item
  \(\lim_{x \to \infty} \left(1 + \frac{a}{x}\right)^{bx} = e^{ab}\)
\item
  \(\lim_{x \to a} \frac{x^n - a^n}{x - a} = na^{n-1}\)
\end{itemize}

\subsubsection{\texorpdfstring{\textbf{Vectors}}{Vectors}}\label{vectors}

\begin{itemize}
\tightlist
\item
  Dot product: \(\vec{a} \cdot \vec{b} = |\vec{a}||\vec{b}|\cos\theta\)
\item
  Cross product:
  \(|\vec{a} \times \vec{b}| = |\vec{a}||\vec{b}|\sin\theta\)
\item
  Work done: \(W = \vec{F} \cdot \vec{s}\)
\end{itemize}

\subsubsection{\texorpdfstring{\textbf{Circle}}{Circle}}\label{circle}

\begin{itemize}
\tightlist
\item
  Standard form: \((x-h)^2 + (y-k)^2 = r^2\)
\item
  Area: \(\pi r^2\)
\item
  Tangent at \((x_1, y_1)\):
  \(xx_1 + yy_1 + g(x+x_1) + f(y+y_1) + c = 0\)
\end{itemize}

\subsection*{Problem-solving
Strategies}\label{problem-solving-strategies}

\textbf{For Determinants:}

\begin{itemize}
\tightlist
\item
  Expand along the row/column with most zeros
\item
  Factor out common terms first
\end{itemize}

\textbf{For Limits:}

\begin{itemize}
\tightlist
\item
  Check for \(\frac{0}{0}\) or \(\frac{\infty}{\infty}\) forms
\item
  Use L'Hôpital's rule or factorization
\item
  Recognize standard limit forms
\end{itemize}

\textbf{For Vectors:}

\begin{itemize}
\tightlist
\item
  Use component form for calculations
\item
  Remember cross product gives perpendicular vector
\item
  Dot product = 0 for perpendicular vectors
\end{itemize}

\subsection*{Common Mistakes to Avoid}\label{common-mistakes-to-avoid}

\begin{itemize}
\tightlist
\item
  \textbf{Sign errors} in determinant expansion
\item
  \textbf{Forgetting degree-radian conversion}: \(180^\circ = \pi\) radians
\item
  \textbf{Not simplifying} trigonometric expressions using identities
\item
  \textbf{Wrong limit evaluation} - always check if direct substitution
  works first
\item
  \textbf{Vector operations} - don't confuse dot and cross products
\end{itemize}

\subsection*{Exam Tips}\label{exam-tips}

\begin{itemize}
\tightlist
\item
  \textbf{Time management}: Spend 1-2 minutes per mark
\item
  \textbf{Show all steps} for partial credit
\item
  \textbf{Check answers} by substitution where possible
\item
  \textbf{Use standard values} for trigonometric functions
\item
  \textbf{Draw diagrams} for vector and geometry problems
\end{itemize}


\end{document}
