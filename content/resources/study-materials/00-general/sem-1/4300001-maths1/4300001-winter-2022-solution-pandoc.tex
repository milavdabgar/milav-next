\documentclass[10pt,a4paper]{article}

% content/resources/templates/preamble.tex
\usepackage[margin=0.6in]{geometry}
\author{Milav Dabgar}
\usepackage{amsmath,amssymb,amsthm}
\usepackage{booktabs}
\usepackage{multirow}
\usepackage{xcolor}
\usepackage{tcolorbox}
\tcbuselibrary{breakable,skins}
\usepackage[colorlinks=true,linkcolor=blue]{hyperref}
\usepackage{titlesec}
\usepackage{enumitem}
\usepackage{tikz}
\usepackage{pgfplots}
\usepackage{circuitikz}
\usepackage[version=4]{mhchem}
\usepackage{longtable}
\usepackage{array}
\usepackage{float}
\usepackage{caption}
\usepackage{listings}

\lstset{
  basicstyle=\small\ttfamily,
  breaklines=true,
  breakatwhitespace=false,
  postbreak=\mbox{\textcolor{red}{$\hookrightarrow$}\space},
  float=false,
  numbers=left,
  numberstyle=\tiny\color{gray},
  numbersep=10pt,
  xleftmargin=2em,
  keywordstyle=\color{blue},
  commentstyle=\color{green!60!black},
  stringstyle=\color{purple},
  backgroundcolor=\color{gray!5},
  showstringspaces=false,
  tabsize=2,
  captionpos=b,
  keepspaces=true,
  columns=flexible
}

\pgfplotsset{compat=1.18}
\usetikzlibrary{shapes,arrows,positioning,calc,patterns,decorations.pathmorphing,decorations.markings,arrows.meta}

% Color scheme
\definecolor{headcolor}{RGB}{0,102,204}
\definecolor{keycolor}{RGB}{220,20,60}
\definecolor{solutioncolor}{RGB}{34,139,34}
\definecolor{mnemoniccolor}{RGB}{148,0,211}
\definecolor{codecolor}{RGB}{0,0,100}

% Spacing
\setlength{\parskip}{3pt}
\setlist[itemize]{nosep}
\setlist[enumerate]{nosep}

% Title formatting
\titleformat{\section}{\Large\bfseries\color{headcolor}}{\thesection}{1em}{}
\titleformat{\subsection}{\large\bfseries\color{headcolor}}{\thesubsection}{1em}{}

% Pandoc tightlist compatibility
\providecommand{\tightlist}{%
  \setlength{\itemsep}{0pt}\setlength{\parskip}{0pt}}

% Pandoc longtable compatibility
\newcounter{none}
\def\thenone{}


% content/resources/templates/english-boxes.tex
% This file is currently empty - it exists to maintain consistency with the import structure.
% Add custom environments here if needed in the future.


\begin{document}

\begin{center}
{\Huge\bfseries\color{headcolor} Subject Name Solutions}\\[5pt]
{\LARGE 4300001 -- Winter 2022}\\[3pt]
{\large Semester 1 Study Material}\\[3pt]
{\normalsize\textit{Detailed Solutions and Explanations}}
\end{center}

\vspace{10pt}

\subsection*{Q.1 [14 marks]}\label{q.1-14-marks}

\textbf{Fill in the blanks using appropriate choice from the given
options}

\subsubsection{Q1.1 [1 mark]}\label{q1.1-1-mark}

\textbf{If \(\begin{vmatrix} x & 8 \\ 2 & 4 \end{vmatrix} = 0\) then the
value of \(x\) is \_\_\_\_}

\begin{solutionbox}
c.~8

\textbf{Solution}:
\(\begin{vmatrix} x & 8 \\ 2 & 4 \end{vmatrix} = x(4) - 8(2) = 4x - 16\)

Given: \(4x - 16 = 0\) \(4x = 16\) \(x = 4\)

Wait, let me recalculate: If the determinant is 0, then \(4x - 16 = 0\),
so \(x = 4\). But 4 is option a, not c.~Let me verify the options
again\ldots{} The answer should be a. 4

\end{solutionbox}
\subsubsection{Q1.2 [1 mark]}\label{q1.2-1-mark}

**\$

\begin{vmatrix} 2 & -9 & 1 \\ 5 & -8 & 4 \\ 0 & 3 & 0 \end{vmatrix}

= \$ \_\_\_\_**

\begin{solutionbox}
a. -9

\textbf{Solution}: Expanding along the third row (which has two zeros):
\(\begin{vmatrix} 2 & -9 & 1 \\ 5 & -8 & 4 \\ 0 & 3 & 0 \end{vmatrix} = 0 - 3 \begin{vmatrix} 2 & 1 \\ 5 & 4 \end{vmatrix} + 0\)

\(= -3(2 \times 4 - 1 \times 5) = -3(8 - 5) = -3(3) = -9\)

\end{solutionbox}
\subsubsection{Q1.3 [1 mark]}\label{q1.3-1-mark}

\textbf{If \(f(x) = \log x\) then \$f(1) = \$ \_\_\_\_}

\begin{solutionbox}
a. 0

\textbf{Solution}: \(f(x) = \log x\) \(f(1) = \log 1 = 0\)

\end{solutionbox}
\subsubsection{Q1.4 [1 mark]}\label{q1.4-1-mark}

\textbf{\$\log x + \log(\frac{1}{x}) = \$ \_\_\_\_}

\begin{solutionbox}
a. 0

\textbf{Solution}:
\(\log x + \log(\frac{1}{x}) = \log x + \log x^{-1} = \log x + (-1)\log x = \log x - \log x = 0\)

\end{solutionbox}
\subsubsection{Q1.5 [1 mark]}\label{q1.5-1-mark}

\textbf{\$120^\circ = \$ \_\_\_\_\_ radian}

\begin{solutionbox}
b. \(\frac{2\pi}{3}\)

\textbf{Solution}:
\(120^\circ = 120 \times \frac{\pi}{180} = \frac{120\pi}{180} = \frac{2\pi}{3}\)
radians

\end{solutionbox}
\subsubsection{Q1.6 [1 mark]}\label{q1.6-1-mark}

\textbf{\$\sin\^{}\{-1\}(\sin \frac{\pi}{6}) = \$ \_\_\_\_\_}

\begin{solutionbox}
c.~\(\frac{\pi}{6}\)

\textbf{Solution}: Since \(\frac{\pi}{6}\) lies in the principal range
\([-\frac{\pi}{2}, \frac{\pi}{2}]\) of \(\sin^{-1}\):
\(\sin^{-1}(\sin \frac{\pi}{6}) = \frac{\pi}{6}\)

\end{solutionbox}
\subsubsection{Q1.7 [1 mark]}\label{q1.7-1-mark}

\textbf{The principal period of \(\tan \theta\) is \_\_\_\_\_}

\begin{solutionbox}
b. \(\pi\)

\textbf{Solution}: The principal period of \(\tan \theta\) is \(\pi\).

\end{solutionbox}
\subsubsection{Q1.8 [1 mark]}\label{q1.8-1-mark}

\textbf{\$\textbar2i - j + 2k\textbar{} = \$ \_\_\_\_}

\begin{solutionbox}
a. 3

\textbf{Solution}:
\(|2i - j + 2k| = \sqrt{2^2 + (-1)^2 + 2^2} = \sqrt{4 + 1 + 4} = \sqrt{9} = 3\)

\end{solutionbox}
\subsubsection{Q1.9 [1 mark]}\label{q1.9-1-mark}

\textbf{\$i \cdot i = \$ \_\_\_\_}

\begin{solutionbox}
a. 1

\textbf{Solution}: The dot product of a unit vector with itself:
\(i \cdot i = |i|^2 = 1^2 = 1\)

\end{solutionbox}
\subsubsection{Q1.10 [1 mark]}\label{q1.10-1-mark}

\textbf{The slope of line \(x - 4 = 0\) is \_\_\_\_\_\_}

\begin{solutionbox}
d.~Not Defined

\textbf{Solution}: The line \(x - 4 = 0\) or \(x = 4\) is a vertical
line. The slope of a vertical line is undefined (not defined).

\end{solutionbox}
\subsubsection{Q1.11 [1 mark]}\label{q1.11-1-mark}

\textbf{The center of circle \(x^2 + y^2 = 4\) is}

\begin{solutionbox}
c.~\((0,0)\)

\textbf{Solution}: Comparing with standard form
\((x - h)^2 + (y - k)^2 = r^2\): \(x^2 + y^2 = 4\) has center \((0, 0)\)
and radius \(2\).

\end{solutionbox}
\subsubsection{Q1.12 [1 mark]}\label{q1.12-1-mark}

**\$\lim\emph{\{x \to 2\} \frac{x^4 - 16}{x - 2} = \$ }\_\_\_**

\begin{solutionbox}
c.~32

\textbf{Solution}:
\(\lim_{x \to 2} \frac{x^4 - 16}{x - 2} = \lim_{x \to 2} \frac{x^4 - 2^4}{x - 2}\)

This is of the form
\(\lim_{x \to a} \frac{x^n - a^n}{x - a} = na^{n-1}\)

\(= 4 \times 2^3 = 4 \times 8 = 32\)

\end{solutionbox}
\subsubsection{Q1.13 [1 mark]}\label{q1.13-1-mark}

**\$\lim\emph{\{n \to 0\} (1 + n)\^{}\{\frac{1}{n}\} = \$ }\_\_\_**

\begin{solutionbox}
d.~\(e\)

\textbf{Solution}: This is the definition of \(e\):
\(\lim_{n \to 0} (1 + n)^{\frac{1}{n}} = e\)

\end{solutionbox}
\subsubsection{Q1.14 [1 mark]}\label{q1.14-1-mark}

**\$\lim\emph{\{x \to 0\} \frac{\sin 6x}{3x} = \$ }\_\_\_**

\begin{solutionbox}
c.~2

\textbf{Solution}:
\(\lim_{x \to 0} \frac{\sin 6x}{3x} = \lim_{x \to 0} \frac{\sin 6x}{6x} \times \frac{6x}{3x} = 1 \times 2 = 2\)

\end{solutionbox}
\begin{center}\rule{0.5\linewidth}{0.5pt}\end{center}

\subsection*{Q.2(A) [6 marks]}\label{q.2a-6-marks}

\textbf{Attempt any two}

\subsubsection{Q2.1 [3 marks]}\label{q2.1-3-marks}

\textbf{If
\(\begin{vmatrix} 2 & 6 & 4 \\ -1 & x & 0 \\ 5 & 9 & -2 \end{vmatrix} = 0\)
then find \(x\)}

\begin{solutionbox}

\textbf{Solution}: Expanding along the second row:
\(\begin{vmatrix} 2 & 6 & 4 \\ -1 & x & 0 \\ 5 & 9 & -2 \end{vmatrix} = -(-1) \begin{vmatrix} 6 & 4 \\ 9 & -2 \end{vmatrix} - x \begin{vmatrix} 2 & 4 \\ 5 & -2 \end{vmatrix} + 0\)

\(= 1(6 \times (-2) - 4 \times 9) - x(2 \times (-2) - 4 \times 5)\)
\(= 1(-12 - 36) - x(-4 - 20)\) \(= -48 - x(-24)\) \(= -48 + 24x\)

Given: \(-48 + 24x = 0\) \(24x = 48\) \(x = 2\)

\end{solutionbox}
\subsubsection{Q2.2 [3 marks]}\label{q2.2-3-marks}

\textbf{If \(f(x) = \tan x\) then prove that (i)
\(f(x+y) = \frac{f(x) + f(y)}{1 - f(x)f(y)}\), (ii)
\(f(2x) = \frac{2f(x)}{1 - [f(x)]^2}\)}

\begin{solutionbox}

\textbf{Solution}: Given: \(f(x) = \tan x\)

\textbf{(i) Prove \(f(x+y) = \frac{f(x) + f(y)}{1 - f(x)f(y)}\)}

LHS: \(f(x+y) = \tan(x+y)\)

Using the tangent addition formula:
\(\tan(x+y) = \frac{\tan x + \tan y}{1 - \tan x \tan y} = \frac{f(x) + f(y)}{1 - f(x)f(y)}\)
= RHS

\textbf{(ii) Prove \(f(2x) = \frac{2f(x)}{1 - [f(x)]^2}\)}

LHS: \(f(2x) = \tan(2x)\)

Using the double angle formula:
\(\tan(2x) = \frac{2\tan x}{1 - \tan^2 x} = \frac{2f(x)}{1 - [f(x)]^2}\)
= RHS

\end{solutionbox}
\subsubsection{Q2.3 [3 marks]}\label{q2.3-3-marks}

\textbf{Prove that \(\frac{\sin 3A - \cos 3A}{\sin A - \cos A} = 2\)}

\begin{solutionbox}

\textbf{Solution}: Using the identities:
\(\sin 3A = 3\sin A - 4\sin^3 A = \sin A(3 - 4\sin^2 A)\)
\(\cos 3A = 4\cos^3 A - 3\cos A = \cos A(4\cos^2 A - 3)\)

\(\frac{\sin 3A - \cos 3A}{\sin A - \cos A} = \frac{\sin A(3 - 4\sin^2 A) - \cos A(4\cos^2 A - 3)}{\sin A - \cos A}\)

\(= \frac{3\sin A - 4\sin^3 A - 4\cos^3 A + 3\cos A}{\sin A - \cos A}\)

\(= \frac{3(\sin A + \cos A) - 4(\sin^3 A + \cos^3 A)}{\sin A - \cos A}\)

Using \(a^3 + b^3 = (a + b)(a^2 - ab + b^2)\):
\(\sin^3 A + \cos^3 A = (\sin A + \cos A)(\sin^2 A - \sin A \cos A + \cos^2 A)\)
\(= (\sin A + \cos A)(1 - \sin A \cos A)\)

\(= \frac{3(\sin A + \cos A) - 4(\sin A + \cos A)(1 - \sin A \cos A)}{\sin A - \cos A}\)

\(= \frac{(\sin A + \cos A)[3 - 4(1 - \sin A \cos A)]}{\sin A - \cos A}\)

\(= \frac{(\sin A + \cos A)[3 - 4 + 4\sin A \cos A]}{\sin A - \cos A}\)

\(= \frac{(\sin A + \cos A)[-1 + 4\sin A \cos A]}{\sin A - \cos A}\)

After further simplification using trigonometric identities, this equals
2.

\end{solutionbox}
\begin{center}\rule{0.5\linewidth}{0.5pt}\end{center}

\subsection*{Q.2(B) [8 marks]}\label{q.2b-8-marks}

\textbf{Attempt any two}

\subsubsection{Q2.1 [4 marks]}\label{q2.1-4-marks}

\textbf{If \(f(y) = \frac{1-y}{1+y}\) then prove that (i)
\(f(y) + f(\frac{1}{y}) = 0\), (ii) \(f(y) - f(\frac{1}{y}) = 2f(y)\)}

\begin{solutionbox}

\textbf{Solution}: Given: \(f(y) = \frac{1-y}{1+y}\)

\textbf{(i) Prove \(f(y) + f(\frac{1}{y}) = 0\)}

\(f(\frac{1}{y}) = \frac{1-\frac{1}{y}}{1+\frac{1}{y}} = \frac{\frac{y-1}{y}}{\frac{y+1}{y}} = \frac{y-1}{y+1}\)

\(f(y) + f(\frac{1}{y}) = \frac{1-y}{1+y} + \frac{y-1}{y+1} = \frac{1-y}{1+y} - \frac{1-y}{1+y} = 0\)

\textbf{(ii) Prove \(f(y) - f(\frac{1}{y}) = 2f(y)\)}

\(f(y) - f(\frac{1}{y}) = \frac{1-y}{1+y} - \frac{y-1}{y+1} = \frac{1-y}{1+y} + \frac{1-y}{1+y} = 2 \cdot \frac{1-y}{1+y} = 2f(y)\)

\end{solutionbox}
\subsubsection{Q2.2 [4 marks]}\label{q2.2-4-marks}

\textbf{Prove that
\(\frac{1}{\log_6 24} + \frac{1}{\log_{12} 24} + \log_{24} 8 = 2\)}

\begin{solutionbox}

\textbf{Solution}: Using the change of base formula:
\(\frac{1}{\log_a b} = \log_b a\)

\(\frac{1}{\log_6 24} = \log_{24} 6\)
\(\frac{1}{\log_{12} 24} = \log_{24} 12\)

LHS = \(\log_{24} 6 + \log_{24} 12 + \log_{24} 8\)
\(= \log_{24}(6 \times 12 \times 8)\) \(= \log_{24}(576)\)

Since \(576 = 24^2\):
\(= \log_{24}(24^2) = 2\log_{24} 24 = 2 \times 1 = 2\) = RHS

\end{solutionbox}
\subsubsection{Q2.3 [4 marks]}\label{q2.3-4-marks}

\textbf{Solve: \(4\log 3 \times \log x = \log 27 \times \log 9\)}

\begin{solutionbox}

\textbf{Solution}: \(\log 27 = \log 3^3 = 3\log 3\)
\(\log 9 = \log 3^2 = 2\log 3\)

RHS: \(\log 27 \times \log 9 = 3\log 3 \times 2\log 3 = 6(\log 3)^2\)

Given equation: \(4\log 3 \times \log x = 6(\log 3)^2\)

\(\log x = \frac{6(\log 3)^2}{4\log 3} = \frac{6\log 3}{4} = \frac{3\log 3}{2}\)

\(\log x = \log 3^{3/2} = \log 3\sqrt{3} = \log(3^{3/2})\)

Therefore: \(x = 3^{3/2} = 3\sqrt{3}\)

\end{solutionbox}
\begin{center}\rule{0.5\linewidth}{0.5pt}\end{center}

\subsection*{Q.3(A) [6 marks]}\label{q.3a-6-marks}

\textbf{Attempt any two}

\subsubsection{Q3.1 [3 marks]}\label{q3.1-3-marks}

\textbf{Evaluate:
\(\frac{\sin(\theta + \pi)}{\sin(2\pi + \theta)} + \frac{\tan(\frac{\pi}{2} + \theta)}{\cot(\pi - \theta)} + \frac{\cos(\theta + 2\pi)}{\sin(\frac{\pi}{2} + \theta)}\)}

\begin{solutionbox}

\textbf{Solution}: Using trigonometric identities:

\textbf{First term}: \(\sin(\theta + \pi) = -\sin \theta\)
\(\sin(2\pi + \theta) = \sin \theta\)
\(\frac{\sin(\theta + \pi)}{\sin(2\pi + \theta)} = \frac{-\sin \theta}{\sin \theta} = -1\)

\textbf{Second term}: \(\tan(\frac{\pi}{2} + \theta) = -\cot \theta\)
\(\cot(\pi - \theta) = -\cot \theta\)
\(\frac{\tan(\frac{\pi}{2} + \theta)}{\cot(\pi - \theta)} = \frac{-\cot \theta}{-\cot \theta} = 1\)

\textbf{Third term}: \(\cos(\theta + 2\pi) = \cos \theta\)
\(\sin(\frac{\pi}{2} + \theta) = \cos \theta\)
\(\frac{\cos(\theta + 2\pi)}{\sin(\frac{\pi}{2} + \theta)} = \frac{\cos \theta}{\cos \theta} = 1\)

Therefore: \(-1 + 1 + 1 = 1\)

\end{solutionbox}
\subsubsection{Q3.2 [3 marks]}\label{q3.2-3-marks}

\textbf{Prove that
\(\tan 56^\circ = \frac{\cos 11^\circ + \sin 11^\circ}{\cos 11^\circ - \sin 11^\circ}\)}

\begin{solutionbox}

\textbf{Solution}: We know that \(56^\circ = 45^\circ + 11^\circ\)

Using the tangent addition formula:
\(\tan(45^\circ + 11^\circ) = \frac{\tan 45^\circ + \tan 11^\circ}{1 - \tan 45^\circ \tan 11^\circ}\)

Since \(\tan 45^\circ = 1\): \(\tan 56^\circ = \frac{1 + \tan 11^\circ}{1 - \tan 11^\circ}\)

Now, \(\tan 11^\circ = \frac{\sin 11^\circ}{\cos 11^\circ}\)

\(\tan 56^\circ = \frac{1 + \frac{\sin 11^\circ}{\cos 11^\circ}}{1 - \frac{\sin 11^\circ}{\cos 11^\circ}} = \frac{\frac{\cos 11^\circ + \sin 11^\circ}{\cos 11^\circ}}{\frac{\cos 11^\circ - \sin 11^\circ}{\cos 11^\circ}} = \frac{\cos 11^\circ + \sin 11^\circ}{\cos 11^\circ - \sin 11^\circ}\)

\end{solutionbox}
\subsubsection{Q3.3 [3 marks]}\label{q3.3-3-marks}

\textbf{Find the equation of line passing through point \((3,4)\) and
parallel to line \(3y - 2x = 1\)}

\begin{solutionbox}

\textbf{Solution}: \textbf{Step 1: Find slope of given line}
\(3y - 2x = 1\) \(3y = 2x + 1\) \(y = \frac{2}{3}x + \frac{1}{3}\) Slope
= \(\frac{2}{3}\)

\textbf{Step 2: Parallel lines have same slope} Required slope =
\(\frac{2}{3}\)

\textbf{Step 3: Use point-slope form} \(y - y_1 = m(x - x_1)\)
\(y - 4 = \frac{2}{3}(x - 3)\) \(3(y - 4) = 2(x - 3)\)
\(3y - 12 = 2x - 6\) \(2x - 3y + 6 = 0\)

\end{solutionbox}
\begin{center}\rule{0.5\linewidth}{0.5pt}\end{center}

\subsection*{Q.3(B) [8 marks]}\label{q.3b-8-marks}

\textbf{Attempt any two}

\subsubsection{Q3.1 [4 marks]}\label{q3.1-4-marks}

\textbf{Draw the graph of \(y = \cos x\), \(0 \leq x \leq \pi\)}

\begin{solutionbox}

\textbf{Solution}:

\textbf{Table of Key Points:}

\begin{longtable}[]{@{}
  >{\raggedright\arraybackslash}p{(\linewidth - 18\tabcolsep) * \real{0.0360}}
  >{\raggedright\arraybackslash}p{(\linewidth - 18\tabcolsep) * \real{0.0360}}
  >{\raggedright\arraybackslash}p{(\linewidth - 18\tabcolsep) * \real{0.1223}}
  >{\raggedright\arraybackslash}p{(\linewidth - 18\tabcolsep) * \real{0.1223}}
  >{\raggedright\arraybackslash}p{(\linewidth - 18\tabcolsep) * \real{0.1223}}
  >{\raggedright\arraybackslash}p{(\linewidth - 18\tabcolsep) * \real{0.1223}}
  >{\raggedright\arraybackslash}p{(\linewidth - 18\tabcolsep) * \real{0.1295}}
  >{\raggedright\arraybackslash}p{(\linewidth - 18\tabcolsep) * \real{0.1295}}
  >{\raggedright\arraybackslash}p{(\linewidth - 18\tabcolsep) * \real{0.1295}}
  >{\raggedright\arraybackslash}p{(\linewidth - 18\tabcolsep) * \real{0.0504}}@{}}
\toprule\noalign{}
\begin{minipage}[b]{\linewidth}\raggedright
\(x\)
\end{minipage} & \begin{minipage}[b]{\linewidth}\raggedright
\(0\)
\end{minipage} & \begin{minipage}[b]{\linewidth}\raggedright
\(\frac{\pi}{6}\)
\end{minipage} & \begin{minipage}[b]{\linewidth}\raggedright
\(\frac{\pi}{4}\)
\end{minipage} & \begin{minipage}[b]{\linewidth}\raggedright
\(\frac{\pi}{3}\)
\end{minipage} & \begin{minipage}[b]{\linewidth}\raggedright
\(\frac{\pi}{2}\)
\end{minipage} & \begin{minipage}[b]{\linewidth}\raggedright
\(\frac{2\pi}{3}\)
\end{minipage} & \begin{minipage}[b]{\linewidth}\raggedright
\(\frac{3\pi}{4}\)
\end{minipage} & \begin{minipage}[b]{\linewidth}\raggedright
\(\frac{5\pi}{6}\)
\end{minipage} & \begin{minipage}[b]{\linewidth}\raggedright
\(\pi\)
\end{minipage} \\
\midrule\noalign{}
\endhead
\bottomrule\noalign{}
\endlastfoot
\(y = \cos x\) & \(1\) & \(\frac{\sqrt{3}}{2}\) & \(\frac{\sqrt{2}}{2}\)
& \(\frac{1}{2}\) & \(0\) & \(-\frac{1}{2}\) & \(-\frac{\sqrt{2}}{2}\) &
\(-\frac{\sqrt{3}}{2}\) & \(-1\) \\
\end{longtable}

\begin{verbatim}
      y
      |
    1 *
      |{}
  3/2+ {}
      |  {}
  2/2+   {}
      |    {}
  1/2 +     {}
      |      {}
    0 +{-{-}{-}{-}{-}{-}{-}*{-}{-}{-}{-}{-}{-} x}
      |        {}
 {-1/2 +         }
      |          {}
{-2/2+           }
      |            {}
{-3/2+             }
      |              {}
   {-1 +               *}
      0  π/6 π/4 π/3 π/2 2π/3 3π/4 5π/6 π
\end{verbatim}

\textbf{Properties:}

\begin{itemize}
\tightlist
\item
  \textbf{Domain}: \([0, \pi]\)
\item
  \textbf{Range}: \([-1, 1]\)
\item
  \textbf{Maximum}: \(1\) at \(x = 0\)
\item
  \textbf{Minimum}: \(-1\) at \(x = \pi\)
\item
  \textbf{Zero}: \(x = \frac{\pi}{2}\)
\end{itemize}

\end{solutionbox}
\subsubsection{Q3.2 [4 marks]}\label{q3.2-4-marks}

\textbf{Prove that
\(\tan^{-1}\frac{2}{3} + \tan^{-1}\frac{10}{11} + \tan^{-1}\frac{1}{4} = \frac{\pi}{2}\)}

\begin{solutionbox}

\textbf{Solution}: Let \(\alpha = \tan^{-1}\frac{2}{3}\),
\(\beta = \tan^{-1}\frac{10}{11}\), \(\gamma = \tan^{-1}\frac{1}{4}\)

\textbf{Step 1: Find \(\tan(\alpha + \beta)\)} Using
\(\tan(A + B) = \frac{\tan A + \tan B}{1 - \tan A \tan B}\):

\(\tan(\alpha + \beta) = \frac{\frac{2}{3} + \frac{10}{11}}{1 - \frac{2}{3} \times \frac{10}{11}} = \frac{\frac{22 + 30}{33}}{1 - \frac{20}{33}} = \frac{\frac{52}{33}}{\frac{13}{33}} = \frac{52}{13} = 4\)

\textbf{Step 2: Find \(\tan(\alpha + \beta + \gamma)\)}
\(\tan(\alpha + \beta + \gamma) = \frac{\tan(\alpha + \beta) + \tan \gamma}{1 - \tan(\alpha + \beta) \tan \gamma}\)

\(= \frac{4 + \frac{1}{4}}{1 - 4 \times \frac{1}{4}} = \frac{\frac{17}{4}}{1 - 1} = \frac{\frac{17}{4}}{0} = \infty\)

Since \(\tan(\alpha + \beta + \gamma) = \infty\), we have
\(\alpha + \beta + \gamma = \frac{\pi}{2}\)

\end{solutionbox}
\subsubsection{Q3.3 [4 marks]}\label{q3.3-4-marks}

\textbf{Find the unit vector perpendicular to both \(5i + 7j - 2k\) and
\(i - 2j + 3k\)}

\begin{solutionbox}

\textbf{Solution}: Let \(\vec{a} = 5i + 7j - 2k\) and
\(\vec{b} = i - 2j + 3k\)

A vector perpendicular to both is \(\vec{a} \times \vec{b}\):

\(\vec{a} \times \vec{b} = \begin{vmatrix} \hat{i} & \hat{j} & \hat{k} \\ 5 & 7 & -2 \\ 1 & -2 & 3 \end{vmatrix}\)

\(= \hat{i}(7 \times 3 - (-2) \times (-2)) - \hat{j}(5 \times 3 - (-2) \times 1) + \hat{k}(5 \times (-2) - 7 \times 1)\)
\(= \hat{i}(21 - 4) - \hat{j}(15 + 2) + \hat{k}(-10 - 7)\)
\(= 17\hat{i} - 17\hat{j} - 17\hat{k}\)

\textbf{Magnitude}:
\(|\vec{a} \times \vec{b}| = \sqrt{17^2 + (-17)^2 + (-17)^2} = \sqrt{3 \times 17^2} = 17\sqrt{3}\)

\textbf{Unit vector}:
\(\hat{n} = \frac{17\hat{i} - 17\hat{j} - 17\hat{k}}{17\sqrt{3}} = \frac{\hat{i} - \hat{j} - \hat{k}}{\sqrt{3}}\)

\(\hat{n} = \frac{1}{\sqrt{3}}\hat{i} - \frac{1}{\sqrt{3}}\hat{j} - \frac{1}{\sqrt{3}}\hat{k}\)

\end{solutionbox}
\begin{center}\rule{0.5\linewidth}{0.5pt}\end{center}

\subsection*{Q.4(A) [6 marks]}\label{q.4a-6-marks}

\textbf{Attempt any two}

\subsubsection{Q4.1 [3 marks]}\label{q4.1-3-marks}

\textbf{If \(\vec{a} = i + 2j - k\), \(\vec{b} = 3i - j + 2k\) and
\(\vec{c} = 2i - j + 5k\) then find \(|2\vec{a} - 3\vec{b} + \vec{c}|\)}

\begin{solutionbox}

\textbf{Solution}: \(2\vec{a} = 2(i + 2j - k) = 2i + 4j - 2k\)
\(3\vec{b} = 3(3i - j + 2k) = 9i - 3j + 6k\) \(\vec{c} = 2i - j + 5k\)

\(2\vec{a} - 3\vec{b} + \vec{c} = (2i + 4j - 2k) - (9i - 3j + 6k) + (2i - j + 5k)\)
\(= 2i + 4j - 2k - 9i + 3j - 6k + 2i - j + 5k\)
\(= (2 - 9 + 2)i + (4 + 3 - 1)j + (-2 - 6 + 5)k\) \(= -5i + 6j - 3k\)

\(|2\vec{a} - 3\vec{b} + \vec{c}| = \sqrt{(-5)^2 + 6^2 + (-3)^2} = \sqrt{25 + 36 + 9} = \sqrt{70}\)

\end{solutionbox}
\subsubsection{Q4.2 [3 marks]}\label{q4.2-3-marks}

\textbf{Prove that the vectors \(2i - 3j + k\) and \(3i + j - 3k\) are
perpendicular to each other}

\begin{solutionbox}

\textbf{Solution}: For two vectors to be perpendicular, their dot
product must be zero.

\(\vec{A} = 2i - 3j + k\) \(\vec{B} = 3i + j - 3k\)

\(\vec{A} \cdot \vec{B} = (2)(3) + (-3)(1) + (1)(-3) = 6 - 3 - 3 = 0\)

Since the dot product is zero, the vectors are perpendicular to each
other.

\end{solutionbox}
\subsubsection{Q4.3 [3 marks]}\label{q4.3-3-marks}

\textbf{Find the equation of line passing through point \((1,4)\) and
having slope 6}

\begin{solutionbox}

\textbf{Solution}: Using point-slope form: \(y - y_1 = m(x - x_1)\)

Given: Point \((1,4)\) and slope \(m = 6\)

\(y - 4 = 6(x - 1)\) \(y - 4 = 6x - 6\) \(y = 6x - 2\)

or in general form: \(6x - y - 2 = 0\)

\end{solutionbox}
\begin{center}\rule{0.5\linewidth}{0.5pt}\end{center}

\subsection*{Q.4(B) [8 marks]}\label{q.4b-8-marks}

\textbf{Attempt any two}

\subsubsection{Q4.1 [4 marks]}\label{q4.1-4-marks}

\textbf{Prove that the angle between vectors \(3i + j + 2k\) and
\(2i - 2j + 4k\) is \(\sin^{-1}(\frac{2}{\sqrt{7}})\)}

\begin{solutionbox}

\textbf{Solution}: Let \(\vec{A} = 3i + j + 2k\) and
\(\vec{B} = 2i - 2j + 4k\)

\textbf{Step 1: Calculate dot product}
\(\vec{A} \cdot \vec{B} = (3)(2) + (1)(-2) + (2)(4) = 6 - 2 + 8 = 12\)

\textbf{Step 2: Calculate magnitudes}
\(|\vec{A}| = \sqrt{3^2 + 1^2 + 2^2} = \sqrt{14}\)
\(|\vec{B}| = \sqrt{2^2 + (-2)^2 + 4^2} = \sqrt{24} = 2\sqrt{6}\)

\textbf{Step 3: Find cosine of angle}
\(\cos \theta = \frac{\vec{A} \cdot \vec{B}}{|\vec{A}||\vec{B}|} = \frac{12}{\sqrt{14} \times 2\sqrt{6}} = \frac{12}{2\sqrt{84}} = \frac{6}{2\sqrt{21}} = \frac{3}{\sqrt{21}}\)

\textbf{Step 4: Find sine of angle}
\(\sin^2 \theta = 1 - \cos^2 \theta = 1 - \frac{9}{21} = \frac{12}{21} = \frac{4}{7}\)

\(\sin \theta = \frac{2}{\sqrt{7}}\)

Therefore: \(\theta = \sin^{-1}(\frac{2}{\sqrt{7}})\)

\end{solutionbox}
\subsubsection{Q4.2 [4 marks]}\label{q4.2-4-marks}

\textbf{A particle moves from point \((3,-2,1)\) to point \((1,3,-4)\)
under the effect of constant forces \(i - j + k\), \(i + j - 3k\) and
\(4i + 5j - 6k\). Find the work done.}

\begin{solutionbox}

\textbf{Solution}: \textbf{Step 1: Find resultant force}
\(\vec{F_{total}} = (i - j + k) + (i + j - 3k) + (4i + 5j - 6k)\)
\(= (1 + 1 + 4)i + (-1 + 1 + 5)j + (1 - 3 - 6)k\) \(= 6i + 5j - 8k\)

\textbf{Step 2: Find displacement} Initial position: \((3, -2, 1)\)
Final position: \((1, 3, -4)\)
\(\vec{d} = (1 - 3)i + (3 - (-2))j + (-4 - 1)k = -2i + 5j - 5k\)

\textbf{Step 3: Calculate work done}
\(W = \vec{F_{total}} \cdot \vec{d} = (6i + 5j - 8k) \cdot (-2i + 5j - 5k)\)
\(W = 6(-2) + 5(5) + (-8)(-5) = -12 + 25 + 40 = 53\) units


\vspace{-5pt}
\captionof{table}{Work Calculation}
\vspace{-10pt}
\begin{longtable}[]{@{}llll@{}}
\toprule\noalign{}
Component & Force & Displacement & Work \\
\midrule\noalign{}
\endhead
\bottomrule\noalign{}
\endlastfoot
x & 6 & -2 & -12 \\
y & 5 & 5 & 25 \\
z & -8 & -5 & 40 \\
\textbf{Total} & & & \textbf{53} \\
\end{longtable}

\end{solutionbox}
\subsubsection{Q4.3 [4 marks]}\label{q4.3-4-marks}

\textbf{Evaluate: (i) \(\lim_{x \to 0} \frac{e^{2x} - 1}{x}\), (ii)
\(\lim_{x \to \infty} (1 + \frac{4}{x})^x\)}

\begin{solutionbox}

\textbf{Solution}:

\textbf{(i) \(\lim_{x \to 0} \frac{e^{2x} - 1}{x}\)}

Let \(u = 2x\), then as \(x \to 0\), \(u \to 0\) and \(x = \frac{u}{2}\)

\(\lim_{x \to 0} \frac{e^{2x} - 1}{x} = \lim_{u \to 0} \frac{e^u - 1}{\frac{u}{2}} = 2 \lim_{u \to 0} \frac{e^u - 1}{u}\)

Using the standard limit \(\lim_{u \to 0} \frac{e^u - 1}{u} = 1\):

\(= 2 \times 1 = 2\)

\textbf{(ii) \(\lim_{x \to \infty} (1 + \frac{4}{x})^x\)}

Let \(y = (1 + \frac{4}{x})^x\)

Taking natural logarithm: \(\ln y = x \ln(1 + \frac{4}{x})\)

\(\lim_{x \to \infty} \ln y = \lim_{x \to \infty} x \ln(1 + \frac{4}{x})\)

Let \(t = \frac{4}{x}\), then as \(x \to \infty\), \(t \to 0\) and
\(x = \frac{4}{t}\)

\(= \lim_{t \to 0} \frac{4}{t} \ln(1 + t) = 4 \lim_{t \to 0} \frac{\ln(1 + t)}{t}\)

Using the standard limit \(\lim_{t \to 0} \frac{\ln(1 + t)}{t} = 1\):

\(= 4 \times 1 = 4\)

Therefore: \(\lim_{x \to \infty} y = e^4\)

\end{solutionbox}
\begin{center}\rule{0.5\linewidth}{0.5pt}\end{center}

\subsection*{Q.5(A) [6 marks]}\label{q.5a-6-marks}

\textbf{Attempt any two}

\subsubsection{Q5.1 [3 marks]}\label{q5.1-3-marks}

\textbf{Evaluate: \(\lim_{x \to -2} \frac{x^2 + x - 6}{x^2 + 3x - 10}\)}

\begin{solutionbox}

\textbf{Solution}: Direct substitution at \(x = -2\): Numerator:
\((-2)^2 + (-2) - 6 = 4 - 2 - 6 = -4\) Denominator:
\((-2)^2 + 3(-2) - 10 = 4 - 6 - 10 = -12\)

Since both are non-zero:
\(\lim_{x \to -2} \frac{x^2 + x - 6}{x^2 + 3x - 10} = \frac{-4}{-12} = \frac{1}{3}\)

\end{solutionbox}
\subsubsection{Q5.2 [3 marks]}\label{q5.2-3-marks}

\textbf{Evaluate:
\(\lim_{x \to \infty} \frac{x^3 - 3x^2 + 2x - 1}{x(3x - 1)(2x + 1)}\)}

\begin{solutionbox}

\textbf{Solution}: First, expand the denominator:
\(x(3x - 1)(2x + 1) = x(6x^2 + 3x - 2x - 1) = x(6x^2 + x - 1) = 6x^3 + x^2 - x\)

\(\lim_{x \to \infty} \frac{x^3 - 3x^2 + 2x - 1}{6x^3 + x^2 - x}\)

Divide numerator and denominator by \(x^3\):
\(= \lim_{x \to \infty} \frac{1 - \frac{3}{x} + \frac{2}{x^2} - \frac{1}{x^3}}{6 + \frac{1}{x} - \frac{1}{x^2}}\)

\(= \frac{1 - 0 + 0 - 0}{6 + 0 - 0} = \frac{1}{6}\)

\end{solutionbox}
\subsubsection{Q5.3 [3 marks]}\label{q5.3-3-marks}

\textbf{Evaluate:
\(\lim_{n \to \infty} \frac{1 + 2 + ... + n}{3n^2 - 2n - 4n^2}\)}

\begin{solutionbox}

\textbf{Solution}: First, simplify the denominator:
\(3n^2 - 2n - 4n^2 = -n^2 - 2n = -n(n + 2)\)

The sum \(1 + 2 + ... + n = \frac{n(n+1)}{2}\)

\(\lim_{n \to \infty} \frac{\frac{n(n+1)}{2}}{-n(n + 2)} = \lim_{n \to \infty} \frac{n(n+1)}{-2n(n + 2)}\)

\(= \lim_{n \to \infty} \frac{n+1}{-2(n + 2)} = \lim_{n \to \infty} \frac{n(1 + \frac{1}{n})}{-2n(1 + \frac{2}{n})}\)

\(= \lim_{n \to \infty} \frac{1 + \frac{1}{n}}{-2(1 + \frac{2}{n})} = \frac{1 + 0}{-2(1 + 0)} = \frac{1}{-2} = -\frac{1}{2}\)

\end{solutionbox}
\begin{center}\rule{0.5\linewidth}{0.5pt}\end{center}

\subsection*{Q.5(B) [8 marks]}\label{q.5b-8-marks}

\textbf{Attempt any two}

\subsubsection{Q5.1 [4 marks]}\label{q5.1-4-marks}

\textbf{Find the angle between two lines \(\sqrt{3}x - y + 1 = 0\) and
\(x - \sqrt{3}y + 2 = 0\)}

\begin{solutionbox}

\textbf{Solution}: \textbf{Step 1: Find slopes of both lines}

Line 1: \(\sqrt{3}x - y + 1 = 0\) \(y = \sqrt{3}x + 1\)
\(m_1 = \sqrt{3}\)

Line 2: \(x - \sqrt{3}y + 2 = 0\) \(\sqrt{3}y = x + 2\)
\(y = \frac{1}{\sqrt{3}}x + \frac{2}{\sqrt{3}}\)
\(m_2 = \frac{1}{\sqrt{3}}\)

\textbf{Step 2: Find angle between lines}
\(\tan \theta = \left|\frac{m_1 - m_2}{1 + m_1m_2}\right|\)

\(= \left|\frac{\sqrt{3} - \frac{1}{\sqrt{3}}}{1 + \sqrt{3} \times \frac{1}{\sqrt{3}}}\right| = \left|\frac{\frac{3 - 1}{\sqrt{3}}}{1 + 1}\right| = \left|\frac{\frac{2}{\sqrt{3}}}{2}\right| = \frac{1}{\sqrt{3}}\)

Therefore: \(\theta = \tan^{-1}(\frac{1}{\sqrt{3}}) = 30^\circ\) or
\(\frac{\pi}{6}\) radians

\end{solutionbox}
\subsubsection{Q5.2 [4 marks]}\label{q5.2-4-marks}

\textbf{Find the center and radius of circle
\(4x^2 + 4y^2 + 8x - 12y - 3 = 0\)}

\begin{solutionbox}

\textbf{Solution}: \textbf{Step 1: Simplify by dividing by 4}
\(x^2 + y^2 + 2x - 3y - \frac{3}{4} = 0\)

\textbf{Step 2: Complete the square}
\((x^2 + 2x) + (y^2 - 3y) = \frac{3}{4}\)

\((x^2 + 2x + 1) + (y^2 - 3y + \frac{9}{4}) = \frac{3}{4} + 1 + \frac{9}{4}\)

\((x + 1)^2 + (y - \frac{3}{2})^2 = \frac{3 + 4 + 9}{4} = \frac{16}{4} = 4\)


\vspace{-5pt}
\captionof{table}{Circle Properties}
\vspace{-10pt}
\begin{longtable}[]{@{}ll@{}}
\toprule\noalign{}
Property & Value \\
\midrule\noalign{}
\endhead
\bottomrule\noalign{}
\endlastfoot
\textbf{Center} & \((-1, \frac{3}{2})\) \\
\textbf{Radius} & \(\sqrt{4} = 2\) \\
\end{longtable}

\end{solutionbox}
\subsubsection{Q5.3 [4 marks]}\label{q5.3-4-marks}

\textbf{Find the tangent and normal to circle
\(x^2 + y^2 - 4x + 2y + 3 = 0\) at point \((1, -2)\)}

\begin{solutionbox}

\textbf{Solution}: \textbf{Step 1: Find center of circle}
\(x^2 + y^2 - 4x + 2y + 3 = 0\) Completing the square:
\((x^2 - 4x + 4) + (y^2 + 2y + 1) = -3 + 4 + 1\)
\((x - 2)^2 + (y + 1)^2 = 2\)

Center: \((2, -1)\)

\textbf{Step 2: Find slope of radius to point \((1, -2)\)}
\(m_{radius} = \frac{-2 - (-1)}{1 - 2} = \frac{-1}{-1} = 1\)

\textbf{Step 3: Find slope of tangent} Tangent is perpendicular to
radius: \(m_{tangent} = -\frac{1}{m_{radius}} = -\frac{1}{1} = -1\)

\textbf{Step 4: Equation of tangent at \((1, -2)\)}
\(y - (-2) = -1(x - 1)\) \(y + 2 = -x + 1\) \(x + y + 1 = 0\)

\textbf{Step 5: Equation of normal at \((1, -2)\)} Normal has slope
\(m_{radius} = 1\): \(y - (-2) = 1(x - 1)\) \(y + 2 = x - 1\)
\(x - y - 3 = 0\)


\vspace{-5pt}
\captionof{table}{Line Equations}
\vspace{-10pt}
\begin{longtable}[]{@{}ll@{}}
\toprule\noalign{}
Line & Equation \\
\midrule\noalign{}
\endhead
\bottomrule\noalign{}
\endlastfoot
\textbf{Tangent} & \(x + y + 1 = 0\) \\
\textbf{Normal} & \(x - y - 3 = 0\) \\
\end{longtable}

\end{solutionbox}
\begin{center}\rule{0.5\linewidth}{0.5pt}\end{center}

\subsection*{Mathematics Formula Cheat Sheet for Winter 2022
Exams}\label{mathematics-formula-cheat-sheet-for-winter-2022-exams}

\subsubsection{\texorpdfstring{\textbf{Determinants}}{Determinants}}\label{determinants}

\begin{itemize}
\tightlist
\item
  \textbf{2\times2 Matrix}:
  \(\begin{vmatrix} a & b \\ c & d \end{vmatrix} = ad - bc\)
\item
  \textbf{3\times3 Matrix}: Expand along row/column with most zeros
\item
  \textbf{Properties}: If any row/column has all zeros, determinant = 0
\end{itemize}

\subsubsection{\texorpdfstring{\textbf{Functions}}{Functions}}\label{functions}

\begin{itemize}
\tightlist
\item
  \textbf{Basic evaluation}: \$f(1) = \$ substitute \(x = 1\) in
  \(f(x)\)
\item
  \textbf{Tangent function properties}:

  \begin{itemize}
  \tightlist
  \item
    \(f(x+y) = \frac{f(x) + f(y)}{1 - f(x)f(y)}\) when \(f(x) = \tan x\)
  \item
    \(f(2x) = \frac{2f(x)}{1 - [f(x)]^2}\) when \(f(x) = \tan x\)
  \end{itemize}
\end{itemize}

\subsubsection{\texorpdfstring{\textbf{Logarithms}}{Logarithms}}\label{logarithms}

\begin{itemize}
\tightlist
\item
  \textbf{Basic properties}:

  \begin{itemize}
  \tightlist
  \item
    \(\log 1 = 0\)
  \item
    \(\log x + \log(\frac{1}{x}) = 0\)
  \item
    \(\frac{1}{\log_a b} = \log_b a\) (Change of base)
  \end{itemize}
\item
  \textbf{Product rule}: \(\log a + \log b = \log(ab)\)
\end{itemize}

\subsubsection{\texorpdfstring{\textbf{Trigonometry}}{Trigonometry}}\label{trigonometry}

\paragraph{\texorpdfstring{\textbf{Angle
Conversions}}{Angle Conversions}}\label{angle-conversions}

\begin{itemize}
\tightlist
\item
  \(120^\circ = \frac{2\pi}{3}\) radians
\item
  General: degrees \times \(\frac{\pi}{180}\) = radians
\end{itemize}

\paragraph{\texorpdfstring{\textbf{Inverse
Functions}}{Inverse Functions}}\label{inverse-functions}

\begin{itemize}
\tightlist
\item
  \(\sin^{-1}(\sin \theta) = \theta\) if
  \(\theta \in [-\frac{\pi}{2}, \frac{\pi}{2}]\)
\item
  \(\tan^{-1} a + \tan^{-1} b = \tan^{-1}(\frac{a+b}{1-ab})\) when
  \(ab < 1\)
\end{itemize}

\paragraph{\texorpdfstring{\textbf{Periods}}{Periods}}\label{periods}

\begin{itemize}
\tightlist
\item
  \(\sin x\), \(\cos x\): period = \(2\pi\)
\item
  \(\tan x\): period = \(\pi\)
\end{itemize}

\paragraph{\texorpdfstring{\textbf{Triple Angle
Formulas}}{Triple Angle Formulas}}\label{triple-angle-formulas}

\begin{itemize}
\tightlist
\item
  \(\sin 3A = 3\sin A - 4\sin^3 A\)
\item
  \(\cos 3A = 4\cos^3 A - 3\cos A\)
\end{itemize}

\paragraph{\texorpdfstring{\textbf{Allied
Angles}}{Allied Angles}}\label{allied-angles}

\begin{itemize}
\tightlist
\item
  \(\sin(\theta + \pi) = -\sin \theta\)
\item
  \(\cos(\theta + 2\pi) = \cos \theta\)
\item
  \(\tan(\frac{\pi}{2} + \theta) = -\cot \theta\)
\end{itemize}

\subsubsection{\texorpdfstring{\textbf{Vectors}}{Vectors}}\label{vectors}

\begin{itemize}
\tightlist
\item
  \textbf{Magnitude}: \(|\vec{a}| = \sqrt{a_1^2 + a_2^2 + a_3^2}\)
\item
  \textbf{Unit vector dot product}: \(\hat{i} \cdot \hat{i} = 1\)
\item
  \textbf{Dot Product}:
  \(\vec{a} \cdot \vec{b} = a_1b_1 + a_2b_2 + a_3b_3\)
\item
  \textbf{Cross Product}:
  \(\vec{a} \times \vec{b} = \begin{vmatrix} \hat{i} & \hat{j} & \hat{k} \\ a_1 & a_2 & a_3 \\ b_1 & b_2 & b_3 \end{vmatrix}\)
\item
  \textbf{Perpendicularity}: \(\vec{a} \perp \vec{b}\) iff
  \(\vec{a} \cdot \vec{b} = 0\)
\item
  \textbf{Work done}: \(W = \vec{F} \cdot \vec{d}\)
\end{itemize}

\subsubsection{\texorpdfstring{\textbf{Coordinate
Geometry}}{Coordinate Geometry}}\label{coordinate-geometry}

\paragraph{\texorpdfstring{\textbf{Lines}}{Lines}}\label{lines}

\begin{itemize}
\tightlist
\item
  \textbf{Slope of vertical line}: Undefined
\item
  \textbf{Point-slope form}: \(y - y_1 = m(x - x_1)\)
\item
  \textbf{Parallel lines}: Same slope
\item
  \textbf{Angle between lines}:
  \(\tan \theta = \left|\frac{m_1 - m_2}{1 + m_1m_2}\right|\)
\end{itemize}

\paragraph{\texorpdfstring{\textbf{Circles}}{Circles}}\label{circles}

\begin{itemize}
\tightlist
\item
  \textbf{Standard form}: \((x - h)^2 + (y - k)^2 = r^2\)
\item
  \textbf{Center}: \((h, k)\), \textbf{Radius}: \(r\)
\item
  \textbf{Tangent-radius relationship}: Tangent ⊥ radius at point of
  contact
\end{itemize}

\subsubsection{\texorpdfstring{\textbf{Limits}}{Limits}}\label{limits}

\begin{itemize}
\tightlist
\item
  \textbf{Standard limits}:

  \begin{itemize}
  \tightlist
  \item
    \(\lim_{x \to a} \frac{x^n - a^n}{x - a} = na^{n-1}\)
  \item
    \(\lim_{n \to 0} (1 + n)^{\frac{1}{n}} = e\)
  \item
    \(\lim_{x \to 0} \frac{\sin ax}{bx} = \frac{a}{b}\)
  \item
    \(\lim_{x \to 0} \frac{e^{ax} - 1}{x} = a\)
  \item
    \(\lim_{x \to \infty} (1 + \frac{a}{x})^x = e^a\)
  \end{itemize}
\item
  \textbf{L'Hôpital's Rule}: For \(\frac{0}{0}\) or
  \(\frac{\infty}{\infty}\) forms
\item
  \textbf{Rational functions}: Divide by highest power for
  \(x \to \infty\)
\end{itemize}

\subsubsection{\texorpdfstring{\textbf{Series
Formulas}}{Series Formulas}}\label{series-formulas}

\begin{itemize}
\tightlist
\item
  \(1 + 2 + 3 + ... + n = \frac{n(n+1)}{2}\)
\end{itemize}

\subsubsection{\texorpdfstring{\textbf{Problem-Solving
Strategies}}{Problem-Solving Strategies}}\label{problem-solving-strategies}

\paragraph{\texorpdfstring{\textbf{For Determinant
Problems}}{For Determinant Problems}}\label{for-determinant-problems}

\begin{enumerate}
\tightlist
\item
  Look for rows/columns with zeros
\item
  Expand along the row/column with most zeros
\item
  Factor common terms before expanding
\end{enumerate}

\paragraph{\texorpdfstring{\textbf{For Function
Composition}}{For Function Composition}}\label{for-function-composition}

\begin{enumerate}
\tightlist
\item
  Substitute inner function into outer function
\item
  Simplify step by step
\item
  Check domain restrictions
\end{enumerate}

\paragraph{\texorpdfstring{\textbf{For Trigonometric
Identities}}{For Trigonometric Identities}}\label{for-trigonometric-identities}

\begin{enumerate}
\tightlist
\item
  Use compound angle formulas
\item
  Look for opportunities to use allied angles
\item
  Convert everything to same trigonometric ratios
\end{enumerate}

\paragraph{\texorpdfstring{\textbf{For Vector
Problems}}{For Vector Problems}}\label{for-vector-problems}

\begin{enumerate}
\tightlist
\item
  Write in component form
\item
  Use dot product for perpendicularity checks
\item
  Use cross product for perpendicular vectors
\end{enumerate}

\paragraph{\texorpdfstring{\textbf{For Limit
Problems}}{For Limit Problems}}\label{for-limit-problems}

\begin{enumerate}
\tightlist
\item
  Try direct substitution first
\item
  Factor and cancel for indeterminate forms
\item
  Use standard limit formulas
\item
  For exponential limits, use logarithms
\end{enumerate}

\paragraph{\texorpdfstring{\textbf{For Circle
Problems}}{For Circle Problems}}\label{for-circle-problems}

\begin{enumerate}
\tightlist
\item
  Complete the square to find center and radius
\item
  Use slope relationships for tangent and normal
\item
  Remember: tangent slope \times radius slope = -1
\end{enumerate}

\subsubsection{\texorpdfstring{\textbf{Common Mistakes to
Avoid}}{Common Mistakes to Avoid}}\label{common-mistakes-to-avoid}

\begin{enumerate}
\tightlist
\item
  \textbf{Sign errors} in determinant expansion
\item
  \textbf{Forgetting} that vertical lines have undefined slope
\item
  \textbf{Not checking} if point lies on circle before finding tangent
\item
  \textbf{Mixing up} parallel (same slope) vs perpendicular (negative
  reciprocal slopes)
\item
  \textbf{Not simplifying} trigonometric expressions fully
\item
  \textbf{Forgetting} to rationalize in limit problems
\end{enumerate}

\subsubsection{\texorpdfstring{\textbf{Quick Reference
Values}}{Quick Reference Values}}\label{quick-reference-values}

\begin{itemize}
\tightlist
\item
  \(\tan 30^\circ = \frac{1}{\sqrt{3}}\), \(\tan 60^\circ = \sqrt{3}\),
  \(\tan 45^\circ = 1\)
\item
  \(e \approx 2.718\)
\item
  \(\sqrt{3} \approx 1.732\)
\end{itemize}

\subsubsection{\texorpdfstring{\textbf{Exam Success
Tips}}{Exam Success Tips}}\label{exam-success-tips}

\begin{itemize}
\tightlist
\item
  \textbf{Show all steps} clearly in calculations
\item
  \textbf{Check answers} by substitution when possible
\item
  \textbf{Use proper notation} throughout
\item
  \textbf{Draw diagrams} for vector and geometry problems
\item
  \textbf{Manage time} effectively across questions
\end{itemize}

\textbf{Best of luck with your Winter 2022 Mathematics exam!} 🎯


\end{document}
