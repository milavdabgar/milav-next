\documentclass[10pt,a4paper]{article}

% content/resources/templates/preamble.tex
\usepackage[margin=0.6in]{geometry}
\author{Milav Dabgar}
\usepackage{amsmath,amssymb,amsthm}
\usepackage{booktabs}
\usepackage{multirow}
\usepackage{xcolor}
\usepackage{tcolorbox}
\tcbuselibrary{breakable,skins}
\usepackage[colorlinks=true,linkcolor=blue]{hyperref}
\usepackage{titlesec}
\usepackage{enumitem}
\usepackage{tikz}
\usepackage{pgfplots}
\usepackage{circuitikz}
\usepackage[version=4]{mhchem}
\usepackage{longtable}
\usepackage{array}
\usepackage{float}
\usepackage{caption}
\usepackage{listings}

\lstset{
  basicstyle=\small\ttfamily,
  breaklines=true,
  breakatwhitespace=false,
  postbreak=\mbox{\textcolor{red}{$\hookrightarrow$}\space},
  float=false,
  numbers=left,
  numberstyle=\tiny\color{gray},
  numbersep=10pt,
  xleftmargin=2em,
  keywordstyle=\color{blue},
  commentstyle=\color{green!60!black},
  stringstyle=\color{purple},
  backgroundcolor=\color{gray!5},
  showstringspaces=false,
  tabsize=2,
  captionpos=b,
  keepspaces=true,
  columns=flexible
}

\pgfplotsset{compat=1.18}
\usetikzlibrary{shapes,arrows,positioning,calc,patterns,decorations.pathmorphing,decorations.markings,arrows.meta}

% Color scheme
\definecolor{headcolor}{RGB}{0,102,204}
\definecolor{keycolor}{RGB}{220,20,60}
\definecolor{solutioncolor}{RGB}{34,139,34}
\definecolor{mnemoniccolor}{RGB}{148,0,211}
\definecolor{codecolor}{RGB}{0,0,100}

% Spacing
\setlength{\parskip}{3pt}
\setlist[itemize]{nosep}
\setlist[enumerate]{nosep}

% Title formatting
\titleformat{\section}{\Large\bfseries\color{headcolor}}{\thesection}{1em}{}
\titleformat{\subsection}{\large\bfseries\color{headcolor}}{\thesubsection}{1em}{}

% Pandoc tightlist compatibility
\providecommand{\tightlist}{%
  \setlength{\itemsep}{0pt}\setlength{\parskip}{0pt}}

% Pandoc longtable compatibility
\newcounter{none}
\def\thenone{}


% content/resources/templates/gujarati-boxes.tex
\usepackage{fontspec}
\usepackage{polyglossia}

% Set Gujarati as main language (document is primarily in Gujarati)
% Note: gloss-gujarati.ldf doesn't exist in polyglossia, but it will use hyphenation patterns
\setdefaultlanguage{gujarati}
\setotherlanguage{english}

% Configure Gujarati font properly
% Use Language=Default to prevent polyglossia from trying to add language-specific features
% that don't exist for Gujarati, which causes "empty feature" warnings
\newfontfamily\gujaratifont[Script=Gujarati,AutoFakeBold=2.5,AutoFakeSlant=0.3]{Noto Sans Gujarati}
\setmainfont[Script=Gujarati,AutoFakeBold=2.5,AutoFakeSlant=0.3]{Noto Sans Gujarati}
% Use Noto Sans Gujarati for monospace to support Gujarati in text
\setmonofont[Scale=0.9]{Noto Sans Gujarati}

% Configure English to use the same font
\newfontfamily\englishfont[Script=Gujarati,AutoFakeBold=2.5,AutoFakeSlant=0.3]{Noto Sans Gujarati}

% Translations for polyglossia
\gappto\captionsgujarati{
  \renewcommand{\tablename}{કોષ્ટક}
  \renewcommand{\figurename}{આકૃતિ}
}

% Helper for TikZ nodes to ensure Gujarati font
\newcommand{\gu}[1]{{\gujaratifont #1}}

% Custom environments
\newtcolorbox{solutionbox}{
    breakable,
    enhanced,
    colback=solutioncolor!5!white,
    colframe=solutioncolor!75!black,
    fonttitle=\bfseries,
    title=જવાબ
}

\newtcolorbox{solutionboxnobreak}{
 colback=solutioncolor!5!white,
 colframe=solutioncolor!75!black,
 fonttitle=\bfseries,
 title=જવાબ
}

\newtcolorbox{keyformula}{
 breakable,
 enhanced,
 colback=keycolor!5!white,
 colframe=keycolor!75!black,
 fonttitle=\bfseries,
 title=રાસાયણિક સમીકરણ/સૂત્ર
}

\newtcolorbox{mnemonicbox}{
 breakable,
 enhanced,
 colback=mnemoniccolor!5!white,
 colframe=mnemoniccolor!75!black,
 fonttitle=\bfseries,
 title=મેમરી ટ્રીક
}


\begin{document}

\begin{center}
{\Huge\bfseries\color{headcolor} Subject Name (Gujarati)}\\[5pt]
{\LARGE 4300021 -- Winter 2023}\\[3pt]
{\large Semester 1 Study Material}\\[3pt]
{\normalsize\textit{Detailed Solutions and Explanations}}
\end{center}

\vspace{10pt}

\section*{Gujarati Translation}\label{gujarati-translation}

\subsection*{પ્રશ્ન 1(અ) [3
ગુણ]}\label{uxaaauxab0uxab6uxaa8-1uxa85-3-uxa97uxaa3}

\textbf{આંત્રપ્રેન્યોરશિપ અને ઈન્ટ્રાપ્રેન્યોરશિપ વચ્ચે સરખામણી આપો.}

\begin{solutionbox}

\begin{longtable}[]{@{}lll@{}}
\toprule\noalign{}
\textbf{પાસું} & \textbf{આંત્રપ્રેન્યોરશિપ} & \textbf{ઈન્ટ્રાપ્રેન્યોરશિપ} \\
\midrule\noalign{}
\endhead
\bottomrule\noalign{}
\endlastfoot
\textbf{વ્યાખ્યા} & પોતાનો વ્યવસાય શરૂ કરવો & હાલની સંસ્થામાં નવીનતા \\
\textbf{જોખમ} & વ્યક્તિગત નાણાકીય જોખમ & સંસ્થા જોખમ લે છે \\
\textbf{સંસાધનો} & પોતાના/ઉધાર લીધેલા & કંપની પૂરા પાડે છે \\
\end{longtable}

\end{solutionbox}
\begin{mnemonicbox}
``બાહ્ય વિરુદ્ધ આંતરિક નવીનતા''

\end{mnemonicbox}
\subsection*{પ્રશ્ન 1(બ) [4
ગુણ]}\label{uxaaauxab0uxab6uxaa8-1uxaac-4-uxa97uxaa3}

\textbf{ઉદ્યોગસાહસિકતાની લાક્ષણિકતાઓ અને કાર્યોની ચર્ચા કરો}

\begin{solutionbox}

\textbf{લાક્ષણિકતાઓ:}

\begin{itemize}
\tightlist
\item
  \textbf{જોખમ લેવાની ક્ષમતા}: હિસાબી વ્યાપારી જોખમો લેવાની તૈયારી
\item
  \textbf{નવીનતા}: નવા ઉત્પાદનો, સેવાઓ અથવા પ્રક્રિયાઓ બનાવવી
\item
  \textbf{નેતૃત્વ કુશળતા}: ટીમને માર્ગદર્શન અને પ્રેરણા આપવાની ક્ષમતા
\end{itemize}

\textbf{કાર્યો:}

\begin{itemize}
\tightlist
\item
  \textbf{રોજગાર સર્જન}: સમાજ માટે રોજગારની તકો બનાવવી
\item
  \textbf{આર્થિક વિકાસ}: GDP અને રાષ્ટ્રીય વૃદ્ધિમાં યોગદાન
\item
  \textbf{નવીનતાનું કેન્દ્ર}: નવી ટેકનોલોજી અને ઉકેલો રજૂ કરવા
\end{itemize}

\end{solutionbox}
\begin{mnemonicbox}
``જોખમ નવીનતા નેતૃત્વ રોજગાર વિકાસ નવીનતા''

\end{mnemonicbox}
\subsection*{પ્રશ્ન 1(ક) [7
ગુણ]}\label{uxaaauxab0uxab6uxaa8-1uxa95-7-uxa97uxaa3}

\textbf{7-M સંસાધનોને ઓળખો અને વિગતવાર ચર્ચા કરો.}

\begin{solutionbox}

\begin{longtable}[]{@{}lll@{}}
\toprule\noalign{}
\textbf{સંસાધન} & \textbf{વર્ણન} & \textbf{મહત્વ} \\
\midrule\noalign{}
\endhead
\bottomrule\noalign{}
\endlastfoot
\textbf{Man (માનવી)} & માનવ સંસાધનો અને કર્મચારીઓ & કામકાજ માટે મુખ્ય સંપત્તિ \\
\textbf{Money (પૈસા)} & નાણાકીય મૂડી અને ભંડોળ & વ્યાપારી કામકાજ માટે જરૂરી \\
\textbf{Material (સામગ્રી)} & કાચો માલ અને પુરવઠો & ઉત્પાદન આવશ્યકતાઓ \\
\textbf{Machine (મશીન)} & સાધનો અને ટેકનોલોજી & કામકાજની કાર્યક્ષમતા \\
\textbf{Method (પદ્ધતિ)} & પ્રક્રિયાઓ અને કાર્યવિધિઓ & વ્યવસ્થિત અભિગમ \\
\textbf{Market (બજાર)} & ગ્રાહક આધાર અને માંગ & આવકનું ઉત્પાદન \\
\textbf{Management (સંચાલન)} & આયોજન અને સંકલન & એકંદર વ્યાપારી નિયંત્રણ \\
\end{longtable}

\end{solutionbox}
\begin{mnemonicbox}
``અનેક આધુનિક મેનેજરો પૈસા બનાવવા બજારોનું સંચાલન કરે છે''

\end{mnemonicbox}
\subsection*{પ્રશ્ન 1(ક) OR [7
ગુણ]}\label{uxaaauxab0uxab6uxaa8-1uxa95-or-7-uxa97uxaa3}

\textbf{સ્ટાર્ટ અપ ઇન્ડિયા નોંધણી પ્રક્રિયા લખો.}

\begin{solutionbox}

\textbf{સ્ટાર્ટ-અપ ઇન્ડિયા નોંધણીના પગલાં:}

\begin{enumerate}
\tightlist
\item
  \textbf{ઓનલાઇન નોંધણી}: www.startupindia.gov.in ની મુલાકાત લો
\item
  \textbf{દસ્તાવેજ તૈયારી}:

  \begin{itemize}
  \tightlist
  \item
    નિગમીકરણનું પ્રમાણપત્ર
  \item
    એન્ટિટીનું PAN કાર્ડ
  \item
    વ્યવસાયનું સંક્ષિપ્ત વર્ણન
  \end{itemize}
\item
  \textbf{પાત્રતાના માપદંડો}:

  \begin{itemize}
  \tightlist
  \item
    એન્ટિટીની ઉંમર 10 વર્ષથી ઓછી
  \item
    વાર્ષિક ટર્નઓવર ₹100 કરોડથી ઓછું
  \item
    નવીનતા/સુધારા તરફ કામ કરવું
  \end{itemize}
\item
  \textbf{અરજી સબમિશન}: જરૂરી દસ્તાવેજો સાથે ઓનલાઇન ફોર્મ ભરવો
\item
  \textbf{ચકાસણી પ્રક્રિયા}: સરકારી સમીક્ષા અને મંજૂરી
\item
  \textbf{પ્રમાણપત્ર આપવું}: માન્યતા પ્રમાણપત્ર પ્રાપ્ત કરવું
\end{enumerate}

\textbf{ફાયદાઓ:}

\begin{itemize}
\tightlist
\item
  \textbf{કર મુક્તિ} સતત 3 વર્ષ માટે
\item
  \textbf{ઝડપી પેટન્ટ} અરજી પ્રક્રિયા
\item
  \textbf{કમ્પ્લાયન્સ ઘટાડો} લેબર અને પર્યાવરણ કાયદા હેઠળ
\end{itemize}

\end{solutionbox}
\begin{mnemonicbox}
``ઓનલાઇન દસ્તાવેજ પાત્રતા અરજી ચકાસણી પ્રમાણપત્ર ફાયદાઓ''

\end{mnemonicbox}
\subsection*{પ્રશ્ન 2(અ) [3
ગુણ]}\label{uxaaauxab0uxab6uxaa8-2uxa85-3-uxa97uxaa3}

\textbf{બજાર સંશોધનની પદ્ધતિઓની સૂચિ બનાવો.}

\begin{solutionbox}

\textbf{પ્રાથમિક સંશોધન પદ્ધતિઓ:}

\begin{itemize}
\tightlist
\item
  \textbf{સર્વે}: ગ્રાહક ડેટા એકત્રિત કરવા માટે પ્રશ્નાવલી
\item
  \textbf{ઇન્ટરવ્યુ}: લક્ષ્ય પ્રેક્ષકો સાથે સીધી વાતચીત
\item
  \textbf{ફોકસ ગ્રુપ}: પ્રતિસાદ માટે જૂથ ચર્ચાઓ
\end{itemize}

\textbf{દ્વિતીયક સંશોધન પદ્ધતિઓ:}

\begin{itemize}
\tightlist
\item
  \textbf{ઓનલાઇન સંશોધન}: ઇન્ટરનેટ આધારિત ડેટા સંગ્રહ
\item
  \textbf{પ્રકાશિત અહેવાલો}: ઉદ્યોગ વિશ્લેષણ અને અભ્યાસો
\item
  \textbf{સરકારી ડેટા}: સત્તાવાર સ્ત્રોતોથી આંકડાકીય માહિતી
\end{itemize}

\end{solutionbox}
\begin{mnemonicbox}
``સર્વે ઇન્ટરવ્યુ ફોકસ ઓનલાઇન પ્રકાશિત સરકારી''

\end{mnemonicbox}
\subsection*{પ્રશ્ન 2(બ) [4
ગુણ]}\label{uxaaauxab0uxab6uxaa8-2uxaac-4-uxa97uxaa3}

\textbf{ઉત્પાદન જીવન ચક્ર દોરો અને સમજાવો.}

\begin{solutionbox}

\begin{verbatim}
Introduction  Growth  Maturity  Decline
     |          |         |         |
   Low Sales  Rising    Peak Sales  Declining
   High Costs Profits   Saturation   Sales
\end{verbatim}

\textbf{તબક્કાઓ:}

\begin{itemize}
\tightlist
\item
  \textbf{પરિચય}: ઊંચા માર્કેટિંગ ખર્ચ સાથે ઉત્પાદન લોન્ચ
\item
  \textbf{વૃદ્ધિ}: ઝડપી વેચાણ વધારો અને બજારમાં સ્વીકૃતિ
\item
  \textbf{પરિપક્વતા}: તીવ્ર સ્પર્ધા સાથે ટોચના વેચાણ
\item
  \textbf{ઘટાડો}: માંગમાં ઘટાડો અને અંતે તબક્કાબંધ
\end{itemize}

\end{solutionbox}
\begin{mnemonicbox}
``હું મારા સપના વધારું છું''

\end{mnemonicbox}
\subsection*{પ્રશ્ન 2(ક) [7
ગુણ]}\label{uxaaauxab0uxab6uxaa8-2uxa95-7-uxa97uxaa3}

\textbf{માર્કેટિંગના 4 P ને ઓળખો અને ચર્ચા કરો.}

\begin{solutionbox}

\begin{longtable}[]{@{}
  >{\raggedright\arraybackslash}p{(\linewidth - 6\tabcolsep) * \real{0.1458}}
  >{\raggedright\arraybackslash}p{(\linewidth - 6\tabcolsep) * \real{0.1875}}
  >{\raggedright\arraybackslash}p{(\linewidth - 6\tabcolsep) * \real{0.2292}}
  >{\raggedright\arraybackslash}p{(\linewidth - 6\tabcolsep) * \real{0.4375}}@{}}
\toprule\noalign{}
\begin{minipage}[b]{\linewidth}\raggedright
\textbf{P}
\end{minipage} & \begin{minipage}[b]{\linewidth}\raggedright
\textbf{તત્વ}
\end{minipage} & \begin{minipage}[b]{\linewidth}\raggedright
\textbf{વર્ણન}
\end{minipage} & \begin{minipage}[b]{\linewidth}\raggedright
\textbf{મુખ્ય વિચારણાઓ}
\end{minipage} \\
\midrule\noalign{}
\endhead
\bottomrule\noalign{}
\endlastfoot
\textbf{Product (ઉત્પાદન)} & ઓફર કરવામાં આવતા માલ/સેવાઓ & લક્ષણો, ગુણવત્તા,
બ્રાન્ડિંગ & ગ્રાહકની જરૂરિયાતોની સંતુષ્ટિ \\
\textbf{Price (કિંમત)} & ગ્રાહકને ખર્ચ & કિંમત વ્યૂહરચના, છૂટ & સ્પર્ધાત્મક
સ્થિતિ \\
\textbf{Place (સ્થળ)} & વિતરણ ચેનલ્સ & ઉત્પાદન ક્યાં વેચાય છે & ગ્રાહકો માટે
પહોંચ \\
\textbf{Promotion (પ્રમોશન)} & માર્કેટિંગ કમ્યુનિકેશન & જાહેરાત, વેચાણ પ્રમોશન &
બ્રાન્ડ જાગૃતિ સર્જન \\
\end{longtable}

\textbf{એકીકરણ:} અસરકારક માર્કેટિંગ વ્યૂહરચના માટે તમામ 4 P એ એકસાથે કામ કરવું પડે
છે.

\end{solutionbox}
\begin{mnemonicbox}
``લોકો ઉત્પાદનો યોગ્ય રીતે ખરીદે છે''

\end{mnemonicbox}
\subsection*{પ્રશ્ન 2(અ) OR [3
ગુણ]}\label{uxaaauxab0uxab6uxaa8-2uxa85-or-3-uxa97uxaa3}

\textbf{B2B, ઈ-કોમર્સ અને GeM ની ચર્ચા કરો.}

\begin{solutionbox}

\begin{longtable}[]{@{}lll@{}}
\toprule\noalign{}
\textbf{પ્રકાર} & \textbf{સંપૂર્ણ નામ} & \textbf{વર્ણન} \\
\midrule\noalign{}
\endhead
\bottomrule\noalign{}
\endlastfoot
\textbf{B2B} & Business to Business & કંપનીઓ વચ્ચેનો વેપાર \\
\textbf{E-commerce} & Electronic Commerce & ઓનલાઇન ખરીદી અને વેચાણ \\
\textbf{GeM} & Government e-Marketplace & સરકારી પ્રાપ્તિ પોર્ટલ \\
\end{longtable}

\textbf{મુખ્ય લક્ષણો:}

\begin{itemize}
\tightlist
\item
  \textbf{B2B}: મોટા પ્રમાણમાં વ્યવહારો, લાંબા ગાળાના સંબંધો
\item
  \textbf{E-commerce}: ડિજિટલ પ્લેટફોર્મ, વૈશ્વિક પહોંચ
\item
  \textbf{GeM}: પારદર્શક સરકારી ખરીદી, સ્પર્ધાત્મક કિંમત
\end{itemize}

\end{solutionbox}
\begin{mnemonicbox}
``વ્યવસાયો ઇલેક્ટ્રોનિક રીતે ખરીદે, સરકાર ઈ-માર્કેટ''

\end{mnemonicbox}
\subsection*{પ્રશ્ન 2(બ) OR [4
ગુણ]}\label{uxaaauxab0uxab6uxaa8-2uxaac-or-4-uxa97uxaa3}

\textbf{વ્યવસાય બનાવવા અને શરૂ કરવા માટેની યોજનાઓ પર એક નોંધ લખો}

\begin{solutionbox}

\textbf{વ્યાપાર સર્જન યોજનાઓ:}

\textbf{બજાર વિશ્લેષણ:}

\begin{itemize}
\tightlist
\item
  \textbf{લક્ષ્ય ગ્રાહકો}: પ્રાથમિક પ્રેક્ષકોને ઓળખવા
\item
  \textbf{સ્પર્ધા અભ્યાસ}: હાલના ખેલાડીઓનું વિશ્લેષણ
\item
  \textbf{બજારનું કદ}: સંભવિત આવકનું નિર્ધારણ
\end{itemize}

\textbf{નાણાકીય આયોજન:}

\begin{itemize}
\tightlist
\item
  \textbf{મૂડીની જરૂરિયાતો}: પ્રારંભિક રોકાણની જરૂર
\item
  \textbf{આવકના અંદાજો}: અપેક્ષિત આવકના સ્ત્રોતો
\item
  \textbf{બ્રેક-ઇવન વિશ્લેષણ}: નફાકારકતાની સમયમર્યાદા
\end{itemize}

\textbf{કામકાજનું સેટઅપ:}

\begin{itemize}
\tightlist
\item
  \textbf{સ્થળ પસંદગી}: વ્યૂહાત્મક સ્થિતિ
\item
  \textbf{સંસાધન ફાળવણી}: માનવ અને ભૌતિક સંસાધનો
\item
  \textbf{કાનૂની અનુપાલન}: લાઇસન્સ અને નોંધણીઓ
\end{itemize}

\end{solutionbox}
\begin{mnemonicbox}
``બજાર નાણા કામકાજ = વ્યાપારની સફળતા''

\end{mnemonicbox}
\subsection*{પ્રશ્ન 2(ક) OR [7
ગુણ]}\label{uxaaauxab0uxab6uxaa8-2uxa95-or-7-uxa97uxaa3}

\textbf{જોખમ અને SWOT વિશ્લેષણની કલ્પના સમજાવો.}

\begin{solutionbox}

\textbf{જોખમની કલ્પના:} જોખમ એ અનિશ્ચિતતા છે જે વ્યાપારી પરિણામોને સકારાત્મક અને
નકારાત્મક બંને રીતે અસર કરી શકે છે.

\textbf{વ્યાપારિક જોખમોના પ્રકારો:}

\begin{itemize}
\tightlist
\item
  \textbf{નાણાકીય જોખમ}: રોકડ પ્રવાહ અને ભંડોળની સમસ્યાઓ
\item
  \textbf{બજાર જોખમ}: માંગની વધઘટ અને સ્પર્ધા
\item
  \textbf{કામકાજી જોખમ}: ઉત્પાદન અને સેવા વિતરણની સમસ્યાઓ
\end{itemize}

\textbf{SWOT વિશ્લેષણ:}

\begin{longtable}[]{@{}ll@{}}
\toprule\noalign{}
\textbf{આંતરિક પરિબળો} & \textbf{બાહ્ય પરિબળો} \\
\midrule\noalign{}
\endhead
\bottomrule\noalign{}
\endlastfoot
\textbf{શક્તિઓ (Strengths)} & \textbf{તકો (Opportunities)} \\
- મુખ્ય ક્ષમતાઓ & - બજાર વૃદ્ધિ \\
- અનન્ય સંસાધનો & - નવી ટેકનોલોજીઓ \\
\textbf{નબળાઈઓ (Weaknesses)} & \textbf{ધમકીઓ (Threats)} \\
- કુશળતાના ગાબડા & - સ્પર્ધા \\
- સંસાધન મર્યાદાઓ & - આર્થિક ફેરફારો \\
\end{longtable}

\textbf{જોખમ ઘટાડવાની વ્યૂહરચનાઓ:}

\begin{itemize}
\tightlist
\item
  \textbf{વૈવિધ્યકરણ}: વિવિધ ક્ષેત્રોમાં જોખમો ફેલાવવા
\item
  \textbf{વીમો}: બીમા કંપનીઓને જોખમ સ્થાનાંતરિત કરવું
\item
  \textbf{આકસ્મિક આયોજન}: અણધાર્યા પરિસ્થિતિઓ માટે તૈયારી
\end{itemize}

\end{solutionbox}
\begin{mnemonicbox}
``બળવાન નબળા તકો ધમકાવે = SWOT''

\end{mnemonicbox}
\subsection*{પ્રશ્ન 3(અ) [3
ગુણ]}\label{uxaaauxab0uxab6uxaa8-3uxa85-3-uxa97uxaa3}

\textbf{સહકારી પ્રકારની સંસ્થા પર ટૂંકી નોંધ લખો.}

\begin{solutionbox}

\textbf{સહકારી સંસ્થા:}

\begin{itemize}
\tightlist
\item
  \textbf{વ્યાખ્યા}: પરસ્પર લાભ માટે લોકોનું સ્વૈચ્છિક સંગઠન
\item
  \textbf{માલિકી}: સભ્યો દ્વારા સામૂહિક માલિકી
\item
  \textbf{નિયંત્રણ}: સમાન મતદાન અધિકારો સાથે લોકશાહી સંચાલન
\end{itemize}

\textbf{લાક્ષણિકતાઓ:}

\begin{itemize}
\tightlist
\item
  \textbf{સભ્ય સહભાગિતા}: નિર્ણય લેવામાં સક્રિય સંડોવણી
\item
  \textbf{નફાની વહેંચણી}: સભ્યો વચ્ચે લાભોનું વિતરણ
\item
  \textbf{સામાજિક હેતુ}: સમુદાયિક કલ્યાણ પર ધ્યાન
\end{itemize}

\textbf{ઉદાહરણો:} કૃષિ સહકારી મંડળીઓ, ક્રેડિટ યુનિયનો, હાઉસિંગ સોસાયટીઓ

\end{solutionbox}
\begin{mnemonicbox}
``સામૂહિક માલિકી સાથે લોકશાહી સંચાલન''

\end{mnemonicbox}
\subsection*{પ્રશ્ન 3(બ) [4
ગુણ]}\label{uxaaauxab0uxab6uxaa8-3uxaac-4-uxa97uxaa3}

\textbf{મેનેજમેન્ટના કાર્યોની સૂચિ આપો અને તે બધાને વ્યાખ્યાયિત કરો.}

\begin{solutionbox}

\begin{longtable}[]{@{}
  >{\raggedright\arraybackslash}p{(\linewidth - 4\tabcolsep) * \real{0.2273}}
  >{\raggedright\arraybackslash}p{(\linewidth - 4\tabcolsep) * \real{0.2955}}
  >{\raggedright\arraybackslash}p{(\linewidth - 4\tabcolsep) * \real{0.4773}}@{}}
\toprule\noalign{}
\begin{minipage}[b]{\linewidth}\raggedright
\textbf{કાર્ય}
\end{minipage} & \begin{minipage}[b]{\linewidth}\raggedright
\textbf{વ્યાખ્યા}
\end{minipage} & \begin{minipage}[b]{\linewidth}\raggedright
\textbf{મુખ્ય પ્રવૃત્તિઓ}
\end{minipage} \\
\midrule\noalign{}
\endhead
\bottomrule\noalign{}
\endlastfoot
\textbf{આયોજન (Planning)} & ઉદ્દેશ્યો અને વ્યૂહરચનાઓ નક્કી કરવી & લક્ષ્ય નિર્ધારણ,
આગાહી, બજેટ \\
\textbf{સંગઠન (Organizing)} & સંસાધનો અને માળખાંની ગોઠવણી & વિભાગીકરણ,
સોંપણી, સંકલન \\
\textbf{સ્ટાફિંગ (Staffing)} & માનવ સંસાધન વ્યવસ્થાપન & ભરતી, તાલીમ, કામગીરી
મૂલ્યાંકન \\
\textbf{દિશા નિર્દેશન (Directing)} & કર્મચારીઓનું નેતૃત્વ અને પ્રેરણા & સંવાદ, નેતૃત્વ,
દેખરેખ \\
\textbf{નિયંત્રણ (Controlling)} & કામગીરીનું નિરીક્ષણ અને સુધારો & કામગીરી
માપન, પ્રતિસાદ, સુધારણા \\
\end{longtable}

\end{solutionbox}
\begin{mnemonicbox}
``યોગ્ય સંગઠન સ્ટાફ દિશા નિયંત્રણને સમર્થન આપે છે''

\end{mnemonicbox}
\subsection*{પ્રશ્ન 3(ક) [7
ગુણ]}\label{uxaaauxab0uxab6uxaa8-3uxa95-7-uxa97uxaa3}

\textbf{માલિકીના પ્રકારોનું વર્ણન કરો અને કોઈપણ ત્રણને વિગતવાર સમજાવો.}

\begin{solutionbox}

\textbf{વ્યાપારિક માલિકીના પ્રકારો:}

\begin{longtable}[]{@{}llll@{}}
\toprule\noalign{}
\textbf{પ્રકાર} & \textbf{માલિકી} & \textbf{જવાબદારી} &
\textbf{નિયંત્રણ} \\
\midrule\noalign{}
\endhead
\bottomrule\noalign{}
\endlastfoot
\textbf{એકલ માલિકી} & એક માલિક & અમર્યાદિત & સંપૂર્ણ \\
\textbf{ભાગીદારી} & 2+ ભાગીદારો & અમર્યાદિત & વહેંચાયેલ \\
\textbf{કંપની} & શેરધારકો & મર્યાદિત & બોર્ડ ઓફ ડિરેક્ટર્સ \\
\textbf{સહકારી} & સભ્યો & મર્યાદિત & લોકશાહી \\
\end{longtable}

\textbf{વિગતવાર સમજૂતી:}

\textbf{1. એકલ માલિકી:}

\begin{itemize}
\tightlist
\item
  \textbf{ફાયદાઓ}: સરળ રચના, સંપૂર્ણ નિયંત્રણ, કર લાભો
\item
  \textbf{નુકસાનો}: અમર્યાદિત જવાબદારી, મર્યાદિત સંસાધનો, વ્યાપારિક નિરંતરતાની
  સમસ્યાઓ
\item
  \textbf{અનુકૂળ}: નાના વ્યવસાયો, વ્યાવસાયિક સેવાઓ
\end{itemize}

\textbf{2. ભાગીદારી:}

\begin{itemize}
\tightlist
\item
  \textbf{ફાયદાઓ}: વહેંચાયેલા સંસાધનો, વિશિષ્ટ કુશળતા, સરળ રચના
\item
  \textbf{નુકસાનો}: અમર્યાદિત જવાબદારી, સંઘર્ષની સંભાવના, વહેંચાયેલા નફા
\item
  \textbf{પ્રકારો}: સામાન્ય ભાગીદારી, મર્યાદિત ભાગીદારી
\end{itemize}

\textbf{3. કંપની:}

\begin{itemize}
\tightlist
\item
  \textbf{ફાયદાઓ}: મર્યાદિત જવાબદારી, શાશ્વત અસ્તિત્વ, સરળ મૂડી ઊભી કરવી
\item
  \textbf{નુકસાનો}: જટિલ નિયમો, બેવડો કર, નિયંત્રણ ગુમાવવું
\item
  \textbf{પ્રકારો}: ખાનગી મર્યાદિત, જાહેર મર્યાદિત
\end{itemize}

\end{solutionbox}
\begin{mnemonicbox}
``એકલા ભાગીદારો કંપનીઓ સહકાર કરે છે''

\end{mnemonicbox}
\subsection*{પ્રશ્ન 3(અ) OR [3
ગુણ]}\label{uxaaauxab0uxab6uxaa8-3uxa85-or-3-uxa97uxaa3}

\textbf{વિવિધ લીડરશિપ મોડલ્સ સમજાવો.}

\begin{solutionbox}

\textbf{નેતૃત્વ મોડલ્સ:}

\begin{longtable}[]{@{}
  >{\raggedright\arraybackslash}p{(\linewidth - 4\tabcolsep) * \real{0.2683}}
  >{\raggedright\arraybackslash}p{(\linewidth - 4\tabcolsep) * \real{0.2927}}
  >{\raggedright\arraybackslash}p{(\linewidth - 4\tabcolsep) * \real{0.4390}}@{}}
\toprule\noalign{}
\begin{minipage}[b]{\linewidth}\raggedright
\textbf{મોડેલ}
\end{minipage} & \begin{minipage}[b]{\linewidth}\raggedright
\textbf{અભિગમ}
\end{minipage} & \begin{minipage}[b]{\linewidth}\raggedright
\textbf{શ્રેષ્ઠ ઉપયોગ}
\end{minipage} \\
\midrule\noalign{}
\endhead
\bottomrule\noalign{}
\endlastfoot
\textbf{સ્વૈરાચારી} & નેતા બધા નિર્ણયો લે છે & કટોકટીની પરિસ્થિતિઓ, ઝડપી નિર્ણયો
જરૂરી \\
\textbf{લોકશાહી} & સહભાગિતાપૂર્ણ નિર્ણય લેવું & ટીમનું ઇનપુટ મૂલ્યવાન, સમય ઉપલબ્ધ \\
\textbf{છૂટક હાથ} & હાથ છોડીને અભિગમ & અનુભવી ટીમ, સર્જનાત્મક કામ \\
\end{longtable}

\textbf{આધુનિક મોડલ્સ:}

\begin{itemize}
\tightlist
\item
  \textbf{રૂપાંતરણીય}: પ્રેરણાદાયક દ્રષ્ટિ અને પરિવર્તન
\item
  \textbf{વ્યવહારિક}: પુરસ્કાર-સજા આધારિત
\item
  \textbf{પરિસ્થિતિગત}: પરિસ્થિતિ મુજબ શૈલી સ્વીકારવી
\end{itemize}

\end{solutionbox}
\begin{mnemonicbox}
``સ્વૈરાચારી લોકશાહી છૂટક રૂપાંતર વ્યવહાર પરિસ્થિતિ''

\end{mnemonicbox}
\subsection*{પ્રશ્ન 3(બ) OR [4
ગુણ]}\label{uxaaauxab0uxab6uxaa8-3uxaac-or-4-uxa97uxaa3}

\textbf{વહીવટ અને સંચાલન વચ્ચેનો તફાવત આપો}

\begin{solutionbox}

\begin{longtable}[]{@{}lll@{}}
\toprule\noalign{}
\textbf{પાસું} & \textbf{વહીવટ (Administration)} & \textbf{સંચાલન
(Management)} \\
\midrule\noalign{}
\endhead
\bottomrule\noalign{}
\endlastfoot
\textbf{ધ્યાન} & નીતિ નિર્માણ & નીતિ અમલીકરણ \\
\textbf{સ્તર} & ઉચ્ચ સ્તરનું કાર્ય & મધ્યમ સ્તરનું કાર્ય \\
\textbf{પ્રકૃતિ} & આયોજન અને વિચારણા & કરવું અને અમલ \\
\textbf{અવકાશ} & વ્યાપક સંગઠનાત્મક & વિશિષ્ટ વિભાગીય \\
\end{longtable}

\textbf{મુખ્ય તફાવતો:}

\begin{itemize}
\tightlist
\item
  \textbf{વહીવટ}: વ્યૂહાત્મક, લાંબા ગાળાનું, કલ્પનાત્મક
\item
  \textbf{સંચાલન}: કામકાજી, ટૂંકા ગાળાનું, વ્યાવહારિક
\end{itemize}

\textbf{સંબંધ:} વહીવટ દિશા નક્કી કરે છે, સંચાલન યોજનાઓનો અમલ કરે છે

\end{solutionbox}
\begin{mnemonicbox}
``વહીવટ આયોજન કરે, સંચાલન અમલ કરે''

\end{mnemonicbox}
\subsection*{પ્રશ્ન 3(ક) OR [7
ગુણ]}\label{uxaaauxab0uxab6uxaa8-3uxa95-or-7-uxa97uxaa3}

\textbf{ઉદ્યોગ, વાણિજ્ય અને વ્યવસાય વચ્ચેના તફાવતની કલ્પના સમજાવો.}

\begin{solutionbox}

\begin{longtable}[]{@{}
  >{\raggedright\arraybackslash}p{(\linewidth - 6\tabcolsep) * \real{0.2000}}
  >{\raggedright\arraybackslash}p{(\linewidth - 6\tabcolsep) * \real{0.2167}}
  >{\raggedright\arraybackslash}p{(\linewidth - 6\tabcolsep) * \real{0.3667}}
  >{\raggedright\arraybackslash}p{(\linewidth - 6\tabcolsep) * \real{0.2167}}@{}}
\toprule\noalign{}
\begin{minipage}[b]{\linewidth}\raggedright
\textbf{કલ્પના}
\end{minipage} & \begin{minipage}[b]{\linewidth}\raggedright
\textbf{વ્યાખ્યા}
\end{minipage} & \begin{minipage}[b]{\linewidth}\raggedright
\textbf{પ્રાથમિક પ્રવૃત્તિ}
\end{minipage} & \begin{minipage}[b]{\linewidth}\raggedright
\textbf{ઉદાહરણો}
\end{minipage} \\
\midrule\noalign{}
\endhead
\bottomrule\noalign{}
\endlastfoot
\textbf{ઉદ્યોગ} & માલનું ઉત્પાદન & ઉત્પાદન, પ્રક્રિયા & સ્ટીલ, કાપડ, રસાયણો \\
\textbf{વાણિજ્ય} & માલનું વિતરણ & વેપાર, પરિવહન & જથ્થાબંધ, છૂટક, લોજિસ્ટિક્સ \\
\textbf{વ્યવસાય} & એકંદર આર્થિક પ્રવૃત્તિ & ઉત્પાદન + વિતરણ & સંપૂર્ણ એન્ટરપ્રાઇઝ
કામકાજ \\
\end{longtable}

\textbf{ઉદ્યોગના વર્ગો:}

\begin{itemize}
\tightlist
\item
  \textbf{પ્રાથમિક}: કાચા માલનું નિષ્કર્ષણ (ખાણકામ, કૃષિ)
\item
  \textbf{દ્વિતીયક}: ઉત્પાદન અને પ્રક્રિયા
\item
  \textbf{તૃતીયક}: સેવાઓ (બેંકિંગ, શિક્ષણ, આરોગ્ય)
\end{itemize}

\textbf{વાણિજ્યના કાર્યો:}

\begin{itemize}
\tightlist
\item
  \textbf{વેપાર}: ખરીદી અને વેચાણની પ્રવૃત્તિઓ
\item
  \textbf{સહાયક}: સહાયક સેવાઓ (પરિવહન, વીમો, બેંકિંગ)
\end{itemize}

\textbf{વ્યવસાયિક એકીકરણ:}

\begin{itemize}
\tightlist
\item
  \textbf{વર્ટિકલ}: ઉદ્યોગ + વાણિજ્ય એકીકરણ
\item
  \textbf{હોરિઝોન્ટલ}: સમાન સ્તરે વૈવિધ્યકરણ
\end{itemize}

\end{solutionbox}
\begin{mnemonicbox}
``ઉદ્યોગ બનાવે, વાણિજ્ય વિતરિત કરે, વ્યવસાય એકીકૃત કરે''

\end{mnemonicbox}
\subsection*{પ્રશ્ન 4(અ) [3
ગુણ]}\label{uxaaauxab0uxab6uxaa8-4uxa85-3-uxa97uxaa3}

\textbf{આ શબ્દો સમજાવો: 1.કરાર 2.કોપીરાઈટ}

\begin{solutionbox}

\begin{longtable}[]{@{}
  >{\raggedright\arraybackslash}p{(\linewidth - 4\tabcolsep) * \real{0.2250}}
  >{\raggedright\arraybackslash}p{(\linewidth - 4\tabcolsep) * \real{0.3250}}
  >{\raggedright\arraybackslash}p{(\linewidth - 4\tabcolsep) * \real{0.4500}}@{}}
\toprule\noalign{}
\begin{minipage}[b]{\linewidth}\raggedright
\textbf{શબ્દ}
\end{minipage} & \begin{minipage}[b]{\linewidth}\raggedright
\textbf{વ્યાખ્યા}
\end{minipage} & \begin{minipage}[b]{\linewidth}\raggedright
\textbf{મુખ્ય લક્ષણો}
\end{minipage} \\
\midrule\noalign{}
\endhead
\bottomrule\noalign{}
\endlastfoot
\textbf{કરાર} & પક્ષો વચ્ચેનો કાનૂની કરાર & બંધનકર્તા, લાગુ કરી શકાય, પરસ્પર
જવાબદારીઓ \\
\textbf{કોપીરાઈટ} & બૌદ્ધિક સંપદા સુરક્ષા & સર્જનાત્મક કાર્યો, વિશેષ અધિકારો,
મર્યાદિત અવધિ \\
\end{longtable}

\textbf{કરારના તત્વો:}

\begin{itemize}
\tightlist
\item
  \textbf{ઓફર અને સ્વીકૃતિ}: સ્પષ્ટ શરતો પર સંમતિ
\item
  \textbf{વિચારણા}: પક્ષો વચ્ચે મૂલ્યનું આદાનપ્રદાન
\item
  \textbf{કાનૂની ક્ષમતા}: પક્ષો કાનૂની રીતે સક્ષમ હોવા જોઈએ
\end{itemize}

\textbf{કોપીરાઈટ સુરક્ષા:}

\begin{itemize}
\tightlist
\item
  \textbf{અવધિ}: સામાન્ય રીતે જીવનકાળ + 70 વર્ષ
\item
  \textbf{અધિકારો}: પુનઃઉત્પાદન, વિતરણ, જાહેર પ્રદર્શન
\item
  \textbf{નોંધણી}: આવશ્યક નથી પરંતુ ભલામણ કરવામાં આવે છે
\end{itemize}

\end{solutionbox}
\begin{mnemonicbox}
``કરાર બંધે, કોપીરાઈટ સુરક્ષિત કરે''

\end{mnemonicbox}
\subsection*{પ્રશ્ન 4(બ) [4
ગુણ]}\label{uxaaauxab0uxab6uxaa8-4uxaac-4-uxa97uxaa3}

\textbf{સ્ટાર્ટઅપ ઇન્ક્યુબેશન સેન્ટર અને મોડાલિટીઝ પર એક નોંધ આપો.}

\begin{solutionbox}

\textbf{સ્ટાર્ટઅપ ઇન્ક્યુબેશન સેન્ટર્સ:}

\begin{itemize}
\tightlist
\item
  \textbf{હેતુ}: પ્રારંભિક તબક્કાના સ્ટાર્ટઅપ્સને સંસાધનો અને માર્ગદર્શન સાથે સહાય
\item
  \textbf{સેવાઓ}: માર્ગદર્શન, ભંડોળ, કાર્યક્ષેત્ર, નેટવર્કિંગ
\item
  \textbf{અવધિ}: સામાન્ય રીતે 6 મહિનાથી 2 વર્ષ
\end{itemize}

\textbf{મુખ્ય મોડાલિટીઝ:}

\textbf{પૂર્વ-ઇન્ક્યુબેશન:}

\begin{itemize}
\tightlist
\item
  \textbf{આઈડિયા વેલિડેશન}: બજાર સંશોધન અને શક્યતા
\item
  \textbf{ટીમ રચના}: મુખ્ય ટીમ બનાવવી
\item
  \textbf{પ્રોટોટાઇપ વિકાસ}: MVP બનાવવું
\end{itemize}

\textbf{ઇન્ક્યુબેશન તબક્કો:}

\begin{itemize}
\tightlist
\item
  \textbf{બિઝનેસ મોડેલ સુધારણા}: આવકના મોડેલનો વિકાસ
\item
  \textbf{બજાર પરીક્ષણ}: ગ્રાહક વેલિડેશન
\item
  \textbf{ભંડોળ તૈયારી}: રોકાણકાર પિચ તૈયારી
\end{itemize}

\textbf{પોસ્ટ-ઇન્ક્યુબેશન:}

\begin{itemize}
\tightlist
\item
  \textbf{એલ્યુમ્નાઈ નેટવર્ક}: સતત સહાય અને જોડાણો
\item
  \textbf{ફોલો-અપ ભંડોળ}: સિરીઝ A તૈયારી
\item
  \textbf{સ્કેલિંગ સહાય}: વૃદ્ધિ વ્યૂહરચના સહાયતા
\end{itemize}

\end{solutionbox}
\begin{mnemonicbox}
``પૂર્વ-ઇન્ક્યુબેટ, ઇન્ક્યુબેટ, પોસ્ટ-સપોર્ટ સ્ટાર્ટઅપ્સ''

\end{mnemonicbox}
\subsection*{પ્રશ્ન 4(ક) [7
ગુણ]}\label{uxaaauxab0uxab6uxaa8-4uxa95-7-uxa97uxaa3}

\textbf{રાજ્ય સ્તરની એજન્સીઓની યાદી બનાવો જે સ્ટાર્ટ-અપ્સને સમર્થન આપે છે અને તેમની
કાર્યક્ષમતાનું વર્ણન કરો}

\begin{solutionbox}

\textbf{ગુજરાત રાજ્ય સહાય એજન્સીઓ:}

\begin{longtable}[]{@{}
  >{\raggedright\arraybackslash}p{(\linewidth - 4\tabcolsep) * \real{0.2500}}
  >{\raggedright\arraybackslash}p{(\linewidth - 4\tabcolsep) * \real{0.3542}}
  >{\raggedright\arraybackslash}p{(\linewidth - 4\tabcolsep) * \real{0.3958}}@{}}
\toprule\noalign{}
\begin{minipage}[b]{\linewidth}\raggedright
\textbf{એજન્સી}
\end{minipage} & \begin{minipage}[b]{\linewidth}\raggedright
\textbf{સંપૂર્ણ નામ}
\end{minipage} & \begin{minipage}[b]{\linewidth}\raggedright
\textbf{મુખ્ય કાર્યો}
\end{minipage} \\
\midrule\noalign{}
\endhead
\bottomrule\noalign{}
\endlastfoot
\textbf{SSIP} & Student Startup \& Innovation Policy & વિદ્યાર્થી
ઉદ્યોગસાહસિક સહાય, ભંડોળ \\
\textbf{iHub Gujarat} & Innovation Hub Gujarat & ઇન્ક્યુબેશન, માર્ગદર્શન,
નેટવર્કિંગ \\
\textbf{GUSEC} & Gujarat University Startup \& Entrepreneurship Council
& યુનિવર્સિટી સ્તરે સ્ટાર્ટઅપ પ્રમોશન \\
\textbf{GIDC} & Gujarat Industrial Development Corporation & ઔદ્યોગિક
ઇન્ફ્રાસ્ટ્રક્ચર, જમીન ફાળવણી \\
\end{longtable}

\textbf{વિગતવાર કાર્યક્ષમતાઓ:}

\textbf{SSIP Gujarat:}

\begin{itemize}
\tightlist
\item
  \textbf{ભંડોળ સહાય}: વિદ્યાર્થી સ્ટાર્ટઅપ્સ માટે ₹2 લાખ સુધી
\item
  \textbf{ઇન્ક્યુબેશન સુવિધાઓ}: કાર્યક્ષેત્ર અને સાધનોની પહોંચ
\item
  \textbf{માર્ગદર્શન કાર્યક્રમો}: ઉદ્યોગ નિષ્ણાત માર્ગદર્શન
\item
  \textbf{IPR સહાય}: પેટન્ટ ફાઇલિંગ સહાયતા
\end{itemize}

\textbf{iHub Gujarat:}

\begin{itemize}
\tightlist
\item
  \textbf{સ્ટાર્ટઅપ ઇકોસિસ્ટમ}: સંપૂર્ણ ઉદ્યોગસાહસિકતા સહાય
\item
  \textbf{ટેકનોલોજી ટ્રાન્સફર}: સંશોધનથી બજાર તરફ રૂપાંતરણ
\item
  \textbf{રોકાણકાર જોડાણો}: ભંડોળ સુવિધા
\item
  \textbf{ઉદ્યોગ ભાગીદારી}: કોર્પોરેટ સહયોગ
\end{itemize}

\textbf{GUSEC:}

\begin{itemize}
\tightlist
\item
  \textbf{વિદ્યાર્થી સંડોવણી}: કેમ્પસ ઉદ્યોગસાહસિકતા કાર્યક્રમો
\item
  \textbf{કુશળતા વિકાસ}: ઉદ્યોગસાહસિકતા શિક્ષણ
\item
  \textbf{સ્પર્ધા આયોજન}: સ્ટાર્ટઅપ હરિફાઈ અને પિચ
\item
  \textbf{નેટવર્ક બિલ્ડિંગ}: એલ્યુમ્નાઈ ઉદ્યોગસાહસિક જોડાણો
\end{itemize}

\textbf{GIDC:}

\begin{itemize}
\tightlist
\item
  \textbf{ઇન્ફ્રાસ્ટ્રક્ચર}: ઔદ્યોગિક જમીન અને સુવિધાઓ
\item
  \textbf{પોલિસી સપોર્ટ}: સરકારી યોજનાઓ અને પ્રોત્સાહનો
\item
  \textbf{વન-સ્ટોપ સર્વિસ}: તમામ લાઇસન્સ અને ક્લિયરન્સ
\item
  \textbf{ઇન્ડસ્ટ્રીયલ પાર્ક}: વિશિષ્ટ ઉદ્યોગ ક્લસ્ટર્સ
\end{itemize}

\textbf{પ્રભાવ માપદંડ:}

\begin{itemize}
\tightlist
\item
  \textbf{વાર્ષિક સમર્થિત સ્ટાર્ટઅપ્સની સંખ્યા}
\item
  \textbf{સમર્થિત સાહસો દ્વારા રોજગાર સર્જન}
\item
  \textbf{ઇન્ક્યુબેટેડ કંપનીઓનું આવક ઉત્પાદન}
\item
  \textbf{સ્નાતક સ્ટાર્ટઅપ્સની સફળતા દર}
\end{itemize}

\end{solutionbox}
\begin{mnemonicbox}
``SSIP iHub GUSEC GIDC ગુજરાત સ્ટાર્ટઅપ્સને સપોર્ટ કરે''

\end{mnemonicbox}
\subsection*{પ્રશ્ન 4(અ) OR [3
ગુણ]}\label{uxaaauxab0uxab6uxaa8-4uxa85-or-3-uxa97uxaa3}

\textbf{આ શબ્દો સમજાવો: 1.IPR 2.ટ્રેડમાર્ક્સ}

\begin{solutionbox}

\begin{longtable}[]{@{}
  >{\raggedright\arraybackslash}p{(\linewidth - 4\tabcolsep) * \real{0.1698}}
  >{\raggedright\arraybackslash}p{(\linewidth - 4\tabcolsep) * \real{0.4717}}
  >{\raggedright\arraybackslash}p{(\linewidth - 4\tabcolsep) * \real{0.3585}}@{}}
\toprule\noalign{}
\begin{minipage}[b]{\linewidth}\raggedright
\textbf{શબ્દ}
\end{minipage} & \begin{minipage}[b]{\linewidth}\raggedright
\textbf{સંપૂર્ણ નામ/વ્યાખ્યા}
\end{minipage} & \begin{minipage}[b]{\linewidth}\raggedright
\textbf{સુરક્ષા અવકાશ}
\end{minipage} \\
\midrule\noalign{}
\endhead
\bottomrule\noalign{}
\endlastfoot
\textbf{IPR} & Intellectual Property Rights & વિચારો, શોધો, સર્જનાત્મક
કાર્યો \\
\textbf{ટ્રેડમાર્ક્સ} & બ્રાન્ડ ઓળખ ચિહ્નો & નામો, લોગો, પ્રતીકો, સ્લોગન્સ \\
\end{longtable}

\textbf{IPR વર્ગો:}

\begin{itemize}
\tightlist
\item
  \textbf{પેટન્ટ્સ}: તકનીકી શોધો (20 વર્ષ)
\item
  \textbf{કોપીરાઈટ્સ}: સર્જનાત્મક અભિવ્યક્તિઓ (જીવનકાળ + 70 વર્ષ)
\item
  \textbf{ટ્રેડમાર્ક્સ}: બ્રાન્ડ ઓળખકર્તા (10 વર્ષ, નવીકરણ યોગ્ય)
\end{itemize}

\textbf{ટ્રેડમાર્કની લક્ષણો:}

\begin{itemize}
\tightlist
\item
  \textbf{વિશિષ્ટતા}: અનન્ય બ્રાન્ડ ઓળખ
\item
  \textbf{વ્યાપારિક ઉપયોગ}: વ્યાપારિક ઓળખાણનો હેતુ
\item
  \textbf{નોંધણી}: નોંધણી દ્વારા કાનૂની સુરક્ષા
\end{itemize}

\end{solutionbox}
\begin{mnemonicbox}
``IPR સુરક્ષિત કરે, ટ્રેડમાર્ક્સ ઓળખે''

\end{mnemonicbox}
\subsection*{પ્રશ્ન 4(બ) OR [4
ગુણ]}\label{uxaaauxab0uxab6uxaa8-4uxaac-or-4-uxa97uxaa3}

\textbf{સ્ટાર્ટ-અપમાં રોકાણકારની ભૂમિકા વ્યાખ્યાયિત કરો.}

\begin{solutionbox}

\textbf{સ્ટાર્ટઅપ્સમાં રોકાણકારની ભૂમિકાઓ:}

\textbf{નાણાકીય સહાય:}

\begin{itemize}
\tightlist
\item
  \textbf{સીડ ફંડિંગ}: બિઝનેસ લોન્ચ માટે પ્રારંભિક મૂડી
\item
  \textbf{ગ્રોથ કેપિટલ}: વિસ્તરણ અને સ્કેલિંગ ફંડ્સ
\item
  \textbf{બ્રિજ ફાઇનાન્સિંગ}: રાઉન્ડ્સ વચ્ચે વચગાળાનું ભંડોળ
\end{itemize}

\textbf{વ્યૂહાત્મક માર્ગદર્શન:}

\begin{itemize}
\tightlist
\item
  \textbf{બિઝનેસ માર્ગદર્શન}: ઉદ્યોગ અનુભવ શેરિંગ
\item
  \textbf{નેટવર્ક એક્સેસ}: ગ્રાહકો અને ભાગીદારો સાથે જોડાણો
\item
  \textbf{બજાર અંતર્દૃષ્ટિ}: ઉદ્યોગ જ્ઞાન અને વલણો
\end{itemize}

\textbf{કામકાજી સહાય:}

\begin{itemize}
\tightlist
\item
  \textbf{ટીમ બિલ્ડિંગ}: ભરતી અને પ્રતિભા સંપાદનની સલાહ
\item
  \textbf{ટેકનોલોજી માર્ગદર્શન}: તકનીકી આર્કિટેક્ચર સૂચનો
\item
  \textbf{કાનૂની અનુપાલન}: નિયમનકારી અને અનુપાલન સહાય
\end{itemize}

\textbf{જોખમ વ્યવસ્થાપન:}

\begin{itemize}
\tightlist
\item
  \textbf{ડ્યુ ડિલિજન્સ}: બિઝનેસ મોડેલ વેલિડેશન
\item
  \textbf{પ્રદર્શન મોનિટરિંગ}: નિયમિત પ્રગતિ ટ્રેકિંગ
\item
  \textbf{એક્ઝિટ સ્ટ્રેટેજી}: રોકાણ પુનઃપ્રાપ્તિ માટે આયોજન
\end{itemize}

\textbf{રોકાણકારોના પ્રકારો:}

\begin{itemize}
\tightlist
\item
  \textbf{એન્જલ ઇન્વેસ્ટર્સ}: વ્યક્તિગત ઉચ્ચ નેટવર્થ રોકાણકારો
\item
  \textbf{વેન્ચર કેપિટલ}: વ્યાવસાયિક રોકાણ કંપનીઓ
\item
  \textbf{કોર્પોરેટ ઇન્વેસ્ટર્સ}: વ્યૂહાત્મક ઉદ્યોગ ખેલાડીઓ
\end{itemize}

\end{solutionbox}
\begin{mnemonicbox}
``નાણા વ્યૂહરચના કામકાજ જોખમ = રોકાણકાર ભૂમિકાઓ''

\end{mnemonicbox}
\subsection*{પ્રશ્ન 4(ક) OR [7
ગુણ]}\label{uxaaauxab0uxab6uxaa8-4uxa95-or-7-uxa97uxaa3}

\textbf{રાષ્ટ્રીય સ્તરની એજન્સીઓની યાદી બનાવો જે સ્ટાર્ટ-અપ્સને સમર્થન આપે છે અને
તેમની કાર્યક્ષમતાનું વર્ણન કરો.}

\begin{solutionbox}

\textbf{રાષ્ટ્રીય સ્ટાર્ટઅપ સહાય એજન્સીઓ:}

\begin{longtable}[]{@{}lll@{}}
\toprule\noalign{}
\textbf{એજન્સી} & \textbf{મંત્રાલય/વિભાગ} & \textbf{પ્રાથમિક ફોકસ} \\
\midrule\noalign{}
\endhead
\bottomrule\noalign{}
\endlastfoot
\textbf{Startup India} & DPIIT, વાણિજ્ય મંત્રાલય & નીતિ ફ્રેમવર્ક અને
ઇકોસિસ્ટમ \\
\textbf{BIRAC} & બાયોટેકનોલોજી વિભાગ & બાયોટેકનોલોજી નવીનતા \\
\textbf{TDB} & વિજ્ઞાન અને ટેકનોલોજી વિભાગ & ટેકનોલોજી વિકાસ \\
\textbf{SIDBI} & નાણાકીય સેવાઓ & MSME અને સ્ટાર્ટઅપ ભંડોળ \\
\end{longtable}

\textbf{વિગતવાર કાર્યક્ષમતાઓ:}

\textbf{Startup India:}

\begin{itemize}
\tightlist
\item
  \textbf{નીતિ ઘડતર}: રાષ્ટ્રીય સ્ટાર્ટઅપ નીતિ ફ્રેમવર્ક
\item
  \textbf{માન્યતા કાર્યક્રમ}: સત્તાવાર સ્ટાર્ટઅપ પ્રમાણપત્ર
\item
  \textbf{કર લાભો}: પાત્ર સ્ટાર્ટઅપ્સ માટે 3-વર્ષની કર મુક્તિ
\item
  \textbf{નિયમનકારી સહાય}: સિંગલ-પોઇન્ટ ક્લિયરન્સ સિસ્ટમ
\item
  \textbf{ભંડોળ સુવિધા}: Fund of Funds યોજના (₹10,000 કરોડ)
\end{itemize}

\textbf{BIRAC (Biotechnology Industry Research Assistance Council):}

\begin{itemize}
\tightlist
\item
  \textbf{બાયોટેક નવીનતા}: બાયોટેક સ્ટાર્ટઅપ્સ અને સંશોધનને સમર્થન
\item
  \textbf{ભંડોળ યોજનાઓ}: SBIRI, SPARSH, BIG કાર્યક્રમો
\item
  \textbf{ઉદ્યોગ ભાગીદારી}: શિક્ષણ-ઉદ્યોગ સહયોગ
\item
  \textbf{ટેકનોલોજી અનુવાદ}: સંશોધનથી બજાર તરફ રૂપાંતરણ
\end{itemize}

\textbf{TDB (Technology Development Board):}

\begin{itemize}
\tightlist
\item
  \textbf{ટેકનોલોજી કોમર્શિયલાઇઝેશન}: સંશોધનને ઉત્પાદનોમાં રૂપાંતરિત કરવું
\item
  \textbf{નાણાકીય સહાય}: ટેકનોલોજી વિકાસ માટે લોન અને ગ્રાન્ટ
\item
  \textbf{ઉદ્યોગ સહાય}: ઉત્પાદન ટેકનોલોજી સહાયતા
\item
  \textbf{નવીનતા પ્રમોશન}: તકનીકી નવીનતાને સમર્થન
\end{itemize}

\textbf{SIDBI (Small Industries Development Bank of India):}

\begin{itemize}
\tightlist
\item
  \textbf{નાણાકીય સહાય}: લોન અને ક્રેડિટ સુવિધાઓ
\item
  \textbf{MSME ફોકસ}: નાના અને મધ્યમ ઉદ્યોગ વિકાસ
\item
  \textbf{સ્ટાર્ટઅપ ભંડોળ}: વેન્ચર કેપિટલ અને ગ્રોથ કેપિટલ
\item
  \textbf{ઇકોસિસ્ટમ વિકાસ}: ઇન્ક્યુબેટર અને એક્સેલેરેટર સહાય
\end{itemize}

\textbf{અતિરિક્ત એજન્સીઓ:}

\begin{itemize}
\tightlist
\item
  \textbf{NSTEDB}: National Science \& Technology Entrepreneurship
  Development Board
\item
  \textbf{MSME}: Micro, Small and Medium Enterprises મંત્રાલય
\item
  \textbf{Atal Innovation Mission}: નવીનતા અને ઉદ્યોગસાહસિકતા પ્રમોશન
\end{itemize}

\textbf{સફળતાના મેટ્રિક્સ:}

\begin{itemize}
\tightlist
\item
  \textbf{સ્ટાર્ટઅપ નોંધણીઓ}: 70,000+ માન્યતા પ્રાપ્ત સ્ટાર્ટઅપ્સ
\item
  \textbf{રોજગાર સર્જન}: લાખો રોજગારની તકો
\item
  \textbf{ભંડોળ સુવિધા}: અબજો રોકાણ એકત્રીકરણ
\item
  \textbf{ઇકોસિસ્ટમ વિકાસ}: હજારો ઇન્ક્યુબેટર અને એક્સેલેરેટર
\end{itemize}

\end{solutionbox}
\begin{mnemonicbox}
``Startup BIRAC TDB SIDBI = રાષ્ટ્રીય સહાય પ્રણાલી''

\end{mnemonicbox}
\subsection*{પ્રશ્ન 5(અ) [3
ગુણ]}\label{uxaaauxab0uxab6uxaa8-5uxa85-3-uxa97uxaa3}

\textbf{આ શરતો સમજાવો: 1.બ્રેક ઇવન પોઇન્ટ 2.રોકાણ પર વળતર 3.વેચાણ પર વળતર.}

\begin{solutionbox}

\begin{longtable}[]{@{}
  >{\raggedright\arraybackslash}p{(\linewidth - 4\tabcolsep) * \real{0.2812}}
  >{\raggedright\arraybackslash}p{(\linewidth - 4\tabcolsep) * \real{0.4062}}
  >{\raggedright\arraybackslash}p{(\linewidth - 4\tabcolsep) * \real{0.3125}}@{}}
\toprule\noalign{}
\begin{minipage}[b]{\linewidth}\raggedright
\textbf{શરત}
\end{minipage} & \begin{minipage}[b]{\linewidth}\raggedright
\textbf{ફોર્મુલા}
\end{minipage} & \begin{minipage}[b]{\linewidth}\raggedright
\textbf{અર્થ}
\end{minipage} \\
\midrule\noalign{}
\endhead
\bottomrule\noalign{}
\endlastfoot
\textbf{બ્રેક ઇવન પોઇન્ટ} & નિશ્ચિત ખર્ચ \div (કિંમત - ચલ ખર્ચ) & બધા ખર્ચને આવરવા
માટેના એકમો \\
\textbf{રોકાણ પર વળતર} & (લાભ-ખર્ચ) \div ખર્ચ \times 100 & રોકાણ કરેલી મૂડી પર
ટકાવારી વળતર \\
\textbf{વેચાણ પર વળતર} & ચોખ્ખી આવક \div વેચાણ \times 100 & નફાના માર્જિનની
ટકાવારી \\
\end{longtable}

\textbf{બ્રેક ઇવન વિશ્લેષણ:}

\begin{itemize}
\tightlist
\item
  \textbf{નિશ્ચિત ખર્ચ}: ભાડું, પગાર, વીમો
\item
  \textbf{ચલ ખર્ચ}: કાચો માલ, એકમ દીઠ ઉપયોગિતાઓ
\item
  \textbf{યોગદાન માર્જિન}: એકમ દીઠ કિંમત માઈનસ ચલ ખર્ચ
\end{itemize}

\textbf{ROI મહત્વ:}

\begin{itemize}
\tightlist
\item
  \textbf{રોકાણ કાર્યક્ષમતા}: રોકાણની કામગીરી માપે છે
\item
  \textbf{તુલના સાધન}: વિવિધ રોકાણ વિકલ્પોની તુલના
\item
  \textbf{નિર્ણય લેવું}: ભાવિ રોકાણના નિર્ણયોનું માર્ગદર્શન
\end{itemize}

\textbf{ROS મહત્વ:}

\begin{itemize}
\tightlist
\item
  \textbf{નફાકારકતાનું માપદંડ}: કામકાજની કાર્યક્ષમતા દર્શાવે છે
\item
  \textbf{ઉદ્યોગ તુલના}: સ્પર્ધકો સાથે બેન્ચમાર્ક
\item
  \textbf{વલણ વિશ્લેષણ}: સમય સાથે કામગીરી ટ્રેક કરવું
\end{itemize}

\end{solutionbox}
\begin{mnemonicbox}
``બ્રેક ઇવન રોકાણ વેચાણ પર વળતર''

\end{mnemonicbox}
\subsection*{પ્રશ્ન 5(બ) [4
ગુણ]}\label{uxaaauxab0uxab6uxaa8-5uxaac-4-uxa97uxaa3}

\textbf{આયાત-નિકાસ નીતિ પર ટૂંકી નોંધ લખો}

\begin{solutionbox}

\textbf{ભારતની આયાત-નિકાસ નીતિ (EXIM નીતિ):}

\textbf{ઉદ્દેશ્યો:}

\begin{itemize}
\tightlist
\item
  \textbf{વેપાર પ્રમોશન}: આંતરરાષ્ટ્રીય વેપારનું પ્રમાણ વધારવું
\item
  \textbf{નિકાસ વૃદ્ધિ}: નિકાસ કમાણી અને સ્પર્ધાત્મકતા વધારવી
\item
  \textbf{આર્થિક વિકાસ}: ઉત્પાદન અને રોજગાર સર્જનને સમર્થન
\end{itemize}

\textbf{મુખ્ય લક્ષણો:}

\textbf{નિકાસ પ્રમોશન:}

\begin{itemize}
\tightlist
\item
  \textbf{નિકાસ પ્રોત્સાહનો}: ડ્યુટી ડ્રોબેક, MEIS યોજનાઓ
\item
  \textbf{વિશેષ આર્થિક ઝોન}: કરમુક્ત નિકાસ ઉત્પાદન
\item
  \textbf{નિકાસ ફાઇનાન્સિંગ}: ક્રેડિટ સુવિધાઓ અને વીમો
\end{itemize}

\textbf{આયાત વ્યવસ્થાપન:}

\begin{itemize}
\tightlist
\item
  \textbf{આયાત લાઇસન્સિંગ}: સંવેદનશીલ વસ્તુઓની નિયંત્રિત આયાત
\item
  \textbf{ડ્યુટી સ્ટ્રક્ચર}: ટેરિફ દરો અને કસ્ટમ પ્રક્રિયાઓ
\item
  \textbf{ગુણવત્તા ધોરણો}: BIS અને અન્ય ગુણવત્તા આવશ્યકતાઓ
\end{itemize}

\textbf{વેપાર સુવિધા:}

\begin{itemize}
\tightlist
\item
  \textbf{ડિજિટલ પ્લેટફોર્મ}: ઓનલાઇન નિકાસ-આયાત પ્રક્રિયાઓ
\item
  \textbf{સિંગલ વિન્ડો}: એકીકૃત ક્લિયરન્સ પ્રણાલી
\item
  \textbf{વેપાર કરારો}: દ્વિપક્ષીય અને બહુપક્ષીય કરારો
\end{itemize}

\textbf{વર્તમાન ફોકસ ક્ષેત્રો:}

\begin{itemize}
\tightlist
\item
  \textbf{મેક ઇન ઇન્ડિયા}: સ્થાનિક ઉત્પાદનને પ્રોત્સાહન
\item
  \textbf{ડિજિટલ ઇન્ડિયા}: ટેકનોલોજી-સક્ષમ વેપાર પ્રક્રિયાઓ
\item
  \textbf{આત્મનિર્ભર ભારત}: સ્વનિર્ભરતા અને આયાત વિકલ્પ
\end{itemize}

\end{solutionbox}
\begin{mnemonicbox}
``નિકાસ આયાત નીતિ વેપાર સુવિધાને પ્રોત્સાહન આપે છે''

\end{mnemonicbox}
\subsection*{પ્રશ્ન 5(ક) [7
ગુણ]}\label{uxaaauxab0uxab6uxaa8-5uxa95-7-uxa97uxaa3}

\textbf{CSR અને આર્થિક કામગીરી વચ્ચેના જોડાણનું વર્ણન કરો.}

\begin{solutionbox}

\textbf{કોર્પોરેટ સામાજિક જવાબદારી (CSR) અને આર્થિક કામગીરીનું કડી:}

\textbf{પ્રત્યક્ષ આર્થિક લાભો:}

\begin{longtable}[]{@{}
  >{\raggedright\arraybackslash}p{(\linewidth - 4\tabcolsep) * \real{0.4000}}
  >{\raggedright\arraybackslash}p{(\linewidth - 4\tabcolsep) * \real{0.3556}}
  >{\raggedright\arraybackslash}p{(\linewidth - 4\tabcolsep) * \real{0.2444}}@{}}
\toprule\noalign{}
\begin{minipage}[b]{\linewidth}\raggedright
\textbf{CSR પ્રવૃત્તિ}
\end{minipage} & \begin{minipage}[b]{\linewidth}\raggedright
\textbf{આર્થિક અસર}
\end{minipage} & \begin{minipage}[b]{\linewidth}\raggedright
\textbf{માપદંડ}
\end{minipage} \\
\midrule\noalign{}
\endhead
\bottomrule\noalign{}
\endlastfoot
\textbf{કર્મચારી કલ્યાણ} & ઊંચી ઉત્પાદકતા, ઓછું ટર્નઓવર & ખર્ચ બચત, કાર્યક્ષમતા
વધારો \\
\textbf{પર્યાવરણ પહેલ} & સંસાધન કાર્યક્ષમતા, કચરો ઘટાડો & ખર્ચ ઘટાડો,
ટકાઉપણું \\
\textbf{સમુદાય વિકાસ} & બજાર વિસ્તરણ, બ્રાન્ડ વફાદારી & આવક વૃદ્ધિ, ગ્રાહક
જાળવણી \\
\end{longtable}

\textbf{પરોક્ષ આર્થિક લાભો:}

\textbf{બ્રાન્ડ વેલ્યુ એન્હાન્સમેન્ટ:}

\begin{itemize}
\tightlist
\item
  \textbf{ગ્રાહક વફાદારી}: પુનરાવર્તિત ખરીદીઓ અને રેફરલ્સમાં વધારો
\item
  \textbf{પ્રીમિયમ પ્રાઇસિંગ}: નૈતિક ઉત્પાદનો માટે ઊંચી કિંમતો વસૂલવાની ક્ષમતા
\item
  \textbf{બજાર ભિન્નતા}: સભાન બજારોમાં સ્પર્ધાત્મક લાભ
\end{itemize}

\textbf{જોખમ વ્યવસ્થાપન:}

\begin{itemize}
\tightlist
\item
  \textbf{નિયમનકારી અનુપાલન}: દંડ અને કાનૂની ખર્ચથી બચવું
\item
  \textbf{પ્રતિષ્ઠા સુરક્ષા}: સામાજિક મુદ્દાઓથી બ્રાન્ડ નુકસાન અટકાવવું
\item
  \textbf{હિતધારક સંબંધો}: રોકાણકારો અને ભાગીદારો સાથે વિશ્વાસ નિર્માણ
\end{itemize}

\textbf{લાંબા ગાળાની આર્થિક કામગીરી:}

\textbf{ટકાઉ વૃદ્ધિ:}

\begin{itemize}
\tightlist
\item
  \textbf{નવીનતાનું ચાલક}: CSR પહેલો ઘણીવાર નવીન ઉકેલો તરફ દોરી જાય છે
\item
  \textbf{બજાર પ્રવેશ}: આંતરરાષ્ટ્રીય બજારો માટે ESG માપદંડો પૂરા કરવા
\item
  \textbf{રોકાણ આકર્ષણ}: ESG-કેન્દ્રિત રોકાણકારો જવાબદાર કંપનીઓને પસંદ કરે છે
\end{itemize}

\textbf{સંશોધન પુરાવા:}

\begin{itemize}
\tightlist
\item
  \textbf{કામગીરી સહસંબંધ}: અભ્યાસો CSR અને નાણાકીય કામગીરી વચ્ચે સકારાત્મક
  સહસંબંધ દર્શાવે છે
\item
  \textbf{રોકાણકાર પસંદગી}: ESG-અનુપાલિત કંપનીઓ વધુ રોકાણ આકર્ષે છે
\item
  \textbf{બજાર મૂલ્યાંકન}: જવાબદાર કંપનીઓનું ઘણીવાર ઊંચું બજાર મૂલ્યાંકન હોય છે
\end{itemize}

\textbf{CSR-આર્થિક કામગીરી ચક્ર:}

\begin{itemize}
\tightlist
\item
  \textbf{CSR માં રોકાણ} \rightarrow \textbf{કામકાજી સુધારાઓ} \rightarrow \textbf{નાણાકીય
  કામગીરી} \rightarrow \textbf{વધુ CSR રોકાણ}
\end{itemize}

\textbf{અમલીકરણ વ્યૂહરચના:}

\begin{itemize}
\tightlist
\item
  \textbf{વ્યૂહાત્મક ગોઠવણી}: CSR ને વ્યાપારિક ઉદ્દેશ્યો સાથે સંરેખિત કરવી
\item
  \textbf{માપન પ્રણાલીઓ}: સામાજિક અને આર્થિક બંને અસરોને ટ્રેક કરવી
\item
  \textbf{હિતધારક સંડોવણી}: CSR આયોજનમાં તમામ હિતધારકોને સામેલ કરવા
\item
  \textbf{સતત સુધારણા}: CSR કાર્યક્રમોની નિયમિત સમીક્ષા અને વૃદ્ધિ
\end{itemize}

\textbf{પડકારો:}

\begin{itemize}
\tightlist
\item
  \textbf{ટૂંકા ગાળાના ખર્ચ}: પ્રારંભિક રોકાણ તાત્કાલિક નફાને અસર કરી શકે છે
\item
  \textbf{માપન મુશ્કેલી}: સામાજિક અસરનું પ્રમાણીકરણ જટિલ હોઈ શકે છે
\item
  \textbf{હિતધારક અપેક્ષાઓ}: વિવિધ હિતધારક માંગણીઓમાં સંતુલન
\end{itemize}

\textbf{સફળતાના પરિબળો:}

\begin{itemize}
\tightlist
\item
  \textbf{નેતૃત્વ પ્રતિબદ્ધતા}: CSR પહેલો માટે ટોચના સંચાલનનું સમર્થન
\item
  \textbf{એકીકરણ}: CSR ને વ્યાપારિક વ્યૂહરચના અને કામકાજમાં જોડવું
\item
  \textbf{પારદર્શિતા}: CSR અસરની નિયમિત રિપોર્ટિંગ અને સંવાદ
\item
  \textbf{નવીનતા}: વ્યાપારિક નવીનતા માટે CSR નો ઉપયોગ કરવો
\end{itemize}

\end{solutionbox}
\begin{mnemonicbox}
``CSR ટકાઉ વળતર બનાવે છે''

\end{mnemonicbox}
\subsection*{પ્રશ્ન 5(અ) OR [3
ગુણ]}\label{uxaaauxab0uxab6uxaa8-5uxa85-or-3-uxa97uxaa3}

\textbf{નાદારી અને અવગણના પર એક નોંધ લખો.}

\begin{solutionbox}

\textbf{નાદારી:}

\begin{itemize}
\tightlist
\item
  \textbf{વ્યાખ્યા}: કાનૂની પ્રક્રિયા જ્યારે વ્યવસાય નાણાકીય જવાબદારીઓ પૂરી કરી
  શકતો નથી
\item
  \textbf{પ્રકારો}: સ્વૈચ્છિક (સ્વ-પ્રારંભિત) અથવા અનૈચ્છિક (લેણદાર-પ્રારંભિત)
\item
  \textbf{પ્રક્રિયા}: કોર્ટની દેખરેખ હેઠળ સંપત્તિ લિક્વિડેશન અથવા પુનર્ગઠન
\end{itemize}

\textbf{અવગણવાની વ્યૂહરચનાઓ:}

\begin{itemize}
\tightlist
\item
  \textbf{રોકડ પ્રવાહ વ્યવસ્થાપન}: પર્યાપ્ત કાર્યકારી મૂડી જાળવવી
\item
  \textbf{દેવું પુનઃરચના}: લેણદારો સાથે ચુકવણીની શરતોની વાટાઘાટ
\item
  \textbf{ખર્ચ ઘટાડો}: બિનજરૂરી ખર્ચ કાપવો અને કાર્યક્ષમતા સુધારવી
\end{itemize}

\textbf{કાનૂની ફ્રેમવર્ક:}

\begin{itemize}
\tightlist
\item
  \textbf{નાદારી અને નાદારી કોડ (IBC)}: ભારતીય નાદારી કાયદો
\item
  \textbf{રિઝોલ્યુશન પ્રક્રિયા}: રિઝોલ્યુશન માટે 180-દિવસની સમયમર્યાદા
\item
  \textbf{હિતધારક સુરક્ષા}: લેણદારો અને દેવાદારો માટે સંતુલિત અભિગમ
\end{itemize}

\end{solutionbox}
\begin{mnemonicbox}
``નાદાર વ્યવસાયો રોકડ નિયંત્રણ દ્વારા ટાળે છે''

\end{mnemonicbox}
\subsection*{પ્રશ્ન 5(બ) OR [4
ગુણ]}\label{uxaaauxab0uxab6uxaa8-5uxaac-or-4-uxa97uxaa3}

\textbf{બિઝનેસ એથિક્સનું મહત્વ લખો}

\begin{solutionbox}

\textbf{બિઝનેસ એથિક્સનું મહત્વ:}

\textbf{હિતધારક વિશ્વાસ:}

\begin{itemize}
\tightlist
\item
  \textbf{ગ્રાહક વિશ્વાસ}: નૈતિક પ્રથાઓ ગ્રાહક વફાદારી બનાવે છે
\item
  \textbf{રોકાણકાર શ્રદ્ધા}: પારદર્શી કામકાજ રોકાણ આકર્ષે છે
\item
  \textbf{કર્મચારી સંતુષ્ટિ}: નૈતિક કાર્યક્ષેત્ર જાળવણી સુધારે છે
\end{itemize}

\textbf{કાનૂની અનુપાલન:}

\begin{itemize}
\tightlist
\item
  \textbf{નિયમનકારી પાલન}: કાનૂની દંડ અને પ્રતિબંધોથી બચવું
\item
  \textbf{જોખમ ઘટાડો}: નૈતિક કૌભાંડ અને કટોકટીઓ અટકાવવી
\item
  \textbf{પ્રતિષ્ઠા સુરક્ષા}: સકારાત્મક બ્રાન્ડ છબી જાળવવી
\end{itemize}

\textbf{સ્પર્ધાત્મક લાભ:}

\begin{itemize}
\tightlist
\item
  \textbf{બજાર ભિન્નતા}: નૈતિક બ્રાન્ડ્સ બજારમાં અલગ પડે છે
\item
  \textbf{પ્રીમિયમ પોઝિશનિંગ}: નૈતિક ઉત્પાદનો ઊંચી કિંમતો મેળવી શકે છે
\item
  \textbf{ટકાઉ વૃદ્ધિ}: નૈતિક પ્રથાઓ દ્વારા લાંબા ગાળાની સફળતા
\end{itemize}

\textbf{સામાજિક અસર:}

\begin{itemize}
\tightlist
\item
  \textbf{સમુદાય વિકાસ}: સામાજિક કલ્યાણમાં યોગદાન
\item
  \textbf{પર્યાવરણ જવાબદારી}: ટકાઉ વ્યાપારિક પ્રથાઓ
\item
  \textbf{આર્થિક યોગદાન}: ન્યાયી વ્યાપારિક પ્રથાઓ આર્થિક વૃદ્ધિને સમર્થન આપે છે
\end{itemize}

\end{solutionbox}
\begin{mnemonicbox}
``નૈતિકતા વિશ્વાસ, અનુપાલન, લાભ, સામાજિક અસર બનાવે છે''

\end{mnemonicbox}
\subsection*{પ્રશ્ન 5(ક) OR [7
ગુણ]}\label{uxaaauxab0uxab6uxaa8-5uxa95-or-7-uxa97uxaa3}

\textbf{પ્રોજેક્ટ રિપોર્ટ લેખનના પગલાં અને ફોર્મેટ આપો}

\begin{solutionbox}

\textbf{પ્રોજેક્ટ રિપોર્ટ લેખનના પગલાં:}

\textbf{લેખન પૂર્વે તબક્કો:}

\begin{enumerate}
\tightlist
\item
  \textbf{પ્રોજેક્ટ આયોજન}: અવકાશ, ઉદ્દેશ્યો અને ડિલિવરેબલ્સ વ્યાખ્યાયિત કરવા
\item
  \textbf{ડેટા સંગ્રહ}: સંબંધિત માહિતી અને સંશોધન એકત્રિત કરવું
\item
  \textbf{વિશ્લેષણ}: એકત્રિત કરેલા ડેટાને પ્રક્રિયા અને વિશ્લેષણ કરવું
\item
  \textbf{માળખું આયોજન}: સામગ્રીને તાર્કિક રીતે ગોઠવવી
\end{enumerate}

\textbf{લેખન તબક્કો:} 5. \textbf{ડ્રાફ્ટ તૈયારી}: ફોર્મેટ અનુસાર પ્રારંભિક વર્ઝન
લખવું 6. \textbf{સામગ્રી વિકાસ}: વિગતો સાથે દરેક વિભાગનું વિસ્તરણ 7.
\textbf{સમીક્ષા અને સુધારણા}: ચોકસાઈ અને સંપૂર્ણતા માટે તપાસવું 8. \textbf{અંતિમ
ફોર્મેટિંગ}: સુસંગત ફોર્મેટિંગ અને શૈલી લાગુ કરવી

\textbf{પ્રોજેક્ટ રિપોર્ટ ફોર્મેટ:}

\begin{verbatim}
1. શીર્ષક પૃષ્ઠ (TITLE PAGE)
   - પ્રોજેક્ટ શીર્ષક
   - લેખક નામ(ઓ)
   - સંસ્થા/સંગઠન
   - સબમિશનની તારીખ

2. કાર્યકારી સારાંશ (EXECUTIVE SUMMARY)
   - પ્રોજેક્ટ ઝાંખી (1-2 પૃષ્ઠો)
   - મુખ્ય તારણો અને ભલામણો
   - અપેક્ષિત પરિણામો

3. સામગ્રી સૂચિ (TABLE OF CONTENTS)
   - અધ્યાય/વિભાગ શીર્ષકો
   - પૃષ્ઠ નંબરો
   - આકૃતિઓ અને કોષ્ટકોની સૂચિ

4. પરિચય (INTRODUCTION)
   - પૃષ્ઠભૂમિ અને સંદર્ભ
   - સમસ્યા નિવેદન
   - ઉદ્દેશ્યો અને અવકાશ
   - પદ્ધતિશાસ્ત્રની ઝાંખી

5. સાહિત્ય સમીક્ષા (LITERATURE REVIEW)
   - હાલના સંશોધન અને અભ્યાસો
   - ગેપ વિશ્લેષણ
   - સૈદ્ધાંતિક ફ્રેમવર્ક

6. પદ્ધતિશાસ્ત્ર (METHODOLOGY)
   - સંશોધન અભિગમ
   - ડેટા સંગ્રહ પદ્ધતિઓ
   - વિશ્લેષણ તકનીકો
   - મર્યાદાઓ

7. વિશ્લેષણ અને તારણો (ANALYSIS AND FINDINGS)
   - ડેટા પ્રસ્તુતિ
   - પરિણામો અને અર્થઘટન
   - મુખ્ય અંતર્દૃષ્ટિ

8. ભલામણો (RECOMMENDATIONS)
   - કાર્યવાહીલાયક સૂચનો
   - અમલીકરણ યોજના
   - અપેક્ષિત લાભો

9. નિષ્કર્ષ (CONCLUSION)
   - તારણોનો સારાંશ
   - ઉદ્દેશ્યોની સિદ્ધિ
   - ભાવિ અવકાશ

10. સંદર્ભો (REFERENCES)
    - ગ્રંથસૂચિ
    - ટાંકવામાં આવેલા સ્ત્રોતો
    - પરિશિષ્ટો (જો કોઈ હોય)
\end{verbatim}

\textbf{લેખન માર્ગદર્શિકા:}

\textbf{સામગ્રી ગુણવત્તા:}

\begin{itemize}
\tightlist
\item
  \textbf{સ્પષ્ટતા}: સરળ, સ્પષ્ટ ભાષાનો ઉપયોગ
\item
  \textbf{ચોકસાઈ}: તથ્યલક્ષી શુદ્ધતા સુનિશ્ચિત કરવી
\item
  \textbf{સંબંધિતતા}: માત્ર સુસંગત માહિતી સામેલ કરવી
\item
  \textbf{તાર્કિક પ્રવાહ}: સુસંગત માળખું જાળવવો
\end{itemize}

\textbf{ફોર્મેટિંગ ધોરણો:}

\begin{itemize}
\tightlist
\item
  \textbf{ફોન્ટ}: Times New Roman 12pt અથવા Arial 11pt
\item
  \textbf{સ્પેસિંગ}: 1.5 લાઇન સ્પેસિંગ
\item
  \textbf{માર્જિન}: બધી બાજુએ 1 ઇંચ
\item
  \textbf{પૃષ્ઠ નંબરિંગ}: આખા દસ્તાવેજમાં સુસંગત
\end{itemize}

\textbf{દૃશ્ય તત્વો:}

\begin{itemize}
\tightlist
\item
  \textbf{કોષ્ટકો}: ડેટા પ્રસ્તુતિ માટે
\item
  \textbf{ચાર્ટ/ગ્રાફ}: ટ્રેન્ડ વિશ્લેષણ માટે
\item
  \textbf{ડાયાગ્રામ}: પ્રક્રિયા સમજૂતી માટે
\item
  \textbf{છબીઓ}: કલ્પના સ્પષ્ટીકરણ માટે
\end{itemize}

\begin{center}
\textbf{Mermaid Diagram (Code)}
\begin{verbatim}
{Shaded}
{Highlighting}[]
graph TD
    A[Project Report] {-{-}{} B[Title Page]}
    A {-{-}{} C[Executive Summary]}
    A {-{-}{} D[Introduction]}
    A {-{-}{} E[Literature Review]}
    A {-{-}{} F[Methodology]}
    A {-{-}{} G[Analysis \& Findings]}
    A {-{-}{} H[Recommendations]}
    A {-{-}{} I[Conclusion]}
    A {-{-}{} J[References]}
    
    B {-{-}{} B1[Project Title{}br/{}Author Details{}br/{}Date]}
    C {-{-}{} C1[Overview{}br/{}Key Findings{}br/{}Outcomes]}
    D {-{-}{} D1[Background{}br/{}Problem{}br/{}Objectives]}
    E {-{-}{} E1[Research Review{}br/{}Gap Analysis{}br/{}Framework]}
    F {-{-}{} F1[Approach{}br/{}Data Collection{}br/{}Analysis Methods]}
    G {-{-}{} G1[Data Presentation{}br/{}Results{}br/{}Insights]}
    H {-{-}{} H1[Suggestions{}br/{}Implementation{}br/{}Benefits]}
    I {-{-}{} I1[Summary{}br/{}Achievement{}br/{}Future Scope]}
    J {-{-}{} J1[Bibliography{}br/{}Citations{}br/{}Appendices]}
{Highlighting}
{Shaded}
\end{verbatim}
\end{center}

\textbf{ગુણવત્તા ચેકલિસ્ટ:}

\begin{itemize}
\tightlist
\item
  \textbf{સંપૂર્ણતા}: બધા જરૂરી વિભાગો સામેલ
\item
  \textbf{સુસંગતતા}: આખા દસ્તાવેજમાં એકસમાન ફોર્મેટિંગ
\item
  \textbf{ચોકસાઈ}: તથ્યો અને આંકડાઓ ચકાસાયેલા
\item
  \textbf{સંબંધિતતા}: ઉદ્દેશ્યો સાથે સંરેખિત સામગ્રી
\end{itemize}

\textbf{સામાન્ય ભૂલો ટાળવી:}

\begin{itemize}
\tightlist
\item
  \textbf{સાહિત્યિક ચોરી}: હંમેશા સ્ત્રોતોને યોગ્ય રીતે ટાંકવા
\item
  \textbf{નબળું માળખું}: તાર્કિક પ્રવાહ જાળવવો
\item
  \textbf{અસુસંગત ફોર્મેટિંગ}: માનક માર્ગદર્શિકા અનુસરવી
\item
  \textbf{અપર્યાપ્ત વિશ્લેષણ}: પર્યાપ્ત ગહનતા પ્રદાન કરવી
\end{itemize}

\textbf{સમીક્ષા પ્રક્રિયા:}

\begin{enumerate}
\tightlist
\item
  \textbf{સ્વ-સમીક્ષા}: લેખક ભૂલો અને સંપૂર્ણતા તપાસે છે
\item
  \textbf{સાથીદાર સમીક્ષા}: સામગ્રી અને સ્પષ્ટતા પર સહકર્મીઓનો પ્રતિસાદ
\item
  \textbf{નિષ્ણાત સમીક્ષા}: વિષય વિશેષજ્ઞ માન્યતા
\item
  \textbf{અંતિમ પ્રૂફરીડિંગ}: વ્યાકરણ અને ફોર્મેટિંગ તપાસ
\end{enumerate}

\end{solutionbox}
\begin{mnemonicbox}
``શીર્ષક કાર્યકારી પરિચય સાહિત્ય પદ્ધતિ વિશ્લેષણ ભલામણો
નિષ્કર્ષ સંદર્ભો''

\end{mnemonicbox}

\end{document}
