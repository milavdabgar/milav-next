\documentclass[10pt,a4paper]{article}

% content/resources/templates/preamble.tex
\usepackage[margin=0.6in]{geometry}
\author{Milav Dabgar}
\usepackage{amsmath,amssymb,amsthm}
\usepackage{booktabs}
\usepackage{multirow}
\usepackage{xcolor}
\usepackage{tcolorbox}
\tcbuselibrary{breakable,skins}
\usepackage[colorlinks=true,linkcolor=blue]{hyperref}
\usepackage{titlesec}
\usepackage{enumitem}
\usepackage{tikz}
\usepackage{pgfplots}
\usepackage{circuitikz}
\usepackage[version=4]{mhchem}
\usepackage{longtable}
\usepackage{array}
\usepackage{float}
\usepackage{caption}
\usepackage{listings}

\lstset{
  basicstyle=\small\ttfamily,
  breaklines=true,
  breakatwhitespace=false,
  postbreak=\mbox{\textcolor{red}{$\hookrightarrow$}\space},
  float=false,
  numbers=left,
  numberstyle=\tiny\color{gray},
  numbersep=10pt,
  xleftmargin=2em,
  keywordstyle=\color{blue},
  commentstyle=\color{green!60!black},
  stringstyle=\color{purple},
  backgroundcolor=\color{gray!5},
  showstringspaces=false,
  tabsize=2,
  captionpos=b,
  keepspaces=true,
  columns=flexible
}

\pgfplotsset{compat=1.18}
\usetikzlibrary{shapes,arrows,positioning,calc,patterns,decorations.pathmorphing,decorations.markings,arrows.meta}

% Color scheme
\definecolor{headcolor}{RGB}{0,102,204}
\definecolor{keycolor}{RGB}{220,20,60}
\definecolor{solutioncolor}{RGB}{34,139,34}
\definecolor{mnemoniccolor}{RGB}{148,0,211}
\definecolor{codecolor}{RGB}{0,0,100}

% Spacing
\setlength{\parskip}{3pt}
\setlist[itemize]{nosep}
\setlist[enumerate]{nosep}

% Title formatting
\titleformat{\section}{\Large\bfseries\color{headcolor}}{\thesection}{1em}{}
\titleformat{\subsection}{\large\bfseries\color{headcolor}}{\thesubsection}{1em}{}

% Pandoc tightlist compatibility
\providecommand{\tightlist}{%
  \setlength{\itemsep}{0pt}\setlength{\parskip}{0pt}}

% Pandoc longtable compatibility
\newcounter{none}
\def\thenone{}


% content/resources/templates/english-boxes.tex
% This file is currently empty - it exists to maintain consistency with the import structure.
% Add custom environments here if needed in the future.


\begin{document}

\begin{center}
{\Huge\bfseries\color{headcolor} Subject Name Solutions}\\[5pt]
{\LARGE 4300021 -- Winter 2023}\\[3pt]
{\large Semester 1 Study Material}\\[3pt]
{\normalsize\textit{Detailed Solutions and Explanations}}
\end{center}

\vspace{10pt}

\subsection*{Question 1(a) [3 marks]}\label{q1a}

\textbf{Give Comparison between Entrepreneurship and Intrapreneurship.}

\begin{solutionbox}

\begin{longtable}[]{@{}
  >{\raggedright\arraybackslash}p{(\linewidth - 4\tabcolsep) * \real{0.2222}}
  >{\raggedright\arraybackslash}p{(\linewidth - 4\tabcolsep) * \real{0.3889}}
  >{\raggedright\arraybackslash}p{(\linewidth - 4\tabcolsep) * \real{0.3889}}@{}}
\toprule\noalign{}
\begin{minipage}[b]{\linewidth}\raggedright
\textbf{Aspect}
\end{minipage} & \begin{minipage}[b]{\linewidth}\raggedright
\textbf{Entrepreneurship}
\end{minipage} & \begin{minipage}[b]{\linewidth}\raggedright
\textbf{Intrapreneurship}
\end{minipage} \\
\midrule\noalign{}
\endhead
\bottomrule\noalign{}
\endlastfoot
\textbf{Definition} & Starting own business with personal risk &
Innovation within existing organization \\
\textbf{Risk} & Personal financial risk & Organization bears risk \\
\textbf{Resources} & Own/borrowed resources & Company provides
resources \\
\end{longtable}

\end{solutionbox}
\begin{mnemonicbox}
``EXternal vs INternal innovation''

\end{mnemonicbox}
\subsection*{Question 1(b) [4 marks]}\label{q1b}

\textbf{Discuss characteristics and functions of Entrepreneurship}

\begin{solutionbox}

\textbf{Characteristics:}

\begin{itemize}
\tightlist
\item
  \textbf{Risk-taking ability}: Willingness to take calculated business
  risks
\item
  \textbf{Innovation}: Creating new products, services, or processes
\item
  \textbf{Leadership skills}: Ability to guide and motivate teams
\end{itemize}

\textbf{Functions:}

\begin{itemize}
\tightlist
\item
  \textbf{Job Creation}: Generates employment opportunities for society
\item
  \textbf{Economic Development}: Contributes to GDP and national growth
\item
  \textbf{Innovation catalyst}: Introduces new technologies and
  solutions
\end{itemize}

\end{solutionbox}
\begin{mnemonicbox}
``RIL Creates Jobs Economically \& Innovatively''

\end{mnemonicbox}
\subsection*{Question 1(c) [7 marks]}\label{q1c}

\textbf{Identify and discuss 7-M Resources in detail.}

\begin{solutionbox}

\begin{longtable}[]{@{}
  >{\raggedright\arraybackslash}p{(\linewidth - 4\tabcolsep) * \real{0.2826}}
  >{\raggedright\arraybackslash}p{(\linewidth - 4\tabcolsep) * \real{0.3696}}
  >{\raggedright\arraybackslash}p{(\linewidth - 4\tabcolsep) * \real{0.3478}}@{}}
\toprule\noalign{}
\begin{minipage}[b]{\linewidth}\raggedright
\textbf{Resource}
\end{minipage} & \begin{minipage}[b]{\linewidth}\raggedright
\textbf{Description}
\end{minipage} & \begin{minipage}[b]{\linewidth}\raggedright
\textbf{Importance}
\end{minipage} \\
\midrule\noalign{}
\endhead
\bottomrule\noalign{}
\endlastfoot
\textbf{Man} & Human resources and workforce & Core asset for
operations \\
\textbf{Money} & Financial capital and funding & Essential for business
operations \\
\textbf{Material} & Raw materials and supplies & Production
requirements \\
\textbf{Machine} & Equipment and technology & Operational efficiency \\
\textbf{Method} & Processes and procedures & Systematic approach \\
\textbf{Market} & Customer base and demand & Revenue generation \\
\textbf{Management} & Planning and coordination & Overall business
control \\
\end{longtable}

\begin{verbatim}
mindmap
  root((7{-M Resources))}
    Man
      Skills
      Experience
    Money
      Capital
      Investment
    Material
      Raw Materials
      Inventory
    Machine
      Equipment
      Technology
    Method
      Processes
      Systems
    Market
      Customers
      Demand
    Management
      Planning
      Control
\end{verbatim}

\end{solutionbox}
\begin{mnemonicbox}
``Many Modern Managers Make Money Managing Markets''

\end{mnemonicbox}
\subsection*{Question 1(c) OR [7
marks]}\label{q1c}

\textbf{Write down the Start Up India Registration process.}

\begin{solutionbox}

\textbf{Start-up India Registration Steps:}

\begin{enumerate}
\tightlist
\item
  \textbf{Online Registration}: Visit www.startupindia.gov.in
\item
  \textbf{Document Preparation}:

  \begin{itemize}
  \tightlist
  \item
    Certificate of Incorporation
  \item
    PAN Card of entity
  \item
    Brief description of business
  \end{itemize}
\item
  \textbf{Eligibility Criteria}:

  \begin{itemize}
  \tightlist
  \item
    Entity age less than 10 years
  \item
    Annual turnover less than ₹100 crore
  \item
    Working towards innovation/improvement
  \end{itemize}
\item
  \textbf{Application Submission}: Complete online form with required
  documents
\item
  \textbf{Verification Process}: Government review and approval
\item
  \textbf{Certificate Issuance}: Receive recognition certificate
\end{enumerate}

\textbf{Benefits:}

\begin{itemize}
\tightlist
\item
  \textbf{Tax exemptions} for 3 consecutive years
\item
  \textbf{Fast-track patent} application process
\item
  \textbf{Compliance reduction} under labor and environment laws
\end{itemize}

\end{solutionbox}
\begin{mnemonicbox}
``Online Documents Eligibility Application
Verification Certificate Benefits''

\end{mnemonicbox}
\subsection*{Question 2(a) [3 marks]}\label{q2a}

\textbf{List Methods of Market Research.}

\begin{solutionbox}

\textbf{Primary Research Methods:}

\begin{itemize}
\tightlist
\item
  \textbf{Surveys}: Questionnaires to collect customer data
\item
  \textbf{Interviews}: Direct interaction with target audience
\item
  \textbf{Focus Groups}: Group discussions for feedback
\end{itemize}

\textbf{Secondary Research Methods:}

\begin{itemize}
\tightlist
\item
  \textbf{Online Research}: Internet-based data collection
\item
  \textbf{Published Reports}: Industry analysis and studies
\item
  \textbf{Government Data}: Statistical information from official
  sources
\end{itemize}

\end{solutionbox}
\begin{mnemonicbox}
``Survey Interview Focus Online Published
Government''

\end{mnemonicbox}
\subsection*{Question 2(b) [4 marks]}\label{q2b}

\textbf{Draw and Explain Product Life Cycle.}

\begin{solutionbox}

\begin{center}
\textbf{Mermaid Diagram (Code)}
\begin{verbatim}
{Shaded}
{Highlighting}[]
graph LR
    A[Introduction] {-{-}{} B[Growth]}
    B {-{-}{} C[Maturity]}
    C {-{-}{} D[Decline]}
    
    A {-{-}{} A1[Low Sales{}br/{}High Costs]}
    B {-{-}{} B1[Rising Sales{}br/{}Profits Increase]}
    C {-{-}{} C1[Peak Sales{}br/{}Market Saturation]}
    D {-{-}{} D1[Declining Sales{}br/{}Phase Out]}
{Highlighting}
{Shaded}
\end{verbatim}
\end{center}

\textbf{Stages:}

\begin{itemize}
\tightlist
\item
  \textbf{Introduction}: Product launch with high marketing costs
\item
  \textbf{Growth}: Rapid sales increase and market acceptance
\item
  \textbf{Maturity}: Peak sales with intense competition
\item
  \textbf{Decline}: Decreasing demand and eventual phase-out
\end{itemize}

\end{solutionbox}
\begin{mnemonicbox}
``I Grow My Dreams''

\end{mnemonicbox}
\subsection*{Question 2(c) [7 marks]}\label{q2c}

\textbf{Identify and discuss 4 P's of Marketing.}

\begin{solutionbox}

\begin{longtable}[]{@{}
  >{\raggedright\arraybackslash}p{(\linewidth - 6\tabcolsep) * \real{0.1186}}
  >{\raggedright\arraybackslash}p{(\linewidth - 6\tabcolsep) * \real{0.2203}}
  >{\raggedright\arraybackslash}p{(\linewidth - 6\tabcolsep) * \real{0.2881}}
  >{\raggedright\arraybackslash}p{(\linewidth - 6\tabcolsep) * \real{0.3729}}@{}}
\toprule\noalign{}
\begin{minipage}[b]{\linewidth}\raggedright
\textbf{P}
\end{minipage} & \begin{minipage}[b]{\linewidth}\raggedright
\textbf{Element}
\end{minipage} & \begin{minipage}[b]{\linewidth}\raggedright
\textbf{Description}
\end{minipage} & \begin{minipage}[b]{\linewidth}\raggedright
\textbf{Key Considerations}
\end{minipage} \\
\midrule\noalign{}
\endhead
\bottomrule\noalign{}
\endlastfoot
\textbf{Product} & Goods/Services offered & Features, quality, branding
& Customer needs satisfaction \\
\textbf{Price} & Cost to customer & Pricing strategy, discounts &
Competitive positioning \\
\textbf{Place} & Distribution channels & Where product is sold &
Accessibility to customers \\
\textbf{Promotion} & Marketing communication & Advertising, sales
promotion & Brand awareness creation \\
\end{longtable}

\begin{center}
\textbf{Mermaid Diagram (Code)}
\begin{verbatim}
{Shaded}
{Highlighting}[]
graph TD
    A[Marketing Mix] {-{-}{} B[Product]}
    A {-{-}{} C[Price]}
    A {-{-}{} D[Place]}
    A {-{-}{} E[Promotion]}
    
    B {-{-}{} B1[Features{}br/{}Quality{}br/{}Branding]}
    C {-{-}{} C1[Strategy{}br/{}Discounts{}br/{}Value]}
    D {-{-}{} D1[Channels{}br/{}Location{}br/{}Access]}
    E {-{-}{} E1[Advertising{}br/{}PR{}br/{}Sales]}
{Highlighting}
{Shaded}
\end{verbatim}
\end{center}

\textbf{Integration:} All 4 P's must work together for effective
marketing strategy.

\end{solutionbox}
\begin{mnemonicbox}
``People Purchase Products Properly''

\end{mnemonicbox}
\subsection*{Question 2(a) OR [3
marks]}\label{q2a}

\textbf{Discuss B2B, E-commerce and GeM.}

\begin{solutionbox}

\begin{longtable}[]{@{}lll@{}}
\toprule\noalign{}
\textbf{Type} & \textbf{Full Form} & \textbf{Description} \\
\midrule\noalign{}
\endhead
\bottomrule\noalign{}
\endlastfoot
\textbf{B2B} & Business to Business & Trade between companies \\
\textbf{E-commerce} & Electronic Commerce & Online buying and selling \\
\textbf{GeM} & Government e-Marketplace & Government procurement
portal \\
\end{longtable}

\textbf{Key Features:}

\begin{itemize}
\tightlist
\item
  \textbf{B2B}: Bulk transactions, long-term relationships
\item
  \textbf{E-commerce}: Digital platforms, global reach
\item
  \textbf{GeM}: Transparent government purchases, competitive pricing
\end{itemize}

\end{solutionbox}
\begin{mnemonicbox}
``Businesses Buy Electronically, Government
e-Markets''

\end{mnemonicbox}
\subsection*{Question 2(b) OR [4
marks]}\label{q2b}

\textbf{Write a note on the plans for creating and starting the
business}

\begin{solutionbox}

\textbf{Business Creation Plans:}

\textbf{Market Analysis:}

\begin{itemize}
\tightlist
\item
  \textbf{Target customers}: Identify primary audience
\item
  \textbf{Competition study}: Analyze existing players
\item
  \textbf{Market size}: Determine potential revenue
\end{itemize}

\textbf{Financial Planning:}

\begin{itemize}
\tightlist
\item
  \textbf{Capital requirements}: Initial investment needed
\item
  \textbf{Revenue projections}: Expected income streams
\item
  \textbf{Break-even analysis}: Profitability timeline
\end{itemize}

\textbf{Operational Setup:}

\begin{itemize}
\tightlist
\item
  \textbf{Location selection}: Strategic positioning
\item
  \textbf{Resource allocation}: Human and material resources
\item
  \textbf{Legal compliance}: Licenses and registrations
\end{itemize}

\end{solutionbox}
\begin{mnemonicbox}
``Market Finance Operations = Business Success''

\end{mnemonicbox}
\subsection*{Question 2(c) OR [7
marks]}\label{q2c}

\textbf{Explain the concept of Risk and SWOT analysis.}

\begin{solutionbox}

\textbf{Risk Concept:} Risk is uncertainty that can affect business
outcomes, both positively and negatively.

\textbf{Types of Business Risks:}

\begin{itemize}
\tightlist
\item
  \textbf{Financial Risk}: Cash flow and funding issues
\item
  \textbf{Market Risk}: Demand fluctuations and competition
\item
  \textbf{Operational Risk}: Production and service delivery problems
\end{itemize}

\textbf{SWOT Analysis:}

\begin{longtable}[]{@{}ll@{}}
\toprule\noalign{}
\textbf{Internal Factors} & \textbf{External Factors} \\
\midrule\noalign{}
\endhead
\bottomrule\noalign{}
\endlastfoot
\textbf{Strengths} & \textbf{Opportunities} \\
- Core competencies & - Market growth \\
- Unique resources & - New technologies \\
\textbf{Weaknesses} & \textbf{Threats} \\
- Skill gaps & - Competition \\
- Resource limitations & - Economic changes \\
\end{longtable}

\begin{center}
\textbf{Mermaid Diagram (Code)}
\begin{verbatim}
{Shaded}
{Highlighting}[]
graph TD
    A[SWOT Analysis] {-{-}{} B[Strengths]}
    A {-{-}{} C[Weaknesses]}
    A {-{-}{} D[Opportunities]}
    A {-{-}{} E[Threats]}
    
    B {-{-}{} B1[Internal Positive]}
    C {-{-}{} C1[Internal Negative]}
    D {-{-}{} D1[External Positive]}
    E {-{-}{} E1[External Negative]}
{Highlighting}
{Shaded}
\end{verbatim}
\end{center}

\textbf{Risk Mitigation Strategies:}

\begin{itemize}
\tightlist
\item
  \textbf{Diversification}: Spread risks across different areas
\item
  \textbf{Insurance}: Transfer risk to insurance companies
\item
  \textbf{Contingency planning}: Prepare for unexpected situations
\end{itemize}

\end{solutionbox}
\begin{mnemonicbox}
``Strong Weak Opportunities Threaten = SWOT''

\end{mnemonicbox}
\subsection*{Question 3(a) [3 marks]}\label{q3a}

\textbf{Write short note on cooperative type organization.}

\begin{solutionbox}

\textbf{Cooperative Organization:}

\begin{itemize}
\tightlist
\item
  \textbf{Definition}: Voluntary association of people for mutual
  benefit
\item
  \textbf{Ownership}: Collectively owned by members
\item
  \textbf{Control}: Democratic management with equal voting rights
\end{itemize}

\textbf{Characteristics:}

\begin{itemize}
\tightlist
\item
  \textbf{Member participation}: Active involvement in decision-making
\item
  \textbf{Profit sharing}: Benefits distributed among members
\item
  \textbf{Social purpose}: Focus on community welfare
\end{itemize}

\textbf{Examples:} Agricultural cooperatives, credit unions, housing
societies

\end{solutionbox}
\begin{mnemonicbox}
``Collective Ownership with Democratic Management''

\end{mnemonicbox}
\subsection*{Question 3(b) [4 marks]}\label{q3b}

\textbf{Give a list of functions of management and define all of them.}

\begin{solutionbox}

\begin{longtable}[]{@{}
  >{\raggedright\arraybackslash}p{(\linewidth - 4\tabcolsep) * \real{0.2708}}
  >{\raggedright\arraybackslash}p{(\linewidth - 4\tabcolsep) * \real{0.3333}}
  >{\raggedright\arraybackslash}p{(\linewidth - 4\tabcolsep) * \real{0.3958}}@{}}
\toprule\noalign{}
\begin{minipage}[b]{\linewidth}\raggedright
\textbf{Function}
\end{minipage} & \begin{minipage}[b]{\linewidth}\raggedright
\textbf{Definition}
\end{minipage} & \begin{minipage}[b]{\linewidth}\raggedright
\textbf{Key Activities}
\end{minipage} \\
\midrule\noalign{}
\endhead
\bottomrule\noalign{}
\endlastfoot
\textbf{Planning} & Setting objectives and strategies & Goal setting,
forecasting, budgeting \\
\textbf{Organizing} & Arranging resources and structure &
Departmentation, delegation, coordination \\
\textbf{Staffing} & Human resource management & Recruitment, training,
performance evaluation \\
\textbf{Directing} & Leading and motivating employees & Communication,
leadership, supervision \\
\textbf{Controlling} & Monitoring and correcting performance &
Performance measurement, feedback, correction \\
\end{longtable}

\end{solutionbox}
\begin{mnemonicbox}
``Proper Organization Supports Directed Control''

\end{mnemonicbox}
\subsection*{Question 3(c) [7 marks]}\label{q3c}

\textbf{Describe types of Ownership and explain any three in detail.}

\begin{solutionbox}

\textbf{Types of Business Ownership:}

\begin{longtable}[]{@{}llll@{}}
\toprule\noalign{}
\textbf{Type} & \textbf{Ownership} & \textbf{Liability} &
\textbf{Control} \\
\midrule\noalign{}
\endhead
\bottomrule\noalign{}
\endlastfoot
\textbf{Sole Proprietorship} & Single owner & Unlimited & Complete \\
\textbf{Partnership} & 2+ partners & Unlimited & Shared \\
\textbf{Company} & Shareholders & Limited & Board of Directors \\
\textbf{Cooperative} & Members & Limited & Democratic \\
\end{longtable}

\textbf{Detailed Explanation:}

\textbf{1. Sole Proprietorship:}

\begin{itemize}
\tightlist
\item
  \textbf{Advantages}: Easy formation, complete control, tax benefits
\item
  \textbf{Disadvantages}: Unlimited liability, limited resources,
  business continuity issues
\item
  \textbf{Suitable for}: Small businesses, professional services
\end{itemize}

\textbf{2. Partnership:}

\begin{itemize}
\tightlist
\item
  \textbf{Advantages}: Shared resources, specialized skills, easy
  formation
\item
  \textbf{Disadvantages}: Unlimited liability, conflict potential,
  shared profits
\item
  \textbf{Types}: General partnership, limited partnership
\end{itemize}

\textbf{3. Company:}

\begin{itemize}
\tightlist
\item
  \textbf{Advantages}: Limited liability, perpetual existence, easier
  capital raising
\item
  \textbf{Disadvantages}: Complex regulations, double taxation, loss of
  control
\item
  \textbf{Types}: Private limited, public limited
\end{itemize}

\end{solutionbox}
\begin{mnemonicbox}
``Single Partners Companies Cooperate''

\end{mnemonicbox}
\subsection*{Question 3(a) OR [3
marks]}\label{q3a}

\textbf{Explain different Leadership Models.}

\begin{solutionbox}

\textbf{Leadership Models:}

\begin{longtable}[]{@{}
  >{\raggedright\arraybackslash}p{(\linewidth - 4\tabcolsep) * \real{0.2500}}
  >{\raggedright\arraybackslash}p{(\linewidth - 4\tabcolsep) * \real{0.3182}}
  >{\raggedright\arraybackslash}p{(\linewidth - 4\tabcolsep) * \real{0.4318}}@{}}
\toprule\noalign{}
\begin{minipage}[b]{\linewidth}\raggedright
\textbf{Model}
\end{minipage} & \begin{minipage}[b]{\linewidth}\raggedright
\textbf{Approach}
\end{minipage} & \begin{minipage}[b]{\linewidth}\raggedright
\textbf{Best Used When}
\end{minipage} \\
\midrule\noalign{}
\endhead
\bottomrule\noalign{}
\endlastfoot
\textbf{Autocratic} & Leader makes all decisions & Crisis situations,
quick decisions needed \\
\textbf{Democratic} & Participative decision-making & Team input
valuable, time available \\
\textbf{Laissez-faire} & Hands-off approach & Experienced team, creative
work \\
\end{longtable}

\textbf{Modern Models:}

\begin{itemize}
\tightlist
\item
  \textbf{Transformational}: Inspiring vision and change
\item
  \textbf{Transactional}: Reward-punishment based
\item
  \textbf{Situational}: Adapts style to situation
\end{itemize}

\end{solutionbox}
\begin{mnemonicbox}
``Auto Demo Laissez Transform Transact Situate''

\end{mnemonicbox}
\subsection*{Question 3(b) OR [4
marks]}\label{q3b}

\textbf{Give the difference between Administration and Management}

\begin{solutionbox}

\begin{longtable}[]{@{}lll@{}}
\toprule\noalign{}
\textbf{Aspect} & \textbf{Administration} & \textbf{Management} \\
\midrule\noalign{}
\endhead
\bottomrule\noalign{}
\endlastfoot
\textbf{Focus} & Policy formulation & Policy implementation \\
\textbf{Level} & Top level function & Middle level function \\
\textbf{Nature} & Planning and thinking & Doing and executing \\
\textbf{Scope} & Broader organizational & Specific departmental \\
\end{longtable}

\textbf{Key Differences:}

\begin{itemize}
\tightlist
\item
  \textbf{Administration}: Strategic, long-term, conceptual
\item
  \textbf{Management}: Operational, short-term, practical
\end{itemize}

\textbf{Relationship:} Administration sets direction, Management
executes plans

\end{solutionbox}
\begin{mnemonicbox}
``Admin Plans, Management Implements''

\end{mnemonicbox}
\subsection*{Question 3(c) OR [7
marks]}\label{q3c}

\textbf{Explain the concept of difference between industry, commerce and
business.}

\begin{solutionbox}

\begin{longtable}[]{@{}
  >{\raggedright\arraybackslash}p{(\linewidth - 6\tabcolsep) * \real{0.2031}}
  >{\raggedright\arraybackslash}p{(\linewidth - 6\tabcolsep) * \real{0.2500}}
  >{\raggedright\arraybackslash}p{(\linewidth - 6\tabcolsep) * \real{0.3281}}
  >{\raggedright\arraybackslash}p{(\linewidth - 6\tabcolsep) * \real{0.2188}}@{}}
\toprule\noalign{}
\begin{minipage}[b]{\linewidth}\raggedright
\textbf{Concept}
\end{minipage} & \begin{minipage}[b]{\linewidth}\raggedright
\textbf{Definition}
\end{minipage} & \begin{minipage}[b]{\linewidth}\raggedright
\textbf{Primary Activity}
\end{minipage} & \begin{minipage}[b]{\linewidth}\raggedright
\textbf{Examples}
\end{minipage} \\
\midrule\noalign{}
\endhead
\bottomrule\noalign{}
\endlastfoot
\textbf{Industry} & Production of goods & Manufacturing, processing &
Steel, textiles, chemicals \\
\textbf{Commerce} & Distribution of goods & Trading, transportation &
Wholesale, retail, logistics \\
\textbf{Business} & Overall economic activity & Production +
distribution & Complete enterprise operations \\
\end{longtable}

\textbf{Industry Categories:}

\begin{itemize}
\tightlist
\item
  \textbf{Primary}: Raw material extraction (mining, agriculture)
\item
  \textbf{Secondary}: Manufacturing and processing
\item
  \textbf{Tertiary}: Services (banking, education, healthcare)
\end{itemize}

\textbf{Commerce Functions:}

\begin{itemize}
\tightlist
\item
  \textbf{Trade}: Buying and selling activities
\item
  \textbf{Auxiliaries}: Supporting services (transport, insurance,
  banking)
\end{itemize}

\textbf{Business Integration:}

\begin{itemize}
\tightlist
\item
  \textbf{Vertical}: Industry + Commerce integration
\item
  \textbf{Horizontal}: Same level diversification
\end{itemize}

\begin{center}
\textbf{Mermaid Diagram (Code)}
\begin{verbatim}
{Shaded}
{Highlighting}[]
graph LR
    A[Business] {-{-}{} B[Industry]}
    A {-{-}{} C[Commerce]}
    
    B {-{-}{} B1[Primary{}br/{}Secondary{}br/{}Tertiary]}
    C {-{-}{} C1[Trade{}br/{}Auxiliaries]}
    
    B1 {-{-}{} B2[Raw Materials{}br/{}Manufacturing{}br/{}Services]}
    C1 {-{-}{} C2[Buy/Sell{}br/{}Support Services]}
{Highlighting}
{Shaded}
\end{verbatim}
\end{center}

\end{solutionbox}
\begin{mnemonicbox}
``Industry Creates, Commerce Distributes, Business
Integrates''

\end{mnemonicbox}
\subsection*{Question 4(a) [3 marks]}\label{q4a}

\textbf{Explain following terms: 1.Contracts 2.Copyrights}

\begin{solutionbox}

\begin{longtable}[]{@{}
  >{\raggedright\arraybackslash}p{(\linewidth - 4\tabcolsep) * \real{0.2273}}
  >{\raggedright\arraybackslash}p{(\linewidth - 4\tabcolsep) * \real{0.3636}}
  >{\raggedright\arraybackslash}p{(\linewidth - 4\tabcolsep) * \real{0.4091}}@{}}
\toprule\noalign{}
\begin{minipage}[b]{\linewidth}\raggedright
\textbf{Term}
\end{minipage} & \begin{minipage}[b]{\linewidth}\raggedright
\textbf{Definition}
\end{minipage} & \begin{minipage}[b]{\linewidth}\raggedright
\textbf{Key Features}
\end{minipage} \\
\midrule\noalign{}
\endhead
\bottomrule\noalign{}
\endlastfoot
\textbf{Contracts} & Legal agreement between parties & Binding,
enforceable, mutual obligations \\
\textbf{Copyrights} & Intellectual property protection & Creative works,
exclusive rights, limited duration \\
\end{longtable}

\textbf{Contract Elements:}

\begin{itemize}
\tightlist
\item
  \textbf{Offer and acceptance}: Clear terms agreed upon
\item
  \textbf{Consideration}: Value exchange between parties
\item
  \textbf{Legal capacity}: Parties must be legally capable
\end{itemize}

\textbf{Copyright Protection:}

\begin{itemize}
\tightlist
\item
  \textbf{Duration}: Generally lifetime + 70 years
\item
  \textbf{Rights}: Reproduction, distribution, public performance
\item
  \textbf{Registration}: Not mandatory but recommended
\end{itemize}

\end{solutionbox}
\begin{mnemonicbox}
``Contracts Bind, Copyrights Protect''

\end{mnemonicbox}
\subsection*{Question 4(b) [4 marks]}\label{q4b}

\textbf{Give a note on startup incubation center and Modalities.}

\begin{solutionbox}

\textbf{Startup Incubation Centers:}

\begin{itemize}
\tightlist
\item
  \textbf{Purpose}: Support early-stage startups with resources and
  guidance
\item
  \textbf{Services}: Mentorship, funding, workspace, networking
\item
  \textbf{Duration}: Typically 6 months to 2 years
\end{itemize}

\textbf{Key Modalities:}

\textbf{Pre-incubation:}

\begin{itemize}
\tightlist
\item
  \textbf{Idea validation}: Market research and feasibility
\item
  \textbf{Team formation}: Building core team
\item
  \textbf{Prototype development}: MVP creation
\end{itemize}

\textbf{Incubation Phase:}

\begin{itemize}
\tightlist
\item
  \textbf{Business model refinement}: Revenue model development
\item
  \textbf{Market testing}: Customer validation
\item
  \textbf{Funding preparation}: Investor pitch preparation
\end{itemize}

\textbf{Post-incubation:}

\begin{itemize}
\tightlist
\item
  \textbf{Alumni network}: Continued support and connections
\item
  \textbf{Follow-up funding}: Series A preparation
\item
  \textbf{Scaling support}: Growth strategy assistance
\end{itemize}

\end{solutionbox}
\begin{mnemonicbox}
``Pre-incubate, Incubate, Post-support Startups''

\end{mnemonicbox}
\subsection*{Question 4(c) [7 marks]}\label{q4c}

\textbf{List State level agencies which supports start-ups and describe
their functionalities}

\begin{solutionbox}

\textbf{Gujarat State Support Agencies:}

\begin{longtable}[]{@{}
  >{\raggedright\arraybackslash}p{(\linewidth - 4\tabcolsep) * \real{0.2609}}
  >{\raggedright\arraybackslash}p{(\linewidth - 4\tabcolsep) * \real{0.3261}}
  >{\raggedright\arraybackslash}p{(\linewidth - 4\tabcolsep) * \real{0.4130}}@{}}
\toprule\noalign{}
\begin{minipage}[b]{\linewidth}\raggedright
\textbf{Agency}
\end{minipage} & \begin{minipage}[b]{\linewidth}\raggedright
\textbf{Full Form}
\end{minipage} & \begin{minipage}[b]{\linewidth}\raggedright
\textbf{Key Functions}
\end{minipage} \\
\midrule\noalign{}
\endhead
\bottomrule\noalign{}
\endlastfoot
\textbf{SSIP} & Student Startup \& Innovation Policy & Student
entrepreneur support, funding \\
\textbf{iHub Gujarat} & Innovation Hub Gujarat & Incubation, mentorship,
networking \\
\textbf{GUSEC} & Gujarat University Startup \& Entrepreneurship Council
& University-level startup promotion \\
\textbf{GIDC} & Gujarat Industrial Development Corporation & Industrial
infrastructure, land allocation \\
\end{longtable}

\textbf{Detailed Functionalities:}

\textbf{SSIP Gujarat:}

\begin{itemize}
\tightlist
\item
  \textbf{Funding support}: Up to ₹2 lakh for student startups
\item
  \textbf{Incubation facilities}: Workspace and equipment access
\item
  \textbf{Mentorship programs}: Industry expert guidance
\item
  \textbf{IPR support}: Patent filing assistance
\end{itemize}

\textbf{iHub Gujarat:}

\begin{itemize}
\tightlist
\item
  \textbf{Startup ecosystem}: Complete entrepreneurship support
\item
  \textbf{Technology transfer}: Research to market conversion
\item
  \textbf{Investor connections}: Funding facilitation
\item
  \textbf{Industry partnerships}: Corporate collaboration
\end{itemize}

\textbf{GUSEC:}

\begin{itemize}
\tightlist
\item
  \textbf{Student engagement}: Campus entrepreneurship programs
\item
  \textbf{Skill development}: Entrepreneurship education
\item
  \textbf{Competition organization}: Startup contests and pitches
\item
  \textbf{Network building}: Alumni entrepreneur connections
\end{itemize}

\begin{center}
\textbf{Mermaid Diagram (Code)}
\begin{verbatim}
{Shaded}
{Highlighting}[]
graph TD
    A[State Startup Support] {-{-}{} B[SSIP]}
    A {-{-}{} C[iHub Gujarat]}
    A {-{-}{} D[GUSEC]}
    A {-{-}{} E[GIDC]}
    
    B {-{-}{} B1[Student Focus{}br/{}Funding{}br/{}Incubation]}
    C {-{-}{} C1[Complete Ecosystem{}br/{}Tech Transfer{}br/{}Investors]}
    D {-{-}{} D1[University Level{}br/{}Skills{}br/{}Competitions]}
    E {-{-}{} E1[Infrastructure{}br/{}Industrial Support]}
{Highlighting}
{Shaded}
\end{verbatim}
\end{center}

\textbf{Impact Measurement:}

\begin{itemize}
\tightlist
\item
  \textbf{Number of startups supported} annually
\item
  \textbf{Job creation} through supported ventures
\item
  \textbf{Revenue generation} of incubated companies
\item
  \textbf{Success rate} of graduated startups
\end{itemize}

\end{solutionbox}
\begin{mnemonicbox}
``SSIP iHub GUSEC GIDC Support Gujarat Startups''

\end{mnemonicbox}
\subsection*{Question 4(a) OR [3
marks]}\label{q4a}

\textbf{Explain following terms: 1.IPR 2.Trademarks}

\begin{solutionbox}

\begin{longtable}[]{@{}
  >{\raggedright\arraybackslash}p{(\linewidth - 4\tabcolsep) * \real{0.1786}}
  >{\raggedright\arraybackslash}p{(\linewidth - 4\tabcolsep) * \real{0.4464}}
  >{\raggedright\arraybackslash}p{(\linewidth - 4\tabcolsep) * \real{0.3750}}@{}}
\toprule\noalign{}
\begin{minipage}[b]{\linewidth}\raggedright
\textbf{Term}
\end{minipage} & \begin{minipage}[b]{\linewidth}\raggedright
\textbf{Full Form/Definition}
\end{minipage} & \begin{minipage}[b]{\linewidth}\raggedright
\textbf{Protection Scope}
\end{minipage} \\
\midrule\noalign{}
\endhead
\bottomrule\noalign{}
\endlastfoot
\textbf{IPR} & Intellectual Property Rights & Ideas, inventions,
creative works \\
\textbf{Trademarks} & Brand identification marks & Names, logos,
symbols, slogans \\
\end{longtable}

\textbf{IPR Categories:}

\begin{itemize}
\tightlist
\item
  \textbf{Patents}: Technical inventions (20 years)
\item
  \textbf{Copyrights}: Creative expressions (lifetime + 70 years)
\item
  \textbf{Trademarks}: Brand identifiers (10 years, renewable)
\end{itemize}

\textbf{Trademark Features:}

\begin{itemize}
\tightlist
\item
  \textbf{Distinctiveness}: Unique brand identification
\item
  \textbf{Commercial use}: Business identification purpose
\item
  \textbf{Registration}: Legal protection through registration
\end{itemize}

\end{solutionbox}
\begin{mnemonicbox}
``IPR Protects, Trademarks Identify''

\end{mnemonicbox}
\subsection*{Question 4(b) OR [4
marks]}\label{q4b}

\textbf{Define the role of Investor in start-up.}

\begin{solutionbox}

\textbf{Investor Roles in Startups:}

\textbf{Financial Support:}

\begin{itemize}
\tightlist
\item
  \textbf{Seed funding}: Initial capital for business launch
\item
  \textbf{Growth capital}: Expansion and scaling funds
\item
  \textbf{Bridge financing}: Interim funding between rounds
\end{itemize}

\textbf{Strategic Guidance:}

\begin{itemize}
\tightlist
\item
  \textbf{Business mentorship}: Industry experience sharing
\item
  \textbf{Network access}: Connections to customers and partners
\item
  \textbf{Market insights}: Industry knowledge and trends
\end{itemize}

\textbf{Operational Support:}

\begin{itemize}
\tightlist
\item
  \textbf{Team building}: Hiring and talent acquisition advice
\item
  \textbf{Technology guidance}: Technical architecture suggestions
\item
  \textbf{Legal compliance}: Regulatory and compliance support
\end{itemize}

\textbf{Risk Management:}

\begin{itemize}
\tightlist
\item
  \textbf{Due diligence}: Business model validation
\item
  \textbf{Performance monitoring}: Regular progress tracking
\item
  \textbf{Exit strategy}: Planning for investment recovery
\end{itemize}

\textbf{Types of Investors:}

\begin{itemize}
\tightlist
\item
  \textbf{Angel investors}: Individual high-net-worth investors
\item
  \textbf{Venture capital}: Professional investment firms
\item
  \textbf{Corporate investors}: Strategic industry players
\end{itemize}

\end{solutionbox}
\begin{mnemonicbox}
``Finance Strategy Operations Risk = Investor Roles''

\end{mnemonicbox}
\subsection*{Question 4(c) OR [7
marks]}\label{q4c}

\textbf{List National level agencies which support start-ups and
describe their functionalities.}

\begin{solutionbox}

\textbf{National Startup Support Agencies:}

\begin{longtable}[]{@{}
  >{\raggedright\arraybackslash}p{(\linewidth - 4\tabcolsep) * \real{0.2182}}
  >{\raggedright\arraybackslash}p{(\linewidth - 4\tabcolsep) * \real{0.4364}}
  >{\raggedright\arraybackslash}p{(\linewidth - 4\tabcolsep) * \real{0.3455}}@{}}
\toprule\noalign{}
\begin{minipage}[b]{\linewidth}\raggedright
\textbf{Agency}
\end{minipage} & \begin{minipage}[b]{\linewidth}\raggedright
\textbf{Ministry/Department}
\end{minipage} & \begin{minipage}[b]{\linewidth}\raggedright
\textbf{Primary Focus}
\end{minipage} \\
\midrule\noalign{}
\endhead
\bottomrule\noalign{}
\endlastfoot
\textbf{Startup India} & DPIIT, Commerce Ministry & Policy framework and
ecosystem \\
\textbf{BIRAC} & Department of Biotechnology & Biotechnology
innovation \\
\textbf{TDB} & Department of Science \& Technology & Technology
development \\
\textbf{SIDBI} & Financial Services & MSME and startup funding \\
\end{longtable}

\textbf{Detailed Functionalities:}

\textbf{Startup India:}

\begin{itemize}
\tightlist
\item
  \textbf{Policy formulation}: National startup policy framework
\item
  \textbf{Recognition program}: Official startup certification
\item
  \textbf{Tax benefits}: 3-year tax exemption for eligible startups
\item
  \textbf{Regulatory support}: Single-point clearance system
\item
  \textbf{Funding facilitation}: Fund of Funds scheme (₹10,000 crores)
\end{itemize}

\textbf{BIRAC (Biotechnology Industry Research Assistance Council):}

\begin{itemize}
\tightlist
\item
  \textbf{Biotech innovation}: Supporting biotech startups and research
\item
  \textbf{Funding schemes}: SBIRI, SPARSH, BIG programs
\item
  \textbf{Industry partnerships}: Academia-industry collaboration
\item
  \textbf{Technology translation}: Research to market conversion
\end{itemize}

\textbf{TDB (Technology Development Board):}

\begin{itemize}
\tightlist
\item
  \textbf{Technology commercialization}: Converting research to products
\item
  \textbf{Financial assistance}: Loans and grants for technology
  development
\item
  \textbf{Industry support}: Manufacturing technology assistance
\item
  \textbf{Innovation promotion}: Supporting technological innovation
\end{itemize}

\textbf{SIDBI (Small Industries Development Bank of India):}

\begin{itemize}
\tightlist
\item
  \textbf{Financial support}: Loans and credit facilities
\item
  \textbf{MSME focus}: Small and medium enterprise development
\item
  \textbf{Startup funding}: Venture capital and growth capital
\item
  \textbf{Ecosystem development}: Incubator and accelerator support
\end{itemize}

\begin{center}
\textbf{Mermaid Diagram (Code)}
\begin{verbatim}
{Shaded}
{Highlighting}[]
graph TD
    A[National Startup Support] {-{-}{} B[Startup India]}
    A {-{-}{} C[BIRAC]}
    A {-{-}{} D[TDB]}
    A {-{-}{} E[SIDBI]}
    
    B {-{-}{} B1[Policy{}br/{}Recognition{}br/{}Tax Benefits{}br/{}Fund of Funds]}
    C {-{-}{} C1[Biotech Innovation{}br/{}SBIRI/SPARSH{}br/{}Industry Partnership]}
    D {-{-}{} D1[Tech Commercialization{}br/{}Financial Assistance{}br/{}Innovation Support]}
    E {-{-}{} E1[MSME Loans{}br/{}Venture Capital{}br/{}Ecosystem Development]}
{Highlighting}
{Shaded}
\end{verbatim}
\end{center}

\textbf{Additional Agencies:}

\begin{itemize}
\tightlist
\item
  \textbf{NSTEDB}: National Science \& Technology Entrepreneurship
  Development Board
\item
  \textbf{MSME}: Ministry of Micro, Small and Medium Enterprises
\item
  \textbf{Atal Innovation Mission}: Innovation and entrepreneurship
  promotion
\end{itemize}

\textbf{Success Metrics:}

\begin{itemize}
\tightlist
\item
  \textbf{Startup registrations}: Over 70,000 recognized startups
\item
  \textbf{Job creation}: Millions of employment opportunities
\item
  \textbf{Funding facilitated}: Billions in investment mobilization
\item
  \textbf{Ecosystem development}: Thousands of incubators and
  accelerators
\end{itemize}

\end{solutionbox}
\begin{mnemonicbox}
``Startup BIRAC TDB SIDBI = National Support System''

\end{mnemonicbox}
\subsection*{Question 5(a) [3 marks]}\label{q5a}

\textbf{Explain following terms: 1.Break Even point 2.Return on
Investment 3.Return on Sales.}

\begin{solutionbox}

\begin{longtable}[]{@{}
  >{\raggedright\arraybackslash}p{(\linewidth - 4\tabcolsep) * \real{0.2778}}
  >{\raggedright\arraybackslash}p{(\linewidth - 4\tabcolsep) * \real{0.3611}}
  >{\raggedright\arraybackslash}p{(\linewidth - 4\tabcolsep) * \real{0.3611}}@{}}
\toprule\noalign{}
\begin{minipage}[b]{\linewidth}\raggedright
\textbf{Term}
\end{minipage} & \begin{minipage}[b]{\linewidth}\raggedright
\textbf{Formula}
\end{minipage} & \begin{minipage}[b]{\linewidth}\raggedright
\textbf{Meaning}
\end{minipage} \\
\midrule\noalign{}
\endhead
\bottomrule\noalign{}
\endlastfoot
\textbf{Break Even Point} & Fixed Costs \div (Price - Variable Cost) &
Units to cover all costs \\
\textbf{Return on Investment} & (Gain-Cost) \div Cost \times 100 & Percentage
return on invested capital \\
\textbf{Return on Sales} & Net Income \div Sales \times 100 & Profit margin
percentage \\
\end{longtable}

\textbf{Break Even Analysis:}

\begin{itemize}
\tightlist
\item
  \textbf{Fixed costs}: Rent, salaries, insurance
\item
  \textbf{Variable costs}: Raw materials, utilities per unit
\item
  \textbf{Contribution margin}: Price minus variable cost per unit
\end{itemize}

\textbf{ROI Importance:}

\begin{itemize}
\tightlist
\item
  \textbf{Investment efficiency}: Measures investment performance
\item
  \textbf{Comparison tool}: Compare different investment options
\item
  \textbf{Decision making}: Guide future investment decisions
\end{itemize}

\textbf{ROS Significance:}

\begin{itemize}
\tightlist
\item
  \textbf{Profitability measure}: Shows operational efficiency
\item
  \textbf{Industry comparison}: Benchmark against competitors
\item
  \textbf{Trend analysis}: Track performance over time
\end{itemize}

\end{solutionbox}
\begin{mnemonicbox}
``Break Even Returns On Investment Sales''

\end{mnemonicbox}
\subsection*{Question 5(b) [4 marks]}\label{q5b}

\textbf{Write a short note on Import-Export Policy}

\begin{solutionbox}

\textbf{India's Import-Export Policy (EXIM Policy):}

\textbf{Objectives:}

\begin{itemize}
\tightlist
\item
  \textbf{Trade promotion}: Increase international trade volume
\item
  \textbf{Export growth}: Boost export earnings and competitiveness
\item
  \textbf{Economic development}: Support manufacturing and job creation
\end{itemize}

\textbf{Key Features:}

\textbf{Export Promotion:}

\begin{itemize}
\tightlist
\item
  \textbf{Export incentives}: Duty drawback, MEIS schemes
\item
  \textbf{Special Economic Zones}: Tax-free export manufacturing
\item
  \textbf{Export financing}: Credit facilities and insurance
\end{itemize}

\textbf{Import Management:}

\begin{itemize}
\tightlist
\item
  \textbf{Import licensing}: Controlled import of sensitive items
\item
  \textbf{Duty structure}: Tariff rates and customs procedures
\item
  \textbf{Quality standards}: BIS and other quality requirements
\end{itemize}

\textbf{Trade Facilitation:}

\begin{itemize}
\tightlist
\item
  \textbf{Digital platforms}: Online export-import procedures
\item
  \textbf{Single window}: Unified clearance system
\item
  \textbf{Trade agreements}: Bilateral and multilateral agreements
\end{itemize}

\textbf{Current Focus Areas:}

\begin{itemize}
\tightlist
\item
  \textbf{Make in India}: Promoting domestic manufacturing
\item
  \textbf{Digital India}: Technology-enabled trade processes
\item
  \textbf{Atmanirbhar Bharat}: Self-reliance and import substitution
\end{itemize}

\end{solutionbox}
\begin{mnemonicbox}
``Export Import Policy Promotes Trade Facilitation''

\end{mnemonicbox}
\subsection*{Question 5(c) [7 marks]}\label{q5c}

\textbf{Describe the connection between CSR and Economic Performance.}

\begin{solutionbox}

\textbf{Corporate Social Responsibility (CSR) and Economic Performance
Link:}

\textbf{Direct Economic Benefits:}

\begin{longtable}[]{@{}
  >{\raggedright\arraybackslash}p{(\linewidth - 4\tabcolsep) * \real{0.3333}}
  >{\raggedright\arraybackslash}p{(\linewidth - 4\tabcolsep) * \real{0.3519}}
  >{\raggedright\arraybackslash}p{(\linewidth - 4\tabcolsep) * \real{0.3148}}@{}}
\toprule\noalign{}
\begin{minipage}[b]{\linewidth}\raggedright
\textbf{CSR Activity}
\end{minipage} & \begin{minipage}[b]{\linewidth}\raggedright
\textbf{Economic Impact}
\end{minipage} & \begin{minipage}[b]{\linewidth}\raggedright
\textbf{Measurement}
\end{minipage} \\
\midrule\noalign{}
\endhead
\bottomrule\noalign{}
\endlastfoot
\textbf{Employee welfare} & Higher productivity, lower turnover & Cost
savings, efficiency gains \\
\textbf{Environmental initiatives} & Resource efficiency, waste
reduction & Cost reduction, sustainability \\
\textbf{Community development} & Market expansion, brand loyalty &
Revenue growth, customer retention \\
\end{longtable}

\textbf{Indirect Economic Benefits:}

\textbf{Brand Value Enhancement:}

\begin{itemize}
\tightlist
\item
  \textbf{Customer loyalty}: Increased repeat purchases and referrals
\item
  \textbf{Premium pricing}: Ability to charge higher prices for ethical
  products
\item
  \textbf{Market differentiation}: Competitive advantage in conscious
  markets
\end{itemize}

\textbf{Risk Management:}

\begin{itemize}
\tightlist
\item
  \textbf{Regulatory compliance}: Avoiding penalties and legal costs
\item
  \textbf{Reputation protection}: Preventing brand damage from social
  issues
\item
  \textbf{Stakeholder relations}: Building trust with investors and
  partners
\end{itemize}

\textbf{Long-term Economic Performance:}

\textbf{Sustainable Growth:}

\begin{itemize}
\tightlist
\item
  \textbf{Innovation driver}: CSR initiatives often lead to innovative
  solutions
\item
  \textbf{Market access}: Meeting ESG criteria for international markets
\item
  \textbf{Investment attraction}: ESG-focused investors prefer
  responsible companies
\end{itemize}

\begin{center}
\textbf{Mermaid Diagram (Code)}
\begin{verbatim}
{Shaded}
{Highlighting}[]
graph TD
    A[CSR Activities] {-{-}{} B[Direct Benefits]}
    A {-{-}{} C[Indirect Benefits]}
    A {-{-}{} D[Long{-}term Impact]}
    
    B {-{-}{} B1[Cost Savings{}br/{}Efficiency{}br/{}Productivity]}
    C {-{-}{} C1[Brand Value{}br/{}Risk Management{}br/{}Stakeholder Trust]}
    D {-{-}{} D1[Sustainable Growth{}br/{}Innovation{}br/{}Market Access]}
    
    B1 {-{-}{} E[Economic Performance]}
    C1 {-{-}{} E}
    D1 {-{-}{} E}
{Highlighting}
{Shaded}
\end{verbatim}
\end{center}

\textbf{Research Evidence:}

\begin{itemize}
\tightlist
\item
  \textbf{Performance correlation}: Studies show positive correlation
  between CSR and financial performance
\item
  \textbf{Investor preference}: ESG-compliant companies attract more
  investment
\item
  \textbf{Market valuation}: Responsible companies often have higher
  market valuations
\end{itemize}

\textbf{CSR-Economic Performance Cycle:}

\begin{itemize}
\tightlist
\item
  \textbf{Investment in CSR} \rightarrow \textbf{Operational improvements} \rightarrow
  \textbf{Financial performance} \rightarrow \textbf{More CSR investment}
\end{itemize}

\textbf{Implementation Strategy:}

\begin{itemize}
\tightlist
\item
  \textbf{Strategic alignment}: Align CSR with business objectives
\item
  \textbf{Measurement systems}: Track both social and economic impacts
\item
  \textbf{Stakeholder engagement}: Involve all stakeholders in CSR
  planning
\item
  \textbf{Continuous improvement}: Regular review and enhancement of CSR
  programs
\end{itemize}

\textbf{Challenges:}

\begin{itemize}
\tightlist
\item
  \textbf{Short-term costs}: Initial investment may impact immediate
  profits
\item
  \textbf{Measurement difficulty}: Quantifying social impact can be
  complex
\item
  \textbf{Stakeholder expectations}: Balancing different stakeholder
  demands
\end{itemize}

\textbf{Success Factors:}

\begin{itemize}
\tightlist
\item
  \textbf{Leadership commitment}: Top management support for CSR
  initiatives
\item
  \textbf{Integration}: Embedding CSR into business strategy and
  operations
\item
  \textbf{Transparency}: Regular reporting and communication of CSR
  impact
\item
  \textbf{Innovation}: Using CSR as a driver for business innovation
\end{itemize}

\end{solutionbox}
\begin{mnemonicbox}
``CSR Creates Sustainable Returns''

\end{mnemonicbox}
\subsection*{Question 5(a) OR [3
marks]}\label{q5a}

\textbf{Write a note on Bankruptcy and Avoidance.}

\begin{solutionbox}

\textbf{Bankruptcy:}

\begin{itemize}
\tightlist
\item
  \textbf{Definition}: Legal process when business cannot meet financial
  obligations
\item
  \textbf{Types}: Voluntary (self-initiated) or Involuntary
  (creditor-initiated)
\item
  \textbf{Process}: Asset liquidation or reorganization under court
  supervision
\end{itemize}

\textbf{Avoidance Strategies:}

\begin{itemize}
\tightlist
\item
  \textbf{Cash flow management}: Maintain adequate working capital
\item
  \textbf{Debt restructuring}: Negotiate payment terms with creditors
\item
  \textbf{Cost reduction}: Cut unnecessary expenses and improve
  efficiency
\end{itemize}

\textbf{Legal Framework:}

\begin{itemize}
\tightlist
\item
  \textbf{Insolvency and Bankruptcy Code (IBC)}: Indian bankruptcy law
\item
  \textbf{Resolution process}: 180-day timeline for resolution
\item
  \textbf{Stakeholder protection}: Balanced approach for creditors and
  debtors
\end{itemize}

\end{solutionbox}
\begin{mnemonicbox}
``Bankrupt Businesses Avoid Through Cash Control''

\end{mnemonicbox}
\subsection*{Question 5(b) OR [4
marks]}\label{q5b}

\textbf{Write an importance of Business Ethics.}

\begin{solutionbox}

\textbf{Importance of Business Ethics:}

\textbf{Stakeholder Trust:}

\begin{itemize}
\tightlist
\item
  \textbf{Customer confidence}: Ethical practices build customer loyalty
\item
  \textbf{Investor faith}: Transparent operations attract investment
\item
  \textbf{Employee satisfaction}: Ethical workplace improves retention
\end{itemize}

\textbf{Legal Compliance:}

\begin{itemize}
\tightlist
\item
  \textbf{Regulatory adherence}: Avoiding legal penalties and sanctions
\item
  \textbf{Risk mitigation}: Preventing ethical scandals and crises
\item
  \textbf{Reputation protection}: Maintaining positive brand image
\end{itemize}

\textbf{Competitive Advantage:}

\begin{itemize}
\tightlist
\item
  \textbf{Market differentiation}: Ethical brands stand out in
  marketplace
\item
  \textbf{Premium positioning}: Ethical products can command higher
  prices
\item
  \textbf{Sustainable growth}: Long-term success through ethical
  practices
\end{itemize}

\textbf{Social Impact:}

\begin{itemize}
\tightlist
\item
  \textbf{Community development}: Contributing to societal welfare
\item
  \textbf{Environmental responsibility}: Sustainable business practices
\item
  \textbf{Economic contribution}: Fair business practices support
  economic growth
\end{itemize}

\end{solutionbox}
\begin{mnemonicbox}
``Ethics Builds Trust, Compliance, Advantage, Social
Impact''

\end{mnemonicbox}
\subsection*{Question 5(c) OR [7
marks]}\label{q5c}

\textbf{Give the steps and format of project report writing}

\begin{solutionbox}

\textbf{Project Report Writing Steps:}

\textbf{Pre-Writing Phase:}

\begin{enumerate}
\tightlist
\item
  \textbf{Project planning}: Define scope, objectives, and deliverables
\item
  \textbf{Data collection}: Gather relevant information and research
\item
  \textbf{Analysis}: Process and analyze collected data
\item
  \textbf{Structure planning}: Organize content logically
\end{enumerate}

\textbf{Writing Phase:} 5. \textbf{Draft preparation}: Write initial
version following format 6. \textbf{Content development}: Elaborate each
section with details 7. \textbf{Review and revision}: Check for accuracy
and completeness 8. \textbf{Final formatting}: Apply consistent
formatting and style

\textbf{Project Report Format:}

\begin{verbatim}
1. TITLE PAGE
   - Project title
   - Author name(s)
   - Institution/Organization
   - Date of submission

2. EXECUTIVE SUMMARY
   - Project overview (1-2 pages)
   - Key findings and recommendations
   - Expected outcomes

3. TABLE OF CONTENTS
   - Chapter/section headings
   - Page numbers
   - List of figures and tables

4. INTRODUCTION
   - Background and context
   - Problem statement
   - Objectives and scope
   - Methodology overview

5. LITERATURE REVIEW
   - Existing research and studies
   - Gap analysis
   - Theoretical framework

6. METHODOLOGY
   - Research approach
   - Data collection methods
   - Analysis techniques
   - Limitations

7. ANALYSIS AND FINDINGS
   - Data presentation
   - Results and interpretation
   - Key insights

8. RECOMMENDATIONS
   - Actionable suggestions
   - Implementation plan
   - Expected benefits

9. CONCLUSION
   - Summary of findings
   - Achievement of objectives
   - Future scope

10. REFERENCES
    - Bibliography
    - Sources cited
    - Appendices (if any)
\end{verbatim}

\textbf{Writing Guidelines:}

\textbf{Content Quality:}

\begin{itemize}
\tightlist
\item
  \textbf{Clarity}: Use simple, clear language
\item
  \textbf{Accuracy}: Ensure factual correctness
\item
  \textbf{Relevance}: Include only pertinent information
\item
  \textbf{Logical flow}: Maintain coherent structure
\end{itemize}

\textbf{Formatting Standards:}

\begin{itemize}
\tightlist
\item
  \textbf{Font}: Times New Roman 12pt or Arial 11pt
\item
  \textbf{Spacing}: 1.5 line spacing
\item
  \textbf{Margins}: 1 inch on all sides
\item
  \textbf{Page numbering}: Consistent throughout
\end{itemize}

\textbf{Visual Elements:}

\begin{itemize}
\tightlist
\item
  \textbf{Tables}: For data presentation
\item
  \textbf{Charts/Graphs}: For trend analysis
\item
  \textbf{Diagrams}: For process illustration
\item
  \textbf{Images}: For concept clarification
\end{itemize}

\begin{center}
\textbf{Mermaid Diagram (Code)}
\begin{verbatim}
{Shaded}
{Highlighting}[]
graph TD
    A[Project Report] {-{-}{} B[Title Page]}
    A {-{-}{} C[Executive Summary]}
    A {-{-}{} D[Introduction]}
    A {-{-}{} E[Literature Review]}
    A {-{-}{} F[Methodology]}
    A {-{-}{} G[Analysis \& Findings]}
    A {-{-}{} H[Recommendations]}
    A {-{-}{} I[Conclusion]}
    A {-{-}{} J[References]}
    
    B {-{-}{} B1[Project Title{}br/{}Author Details{}br/{}Date]}
    C {-{-}{} C1[Overview{}br/{}Key Findings{}br/{}Outcomes]}
    D {-{-}{} D1[Background{}br/{}Problem{}br/{}Objectives]}
    E {-{-}{} E1[Research Review{}br/{}Gap Analysis{}br/{}Framework]}
    F {-{-}{} F1[Approach{}br/{}Data Collection{}br/{}Analysis Methods]}
    G {-{-}{} G1[Data Presentation{}br/{}Results{}br/{}Insights]}
    H {-{-}{} H1[Suggestions{}br/{}Implementation{}br/{}Benefits]}
    I {-{-}{} I1[Summary{}br/{}Achievement{}br/{}Future Scope]}
    J {-{-}{} J1[Bibliography{}br/{}Citations{}br/{}Appendices]}
{Highlighting}
{Shaded}
\end{verbatim}
\end{center}

\textbf{Quality Checklist:}

\begin{itemize}
\tightlist
\item
  \textbf{Completeness}: All required sections included
\item
  \textbf{Consistency}: Uniform formatting throughout
\item
  \textbf{Accuracy}: Facts and figures verified
\item
  \textbf{Relevance}: Content aligned with objectives
\end{itemize}

\textbf{Common Mistakes to Avoid:}

\begin{itemize}
\tightlist
\item
  \textbf{Plagiarism}: Always cite sources properly
\item
  \textbf{Poor structure}: Maintain logical flow
\item
  \textbf{Inconsistent formatting}: Follow standard guidelines
\item
  \textbf{Inadequate analysis}: Provide sufficient depth
\end{itemize}

\textbf{Review Process:}

\begin{enumerate}
\tightlist
\item
  \textbf{Self-review}: Author checks for errors and completeness
\item
  \textbf{Peer review}: Colleague feedback on content and clarity
\item
  \textbf{Expert review}: Subject matter expert validation
\item
  \textbf{Final proofreading}: Grammar and formatting check
\end{enumerate}

\end{solutionbox}
\begin{mnemonicbox}
``Title Executive Introduction Literature Methodology
Analysis Recommendations Conclusion References''

\end{mnemonicbox}

\end{document}
