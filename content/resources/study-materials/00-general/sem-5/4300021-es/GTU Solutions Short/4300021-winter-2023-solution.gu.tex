\documentclass{article}

% content/resources/templates/preamble.tex
\usepackage[margin=0.6in]{geometry}
\author{Milav Dabgar}
\usepackage{amsmath,amssymb,amsthm}
\usepackage{booktabs}
\usepackage{multirow}
\usepackage{xcolor}
\usepackage{tcolorbox}
\tcbuselibrary{breakable,skins}
\usepackage[colorlinks=true,linkcolor=blue]{hyperref}
\usepackage{titlesec}
\usepackage{enumitem}
\usepackage{tikz}
\usepackage{pgfplots}
\usepackage{circuitikz}
\usepackage[version=4]{mhchem}
\usepackage{longtable}
\usepackage{array}
\usepackage{float}
\usepackage{caption}
\usepackage{listings}

\lstset{
  basicstyle=\small\ttfamily,
  breaklines=true,
  breakatwhitespace=false,
  postbreak=\mbox{\textcolor{red}{$\hookrightarrow$}\space},
  float=false,
  numbers=left,
  numberstyle=\tiny\color{gray},
  numbersep=10pt,
  xleftmargin=2em,
  keywordstyle=\color{blue},
  commentstyle=\color{green!60!black},
  stringstyle=\color{purple},
  backgroundcolor=\color{gray!5},
  showstringspaces=false,
  tabsize=2,
  captionpos=b,
  keepspaces=true,
  columns=flexible
}

\pgfplotsset{compat=1.18}
\usetikzlibrary{shapes,arrows,positioning,calc,patterns,decorations.pathmorphing,decorations.markings,arrows.meta}

% Color scheme
\definecolor{headcolor}{RGB}{0,102,204}
\definecolor{keycolor}{RGB}{220,20,60}
\definecolor{solutioncolor}{RGB}{34,139,34}
\definecolor{mnemoniccolor}{RGB}{148,0,211}
\definecolor{codecolor}{RGB}{0,0,100}

% Spacing
\setlength{\parskip}{3pt}
\setlist[itemize]{nosep}
\setlist[enumerate]{nosep}

% Title formatting
\titleformat{\section}{\Large\bfseries\color{headcolor}}{\thesection}{1em}{}
\titleformat{\subsection}{\large\bfseries\color{headcolor}}{\thesubsection}{1em}{}

% Pandoc tightlist compatibility
\providecommand{\tightlist}{%
  \setlength{\itemsep}{0pt}\setlength{\parskip}{0pt}}

% Pandoc longtable compatibility
\newcounter{none}
\def\thenone{}


% content/resources/templates/gujarati-boxes.tex
\usepackage{fontspec}
\usepackage{polyglossia}

% Set Gujarati as main language (document is primarily in Gujarati)
% Note: gloss-gujarati.ldf doesn't exist in polyglossia, but it will use hyphenation patterns
\setdefaultlanguage{gujarati}
\setotherlanguage{english}

% Configure Gujarati font properly
% Use Language=Default to prevent polyglossia from trying to add language-specific features
% that don't exist for Gujarati, which causes "empty feature" warnings
\newfontfamily\gujaratifont[Script=Gujarati,AutoFakeBold=2.5,AutoFakeSlant=0.3]{Noto Sans Gujarati}
\setmainfont[Script=Gujarati,AutoFakeBold=2.5,AutoFakeSlant=0.3]{Noto Sans Gujarati}
% Use Noto Sans Gujarati for monospace to support Gujarati in text
\setmonofont[Scale=0.9]{Noto Sans Gujarati}

% Configure English to use the same font
\newfontfamily\englishfont[Script=Gujarati,AutoFakeBold=2.5,AutoFakeSlant=0.3]{Noto Sans Gujarati}

% Translations for polyglossia
\gappto\captionsgujarati{
  \renewcommand{\tablename}{કોષ્ટક}
  \renewcommand{\figurename}{આકૃતિ}
}

% Helper for TikZ nodes to ensure Gujarati font
\newcommand{\gu}[1]{{\gujaratifont #1}}

% Custom environments
\newtcolorbox{solutionbox}{
    breakable,
    enhanced,
    colback=solutioncolor!5!white,
    colframe=solutioncolor!75!black,
    fonttitle=\bfseries,
    title=જવાબ
}

\newtcolorbox{solutionboxnobreak}{
 colback=solutioncolor!5!white,
 colframe=solutioncolor!75!black,
 fonttitle=\bfseries,
 title=જવાબ
}

\newtcolorbox{keyformula}{
 breakable,
 enhanced,
 colback=keycolor!5!white,
 colframe=keycolor!75!black,
 fonttitle=\bfseries,
 title=રાસાયણિક સમીકરણ/સૂત્ર
}

\newtcolorbox{mnemonicbox}{
 breakable,
 enhanced,
 colback=mnemoniccolor!5!white,
 colframe=mnemoniccolor!75!black,
 fonttitle=\bfseries,
 title=મેમરી ટ્રીક
}


% Custom commands for GTU solutions
% This file defines semantic commands for consistent formatting

% Question command with automatic formatting
\newcommand{\question}[2]{%
  \section*{Question #1}%
  \textbf{#2}%
}

% OR question variant
\newcommand{\questionor}[2]{%
  \section*{Question #1 OR}%
  \textbf{#2}%
}

% Proper table environment with caption
\newenvironment{answertable}[1]{%
  \begin{table}[htbp]
  \centering
  \caption{#1}
}{%
  \end{table}
}

% Proper figure environment for diagrams
\newenvironment{answerdiagram}[1]{%
  \begin{figure}[htbp]
  \centering
  \caption{#1}
}{%
  \end{figure}
}

% Semantic markup for key terms
\newcommand{\keyword}[1]{\textbf{#1}}
\newcommand{\code}[1]{\texttt{#1}}
\newcommand{\classname}[1]{\texttt{#1}}
\newcommand{\methodname}[1]{\texttt{#1}}

% Proper quotation marks
\newcommand{\mnemonic}[1]{``#1''}


\title{Entrepreneurship and Start-ups (4300021) - Winter 2023 Solution}
\date{December 01, 2023}

\begin{document}
\maketitle

\questionmarks{1(અ)}{3}{આંત્રપ્રેન્યોરશિપ અને ઈન્ટ્રાપ્રેન્યોરશિપ વચ્ચે સરખામણી આપો.}

\begin{solutionbox}
\begin{center}
\captionof{table}{આંત્રપ્રેન્યોરશિપ vs ઈન્ટ્રાપ્રેન્યોરશિપ}
\begin{tabulary}{\linewidth}{|L|L|L|}
\hline
\textbf{પાસું} & \textbf{આંત્રપ્રેન્યોરશિપ} & \textbf{ઈન્ટ્રાપ્રેન્યોરશિપ} \\ \hline
\textbf{વ્યાખ્યા} & પોતાનો વ્યવસાય શરૂ કરવો & હાલની સંસ્થામાં નવીનતા \\ \hline
\textbf{જોખમ} & વ્યક્તિગત નાણાકીય જોખમ & સંસ્થા જોખમ લે છે \\ \hline
\textbf{સંસાધનો} & પોતાના/ઉધાર લીધેલા & કંપની પૂરા પાડે છે \\ \hline
\end{tabulary}
\end{center}
\end{solutionbox}

\begin{mnemonicbox}
\mnemonic{"બાહ્ય વિરુદ્ધ આંતરિક નવીનતા"}
\end{mnemonicbox}

\questionmarks{1(બ)}{4}{ઉદ્યોગસાહસિકતાની લાક્ષણિકતાઓ અને કાર્યોની ચર્ચા કરો}

\begin{solutionbox}
\textbf{લાક્ષણિકતાઓ:}

\begin{itemize}
    \item \keyword{જોખમ લેવાની ક્ષમતા}: હિસાબી વ્યાપારી જોખમો લેવાની તૈયારી
    \item \keyword{નવીનતા}: નવા ઉત્પાદનો, સેવાઓ અથવા પ્રક્રિયાઓ બનાવવી
    \item \keyword{નેતૃત્વ કુશળતા}: ટીમને માર્ગદર્શન અને પ્રેરણા આપવાની ક્ષમતા
\end{itemize}

\textbf{કાર્યો:}

\begin{itemize}
    \item \keyword{રોજગાર સર્જન}: સમાજ માટે રોજગારની તકો બનાવવી
    \item \keyword{આર્થિક વિકાસ}: GDP અને રાષ્ટ્રીય વૃદ્ધિમાં યોગદાન
    \item \keyword{નવીનતાનું કેન્દ્ર}: નવી ટેકનોલોજી અને ઉકેલો રજૂ કરવા
\end{itemize}
\end{solutionbox}

\begin{mnemonicbox}
\mnemonic{"જોખમ નવીનતા નેતૃત્વ રોજગાર વિકાસ નવીનતા"}
\end{mnemonicbox}

\questionmarks{1(ક)}{7}{7-M સંસાધનોને ઓળખો અને વિગતવાર ચર્ચા કરો.}

\begin{solutionbox}
\begin{center}
\captionof{table}{7-M સંસાધનો}
\begin{tabulary}{\linewidth}{|L|L|L|}
\hline
\textbf{સંસાધન} & \textbf{વર્ણન} & \textbf{મહત્વ} \\ \hline
\textbf{Man (માનવી)} & માનવ સંસાધનો અને કર્મચારીઓ & કામકાજ માટે મુખ્ય સંપત્તિ \\ \hline
\textbf{Money (પૈસા)} & નાણાકીય મૂડી અને ભંડોળ & વ્યાપારી કામકાજ માટે જરૂરી \\ \hline
\textbf{Material (સામગ્રી)} & કાચો માલ અને પુરવઠો & ઉત્પાદન આવશ્યકતાઓ \\ \hline
\textbf{Machine (મશીન)} & સાધનો અને ટેકનોલોજી & કામકાજની કાર્યક્ષમતા \\ \hline
\textbf{Method (પદ્ધતિ)} & પ્રક્રિયાઓ અને કાર્યવિધિઓ & વ્યવસ્થિત અભિગમ \\ \hline
\textbf{Market (બજાર)} & ગ્રાહક આધાર અને માંગ & આવકનું ઉત્પાદન \\ \hline
\textbf{Management (સંચાલન)} & આયોજન અને સંકલન & એકંદર વ્યાપારી નિયંત્રણ \\ \hline
\end{tabulary}
\end{center}

\begin{center}
\begin{tikzpicture}[node distance=2cm]
  \node [gtu block, minimum size=1.5cm] (root) {7-M સંસાધનો};
  \node [gtu block, above=of root] (man) {Man (માનવી)};
  \node [gtu block, above right=of root] (money) {Money (પૈસા)};
  \node [gtu block, right=of root] (material) {Material (સામગ્રી)};
  \node [gtu block, below right=of root] (machine) {Machine (મશીન)};
  \node [gtu block, below=of root] (method) {Method (પદ્ધતિ)};
  \node [gtu block, below left=of root] (market) {Market (બજાર)};
  \node [gtu block, left=of root] (management) {Management (સંચાલન)};
  
  \draw [gtu arrow] (root) -- (man);
  \draw [gtu arrow] (root) -- (money);
  \draw [gtu arrow] (root) -- (material);
  \draw [gtu arrow] (root) -- (machine);
  \draw [gtu arrow] (root) -- (method);
  \draw [gtu arrow] (root) -- (market);
  \draw [gtu arrow] (root) -- (management);
\end{tikzpicture}
\captionof{figure}{7-M સંસાધનો}
\end{center}
\end{solutionbox}

\begin{mnemonicbox}
\mnemonic{"અનેક આધુનિક મેનેજરો પૈસા બનાવવા બજારોનું સંચાલન કરે છે"}
\end{mnemonicbox}

\questionmarks{1(ક) OR}{7}{સ્ટાર્ટ અપ ઇન્ડિયા નોંધણી પ્રક્રિયા લખો.}

\begin{solutionbox}
\textbf{સ્ટાર્ટ-અપ ઇન્ડિયા નોંધણીના પગલાં:}

\begin{enumerate}
    \item \textbf{ઓનલાઇન નોંધણી}: www.startupindia.gov.in ની મુલાકાત લો
    \item \textbf{દસ્તાવેજ તૈયારી}:
    \begin{itemize}
        \item નિગમીકરણનું પ્રમાણપત્ર
        \item એન્ટિટીનું PAN કાર્ડ
        \item વ્યવસાયનું સંક્ષિપ્ત વર્ણન
    \end{itemize}
    \item \textbf{પાત્રતાના માપદંડો}:
    \begin{itemize}
        \item એન્ટિટીની ઉંમર 10 વર્ષથી ઓછી
        \item વાર્ષિક ટર્નઓવર ₹100 કરોડથી ઓછું
        \item નવીનતા/સુધારા તરફ કામ કરવું
    \end{itemize}
    \item \textbf{અરજી સબમિશન}: જરૂરી દસ્તાવેજો સાથે ઓનલાઇન ફોર્મ ભરવો
    \item \textbf{ચકાસણી પ્રક્રિયા}: સરકારી સમીક્ષા અને મંજૂરી
    \item \textbf{પ્રમાણપત્ર આપવું}: માન્યતા પ્રમાણપત્ર પ્રાપ્ત કરવું
\end{enumerate}

\begin{center}
\begin{tikzpicture}[node distance=0.8cm, auto]
    \node [gtu block] (A) {1. ઓનલાઇન નોંધણી};
    \node [gtu block, below=of A] (B) {2. દસ્તાવેજ તૈયારી};
    \node [gtu block, below=of B] (C) {3. પાત્રતાના માપદંડો};
    \node [gtu block, below=of C] (D) {4. અરજી સબમિશન};
    \node [gtu block, below=of D] (E) {5. ચકાસણી પ્રક્રિયા};
    \node [gtu block, below=of E] (F) {6. પ્રમાણપત્ર આપવું};
    
    \draw [gtu arrow] (A) -- (B);
    \draw [gtu arrow] (B) -- (C);
    \draw [gtu arrow] (C) -- (D);
    \draw [gtu arrow] (D) -- (E);
    \draw [gtu arrow] (E) -- (F);
\end{tikzpicture}
\captionof{figure}{નોંધણી પ્રક્રિયા}
\end{center}

\textbf{ફાયદાઓ:}

\begin{itemize}
    \item \keyword{કર મુક્તિ} સતત 3 વર્ષ માટે
    \item \keyword{ઝડપી પેટન્ટ} અરજી પ્રક્રિયા
    \item \keyword{કમ્પ્લાયન્સ ઘટાડો} લેબર અને પર્યાવરણ કાયદા હેઠળ
\end{itemize}
\end{solutionbox}

\begin{mnemonicbox}
\mnemonic{"ઓનલાઇન દસ્તાવેજ પાત્રતા અરજી ચકાસણી પ્રમાણપત્ર ફાયદાઓ"}
\end{mnemonicbox}

\questionmarks{2(અ)}{3}{બજાર સંશોધનની પદ્ધતિઓની સૂચિ બનાવો.}

\begin{solutionbox}
\textbf{પ્રાથમિક સંશોધન પદ્ધતિઓ:}

\begin{itemize}
    \item \keyword{સર્વે}: ગ્રાહક ડેટા એકત્રિત કરવા માટે પ્રશ્નાવલી
    \item \keyword{ઇન્ટરવ્યુ}: લક્ષ્ય પ્રેક્ષકો સાથે સીધી વાતચીત
    \item \keyword{ફોકસ ગ્રુપ}: પ્રતિસાદ માટે જૂથ ચર્ચાઓ
\end{itemize}

\textbf{દ્વિતીયક સંશોધન પદ્ધતિઓ:}

\begin{itemize}
    \item \keyword{ઓનલાઇન સંશોધન}: ઇન્ટરનેટ આધારિત ડેટા સંગ્રહ
    \item \keyword{પ્રકાશિત અહેવાલો}: ઉદ્યોગ વિશ્લેષણ અને અભ્યાસો
    \item \keyword{સરકારી ડેટા}: સત્તાવાર સ્ત્રોતોથી આંકડાકીય માહિતી
\end{itemize}
\end{solutionbox}

\begin{mnemonicbox}
\mnemonic{"સર્વે ઇન્ટરવ્યુ ફોકસ ઓનલાઇન પ્રકાશિત સરકારી"}
\end{mnemonicbox}

\questionmarks{2(બ)}{4}{ઉત્પાદન જીવન ચક્ર દોરો અને સમજાવો.}

\begin{solutionbox}
\begin{center}
\begin{tikzpicture}[node distance=1.5cm, auto]
    \node [gtu state] (A) {પરિચય};
    \node [gtu state, right=of A] (B) {વૃદ્ધિ};
    \node [gtu state, right=of B] (C) {પરિપક્વતા};
    \node [gtu state, right=of C] (D) {ઘટાડો};
    
    \path [gtu arrow] (A) -- (B);
    \path [gtu arrow] (B) -- (C);
    \path [gtu arrow] (C) -- (D);
\end{tikzpicture}
\captionof{figure}{ઉત્પાદન જીવન ચક્ર}
\end{center}

\textbf{તબક્કાઓ:}

\begin{itemize}
    \item \keyword{પરિચય}: ઊંચા માર્કેટિંગ ખર્ચ સાથે ઉત્પાદન લોન્ચ
    \item \keyword{વૃદ્ધિ}: ઝડપી વેચાણ વધારો અને બજારમાં સ્વીકૃતિ
    \item \keyword{પરિપક્વતા}: તીવ્ર સ્પર્ધા સાથે ટોચના વેચાણ
    \item \keyword{ઘટાડો}: માંગમાં ઘટાડો અને અંતે તબક્કાબંધ
\end{itemize}
\end{solutionbox}

\begin{mnemonicbox}
\mnemonic{"હું મારા સપના વધારું છું"}
\end{mnemonicbox}

\questionmarks{2(ક)}{7}{માર્કેટિંગના 4 P ને ઓળખો અને ચર્ચા કરો.}

\begin{solutionbox}
\begin{center}
\captionof{table}{માર્કેટિંગના 4 P}
\begin{tabulary}{\linewidth}{|L|L|L|}
\hline
\textbf{P} & \textbf{તત્વ} & \textbf{વર્ણન} \\ \hline
\textbf{Product} & ઉત્પાદન & લક્ષણો, ગુણવત્તા, બ્રાન્ડિંગ \\ \hline
\textbf{Price} & કિંમત & કિંમત વ્યૂહરચના, છૂટ \\ \hline
\textbf{Place} & સ્થળ & વિતરણ ચેનલ્સ \\ \hline
\textbf{Promotion} & પ્રમોશન & જાહેરાત, વેચાણ પ્રમોશન \\ \hline
\end{tabulary}
\end{center}

\begin{center}
\begin{tikzpicture}[node distance=1.5cm]
  \node [gtu block] (root) {4 P's of Marketing};
  \node [gtu block, below left=2cm of root] (prod) {Product (ઉત્પાદન)};
  \node [gtu block, below right=0.5cm and 2cm of root] (price) {Price (કિંમત)};
  \node [gtu block, below left=0.5cm and 2cm of root] (place) {Place (સ્થળ)};
  \node [gtu block, below right=2cm of root] (prom) {Promotion (પ્રમોશન)};
  
  \draw [gtu arrow] (root) -- (prod);
  \draw [gtu arrow] (root) -- (price);
  \draw [gtu arrow] (root) -- (place);
  \draw [gtu arrow] (root) -- (prom);
\end{tikzpicture}
\captionof{figure}{માર્કેટિંગ મિશ્રણ}
\end{center}
\end{solutionbox}

\begin{mnemonicbox}
\mnemonic{"લોકો ઉત્પાદનો યોગ્ય રીતે ખરીદે છે"}
\end{mnemonicbox}

\questionmarks{2(અ) OR}{3}{B2B, ઈ-કોમર્સ અને GeM ની ચર્ચા કરો.}

\begin{solutionbox}
\begin{center}
\captionof{table}{B2B, E-commerce, GeM}
\begin{tabulary}{\linewidth}{|L|L|L|}
\hline
\textbf{પ્રકાર} & \textbf{સંપૂર્ણ નામ} & \textbf{વર્ણન} \\ \hline
\textbf{B2B} & Business to Business & કંપનીઓ વચ્ચેનો વેપાર \\ \hline
\textbf{E-commerce} & Electronic Commerce & ઓનલાઇન ખરીદી અને વેચાણ \\ \hline
\textbf{GeM} & Government e-Marketplace & સરકારી પ્રાપ્તિ પોર્ટલ \\ \hline
\end{tabulary}
\end{center}


\textbf{મુખ્ય લક્ષણો:}

\begin{itemize}
    \item \textbf{B2B}: મોટા પ્રમાણમાં વ્યવહારો, લાંબા ગાળાના સંબંધો
    \item \textbf{E-commerce}: ડિજિટલ પ્લેટફોર્મ, વૈશ્વિક પહોંચ
    \item \textbf{GeM}: પારદર્શક સરકારી ખરીદી, સ્પર્ધાત્મક કિંમત
\end{itemize}
\end{solutionbox}

\begin{mnemonicbox}
\mnemonic{"વ્યવસાયો ઇલેક્ટ્રોનિક રીતે ખરીદે, સરકાર ઈ-માર્કેટ"}
\end{mnemonicbox}

\questionmarks{2(બ) OR}{4}{વ્યવસાય બનાવવા અને શરૂ કરવા માટેની યોજનાઓ પર એક નોંધ લખો}

\begin{solutionbox}
\textbf{વ્યાપાર સર્જન યોજનાઓ:}

\begin{itemize}
    \item \textbf{બજાર વિશ્લેષણ}:
    \begin{itemize}
        \item \keyword{લક્ષ્ય ગ્રાહકો}: પ્રાથમિક પ્રેક્ષકોને ઓળખવા
        \item \keyword{સ્પર્ધા અભ્યાસ}: હાલના ખેલાડીઓનું વિશ્લેષણ
    \end{itemize}
    \item \textbf{નાણાકીય આયોજન}:
    \begin{itemize}
        \item \keyword{મૂડીની જરૂરિયાતો}: પ્રારંભિક રોકાણ
        \item \keyword{આવકના અંદાજો}: અપેક્ષિત આવક
    \end{itemize}
    \item \textbf{કામકાજનું સેટઅપ}:
    \begin{itemize}
        \item \keyword{સ્થળ પસંદગી}: વ્યૂહાત્મક સ્થિતિ
        \item \keyword{સંસાધન ફાળવણી}: માનવ અને ભૌતિક સંસાધનો
    \end{itemize}
\end{itemize}
\end{solutionbox}

\begin{mnemonicbox}
\mnemonic{"બજાર નાણા કામકાજ = વ્યાપારની સફળતા"}
\end{mnemonicbox}

\questionmarks{2(ક) OR}{7}{જોખમ અને SWOT વિશ્લેષણની કલ્પના સમજાવો.}

\begin{solutionbox}
\textbf{જોખમની કલ્પના:}
જોખમ એ અનિશ્ચિતતા છે જે વ્યાપારી પરિણામોને સકારાત્મક અને નકારાત્મક બંને રીતે અસર કરી શકે છે.

\textbf{વ્યાપારિક જોખમોના પ્રકારો:}
\begin{itemize}
    \item \textbf{નાણાકીય જોખમ}: રોકડ પ્રવાહ અને ભંડોળની સમસ્યાઓ
    \item \textbf{બજાર જોખમ}: માંગની વધઘટ અને સ્પર્ધા
    \item \textbf{કામકાજી જોખમ}: ઉત્પાદન અને સેવા વિતરણની સમસ્યાઓ
\end{itemize}
\textbf{SWOT વિશ્લેષણ:}

\begin{center}
\begin{tikzpicture}[node distance=0cm, outer sep=0pt]
    \node [gtu block, minimum width=4cm, minimum height=2cm] (S) {શક્તિઓ\\(Strengths)};
    \node [gtu block, minimum width=4cm, minimum height=2cm, right=0.5cm of S] (W) {નબળાઈઓ\\(Weaknesses)};
    \node [gtu block, minimum width=4cm, minimum height=2cm, below=0.5cm of S] (O) {તકો\\(Opportunities)};
    \node [gtu block, minimum width=4cm, minimum height=2cm, right=0.5cm of O] (T) {ધમકીઓ\\(Threats)};
    
    \node [above=0.1cm of S] {\textbf{આંતરિક}};
    \node [above=0.1cm of W] {\textbf{આંતરિક}};
    \node [left=0.1cm of S, rotate=90] {\textbf{સકારાત્મક}};
    \node [left=0.1cm of O, rotate=90] {\textbf{નકારાત્મક}};
\end{tikzpicture}
\captionof{figure}{SWOT વિશ્લેષણ}
\end{center}

\begin{center}
\captionof{table}{SWOT વિશ્લેષણ વિગતો}
\begin{tabulary}{\linewidth}{|L|L|}
\hline
\textbf{આંતરિક પરિબળો} & \textbf{બાહ્ય પરિબળો} \\ \hline
\textbf{શક્તિઓ (Strengths)} & \textbf{તકો (Opportunities)} \\
- મુખ્ય ક્ષમતાઓ & - બજાર વૃદ્ધિ \\
- અનન્ય સંસાધનો & - નવી ટેકનોલોજીઓ \\ \hline
\textbf{નબળાઈઓ (Weaknesses)} & \textbf{ધમકીઓ (Threats)} \\
- કુશળતાના ગાબડા & - સ્પર્ધા \\
- સંસાધન મર્યાદાઓ & - આર્થિક ફેરફારો \\ \hline
\end{tabulary}
\end{center}


\textbf{જોખમ ઘટાડવાની વ્યૂહરચનાઓ}: વૈવિધ્યકરણ, વીમો, આકસ્મિક આયોજન.
\end{solutionbox}

\begin{mnemonicbox}
\mnemonic{"બળવાન નબળા તકો ધમકાવે = SWOT"}
\end{mnemonicbox}

\questionmarks{3(અ)}{3}{સહકારી પ્રકારની સંસ્થા પર ટૂંકી નોંધ લખો.}

\begin{solutionbox}
\textbf{સહકારી સંસ્થા:}

\begin{itemize}
    \item \keyword{વ્યાખ્યા}: પરસ્પર લાભ માટે લોકોનું સ્વૈચ્છિક સંગઠન
    \item \keyword{માલિકી}: સભ્યો દ્વારા સામૂહિક માલિકી
    \item \keyword{નિયંત્રણ}: સમાન મતદાન અધિકારો સાથે લોકશાહી સંચાલન
\end{itemize}

\textbf{લાક્ષણિકતાઓ}: સભ્ય સહભાગિતા, નફાની વહેંચણી, સામાજિક હેતુ.
\end{solutionbox}

\begin{mnemonicbox}
\mnemonic{"સામૂહિક માલિકી સાથે લોકશાહી સંચાલન"}
\end{mnemonicbox}

\questionmarks{3(બ)}{4}{મેનેજમેન્ટના કાર્યોની સૂચિ આપો અને તે બધાને વ્યાખ્યાયિત કરો.}

\begin{solutionbox}
\begin{center}
\captionof{table}{મેનેજમેન્ટના કાર્યો}
\begin{tabulary}{\linewidth}{|L|L|L|}
\hline
\textbf{કાર્ય} & \textbf{વ્યાખ્યા} & \textbf{મુખ્ય પ્રવૃત્તિઓ} \\ \hline
\textbf{આયોજન} & ઉદ્દેશ્યો અને વ્યૂહરચનાઓ નક્કી કરવી & લક્ષ્ય નિર્ધારણ \\ \hline
\textbf{સંગઠન} & સંસાધનો અને માળખાંની ગોઠવણી & વિભાગીકરણ \\ \hline
\textbf{સ્ટાફિંગ} & માનવ સંસાધન વ્યવસ્થાપન & ભરતી, તાલીમ \\ \hline
\textbf{દિશા નિર્દેશન} & કર્મચારીઓનું નેતૃત્વ અને પ્રેરણા & નેતૃત્વ, દેખરેખ \\ \hline
\textbf{નિયંત્રણ} & કામગીરીનું નિરીક્ષણ અને સુધારો & કામગીરી માપન \\ \hline
\end{tabulary}
\end{center}
\end{solutionbox}

\begin{mnemonicbox}
\mnemonic{"યોગ્ય સંગઠન સ્ટાફ દિશા નિયંત્રણને સમર્થન આપે છે"}
\end{mnemonicbox}

\questionmarks{3(ક)}{7}{માલિકીના પ્રકારોનું વર્ણન કરો અને કોઈપણ ત્રણને વિગતવાર સમજાવો.}

\begin{solutionbox}
\begin{center}
\captionof{table}{માલિકીના પ્રકારો}
\begin{tabulary}{\linewidth}{|L|L|L|}
\hline
\textbf{પ્રકાર} & \textbf{માલિકી} & \textbf{જવાબદારી} \\ \hline
\textbf{એકલ માલિકી} & એક માલિક & અમર્યાદિત \\ \hline
\textbf{ભાગીદારી} & 2+ ભાગીદારો & અમર્યાદિત \\ \hline
\textbf{કંપની} & શેરધારકો & મર્યાદિત \\ \hline
\end{tabulary}
\end{center}

\textbf{વિગતવાર સમજૂતી:}

\begin{itemize}
    \item \textbf{1. એકલ માલિકી}:
    \begin{itemize}
        \item \textbf{ફાયદાઓ}: સરળ રચના, સંપૂર્ણ નિયંત્રણ, કર લાભો
        \item \keyword{નુકસાનો}: અમર્યાદિત જવાબદારી, મર્યાદિત સંસાધનો
        \item \textbf{અનુકૂળ}: નાના વ્યવસાયો, વ્યાવસાયિક સેવાઓ
    \end{itemize}
    \item \textbf{2. ભાગીદારી}:
    \begin{itemize}
        \item \textbf{ફાયદાઓ}: વહેંચાયેલા સંસાધનો, વિશિષ્ટ કુશળતા
        \item \keyword{નુકસાનો}: અમર્યાદિત જવાબદારી, સંઘર્ષની સંભાવના
        \item \textbf{પ્રકારો}: સામાન્ય ભાગીદારી, મર્યાદિત ભાગીદારી
    \end{itemize}
    \item \textbf{3. કંપની}:
    \begin{itemize}
        \item \textbf{ફાયદાઓ}: મર્યાદિત જવાબદારી, શાશ્વત અસ્તિત્વ
        \item \keyword{નુકસાનો}: જટિલ નિયમો, બેવડો કર, નિયંત્રણ ગુમાવવું
        \item \textbf{પ્રકારો}: ખાનગી મર્યાદિત, જાહેર મર્યાદિત
    \end{itemize}
\end{itemize}
\end{solutionbox}

\begin{mnemonicbox}
\mnemonic{"એકલા ભાગીદારો કંપનીઓ સહકાર કરે છે"}
\end{mnemonicbox}

\questionmarks{3(અ) OR}{3}{વિવિધ લીડરશિપ મોડલ્સ સમજાવો.}

\begin{solutionbox}
\begin{center}
\captionof{table}{નેતૃત્વ મોડલ્સ}
\begin{tabulary}{\linewidth}{|L|L|L|}
\hline
\textbf{મોડેલ} & \textbf{અભિગમ} & \textbf{શ્રેષ્ઠ ઉપયોગ} \\ \hline
\textbf{સ્વૈરાચારી} & નેતા બધા નિર્ણયો લે છે & કટોકટી, ઝડપી નિર્ણયો \\ \hline
\textbf{લોકશાહી} & સહભાગિતાપૂર્ણ નિર્ણય & ટીમ ઇનપુટ મૂલ્યવાન \\ \hline
\textbf{છૂટક હાથ} & હાથ છોડીને અભિગમ & અનુભવી ટીમ \\ \hline
\end{tabulary}
\end{center}
\end{solutionbox}

\begin{mnemonicbox}
\mnemonic{"સ્વૈરાચારી લોકશાહી છૂટક રૂપાંતર વ્યવહાર પરિસ્થિતિ"}
\end{mnemonicbox}

\questionmarks{3(બ) OR}{4}{વહીવટ અને સંચાલન વચ્ચેનો તફાવત આપો}

\begin{solutionbox}
\textbf{મુખ્ય તફાવતો:}
\begin{itemize}
    \item \textbf{વહીવટ}: નીતિ નિર્માણ, ઉચ્ચ સ્તર, આયોજન.
    \item \textbf{સંચાલન}: નીતિ અમલીકરણ, મધ્યમ સ્તર, અમલ.
\end{itemize}
\end{solutionbox}

\begin{mnemonicbox}
\mnemonic{"વહીવટ આયોજન કરે, સંચાલન અમલ કરે"}
\end{mnemonicbox}

\questionmarks{3(ક) OR}{7}{ઉદ્યોગ, વાણિજ્ય અને વ્યવસાય વચ્ચેના તફાવતની કલ્પના સમજાવો.}

\begin{solutionbox}
\begin{center}
\captionof{table}{ઉદ્યોગ, વાણિજ્ય, વ્યવસાય}
\begin{tabulary}{\linewidth}{|L|L|L|}
\hline
\textbf{કલ્પના} & \textbf{વ્યાખ્યા} & \textbf{પ્રાથમિક પ્રવૃત્તિ} \\ \hline
\textbf{ઉદ્યોગ} & માલનું ઉત્પાદન & ઉત્પાદન, પ્રક્રિયા \\ \hline
\textbf{વાણિજ્ય} & માલનું વિતરણ & વેપાર, પરિવહન \\ \hline
\textbf{વ્યવસાય} & એકંદર આર્થિક પ્રવૃત્તિ & ઉત્પાદન + વિતરણ \\ \hline
\end{tabulary}
\end{center}

\begin{center}
\begin{tikzpicture}[node distance=1.5cm, auto]
    \node [gtu block] (A) {વ્યવસાય};
    \node [gtu block, below left=1.5cm of A] (B) {ઉદ્યોગ};
    \node [gtu block, below right=1.5cm of A] (C) {વાણિજ્ય};
    
    \node [gtu block, below=0.8cm of B] (B1) {પ્રાથમિક, દ્વિતીયક, તૃતીયક};
    \node [gtu block, below=0.8cm of C] (C1) {વેપાર, સહાયક};
    
    \draw [gtu arrow] (A) -- (B);
    \draw [gtu arrow] (A) -- (C);
    \draw [gtu arrow] (B) -- (B1);
    \draw [gtu arrow] (C) -- (C1);
\end{tikzpicture}
\captionof{figure}{વ્યવસાય ઘટકો}
\end{center}
\end{solutionbox}

\begin{mnemonicbox}
\mnemonic{"ઉદ્યોગ બનાવે, વાણિજ્ય વિતરિત કરે, વ્યવસાય એકીકૃત કરે"}
\end{mnemonicbox}

\questionmarks{4(અ)}{3}{આ શબ્દો સમજાવો: 1.કરાર 2.કોપીરાઈટ}

\begin{solutionbox}
\begin{center}
\captionof{table}{કરાર vs કોપીરાઈટ}
\begin{tabulary}{\linewidth}{|L|L|L|}
\hline
\textbf{શબ્દ} & \textbf{વ્યાખ્યા} & \textbf{મુખ્ય લક્ષણો} \\ \hline
\textbf{કરાર} & પક્ષો વચ્ચેનો કાનૂની કરાર & બંધનકર્તા, લાગુ કરી શકાય \\ \hline
\textbf{કોપીરાઈટ} & બૌદ્ધિક સંપદા સુરક્ષા & સર્જનાત્મક કાર્યો, વિશેષ અધિકારો \\ \hline
\end{tabulary}
\end{center}
\end{solutionbox}

\begin{mnemonicbox}
\mnemonic{"કરાર બંધે, કોપીરાઈટ સુરક્ષિત કરે"}
\end{mnemonicbox}

\questionmarks{4(બ)}{4}{સ્ટાર્ટઅપ ઇન્ક્યુબેશન સેન્ટર અને મોડાલિટીઝ પર એક નોંધ આપો.}

\begin{solutionbox}
\textbf{સ્ટાર્ટઅપ ઇન્ક્યુબેશન સેન્ટર્સ:} હેતુ સ્ટાર્ટઅપ્સને સહાય કરવાનો છે.

\textbf{મોડાલિટીઝ:}
\begin{itemize}
    \item \textbf{પૂર્વ-ઇન્ક્યુબેશન}: આઈડિયા વેલિડેશન.
    \item \textbf{ઇન્ક્યુબેશન}: બિઝનેસ મોડેલ સુધારણા.
    \item \textbf{પોસ્ટ-ઇન્ક્યુબેશન}: સ્કેલિંગ સહાય.
\end{itemize}
\end{solutionbox}

\begin{mnemonicbox}
\mnemonic{"પૂર્વ-ઇન્ક્યુબેટ, ઇન્ક્યુબેટ, પોસ્ટ-સપોર્ટ સ્ટાર્ટઅપ્સ"}
\end{mnemonicbox}

\questionmarks{4(ક)}{7}{રાજ્ય સ્તરની એજન્સીઓની યાદી બનાવો જે સ્ટાર્ટ-અપ્સને સમર્થન આપે છે અને તેમની કાર્યક્ષમતાનું વર્ણન કરો}

\begin{solutionbox}
\textbf{ગુજરાત રાજ્ય સહાય એજન્સીઓ:}

\begin{center}
\begin{tabulary}{\linewidth}{|L|L|L|}
\hline
\textbf{એજન્સી} & \textbf{સંપૂર્ણ નામ} & \textbf{મુખ્ય કાર્યો} \\ \hline
\textbf{SSIP} & Student Startup Policy & વિદ્યાર્થી ઉદ્યોગસાહસિક સહાય \\ \hline
\textbf{iHub} & Innovation Hub & ઇન્ક્યુબેશન, માર્ગદર્શન \\ \hline
\textbf{GUSEC} & Gujarat University Council & યુનિવર્સિટી સ્તરે પ્રમોશન \\ \hline
\textbf{GIDC} & Industrial Development Corp & ઇન્ફ્રાસ્ટ્રક્ચર, જમીન \\ \hline
\end{tabulary}
\end{center}



\textbf{વિગતવાર કાર્યક્ષમતાઓ:}

\textbf{SSIP Gujarat:}
\begin{itemize}
    \item \textbf{ભંડોળ સહાય}: વિદ્યાર્થી સ્ટાર્ટઅપ્સ માટે ₹2 લાખ સુધી
    \item \textbf{ઇન્ક્યુબેશન સુવિધાઓ}: કાર્યક્ષેત્ર અને સાધનોની પહોંચ
    \item \textbf{માર્ગદર્શન કાર્યક્રમો}: ઉદ્યોગ નિષ્ણાત માર્ગદર્શન
    \item \textbf{IPR સહાય}: પેટન્ટ ફાઇલિંગ સહાયતા
\end{itemize}

\textbf{iHub Gujarat:}
\begin{itemize}
    \item \textbf{સ્ટાર્ટઅપ ઇકોસિસ્ટમ}: સંપૂર્ણ ઉદ્યોગસાહસિકતા સહાય
    \item \textbf{ટેકનોલોજી ટ્રાન્સફર}: સંશોધનથી બજાર તરફ રૂપાંતરણ
    \item \textbf{રોકાણકાર જોડાણો}: ભંડોળ સુવિધા
    \item \textbf{ઉદ્યોગ ભાગીદારી}: કોર્પોરેટ સહયોગ
\end{itemize}

\textbf{GUSEC:}
\begin{itemize}
    \item \textbf{વિદ્યાર્થી સંડોવણી}: કેમ્પસ ઉદ્યોગસાહસિકતા કાર્યક્રમો
    \item \textbf{કુશળતા વિકાસ}: ઉદ્યોગસાહસિકતા શિક્ષણ
    \item \textbf{સ્પર્ધા આયોજન}: સ્ટાર્ટઅપ હરિફાઈ અને પિચ
    \item \textbf{નેટવર્ક બિલ્ડિંગ}: એલ્યુમ્નાઈ ઉદ્યોગસાહસિક જોડાણો
\end{itemize}

\begin{center}
\begin{tikzpicture}[node distance=1.5cm]
  \node [gtu block] (root) {રાજ્ય સ્ટાર્ટઅપ સહાય};
  \node [gtu block, below left=1.5cm of root] (ssip) {SSIP};
  \node [gtu block, below right=0.5cm and 1cm of root] (ihub) {iHub};
  \node [gtu block, below left=0.5cm and 1cm of root] (gusec) {GUSEC};
  \node [gtu block, below right=1.5cm of root] (gidc) {GIDC};
  
  \draw [gtu arrow] (root) -- (ssip);
  \draw [gtu arrow] (root) -- (ihub);
  \draw [gtu arrow] (root) -- (gusec);
  \draw [gtu arrow] (root) -- (gidc);
\end{tikzpicture}
\captionof{figure}{રાજ્ય એજન્સીઓ}
\end{center}

\textbf{પ્રભાવ માપદંડ:}
\begin{itemize}
    \item \textbf{વાર્ષિક સમર્થિત સ્ટાર્ટઅપ્સની સંખ્યા}
    \item \textbf{સમર્થિત સાહસો દ્વારા રોજગાર સર્જન}
    \item \textbf{ઇન્ક્યુબેટેડ કંપનીઓનું આવક ઉત્પાદન}
    \item \textbf{સ્નાતક સ્ટાર્ટઅપ્સની સફળતા દર}
\end{itemize}
\end{solutionbox}

\begin{mnemonicbox}
\mnemonic{"SSIP iHub GUSEC GIDC ગુજરાત સ્ટાર્ટઅપ્સને સપોર્ટ કરે"}
\end{mnemonicbox}

\questionmarks{4(અ) OR}{3}{આ શબ્દો સમજાવો: 1.IPR 2.ટ્રેડમાર્ક્સ}

\begin{solutionbox}
\begin{center}
\captionof{table}{IPR vs ટ્રેડમાર્ક્સ}
\begin{tabulary}{\linewidth}{|L|L|L|}
\hline
\textbf{શબ્દ} & \textbf{વ્યાખ્યા} & \textbf{સુરક્ષા અવકાશ} \\ \hline
\textbf{IPR} & બૌદ્ધિક સંપદા અધિકારો & વિચારો, શોધો \\ \hline
\textbf{ટ્રેડમાર્ક્સ} & બ્રાન્ડ ઓળખ ચિહ્નો & નામો, લોગો \\ \hline
\end{tabulary}
\end{center}


\textbf{IPR વર્ગો:}
\begin{itemize}
    \item \textbf{પેટન્ટ્સ}: તકનીકી શોધો (20 વર્ષ)
    \item \textbf{કોપીરાઈટ્સ}: સર્જનાત્મક અભિવ્યક્તિઓ (જીવનકાળ + 70 વર્ષ)
    \item \textbf{ટ્રેડમાર્ક્સ}: બ્રાન્ડ ઓળખકર્તા (10 વર્ષ, નવીકરણ યોગ્ય)
\end{itemize}

\textbf{ટ્રેડમાર્કની લક્ષણો:}
\begin{itemize}
    \item \textbf{વિશિષ્ટતા}: અનન્ય બ્રાન્ડ ઓળખ
    \item \textbf{વ્યાપારિક ઉપયોગ}: વ્યાપારિક ઓળખાણનો હેતુ
    \item \textbf{નોંધણી}: નોંધણી દ્વારા કાનૂની સુરક્ષા
\end{itemize}
\end{solutionbox}

\begin{mnemonicbox}
\mnemonic{"IPR સુરક્ષિત કરે, ટ્રેડમાર્ક્સ ઓળખે"}
\end{mnemonicbox}

\questionmarks{4(બ) OR}{4}{સ્ટાર્ટ-અપમાં રોકાણકારની ભૂમિકા વ્યાખ્યાયિત કરો.}

\begin{solutionbox}
\textbf{રોકાણકારની ભૂમિકાઓ:}

\begin{itemize}
    \item \textbf{નાણાકીય સહાય}: સીડ ફંડિંગ, ગ્રોથ કેપિટલ
    \item \textbf{વ્યૂહાત્મક માર્ગદર્શન}: નેટવર્ક એક્સેસ, બજાર સૂઝ
    \item \textbf{કામકાજી સહાય}: ટીમ બિલ્ડિંગ, કાનૂની અનુપાલન
    \item \textbf{જોખમ વ્યવસ્થાપન}: ડ્યુ ડિલિજન્સ, પ્રદર્શન મોનિટરિંગ
\end{itemize}

\textbf{રોકાણકારોના પ્રકારો}: એન્જલ ઇન્વેસ્ટર્સ, વેન્ચર કેપિટલ, કોર્પોરેટ ઇન્વેસ્ટર્સ.
\end{solutionbox}

\begin{mnemonicbox}
\mnemonic{"નાણા વ્યૂહરચના કામકાજ જોખમ = રોકાણકાર ભૂમિકાઓ"}
\end{mnemonicbox}

\questionmarks{4(ક) OR}{7}{રાષ્ટ્રીય સ્તરની એજન્સીઓની યાદી બનાવો જે સ્ટાર્ટ-અપ્સને સમર્થન આપે છે અને તેમની કાર્યક્ષમતાનું વર્ણન કરો.}

\begin{solutionbox}
\textbf{રાષ્ટ્રીય સ્ટાર્ટઅપ સહાય એજન્સીઓ:}

\begin{center}
\begin{tabulary}{\linewidth}{|L|L|L|}
\hline
\textbf{એજન્સી} & \textbf{વિભાગ} & \textbf{પ્રાથમિક ફોકસ} \\ \hline
\textbf{Startup India} & DPIIT & નીતિ અને ઇકોસિસ્ટમ \\ \hline
\textbf{BIRAC} & બાયોટેકનોલોજી & બાયોટેક નવીનતા \\ \hline
\textbf{TDB} & વિજ્ઞાન અને ટેક & ટેકનોલોજી વિકાસ \\ \hline
\textbf{SIDBI} & નાણાકીય સેવાઓ & MSME ભંડોળ \\ \hline
\end{tabulary}
\end{center}

\textbf{વિગતવાર કાર્યક્ષમતાઓ:}

\textbf{Startup India:}
\begin{itemize}
    \item \textbf{નીતિ ઘડતર}: રાષ્ટ્રીય સ્ટાર્ટઅપ નીતિ ફ્રેમવર્ક
    \item \textbf{માન્યતા કાર્યક્રમ}: સત્તાવાર સ્ટાર્ટઅપ પ્રમાણપત્ર
    \item \textbf{કર લાભો}: પાત્ર સ્ટાર્ટઅપ્સ માટે 3-વર્ષની કર મુક્તિ
    \item \textbf{નિયમનકારી સહાય}: સિંગલ-પોઇન્ટ ક્લિયરન્સ સિસ્ટમ
    \item \textbf{ભંડોળ સુવિધા}: Fund of Funds યોજના (₹10,000 કરોડ)
\end{itemize}

\textbf{BIRAC (Biotechnology Industry Research Assistance Council):}
\begin{itemize}
    \item \textbf{બાયોટેક નવીનતા}: બાયોટેક સ્ટાર્ટઅપ્સ અને સંશોધનને સમર્થન
    \item \textbf{ભંડોળ યોજનાઓ}: SBIRI, SPARSH, BIG કાર્યક્રમો
    \item \textbf{ઉદ્યોગ ભાગીદારી}: શિક્ષણ-ઉદ્યોગ સહયોગ
    \item \textbf{ટેકનોલોજી અનુવાદ}: સંશોધનથી બજાર તરફ રૂપાંતરણ
\end{itemize}

\textbf{TDB (Technology Development Board):}
\begin{itemize}
    \item \textbf{ટેકનોલોજી કોમર્શિયલાઇઝેશન}: સંશોધનને ઉત્પાદનોમાં રૂપાંતરિત કરવું
    \item \textbf{નાણાકીય સહાય}: ટેકનોલોજી વિકાસ માટે લોન અને ગ્રાન્ટ
    \item \textbf{ઉદ્યોગ સહાય}: ઉત્પાદન ટેકનોલોજી સહાયતા
    \item \textbf{નવીનતા પ્રમોશન}: તકનીકી નવીનતાને સમર્થન
\end{itemize}

\textbf{SIDBI (Small Industries Development Bank of India):}
\begin{itemize}
    \item \textbf{નાણાકીય સહાય}: લોન અને ક્રેડિટ સુવિધાઓ
    \item \textbf{MSME ફોકસ}: નાના અને મધ્યમ ઉદ્યોગ વિકાસ
    \item \textbf{સ્ટાર્ટઅપ ભંડોળ}: વેન્ચર કેપિટલ અને ગ્રોથ કેપિટલ
    \item \textbf{ઇકોસિસ્ટમ વિકાસ}: ઇન્ક્યુબેટર અને એક્સેલેરેટર સહાય
\end{itemize}

\begin{center}
\begin{tikzpicture}[node distance=1.5cm]
  \node [gtu block] (root) {રાષ્ટ્રીય સ્ટાર્ટઅપ સહાય};
  \node [gtu block, below left=1.5cm of root] (si) {Startup India};
  \node [gtu block, below right=0.5cm and 1cm of root] (birac) {BIRAC};
  \node [gtu block, below left=0.5cm and 1cm of root] (tdb) {TDB};
  \node [gtu block, below right=1.5cm of root] (sidbi) {SIDBI};
  
  \draw [gtu arrow] (root) -- (si);
  \draw [gtu arrow] (root) -- (birac);
  \draw [gtu arrow] (root) -- (tdb);
  \draw [gtu arrow] (root) -- (sidbi);
\end{tikzpicture}
\captionof{figure}{રાષ્ટ્રીય એજન્સીઓ}
\end{center}

\textbf{સફળતાના મેટ્રિક્સ:}
\begin{itemize}
    \item \textbf{સ્ટાર્ટઅપ નોંધણીઓ}: 70,000+ માન્યતા પ્રાપ્ત સ્ટાર્ટઅપ્સ
    \item \textbf{રોજગાર સર્જન}: લાખો રોજગારની તકો
    \item \textbf{ભંડોળ સુવિધા}: અબજો રોકાણ એકત્રીકરણ
    \item \textbf{ઇકોસિસ્ટમ વિકાસ}: હજારો ઇન્ક્યુબેટર અને એક્સેલેરેટર
\end{itemize}
\end{solutionbox}

\begin{mnemonicbox}
\mnemonic{"Startup BIRAC TDB SIDBI = રાષ્ટ્રીય સહાય પ્રણાલી"}
\end{mnemonicbox}

\questionmarks{5(અ)}{3}{આ શરતો સમજાવો: 1.બ્રેક ઇવન પોઇન્ટ 2.રોકાણ પર વળતર 3.વેચાણ પર વળતર.}

\begin{solutionbox}
\begin{center}
\captionof{table}{નાણાકીય શરતો}
\begin{tabulary}{\linewidth}{|L|L|L|}
\hline
\textbf{શરત} & \textbf{ફોર્મુલા} & \textbf{અર્થ} \\ \hline
\textbf{BEP} & ખર્ચ / માર્જિન & ખર્ચ આવરવા માટેના એકમો \\ \hline
\textbf{ROI} & (લાભ-ખર્ચ) / ખર્ચ * 100 & મૂડી પર વળતર \\ \hline
\textbf{ROS} & આવક / વેચાણ * 100 & નફાનું માર્જિન \\ \hline
\end{tabulary}
\end{center}


\textbf{બ્રેક ઇવન વિશ્લેષણ:}
\begin{itemize}
    \item \textbf{નિશ્ચિત ખર્ચ}: ભાડું, પગાર, વીમો
    \item \textbf{ચલ ખર્ચ}: કાચો માલ, એકમ દીઠ ઉપયોગિતાઓ
    \item \textbf{યોગદાન માર્જિન}: એકમ દીઠ કિંમત માઈનસ ચલ ખર્ચ
\end{itemize}

\textbf{ROI મહત્વ:}
\begin{itemize}
    \item \textbf{રોકાણ કાર્યક્ષમતા}: રોકાણની કામગીરી માપે છે
    \item \textbf{તુલના સાધન}: વિવિધ રોકાણ વિકલ્પોની તુલના
    \item \textbf{નિર્ણય લેવું}: ભાવિ રોકાણના નિર્ણયોનું માર્ગદર્શન
\end{itemize}

\textbf{ROS મહત્વ:}
\begin{itemize}
    \item \textbf{નફાકારકતાનું માપદંડ}: કામકાજની કાર્યક્ષમતા દર્શાવે છે
    \item \textbf{ઉદ્યોગ તુલના}: સ્પર્ધકો સાથે બેન્ચમાર્ક
    \item \textbf{વલણ વિશ્લેષણ}: સમય સાથે કામગીરી ટ્રેક કરવું
\end{itemize}
\end{solutionbox}

\begin{mnemonicbox}
\mnemonic{"બ્રેક ઇવન રોકાણ વેચાણ પર વળતર"}
\end{mnemonicbox}

\questionmarks{5(બ)}{4}{આયાત-નિકાસ નીતિ પર ટૂંકી નોંધ લખો}

\begin{solutionbox}
\textbf{ભારતની EXIM નીતિ:}
\begin{itemize}
    \item \textbf{ઉદ્દેશ્યો}: વેપાર પ્રમોશન, નિકાસ વૃદ્ધિ, આર્થિક વિકાસ.
    \item \textbf{નિકાસ પ્રમોશન}:
    \begin{itemize}
        \item \keyword{નિકાસ પ્રોત્સાહનો}: ડ્યુટી ડ્રોબેક, MEIS યોજનાઓ
        \item \keyword{વિશેષ આર્થિક ઝોન}: કરમુક્ત નિકાસ ઉત્પાદન
    \end{itemize}
    \item \textbf{આયાત વ્યવસ્થાપન}:
    \begin{itemize}
        \item \keyword{આયાત લાઇસન્સિંગ}: સંવેદનશીલ ઉદ્યોગો માટે
        \item \keyword{ડ્યુટી સ્ટ્રક્ચર}: ટેરિફ દરો
        \item \keyword{ગુણવત્તા ધોરણો}: BIS અને અન્ય
    \end{itemize}
    \item \textbf{વર્તમાન ફોકસ}: મેક ઇન ઇન્ડિયા, ડિજિટલ ઇન્ડિયા, આત્મનિર્ભર ભારત.
\end{itemize}
\end{solutionbox}

\begin{mnemonicbox}
\mnemonic{"નિકાસ આયાત નીતિ વેપાર સુવિધાને પ્રોત્સાહન આપે છે"}
\end{mnemonicbox}

\questionmarks{5(ક)}{7}{CSR અને આર્થિક કામગીરી વચ્ચેના જોડાણનું વર્ણન કરો.}

\begin{solutionbox}
\textbf{પ્રત્યક્ષ આર્થિક લાભો:}
\begin{center}
\captionof{table}{પ્રત્યક્ષ લાભો}
\begin{tabulary}{\linewidth}{|L|L|L|}
\hline
\textbf{CSR પ્રવૃત્તિ} & \textbf{આર્થિક અસર} & \textbf{માપદંડ} \\ \hline
\textbf{કર્મચારી કલ્યાણ} & ઊંચી ઉત્પાદકતા & ખર્ચ બચત \\ \hline
\textbf{પર્યાવરણ પહેલ} & સંસાધન કાર્યક્ષમતા & ખર્ચ ઘટાડો \\ \hline
\textbf{સમુદાય વિકાસ} & બજાર વિસ્તરણ & આવક વૃદ્ધિ \\ \hline
\end{tabulary}
\end{center}

\textbf{પરોક્ષ આર્થિક લાભો:}
\begin{itemize}
    \item \textbf{બ્રાન્ડ વેલ્યુ}: ગ્રાહક વફાદારી, પ્રીમિયમ પ્રાઇસિંગ, બજાર ભિન્નતા.
    \item \textbf{જોખમ વ્યવસ્થાપન}: નિયમનકારી અનુપાલન, પ્રતિષ્ઠા સુરક્ષા, હિતધારક સંબંધો.
\end{itemize}

\textbf{લાંબા ગાળાની આર્થિક કામગીરી:}
\begin{itemize}
    \item \textbf{ટકાઉ વૃદ્ધિ}: નવીનતાનું ચાલક, બજાર પ્રવેશ, રોકાણ આકર્ષણ.
\end{itemize}


\begin{center}
\begin{tikzpicture}[node distance=1.5cm, auto]
    \node [gtu block] (A) {CSR પ્રવૃત્તિઓ};
    \node [gtu block, below left=1.0cm of A] (B) {પ્રત્યક્ષ લાભો};
    \node [gtu block, below=1.0cm of A] (C) {પરોક્ષ લાભો};
    \node [gtu block, below right=1.0cm of A] (D) {લાંબા ગાળાની અસર};
    
    \node [gtu block, below=2.5cm of A] (E) {આર્થિક કામગીરી};
    
    \draw [gtu arrow] (A) -- (B);
    \draw [gtu arrow] (A) -- (C);
    \draw [gtu arrow] (A) -- (D);
    \draw [gtu arrow] (B) -- (E);
    \draw [gtu arrow] (C) -- (E);
    \draw [gtu arrow] (D) -- (E);
\end{tikzpicture}
\captionof{figure}{CSR અને પ્રદર્શન}
\end{center}
\end{solutionbox}

\begin{mnemonicbox}
\mnemonic{"CSR ટકાઉ વળતર બનાવે છે"}
\end{mnemonicbox}

\questionmarks{5(અ) OR}{3}{નાદારી અને અવગણના પર એક નોંધ લખો.}

\begin{solutionbox}
\textbf{નાદારી}: કાનૂની પ્રક્રિયા જ્યારે વ્યવસાય નાણાકીય જવાબદારીઓ પૂરી કરી શકતો નથી.

\textbf{અવગણવાની વ્યૂહરચનાઓ}:
\begin{itemize}
    \item \keyword{રોકડ પ્રવાહ વ્યવસ્થાપન}: કાર્યકારી મૂડી જાળવવી
    \item \keyword{દેવું પુનઃરચના}: ચુકવણીની શરતોની વાટાઘાટ
    \item \keyword{ખર્ચ ઘટાડો}: બિનજરૂરી ખર્ચ કાપવો અને કાર્યક્ષમતા સુધારવી
\end{itemize}

\textbf{કાનૂની ફ્રેમવર્ક}: નાદારી અને નાદારી કોડ (IBC).
\end{solutionbox}

\begin{mnemonicbox}
\mnemonic{"નાદાર વ્યવસાયો રોકડ નિયંત્રણ દ્વારા ટાળે છે"}
\end{mnemonicbox}

\questionmarks{5(બ) OR}{4}{બિઝનેસ એથિક્સનું મહત્વ લખો}

\begin{solutionbox}
\textbf{બિઝનેસ એથિક્સનું મહત્વ:}

\begin{itemize}
    \item \textbf{હિતધારક વિશ્વાસ}: ગ્રાહક વિશ્વાસ, રોકાણકાર શ્રદ્ધા, કર્મચારી સંતુષ્ટિ.
    \item \textbf{કાનૂની અનુપાલન}: નિયમનકારી પાલન, જોખમ ઘટાડો, પ્રતિષ્ઠા સુરક્ષા.
    \item \textbf{સ્પર્ધાત્મક લાભ}: બજાર ભિન્નતા, પ્રીમિયમ પોઝિશનિંગ, ટકાઉ વૃદ્ધિ.
    \item \textbf{સામાજિક અસર}: સમુદાય વિકાસ, પર્યાવરણ જવાબદારી.
\end{itemize}
\end{solutionbox}

\begin{mnemonicbox}
\mnemonic{"નૈતિકતા વિશ્વાસ, અનુપાલન, લાભ, સામાજિક અસર બનાવે છે"}
\end{mnemonicbox}

\questionmarks{5(ક) OR}{7}{પ્રોજેક્ટ રિપોર્ટ લેખનના પગલાં અને ફોર્મેટ આપો}

\begin{solutionbox}
\textbf{પ્રોજેક્ટ રિપોર્ટ લેખનના પગલાં:}

\textbf{લેખન પૂર્વે તબક્કો:}
\begin{enumerate}
    \item \textbf{પ્રોજેક્ટ આયોજન}: અવકાશ, ઉદ્દેશ્યો અને ડિલિવરેબલ્સ
    \item \textbf{ડેટા સંગ્રહ}: સંબંધિત માહિતી અને સંશોધન
    \item \textbf{વિશ્લેષણ}: એકત્રિત કરેલા ડેટાને પ્રક્રિયા
    \item \textbf{માળખું આયોજન}: સામગ્રીને તાર્કિક રીતે ગોઠવવી
\end{enumerate}

\textbf{લેખન તબક્કો:}
\begin{enumerate}
    \setcounter{enumi}{4}
    \item \textbf{ડ્રાફ્ટ તૈયારી}: ફોર્મેટ અનુસાર પ્રારંભિક વર્ઝન
    \item \textbf{સામગ્રી વિકાસ}: વિગતો સાથે વિસ્તરણ
    \item \textbf{સમીક્ષા અને સુધારણા}: ચોકસાઈ માટે તપાસ
    \item \textbf{અંતિમ ફોર્મેટિંગ}: શૈલી લાગુ કરવી
\end{enumerate}

\textbf{પ્રોજેક્ટ રિપોર્ટ ફોર્મેટ:}

\begin{center}
\begin{tikzpicture}[node distance=0.6cm]
  \node [gtu block] (title) {1. શીર્ષક પૃષ્ઠ};
  \node [gtu block, below=of title] (exec) {2. કાર્યકારી સારાંશ};
  \node [gtu block, below=of exec] (intro) {3. પરિચય};
  \node [gtu block, below=of intro] (lit) {4. સાહિત્ય સમીક્ષા};
  \node [gtu block, below=of lit] (method) {5. પદ્ધતિશાસ્ત્ર};
  \node [gtu block, below=of method] (analysis) {6. વિશ્લેષણ અને તારણો};
  \node [gtu block, below=of analysis] (rec) {7. ભલામણો};
  \node [gtu block, below=of rec] (conc) {8. નિષ્કર્ષ};
  \node [gtu block, below=of conc] (ref) {9. સંદર્ભો};
  
  \draw [gtu arrow] (title) -- (exec);
  \draw [gtu arrow] (exec) -- (intro);
  \draw [gtu arrow] (intro) -- (lit);
  \draw [gtu arrow] (lit) -- (method);
  \draw [gtu arrow] (method) -- (analysis);
  \draw [gtu arrow] (analysis) -- (rec);
  \draw [gtu arrow] (rec) -- (conc);
  \draw [gtu arrow] (conc) -- (ref);
\end{tikzpicture}
\captionof{figure}{રિપોર્ટ માળખું}
\end{center}

\begin{center}
\captionof{table}{રિપોર્ટ ફોર્મેટ}
\begin{tabulary}{\linewidth}{|L|L|}
\hline
\textbf{1. શીર્ષક પૃષ્ઠ} & શીર્ષક, લેખક, સંસ્થા, તારીખ \\ \hline
\textbf{2. કાર્યકારી સારાંશ} & ઝાંખી, તારણો, પરિણામો \\ \hline
\textbf{3. સામગ્રી સૂચિ} & શીર્ષકો, પૃષ્ઠ નંબરો \\ \hline
\textbf{4. પરિચય} & પૃષ્ઠભૂમિ, સમસ્યા, ઉદ્દેશ્યો \\ \hline
\textbf{5. સાહિત્ય સમીક્ષા} & સંશોધન, ગેપ વિશ્લેષણ \\ \hline
\textbf{6. પદ્ધતિશાસ્ત્ર} & અભિગમ, ડેટા, વિશ્લેષણ \\ \hline
\textbf{7. વિશ્લેષણ અને તારણો} & ડેટા પ્રસ્તુતિ, પરિણામો \\ \hline
\textbf{8. ભલામણો} & સૂચનો, અમલીકરણ \\ \hline
\textbf{9. નિષ્કર્ષ} & સારાંશ, સિદ્ધિઓ \\ \hline
\textbf{10. સંદર્ભો} & ગ્રંથસૂચિ, પરિશિષ્ટો \\ \hline
\end{tabulary}
\end{center}

\begin{center}
\begin{tikzpicture}[node distance=0.6cm]
  \node [gtu block] (title) {1. શીર્ષક પૃષ્ઠ};
  \node [gtu block, below=of title] (exec) {2. કાર્યકારી સારાંશ};
  \node [gtu block, below=of exec] (intro) {3. પરિચય};
  \node [gtu block, below=of intro] (lit) {4. સાહિત્ય સમીક્ષા};
  \node [gtu block, below=of lit] (method) {5. પદ્ધતિશાસ્ત્ર};
  \node [gtu block, below=of method] (analysis) {6. વિશ્લેષણ અને તારણો};
  \node [gtu block, below=of analysis] (rec) {7. ભલામણો};
  \node [gtu block, below=of rec] (conc) {8. નિષ્કર્ષ};
  \node [gtu block, below=of conc] (ref) {9. સંદર્ભો};
  
  \draw [gtu arrow] (title) -- (exec);
  \draw [gtu arrow] (exec) -- (intro);
  \draw [gtu arrow] (intro) -- (lit);
  \draw [gtu arrow] (lit) -- (method);
  \draw [gtu arrow] (method) -- (analysis);
  \draw [gtu arrow] (analysis) -- (rec);
  \draw [gtu arrow] (rec) -- (conc);
  \draw [gtu arrow] (conc) -- (ref);
\end{tikzpicture}
\captionof{figure}{રિપોર્ટ માળખું}
\end{center}

\textbf{ગુણવત્તા ચેકલિસ્ટ:}
\begin{itemize}
    \item \textbf{સંપૂર્ણતા}: બધા વિભાગો સામેલ
    \item \textbf{સુસંગતતા}: એકસમાન ફોર્મેટિંગ
    \item \textbf{ચોકસાઈ}: તથ્યો ચકાસાયેલા
    \item \textbf{સંબંધિતતા}: ઉદ્દેશ્યો સાથે સંરેખિત
\end{itemize}

\end{solutionbox}

\begin{mnemonicbox}
\mnemonic{"શીર્ષક કાર્યકારી પરિચય સાહિત્ય પદ્ધતિ વિશ્લેષણ ભલામણો નિષ્કર્ષ સંદર્ભો"}
\end{mnemonicbox}

\end{document}
