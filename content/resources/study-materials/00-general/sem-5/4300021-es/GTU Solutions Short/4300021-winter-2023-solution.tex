\documentclass{article}

% content/resources/templates/preamble.tex
\usepackage[margin=0.6in]{geometry}
\author{Milav Dabgar}
\usepackage{amsmath,amssymb,amsthm}
\usepackage{booktabs}
\usepackage{multirow}
\usepackage{xcolor}
\usepackage{tcolorbox}
\tcbuselibrary{breakable,skins}
\usepackage[colorlinks=true,linkcolor=blue]{hyperref}
\usepackage{titlesec}
\usepackage{enumitem}
\usepackage{tikz}
\usepackage{pgfplots}
\usepackage{circuitikz}
\usepackage[version=4]{mhchem}
\usepackage{longtable}
\usepackage{array}
\usepackage{float}
\usepackage{caption}
\usepackage{listings}

\lstset{
  basicstyle=\small\ttfamily,
  breaklines=true,
  breakatwhitespace=false,
  postbreak=\mbox{\textcolor{red}{$\hookrightarrow$}\space},
  float=false,
  numbers=left,
  numberstyle=\tiny\color{gray},
  numbersep=10pt,
  xleftmargin=2em,
  keywordstyle=\color{blue},
  commentstyle=\color{green!60!black},
  stringstyle=\color{purple},
  backgroundcolor=\color{gray!5},
  showstringspaces=false,
  tabsize=2,
  captionpos=b,
  keepspaces=true,
  columns=flexible
}

\pgfplotsset{compat=1.18}
\usetikzlibrary{shapes,arrows,positioning,calc,patterns,decorations.pathmorphing,decorations.markings,arrows.meta}

% Color scheme
\definecolor{headcolor}{RGB}{0,102,204}
\definecolor{keycolor}{RGB}{220,20,60}
\definecolor{solutioncolor}{RGB}{34,139,34}
\definecolor{mnemoniccolor}{RGB}{148,0,211}
\definecolor{codecolor}{RGB}{0,0,100}

% Spacing
\setlength{\parskip}{3pt}
\setlist[itemize]{nosep}
\setlist[enumerate]{nosep}

% Title formatting
\titleformat{\section}{\Large\bfseries\color{headcolor}}{\thesection}{1em}{}
\titleformat{\subsection}{\large\bfseries\color{headcolor}}{\thesubsection}{1em}{}

% Pandoc tightlist compatibility
\providecommand{\tightlist}{%
  \setlength{\itemsep}{0pt}\setlength{\parskip}{0pt}}

% Pandoc longtable compatibility
\newcounter{none}
\def\thenone{}


% content/resources/templates/english-boxes.tex

% Custom environments
\newtcolorbox{solutionbox}{
 breakable,
 enhanced,
 colback=solutioncolor!5!white,
 colframe=solutioncolor!75!black,
 fonttitle=\bfseries,
 title=Solution
}

\newtcolorbox{solutionboxnobreak}{
 colback=solutioncolor!5!white,
 colframe=solutioncolor!75!black,
 fonttitle=\bfseries,
 title=Solution
}

\newtcolorbox{keyformula}{
 breakable,
 enhanced,
 colback=keycolor!5!white,
 colframe=keycolor!75!black,
 fonttitle=\bfseries,
 title=Key Formula
}

\newtcolorbox{mnemonicboxenv}{
 breakable,
 enhanced,
 colback=mnemoniccolor!5!white,
 colframe=mnemoniccolor!75!black,
 fonttitle=\bfseries,
 title=Mnemonic
}

\newcommand{\mnemonicbox}[1]{%
  \begin{mnemonicboxenv}
    #1
  \end{mnemonicboxenv}
}


% Custom commands for GTU solutions
% This file defines semantic commands for consistent formatting

% Question command with automatic formatting
\newcommand{\question}[2]{%
  \section*{Question #1}%
  \textbf{#2}%
}

% OR question variant
\newcommand{\questionor}[2]{%
  \section*{Question #1 OR}%
  \textbf{#2}%
}

% Proper table environment with caption
\newenvironment{answertable}[1]{%
  \begin{table}[htbp]
  \centering
  \caption{#1}
}{%
  \end{table}
}

% Proper figure environment for diagrams
\newenvironment{answerdiagram}[1]{%
  \begin{figure}[htbp]
  \centering
  \caption{#1}
}{%
  \end{figure}
}

% Semantic markup for key terms
\newcommand{\keyword}[1]{\textbf{#1}}
\newcommand{\code}[1]{\texttt{#1}}
\newcommand{\classname}[1]{\texttt{#1}}
\newcommand{\methodname}[1]{\texttt{#1}}

% Proper quotation marks
\newcommand{\mnemonic}[1]{``#1''}


\title{Entrepreneurship and Start-ups (4300021) - Winter 2023 Solution}
\date{December 01, 2023}

\begin{document}
\maketitle

\questionmarks{1(a)}{3}{Give Comparison between Entrepreneurship and Intrapreneurship.}

\begin{solutionbox}
\begin{center}
\captionof{table}{Entrepreneurship vs Intrapreneurship}
\begin{tabulary}{\linewidth}{|L|L|L|}
\hline
\textbf{Aspect} & \textbf{Entrepreneurship} & \textbf{Intrapreneurship} \\ \hline
\textbf{Definition} & Starting own business with personal risk & Innovation within existing organization \\ \hline
\textbf{Risk} & Personal financial risk & Organization bears risk \\ \hline
\textbf{Resources} & Own/borrowed resources & Company provides resources \\ \hline
\end{tabulary}
\end{center}
\end{solutionbox}

\begin{mnemonicbox}
\mnemonic{"EXternal vs INternal innovation"}
\end{mnemonicbox}

\questionmarks{1(b)}{4}{Discuss characteristics and functions of Entrepreneurship}

\begin{solutionbox}
\textbf{Characteristics:}

\begin{itemize}
    \item \keyword{Risk-taking ability}: Willingness to take calculated business risks
    \item \keyword{Innovation}: Creating new products, services, or processes
    \item \keyword{Leadership skills}: Ability to guide and motivate teams
\end{itemize}

\textbf{Functions:}

\begin{itemize}
    \item \keyword{Job Creation}: Generates employment opportunities for society
    \item \keyword{Economic Development}: Contributes to GDP and national growth
    \item \keyword{Innovation catalyst}: Introduces new technologies and solutions
\end{itemize}
\end{solutionbox}

\begin{mnemonicbox}
\mnemonic{"RIL Creates Jobs Economically \& Innovatively"}
\end{mnemonicbox}

\questionmarks{1(c)}{7}{Identify and discuss 7-M Resources in detail.}

\begin{solutionbox}
\begin{center}
\captionof{table}{7-M Resources}
\begin{tabulary}{\linewidth}{|L|L|L|}
\hline
\textbf{Resource} & \textbf{Description} & \textbf{Importance} \\ \hline
\textbf{Man} & Human resources and workforce & Core asset for operations \\ \hline
\textbf{Money} & Financial capital and funding & Essential for business operations \\ \hline
\textbf{Material} & Raw materials and supplies & Production requirements \\ \hline
\textbf{Machine} & Equipment and technology & Operational efficiency \\ \hline
\textbf{Method} & Processes and procedures & Systematic approach \\ \hline
\textbf{Market} & Customer base and demand & Revenue generation \\ \hline
\textbf{Management} & Planning and coordination & Overall business control \\ \hline
\end{tabulary}
\end{center}

\begin{center}
\begin{tikzpicture}[node distance=2cm]
  \node [gtu block, minimum size=1.5cm] (root) {7-M Resources};
  
  \node [gtu block, above=of root] (man) {Man};
  \node [gtu block, above right=of root] (money) {Money};
  \node [gtu block, right=of root] (material) {Material};
  \node [gtu block, below right=of root] (machine) {Machine};
  \node [gtu block, below=of root] (method) {Method};
  \node [gtu block, below left=of root] (market) {Market};
  \node [gtu block, left=of root] (management) {Management};
  
  \draw [gtu arrow] (root) -- (man);
  \draw [gtu arrow] (root) -- (money);
  \draw [gtu arrow] (root) -- (material);
  \draw [gtu arrow] (root) -- (machine);
  \draw [gtu arrow] (root) -- (method);
  \draw [gtu arrow] (root) -- (market);
  \draw [gtu arrow] (root) -- (management);
\end{tikzpicture}
\captionof{figure}{7-M Resources Mindmap}
\end{center}
\end{solutionbox}

\begin{mnemonicbox}
\mnemonic{"Many Modern Managers Make Money Managing Markets"}
\end{mnemonicbox}

\questionmarks{1(c) OR}{7}{Write down the Start Up India Registration process.}

\begin{solutionbox}
\textbf{Start-up India Registration Steps:}

\begin{enumerate}
    \item \textbf{Online Registration}: Visit www.startupindia.gov.in
    \item \textbf{Document Preparation}:
    \begin{itemize}
        \item Certificate of Incorporation
        \item PAN Card of entity
        \item Brief description of business
    \end{itemize}
    \item \textbf{Eligibility Criteria}:
    \begin{itemize}
        \item Entity age less than 10 years
        \item Annual turnover less than ₹100 crore
        \item Working towards innovation/improvement
    \end{itemize}
    \item \textbf{Application Submission}: Complete online form with required documents
    \item \textbf{Verification Process}: Government review and approval
    \item \textbf{Certificate Issuance}: Receive recognition certificate
    \end{enumerate}

\begin{center}
\begin{tikzpicture}[node distance=0.8cm, auto]
    \node [gtu block] (A) {1. Online Registration (startupindia.gov.in)};
    \node [gtu block, below=of A] (B) {2. Document Preparation (Certificate, PAN, Desc.)};
    \node [gtu block, below=of B] (C) {3. Eligibility Check (< 10 yrs, < 100 Cr)};
    \node [gtu block, below=of C] (D) {4. Application Submission};
    \node [gtu block, below=of D] (E) {5. Verification Process};
    \node [gtu block, below=of E] (F) {6. Certificate Issuance};
    
    \draw [gtu arrow] (A) -- (B);
    \draw [gtu arrow] (B) -- (C);
    \draw [gtu arrow] (C) -- (D);
    \draw [gtu arrow] (D) -- (E);
    \draw [gtu arrow] (E) -- (F);
\end{tikzpicture}
\captionof{figure}{Start-up India Registration Process}
\end{center}

\textbf{Benefits:}

\begin{itemize}
    \item \keyword{Tax exemptions} for 3 consecutive years
    \item \keyword{Fast-track patent} application process
    \item \keyword{Compliance reduction} under labor and environment laws
\end{itemize}
\end{solutionbox}

\begin{mnemonicbox}
\mnemonic{"Online Documents Eligibility Application Verification Certificate Benefits"}
\end{mnemonicbox}

\questionmarks{2(a)}{3}{List Methods of Market Research.}

\begin{solutionbox}
\textbf{Primary Research Methods:}

\begin{itemize}
    \item \keyword{Surveys}: Questionnaires to collect customer data
    \item \keyword{Interviews}: Direct interaction with target audience
    \item \keyword{Focus Groups}: Group discussions for feedback
\end{itemize}

\textbf{Secondary Research Methods:}

\begin{itemize}
    \item \keyword{Online Research}: Internet-based data collection
    \item \keyword{Published Reports}: Industry analysis and studies
    \item \keyword{Government Data}: Statistical information from official sources
\end{itemize}
\end{solutionbox}

\begin{mnemonicbox}
\mnemonic{"Survey Interview Focus Online Published Government"}
\end{mnemonicbox}

\questionmarks{2(b)}{4}{Draw and Explain Product Life Cycle.}

\begin{solutionbox}
\begin{center}
\begin{tikzpicture}[node distance=1.5cm, auto]
    \node [gtu state] (A) {Introduction};
    \node [gtu state, right=of A] (B) {Growth};
    \node [gtu state, right=of B] (C) {Maturity};
    \node [gtu state, right=of C] (D) {Decline};
    
    \path [gtu arrow] (A) -- (B);
    \path [gtu arrow] (B) -- (C);
    \path [gtu arrow] (C) -- (D);
\end{tikzpicture}
\captionof{figure}{Product Life Cycle Stages}
\end{center}

\textbf{Stages:}

\begin{itemize}
    \item \keyword{Introduction}: Product launch with high marketing costs
    \item \keyword{Growth}: Rapid sales increase and market acceptance
    \item \keyword{Maturity}: Peak sales with intense competition
    \item \keyword{Decline}: Decreasing demand and eventual phase-out
\end{itemize}
\end{solutionbox}

\begin{mnemonicbox}
\mnemonic{"I Grow My Dreams"}
\end{mnemonicbox}

\questionmarks{2(c)}{7}{Identify and discuss 4 P's of Marketing.}

\begin{solutionbox}
\begin{center}
\captionof{table}{4 P's of Marketing}
\begin{tabulary}{\linewidth}{|L|L|L|}
\hline
\textbf{P} & \textbf{Example Element} & \textbf{Description} \\ \hline
\textbf{Product} & Features, quality, branding & Goods/Services offered to satisfy needs \\ \hline
\textbf{Price} & Strategy, discounts & Cost to customer and competitive position \\ \hline
\textbf{Place} & Channels, location & Distribution channels and accessibility \\ \hline
\textbf{Promotion} & Advertising, PR & Marketing communication and awareness \\ \hline
\end{tabulary}
\end{center}

\begin{center}
\begin{tikzpicture}[node distance=1.5cm]
  \node [gtu block] (root) {4 P's of Marketing};
  \node [gtu block, below left=2cm of root] (prod) {Product};
  \node [gtu block, below right=0.5cm and 2cm of root] (price) {Price};
  \node [gtu block, below left=0.5cm and 2cm of root] (place) {Place};
  \node [gtu block, below right=2cm of root] (prom) {Promotion};
  
  \draw [gtu arrow] (root) -- (prod);
  \draw [gtu arrow] (root) -- (price);
  \draw [gtu arrow] (root) -- (place);
  \draw [gtu arrow] (root) -- (prom);
\end{tikzpicture}
\captionof{figure}{Marketing Mix}
\end{center}

\textbf{Integration:} All 4 P's must work together for effective marketing strategy.
\end{solutionbox}

\begin{mnemonicbox}
\mnemonic{"People Purchase Products Properly"}
\end{mnemonicbox}

\questionmarks{2(a) OR}{3}{Discuss B2B, E-commerce and GeM.}

\begin{solutionbox}
\begin{center}
\captionof{table}{B2B, E-commerce, GeM}
\begin{tabulary}{\linewidth}{|L|L|L|}
\hline
\textbf{Type} & \textbf{Full Form} & \textbf{Description} \\ \hline
\textbf{B2B} & Business to Business & Trade between companies (Bulk transactions) \\ \hline
\textbf{E-commerce} & Electronic Commerce & Online buying and selling (Global reach) \\ \hline
\textbf{GeM} & Government e-Marketplace & Govt procurement portal (Transparent) \\ \hline
\end{tabulary}
\end{center}
\end{center}

\textbf{Key Features:}

\begin{itemize}
    \item \textbf{B2B}: Bulk transactions, long-term relationships
    \item \textbf{E-commerce}: Digital platforms, global reach
    \item \textbf{GeM}: Transparent government purchases, competitive pricing
\end{itemize}
\end{solutionbox}

\begin{mnemonicbox}
\mnemonic{"Businesses Buy Electronically, Government e-Markets"}
\end{mnemonicbox}

\questionmarks{2(b) OR}{4}{Write a note on the plans for creating and starting the business}

\begin{solutionbox}
\textbf{Business Creation Plans:}

\begin{itemize}
    \item \textbf{Market Analysis}:
    \begin{itemize}
        \item \keyword{Target customers}: Identify primary audience
        \item \keyword{Competition study}: Analyze existing players
        \item \keyword{Market size}: Determine potential revenue
    \end{itemize}
    \item \textbf{Financial Planning}:
    \begin{itemize}
        \item \keyword{Capital requirements}: Initial investment needed
        \item \keyword{Revenue projections}: Expected income streams
        \item \keyword{Break-even analysis}: Profitability timeline
    \end{itemize}
    \item \textbf{Operational Setup}:
    \begin{itemize}
        \item \keyword{Location selection}: Strategic positioning
        \item \keyword{Resource allocation}: Human and material resources
        \item \keyword{Legal compliance}: Licenses and registrations
    \end{itemize}
\end{itemize}
\end{solutionbox}

\begin{mnemonicbox}
\mnemonic{"Market Finance Operations = Business Success"}
\end{mnemonicbox}

\questionmarks{2(c) OR}{7}{Explain the concept of Risk and SWOT analysis.}

\begin{solutionbox}
\textbf{Risk Concept:}
Risk is uncertainty that can affect business outcomes, both positively and negatively.

\textbf{Types of Business Risks:}

\begin{itemize}
    \item \keyword{Financial Risk}: Cash flow and funding issues
    \item \keyword{Market Risk}: Demand fluctuations and competition
    \item \keyword{Operational Risk}: Production and service delivery problems
\end{itemize}

\textbf{SWOT Analysis:}

\begin{center}
\begin{tikzpicture}[node distance=0cm, outer sep=0pt]
    \node [gtu block, minimum width=4cm, minimum height=2cm] (S) {Strengths\\(Internal Positive)};
    \node [gtu block, minimum width=4cm, minimum height=2cm, right=0.5cm of S] (W) {Weaknesses\\(Internal Negative)};
    \node [gtu block, minimum width=4cm, minimum height=2cm, below=0.5cm of S] (O) {Opportunities\\(External Positive)};
    \node [gtu block, minimum width=4cm, minimum height=2cm, right=0.5cm of O] (T) {Threats\\(External Negative)};
    
    \node [above=0.1cm of S] {\textbf{Internal}};
    \node [above=0.1cm of W] {\textbf{Internal}};
    \node [left=0.1cm of S, rotate=90] {\textbf{Positive}};
    \node [left=0.1cm of O, rotate=90] {\textbf{Negative}};
\end{tikzpicture}
\captionof{figure}{SWOT Analysis Matrix}
\end{center}

\begin{center}
\captionof{table}{SWOT Factors}
\begin{tabulary}{\linewidth}{|L|L|}
\hline
\textbf{Internal Factors} & \textbf{External Factors} \\ \hline
\textbf{Strengths} & \textbf{Opportunities} \\
- Core competencies & - Market growth \\
- Unique resources & - New technologies \\ \hline
\textbf{Weaknesses} & \textbf{Threats} \\
- Skill gaps & - Competition \\
- Resource limitations & - Economic changes \\ \hline
\end{tabulary}
\end{center}

\textbf{Risk Mitigation Strategies:}

\begin{itemize}
    \item \keyword{Diversification}: Spread risks across different areas
    \item \keyword{Insurance}: Transfer risk to insurance companies
    \item \keyword{Contingency planning}: Prepare for unexpected situations
\end{itemize}
\end{solutionbox}

\begin{mnemonicbox}
\mnemonic{"Strong Weak Opportunities Threaten = SWOT"}
\end{mnemonicbox}

\questionmarks{3(a)}{3}{Write short note on cooperative type organization.}

\begin{solutionbox}
\textbf{Cooperative Organization:}

\begin{itemize}
    \item \keyword{Definition}: Voluntary association of people for mutual benefit
    \item \keyword{Ownership}: Collectively owned by members
    \item \keyword{Control}: Democratic management with equal voting rights
\end{itemize}

\textbf{Characteristics:}

\begin{itemize}
    \item \keyword{Member participation}: Active involvement in decision-making
    \item \keyword{Profit sharing}: Benefits distributed among members
    \item \keyword{Social purpose}: Focus on community welfare
\end{itemize}

\textbf{Examples}: Agricultural cooperatives, credit unions, housing societies
\end{solutionbox}

\begin{mnemonicbox}
\mnemonic{"Collective Ownership with Democratic Management"}
\end{mnemonicbox}

\questionmarks{3(b)}{4}{Give a list of functions of management and define all of them.}

\begin{solutionbox}
\begin{center}
\captionof{table}{Functions of Management}
\begin{tabulary}{\linewidth}{|L|L|L|}
\hline
\textbf{Function} & \textbf{Definition} & \textbf{Key Activities} \\ \hline
\textbf{Planning} & Setting objectives and strategies & Goal setting, forecasting, budgeting \\ \hline
\textbf{Organizing} & Arranging resources and structure & Departmentation, delegation, coordination \\ \hline
\textbf{Staffing} & Human resource management & Recruitment, training, performance evaluation \\ \hline
\textbf{Directing} & Leading and motivating employees & Communication, leadership, supervision \\ \hline
\textbf{Controlling} & Monitoring and correcting performance & Performance measurement, feedback, correction \\ \hline
\end{tabulary}
\end{center}
\end{solutionbox}

\begin{mnemonicbox}
\mnemonic{"Proper Organization Supports Directed Control"}
\end{mnemonicbox}

\questionmarks{3(c)}{7}{Describe types of Ownership and explain any three in detail.}

\begin{solutionbox}
\textbf{Types of Business Ownership:}

\begin{center}
\begin{tabulary}{\linewidth}{|L|L|L|L|}
\hline
\textbf{Type} & \textbf{Ownership} & \textbf{Liability} & \textbf{Control} \\ \hline
\textbf{Sole Proprietorship} & Single owner & Unlimited & Complete \\ \hline
\textbf{Partnership} & 2+ partners & Unlimited & Shared \\ \hline
\textbf{Company} & Shareholders & Limited & Board of Directors \\ \hline
\textbf{Cooperative} & Members & Limited & Democratic \\ \hline
\end{tabulary}
\end{center}

\textbf{Detailed Explanation:}

\begin{itemize}
    \item \textbf{1. Sole Proprietorship}:
    \begin{itemize}
        \item \keyword{Advantages}: Easy formation, complete control, tax benefits
        \item \keyword{Disadvantages}: Unlimited liability, limited resources, business continuity issues
        \item \textbf{Suitable for}: Small businesses, professional services
    \end{itemize}
    
    \item \textbf{2. Partnership}:
    \begin{itemize}
        \item \keyword{Advantages}: Shared resources, specialized skills, easy formation
        \item \keyword{Disadvantages}: Unlimited liability, conflict potential, shared profits
        \item \textbf{Types}: General partnership, limited partnership
    \end{itemize}
    
    \item \textbf{3. Company}:
    \begin{itemize}
        \item \keyword{Advantages}: Limited liability, perpetual existence, easier capital raising
        \item \keyword{Disadvantages}: Complex regulations, double taxation, loss of control
        \item \textbf{Types}: Private limited, public limited
    \end{itemize}
\end{itemize}
\end{solutionbox}

\begin{mnemonicbox}
\mnemonic{"Single Partners Companies Cooperate"}
\end{mnemonicbox}

\questionmarks{3(a) OR}{3}{Explain different Leadership Models.}

\begin{solutionbox}
\textbf{Leadership Models:}

\begin{center}
\begin{tabulary}{\linewidth}{|L|L|L|}
\hline
\textbf{Model} & \textbf{Approach} & \textbf{Best Used When} \\ \hline
\textbf{Autocratic} & Leader makes all decisions & Crisis situations, quick decisions needed \\ \hline
\textbf{Democratic} & Participative decision-making & Team input valuable, time available \\ \hline
\textbf{Laissez-faire} & Hands-off approach & Experienced team, creative work \\ \hline
\end{tabulary}
\end{center}

\textbf{Modern Models:}

\begin{itemize}
    \item \keyword{Transformational}: Inspiring vision and change
    \item \keyword{Transactional}: Reward-punishment based
    \item \keyword{Situational}: Adapts style to situation
\end{itemize}
\end{solutionbox}

\begin{mnemonicbox}
\mnemonic{"Auto Demo Laissez Transform Transact Situate"}
\end{mnemonicbox}

\questionmarks{3(b) OR}{4}{Give the difference between Administration and Management}

\begin{solutionbox}
\begin{center}
\captionof{table}{Administration vs Management}
\begin{tabulary}{\linewidth}{|L|L|L|}
\hline
\textbf{Aspect} & \textbf{Administration} & \textbf{Management} \\ \hline
\textbf{Focus} & Policy formulation & Policy implementation \\ \hline
\textbf{Level} & Top level function & Middle level function \\ \hline
\textbf{Nature} & Planning and thinking & Doing and executing \\ \hline
\textbf{Scope} & Broader organizational & Specific departmental \\ \hline
\end{tabulary}
\end{center}

\textbf{Key Differences:}

\begin{itemize}
    \item \keyword{Administration}: Strategic, long-term, conceptual
    \item \keyword{Management}: Operational, short-term, practical
\end{itemize}

\textbf{Relationship}: Administration sets direction, Management executes plans.
\end{solutionbox}

\begin{mnemonicbox}
\mnemonic{"Admin Plans, Management Implements"}
\end{mnemonicbox}

\questionmarks{3(c) OR}{7}{Explain the concept of difference between industry, commerce and business.}

\begin{solutionbox}
\begin{center}
\captionof{table}{Industry, Commerce, Business}
\begin{tabulary}{\linewidth}{|L|L|L|L|}
\hline
\textbf{Concept} & \textbf{Definition} & \textbf{Primary Activity} & \textbf{Examples} \\ \hline
\textbf{Industry} & Production of goods & Manufacturing, processing & Steel, textiles \\ \hline
\textbf{Commerce} & Distribution of goods & Trading, transportation & Wholesale, retail \\ \hline
\textbf{Business} & Overall economic activity & Production + distribution & Complete enterprise \\ \hline
\end{tabulary}
\end{center}

\begin{center}
\begin{tikzpicture}[node distance=1.5cm, auto]
    \node [gtu block] (A) {Business};
    \node [gtu block, below left=1.5cm of A] (B) {Industry};
    \node [gtu block, below right=1.5cm of A] (C) {Commerce};
    
    \node [gtu block, below=0.8cm of B] (B1) {Primary, Secondary, Tertiary};
    \node [gtu block, below=0.8cm of C] (C1) {Trade, Auxiliaries};
    
    \draw [gtu arrow] (A) -- (B);
    \draw [gtu arrow] (A) -- (C);
    \draw [gtu arrow] (B) -- (B1);
    \draw [gtu arrow] (C) -- (C1);
\end{tikzpicture}
\captionof{figure}{Business Components}
\end{center}
\end{solutionbox}

\begin{mnemonicbox}
\mnemonic{"Industry Creates, Commerce Distributes, Business Integrates"}
\end{mnemonicbox}

\questionmarks{4(a)}{3}{Explain following terms: 1.Contracts 2.Copyrights}

\begin{solutionbox}
\begin{center}
\captionof{table}{Contracts vs Copyrights}
\begin{tabulary}{\linewidth}{|L|L|L|}
\hline
\textbf{Term} & \textbf{Definition} & \textbf{Key Features} \\ \hline
\textbf{Contracts} & Legal agreement between parties & Binding, enforceable, mutual obligations \\ \hline
\textbf{Copyrights} & Intellectual property protection & Creative works, exclusive rights, limited duration \\ \hline
\end{tabulary}
\end{center}

\textbf{Contract Elements}: Offer and acceptance, Consideration, Legal capacity.

\textbf{Copyright Protection}: Duration (lifetime + 70 years), Rights (reproduction, distribution).
\end{solutionbox}

\begin{mnemonicbox}
\mnemonic{"Contracts Bind, Copyrights Protect"}
\end{mnemonicbox}

\questionmarks{4(b)}{4}{Give a note on startup incubation center and Modalities.}

\begin{solutionbox}
\textbf{Startup Incubation Centers:}

\begin{itemize}
    \item \keyword{Purpose}: Support early-stage startups with resources and guidance
    \item \keyword{Services}: Mentorship, funding, workspace, networking
\end{itemize}

\textbf{Key Modalities:}

\begin{itemize}
    \item \textbf{Pre-incubation}: Idea validation, Team formation, Prototype development
    \item \textbf{Incubation Phase}: Business model refinement, Market testing, Funding preparation
    \item \textbf{Post-incubation}: Alumni network, Follow-up funding, Scaling support
\end{itemize}
\end{solutionbox}

\begin{mnemonicbox}
\mnemonic{"Pre-incubate, Incubate, Post-support Startups"}
\end{mnemonicbox}

\questionmarks{4(c)}{7}{List State level agencies which supports start-ups and describe their functionalities}

\begin{solutionbox}
\textbf{Gujarat State Support Agencies:}

\begin{center}
\begin{tabulary}{\linewidth}{|L|L|L|}
\hline
\textbf{Agency} & \textbf{Full Form} & \textbf{Key Functions} \\ \hline
\textbf{SSIP} & Student Startup \& Innovation Policy & Student entrepreneur support, funding \\ \hline
\textbf{iHub} & Innovation Hub Gujarat & Incubation, mentorship, networking \\ \hline
\textbf{GUSEC} & Gujarat University Startup Council & University-level startup promotion \\ \hline
\textbf{GIDC} & Gujarat Industrial Development Corp & Industrial infrastructure, land \\ \hline
\end{tabulary}
\end{center}



\textbf{Detailed Functionalities:}

\textbf{SSIP Gujarat:}
\begin{itemize}
    \item \textbf{Funding support}: Up to Rs. 2 lakh for student startups
    \item \textbf{Incubation facilities}: Workspace and equipment access
    \item \textbf{Mentorship programs}: Industry expert guidance
    \item \textbf{IPR support}: Patent filing assistance
\end{itemize}

\textbf{iHub Gujarat:}
\begin{itemize}
    \item \textbf{Startup ecosystem}: Complete entrepreneurship support
    \item \textbf{Technology transfer}: Research to market conversion
    \item \textbf{Investor connections}: Funding facilitation
    \item \textbf{Industry partnerships}: Corporate collaboration
\end{itemize}

\textbf{GUSEC:}
\begin{itemize}
    \item \textbf{Student engagement}: Campus entrepreneurship programs
    \item \textbf{Skill development}: Entrepreneurship education
    \item \textbf{Competition organization}: Startup contests and pitches
    \item \textbf{Network building}: Alumni entrepreneur connections
\end{itemize}

\begin{center}
\begin{tikzpicture}[node distance=1.5cm]
  \node [gtu block] (root) {State Startup Support};
  \node [gtu block, below left=1.5cm of root] (ssip) {SSIP};
  \node [gtu block, below right=0.5cm and 1cm of root] (ihub) {iHub};
  \node [gtu block, below left=0.5cm and 1cm of root] (gusec) {GUSEC};
  \node [gtu block, below right=1.5cm of root] (gidc) {GIDC};
  
  \draw [gtu arrow] (root) -- (ssip);
  \draw [gtu arrow] (root) -- (ihub);
  \draw [gtu arrow] (root) -- (gusec);
  \draw [gtu arrow] (root) -- (gidc);
\end{tikzpicture}
\captionof{figure}{State Agencies}
\end{center}

\textbf{Impact Measurement:}
\begin{itemize}
    \item \textbf{Number of startups supported} annually
    \item \textbf{Job creation} through supported ventures
    \item \textbf{Revenue generation} of incubated companies
    \item \textbf{Success rate} of graduated startups
\end{itemize}
\end{solutionbox}

\begin{mnemonicbox}
\mnemonic{"SSIP iHub GUSEC GIDC Support Gujarat Startups"}
\end{mnemonicbox}

\questionmarks{4(a) OR}{3}{Explain following terms: 1.IPR 2.Trademarks}

\begin{solutionbox}
\begin{center}
\captionof{table}{IPR vs Trademarks}
\begin{tabulary}{\linewidth}{|L|L|L|}
\hline
\textbf{Term} & \textbf{Definition} & \textbf{Protection Scope} \\ \hline
\textbf{IPR} & Intellectual Property Rights & Ideas, inventions, creative works \\ \hline
\textbf{Trademarks} & Brand identification marks & Names, logos, symbols \\ \hline
\end{tabulary}
\end{center}


\textbf{IPR Categories:}
\begin{itemize}
    \item \textbf{Patents}: Technical inventions (20 years)
    \item \textbf{Copyrights}: Creative expressions (lifetime + 70 years)
    \item \textbf{Trademarks}: Brand identifiers (10 years, renewable)
\end{itemize}

\textbf{Trademark Features:}
\begin{itemize}
    \item \textbf{Distinctiveness}: Unique brand identification
    \item \textbf{Commercial use}: Business identification purpose
    \item \textbf{Registration}: Legal protection through registration
\end{itemize}
\end{solutionbox}

\begin{mnemonicbox}
\mnemonic{"IPR Protects, Trademarks Identify"}
\end{mnemonicbox}

\questionmarks{4(b) OR}{4}{Define the role of Investor in start-up.}

\begin{solutionbox}
\textbf{Investor Roles in Startups:}

\begin{itemize}
    \item \textbf{Financial Support}: Seed funding, Growth capital, Bridge financing
    \item \textbf{Strategic Guidance}: Business mentorship, Network access, Market insights
    \item \textbf{Operational Support}: Team building, Technology guidance, Legal compliance
    \item \textbf{Risk Management}: Due diligence, Performance monitoring, Exit strategy
\end{itemize}

\textbf{Types of Investors}: Angel investors, Venture capital, Corporate investors.
\end{solutionbox}

\begin{mnemonicbox}
\mnemonic{"Finance Strategy Operations Risk = Investor Roles"}
\end{mnemonicbox}

\questionmarks{4(c) OR}{7}{List National level agencies which support start-ups and describe their functionalities.}

\begin{solutionbox}
\textbf{National Startup Support Agencies:}

\begin{center}
\begin{tabulary}{\linewidth}{|L|L|L|}
\hline
\textbf{Agency} & \textbf{Department} & \textbf{Primary Focus} \\ \hline
\textbf{Startup India} & DPIIT & Policy framework and ecosystem \\ \hline
\textbf{BIRAC} & Biotechnology & Biotech innovation \\ \hline
\textbf{TDB} & Science \& Tech & Technology development \\ \hline
\textbf{SIDBI} & Financial Services & MSME and startup funding \\ \hline
\end{tabulary}
\end{center}

\textbf{Detailed Functionalities:}

\textbf{Startup India:}
\begin{itemize}
    \item \textbf{Policy formulation}: National startup policy framework
    \item \textbf{Recognition program}: Official startup certification
    \item \textbf{Tax benefits}: 3-year tax exemption for eligible startups
    \item \textbf{Regulatory support}: Single-point clearance system
    \item \textbf{Funding facilitation}: Fund of Funds scheme (₹10,000 crores)
\end{itemize}

\textbf{BIRAC (Biotechnology Industry Research Assistance Council):}
\begin{itemize}
    \item \textbf{Biotech innovation}: Supporting biotech startups and research
    \item \textbf{Funding schemes}: SBIRI, SPARSH, BIG programs
    \item \textbf{Industry partnerships}: Academia-industry collaboration
    \item \textbf{Technology translation}: Research to market conversion
\end{itemize}

\textbf{TDB (Technology Development Board):}
\begin{itemize}
    \item \textbf{Technology commercialization}: Converting research to products
    \item \textbf{Financial assistance}: Loans and grants for technology development
    \item \textbf{Industry support}: Manufacturing technology assistance
    \item \textbf{Innovation promotion}: Supporting technological innovation
\end{itemize}

\textbf{SIDBI (Small Industries Development Bank of India):}
\begin{itemize}
    \item \textbf{Financial support}: Loans and credit facilities
    \item \textbf{MSME focus}: Small and medium enterprise development
    \item \textbf{Startup funding}: Venture capital and growth capital
    \item \textbf{Ecosystem development}: Incubator and accelerator support
\end{itemize}

\begin{center}
\begin{tikzpicture}[node distance=1.5cm]
  \node [gtu block] (root) {National Startup Support};
  \node [gtu block, below left=1.5cm of root] (si) {Startup India};
  \node [gtu block, below right=0.5cm and 1cm of root] (birac) {BIRAC};
  \node [gtu block, below left=0.5cm and 1cm of root] (tdb) {TDB};
  \node [gtu block, below right=1.5cm of root] (sidbi) {SIDBI};
  
  \draw [gtu arrow] (root) -- (si);
  \draw [gtu arrow] (root) -- (birac);
  \draw [gtu arrow] (root) -- (tdb);
  \draw [gtu arrow] (root) -- (sidbi);
\end{tikzpicture}
\captionof{figure}{National Agencies}
\end{center}

\textbf{Additional Agencies:}
\begin{itemize}
    \item \textbf{NSTEDB}: National Science \& Technology Entrepreneurship Development Board
    \item \textbf{MSME}: Ministry of Micro, Small and Medium Enterprises
    \item \textbf{Atal Innovation Mission}: Innovation and entrepreneurship promotion
\end{itemize}

\textbf{Success Metrics:}
\begin{itemize}
    \item \textbf{Startup registrations}: Over 70,000 recognized startups
    \item \textbf{Job creation}: Millions of employment opportunities
    \item \textbf{Funding facilitated}: Billions in investment mobilization
    \item \textbf{Ecosystem development}: Thousands of incubators and accelerators
\end{itemize}
\end{solutionbox}

\begin{mnemonicbox}
\mnemonic{"Startup BIRAC TDB SIDBI = National Support System"}
\end{mnemonicbox}

\questionmarks{5(a)}{3}{Explain following terms: 1.Break Even point 2.Return on Investment 3.Return on Sales.}

\begin{solutionbox}
\begin{center}
\captionof{table}{Financial Terms}
\begin{tabulary}{\linewidth}{|L|L|L|}
\hline
\textbf{Term} & \textbf{Formula} & \textbf{Meaning} \\ \hline
\textbf{Break Even Point} & Fixed Costs / (Price - Var Cost) & Units to cover all costs \\ \hline
\textbf{ROI} & (Gain-Cost) / Cost * 100 & Return on invested capital \\ \hline
\textbf{ROS} & Net Income / Sales * 100 & Profit margin percentage \\ \hline
\end{tabulary}
\end{center}


\textbf{Break Even Analysis:}
\begin{itemize}
    \item \textbf{Fixed costs}: Rent, salaries, insurance
    \item \textbf{Variable costs}: Raw materials, utilities per unit
    \item \textbf{Contribution margin}: Price minus variable cost per unit
\end{itemize}

\textbf{ROI Importance:}
\begin{itemize}
    \item \textbf{Investment efficiency}: Measures investment performance
    \item \textbf{Comparison tool}: Compare different investment options
    \item \textbf{Decision making}: Guide future investment decisions
\end{itemize}

\textbf{ROS Significance:}
\begin{itemize}
    \item \textbf{Profitability measure}: Shows operational efficiency
    \item \textbf{Industry comparison}: Benchmark against competitors
    \item \textbf{Trend analysis}: Track performance over time
\end{itemize}
\end{solutionbox}

\begin{mnemonicbox}
\mnemonic{"Break Even Returns On Investment Sales"}
\end{mnemonicbox}

\questionmarks{5(b)}{4}{Write a short note on Import-Export Policy}

\begin{solutionbox}
\textbf{India's Import-Export Policy (EXIM Policy):}

\begin{itemize}
    \item \textbf{Objectives}: Trade promotion, Export growth, Economic development.
    \item \textbf{Export Promotion}: Incentives, SEZs, Financing.
    \item \textbf{Import Management}: Licensing, Duty structure, Quality standards.
    \item \textbf{Trade Facilitation}: Digital platforms, Single window, Agreements.
    \item \textbf{Current Focus}: Make in India, Digital India, Atmanirbhar Bharat.
\end{itemize}
\end{solutionbox}

\begin{mnemonicbox}
\mnemonic{"Export Import Policy Promotes Trade Facilitation"}
\end{mnemonicbox}

\questionmarks{5(c)}{7}{Describe the connection between CSR and Economic Performance.}

\begin{solutionbox}
\textbf{Direct Economic Benefits:}

\begin{center}
\captionof{table}{Direct Benefits}
\begin{tabulary}{\linewidth}{|L|L|L|}
\hline
\textbf{CSR Activity} & \textbf{Economic Impact} & \textbf{Measurement} \\ \hline
\textbf{Employee welfare} & Higher productivity, lower turnover & Cost savings, efficiency gains \\ \hline
\textbf{Environmental initiatives} & Resource efficiency, waste reduction & Cost reduction, sustainability \\ \hline
\textbf{Community development} & Market expansion, brand loyalty & Revenue growth, customer retention \\ \hline
\end{tabulary}
\end{center}

\textbf{Indirect Economic Benefits:}

\textbf{Brand Value Enhancement:}
\begin{itemize}
    \item \textbf{Customer loyalty}: Increased repeat purchases and referrals
    \item \textbf{Premium pricing}: Ability to charge higher prices for ethical products
    \item \textbf{Market differentiation}: Competitive advantage in conscious markets
\end{itemize}

\textbf{Risk Management:}
\begin{itemize}
    \item \textbf{Regulatory compliance}: Avoiding penalties and legal costs
    \item \textbf{Reputation protection}: Preventing brand damage from social issues
    \item \textbf{Stakeholder relations}: Building trust with investors and partners
\end{itemize}

\textbf{Long-term Economic Performance:}

\textbf{Sustainable Growth:}
\begin{itemize}
    \item \textbf{Innovation driver}: CSR initiatives often lead to innovative solutions
    \item \textbf{Market access}: Meeting ESG criteria for international markets
    \item \textbf{Investment attraction}: ESG-focused investors prefer responsible companies
\end{itemize}

\begin{center}
\begin{tikzpicture}[node distance=1.5cm, auto]
    \node [gtu block] (A) {CSR Activities};
    \node [gtu block, below left=1.0cm of A] (B) {Direct Benefits};
    \node [gtu block, below=1.0cm of A] (C) {Indirect Benefits};
    \node [gtu block, below right=1.0cm of A] (D) {Long-term Impact};
    
    \node [gtu block, below=2.5cm of A] (E) {Economic Performance};
    
    \draw [gtu arrow] (A) -- (B);
    \draw [gtu arrow] (A) -- (C);
    \draw [gtu arrow] (A) -- (D);
    \draw [gtu arrow] (B) -- (E);
    \draw [gtu arrow] (C) -- (E);
    \draw [gtu arrow] (D) -- (E);
\end{tikzpicture}
\captionof{figure}{CSR and Economic Performance}
\end{center}

\textbf{Research Evidence:}
\begin{itemize}
    \item \textbf{Performance correlation}: Studies show positive correlation between CSR and financial performance
    \item \textbf{Investor preference}: ESG-compliant companies attract more investment
    \item \textbf{Market valuation}: Responsible companies often have higher market valuations
\end{itemize}

\textbf{CSR-Economic Performance Cycle:}
\begin{itemize}
    \item Investment in CSR $\rightarrow$ Operational improvements $\rightarrow$ Financial performance $\rightarrow$ More CSR investment
\end{itemize}

\textbf{Implementation Strategy:}
\begin{itemize}
    \item \textbf{Strategic alignment}: Align CSR with business objectives
    \item \textbf{Measurement systems}: Track both social and economic impacts
    \item \textbf{Stakeholder engagement}: Involve all stakeholders in CSR planning
    \item \textbf{Continuous improvement}: Regular review and enhancement of CSR programs
\end{itemize}

\textbf{Challenges:}
\begin{itemize}
    \item \textbf{Short-term costs}: Initial investment may impact immediate profits
    \item \textbf{Measurement difficulty}: Quantifying social impact can be complex
    \item \textbf{Stakeholder expectations}: Balancing different stakeholder demands
\end{itemize}

\textbf{Success Factors:}
\begin{itemize}
    \item \textbf{Leadership commitment}: Top management support for CSR initiatives
    \item \textbf{Integration}: Embedding CSR into business strategy and operations
    \item \textbf{Transparency}: Regular reporting and communication of CSR impact
    \item \textbf{Innovation}: Using CSR as a driver for business innovation
\end{itemize}
\end{solutionbox}

\begin{mnemonicbox}
\mnemonic{"CSR Creates Sustainable Returns"}
\end{mnemonicbox}

\questionmarks{5(a) OR}{3}{Write a note on Bankruptcy and Avoidance.}

\begin{solutionbox}
\textbf{Bankruptcy}: Legal process (Asset liquidation or reorganization) when a business cannot meet obligations.

\textbf{Avoidance Strategies}:
\begin{itemize}
    \item \keyword{Cash flow management}: Maintain working capital
    \item \keyword{Debt restructuring}: Negotiate payment terms
    \item \keyword{Cost reduction}: Improve efficiency
\end{itemize}
\end{solutionbox}

\begin{mnemonicbox}
\mnemonic{"Bankrupt Businesses Avoid Through Cash Control"}
\end{mnemonicbox}

\questionmarks{5(b) OR}{4}{Write an importance of Business Ethics.}

\begin{solutionbox}
\textbf{Importance of Business Ethics:}

\begin{itemize}
    \item \textbf{Stakeholder Trust}: Customer confidence, Investor faith, Employee satisfaction.
    \item \textbf{Legal Compliance}: Avoiding penalties, Risk mitigation, Reputation protection.
    \item \textbf{Competitive Advantage}: Market differentiation, Premium positioning, Sustainable growth.
    \item \textbf{Social Impact}: Community development, Environmental responsibility.
\end{itemize}
\end{solutionbox}

\begin{mnemonicbox}
\mnemonic{"Ethics Builds Trust, Compliance, Advantage, Social Impact"}
\end{mnemonicbox}

\questionmarks{5(c) OR}{7}{Give the steps and format of project report writing}

\begin{solutionbox}

\textbf{Project Report Writing Steps:}

\textbf{Pre-Writing Phase:}
\begin{enumerate}
    \item \textbf{Project planning}: Define scope, objectives, and deliverables
    \item \textbf{Data collection}: Gather relevant information and research
    \item \textbf{Analysis}: Process and analyze collected data
    \item \textbf{Structure planning}: Organize content logically
\end{enumerate}

\textbf{Writing Phase:}
\begin{enumerate}
    \setcounter{enumi}{4}
    \item \textbf{Draft preparation}: Write initial version following format
    \item \textbf{Content development}: Elaborate each section with details
    \item \textbf{Review and revision}: Check for accuracy and completeness
    \item \textbf{Final formatting}: Apply consistent formatting and style
\end{enumerate}

\textbf{Project Report Format:}

\begin{center}
\begin{tabulary}{\linewidth}{|L|L|}
\hline
\textbf{1. TITLE PAGE} & Project title, Author name(s), Institution, Date \\ \hline
\textbf{2. EXECUTIVE SUMMARY} & Overview, Findings, Outcomes \\ \hline
\textbf{3. TABLE OF CONTENTS} & Headings, Page numbers, Figures list \\ \hline
\textbf{4. INTRODUCTION} & Background, Problem statement, Objectives \\ \hline
\textbf{5. LITERATURE REVIEW} & Existing research, Gap analysis \\ \hline
\textbf{6. METHODOLOGY} & Approach, Data collection, Analysis \\ \hline
\textbf{7. ANALYSIS AND FINDINGS} & Data presentation, Results, Insights \\ \hline
\textbf{8. RECOMMENDATIONS} & Suggestions, Implementation, Benefits \\ \hline
\textbf{9. CONCLUSION} & Summary, Achievements, Future scope \\ \hline
\textbf{10. REFERENCES} & Bibliography, Sources, Appendices \\ \hline
\end{tabulary}
\end{center}

\begin{center}
\begin{tikzpicture}[node distance=0.6cm]
  \node [gtu block] (title) {1. Title Page};
  \node [gtu block, below=of title] (exec) {2. Executive Summary};
  \node [gtu block, below=of exec] (intro) {3. Introduction};
  \node [gtu block, below=of intro] (lit) {4. Literature Review};
  \node [gtu block, below=of lit] (method) {5. Methodology};
  \node [gtu block, below=of method] (analysis) {6. Analysis \& Findings};
  \node [gtu block, below=of analysis] (rec) {7. Recommendations};
  \node [gtu block, below=of rec] (conc) {8. Conclusion};
  \node [gtu block, below=of conc] (ref) {9. References};
  
  \draw [gtu arrow] (title) -- (exec);
  \draw [gtu arrow] (exec) -- (intro);
  \draw [gtu arrow] (intro) -- (lit);
  \draw [gtu arrow] (lit) -- (method);
  \draw [gtu arrow] (method) -- (analysis);
  \draw [gtu arrow] (analysis) -- (rec);
  \draw [gtu arrow] (rec) -- (conc);
  \draw [gtu arrow] (conc) -- (ref);
\end{tikzpicture}
\captionof{figure}{Report Structure}
\end{center}

\textbf{Writing Guidelines:}

\textbf{Content Quality:}
\begin{itemize}
    \item \textbf{Clarity}: Use simple, clear language
    \item \textbf{Accuracy}: Ensure factual correctness
    \item \textbf{Relevance}: Include only pertinent information
    \item \textbf{Logical flow}: Maintain coherent structure
\end{itemize}

\textbf{Formatting Standards:}
\begin{itemize}
    \item \textbf{Font}: Times New Roman 12pt or Arial 11pt
    \item \textbf{Spacing}: 1.5 line spacing
    \item \textbf{Margins}: 1 inch on all sides
    \item \textbf{Page numbering}: Consistent throughout
\end{itemize}

\textbf{Visual Elements:}
\begin{itemize}
    \item \textbf{Tables}: For data presentation
    \item \textbf{Charts/Graphs}: For trend analysis
    \item \textbf{Diagrams}: For process illustration
    \item \textbf{Images}: For concept clarification
\end{itemize}

\textbf{Quality Checklist:}
\begin{itemize}
    \item \textbf{Completeness}: All required sections included
    \item \textbf{Consistency}: Uniform formatting throughout
    \item \textbf{Accuracy}: Facts and figures verified
    \item \textbf{Relevance}: Content aligned with objectives
\end{itemize}

\textbf{Common Mistakes to Avoid:}
\begin{itemize}
    \item \textbf{Plagiarism}: Always cite sources properly
    \item \textbf{Poor structure}: Maintain logical flow
    \item \textbf{Inconsistent formatting}: Follow standard guidelines
    \item \textbf{Inadequate analysis}: Provide sufficient depth
\end{itemize}

\textbf{Review Process:}
\begin{enumerate}
    \item \textbf{Self-review}: Author checks for errors and completeness
    \item \textbf{Peer review}: Colleague feedback on content and clarity
    \item \textbf{Expert review}: Subject matter expert validation
    \item \textbf{Final proofreading}: Grammar and formatting check
\end{enumerate}
\end{solutionbox}

\begin{mnemonicbox}
\mnemonic{"Title Executive Introduction Literature Methodology Analysis Recommendations Conclusion References"}
\end{mnemonicbox}

\end{document}
