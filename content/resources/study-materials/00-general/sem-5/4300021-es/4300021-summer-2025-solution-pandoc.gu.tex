\documentclass[10pt,a4paper]{article}

% content/resources/templates/preamble.tex
\usepackage[margin=0.6in]{geometry}
\author{Milav Dabgar}
\usepackage{amsmath,amssymb,amsthm}
\usepackage{booktabs}
\usepackage{multirow}
\usepackage{xcolor}
\usepackage{tcolorbox}
\tcbuselibrary{breakable,skins}
\usepackage[colorlinks=true,linkcolor=blue]{hyperref}
\usepackage{titlesec}
\usepackage{enumitem}
\usepackage{tikz}
\usepackage{pgfplots}
\usepackage{circuitikz}
\usepackage[version=4]{mhchem}
\usepackage{longtable}
\usepackage{array}
\usepackage{float}
\usepackage{caption}
\usepackage{listings}

\lstset{
  basicstyle=\small\ttfamily,
  breaklines=true,
  breakatwhitespace=false,
  postbreak=\mbox{\textcolor{red}{$\hookrightarrow$}\space},
  float=false,
  numbers=left,
  numberstyle=\tiny\color{gray},
  numbersep=10pt,
  xleftmargin=2em,
  keywordstyle=\color{blue},
  commentstyle=\color{green!60!black},
  stringstyle=\color{purple},
  backgroundcolor=\color{gray!5},
  showstringspaces=false,
  tabsize=2,
  captionpos=b,
  keepspaces=true,
  columns=flexible
}

\pgfplotsset{compat=1.18}
\usetikzlibrary{shapes,arrows,positioning,calc,patterns,decorations.pathmorphing,decorations.markings,arrows.meta}

% Color scheme
\definecolor{headcolor}{RGB}{0,102,204}
\definecolor{keycolor}{RGB}{220,20,60}
\definecolor{solutioncolor}{RGB}{34,139,34}
\definecolor{mnemoniccolor}{RGB}{148,0,211}
\definecolor{codecolor}{RGB}{0,0,100}

% Spacing
\setlength{\parskip}{3pt}
\setlist[itemize]{nosep}
\setlist[enumerate]{nosep}

% Title formatting
\titleformat{\section}{\Large\bfseries\color{headcolor}}{\thesection}{1em}{}
\titleformat{\subsection}{\large\bfseries\color{headcolor}}{\thesubsection}{1em}{}

% Pandoc tightlist compatibility
\providecommand{\tightlist}{%
  \setlength{\itemsep}{0pt}\setlength{\parskip}{0pt}}

% Pandoc longtable compatibility
\newcounter{none}
\def\thenone{}


% content/resources/templates/gujarati-boxes.tex
\usepackage{fontspec}
\usepackage{polyglossia}

% Set Gujarati as main language (document is primarily in Gujarati)
% Note: gloss-gujarati.ldf doesn't exist in polyglossia, but it will use hyphenation patterns
\setdefaultlanguage{gujarati}
\setotherlanguage{english}

% Configure Gujarati font properly
% Use Language=Default to prevent polyglossia from trying to add language-specific features
% that don't exist for Gujarati, which causes "empty feature" warnings
\newfontfamily\gujaratifont[Script=Gujarati,AutoFakeBold=2.5,AutoFakeSlant=0.3]{Noto Sans Gujarati}
\setmainfont[Script=Gujarati,AutoFakeBold=2.5,AutoFakeSlant=0.3]{Noto Sans Gujarati}
% Use Noto Sans Gujarati for monospace to support Gujarati in text
\setmonofont[Scale=0.9]{Noto Sans Gujarati}

% Configure English to use the same font
\newfontfamily\englishfont[Script=Gujarati,AutoFakeBold=2.5,AutoFakeSlant=0.3]{Noto Sans Gujarati}

% Translations for polyglossia
\gappto\captionsgujarati{
  \renewcommand{\tablename}{કોષ્ટક}
  \renewcommand{\figurename}{આકૃતિ}
}

% Helper for TikZ nodes to ensure Gujarati font
\newcommand{\gu}[1]{{\gujaratifont #1}}

% Custom environments
\newtcolorbox{solutionbox}{
    breakable,
    enhanced,
    colback=solutioncolor!5!white,
    colframe=solutioncolor!75!black,
    fonttitle=\bfseries,
    title=જવાબ
}

\newtcolorbox{solutionboxnobreak}{
 colback=solutioncolor!5!white,
 colframe=solutioncolor!75!black,
 fonttitle=\bfseries,
 title=જવાબ
}

\newtcolorbox{keyformula}{
 breakable,
 enhanced,
 colback=keycolor!5!white,
 colframe=keycolor!75!black,
 fonttitle=\bfseries,
 title=રાસાયણિક સમીકરણ/સૂત્ર
}

\newtcolorbox{mnemonicbox}{
 breakable,
 enhanced,
 colback=mnemoniccolor!5!white,
 colframe=mnemoniccolor!75!black,
 fonttitle=\bfseries,
 title=મેમરી ટ્રીક
}


\begin{document}

\begin{center}
{\Huge\bfseries\color{headcolor} Subject Name (Gujarati)}\\[5pt]
{\LARGE 4300021 -- Summer 2025}\\[3pt]
{\large Semester 1 Study Material}\\[3pt]
{\normalsize\textit{Detailed Solutions and Explanations}}
\end{center}

\vspace{10pt}

\subsection*{પ્રશ્ન 1(અ) [3
ગુણ]}\label{uxaaauxab0uxab6uxaa8-1uxa85-3-uxa97uxaa3}

\textbf{ઉદ્યોગસાહસિક અને ઇન્ટ્રાપ્રિન્યોર વચ્ચેનો તફાવત સમજાવો.}

\begin{solutionbox}

\begin{longtable}[]{@{}lll@{}}
\toprule\noalign{}
\textbf{પાસું} & \textbf{ઉદ્યોગસાહસિક} & \textbf{ઇન્ટ્રાપ્રિન્યોર} \\
\midrule\noalign{}
\endhead
\bottomrule\noalign{}
\endlastfoot
\textbf{વ્યાખ્યા} & નવા વ્યવસાય બનાવે છે & વર્તમાન કંપનીમાં કામ કરે છે \\
\textbf{જોખમ} & પોતાનું નાણાકીય જોખમ લે છે & મર્યાદિત જોખમ, કંપની જવાબદાર \\
\textbf{સંસાધનો} & પોતાના સંસાધનોનું આયોજન & કંપનીના સંસાધનોનો ઉપયોગ \\
\textbf{સ્વતંત્રતા} & સંપૂર્ણ સ્વતંત્રતા & કંપનીની નીતિ અનુસાર કામ \\
\textbf{ફાયદો} & બધો નફો મળે છે & પગાર અને બોનસ મળે છે \\
\end{longtable}

\end{solutionbox}
\begin{mnemonicbox}
``બાહ્ય જોખમ વ્યક્તિગત સફળતા બનાવે બાંધકામ જોખમ સંસ્થાકીય
સફળતા''

\end{mnemonicbox}
\begin{center}\rule{0.5\linewidth}{0.5pt}\end{center}

\subsection*{પ્રશ્ન 1(બ) [4
ગુણ]}\label{uxaaauxab0uxab6uxaa8-1uxaac-4-uxa97uxaa3}

\textbf{ઉદ્યોગસાહસિક માટે 7-M સંસાધનો સમજાવો.}

\begin{solutionbox}

ઉદ્યોગસાહસિકતાની સફળતા માટે જરૂરી 7-M સંસાધનો:

\begin{longtable}[]{@{}ll@{}}
\toprule\noalign{}
\textbf{સંસાધન} & \textbf{વિગત} \\
\midrule\noalign{}
\endhead
\bottomrule\noalign{}
\endlastfoot
\textbf{Men (માણસો)} & મનુષ્ય બળ અને કુશળ કર્મચારીઓ \\
\textbf{Money (પૈસા)} & નાણાકીય મૂડી અને ફંડિંગ સ્રોતો \\
\textbf{Material (સામગ્રી)} & કાચો માલ અને ભૌતિક ઇનપુટ \\
\textbf{Machine (મશીન)} & સાધનો અને ટેકનોલોજી \\
\textbf{Method (પદ્ધતિ)} & ઉત્પાદન પ્રક્રિયા અને તકનીકો \\
\textbf{Market (બજાર)} & લક્ષ્ય ગ્રાહકો અને માંગ \\
\textbf{Management (વ્યવસ્થાપન)} & નેતૃત્વ અને સંગઠનાત્મક કુશળતા \\
\end{longtable}

\end{solutionbox}
\begin{mnemonicbox}
``મારા માણસો મશીન પૈસા વ્યવસ્થાપન સામગ્રી બજાર''

\end{mnemonicbox}
\begin{center}\rule{0.5\linewidth}{0.5pt}\end{center}

\subsection*{પ્રશ્ન 1(ક) [7
ગુણ]}\label{uxaaauxab0uxab6uxaa8-1uxa95-7-uxa97uxaa3}

\textbf{સ્ટાર્ટ અપ ઇન્ડિયા યોજના હેઠળ સ્ટાર્ટ અપ તરીકે માન્યતા મેળવવા માટેની
તબક્કાવાર પ્રક્રિયાનું વર્ણન કરો.}

\begin{solutionbox}

\textbf{તબક્કાવાર નોંધણી પ્રક્રિયા:}

\begin{verbatim}
flowchart LR
    A[startup.india.gov.in પર જાઓ] {-{-} B[એકાઉન્ટ બનાવો]}
    B {-{-} C[અરજી ફોર્મ ભરો]}
    C {-{-} D[જરૂરી દસ્તાવેજો અપલોડ કરો]}
    D {-{-} E[અરજી સબમિટ કરો]}
    E {-{-} F[DPIIT દ્વારા સમીક્ષા]}
    F {-{-} G[માન્યતાનું પ્રમાણપત્ર]}
\end{verbatim}

\textbf{જરૂરી દસ્તાવેજો:}

\begin{itemize}
\tightlist
\item
  \textbf{ઇન્કોર્પોરેશન પ્રમાણપત્ર}: કંપની નોંધણીનો પુરાવો
\item
  \textbf{પાર્ટનરશિપ ડીડ}: LLP એન્ટિટી માટે\\
\item
  \textbf{સપોર્ટ પત્ર}: ઇન્ક્યુબેટર/એક્સેલેરેટર તરફથી
\item
  \textbf{ભલામણ પત્ર}: સરકારી સંસ્થા તરફથી
\item
  \textbf{બિઝનેસ પ્લાન}: વિસ્તૃત વ્યવસાયિક વર્ણન
\end{itemize}

\textbf{માન્યતા પછીના મુખ્ય ફાયદા:}

\begin{itemize}
\tightlist
\item
  3 વર્ષ માટે કર મુક્તિ
\item
  પેટન્ટ નોંધણીમાં ઝડપી પ્રક્રિયા
\item
  સરકારી ટેન્ડરમાં પ્રાધાન્ય
\end{itemize}

\end{solutionbox}
\begin{mnemonicbox}
``DPIIT સ્ટાર્ટઅપ્સને અદ્ભુત માન્યતા લાભ આપે છે''

\end{mnemonicbox}
\begin{center}\rule{0.5\linewidth}{0.5pt}\end{center}

\subsection*{પ્રશ્ન 1(ક) અથવા [7
ગુણ]}\label{uxaaauxab0uxab6uxaa8-1uxa95-uxa85uxaa5uxab5-7-uxa97uxaa3}

\textbf{સૂક્ષ્મ, લઘુ અને મધ્યમ ઉદ્યોગોને વ્યાખ્યાયિત કરો. રાષ્ટ્રના આર્થિક વિકાસમાં
MSME ની ભૂમિકા સમજાવો.}

\begin{solutionbox}

\textbf{MSME વર્ગીકરણ સારણી:}

\begin{longtable}[]{@{}lll@{}}
\toprule\noalign{}
\textbf{શ્રેણી} & \textbf{રોકાણ મર્યાદા} & \textbf{ટર્નઓવર મર્યાદા} \\
\midrule\noalign{}
\endhead
\bottomrule\noalign{}
\endlastfoot
\textbf{સૂક્ષ્મ} & ₹1 કરોડ સુધી & ₹5 કરોડ સુધી \\
\textbf{લઘુ} & ₹10 કરોડ સુધી & ₹50 કરોડ સુધી \\
\textbf{મધ્યમ} & ₹50 કરોડ સુધી & ₹250 કરોડ સુધી \\
\end{longtable}

\textbf{આર્થિક વિકાસમાં ભૂમિકા:}

\begin{itemize}
\tightlist
\item
  \textbf{રોજગાર સર્જન}: કુલ રોજગારનો 45\% ભાગ બનાવે છે
\item
  \textbf{નિકાસ યોગદાન}: કુલ નિકાસમાં 40\% યોગદાન
\item
  \textbf{GDP હિસ્સો}: ભારતની GDPમાં 30\% હિસ્સો
\item
  \textbf{નવીનતા હબ}: જમીની સ્તરે નવીનતાને પ્રોત્સાહન
\item
  \textbf{પ્રાદેશિક સંતુલન}: શહેરી-ગ્રામ્ય અંતર ઘટાડે છે
\end{itemize}

\end{solutionbox}
\begin{mnemonicbox}
``MSME બનાવે રોજગાર, નિકાસ, GDP, નવીનતા,
પ્રાદેશિક-સંતુલન''

\end{mnemonicbox}
\begin{center}\rule{0.5\linewidth}{0.5pt}\end{center}

\subsection*{પ્રશ્ન 2(અ) [3
ગુણ]}\label{uxaaauxab0uxab6uxaa8-2uxa85-3-uxa97uxaa3}

\textbf{બિઝનેસ આઇડિયાને વ્યાખ્યાયિત કરો અને તેના વિવિધ સ્રોતોની યાદી બનાવો.}

\begin{solutionbox}

\textbf{બિઝનેસ આઇડિયાની વ્યાખ્યા:} બજારની તકો ઓળખીને અને ઉકેલો સૂચવીને વ્યવસાય શરૂ
કરવાની કલ્પના.

\textbf{બિઝનેસ આઇડિયાના સ્રોતો:}

\begin{itemize}
\tightlist
\item
  \textbf{વ્યક્તિગત અનુભવ}: પોતાની કુશળતા અને રુચિઓ
\item
  \textbf{બજારની ખાલી જગ્યા}: અપૂર્ણ ગ્રાહક જરૂરિયાતો
\item
  \textbf{ટેકનોલોજી ટ્રેન્ડ્સ}: નવા તકનીકી વિકાસ
\item
  \textbf{સામાજિક સમસ્યાઓ}: સમુદાયિક મુદ્દાઓ જેના ઉકેલ જોઈએ
\item
  \textbf{ફ્રેન્ચાઇઝિંગ}: સાબિત વ્યવસાયિક મોડલ
\end{itemize}

\end{solutionbox}
\begin{mnemonicbox}
``વ્યક્તિગત બજાર ટેકનોલોજી સામાજિક ફ્રેન્ચાઇઝ''

\end{mnemonicbox}
\begin{center}\rule{0.5\linewidth}{0.5pt}\end{center}

\subsection*{પ્રશ્ન 2(બ) [4
ગુણ]}\label{uxaaauxab0uxab6uxaa8-2uxaac-4-uxa97uxaa3}

\textbf{SWOT વિશ્લેષણ પર ટૂંકી નોંધ લખો.}

\begin{solutionbox}

\textbf{SWOT વિશ્લેષણના ઘટકો:}

\begin{longtable}[]{@{}ll@{}}
\toprule\noalign{}
\textbf{આંતરિક પરિબળો} & \textbf{બાહ્ય પરિબળો} \\
\midrule\noalign{}
\endhead
\bottomrule\noalign{}
\endlastfoot
\textbf{શક્તિઓ} - આંતરિક ફાયદા & \textbf{તકો} - બાહ્ય તકો \\
\textbf{નબળાઈઓ} - આંતરિક મર્યાદાઓ & \textbf{ધમકીઓ} - બાહ્ય પડકારો \\
\end{longtable}

\textbf{હેતુ:}

\begin{itemize}
\tightlist
\item
  \textbf{વ્યૂહાત્મક આયોજન}: વ્યવસાયિક નિર્ણયોનું માર્ગદર્શન
\item
  \textbf{જોખમ મૂલ્યાંકન}: સંભવિત સમસ્યાઓ ઓળખે છે
\item
  \textbf{સંસાધન વિતરણ}: અસ્કયામતોનો મહત્તમ ઉપયોગ
\item
  \textbf{સ્પર્ધાત્મક વિશ્લેષણ}: બજારની સ્થિતિ સમજે છે
\end{itemize}

\end{solutionbox}
\begin{mnemonicbox}
``મજબૂત નબળી તકો ધમકી આપે છે''

\end{mnemonicbox}
\begin{center}\rule{0.5\linewidth}{0.5pt}\end{center}

\subsection*{પ્રશ્ન 2(ક) [7
ગુણ]}\label{uxaaauxab0uxab6uxaa8-2uxa95-7-uxa97uxaa3}

\textbf{ઉત્પાદન જીવન ચક્ર (PLC) ના વિવિધ તબક્કાઓ સમજાવો.}

\begin{solutionbox}

\begin{center}
\textbf{Mermaid Diagram (Code)}
\begin{verbatim}
{Shaded}
{Highlighting}[]
graph LR
    A[પરિચય] {-{-}{} B[વૃદ્ધિ]}
    B {-{-}{} C[પરિપક્વતા]}
    C {-{-}{} D[ઘટાડો]}
{Highlighting}
{Shaded}
\end{verbatim}
\end{center}

\textbf{PLC તબક્કાઓની સારણી:}

\begin{longtable}[]{@{}lll@{}}
\toprule\noalign{}
\textbf{તબક્કો} & \textbf{લક્ષણો} & \textbf{માર્કેટિંગ વ્યૂહરચના} \\
\midrule\noalign{}
\endhead
\bottomrule\noalign{}
\endlastfoot
\textbf{પરિચય} & ઓછું વેચાણ, વધારે ખર્ચ & જાગૃતિ બનાવવી \\
\textbf{વૃદ્ધિ} & ઝડપી વેચાણ વૃદ્ધિ & બજાર હિસ્સો વધારવો \\
\textbf{પરિપક્વતા} & ઉચ્ચ વેચાણ, તીવ્ર સ્પર્ધા & બજાર સ્થિતિ બચાવવી \\
\textbf{ઘટાડો} & વેચાણ અને નફામાં ઘટાડો & લાભ લેવો અથવા છોડી દેવું \\
\end{longtable}

\textbf{મુખ્ય પરિબળો:}

\begin{itemize}
\tightlist
\item
  \textbf{વેચાણ વોલ્યુમ}: જીવનચક્ર દરમિયાન બદલાય છે
\item
  \textbf{નફાના માર્જિન}: તબક્કા પ્રમાણે બદલાય છે
\item
  \textbf{સ્પર્ધાનું સ્તર}: સમય સાથે વધે છે
\item
  \textbf{માર્કેટિંગ ફોકસ}: દરેક તબક્કા સાથે બદલાય છે
\end{itemize}

\end{solutionbox}
\begin{mnemonicbox}
``પરિચય વૃદ્ધિ પરિપક્વ ઘટે છે''

\end{mnemonicbox}
\begin{center}\rule{0.5\linewidth}{0.5pt}\end{center}

\subsection*{પ્રશ્ન 2(અ) અથવા [3
ગુણ]}\label{uxaaauxab0uxab6uxaa8-2uxa85-uxa85uxaa5uxab5-3-uxa97uxaa3}

\textbf{પ્રોજેક્ટ રિપોર્ટની મૂળભૂત બાબતો શું છે?}

\begin{solutionbox}

\textbf{મૂળભૂત ઘટકો:}

\begin{itemize}
\tightlist
\item
  \textbf{એક્ઝિક્યુટિવ સમરી}: પ્રોજેક્ટની સામાન્ય માહિતી અને મુખ્ય બાબતો
\item
  \textbf{બજાર વિશ્લેષણ}: માંગ અને સ્પર્ધાનો અભ્યાસ
\item
  \textbf{તકનીકી વિગતો}: ઉત્પાદન પ્રક્રિયા અને ટેકનોલોજી
\item
  \textbf{નાણાકીય અંદાજો}: ખર્ચનો અંદાજ અને આવકનું આયોજન
\item
  \textbf{જોખમ મૂલ્યાંકન}: સંભવિત પડકારો અને તેના ઉકેલો
\end{itemize}

\end{solutionbox}
\begin{mnemonicbox}
``દરેક બજાર તકનીકી નાણાકીય જોખમ''

\end{mnemonicbox}
\begin{center}\rule{0.5\linewidth}{0.5pt}\end{center}

\subsection*{પ્રશ્ન 2(બ) અથવા [4
ગુણ]}\label{uxaaauxab0uxab6uxaa8-2uxaac-uxa85uxaa5uxab5-4-uxa97uxaa3}

\textbf{ઈ-કોમર્સના ફાયદા અને ગેરફાયદા સમજાવો.}

\begin{solutionbox}

\textbf{ઈ-કોમર્સના ફાયદા અને પડકારો:}

\begin{longtable}[]{@{}ll@{}}
\toprule\noalign{}
\textbf{ફાયદા} & \textbf{નુકસાનો} \\
\midrule\noalign{}
\endhead
\bottomrule\noalign{}
\endlastfoot
વૈશ્વિક પહોંચ અને 24/7 ઉપલબ્ધતા & સુરક્ષા અને ગોપનીયતાની ચિંતા \\
ઓછા ઓપરેશનલ ખર્ચ & વ્યક્તિગત સ્પર્શનો અભાવ \\
બજારમાં સરળ પ્રવેશ & ટેકનોલોજી પર નિર્ભરતા \\
બહેતર ગ્રાહક ટ્રેકિંગ & ડિલિવરી અને લોજિસ્ટિક્સની સમસ્યાઓ \\
\end{longtable}

\end{solutionbox}
\begin{mnemonicbox}
``વૈશ્વિક ઓછા સરળ બહેતર બનાવે સુરક્ષા વ્યક્તિગત ટેકનોલોજી
ડિલિવરી''

\end{mnemonicbox}
\begin{center}\rule{0.5\linewidth}{0.5pt}\end{center}

\subsection*{પ્રશ્ન 2(ક) અથવા [7
ગુણ]}\label{uxaaauxab0uxab6uxaa8-2uxa95-uxa85uxaa5uxab5-7-uxa97uxaa3}

\textbf{બજાર સંશોધન માટે પ્રશ્નાવલી ડિઝાઇનના વિવિધ પગલાં સમજાવો.}

\begin{solutionbox}

\textbf{પ્રશ્નાવલી ડિઝાઇન પ્રક્રિયા:}

\begin{verbatim}
flowchart LR
    A[સંશોધન હેતુઓ નક્કી કરવા] {-{-} B[જરૂરી માહિતી નક્કી કરવી]}
    B {-{-} C[પ્રશ્નના પ્રકાર પસંદ કરવા]}
    C {-{-} D[પ્રશ્નો વિકસાવવા]}
    D {-{-} E[પ્રશ્નોનો ક્રમ ગોઠવવો]}
    E {-{-} F[પ્રશ્નાવલીનું પૂર્વ પરીક્ષણ]}
    F {-{-} G[અંતિમ ડિઝાઇન]}
\end{verbatim}

\textbf{પ્રશ્ન ડિઝાઇનના તબક્કાઓ:}

\begin{itemize}
\tightlist
\item
  \textbf{ઉદ્દેશ્ય નિર્ધારણ}: સ્પષ્ટ સંશોધન લક્ષ્યો
\item
  \textbf{માહિતી આયોજન}: જરૂરી ડેટાના પ્રકારો
\item
  \textbf{પ્રશ્ન ફોર્મેટિંગ}: ખુલ્લા/બંધ પ્રશ્નો
\item
  \textbf{ભાષા પસંદગી}: સરળ અને સ્પષ્ટ
\item
  \textbf{ક્રમ આયોજન}: તાર્કિક પ્રવાહ
\item
  \textbf{પરીક્ષણ તબક્કો}: પાઇલટ અભ્યાસ
\item
  \textbf{અંતિમ સમીક્ષા}: ભૂલ સુધારો
\end{itemize}

\end{solutionbox}
\begin{mnemonicbox}
``ઉદ્દેશ્ય માહિતી ફોર્મેટ ભાષા ક્રમ પરીક્ષણ સમીક્ષા''

\end{mnemonicbox}
\begin{center}\rule{0.5\linewidth}{0.5pt}\end{center}

\subsection*{પ્રશ્ન 3(અ) [3
ગુણ]}\label{uxaaauxab0uxab6uxaa8-3uxa85-3-uxa97uxaa3}

\textbf{મેનેજમેન્ટ અને એડમિનિસ્ટ્રેશન વચ્ચે તફાવત કરો.}

\begin{solutionbox}

\begin{longtable}[]{@{}lll@{}}
\toprule\noalign{}
\textbf{પાસું} & \textbf{મેનેજમેન્ટ} & \textbf{એડમિનિસ્ટ્રેશન} \\
\midrule\noalign{}
\endhead
\bottomrule\noalign{}
\endlastfoot
\textbf{સ્તર} & મધ્યમ અને નીચલા સ્તરે & ઉચ્ચ સ્તરે \\
\textbf{કાર્ય} & નીતિઓનો અમલ & નીતિ ઘડતર \\
\textbf{કુશળતા} & તકનીકી અને માનવીય કુશળતા & વૈચારિક કુશળતા \\
\textbf{સત્તા} & મર્યાદિત સત્તા & અંતિમ સત્તા \\
\end{longtable}

\end{solutionbox}
\begin{mnemonicbox}
``મધ્યમ અમલ તકનીકી મર્યાદિત બનાવે ઉચ્ચ નીતિ વૈચારિક
અંતિમ''

\end{mnemonicbox}
\begin{center}\rule{0.5\linewidth}{0.5pt}\end{center}

\subsection*{પ્રશ્ન 3(બ) [4
ગુણ]}\label{uxaaauxab0uxab6uxaa8-3uxaac-4-uxa97uxaa3}

\textbf{ખાનગી કંપની અને હેર કંપની વચ્ચે તફાવત કરો.}

\begin{solutionbox}

\begin{longtable}[]{@{}lll@{}}
\toprule\noalign{}
\textbf{લક્ષણ} & \textbf{ખાનગી કંપની} & \textbf{હેર કંપની} \\
\midrule\noalign{}
\endhead
\bottomrule\noalign{}
\endlastfoot
\textbf{સભ્યો} & 2 થી 200 સભ્યો & ઓછામાં ઓછા 7, મહત્તમ મર્યાદા નથી \\
\textbf{શેર ટ્રાન્સફર} & મર્યાદિત ટ્રાન્સફર & મુક્ત ટ્રાન્સફર \\
\textbf{જાહેર ઇશ્યૂ} & જાહેરમાં શેર ઇશ્યૂ કરી શકતી નથી & જાહેરમાં ઇશ્યૂ કરી શકે છે \\
\textbf{માહિતી જાહેરાત} & મર્યાદિત જાહેરાત જરૂરિયાતો & વ્યાપક જાહેરાત \\
\end{longtable}

\end{solutionbox}
\begin{mnemonicbox}
``ખાનગી મર્યાદિત રાખે છે, હેર છૂટ આપે છે''

\end{mnemonicbox}
\begin{center}\rule{0.5\linewidth}{0.5pt}\end{center}

\subsection*{પ્રશ્ન 3(ક) [7
ગુણ]}\label{uxaaauxab0uxab6uxaa8-3uxa95-7-uxa97uxaa3}

\textbf{વ્યવસાયિક પ્રથા માટે બ્રેક-ઇવન વિશ્લેષણ સમજાવો.}

\begin{solutionbox}

\textbf{બ્રેક-ઇવન ફોર્મ્યુલા:} બ્રેક-ઇવન પોઇન્ટ = સ્થિર ખર્ચ \div (એકમ દીઠ વેચાણ કિંમત
- એકમ દીઠ ચલ ખર્ચ)

\begin{center}
\textbf{Mermaid Diagram (Code)}
\begin{verbatim}
{Shaded}
{Highlighting}[]
graph LR
    A[સ્થિર ખર્ચ] {-{-}{} B[ચલ ખર્ચ]}
    B {-{-}{} C[કુલ ખર્ચ]}
    C {-{-}{} D[આવક લાઇન]}
    D {-{-}{} E[બ્રેક{-}ઇવન પોઇન્ટ]}
{Highlighting}
{Shaded}
\end{verbatim}
\end{center}

\textbf{ઘટકો:}

\begin{itemize}
\tightlist
\item
  \textbf{સ્થિર ખર્ચ}: ભાડું, પગાર, વીમો
\item
  \textbf{ચલ ખર્ચ}: કાચો માલ, મજૂરી
\item
  \textbf{વેચાણ કિંમત}: એકમ દીઠ આવક
\item
  \textbf{યોગદાન માર્જિન}: કિંમત માઈનસ ચલ ખર્ચ
\end{itemize}

\textbf{ફાયદા:}

\begin{itemize}
\tightlist
\item
  \textbf{નફો આયોજન}: જરૂરી ન્યૂનતમ વેચાણ નક્કી કરે છે
\item
  \textbf{કિંમત નિર્ધારણ}: કિંમત નક્કી કરવામાં મદદ કરે છે
\item
  \textbf{ખર્ચ નિયંત્રણ}: ખર્ચનું માળખું ઓળખે છે
\item
  \textbf{રોકાણ નિર્ણયો}: પ્રોજેક્ટની વ્યવહારિકતા મૂલ્યાંકન કરે છે
\end{itemize}

\end{solutionbox}
\begin{mnemonicbox}
``સ્થિર ચલ વેચાણ યોગદાન બનાવે નફો કિંમત ખર્ચ રોકાણ''

\end{mnemonicbox}
\begin{center}\rule{0.5\linewidth}{0.5pt}\end{center}

\subsection*{પ્રશ્ન 3(અ) અથવા [3
ગુણ]}\label{uxaaauxab0uxab6uxaa8-3uxa85-uxa85uxaa5uxab5-3-uxa97uxaa3}

\textbf{નેતૃત્વ શું છે? તેના લક્ષણો આપો.}

\begin{solutionbox}

\textbf{નેતૃત્વની વ્યાખ્યા:} સામાન્ય લક્ષ્યો હાંસલ કરવા તરફ અન્ય લોકોને પ્રભાવિત અને
માર્ગદર્શન આપવાની ક્ષમતા.

\textbf{નેતૃત્વના લક્ષણો:}

\begin{itemize}
\tightlist
\item
  \textbf{દ્રષ્ટિ}: સ્પષ્ટ ભવિષ્યની દિશા
\item
  \textbf{સંવાદ}: અસરકારક અંતર્ક્રિયા કુશળતા
\item
  \textbf{પ્રામાણિકતા}: પ્રામાણિક અને નૈતિક વર્તન
\item
  \textbf{આત્મવિશ્વાસ}: નિર્ણયોમાં સ્વયં-વિશ્વાસ
\item
  \textbf{સહાનુભૂતિ}: અન્યોના દ્રષ્ટિકોણને સમજવું
\end{itemize}

\end{solutionbox}
\begin{mnemonicbox}
``દ્રષ્ટિવાન સંવાદકર્તા પ્રામાણિકતા, આત્મવિશ્વાસ, સહાનુભૂતિ
સાથે''

\end{mnemonicbox}
\begin{center}\rule{0.5\linewidth}{0.5pt}\end{center}

\subsection*{પ્રશ્ન 3(બ) અથવા [4
ગુણ]}\label{uxaaauxab0uxab6uxaa8-3uxaac-uxa85uxaa5uxab5-4-uxa97uxaa3}

\textbf{વ્યવસ્થાપનના કાર્યોની યાદી બનાવો અને આયોજન સમજાવો.}

\begin{solutionbox}

\textbf{વ્યવસ્થાપનના કાર્યો:}

\begin{itemize}
\tightlist
\item
  \textbf{આયોજન}: ઉદ્દેશ્યો અને વ્યૂહરચના નક્કી કરવી
\item
  \textbf{સંગઠન}: સંસાધનો અને માળખાની વ્યવસ્થા
\item
  \textbf{સ્ટાફિંગ}: લોકોને ભરતી અને વ્યવસ્થાપન
\item
  \textbf{નિર્દેશન}: કર્મચારીઓને નેતૃત્વ અને પ્રેરણા
\item
  \textbf{નિયંત્રણ}: કામગીરીની દેખરેખ અને સુધારો
\end{itemize}

\textbf{આયોજન પ્રક્રિયા:} આયોજનમાં લક્ષ્યો નક્કી કરવા, વ્યૂહરચના વિકસાવવી અને
ભવિષ્યની પ્રવૃત્તિઓ માટે કાર્ય યોજના બનાવવાનો સમાવેશ થાય છે.

\end{solutionbox}
\begin{mnemonicbox}
``લોકો સંગઠિત સ્ટાફ નિર્દેશિત નિયંત્રિત''

\end{mnemonicbox}
\begin{center}\rule{0.5\linewidth}{0.5pt}\end{center}

\subsection*{પ્રશ્ન 3(ક) અથવા [7
ગુણ]}\label{uxaaauxab0uxab6uxaa8-3uxa95-uxa85uxaa5uxab5-7-uxa97uxaa3}

\textbf{જોઇન્ટ સ્ટોક કંપનીની વિશેષતાઓ સમજાવો અને તેના ફાયદા લખો.}

\begin{solutionbox}

\textbf{જોઇન્ટ સ્ટોક કંપનીની વિશેષતાઓ:}

\begin{longtable}[]{@{}ll@{}}
\toprule\noalign{}
\textbf{વિશેષતા} & \textbf{વર્ણન} \\
\midrule\noalign{}
\endhead
\bottomrule\noalign{}
\endlastfoot
\textbf{અલગ કાનૂની અસ્તિત્વ} & સ્વતંત્ર કાનૂની દરજ્જો \\
\textbf{મર્યાદિત જવાબદારી} & સભ્યોની જવાબદારી મર્યાદિત \\
\textbf{હસ્તાંતરિત શેરો} & માલિકીનું સરળ હસ્તાંતરણ \\
\textbf{શાશ્વત ઉત્તરાધિકાર} & સતત અસ્તિત્વ \\
\textbf{કોમન સીલ} & સત્તાવાર હસ્તાક્ષર \\
\end{longtable}

\textbf{ફાયદા:}

\begin{itemize}
\tightlist
\item
  \textbf{મોટી મૂડી}: પર્યાપ્ત ભંડોળ એકત્રિત કરી શકે છે
\item
  \textbf{જોખમ વિતરણ}: મર્યાદિત જવાબદારીનું સુરક્ષા
\item
  \textbf{વ્યવસાયિક વ્યવસ્થાપન}: કુશળ મેનેજરો
\item
  \textbf{વિશ્વસનીયતા}: જાહેર વિશ્વાસ અને માન્યતા
\item
  \textbf{વૃદ્ધિની તકો}: વિસ્તરણની શક્યતાઓ
\end{itemize}

\end{solutionbox}
\begin{mnemonicbox}
``અલગ મર્યાદિત હસ્તાંતરિત શાશ્વત કોમન આપે મોટી જોખમ
વ્યવસાયિક વિશ્વસનીય વૃદ્ધિ''

\end{mnemonicbox}
\begin{center}\rule{0.5\linewidth}{0.5pt}\end{center}

\subsection*{પ્રશ્ન 4(અ) [3
ગુણ]}\label{uxaaauxab0uxab6uxaa8-4uxa85-3-uxa97uxaa3}

\textbf{નવા ઉદ્યોગસાહસિક વિકાસમાં સહાય માટે કોઈપણ ત્રણ રાજ્ય અથવા રાષ્ટ્રીય
સ્તરની નાણાકીય સંસ્થાઓની નોંધણી કરો.}

\begin{solutionbox}

\textbf{નાણાકીય સંસ્થાઓ:}

\begin{itemize}
\tightlist
\item
  \textbf{SIDBI}: સ્મોલ ઇન્ડસ્ટ્રીઝ ડેવલપમેન્ટ બેન્ક ઓફ ઇન્ડિયા
\item
  \textbf{NABARD}: નેશનલ બેન્ક ફોર એગ્રિકલ્ચર એન્ડ રુરલ ડેવલપમેન્ટ
\item
  \textbf{NSIC}: નેશનલ સ્મોલ ઇન્ડસ્ટ્રીઝ કોર્પોરેશન
\item
  \textbf{રાજ્ય નાણાકીય નિગમો}: રાજ્ય-સ્તરીય ભંડોળ પુરવઠો
\item
  \textbf{MUDRA}: માઇક્રો યુનિટ્સ ડેવલપમેન્ટ \& રિફાઇનાન્સ એજન્સી
\end{itemize}

\end{solutionbox}
\begin{mnemonicbox}
``SIDBI NABARD NSIC રાજ્ય MUDRA''

\end{mnemonicbox}
\begin{center}\rule{0.5\linewidth}{0.5pt}\end{center}

\subsection*{પ્રશ્ન 4(બ) [4
ગુણ]}\label{uxaaauxab0uxab6uxaa8-4uxaac-4-uxa97uxaa3}

\textbf{સ્ટાર્ટઅપ મેન્ટર શું છે? મેન્ટર રાખવાના ફાયદા શું છે?}

\begin{solutionbox}

\textbf{સ્ટાર્ટઅપ મેન્ટરની વ્યાખ્યા:} અનુભવી વ્યવસાયિક જે ઉદ્યોગસાહસિકોને સલાહ,
કનેક્શન અને સહાયતા સાથે માર્ગદર્શન આપે છે.

\textbf{મેન્ટરશિપના ફાયદા:}

\begin{longtable}[]{@{}ll@{}}
\toprule\noalign{}
\textbf{ફાયદો} & \textbf{વર્ણન} \\
\midrule\noalign{}
\endhead
\bottomrule\noalign{}
\endlastfoot
\textbf{અનુભવ શેરિંગ} & ભૂતકાળની ભૂલોથી શીખવું \\
\textbf{નેટવર્ક એક્સેસ} & ઉદ્યોગના કનેક્શનો \\
\textbf{વ્યૂહાત્મક માર્ગદર્શન} & વ્યવસાયિક દિશાની સલાહ \\
\textbf{કુશળતા વિકાસ} & વ્યક્તિગત વૃદ્ધિ સહાયતા \\
\end{longtable}

\end{solutionbox}
\begin{mnemonicbox}
``અનુભવ નેટવર્ક વ્યૂહરચના કુશળતા''

\end{mnemonicbox}
\begin{center}\rule{0.5\linewidth}{0.5pt}\end{center}

\subsection*{પ્રશ્ન 4(ક) [7
ગુણ]}\label{uxaaauxab0uxab6uxaa8-4uxa95-7-uxa97uxaa3}

\textbf{વિવિધ પ્રકારના નેતૃત્વ મોડલો સમજાવો.}

\begin{solutionbox}

\textbf{નેતૃત્વ મોડલો:}

\begin{longtable}[]{@{}lll@{}}
\toprule\noalign{}
\textbf{મોડલ} & \textbf{લક્ષણો} & \textbf{લાગુ પાડવાનું} \\
\midrule\noalign{}
\endhead
\bottomrule\noalign{}
\endlastfoot
\textbf{સત્તાવાદી} & કેન્દ્રીય નિર્ણય લેવાની & કટોકટીની પરિસ્થિતિઓ \\
\textbf{લોકશાહી} & સહભાગી અભિગમ & ટીમ વાતાવરણ \\
\textbf{લેસેઝ-ફેર} & હાથ છોડીને કામ & સર્જનાત્મક પ્રોજેક્ટ્સ \\
\textbf{પરિવર્તનકારી} & પ્રેરણાદાયક નેતૃત્વ & પરિવર્તન વ્યવસ્થાપન \\
\textbf{વ્યવહારિક} & પુરસ્કાર આધારિત સિસ્ટમ & નિયમિત કામકાજ \\
\end{longtable}

\textbf{પસંદગીના પરિબળો:}

\begin{itemize}
\tightlist
\item
  \textbf{પરિસ્થિતિ}: સંદર્ભ શૈલી નક્કી કરે છે
\item
  \textbf{ટીમની પરિપક્વતા}: અનુભવનું સ્તર મહત્વપૂર્ણ
\item
  \textbf{સંસ્થાકીય સંસ્કૃતિ}: કંપનીના મૂલ્યો
\item
  \textbf{કાર્યની જટિલતા}: કામનું સ્વરૂપ
\end{itemize}

\end{solutionbox}
\begin{mnemonicbox}
``સત્તાવાદી લોકશાહી લેસેઝ પરિવર્તનકારી વ્યવહારિક સાથે
પરિસ્થિતિ ટીમ સંસ્થાકીય કાર્ય''

\end{mnemonicbox}
\begin{center}\rule{0.5\linewidth}{0.5pt}\end{center}

\subsection*{પ્રશ્ન 4(અ) અથવા [3
ગુણ]}\label{uxaaauxab0uxab6uxaa8-4uxa85-uxa85uxaa5uxab5-3-uxa97uxaa3}

\textbf{સ્ટાર્ટઅપ ઇન્ક્યુબેટર્સ પર ટૂંકી નોંધ લખો.}

\begin{solutionbox}

\textbf{સ્ટાર્ટઅપ ઇન્ક્યુબેટર્સ:} એવી સંસ્થાઓ જે પ્રારંભિક તબક્કાના સ્ટાર્ટઅપ્સને સંસાધનો,
માર્ગદર્શન અને ભંડોળ સાથે સહાયતા કરે છે.

\textbf{આપવામાં આવતી સેવાઓ:}

\begin{itemize}
\tightlist
\item
  \textbf{ઓફિસ સ્પેસ}: સાઝી કામકાજની સુવિધાઓ
\item
  \textbf{મેન્ટરશિપ}: નિષ્ણાત માર્ગદર્શન અને સલાહ
\item
  \textbf{ફંડિંગ}: સીડ કેપિટલ અને રોકાણ
\item
  \textbf{નેટવર્કિંગ}: ઉદ્યોગના કનેક્શનો
\item
  \textbf{તાલીમ}: કુશળતા વિકાસ કાર્યક્રમો
\end{itemize}

\end{solutionbox}
\begin{mnemonicbox}
``ઓફિસ મેન્ટરશિપ ફંડિંગ નેટવર્કિંગ તાલીમ''

\end{mnemonicbox}
\begin{center}\rule{0.5\linewidth}{0.5pt}\end{center}

\subsection*{પ્રશ્ન 4(બ) અથવા [4
ગુણ]}\label{uxaaauxab0uxab6uxaa8-4uxaac-uxa85uxaa5uxab5-4-uxa97uxaa3}

\textbf{IPR શું છે? તેનું મહત્વ લખો.}

\begin{solutionbox}

\textbf{IPR ની વ્યાખ્યા:} બૌદ્ધિક સંપત્તિ અધિકારો મનના સર્જનને સુરક્ષિત કરે છે જેમાં
શોધ, ડિઝાઇન અને કલાત્મક કૃતિઓનો સમાવેશ થાય છે.

\textbf{મહત્વ:}

\begin{longtable}[]{@{}ll@{}}
\toprule\noalign{}
\textbf{ફાયદો} & \textbf{વર્ણન} \\
\midrule\noalign{}
\endhead
\bottomrule\noalign{}
\endlastfoot
\textbf{નવીનતા સુરક્ષા} & સર્જનાત્મક વિચારોનું રક્ષણ \\
\textbf{વ્યાપારિક મૂલ્ય} & વ્યવસાયિક અસ્કયામતો બનાવે છે \\
\textbf{બજાર લાભ} & સ્પર્ધાત્મક તફાવત \\
\textbf{આવક ઉત્પાદન} & લાઇસન્સિંગની તકો \\
\end{longtable}

\end{solutionbox}
\begin{mnemonicbox}
``નવીનતા વ્યાપારિક બજાર આવક''

\end{mnemonicbox}
\begin{center}\rule{0.5\linewidth}{0.5pt}\end{center}

\subsection*{પ્રશ્ન 4(ક) અથવા [7
ગુણ]}\label{uxaaauxab0uxab6uxaa8-4uxa95-uxa85uxaa5uxab5-7-uxa97uxaa3}

\textbf{નાણાકીય સંગઠન વ્યવસ્થાપન વિશે ચર્ચા કરો.}

\begin{solutionbox}

\textbf{નાણાકીય સંગઠન માળખું:}

\begin{center}
\textbf{Mermaid Diagram (Code)}
\begin{verbatim}
{Shaded}
{Highlighting}[]
graph TD
    A[મુખ્ય નાણાકીય અધિકારી] {-{-}{} B[નાણાકીય નિયંત્રક]}
    A {-{-}{} C[ખજાનચી]}
    B {-{-}{} D[એકાઉન્ટિંગ મેનેજર]}
    B {-{-}{} E[બજેટ મેનેજર]}
    C {-{-}{} F[રોકડ વ્યવસ્થાપક]}
    C {-{-}{} G[રોકાણ વ્યવસ્થાપક]}
{Highlighting}
{Shaded}
\end{verbatim}
\end{center}

\textbf{મુખ્ય કાર્યો:}

\begin{itemize}
\tightlist
\item
  \textbf{નાણાકીય આયોજન}: બજેટ તૈયારી અને અંદાજ
\item
  \textbf{ફંડ મેનેજમેન્ટ}: મૂડી માળખું અને તરલતા
\item
  \textbf{જોખમ વ્યવસ્થાપન}: નાણાકીય જોખમ મૂલ્યાંકન
\item
  \textbf{કામગીરી દેખરેખ}: નાણાકીય વિશ્લેષણ અને રિપોર્ટિંગ
\item
  \textbf{અનુપાલન}: નિયમનકારી આવશ્યકતાઓનું પાલન
\item
  \textbf{રોકાણ નિર્ણયો}: મૂડી વિતરણ વ્યૂહરચના
\end{itemize}

\textbf{સંગઠનાત્મક ફાયદા:}

\begin{itemize}
\tightlist
\item
  \textbf{સ્પષ્ટ જવાબદારી}: નિર્ધારિત ભૂમિકાઓ અને જવાબદારીઓ
\item
  \textbf{કાર્યક્ષમ કામકાજ}: સુવ્યવસ્થિત નાણાકીય પ્રક્રિયાઓ
\item
  \textbf{બહેતર નિયંત્રણ}: વર્ધિત નાણાકીય દેખરેખ
\item
  \textbf{વ્યૂહાત્મક સહાયતા}: માહિતિગર નિર્ણય લેવાની
\end{itemize}

\end{solutionbox}
\begin{mnemonicbox}
``આયોજન ફંડ જોખમ કામગીરી અનુપાલન રોકાણ આપે સ્પષ્ટ કાર્યક્ષમ
બહેતર વ્યૂહાત્મક''

\end{mnemonicbox}
\begin{center}\rule{0.5\linewidth}{0.5pt}\end{center}

\subsection*{પ્રશ્ન 5(અ) [3
ગુણ]}\label{uxaaauxab0uxab6uxaa8-5uxa85-3-uxa97uxaa3}

\textbf{પ્રોજેક્ટ પ્લાનિંગ શું છે? તેના હેતુઓ લખો.}

\begin{solutionbox}

\textbf{પ્રોજેક્ટ પ્લાનિંગની વ્યાખ્યા:} પ્રોજેક્ટના વ્યાપ, ઉદ્દેશ્યો અને લક્ષ્યો હાંસલ
કરવા માટે જરૂરી પગલાંને વ્યાખ્યાયિત કરવાની પ્રક્રિયા.

\textbf{હેતુઓ:}

\begin{itemize}
\tightlist
\item
  \textbf{લક્ષ્ય નિર્ધારણ}: સ્પષ્ટ ઉદ્દેશ્યની વ્યાખ્યા
\item
  \textbf{સંસાધન વિતરણ}: સંસાધનોનો કાર્યક્ષમ ઉપયોગ
\item
  \textbf{સમય વ્યવસ્થાપન}: શેડ્યૂલ વિકાસ
\item
  \textbf{જોખમ ઘટાડવું}: સમસ્યાની અપેક્ષા
\item
  \textbf{ગુણવત્તા ખાતરી}: ધોરણ જાળવણી
\end{itemize}

\end{solutionbox}
\begin{mnemonicbox}
``લક્ષ્ય સંસાધન સમય જોખમ ગુણવત્તા''

\end{mnemonicbox}
\begin{center}\rule{0.5\linewidth}{0.5pt}\end{center}

\subsection*{પ્રશ્ન 5(બ) [4
ગુણ]}\label{uxaaauxab0uxab6uxaa8-5uxaac-4-uxa97uxaa3}

\textbf{પ્રોજેક્ટ ખર્ચ અંદાજ પર ટૂંકી નોંધ લખો.}

\begin{solutionbox}

\textbf{ખર્ચ અંદાજના ઘટકો:}

\begin{longtable}[]{@{}ll@{}}
\toprule\noalign{}
\textbf{ખર્ચનો પ્રકાર} & \textbf{ઉદાહરણો} \\
\midrule\noalign{}
\endhead
\bottomrule\noalign{}
\endlastfoot
\textbf{પ્રત્યક્ષ ખર્ચ} & સામગ્રી, મજૂરી, સાધનો \\
\textbf{અપ્રત્યક્ષ ખર્ચ} & ઓવરહેડ, પ્રશાસન \\
\textbf{સ્થિર ખર્ચ} & ભાડું, વીમો, પગાર \\
\textbf{ચલ ખર્ચ} & કાચો માલ, યુટિલિટીઝ \\
\end{longtable}

\textbf{અંદાજની પદ્ધતિઓ:}

\begin{itemize}
\tightlist
\item
  \textbf{બોટમ-અપ}: વિસ્તૃત પ્રવૃત્તિ ખર્ચ
\item
  \textbf{ટોપ-ડાઉન}: ઉચ્ચ-સ્તરનો અંદાજ
\item
  \textbf{સમાન}: ઐતિહાસિક પ્રોજેક્ટ સરખામણી
\item
  \textbf{પેરામેટ્રિક}: ગાણિતિક મોડેલો
\end{itemize}

\end{solutionbox}
\begin{mnemonicbox}
``પ્રત્યક્ષ અપ્રત્યક્ષ સ્થિર ચલ વાપરીને બોટમ ટોપ સમાન
પેરામેટ્રિક''

\end{mnemonicbox}
\begin{center}\rule{0.5\linewidth}{0.5pt}\end{center}

\subsection*{પ્રશ્ન 5(ક) [7
ગુણ]}\label{uxaaauxab0uxab6uxaa8-5uxa95-7-uxa97uxaa3}

\textbf{વ્યવસાયમાં શક્યતા વિશ્લેષણ શું છે? શક્યતા અભ્યાસમાં આવરી લેવામાં આવતા વિવિધ
વિશ્લેષણ સમજાવો.}

\begin{solutionbox}

\textbf{શક્યતા વિશ્લેષણની વ્યાખ્યા:} અમલીકરણ પહેલાં પ્રોજેક્ટની વ્યવહારિકતાનું
વ્યવસ્થિત મૂલ્યાંકન.

\textbf{શક્યતા વિશ્લેષણના પ્રકારો:}

\begin{longtable}[]{@{}lll@{}}
\toprule\noalign{}
\textbf{વિશ્લેષણનો પ્રકાર} & \textbf{ફોકસ ક્ષેત્ર} & \textbf{મુખ્ય પ્રશ્નો} \\
\midrule\noalign{}
\endhead
\bottomrule\noalign{}
\endlastfoot
\textbf{તકનીકી} & ટેકનોલોજી અને પ્રક્રિયાઓ & શું તે બનાવી શકાય? \\
\textbf{આર્થિક} & નાણાકીય વ્યવહારિકતા & શું તે નફાકારક છે? \\
\textbf{બજાર} & માંગ અને સ્પર્ધા & શું તે વેચાશે? \\
\textbf{કાનૂની} & નિયમનકારી અનુપાલન & શું તે કાયદેસર છે? \\
\textbf{કામકાજી} & અમલીકરણ ક્ષમતા & શું આપણે તેનું સંચાલન કરી શકીએ? \\
\textbf{શેડ્યૂલ} & સમયની મર્યાદાઓ & શું આપણે સમયસર પહોંચાડી શકીએ? \\
\end{longtable}

\textbf{ફાયદા:}

\begin{itemize}
\tightlist
\item
  \textbf{જોખમ ઘટાડવું}: શરૂઆતમાં સંભવિત સમસ્યાઓ ઓળખે છે
\item
  \textbf{સંસાધન ઑપ્ટિમાઇઝેશન}: વ્યર્થ રોકાણને અટકાવે છે
\item
  \textbf{નિર્ણય સહાયતા}: આગળ વધવા/ન વધવાના નિર્ણયો માટે ડેટા પ્રદાન કરે છે
\item
  \textbf{હિસ્સેદારોનો વિશ્વાસ}: રોકાણકારોનો વિશ્વાસ વધારે છે
\end{itemize}

\textbf{પ્રક્રિયા પ્રવાહ:}

\begin{enumerate}
\tightlist
\item
  પ્રોજેક્ટનો વ્યાપ અને ઉદ્દેશ્યો વ્યાખ્યાયિત કરવા
\item
  વિવિધ શક્યતા વિશ્લેષણ હાથ ધરવા
\item
  પરિણામો અને વિકલ્પોનું મૂલ્યાંકન કરવું
\item
  અમલીકરણની ભલામણ કરવી
\end{enumerate}

\end{solutionbox}
\begin{mnemonicbox}
``તકનીકી આર્થિક બજાર કાનૂની કામકાજી શેડ્યૂલ ઘટાડે જોખમ,
ઑપ્ટિમાઇઝ સંસાધન, સહાય નિર્ણય, બનાવે વિશ્વાસ''

\end{mnemonicbox}
\begin{center}\rule{0.5\linewidth}{0.5pt}\end{center}

\subsection*{પ્રશ્ન 5(અ) અથવા [3
ગુણ]}\label{uxaaauxab0uxab6uxaa8-5uxa85-uxa85uxaa5uxab5-3-uxa97uxaa3}

\textbf{નાદારી શું છે? વ્યવસાયમાં નાદારીનાં કારણો લખો.}

\begin{solutionbox}

\textbf{નાદારીની વ્યાખ્યા:} કાનૂની પ્રક્રિયા જ્યાં વ્યક્તિઓ અથવા વ્યવસાયો કે જેઓ દેવું
ચૂકવવામાં અસમર્થ છે તેઓ લેણદારોથી રાહત માંગે છે.

\textbf{વ્યવસાયિક નાદારીના કારણો:}

\begin{itemize}
\tightlist
\item
  \textbf{નબળું નાણાકીય વ્યવસ્થાપન}: અપૂરતું કેશ ફ્લો નિયંત્રણ
\item
  \textbf{બજારી સ્પર્ધા}: બજાર હિસ્સાની હાનિ
\item
  \textbf{આર્થિક મંદી}: બાહ્ય આર્થિક પરિબળો
\item
  \textbf{અતિ-વિસ્તરણ}: યોગ્ય આયોજન વિના ઝડપી વૃદ્ધિ
\item
  \textbf{ઉચ્ચ દેવાનો બોજ}: વધુ પડતા ધિરાણ ખર્ચ
\end{itemize}

\end{solutionbox}
\begin{mnemonicbox}
``નબળું બજાર આર્થિક અતિ ઉચ્ચ''

\end{mnemonicbox}
\begin{center}\rule{0.5\linewidth}{0.5pt}\end{center}

\subsection*{પ્રશ્ન 5(બ) અથવા [4
ગુણ]}\label{uxaaauxab0uxab6uxaa8-5uxaac-uxa85uxaa5uxab5-4-uxa97uxaa3}

\textbf{નિશ્ચિત ખર્ચ અને ચલ ખર્ચ વચ્ચેનો તફાવત જણાવો.}

\begin{solutionbox}

\begin{longtable}[]{@{}lll@{}}
\toprule\noalign{}
\textbf{પાસું} & \textbf{નિશ્ચિત ખર્ચ} & \textbf{ચલ ખર્ચ} \\
\midrule\noalign{}
\endhead
\bottomrule\noalign{}
\endlastfoot
\textbf{સ્વભાવ} & સ્થિર રહે છે & ઉત્પાદન સાથે બદલાય છે \\
\textbf{ઉદાહરણો} & ભાડું, વીમો, પગાર & કાચો માલ, મજૂરી \\
\textbf{વર્તન} & આઉટપુટથી સ્વતંત્ર & સીધા પ્રમાણમાં \\
\textbf{નિયંત્રણ} & બદલવું મુશ્કેલ & સંચાલન સરળ \\
\end{longtable}

\end{solutionbox}
\begin{mnemonicbox}
``નિશ્ચિત રહે છે, ચલ બદલાય છે''

\end{mnemonicbox}
\begin{center}\rule{0.5\linewidth}{0.5pt}\end{center}

\subsection*{પ્રશ્ન 5(ક) અથવા [7
ગુણ]}\label{uxaaauxab0uxab6uxaa8-5uxa95-uxa85uxaa5uxab5-7-uxa97uxaa3}

\textbf{વિવિધ વ્યવસાયિક નીતિશાસ્ત્ર અને તેમના મહત્વનું વર્ણન કરો.}

\begin{solutionbox}

\textbf{વ્યવસાયિક નીતિશાસ્ત્રના પ્રકારો:}

\begin{longtable}[]{@{}
  >{\raggedright\arraybackslash}p{(\linewidth - 4\tabcolsep) * \real{0.4048}}
  >{\raggedright\arraybackslash}p{(\linewidth - 4\tabcolsep) * \real{0.2857}}
  >{\raggedright\arraybackslash}p{(\linewidth - 4\tabcolsep) * \real{0.3095}}@{}}
\toprule\noalign{}
\begin{minipage}[b]{\linewidth}\raggedright
\textbf{નીતિશાસ્ત્રનું ક્ષેત્ર}
\end{minipage} & \begin{minipage}[b]{\linewidth}\raggedright
\textbf{વર્ણન}
\end{minipage} & \begin{minipage}[b]{\linewidth}\raggedright
\textbf{પ્રથાઓ}
\end{minipage} \\
\midrule\noalign{}
\endhead
\bottomrule\noalign{}
\endlastfoot
\textbf{કોર્પોરેટ જવાબદારી} & સમાજ પ્રત્યે કંપનીની ફરજ & CSR પ્રવૃત્તિઓ,
પર્યાવરણીય કાળજી \\
\textbf{કર્મચારી સંબંધો} & કામદારો સાથે ન્યાયી વર્તન & સમાન તકો, સુરક્ષા \\
\textbf{ગ્રાહક સંબંધો} & ક્લાયન્ટ્સ સાથે પ્રામાણિક વ્યવહાર & ગુણવત્તાયુક્ત ઉત્પાદનો,
ન્યાયી કિંમત \\
\textbf{સપ્લાયર સંબંધો} & નૈતિક પ્રાપ્તિ & ન્યાયી કરારો, સમયસર ચુકવણી \\
\textbf{પર્યાવરણીય નીતિશાસ્ત્ર} & ટકાઉ પ્રથાઓ & કચરો ઘટાડવો, લીલી
ટેકનોલોજી \\
\end{longtable}

\textbf{મહત્વ:}

\begin{itemize}
\tightlist
\item
  \textbf{પ્રતિષ્ઠા નિર્માણ}: સકારાત્મક બ્રાન્ડ ઇમેજ બનાવે છે
\item
  \textbf{હિસ્સેદારોનો વિશ્વાસ}: બધા પક્ષોમાં વિશ્વાસ નિર્માણ કરે છે
\item
  \textbf{કાનૂની અનુપાલન}: નિયમનકારી મુદ્દાઓને ટાળે છે
\item
  \textbf{કર્મચારી પ્રેરણા}: કામના વાતાવરણમાં સુધારો
\item
  \textbf{લાંબા ગાળાની સફળતા}: ટકાઉ વૃદ્ધિની ખાતરી
\item
  \textbf{સ્પર્ધાત્મક લાભ}: સ્પર્ધકોથી અલગ પાડે છે
\end{itemize}

\textbf{અમલીકરણ માળખું:}

\begin{enumerate}
\tightlist
\item
  નૈતિક કોડ્સ અને નીતિઓ વિકસાવવી
\item
  કર્મચારીઓને નીતિશાસ્ત્રની તાલીમ આપવી
\item
  રિપોર્ટિંગ મેકેનિઝમ સ્થાપિત કરવું
\item
  નૈતિક પ્રથાઓનું દેખરેખ અને મૂલ્યાંકન કરવું
\item
  જરૂર પડ્યે સુધારાત્મક પગલાં લેવા
\end{enumerate}

\textbf{વ્યવસાય માટેના ફાયદા:}

\begin{itemize}
\tightlist
\item
  \textbf{જોખમ વ્યવસ્થાપન}: નૈતિક કૌભાંડોને અટકાવે છે
\item
  \textbf{ગ્રાહક વફાદારી}: સ્થાયી સંબંધો બનાવે છે
\item
  \textbf{રોકાણકારોનો વિશ્વાસ}: નૈતિક રોકાણકારોને આકર્ષે છે
\item
  \textbf{નિયમનકારી સહાય}: સરકાર સાથે સારા સંબંધો જાળવે છે
\end{itemize}

\end{solutionbox}
\begin{mnemonicbox}
``કોર્પોરેટ કર્મચારી ગ્રાહક સપ્લાયર પર્યાવરણીય બનાવે
પ્રતિષ્ઠા વિશ્વાસ કાનૂની પ્રેરણા સફળતા સ્પર્ધાત્મક જોખમ ગ્રાહક રોકાણકાર નિયમનકારી
દ્વારા''

\end{mnemonicbox}

\end{document}
