\documentclass[10pt,a4paper]{article}

% content/resources/templates/preamble.tex
\usepackage[margin=0.6in]{geometry}
\author{Milav Dabgar}
\usepackage{amsmath,amssymb,amsthm}
\usepackage{booktabs}
\usepackage{multirow}
\usepackage{xcolor}
\usepackage{tcolorbox}
\tcbuselibrary{breakable,skins}
\usepackage[colorlinks=true,linkcolor=blue]{hyperref}
\usepackage{titlesec}
\usepackage{enumitem}
\usepackage{tikz}
\usepackage{pgfplots}
\usepackage{circuitikz}
\usepackage[version=4]{mhchem}
\usepackage{longtable}
\usepackage{array}
\usepackage{float}
\usepackage{caption}
\usepackage{listings}

\lstset{
  basicstyle=\small\ttfamily,
  breaklines=true,
  breakatwhitespace=false,
  postbreak=\mbox{\textcolor{red}{$\hookrightarrow$}\space},
  float=false,
  numbers=left,
  numberstyle=\tiny\color{gray},
  numbersep=10pt,
  xleftmargin=2em,
  keywordstyle=\color{blue},
  commentstyle=\color{green!60!black},
  stringstyle=\color{purple},
  backgroundcolor=\color{gray!5},
  showstringspaces=false,
  tabsize=2,
  captionpos=b,
  keepspaces=true,
  columns=flexible
}

\pgfplotsset{compat=1.18}
\usetikzlibrary{shapes,arrows,positioning,calc,patterns,decorations.pathmorphing,decorations.markings,arrows.meta}

% Color scheme
\definecolor{headcolor}{RGB}{0,102,204}
\definecolor{keycolor}{RGB}{220,20,60}
\definecolor{solutioncolor}{RGB}{34,139,34}
\definecolor{mnemoniccolor}{RGB}{148,0,211}
\definecolor{codecolor}{RGB}{0,0,100}

% Spacing
\setlength{\parskip}{3pt}
\setlist[itemize]{nosep}
\setlist[enumerate]{nosep}

% Title formatting
\titleformat{\section}{\Large\bfseries\color{headcolor}}{\thesection}{1em}{}
\titleformat{\subsection}{\large\bfseries\color{headcolor}}{\thesubsection}{1em}{}

% Pandoc tightlist compatibility
\providecommand{\tightlist}{%
  \setlength{\itemsep}{0pt}\setlength{\parskip}{0pt}}

% Pandoc longtable compatibility
\newcounter{none}
\def\thenone{}


% content/resources/templates/gujarati-boxes.tex
\usepackage{fontspec}
\usepackage{polyglossia}

% Set Gujarati as main language (document is primarily in Gujarati)
% Note: gloss-gujarati.ldf doesn't exist in polyglossia, but it will use hyphenation patterns
\setdefaultlanguage{gujarati}
\setotherlanguage{english}

% Configure Gujarati font properly
% Use Language=Default to prevent polyglossia from trying to add language-specific features
% that don't exist for Gujarati, which causes "empty feature" warnings
\newfontfamily\gujaratifont[Script=Gujarati,AutoFakeBold=2.5,AutoFakeSlant=0.3]{Noto Sans Gujarati}
\setmainfont[Script=Gujarati,AutoFakeBold=2.5,AutoFakeSlant=0.3]{Noto Sans Gujarati}
% Use Noto Sans Gujarati for monospace to support Gujarati in text
\setmonofont[Scale=0.9]{Noto Sans Gujarati}

% Configure English to use the same font
\newfontfamily\englishfont[Script=Gujarati,AutoFakeBold=2.5,AutoFakeSlant=0.3]{Noto Sans Gujarati}

% Translations for polyglossia
\gappto\captionsgujarati{
  \renewcommand{\tablename}{કોષ્ટક}
  \renewcommand{\figurename}{આકૃતિ}
}

% Helper for TikZ nodes to ensure Gujarati font
\newcommand{\gu}[1]{{\gujaratifont #1}}

% Custom environments
\newtcolorbox{solutionbox}{
    breakable,
    enhanced,
    colback=solutioncolor!5!white,
    colframe=solutioncolor!75!black,
    fonttitle=\bfseries,
    title=જવાબ
}

\newtcolorbox{solutionboxnobreak}{
 colback=solutioncolor!5!white,
 colframe=solutioncolor!75!black,
 fonttitle=\bfseries,
 title=જવાબ
}

\newtcolorbox{keyformula}{
 breakable,
 enhanced,
 colback=keycolor!5!white,
 colframe=keycolor!75!black,
 fonttitle=\bfseries,
 title=રાસાયણિક સમીકરણ/સૂત્ર
}

\newtcolorbox{mnemonicbox}{
 breakable,
 enhanced,
 colback=mnemoniccolor!5!white,
 colframe=mnemoniccolor!75!black,
 fonttitle=\bfseries,
 title=મેમરી ટ્રીક
}


\begin{document}

\begin{center}
{\Huge\bfseries\color{headcolor} Subject Name (Gujarati)}\\[5pt]
{\LARGE 4300021 -- Winter 2024}\\[3pt]
{\large Semester 1 Study Material}\\[3pt]
{\normalsize\textit{Detailed Solutions and Explanations}}
\end{center}

\vspace{10pt}

\subsection*{પ્રશ્ન 1(અ) [3
ગુણ]}\label{uxaaauxab0uxab6uxaa8-1uxa85-3-uxa97uxaa3}

\textbf{આંત્રપ્રન્યોર અને મેનેજર વચ્ચેનો તફાવત આપો.}

\begin{solutionbox}

\begin{longtable}[]{@{}
  >{\raggedright\arraybackslash}p{(\linewidth - 4\tabcolsep) * \real{0.2414}}
  >{\raggedright\arraybackslash}p{(\linewidth - 4\tabcolsep) * \real{0.4483}}
  >{\raggedright\arraybackslash}p{(\linewidth - 4\tabcolsep) * \real{0.3103}}@{}}
\toprule\noalign{}
\begin{minipage}[b]{\linewidth}\raggedright
પાસાં
\end{minipage} & \begin{minipage}[b]{\linewidth}\raggedright
આંત્રપ્રન્યોર
\end{minipage} & \begin{minipage}[b]{\linewidth}\raggedright
મેનેજર
\end{minipage} \\
\midrule\noalign{}
\endhead
\bottomrule\noalign{}
\endlastfoot
\textbf{મુખ્ય ભૂમિકા} & નવા વ્યવસાય અને તકો બનાવે છે & હાલની કામગીરીનું સંચાલન કરે
છે \\
\textbf{જોખમ લેવું} & ઉચ્ચ જોખમ લેનાર, અનિશ્ચિતતા સહન કરે છે & ઓછું થી મધ્યમ જોખમ,
માર્ગદર્શિકા અનુસરે છે \\
\textbf{નિર્ણય લેવું} & ઝડપી, સહજ નિર્ણયો & વ્યવસ્થિત, નીતિ આધારિત નિર્ણયો \\
\textbf{ધ્યાન} & નવાચાર અને વૃદ્ધિ & કાર્યક્ષમતા અને નિયંત્રણ \\
\textbf{પારિતોષિક} & નફો અને માલિકી & પગાર અને લાભો \\
\end{longtable}

\end{solutionbox}
\begin{mnemonicbox}
``CRIFO'' - Creates Risk Innovation Focus Ownership

\end{mnemonicbox}
\subsection*{પ્રશ્ન 1(બ) [4
ગુણ]}\label{uxaaauxab0uxab6uxaa8-1uxaac-4-uxa97uxaa3}

\textbf{આંત્રપ્રન્યોરશિપના કોઈપણ ચાર મુખ્ય કાર્યો સમજાવો.}

\begin{solutionbox}

\begin{itemize}
\tightlist
\item
  \textbf{રોજગાર સર્જન}: આંત્રપ્રન્યોર નવા વ્યવસાયો સ્થાપે છે, જેથી અન્ય લોકો માટે
  રોજગારની તકો બને છે
\item
  \textbf{નવાચાર}: તેઓ બજારની જરૂરિયાતો પૂરી કરવા નવા ઉત્પાદનો, સેવાઓ અથવા
  પ્રક્રિયાઓ રજૂ કરે છે
\item
  \textbf{આર્થિક વિકાસ}: સંપત્તિ પેદા કરે છે, GDPમાં યોગદાન આપે છે અને આર્થિક વૃદ્ધિને
  ઉત્તેજિત કરે છે
\item
  \textbf{જોખમ લેવું}: સંભવિત નફા માટે વ્યવસાયિક અનિશ્ચિતતા અને નાણાકીય જોખમો
  સ્વીકારે છે
\end{itemize}

\textbf{ડાયાગ્રામ:}

\begin{center}
\textbf{Mermaid Diagram (Code)}
\begin{verbatim}
{Shaded}
{Highlighting}[]
graph TD
    A[આંત્રપ્રન્યોરશિપ કાર્યો] {-{-}{} B[રોજગાર સર્જન]}
    A {-{-}{} C[નવાચાર]}
    A {-{-}{} D[આર્થિક વિકાસ]}
    A {-{-}{} E[જોખમ લેવું]}
    B {-{-}{} F[રોજગાર નિર્માણ]}
    C {-{-}{} G[નવા ઉત્પાદનો/સેવાઓ]}
    D {-{-}{} H[GDP વૃદ્ધિ]}
    E {-{-}{} I[વ્યવસાયિક અનિશ્ચિતતા]}
{Highlighting}
{Shaded}
\end{verbatim}
\end{center}

\end{solutionbox}
\begin{mnemonicbox}
``રોનાજો'' - રોજગાર નવાચાર આર્થિક જોખમ

\end{mnemonicbox}
\subsection*{પ્રશ્ન 1(ક) [7
ગુણ]}\label{uxaaauxab0uxab6uxaa8-1uxa95-7-uxa97uxaa3}

\textbf{MSMEs ભારત દેશના આર્થિક વિકાસ માટે કેઈ રીતે અગત્યના છે?}

\begin{solutionbox}

\begin{longtable}[]{@{}ll@{}}
\toprule\noalign{}
યોગદાનનું ક્ષેત્ર & મહત્વ \\
\midrule\noalign{}
\endhead
\bottomrule\noalign{}
\endlastfoot
\textbf{રોજગાર નિર્માણ} & કૃષિ પછી બીજા ક્રમનો સૌથી મોટો રોજગાર દાતા \\
\textbf{ઔદ્યોગિક ઉત્પાદન} & મેન્યુફેક્ચરિંગ આઉટપુટમાં 45\% યોગદાન \\
\textbf{નિકાસ આવક} & કુલ નિકાસમાં 40\% હિસ્સો \\
\textbf{GDP યોગદાન} & ભારતના GDPમાં લગભગ 30\% યોગદાન \\
\textbf{ગ્રામીણ વિકાસ} & સંતુલિત પ્રાદેશિક વૃદ્ધિને પ્રોત્સાહન \\
\end{longtable}

\begin{itemize}
\tightlist
\item
  \textbf{મેન્યુફેક્ચરિંગ લવચીકતા}: બજારના ફેરફારો અને ગ્રાહકની જરૂરિયાતોને ઝડપથી
  અનુકૂલન
\item
  \textbf{નવાચાર હબ}: મોટા ઉદ્યોગોને સપ્લાયર અને વેન્ડર તરીકે આધાર આપે છે
\item
  \textbf{ઉદ્યમિતા વિકાસ}: વ્યક્તિગત વ્યવસાય માલિકી અને સ્વરોજગારને પ્રોત્સાહન
\end{itemize}

\end{solutionbox}
\begin{mnemonicbox}
``રો-ઔ-ની-જી-ગ્રા'' - રોજગાર ઔદ્યોગિક નિકાસ જીડીપી
ગ્રામીણ

\end{mnemonicbox}
\subsection*{પ્રશ્ન 1(ક) OR [7
ગુણ]}\label{uxaaauxab0uxab6uxaa8-1uxa95-or-7-uxa97uxaa3}

\textbf{ડિપ્લોમાના વિદ્યાર્થીને પોતાનું સ્ટાર્ટ-અપ શરૂ કરવામાં સ્ટુડેન્ટ સ્ટાર્ટ-અપ એન્ડ
ઇનોવેશન પોલીસી (SSIP) કેવી રીતે મદદરૂપ થાય છે?}

\begin{solutionbox}

\begin{longtable}[]{@{}ll@{}}
\toprule\noalign{}
SSIP લાભો & વર્ણન \\
\midrule\noalign{}
\endhead
\bottomrule\noalign{}
\endlastfoot
\textbf{નાણાકીય સહાય} & સીડ ફંડિંગ અને ₹2 લાખ સુધીની ગ્રાન્ટ \\
\textbf{ઇન્ક્યુબેશન સેન્ટર} & ગુજરાતભરમાં 50+ ઇન્ક્યુબેશન સેન્ટરની સુવિધા \\
\textbf{માર્ગદર્શન} & ઉદ્યોગ નિષ્ણાતોનું માર્ગદર્શન અને સલાહ \\
\textbf{માળખાકીય સુવિધા} & મફત કો-વર્કિંગ સ્પેસ અને સાધનોની સુવિધા \\
\textbf{કૌશલ્ય વિકાસ} & ઉદ્યમિતા તાલીમ કાર્યક્રમો \\
\end{longtable}

\begin{itemize}
\tightlist
\item
  \textbf{શૈક્ષણિક એકીકરણ}: સ્ટાર્ટ-અપ પ્રવૃત્તિઓને શૈક્ષણિક ક્રેડિટ તરીકે ગણવામાં આવે
  છે
\item
  \textbf{IPR સહાય}: પેટન્ટ ફાઇલિંગ અને બૌદ્ધિક સંપદા સુરક્ષામાં મદદ
\item
  \textbf{બજાર પહોંચ}: રોકાણકારો અને ઉદ્યોગ ભાગીદારો સાથે નેટવર્કિંગની તકો
\end{itemize}

\end{solutionbox}
\begin{mnemonicbox}
``ના-ઇ-મા-મા-કૌ'' - નાણાકીય ઇન્ક્યુબેશન માર્ગદર્શન
માળખાકીય કૌશલ્ય

\end{mnemonicbox}
\subsection*{પ્રશ્ન 2(અ) [3
ગુણ]}\label{uxaaauxab0uxab6uxaa8-2uxa85-3-uxa97uxaa3}

\textbf{પ્રોજેક્ટ રિપોર્ટ એટલે શું? પ્રોજેક્ટના અમલીકરણમાં તેનું મહત્વ સમજાવો.}

\begin{solutionbox}

\textbf{પ્રોજેક્ટ રિપોર્ટ} એ એક વ્યાપક દસ્તાવેજ છે જેમાં પ્રસ્તાવિત વ્યવસાયિક સાહસની
તકનીકી, નાણાકીય અને વ્યાપારિક પાસાઓની વિગતવાર માહિતી હોય છે.

\textbf{મહત્વ:}

\begin{itemize}
\tightlist
\item
  \textbf{લોન મંજૂરી}: બેંકોને ધિરાણ નિર્ણયો માટે પ્રોજેક્ટ રિપોર્ટની જરૂર હોય છે
\item
  \textbf{સંસાધન આયોજન}: સંસાધનો અને માનવબળની યોગ્ય ફાળવણીમાં મદદ કરે છે
\item
  \textbf{જોખમ મૂલ્યાંકન}: સંભવિત પડકારો અને તેના ઉકેલની ઓળખ કરે છે
\end{itemize}

\end{solutionbox}
\begin{mnemonicbox}
``લોસાજો'' - લોન સંસાધન જોખમ

\end{mnemonicbox}
\subsection*{પ્રશ્ન 2(બ) [4
ગુણ]}\label{uxaaauxab0uxab6uxaa8-2uxaac-4-uxa97uxaa3}

\textbf{બ્રેક-ઈવન પોઈન્ટ (in terms of sales revenue)ની ગણતરી કેઈ રીતે કરશો?
તેનું graphical representation ઉદાહરણ સહિત આપો.}

\begin{solutionbox}

\textbf{સૂત્ર:} બ્રેક-ઈવન પોઈન્ટ (વેચાણ) = સ્થિર ખર્ચ \div યોગદાન માર્જિન રેશિયો

જ્યાં: યોગદાન માર્જિન રેશિયો = (વેચાણ - વેરિયેબલ કોસ્ટ) \div વેચાણ

\textbf{ઉદાહરણ:}

\begin{itemize}
\tightlist
\item
  સ્થિર ખર્ચ = ₹50,000
\item
  યુનિટ દીઠ વેચાણ કિંમત = ₹100
\item
  યુનિટ દીઠ વેરિયેબલ કોસ્ટ = ₹60
\item
  યુનિટ દીઠ યોગદાન = ₹40
\item
  યોગદાન માર્જિન રેશિયો = 40\%
\item
  બ્રેક-ઈવન વેચાણ = ₹50,000 \div 0.40 = ₹1,25,000
\end{itemize}

\textbf{ડાયાગ્રામ:}

\begin{verbatim}
    આવક/ખર્ચ (₹)
         |
    2,00,000 |     /
             |    /  કુલ આવક
    1,50,000 |   /
             |  /
    1,25,000 |./\_\_\_\_\_ બ્રેક{-ઈવન પોઈન્ટ}
             |/     
    1,00,000 |      કુલ ખર્ચ
             |     /
     50,000  |\_\_\_\_/\_\_\_\_\_ સ્થિર ખર્ચ
             |
             +{-{-}{-}{-}{-}{-}{-}{-}{-}{-}{-}{-}{-}{-}{-}{-}{-}{-}{-}{-}{-}{-}{-}{-} વેચાયેલા યુનિટ}
             0   500  1,250  2,000
\end{verbatim}

\end{solutionbox}
\begin{mnemonicbox}
``સ્થ-યો-રે'' - સ્થિર યોગદાન રેશિયો

\end{mnemonicbox}
\subsection*{પ્રશ્ન 2(ક) [7
ગુણ]}\label{uxaaauxab0uxab6uxaa8-2uxa95-7-uxa97uxaa3}

\textbf{માર્કેટ સર્વેની જરૂરિયાત સમજાવો તથા માર્કેટ સર્વેની માર્કેટ ટેસ્ટ મેથડ
સમજાવો.}

\begin{solutionbox}

\textbf{માર્કેટ સર્વેની જરૂરિયાત:}

\begin{longtable}[]{@{}ll@{}}
\toprule\noalign{}
હેતુ & વર્ણન \\
\midrule\noalign{}
\endhead
\bottomrule\noalign{}
\endlastfoot
\textbf{માંગ મૂલ્યાંકન} & ગ્રાહકોની જરૂરિયાતો અને પસંદગીઓ સમજવી \\
\textbf{સ્પર્ધા વિશ્લેષણ} & સ્પર્ધકોની વ્યૂહરચના અને કિંમતોનો અભ્યાસ \\
\textbf{બજારનું કદ} & કુલ સંબોધિત બજારનો અંદાજ \\
\textbf{કિંમત વ્યૂહરચના} & શ્રેષ્ઠ કિંમત બિંદુઓ નક્કી કરવા \\
\end{longtable}

\textbf{માર્કેટ ટેસ્ટ મેથડ:}

\begin{itemize}
\tightlist
\item
  \textbf{ટેસ્ટ માર્કેટિંગ}: મર્યાદિત ભૌગોલિક વિસ્તારમાં ઉત્પાદન લોન્ચ કરવું
\item
  \textbf{ફોકસ ગ્રુપ}: લક્ષ્ય ગ્રાહકો સાથે ચર્ચા કરવી
\item
  \textbf{પાયલટ અભ્યાસ}: પસંદ કરેલા ગ્રાહકો સાથે ઓછા પાયે ઉત્પાદન પરીક્ષણ
\item
  \textbf{ઓનલાઇન સર્વે}: વ્યાપક પહોંચ માટે ડિજિટલ પ્રશ્નાવલી
\end{itemize}

\begin{center}
\textbf{Mermaid Diagram (Code)}
\begin{verbatim}
{Shaded}
{Highlighting}[]
graph TD
    A[માર્કેટ સર્વેની જરૂર] {-{-}{} B[માંગ મૂલ્યાંકન]}
    A {-{-}{} C[સ્પર્ધા વિશ્લેષણ]}
    A {-{-}{} D[બજારનું કદ]}
    A {-{-}{} E[કિંમત વ્યૂહરચના]}
    
    F[માર્કેટ ટેસ્ટ મેથડ] {-{-}{} G[ટેસ્ટ માર્કેટિંગ]}
    F {-{-}{} H[ફોકસ ગ્રુપ]}
    F {-{-}{} I[પાયલટ અભ્યાસ]}
    F {-{-}{} J[ઓનલાઇન સર્વે]}
{Highlighting}
{Shaded}
\end{verbatim}
\end{center}

\end{solutionbox}
\begin{mnemonicbox}
``મા-સ્પ-બ-કિ'' - માંગ સ્પર્ધા બજાર કિંમત

\end{mnemonicbox}
\subsection*{પ્રશ્ન 2(અ) OR [3
ગુણ]}\label{uxaaauxab0uxab6uxaa8-2uxa85-or-3-uxa97uxaa3}

\textbf{માર્કેટિંગ પ્લાન એટલે શું? ટૂંકમાં સમજાવો.}

\begin{solutionbox}

\textbf{માર્કેટિંગ પ્લાન} એ એક વ્યૂહરચનાત્મક દસ્તાવેજ છે જેમાં વ્યવસાય તેના ઉત્પાદનો
અથવા સેવાઓને લક્ષ્ય ગ્રાહકોને કેવી રીતે પ્રમોટ અને વેચશે તેની રૂપરેખા હોય છે.

\textbf{ઘટકો:}

\begin{itemize}
\tightlist
\item
  \textbf{બજાર વિશ્લેષણ}: ગ્રાહક વસ્તી વિષયક અને વર્તન અભ્યાસ
\item
  \textbf{માર્કેટિંગ મિક્સ}: ઉત્પાદ, કિંમત, સ્થાન, પ્રમોશન વ્યૂહરચના
\item
  \textbf{બજેટ ફાળવણી}: માર્કેટિંગ પ્રવૃત્તિઓ માટે નાણાકીય સંસાધનો
\end{itemize}

\end{solutionbox}
\begin{mnemonicbox}
``બ-મા-બ'' - બજાર માર્કેટિંગ બજેટ

\end{mnemonicbox}
\subsection*{પ્રશ્ન 2(બ) OR [4
ગુણ]}\label{uxaaauxab0uxab6uxaa8-2uxaac-or-4-uxa97uxaa3}

\textbf{ગુજરાતના કોઈ શહેરી વિસ્તારમાં e-bike મેન્યુફેક્ચર કરતી કંપનીનું SWOT
analysis કરો.}

\begin{solutionbox}

\begin{longtable}[]{@{}
  >{\raggedright\arraybackslash}p{(\linewidth - 2\tabcolsep) * \real{0.3636}}
  >{\raggedright\arraybackslash}p{(\linewidth - 2\tabcolsep) * \real{0.6364}}@{}}
\toprule\noalign{}
\begin{minipage}[b]{\linewidth}\raggedright
SWOT વિશ્લેષણ
\end{minipage} & \begin{minipage}[b]{\linewidth}\raggedright
E-bike મેન્યુફેક્ચરિંગ કંપની
\end{minipage} \\
\midrule\noalign{}
\endhead
\bottomrule\noalign{}
\endlastfoot
\textbf{શક્તિઓ (Strengths)} & • ઇલેક્ટ્રિક વાહનો માટે સરકારી સહાય• વધતી
પર્યાવરણીય જાગૃતિ• પેટ્રોલ વાહનો કરતાં ઓછો ઓપરેટિંગ ખર્ચ \\
\textbf{નબળાઈઓ (Weaknesses)} & • ઉચ્ચ પ્રારંભિક રોકાણ• મર્યાદિત ચાર્જિંગ
ઇન્ફ્રાસ્ટ્રક્ચર• બેટરી રિપ્લેસમેન્ટ ખર્ચ \\
\textbf{તકો (Opportunities)} & • FAME યોજના સબસિડી• શહેરી પ્રદૂષણની ચિંતા•
વધતા ઇંધણ ભાવ \\
\textbf{જોખમો (Threats)} & • સ્થાપિત પ્લેયર્સ તરફથી સ્પર્ધા• ટેકનોલોજી
અપ્રચલિતતા• આર્થિક મંદીથી ખરીદ શક્તિ પર અસર \\
\end{longtable}

\end{solutionbox}
\begin{mnemonicbox}
``સ-ન-ત-જો'' - શક્તિઓ નબળાઈઓ તકો જોખમો

\end{mnemonicbox}
\subsection*{પ્રશ્ન 2(ક) OR [7
ગુણ]}\label{uxaaauxab0uxab6uxaa8-2uxa95-or-7-uxa97uxaa3}

\textbf{ઇનોવેશન એટલે શું? કોઈપણ પ્રોડક્ટ અથવા પ્રોસેસ અથવા સર્વિસના ઓછામાં ઓછા પાંચ
ઇનોવેશન્સનું લિસ્ટ આપો.}

\begin{solutionbox}

\textbf{ઇનોવેશન} એ નવા અથવા સુધારેલા ઉત્પાદનો, સેવાઓ અથવા પ્રક્રિયાઓ બનાવવાની
પ્રક્રિયા છે જે ગ્રાહકોને મૂલ્ય અને સંસ્થાઓને સ્પર્ધાત્મક લાભ પૂરો પાડે છે.

\textbf{પાંચ ઉત્પાદન/સેવા ઇનોવેશન:}

\begin{longtable}[]{@{}
  >{\raggedright\arraybackslash}p{(\linewidth - 4\tabcolsep) * \real{0.4167}}
  >{\raggedright\arraybackslash}p{(\linewidth - 4\tabcolsep) * \real{0.2917}}
  >{\raggedright\arraybackslash}p{(\linewidth - 4\tabcolsep) * \real{0.2917}}@{}}
\toprule\noalign{}
\begin{minipage}[b]{\linewidth}\raggedright
ઇનોવેશન
\end{minipage} & \begin{minipage}[b]{\linewidth}\raggedright
પ્રકાર
\end{minipage} & \begin{minipage}[b]{\linewidth}\raggedright
વર્ણન
\end{minipage} \\
\midrule\noalign{}
\endhead
\bottomrule\noalign{}
\endlastfoot
\textbf{UPI પેમેન્ટ સિસ્ટમ} & સેવા & ડિજિટલ પેમેન્ટ પ્લેટફોર્મ જેણે વ્યવહારોમાં ક્રાંતિ
લાવી \\
\textbf{ટેસ્લા ઇલેક્ટ્રિક કાર} & ઉત્પાદ & સ્વાયત્ત સુવિધાઓ સાથે ટકાઉ ઓટોમોટિવ
ટેકનોલોજી \\
\textbf{નેટફ્લિક્સ સ્ટ્રીમિંગ} & સેવા & માંગ મુજબ મનોરંજન ડિલિવરી મોડલ \\
\textbf{3D પ્રિન્ટિંગ} & પ્રક્રિયા & એડિટિવ મેન્યુફેક્ચરિંગ ટેકનોલોજી \\
\textbf{ઝૂમ વિડિયો કોલિંગ} & સેવા & વર્ચ્યુઅલ મીટિંગ માટે રિમોટ કમ્યુનિકેશન
પ્લેટફોર્મ \\
\end{longtable}

\begin{itemize}
\tightlist
\item
  \textbf{મૂલ્ય સર્જન}: દરેક ઇનોવેશને હાલની ગ્રાહક સમસ્યાઓનો ઉકેલ લાવ્યો
\item
  \textbf{બજાર વિક્ષેપ}: પરંપરાગત બિઝનેસ મોડલ અને વપરાશકર્તા વર્તનમાં ફેરફાર કર્યો
\item
  \textbf{ટેકનોલોજી એકીકરણ}: વર્ધિત વપરાશકર્તા અનુભવ માટે બહુવિધ ટેકનોલોજીઓનું
  સંયોજન
\end{itemize}

\end{solutionbox}
\begin{mnemonicbox}
``UNT3Z'' - UPI Netflix Tesla 3D Zoom

\end{mnemonicbox}
\subsection*{પ્રશ્ન 3(અ) [3
ગુણ]}\label{uxaaauxab0uxab6uxaa8-3uxa85-3-uxa97uxaa3}

\textbf{ભાગીદારી પેઢી પર ટૂંકનોંધ લખો.}

\begin{solutionbox}

\textbf{ભાગીદારી પેઢી} એ એક વ્યવસાયિક માળખું છે જ્યાં બે અથવા વધુ વ્યક્તિઓ નફા માટે
સંયુક્ત રીતે વ્યવસાયનું માલિકી અને સંચાલન કરે છે.

\textbf{મુખ્ય લક્ષણો:}

\begin{itemize}
\tightlist
\item
  \textbf{વહેંચાયેલ માલિકી}: બહુવિધ ભાગીદારો મૂડી અને કુશળતાનું યોગદાન આપે છે
\item
  \textbf{સંયુક્ત જવાબદારી}: ભાગીદારો વ્યવસાયિક દેવા માટે વ્યક્તિગત રીતે જવાબદાર
  છે
\item
  \textbf{નફાની વહેંચણી}: ભાગીદારી કરાર અનુસાર કમાણીનું વિતરણ
\end{itemize}

\end{solutionbox}
\begin{mnemonicbox}
``વ-સ-ન'' - વહેંચાયેલ સંયુક્ત નફો

\end{mnemonicbox}
\subsection*{પ્રશ્ન 3(બ) [4
ગુણ]}\label{uxaaauxab0uxab6uxaa8-3uxaac-4-uxa97uxaa3}

\textbf{મેનેજમેન્ટના `staffing' કાર્યમાં સમાવિષ્ટ વિવિધ activities સમજાવો.}

\begin{solutionbox}

\begin{longtable}[]{@{}
  >{\raggedright\arraybackslash}p{(\linewidth - 4\tabcolsep) * \real{0.5278}}
  >{\raggedright\arraybackslash}p{(\linewidth - 4\tabcolsep) * \real{0.1944}}
  >{\raggedright\arraybackslash}p{(\linewidth - 4\tabcolsep) * \real{0.2778}}@{}}
\toprule\noalign{}
\begin{minipage}[b]{\linewidth}\raggedright
Staffing પ્રવૃત્તિ
\end{minipage} & \begin{minipage}[b]{\linewidth}\raggedright
વર્ણન
\end{minipage} & \begin{minipage}[b]{\linewidth}\raggedright
ઉદાહરણ
\end{minipage} \\
\midrule\noalign{}
\endhead
\bottomrule\noalign{}
\endlastfoot
\textbf{ભરતી (Recruitment)} & સંભવિત ઉમેદવારોને આકર્ષવા & LinkedIn પર
નોકરીની જાહેરાત પોસ્ટ કરવી \\
\textbf{પસંદગી (Selection)} & યોગ્ય ઉમેદવારોની પસંદગી & ઇન્ટરવ્યુ અને યોગ્યતા
પરીક્ષા લેવી \\
\textbf{તાલીમ (Training)} & કૌશલ્ય વિકાસ કાર્યક્રમો & નવા કર્મચારીની
ઓરિએન્ટેશન સેશન \\
\textbf{કામગીરી મૂલ્યાંકન} & કર્મચારીની કામગીરીનું મૂલ્યાંકન & વાર્ષિક કામગીરી
સમીક્ષા \\
\end{longtable}

\begin{itemize}
\tightlist
\item
  \textbf{નિયુક્તિ}: યોગ્ય વ્યક્તિને યોગ્ય જગ્યાએ સોંપવું
\item
  \textbf{પ્રમોશન}: કામગીરી અને અનુભવ આધારે કારકિર્દી વૃદ્ધિ
\item
  \textbf{વળતર}: ન્યાયપૂર્ણ વેતન અને લાભોના પેકેજ નક્કી કરવા
\end{itemize}

\end{solutionbox}
\begin{mnemonicbox}
``ભ-પ-તા-કા-ની-પ્ર-વ'' - ભરતી પસંદગી તાલીમ કામગીરી
નિયુક્તિ પ્રમોશન વળતર

\end{mnemonicbox}
\subsection*{પ્રશ્ન 3(ક) [7
ગુણ]}\label{uxaaauxab0uxab6uxaa8-3uxa95-7-uxa97uxaa3}

\textbf{આપખુદ નેતાગીરી સમજાવો અને તેના ફાયદાઓ જણાવો.}

\begin{solutionbox}

\textbf{આપખુદ નેતાગીરી} એ એક મેનેજમેન્ટ શૈલી છે જ્યાં નેતા ટીમના સભ્યોની સલાહ લીધા
વિના સ્વતંત્ર રીતે તમામ નિર્ણયો લે છે.

\textbf{લાક્ષણિકતાઓ:}

\begin{itemize}
\tightlist
\item
  \textbf{કેન્દ્રીકૃત નિર્ણય લેવું}: નેતા પાસે સંપૂર્ણ સત્તા અને નિયંત્રણ
\item
  \textbf{સ્પષ્ટ કમાન્ડ ચેઇન}: સુ-વ્યાખ્યાયિત વંશવેલો અને રિપોર્ટિંગ માળખું
\item
  \textbf{મર્યાદિત કર્મચારી ઇનપુટ}: નિર્ણય લેવાની પ્રક્રિયામાં ન્યૂનતમ સહભાગિતા
\end{itemize}

\textbf{ફાયદાઓ:}

\begin{longtable}[]{@{}
  >{\raggedright\arraybackslash}p{(\linewidth - 2\tabcolsep) * \real{0.5000}}
  >{\raggedright\arraybackslash}p{(\linewidth - 2\tabcolsep) * \real{0.5000}}@{}}
\toprule\noalign{}
\begin{minipage}[b]{\linewidth}\raggedright
ફાયદો
\end{minipage} & \begin{minipage}[b]{\linewidth}\raggedright
વર્ણન
\end{minipage} \\
\midrule\noalign{}
\endhead
\bottomrule\noalign{}
\endlastfoot
\textbf{ઝડપી નિર્ણયો} & લાંબી સલાહમશવરા વિના ઝડપી સમસ્યા-સમાધાન \\
\textbf{સ્પષ્ટ દિશા} & કર્મચારીઓ બરાબર જાણે છે કે શું અપેક્ષિત છે \\
\textbf{કટોકટી વ્યવસ્થાપન} & તાત્કાલિક કાર્યવાહીની જરૂર હોય તેવી કટોકટી
દરમિયાન અસરકારક \\
\textbf{ઉત્પાદકતા} & માળખાકીય કાર્ય વાતાવરણને કારણે વધુ આઉટપુટ \\
\textbf{જવાબદારી} & પરિણામો માટે એક જ બિંદુની જવાબદારી \\
\end{longtable}

\begin{center}
\textbf{Mermaid Diagram (Code)}
\begin{verbatim}
{Shaded}
{Highlighting}[]
graph TD
    A[આપખુદ નેતાગીરી] {-{-}{} B[કેન્દ્રીકૃત નિર્ણયો]}
    A {-{-}{} C[સ્પષ્ટ કમાન્ડ ચેઇન]}
    A {-{-}{} D[મર્યાદિત કર્મચારી ઇનપુટ]}
    
    E[ફાયદાઓ] {-{-}{} F[ઝડપી નિર્ણયો]}
    E {-{-}{} G[સ્પષ્ટ દિશા]}
    E {-{-}{} H[કટોકટી વ્યવસ્થાપન]}
    E {-{-}{} I[વધુ ઉત્પાદકતા]}
    E {-{-}{} J[એક જવાબદારી]}
{Highlighting}
{Shaded}
\end{verbatim}
\end{center}

\end{solutionbox}
\begin{mnemonicbox}
``ઝ-સ્પ-ક-ઉ-જ'' - ઝડપી સ્પષ્ટ કટોકટી ઉત્પાદકતા જવાબદારી

\end{mnemonicbox}
\subsection*{પ્રશ્ન 3(અ) OR [3
ગુણ]}\label{uxaaauxab0uxab6uxaa8-3uxa85-or-3-uxa97uxaa3}

\textbf{જોઈન્ટ સ્ટોક કંપની પર ટૂંકનોંધ લખો.}

\begin{solutionbox}

\textbf{જોઈન્ટ સ્ટોક કંપની} એ એક વ્યવસાયિક સંસ્થા છે જ્યાં મૂડી બહુવિધ શેરહોલ્ડરોની
માલિકીના શેરોમાં વહેંચાયેલું હોય છે.

\textbf{મુખ્ય લક્ષણો:}

\begin{itemize}
\tightlist
\item
  \textbf{મર્યાદિત જવાબદારી}: શેરહોલ્ડરોની જવાબદારી તેમના રોકાણ સુધી મર્યાદિત
\item
  \textbf{ટ્રાન્સફરેબલ શેર}: માલિકી સહજતાથી ખરીદી-વેચી શકાય
\item
  \textbf{અલગ કાનૂની એન્ટિટી}: કંપની તેના માલિકોથી સ્વતંત્ર રીતે અસ્તિત્વ ધરાવે છે
\end{itemize}

\end{solutionbox}
\begin{mnemonicbox}
``મ-ટ્રા-અ'' - મર્યાદિત ટ્રાન્સફરેબલ અલગ

\end{mnemonicbox}
\subsection*{પ્રશ્ન 3(બ) OR [4
ગુણ]}\label{uxaaauxab0uxab6uxaa8-3uxaac-or-4-uxa97uxaa3}

\textbf{મેનેજમેન્ટના `organizing' કાર્યમાં સમાવિષ્ટ વિવિધ activities સમજાવો.}

\begin{solutionbox}

\begin{longtable}[]{@{}
  >{\raggedright\arraybackslash}p{(\linewidth - 4\tabcolsep) * \real{0.5526}}
  >{\raggedright\arraybackslash}p{(\linewidth - 4\tabcolsep) * \real{0.1842}}
  >{\raggedright\arraybackslash}p{(\linewidth - 4\tabcolsep) * \real{0.2632}}@{}}
\toprule\noalign{}
\begin{minipage}[b]{\linewidth}\raggedright
Organizing પ્રવૃત્તિ
\end{minipage} & \begin{minipage}[b]{\linewidth}\raggedright
વર્ણન
\end{minipage} & \begin{minipage}[b]{\linewidth}\raggedright
ઉદાહરણ
\end{minipage} \\
\midrule\noalign{}
\endhead
\bottomrule\noalign{}
\endlastfoot
\textbf{જોબ ડિઝાઇન} & ભૂમિકાઓ અને જવાબદારીઓ વ્યાખ્યાયિત કરવી & માર્કેટિંગ મેનેજર
માટે જોબ ડિસ્ક્રિપ્શન બનાવવું \\
\textbf{વિભાગીકરણ} & સમાન પ્રવૃત્તિઓનું જૂથીકરણ & HR, ફાઇનાન્સ અને ઓપરેશન્સ વિભાગ
બનાવવા \\
\textbf{સત્તા સોંપણી} & સત્તા અને જવાબદારી સોંપવી & મેનેજર ટીમ લીડ્સને બજેટ મંજૂરી
સોંપે છે \\
\textbf{સંકલન} & સરળ વર્કફ્લો સુનિશ્ચિત કરવું & સાપ્તાહિક આંતર-વિભાગીય મીટિંગ્સ \\
\end{longtable}

\begin{itemize}
\tightlist
\item
  \textbf{સંસાધન ફાળવણી}: નાણાકીય અને માનવ સંસાધનોનું કાર્યક્ષમ વિતરણ
\item
  \textbf{નિયંત્રણ વિસ્તાર}: મેનેજર દીઠ ઉપકર્મચારીઓની સંખ્યા નક્કી કરવી
\item
  \textbf{કમાન્ડની એકતા}: દરેક કર્મચારી એક ઉપરી અધિકારીને રિપોર્ટ કરે
\end{itemize}

\end{solutionbox}
\begin{mnemonicbox}
``જો-વિ-સ-સં-સં-ની-ક'' - જોબ વિભાગીકરણ સત્તા સંકલન સંસાધન
નિયંત્રણ કમાન્ડ

\end{mnemonicbox}
\subsection*{પ્રશ્ન 3(ક) OR [7
ગુણ]}\label{uxaaauxab0uxab6uxaa8-3uxa95-or-7-uxa97uxaa3}

\textbf{લોકશાહી નેતાગીરી સમજાવો અને તેના ફાયદાઓ જણાવો.}

\begin{solutionbox}

\textbf{લોકશાહી નેતાગીરી} એ એક મેનેજમેન્ટ શૈલી છે જ્યાં નેતાઓ નિર્ણય લેવાની
પ્રક્રિયાઓમાં ટીમના સભ્યોને સામેલ કરે છે અને સહભાગિતાને પ્રોત્સાહન આપે છે.

\textbf{લાક્ષણિકતાઓ:}

\begin{itemize}
\tightlist
\item
  \textbf{સહભાગી નિર્ણય લેવું}: ટીમના સભ્યો સમસ્યા-સમાધાનમાં યોગદાન આપે છે
\item
  \textbf{ખુલ્લો વાતચીત}: નેતાઓ અને કર્મચારીઓ વચ્ચે દ્વિ-માર્ગી વાતચીત
\item
  \textbf{વહેંચાયેલ જવાબદારી}: પરિણામો અને પરિણામોની સામૂહિક માલિકી
\end{itemize}

\textbf{ફાયદાઓ:}

\begin{longtable}[]{@{}ll@{}}
\toprule\noalign{}
ફાયદો & વર્ણન \\
\midrule\noalign{}
\endhead
\bottomrule\noalign{}
\endlastfoot
\textbf{વધુ જોબ સંતોષ} & કર્મચારીઓ મૂલ્યવાન અને સાંભળ્યા હોવાનું અનુભવે છે \\
\textbf{વધુ સારા ગુણવત્તાના નિર્ણયો} & બહુવિધ દ્રષ્ટિકોણ નિર્ણયની ગુણવત્તા સુધારે
છે \\
\textbf{સુધારેલ સર્જનાત્મકતા} & વિવિધ વિચારો અને નવાચારી ઉકેલો \\
\textbf{ટીમ બિલ્ડિંગ} & મજબૂત સહયોગ અને વિશ્વાસ \\
\textbf{કર્મચારી વિકાસ} & સહભાગિતા દ્વારા કૌશલ્ય વૃદ્ધિ \\
\end{longtable}

\begin{center}
\textbf{Mermaid Diagram (Code)}
\begin{verbatim}
{Shaded}
{Highlighting}[]
graph TD
    A[લોકશાહી નેતાગીરી] {-{-}{} B[સહભાગી નિર્ણયો]}
    A {-{-}{} C[ખુલ્લો વાતચીત]}
    A {-{-}{} D[વહેંચાયેલ જવાબદારી]}
    
    E[ફાયદાઓ] {-{-}{} F[જોબ સંતોષ]}
    E {-{-}{} G[ગુણવત્તા નિર્ણયો]}
    E {-{-}{} H[સર્જનાત્મકતા]}
    E {-{-}{} I[ટીમ બિલ્ડિંગ]}
    E {-{-}{} J[કર્મચારી વિકાસ]}
{Highlighting}
{Shaded}
\end{verbatim}
\end{center}

\end{solutionbox}
\begin{mnemonicbox}
``જો-ગુ-સ-ટી-ક'' - જોબ ગુણવત્તા સર્જનાત્મકતા ટીમ કર્મચારી

\end{mnemonicbox}
\subsection*{પ્રશ્ન 4(અ) [3
ગુણ]}\label{uxaaauxab0uxab6uxaa8-4uxa85-3-uxa97uxaa3}

\textbf{જિલ્લા ઉદ્યોગ કેન્દ્રના વિવિધ કાર્યો લખો.}

\begin{solutionbox}

\begin{itemize}
\tightlist
\item
  \textbf{નોંધણી સેવાઓ}: MSME નોંધણી અને વિવિધ લાઇસન્સ મંજૂરીઓ
\item
  \textbf{નાણાકીય સહાય}: લોન અને સરકારી યોજના અરજીઓ માટે માર્ગદર્શન
\item
  \textbf{તકનીકી સહાય}: તકનીકી માર્ગદર્શન અને સલાહકારી સેવાઓ પૂરી પાડવી
\end{itemize}

\end{solutionbox}
\begin{mnemonicbox}
``નો-ના-ત'' - નોંધણી નાણાકીય તકનીકી

\end{mnemonicbox}
\subsection*{પ્રશ્ન 4(બ) [4
ગુણ]}\label{uxaaauxab0uxab6uxaa8-4uxaac-4-uxa97uxaa3}

\textbf{રાજ્ય કક્ષાના કોઈપણ બે ઇન્ક્યુબેટરના નામ આપી તેમના કાર્યો લખો.}

\begin{solutionbox}

\begin{longtable}[]{@{}
  >{\raggedright\arraybackslash}p{(\linewidth - 2\tabcolsep) * \real{0.6500}}
  >{\raggedright\arraybackslash}p{(\linewidth - 2\tabcolsep) * \real{0.3500}}@{}}
\toprule\noalign{}
\begin{minipage}[b]{\linewidth}\raggedright
ઇન્ક્યુબેટર
\end{minipage} & \begin{minipage}[b]{\linewidth}\raggedright
કાર્યો
\end{minipage} \\
\midrule\noalign{}
\endhead
\bottomrule\noalign{}
\endlastfoot
\textbf{i-HUB ગુજરાત} & • સ્ટાર્ટઅપ મેન્ટરિંગ અને એક્સેલેરેશન કાર્યક્રમો• ફંડિંગ સપોર્ટ
અને રોકાણકારો સાથે કનેક્શન• ઇન્ફ્રાસ્ટ્રક્ચર અને કો-વર્કિંગ સ્પેસ સુવિધાઓ \\
\textbf{CIIE અમદાવાદ} & • ટેકનોલોજી કોમર્શિયલાઇઝેશન સપોર્ટ• ઇન્ડસ્ટ્રી-એકેડેમિયા
સહયોગ• વધતા સ્ટાર્ટઅપ્સ માટે સ્કેલ-અપ કાર્યક્રમો \\
\end{longtable}

\textbf{સામાન્ય કાર્યો:}

\begin{itemize}
\tightlist
\item
  \textbf{મેન્ટરશિપ કાર્યક્રમો}: ઉદ્યોગ વ્યાવસાયિકો તરફથી નિષ્ણાત માર્ગદર્શન
\item
  \textbf{નેટવર્કિંગ ઇવેન્ટ્સ}: સ્ટાર્ટઅપ્સને રોકાણકારો અને ભાગીદારો સાથે જોડવા
\end{itemize}

\end{solutionbox}
\begin{mnemonicbox}
``મે-ફં-ઇન-ને'' - મેન્ટરિંગ ફંડિંગ ઇન્ફ્રાસ્ટ્રક્ચર નેટવर્કિંગ

\end{mnemonicbox}
\subsection*{પ્રશ્ન 4(ક) [7
ગુણ]}\label{uxaaauxab0uxab6uxaa8-4uxa95-7-uxa97uxaa3}

\textbf{સ્ટાર્ટ-અપ ઇકો સિસ્ટમ એટલે શું? તેના વિવિધ કાર્યો અને ઘટકોનું લિસ્ટ આપો.}

\begin{solutionbox}

\textbf{સ્ટાર્ટ-અપ ઇકોસિસ્ટમ} એ પરસ્પર જોડાયેલ સંસ્થાઓ, વ્યક્તિઓ અને સંસાધનોનું નેટવર્ક
છે જે ઉદ્યમિતા અને સ્ટાર્ટઅપ વિકાસને સમર્થન આપે છે.

\textbf{મુખ્ય ઘટકો:}

\begin{longtable}[]{@{}ll@{}}
\toprule\noalign{}
ઘટક & વર્ણન \\
\midrule\noalign{}
\endhead
\bottomrule\noalign{}
\endlastfoot
\textbf{આંત્રપ્રન્યોર} & નવા સાહસો શરૂ કરતા દ્રષ્ટિકોણ ધરાવતા વ્યક્તિઓ \\
\textbf{રોકાણકારો} & ફંડિંગ પૂરું પાડતા એન્જેલ રોકાણકારો, VCs \\
\textbf{ઇન્ક્યુબેટર/એક્સેલેરેટર} & પ્રારંભિક તબક્કાના સ્ટાર્ટઅપ્સ માટે સહાયક સંસ્થાઓ \\
\textbf{સરકાર} & નીતિ નિર્માતાઓ અને નિયમનકારી સંસ્થાઓ \\
\textbf{શૈક્ષણિક સંસ્થાઓ} & યુનિવર્સિટીઓ અને સંશોધન કેન્દ્રો \\
\textbf{સેવા પ્રદાતાઓ} & કાનૂની, એકાઉન્ટિંગ, કન્સલ્ટિંગ ફર્મ \\
\end{longtable}

\textbf{પ્રવૃત્તિઓ:}

\begin{itemize}
\tightlist
\item
  \textbf{મેન્ટરિંગ સેશન}: અનુભવી આંત્રપ્રન્યોર તરફથી નિયમિત માર્ગદર્શન
\item
  \textbf{નેટવર્કિંગ ઇવેન્ટ્સ}: સ્ટાર્ટઅપ મીટઅપ્સ અને રોકાણકાર પિચ સેશન
\item
  \textbf{ફંડિંગ રાઉન્ડ્સ}: સીડ, સિરીઝ A, B ફંડિંગ તકો
\item
  \textbf{કૌશલ્ય વિકાસ}: તકનીકી અને બિઝનેસ તાલીમ કાર્યક્રમો
\end{itemize}

\begin{center}
\textbf{Mermaid Diagram (Code)}
\begin{verbatim}
{Shaded}
{Highlighting}[]
graph TD
    A[સ્ટાર્ટઅપ ઇકોસિસ્ટમ] {-{-}{} B[આંત્રપ્રન્યોર]}
    A {-{-}{} C[રોકાણકારો]}
    A {-{-}{} D[ઇન્ક્યુબેટર]}
    A {-{-}{} E[સરકાર]}
    A {-{-}{} F[શૈક્ષણિક સંસ્થાઓ]}
    A {-{-}{} G[સેવા પ્રદાતાઓ]}
    
    H[પ્રવૃત્તિઓ] {-{-}{} I[મેન્ટરિંગ]}
    H {-{-}{} J[નેટવર્કિંગ]}
    H {-{-}{} K[ફંડિંગ]}
    H {-{-}{} L[કૌશલ્ય વિકાસ]}
{Highlighting}
{Shaded}
\end{verbatim}
\end{center}

\end{solutionbox}
\begin{mnemonicbox}
``આં-રો-ઇ-સ-શૈ-સે'' - આંત્રપ્રન્યોર રોકાણકારો ઇન્ક્યુબેટર સરકાર
શૈક્ષણિક સેવા

\end{mnemonicbox}
\subsection*{પ્રશ્ન 4(અ) OR [3
ગુણ]}\label{uxaaauxab0uxab6uxaa8-4uxa85-or-3-uxa97uxaa3}

\textbf{સ્મોલ ઇન્ડસ્ટ્રીઝ ડેવલોપમેન્ટ બેન્ક ઓફ ઇન્ડિયા (SIDBI)ના વિવિધ કાર્યો
જણાવો.}

\begin{solutionbox}

\begin{itemize}
\tightlist
\item
  \textbf{નાણાકીય સેવાઓ}: MSMEs અને સ્ટાર્ટઅપ્સને પ્રત્યક્ષ અને અપ્રત્યક્ષ ધિરાણ
\item
  \textbf{વિકાસ સેવાઓ}: ક્ષમતા નિર્માણ અને કૌશલ્ય વિકાસ કાર્યક્રમો
\item
  \textbf{પ્રમોશનલ પ્રવૃત્તિઓ}: બજાર વિકાસ અને ટેકનોલોજી અપગ્રેડેશન સપોર્ટ
\end{itemize}

\end{solutionbox}
\begin{mnemonicbox}
``ના-વિ-પ્ર'' - નાણાકીય વિકાસ પ્રમોશનલ

\end{mnemonicbox}
\subsection*{પ્રશ્ન 4(બ) OR [4
ગુણ]}\label{uxaaauxab0uxab6uxaa8-4uxaac-or-4-uxa97uxaa3}

\textbf{રાષ્ટ્રીય કક્ષાના કોઈપણ બે ઇન્ક્યુબેટરના નામ આપી તેમના કાર્યો લખો.}

\begin{solutionbox}

\begin{longtable}[]{@{}
  >{\raggedright\arraybackslash}p{(\linewidth - 2\tabcolsep) * \real{0.6500}}
  >{\raggedright\arraybackslash}p{(\linewidth - 2\tabcolsep) * \real{0.3500}}@{}}
\toprule\noalign{}
\begin{minipage}[b]{\linewidth}\raggedright
ઇન્ક્યુબેટર
\end{minipage} & \begin{minipage}[b]{\linewidth}\raggedright
કાર્યો
\end{minipage} \\
\midrule\noalign{}
\endhead
\bottomrule\noalign{}
\endlastfoot
\textbf{T-Hub હૈદરાબાદ} & • ભારતનું સૌથી મોટું સ્ટાર્ટઅપ ઇન્ક્યુબેટર• ટેકનોલોજી
ઇનોવેશન અને R\&D સપોર્ટ• ગ્લોબલ માર્કેટ એક્સેસ કાર્યક્રમો \\
\textbf{NASSCOM 10,000 સ્ટાર્ટઅપ્સ} & • સમગ્ર ભારતમાં સ્ટાર્ટઅપ એક્સેલેરેશન
કાર્યક્રમ• કોર્પોરેટ પાર્ટનરશિપ સુવિધા• ઇકોસિસ્ટમ બિલ્ડિંગ અને નીતિ વકીલાત \\
\end{longtable}

\textbf{સામાન્ય કાર્યો:}

\begin{itemize}
\tightlist
\item
  \textbf{એક્સેલેરેશન કાર્યક્રમો}: સઘન સ્ટાર્ટઅપ વિકાસ અને મેન્ટરિંગ
\item
  \textbf{કોર્પોરેટ કનેક્શન્સ}: સ્ટાર્ટઅપ્સને મોટા એન્ટરપ્રાઇઝ ભાગીદારો સાથે જોડવા
\end{itemize}

\end{solutionbox}
\begin{mnemonicbox}
``ટે-ઇ-ગ્લો-એ-કો'' - ટેકનોલોજી ઇનોવેશન ગ્લોબલ એક્સેલેરેશન
કોર્પોરેટ

\end{mnemonicbox}
\subsection*{પ્રશ્ન 4(ક) OR [7
ગુણ]}\label{uxaaauxab0uxab6uxaa8-4uxa95-or-7-uxa97uxaa3}

\textbf{સ્ટાર્ટ-અપને નિષ્ફળ જતું અટકાવવા કયાં પગલા લેવા જોઈએ? ટૂંકમાં સમજાવો.}

\begin{solutionbox}

\textbf{સ્ટાર્ટઅપ નિષ્ફળતા અટકાવવાના પગલાં:}

\begin{longtable}[]{@{}ll@{}}
\toprule\noalign{}
પગલું & વર્ણન \\
\midrule\noalign{}
\endhead
\bottomrule\noalign{}
\endlastfoot
\textbf{બજાર સંશોધન} & ગ્રાહકોની જરૂરિયાતો અને બજારની માંગની સંપૂર્ણ સમજ \\
\textbf{નાણાકીય આયોજન} & યોગ્ય કેશ ફ્લો મેનેજમેન્ટ અને ફંડિંગ વ્યૂહરચના \\
\textbf{ટીમ બિલ્ડિંગ} & કુશળ અને પ્રતિબદ્ધ ટીમના સભ્યોની નિમણૂક \\
\textbf{પ્રોડક્ટ વેલિડેશન} & સંપૂર્ણ લોન્ચ પહેલાં પ્રોડક્ટ-માર્કેટ ફિટનું પરીક્ષણ \\
\textbf{ગ્રાહક ફોકસ} & સતત ગ્રાહક પ્રતિસાદ અને સંતોષ \\
\end{longtable}

\begin{itemize}
\tightlist
\item
  \textbf{જોખમ વ્યવસ્થાપન}: સંભવિત જોખમો અને નિવારણ વ્યૂહરચનાની ઓળખ
\item
  \textbf{અનુકૂલનશીલતા}: બજારના ફેરફારો આધારે પિવટ કરવાની લવચીકતા
\item
  \textbf{કાનૂની અનુપાલન}: યોગ્ય નોંધણી અને નિયમનકારી પાલન
\end{itemize}

\begin{center}
\textbf{Mermaid Diagram (Code)}
\begin{verbatim}
{Shaded}
{Highlighting}[]
graph TD
    A[સ્ટાર્ટઅપ નિષ્ફળતા અટકાવવા] {-{-}{} B[બજાર સંશોધન]}
    A {-{-}{} C[નાણાકીય આયોજન]}
    A {-{-}{} D[ટીમ બિલ્ડિંગ]}
    A {-{-}{} E[પ્રોડક્ટ વેલિડેશન]}
    A {-{-}{} F[ગ્રાહક ફોકસ]}
    A {-{-}{} G[જોખમ વ્યવસ્થાપન]}
    A {-{-}{} H[અનુકૂલનશીલતા]}
    A {-{-}{} I[કાનૂની અનુપાલન]}
{Highlighting}
{Shaded}
\end{verbatim}
\end{center}

\end{solutionbox}
\begin{mnemonicbox}
``બ-ના-ટી-પ્ર-ગ્રા-જો-અ-કા'' - બજાર નાણાકીય ટીમ પ્રોડક્ટ
ગ્રાહક જોખમ અનુકૂલનશીલતા કાનૂની

\end{mnemonicbox}
\subsection*{પ્રશ્ન 5(અ) [3
ગુણ]}\label{uxaaauxab0uxab6uxaa8-5uxa85-3-uxa97uxaa3}

\textbf{રોકાણ પર વળતર (ROI)ની ગણતરી કેઈ રીતે થાય છે તે બતાવો.}

\begin{solutionbox}

\textbf{ROI સૂત્ર:} ROI = (નેટ પ્રોફિટ \div કુલ રોકાણ) \times 100

\textbf{ઉદાહરણ:}

\begin{itemize}
\tightlist
\item
  રોકાણ = ₹1,00,000
\item
  નેટ પ્રોફિટ = ₹20,000
\item
  ROI = (20,000 \div 1,00,000) \times 100 = 20\%
\end{itemize}

\end{solutionbox}
\begin{mnemonicbox}
``ને-કુ-સો'' - નેટ કુલ સો

\end{mnemonicbox}
\subsection*{પ્રશ્ન 5(બ) [4
ગુણ]}\label{uxaaauxab0uxab6uxaa8-5uxaac-4-uxa97uxaa3}

\textbf{શક્તા અભ્યાસમાં ટેકનીકલ એનાલીસીસની અગત્ય સમજાવો.}

\begin{solutionbox}

\begin{longtable}[]{@{}ll@{}}
\toprule\noalign{}
મહત્વ & વર્ણન \\
\midrule\noalign{}
\endhead
\bottomrule\noalign{}
\endlastfoot
\textbf{ટેકનોલોજી આકારણી} & તકનીકી વ્યવહાર્યતા અને જરૂરિયાતોનું મૂલ્યાંકન \\
\textbf{સંસાધન આયોજન} & મશીનરી, સાધનો અને ઇન્ફ્રાસ્ટ્રક્ચરની જરૂરિયાતો નક્કી
કરવી \\
\textbf{પ્રક્રિયા ડિઝાઇન} & શ્રેષ્ઠ ઉત્પાદન પદ્ધતિઓ અને વર્કફ્લો \\
\textbf{ગુણવત્તા ધોરણો} & ઉત્પાદન ઉદ્યોગ વિશિષ્ટતાઓને પૂર્ણ કરે તેની ખાતરી \\
\end{longtable}

\end{solutionbox}
\begin{mnemonicbox}
``ટે-સં-પ્ર-ગુ'' - ટેકનોલોજી સંસાધન પ્રક્રિયા ગુણવત્ત

\end{mnemonicbox}
\subsection*{પ્રશ્ન 5(ક) [7
ગુણ]}\label{uxaaauxab0uxab6uxaa8-5uxa95-7-uxa97uxaa3}

\textbf{કોપોરેટ સામાજિક જવાબદારીની લાક્ષણિકતાઓ વર્ણવો.}

\begin{solutionbox}

\textbf{કોપોરેટ સામાજિક જવાબદારી (CSR)} એ વ્યાવસાયિક પ્રથાઓનો સંદર્ભ છે જેમાં
એવી પહેલોનો સમાવેશ થાય છે જે સમાજને લાભ આપે છે અને નૈતિક કામગીરી પ્રત્યે પ્રતિબદ્ધતા
દર્શાવે છે.

\textbf{મુખ્ય લાક્ષણિકતાઓ:}

\begin{longtable}[]{@{}
  >{\raggedright\arraybackslash}p{(\linewidth - 2\tabcolsep) * \real{0.6500}}
  >{\raggedright\arraybackslash}p{(\linewidth - 2\tabcolsep) * \real{0.3500}}@{}}
\toprule\noalign{}
\begin{minipage}[b]{\linewidth}\raggedright
લાક્ષણિકતા
\end{minipage} & \begin{minipage}[b]{\linewidth}\raggedright
વર્ણન
\end{minipage} \\
\midrule\noalign{}
\endhead
\bottomrule\noalign{}
\endlastfoot
\textbf{સ્વૈચ્છિક સ્વભાવ} & કાનૂની જરૂરિયાતોથી આગળ, સ્વ-લાદેલી પ્રતિબદ્ધતાઓ \\
\textbf{સ્ટેકહોલ્ડર દિશા} & માત્ર શેરહોલ્ડરો જ નહીં, પણ તમામ સ્ટેકહોલ્ડરો પરની
અસરનો વિચાર \\
\textbf{ત્રિપલ બોટમ લાઇન} & લોકો, પ્લેનેટ અને પ્રોફિટ પર ધ્યાન \\
\textbf{ટકાઉ પ્રથાઓ} & લાંબા ગાળાની પર્યાવરણીય અને સામાજિક ટકાઉપણું \\
\textbf{પારદર્શિતા} & ખુલ્લા રિપોર્ટિંગ અને જવાબદારી \\
\end{longtable}

\begin{itemize}
\tightlist
\item
  \textbf{સમુદાય વિકાસ}: શિક્ષણ, આરોગ્યસંભાળ અને ઇન્ફ્રાસ્ટ્રક્ચર પ્રોજેક્ટ્સ
\item
  \textbf{પર્યાવરણ સંરક્ષણ}: પ્રદૂષણ નિયંત્રણ અને સંસાધન સંરક્ષણ
\item
  \textbf{કર્મચારી કલ્યાણ}: ન્યાયપૂર્ણ વેતન, સુરક્ષિત કાર્ય પરિસ્થિતિઓ, કૌશલ્ય
  વિકાસ
\end{itemize}

\begin{center}
\textbf{Mermaid Diagram (Code)}
\begin{verbatim}
{Shaded}
{Highlighting}[]
graph TD
    A[CSR લાક્ષણિકતાઓ] {-{-}{} B[સ્વૈચ્છિક સ્વભાવ]}
    A {-{-}{} C[સ્ટેકહોલ્ડર દિશા]}
    A {-{-}{} D[ત્રિપલ બોટમ લાઇન]}
    A {-{-}{} E[ટકાઉ પ્રથાઓ]}
    A {-{-}{} F[પારદર્શિતા]}
    
    G[CSR પ્રવૃત્તિઓ] {-{-}{} H[સમુદાય વિકાસ]}
    G {-{-}{} I[પર્યાવરણ સંરક્ષણ]}
    G {-{-}{} J[કર્મચારી કલ્યાણ]}
{Highlighting}
{Shaded}
\end{verbatim}
\end{center}

\end{solutionbox}
\begin{mnemonicbox}
``સ્વ-સ્ટે-ત્રિ-ટ-પા'' - સ્વૈચ્છિક સ્ટેકહોલ્ડર ત્રિપલ ટકાઉ
પારદર્શિતા

\end{mnemonicbox}
\subsection*{પ્રશ્ન 5(અ) OR [3
ગુણ]}\label{uxaaauxab0uxab6uxaa8-5uxa85-or-3-uxa97uxaa3}

\textbf{વેચાણ પર વળતર (ROS)ની ગણતરી કેઈ રીતે થાય છે તે બતાવો.}

\begin{solutionbox}

\textbf{ROS સૂત્ર:} ROS = (નેટ પ્રોફિટ \div નેટ સેલ્સ) \times 100

\textbf{ઉદાહરણ:}

\begin{itemize}
\tightlist
\item
  નેટ સેલ્સ = ₹5,00,000
\item
  નેટ પ્રોફિટ = ₹50,000
\item
  ROS = (50,000 \div 5,00,000) \times 100 = 10\%
\end{itemize}

\end{solutionbox}
\begin{mnemonicbox}
``ને-સે-સો'' - નેટ સેલ્સ સો

\end{mnemonicbox}
\subsection*{પ્રશ્ન 5(બ) OR [4
ગુણ]}\label{uxaaauxab0uxab6uxaa8-5uxaac-or-4-uxa97uxaa3}

\textbf{શક્તા અભ્યાસમાં માર્કેટ એનાલીસીસની અગત્ય સમજાવો.}

\begin{solutionbox}

\begin{longtable}[]{@{}ll@{}}
\toprule\noalign{}
મહત્વ & વર્ણન \\
\midrule\noalign{}
\endhead
\bottomrule\noalign{}
\endlastfoot
\textbf{માંગ પૂર્વાનુમાન} & ભવિષ્યના બજારના આકાર અને વૃદ્ધિનો અંદાજ \\
\textbf{સ્પર્ધા આકારણી} & સ્પર્ધાત્મક વાતાવરણની સમજ \\
\textbf{કિંમત વ્યૂહરચના} & શ્રેષ્ઠ કિંમત બિંદુઓ નક્કી કરવા \\
\textbf{બજાર વિભાજન} & લક્ષ્ય ગ્રાહક જૂથોની ઓળખ \\
\end{longtable}

\end{solutionbox}
\begin{mnemonicbox}
``મા-સ્પ-કિ-બ'' - માંગ સ્પર્ધા કિંમત બજાર

\end{mnemonicbox}
\subsection*{પ્રશ્ન 5(ક) OR [7
ગુણ]}\label{uxaaauxab0uxab6uxaa8-5uxa95-or-7-uxa97uxaa3}

\textbf{નૈતિકતાની લાક્ષણિકતાઓ વર્ણવો.}

\begin{solutionbox}

\textbf{નૈતિકતા} એ નૈતિક સિદ્ધાંતો છે જે વ્યક્તિગત અને વ્યાવસાયિક સંદર્ભોમાં વર્તન અને
નિર્ણય લેવાનું શાસન કરે છે.

\textbf{મુખ્ય લાક્ષણિકતાઓ:}

\begin{longtable}[]{@{}ll@{}}
\toprule\noalign{}
લાક્ષણિકતા & વર્ણન \\
\midrule\noalign{}
\endhead
\bottomrule\noalign{}
\endlastfoot
\textbf{સાર્વત્રિક સિદ્ધાંતો} & સંસ્કૃતિઓ અને પરિસ્થિતિઓમાં લાગુ પડે છે \\
\textbf{નૈતિક ધોરણો} & સાચા અને ખોટાની વિભાવનાઓ પર આધારિત \\
\textbf{સ્વૈચ્છિક અનુપાલન} & બાહ્ય બળને બદલે આંતરિક પ્રેરણા \\
\textbf{પરિણામલક્ષી વિચારસરણી} & પરિણામો અને પ્રભાવોનો વિચાર \\
\textbf{સ્ટેકહોલ્ડર વિચારણા} & બધા અસરગ્રસ્ત પક્ષોનો હિસાબ \\
\end{longtable}

\begin{itemize}
\tightlist
\item
  \textbf{સુસંગતતા}: નૈતિક વર્તન તમામ પરિસ્થિતિઓમાં સ્થિર રહે છે
\item
  \textbf{પારદર્શિતા}: ખુલ્લી અને પ્રામાણિક વાતચીત અને ક્રિયાઓ
\item
  \textbf{જવાબદારી}: નિર્ણયો અને તેમના પરિણામો માટે જવાબદારી લેવી
\end{itemize}

\begin{center}
\textbf{Mermaid Diagram (Code)}
\begin{verbatim}
{Shaded}
{Highlighting}[]
graph TD
    A[નૈતિકતા લાક્ષણિકતાઓ] {-{-}{} B[સાર્વત્રિક સિદ્ધાંતો]}
    A {-{-}{} C[નૈતિક ધોરણો]}
    A {-{-}{} D[સ્વૈચ્છિક અનુપાલન]}
    A {-{-}{} E[પરિણામલક્ષી વિચારસરણી]}
    A {-{-}{} F[સ્ટેકહોલ્ડર વિચારણા]}
    A {-{-}{} G[સુસંગતતા]}
    A {-{-}{} H[પારદર્શિતા]}
    A {-{-}{} I[જવાબદારી]}
{Highlighting}
{Shaded}
\end{verbatim}
\end{center}

\end{solutionbox}
\begin{mnemonicbox}
``સા-નૈ-સ્વ-પ-સ્ટે-સુ-પા-જ'' - સાર્વત્રિક નૈતિક સ્વૈચ્છિક
પરિણામલક્ષી સ્ટેકહોલ્ડર સુસંગતતા પારદર્શિતા જવાબદારી

\end{mnemonicbox}

\end{document}
