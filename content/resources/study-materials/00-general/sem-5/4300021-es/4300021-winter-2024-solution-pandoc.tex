\documentclass[10pt,a4paper]{article}

% content/resources/templates/preamble.tex
\usepackage[margin=0.6in]{geometry}
\author{Milav Dabgar}
\usepackage{amsmath,amssymb,amsthm}
\usepackage{booktabs}
\usepackage{multirow}
\usepackage{xcolor}
\usepackage{tcolorbox}
\tcbuselibrary{breakable,skins}
\usepackage[colorlinks=true,linkcolor=blue]{hyperref}
\usepackage{titlesec}
\usepackage{enumitem}
\usepackage{tikz}
\usepackage{pgfplots}
\usepackage{circuitikz}
\usepackage[version=4]{mhchem}
\usepackage{longtable}
\usepackage{array}
\usepackage{float}
\usepackage{caption}
\usepackage{listings}

\lstset{
  basicstyle=\small\ttfamily,
  breaklines=true,
  breakatwhitespace=false,
  postbreak=\mbox{\textcolor{red}{$\hookrightarrow$}\space},
  float=false,
  numbers=left,
  numberstyle=\tiny\color{gray},
  numbersep=10pt,
  xleftmargin=2em,
  keywordstyle=\color{blue},
  commentstyle=\color{green!60!black},
  stringstyle=\color{purple},
  backgroundcolor=\color{gray!5},
  showstringspaces=false,
  tabsize=2,
  captionpos=b,
  keepspaces=true,
  columns=flexible
}

\pgfplotsset{compat=1.18}
\usetikzlibrary{shapes,arrows,positioning,calc,patterns,decorations.pathmorphing,decorations.markings,arrows.meta}

% Color scheme
\definecolor{headcolor}{RGB}{0,102,204}
\definecolor{keycolor}{RGB}{220,20,60}
\definecolor{solutioncolor}{RGB}{34,139,34}
\definecolor{mnemoniccolor}{RGB}{148,0,211}
\definecolor{codecolor}{RGB}{0,0,100}

% Spacing
\setlength{\parskip}{3pt}
\setlist[itemize]{nosep}
\setlist[enumerate]{nosep}

% Title formatting
\titleformat{\section}{\Large\bfseries\color{headcolor}}{\thesection}{1em}{}
\titleformat{\subsection}{\large\bfseries\color{headcolor}}{\thesubsection}{1em}{}

% Pandoc tightlist compatibility
\providecommand{\tightlist}{%
  \setlength{\itemsep}{0pt}\setlength{\parskip}{0pt}}

% Pandoc longtable compatibility
\newcounter{none}
\def\thenone{}


% content/resources/templates/english-boxes.tex
% This file is currently empty - it exists to maintain consistency with the import structure.
% Add custom environments here if needed in the future.


\begin{document}

\begin{center}
{\Huge\bfseries\color{headcolor} Subject Name Solutions}\\[5pt]
{\LARGE 4300021 -- Winter 2024}\\[3pt]
{\large Semester 1 Study Material}\\[3pt]
{\normalsize\textit{Detailed Solutions and Explanations}}
\end{center}

\vspace{10pt}

\subsection*{Question 1(a) [3 marks]}\label{q1a}

\textbf{Distinguish between Entrepreneur and Manager.}

\begin{solutionbox}

\begin{longtable}[]{@{}
  >{\raggedright\arraybackslash}p{(\linewidth - 4\tabcolsep) * \real{0.2667}}
  >{\raggedright\arraybackslash}p{(\linewidth - 4\tabcolsep) * \real{0.4333}}
  >{\raggedright\arraybackslash}p{(\linewidth - 4\tabcolsep) * \real{0.3000}}@{}}
\toprule\noalign{}
\begin{minipage}[b]{\linewidth}\raggedright
Aspect
\end{minipage} & \begin{minipage}[b]{\linewidth}\raggedright
Entrepreneur
\end{minipage} & \begin{minipage}[b]{\linewidth}\raggedright
Manager
\end{minipage} \\
\midrule\noalign{}
\endhead
\bottomrule\noalign{}
\endlastfoot
\textbf{Primary Role} & Creates new ventures and opportunities &
Administers existing operations \\
\textbf{Risk Taking} & High risk-taker, bears uncertainty & Low to
moderate risk, follows guidelines \\
\textbf{Decision Making} & Quick, intuitive decisions & Systematic,
policy-based decisions \\
\textbf{Focus} & Innovation and growth & Efficiency and control \\
\textbf{Rewards} & Profit and ownership & Salary and benefits \\
\end{longtable}

\end{solutionbox}
\begin{mnemonicbox}
``CRIFO'' - Creates Risk Innovation Focus Ownership

\end{mnemonicbox}
\subsection*{Question 1(b) [4 marks]}\label{q1b}

\textbf{Explain any four functions of Entrepreneurship.}

\begin{solutionbox}

\begin{itemize}
\tightlist
\item
  \textbf{Job Creation}: Entrepreneurs establish new businesses,
  creating employment opportunities for others
\item
  \textbf{Innovation}: They introduce new products, services, or
  processes to meet market needs
\item
  \textbf{Economic Development}: Generate wealth, contribute to GDP, and
  stimulate economic growth
\item
  \textbf{Risk Taking}: Accept business uncertainties and financial
  risks for potential profits
\end{itemize}

\textbf{Diagram:}

\begin{center}
\textbf{Mermaid Diagram (Code)}
\begin{verbatim}
{Shaded}
{Highlighting}[]
graph TD
    A[Entrepreneurship Functions] {-{-}{} B[Job Creation]}
    A {-{-}{} C[Innovation]}
    A {-{-}{} D[Economic Development]}
    A {-{-}{} E[Risk Taking]}
    B {-{-}{} F[Employment Generation]}
    C {-{-}{} G[New Products/Services]}
    D {-{-}{} H[GDP Growth]}
    E {-{-}{} I[Business Uncertainty]}
{Highlighting}
{Shaded}
\end{verbatim}
\end{center}

\end{solutionbox}
\begin{mnemonicbox}
``JIER'' - Job Innovation Economic Risk

\end{mnemonicbox}
\subsection*{Question 1(c) [7 marks]}\label{q1c}

\textbf{How MSMEs are important in the development of economy of India?}

\begin{solutionbox}

\begin{longtable}[]{@{}
  >{\raggedright\arraybackslash}p{(\linewidth - 2\tabcolsep) * \real{0.6000}}
  >{\raggedright\arraybackslash}p{(\linewidth - 2\tabcolsep) * \real{0.4000}}@{}}
\toprule\noalign{}
\begin{minipage}[b]{\linewidth}\raggedright
Contribution Area
\end{minipage} & \begin{minipage}[b]{\linewidth}\raggedright
Importance
\end{minipage} \\
\midrule\noalign{}
\endhead
\bottomrule\noalign{}
\endlastfoot
\textbf{Employment Generation} & Second largest employer after
agriculture \\
\textbf{Industrial Production} & Contributes 45\% of manufacturing
output \\
\textbf{Export Earnings} & Accounts for 40\% of total exports \\
\textbf{GDP Contribution} & Contributes around 30\% to India's GDP \\
\textbf{Rural Development} & Promotes balanced regional growth \\
\end{longtable}

\begin{itemize}
\tightlist
\item
  \textbf{Manufacturing Flexibility}: Quick adaptation to market changes
  and customer requirements
\item
  \textbf{Innovation Hub}: Supports large industries as suppliers and
  vendors
\item
  \textbf{Entrepreneurship Development}: Encourages individual business
  ownership and self-employment
\end{itemize}

\end{solutionbox}
\begin{mnemonicbox}
``EIGER-MIE'' - Employment Industrial GDP Export
Rural Manufacturing Innovation Entrepreneurship

\end{mnemonicbox}
\subsection*{Question 1(c) OR [7
marks]}\label{q1c}

\textbf{How Student Start-up and Innovation Policy (SSIP) helps diploma
students to start their own start-up?}

\begin{solutionbox}

\begin{longtable}[]{@{}
  >{\raggedright\arraybackslash}p{(\linewidth - 2\tabcolsep) * \real{0.5357}}
  >{\raggedright\arraybackslash}p{(\linewidth - 2\tabcolsep) * \real{0.4643}}@{}}
\toprule\noalign{}
\begin{minipage}[b]{\linewidth}\raggedright
SSIP Benefits
\end{minipage} & \begin{minipage}[b]{\linewidth}\raggedright
Description
\end{minipage} \\
\midrule\noalign{}
\endhead
\bottomrule\noalign{}
\endlastfoot
\textbf{Financial Support} & Seed funding and grants up to ₹2 lakhs \\
\textbf{Incubation Centers} & Access to 50+ incubation centers across
Gujarat \\
\textbf{Mentorship} & Industry expert guidance and counseling \\
\textbf{Infrastructure} & Free co-working spaces and equipment access \\
\textbf{Skill Development} & Entrepreneurship training programs \\
\end{longtable}

\begin{itemize}
\tightlist
\item
  \textbf{Academic Integration}: Start-up activities counted as academic
  credits
\item
  \textbf{IPR Support}: Help in patent filing and intellectual property
  protection
\item
  \textbf{Market Access}: Networking opportunities with investors and
  industry partners
\end{itemize}

\end{solutionbox}
\begin{mnemonicbox}
``FIMSAIM'' - Financial Incubation Mentorship Skill
Academic IPR Market

\end{mnemonicbox}
\subsection*{Question 2(a) [3 marks]}\label{q2a}

\textbf{What is project report? Show its importance in project
implementation.}

\begin{solutionbox}

A \textbf{project report} is a comprehensive document containing
detailed information about a proposed business venture including
technical, financial, and commercial aspects.

\textbf{Importance:}

\begin{itemize}
\tightlist
\item
  \textbf{Loan Approval}: Banks require project reports for financing
  decisions
\item
  \textbf{Resource Planning}: Helps in proper allocation of resources
  and manpower
\item
  \textbf{Risk Assessment}: Identifies potential challenges and
  mitigation strategies
\end{itemize}

\end{solutionbox}
\begin{mnemonicbox}
``LRR'' - Loan Resource Risk

\end{mnemonicbox}
\subsection*{Question 2(b) [4 marks]}\label{q2b}

\textbf{How the Break-Even Point (in terms of sales revenue) is
calculated? Also show graphical representation with example.}

\begin{solutionbox}

\textbf{Formula:} Break-Even Point (Sales) = Fixed Costs \div Contribution
Margin Ratio

Where: Contribution Margin Ratio = (Sales - Variable Costs) \div Sales

\textbf{Example:}

\begin{itemize}
\tightlist
\item
  Fixed Costs = ₹50,000
\item
  Selling Price per unit = ₹100
\item
  Variable Cost per unit = ₹60
\item
  Contribution per unit = ₹40
\item
  Contribution Margin Ratio = 40\%
\item
  Break-Even Sales = ₹50,000 \div 0.40 = ₹1,25,000
\end{itemize}

\textbf{Diagram:}

\begin{verbatim}
    Revenue/Costs (₹)
         |
    2,00,000 |     /
             |    /  Total Revenue
    1,50,000 |   /
             |  /
    1,25,000 |./\_\_\_\_\_ Break{-Even Point}
             |/     
    1,00,000 |      Total Costs
             |     /
     50,000  |\_\_\_\_/\_\_\_\_\_ Fixed Costs
             |
             +{-{-}{-}{-}{-}{-}{-}{-}{-}{-}{-}{-}{-}{-}{-}{-}{-}{-}{-}{-}{-}{-}{-}{-} Units Sold}
             0   500  1,250  2,000
\end{verbatim}

\end{solutionbox}
\begin{mnemonicbox}
``FCR'' - Fixed Costs Contribution Ratio

\end{mnemonicbox}
\subsection*{Question 2(c) [7 marks]}\label{q2c}

\textbf{Explain the need of market survey and also explain the market
test method of market survey.}

\begin{solutionbox}

\textbf{Need of Market Survey:}

\begin{longtable}[]{@{}ll@{}}
\toprule\noalign{}
Purpose & Description \\
\midrule\noalign{}
\endhead
\bottomrule\noalign{}
\endlastfoot
\textbf{Demand Assessment} & Understand customer needs and
preferences \\
\textbf{Competition Analysis} & Study competitor strategies and
pricing \\
\textbf{Market Size} & Estimate total addressable market \\
\textbf{Pricing Strategy} & Determine optimal price points \\
\end{longtable}

\textbf{Market Test Method:}

\begin{itemize}
\tightlist
\item
  \textbf{Test Marketing}: Launch product in limited geographic area
\item
  \textbf{Focus Groups}: Conduct discussions with target customers
\item
  \textbf{Pilot Studies}: Small-scale product trials with selected
  customers
\item
  \textbf{Online Surveys}: Digital questionnaires for broader reach
\end{itemize}

\begin{center}
\textbf{Mermaid Diagram (Code)}
\begin{verbatim}
{Shaded}
{Highlighting}[]
graph TD
    A[Market Survey Need] {-{-}{} B[Demand Assessment]}
    A {-{-}{} C[Competition Analysis]}
    A {-{-}{} D[Market Size]}
    A {-{-}{} E[Pricing Strategy]}
    
    F[Market Test Methods] {-{-}{} G[Test Marketing]}
    F {-{-}{} H[Focus Groups]}
    F {-{-}{} I[Pilot Studies]}
    F {-{-}{} J[Online Surveys]}
{Highlighting}
{Shaded}
\end{verbatim}
\end{center}

\end{solutionbox}
\begin{mnemonicbox}
``DCMP-TFPO'' - Demand Competition Market Pricing
Test Focus Pilot Online

\end{mnemonicbox}
\subsection*{Question 2(a) OR [3
marks]}\label{q2a}

\textbf{What is marketing plan? Explain in brief.}

\begin{solutionbox}

A \textbf{marketing plan} is a strategic document outlining how a
business will promote and sell its products or services to target
customers.

\textbf{Components:}

\begin{itemize}
\tightlist
\item
  \textbf{Market Analysis}: Customer demographics and behavior study
\item
  \textbf{Marketing Mix}: Product, Price, Place, Promotion strategies
\item
  \textbf{Budget Allocation}: Financial resources for marketing
  activities
\end{itemize}

\end{solutionbox}
\begin{mnemonicbox}
``AMB'' - Analysis Mix Budget

\end{mnemonicbox}
\subsection*{Question 2(b) OR [4
marks]}\label{q2b}

\textbf{Prepare SWOT analysis for a company manufacturing e-bike in
urban region in Gujarat.}

\begin{solutionbox}

\begin{longtable}[]{@{}
  >{\raggedright\arraybackslash}p{(\linewidth - 2\tabcolsep) * \real{0.3488}}
  >{\raggedright\arraybackslash}p{(\linewidth - 2\tabcolsep) * \real{0.6512}}@{}}
\toprule\noalign{}
\begin{minipage}[b]{\linewidth}\raggedright
SWOT Analysis
\end{minipage} & \begin{minipage}[b]{\linewidth}\raggedright
E-bike Manufacturing Company
\end{minipage} \\
\midrule\noalign{}
\endhead
\bottomrule\noalign{}
\endlastfoot
\textbf{Strengths} & • Government support for electric vehicles• Growing
environmental awareness• Lower operating costs than petrol vehicles \\
\textbf{Weaknesses} & • High initial investment• Limited charging
infrastructure• Battery replacement costs \\
\textbf{Opportunities} & • FAME scheme subsidies• Urban pollution
concerns• Rising fuel prices \\
\textbf{Threats} & • Competition from established players• Technology
obsolescence• Economic slowdown affecting purchasing power \\
\end{longtable}

\end{solutionbox}
\begin{mnemonicbox}
``SWOT-GILH'' - Strengths Weaknesses Opportunities
Threats Government Infrastructure Low High

\end{mnemonicbox}
\subsection*{Question 2(c) OR [7
marks]}\label{q2c}

\textbf{What is innovation? List any five innovations of any product or
process or service.}

\begin{solutionbox}

\textbf{Innovation} is the process of creating new or improved products,
services, or processes that provide value to customers and competitive
advantage to organizations.

\textbf{Five Product/Service Innovations:}

\begin{longtable}[]{@{}
  >{\raggedright\arraybackslash}p{(\linewidth - 4\tabcolsep) * \real{0.3871}}
  >{\raggedright\arraybackslash}p{(\linewidth - 4\tabcolsep) * \real{0.1935}}
  >{\raggedright\arraybackslash}p{(\linewidth - 4\tabcolsep) * \real{0.4194}}@{}}
\toprule\noalign{}
\begin{minipage}[b]{\linewidth}\raggedright
Innovation
\end{minipage} & \begin{minipage}[b]{\linewidth}\raggedright
Type
\end{minipage} & \begin{minipage}[b]{\linewidth}\raggedright
Description
\end{minipage} \\
\midrule\noalign{}
\endhead
\bottomrule\noalign{}
\endlastfoot
\textbf{UPI Payment System} & Service & Digital payment platform
revolutionizing transactions \\
\textbf{Tesla Electric Cars} & Product & Sustainable automotive
technology with autonomous features \\
\textbf{Netflix Streaming} & Service & On-demand entertainment delivery
model \\
\textbf{3D Printing} & Process & Additive manufacturing technology \\
\textbf{Zoom Video Calling} & Service & Remote communication platform
for virtual meetings \\
\end{longtable}

\begin{itemize}
\tightlist
\item
  \textbf{Value Creation}: Each innovation solved existing customer
  problems
\item
  \textbf{Market Disruption}: Changed traditional business models and
  user behavior
\item
  \textbf{Technology Integration}: Combined multiple technologies for
  enhanced user experience
\end{itemize}

\end{solutionbox}
\begin{mnemonicbox}
``UNTZI-VTM'' - UPI Netflix Tesla Zoom Innovation
Value Technology Market

\end{mnemonicbox}
\subsection*{Question 3(a) [3 marks]}\label{q3a}

\textbf{Write short note on partnership firm.}

\begin{solutionbox}

A \textbf{partnership firm} is a business structure where two or more
individuals jointly own and operate a business for profit.

\textbf{Key Features:}

\begin{itemize}
\tightlist
\item
  \textbf{Shared Ownership}: Multiple partners contribute capital and
  expertise
\item
  \textbf{Joint Liability}: Partners are personally liable for business
  debts
\item
  \textbf{Profit Sharing}: Earnings distributed according to partnership
  agreement
\end{itemize}

\end{solutionbox}
\begin{mnemonicbox}
``SJP'' - Shared Joint Profit

\end{mnemonicbox}
\subsection*{Question 3(b) [4 marks]}\label{q3b}

\textbf{Explain the various activities involved in `staffing' function
of management with example.}

\begin{solutionbox}

\begin{longtable}[]{@{}
  >{\raggedright\arraybackslash}p{(\linewidth - 4\tabcolsep) * \real{0.4500}}
  >{\raggedright\arraybackslash}p{(\linewidth - 4\tabcolsep) * \real{0.3250}}
  >{\raggedright\arraybackslash}p{(\linewidth - 4\tabcolsep) * \real{0.2250}}@{}}
\toprule\noalign{}
\begin{minipage}[b]{\linewidth}\raggedright
Staffing Activity
\end{minipage} & \begin{minipage}[b]{\linewidth}\raggedright
Description
\end{minipage} & \begin{minipage}[b]{\linewidth}\raggedright
Example
\end{minipage} \\
\midrule\noalign{}
\endhead
\bottomrule\noalign{}
\endlastfoot
\textbf{Recruitment} & Attracting potential candidates & Posting job
advertisements on LinkedIn \\
\textbf{Selection} & Choosing suitable candidates & Conducting
interviews and aptitude tests \\
\textbf{Training} & Skill development programs & New employee
orientation sessions \\
\textbf{Performance Appraisal} & Evaluating employee performance &
Annual performance reviews \\
\end{longtable}

\begin{itemize}
\tightlist
\item
  \textbf{Placement}: Assigning right person to right job position
\item
  \textbf{Promotion}: Career advancement based on performance and
  experience
\item
  \textbf{Compensation}: Determining fair wages and benefits package
\end{itemize}

\end{solutionbox}
\begin{mnemonicbox}
``RSTPPC'' - Recruitment Selection Training
Performance Placement Promotion Compensation

\end{mnemonicbox}
\subsection*{Question 3(c) [7 marks]}\label{q3c}

\textbf{Explain the autocratic leadership and state its advantages.}

\begin{solutionbox}

\textbf{Autocratic Leadership} is a management style where the leader
makes all decisions independently without consulting team members.

\textbf{Characteristics:}

\begin{itemize}
\tightlist
\item
  \textbf{Centralized Decision Making}: Leader has complete authority
  and control
\item
  \textbf{Clear Chain of Command}: Well-defined hierarchy and reporting
  structure
\item
  \textbf{Limited Employee Input}: Minimal participation in
  decision-making process
\end{itemize}

\textbf{Advantages:}

\begin{longtable}[]{@{}
  >{\raggedright\arraybackslash}p{(\linewidth - 2\tabcolsep) * \real{0.4583}}
  >{\raggedright\arraybackslash}p{(\linewidth - 2\tabcolsep) * \real{0.5417}}@{}}
\toprule\noalign{}
\begin{minipage}[b]{\linewidth}\raggedright
Advantage
\end{minipage} & \begin{minipage}[b]{\linewidth}\raggedright
Description
\end{minipage} \\
\midrule\noalign{}
\endhead
\bottomrule\noalign{}
\endlastfoot
\textbf{Quick Decisions} & Faster problem-solving without lengthy
consultations \\
\textbf{Clear Direction} & Employees know exactly what is expected \\
\textbf{Crisis Management} & Effective during emergencies requiring
immediate action \\
\textbf{Productivity} & Higher output due to structured work
environment \\
\textbf{Accountability} & Single point of responsibility for outcomes \\
\end{longtable}

\begin{center}
\textbf{Mermaid Diagram (Code)}
\begin{verbatim}
{Shaded}
{Highlighting}[]
graph TD
    A[Autocratic Leadership] {-{-}{} B[Centralized Decisions]}
    A {-{-}{} C[Clear Chain of Command]}
    A {-{-}{} D[Limited Employee Input]}
    
    E[Advantages] {-{-}{} F[Quick Decisions]}
    E {-{-}{} G[Clear Direction]}
    E {-{-}{} H[Crisis Management]}
    E {-{-}{} I[Higher Productivity]}
    E {-{-}{} J[Single Accountability]}
{Highlighting}
{Shaded}
\end{verbatim}
\end{center}

\end{solutionbox}
\begin{mnemonicbox}
``QCCPA'' - Quick Clear Crisis Productivity
Accountability

\end{mnemonicbox}
\subsection*{Question 3(a) OR [3
marks]}\label{q3a}

\textbf{Write short note on joint stock company.}

\begin{solutionbox}

A \textbf{joint stock company} is a business organization where capital
is divided into shares owned by multiple shareholders.

\textbf{Key Features:}

\begin{itemize}
\tightlist
\item
  \textbf{Limited Liability}: Shareholders' liability limited to their
  investment
\item
  \textbf{Transferable Shares}: Ownership can be easily bought and sold
\item
  \textbf{Separate Legal Entity}: Company exists independently of its
  owners
\end{itemize}

\end{solutionbox}
\begin{mnemonicbox}
``LTS'' - Limited Transferable Separate

\end{mnemonicbox}
\subsection*{Question 3(b) OR [4
marks]}\label{q3b}

\textbf{Explain the various activities involved in `organizing' function
of management with example.}

\begin{solutionbox}

\begin{longtable}[]{@{}
  >{\raggedright\arraybackslash}p{(\linewidth - 4\tabcolsep) * \real{0.4634}}
  >{\raggedright\arraybackslash}p{(\linewidth - 4\tabcolsep) * \real{0.3171}}
  >{\raggedright\arraybackslash}p{(\linewidth - 4\tabcolsep) * \real{0.2195}}@{}}
\toprule\noalign{}
\begin{minipage}[b]{\linewidth}\raggedright
Organizing Activity
\end{minipage} & \begin{minipage}[b]{\linewidth}\raggedright
Description
\end{minipage} & \begin{minipage}[b]{\linewidth}\raggedright
Example
\end{minipage} \\
\midrule\noalign{}
\endhead
\bottomrule\noalign{}
\endlastfoot
\textbf{Job Design} & Defining roles and responsibilities & Creating job
descriptions for marketing manager \\
\textbf{Departmentalization} & Grouping similar activities & Forming HR,
Finance, and Operations departments \\
\textbf{Delegation} & Assigning authority and responsibility & Manager
delegating budget approval to team leads \\
\textbf{Coordination} & Ensuring smooth workflow & Weekly
inter-department meetings \\
\end{longtable}

\begin{itemize}
\tightlist
\item
  \textbf{Resource Allocation}: Distributing financial and human
  resources efficiently
\item
  \textbf{Span of Control}: Determining number of subordinates per
  manager
\item
  \textbf{Unity of Command}: Each employee reports to one superior
\end{itemize}

\end{solutionbox}
\begin{mnemonicbox}
``JDDCRSU'' - Job Departmentalization Delegation
Coordination Resource Span Unity

\end{mnemonicbox}
\subsection*{Question 3(c) OR [7
marks]}\label{q3c}

\textbf{Explain the democratic leadership and state its advantages.}

\begin{solutionbox}

\textbf{Democratic Leadership} is a management style where leaders
involve team members in decision-making processes and encourage
participation.

\textbf{Characteristics:}

\begin{itemize}
\tightlist
\item
  \textbf{Participative Decision Making}: Team members contribute to
  problem-solving
\item
  \textbf{Open Communication}: Two-way communication between leaders and
  employees
\item
  \textbf{Shared Responsibility}: Collective ownership of outcomes and
  results
\end{itemize}

\textbf{Advantages:}

\begin{longtable}[]{@{}
  >{\raggedright\arraybackslash}p{(\linewidth - 2\tabcolsep) * \real{0.4583}}
  >{\raggedright\arraybackslash}p{(\linewidth - 2\tabcolsep) * \real{0.5417}}@{}}
\toprule\noalign{}
\begin{minipage}[b]{\linewidth}\raggedright
Advantage
\end{minipage} & \begin{minipage}[b]{\linewidth}\raggedright
Description
\end{minipage} \\
\midrule\noalign{}
\endhead
\bottomrule\noalign{}
\endlastfoot
\textbf{Higher Job Satisfaction} & Employees feel valued and heard \\
\textbf{Better Quality Decisions} & Multiple perspectives improve
decision quality \\
\textbf{Improved Creativity} & Diverse ideas and innovative solutions \\
\textbf{Team Building} & Stronger collaboration and trust \\
\textbf{Employee Development} & Skills enhancement through
participation \\
\end{longtable}

\begin{center}
\textbf{Mermaid Diagram (Code)}
\begin{verbatim}
{Shaded}
{Highlighting}[]
graph TD
    A[Democratic Leadership] {-{-}{} B[Participative Decisions]}
    A {-{-}{} C[Open Communication]}
    A {-{-}{} D[Shared Responsibility]}
    
    E[Advantages] {-{-}{} F[Job Satisfaction]}
    E {-{-}{} G[Quality Decisions]}
    E {-{-}{} H[Creativity]}
    E {-{-}{} I[Team Building]}
    E {-{-}{} J[Employee Development]}
{Highlighting}
{Shaded}
\end{verbatim}
\end{center}

\end{solutionbox}
\begin{mnemonicbox}
``JQCTE'' - Job Quality Creativity Team Employee

\end{mnemonicbox}
\subsection*{Question 4(a) [3 marks]}\label{q4a}

\textbf{State various functions of District Industries center.}

\begin{solutionbox}

\begin{itemize}
\tightlist
\item
  \textbf{Registration Services}: MSME registration and various license
  approvals
\item
  \textbf{Financial Assistance}: Guidance for loans and government
  scheme applications
\item
  \textbf{Technical Support}: Providing technical guidance and
  consultancy services
\end{itemize}

\end{solutionbox}
\begin{mnemonicbox}
``RFT'' - Registration Financial Technical

\end{mnemonicbox}
\subsection*{Question 4(b) [4 marks]}\label{q4b}

\textbf{Identify any two state level incubators and write their
functions.}

\begin{solutionbox}

\begin{longtable}[]{@{}
  >{\raggedright\arraybackslash}p{(\linewidth - 2\tabcolsep) * \real{0.5000}}
  >{\raggedright\arraybackslash}p{(\linewidth - 2\tabcolsep) * \real{0.5000}}@{}}
\toprule\noalign{}
\begin{minipage}[b]{\linewidth}\raggedright
Incubator
\end{minipage} & \begin{minipage}[b]{\linewidth}\raggedright
Functions
\end{minipage} \\
\midrule\noalign{}
\endhead
\bottomrule\noalign{}
\endlastfoot
\textbf{i-HUB Gujarat} & • Startup mentoring and acceleration programs•
Funding support and investor connections• Infrastructure and co-working
space facilities \\
\textbf{CIIE Ahmedabad} & • Technology commercialization support•
Industry-academia collaboration• Scale-up programs for growing
startups \\
\end{longtable}

\textbf{Common Functions:}

\begin{itemize}
\tightlist
\item
  \textbf{Mentorship Programs}: Expert guidance from industry
  professionals
\item
  \textbf{Networking Events}: Connecting startups with investors and
  partners
\end{itemize}

\end{solutionbox}
\begin{mnemonicbox}
``MFIN'' - Mentoring Funding Infrastructure
Networking

\end{mnemonicbox}
\subsection*{Question 4(c) [7 marks]}\label{q4c}

\textbf{What is start-up eco system? List various activities and
elements of start-up eco system.}

\begin{solutionbox}

\textbf{Start-up Ecosystem} is a interconnected network of
organizations, individuals, and resources that support entrepreneurship
and startup development.

\textbf{Key Elements:}

\begin{longtable}[]{@{}
  >{\raggedright\arraybackslash}p{(\linewidth - 2\tabcolsep) * \real{0.4091}}
  >{\raggedright\arraybackslash}p{(\linewidth - 2\tabcolsep) * \real{0.5909}}@{}}
\toprule\noalign{}
\begin{minipage}[b]{\linewidth}\raggedright
Element
\end{minipage} & \begin{minipage}[b]{\linewidth}\raggedright
Description
\end{minipage} \\
\midrule\noalign{}
\endhead
\bottomrule\noalign{}
\endlastfoot
\textbf{Entrepreneurs} & Visionary individuals starting new ventures \\
\textbf{Investors} & Angel investors, VCs providing funding \\
\textbf{Incubators/Accelerators} & Support organizations for early-stage
startups \\
\textbf{Government} & Policy makers and regulatory bodies \\
\textbf{Educational Institutions} & Universities and research centers \\
\textbf{Service Providers} & Legal, accounting, consulting firms \\
\end{longtable}

\textbf{Activities:}

\begin{itemize}
\tightlist
\item
  \textbf{Mentoring Sessions}: Regular guidance from experienced
  entrepreneurs
\item
  \textbf{Networking Events}: Startup meetups and investor pitch
  sessions
\item
  \textbf{Funding Rounds}: Seed, Series A, B funding opportunities
\item
  \textbf{Skill Development}: Technical and business training programs
\end{itemize}

\begin{center}
\textbf{Mermaid Diagram (Code)}
\begin{verbatim}
{Shaded}
{Highlighting}[]
graph TD
    A[Startup Ecosystem] {-{-}{} B[Entrepreneurs]}
    A {-{-}{} C[Investors]}
    A {-{-}{} D[Incubators]}
    A {-{-}{} E[Government]}
    A {-{-}{} F[Educational Institutions]}
    A {-{-}{} G[Service Providers]}
    
    H[Activities] {-{-}{} I[Mentoring]}
    H {-{-}{} J[Networking]}
    H {-{-}{} K[Funding]}
    H {-{-}{} L[Skill Development]}
{Highlighting}
{Shaded}
\end{verbatim}
\end{center}

\end{solutionbox}
\begin{mnemonicbox}
``EIGEES-MNFS'' - Entrepreneurs Investors Government
Education Service Mentoring Networking Funding Skill

\end{mnemonicbox}
\subsection*{Question 4(a) OR [3
marks]}\label{q4a}

\textbf{State various functions of Small industries Development Bank of
India (SIDBI).}

\begin{solutionbox}

\begin{itemize}
\tightlist
\item
  \textbf{Financial Services}: Direct and indirect lending to MSMEs and
  startups
\item
  \textbf{Development Services}: Capacity building and skill development
  programs
\item
  \textbf{Promotional Activities}: Market development and technology
  upgradation support
\end{itemize}

\end{solutionbox}
\begin{mnemonicbox}
``FDP'' - Financial Development Promotional

\end{mnemonicbox}
\subsection*{Question 4(b) OR [4
marks]}\label{q4b}

\textbf{Identify any two national level incubators and write their
functions.}

\begin{solutionbox}

\begin{longtable}[]{@{}
  >{\raggedright\arraybackslash}p{(\linewidth - 2\tabcolsep) * \real{0.5000}}
  >{\raggedright\arraybackslash}p{(\linewidth - 2\tabcolsep) * \real{0.5000}}@{}}
\toprule\noalign{}
\begin{minipage}[b]{\linewidth}\raggedright
Incubator
\end{minipage} & \begin{minipage}[b]{\linewidth}\raggedright
Functions
\end{minipage} \\
\midrule\noalign{}
\endhead
\bottomrule\noalign{}
\endlastfoot
\textbf{T-Hub Hyderabad} & • India's largest startup incubator•
Technology innovation and R\&D support• Global market access programs \\
\textbf{NASSCOM 10,000 Startups} & • Pan-India startup acceleration
program• Corporate partnership facilitation• Ecosystem building and
policy advocacy \\
\end{longtable}

\textbf{Common Functions:}

\begin{itemize}
\tightlist
\item
  \textbf{Acceleration Programs}: Intensive startup development and
  mentoring
\item
  \textbf{Corporate Connections}: Linking startups with large enterprise
  partners
\end{itemize}

\end{solutionbox}
\begin{mnemonicbox}
``TIGAC'' - Technology Innovation Global Acceleration
Corporate

\end{mnemonicbox}
\subsection*{Question 4(c) OR [7
marks]}\label{q4c}

\textbf{Which steps should be taken to avoid failure of start-up?
Explain in brief.}

\begin{solutionbox}

\textbf{Steps to Avoid Startup Failure:}

\begin{longtable}[]{@{}
  >{\raggedright\arraybackslash}p{(\linewidth - 2\tabcolsep) * \real{0.3158}}
  >{\raggedright\arraybackslash}p{(\linewidth - 2\tabcolsep) * \real{0.6842}}@{}}
\toprule\noalign{}
\begin{minipage}[b]{\linewidth}\raggedright
Step
\end{minipage} & \begin{minipage}[b]{\linewidth}\raggedright
Description
\end{minipage} \\
\midrule\noalign{}
\endhead
\bottomrule\noalign{}
\endlastfoot
\textbf{Market Research} & Thorough understanding of customer needs and
market demand \\
\textbf{Financial Planning} & Proper cash flow management and funding
strategies \\
\textbf{Team Building} & Hiring skilled and committed team members \\
\textbf{Product Validation} & Testing product-market fit before full
launch \\
\textbf{Customer Focus} & Continuous customer feedback and
satisfaction \\
\end{longtable}

\begin{itemize}
\tightlist
\item
  \textbf{Risk Management}: Identifying potential threats and mitigation
  strategies
\item
  \textbf{Adaptability}: Flexibility to pivot based on market changes
\item
  \textbf{Legal Compliance}: Proper registration and regulatory
  adherence
\end{itemize}

\begin{center}
\textbf{Mermaid Diagram (Code)}
\begin{verbatim}
{Shaded}
{Highlighting}[]
graph TD
    A[Avoid Startup Failure] {-{-}{} B[Market Research]}
    A {-{-}{} C[Financial Planning]}
    A {-{-}{} D[Team Building]}
    A {-{-}{} E[Product Validation]}
    A {-{-}{} F[Customer Focus]}
    A {-{-}{} G[Risk Management]}
    A {-{-}{} H[Adaptability]}
    A {-{-}{} I[Legal Compliance]}
{Highlighting}
{Shaded}
\end{verbatim}
\end{center}

\end{solutionbox}
\begin{mnemonicbox}
``MFTPCRAL'' - Market Financial Team Product Customer
Risk Adaptability Legal

\end{mnemonicbox}
\subsection*{Question 5(a) [3 marks]}\label{q5a}

\textbf{How Return on Investment (ROI) is calculated?}

\begin{solutionbox}

\textbf{ROI Formula:} ROI = (Net Profit \div Total Investment) \times 100

\textbf{Example:}

\begin{itemize}
\tightlist
\item
  Investment = ₹1,00,000
\item
  Net Profit = ₹20,000
\item
  ROI = (20,000 \div 1,00,000) \times 100 = 20\%
\end{itemize}

\end{solutionbox}
\begin{mnemonicbox}
``NTH'' - Net Total Hundred

\end{mnemonicbox}
\subsection*{Question 5(b) [4 marks]}\label{q5b}

\textbf{Show the significance of technical analysis in feasibility
study.}

\begin{solutionbox}

\begin{longtable}[]{@{}
  >{\raggedright\arraybackslash}p{(\linewidth - 2\tabcolsep) * \real{0.5185}}
  >{\raggedright\arraybackslash}p{(\linewidth - 2\tabcolsep) * \real{0.4815}}@{}}
\toprule\noalign{}
\begin{minipage}[b]{\linewidth}\raggedright
Significance
\end{minipage} & \begin{minipage}[b]{\linewidth}\raggedright
Description
\end{minipage} \\
\midrule\noalign{}
\endhead
\bottomrule\noalign{}
\endlastfoot
\textbf{Technology Assessment} & Evaluating technical viability and
requirements \\
\textbf{Resource Planning} & Determining machinery, equipment, and
infrastructure needs \\
\textbf{Process Design} & Optimal production methods and workflow \\
\textbf{Quality Standards} & Ensuring product meets industry
specifications \\
\end{longtable}

\end{solutionbox}
\begin{mnemonicbox}
``TRPQ'' - Technology Resource Process Quality

\end{mnemonicbox}
\subsection*{Question 5(c) [7 marks]}\label{q5c}

\textbf{Explain the characteristics of corporate social responsibility.}

\begin{solutionbox}

\textbf{Corporate Social Responsibility (CSR)} refers to business
practices involving initiatives that benefit society and demonstrate
commitment to ethical operations.

\textbf{Key Characteristics:}

\begin{longtable}[]{@{}
  >{\raggedright\arraybackslash}p{(\linewidth - 2\tabcolsep) * \real{0.5517}}
  >{\raggedright\arraybackslash}p{(\linewidth - 2\tabcolsep) * \real{0.4483}}@{}}
\toprule\noalign{}
\begin{minipage}[b]{\linewidth}\raggedright
Characteristic
\end{minipage} & \begin{minipage}[b]{\linewidth}\raggedright
Description
\end{minipage} \\
\midrule\noalign{}
\endhead
\bottomrule\noalign{}
\endlastfoot
\textbf{Voluntary Nature} & Beyond legal requirements, self-imposed
commitments \\
\textbf{Stakeholder Orientation} & Considering impact on all
stakeholders, not just shareholders \\
\textbf{Triple Bottom Line} & Focus on People, Planet, and Profit \\
\textbf{Sustainable Practices} & Long-term environmental and social
sustainability \\
\textbf{Transparency} & Open reporting and accountability \\
\end{longtable}

\begin{itemize}
\tightlist
\item
  \textbf{Community Development}: Education, healthcare, and
  infrastructure projects
\item
  \textbf{Environmental Protection}: Pollution control and resource
  conservation
\item
  \textbf{Employee Welfare}: Fair wages, safe working conditions, skill
  development
\end{itemize}

\begin{center}
\textbf{Mermaid Diagram (Code)}
\begin{verbatim}
{Shaded}
{Highlighting}[]
graph TD
    A[CSR Characteristics] {-{-}{} B[Voluntary Nature]}
    A {-{-}{} C[Stakeholder Orientation]}
    A {-{-}{} D[Triple Bottom Line]}
    A {-{-}{} E[Sustainable Practices]}
    A {-{-}{} F[Transparency]}
    
    G[CSR Activities] {-{-}{} H[Community Development]}
    G {-{-}{} I[Environmental Protection]}
    G {-{-}{} J[Employee Welfare]}
{Highlighting}
{Shaded}
\end{verbatim}
\end{center}

\end{solutionbox}
\begin{mnemonicbox}
``VSTST-CEE'' - Voluntary Stakeholder Triple
Sustainable Transparency Community Environmental Employee

\end{mnemonicbox}
\subsection*{Question 5(a) OR [3
marks]}\label{q5a}

\textbf{How Return on Sales (ROS) is calculated?}

\begin{solutionbox}

\textbf{ROS Formula:} ROS = (Net Profit \div Net Sales) \times 100

\textbf{Example:}

\begin{itemize}
\tightlist
\item
  Net Sales = ₹5,00,000
\item
  Net Profit = ₹50,000
\item
  ROS = (50,000 \div 5,00,000) \times 100 = 10\%
\end{itemize}

\end{solutionbox}
\begin{mnemonicbox}
``NSH'' - Net Sales Hundred

\end{mnemonicbox}
\subsection*{Question 5(b) OR [4
marks]}\label{q5b}

\textbf{Show the significance of market analysis in feasibility study.}

\begin{solutionbox}

\begin{longtable}[]{@{}ll@{}}
\toprule\noalign{}
Significance & Description \\
\midrule\noalign{}
\endhead
\bottomrule\noalign{}
\endlastfoot
\textbf{Demand Forecasting} & Estimating future market size and
growth \\
\textbf{Competition Assessment} & Understanding competitive landscape \\
\textbf{Pricing Strategy} & Determining optimal price points \\
\textbf{Market Segmentation} & Identifying target customer groups \\
\end{longtable}

\end{solutionbox}
\begin{mnemonicbox}
``DCPM'' - Demand Competition Pricing Market

\end{mnemonicbox}
\subsection*{Question 5(c) OR [7
marks]}\label{q5c}

\textbf{Explain the characteristics of ethics.}

\begin{solutionbox}

\textbf{Ethics} are moral principles that govern behavior and
decision-making in personal and professional contexts.

\textbf{Key Characteristics:}

\begin{longtable}[]{@{}
  >{\raggedright\arraybackslash}p{(\linewidth - 2\tabcolsep) * \real{0.5517}}
  >{\raggedright\arraybackslash}p{(\linewidth - 2\tabcolsep) * \real{0.4483}}@{}}
\toprule\noalign{}
\begin{minipage}[b]{\linewidth}\raggedright
Characteristic
\end{minipage} & \begin{minipage}[b]{\linewidth}\raggedright
Description
\end{minipage} \\
\midrule\noalign{}
\endhead
\bottomrule\noalign{}
\endlastfoot
\textbf{Universal Principles} & Apply across cultures and situations \\
\textbf{Moral Standards} & Based on concepts of right and wrong \\
\textbf{Voluntary Compliance} & Internal motivation rather than external
force \\
\textbf{Consequential Thinking} & Considering outcomes and impacts \\
\textbf{Stakeholder Consideration} & Accounting for all affected
parties \\
\end{longtable}

\begin{itemize}
\tightlist
\item
  \textbf{Consistency}: Ethical behavior remains constant across
  situations
\item
  \textbf{Transparency}: Open and honest communication and actions
\item
  \textbf{Accountability}: Taking responsibility for decisions and their
  consequences
\end{itemize}

\begin{center}
\textbf{Mermaid Diagram (Code)}
\begin{verbatim}
{Shaded}
{Highlighting}[]
graph TD
    A[Ethics Characteristics] {-{-}{} B[Universal Principles]}
    A {-{-}{} C[Moral Standards]}
    A {-{-}{} D[Voluntary Compliance]}
    A {-{-}{} E[Consequential Thinking]}
    A {-{-}{} F[Stakeholder Consideration]}
    A {-{-}{} G[Consistency]}
    A {-{-}{} H[Transparency]}
    A {-{-}{} I[Accountability]}
{Highlighting}
{Shaded}
\end{verbatim}
\end{center}

\end{solutionbox}
\begin{mnemonicbox}
``UMVCSCTA'' - Universal Moral Voluntary
Consequential Stakeholder Consistency Transparency Accountability

\end{mnemonicbox}

\end{document}
