%% METADATA
%% subject-code: SUBJECT001
%% subject-name: Subject Name
%% semester: 1
%% examination: Sample-2025
%% date: 01-01-2025
%% language: gujarati
%% description: Sample solution guide demonstrating LaTeX conventions for GTU exam solutions in Gujarati
%% summary: Complete Gujarati example showing proper structure, formatting, and content organization
%% tags: study-material, solutions, sample, gtu, subject001, gujarati
%% END METADATA

\documentclass{article}
% GTU Solutions - Gujarati Preamble
% Includes common preamble + Gujarati font setup

% Basic setup
\usepackage[margin=1in]{geometry}
\author{Milav Dabgar}

% Math and tables
\usepackage{amsmath,amssymb,amsthm}
\usepackage{booktabs}
\usepackage{tabularx}
\usepackage{graphicx}
\usepackage{float}  % Required for [H] float placement

% Code listings with syntax highlighting
\usepackage{xcolor}
\usepackage{listings}
\lstset{
  basicstyle=\small\ttfamily,
  breaklines=true,
  numbers=left,
  numberstyle=\tiny\color{gray},
  xleftmargin=2em,
  frame=single,
  showstringspaces=false,
  tabsize=2,
  keywordstyle=\color{blue},
  commentstyle=\color{green!60!black},
  stringstyle=\color{purple}
}

% Optional: TikZ for diagrams (remove if not needed)
\usepackage{tikz}
\usepackage{circuitikz}
\usetikzlibrary{shapes,arrows,positioning,calc}

% Header/footer with author and website
\usepackage{fancyhdr}
\usepackage{lastpage}

\pagestyle{fancy}
\fancyhf{}
\fancyhead[L]{\small\itshape\leftmark}
\fancyhead[R]{\small Milav Dabgar}
\fancyfoot[L]{\small\href{https://www.milav.in}{www.milav.in}}
\fancyfoot[R]{\small Page \thepage\ of \pageref{LastPage}}
\renewcommand{\headrulewidth}{0.4pt}
\renewcommand{\footrulewidth}{0.4pt}

% Hyperref (load before fontspec for Gujarati)
\usepackage[
  colorlinks=true,
  linkcolor=blue,
  urlcolor=blue,
  citecolor=blue,
  pdfauthor={Milav Dabgar},
  pdfsubject={GTU Exam Solutions},
  pdfkeywords={GTU, Java, Programming, Solutions, Gujarati},
  bookmarks=true
]{hyperref}

% Gujarati font setup
\usepackage{fontspec}
\usepackage{polyglossia}
\setdefaultlanguage{gujarati}
\setotherlanguage{english}
\newfontfamily\gujaratifont[Script=Gujarati,AutoFakeBold=2.5,AutoFakeSlant=0.3]{Noto Sans Gujarati}
\setmainfont[Script=Gujarati,AutoFakeBold=2.5,AutoFakeSlant=0.3]{Noto Sans Gujarati}
\setmonofont[Scale=0.9]{Noto Sans Gujarati}
\newfontfamily\englishfont[Script=Gujarati,AutoFakeBold=2.5,AutoFakeSlant=0.3]{Noto Sans Gujarati}
\gappto\captionsgujarati{
  \renewcommand{\tablename}{કોષ્ટક}
  \renewcommand{\figurename}{આકૃતિ}
}
\newcommand{\gu}[1]{{\gujaratifont #1}}


\title{Subject Name (SUBJECT001) - Sample Term Solution}
\date{મહિના દિવસ, વર્ષ}

% PDF Metadata
\hypersetup{
  pdftitle={Subject Name (SUBJECT001) - Sample Term Solution (Gujarati)},
  pdfsubject={GTU Exam Solution - Sample-2025},
  pdfauthor={Milav Dabgar},
  pdfkeywords={study-material, solutions, sample, gtu, subject001, gujarati},
  pdfcreator={XeLaTeX},
  pdflang={gu}
}

\begin{document}
\maketitle

% ========================================
% QUESTION 1(a): Programming Code (3 marks)
% Demonstrates: lstlisting for code, straight quotes in code, ~100 words
% ========================================

\section{Question 1}
\subsection{Question 1(અ) [3 marks]}
ત્રણ numbers માંથી maximum શોધવા માટે જાવા પ્રોગ્રામ લખો.

\subsubsection{Solution}
ત્રણ નંબરોમાંથી \textbf{મેક્સિમમ} શોધવા માટે, અમે વેલ્યુઝની સરખામણી કરવા માટે \textbf{કન્ડિશનલ સ્ટેટમેન્ટ્સ} (if-else) નો ઉપયોગ કરીએ છીએ. પ્રોગ્રામ ત્રણ નંબરો ઇનપુટ તરીકે લે છે અને તેમાંથી ``\emph{સૌથી મોટી વેલ્યુ}'' પરત કરે છે.

\paragraph{જાવા પ્રોગ્રામ:}
\begin{lstlisting}[language=Java,caption={ત્રણ Numbers માંથી Maximum શોધો}]
public class MaxOfThree {
    public static void main(String[] args) {
        int a = 25, b = 40, c = 15;
        int max;
        
        // Compare first two numbers
        if (a > b) {
            max = a;
        } else {
            max = b;
        }
        
        // Compare result with third number
        if (c > max) {
            max = c;
        }
        
        System.out.println("Maximum number is: " + max);
    }
}
\end{lstlisting}

\paragraph{આઉટપુટ:}
\begin{verbatim}
Maximum number is: 40
\end{verbatim}

\paragraph{મુખ્ય મુદ્દાઓ:}
\begin{description}
    \item[લોજિક:] પ્રથમ \texttt{a} અને \texttt{b} ની સરખામણી કરો, મોટી વેલ્યુ ને \texttt{max} માં સ્ટોર કરો
    \item[બીજી સરખામણી:] અંતિમ મેક્સિમમ મેળવવા માટે \texttt{max} ને \texttt{c} સાથે સરખાવો
    \item[વૈકલ્પિક:] સંક્ષિપ્ત કોડ માટે \texttt{Math.max(a, Math.max(b, c))} નો ઉપયોગ કરી શકાય
\end{description}

\paragraph{મેમરી ટ્રીક:}
\emph{MAX: જોડીમાં સરખાવો, પરીક્ષણે મહત્તમ અપડેટ કરો}

% ========================================
% QUESTION 1(b): Mathematics Calculation (4 marks)
% Demonstrates: Math equations, step-by-step calculation, inline \texttt{}, ~150 words
% ========================================

\subsection{Question 1(બ) [4 marks]}
આરસી લો-પાસ ફિલ્ટર નું કટઓફ ફ્રિક્વન્સી શોધો જ્યાં \(R = 1.5\,k\Omega\) અને \(C = 100\,nF\) છે. તેમજ કટઓફ ફ્રિક્વન્સી પર જો ઇનપુટ 10V હોય તો આઉટપુટ વોલ્ટેજ શોધો.

\subsubsection{Solution}
\paragraph{આપેલ માહિતી:}
\begin{itemize}
    \item રેઝિસ્ટન્સ: \(R = 1.5\,k\Omega = 1500\,\Omega\)
    \item કેપેસિટન્સ: \(C = 100\,nF = 100 \times 10^{-9}\,F\)
    \item ઇનપુટ વોલ્ટેજ: \(V_{in} = 10\,V\)
\end{itemize}

\paragraph{પગલું 1: કટઓફ ફ્રિક્વન્સી ની ગણતરી}
આરસી લો-પાસ ફિલ્ટર માટે \textbf{કટઓફ ફ્રિક્વન્સી} નો ફોર્મ્યુલા છે:
\[f_c = \frac{1}{2\pi RC}\]

મૂલ્યો મૂકીએ:
\[f_c = \frac{1}{2\pi \times 1500 \times 100 \times 10^{-9}}\]
\[f_c = \frac{1}{2\pi \times 1.5 \times 10^{-4}}\]
\[f_c = \frac{1}{9.42 \times 10^{-4}} = 1061.57\,Hz \approx 1.06\,kHz\]

\paragraph{પગલું 2: કટઓફ પર આઉટપુટ વોલ્ટેજ}
કટઓફ ફ્રિક્વન્સી પર, આઉટપુટ વોલ્ટેજ એ ઇનપુટ વોલ્ટેજ ના \textbf{0.707 ગણા} (અથવા \(\frac{1}{\sqrt{2}}\)) હોય છે:
\[V_{out} = 0.707 \times V_{in} = 0.707 \times 10 = 7.07\,V\]

\paragraph{પરિણામો:}
\begin{description}
    \item[કટઓફ ફ્રિક્વન્સી:] \(f_c = 1.06\,kHz\)
    \item[આઉટપુટ વોલ્ટેજ:] \(V_{out} = 7.07\,V\) કટઓફ પર
    \item[એટેન્યુએશન:] \(-3\,dB\) કટઓફ ફ્રિક્વન્સી પર
    \item[ફેઝ શિફ્ટ:] \(-45^\circ\) કટઓફ ફ્રિક્વન્સી પર
\end{description}

\paragraph{મેમરી ટ્રીક:}
\emph{RC-Formula:} \(f_c = \frac{1}{2\pi RC}\), \(V_{out} = 0.707 \times V_{in}\) at \(f_c\)

% ========================================
% QUESTION 1(c): Comparison Table (7 marks)
% Demonstrates: Comprehensive comparison, \tabulary{} with caption at TOP, ~250 words
% ========================================

\subsection{Question 1(ક) [7 marks]}
એક્ટિવ અને પેસિવ ઇલેક્ટ્રોનિક કોમ્પોનન્ટ્સ ની યોગ્ય ઉદાહરણો સાથે તુલના કરો.

\subsubsection{Solution}
ઇલેક્ટ્રોનિક કોમ્પોનન્ટ્સ ને \textbf{એક્ટિવ} અને \textbf{પેસિવ} કેટેગરીમાં વર્ગીકૃત કરવામાં આવે છે જે તેમની ઇલેક્ટ્રિકલ એનર્જી ને કંટ્રોલ અથવા એમ્પ્લિફાય કરવાની ક્ષમતા પર આધારિત છે.

\begin{table}[H]
\centering
\caption{Active vs Passive Components Comparison}
\begin{tabularx}{\textwidth}{lXX}
\toprule
\textbf{લાક્ષણિકતા} & \textbf{Active Components} & \textbf{Passive Components} \\
\midrule
Energy Source & બાહ્ય power source જરૂરી & બાહ્ય power જરૂરી નથી \\
Control Ability & Current flow ને control/amplify કરી શકે & Amplify નહીં, ફક્ત regulate કરે \\
Directionality & સામાન્ય રીતે unidirectional & Bidirectional \\
Power Gain & Power gain આપે (\(>1\)) & Power gain હંમેશા \(\leq 1\) \\
ઉદાહરણો & Transistors (BJT, FET), Diodes (LED, Zener), ICs (Op-Amp, 555), SCR & Resistors, Capacitors, Inductors, Transformers \\
કાર્ય & Amplification, switching, oscillation, rectification & Resistance, capacitance, inductance, filtering \\
Linearity & Linear અથવા non-linear હોઈ શકે & સામાન્ય રીતે linear \\
\bottomrule
\end{tabularx}
\end{table}

\paragraph{Active Components વિગતવાર:}
\begin{description}
    \item[Transistors:] Amplification અને switching માટે વપરાય છે. BJT current control વાપરે, FET voltage control વાપરે છે.
    \item[Diodes:] એક દિશામાં current ને પસાર થવા દે છે. LED પ્રકાશ બહાર કાઢે, Zener voltage regulate કરે છે.
    \item[ICs:] Integrated circuits જેવા કે \texttt{555 timer} (oscillator), op-amps (amplifier).
    \item[Power Requirement:] બધા active components ને ચાલુ થવા માટે DC bias/supply જરૂરી.
\end{description}

\paragraph{Passive Components વિગતવાર:}
\begin{description}
    \item[Resistors:] Current flow નો વિરોધ કરે, power ને heat તરીકે dissipate કરે. મૂલ્ય \(\Omega\) માં.
    \item[Capacitors:] Electric field માં energy સંગ્રહ કરે. મૂલ્ય Farads (F) માં, DC block કરે, AC પસાર કરે.
    \item[Inductors:] Magnetic field માં energy સંગ્રહ કરે. મૂલ્ય Henry (H) માં, AC ફેરફારોનો વિરોધ કરે.
    \item[Transformers:] Magnetic coupling દ્વારા circuits વચ્ચે energy transfer કરે.
\end{description}

\paragraph{મુખ્ય તફાવત:}
મૂળભૂત તફાવત એ છે કે active components circuit માં \emph{power inject} કરી શકે (amplification), જ્યારે passive components ફક્ત energy \emph{absorb અથવા store} કરી શકે, તેને ક્યારેય વધારી શકતા નથી.

\paragraph{મેમરી ટ્રીક:}
\emph{ACTIVE = Amplify, Control, Transform; PASSIVE = Resist, Store, Filter}

% ========================================
% QUESTION 1(c OR): Alternative Question (7 marks)
% Demonstrates: OR question format, TikZ diagram with gtu styles, lstlisting, ~250 words
% ========================================

\subsection{Question 1(ક OR) [7 marks]}
Half-wave rectifier circuit ને input અને output waveforms સાથે દોરો અને તેનું કાર્ય સમજાવો.

\subsubsection{Solution}
\textbf{Half-wave rectifier} AC voltage ને pulsating DC માં રૂપાંતરિત કરે છે input AC waveform ના ફક્ત એક half-cycle (positive અથવા negative) ને પસાર થવા દઈને.

\paragraph{Circuit Diagram:}
\begin{figure}[H]
\centering
\begin{circuitikz}[scale=1.2]
    % AC Source
    \draw (0,0) to[sV, l=\(V_{in}\)] (0,2);
    \draw (0,2) to[short] (2,2);
    
    % Diode
    \draw (2,2) to[D*, l=\(D\)] (4,2);
    
    % Load Resistor
    \draw (4,2) to[short] (5,2);
    \draw (5,2) to[R, l=\(R_L\)] (5,0);
    \draw (5,0) to[short] (0,0);
    
    % Output voltage measurement
    \draw (4.5,2) to[short, *-] (4.5,2.5);
    \node at (4.5,2.7) {\(V_{out}\)};
    \draw (4.5,0) to[short, *-] (4.5,-0.5);
    \node[ground] at (4.5,-0.5) {};
\end{circuitikz}
\caption{Half-Wave Rectifier Circuit}
\end{figure}

\paragraph{કાર્ય સિદ્ધાંત:}
\begin{description}
    \item[Positive Half-Cycle:] જ્યારે input AC positive હોય, diode forward-biased (conduct) થાય છે. Current load resistor \(R_L\) માંથી વહે છે, output voltage ઉત્પન્ન કરે છે.
    \item[Negative Half-Cycle:] જ્યારે input AC negative હોય, diode reverse-biased (block) થાય છે. કોઈ current વહેતું નથી, output voltage શૂન્ય હોય છે.
    \item[પરિણામ:] Output પર ફક્ત positive half-cycles દેખાય છે, pulsating DC બનાવે છે.
\end{description}

\paragraph{Waveform Representation:}
\begin{figure}[H]
\centering
\begin{tikzpicture}[scale=0.9]
    % Input waveform
    \draw[->] (0,0) -- (6.5,0) node[right] {\(t\)};
    \draw[->] (0,-1.5) -- (0,1.8) node[above] {\(V_{in}\)};
    \draw[thick, blue] (0,0) sin (1,1) cos (2,0) sin (3,-1) cos (4,0) sin (5,1) cos (6,0);
    \node at (3,1.5) {Input AC};
    
    % Output waveform
    \begin{scope}[yshift=-3.5cm]
        \draw[->] (0,0) -- (6.5,0) node[right] {\(t\)};
        \draw[->] (0,-0.3) -- (0,1.8) node[above] {\(V_{out}\)};
        \draw[thick, red] (0,0) sin (1,1) cos (2,0);
        \draw[thick, red] (2,0) -- (4,0);
        \draw[thick, red] (4,0) sin (5,1) cos (6,0);
        \node at (3,1.5) {Output Pulsating DC};
    \end{scope}
\end{tikzpicture}
\caption{Input and Output Waveforms}
\end{figure}

\paragraph{મુખ્ય પરિમાણો:}
\begin{description}
    \item[Efficiency:] \(\eta = 40.6\%\) (સૈદ્ધાંતિક મહત્તમ)
    \item[Ripple Factor:] \(r = 1.21\) (ઉચ્ચ ripple content)
    \item[Peak Inverse Voltage (PIV):] \(PIV = V_m\) (diode પર મહત્તમ reverse voltage)
    \item[DC Output:] \(V_{DC} = \frac{V_m}{\pi} = 0.318 V_m\) જ્યાં \(V_m\) એ peak AC voltage છે
\end{description}

\paragraph{એપ્લિકેશન્સ:}
Half-wave rectifiers નો ઉપયોગ low-power applications માં થાય છે જેવા કે battery charging, signal demodulation, અને voltage multipliers. તેઓ poor efficiency ને કારણે high-power applications માટે \emph{યોગ્ય નથી}.

\paragraph{મેમરી ટ્રીક:}
\emph{HWR: Half-Wave = અડધું output, 40.6\% efficiency, PIV = Vm}

\end{document}
