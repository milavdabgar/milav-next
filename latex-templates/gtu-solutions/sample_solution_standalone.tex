%% METADATA
%% subject-code: SUBJECT001
%% subject-name: Subject Name
%% semester: 1
%% examination: Sample-2025
%% date: 01-01-2025
%% description: Standalone LaTeX solution with ZERO custom dependencies
%% summary: Pure standard LaTeX with no external files or custom commands
%% tags: study-material, solutions, sample, gtu, subject001, standalone
%% END METADATA

\documentclass[11pt]{article}

% Essential packages only
\usepackage[margin=1in]{geometry}
\usepackage{amsmath,amssymb,amsthm}
\usepackage{booktabs}
\usepackage{tabularx}
\usepackage{graphicx}
\usepackage{listings}
\usepackage{hyperref}

% Listings configuration for code
\lstset{
  basicstyle=\ttfamily\small,
  breaklines=true,
  frame=single,
  numbers=left,
  numberstyle=\tiny,
  showstringspaces=false
}

% Hyperref configuration
\hypersetup{
  colorlinks=true,
  linkcolor=blue,
  urlcolor=blue,
  citecolor=blue
}

\title{Subject Name (SUBJECT001) - Sample Term Solution}
\author{Milav Dabgar}
\date{Month Day, Year}

\begin{document}
\maketitle

\section{Question 1}

\subsection{Question 1(a) [3 marks]}
\textbf{Write a Java program to find the maximum of three numbers.}

\subsubsection{Solution}

To find the \textbf{maximum} of three numbers, we use \textbf{conditional statements} (if-else) to compare values. The program takes three numbers as input and returns the ``largest value'' among them.

\paragraph{Java Program:}
\begin{lstlisting}[language=Java]
public class MaxOfThree {
    public static void main(String[] args) {
        int a = 25, b = 40, c = 15;
        int max;
        
        // Compare first two numbers
        if (a > b) {
            max = a;
        } else {
            max = b;
        }
        
        // Compare result with third number
        if (c > max) {
            max = c;
        }
        
        System.out.println("Maximum number is: " + max);
    }
}
\end{lstlisting}

\paragraph{Output:}
\begin{verbatim}
Maximum number is: 40
\end{verbatim}

\paragraph{Key Points:}
\begin{itemize}
    \item \textbf{Logic}: First compare \texttt{a} and \texttt{b}, store larger in \texttt{max}
    \item \textbf{Second Comparison}: Compare \texttt{max} with \texttt{c} to get final maximum
    \item \textbf{Alternative}: Can use \texttt{Math.max(a, Math.max(b, c))} for concise code
\end{itemize}

\begin{quote}
\textbf{Mnemonic:} \emph{MAX: Compare in pairs, update Maximum At eXamination}
\end{quote}

\subsection{Question 1(b) [4 marks]}
\textbf{Calculate the cutoff frequency of an RC low-pass filter with $R = 1.5\,k\Omega$ and $C = 100\,nF$. Also find the output voltage at cutoff frequency if input is 10V.}

\subsubsection{Solution}

\paragraph{Given Data:}
\begin{itemize}
    \item Resistance: $R = 1.5\,k\Omega = 1500\,\Omega$
    \item Capacitance: $C = 100\,nF = 100 \times 10^{-9}\,F$
    \item Input Voltage: $V_{in} = 10\,V$
\end{itemize}

\paragraph{Step 1: Calculate Cutoff Frequency}

The \textbf{cutoff frequency} formula for RC low-pass filter is:
\[f_c = \frac{1}{2\pi RC}\]

Substituting values:
\[f_c = \frac{1}{2\pi \times 1500 \times 100 \times 10^{-9}}\]
\[f_c = \frac{1}{2\pi \times 1.5 \times 10^{-4}}\]
\[f_c = \frac{1}{9.42 \times 10^{-4}} = 1061.57\,Hz \approx 1.06\,kHz\]

\paragraph{Step 2: Calculate Output Voltage at Cutoff}

At cutoff frequency, output voltage is \textbf{0.707 times} (or $\frac{1}{\sqrt{2}}$) the input voltage:
\[V_{out} = 0.707 \times V_{in} = 0.707 \times 10 = 7.07\,V\]

\paragraph{Results:}
\begin{itemize}
    \item \textbf{Cutoff Frequency}: $f_c = 1.06\,kHz$
    \item \textbf{Output Voltage}: $V_{out} = 7.07\,V$ at cutoff
    \item \textbf{Attenuation}: $-3\,dB$ at cutoff frequency
    \item \textbf{Phase Shift}: $-45^\circ$ at cutoff frequency
\end{itemize}

\begin{quote}
\textbf{Mnemonic:} \emph{RC-Formula: fc = 1/(2$\pi$ RC), Vout = 0.707 Vin at fc}
\end{quote}

\subsection{Question 1(c) [7 marks]}
\textbf{Compare active and passive electronic components with suitable examples.}

\subsubsection{Solution}

Electronic components are classified into \textbf{active} and \textbf{passive} categories based on their ability to control or amplify electrical energy.

\begin{table}[h]
\centering
\caption{Active vs Passive Components Comparison}
\begin{tabularx}{\textwidth}{lXX}
\toprule
\textbf{Characteristic} & \textbf{Active Components} & \textbf{Passive Components} \\
\midrule
Energy Source & Require external power source & Do not require external power \\
Control Ability & Can control/amplify current flow & Cannot amplify, only regulate \\
Directionality & Usually unidirectional & Bidirectional \\
Power Gain & Provide power gain ($>1$) & Power gain is always $\leq 1$ \\
Examples & Transistors (BJT, FET), Diodes (LED, Zener), ICs (Op-Amp, 555), SCR & Resistors, Capacitors, Inductors, Transformers \\
Function & Amplification, switching, oscillation, rectification & Resistance, capacitance, inductance, filtering \\
Linearity & Can be linear or non-linear & Generally linear \\
\bottomrule
\end{tabularx}
\end{table}

\paragraph{Active Components in Detail:}
\begin{itemize}
    \item \textbf{Transistors}: Used for amplification and switching. BJT uses current control, FET uses voltage control.
    \item \textbf{Diodes}: Allow current in one direction. LED emits light, Zener regulates voltage.
    \item \textbf{ICs}: Integrated circuits like 555 timer (oscillator), op-amps (amplifier).
    \item \textbf{Power Requirement}: All active components need DC bias/supply to operate.
\end{itemize}

\paragraph{Passive Components in Detail:}
\begin{itemize}
    \item \textbf{Resistors}: Oppose current flow, dissipate power as heat. Value in $\Omega$.
    \item \textbf{Capacitors}: Store energy in electric field. Value in Farads (F), block DC, pass AC.
    \item \textbf{Inductors}: Store energy in magnetic field. Value in Henry (H), oppose AC changes.
    \item \textbf{Transformers}: Transfer energy between circuits via magnetic coupling.
\end{itemize}

\paragraph{Key Distinction:}
The fundamental difference is that active components can ``inject power'' into a circuit (amplification), while passive components can only ``absorb or store'' energy, never increase it.

\begin{quote}
\textbf{Mnemonic:} \emph{ACTIVE = Amplify, Control, Transform; PASSIVE = Resist, Store, Filter}
\end{quote}

\subsection{Question 1(c OR) [7 marks]}
\textbf{Draw and explain the working of a half-wave rectifier circuit with input and output waveforms.}

\subsubsection{Solution}

A \textbf{half-wave rectifier} converts AC voltage to pulsating DC by allowing only one half-cycle (positive or negative) of the input AC waveform to pass through.

\paragraph{Circuit Description:}
The circuit consists of:
\begin{itemize}
    \item AC voltage source ($V_{in}$)
    \item Diode (D)
    \item Load resistor ($R_L$)
\end{itemize}

\paragraph{Working Principle:}
\begin{enumerate}
    \item \textbf{Positive Half-Cycle}: When input AC is positive, diode is forward-biased (conducts). Current flows through load resistor $R_L$, producing output voltage.
    \item \textbf{Negative Half-Cycle}: When input AC is negative, diode is reverse-biased (blocks). No current flows, output voltage is zero.
    \item \textbf{Result}: Only positive half-cycles appear at output, creating pulsating DC.
\end{enumerate}

\paragraph{Key Parameters:}
\begin{itemize}
    \item \textbf{Efficiency}: $\eta = 40.6\%$ (theoretical maximum)
    \item \textbf{Ripple Factor}: $r = 1.21$ (high ripple content)
    \item \textbf{Peak Inverse Voltage (PIV)}: $PIV = V_m$ (maximum reverse voltage across diode)
    \item \textbf{DC Output}: $V_{DC} = \frac{V_m}{\pi} = 0.318 V_m$ where $V_m$ is peak AC voltage
\end{itemize}

\paragraph{Applications:}
Half-wave rectifiers are used in low-power applications such as battery charging, signal demodulation, and voltage multipliers. They are \textit{not suitable} for high-power applications due to poor efficiency.

\begin{quote}
\textbf{Mnemonic:} \emph{HWR: Half-Wave = Half output, 40.6\% efficiency, PIV = Vm}
\end{quote}

\end{document}
