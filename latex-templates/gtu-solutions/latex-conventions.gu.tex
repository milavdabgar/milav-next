\documentclass{article}
% Adjust the relative path to point to the latex-templates directory
% Example (for files deep in content/resources/...): 
% Absolute paths for template files

% content/resources/templates/preamble.tex
\usepackage[margin=0.6in]{geometry}
\author{Milav Dabgar}
\usepackage{amsmath,amssymb,amsthm}
\usepackage{booktabs}
\usepackage{multirow}
\usepackage{xcolor}
\usepackage{tcolorbox}
\tcbuselibrary{breakable,skins}
\usepackage[colorlinks=true,linkcolor=blue]{hyperref}
\usepackage{titlesec}
\usepackage{enumitem}
\usepackage{tikz}
\usepackage{pgfplots}
\usepackage{circuitikz}
\usepackage[version=4]{mhchem}
\usepackage{longtable}
\usepackage{array}
\usepackage{float}
\usepackage{caption}
\usepackage{listings}

\lstset{
  basicstyle=\small\ttfamily,
  breaklines=true,
  breakatwhitespace=false,
  postbreak=\mbox{\textcolor{red}{$\hookrightarrow$}\space},
  float=false,
  numbers=left,
  numberstyle=\tiny\color{gray},
  numbersep=10pt,
  xleftmargin=2em,
  keywordstyle=\color{blue},
  commentstyle=\color{green!60!black},
  stringstyle=\color{purple},
  backgroundcolor=\color{gray!5},
  showstringspaces=false,
  tabsize=2,
  captionpos=b,
  keepspaces=true,
  columns=flexible
}

\pgfplotsset{compat=1.18}
\usetikzlibrary{shapes,arrows,positioning,calc,patterns,decorations.pathmorphing,decorations.markings,arrows.meta}

% Color scheme
\definecolor{headcolor}{RGB}{0,102,204}
\definecolor{keycolor}{RGB}{220,20,60}
\definecolor{solutioncolor}{RGB}{34,139,34}
\definecolor{mnemoniccolor}{RGB}{148,0,211}
\definecolor{codecolor}{RGB}{0,0,100}

% Spacing
\setlength{\parskip}{3pt}
\setlist[itemize]{nosep}
\setlist[enumerate]{nosep}

% Title formatting
\titleformat{\section}{\Large\bfseries\color{headcolor}}{\thesection}{1em}{}
\titleformat{\subsection}{\large\bfseries\color{headcolor}}{\thesubsection}{1em}{}

% Pandoc tightlist compatibility
\providecommand{\tightlist}{%
  \setlength{\itemsep}{0pt}\setlength{\parskip}{0pt}}

% Pandoc longtable compatibility
\newcounter{none}
\def\thenone{}


% content/resources/templates/gujarati-boxes.tex
\usepackage{fontspec}
\usepackage{polyglossia}

% Set Gujarati as main language (document is primarily in Gujarati)
% Note: gloss-gujarati.ldf doesn't exist in polyglossia, but it will use hyphenation patterns
\setdefaultlanguage{gujarati}
\setotherlanguage{english}

% Configure Gujarati font properly
% Use Language=Default to prevent polyglossia from trying to add language-specific features
% that don't exist for Gujarati, which causes "empty feature" warnings
\newfontfamily\gujaratifont[Script=Gujarati,AutoFakeBold=2.5,AutoFakeSlant=0.3]{Noto Sans Gujarati}
\setmainfont[Script=Gujarati,AutoFakeBold=2.5,AutoFakeSlant=0.3]{Noto Sans Gujarati}
% Use Noto Sans Gujarati for monospace to support Gujarati in text
\setmonofont[Scale=0.9]{Noto Sans Gujarati}

% Configure English to use the same font
\newfontfamily\englishfont[Script=Gujarati,AutoFakeBold=2.5,AutoFakeSlant=0.3]{Noto Sans Gujarati}

% Translations for polyglossia
\gappto\captionsgujarati{
  \renewcommand{\tablename}{કોષ્ટક}
  \renewcommand{\figurename}{આકૃતિ}
}

% Helper for TikZ nodes to ensure Gujarati font
\newcommand{\gu}[1]{{\gujaratifont #1}}

% Custom environments
\newtcolorbox{solutionbox}{
    breakable,
    enhanced,
    colback=solutioncolor!5!white,
    colframe=solutioncolor!75!black,
    fonttitle=\bfseries,
    title=જવાબ
}

\newtcolorbox{solutionboxnobreak}{
 colback=solutioncolor!5!white,
 colframe=solutioncolor!75!black,
 fonttitle=\bfseries,
 title=જવાબ
}

\newtcolorbox{keyformula}{
 breakable,
 enhanced,
 colback=keycolor!5!white,
 colframe=keycolor!75!black,
 fonttitle=\bfseries,
 title=રાસાયણિક સમીકરણ/સૂત્ર
}

\newtcolorbox{mnemonicbox}{
 breakable,
 enhanced,
 colback=mnemoniccolor!5!white,
 colframe=mnemoniccolor!75!black,
 fonttitle=\bfseries,
 title=મેમરી ટ્રીક
}


% Custom commands for GTU solutions
% This file defines semantic commands for consistent formatting

% Question command with automatic formatting
\newcommand{\question}[2]{%
  \section*{Question #1}%
  \textbf{#2}%
}

% OR question variant
\newcommand{\questionor}[2]{%
  \section*{Question #1 OR}%
  \textbf{#2}%
}

% Proper table environment with caption
\newenvironment{answertable}[1]{%
  \begin{table}[htbp]
  \centering
  \caption{#1}
}{%
  \end{table}
}

% Proper figure environment for diagrams
\newenvironment{answerdiagram}[1]{%
  \begin{figure}[htbp]
  \centering
  \caption{#1}
}{%
  \end{figure}
}

% Semantic markup for key terms
\newcommand{\keyword}[1]{\textbf{#1}}
\newcommand{\code}[1]{\texttt{#1}}
\newcommand{\classname}[1]{\texttt{#1}}
\newcommand{\methodname}[1]{\texttt{#1}}

% Proper quotation marks
\newcommand{\mnemonic}[1]{``#1''}


\title{વિષયનું નામ (કોડ) - પરીક્ષા ટર્મ સોલ્યુશન}
\date{મહિનો દિવસ, વર્ષ}

\begin{document}
\maketitle

% ==================================================================
% GUIDELINES for 100/100 Quality Solutions (Gujarati Version)
% 1. Use semantic commands: \questionmarks, \solutionbox, \mnemonicbox.
% 2. Use \code{...} for inline code, NO \texttt.
% 3. Use LaTeX smart quotes in text: ``Double'' and `Single'.
% 4. Use straight quotes in code listings: "String".
% 5. Tables: Caption at TOP. Use \tabulary for adaptive width (preferred).
% 6. Figures: Caption at BOTTOM, use \captionof{figure}{...}.
% 7. TikZ: Use standard TikZ node styles with semantic inline styling.
% 8. Gujarati text: Use gujarati-boxes.tex for proper font rendering.
% ==================================================================

\questionmarks{1(અ)}{3}{``સ્માર્ટ કોટ્સ'' અને ઇનલાઇન \code{code} સ્ટાઇલિંગનો ઉપયોગ સમજાવો.}

\begin{solutionbox}
જ્યારે ટેક્સ્ટ લખો છો, ત્યારે હંમેશા LaTeX સ્માર્ટ કોટ્સનો ઉપયોગ કરો. ડબલ કોટ્સ માટે, બે બેકટિક્સ અને બે સિંગલ કોટ્સનો ઉપયોગ કરો: ``આ રીતે''. સિંગલ કોટ્સ માટે, એક બેકટિક અને એક સિંગલ કોટનો ઉપયોગ કરો: `આ રીતે'.

ઇનલાઇન કોડ ફ્રેગમેન્ટ્સ માટે, \code{\\code\{\}} કમાન્ડનો ઉપયોગ કરો. \code{\\texttt\{\}} નો ઉપયોગ ન કરો.

\begin{itemize}
    \item \keyword{સુસંગતતા}: બધી ફાઇલો વ્યાવસાયિક દેખાય તેની ખાતરી કરે છે.
    \item \keyword{ટાઇપોગ્રાફી}: સ્માર્ટ કોટ્સ ટેક્સ્ટમાં સ્ટ્રેટ કોટ્સ કરતાં વધુ સારા દેખાય છે.
\end{itemize}
\end{solutionbox}

\begin{mnemonicbox}
\mnemonic{SQC: Smart Quotes in Content, Straight Quotes in Code}
\end{mnemonicbox}

\questionmarks{1(બ)}{4}{ટેબલનો ઉપયોગ કરીને Abstract Class અને Interface વચ્ચે તફાવત દર્શાવો.}

\begin{solutionbox}
તુલના માટે \code{table} એન્વાયર્નમેન્ટનો ઉપયોગ કરો. નોંધ કરો કે કેપ્શન ટેબલની \textbf{ઉપર} મૂકવામાં આવે છે.

અમે સ્ટાન્ડર્ડ એન્વાયર્નમેન્ટ તરીકે \code{tabulary} નો ઉપયોગ કરીએ છીએ કારણ કે તે સામગ્રીની પહોળાઈને આપોઆપ અનુકૂલિત કરે છે. જરૂર મુજબ લપેટાતા લેફ્ટ-એલાઇન કૉલમ્સ માટે \code{L}, સેન્ટર માટે \code{C} વગેરેનો ઉપયોગ કરો.

\textbf{ઉદાહરણ 1: ટેક્સ્ટ લપેટવું (એડેપ્ટિવ)}
\begin{center}
\captionof{table}{Abstract Class vs Interface}
\begin{tabulary}{\linewidth}{|L|L|L|}
\hline
\textbf{ફીચર} & \textbf{Abstract Class} & \textbf{Interface} \\ \hline
મેથડ & અમલીકરણ હોઈ શકે છે & એબ્સ્ટ્રેક્ટ (મોટે ભાગે) \\ \hline
વેરિએબલ્સ & કોઈપણ પ્રકાર & \code{public static final} \\ \hline
ઇનહેરિટ & \code{extends} કીવર્ડ & \code{implements} કીવર્ડ \\ \hline
\end{tabulary}
\end{center}

\textbf{ઉદાહરણ 2: કોમ્પેક્ટ ડેટા (ઓટો-શ્રિંક્સ)}
\begin{center}
\captionof{table}{સરળ ડેટા તુલના}
\begin{tabulary}{\linewidth}{|L|L|}
\hline
\textbf{પ્રકાર} & \textbf{સાઇઝ} \\ \hline
int & 4 બાઇટ્સ \\ \hline
long & 8 બાઇટ્સ \\ \hline
\end{tabulary}
\end{center}
\end{solutionbox}

\begin{mnemonicbox}
\mnemonic{MVI: Methods Variables Inheritance differences}
\end{mnemonicbox}

\questionmarks{1(ક)}{7}{થ્રેડ લાઇફ સાઇકલ દોરો અને થ્રેડ બનાવવાનું દર્શાવવા માટે પ્રોગ્રામ લખો.}

\begin{solutionbox}
ડાયાગ્રામ માટે, standard TikZ inline styles વાપરો. Caption \textbf{નીચે} આવે.

\begin{center}
\begin{tikzpicture}[node distance=1.5cm, auto]
    % Explicit node styles વાપરો: circle, rectangle, diamond, etc.
    \node [circle, draw, fill=blue!10, minimum width=2.5em] (N) {New};
    \node [circle, draw, fill=blue!10, minimum width=2.5em, right=1.5cm of N] (R) {Runnable};
    \node [circle, draw, fill=blue!10, minimum width=2.5em, right=1.5cm of R] (Rn) {Running};
    
    \path [draw, -latex] (N) -- (R);
    \path [draw, -latex] (R) -- (Rn);
\end{tikzpicture}
\captionof{figure}{સરળ Thread Life Cycle}
\end{center}

કોડ માટે, \code{lstlisting} એન્વાયર્નમેન્ટનો ઉપયોગ કરો. ખાતરી કરો કે તમે કોડ બ્લોકની અંદર \textbf{સ્ટ્રેટ કોટ્સ} (\code{"}) નો ઉપયોગ કરો છો જેથી ટેક્સ્ટ યોગ્ય રીતે કૉપિ કરી શકાય.

\begin{lstlisting}[language=Java,caption={થ્રેડ બનાવવાનું ઉદાહરણ}]
class MyThread extends Thread {
    public void run() {
        // યોગ્ય: સ્ટ્રેટ કોટ્સનો ઉપયોગ
        System.out.println("Thread is running...");
    }
}

public class Main {
    public static void main(String[] args) {
        MyThread t1 = new MyThread();
        t1.start(); // થ્રેડ શરૂ કરો
    }
}
\end{lstlisting}
\end{solutionbox}

\begin{mnemonicbox}
\mnemonic{NRR: New Runnable Running states}
\end{mnemonicbox}

\questionmarks{1(ક OR)}{7}{અમે વૈકલ્પિક પ્રશ્નો કેવી રીતે હેન્ડલ કરીએ છીએ?}

\begin{solutionbox}
પ્રશ્ન નંબર આર્ગ્યુમેન્ટમાં ફક્ત \code{OR} સફિક્સનો ઉપયોગ કરો. આ અંતિમ દસ્તાવેજમાં યોગ્ય ઇન્ડેક્સિંગ સુનિશ્ચિત કરે છે.

\begin{itemize}
    \item \keyword{લવચીકતા}: GTU ના વૈકલ્પિક પ્રશ્ન ફોર્મેટને સપોર્ટ કરે છે.
    \item \keyword{એપોસ્ટ્રોફી}: સિંગલ કોટ \code{'} નો ઉપયોગ કરો: તે સરળ છે!
\end{itemize}
\end{solutionbox}

\end{document}
