\documentclass{article}
% Adjust the relative path to point to the latex-templates directory
% Example (for files deep in content/resources/...): 
% 
% content/resources/templates/preamble.tex
\usepackage[margin=0.6in]{geometry}
\author{Milav Dabgar}
\usepackage{amsmath,amssymb,amsthm}
\usepackage{booktabs}
\usepackage{multirow}
\usepackage{xcolor}
\usepackage{tcolorbox}
\tcbuselibrary{breakable,skins}
\usepackage[colorlinks=true,linkcolor=blue]{hyperref}
\usepackage{titlesec}
\usepackage{enumitem}
\usepackage{tikz}
\usepackage{pgfplots}
\usepackage{circuitikz}
\usepackage[version=4]{mhchem}
\usepackage{longtable}
\usepackage{array}
\usepackage{float}
\usepackage{caption}
\usepackage{listings}

\lstset{
  basicstyle=\small\ttfamily,
  breaklines=true,
  breakatwhitespace=false,
  postbreak=\mbox{\textcolor{red}{$\hookrightarrow$}\space},
  float=false,
  numbers=left,
  numberstyle=\tiny\color{gray},
  numbersep=10pt,
  xleftmargin=2em,
  keywordstyle=\color{blue},
  commentstyle=\color{green!60!black},
  stringstyle=\color{purple},
  backgroundcolor=\color{gray!5},
  showstringspaces=false,
  tabsize=2,
  captionpos=b,
  keepspaces=true,
  columns=flexible
}

\pgfplotsset{compat=1.18}
\usetikzlibrary{shapes,arrows,positioning,calc,patterns,decorations.pathmorphing,decorations.markings,arrows.meta}

% Color scheme
\definecolor{headcolor}{RGB}{0,102,204}
\definecolor{keycolor}{RGB}{220,20,60}
\definecolor{solutioncolor}{RGB}{34,139,34}
\definecolor{mnemoniccolor}{RGB}{148,0,211}
\definecolor{codecolor}{RGB}{0,0,100}

% Spacing
\setlength{\parskip}{3pt}
\setlist[itemize]{nosep}
\setlist[enumerate]{nosep}

% Title formatting
\titleformat{\section}{\Large\bfseries\color{headcolor}}{\thesection}{1em}{}
\titleformat{\subsection}{\large\bfseries\color{headcolor}}{\thesubsection}{1em}{}

% Pandoc tightlist compatibility
\providecommand{\tightlist}{%
  \setlength{\itemsep}{0pt}\setlength{\parskip}{0pt}}

% Pandoc longtable compatibility
\newcounter{none}
\def\thenone{}


% content/resources/templates/preamble.tex
\usepackage[margin=0.6in]{geometry}
\author{Milav Dabgar}
\usepackage{amsmath,amssymb,amsthm}
\usepackage{booktabs}
\usepackage{multirow}
\usepackage{xcolor}
\usepackage{tcolorbox}
\tcbuselibrary{breakable,skins}
\usepackage[colorlinks=true,linkcolor=blue]{hyperref}
\usepackage{titlesec}
\usepackage{enumitem}
\usepackage{tikz}
\usepackage{pgfplots}
\usepackage{circuitikz}
\usepackage[version=4]{mhchem}
\usepackage{longtable}
\usepackage{array}
\usepackage{float}
\usepackage{caption}
\usepackage{listings}

\lstset{
  basicstyle=\small\ttfamily,
  breaklines=true,
  breakatwhitespace=false,
  postbreak=\mbox{\textcolor{red}{$\hookrightarrow$}\space},
  float=false,
  numbers=left,
  numberstyle=\tiny\color{gray},
  numbersep=10pt,
  xleftmargin=2em,
  keywordstyle=\color{blue},
  commentstyle=\color{green!60!black},
  stringstyle=\color{purple},
  backgroundcolor=\color{gray!5},
  showstringspaces=false,
  tabsize=2,
  captionpos=b,
  keepspaces=true,
  columns=flexible
}

\pgfplotsset{compat=1.18}
\usetikzlibrary{shapes,arrows,positioning,calc,patterns,decorations.pathmorphing,decorations.markings,arrows.meta}

% Color scheme
\definecolor{headcolor}{RGB}{0,102,204}
\definecolor{keycolor}{RGB}{220,20,60}
\definecolor{solutioncolor}{RGB}{34,139,34}
\definecolor{mnemoniccolor}{RGB}{148,0,211}
\definecolor{codecolor}{RGB}{0,0,100}

% Spacing
\setlength{\parskip}{3pt}
\setlist[itemize]{nosep}
\setlist[enumerate]{nosep}

% Title formatting
\titleformat{\section}{\Large\bfseries\color{headcolor}}{\thesection}{1em}{}
\titleformat{\subsection}{\large\bfseries\color{headcolor}}{\thesubsection}{1em}{}

% Pandoc tightlist compatibility
\providecommand{\tightlist}{%
  \setlength{\itemsep}{0pt}\setlength{\parskip}{0pt}}

% Pandoc longtable compatibility
\newcounter{none}
\def\thenone{}


% content/resources/templates/english-boxes.tex

% Custom environments
\newtcolorbox{solutionbox}{
 breakable,
 enhanced,
 colback=solutioncolor!5!white,
 colframe=solutioncolor!75!black,
 fonttitle=\bfseries,
 title=Solution
}

\newtcolorbox{solutionboxnobreak}{
 colback=solutioncolor!5!white,
 colframe=solutioncolor!75!black,
 fonttitle=\bfseries,
 title=Solution
}

\newtcolorbox{keyformula}{
 breakable,
 enhanced,
 colback=keycolor!5!white,
 colframe=keycolor!75!black,
 fonttitle=\bfseries,
 title=Key Formula
}

\newtcolorbox{mnemonicboxenv}{
 breakable,
 enhanced,
 colback=mnemoniccolor!5!white,
 colframe=mnemoniccolor!75!black,
 fonttitle=\bfseries,
 title=Mnemonic
}

\newcommand{\mnemonicbox}[1]{%
  \begin{mnemonicboxenv}
    #1
  \end{mnemonicboxenv}
}


% Custom commands for GTU solutions
% This file defines semantic commands for consistent formatting

% Question command with automatic formatting
\newcommand{\question}[2]{%
  \section*{Question #1}%
  \textbf{#2}%
}

% OR question variant
\newcommand{\questionor}[2]{%
  \section*{Question #1 OR}%
  \textbf{#2}%
}

% Proper table environment with caption
\newenvironment{answertable}[1]{%
  \begin{table}[htbp]
  \centering
  \caption{#1}
}{%
  \end{table}
}

% Proper figure environment for diagrams
\newenvironment{answerdiagram}[1]{%
  \begin{figure}[htbp]
  \centering
  \caption{#1}
}{%
  \end{figure}
}

% Semantic markup for key terms
\newcommand{\keyword}[1]{\textbf{#1}}
\newcommand{\code}[1]{\texttt{#1}}
\newcommand{\classname}[1]{\texttt{#1}}
\newcommand{\methodname}[1]{\texttt{#1}}

% Proper quotation marks
\newcommand{\mnemonic}[1]{``#1''}


\title{Subject Name (Code) - Exam Term Solution}
\date{Month Day, Year}

\begin{document}
\maketitle

% ==================================================================
% GUIDELINES for 100/100 Quality Solutions
% 1. Use semantic commands: \questionmarks, \solutionbox, \mnemonicbox.
% 2. Use \code{...} for inline code, NO \texttt.
% 3. Use LaTeX smart quotes in text: ``Double'' and `Single'.
% 4. Use straight quotes in code listings: "String".
% 5. Tables: Caption at TOP. Use \tabulary for adaptive width (preferred).
% 6. Figures: Caption at BOTTOM, use \captionof{figure}{...}.
% 7. TikZ: Use global styles (gtu block, gtu arrow, gtu state, etc.).
% ==================================================================

\questionmarks{1(a)}{3}{Explain the usage of ``Smart Quotes'' and inline \code{code} styling.}

\begin{solutionbox}
When writing text, always use LaTeX smart quotes. For double quotes, use two backticks and two single quotes: ``Like This''. For single quotes, use one backtick and one single quote: `Like This'.

For inline code fragments, use the \code{\\code\{\}} command. Do not use \code{\\texttt\{\}}.

\begin{itemize}
    \item \keyword{Consistency}: Ensures all files look professional.
    \item \keyword{Typography}: Smart quotes look better than straight quotes in text.
\end{itemize}
\end{solutionbox}

\begin{mnemonicbox}
\mnemonic{SQC: Smart Quotes in Content, Straight Quotes in Code}
\end{mnemonicbox}

\questionmarks{1(b)}{4}{Differentiate between Abstract Class and Interface using a table.}

\begin{solutionbox}
Use the \code{table} environment for comparisons. Note that the caption is placed \textbf{above} the table.

\textbf{Option 1: Full Width (Text Wrapping)} - Use \code{tabularx} when you have long text.
\begin{center}
\captionof{table}{Abstract Class vs Interface}
\begin{tabularx}{\linewidth}{|l|X|X|}
\hline
\textbf{Feature} & \textbf{Abstract Class} & \textbf{Interface} \\ \hline
Method & Can have implementation & Abstract (mostly) \\ \hline
Vars & Any type & \code{public static final} \\ \hline
Inherit & \code{extends} keyword & \code{implements} keyword \\ \hline
\end{tabularx}
\end{center}

\textbf{Option 2: Natural Width (Compact)} - Use \code{tabular} for simple data.
\begin{center}
\captionof{table}{Simple Data Comparison}
\begin{tabular}{|l|l|}
\hline
\textbf{Type} & \textbf{Size} \\ \hline
int & 4 bytes \\ \hline
long & 8 bytes \\ \hline
\end{tabular}
\end{center}

\textbf{Option 3: Adaptive Width (Preferred)} - Use \code{tabulary} for smart sizing.
\begin{center}
\captionof{table}{Adaptive Table (Shrinks or Wraps)}
\begin{tabulary}{\linewidth}{|L|L|}
\hline
\textbf{Feature} & \textbf{Description} \\ \hline
Smart Sizing & This table will shrink if content is small. \\ \hline
Wrapping & It will also wrap automatically if the content is very long, like this sentence right here. \\ \hline
\end{tabulary}
\end{center}
\end{solutionbox}

\begin{mnemonicbox}
\mnemonic{MVI: Methods Variables Inheritance differences}
\end{mnemonicbox}

\questionmarks{1(c)}{7}{Draw the thread life cycle and write a program to demonstrate thread creation.}

\begin{solutionbox}
For diagrams, use the global \code{gtu} styles defined in \code{preamble.tex}. Caption goes \textbf{below} the figure.

\begin{center}
\begin{tikzpicture}[node distance=1.5cm, auto]
    % Available styles: gtu block, gtu state, gtu decision, gtu loop, gtu database, gtu arrow
    \node [gtu state] (N) {New};
    \node [gtu state, right=1.5cm of N] (R) {Runnable};
    \node [gtu state, right=1.5cm of R] (Rn) {Running};
    
    \path [gtu arrow] (N) -- (R);
    \path [gtu arrow] (R) -- (Rn);
\end{tikzpicture}
\captionof{figure}{Simplified Thread Life Cycle}
\end{center}

For code, use the \code{lstlisting} environment. Ensure you use \textbf{straight quotes} (\code{"}) inside the code block so text can be copied correctly.

\begin{lstlisting}[language=Java,caption={Thread Creation Example}]
class MyThread extends Thread {
    public void run() {
        // correct: using straight quotes
        System.out.println("Thread is running...");
    }
}

public class Main {
    public static void main(String[] args) {
        MyThread t1 = new MyThread();
        t1.start(); // Start the thread
    }
}
\end{lstlisting}
\end{solutionbox}

\begin{mnemonicbox}
\mnemonic{NRR: New Runnable Running states}
\end{mnemonicbox}

\questionmarks{1(c OR)}{7}{How do we handle alternative questions?}

\begin{solutionbox}
Simply use the \code{OR} suffix in the question number argument. This ensures proper indexing in the final document.

\begin{itemize}
    \item \keyword{Flexibility}: Supports GTU's optional question format.
    \item \keyword{Apostrophes}: Use single quote \code{'}: It's easy!
\end{itemize}
\end{solutionbox}

\end{document}
