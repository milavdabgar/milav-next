%% METADATA
%% subject-code: SUBJECT001
%% subject-name: Subject Name
%% semester: 1
%% examination: Sample-2025
%% date: 01-01-2025
%% language: gujarati
%% description: Standalone Gujarati LaTeX solution with ZERO custom dependencies
%% summary: Pure standard LaTeX with no external files or custom commands
%% tags: study-material, solutions, sample, gtu, subject001, standalone, gujarati
%% END METADATA

\documentclass[11pt]{article}

% Essential packages only
\usepackage[margin=1in]{geometry}
\usepackage{amsmath,amssymb,amsthm}
\usepackage{booktabs}
\usepackage{tabularx}
\usepackage{graphicx}
\usepackage{listings}
\usepackage{fontspec}
\usepackage{polyglossia}
\usepackage{hyperref}

% Gujarati font and language setup
\setmainlanguage{gujarati}
\setotherlanguage{english}
\setmainfont{Noto Sans Gujarati}

% Listings configuration for code
\lstset{
  basicstyle=\ttfamily\small,
  breaklines=true,
  frame=single,
  numbers=left,
  numberstyle=\tiny,
  showstringspaces=false
}

% Hyperref configuration
\hypersetup{
  colorlinks=true,
  linkcolor=blue,
  urlcolor=blue,
  citecolor=blue
}

\title{Subject Name (SUBJECT001) - Sample Term Solution}
\author{Milav Dabgar}
\date{મહિના દિવસ, વર્ષ}

\begin{document}
\maketitle

\section{Question 1}

\subsection{Question 1(અ) [3 marks]}
\textbf{ત્રણ numbers માંથી maximum શોધવા માટે Java program લખો.}

\subsubsection{Solution}

ત્રણ numbers માંથી \textbf{maximum} શોધવા માટે, અમે values ની સરખામણી કરવા માટે \textbf{conditional statements} (if-else) નો ઉપયોગ કરીએ છીએ. Program ત્રણ numbers input તરીકે લે છે અને તેમાંથી ``સૌથી મોટી value'' પરત કરે છે.

\paragraph{Java Program:}
\begin{lstlisting}[language=Java]
public class MaxOfThree {
    public static void main(String[] args) {
        int a = 25, b = 40, c = 15;
        int max;
        
        // Compare first two numbers
        if (a > b) {
            max = a;
        } else {
            max = b;
        }
        
        // Compare result with third number
        if (c > max) {
            max = c;
        }
        
        System.out.println("Maximum number is: " + max);
    }
}
\end{lstlisting}

\paragraph{Output:}
\begin{verbatim}
Maximum number is: 40
\end{verbatim}

\paragraph{મુખ્ય મુદ્દાઓ:}
\begin{itemize}
    \item \textbf{Logic}: પ્રથમ \texttt{a} અને \texttt{b} ની સરખામણી કરો, મોટી value ને \texttt{max} માં store કરો
    \item \textbf{બીજી સરખામણી}: અંતિમ maximum મેળવવા માટે \texttt{max} ને \texttt{c} સાથે સરખાવો
    \item \textbf{વૈકલ્પિક}: Concise code માટે \texttt{Math.max(a, Math.max(b, c))} નો ઉપયોગ કરી શકાય
\end{itemize}

\begin{quote}
\textbf{મેમરી ટ્રીક:} \emph{MAX: જોડીમાં સરખાવો, પરીક્ષણે મહત્તમ અપડેટ કરો}
\end{quote}

\subsection{Question 1(બ) [4 marks]}
\textbf{RC low-pass filter નું cutoff frequency શોધો જ્યાં $R = 1.5\,k\Omega$ અને $C = 100\,nF$ છે. તેમજ cutoff frequency પર જો input 10V હોય તો output voltage શોધો.}

\subsubsection{Solution}

\paragraph{આપેલ માહિતી:}
\begin{itemize}
    \item Resistance: $R = 1.5\,k\Omega = 1500\,\Omega$
    \item Capacitance: $C = 100\,nF = 100 \times 10^{-9}\,F$
    \item Input Voltage: $V_{in} = 10\,V$
\end{itemize}

\paragraph{પગલું 1: Cutoff Frequency ની ગણતરી}

RC low-pass filter માટે \textbf{cutoff frequency} નો formula છે:
\[f_c = \frac{1}{2\pi RC}\]

મૂલ્યો મૂકીએ:
\[f_c = \frac{1}{2\pi \times 1500 \times 100 \times 10^{-9}}\]
\[f_c = \frac{1}{2\pi \times 1.5 \times 10^{-4}}\]
\[f_c = \frac{1}{9.42 \times 10^{-4}} = 1061.57\,Hz \approx 1.06\,kHz\]

\paragraph{પગલું 2: Cutoff પર Output Voltage}

Cutoff frequency પર, output voltage એ input voltage ના \textbf{0.707 ગણા} (અથવા $\frac{1}{\sqrt{2}}$) હોય છે:
\[V_{out} = 0.707 \times V_{in} = 0.707 \times 10 = 7.07\,V\]

\paragraph{પરિણામો:}
\begin{itemize}
    \item \textbf{Cutoff Frequency}: $f_c = 1.06\,kHz$
    \item \textbf{Output Voltage}: $V_{out} = 7.07\,V$ cutoff પર
    \item \textbf{Attenuation}: $-3\,dB$ cutoff frequency પર
    \item \textbf{Phase Shift}: $-45^\circ$ cutoff frequency પર
\end{itemize}

\begin{quote}
\textbf{મેમરી ટ્રીક:} \emph{RC-Formula: fc = 1/(2$\pi$ RC), Vout = 0.707 Vin at fc}
\end{quote}

\subsection{Question 1(ક) [7 marks]}
\textbf{Active અને passive electronic components ની યોગ્ય ઉદાહરણો સાથે તુલના કરો.}

\subsubsection{Solution}

Electronic components ને \textbf{active} અને \textbf{passive} કેટેગરીમાં વર્ગીકૃત કરવામાં આવે છે જે તેમની electrical energy ને control અથવા amplify કરવાની ક્ષમતા પર આધારિત છે.

\begin{table}[h]
\centering
\caption{Active vs Passive Components Comparison}
\begin{tabularx}{\textwidth}{lXX}
\toprule
\textbf{લાક્ષણિકતા} & \textbf{Active Components} & \textbf{Passive Components} \\
\midrule
Energy Source & બાહ્ય power source જરૂરી & બાહ્ય power જરૂરી નથી \\
Control Ability & Current flow ને control/amplify કરી શકે & Amplify નહીં, ફક્ત regulate કરે \\
Directionality & સામાન્ય રીતે unidirectional & Bidirectional \\
Power Gain & Power gain આપે ($>1$) & Power gain હંમેશા $\leq 1$ \\
ઉદાહરણો & Transistors (BJT, FET), Diodes (LED, Zener), ICs (Op-Amp, 555), SCR & Resistors, Capacitors, Inductors, Transformers \\
કાર્ય & Amplification, switching, oscillation, rectification & Resistance, capacitance, inductance, filtering \\
Linearity & Linear અથવા non-linear હોઈ શકે & સામાન્ય રીતે linear \\
\bottomrule
\end{tabularx}
\end{table}

\paragraph{Active Components વિગતવાર:}
\begin{itemize}
    \item \textbf{Transistors}: Amplification અને switching માટે વપરાય છે. BJT current control વાપરે, FET voltage control વાપરે છે.
    \item \textbf{Diodes}: એક દિશામાં current ને પસાર થવા દે છે. LED પ્રકાશ બહાર કાઢે, Zener voltage regulate કરે છે.
    \item \textbf{ICs}: Integrated circuits જેવા કે 555 timer (oscillator), op-amps (amplifier).
    \item \textbf{Power Requirement}: બધા active components ને ચાલુ થવા માટે DC bias/supply જરૂરી.
\end{itemize}

\paragraph{Passive Components વિગતવાર:}
\begin{itemize}
    \item \textbf{Resistors}: Current flow નો વિરોધ કરે, power ને heat તરીકે dissipate કરે. મૂલ્ય $\Omega$ માં.
    \item \textbf{Capacitors}: Electric field માં energy સંગ્રહ કરે. મૂલ્ય Farads (F) માં, DC block કરે, AC પસાર કરે.
    \item \textbf{Inductors}: Magnetic field માં energy સંગ્રહ કરે. મૂલ્ય Henry (H) માં, AC ફેરફારોનો વિરોધ કરે.
    \item \textbf{Transformers}: Magnetic coupling દ્વારા circuits વચ્ચે energy transfer કરે.
\end{itemize}

\paragraph{મુખ્ય તફાવત:}
મૂળભૂત તફાવત એ છે કે active components circuit માં ``power inject'' કરી શકે (amplification), જ્યારે passive components ફક્ત energy ``absorb અથવા store'' કરી શકે, તેને ક્યારેય વધારી શકતા નથી.

\begin{quote}
\textbf{મેમરી ટ્રીક:} \emph{ACTIVE = Amplify, Control, Transform; PASSIVE = Resist, Store, Filter}
\end{quote}

\subsection{Question 1(ક OR) [7 marks]}
\textbf{Half-wave rectifier circuit ને input અને output waveforms સાથે દોરો અને તેનું કાર્ય સમજાવો.}

\subsubsection{Solution}

\textbf{Half-wave rectifier} AC voltage ને pulsating DC માં રૂપાંતરિત કરે છે input AC waveform ના ફક્ત એક half-cycle (positive અથવા negative) ને પસાર થવા દઈને.

\paragraph{Circuit વર્ણન:}
Circuit માં સમાવેશ થાય છે:
\begin{itemize}
    \item AC voltage source ($V_{in}$)
    \item Diode (D)
    \item Load resistor ($R_L$)
\end{itemize}

\paragraph{કાર્ય સિદ્ધાંત:}
\begin{enumerate}
    \item \textbf{Positive Half-Cycle}: જ્યારે input AC positive હોય, diode forward-biased (conduct) થાય છે. Current load resistor $R_L$ માંથી વહે છે, output voltage ઉત્પન્ન કરે છે.
    \item \textbf{Negative Half-Cycle}: જ્યારે input AC negative હોય, diode reverse-biased (block) થાય છે. કોઈ current વહેતું નથી, output voltage શૂન્ય હોય છે.
    \item \textbf{પરિણામ}: Output પર ફક્ત positive half-cycles દેખાય છે, pulsating DC બનાવે છે.
\end{enumerate}

\paragraph{મુખ્ય પરિમાણો:}
\begin{itemize}
    \item \textbf{Efficiency}: $\eta = 40.6\%$ (સૈદ્ધાંતિક મહત્તમ)
    \item \textbf{Ripple Factor}: $r = 1.21$ (ઉચ્ચ ripple content)
    \item \textbf{Peak Inverse Voltage (PIV)}: $PIV = V_m$ (diode પર મહત્તમ reverse voltage)
    \item \textbf{DC Output}: $V_{DC} = \frac{V_m}{\pi} = 0.318 V_m$ જ્યાં $V_m$ એ peak AC voltage છે
\end{itemize}

\paragraph{એપ્લિકેશન્સ:}
Half-wave rectifiers નો ઉપયોગ low-power applications માં થાય છે જેવા કે battery charging, signal demodulation, અને voltage multipliers. તેઓ poor efficiency ને કારણે high-power applications માટે \textit{યોગ્ય નથી}.

\begin{quote}
\textbf{મેમરી ટ્રીક:} \emph{HWR: Half-Wave = અડધું output, 40.6\% efficiency, PIV = Vm}
\end{quote}

\end{document}
