%% METADATA
%% subject-code: SUBJECT001
%% subject-name: Subject Name
%% semester: 1
%% examination: Sample-2025
%% date: 01-01-2025
%% language: gujarati
%% description: Sample solution guide demonstrating LaTeX conventions for GTU exam solutions in Gujarati
%% summary: Complete Gujarati example showing proper structure, formatting, and content organization
%% tags: study-material, solutions, sample, gtu, subject001, gujarati
%% END METADATA

\documentclass{article}
% GTU Solutions - Gujarati Preamble
% Includes common preamble + Gujarati font setup

% Basic setup
\usepackage[margin=1in]{geometry}
\author{Milav Dabgar}

% Math and tables
\usepackage{amsmath,amssymb,amsthm}
\usepackage{booktabs}
\usepackage{tabularx}
\usepackage{graphicx}
\usepackage{float}  % Required for [H] float placement

% Code listings with syntax highlighting
\usepackage{xcolor}
\usepackage{listings}
\lstset{
  basicstyle=\small\ttfamily,
  breaklines=true,
  numbers=left,
  numberstyle=\tiny\color{gray},
  xleftmargin=2em,
  frame=single,
  showstringspaces=false,
  tabsize=2,
  keywordstyle=\color{blue},
  commentstyle=\color{green!60!black},
  stringstyle=\color{purple}
}

% Optional: TikZ for diagrams (remove if not needed)
\usepackage{tikz}
\usepackage{circuitikz}
\usetikzlibrary{shapes,arrows,positioning,calc}

% Header/footer with author and website
\usepackage{fancyhdr}
\usepackage{lastpage}

\pagestyle{fancy}
\fancyhf{}
\fancyhead[L]{\small\itshape\leftmark}
\fancyhead[R]{\small Milav Dabgar}
\fancyfoot[L]{\small\href{https://www.milav.in}{www.milav.in}}
\fancyfoot[R]{\small Page \thepage\ of \pageref{LastPage}}
\renewcommand{\headrulewidth}{0.4pt}
\renewcommand{\footrulewidth}{0.4pt}

% Hyperref (load before fontspec for Gujarati)
\usepackage[
  colorlinks=true,
  linkcolor=blue,
  urlcolor=blue,
  citecolor=blue,
  pdfauthor={Milav Dabgar},
  pdfsubject={GTU Exam Solutions},
  pdfkeywords={GTU, Java, Programming, Solutions, Gujarati},
  bookmarks=true
]{hyperref}

% Gujarati font setup
\usepackage{fontspec}
\usepackage{polyglossia}
\setdefaultlanguage{gujarati}
\setotherlanguage{english}
\newfontfamily\gujaratifont[Script=Gujarati,AutoFakeBold=2.5,AutoFakeSlant=0.3]{Noto Sans Gujarati}
\setmainfont[Script=Gujarati,AutoFakeBold=2.5,AutoFakeSlant=0.3]{Noto Sans Gujarati}
\setmonofont[Scale=0.9]{Noto Sans Gujarati}
\newfontfamily\englishfont[Script=Gujarati,AutoFakeBold=2.5,AutoFakeSlant=0.3]{Noto Sans Gujarati}
\gappto\captionsgujarati{
  \renewcommand{\tablename}{કોષ્ટક}
  \renewcommand{\figurename}{આકૃતિ}
}
\newcommand{\gu}[1]{{\gujaratifont #1}}



\title{Subject Name (SUBJECT001) - Sample Term Solution}
\date{મહિના દિવસ, વર્ષ}

% PDF Metadata
\hypersetup{
  pdftitle={Subject Name (SUBJECT001) - Sample Term Solution (Gujarati)},
  pdfsubject={GTU Exam Solution - Sample-2025},
  pdfauthor={Milav Dabgar},
  pdfkeywords={study-material, solutions, sample, gtu, subject001, gujarati},
  pdfcreator={XeLaTeX}
}

\begin{document}

\maketitle

\setcounter{tocdepth}{5}
\tableofcontents
\newpage

% ========================================
% પ્રશ્ન 1: સંક્ષિપ્ત પ્રશ્નો (14 ગુણ)
% ========================================

\section{પ્રશ્ન 1}

\subsection{પ્રશ્ન 1(a) [3 ગુણ]}
\textbf{ત્રણ નંબરોમાંથી મેક્સિમમ શોધવા માટે જાવા પ્રોગ્રામ લખો.}

\subsubsection{ઉકેલ}
ત્રણ નંબરોમાંથી \textbf{મેક્સિમમ} શોધવા માટે, અમે વેલ્યુઝની સરખામણી કરવા માટે \textbf{કન્ડિશનલ સ્ટેટમેન્ટ્સ} (if-else) નો ઉપયોગ કરીએ છીએ. પ્રોગ્રામ ત્રણ નંબરો ઇનપુટ તરીકે લે છે અને તેમાંથી ``\emph{સૌથી મોટી વેલ્યુ}'' પરત કરે છે.

\begin{lstlisting}[language=Java, caption={ત્રણ નંબરોમાંથી મેક્સિમમ શોધવા માટે જાવા પ્રોગ્રામ}]
public class MaxOfThree {
    public static void main(String[] args) {
        int a = 10, b = 25, c = 15;
        int max;
        
        // Compare a and b, store the larger in max
        if (a > b) {
            max = a;
        } else {
            max = b;
        }
        
        // Compare max with c to get the final maximum
        if (c > max) {
            max = c;
        }
        
        System.out.println("Maximum value: " + max);
    }
}
\end{lstlisting}

\paragraph{આઉટપુટ:}
\begin{verbatim}
Maximum value: 25
\end{verbatim}

\paragraph{મુખ્ય મુદ્દાઓ:}
\begin{description}
    \item[લોજિક:] પ્રથમ \texttt{a} અને \texttt{b} ની સરખામણી કરો, મોટી વેલ્યુ ને \texttt{max} માં સ્ટોર કરો
    \item[બીજી સરખામણી:] અંતિમ મેક્સિમમ મેળવવા માટે \texttt{max} ને \texttt{c} સાથે સરખાવો
    \item[વૈકલ્પિક:] સંક્ષિપ્ત કોડ માટે \texttt{Math.max(a, Math.max(b, c))} નો ઉપયોગ કરી શકાય
\end{description}

\paragraph{મેમરી ટ્રીક:}
\emph{``કંપેર ટુ-એન્ડ-ટુ, સ્ટોર ધ બેસ્ટ, ફાઇનલ ચેક વિથ ધ રેસ્ટ!''}

% ========================================
% પ્રશ્ન 1(b): આરસી ફિલ્ટર ડિઝાઇન (4 ગુણ)
% ========================================

\subsection{પ્રશ્ન 1(b) [4 ગુણ]}
\textbf{આરસી લો-પાસ ફિલ્ટર નું કટઓફ ફ્રિક્વન્સી શોધો જ્યાં \(R = 1.5\,k\Omega\) અને \(C = 100\,nF\) છે. તેમજ કટઓફ ફ્રિક્વન્સી પર જો ઇનપુટ 10V હોય તો આઉટપુટ વોલ્ટેજ શોધો.}

\subsubsection{ઉકેલ}

\paragraph{આપેલ:}
\begin{itemize}
    \item રેઝિસ્ટન્સ: \(R = 1.5\,k\Omega = 1500\,\Omega\)
    \item કેપેસિટન્સ: \(C = 100\,nF = 100 \times 10^{-9}\,F\)
    \item ઇનપુટ વોલ્ટેજ: \(V_{in} = 10\,V\)
\end{itemize}

\paragraph{પગલું 1: કટઓફ ફ્રિક્વન્સી ની ગણતરી}
આરસી લો-પાસ ફિલ્ટર માટે \textbf{કટઓફ ફ્રિક્વન્સી} નો ફોર્મ્યુલા છે:

\[
f_c = \frac{1}{2\pi RC}
\]

વેલ્યુઝ મૂકીએ:

\[
f_c = \frac{1}{2 \times 3.1416 \times 1500 \times 100 \times 10^{-9}}
\]

\[
f_c = \frac{1}{9.4248 \times 10^{-4}} = 1061.03\,Hz \approx 1.06\,kHz
\]

\paragraph{પગલું 2: કટઓફ પર આઉટપુટ વોલ્ટેજ}
કટઓફ ફ્રિક્વન્સી પર, આઉટપુટ વોલ્ટેજ એ ઇનપુટ વોલ્ટેજ ના \textbf{0.707 ગણા} (અથવા \(\frac{1}{\sqrt{2}}\)) હોય છે:

\[
V_{out} = 0.707 \times V_{in} = 0.707 \times 10 = 7.07\,V
\]

\paragraph{પરિણામો:}
\begin{description}
    \item[કટઓફ ફ્રિક્વન્સી:] \(f_c = 1.06\,kHz\)
    \item[આઉટપુટ વોલ્ટેજ:] \(V_{out} = 7.07\,V\) કટઓફ પર
    \item[એટેન્યુએશન:] \(-3\,dB\) કટઓફ ફ્રિક્વન્સી પર
    \item[ફેઝ શિફ્ટ:] \(-45^\circ\) કટઓફ ફ્રિક્વન્સી પર
\end{description}

\subparagraph{ફિલ્ટર વર્તન:}
કટઓફ નીચે, સિગ્નલ ઓછા એટેન્યુએશન સાથે પાસ થાય છે. કટઓફ ઉપર, એટેન્યુએશન \(-20\,dB/decade\) રોલ-ઓફ રેટ પર વધે છે.

\paragraph{મેમરી ટ્રીક:}
\emph{``\textbf{ફ્રિક્વન્સી = વન બાય ટુ-પાય-આરસી}, \textbf{આઉટપુટ = પોઇન્ટ-સેવન-ઓ-સેવન ટાઇમ્સ ઇનપુટ}''}

% ========================================
% પ્રશ્ન 1(c): એક્ટિવ અને પેસિવ કોમ્પોનન્ટ્સ (7 ગુણ)
% ========================================

\subsection{પ્રશ્ન 1(c) [7 ગુણ]}
\textbf{એક્ટિવ અને પેસિવ ઇલેક્ટ્રોનિક કોમ્પોનન્ટ્સ ની યોગ્ય ઉદાહરણો સાથે તુલના કરો.}

\subsubsection{ઉકેલ}

ઇલેક્ટ્રોનિક કોમ્પોનન્ટ્સ ને \textbf{એક્ટિવ} અને \textbf{પેસિવ} કેટેગરીમાં વર્ગીકૃત કરવામાં આવે છે જે તેમની ઇલેક્ટ્રિકલ એનર્જી ને કંટ્રોલ અથવા એમ્પ્લિફાય કરવાની ક્ષમતા પર આધારિત છે.

\begin{table}[H]
\centering
\caption{એક્ટિવ બનામ પેસિવ કોમ્પોનન્ટ્સ તુલના}
\begin{tabularx}{\textwidth}{lXX}
\toprule
\textbf{લક્ષણ} & \textbf{એક્ટિવ કોમ્પોનન્ટ્સ} & \textbf{પેસિવ કોમ્પોનન્ટ્સ} \\
\midrule
એનર્જી સ્રોત & બાહ્ય પાવર સ્રોત જરૂરી & બાહ્ય પાવર જરૂરી નથી \\
કંટ્રોલ ક્ષમતા & કરંટ ફ્લો ને કંટ્રોલ/એમ્પ્લિફાય કરી શકે & એમ્પ્લિફાય કરી શકતા નથી, ફક્ત રેગ્યુલેટ કરે \\
દિશા & સામાન્ય રીતે યુનિડાયરેક્શનલ & બાયડાયરેક્શનલ \\
પાવર ગેઇન & પાવર ગેઇન પ્રદાન કરે (\(>1\)) & પાવર ગેઇન હંમેશા \(\leq 1\) \\
ઉદાહરણો & ટ્રાન્ઝિસ્ટર્સ (BJT, FET), ડાયોડ્સ (LED, ઝેનર), ICs (Op-Amp, 555), SCR & રેઝિસ્ટર્સ, કેપેસિટર્સ, ઇન્ડક્ટર્સ, ટ્રાન્સફોર્મર્સ \\
ફંક્શન & એમ્પ્લિફિકેશન, સ્વિચિંગ, ઓસિલેશન, રેક્ટિફિકેશન & રેઝિસ્ટન્સ, કેપેસિટન્સ, ઇન્ડક્ટન્સ, ફિલ્ટરિંગ \\
લિનિઅરિટી & લિનિઅર અથવા નોન-લિનિઅર હોઈ શકે & સામાન્ય રીતે લિનિઅર \\
\bottomrule
\end{tabularx}
\end{table}

\paragraph{એક્ટિવ કોમ્પોનન્ટ્સ વિગતવાર:}
\begin{description}
    \item[ટ્રાન્ઝિસ્ટર્સ:] એમ્પ્લિફિકેશન અને સ્વિચિંગ માટે વપરાય છે. BJT કરંટ કંટ્રોલ વાપરે છે, FET વોલ્ટેજ કંટ્રોલ વાપરે છે.
    \item[ડાયોડ્સ:] એક દિશામાં કરંટ વહેવા દે છે. LED પ્રકાશ ઉત્સર્જન કરે છે, ઝેનર વોલ્ટેજ રેગ્યુલેટ કરે છે.
    \item[ICs:] ઇન્ટીગ્રેટેડ સર્કિટ્સ જેવા કે \texttt{555 timer} (ઓસિલેટર), op-amps (એમ્પ્લિફાયર).
    \item[પાવર જરૂરિયાત:] બધા એક્ટિવ કોમ્પોનન્ટ્સને કાર્ય કરવા માટે DC બાયસ/સપ્લાય જોઈએ છે.
\end{description}

\subparagraph{ટ્રાન્ઝિસ્ટર પ્રકારો:}
BJT (Bipolar Junction Transistor) માં NPN અને PNP વેરિઅન્ટ્સ છે. FET (Field Effect Transistor) માં JFET અને MOSFET પ્રકારો સામેલ છે.

\paragraph{પેસિવ કોમ્પોનન્ટ્સ વિગતવાર:}
\begin{description}
    \item[રેઝિસ્ટર્સ:] કરંટ ફ્લો નો વિરોધ કરે છે, પાવર ને હીટ તરીકે વિતરિત કરે છે. વેલ્યુ \(\Omega\) માં.
    \item[કેપેસિટર્સ:] ઇલેક્ટ્રિક ફીલ્ડમાં એનર્જી સ્ટોર કરે છે. વેલ્યુ ફેરાડ્સ (F) માં, DC ને બ્લોક કરે, AC ને પાસ કરે.
    \item[ઇન્ડક્ટર્સ:] મેગ્નેટિક ફીલ્ડમાં એનર્જી સ્ટોર કરે છે. વેલ્યુ હેન્રી (H) માં, AC ચેંજીસનો વિરોધ કરે.
    \item[ટ્રાન્સફોર્મર્સ:] મેગ્નેટિક કપલિંગ દ્વારા સર્કિટ્સ વચ્ચે એનર્જી ટ્રાન્સફર કરે છે.
\end{description}

\subparagraph{રેઝિસ્ટર પ્રકારો:}
ફિક્સ્ડ રેઝિસ્ટર્સમાં carbon composition, metal film, અને wire-wound પ્રકારો સામેલ છે. variable રેઝિસ્ટર્સ potentiometers અને rheostats છે.

\subparagraph{કેપેસિટર પ્રકારો:}
કેપેસિટર્સમાં ઇલેક્ટ્રોલિટિક (પોલરાઇઝ્ડ, હાય કેપેસિટન્સ), સિરામિક (નાના, સ્ટેબલ), અને ફિલ્મ (પ્રિસિઝન) પ્રકારો સામેલ છે.

\paragraph{મુખ્ય તફાવત:}
મૂળભૂત તફાવત એ છે કે એક્ટિવ કોમ્પોનન્ટ્સ સર્કિટમાં \emph{પાવર ઇન્જેક્ટ} કરી શકે (એમ્પ્લિફિકેશન), જ્યારે પેસિવ કોમ્પોનન્ટ્સ ફક્ત એનર્જી \emph{શોષી અથવા સ્ટોર} કરી શકે, તેને ક્યારેય વધારી શકતા નથી.

\paragraph{મેમરી ટ્રીક:}
\emph{ACTIVE = Amplify, Control, Transform; PASSIVE = Resist, Store, Filter}

% ========================================
% પ્રશ્ન 1(c) OR: વૈકલ્પિક પ્રશ્ન (7 ગુણ)
% ========================================

\subsection{પ્રશ્ન 1(c) OR [7 ગુણ]}
\textbf{હાફ-વેવ રેક્ટિફાયર સર્કિટની કાર્યપદ્ધતિ ઇનપુટ અને આઉટપુટ વેવફોર્મ્સ સાથે દોરો અને સમજાવો.}

\subsubsection{ઉકેલ}

\textbf{હાફ-વેવ રેક્ટિફાયર} AC વોલ્ટેજને પલ્સેટિંગ DC માં કન્વર્ટ કરે છે જે ફક્ત ઇનપુટ AC વેવફોર્મના એક હાફ-સાઇકલ (પોઝિટિવ અથવા નેગેટિવ) ને પાસ થવા દે છે.

\paragraph{સર્કિટ ડાયાગ્રામ:}
\begin{figure}[H]
\centering
\begin{circuitikz}[scale=1.2]
    % AC સ્રોત
    \draw (0,0) to[sV, l=\(V_{in}\)] (0,2);
    \draw (0,2) to[short] (2,2);
    
    % ડાયોડ
    \draw (2,2) to[D*, l=\(D\)] (4,2);
    
    % લોડ રેઝિસ્ટર
    \draw (4,2) to[short] (5,2);
    \draw (5,2) to[R, l=\(R_L\)] (5,0);
    \draw (5,0) to[short] (0,0);
    
    % આઉટપુટ વોલ્ટેજ માપન
    \draw (4.5,2) to[short, *-] (4.5,2.5);
    \node at (4.5,2.7) {\(V_{out}\)};
    \draw (4.5,0) to[short, *-] (4.5,-0.5);
    \node[ground] at (4.5,-0.5) {};
\end{circuitikz}
\caption{હાફ-વેવ રેક્ટિફાયર સર્કિટ}
\end{figure}

\paragraph{કાર્યપદ્ધતિ:}
\begin{description}
    \item[પોઝિટિવ હાફ-સાઇકલ:] જ્યારે ઇનપુટ AC પોઝિટિવ હોય, ડાયોડ ફોરવર્ડ-બાયસ્ડ (સંચાલન) થાય છે. કરંટ લોડ રેઝિસ્ટર \(R_L\) માંથી વહે છે અને આઉટપુટ વોલ્ટેજ ઉત્પન્ન કરે છે.
    \item[નેગેટિવ હાફ-સાઇકલ:] જ્યારે ઇનપુટ AC નેગેટિવ હોય, ડાયોડ રિવર્સ-બાયસ્ડ (બ્લોક) થાય છે. કોઈ કરંટ વહેતો નથી, આઉટપુટ વોલ્ટેજ શૂન્ય છે.
    \item[પરિણામ:] ફક્ત પોઝિટિવ હાફ-સાઇકલ્સ આઉટપુટ પર દેખાય છે અને પલ્સેટિંગ DC બનાવે છે.
\end{description}

\paragraph{વેવફોર્મ રેપ્રેઝન્ટેશન:}
\begin{figure}[H]
\centering
\begin{tikzpicture}[scale=0.9]
    % ઇનપુટ વેવફોર્મ
    \draw[->] (0,0) -- (6.5,0) node[right] {\(t\)};
    \draw[->] (0,-1.5) -- (0,1.8) node[above] {\(V_{in}\)};
    \draw[thick, blue] (0,0) sin (1,1) cos (2,0) sin (3,-1) cos (4,0) sin (5,1) cos (6,0);
    \node at (3,1.5) {ઇનપુટ AC};
    
    % આઉટપુટ વેવફોર્મ
    \begin{scope}[yshift=-3.5cm]
        \draw[->] (0,0) -- (6.5,0) node[right] {\(t\)};
        \draw[->] (0,-0.3) -- (0,1.8) node[above] {\(V_{out}\)};
        \draw[thick, red] (0,0) sin (1,1) cos (2,0);
        \draw[thick, red] (2,0) -- (4,0);
        \draw[thick, red] (4,0) sin (5,1) cos (6,0);
        \node at (3,1.5) {આઉટપુટ પલ્સેટિંગ DC};
    \end{scope}
\end{tikzpicture}
\caption{ઇનપુટ અને આઉટપુટ વેવફોર્મ્સ}
\end{figure}

\paragraph{મુખ્ય પેરામીટર્સ:}
\begin{description}
    \item[કાર્યક્ષમતા:] \(\eta = 40.6\%\) (થિયરેટિકલ મેક્સિમમ)
    \item[રિપલ ફેક્ટર:] \(r = 1.21\) (હાય રિપલ કન્ટેન્ટ)
    \item[પીક ઇનવર્સ વોલ્ટેજ (PIV):] \(PIV = V_m\) (ડાયોડ પર મેક્સિમમ રિવર્સ વોલ્ટેજ)
    \item[DC આઉટપુટ:] \(V_{DC} = \frac{V_m}{\pi} = 0.318 V_m\) જ્યાં \(V_m\) એ પીક AC વોલ્ટેજ છે
\end{description}

\subparagraph{કાર્યક્ષમતા ડેરિવેશન:}
કાર્યક્ષમતા \(\eta = \frac{P_{DC}}{P_{AC}} = \frac{(V_{DC})^2/R_L}{(V_{rms})^2/R_L} = \frac{(V_m/\pi)^2}{(V_m/2)^2} = \frac{4}{\pi^2} = 0.406 = 40.6\%\)

\paragraph{એપ્લિકેશન્સ:}
હાફ-વેવ રેક્ટિફાયર્સ લો-પાવર એપ્લિકેશન્સમાં વપરાય છે જેવા કે બેટરી ચાર્જિંગ, સિગ્નલ ડિમોડ્યુલેશન, અને વોલ્ટેજ મલ્ટિપ્લાયર્સ. તેઓ હાય-પાવર એપ્લિકેશન્સ માટે \emph{યોગ્ય નથી} કારણ કે ઓછી કાર્યક્ષમતા.

\paragraph{મેમરી ટ્રીક:}
\emph{HWR: Half-Wave = Half output, 40.6\% efficiency, PIV = Vm}

% ========================================

\end{document}
