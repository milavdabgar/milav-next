%% METADATA
%% subject-code: SUBJECT001
%% subject-name: Subject Name
%% semester: 1
%% examination: Sample-2025
%% date: 01-01-2025
%% language: gujarati
%% description: Sample solution guide demonstrating LaTeX conventions for GTU exam solutions in Gujarati
%% summary: Complete Gujarati example showing proper structure, formatting, and content organization
%% tags: study-material, solutions, sample, gtu, subject001, gujarati
%% END METADATA

\documentclass{article}

% content/resources/templates/preamble.tex
\usepackage[margin=0.6in]{geometry}
\author{Milav Dabgar}
\usepackage{amsmath,amssymb,amsthm}
\usepackage{booktabs}
\usepackage{multirow}
\usepackage{xcolor}
\usepackage{tcolorbox}
\tcbuselibrary{breakable,skins}
\usepackage[colorlinks=true,linkcolor=blue]{hyperref}
\usepackage{titlesec}
\usepackage{enumitem}
\usepackage{tikz}
\usepackage{pgfplots}
\usepackage{circuitikz}
\usepackage[version=4]{mhchem}
\usepackage{longtable}
\usepackage{array}
\usepackage{float}
\usepackage{caption}
\usepackage{listings}

\lstset{
  basicstyle=\small\ttfamily,
  breaklines=true,
  breakatwhitespace=false,
  postbreak=\mbox{\textcolor{red}{$\hookrightarrow$}\space},
  float=false,
  numbers=left,
  numberstyle=\tiny\color{gray},
  numbersep=10pt,
  xleftmargin=2em,
  keywordstyle=\color{blue},
  commentstyle=\color{green!60!black},
  stringstyle=\color{purple},
  backgroundcolor=\color{gray!5},
  showstringspaces=false,
  tabsize=2,
  captionpos=b,
  keepspaces=true,
  columns=flexible
}

\pgfplotsset{compat=1.18}
\usetikzlibrary{shapes,arrows,positioning,calc,patterns,decorations.pathmorphing,decorations.markings,arrows.meta}

% Color scheme
\definecolor{headcolor}{RGB}{0,102,204}
\definecolor{keycolor}{RGB}{220,20,60}
\definecolor{solutioncolor}{RGB}{34,139,34}
\definecolor{mnemoniccolor}{RGB}{148,0,211}
\definecolor{codecolor}{RGB}{0,0,100}

% Spacing
\setlength{\parskip}{3pt}
\setlist[itemize]{nosep}
\setlist[enumerate]{nosep}

% Title formatting
\titleformat{\section}{\Large\bfseries\color{headcolor}}{\thesection}{1em}{}
\titleformat{\subsection}{\large\bfseries\color{headcolor}}{\thesubsection}{1em}{}

% Pandoc tightlist compatibility
\providecommand{\tightlist}{%
  \setlength{\itemsep}{0pt}\setlength{\parskip}{0pt}}

% Pandoc longtable compatibility
\newcounter{none}
\def\thenone{}


% content/resources/templates/gujarati-boxes.tex
\usepackage{fontspec}
\usepackage{polyglossia}

% Set Gujarati as main language (document is primarily in Gujarati)
% Note: gloss-gujarati.ldf doesn't exist in polyglossia, but it will use hyphenation patterns
\setdefaultlanguage{gujarati}
\setotherlanguage{english}

% Configure Gujarati font properly
% Use Language=Default to prevent polyglossia from trying to add language-specific features
% that don't exist for Gujarati, which causes "empty feature" warnings
\newfontfamily\gujaratifont[Script=Gujarati,AutoFakeBold=2.5,AutoFakeSlant=0.3]{Noto Sans Gujarati}
\setmainfont[Script=Gujarati,AutoFakeBold=2.5,AutoFakeSlant=0.3]{Noto Sans Gujarati}
% Use Noto Sans Gujarati for monospace to support Gujarati in text
\setmonofont[Scale=0.9]{Noto Sans Gujarati}

% Configure English to use the same font
\newfontfamily\englishfont[Script=Gujarati,AutoFakeBold=2.5,AutoFakeSlant=0.3]{Noto Sans Gujarati}

% Translations for polyglossia
\gappto\captionsgujarati{
  \renewcommand{\tablename}{કોષ્ટક}
  \renewcommand{\figurename}{આકૃતિ}
}

% Helper for TikZ nodes to ensure Gujarati font
\newcommand{\gu}[1]{{\gujaratifont #1}}

% Custom environments
\newtcolorbox{solutionbox}{
    breakable,
    enhanced,
    colback=solutioncolor!5!white,
    colframe=solutioncolor!75!black,
    fonttitle=\bfseries,
    title=જવાબ
}

\newtcolorbox{solutionboxnobreak}{
 colback=solutioncolor!5!white,
 colframe=solutioncolor!75!black,
 fonttitle=\bfseries,
 title=જવાબ
}

\newtcolorbox{keyformula}{
 breakable,
 enhanced,
 colback=keycolor!5!white,
 colframe=keycolor!75!black,
 fonttitle=\bfseries,
 title=રાસાયણિક સમીકરણ/સૂત્ર
}

\newtcolorbox{mnemonicbox}{
 breakable,
 enhanced,
 colback=mnemoniccolor!5!white,
 colframe=mnemoniccolor!75!black,
 fonttitle=\bfseries,
 title=મેમરી ટ્રીક
}


% Custom commands for GTU solutions
% This file defines semantic commands for consistent formatting

% Question command with automatic formatting
\newcommand{\question}[2]{%
  \section*{Question #1}%
  \textbf{#2}%
}

% OR question variant
\newcommand{\questionor}[2]{%
  \section*{Question #1 OR}%
  \textbf{#2}%
}

% Proper table environment with caption
\newenvironment{answertable}[1]{%
  \begin{table}[htbp]
  \centering
  \caption{#1}
}{%
  \end{table}
}

% Proper figure environment for diagrams
\newenvironment{answerdiagram}[1]{%
  \begin{figure}[htbp]
  \centering
  \caption{#1}
}{%
  \end{figure}
}

% Semantic markup for key terms
\newcommand{\keyword}[1]{\textbf{#1}}
\newcommand{\code}[1]{\texttt{#1}}
\newcommand{\classname}[1]{\texttt{#1}}
\newcommand{\methodname}[1]{\texttt{#1}}

% Proper quotation marks
\newcommand{\mnemonic}[1]{``#1''}


\title{Subject Name (SUBJECT001) - Sample Term Solution}
\date{મહિના દિવસ, વર્ષ}

% PDF Metadata
\hypersetup{
  pdftitle={Subject Name (SUBJECT001) - Sample Term Solution (Gujarati)},
  pdfsubject={GTU Exam Solution - Sample-2025},
  pdfauthor={Milav Dabgar},
  pdfkeywords={study-material, solutions, sample, gtu, subject001, gujarati},
  pdfcreator={XeLaTeX},
  pdflang={gu}
}

\begin{document}
\maketitle

% ========================================
% QUESTION 1(a): Programming Code (3 marks)
% Demonstrates: lstlisting for code, straight quotes in code, ~100 words
% ========================================

\questionmarks{1(અ)}{3}{ત્રણ numbers માંથી maximum શોધવા માટે Java program લખો.}

\begin{solutionbox}
ત્રણ numbers માંથી \keyword{maximum} શોધવા માટે, અમે values ની સરખામણી કરવા માટે \keyword{conditional statements} (if-else) નો ઉપયોગ કરીએ છીએ. Program ત્રણ numbers input તરીકે લે છે અને તેમાંથી ``સૌથી મોટી value'' પરત કરે છે.

\textbf{Java Program:}
\begin{lstlisting}[language=Java,caption={ત્રણ Numbers માંથી Maximum શોધો}]
public class MaxOfThree {
    public static void main(String[] args) {
        int a = 25, b = 40, c = 15;
        int max;
        
        // Compare first two numbers
        if (a > b) {
            max = a;
        } else {
            max = b;
        }
        
        // Compare result with third number
        if (c > max) {
            max = c;
        }
        
        System.out.println("Maximum number is: " + max);
    }
}
\end{lstlisting}

\textbf{Output:}
\begin{verbatim}
Maximum number is: 40
\end{verbatim}

\textbf{મુખ્ય મુદ્દાઓ:}
\begin{itemize}
    \item \keyword{Logic}: પ્રથમ \code{a} અને \code{b} ની સરખામણી કરો, મોટી value ને \code{max} માં store કરો
    \item \keyword{બીજી સરખામણી}: અંતિમ maximum મેળવવા માટે \code{max} ને \code{c} સાથે સરખાવો
    \item \keyword{વૈકલ્પિક}: Concise code માટે \code{Math.max(a, Math.max(b, c))} નો ઉપયોગ કરી શકાય
\end{itemize}
\end{solutionbox}

\begin{mnemonicbox}
\mnemonic{MAX: જોડીમાં સરખાવો, પરીક્ષણે મહત્તમ અપડેટ કરો}
\end{mnemonicbox}

% ========================================
% QUESTION 1(b): Mathematics Calculation (4 marks)
% Demonstrates: Math equations, step-by-step calculation, inline \code{}, ~150 words
% ========================================

\questionmarks{1(બ)}{4}{RC low-pass filter નું cutoff frequency શોધો જ્યાં $R = 1.5\,k\Omega$ અને $C = 100\,nF$ છે. તેમજ cutoff frequency પર જો input 10V હોય તો output voltage શોધો.}

\begin{solutionbox}
\textbf{આપેલ માહિતી:}
\begin{itemize}
    \item Resistance: $R = 1.5\,k\Omega = 1500\,\Omega$
    \item Capacitance: $C = 100\,nF = 100 \times 10^{-9}\,F$
    \item Input Voltage: $V_{in} = 10\,V$
\end{itemize}

\textbf{પગલું 1: Cutoff Frequency ની ગણતરી}
RC low-pass filter માટે \keyword{cutoff frequency} નો formula છે:
\[f_c = \frac{1}{2\pi RC}\]

મૂલ્યો મૂકીએ:
\[f_c = \frac{1}{2\pi \times 1500 \times 100 \times 10^{-9}}\]
\[f_c = \frac{1}{2\pi \times 1.5 \times 10^{-4}}\]
\[f_c = \frac{1}{9.42 \times 10^{-4}} = 1061.57\,Hz \approx 1.06\,kHz\]

\textbf{પગલું 2: Cutoff પર Output Voltage}
Cutoff frequency પર, output voltage એ input voltage કરતાં \keyword{0.707 ગણું} (અથવા $\frac{1}{\sqrt{2}}$) હોય છે:
\[V_{out} = 0.707 \times V_{in} = 0.707 \times 10 = 7.07\,V\]

\textbf{પરિણામો:}
\begin{itemize}
    \item \keyword{Cutoff Frequency}: $f_c = 1.06\,kHz$
    \item \keyword{Output Voltage}: $V_{out} = 7.07\,V$ cutoff પર
    \item \keyword{Attenuation}: $-3\,dB$ cutoff frequency પર
    \item \keyword{Phase Shift}: $-45^\circ$ cutoff frequency પર
\end{itemize}
\end{solutionbox}

\begin{mnemonicbox}
\mnemonic{RC-Formula: fc = 1/(2$\pi$ RC), Vout = 0.707 Vin at fc}
\end{mnemonicbox}

% ========================================
% QUESTION 1(c): Comparison Table (7 marks)
% Demonstrates: Comprehensive comparison, \tabulary{} with caption at TOP, ~250 words
% ========================================

\questionmarks{1(ક)}{7}{Active અને passive electronic components ની યોગ્ય ઉદાહરણો સાથે તુલના કરો.}

\begin{solutionbox}
Electronic components ને \keyword{active} અને \keyword{passive} કેટેગરીમાં વર્ગીકૃત કરવામાં આવે છે જે તેમની electrical energy ને control અથવા amplify કરવાની ક્ષમતા પર આધારિત છે.

\begin{center}
\captionof{table}{Active vs Passive Components Comparison}
\begin{tabulary}{\linewidth}{|L|L|L|}
\hline
\textbf{લાક્ષણિકતા} & \textbf{Active Components} & \textbf{Passive Components} \\ \hline
Energy Source & બાહ્ય power source જરૂરી & બાહ્ય power જરૂરી નથી \\ \hline
Control Ability & Current flow ને control/amplify કરી શકે & Amplify નહીં, ફક્ત regulate કરે \\ \hline
Directionality & સામાન્ય રીતે unidirectional & Bidirectional \\ \hline
Power Gain & Power gain આપે ($>1$) & Power gain હંમેશા $\leq 1$ \\ \hline
ઉદાહરણો & Transistors (BJT, FET), Diodes (LED, Zener), ICs (Op-Amp, 555 Timer), SCR & Resistors, Capacitors, Inductors, Transformers \\ \hline
કાર્ય & Amplification, switching, oscillation, rectification & Resistance, capacitance, inductance, filtering \\ \hline
Linearity & Linear અથવા non-linear હોઈ શકે & સામાન્ય રીતે linear \\ \hline
\end{tabulary}
\end{center}

\textbf{Active Components વિગતવાર:}
\begin{itemize}
    \item \keyword{Transistors}: Amplification અને switching માટે વપરાય છે. BJT current control વાપરે, FET voltage control વાપરે છે.
    \item \keyword{Diodes}: એક દિશામાં current ને પસાર થવા દે છે. LED પ્રકાશ બહાર કાઢે, Zener voltage regulate કરે છે.
    \item \keyword{ICs}: Integrated circuits જેવા કે \code{555 timer} (oscillator), op-amps (amplifier).
    \item \keyword{Power Requirement}: બધા active components ને ચાલુ થવા માટે DC bias/supply જરૂરી.
\end{itemize}

\textbf{Passive Components વિગતવાર:}
\begin{itemize}
    \item \keyword{Resistors}: Current flow નો વિરોધ કરે, power ને heat તરીકે dissipate કરે. મૂલ્ય $\Omega$ માં.
    \item \keyword{Capacitors}: Electric field માં energy સંગ્રહ કરે. મૂલ્ય Farads (F) માં, DC block કરે, AC પસાર કરે.
    \item \keyword{Inductors}: Magnetic field માં energy સંગ્રહ કરે. મૂલ્ય Henry (H) માં, AC ફેરફારોનો વિરોધ કરે.
    \item \keyword{Transformers}: Magnetic coupling દ્વારા circuits વચ્ચે energy transfer કરે.
\end{itemize}

\textbf{મુખ્ય તફાવત:}
મૂળભૂત તફાવત એ છે કે active components circuit માં ``power inject'' કરી શકે (amplification), જ્યારે passive components ફક્ત energy ``absorb અથવા store'' કરી શકે, તેને ક્યારેય વધારી શકતા નથી.
\end{solutionbox}

\begin{mnemonicbox}
\mnemonic{ACTIVE = Amplify, Control, Transform; PASSIVE = Resist, Store, Filter}
\end{mnemonicbox}

% ========================================
% QUESTION 1(c OR): Alternative Question (7 marks)
% Demonstrates: OR question format, TikZ diagram with gtu styles, lstlisting, ~250 words
% ========================================

\questionmarks{1(ક OR)}{7}{Half-wave rectifier circuit ને input અને output waveforms સાથે દોરો અને તેનું કાર્ય સમજાવો.}

\begin{solutionbox}
\keyword{Half-wave rectifier} AC voltage ને pulsating DC માં રૂપાંતરિત કરે છે input AC waveform ના ફક્ત એક half-cycle (positive અથવા negative) ને પસાર થવા દઈને.

\textbf{Circuit Diagram:}
\begin{center}
\begin{circuitikz}[scale=1.2]
    % AC Source
    \draw (0,0) to[sV, l=$V_{in}$] (0,2);
    \draw (0,2) to[short] (2,2);
    
    % Diode
    \draw (2,2) to[D*, l=$D$] (4,2);
    
    % Load Resistor
    \draw (4,2) to[short] (5,2);
    \draw (5,2) to[R, l=$R_L$] (5,0);
    \draw (5,0) to[short] (0,0);
    
    % Output voltage measurement
    \draw (4.5,2) to[short, *-] (4.5,2.5);
    \node at (4.5,2.7) {$V_{out}$};
    \draw (4.5,0) to[short, *-] (4.5,-0.5);
    \node[ground] at (4.5,-0.5) {};
\end{circuitikz}
\captionof{figure}{Half-Wave Rectifier Circuit}
\end{center}

\textbf{કાર્ય સિદ્ધાંત:}
\begin{enumerate}
    \item \keyword{Positive Half-Cycle}: જ્યારે input AC positive હોય, diode forward-biased (conduct) થાય છે. Current load resistor $R_L$ માંથી વહે છે, output voltage ઉત્પન્ન કરે છે.
    \item \keyword{Negative Half-Cycle}: જ્યારે input AC negative હોય, diode reverse-biased (block) થાય છે. કોઈ current વહેતું નથી, output voltage શૂન્ય હોય છે.
    \item \keyword{પરિણામ}: Output પર ફક્ત positive half-cycles દેખાય છે, pulsating DC બનાવે છે.
\end{enumerate}

\textbf{Waveform Representation:}
\begin{center}
\begin{tikzpicture}[scale=0.9]
    % Input waveform
    \draw[->] (0,0) -- (6.5,0) node[right] {$t$};
    \draw[->] (0,-1.5) -- (0,1.8) node[above] {$V_{in}$};
    \draw[thick, blue] (0,0) sin (1,1) cos (2,0) sin (3,-1) cos (4,0) sin (5,1) cos (6,0);
    \node at (3,1.5) {Input AC};
    
    % Output waveform
    \begin{scope}[yshift=-3.5cm]
        \draw[->] (0,0) -- (6.5,0) node[right] {$t$};
        \draw[->] (0,-0.3) -- (0,1.8) node[above] {$V_{out}$};
        \draw[thick, red] (0,0) sin (1,1) cos (2,0);
        \draw[thick, red] (2,0) -- (4,0);
        \draw[thick, red] (4,0) sin (5,1) cos (6,0);
        \node at (3,1.5) {Output Pulsating DC};
    \end{scope}
\end{tikzpicture}
\captionof{figure}{Input અને Output Waveforms}
\end{center}

\textbf{મુખ્ય પરિમાણો:}
\begin{itemize}
    \item \keyword{Efficiency}: $\eta = 40.6\%$ (સૈદ્ધાંતિક મહત્તમ)
    \item \keyword{Ripple Factor}: $r = 1.21$ (ઉચ્ચ ripple content)
    \item \keyword{Peak Inverse Voltage (PIV)}: $PIV = V_m$ (diode પર મહત્તમ reverse voltage)
    \item \keyword{DC Output}: $V_{DC} = \frac{V_m}{\pi} = 0.318 V_m$ જ્યાં $V_m$ એ peak AC voltage છે
\end{itemize}

\textbf{એપ્લિકેશન્સ:}
Half-wave rectifiers નો ઉપયોગ low-power applications માં થાય છે જેવા કે battery charging, signal demodulation, અને voltage multipliers. તેઓ poor efficiency ને કારણે high-power applications માટે \textit{યોગ્ય નથી}.
\end{solutionbox}

\begin{mnemonicbox}
\mnemonic{HWR: Half-Wave = અડધું output, 40.6\% efficiency, PIV = Vm}
\end{mnemonicbox}

\end{document}
