
% content/resources/templates/gujarati-boxes.tex
\usepackage{fontspec}
\usepackage{polyglossia}
\setmainlanguage{gujarati}
\setotherlanguage{english}
\newfontfamily\gujaratifont[Script=Gujarati,AutoFakeBold=2.5,AutoFakeSlant=0.3]{Noto Sans Gujarati}
\setmainfont[Script=Gujarati,AutoFakeBold=2.5,AutoFakeSlant=0.3]{Noto Sans Gujarati}
\setmonofont[Script=Gujarati,AutoFakeBold=2.5,AutoFakeSlant=0.3]{Noto Sans Gujarati}

% Translations for polyglossia
\gappto\captionsgujarati{
  \renewcommand{\tablename}{કોષ્ટક}
  \renewcommand{\figurename}{આકૃતિ}
}

% Redefine question commands for Gujarati
\AtBeginDocument{
  \renewcommand{\question}[2]{%
    \section*{પ્રશ્ન #1}%
    \vspace{0.3em}%
    {\large\textbf{#2}}%
    \vspace{0.5em}%
  }
  \renewcommand{\questionor}[2]{%
    \section*{પ્રશ્ન #1 અથવા}%
    \vspace{0.3em}%
    {\large\textbf{#2}}%
    \vspace{0.5em}%
  }
  \renewcommand{\questionmarks}[3]{%
    \section*{પ્રશ્ન #1 [#2 ગુણ]}%
    \addcontentsline{toc}{section}{પ્રશ્ન #1: #3}%
    \markboth{પ્રશ્ન #1}{પ્રશ્ન #1}%
    \vspace{0.3em}%
    {\large\textbf{#3}}%
    \vspace{0.5em}%
  }
}

% Custom environments
\newtcolorbox{solutionbox}{
 breakable,
 colback=solutioncolor!5!white,
 colframe=solutioncolor!75!black,
 fonttitle=\bfseries,
 title=જવાબ
}

\newtcolorbox{solutionboxnobreak}{
 colback=solutioncolor!5!white,
 colframe=solutioncolor!75!black,
 fonttitle=\bfseries,
 title=જવાબ
}

\newtcolorbox{keyformula}{
 colback=keycolor!5!white,
 colframe=keycolor!75!black,
 fonttitle=\bfseries,
 title=રાસાયણિક સમીકરણ/સૂત્ર
}

\newtcolorbox{mnemonicbox}{
 colback=mnemoniccolor!5!white,
 colframe=mnemoniccolor!75!black,
 fonttitle=\bfseries,
 title=મેમરી ટ્રીક
}
