\documentclass{article}
\usepackage{tabularx}
\usepackage{tabulary}
\usepackage[margin=1in]{geometry}
\usepackage{caption}

\begin{document}

\section*{Tabularx (Always Full Width)}
\begin{center}
\begin{tabularx}{\linewidth}{|l|X|}
\hline
Short & Data \\ \hline
Short & Data \\ \hline
\end{tabularx}
\end{center}

\section*{Tabulary (Adaptive)}
% L = Left align wrapping, C = Center wrapping, R = Right wrapping, J = Justified wrapping
% l, c, r = standard non-wrapping (like tabular)
\begin{center}
\textbf{Case 1: Short Data (Should be compact)}
\begin{tabulary}{\linewidth}{|l|L|}
\hline
Short & Data \\ \hline
Short & Data \\ \hline
\end{tabulary}

\vspace{1cm}

\textbf{Case 2: Long Data (Should expand and wrap)}
\begin{tabulary}{\linewidth}{|l|L|}
\hline
Feature & Description \\ \hline
Wrapping & This is a very long text that should definitely wrap because it is inside a tabulary environment with the L column type, which allows wrapping. \\ \hline
\end{tabulary}
\end{center}

\end{document}
